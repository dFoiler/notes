\documentclass[../notes.tex]{subfiles}

\begin{document}

% !TEX root = ../notes.tex

Today we will review.

\subsection{Flat and Projective Modules}
Recall the following equivalent conditions for an $R$-module $M$ to be flat.
\begin{itemize}
	\item $N\into N'$ remains an injection as $M\otimes_RN\into M\otimes_RN'$.
	\item $\op{Tor}_1^R(M,-)$ vanishes.
	\item $\op{Tor}_1^R(M,R/I)$ vanishes for any finitely generated ideal $I$.
\end{itemize}
Here are some equivalent conditions for an $R$-module $M$ to be projective.
\begin{itemize}
	\item $\op{Hom}_R(M,-)$ is exact.
	\item There is some $N$ for which $M\oplus N$ is free.
\end{itemize}
One way to see that a projective module is to flat is to note that $\op{Tor}_1^R(M,-)$ vanishes easily because the (augmented) projective resolution for $M$ is simply
\[\cdots\to0\to0\to M\to0.\]
Other examples are flat modules tend to be localizations or tensor products of such things. For example, $\QQ$ is a flat but not projective module. It does turn out that finitely presented flat modules are projective, which one can see because in this case both flat and projective are the same as being locally free.

\subsection{Invertible Modules}
Invertible modules are finitely generated modules which are locally isomorphic to $R_\mf p$ (namely, upon localizing at a prime $\mf p$). For example, take $R\coloneqq\ZZ[\sqrt{-5}]$. Then the invertible modules are the fractional ideals of $\QQ(\sqrt{-5})$, which generate $C(R)$.

Notably, $\ZZ[\sqrt{-5}]$ is normal (it's the normal closure of $\ZZ$) and of dimension $1$ (it's integral over the one-dimensional ring $\ZZ$), so $R$ is a Dedekind domain. Thus, all nonzero ideals are invertible, and there is unique prime factorization of invertible ideals, so our invertible ideals in $C(R)$ are freely generated by our prime ideals.

However, $\ZZ[\sqrt{-5}]$ is not a principal ideal domain because it is not a unique factorization domain, as witnessed by
\[\left(1+\sqrt{-5}\right)\left(1-\sqrt{-5}\right)=2\cdot3.\]
For example, this means that the Picard group $\op{Pic}R$ is nontrivial.

As another example, take $R\coloneqq\ZZ[\sqrt5]$. This ring is not normal, so we might be able to find an ideal which is not invertible. For example, we can see that the kernel of the map
\[\ZZ[\sqrt5]\to\ZZ/2\ZZ\]
by taking $\sqrt5\mapsto1$ is $I\coloneqq\left(2,1+\sqrt5\right)$. As such, we conclude that $I$ is a maximal ideal. To see that $I$ is invertible, we proceed by the definitions: if we localize at an ideal $\mf p$ away from $I$, then we will get $I_\mf p=R_\mf p$ set-theoretically because $I$ will have an element outside $\mf p$.

Lastly, with $\mf p=I$, we need to show that $I_I$ is not principal, for which we check instead that
\[\dim_{R_I/I_I}I_I/I_I^2\ne1.\]
Now, $R/I\cong\FF_2$, so $R_I/I_I\cong\FF_2$ still. Well, we compute
\[I^2=\left(4,2+2\sqrt5,6+2\sqrt5\right)=\left(4,2+2\sqrt5\right).\]
Then $I/I^2$ will have the elements $2$ and $1+\sqrt5$ remaining linearly independent, so we are done.
\begin{remark}
	Back in $\ZZ[\sqrt{-5}]$, the ideal $\mf p\coloneqq\left(2,1+\sqrt{-5}\right)$ is still invertible. The difference is that $\mf p_\mf p$ is now still principal, generated by $1+\sqrt{-5}$ because the other generator has
	\[\left(1+\sqrt{-5}\right)\cdot\frac{1-\sqrt{-5}}3=2.\]
\end{remark}

\subsection{Grassmannian}
Let's do an example of a projective variety: Grassmannians $\op{Gr}(p,n)$. 
\begin{definition}[Grassmannian]
	Set $k$ be an algebraically closed field. The space $\op{Gr}(p,n)$ consists of the $p$-dimensional subspaces of $k^n$.
\end{definition}
\begin{example}
	Recall
	\[\PP^{n-1}=\op{Gr}(1,n),\]
	meaning that $\PP^{n-1}$ is made up of the $1$-dimensional subspaces of $k^n$.
\end{example}
We would like to show that Grassmannians are projective varieties.
\begin{exe}
	Set $X\coloneqq\op{Gr}(2,4)$. We show $X$ is projective and compute $\dim X$.
\end{exe}
\begin{proof}
	We are looking at planes in $k^4$. For concreteness, give $k^4$ the standard basis $\{e_1,e_2,e_3,e_4\}$. As such, we can enumerate planes $V\subseteq k^4$ by their basis
	\[\begin{bmatrix}
		a_1 \\
		a_2 \\
		a_3 \\
		a_4
	\end{bmatrix}\qquad\text{and}\qquad\begin{bmatrix}
		b_1 \\
		b_2 \\
		b_3 \\
		b_4
	\end{bmatrix}.\]
	We glue these together into matrices
	\[M\coloneqq\begin{bmatrix}
		a_1 & b_1 \\
		a_2 & b_2 \\
		a_3 & b_3 \\
		a_4 & b_4
	\end{bmatrix}.\]
	Of course, for this to make a plane, we need the matrix to have full rank, which means that there is no nontrivial relation among the columns; this cuts out a Zariski open set from $k^8$. Let $M_0(k)$ be the set of all such matrices.

	Of course, it is possible for two matrices $M,M'\in M_0(k)$ to generate the same plane, which will happen whenever there is a way to make the columns of $M'$ as a linear combination from the columns of $M$. Equivalently, we are asking for a matrix $A\in\op{GL}_2(k)$ such that
	\[M'=MA.\]
	In general, there is a $\op{GL}_2(k)$-action on $M_0(k)$, so our planes can be nicely parameterized as elements of $X=M_0(k)/\op{GL}_2(k)$.

	To make $X$ projective, we set
	\[p_{ij}=\deg\begin{bmatrix}
		a_i & b_i \\
		a_j & b_j
	\end{bmatrix}\]
	using the coordinates for $M$ above. Of course, scaling $M$ merely changes the $p_{ij}$ but does not change the plane (and the condition that our matrices have full rank means that at least one of the $p_{ij}$ is nonzero), so we can be assured a well-defined map
	\[X\to\PP^5(k)\]
	by mapping to our coordinates $(p_{12}:p_{13}:p_{14}:p_{23}:p_{24}:p_{34})$. We now turn to computing the dimension of $X$.

	For example, suppose $p_{12}\ne0$. After doing some shifting via our $\op{GL}_2(k)$-action, we may assume that we have a matrix of the form
	\[\begin{bmatrix}
		1 & 0 \\
		0 & 1 \\
		x_1 & y_1 \\
		x_2 & y_2
	\end{bmatrix}.\]
	In particular, localizing our space $X$ at $p_{12}=1$, we will simply get $\AA^2(k)$ for the bottom four coordinates.
	
	Running the above argument through for all coordinates $p_{ij}$, we have covered $X$ with six copies of $\AA^4(k)$, so $X$ is a four-dimensional projective variety.
	\begin{remark}
		In general, $\dim\op{Gr}(p,n)=p(n-p)$.
	\end{remark}
	We now turn to writing down equations for $X$. We start by noting that
	\[p_{12}p_{34}-p_{13}p_{24}+p_{14}p_{23}=0\tag{$*$}\label{eq:grassmannian}\]
	essentially by directly expanding the definition of the $p_{ij}$. A vaguely smarter way is to optimize our computation by using our affine charts of $\AA^4(k)$.

	It remains to see that \autoref{eq:grassmannian} is our only equation. Well, we construct our matrix backwards by hand. Without loss of generality, take $p_{12}=1$ (after some scaling and rearranging), and we just use the matrix
	\[\begin{bmatrix}
		1 & 0 \\
		0 & 1 \\
		-p_{23} & p_{13} \\
		-p_{24} & p_{14}
	\end{bmatrix}.\]
	So indeed, \autoref{eq:grassmannian} is sharp enough to cut out $X$.
\end{proof}
\begin{remark}
	In general, the Grassmannians are all quadratic varieties.
\end{remark}

\subsection{The Associated Graded Ring}
Recall the associated graded ring
\[\op{gr}_IR\coloneqq\bigoplus_{n\ge0}I^n/I^{n+1}.\]
Geometrically, we might imagine $R$ is an affine domain and $I$ a maximal ideal; then $\op{gr}_IR$ is the ring of functions on the tangent cone.
\begin{exe}
	We work out the tangent cone for $y^2=x^2(x+1)$ at $(0,0)$.
\end{exe}
\begin{proof}
	Here is our image.
	\begin{center}
		\begin{asy}
			unitsize(1cm);
			import graph;
			real x(real t)
			{
				return t*t-1;
			}
			real y(real t)
			{
				return t*(t*t-1);
			}
			draw(graph(x, y,-1.521,1.521));
			draw((-1.5,0) -- (2,0), dotted); label("$x$", (2,0), E);
			draw((0,-2) -- (0,2), dotted); label("$y$", (0,2), N);
			dot((0,0), red);
			draw((-0.5,-0.5)--(0.5,0.5), red);
			draw((-0.5,0.5)--(0.5,-0.5), red);
		\end{asy}
	\end{center}
	We expect the tangent cone to be the union of the two lines $y=x$ and $y=-x$.

	On the algebraic side, we are setting
	\[R\coloneqq\frac{k[x,y]}{\left(y^2-x^2(x+1)\right)}\qquad\text{and}\qquad\mf m=(x,y),\]
	and we want to study $\op{gr}_\mf mR$. To begin, we note that $R$ is not normal because $t\coloneqq y/x$ satisfies the monic polynomial
	\[t^2-(x+1)=0.\]
	As such, our normalization will have to include $t$; concretely, our normalization map takes
	\[x\mapsto t^2-1\qquad\text{and}\qquad y\mapsto t\left(t^2-1\right),\]
	which will turn out to embed $R\into k[t]$, with $R\cong k\left[t^2-1,t^3-t\right]$. In particular,
	\[\mf m=\left(t^2-1,t^3-t\right)\]
	under this embedding, so $t^2-1$ and $t^3-t$ will generate $\mf m/\mf m^2$. However, multiplying these two generators together will kill them in $\mf m^2/\mf m^3$, so we see
	\[\op{gr}_\mf mR\cong k[z,w]/(zw)\]
	after a little reparameterization. The $z=0$ and $w=0$ correspond to $t=1$ and $t=-1$, which are our tangent cone lines $y=\pm x$.
\end{proof}

\subsection{Open Maps}
As a last remark to close out class, we do a little more algebraic geometry. 
\begin{theorem}
	Suppose $\varphi\colon X\to Y$ is a map of affine varieties, and suppose accordingly that $A(Y)$ is normal and that $A(X)$ is integral over $A(Y)$ by the map
	\[\varphi^*\colon A(Y)\to A(X).\]
	In this case $\varphi$ is an open map.
\end{theorem}
\begin{proof}
	Let $U\subseteq X$ be Zariski open so that $U=X\setminus V(f)$ for some element $f\in A(X)$. We need to show that
	\[Y\setminus\varphi(U)=Y\setminus Z(f)\]
	is closed in $Y$. Translating this into algebra, we are asking which maximal ideals $\mf m\subseteq A(Y)$ to contain $f$ so that this localization goes through correctly.
	
	We leave the rest of the proof as an exercise; the main idea is that $f$ satisfies a monic polynomial in elements of $\im\varphi^*$, so we should show that $Y\setminus\varphi(U)$ is given by the ideal generated by the coefficients of this monic polynomial. Then we use the geometric $AKLB$ set-up.
\end{proof}
\begin{remark}
	The final is expected to emphasize dimension theory (but not exclusively dimension theory). It will be about 8 problems.
\end{remark}

\end{document}