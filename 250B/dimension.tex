% LTeX: enabled=false

\documentclass{article}
\usepackage[utf8]{inputenc}

\newcommand{\nirpdftitle}{Dimension Theory}
\usepackage{import}
\inputfrom{..}{nir}

\pagestyle{contentpage}

\title{Dimension Theory Speedrun}
\author{Nir Elber}
\date{Spring 2022}
\rhead{\textit{250B: DIM. SPEEDRUN}}
\lhead{}

\begin{document}

\maketitle

\begin{abstract}
	\noindent This document collects a variety of dimension-computing results from Eisenbud's \textit{Commutative Algebra: with a View Toward Algebraic Geometry}. All references are to this book.
\end{abstract}

\setcounter{secnumdepth}{4}
\setcounter{tocdepth}{4}
\tableofcontents

\newpage

\section{Definitions}

\subsection{Kinds of Dimension}
\begin{defi*}[Dimension] \label{def:krulldim}
	The \textit{Krull dimension} of a ring $R$, denoted $\dim R$, is the supremum of the length $r$ of a chain of distinct primes
	\[\mf p_0\subsetneq\mf p_1\subsetneq\cdots\subsetneq\mf p_r.\]
\end{defi*}
\begin{definition*}[Dimension, ideals]
	Fix a ring $R$ and an ideal $I\subseteq R$. Then we define the \textit{dimension} of an ideal $I$ to be $\dim I\coloneqq\dim R/I$.
\end{definition*}
\begin{lemma*} \label{lem:dimisascend}
	Fix an ideal $I$ of a ring $R$. Then $\dim I$ is equal to the length of the longest chain of primes
	\[I\subseteq\mf p_0\subsetneq\mf p_1\subsetneq\cdots\subsetneq\mf p_d\]
	in $R$.
\end{lemma*}
\begin{definition*}[Codimension]
	Fix $I$ a proper ideal of a ring $R$.
	\begin{itemize}
		\item If $I=\mf p$ is a prime ideal of $R$, then we define the \textit{codimension} as $\codim\mf p\coloneqq\dim R_\mf p$.
		\item More generally, we define the \textit{codimension} as
		\[\codim I\coloneqq\min_{\mf p\subseteq I}\codim\mf p,\]
		where the minimum is over all prime ideals $\mf p$ containing $I$.
	\end{itemize}
\end{definition*}
\begin{lemma*} \label{lem:codimisdescend}
	Fix a prime ideal $\mf p$ of a ring $R$. Then $\codim\mf p$ is equal to the length of the longest chain of primes
	\[\mf p_0\subsetneq\mf p_1\subsetneq\cdots\subsetneq\mf p_d=\mf p,\]
	where $\mf p$ is included in the chain; i.e., $\codim\mf p=d$ here.
\end{lemma*}
\begin{definition*}[Dimension, modules]
	Given an $R$-module $M$, we define the \textit{dimension} of $M$ as $\dim M:=\dim R/\op{Ann}M$.
\end{definition*}

\subsection{Kinds of Rings}
\begin{definition*}[Regular]
	Fix a local ring $R$ of dimension $d\coloneqq\dim R$. Further, let $\mf m$ be the maximal ideal of $R$. Then $R$ is \textit{regular} if and only if there exist elements $\{f_1,\ldots,f_d\}\subseteq R$ such that
	\[\mf m=(f_1,\ldots,f_d).\]
\end{definition*}
\begin{remark*}[Corollary 10.14]
	Regular local rings are integral domains.
\end{remark*}
\begin{definition*}[Discrete valuation ring]
	A \textit{discrete valuation ring} is an integral domain $R$ equipped with a valuation $\nu\colon K(R)^\times\to\ZZ$.
\end{definition*}
\begin{proposition*}[Proposition 11.1] \label{prop:dvrgrabbag}
	Fix a Noetherian ring $R$. The following are equivalent.
	\begin{itemize}
		\item $R$ is a discrete valuation ring.
		\item $R$ is a field or regular local ring of dimension $1$.
	\end{itemize}
\end{proposition*}
\begin{definition*}[Dedekind]
	A \textit{Dedekind domain} is a Noetherian normal domain of dimension $1$.
\end{definition*}

\newpage
\section{Theorems}

\subsection{First Results}
\begin{proposition*} \label{prop:upperbounddimension}
	Fix $I$ an ideal of a ring $R$. Then
	\[\dim I+\codim I\le\dim R.\]
\end{proposition*}
\begin{remark*}
	Equality for the above holds when $R$ is an affine domain, by Corollary 13.4.
\end{remark*}
\begin{lemma*} \label{lem:diminclusionbounds}
	Fix ideals $I$ and $J$ in a ring $R$. If $I\subseteq J$, then
	\[\dim I\ge\dim J\qquad\text{and}\qquad\codim I\le\codim J.\]
	Similarly, if $\mf p$ and $\mf q$ are primes with $\mf p\subseteq\mf q$, then $\codim\mf p\le\codim\mf q$, with equality if and only if $\mf p=\mf q$.
\end{lemma*}

\subsection{Localization, Completion, Polynomials}
\begin{theorem*} \label{thm:dimislocal}
	Fix a ring $R$. Then
	\[\dim R=\max_{\mf p\in\op{Spec}R}\dim R_\mf p.\]
	In other words, dimension is a local quantity.
\end{theorem*}
\begin{corollary*}[Corollary 10.12]
	Fix a local Noetherian ring $R$ and a finitely generated module $M$. Then $\dim R=\dim\widehat R$.
\end{corollary*}
\begin{remark*}
	Corollary 10.12 is strictly weaker than Corollary 12.15.
\end{remark*}
\begin{cor*}[Corollary 10.13] \label{prop:polyringdimension}
	Fix a Noetherian ring $R$ with finite dimension. Then $\dim R[x]=\dim R+1$.
\end{cor*}

\subsection{Comparing Rings}
\begin{proposition*}[Proposition 9.2] \label{prop:intextensiondimen}
	Fix a ring homomorphism $\varphi\colon R\to S$ which makes $S$ into an integral $R$-algebra. Then, for any $\mf p\in\op{Spec}R$ such that $\ker\varphi\subseteq\mf p$, there exists $\mf q\in\op{Spec}S$ such that
	\[\mf p=\varphi^{-1}(\mf q).\]
	In fact, for any ideal $I\subseteq S$, we have $\dim S/I=\dim R/\varphi^{-1}(I)$. In particular, if $\varphi\colon R\to S$ is injective, then $\dim R=\dim S$.
\end{proposition*}
\begin{lem*} \label{lem:localizedim}
	Fix a ring $R$ and a multiplicatively closed subset $U\subseteq R$. Further, set $S:=R\left[U^{-1}\right]$ with the natural map $\varphi:R\to S$. Then, for any prime $\mf p\subseteq R\left[U^{-1}\right]$, we have
	\[\codim\varphi^{-1}(\mf p)=\codim\mf p.\]
\end{lem*}
\begin{theorem*}[Theorem 10.10] \label{prop:ranknullity}
	Fix two local rings $R$ and $S$ with maximal ideals $\mf m$ and $\mf n$, respectively. Given a map $\varphi:R\to S$ of local rings so that $\varphi(\mf m)\subseteq\mf n$, we have
	\[\dim S\le\dim R+\dim S/\mf mS\]
	In fact, if $S$ is a flat as an $R$-module, then we have equality.
\end{theorem*}

\subsection{Generating Elements}
\begin{theorem*}[Theorem 10.2] \label{thm:pit}
	Fix a Noetherian ring $R$. Given an ideal $(x_1,\ldots,x_s)\in R$, suppose $\mf p$ is a minimal prime over $(x_1,\ldots,x_s)$. Then
	\[\codim\mf p\le s.\]
\end{theorem*}
\begin{cor*}[Corollary 10.5] \label{cor:pitconverse}
	Fix a prime ideal $\mf p$ of a Noetherian ring $R$ with codimension $r$. Then there are elements $x_1,\ldots,x_r$ such that $\mf p$ is minimal over $(x_1,\ldots,x_r)$, and in fact $\codim(x_1,\ldots,x_r)=r$.
\end{cor*}
\begin{proposition*}[Proposition 10.8] \label{prop:betterlocaldim}
	Fix a local ring $R$ with maximal ideal $\mf m$. Then $\dim R$ is the minimal $d\in\NN$ such that there exist generators $f_1,\ldots,f_d$ so that
	\[\mf m^n\subseteq(f_1,\ldots,f_d)\subseteq\mf m\]
	for some $n$.
\end{proposition*}
\begin{cor*} \label{prop:reglocaldimisdim}
	Let $R$ be a Noetherian regular local ring with maximal ideal $\mf m$. Then
	\[\dim_{R/\mf m}\mf m/\mf m^2=\dim R.\]
\end{cor*}

\subsection{Dimension for Modules}
\begin{proposition*}[Proposition 10.8] \label{prop:generatorsdimension}
	Fix a Noetherian local ring $R$ with maximal ideal $\mf m$ and an $R$-module $M$. Then $\dim M$ is equal to the minimal $d$ such that there is some proper ideal $(f_1,\ldots,f_d)\subseteq R$ with finite colength on $M$.
\end{proposition*}
\begin{cor*}[Corollary 10.9] \label{prop:killoneelement}
	Fix a Noetherian local ring $R$ with maximal ideal $\mf m$ and an $R$-module $M$. Given $x\in\mf m$, we have
	\[\dim M/xM\ge\dim M-1.\]
\end{cor*}
\begin{cor*}[Corollary 12.5]
	Fix a Noetherian local ring $R$. Given a finitely generated $R$-module $M$ with parameter ideal $\mf q$,
	\[\dim M=\dim\widehat M_\mf q=\dim(\op{gr}_\mf qM)_\mf P.\]
	Here, $\mf P\subseteq\op{gr}_\mf qM$ is the irrelevant ideal.
\end{cor*}

\subsection{The Hilbert Function}
\begin{definition*}[Hibert--Samuel function]
	Fix a local Noetherian ring $R$ with finitely generated $R$-module $M$ and some prime of finite colength $\mf q$. Then we define the \textit{Hilbert--Samuel function} by
	\[H_{\mf q,M}(n):=\ell\left(\mf q^nM/\mf q^{n+1}\right).\]
\end{definition*}
\begin{lemma*}
	Fix a local Noetherian ring $R$ with maximal ideal $\mf m$. Further suppose that there is a map $k\into R$ such that the composite
	\[k\into R\onto R/\mf m\]
	is an isomorphism. Then, for any finitely generated $R$-module $M$ of finite length,
	\[\ell_R(M)=\dim_kM.\]
\end{lemma*}
% \begin{lemma*}
% 	Fix a Noetherian affine ring $R$ over an algebraically closed field $k$. Then, given an $R$-module $M$ of finite length, we have
% 	\[\ell_R(M)=\dim_kM.\]
% \end{lemma*}
\begin{theorem*}[Theorem 12.4]
	Fix a local Noetherian ring $R$ with unique maximal ideal $\mf m$. Further, take a finitely generated module $M$ and an ideal $\mf q$ of finite colength on $M$. Then
	\[\dim M=1+\deg P_{\mf q,M}.\]
\end{theorem*}
\begin{cor*}[Corollary 13.7]
	Fix a Noetherian graded ring $R\coloneqq R_0\oplus R_1\oplus\cdots$. Then $\dim R$ is the supremum of $\dim R_\mf p$ for all homogeneous prime ideals $\mf p$.

	Thus, if $R_0$ is a field, then
	\[\dim R=1+\deg P_R,\]
	where $P_R$ is the Hilbert polynomial for $R$.
\end{cor*}

\subsection{Affine Domains}
\begin{theorem*}[Theorem 13.3] \label{thm:noethernormal}
	Fix an affine ring $R$ of dimension $d$. Given a chain
	\[I_1\subseteq I_2\subseteq\cdots\subseteq I_m\subseteq R\]
	with $d_j:=\dim I_j$ such that $\{d_j\}_{j=0}^m$ is strictly decreasing and $d_m>0$. Then there is a subring $S\subseteq R$ such that
	\begin{listalph}
		\item $S\cong k[x_1,\ldots,x_d]$,
		\item $R$ is finite over $S$, and
		\item any ideal $I_j$ has $S\cap I_j=(x_{d_j+1},\ldots,x_d)$.
	\end{listalph}
\end{theorem*}
\begin{theorem*}[Theorem A]
	Fix an affine domain $R$ over a field $k$. Then
	\[\dim R=\op{transcendence~degree}_kR.\]
\end{theorem*}
\begin{corollary*}[Corollary 13.4]
	Fix an affine domain $R$. Given an ideal $I\subseteq R$, we have
	\[\dim I+\codim I=\dim R.\]
\end{corollary*}
\begin{corollary*}[Corollary 13.5]
	Suppose that we have an inclusion $R\subseteq T$ of affine domains over $k$. Then
	\[\dim T=\dim R+\dim K(R)\otimes_RT.\]
\end{corollary*}
\begin{cor*}[Corollary 13.11]
	Fix an affine domain $R$. If $f\in R\setminus\{0\}$ is not a unit, then
	\[\dim R/(f)=\dim R-1.\]
\end{cor*}


\end{document}