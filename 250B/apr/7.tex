% !TEX root = ../notes.tex

We continue.

\subsection{Fractional Ideals}
We continue discussing fractional ideals. Last time we showed the following results.
\begin{lemma} \label{lem:classifyinvertiblemods}
	Fix a Noetherian domain $R$. Then $M$ is invertible if and only if $M$ is isomorphic to some nonzero fractional ideal.
\end{lemma}
We were also in the middle of the following proof, which we will finish today.
\begin{lemma}
	Fix a Noetherian domain $R$. If $I$ and $J$ are nonzero fractional ideals, then
	\[IJ\cong I\otimes_RJ\qquad\text{and}\qquad I^{-1}J\cong\op{Hom}(I.J).\]
\end{lemma}
\begin{proof}
	The map
	\[I\otimes_RJ\to IJ\]
	is by $a\otimes b\mapsto ab$. This is of course surjective, so we just need injectivity. It suffices to show injectivity upon localizing by any prime $\mf p$. But now we are looking at the map
	\[I_\mf p\otimes_{R_\mf p}J_\mf p\to(IJ)_\mf p,\]
	which is injective because $I_\mf p$ and $J_\mf p$ are both free $R_\mf p$-modules (by definition), so we get the injection here automatically because $R_\mf p$ is an integral domain.

	The map
	\[I^{-1}J\to\op{Hom}_R(I,J)\]
	is by sending $t$ to $\mu_t:x\mapsto tx$. This map is injective because $\mu_{t_1}=\mu_{t_2}$ implies they are equal on $a\in I$ (say), so $t_1=t_2$ because $R$ is an integral domain. In fact, we can even say that $\mu_t$ is injective for each nonzero $t$.

	It remains to show surjectivity. Well, pick up some $R$-module homomorphism $\varphi:I\to J$. Now, for some $f\in I\setminus\{0\}$, suppose $\varphi(f)=g$ so that we may consider
	\[\mu_{g/f}(f)=g,\]
	and we can check that $\mu_{g/f}=\varphi$ everywhere by some computation.
\end{proof}
Here is another helpful result.
\begin{lemma}
	Fix a Noetherian domain $R$. Then $I\subseteq K(R)$ is an invertible fractional ideal if and only if $I^{-1}I=R$.
\end{lemma}
\begin{proof}
	On one hand, we see that $I$ being invertible implies that $I^{-1}I\cong\op{Hom}_R(I,I)\cong R$. On the other hand, suppose $I^{-1}I=R$. Localizing gives us
	\[I_\mf pI^{-1}_\mf p=R_\mf p.\]
	But then $vI_\mf p\not\subseteq\mf pR_\mf p$ for some $v\in I^{-1}$, so we can conclude $vI_\mf p=R_\mf p$, so $I$ is indeed locally free.
\end{proof}
As such, we are able to build the following group.
\begin{definition}[Cartier divisors]
	Fix a Noetherian domain $R$. Then a \textit{Cartier divisor} is an invertible fractional ideal.
\end{definition}
From the above results, the Cartier divisors are an abelian group with respect to multiplication, which we all $C(R)$.

Now, we note that we have a homomorphism
\[C(R)\to\op{Pic}R\]
by $I\mapsto[I]$. Notably, \autoref{lem:classifyinvertiblemods} tells us that this homomorphism is surjective, and its kernel consists of ideals $I$ such that $[I]=[R]$, which means $I\cong R$ (as $R$-modules), which means $I$ is principal, generated by some element of $K(R)$. Thus, we have the exact sequence
\[K(R)^\times\to C(R)\to\op{Pic}R\to0.\]
We would like to make this have $0$s on the end, so we note that $a\in K(R)^\times$ will have $(a)=R$ if and only if $a\in R^\times$, so we get to write
\[0\to R^\times\to K(R)^\times\to C(R)\to\op{Pic}R\to0.\]
As such, we have a way to measure $\op{Pic}R$ by objects only internal to $K(R)$.

To make this behave a little better, we pick up the following lemma.
\begin{lemma}
	The group $C(R)$ is generated by invertible ideals $I\subseteq R$.
\end{lemma}
\begin{proof}
	The point is to multiply an arbitrary invertible ideal from $K(R)$ to $R$. Indeed, any invertible fractional ideal $I\in C(R)$ will at least live in $K(R)$. Picking up any nonzero $a\in I\times R$, we note that
	\[I=\left(a^{-1}\right)\cdot(aI),\]
	and $aI\subseteq R$ by construction of $a$. So we are indeed able to generate $C(R)$ as an $R$-module by these invertible ideals.
\end{proof}
Let's see some examples.
\begin{example}
	Fix a principal ideal domain $R$. Then every ideal is principal and hence isomorphic to $R$, so $\op{Pic}R=0$. Namely, $C(R)$ only consists of principal ideals.
\end{example}
\begin{exe}
	We discuss $\op{Pic}\ZZ[\sqrt{-5}]$.
\end{exe}
\begin{proof}
	Fix the Noetherian domain $R=\ZZ[\sqrt{-5}]$. This is normal because it is the integral closure of $\ZZ$ in $\QQ(\sqrt{-5})$, as we showed on the homework. This is dimension $1$ because $R$ is integral over $\ZZ$, and $\dim\ZZ=1$. However, $R$ is not a principal ideal domain because it is not factorial, as
	\[\left(1+\sqrt{-5}\right)\left(1-\sqrt{-5}\right)=2\cdot3\]
	shows. In particular, the ideal $\mf p:=\left(2,1+\sqrt{-5}\right)$ is not principal. In fact, $\ZZ/\mf p=\ZZ/2\ZZ$ is a field,\footnote{Track through the map $\ZZ\to\ZZ[\sqrt{-5}]/\mf p$, and we can note that it is surjective and has kernel $(2)$.} so $\mf p$ is maximal.

	We will take on faith that $\mf p$ is not principal because just look at it. To show that $\mf p$ is invertible, we note that localizing at any prime which is not $\mf p$ will automatically trivialize, so we have left to study
	\[\mf pR_\mf p\subseteq R_\mf p.\]
	But in $R_\mf p$, we see that
	\[2=\frac13\cdot\left(1-\sqrt{-5}\right)\left(1+\sqrt{-5}\right),\]
	so
	\[\mf pR_\mf p=\left(1+\sqrt{-5}\right),\]
	which is indeed principal.

	Thus, we have a nontrivial element of $\op{Pic}\ZZ[\sqrt{-5}]$. We can also compute
	\[\mf p^2=\left(2,1+\sqrt{-5}\right)^2=\left(4,2+2\sqrt{-5},-4+2\sqrt{-5}\right)=\left(4,2+2\sqrt{-5},-6\right)=\left(2,2+2\sqrt{-5},-6\right)=(2),\]
	so this is indeed principal. So $\mf p$ is of order $2$ in $\op{Pic}\ZZ[\sqrt{-5}]$. In fact, this is an isomorphism, which one can see by taking Math 254A.
\end{proof}
\begin{ex}
	Fix $R:=k[x,y]/\left(y^2-x^3\right)\cong k\left[t^2,t^3\right]$ so that $k[t]$ is the normalization of $R$. Now, any ideal of $k[t]$ is principal, so $\op{Pic}k[t]=0$. However, for any invertible ideal $I$ of $k[t]$, then $I\cap k\left[t^2,t^3\right]$ will remain invertible by tracking through the definition. For example, if we take $1+at$ as $a$ varies over $k$, we have a map
	\[k\to\op{Pic}R\]
	by $a\mapsto(1+at)$, which turns out to be an isomorphism. For more, see exercises 11.15 and 11.16.
\end{ex}
\begin{remark}
	It is not technically necessary for $R$ to be a domain in the above results, but the proofs are more annoying. Namely, instead of using the fraction field $K(R)$, one should use the total quotient $K(R)$.
\end{remark}

\subsection{Divisors}
We now talk about divisors a little more generally. We pick up the following definition.
\begin{definition}[Pure codimension]
	Fix a Noetherian domain $R$. Then $I\subseteq R$ has \textit{pure codimension $1$} if and only if every prime associated to $I$ has codimension $1$.
\end{definition}
\begin{theorem} \label{thm:upfforideals}
	Fix a Noetherian domain $R$ such that $R_\mf m$ is factorial for each maximal ideal $\mf m$. Then the following are true.
	\begin{listalph}
		\item An ideal $I\subseteq R$ is invertible if and only $I$ has pure codimension $1$.
		\item An invertible fractional ideal $I$ can be written uniquely as
		\[I=\mf p_1^{m_1}\cdots\mf p_n^{m_n},\]
		for distinct prime ideals $\mf p_1,\ldots,\mf p_n$ of codimension $1$.
	\end{listalph}
\end{theorem}
We will prove this momentarily, but let's talk about some consequences.
\begin{corollary}
	Fix a Noetherian domain $R$. Then $C(R)$ is a free abelian group generated by prime ideals $\mf p$ of codimension $1$.
\end{corollary}
\begin{proof}
	This follows directly from part (b) of the theorem.
\end{proof}
Here is the case that number theorists care about.
\begin{definition}[Dedekind]
	A \textit{Dedekind domain} is a Noetherian normal domain of dimension $1$.
\end{definition}
Notably, in a Dedekind domain, all primes of codimension $1$ are maximal, which are all now invertible by (a) of the theorem. In particular, $R_\mf m$ is indeed factorial for all maximal ideals $\mf m$ because we showed last class that a Noetherian domain being normal is equivalent to all the primes $\mf p$ associated to a principal ideal has $\mf pR_\mf p\subseteq R_\mf p$ principal, which makes $R_\mf p$ a discrete valuation ring and in particular factorial.

We now prove our theorem.
\begin{proof}[Proof of \autoref{thm:upfforideals}]
	We go one at a time.
	\begin{listalph}
		\item Fix $I$ an invertible fractional ideal. Then $R_\mf m$ is factorial, so we showed a while ago that this implies $\mf mR_\mf m$ (which is a codimension-$1$ prime) must be principal, so we are done.

		We  now show the other direction. Well, if $\mf p$ is a prime of codimension $1$, then place $\mf p$ in some maximal ideal $\mf m$, and we see that $\mf p_\mf m$ is principal and hence codimension $1$ in the factorial ring $R_\mf m$. This finishes this direction.

		\item Fix an invertible fractional ideal $I$. Then we know that any prime $\mf p$ associated to $I$ has codimension $1$, by part (a). To start, we show that $I$ is a finite product of primes. Well, otherwise we could find an ideal $I$ of $R$ maximal with respect to not being a product of primes, and place $I$ in a maximal ideal $\mf m$. Of course, $I\subsetneq\mf m$ because $\mf m$ is its own factorization, so we look at
		\[\mf m^{-1}I\subsetneq R.\]
		Notably, $\mf m^{-1}I\supsetneq I$ would imply that $\mf m^{-1}I$ would have a factorization into primes, giving $I$ a factorization into primes.

		So we have left to show $I\subsetneq\mf m^{-1}I$ require using that $R$ is normal. In particular, $\mf m^{-1}I=I$ would imply, by the Cayley--Hamilton theorem, we have that every element $x\in\mf m^{-1}$ is integral over $R$ and hence is in $R$, so $\mf m^{-1}=R$, which does not make sense.

		Lastly, we show uniqueness. Well, if
		\[\prod_{k=1}^m\mf p_k=\prod_{\ell=1}^n\mf q_\ell,\]
		we pick up some $\mf q_n$, and by the product, we can say that some $\mf p_k$ contains $\mf q_1$. But $\mf p_k$ has codimension $1$, so $\mf p_k=\mf q_1$, so we can cancel from both sides and then induct downwards.
		\qedhere
	\end{listalph}
\end{proof}
With the above in mind, we see that we are justified in only caring about the primes of codimension $1$. This gives us the following definition.
\begin{definition}[Divisor]
	Fix a Noetherian domain $R$. Then the group of \textit{divisors} $\op{Div}R$ is the free abelian group generated by all primes of codimension $1$ (as letters).
\end{definition}
Notably, there is a good homomorphism
\[\varphi:C(R)\to\op{Div}R,\]
though they are not the same. To see this, take an invertible ideal $I\in C(R)$ and then set
\[\varphi(I):=\sum_\mf p\ell(R_\mf p/I_\mf p)[\mf p].\]
Notably, the length $\ell(R_\mf p/I_\mf p)$ is finite because $\dim R_\mf p/I_\mf p=0$ (making $R_\mf p/I_\mf p$ Artinian) by the principal ideal theorem: we get $\dim R_\mf p=1$ and $\dim I_\mf p=1$, so we bound $\dim R_\mf p/I_\mf p$ down to $0$. It requires some work to show that $\varphi$ is a homomorphism. Namely, we have to show that
\[\ell(R_\mf p/(IJ)_\mf p)\stackrel?=\ell(R_\mf p/I_\mf p)+\ell(R_\mf p/I_\mf p).\]
We are able to force $I_\mf p$ and $J_\mf p$ to be principal by using our theory of modules of finite length, so by replacing $R$ with $R_\mf p$, we are showing
\[\ell(R/(IJ))\stackrel?=\ell(R/I)+\ell(R/J),\]
where $I=(a)$ and $J=(b)$. But then we can build the filtration for $R/(IJ)$ by hand by zippering the filtrations for $R/I$ and $R/J$ together.
\begin{remark}
	The homomorphism $\varphi$ is in general not injective, but it will be injective when $R$ is also normal. The main idea is that, if $R$ is normal, then $R_\mf p$ will be factorial and in particular a discrete valuation ring, so $\ell(R_\mf p/I_\mf p)$ vanishing everywhere forces $I$ to vanish.
\end{remark}