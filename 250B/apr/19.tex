\documentclass[../notes.tex]{subfiles}

\begin{document}

% !TEX root = ../notes.tex

Welcome back.

\subsection{The Nullstellensatz Two, Electric Boogaloo}
Last time we showed Noether normalization. Today we go over some consequences. Recall the statement of Noether normalization.
\begin{theorem}[Noether normalization]
	Fix an affine ring $R$ of dimension $d$. Given a chain
	\[I_1\subseteq I_2\subseteq\cdots\subseteq I_m\subseteq R\]
	with $d_j:=\dim I_j$ such that $\{d_j\}_{j=0}^m$ is strictly decreasing. Then there is a subring $S\subseteq R$ such that
	\begin{listalph}
		\item $S\cong k[x_1,\ldots,x_d]$,
		\item $R$ is finite over $S$, and
		\item any ideal $I_j$ has $S\cap I_j=(x_{d_j+1},\ldots,x_d)$.
	\end{listalph}
\end{theorem}
As an example application, we show Hilbert's Nullstellensatz.
\begin{theorem}[Nullstellensatz]
	Fix $R$ an (affine) $k$-algebra.
	\begin{listalph}
		\item Given a maximal ideal $\mf m\subseteq R$, then $R/\mf m$ is a finite extension of $k$. In particular, if $k$ is algebraically closed, then $R/\mf m\cong k$.
		\item The ring $R$ is Jacobson: any prime ideal is the intersection of maximal ideals.
	\end{listalph}
\end{theorem}
From here one can deduce the usual Nullstellensatz. Anyway, let's prove this.
\begin{proof}
	We go one at a time.
	\begin{listalph}
		\item The dimension of $R/\mf m$ is $0$ because $R/\mf m$ is a field, but $R/\mf m$ is an integral extension of $k$ (because $R$ is) while being of finite length (and finitely generated as a $k$-algebra), so $R/\mf m$ is a finite extension of $k$.
		\item As usual, pick up a prime $\mf p$, and we need to show
		\[\mf p\stackrel?=\bigcap_{\mf m\supseteq\mf p}\mf m.\]
		Certainly we have $\subseteq$, so we show $\supseteq$. As such, for each $f\notin\mf p$, we need to show that there exists a maximal ideal $\mf m$ containing $\mf p$ but not $f$.

		Now, taking the quotient by $\mf p$, we may assume that $R$ is an integral domain and that $\mf p=(0)$. In particular, we need to show that $f\ne0$ is avoided by some maximal ideal. However, we showed last time that
		\[\dim R/(f)=\dim R-1.\]
		Running through Noether normalization, we can choose $S\cong k[x_1,\ldots,x_d]$ so that $S\cap(f)=(x_1)$. So to get our maximal ideal, we choose
		\[(x_1-1,x_2,x_3,\ldots,x_d)\subseteq S\]
		and then lift to $R$ to some maximal ideal $\mf n$. Notably, $f\in\mf n$ implies that $\mf n\cap S$ contains $x_1$, which doesn't make sense by the above construction. This finishes.
		\qedhere
	\end{listalph}
\end{proof}

\subsection{The Geometric \texorpdfstring{$AKLB$}{AKLB} Set-Up}
Here is another application.
\begin{proposition}[\texorpdfstring{$AKLB$}{} for geometers] \label{prop:geometricaklb}
	Fix an affine domain $R$ over a field $k$. Now, set $L$ to be a finite extension of $K:=K(R)$, and let $T$ be the integral closure of $R$ in $L$. Then $T$ is a finitely generated $R$-module; in particular, $T$ is an affine domain.
\end{proposition}
\begin{proof}
	Here is the image.
	% https://q.uiver.app/?q=WzAsNCxbMSwwLCJMIl0sWzEsMSwiSyJdLFswLDEsIlIiXSxbMCwwLCJTIl0sWzIsMywiIiwwLHsic3R5bGUiOnsiaGVhZCI6eyJuYW1lIjoibm9uZSJ9fX1dLFsxLDAsIiIsMCx7InN0eWxlIjp7ImhlYWQiOnsibmFtZSI6Im5vbmUifX19XSxbMiwxLCJcXHN1YnNldGVxIiwzLHsic3R5bGUiOnsiYm9keSI6eyJuYW1lIjoibm9uZSJ9LCJoZWFkIjp7Im5hbWUiOiJub25lIn19fV0sWzMsMCwiXFxzdWJzZXRlcSIsMyx7InN0eWxlIjp7ImJvZHkiOnsibmFtZSI6Im5vbmUifSwiaGVhZCI6eyJuYW1lIjoibm9uZSJ9fX1dXQ==&macro_url=https%3A%2F%2Fraw.githubusercontent.com%2FdFoiler%2Fnotes%2Fmaster%2Fnir.tex
	\[\begin{tikzcd}
		T & L \\
		R & K
		\arrow[no head, from=2-1, to=1-1]
		\arrow[no head, from=2-2, to=1-2]
		\arrow["\subseteq"{marking}, draw=none, from=2-1, to=2-2]
		\arrow["\subseteq"{marking}, draw=none, from=1-1, to=1-2]
	\end{tikzcd}\]
	To begin, we use Noether normalization to force $S$ to be a polynomial ring. Namely, $T$ is still integral over $S$ (because of the chain $S\subseteq R\subseteq T$), and it still suffices to show that $T$ is finite over $S$.

	As such, we may assume that $R$ is a unique factorization domain and in particular is normal. As another reduction, by taking the normal closure of $L$, we would only make $T$ bigger, so it suffices to take $L/K$ to be a normal extension of fields.

	We would like our extension to be Galois, but for this we must fight with the separability condition. In particular, $L/K$ is an inseparable extension if an only if there is an irreducible polynomial $\pi\in K[x]$ with a multiple root $\alpha$; here, this element $\alpha$ is called purely inseparable. This will in fact mean that $\alpha{(\op{char}K)^n}\in K$ for some $n$.

	Nonetheless, with $L/K$ any normal extension, we can talk about its automorphism group $G$ and then build the following tower.
	% https://q.uiver.app/?q=WzAsMyxbMCwwLCJMIl0sWzAsMSwiTF5HIl0sWzAsMiwiRyJdLFsyLDEsIlxcdGV4dHtHYWxvaXN9IiwwLHsic3R5bGUiOnsiaGVhZCI6eyJuYW1lIjoibm9uZSJ9fX1dLFsxLDAsIlxcdGV4dHtwdXJlbHkgaW5zZXB9Il1d&macro_url=https%3A%2F%2Fraw.githubusercontent.com%2FdFoiler%2Fnotes%2Fmaster%2Fnir.tex
	\[\begin{tikzcd}
		L \\
		{L^G} \\
		G
		\arrow["{\text{Galois}}"', no head, from=3-1, to=2-1]
		\arrow["{\text{purely insep}}"', no head, from=2-1, to=1-1]
	\end{tikzcd}\]
	Namely, we want to not think very hard about $L/L^G$, but we must.
	\begin{example}
		The extension $k(t)\subseteq k\left(t^{1/p}\right)$, where $p:=\op{char}k>0$, is a purely inseparable extension.
	\end{example}
	Anyway, this tower that we may assume that $L/K$ is either Galois or purely inseparable. We do these cases separately.
	\begin{itemize}
		\item Take $L/K$ to be purely inseparable. Here, $K=k(x_1,\ldots,x_d)$. Because $L/K$ is finite and purely inseparable, we can find some $q$ (which is a large power of $p$) so that $L$ is contained in
		\[k'\left(x_1^{1/q},\ldots,x_d^{1/q}\right),\]
		where in $k'$ we have to possibly add in some $q$th roots to make this legal. In particular, we take $L=K\left(\alpha_1^{1/q},\ldots,\alpha_d^{1/q}\right)$ and then take out the $q$th roots of variables we need using the fact that $(x+y)^q=x^q+y^q$.

		As such, it suffices to show our result in the case where $L=k'\left(x_1^{1/q},\ldots,x_d^{1/q}\right)$. But now we can exactly describe our integral closure as
		\[T=k'\left[x_1^{1/q},\ldots,x_d^{1/q}\right],\]
		which is finite over $R$ because $k'/k$ is finite, and each of the $x_i^{1/q}$ is the root of a monic polynomial $t^q-x_i=0$ over $R$. This finishes.

		\item Now take $L/K$ to be a Galois extension. We pick up the following lemma. We simply outsource the logic here to the following lemma.
	\end{itemize}
	\begin{lemma}
		Fix a normal Noetherian domain $R$. Now, set $L$ to be a finite, Galois extension of $K:=K(R)$ where $G:=\op{Gal}(L/K)$, and let $T$ be the integral closure of $R$ in $L$. Then $T$ is a finitely generated $R$-module.
	\end{lemma}
	\begin{proof}
		Note $R\subseteq K=L^G$, and $R$ is normal (i.e., is its own Galois closure in $K$), so it follows that $R^G=R$. Now, by clearing denominators, we may choose elements $b_1,\ldots,b_n$ to be a basis of $L$ over $K$. Additionally, we enumerate the elements of $G$ as
		\[\{g_1,\ldots,g_n\},\]
		where $n:=\#\op{Gal}(L/K)=[L:K]$. Now, the Galois action preserves integrality, so the matrix
		\[M:=\begin{bmatrix}
			g_1b_1 & \cdots & g_1b_n \\
			\vdots & \ddots & \vdots \\
			g_nb_1 & \cdots & g_nb_n
		\end{bmatrix}.\]
		Now, set $d:=\det M$. We make the following observations about $d$.
		\begin{itemize}
			\item Note that $d\ne0$ because this would require a relation of the rows
			\[\sum_{i=1}^na_ig_ib_j=0\]
			for all $j$, so in particular $\sum_ia_ig_i=0$ (as a function $L\to K$!), which contradicts linear independence of automorphisms.\footnote{This linear independence is Artin's lemma. As a sketch of the proof, if there is a nontrivial relation, choose the smallest relation, and then rearrange by plugging things in to subtract off and get a smaller relation.}
			\item We do see that $d^2\in R$ because, picking up any $g\in G$ to the matrix $M$ will merely permute the rows of $M$, so $gd=\pm d$. In particular, $d^2$ is fixed by $G$ and hence lives in $K$. Because $d$ is a linear combination of integral elements (namely, elements that live in $T$), it is integral over $R$, so $d\in R$ by normality.
			\item Lastly, we note that $T\subseteq\frac1{d^2}R$. Well, pick up some $b\in T$ and write
			\[b=\sum_{i=1}^na_ib_i\]
			with $a_i\in K$ using the fact that the $\{b_i\}$ are a basis of $L/K$. However, pushing this ``vector'' through $M$, the $i$th component comes out to
			\[\sum_{j=1}^ng_i(b_j)\cdot c_j=g_i\left(\sum_{j=1}^nc_jb_j\right)=g_i(b)\in T.\]
			In particular, letting $M^*$ denote the adjugate matrix of $M$, we see that pushing $M^*$ through the above will show that all coefficients live in $\frac 1dR\subseteq\frac1{d^2}R$ (because $M^*M=dI$). Thus, we get $c_i\in\frac1dR$, which is what we wanted.
		\end{itemize}
		The last claim shows that $T$ is a submodule of a finitely generated $R$-module, which shows that $T$ is finite over $R$, as needed.
	\end{proof}
	The above lemma finishes the last case of the proof.
\end{proof}
Here is an example application.
\begin{exe}
	Fix an algebraically closed field $k$ of characteristic $0$. Then consider the fraction field of $k\bb x$ as
	\[k\bb x\subseteq k((x)).\]
	However, $k((x))$ no longer needs to be algebraically closed, so let $L$ be some finite extension. Then, setting $T$ to be the integral closure of $R$ in $L$, we get that $T$ is finite over $R$ by \autoref{prop:geometricaklb}. In fact, we show $T\cong k\bb{x^{1/n}}$ for some $n$.
\end{exe}
\begin{proof}
	Note that $T$ is normal, local, and finitely generated over the ring $R$. Letting $\mf m$ be its maximal ideal, we can get $\mf m=(\pi)$ for some element $\pi$ by pushing a little. As such, $T$ we see that $T$ is a discrete valuation ring. Further, because $T$ is finite over $R$, we get some $n$ such that
	\[\pi^n\]
	can be written as lower terms. But this requires $\pi^n=ux$ for some unit $u$, and the unit $u$ has an $n$th power by Hensel's lemma (here is where we use $\op{char}k=0$), so we get that we can replace $\pi$ by $t^{1/n}$. This finishes.
\end{proof}
\begin{remark}
	This classifies all algebraic extensions of $k((x))$ as $k((x^{1/n}))$, so the algebraic closure of $k((x))$ is simply
	\[\bigcup_{n\ge1}k((x^{1/n})),\]
	which is pretty nice.
\end{remark}
\begin{example}
	Fix some $f\in\CC[x,y]$. Writing down $f\in\CC[x][y]$ as a polynomial in $y$, we may write
	\[f(x,y)=\sum_{i=1}^np_n(x)y^n.\]
	Taking the completion to $\CC\bb x[y]$ (i.e., looking locally at $x=0$), we hope that $f$ is still irreducible; otherwise, we can just take some irreducible factor. Now, we can directly solve for $y$ in this system as some series
	\[\sum_{i=-m}^\infty c_ix^{i/n}.\]
	Now, if $p_n$ is monic, then we can assert that $y$ lives in the integral closure of $R$, so $q$ is a bona fide power series.
\end{example}

\subsection{Invariant Theory}
Let's return to discussing invariant theory for the end of class; recall that, in an invariant theory, we are interested in studying the group invariants of some group action on an affine ring. We have the following result.
\begin{theorem}
	Fix a finite group $G$ and an affine ring $T$ with a $G$-action. Then $T^G$ is also an affine ring.
\end{theorem}
\begin{proof}
	In some sense, we are ``quotienting'' by the $G$-action and hoping that we recover an affine ring. Very quickly, note that $T$ is integral over $T^G$, which we get because $G$ is finite. Namely, for $a\in T$, we simply use the polynomial
	\[\prod_{g\in G}(x-ga),\]
	which is monic and in $T^G[x]$ because multiplying by some $g\in G$ will merely permute the terms of the product.

	We now note that, because $T$ is an affine ring, we suppose that the elements $\{y_1,\ldots,y_m\}$ generate $T$. Now, we let $S$ be the $k$-algebra generated by the elementary symmetric polynomials in the $g_iy_j$, of which there are still finitely many while maintaining $S\subseteq T^G$. Now, $T$ is a finitely generated $S$-module because the generators of $S$ are roots of the polynomials
	\[\prod_{g\in G}(x-gy_j)\in S[x],\]
	so $T$ is finite over $S$, meaning that $T^G$ is finite over $S$. It follows that $T^G$ is an affine ring.
\end{proof}
Our last topic is on elimination theory, which is what we will spend the next two lectures on. We will have a total of 13 homeworks.

\end{document}