\documentclass[../notes.tex]{subfiles}

\begin{document}

% !TEX root = ../notes.tex

Welcome back.

\subsection{Discrete Valuation Rings}
This subsection will focus on regular local rings $R$ of dimension $1$, which turn out to be ``discrete valuation rings.'' For example, \autoref{prop:regularimpliesdomain} tells us that such rings $R$ must be integral domains, so we have some inkling that such rings must be nice. Further, the maximal ideal $\mf m\subseteq R$ must be principal because $R$ is regular, so we set
\[\mf m=(\pi)\]
for some $\pi\in R$. This turns out to make $R$ very nice.
\begin{proposition} \label{prop:altdvr}
	Fix a Noetherian regular local ring $R$ of dimension $1$, with maximal ideal $\mf m\coloneqq(\pi)$. Then all nonzero ideals of $R$ are of the form $\left(\pi^k\right)$ for some natural number $k$. In particular, $R$ is a principal ideal domain.
\end{proposition}
\begin{proof}
	The key is to show that any $r\in R\setminus\{0\}$ can be written as $u\pi^n$ for some $n\in\NN$ and $u\in R^\times=R\setminus(\pi)$. For this, we use the Krull intersection theorem: by \autoref{cor:krullintersect}, we see that
	\[\bigcap_{n\ge0}\left(\pi^n\right)=\bigcap_{n\ge0}\left(\pi\right)^n=0.\]
	Thus, $r\ne0$ is not in the intersection of all the $\pi^n$ even though certainly $r\in\left(\pi^0\right)=R$, so there exists some least $n\ge1$ such that $r\notin\left(\pi^{n+1}\right)$. Notably, because $n$ is the least such, we have
	\[r\in\left(\pi^n\right)\setminus\left(\pi^{n+1}\right).\]
	As such, we can write $r=u\pi^n$, but $u\notin(\pi)$ because this would imply $r\in\left(\pi^{n+1}\right)$. This is what we wanted.

	We take a moment to note that representation as $u\pi^n$ is in fact unique: suppose $u\pi^a=v\pi^b$ for $u,v\notin(\pi)$ and $a,b\in\NN$. Without loss of generality, take $a\ge b$ so that $R$ being an integral domain (by \autoref{prop:regularimpliesdomain}) gives
	\[u\pi^{a-b}=v\notin(\pi).\]
	Thus, $a-b<1$, so $a=b$ follows. In this case, the above equation reads $a=b$.
	
	In total, we have written all elements $r\in R\setminus\{0\}$ uniquely in the form $u\pi^n$ for some $u\notin(\pi)$ and $n\in\NN$. As such, we may define the function $\nu\colon R\setminus\{0\}\to\NN$ by
	\[\nu(r)\coloneqq n.\]
	In particular, $r/\pi^{\nu(r)}=u\in R^\times$ implies that $(r)=\left(\pi^{\nu(r)}\right)$.
	
	We are now ready to attack the proof directly. Suppose that $I\subseteq R$ is a nonzero ideal. Then we set
	\[\nu\coloneqq\min\{\nu(a):a\in I\setminus\{0\}\},\]
	where the minimum is well-defined because $I\ne(0)$; in particular, this is some $x\in I$ with $\nu(x)=\nu$. Now, $a\ge b$ implies that $\left(\pi^a\right)\subseteq\left(\pi^b\right)$, so we write
	\[I=\bigcup_{r\in I\setminus\{0\}}(r)=\bigcup_{r\in I\setminus\{0\}}\left(\pi^{\nu(r)}\right)\subseteq\bigcup_{r\in I}\left(\pi^\nu\right)=\left(\pi^\nu\right)=(x)\subseteq I.\]
	Thus, equalities follow, and we get $I=(x)=\left(\pi^\nu\right)$, which finishes the proof.
\end{proof}
As mentioned above, these rings are actually called discrete valuation rings. Let's explain this.
\begin{definition}[Totally ordered group]
	A group $\Gamma$ endowed with a total order $\ge$ is a \textit{totally ordered group} if and only if $a\ge b$ for $a,b\in G$ implies
	\[ac\ge bc\qquad\text{and}\qquad ca\ge cb\]
	for any $c\in G$.
\end{definition}
\begin{example} \label{ex:zisordered}
	The group $(\ZZ,+)$ is an ordered group under the usual ordering $\ge$. The coherence check is that $a\ge b$ implies $a+c\ge b+c$.
\end{example}
\begin{example}
	Exactly analogously to \autoref{ex:zisordered}, $(\RR,+)$ is an ordered group under the usual ordering $\ge$; the coherence check is the same.
\end{example}
% As an aside, recall that a totally ordered set $(\Gamma,\ge)$ comes with an order topology with subbasis of open sets given by
% \[(a,\infty)\coloneqq\{g\in\Gamma:g>a\}\qquad\text{and}\qquad(-\infty,a)\coloneqq\{g\in\Gamma:a>g\}.\]
% Now, it turns out that the definition of a totally ordered group turns $\Gamma$ into a topological group.
% \begin{lemma}
% 	Fix a totally ordered group $\Gamma$ with a total order $\ge$, and endow the set $\Gamma$ with the order topology. Then $\Gamma$ is a topological group.
% \end{lemma}
% \begin{proof}
% 	Suppose that $\Gamma$ is a totally ordered group. Let $\mu\colon\Gamma^2\to\Gamma$ and $i\colon\Gamma\to\Gamma$ be the multiplication and inversion maps, respectively. We need to show that $\mu$ and $i$ are continuous.
	
% 	Quickly, observe that if any interval
% 	\[(a,b)=\{g\in\Gamma:b>g>a\}\]
% 	is a singleton $\{g\}$, then in fact $\Gamma$ has the discrete topology: for any $h\in\Gamma$, we have $h^{-1}bg>x>h^{-1}ag$ if and only if
% 	\[b>h^{-1}xg^{-1}>a\]
% 	if and only if $h^{-1}xg^{-1}=g$ if and only if $x=h$, so $\{h\}=\left(h^{-1}bg,h^{-1}ag\right)$ is open. In these cases, $\mu$ and $i$ are trivially continuous because all maps are continuous in the discrete topology.

% 	We now suppose that all nonempty intervals $(a,b)$ contain more than one element. We now run our checks.
% 	\begin{itemize}
% 		\item To show $\mu$ is continuous, it suffices to show that $\mu^{-1}((-\infty,a))$ and $\mu^{-1}((a,\infty))$ are open; we will show that $\mu^{-1}((a,\infty))$ is open, and the other is analogous.

% 		Well, suppose $(g,h)\in\mu^{-1}((a,\infty))$; we will put an open neighborhood around the point $(g,h)\in\Gamma^2$ inside $\mu^{-1}((a,\infty))$. To start, we know that $gh>a$. We have two cases.
% 		\begin{itemize}
% 			\item If there are no elements strictly between $gh$ and $a$, then we claim that
% 			\[U\coloneqq\left(ah^{-1},\infty\right)\times\left(g^{-1}a,\infty\right)\]
% 			is the open set we are looking for. Surely $gh>a$ implies $g>ah^{-1}$ and $h>g^{-1}a$, so the point $(g,h)$ lives in $U$. Further, $U$ is open as the product of sub-basis elements.
			
% 			It remains to show $U\subseteq\mu^{-1}((a,\infty))$. Well, given a point $(x,y)\in U$, then $x>ah^{-1}$ and $y>g^{-1}a$. If $g>x$, we see $gh>xh>a$, which contradicts our assumption. Similarly, if $h>y$, we see $gh>gy>a$, again a contradiction. So instead $x\ge g$ and $y\ge h$, so it follows that $xy\ge gh>a$. Thus, $U\subseteq\mu^{-1}((a,\infty))$ follows.

% 			\item Otherwise, there exists some element $z$ strictly between $gh$ and $a$. We claim that
% 			\[U\coloneqq\left(zh^{-1},\infty\right)\times\left(hz^{-1}a,\infty\right)\]
% 			is the open set we are looking for. Again, $gh>z$ implies $g>zh^{-1}$ and $h>hz^{-1}a$ because $z>a$, so $(g,h)\in U$; and $U$ is open because it is the product of subbasis elements.

% 			It remains to show $U\subseteq\mu^{-1}((a,\infty))$. Well, given a point $(x,y)\in U$, we see $x>zh^{-1}$ and $y>hz^{-1}a$ implies
% 			\[xy>zh^{-1}y>zh^{-1}hz^{-1}a=a,\]
% 			giving $\mu((x,y))\in(a,\infty)$, finishing.
% 		\end{itemize}

% 		\item To show $i$ is continuous, it suffices to show that $i^{-1}((-\infty,a))$ and $i^{-1}((a,\infty))$ are open for any $a$; we will show that $i^{-1}((a,\infty))$ is open, and the other is analogous.
		
% 		Well, $g\in i^{-1}((a,\infty))$ if and only if $g^{-1}>a$ if and only if $a^{-1}>g^{-1}$ if and only if $g\in\left(a^{-1},-\infty\right)$. Thus, $i^{-1}((a,\infty))=\left(-\infty,a^{-1}\right)$ is open.
% 	\end{itemize}
% 	The above checks complete the proof.
% \end{proof}
% \begin{remark}[Nir]
% 	Recall that a totally ordered set $(\Gamma,\ge)$ comes with an order topology with basis of open sets given by
% 	\[(a,b)\coloneqq\{g\in\Gamma:b>g>a\}.\]
% 	The definition of being a totally ordered group essentially turns $\Gamma$ into a topological group under this topology. Here are our checks; let $\mu\colon\Gamma^2\to\Gamma$ be multiplication and $i\colon\Gamma\to\Gamma$ be inversion.
% 	\begin{itemize}
% 		\item To show $\mu$ is continuous, it suffices to show $\mu^{-1}((a,b))$ is open for any $b>a$. Well, given $(g,h)\in\mu^{-1}((a,b))$, we have $b>gh>a$.
% 		\item To show $i$ is continuous, it suffices to show that $i^{-1}((a,b))$ is open for any $b>a$. Well, $g\in i^{-1}((a,b))$ if and only if $g^{-1}\in (a,b)$ if and only if $b>g^{-1}$ and $g^{-1}>a$ if and only if $a^{-1}>g>b^{-1}$. Thus, $i^{-1}((a,b))=\left(b^{-1},a^{-1}\right)$ is open.
% 	\end{itemize}
% \end{remark}
The reason we have defined totally ordered groups is to be able to define valuations.
\begin{definition}[Valuation]
	Fix a domain $R$ and totally ordered group $\Gamma$. A \textit{valuation} $\nu$ is a group homomorphism $\nu\colon K(R)^\times\to\Gamma$ satisfying the following.
	\begin{itemize}
		\item We have $\nu(a+b)\ge\min\{\nu(a),\nu(b)\}$.
		\item We have $R=\{x\in K(R)^\times:\nu(x)\ge0\}\cup\{0\}$.
	\end{itemize}
\end{definition}
When the codomain of our valuation is $\ZZ$, these rings have a special name.
\begin{definition}[Discrete valuation ring]
	A \textit{discrete valuation ring} is an integral domain $R$ equipped with a valuation $\nu\colon K(R)^\times\to\ZZ$.
\end{definition}
\begin{example} \label{ex:fieldisdvr}
	Suppose that $R$ is a field. Then we claim $R$ is a discrete valuation ring. Indeed, set $\nu\colon K(R)^\times\to\ZZ$ by $\nu(x)=0$ always. We have the following checks.
	\begin{itemize}
		\item Homomorphism: note $\nu(a)+\nu(b)=0=\nu(a+b)$ for any $a,b\in K(R)^\times$.
		\item Note $\nu(a+b)=0\ge0=\min\{\nu(a),\nu(b)\}$ for any $a,b\in K(R)^\times$.
		\item Lastly, note $\nu(x)=0\ge0$ for all $x\in K(R)^\times$, so
		\[\{x\in K(R)^\times:\nu(x)\ge0\}\cup\{0\}=K(R)^\times\cup\{0\}=K(R)=R,\]
		where the last equality is because $R$ is a field.
	\end{itemize}
\end{example}
We are now ready for our main result.
\begin{proposition} \label{prop:dvrgrabbag}
	Fix a Noetherian ring $R$. The following are equivalent.
	\begin{itemize}
		\item $R$ is a discrete valuation ring.
		\item $R$ is a field or regular local ring of dimension $1$.
	\end{itemize}
\end{proposition}
\begin{proof}
	We show the directions independently.
	\begin{itemize}
		\item In one direction, suppose $R$ is a regular local ring of dimension $1$ or a field. If $R$ is a field, then $R$ is a discrete valuation ring by \autoref{ex:fieldisdvr}.

		Otherwise, take $R$ to be a regular local ring of dimension $1$; as such, suppose $(\pi)$ is maximal. Then most of the heavy lifting is done by \autoref{prop:altdvr}, in which we defined a function $\nu\colon R\setminus\{0\}\to\NN$ such that $\nu(r)$ is the unique nonnegative integer such that
		\[r/\pi^{\nu(r)}\in R^\times.\]
		Quickly, we check $\nu$ is additive: taking $r,s\in R$, we see that $r/\pi^{\nu(r)},s/\pi^{\nu(s)}\in R^\times$ so that
		\[\frac{rs}{\pi^{\nu(r)+\nu(s)}}\in R^\times\]
		so that uniqueness of $\nu$ forces $\nu(r+s)=\nu(r)-\nu(s)$. Now, to extend this to all of $K(R)^\times$, we simply define
		\[\nu(r/s)\coloneqq\nu(r)-\nu(s)\]
		for $r/s\in K(R)^\times$ (note $r,s\ne0$). Observe that each $r\in R$ has $\nu(r/1)=\nu(r)-\nu(1)=\nu(r)$, so $\nu$ does indeed extend $\nu$.
		
		Now, to see that $\nu$ is well-defined, note $\frac rs=\frac{r'}{s'}$ implies $rs'=r's$ because $R$ is an integral domain, so
		\[\nu(r)-\nu(s)=\nu(r')-\nu(s').\]
		As such, we have a well-defined function $\nu\colon K(R)^\times\to\ZZ$. To check that this is a homomorphism, set $\frac ab,\frac cd\in K(R)^\times$. Then, because we already showed that $\nu$ is additive on $R$, we see
		\[\nu\left(\frac ab\cdot\frac cd\right)=\nu\left(\frac{ac}{bd}\right)=\nu(ac)-\nu(bd)=\big(\nu(a)-\nu(b)\big)+\big(\nu(c)-\nu(d)\big)=\nu\left(\frac ab\right)+\nu\left(\frac cd\right).\]
		So we do have a homomorphism $\nu\colon K(R)^\times\to\ZZ$.
		
		Before continuing, observe that some $\frac ab\in K(R)^\times$ will have $a=u\pi^{\nu(a)}$ and $b=v\pi^{\nu(b)}$ for $u,v\in R^\times$, so
		\[\frac ab=\frac{u\pi^{\nu(a)}}{v\pi^{\nu(b)}}=uv^{-1}\pi^{\nu(a)-\nu(b)}=uv^{-1}\cdot\pi^{\nu(a/b)}\in R^\times.\]
		In particular, $\nu(a/b)$ is the correct power in $\ZZ$ to divide out from $\frac ab$ by to get into $R^\times$. As usual, this is unique: if $u\pi^m=v\pi^n$ for $u,v\in R^\times$ and $m\ge n$ (without loss of generality), then $\pi^{m-n}=u^{-1}v\in R^\times$, so $0\le m-n<1$, so $m=n$.

		We now have the following remaining checks.
		\begin{itemize}
			\item Given $\frac ab,\frac cd\in K(R)^\times$, and suppose that $\nu(c/d)\ge\nu(a/b)$ without loss of generality. Then
			\[\frac{a/b}{\pi^{\nu(a/b)}}\in R\qquad\text{and}\qquad\frac{c/d}{\pi^{\nu(a/b)}}=\pi^{\nu(c/d)-\nu(a/b)}\cdot\frac{c/d}{\pi^{\nu(c/d)}}\in R,\]
			so it follows
			\[\frac{a/b+c/d}{\pi^{\nu(a/b)}}\in R,\]
			so $\nu(a/b+c/d)\ge\nu(a/b)$.
			\[\nu\left(\frac ab+\frac cd\right)=\nu\left(\frac{ad+bc}{bd}\right)=\nu(ad+bc)-\nu(bd).\]
			\item Certainly $r\in R\setminus\{0\}$ has $\nu(r)\ge0$ by definition of $\nu$, so
			\[R\subseteq\{x\in K(R)^\times:\nu(x)\ge0\}\cup\{0\}.\]
			Conversely, of course $0\in R$, and any $\frac rs\in K(R)^\times$ with $\nu(r/s)=\nu(r)-\nu(s)\ge0$ has $r=u\pi^{\nu(r)}$ and $s=v\pi^{\nu(s)}$ for some $u,v\in R^\times$. As such,
			\[\frac rs=\frac{u\pi^{\nu(r)}}{v\pi^{\nu(s)}}=\frac{uv^{-1}\pi^{\nu(r)-\nu(s)}}1\in R,\]
			so we also get
			\[\{x\in K(R)^\times:\nu(x)\ge0\}\cup\{0\}\subseteq R.\]
		\end{itemize}
		The above checks verify that $R$ is in fact a discrete valuation ring with valuation $\nu$.

		\item Suppose $R$ is a discrete valuation ring, so pick up our valuation $\nu\colon K(R)^\times\to\ZZ$. If $\im\nu=\{0\}$, then
		\[R=\{x\in K(R)^\times:\nu(x)\ge0\}\cup\{0\}=K(R),\]
		so $R$ is a field.

		Otherwise, suppose $R$ is not a field and that $\im\nu\subseteq\ZZ$ has a nonzero element and hence a positive element by signing, so suppose $d$ is the least such positive element. As such, find any
		\[\pi\in\nu^{-1}(\{d\}).\]
		We now proceed with the following steps.
		\begin{itemize}
			\item We claim $R^\times=\nu^{-1}(\{0\})$. In one direction, if $r\in R^\times$, find $s\in R$ with $rs=1$ so that certainly $r,s\ne0$ and
			\[\nu(r)+\nu(s)=\nu(rs)=\nu(1)=0,\]
			so $\nu(s)\ge0$ (from $s\in R$) implies $\nu(r)\le0$. But still $\nu(r)\ge0$, so we conclude $\nu(r)=0$.

			Conversely, if $x\in K(R)^\times$ has $\nu(x)=0$, then $\nu\left(x^{-1}\right)=-\nu(x)=0\ge0$ as well, so we see that both $x$ and $x^{-1}$ live in $R$, so $x\cdot x^{-1}=1$ witnesses that $x$ is a unit in $R$.
			
			\item We claim that
			\[(\pi)=\{0\}\cup\{x\in K(R)^\times:\nu(x)>0\}.\]
			Certainly any $\pi r\in(\pi)$ has $r=0$ or $\nu(\pi r)=\nu(\pi)+\nu(r)\ge d+0>0$. Conversely, if $x\in K(R)^\times$ has $\nu(x)>0$, then of course $x\in R$. Also, $\nu(\pi)=d$ is as small as possible, so $\nu(x)\ge\nu(\pi)$, so $\nu(x/\pi)\ge0$ gives $x/\pi\in R$ as well. Thus,
			\[x=(x/\pi)\cdot\pi\]
			witnesses $x\in(\pi)$.
		\end{itemize}
		The above two claims show that
		\begin{align*}
			R &= \{0\}\cup\{x\in K(R)^\times:\nu(x)\ge0\} \\
			&= \left(\{0\}\cup\{x\in K(R)^\times:\nu(x)>0\}\right)\sqcup\{x\in K(R)^\times:\nu(x)=0\} \\
			&=(\pi)\sqcup R^\times,
		\end{align*}
		so $R$ is local with unique maximal ideal $(\pi)$. Further, note that $\dim R=\codim(\pi)\le1$ by \autoref{thm:pit1}, but $\dim R>0$ because of the chain of primes $(0)\subsetneq(\pi)$. Thus, $R$ is a regular local ring with dimension $1$, regular because its maximal ideal is principal.
	\end{itemize}
	The above directions complete the proof.
\end{proof}
As such, here is another example.
\begin{example}
	The ring $\ZZ_p=\widehat{\ZZ}_{(p)}$ is a discrete valuation ring. Indeed, $\ZZ_p$ is local with maximal ideal $\widehat{(p)}$ by \autoref{lem:completeislocal}, and $\widehat{(p)}=p\ZZ_p$ by \autoref{prop:getbetterinducedfiltration}. Further, $\dim\ZZ_p=1$ because $\dim\ZZ_p\ge1$ because of the chain of primes $(0)\subsetneq p\ZZ_p$ while
	\[\dim\ZZ_p=\codim p\ZZ_p\le1\]
	by \autoref{thm:pit1}. Thus, $\ZZ_p$ is a regular local ring with dimension $1$, so we finish by \autoref{prop:dvrgrabbag}.
\end{example}
\begin{remark}
	Tracking \autoref{prop:dvrgrabbag} through, we see that our valuation $\nu\colon\QQ_p\to\ZZ$ takes $q\in\QQ_p$ to the value $n$ such that $q=u\pi^n$ with $u\in\ZZ_p^\times$. In fact, we can see that the function
	\[d(a,b):=p^{-\nu(a-b)}\]
	is exactly the metric defined in \autoref{lem:krullmetric}, so $\ZZ_p$ is a metric space defined by this valuation, actually complete by \autoref{prop:unitecompletion}.
\end{remark}

\subsection{Normal Domains}
Just for fun, let's provide a criterion to have a normal domain. To set up our discussion, recall that unique factorization domains are normal by \autoref{prop:ufdnormal}. As such, we will imitate the following result.
\equivufdprop*
\noindent The idea is to weaken this condition to give us normality. In particular, recall that a prime $\mf p$ is associated to the ideal $I$ if and only if $\mf p\in\op{Ass}R/I$ if and only if there exists $x\in R$ such that
\[\mf p=\op{Ann}_R[x]_I=\{r:rx\in I\}.\]
Notably, we this would imply that $x\notin I$ and hence $[x]_I\ne[0]_I$.

Anyway, here is our statement.
\begin{theorem}
	Fix a Noetherian domain $R$. Then $R$ is normal if and only if either of the following conditions hold.
	\begin{listalph}
		\item For each prime $\mf p$ associated to a principal ideal, the ideal $\mf pR_\mf p\subseteq R_\mf p$ is a principal ideal.
		\item For each codimension-$1$ prime $\mf p$, the localization $R_\mf p$ is a discrete valuation ring. Further, each nonzero prime $\mf p$ associated to a principal ideal has codimension $1$.
	\end{listalph}
\end{theorem}
\begin{proof}
	We start by showing that each of the conditions (a) and (b) are equivalent.
	\begin{itemize}
		\item We show (a) implies (b). There are two sentences to check.
		\begin{itemize}
			\item Given a codimension-$1$ prime $\mf p$, \autoref{cor:pitconverse} tells us that $\mf p$ is minimal over some principal ideal $(a)$. But $(a)=\op{Ann}R/(a)$, so \autoref{prop:minassprimes} shows that $\mf p$ is associated to the $R$-module $R/(a)$, which means that $\mf p$ is associated to the principal ideal $(a)$.

			As such, by (a), we see that $\mf pR_\mf p$ is a principal ideal. This implies that $R_\mf p$ is a regular local ring of dimension $1$: note $R_\mf p$ is local with maximal ideal $\mf pR_\mf p$, and we have $\dim R_\mf p=\codim\mf p=1$ by construction of $\mf p$. Thus, $\mf pR_\mf p$ being principal verifies that $R_\mf p$ is regular. It follows $R_\mf p$ is a discrete valuation ring by \autoref{prop:dvrgrabbag}.

			\item Pick up a nonzero prime $\mf p$ associated to a principal ideal. By (a), we see $\mf pR_\mf p\subseteq R_\mf p$ is a principal ideal and hence minimal over a principal ideal (namely, itself), so
			\[\codim\mf pR_\mf p\le1\]
			by \autoref{thm:pit1}. But certainly $\mf p\supsetneq(0)$ in $R$ because $\mf p$ is nonzero, and $(0)$ is prime in the integral domain $R$, so $\codim\mf p\ge1$ by \autoref{lem:codimisdescend}.
			
			Thus, \autoref{lem:dimiscodim} grants the inequalities
			\[1\le\codim\mf p=\dim R_\mf p=\codim\mf pR_\mf p\le1\]
			because $R_\mf p$ is local with maximal ideal $\mf pR_\mf p$. So indeed, $\codim\mf p=1$, which is what we wanted.
		\end{itemize}

		\item We show (b) implies (a). Fix a prime $\mf p$ associated to a principal ideal, and we check that $\mf pR_\mf p$ is principal. If $\mf p=(0)$, then $\mf pR_\mf p=(0)$ is principal.

		Otherwise, $\mf p$ is nonzero, so the second sentence of (b) tells us that $\mf p$ has codimension $1$, from which the first sentence tells us that $R_\mf p$ is a discrete valuation ring. But then $R_\mf p$ is a regular local ring of dimension $1$ by \autoref{prop:dvrgrabbag}, so the maximal ideal $\mf pR_\mf p$ must be principal, which is what we wanted.
	\end{itemize}
	We now show that (a) is equivalent to $R$ being normal. We start with the backwards direction: suppose (a) holds, and we'll show $R$ is normal. To start, we pick up the following lemma.
	\begin{lemma}
		Fix a Noetherian domain $R$. Given $x\in K(R)$, then $x\in R$ if and only if $x\in R_\mf p$ for all primes $\mf p$ associated to a principal ideal. In other words,
		\[R=\bigcap_\mf pR_\mf p,\]
		where $\mf p$ varies over primes associated to a principal ideal.
	\end{lemma}
	\begin{proof}
		Of course, $x\in R$ implies that $x=x/1\in R_\mf p$ for each prime $\mf p$ and therefore for each prime $\mf p$ associated to a principal ideal.

		Conversely, suppose $x\notin R$ has $x=\frac ab$. Then we are given $a\notin(b)$. Now, we showed a while ago that, in an $R$-module $M$, we have $m=0$ if and only if $\frac m1=\frac01$ in $R_\mf p$ for all $\mf p\in\op{Ass}M$. As such, working with $M:=R/(x)$, we see that $[a]_{(x)}\ne[0]_{(x)}$, so there exists a prime $\mf p$ associated to $R/(b)$ (i.e., associated to the ideal $(b)$) with $a\notin(b)_\mf p$, so $x\notin R_\mf p$.
	\end{proof}
	Thus, the hypothesis tells us that each $R_\mf p$ is a discrete valuation ring and hence a principal ideal domain and hence a unique factorization domain and hence normal.\footnote{One can show this somewhat more directly by building a monic polynomial with some $u\pi^m$ as a root and then arguing about the maximal ideal, but we won't bother.} Thus, because the intersection of normal domains is normal, we deduce that $R$ is normal.

	We now show the forwards direction. Suppose that $R$ is normal, and let $\mf p$ be some prime associated to a principal ideal $(a)$. We would like to show that $\mf pR_\mf p$ is principal; because $\mf pR_\mf p$ is still associated to $(a)R_\mf p$, we see that we may replace $R$ and $\mf p$ with $R_\mf p$ and $\mf pR_\mf p$ so that $R$ is local with maximal ideal $\mf p$.

	To continue, we pick up the following definition.
	\begin{definition}
		A \textit{fractional ideal} is an $R$-submodule of $K(R)$.
	\end{definition}
	As such, we set
	\[\mf p^{-1}:=\{x\in K(R):x\mf p\subseteq R\}.\]
	Our goal is to show that $\mf p\mf p^{-1}=R$. Certainly $\mf p^{-1}\mf p\subseteq R$ by definition, and we can see that $\mf p^{-1}\supseteq R$ implies $\mf p\mf p^{-1}\subseteq R$. Now, $\mf p\mf p^{-1}$ can be checked to be an $R$-ideal, so because $\mf p$ is currently maximal, $\mf p\mf p^{-1}\in\{\mf p,R\}$.

	Now, suppose for the sake of contradiction that $\mf p^{-1}\mf p=\mf p$. Well, any $x\in\mf p^{-1}$ is integral by the Cayley--Hamilton theorem, so $x\in R$, so we have shown $\mf p^{-1}\subseteq R$. But this does not make sense: $\mf p$ is associated to $(a)$ by some element $[b]_{(a)}$, but then $b\mf p\subseteq(a)$, so $a^{-1}b\mf p\subseteq R$, so $a^{-1}b\in R\setminus\mf p$.

	But now $\mf p^{-1}\mf p=R$ shows that there exists $\frac ab$ such that $\frac xy\mf p=R$ for some unit $x$, so $\frac1y\mf p=R$, so $\mf p=(y)$. This finishes the proof.
\end{proof}
\begin{remark}
	It is possible for $\mf p$ in the proof to not be principal but still have $\mf pR_\mf o$ be principal.
\end{remark}
As a corollary of the proof, we get the following results.
\begin{corollary}
	Fix a Noetherian domain $R$. If $R$ is normal, then
	\[R=\bigcap_{\mf p}R_\mf p,\]
	where the intersection is over all primes $\mf p$ of codimension $1$.
\end{corollary}
\begin{corollary}
	Fix $X$ an affine algebraic variety such that $A(X)$ is a normal domain. If we have a subvariety $Y\subseteq X$ is such that $A(Y)$ is of codimension at least $2$, then $A(X-Y)=A(X)$.
\end{corollary}
\begin{proof}
	Suppose that $\mf q$ is the prime ideal corresponding to the variety $Y$. Then $A(X-Y)=A(X)_\mf q$, so taking the intersection finishes. % \todo{what}
\end{proof}

\subsection{Invertible Modules}
For the following discussion, we will take $R$ to be a Noetherian domain, for intuition. We have the following definition.
\begin{definition}[Invertible module]
	An $R$-module $M$ is \textit{invertible} if and only if all prime ideals $\mf p\subseteq R$ has $M_\mf p\cong R_\mf p$.
\end{definition}
It turns out that these are all fractional ideals in the case where $R$ is a Noetherian domain. Before that, here are some examples.
\begin{example}
	A principal ideal $(f)\subseteq R$ is invertible.
\end{example}
\begin{example}
	If $M$ and $N$ are invertible $R$-modules, then any prime $\mf p$ will have
	\[(M\otimes_RN)_\mf p\cong M_\mf p\otimes_{R_\mf p}N_\mf p\cong R_\mf p\otimes_{R_\mf p}R_\mf p\cong R_\mf p,\]
	so $M\otimes_RN$ is also invertible.
\end{example}
\begin{example}
	If $M$ is an invertible, finitely generated $R$-module, then $M^*:=\op{Hom}_R(M,R)$ is also invertible. In particular, because $R$ is Noetherian, $M$ is finitely presented, so
	\[R_\mf p\cong\op{Hom}_{R_\mf p}(R_\mf p,R_\mf p)\cong\op{Hom}_{R_\mf p}(M_\mf p,R_\mf p)\cong\op{Hom}_R(M,R)_\mf p.\]
\end{example}
To start our discussion, here is a lemma.
\begin{lemma}
	Fix a Noetherian domain $R$. An $R$-module $M$ is invertible if and only if the map
	\[\mu:M^*\otimes_RM\to R\]
	by $\varphi\otimes m\mapsto\varphi(m)$ is an isomorphism.
\end{lemma}
\begin{proof}
	It suffices to work with the case that $\mu_\mf p$ is an isomorphism for all primes $\mf p$. By running through the isomorphisms in the examples, we see that we are asking for
	\[\mu_\mf p:(M_\mf p)^*\otimes_{R_\mf p}M_\mf p\to R_\mf p\]
	is an isomorphism for all primes $\mf p$.

	In particular, we are allowed to assume that $R$ is local with maximal ideal $\mf p$. In one direction, suppose that $\mu$ is an isomorphism. By surjectivity, we are promised some
	\[\mu\Bigg(\sum_{i=1}^n\varphi_i\otimes a_i\Bigg)=1.\]
	In particular, there exists $i$ such that $\varphi_i(a_i)\notin\mf p$, but $R\setminus\mf p$ are all units, so we can force $\varphi(a)=1$ for some $\varphi$ and $a$. Now, living in a local ring thus forces by $\varphi$ to show that
	\[M\cong R\oplus\ker\varphi,\]
	but $\ker\varphi$ is trivial because any kernel would have to show up in the kernel of $\mu$, which is trivial by hypothesis.

	We don't discuss the other direction.
\end{proof}
\begin{remark}
	We can see that $M$ will be generated by the elements $a_i$ in the summation
	\[\mu\Bigg(\sum_{i=1}^n\varphi_i\otimes a_i\Bigg)=1.\]
	Thus, $M$ should be finitely generated.
\end{remark}
This discussion gives us the following definition.
\begin{definition}[Picard group]
	Fix a Noetherian domain $R$. Then $\op{Pic}R$ is the group of isomorphism classes of invertible $R$-modules.
\end{definition}
\begin{remark}
	The Picard group loosely corresponds to line bundles.
\end{remark}
To be explicit, the group operation of $\op{Pic}R$ is by
\[[X]\cdot[Y]:=[X\otimes_RY],\]
our identity element is $[R]$, and the inverses are $[X]^{-1}:=\left[X^*\right]$.

\subsection{The Class Group}
To close out class, we discuss the connection to fractional ideals.
\begin{lemma}
	Fix a Noetherian domain $R$. Then $M$ is invertible if and only if $M$ is isomorphic to some nonzero fractional ideal.
\end{lemma}
\begin{proof}
	The idea is to embed $M$ into $K(R)$ to extract our fractional ideal. Well, the embedding $R\to K(R)$ gives us an embedding
	\[M\to K(R)\otimes_RM.\]
	But now, $K(R)\otimes_RM\cong K(R)$ because $K(R)\otimes_RM$ is an invertible module over $K(R)$, which must be isomorphic to $K(R)$ because $K(R)$ only has the localization at the prime $(0)$ (which does nothing).

	As such, we have placed $M$ as an $R$-submodule of $K(R)$ and hence is isomorphic to a nonzero fractional ideal.
\end{proof}
As such, we can give an alternate characterization of the Picard group.
\begin{lemma}
	Fix a Noetherian domain $R$. If $I$ and $J$ are fractional ideals, then
	\[IJ\cong I\otimes_RJ\qquad\text{and}\qquad I^{-1}J\cong\op{Hom}(I.J).\]
\end{lemma}
\begin{proof}
	The isomorphism $I\otimes_RJ\cong IJ$ is by $a\otimes b\mapsto ab$. That $I^{-1}J\cong\op{Hom}_R(I,J)$ follows from carefully considering the localizations.
\end{proof}
Thus, modding out by principal ideals from the fractional ideals gives us the Picard group back again.

\end{document}