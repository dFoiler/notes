% !TEX root = ../notes.tex

Welcome back.

\subsection{Regular Rings}
Today we are mostly talking about regular local rings of dimension $1$. Concretely, we have a ring $R$ with a unique maximal ideal $\mf m$ (by being local), and because $R$ is regular, we have
\[\mf m=(\pi)\]
for some $\pi\in R$. As we showed last time, all regular local rings are domains, but in fact we now claim that regular local rings of dimension $1$ are principal ideal domains. In fact, we have stronger.
\begin{proposition} \label{prop:altdvr}
	Fix a Noetherian, regular, local ring $R$ of dimension $1$, with maximal ideal $(\pi)$. Then all nonzero ideals of $R$ are of the form $\left(\pi^k\right)$ for some natural number $k$.
\end{proposition}
\begin{proof}
	Set $K(R)$ to be the quotient field of $R$. We show that any nonzero element of the quotient field $r\in K(R)^\times$ can be written as $u\pi^n$ for some $u\in R\setminus(\pi)$ and $n\in\ZZ$. This will be enough to finish the proof because we more or less have a very good unique prime factorization.

	We start by taking $r\in K(R)$. By the Krull intersection theorem, we see that
	\[\bigcap_{n\ge1}(\pi)^n=(0),\]
	so because $r\ne0$, we deduce that $r$ must live in $\left(\pi^n\right)\setminus\left(pi^{n+1}\right)$ for some $n$. Thus, setting $r=u\pi^n$, we see that $u\in R\setminus(\pi)$, so $u$ is a unit, as needed.

	For general elements $\frac rs\in K(R)$, we simply take the quotient of the representations of $r$ and $s$ to finish. This finishes the proof.
\end{proof}
These rings are actually called discrete valuation rings. Let's explain this terminology.
\begin{definition}
	A group $\Gamma$ endowed with a total order $\ge$ is a \textit{totally ordered group} if and only if the set
	\[\{\gamma:\gamma\ge0\}\]
	is closed under the operation of $\Gamma$, and $\gamma_1\ge\gamma_2$ is implied by $\gamma_1\gamma_2^{-1}\ge0$.
\end{definition}
\begin{example}
	The rings $\ZZ$ and $\RR$ are ordered groups.
\end{example}
It happens that the discrete, countable totally ordered groups are all $\ZZ$.
\begin{definition}[Valuation]
	Fix a domain $R$ and totally ordered group $\Gamma$. A \textit{valuation} $\nu$ is a group homomorphism $\nu:K(R)^\times\to\Gamma$ satisfying the following.
	\begin{itemize}
		\item We have $\nu(a+b)\ge\min\{\nu(a),\nu(b)\}$.
		\item We have $R=\{x\in K(R):\nu(x)\ge0\}$.
	\end{itemize}
\end{definition}
\begin{definition}[Discrete valuation ring]
	A \textit{valuation ring} is an integral domain $R$ equipped with a valuation $\nu:K(R)^\times\to\ZZ$.
\end{definition}
And here are our examples.
\begin{example}
	Fix a regular, local ring $R$ of dimension $1$. Then we described in the proof of \autoref{prop:altdvr} a way to write elements $x\in K(R)^\times$ in the form $u\pi^n$ where $n\in\ZZ$. Defining $\nu(x):=n$ equips $R$ with a discrete valuation.
\end{example}
\begin{example}
	The ring $\ZZ_p$ is a discrete valuation ring. Indeed, $K(\ZZ_p)=\QQ_p$, and in the same way above we can write any $x\in\ZZ_p$ as $up^n$ for a unit $u$ and integer $n$. Setting $\nu(x):=n$ provides our valuation. In fact, the function
	\[d(a,b):=p^{-\nu(a-b)}\]
	turns $\QQ_p$ into a metric space, in fact complete with respect to this metric $d$.
\end{example}
\begin{remark}[Nir]
	It also turns out that discrete valuation rings are regular, local rings of dimension $1$ as well.
\end{remark}

\subsection{Normal Domains}
Just for fun, let's provide a criterion to have a normal domain. To start, recall that unique factorization domains are normal already. As such, we recall the following statement.
\begin{prop}
	A Noetherian domain $R$ is a unique factorization domain if and only if every prime $\mf p$ minimal over some principal ideal $(a)$ is itself principal.
\end{prop}
The idea is to weaken this condition to give us normality. In particular, recall that a prime $\mf p$ is associated to the ideal $I$ if and only if $\mf p\in\op{Ass}R/I$ if and only if there exists $x\in R$ such that
\[\mf p=\{r:rx\in I\}.\]
Notably, we this would imply that $x\notin I$ and hence $[x]_I\ne[0]_I$.

Anyway, here is our statement.
\begin{theorem}
	Fix a Noetherian domain $R$. Then $R$ is normal if and only if either of the following conditions hold.
	\begin{listalph}
		\item All primes $\mf p$ associated to a principal ideal $(a)\subseteq R$ have $\mf pR_\mf p\subseteq R_\mf p$ is a principal ideal.
		\item Every localization of $R$ at a codimension-$1$ prime $\mf p$ is a regular local ring.
	\end{listalph}
\end{theorem}
\begin{proof}
	The condition (a) is equivalent to $R_\mf p$ being a discrete valuation ring: note that we already know that $R_\mf p$ is local with maximal ideal $\mf pR_\mf p$. Now, because codimension-$1$ primes are the ones minimal over principal rings, we see that (a) and (b) are equivalent.

	We now show the backwards direction. We pick up the following lemma.
	\begin{lemma}
		Fix a Noetherian domain $R$. Then $x\in K(R)$ has $x\in R$ if and only if $x\in R_\mf p$ for all primes $\mf p$ associated to some principal ideal. In other words,
		\[R=\bigcap_\mf pR_\mf p,\]
		where $\mf p$ varies over primes associated to a principal ideal.
	\end{lemma}
	\begin{proof}
		Of course, $x\in R$ implies that $x\in R_\mf p$ for each $\mf p$.

		In the reverse direction, suppose $x\notin R$ has $x=\frac ab$. Then we are given $a\notin(b)$. Now, we showed a while ago that, in an $R$-module $M$, we have $m=0$ if and only if $\frac m1=\frac01$ in $R_\mf p$ for all $\mf p\in\op{Ass}M$. As such, working with $M:=R/(x)$, we see that $[a]_{(x)}\ne[0]_{(x)}$, so there exists a prime $\mf p$ associated to $R/(b)$ (i.e., associated to the ideal $(b)$) with $a\notin(b)_\mf p$, so $x\notin R_\mf p$.
	\end{proof}
	Thus, the hypothesis tells us that each $R_\mf p$ is a discrete valuation ring and hence a principal ideal domain and hence a unique factorization domain and hence normal.\footnote{One can show this somewhat more directly by building a monic polynomial with some $u\pi^m$ as a root and then arguing about the maximal ideal, but we won't bother.} Thus, because the intersection of normal domains is normal, we deduce that $R$ is normal.

	We now show the forwards direction. Suppose that $R$ is normal, and let $\mf p$ be some prime associated to a principal ideal $(a)$. We would like to show that $\mf pR_\mf p$ is principal; because $\mf pR_\mf p$ is still associated to $(a)R_\mf p$, we see that we may replace $R$ and $\mf p$ with $R_\mf p$ and $\mf pR_\mf p$ so that $R$ is local with maximal ideal $\mf p$.

	To continue, we pick up the following definition.
	\begin{definition}
		A \textit{fractional ideal} is an $R$-submodule of $K(R)$.
	\end{definition}
	As such, we set
	\[\mf p^{-1}:=\{x\in K(R):x\mf p\subseteq R\}.\]
	Notably, $\mf p\mf p^{-1}$ will be contained in $R$, and it contains $\mf p$. So it is either $\mf p$ or $R$.

	Suppose for the sake of contradiction that $\mf p^{-1}\mf p=\mf p$. Well, any $x\in\mf p^{-1}$ is integral by the Cayley--Hamilton theorem, so $x\in R$, so we have shown $\mf p^{-1}\subseteq R$. But this does not make sense: $\mf p$ is associated to $(a)$ by some element $[b]_{(a)}$, but then $b\mf p\subseteq(a)$, so $a^{-1}b\mf p\subseteq R$, so $a^{-1}b\in R\setminus\mf p$.

	But now $\mf p^{-1}\mf p=R$ shows that there exists $\frac ab$ such that $\frac xy\mf p=R$ for some unit $x$, so $\frac1y\mf p=R$, so $\mf p=(y)$. This finishes the proof.
\end{proof}
\begin{remark}
	It is possible for $\mf p$ in the proof to not be principal but still have $\mf pR_\mf o$ be principal.
\end{remark}
As a corollary of the proof, we get the following results.
\begin{corollary}
	Fix a Noetherian domain $R$. If $R$ is normal, then
	\[R=\bigcap_{\mf p}R_\mf p,\]
	where the intersection is over all primes $\mf p$ of codimension $1$.
\end{corollary}
\begin{corollary}
	Fix $X$ an affine algebraic variety such that $A(X)$ is a normal domain. If we have a subvariety $Y\subseteq X$ is such that $A(Y)$ is of codimension at least $2$, then $A(X-Y)=A(X)$.
\end{corollary}
\begin{proof}
	Suppose that $\mf q$ is the prime ideal corresponding to the variety $Y$. Then $A(X-Y)=A(X)_\mf q$, so taking the intersection finishes. % \todo{what}
\end{proof}

\subsection{Invertible Modules}
For the following discussion, we will take $R$ to be a Noetherian domain, for intuition. We have the following definition.
\begin{definition}[Invertible module]
	An $R$-module $M$ is \textit{invertible} if and only if all prime ideals $\mf p\subseteq R$ has $M_\mf p\cong R_\mf p$.
\end{definition}
It turns out that these are all fractional ideals in the case where $R$ is a Noetherian domain. Before that, here are some examples.
\begin{example}
	A principal ideal $(f)\subseteq R$ is invertible.
\end{example}
\begin{example}
	If $M$ and $N$ are invertible $R$-modules, then any prime $\mf p$ will have
	\[(M\otimes_RN)_\mf p\cong M_\mf p\otimes_{R_\mf p}N_\mf p\cong R_\mf p\otimes_{R_\mf p}R_\mf p\cong R_\mf p,\]
	so $M\otimes_RN$ is also invertible.
\end{example}
\begin{example}
	If $M$ is an invertible, finitely generated $R$-module, then $M^*:=\op{Hom}_R(M,R)$ is also invertible. In particular, because $R$ is Noetherian, $M$ is finitely presented, so
	\[R_\mf p\cong\op{Hom}_{R_\mf p}(R_\mf p,R_\mf p)\cong\op{Hom}_{R_\mf p}(M_\mf p,R_\mf p)\cong\op{Hom}_R(M,R)_\mf p.\]
\end{example}
To start our discussion, here is a lemma.
\begin{lemma}
	Fix a Noetherian domain $R$. An $R$-module $M$ is invertible if and only if the map
	\[\mu:M^*\otimes_RM\to R\]
	by $\varphi\otimes m\mapsto\varphi(m)$ is an isomorphism.
\end{lemma}
\begin{proof}
	It suffices to work with the case that $\mu_\mf p$ is an isomorphism for all primes $\mf p$. By running through the isomorphisms in the examples, we see that we are asking for
	\[\mu_\mf p:(M_\mf p)^*\otimes_{R_\mf p}M_\mf p\to R_\mf p\]
	is an isomorphism for all primes $\mf p$.

	In particular, we are allowed to assume that $R$ is local with maximal ideal $\mf p$. In one direction, suppose that $\mu$ is an isomorphism. By surjectivity, we are promised some
	\[\mu\Bigg(\sum_{i=1}^n\varphi_i\otimes a_i\Bigg)=1.\]
	In particular, there exists $i$ such that $\varphi_i(a_i)\notin\mf p$, but $R\setminus\mf p$ are all units, so we can force $\varphi(a)=1$ for some $\varphi$ and $a$. Now, living in a local ring thus forces by $\varphi$ to show that
	\[M\cong R\oplus\ker\varphi,\]
	but $\ker\varphi$ is trivial because any kernel would have to show up in the kernel of $\mu$, which is trivial by hypothesis.

	We don't discuss the other direction.
\end{proof}
\begin{remark}
	We can see that $M$ will be generated by the elements $a_i$ in the summation
	\[\mu\Bigg(\sum_{i=1}^n\varphi_i\otimes a_i\Bigg)=1.\]
	Thus, $M$ should be finitely generated.
\end{remark}
This discussion gives us the following definition.
\begin{definition}[Picard group]
	Fix a Noetherian domain $R$. Then $\op{Pic}R$ is the group of isomorphism classes of invertible $R$-modules.
\end{definition}
\begin{remark}
	The Picard group loosely corresponds to line bundles.
\end{remark}
To be explicit, the group operation of $\op{Pic}R$ is by
\[[X]\cdot[Y]:=[X\otimes_RY],\]
our identity element is $[R]$, and the inverses are $[X]^{-1}:=\left[X^*\right]$.

\subsection{The Class Group}
To close out class, we discuss the connection to fractional ideals.
\begin{lemma}
	Fix a Noetherian domain $R$. Then $M$ is invertible if and only if $M$ is isomorphic to some nonzero fractional ideal.
\end{lemma}
\begin{proof}
	The idea is to embed $M$ into $K(R)$ to extract our fractional ideal. Well, the embedding $R\to K(R)$ gives us an embedding
	\[M\to K(R)\otimes_RM.\]
	But now, $K(R)\otimes_RM\cong K(R)$ because $K(R)\otimes_RM$ is an invertible module over $K(R)$, which must be isomorphic to $K(R)$ because $K(R)$ only has the localization at the prime $(0)$ (which does nothing).

	As such, we have placed $M$ as an $R$-submodule of $K(R)$ and hence is isomorphic to a nonzero fractional ideal.
\end{proof}
As such, we can give an alternate characterization of the Picard group.
\begin{lemma}
	Fix a Noetherian domain $R$. If $I$ and $J$ are fractional ideals, then
	\[IJ\cong I\otimes_RJ\qquad\text{and}\qquad I^{-1}J\cong\op{Hom}(I.J).\]
\end{lemma}
\begin{proof}
	The isomorphism $I\otimes_RJ\cong IJ$ is by $a\otimes b\mapsto ab$. That $I^{-1}J\cong\op{Hom}_R(I,J)$ follows from carefully considering the localizations.
\end{proof}
Thus, modding out by principal ideals from the fractional ideals gives us the Picard group back again.