% !TEX root = ../notes.tex

We will finish up homological algebra today by talking more about flatness. We will then shift over to talk about completions.

\subsection{A Criterion for Flatness}
Let's review the following result.
\flatcondition*
\begin{proof}
	The second statement follows from \autoref{lem:quickflatcomputation}, so the real content is in the first statement. Of course, if $M$ is flat, then any finitely generated $R$-ideal $I$ makes $R/I$ into an $R$-module, so $\op{Tor}_1^R(R/I,M)=0$ by \autoref{thm:flatisacyclic}.
	
	Thus, we want to show that, if $\op{Tor}_1^R(R/I,M)=0$ for any finitely generated $R$-ideal $I$, then $M$ is flat. We proceed with the following steps. The main point is to use the fact that tensor products use only finite sums.
	\begin{enumerate}
		\item We show $\op{Tor}_1^R(R/I,M)=0$ for all $R$-ideals $I$. By \autoref{lem:quickflatcomputation}, this is equivalent to showing that $I\otimes_RM\to M$ is injective for all $R$-ideals $I$. For this, we proceed by contraposition: we claim that if we can find some $R$-ideal $I$ such that $I\otimes_RM\to M$ has nontrivial kernel, then we can find a finitely generated $I$ such that $I\otimes_RM\to M$ has nontrivial kernel, which is enough by \autoref{lem:quickflatcomputation}.
		
		Well, if we have a nontrivial element $x$ of the kernel of $\varphi:I\otimes M\to_RM$, then we can write
		\[x=\sum_{i=1}^nx_i\otimes m_i\]
		for some $x_1,\ldots,x_n\in I$ and $m_1,\ldots,m_i\in M$. In particular, we are told that
		\[\varphi(x)=\sum_{i=1}^nx_im_i=0\]
		even though $x\ne0$.

		Now, we set $I':=(x_1,\ldots,x_n)$ and consider the natural map $\varphi':I'\otimes_RM\to M$; we claim that $\varphi'$ has nontrivial kernel, which will finish because $I'$ is a finitely generated ideal. Well, we still have
		\[\varphi'\left(\sum_{i=1}^nx_i\otimes m_i\right)=\sum_{i=1}^nx_im_i=0,\]
		so this element $x':=\sum_ix_i\otimes m_i\in I'\otimes_RM$ still lives in the kernel. But $x$ was nonzero in $I\otimes_RM$, so it will remain nonzero in this restriction: namely, the map $I'\into I$ induces a map $I'\otimes_RM\to I\otimes_RM$ by $r\otimes m\mapsto r\otimes m$, through which we see that
		\[x'\mapsto\sum_{i=1}^nx_i\otimes m_i=x.\]
		So if $x'=0$, then $x=0$, which is false, so we instead see that $x'$ is indeed a nontrivial element of our kernel.

		\item We next claim that $\op{Tor}_1^R(N,M)=0$ for any finitely generated $R$-module $N$. We proceed by induction on the number of generators for $N$. If we use zero generators, then $N=0$, so $\op{Tor}_1^R(N,M)=\op{Tor}_1^R(0,M)=0$ by \autoref{lem:torprojective}.

		Otherwise, suppose that all $R$-modules generated by $n$ elements have vanishing $\op{Tor}_1^R(-,M)$. Now, we set any $R$-module $N=(x_1,\ldots,x_{n+1})$ generated by $n+1$ elements, and we fix
		\[N':=(x_1,\ldots,x_n)\subseteq N\]
		to be generated by $N'$ elements. Then any element of $N/N'$ can be written as
		\[\sum_{i=1}^{n+1}r_ix_i+N'=r_{n+1}x_{n+1}+\underbrace{\sum_{i=1}^nr_ix_i}_{\in N'}+N'=r_{n+1}x_{n+1},\]
		so $N/N'$ is generated by $x_{n+1}+N'$. In particular, there is a surjection $R\onto N/N'$ by $1\mapsto x_{n+1}+N'$, so letting $I$ be the kernel of this map, we have an isomorphism $R/I\cong N/N'$.

		Now, consider the short exact sequence
		\[0\to N'\to N\to N/N'\to 0.\]
		Applying \autoref{thm:torles}, we get an exact sequence
		\[\op{Tor}_1^R(N',M)\to\op{Tor}_1^R(N,M)\to\op{Tor}_1^R(N/N',M).\]
		To finish, note $\op{Tor}_1^R(N',M)=0$ by the inductive hypothesis and $\op{Tor}_1^R(N/N',M)=\op{Tor}_1^R(R/I,M)=0$ by the step above, so $\op{Tor}_1^R(N,M)=0$ follows by exactness.

		\item We claim that, if $A$ and $B$ are finitely generated $R$-modules equipped with an embedding $\iota:A\into B$, then the induced map
		\[\iota\otimes M:A\otimes_RM\to B\otimes_RM\]
		is still injective.

		The point is that $\op{Tor}_1^R(B/\iota A,M)=0$. Indeed, because $B$ is finitely generated, there is an integer $b$ with a surjection $R^b\onto B\onto B/\iota A$, so $B/\iota A$ is finitely generated, so $\op{Tor}_1^R(B/\iota A,M)=0$ follows from the previous part.

		But now, we see that we have a short exact sequence
		\[0\to A\stackrel\iota\to B\to B/\iota A\to 0,\]
		which by \autoref{thm:torles} gives us an exact sequence
		\[\op{Tor}_1^R(B/\iota A,M)\to\op{Tor}_0^R(A,M)\to\op{Tor}_0^R(B,M).\]
		Thus, $\op{Tor}_1^R(B/\iota A,M)=0$ forces $\op{Tor}_0^R(A,M)\to\op{Tor}_0^R(B,M)$ to be injective, but \autoref{rem:lzerofunctorial} tells us that this map is $\iota\otimes M:A\otimes_RM\to B\otimes_RM$. So we are done.

		\item Lastly, we show that $M$ is flat, given the previous step. By \autoref{rem:easierflat}, it suffices to show that any embedding $\iota:A\into B$ induces an injective map $\iota\otimes M:A\otimes_RM\to B\otimes_RM$, so the content is in reducing this condition to the finitely generated case, which is the previous step.

		Well, we proceed by contraposition: suppose that $M$ is not flat so that \autoref{rem:easierflat} promises us an embedding of $R$-modules $\iota:A\into B$ such that $\iota\otimes M$ is not injective. This means that we have a nonzero element $x\in A\otimes_RM$, of the form
		\[x:=\sum_{i=1}^na_i\otimes m_i\]
		such that
		\[\sum_{i=1}^n\iota a_i\otimes m_i=(\iota\otimes M)(x)=0.\]
		Well, we set $A':=(a_1,\ldots,a_n)$ and $B':=(\iota a_1,\ldots,\iota a_n)$, which are now both finitely generated. Notably, $\iota$ will restrict to a map $\iota'A'\to B'$, where $\im\iota'\subseteq B'$ because any element $\sum_ir_ia_i$ of $A'$ has
		\[\iota'\Bigg(\sum_{i=1}^nr_ia_i\Bigg)=\sum_{i=1}^nr_i\iota'(a_i)\in(\iota a_1,\ldots,\iota a_n)=B'.\]
		We now claim that $\iota'\otimes M$ has nontrivial kernel, which will finish because $A'$ and $B'$ are finitely generated. Namely, consider the element
		\[x':=\sum_{i=1}^na_i\otimes m_i\in A'\otimes_RM.\]
		Note that $x'\in\ker(\iota'\otimes M)$ because
		\[\iota'(x')=\sum_{i=1}^n\iota'a_i\otimes m_i=\sum_{i=1}^n\iota a_i\otimes m_i=\iota(x)=0.\]
		Further, we note that $A'\subseteq A$ and $B'\subseteq B$ induces a map $A'\otimes_RB'\to A\otimes_RB$ by $a\otimes b\mapsto a\otimes b$ (i.e., by restricting the identity), upon which the element $x'$ goes to
		\[x'=\sum_{i=1}^na_i\otimes m_i\mapsto\sum_{i=1}^na_i\otimes m_i=x,\]
		so $x\ne0$ forces $x'\ne0$. So we see that $x'$ is the nontrivial element of our kernel. This finishes.
		\qedhere
	\end{enumerate}
\end{proof}
\noindent Let's see some examples.
\begin{exe}
	Set $R:=k[x]/\left(x^2\right)$. We show that an $R$-module $M$ is flat if and only if $M$ is free.
\end{exe}
\begin{proof}
	We use \autoref{thm:flatcondition}. Certainly if $I=R$ or $I=0$ we are done. The only other ideal to check is $(x)$, so we just need to verify that
	\[(x)M\to M\]
	is injective. In particular, we need to check that $\ker(x)/\im(x)=0$, which amounts to verifying that $M$ is free because $x$ is the only element that could provide us with a kernel.
\end{proof}
\begin{remark}[Serganova]
	In fact, we can show that any $R$-module $M$ can be written as $M_0\oplus F$ where $M_0\cong\ker(x)/\im(x)$ and $F$ is free.
\end{remark}
\begin{exe}
	Fix $R$ a principal ideal domain. Then an $R$-module $M$ is flat if and only if $M$ is torsion-free.
\end{exe}
\begin{proof}
	All ideals take the form $(a)$, but $(a)\cong R$ because $R$ is a principal ideal domain, so we are merely verifying that the map $R\otimes_RM\to M$ is injective, which is true if and only if $M$ is torsion free.
\end{proof}
\begin{example}
	If a $\ZZ$-module $M$ is finitely generated and torsion free, then $M$ must actually be free by the classification of finitely generated abelian groups. Also, $\QQ$ is torsion free and hence flat.
\end{example}

\subsection{Flatness Locally}
We note the following.
\begin{lemma}
	Fix $R$ a ring and $\mf p$ a prime. If $M$ is a flat $R$-module, then $M_\mf p$ is a flat $R_\mf p$-module.
\end{lemma}
\begin{proof}
	The main point is that $M_\mf p\cong R_\mf p\otimes_RM$, where $M$ is a flat $R$-module. Namely, we will show that $R_\mf p\otimes M$ is a flat $R_\mf p$-module. By \autoref{rem:easierflat}, it suffices to pick up any inclusion $\iota:A\to B$ if $R_\mf p$ and show that the induced map
	\[\iota\otimes(R_\mf p\otimes M):A\otimes_{R_\mf p}(R_\mf p\otimes M)\to B\otimes_{R_\mf p}(R_\mf p\otimes M).\]
	Indeed, the key is showing that the following diagram commutes.
	% https://q.uiver.app/?q=WzAsNCxbMCwwLCIoQVxcb3RpbWVzX3tSX1xcbWYgcH1SX1xcbWYgcClcXG90aW1lc19STSJdLFsxLDAsIihCXFxvdGltZXNfe1JfXFxtZiBwfVJfXFxtZiBwKVxcb3RpbWVzX1JNIl0sWzAsMSwiQVxcb3RpbWVzX3tSX1xcbWYgcH0oUl9cXG1mIHBcXG90aW1lc19STSkiXSxbMSwxLCJBXFxvdGltZXNfe1JfXFxtZiBwfShSX1xcbWYgcFxcb3RpbWVzX1JNKSJdLFswLDEsIihcXGlvdGFcXG90aW1lcyBSX1xcbWYgcClcXG90aW1lcyBNIl0sWzAsMl0sWzIsMywiXFxpb3RhXFxvdGltZXMoUl9cXG1mIHBcXG90aW1lcyBNKSIsMl0sWzEsM11d&macro_url=https%3A%2F%2Fraw.githubusercontent.com%2FdFoiler%2Fnotes%2Fmaster%2Fnir.tex
	\[\begin{tikzcd}
		{(A\otimes_{R_\mf p}R_\mf p)\otimes_RM} & {(B\otimes_{R_\mf p}R_\mf p)\otimes_RM} \\
		{A\otimes_{R_\mf p}(R_\mf p\otimes_RM)} & {A\otimes_{R_\mf p}(R_\mf p\otimes_RM)}
		\arrow["{(\iota\otimes R_\mf p)\otimes M}", from=1-1, to=1-2]
		\arrow[from=1-1, to=2-1]
		\arrow["{\iota\otimes(R_\mf p\otimes M)}"', from=2-1, to=2-2]
		\arrow[from=1-2, to=2-2]
	\end{tikzcd}\tag{$*$}\label{eq:assdiagramcommute}\]
	The vertical morphisms are the isomorphisms from \autoref{prop:tensorassociate}. To see that this diagram commutes, note that along the top we take $(a\otimes r)\otimes m$ to $(\iota a\otimes r)\otimes m$ to $\iota a\otimes(r\otimes m)$. Then along the bottom we take $(a\otimes r)\otimes m$ to $a\otimes(r\otimes m)$ to $\iota a\otimes(r\otimes m)$, which is the same.

	Now, the map
	\[(\iota\otimes R_\mf p)\otimes M:(A\otimes_{R_\mf p}R_\mf p)\otimes_RM\to(B\otimes_{R_\mf p}R_\mf p)\otimes_RM\]
	is injective as an $R$-module homomorphism. In particular, it is injective as a function of sets, so it is injective as an $R_\mf p$-module homomorphism. Thus, the top arrow of \autoref{eq:assdiagramcommute} is injective, so the bottom arrow
	\[\iota\otimes(R_\mf p\otimes M):A\otimes_{R_\mf p}(R_\mf p\otimes M)\to B\otimes_{R_\mf p}(R_\mf p\otimes M)\]
	is also injective. This finishes.
\end{proof}
We might hope the converse holds. Indeed, it does.
\begin{proposition} \label{prop:locallyflat}
	Fix $R$ a ring and $M$ an $R$-module. If $M_\mf p$ is a flat $R_\mf p$-module for all primes $\mf p\subseteq R$, then $M$ is also flat.
\end{proposition}
\begin{proof}
	By \autoref{rem:easierflat}, it suffices to fix some inclusion $\iota:A\into B$ so that we want to show that
	\[M\otimes\iota:M\otimes_RA\to M\otimes_RB\]
	is also an inclusion. By \autoref{cor:liftinglocal}, it will be enough to show that $(M\otimes\iota)_\mf p$ is injective for all primes $\mf p\subseteq R$. For this, we fix any prime $\mf p$ and draw the following diagram.
	% https://q.uiver.app/?q=WzAsNCxbMCwwLCJNX1xcbWYgcFxcb3RpbWVzX3tSX1xcbWYgcH1BX1xcbWYgcCJdLFswLDEsIihNXFxvdGltZXNfUkEpX3tcXG1mIHB9Il0sWzEsMCwiTV9cXG1mIHBcXG90aW1lc197Ul9cXG1mIHB9Ql9cXG1mIHAiXSxbMSwxLCIoTVxcb3RpbWVzX1JCKV97XFxtZiBwfSJdLFswLDFdLFsyLDNdLFswLDIsIk1fXFxtZiBwXFxvdGltZXNcXGlvdGFfXFxtZiBwIl0sWzEsMywiKE1cXG90aW1lc1xcaW90YSlfXFxtZiBwIiwyXV0=&macro_url=https%3A%2F%2Fraw.githubusercontent.com%2FdFoiler%2Fnotes%2Fmaster%2Fnir.tex
	\[\begin{tikzcd}
		{M_\mf p\otimes_{R_\mf p}A_\mf p} & {M_\mf p\otimes_{R_\mf p}B_\mf p} \\
		{(M\otimes_RA)_{\mf p}} & {(M\otimes_RB)_{\mf p}}
		\arrow[from=1-1, to=2-1]
		\arrow[from=1-2, to=2-2]
		\arrow["{M_\mf p\otimes\iota_\mf p}", from=1-1, to=1-2]
		\arrow["{(M\otimes\iota)_\mf p}"', from=2-1, to=2-2]
	\end{tikzcd}\]
	This diagram commutes by \autoref{rem:localizetensorfunctorial}, and the bottom arrow is $(M\otimes\iota)_\mf p$ because this is our induced arrow. Thus, we see that it suffices to show that the top arrow is injective.

	Well, $A\into B$ is injective, so \autoref{prop:localexact} tells us that the map $\iota_\mf p:A_\mf p\into B_\mf p$ is injective. But now, $M_\mf p$ is a flat $R_\mf p$-module, so this causes the map
	\[M_\mf p\otimes\iota_\mf p:M_\mf p\otimes_{R_\mf p}A_\mf p\to M_\mf p\otimes_{R_\mf p}B_\mf p\]
	to be injective. This finishes.
\end{proof}
So we are motivated to study how flat modules behave under localization.
\begin{prop} \label{prop:flatislocallyfree}
	Fix a local ring $R$ with maximal ideal $\mf p$. Further, let $M$ be a finitely presented $R$-module. If $M$ is flat, then $M$ is free.
\end{prop}
\begin{proof}
	The idea is to use \autoref{cor:quotientdimension}. For psychological reasons, we will pick up the following lemma, merely because it will make our lives a little easier; it is a surprise tool that will help us later.
	\begin{lemma} \label{lem:finitepresentses}
		Fix a ring $R$ and a short exact sequence
		\[0\to A\to B\to C\to 0\]
		of $R$-modules. If $C$ is finitely presented, and $B$ is finitely generated, then $A$ is also finitely generated.
	\end{lemma}
	\begin{proof}
		The main point is to use the Snake lemma. Because $C$ is finitely presented, we are promised an exact sequence
		\[R^m\to R^n\to C\to 0.\]
		As such, we lay our two exact sequences on top of each other, as follows.
		% https://q.uiver.app/?q=WzAsOSxbMSwwLCJSXm0iXSxbMiwwLCJSXm4iXSxbMywwLCJDIl0sWzQsMCwiMCJdLFswLDEsIjAiXSxbMSwxLCJBIl0sWzIsMSwiQiJdLFszLDEsIkMiXSxbNCwxLCIwIl0sWzIsNywiIiwwLHsibGV2ZWwiOjIsInN0eWxlIjp7ImhlYWQiOnsibmFtZSI6Im5vbmUifX19XSxbMCwxXSxbMSwyXSxbMiwzXSxbNCw1XSxbNSw2XSxbNiw3XSxbNyw4XSxbMSw2LCIiLDEseyJzdHlsZSI6eyJib2R5Ijp7Im5hbWUiOiJkYXNoZWQifX19XSxbMCw1LCIiLDEseyJzdHlsZSI6eyJib2R5Ijp7Im5hbWUiOiJkYXNoZWQifX19XV0=
		\[\begin{tikzcd}
			& {R^m} & {R^n} & C & 0 \\
			0 & A & B & C & 0
			\arrow[Rightarrow, no head, from=1-4, to=2-4]
			\arrow[from=1-2, to=1-3]
			\arrow[from=1-3, to=1-4]
			\arrow[from=1-4, to=1-5]
			\arrow[from=2-1, to=2-2]
			\arrow[from=2-2, to=2-3]
			\arrow[from=2-3, to=2-4]
			\arrow[from=2-4, to=2-5]
			\arrow[dashed, from=1-3, to=2-3]
			\arrow[dashed, from=1-2, to=2-2]
		\end{tikzcd}\]
		Because $R^n$ is free and hence projective (by \autoref{lem:projgrabbag}, say), we get a map $R^n\to B$ making the right square= commute, and so the diagram will induce a map $R^m\to A$ making the total diagram commute. Now, \autoref{lem:snake} gives us an exact sequence
		\[\ker(C\to C)\to\coker\left(R^m\to A\right)\to\coker\left(R^n\to B\right)\to\coker(C\to C).\]
		Because $C\to C$ is $\id_C$, these terms are zero, so the above is in fact an isomorphism $\coker\left(R^m\to A\right)\cong\coker\left(R^n\to B\right)$.
		
		As such, we note that $B$ will project onto $\coker\left(R^n\to B\right)$; thus, $\coker\left(R^n\to B\right)$ and so $\coker\left(R^m\to A\right)$ is finitely generated. Further, $\im\left(R^m\to A\right)$ is (by definition) projected onto by $R^m$ and therefore finitely generated, so we have the short exact sequence
		\[0\to\im\left(R^m\to A\right)\to A\to\coker\left(R^m\to A\right)\to0,\]
		where the ends are finitely generated. By building finitely generated free resolutions via \autoref{rem:fingenprojresolution} (e.g., with $R^x\onto\im\left(R^m\to A\right)$ and $R^y\onto\coker\left(R^m\to A\right)$) for either ends, \autoref{prop:horseshoe} promises us a surjection $R^{x+y}\onto A$ making the following diagram commute.
		% https://q.uiver.app/?q=WzAsMTAsWzAsMCwiMCJdLFsxLDAsIlxcaW1cXGxlZnQoUl5tXFx0byBBXFxyaWdodCkiXSxbMywwLCJcXGNva2VyXFxsZWZ0KFJebVxcdG8gQVxccmlnaHQpIl0sWzIsMCwiQSJdLFs0LDAsIjAiXSxbMCwxLCIwIl0sWzEsMSwiUl54Il0sWzMsMSwiUl55Il0sWzIsMSwiUl57eCt5fSJdLFs0LDEsIjAiXSxbNSw2XSxbNiw4XSxbOCw3XSxbNyw5XSxbMCwxXSxbNiwxXSxbMSwzXSxbOCwzXSxbMywyXSxbNywyXSxbMiw0XV0=&macro_url=https%3A%2F%2Fraw.githubusercontent.com%2FdFoiler%2Fnotes%2Fmaster%2Fnir.tex
		\[\begin{tikzcd}
			0 & {\im\left(R^m\to A\right)} & A & {\coker\left(R^m\to A\right)} & 0 \\
			0 & {R^x} & {R^{x+y}} & {R^y} & 0
			\arrow[from=2-1, to=2-2]
			\arrow[from=2-2, to=2-3]
			\arrow[from=2-3, to=2-4]
			\arrow[from=2-4, to=2-5]
			\arrow[from=1-1, to=1-2]
			\arrow[from=2-2, to=1-2]
			\arrow[from=1-2, to=1-3]
			\arrow[from=2-3, to=1-3]
			\arrow[from=1-3, to=1-4]
			\arrow[from=2-4, to=1-4]
			\arrow[from=1-4, to=1-5]
		\end{tikzcd}\]
		So $A$ is in fact finitely generated (the middle arrow is a surjection either by construction or by \autoref{lem:snake}), which is what we wanted.
	\end{proof}
	Now, $M$ is finitely generated, so we can find a free module of finite rank $R^n$ with a surjection $\pi_0:R^n\onto M$. Of course, $n$ might be too large, so we would like to refine our $n$; we do this by looking at $M/\mf pM$.
	
	In particular, tensoring by $-\otimes_RR/\mf p$, we get an induced surjection
	\[(R/\mf p)^n\cong(R\otimes_RR/\mf p)^n\cong R^n\otimes_RR/\mf p\stackrel{\pi_0}\onto M\otimes_RR/\mf p\cong M/\mf pM,\]
	where the last isomorphism is by \autoref{prop:tensorquotient}. This is all to say that $M/\mf pM$ is a finite-dimensional $R/\mf p$-vector space, so give it a dimension $d$ with basis
	\[\{m_1+\mf pM,\ldots,m_d+\mf pM\}.\]
	In particular, \autoref{cor:quotientdimension} tells us that these elements $\{m_1,\ldots,m_d\}$ will generate $M$, so we get a surjection
	\[\pi:R^d\onto M\]
	by sending the basis vector $e_i\in R^d$ to $\pi:e_i\mapsto m_i$.
	
	We would like to show that $\ker\pi$ is zero. Well, setting $K:=\ker\pi$, we get a short exact sequence
	\[0\to K\to R^d\to M\to0,\]
	which by \autoref{lem:finitepresentses} forces $K$ to be finitely generated. We will use Nakayama's lemma to get us all the way to $K=0$. We draw the following diagram.
	% https://q.uiver.app/?q=WzAsMTUsWzAsMCwiXFxvcHtUb3J9XzFeUihNLFIvXFxtZiBwKSJdLFsxLDAsIlxcb3B7VG9yfV8wXlIoSyxSL1xcbWYgcCkiXSxbMiwwLCJcXG9we1Rvcn1fMF5SXFxsZWZ0KFJeZCxSL1xcbWYgcFxccmlnaHQpIl0sWzMsMCwiXFxvcHtUb3J9XzBeUihNLFIvXFxtZiBwKSJdLFs0LDAsIjAiXSxbMSwxLCJLXFxvdGltZXNfUlIvXFxtZiBwIl0sWzIsMSwiUl5kXFxvdGltZXNfUlIvXFxtZiBwIl0sWzMsMSwiTVxcb3RpbWVzX1JSL1xcbWYgcCJdLFs0LDEsIjAiXSxbNCwyLCIwIl0sWzEsMiwiSy9cXG1mIHBLIl0sWzIsMiwiUl5kL1xcbWYgcFJeZCJdLFszLDIsIk0vXFxtZiBwTSJdLFswLDEsIjAiXSxbMCwyLCIwIl0sWzAsMV0sWzEsMl0sWzIsM10sWzMsNF0sWzEsNSwiIiwwLHsibGV2ZWwiOjIsInN0eWxlIjp7ImhlYWQiOnsibmFtZSI6Im5vbmUifX19XSxbMiw2LCIiLDAseyJsZXZlbCI6Miwic3R5bGUiOnsiaGVhZCI6eyJuYW1lIjoibm9uZSJ9fX1dLFszLDcsIiIsMCx7ImxldmVsIjoyLCJzdHlsZSI6eyJoZWFkIjp7Im5hbWUiOiJub25lIn19fV0sWzUsMTBdLFs2LDExXSxbNywxMl0sWzUsNl0sWzYsN10sWzcsOF0sWzEwLDExXSxbMTEsMTJdLFsxMiw5XSxbMTMsNV0sWzE0LDEwXV0=&macro_url=https%3A%2F%2Fraw.githubusercontent.com%2FdFoiler%2Fnotes%2Fmaster%2Fnir.tex
	\[\begin{tikzcd}
		{\op{Tor}_1^R(M,R/\mf p)} & {\op{Tor}_0^R(K,R/\mf p)} & {\op{Tor}_0^R\left(R^d,R/\mf p\right)} & {\op{Tor}_0^R(M,R/\mf p)} & 0 \\
		0 & {K\otimes_RR/\mf p} & {R^d\otimes_RR/\mf p} & {M\otimes_RR/\mf p} & 0 \\
		0 & {K/\mf pK} & {R^d/\mf pR^d} & {M/\mf pM} & 0
		\arrow[from=1-1, to=1-2]
		\arrow[from=1-2, to=1-3]
		\arrow[from=1-3, to=1-4]
		\arrow[from=1-4, to=1-5]
		\arrow[Rightarrow, no head, from=1-2, to=2-2]
		\arrow[Rightarrow, no head, from=1-3, to=2-3]
		\arrow[Rightarrow, no head, from=1-4, to=2-4]
		\arrow[from=2-2, to=3-2]
		\arrow[from=2-3, to=3-3]
		\arrow[from=2-4, to=3-4]
		\arrow[from=2-2, to=2-3]
		\arrow[from=2-3, to=2-4]
		\arrow[from=2-4, to=2-5]
		\arrow[from=3-2, to=3-3]
		\arrow[from=3-3, to=3-4]
		\arrow[from=3-4, to=3-5]
		\arrow[from=2-1, to=2-2]
		\arrow[from=3-1, to=3-2]
	\end{tikzcd}\tag{$*$}\label{eq:bigtensordiagram}\]
	Note that the top two rows of the diagram commute by \autoref{rem:lzerofunctorial}. The bottom two rows of the diagram commute and feature vertical isomorphisms by \autoref{rem:tensorquotientfunctorial}.
	
	Additionally, the top row is exact by \autoref{thm:torles}, but we see that $\op{Tor}_1^R(M,R/\mf p)\cong\op{Tor}_1^R(R/\mf p,M)$ by \autoref{lem:tor1commute}, which is $\op{Tor}_1^R(R/\mf p,M)=0$ by \autoref{thm:flatisacyclic}, so in fact the top row is short exact, so it follows by the commutativity of the diagram and the fact that all vertical morphisms are isomorphisms that all the rows are exact.
	
	Now, $\dim_{R/\mf p}M/\mf pM=d$, and $R^d\otimes_RR/\mf p\cong(R\otimes_RR/\mf p)^d\cong(R/\mf p)^d$ is also of dimension $d$, so the fact that the map $R^d/\mf pR^d\onto M/\mf pM$ is surjective forces this to be an isomorphism. So the exactness of the bottom row of \autoref{eq:bigtensordiagram} thus forces $K/\mf pK=0$. Because $R$ is local with maximal ideal $\mf p$, we conclude that $K=0$ by \autoref{thm:nakayama}. This gives $R^d\cong M$, finishing.
\end{proof}

\subsection{Flatness and Projectivity}
As a consequence of all of our hard work, we get the following lovely result.
\begin{theorem}
	Fix a ring $R$ and a finitely presented module $M$. Then $M$ is flat if and only if $M$ is projective.
\end{theorem}
\begin{proof}
	We claim that the following conditions are all equivalent.
	\begin{listalph}
		\item $M$ is flat.
		\item $M_\mf p$ is flat for all prime ideal $\mf p\subseteq R$.
		\item $M_\mf p$ is free for all prime ideals $\mf p\subseteq R$.
		\item $M$ is projective.
	\end{listalph}
	The fact that (a) implies (b) is by \autoref{prop:locallyflat}. The implication (b) to (c) is by \autoref{prop:flatislocallyfree} because $R_\mf p$ is a local ring (with maximal ideal by $\mf pR_\mf p$), by \autoref{prop:localizetolocal}. Then (c) implies (d) by \autoref{prop:projislocallyfree}. Lastly, (d) implies (a) by \autoref{prop:projisflat}.
\end{proof}
\begin{remark}
	The above result is amazing: we proved this essentially by looking at the module $M$ locally at all primes, but the final result has nothing to do with localization!
\end{remark}
However, the finitely generated condition is necessary.
\begin{exe}
	We show that $\QQ$ is a flat but not projective $\ZZ$-module.
\end{exe}
\begin{proof}
	We will be brief. The $\ZZ$-module $\QQ=\ZZ_{(0)}$ is flat (by \autoref{cor:localflat}), but $\QQ$ is not projective. To see this, note that each $q\in\QQ$ creates a map $\pi_q:\ZZ\to\QQ$
	\[\pi_q:k\mapsto kq.\]
	These glue together to create a map $\pi:\ZZ^{\oplus\QQ}\to\QQ$, which is surjective because each $x\in\QQ$ has
	\[\pi\left(\{1_{q=x}\}_{q\in\QQ}\right)=\sum_{q\in\QQ}\pi_q(1_{q=x})=\sum_{q\in\QQ}1_{q=x}q=q.\]
	However, $\op{Hom}_\ZZ\left(\QQ,\ZZ^{\oplus\QQ}\right)=0$ because the projection onto any coordinate gives a map $\QQ\to\ZZ$, which must be $0$: if $\varphi:\QQ\to\ZZ$ is a morphism such that $\varphi(q)=n\ne0$, then $2n\varphi(q/2n)=\varphi(q)=n$, so $\varphi(q/2n)=1/2\notin\ZZ$, which does not make sense.
	
	Thus, the short exact sequence
	\[0\to\ker\pi\to\ZZ^{\oplus\QQ}\stackrel\pi\to\QQ\to0\]
	cannot split because the only possible lift for $\pi$ is $0$, which doesn't work. So $\QQ$ is not projective by part(b) of \autoref{lem:projgrabbag}.
\end{proof}