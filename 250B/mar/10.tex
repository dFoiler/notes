% !TEX root = ../notes.tex

We continue discussing completion.

\subsection{Refining Inverse Limits}
Recall that we had a notion of completion for modules as follows.
\modcomplete*
\noindent Here is our primary example.
\begin{example}
	If we fix an ideal $I\subseteq R$, then we are granted an $I$-adic filtration of $M$, which gives the $I$-adic completion of $M$. In particular, this is an $\widehat R_I$-module by \autoref{lem:completemodovercompletering}.
\end{example}
We are going to want some freedom in changing our exact filtration, so we have the following sequence of lemmas. To start, subsequences don't do anything to our inverse limit.
\begin{lemma} \label{lem:subsequencefiltration}
	Fix an $R$-module $M$ with a filtration $\mathcal J=\{M_k\}_{k\in\NN}$. Then, for any strictly increasing function $\alpha:\NN\to\NN$, we have
	\[\limit M/M_{\alpha k}\cong\limit M/M_k.\]
\end{lemma}
\begin{proof}
	We proceed by force. For any $k$, we note that $\alpha k\ge k$ (e.g., by induction because $\alpha$ is strictly increasing), so $M_k\supseteq M_{\alpha k}$, so we have an induced map
	\[\psi_k:M/M_{\alpha k}\onto M/M_k.\]
	As such, we define the map $\psi:\prod_kM/M_{\alpha k}\to\prod_kM/M_k$ by
	\[\psi:\{m_k+M_{\alpha k}\}_{k\in\NN}\mapsto\{m_k+M_k\}_{k\in\NN}\]
	created by gluing the maps $\psi_k$. We claim that $\psi$ descends to the desired isomorphism: let $\varphi$ be the restriction of $\psi$ to $\limit M/M_{\alpha k}$. We have the following checks on $\varphi$.
	\begin{itemize}
		\item The image of $\varphi$ is contained in $\limit M/M_k$. Indeed, fix some $\{m_k+M_{\alpha k}\}_{k\in\NN}$, which goes to $\{m_k+M_k\}_{k\in\NN}$. For any $i>j$, we need to show that
		\[m_i\equiv m_j\pmod{M_i}.\]
		Well, by hypothesis, we see that $m_i\equiv m_j\pmod{M_{\alpha i}}$, so $m_i-m_j\in M_{\alpha i}\subseteq M_i$, which is what we wanted.
		\item The image of $\varphi$ contains $\limit M/M_k$. For this, we create an inverse map of sets. Indeed, fix some $\{m_k+M_k\}_{k\in\NN}$, and we note that
		\[\varphi:\{m_{\alpha k}+M_{\alpha k}\}_{k\in\NN}\mapsto\{m_{\alpha k}+M_k\}_{k\in\NN}.\]
		So we claim that $\{m_{\alpha k}+M_{\alpha k}\}_{k\in\NN}$ is the desired input. First, this input is valid: for any $i>j$, we see that $m_{\alpha i}\equiv m_{\alpha j}\pmod{M_{\alpha i}}$ by hypothesis on the $m_i$.

		But second, we note that $\alpha k\ge k$ implies that
		\[m_{\alpha k}\equiv m_k\pmod{M_k},\]
		so $\{m_{\alpha k}+M_k\}_{k\in\NN}=\{m_k+M_k\}_{k\in\NN}$, so we have indeed hit the desired coset.
		\item The kernel of $\varphi$ is trivial. Indeed, suppose that $\{m_k+M_{\alpha k}\}_{k\in\NN}$ goes to $0$ under $\varphi$, which means that
		\[m_k+M_k=0+M_k\]
		for each $k$. But this implies that $m_{\alpha k}\in M_{\alpha k}$ while $\alpha k\ge k$, so the hypothesis on the $m_k$ implies that
		\[m_k\equiv m_{\alpha k}\equiv0\pmod{M_{\alpha k}},\]
		so the $m_k+M_{\alpha k}$ all vanish. This finishes.
	\end{itemize}
	Thus, we have shown that $\varphi$ will inject onto $\limit M/M_k$ and therefore witnesses the needed isomorphism
	\[\limit M/M_{\alpha k}\cong\limit M/M_k.\]
	This finishes.
\end{proof}
Next, we note that containment is fairly well-behaved.
\begin{lemma} \label{lem:containedfiltration}
	Fix an $R$-module $M$ with filtrations $\mathcal J=\{M_k\}_{k\in\NN}$ and $\mathcal J'=\{M_k'\}_{k\in\NN}$ such that $M_k\subseteq M_k'$ for each $k$. Then the map
	\[\{m_k+M_k\}_{k\in\NN}\mapsto\{m_k+M_k'\}_{k\in\NN}\]
	defines a morphism $\limit M/M_k\to\limit M/M_k'$.
\end{lemma}
\begin{proof}
	For any fixed $k$, that $M_k\subseteq M_k'$ induces a morphism $\psi_k:M/M_k\to M/M_k'$ by
	\[\psi_k:m_k+M_k\to m_k+M_k'.\]
	These glue together to a morphism
	\[\psi:\prod_{k\in\NN}M/M_k\to\prod_{k\in\NN}M/M_k'.\]
	Let $\varphi$ be the restriction of this map to $\limit M/M_k$, and we need to show $\im\varphi\subseteq\limit M/M_k'$. Well, fix $\{m_k+M_k\}_{k\in\NN}$ in $\limit M/M_k$. Then, for any $i>j$, we see that
	\[m_i-m_j\in M_j\subseteq M_j',\]
	so it follows
	\[\{m_k+M_k'\}_{k\in\NN}\in\limit M/M_k',\]
	which is what we wanted.
\end{proof}
We can then synthesize the above two lemmas to give the following refinement result.
\begin{lemma} \label{lem:samecompletion}
	Fix an $R$-module $M$ with filtrations $\mathcal J=\{M_k\}_{k\in\NN}$ and $\mathcal J'=\{M_k'\}_{k\in\NN}$. Further, suppose that, for all $i$, there exists $j$ such that $M_i\supseteq M_j'$; similarly, there exists (perhaps another) $j$ such that $M_i'\supseteq M_j$. Then we have an isomorphism
	\[\limit M/M_i\cong\limit M/M_i'\]
\end{lemma}
\begin{proof}
	In general, if we have two inverse limits, the way to define an inverse limit is to define a map into each of the components. To manifest this idea, we pick up strictly increasing $\alpha,\beta,\gamma:\NN\to\NN$ such that
	\[M_j\supseteq M_{\alpha(j)}'\supseteq M_{\beta(j)}\supseteq M'_{\gamma(j)}.\tag{$*$}\label{eq:inclusions}\]
	To show that such $\alpha,\beta,\gamma$ all actually exist, we proceed inductively: we can start with $\alpha(0)=\beta(0)=\gamma(0)$ because $M_0=M_0'=M$. Then, if we have defined all three up to $n\in\NN$, we increment in three steps.
	\begin{itemize}
		\item We find some $n'$ such that $M_n\supseteq M_{n'}'$. Then we can set $\alpha(n+1)=\max\{n',\alpha n+1\}$, which works because $M_{n'}'\supseteq M_{\alpha(n+1)}$ while $\alpha(n+1)>\alpha(n)$.
		\item We find some $n'$ such that $M_{\alpha(n+1)}'\supseteq M_{n'}$. Then we can set $\beta(n+1)=\max\{n',\beta n+1\}$, which works because $M_{\alpha(n+1)}'\supseteq M_{\beta(n+1)}$ while $\beta(n+1)>\beta(n)$.
		\item We find some $n'$ such that $M_{\beta(n+1)}\supseteq M_{n'}'$. Then we can set $\gamma(n+1)=\max\{n',\gamma n+1\}$, which works because $M_{\beta(n+1)}\supseteq M_{\gamma(n+1)}'$ while $\gamma(n+1)>\gamma(n)$.
	\end{itemize}
	Anyways, the point is that our $\alpha,\beta,\gamma$ induce morphisms
	\[\limit M/M'_{\gamma(j)}\to\limit M/M_{\beta(j)}\to\limit M/M'_{\alpha(j)}\to\limit M/M_j\]
	by \autoref{lem:containedfiltration}. In fact, \autoref{lem:subsequencefiltration} lets us remove the $\alpha,\beta,\gamma$ to set up the following commutative diagram.
	% https://q.uiver.app/?q=WzAsOCxbMCwwLCJcXGxpbWl0IE0vTSdfe1xcZ2FtbWEoail9Il0sWzEsMCwiXFxsaW1pdCBNL01fe1xcYmV0YShqKX0iXSxbMiwwLCJcXGxpbWl0IE0vTSdfe1xcYWxwaGEoail9Il0sWzMsMCwiXFxsaW1pdCBNL01fe2p9Il0sWzAsMSwiXFxsaW1pdCBNL01fe2p9JyJdLFsxLDEsIlxcbGltaXQgTS9NX3tqfSJdLFszLDEsIlxcbGltaXQgTS9NX3tqfSJdLFsyLDEsIlxcbGltaXQgTS9NX3tqfSciXSxbMCw0XSxbMSw1XSxbMiw3XSxbMyw2XSxbMCwxXSxbMSwyXSxbMiwzXSxbNCw1LCJmIiwwLHsic3R5bGUiOnsiYm9keSI6eyJuYW1lIjoiZGFzaGVkIn19fV0sWzUsNywiZyIsMCx7InN0eWxlIjp7ImJvZHkiOnsibmFtZSI6ImRhc2hlZCJ9fX1dLFs3LDYsImgiLDAseyJzdHlsZSI6eyJib2R5Ijp7Im5hbWUiOiJkYXNoZWQifX19XV0=&macro_url=https%3A%2F%2Fraw.githubusercontent.com%2FdFoiler%2Fnotes%2Fmaster%2Fnir.tex
	\[\begin{tikzcd}
		{\limit M/M'_{\gamma(j)}} & {\limit M/M_{\beta(j)}} & {\limit M/M'_{\alpha(j)}} & {\limit M/M_{j}} \\
		{\limit M/M_{j}'} & {\limit M/M_{j}} & {\limit M/M_{j}'} & {\limit M/M_{j}}
		\arrow[from=1-1, to=2-1]
		\arrow[from=1-2, to=2-2]
		\arrow[from=1-3, to=2-3]
		\arrow[from=1-4, to=2-4]
		\arrow[from=1-1, to=1-2]
		\arrow[from=1-2, to=1-3]
		\arrow[from=1-3, to=1-4]
		\arrow["f", dashed, from=2-1, to=2-2]
		\arrow["g", dashed, from=2-2, to=2-3]
		\arrow["h", dashed, from=2-3, to=2-4]
	\end{tikzcd}\]
	Namely, the vertical morphisms are isomorphisms and therefore induce the morphisms on the bottom row. We claim that $gf$ and $hg$ are both the identity. Indeed, tracking through the morphisms in the commutative diagram, we see that $fg$ moves as follows.
	% https://q.uiver.app/?q=WzAsOCxbMCwwLCJcXHttX3tcXGdhbW1hIGt9K01fe1xcZ2FtbWEga30nXFx9X3trXFxpblxcTk59Il0sWzEsMCwiXFx7bV97XFxnYW1tYSBrfStNX3tcXGJldGEga31cXH1fe2tcXGluXFxOTn0iXSxbMiwwLCJcXHttX3tcXGdhbW1hIGt9K01fe1xcYWxwaGEga30nXFx9X3trXFxpblxcTk59Il0sWzMsMCwiXFxidWxsZXQiXSxbMCwxLCJcXHttX2srTV9rJ1xcfV97a1xcaW5cXE5OfSJdLFsxLDEsIlxcYnVsbGV0Il0sWzMsMSwiXFxidWxsZXQiXSxbMiwxLCJcXHttX3tcXGdhbW1hIGt9K01fe2t9J1xcfV97a1xcaW5cXE5OfSJdLFswLDRdLFsxLDVdLFsyLDddLFszLDZdLFswLDFdLFsxLDJdLFsyLDNdLFs0LDUsImYiLDAseyJzdHlsZSI6eyJib2R5Ijp7Im5hbWUiOiJkYXNoZWQifX19XSxbNSw3LCJnIiwwLHsic3R5bGUiOnsiYm9keSI6eyJuYW1lIjoiZGFzaGVkIn19fV0sWzcsNiwiaCIsMCx7InN0eWxlIjp7ImJvZHkiOnsibmFtZSI6ImRhc2hlZCJ9fX1dXQ==&macro_url=https%3A%2F%2Fraw.githubusercontent.com%2FdFoiler%2Fnotes%2Fmaster%2Fnir.tex
	\[\begin{tikzcd}
		{\{m_{\gamma k}+M_{\gamma k}'\}_{k\in\NN}} & {\{m_{\gamma k}+M_{\beta k}\}_{k\in\NN}} & {\{m_{\gamma k}+M_{\alpha k}'\}_{k\in\NN}} & \bullet \\
		{\{m_k+M_k'\}_{k\in\NN}} & \bullet & {\{m_{\gamma k}+M_{k}'\}_{k\in\NN}} & \bullet
		\arrow[from=1-1, to=2-1]
		\arrow[from=1-2, to=2-2]
		\arrow[from=1-3, to=2-3]
		\arrow[from=1-4, to=2-4]
		\arrow[from=1-1, to=1-2]
		\arrow[from=1-2, to=1-3]
		\arrow[from=1-3, to=1-4]
		\arrow["f", dashed, from=2-1, to=2-2]
		\arrow["g", dashed, from=2-2, to=2-3]
		\arrow["h", dashed, from=2-3, to=2-4]
	\end{tikzcd}\]
	This is the identity because $\gamma k\ge k$ (because $\gamma$ is strictly increasing), so $m_k+M_k'=m_{\gamma k}+M_k'$ for any $k$.
	
	Similarly, we track $hg$ as follows.
	% https://q.uiver.app/?q=WzAsOCxbMCwwLCJcXGJ1bGxldCJdLFsxLDAsIlxce21fe1xcYmV0YSBrfStNX3tcXGJldGEga31cXH1fe2tcXGluXFxOTn0iXSxbMiwwLCJcXHttX3tcXGJldGEga30rTV97XFxhbHBoYSBrfSdcXH1fe2tcXGluXFxOTn0iXSxbMywwLCJcXHttX3tcXGJldGEga30rTV97a31cXH1fe2tcXGluXFxOTn0iXSxbMCwxLCJcXGJ1bGxldCJdLFsxLDEsIlxce21faytNX2tcXH1fe2tcXGluXFxOTn0iXSxbMywxLCJcXHttX3tcXGJldGEga30rTV97a31cXH1fe2tcXGluXFxOTn0iXSxbMiwxLCJcXGJ1bGxldCJdLFswLDRdLFsxLDVdLFsyLDddLFszLDZdLFswLDFdLFsxLDJdLFsyLDNdLFs0LDUsImYiLDAseyJzdHlsZSI6eyJib2R5Ijp7Im5hbWUiOiJkYXNoZWQifX19XSxbNSw3LCJnIiwwLHsic3R5bGUiOnsiYm9keSI6eyJuYW1lIjoiZGFzaGVkIn19fV0sWzcsNiwiaCIsMCx7InN0eWxlIjp7ImJvZHkiOnsibmFtZSI6ImRhc2hlZCJ9fX1dXQ==&macro_url=https%3A%2F%2Fraw.githubusercontent.com%2FdFoiler%2Fnotes%2Fmaster%2Fnir.tex
	\[\begin{tikzcd}
		\bullet & {\{m_{\beta k}+M_{\beta k}\}_{k\in\NN}} & {\{m_{\beta k}+M_{\alpha k}'\}_{k\in\NN}} & {\{m_{\beta k}+M_{k}\}_{k\in\NN}} \\
		\bullet & {\{m_k+M_k\}_{k\in\NN}} & \bullet & {\{m_{\beta k}+M_{k}\}_{k\in\NN}}
		\arrow[from=1-1, to=2-1]
		\arrow[from=1-2, to=2-2]
		\arrow[from=1-3, to=2-3]
		\arrow[from=1-4, to=2-4]
		\arrow[from=1-1, to=1-2]
		\arrow[from=1-2, to=1-3]
		\arrow[from=1-3, to=1-4]
		\arrow["f", dashed, from=2-1, to=2-2]
		\arrow["g", dashed, from=2-2, to=2-3]
		\arrow["h", dashed, from=2-3, to=2-4]
	\end{tikzcd}\]
	Again, this is the identity because $\beta k\ge k$ everywhere implies that $m_k+M_k=m_{\beta k}+M_k$ for any $k$.

	Thus, we see that $g$ is a morphism with both a left and a right inverse, so it is an isomorphism. (For example, a left inverse shows that $g$ is injective, and a right inverse shows that $g$ is surjective.) This is what we wanted to prove, so we are done.
\end{proof}
\begin{remark}[Nir]
	Another way to prove \autoref{lem:samecompletion} is to note that both filtrations will induce the same topology on $M$ and therefore will be isomorphic as completions.
\end{remark}

\subsection{Completion for Modules}
We now return to talking about completions. We want to show that $\widehat R_I$ is flat, so we will want to talk about short exact sequences.
\begin{lemma} \label{lem:fgcompletionexactness}
	Fix $R$ a Noetherian ring and an ideal $I$. Further, suppose that we have a short exact sequence
	\[0\to A\to B\to C\to 0\]
	of finitely generated $R$-modules. Then we have a short exact sequence
	\[0\to\widehat A\to\widehat B\to\widehat C\to0\]
	of completions at $I$.
\end{lemma}
\begin{proof}
	To show that
	\begin{equation}
		0\to\widehat A\to\widehat B\to\widehat C\to0 \label{eq:desiredses}
	\end{equation}
	is a short exact sequence, we will show that $\widehat B\onto\widehat C$ by hand and then use the fact that inverse limits commute with kernels to get exactness at $\widehat A$ and $\widehat B$.

	Let's start with exactness at $\widehat C$ in \autoref{eq:desiredses}. Label our short exact sequence by
	\[0\to A\stackrel\iota\to B\stackrel\pi\to C\to 0\]
	and tensor by $-\otimes R/I^s$ to get a right-exact sequence
	\begin{equation}
		A/I^sA\to B/I^sB\to C/I^sC\to0. \label{eq:rightexactseq}
	\end{equation}
	With this in mind, fix some $\{c_i\}_{i\in\NN}\in\widehat C$. We want to find $\{b_i\}_{i\in\NN}\in\widehat B$ such that $\pi(b_i)\equiv c_i\pmod{I^iC}$ for all $i\in\NN$. We proceed inductively, starting with $b_0=0$, which is legal because $c_0\equiv0\pmod C$.
	
	Now suppose we already have $\{b_i\}_{i<s}$ for $s>0$ which cohere properly, and we construct $b_s$ with $\pi b_s\equiv c_s\pmod{I^sC}$ and $b_s\equiv b_{s-1}\pmod{I^{s-1}B}$. Well, we start by simply guessing: lift $c_s\in C$ up through $\pi\colon B\onto C$ to some $\widetilde b_s\in B$ such that $\pi\widetilde b_s=c_s$. The concern here is that we might not have $\widetilde b_s\equiv b_{s-1}$, but we must be pretty close because
	\[\pi\big(\widetilde b_s-b_{s-1}+I^{s-1}B\big)=c_s-c_{s-1}+I^{s-1}C=c_{s-1}-c_{s-1}+I^{s-1}C=0+I^{s-1}C.\]
	Thus, exactness of \autoref{eq:rightexactseq} promises us $a_s\in A$ such that $\iota a+I^{s-1}B=\iota(a+I^{s-1}A)=\widetilde b_s-b_{s-1}+I^{s-1}B$. As such, we set
	\[b_s\coloneqq\widetilde b_s-\iota a.\]
	Then $b_s-b_{s-1}\in I^{s-1}B$ by construction of $a$, and $\pi(b_s)=\pi(\widetilde b_s)+\pi(\iota a)\equiv c_s\pmod{I^sC}$ by construction of $\widetilde b_s$.

	We now turn to showing exactness at $\widehat A$ and $\widehat B$ in \autoref{eq:desiredses}. We would like to write down some left-exact sequence to take the limit of, and the best thing we can write down is
	\begin{equation}
		0\to A/{\iota^{-1}(I^sB)}\to B/{I^sB}\to C/{I^sC}\to0. \label{eq:almostcompletedses}
	\end{equation}
	To see that this is exact, we note that \autoref{eq:rightexactseq} promises us that
	\[A\to B/I^sB\to C/I^sC\to 0\]
	is right-exact, so it remains to make the map $A\to B/I^sB$ injective, for which we merely mod out by the kernel. Because $\iota\colon A\to B$ is injective, the kernel of $A\to B/I^sB$ can be simply be written as $\iota^{-1}(I^sB)$.

	We even note that \autoref{eq:almostcompletedses} is natural in $s$: if $t\ge s$, then we induce a morphism of short exact sequences in the obvious way as follows.
	% https://q.uiver.app/?q=WzAsMTAsWzAsMCwiMCJdLFsxLDAsIkEvXFxpb3RhXnstMX0oSV50QikiXSxbMiwwLCJCL0ledEIiXSxbMywwLCJDL0ledEMiXSxbNCwwLCIwIl0sWzAsMSwiMCJdLFsxLDEsIkEvXFxpb3RhXnstMX0oSV5zQikiXSxbMiwxLCJCL0lec0IiXSxbMywxLCJDL0lec1MiXSxbNCwxLCIwIl0sWzAsMV0sWzEsMiwiXFxpb3RhIl0sWzIsMywiXFxwaSJdLFszLDRdLFs1LDZdLFs2LDcsIlxcaW90YSJdLFs3LDgsIlxccGkiXSxbOCw5XSxbMyw4LCIiLDEseyJzdHlsZSI6eyJoZWFkIjp7Im5hbWUiOiJlcGkifX19XSxbMiw3LCIiLDEseyJzdHlsZSI6eyJoZWFkIjp7Im5hbWUiOiJlcGkifX19XSxbMSw2LCIiLDEseyJzdHlsZSI6eyJoZWFkIjp7Im5hbWUiOiJlcGkifX19XV0=&macro_url=https%3A%2F%2Fraw.githubusercontent.com%2FdFoiler%2Fnotes%2Fmaster%2Fnir.tex
	\[\begin{tikzcd}
		0 & {A/\iota^{-1}(I^tB)} & {B/I^tB} & {C/I^tC} & 0 \\
		0 & {A/\iota^{-1}(I^sB)} & {B/I^sB} & {C/I^sS} & 0
		\arrow[from=1-1, to=1-2]
		\arrow["\iota", from=1-2, to=1-3]
		\arrow["\pi", from=1-3, to=1-4]
		\arrow[from=1-4, to=1-5]
		\arrow[from=2-1, to=2-2]
		\arrow["\iota", from=2-2, to=2-3]
		\arrow["\pi", from=2-3, to=2-4]
		\arrow[from=2-4, to=2-5]
		\arrow[two heads, from=1-4, to=2-4]
		\arrow[two heads, from=1-3, to=2-3]
		\arrow[two heads, from=1-2, to=2-2]
	\end{tikzcd}\]
	As such, we are actually inducing morphisms of inverse limits, so because inverse limits commute with inverse limits and in particular commute with limits, we get the left-exact sequence
	\[0\to\limit\frac A{\iota^{-1}(I^sB)}\stackrel\iota\to\limit\frac B{I^sB}\stackrel\pi\to\limit\frac C{I^sC}.\]
	To finish, we show that our left term is $\widehat A$. This is an application of the Artin--Rees lemma (!): by \autoref{thm:artinrees}, the filtration
	\[\iota^{-1}(I^sB)\]
	is an $I$-stable filtration. In other words, there is an $n\in\NN$ such that $I^k\iota^{-1}(I^nB)=\iota^{-1}(I^{n+k}B)$ for all $k\in\NN$. So, on one hand, exactness of \autoref{eq:rightexactseq} implies
	\[\iota^{-1}(I^sB)\supseteq I^sA,\]
	but on the other hand, the above argument yields
	\[I^sA\supseteq I^s\iota^{-1}(I^nB)\supseteq\iota^{-1}(I^{s+n}B).\]
	Applying \autoref{lem:samecompletion} finishes.
\end{proof}
\begin{remark}
	We used the condition that $A,B,C$ are finitely generated in the application of \autoref{thm:artinrees}.
\end{remark}
The point of this is that we see completion is an exact functor. In fact, as with localization, there is a flat module hiding in the background.
\completetensor*
\begin{proof}
	We show the parts individually.
	\begin{listalph}
		\item This proof will occur in steps.
		\begin{enumerate}
			\item We define the isomorphism. Note that there is at least a canonical morphism $\eta_M\colon\widehat R_I\otimes_RM\to\widehat M_I$ by
			\[\eta_M\left(\{r_n\}_{n\in\NN}\otimes m\right)\coloneqq\{r_nm\}_{n\in\NN}.\]
			Technically, we ought to show that the map $\widehat R_I\times M\to\widehat M_I$ given by
			\[\left(\{r_n\}_{n\in\NN},m\right)\mapsto\{r_nm\}_{n\in\NN}\]
			is well-defined and $R$-bilinear. We note that this is well-defined because $i>j$ gives $r_i\equiv r_j\pmod{I^j}$ and hence $r_im\equiv r_jm\pmod{I^jM}$, so $\{r_nm\}_{n\in\NN}\in\widehat M_I$. However, we will not bother with checking that this map is $R$-bilinear because look at it.
	
			\item We establish naturality for our morphisms. Given an $R$-module morphism $\varphi\colon A\to B$, we can just check that the following naturality square commutes.
			% https://q.uiver.app/?q=WzAsOCxbMCwwLCJBXFxvdGltZXNfUlxcd2lkZWhhdCBSX0kiXSxbMSwwLCJcXHdpZGVoYXQgQV9JIl0sWzAsMSwiQlxcb3RpbWVzX1JcXHdpZGVoYXQgUl9JIl0sWzEsMSwiXFx3aWRlaGF0IEJfSSJdLFsyLDAsImFcXG90aW1lc1xce3JfblxcfV9uIl0sWzMsMCwiXFx7cl9uYVxcfV9uIl0sWzIsMSwiXFx2YXJwaGkoYSlcXG90aW1lc1xce3JfblxcfV9uIl0sWzMsMSwiXFx7XFx2YXJwaGkocl9uYSlcXH1fbiJdLFswLDEsIlxcZXRhX0EiXSxbMiwzLCJcXGV0YV9CIl0sWzAsMiwiXFx2YXJwaGkiLDJdLFsxLDMsIlxcdmFycGhpIiwyXSxbNCw1LCIiLDIseyJzdHlsZSI6eyJ0YWlsIjp7Im5hbWUiOiJtYXBzIHRvIn19fV0sWzUsNywiIiwyLHsic3R5bGUiOnsidGFpbCI6eyJuYW1lIjoibWFwcyB0byJ9fX1dLFs0LDYsIiIsMCx7InN0eWxlIjp7InRhaWwiOnsibmFtZSI6Im1hcHMgdG8ifX19XSxbNiw3LCIiLDAseyJzdHlsZSI6eyJ0YWlsIjp7Im5hbWUiOiJtYXBzIHRvIn19fV1d&macro_url=https%3A%2F%2Fraw.githubusercontent.com%2FdFoiler%2Fnotes%2Fmaster%2Fnir.tex
			\[\begin{tikzcd}
				{A\otimes_R\widehat R_I} & {\widehat A_I} & {a\otimes\{r_n\}_n} & {\{r_na\}_n} \\
				{B\otimes_R\widehat R_I} & {\widehat B_I} & {\varphi(a)\otimes\{r_n\}_n} & {\{\varphi(r_na)\}_n}
				\arrow["{\eta_A}", from=1-1, to=1-2]
				\arrow["{\eta_B}", from=2-1, to=2-2]
				\arrow["\varphi"', from=1-1, to=2-1]
				\arrow["\varphi"', from=1-2, to=2-2]
				\arrow[maps to, from=1-3, to=1-4]
				\arrow[maps to, from=1-4, to=2-4]
				\arrow[maps to, from=1-3, to=2-3]
				\arrow[maps to, from=2-3, to=2-4]
			\end{tikzcd}\]
			Thus, we have exhibited a natural transformation $\eta_\bullet$ which we would like to upgrade to a natural isomorphism on finitely generated $R$-modules.
			
			\item We show that $\eta_\bullet$ is an isomorphism for finitely generated free $R$-modules $R^n$. Indeed, we can track the sequence of isomorphisms
			\[\arraycolsep=1.4pt\begin{array}{cccccccccccccc}
				R^n\otimes_R\widehat R_I &\simeq& \widehat R_I^n &=& \big(\limit R/I^s\big)^n &\simeq& \limit (R/I^s)^n &\simeq& \limit R^n/I^sR^n &=& \widehat{R^n}_I   \\
				(x_i)_i\otimes\{r_s+I^sR\}_s &\mapsto& && \big(\{r_sx_i+I^sR\}_s\big)_i &\mapsto& \big\{(r_sx_i+I^sR)_i\big\}_s &\mapsto& \big\{r_s(x_i)_i+I^sR^n\big\}_s
			\end{array}\]
			which we can see is $\eta_{R^n}$. Thus, $\eta_\bullet$ is an isomorphism of finitely generated free $R$-modules.
	
			\item Lastly, to show that $\eta_M$ is an isomorphism for arbitrary finitely generated $R$-modules $M$, we note that $R$ being Noetherian implies that $M$ is actually finitely presented, so we can find finitely generated free $R$-modules $F$ and $G$ making the sequence
			\[G\stackrel\iota\to F\stackrel\pi\to M\to 0\]
			right-exact. Now, naturality of $\eta_\bullet$ induces the morphism of right-exact sequences as follows.
			% https://q.uiver.app/?q=WzAsOCxbMCwwLCJHXFxvdGltZXNfUlxcd2lkZWhhdCBSX0kiXSxbMSwwLCJGXFxvdGltZXNfUlxcd2lkZWhhdCBSX0kiXSxbMiwwLCJNXFxvdGltZXNfUlxcd2lkZWhhdCBSX0kiXSxbMCwxLCJcXHdpZGVoYXQgR19JIl0sWzEsMSwiXFx3aWRlaGF0IEZfSSJdLFsyLDEsIlxcd2lkZWhhdCBNX0kiXSxbMywwLCIwIl0sWzMsMSwiMCJdLFswLDEsIlxcaW90YSJdLFsxLDIsIlxccGkiXSxbMiw2XSxbMyw0LCJcXGlvdGEiXSxbNCw1LCJcXHBpIl0sWzUsN10sWzAsMywiXFxldGFfRyJdLFsxLDQsIlxcZXRhX0YiXSxbMiw1LCJcXGV0YV9NIl1d
			\[\begin{tikzcd}
				{G\otimes_R\widehat R_I} & {F\otimes_R\widehat R_I} & {M\otimes_R\widehat R_I} & 0 \\
				{\widehat G_I} & {\widehat F_I} & {\widehat M_I} & 0
				\arrow["\iota", from=1-1, to=1-2]
				\arrow["\pi", from=1-2, to=1-3]
				\arrow[from=1-3, to=1-4]
				\arrow["\iota", from=2-1, to=2-2]
				\arrow["\pi", from=2-2, to=2-3]
				\arrow[from=2-3, to=2-4]
				\arrow["{\eta_G}", from=1-1, to=2-1]
				\arrow["{\eta_F}", from=1-2, to=2-2]
				\arrow["{\eta_M}", from=1-3, to=2-3]
			\end{tikzcd}\]
			Now, the two morphisms on the left are isomorphisms by the previous step, so $\eta_M$ is an isomorphism as well. This completes the proof.
		\end{enumerate}
		\item It might look like we can just combine (a) with \autoref{lem:fgcompletionexactness} to show flatness straight from the definition, but this only gives the exactness when all relevant modules are finitely generated. However, this is no problem: we use \autoref{thm:flatcondition}. Namely, it suffices to show that the natural map
		\[J\otimes_R\widehat R_I\to R\otimes_R\widehat R_I\]
		is injective for all finitely generated $R$-ideals $J\subseteq R$. To show this, we note the short exact sequence
		\[0\to J\stackrel\iota\to R\stackrel\pi\to R/J\to0\]
		gives the following diagram because $R$ and $J$ are finitely generated.
		% https://q.uiver.app/?q=WzAsMTAsWzAsMCwiMCJdLFsxLDAsIkpcXG90aW1lc19SXFx3aWRlaGF0IFJfSSJdLFsyLDAsIlJcXG90aW1lc19SXFx3aWRlaGF0IFJfSSJdLFsxLDEsIlxcd2lkZWhhdCBKX0kiXSxbMiwxLCJcXHdpZGVoYXQgUl9JIl0sWzMsMCwiKFIvSilcXG90aW1lc19SXFx3aWRlaGF0IFJfSSJdLFszLDEsIlxcd2lkZWhhdHtSL0p9X0kiXSxbNCwwLCIwIl0sWzQsMSwiMCJdLFswLDEsIjAiXSxbMCwxXSxbMSwyLCJcXGlvdGEiXSxbMiw1LCJcXHBpIl0sWzUsN10sWzksM10sWzMsNCwiXFxpb3RhIl0sWzQsNiwiXFxwaSJdLFs2LDhdLFsxLDMsIlxcZXRhX0oiXSxbMiw0LCJcXGV0YV9SIl0sWzUsNiwiXFxldGFfe1IvSn0iXV0=&macro_url=https%3A%2F%2Fraw.githubusercontent.com%2FdFoiler%2Fnotes%2Fmaster%2Fnir.tex
		\[\begin{tikzcd}
			0 & {J\otimes_R\widehat R_I} & {R\otimes_R\widehat R_I} & {(R/J)\otimes_R\widehat R_I} & 0 \\
			0 & {\widehat J_I} & {\widehat R_I} & {\widehat{R/J}_I} & 0
			\arrow[from=1-1, to=1-2]
			\arrow["\iota", from=1-2, to=1-3]
			\arrow["\pi", from=1-3, to=1-4]
			\arrow[from=1-4, to=1-5]
			\arrow[from=2-1, to=2-2]
			\arrow["\iota", from=2-2, to=2-3]
			\arrow["\pi", from=2-3, to=2-4]
			\arrow[from=2-4, to=2-5]
			\arrow["{\eta_J}", from=1-2, to=2-2]
			\arrow["{\eta_R}", from=1-3, to=2-3]
			\arrow["{\eta_{R/J}}", from=1-4, to=2-4]
		\end{tikzcd}\]
		Explicitly, the bottom row is exact by \autoref{lem:fgcompletionexactness}, and all vertical morphisms are isomorphisms giving commuting squares by (a), so the top row is exact as well.
	\end{listalph}
	The above parts finish the proof.
\end{proof}
\begin{remark}[Nir]
	It is in fact necessary that $M$ be finitely generated. For example, take $R=\ZZ$ and $I=(p)$ and $M=\QQ$. In this case, $\widehat R_I\otimes_RM=\QQ_p$, but $\widehat M_I=0$ because $M/I^sM=\QQ/\left(p^s\right)\QQ=\QQ/\QQ=0$ for all $s\in\NN$.
\end{remark}

\subsection{Examples of Hensel's Lemma}
Let's continue talking about number theory. Hensel's lemma is a way to lift solutions to polynomial equations from quotients up to complete rings. More precisely, we have the following.
\begin{theorem}[Hensel's lemma] \label{thm:hensel}
	Suppose that $R$ is a ring complete with respect to an $I$-adic filtration, and pick up a polynomial $f(x)\in R[x]$. Now, suppose we have $a\in R$ such that
	\[f(a)\equiv0\pmod{f'(a)^2I}.\]
	Then there exists $b\in R$ such that $b\equiv a\pmod{f'(a)m}$ and $f(b)=0$.
\end{theorem}
We do a few examples before proving the lemma.
\begin{exe}
	We solve for $x\in k\bb{t}$ in the equation $x^2=1+t$, where $k\bb{t}$ is complete with respect to $(t)$.
\end{exe}
\begin{proof}
	Note that $x_0=1$ is a solution in $R/(t)$. So we hope that we can find a solution $u\in k\bb{t}$ such that $u\equiv1\pmod t$ such that $u^2=1+t$. Well, from the general binomial theorem, we can write
	\[\sqrt{1+t}=\sum_{k=0}^\infty\binom{1/2}kt^k.\]
	We can check that this works.
\end{proof}
\begin{exe}
	Fix $a\in\ZZ_p$ for an odd prime $p$. We discuss when we can solve $x^2\equiv a$.
\end{exe}
\begin{proof}
	If $a=0$, then we are done. Otherwise, write $a=bp^n$ where $b\in\ZZ_p\setminus p\ZZ_p$. If $n$ is odd, there is no solution; so we let $n=2k$ and write
	\[\left(x/p^k\right)^2=b,\]
	so we are solving $y^2=b$, where $b\in\ZZ_p\setminus p\ZZ_p$. Now, if a solution is to exist, then we require $b\pmod p$ to be a perfect square, so find $x_0\in\FF_p$ such that 
	\[x_0^2\equiv b\pmod p.\]
	To check that we can lift by Hensel's lemma, we need to check the derivative, but when $p$ is odd, then our derivative is $2x_0$, which is nonzero because $x_0$ is nonzero.

	Let's actually show how we can solve this. Well, expand out $x$ in a $p$-adic series as
	\[\left(\sum_{k=0}^\infty x_kp^k\right)^2=b=:\sum_{k=0}^\infty b_kp^k.\]
	We already have $x_0$. For $x_1$, we check the linear term to find
	\[2x_0x_1\equiv b_1\pmod p,\]
	from which we extract $b_1$. More generally, this term reads as
	\[x_0x_n+\sum_{k=1}^nx_kx_{n-k}=b_n,\]
	from which we can solve for $x_n$ recursively.
\end{proof}
\begin{exe}
	We show that $x^2=b$ has a solution in $\ZZ_2$ if $b$ is an odd perfect square$\pmod8$. In other words, we require $b\equiv1\pmod8$.
\end{exe}
\begin{proof}
	Simply use Hensel's lemma, but now $f'(a)^2\cdot2$ is divisible by a factor of $8$.
\end{proof}

\subsection{Proof of Hensel's Lemma}
With sufficient motivation, we now turn to a proof of \autoref{thm:hensel}. We have the following universal property.
\begin{proposition}
	Fix $S$ an $R$-algebra such that $S$ is complete with respect to an ideal $I\subseteq S$. If $I=(f_1,\ldots,f_n)$ is finitely generated, then there is a unique homomorphism
	\[\varphi:R\bb{x_1,\ldots,x_n}\to S\]
	such that $x_\bullet\mapsto f_\bullet$, and $\varphi$ is continuous under the induced $I$-adic topology. In fact, the following hold.
	\begin{itemize}
		\item If $R\to S/I$ is surjective, then $\varphi$ is surjective.
		\item If the induced map $R[x_1,\ldots,x_n]\to\op{gr}_IS$, then $\varphi$ is injective.
	\end{itemize}
\end{proposition}
\begin{remark}
	This is intended to be an analog for the universal property of polynomial algebras.
\end{remark}
\begin{proof}
	To construct $\varphi$, it suffices to note that $R\bb{x_1,\ldots,x_n}$ is the completion of $R[x_1,\ldots,x_n]$ with respect to the ideal $\mf m=(x_1,\ldots,x_n)$ and then construct a system of maps
	\[\varphi_k:\frac{R[x_1,\ldots,x_n]}{\mf m^k}\to\frac S{I^k}.\]
	Alternatively, we can note that the restricted map on $R[x_1,\ldots,x_n]\to S$ is forced and use continuity to fill in for the rest of $R\bb{x_1,\ldots,x_n}$.

	For the surjectivity check, we note that we can lift to
	\[\varphi_k:\frac{R[x_1,\ldots,x_n]}{\mf m^k}\to\frac S{I^k}\]
	is surjective, so going to the completion provides the result.

	Lastly, we note that the condition tells us that
	\[\bigcap_iI^i=0\implies\bigcap\mf m^i=0.\]
	To finish our injectivity check, we note more generally that if $\varphi:A\to B$ is a map of filtered algebras, then we can build an associated map $\op{gr}\varphi:\op{gr}A\to\op{gr}B$. Then if $\op{gr}\varphi$ is injective (as seen above), then $\varphi$ is also injective. This gives the result after some care.
\end{proof}
\begin{corollary}
	Fix $\varphi:R\bb x\to R\bb x$ some morphism. Further, find $f\in(x)$ such that $f\equiv x\pmod{x^2}$. Then if $\varphi(x)=f$ and $\varphi(r)=r$ for $r\in R$, then $\varphi$ is an isomorphism.
\end{corollary}
\begin{proof}
	Use the previous lemma to construct $\varphi$ and then run the previous surjectivity and injectivity check.
\end{proof}
\begin{remark}
	In fact, there is an explicit inverse map for this $\varphi$.
\end{remark}
We are now ready to prove \autoref{thm:hensel}.
\begin{proof}[Proof of \autoref{thm:hensel}]
	We use Newton's lemma to build our solution $b$. For ease of mind, we set $e:=f'(a)$ so that we know
	\[f(a)\equiv0\pmod{e^2m}.\]
	Now, we can write $f(a+ex)$, which upon expansion via the binomial theorem looks like
	\[f(a+ex)=f(a)+f'(a)ex+h(x)(ex)^2\]
	for some $h\in R[x]$. Using $f'(a)=e$, we get
	\[f(a+ex)=f(a)+e^2\left(x+x^2h(x)\right).\]
	Now, consider the homomorphism $\varphi:R\bb x\to R\bb x$ by $\varphi(x):=x+x^2h(x)$, but the previous corollary tells us that $\varphi$ is an isomorphism! So we see
	\[f\left(a+e\varphi^{-1}(x)\right)=f(a)+e^2x\]
	by plugging in. To finish, we build $\psi:R\bb x\to R$ by $\psi(x)=-c$, where $f(a)=e^2c$ and can compute that
	\[f\left(a+e\psi\varphi^{-1}(x)\right)=f(a)-e^2c=0,\]
	which finishes.
\end{proof}
\begin{remark}
	We can show that the solution above is unique, provided that $f'(a)$ is not a zero-divisor. We will omit this proof.
\end{remark}
% \begin{center}
% 	\begin{asy}
% 		import graph;
% 		unitsize(4cm);
% 		real f(real x)
% 		{
% 			return x*x*x - x + 0.1;
% 		}
% 		draw((-0.6,0)--(2,0)); draw((0,-0.6)--(0,2));
% 		draw(graph(f,-0.6,1.5), blue);

% 		real a0 = 1.3;
% 		real a1 = a0 - (a0*a0*a0 - a0) / (3*a0*a0 - 1);

% 		pair v = (a0, f(a0)) - (a1, 0);
% 		draw((a1, 0) -- (a1, 0) + 1.8*v, red);

% 		draw((a0,0) -- (a0,f(a0)), dashed);
% 		draw((a1,0) -- (a1,f(a1)), dashed);

% 		dot("$x_0$", (a0,0), S); dot("$x_1$", (a1, 0), S);

% 		dot("$(x_0,f(x_0))$", (a0, f(a0)), WNW);
% 		dot("$(x_{1},f(x_{1}))$", (a1, f(a1)), WNW);

% 		label("\color{red}$y-f(x_0)=f'(x_0)(x-x_0)$", (a1,0) + 1.5*v, W);
% 	\end{asy}
% \end{center}