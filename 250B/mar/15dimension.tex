% !TEX root = ../notes.tex

We continue today's lecture by transitioning over to dimension theory.

\subsection{Krull Dimension}
Let's talk about some properties that we want out of our dimension. Here's a starting example.
\begin{defi}[Dimension, vector spaces]
	The dimension of a (finite-dimensional) vector space $V$ is the length of a maximal chain of distinct subspaces
	\[0=V_0\subsetneq V_1\subsetneq\cdots\subsetneq V_n=V.\]
\end{defi}
\begin{remark}
	To see that these align, suppose that $V$ is finite-dimensional so that the definition makes sense. Then we can give $V$ a basis by choosing a vector from each $V_k\setminus V_{k-1}$. We will not check formally that this works.
\end{remark}
What's impressive about this definition is that we have even managed to remove the data of the ground field of our vector space!

In analogy with this, we have the following algebraic definition of dimension.
\begin{restatable}[Krull dimension]{definition}{krulldimdef} \label{def:krulldim}
	The \textit{Krull dimension} of a ring $R$, denoted $\dim R$, is the supremum of the length $r$ of a chain of distinct primes
	\[\mf p_0\subsetneq\mf p_1\subsetneq\cdots\subsetneq\mf p_r.\]
\end{restatable}
Here are some examples to get used to this definition.
\begin{example}
	Fields have dimension $0$.
\end{example}
\begin{example}
	In general, if a Noetherian ring $R$ has $\dim R=0$, then all primes are maximal, so we know from our discussion of modules of finite length that $R$ is Artinian. In fact, we know $R$ is Artinian if and only if all primes are maximal and $R$ is Noetherian.
\end{example}
\begin{example}
	If $R$ is a principal ideal domain which is not a field, then we showed in our proof that principal ideal domains are unique factorization domains that all nonzero prime ideals are maximal. Thus, the largest possible chain in $R$ takes the form
	\[(0)\subsetneq\mf p\subsetneq R,\]
	so $R$ has dimension $1$.
\end{example}
\begin{example}
	The ring $k[x]$ is a principal ideal domain and not a field and therefore has dimension $1$. More generally,
	\[\dim k[x_1,\ldots,x_n]=n,\]
	but we will not prove this yet.
\end{example}

\subsection{Motivating Dimension}
One way to be convinced that \autoref{def:krulldim} is the right definition of dimension is to write down some axioms that we want out of our dimension and try to use these to characterize dimension. Here are some axioms that Eisenbud provides.
\begin{itemize}
	\item We want dimension to be a property determined locally; for example, the dimension of the union of a plane and a line should be $2$ because of the plane. Because localization and completions are intended to be ways to look very locally at a point (geometrically speaking), we ask for
	\[\dim R=\sup_{\mf p\in\op{Spec}R}\dim R_\mf p\qquad\text{and}\qquad\dim R_\mf p=\dim\widehat R_\mf p.\]
	\item We don't want to have to deal with nilpotent elements. In some sense, nilpotent elements correspond to differentials, but they shouldn't affect the dimension of our space. As such, if $I\subseteq R$ is a nilpotent ideal, we will require that
	\[\dim(R/I)=\dim R.\]
	\item Small changes in our base ring should not affect the dimension either. For example, the rings $k$ and $k[x]/\left(x^2+1\right)$ should have the same dimension (they are, roughly speaking, just lines), even though the latter ring is certainly bigger in some sense. To codify this, if $S$ is a finite $R$-algebra containing $R$, we will require that
	\[\dim S=\dim R.\]
	\item We want the dimension of the coordinate ring of affine $n$-space to be $n$; additionally, the dimension should be uniform across all of $n$-space. So after taking completion at $(x_1,\ldots,x_n)$, this amounts to requiring
	\[\dim k[[x_1,\ldots,x_n]]=n.\]
\end{itemize}
It turns out that these properties completely characterize the dimension.

\subsection{Other Characterizations}
Here are some other characterizations of the dimension.
\begin{theorem}
	Fix $R:=k[x_1,\ldots,x_n]/\mf p$ (i.e., $R$ is a finitely generated $k$-algebra), where $\mf p\subseteq k[x_1,\ldots,x_n]$ is a prime so that $R$ is a domain. Then $\dim R$ is equal to the transcendence degree of $K(R)$ over $k$. In fact, in this case, $\dim R$ is the length of all maximal chains of distinct primes.
\end{theorem}
Note that the previous theorem is somewhat agnostic about the case where $R$ is not an integral domain: for example, the ring
\[k[x]\times k[y,z],\]
which can be thought of the ring of functions of the (disjoint) union of a line and a plane. Indeed, in this disjoint union, we wouldn't even expect all maximal chains to have the same length because taking the union along the line should prevent us from being able to see the plane.

We can use the above result in a stronger sense to justify our axioms above.
\begin{theorem}[Noether Normalization]
	Fix $R:=k[x_1,\ldots,x_n]/\mf p$ (i.e., $R$ is a finitely generated $k$-algebra), where $\mf p\subseteq k[x_1,\ldots,x_n]$ is a prime so that $R$ is a domain. Further, fix
	\[\mf p_0\subsetneq\mf p_1\subsetneq\cdots\subsetneq\mf p_n\]
	a maximal chain of primes. Then there exists a subring $S\subseteq R$ such that $S\cong k[x_1,\ldots,x_n]$ and $\mf p_k\cap S=(x_1,\ldots,x_k)$.  
\end{theorem}
In particular, finitely generated $k$-algebras have their dimension intimately connected with some finite subring, akin to our third axiom.

We close with a more computational way to look at the dimension. Recall that if $R$ is a local Noetherian ring with maximal ideal $\mf m$, then we can define its Hilbert function by
\[H_R(n):=\dim_{R/\mf m}\mf m^n/\mf m^{n+1},\]
which we know to be equal to a polynomial $P_R$ for sufficiently large values of $n$. Then the following is true.
\begin{theorem}
	Fix $R$ a local Noetherian ring. Then $\dim R=1+\deg P_R$.
\end{theorem}
We can be convinced of this by running the example $R=k[[x_1,\ldots,x_n]]$.