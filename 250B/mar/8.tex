% !TEX root = ../notes.tex

Let's begin.

\subsection{A Criterion for Flatness}
Let's review the following result.
\begin{theorem} \label{thm:flatcondition}
	An $R$-module $M$ is flat if and only if $\op{Tor}_1^R(R/I,M)=0$ for all finitely generated ideals $I\subseteq R$. Equivalently, $\op{Tor}_1^R(R/I,M)=0$ if and only if the natural map $I\otimes M\to M$ is injective.
\end{theorem} % \todo{restatable}
\begin{proof}
	We proceed with the following steps.
	\begin{enumerate}
		\item From the long exact sequence of $\op{Tor}$, the injectivity of $I\otimes M\to M$ is equivalent to
		\[\op{Tor}_1^R(R/I,M)=0\]
		for all finitely generated ideals $I$. In particular, we use the short exact sequence
		\[0\to I\to R\to R/I\to 0\]
		and consider the long exact sequence we get upon applying $-\otimes_RM$.
		\item We extend to all ideals $I=(x_\alpha)_{\alpha\in\lambda}$. Well, if we have a nontrivial element $x\otimes m$ of the kernel of $I\otimes M\to M$, then we can write
		\[x=\sum_{i=1}^nr_ix_{\alpha_i}\otimes m,\]
		because $x$ only uses finitely many of the $x_\alpha$, so we see that the finitely generated (!) ideal $(x_{\alpha_1},\ldots,x_{\alpha_n})$ will also have a kernel.
		\item We reduce checking that, for all submodule $N'\into N$, we have that $M\otimes_RN'\into M\otimes_RN$, to the case where $N$ is finitely generated. Well, if there is a nontrivial kernel, then we write a nontrivial element of the kernel as
		\[\sum_{i=1}^nm_i\otimes n_i'\mapsto0,\]
		so the same trick lets us assume that both $M$ and $N'$ are finitely generated.

		\item We would like to check that $\op{Tor}_1^R(M,N)=0$ for finitely generated $N$. Well, let $n$ be the minimal number of generators for $N$. For $n=0$, we have $N=0$ and are done. For $n=1$, we use step 1. Then for our induction, we write
		\[0\to N''\to N\to N'\to0\]
		where $N''$ and $N'$ have fewer than $n$ generators. Then the long exact sequence tells us that
		\[\op{Tor}_1^R(M,N'')\to\op{Tor}_1^R(M,N)\to\op{Tor}_1^R(M,N)\]
		is exact, so the induction forces the left and right terms to vanish, so $\op{Tor}_1^R(M,N)=0$.
		\qedhere
	\end{enumerate}
\end{proof}
Let's see some example.
\begin{exe}
	Set $R:=k[x]/\left(x^2\right)$. We show that an $R$-module $M$ is flat if and only if $M$ is free.
\end{exe}
\begin{proof}
	We use \autoref{thm:flatcondition}. Certainly if $I=R$ or $I=0$ we are done. The only other ideal to check is $(x)$, so we just need to verify that
	\[(x)M\to M\]
	is injective. In particular, we need to check that $\ker(x)/\im(x)=0$, which amounts to verifying that $M$ is free because $x$ is the only element that could provide us with a kernel.
\end{proof}
\begin{remark}[Serganova]
	In fact, we can show that any $R$-module $M$ can be written as $M_0\oplus F$ where $M_0\cong\ker(x)/\im(x)$ and $F$ is free.
\end{remark}
\begin{exe}
	Fix $R$ a principal ideal domain. Then an $R$-module $M$ is flat if and only if $M$ is torsion-free.
\end{exe}
\begin{proof}
	All ideals take the form $(a)$, but $(a)\cong R$ because $R$ is a principal ideal domain, so we are merely verifying that the map $R\otimes_RM\to M$ is injective, which is true if and only if $M$ is torsion free.
\end{proof}
\begin{example}
	If a $\ZZ$-module $M$ is finitely generated and torsion free, then $M$ must actually be free. Also, $\QQ$ is torsion free and hence flat.
\end{example}

\subsection{Flatness Locally}
We note the following.
\begin{lemma}
	Fix $R$ a ring and $\mf p$ a prime. If $M$ is a flat $R$-module, then $M_\mf p$ is a flat $R_\mf p$-module.
\end{lemma}
\begin{proof}
	Use the $\op{Tor}$ condition for flatness and note that $R_\mf p$ being flat allows us to simply tensor in the projective resolution for $M$ to give a projective resolution for $M_\mf p$. Alternatively, simply note that $R_\mf p\otimes_RM$ is the tensor product of two flat modules.
\end{proof}
We might hope the converse holds. Indeed, it does.
\begin{proposition}
	Fix $R$ a ring and $M$ an $R$-module. If $M_\mf p$ is flat for all primes $\mf p$, then $M$ is also flat.
\end{proposition}
\begin{proof}
	Well, fix some inclusion $N\subseteq N'$ so that we want to show that
	\[M\otimes_R N'\to M\otimes_RN\]
	is also an inclusion. Well, we know that, upon localization, we have an inclusion
	\[M_\mf p\otimes_{R_\mf p}N'_\mf p\to M_\mf p\otimes_{R_\mf p}N_\mf p,\]
	so because this is an inclusion locally, it becomes an inclusion globally as we showed a while ago.
\end{proof}
So we are motivated to study how flat modules behave under localization.
\begin{prop}
	Fix $R$ a local ring with maximal ideal $\mf p$. Further, let $M$ be a finitely presented $R$-module. If $M$ is flat, then $M$ is free.
\end{prop}
\begin{proof}
	The idea is to use Nakayama's lemma. Because $M$ is finitely presented, we can build a short exact sequence
	\[0\to N\to F\to M\to 0,\]
	where $N$ is finitely generated, and $F$ is free. Upon tensoring with $R/\mf p$, we get the right-exact sequence
	\[N/\mf pN\to F/\mf pF\to M/\mf pM\to 0.\]
	Now, choose $F$ such that $\dim F/\mf pF=\dim M/\mf pM$, for otherwise we could use Nakayama's lemma to generate $F$ by fewer (namely, $\dim F/\mf pF$ many) elements.
	
	It follows that $N/\mf pN$ must vanish, so because $N$ is finitely generated, Nakayama's lemma promises that $N=0$. Thus, $F=M$, so $M$ is free, so we are done.
\end{proof}
\begin{remark}
	In fact, any projective module over a local ring is free. The case for finitely generated modules is on the homework. The homework also includes some examples of flatness checks.
\end{remark}

\subsection{Completions, Algebraically}
We shift gears to talk about completion. The idea is to give a ring a topology by more or less by thinking about a filtration like a fundamental system of neighborhoods around the identity.
\begin{definition}[Completion, rings]
	Fix $R$ with a filtration $\mathcal J$ given by
	\[R=I_0\supseteq I_1\supseteq I_2\supseteq\cdots.\]
	Then we define the \textit{completion $\widehat R_\mathcal J$} as a subring of $\prod_{s\in\NN}R/\mf I_s$ by
	\[\widehat R_\mathcal J=\left\{(r_0,r_1,\ldots)\in\prod_{s\in\NN}R/I_s:r_i\equiv r_j\pmod{I_j}\text{ for }i>j\right\}.\]
\end{definition}
As usual, the most interesting filtrations for us will be the $I$-adic filtrations, especially when $I$ is a maximal ideal.

We have the following check.
\begin{lemma}
	Fix $R$ with a filtration $\mathcal J$. Then $\widehat R_\mathcal J$ is a ring.
\end{lemma}
\begin{proof}
	For concreteness, label our filtration $\mathcal J$ by
	\[R=I_0\supseteq I_1\supseteq I_2\supseteq\cdots.\]
	The ring structure of $\widehat R_\mathcal J$ is inherited from the product $\prod_{s\in\NN}R/\mf I_s$, so we merely have to run checks for being a subring.
	\begin{itemize}
		\item We note that the identity element of $\prod_{s\in\NN}R/I_s$ is $\{[1]_{I_s}\}_{s\in\NN}$. To see that this is in $\widehat R_\mathcal J$, we see that
		\[[1]_{I_n}\equiv[1]_{I_m}\pmod{I_m}\]
		for $n>m$.
		\item We show closure under addition and multiplication. Indeed, if $\{r_n\}_{n\in\NN},\{s_n\}_{n\in\NN}\in\widehat R_\mathcal J$, then the sum is
		\[\{r_n+s_n\}_{n\in\NN}\]
		To see that this lives in $\widehat R_\mathcal J$, we see that $n>m$ has $r_s\equiv r_m\pmod{I_m}$ and $s_n\equiv s_m\pmod{I_m}$, so
		\[r_n+s_n\equiv r_m+s_m\pmod{I_m}\qquad r_ss_n\equiv r_ms_m\pmod{I_m},\]
		which finishes.
		\item We show closure under negation. Indeed, if $\{r_n\}_{n\in\NN}\in\widehat R_\mathcal J$, then $r_n\equiv r_m\pmod{I_m}$ for $n>m$, so
		\[-r_n\equiv-r_m\pmod{I_m},\]
		so $\{-r_n\}_{n\in\NN}$ lives in $\widehat R_\mathcal J$. This is our negative element in $\prod_{s\in\NN}R/I_s$, so we are done.
	\end{itemize}
	The above checks show that we have a subring.
\end{proof}

Let's see some examples.
\begin{exe}
	Fix $R:=k[x]$ and $\mf m:=(x)$ a maximal ideal. We claim $\widehat R_\mf m\cong k[[x]]$.
\end{exe}
\begin{proof}
	We see $\widehat R_\mf m$ consists of sequences of polynomials $\{p_n(x)\}_{n\in\NN}$ such that
	\[p_n\equiv p_m\pmod{x^m}\]
	for $n>m$. To be explicit, we set
	\[p_n(x):=\sum_{k=0}^\infty a_{n,k}x^k\]
	so that we know
	\[\sum_{k=0}^\infty a_{n,k}x^k=p_n(x)\equiv p_m(x)=\sum_{k=0}^\infty a_{m,k}x^k\pmod{x^m}.\]
	In particular, we see that are forced to have $a_{n,k}=a_{m,k}$ for $k<m$ because the above equation tells us that $x^m$ divides
	\[\sum_{k=0}^{m-1}(a_{n,k}-a_{m,k})x^k,\]
	which can only be possible if all these coefficients vanish by degree arguments.

	We are ready to construct our isomorphism $\widehat R_\mf m\cong k[[x]]$: we take the above sequence $\{p_n(x)\}_{n\in\NN}$. We set
	\[\varphi\left(\{p_n(x)\}_{x\in\NN}\right):=\sum_{k=0}^\infty a_{k+1,k}x^k.\]
	This map is of course well-defined. Our inverse map is
	\[\psi\left(\sum_{k=0}^\infty a_kx^k\right)=\left\{\sum_{k=0}^{n-1}a_kx^k\right\}_{n\in\NN}.\]
	We run the following checks.
	\begin{itemize}
		\item We check that $\psi$ is well-defined. Indeed, if $n>m$, then we see
		\[\sum_{k=0}^{n-1}a_kx^k\equiv\sum_{k=0}^{m-1}a_kx^k\pmod{x^m},\]
		which is what we need to live $\widehat R_\mf m$.
		\item We check $\varphi\circ\psi$ is the identity. Indeed, we see that
		\[\varphi\left(\psi\left(\sum_{k=0}^\infty a_kx^k\right)\right)=\varphi\Bigg(\Bigg\{\sum_{k=0}^{n-1}a_kx^k\Bigg\}_{n\in\NN}\Bigg)=\sum_{k=0}^\infty a_{k,k+1}x^k,\]
		but $a_{k,k+1}=a_k$ by construction, so we are done.
		\item We check $\psi\circ\varphi$ is the identity. Indeed, we see that
		\[\psi\left(\varphi\left(\Bigg\{\sum_{k=0}^\infty a_{n,k}x^k\Bigg\}_{n\in\NN}\right)\right)=\psi\left(\sum_{k=0}^\infty a_{k+1,k}x^k\right)=\left\{\sum_{k=0}^{n-1}a_{k+1,k}x^k\right\}_{n\in\NN}.\]
		To see that this is identity, we need to show that
		\[\sum_{k=0}^\infty a_{n,k}x^k\equiv\sum_{k=0}^{n-1}a_{k+1,k}x^k\pmod{x^n},\]
		for which we have to show that $a_{n,k}=a_{k+1,k}$ for $k<n$. But by definition of $\widehat R_\mf m$, we see that
		\[\sum_{k=0}^\infty a_{n,k}x^k\equiv\sum_{k=0}^\infty a_{k+1,k}x^k\pmod{x^{k+1}},\]
		so we get the desired result upon comparing the $x^k$ term above.
		\item We check that $\psi$ preserves addition. This is a matter of force. We write
		\begin{align*}
			\varphi\left(\sum_{k=0}^\infty a_kx^k+\sum_{k=0}^\infty b_kx^k\right) &= \varphi\left(\sum_{k=0}^\infty(a_k+b_k)x^k\right) \\
			&= \left\{\sum_{k=0}^{n-1}(a_k+b_k)x^k\right\}_{n\in\NN} \\
			&= \left\{\sum_{k=0}^{n-1}a_kx^k\right\}_{n\in\NN}+\left\{\sum_{k=0}^{n-1}b_kx^k\right\}_{n\in\NN} \\
			&= \varphi\left(\sum_{k=0}^\infty a_kx^k\right)+\varphi\left(\sum_{k=0}^\infty b_kx^k\right).
		\end{align*}
		\item We check that $\psi$ preserves multiplication. Again, this is a matter of force. We write
		\begin{align*}
			\varphi\left(\sum_{k=0}^\infty a_kx^k\cdot\sum_{k=0}^\infty b_kx^k\right) &= \varphi\left(\sum_{m=0}^\infty\sum_{k+\ell=m}(a_kb_\ell)x^m\right) \\
			&= \left\{\sum_{m=0}^{n-1}\sum_{k+\ell=m}(a_kb_\ell)x^m\right\}_{n\in\NN} \\
			&= \left\{\sum_{m=0}^{n-1}a_kx^k\cdot\sum_{\ell=0}^{n-1}a_\ell x^\ell\right\}_{n\in\NN},
		\end{align*}
		where in the last equality we have used the fact that the terms of degree at least $x^n$ will vanish in the $n$th term. Continuing we see
		\begin{align*}
			\varphi\left(\sum_{k=0}^\infty a_kx^k\cdot\sum_{k=0}^\infty b_kx^k\right) &= \left\{\sum_{k=0}^{n-1}a_kx^k\right\}_{n\in\NN}\cdot\left\{\sum_{k=0}^{n-1}b_kx^k\right\}_{n\in\NN} \\
			&= \varphi\left(\sum_{k=0}^\infty a_kx^k\right)\varphi\left(\sum_{k=0}^\infty b_kx^k\right).
		\end{align*}
		\item We check that $\psi$ preserves identities. Well, we note $\psi(1)=\{1\}_{n\in\NN}$.
	\end{itemize}
	The first three checks shows that $\psi$ is bijective, and the last three checks show that $\psi$ is a homomorphism. So we are done.
\end{proof}
% \begin{example}
% 	Fix $R:=k[x]$ with $\mf m:=(x)$ a maximal ideal. Then 
% 	for $n>m$. These sequences just define formal power series, where the last $n$ terms of the power series are determined by $p_n$, and the coherence above guarantees that this is well-defined.
% \end{example}
\begin{example}
	Fix $R:=\ZZ$ with $\mf m:=(p)$. Then the ring $\widehat R_\mf m$ is called the $p$-adic integers, more commonly denoted $\ZZ_p$. This ring consists of sequences $\{b_n\}_{n\in\NN}$ which behave as ``formal power series'' in $p$ in the following way. Note $b_{n+1}-b_n\equiv0\pmod{p^n}$, so we can set $a_n:=\frac{b_{n+1}-b_n}{p^n}\in\ZZ/p\ZZ$; in particular, we see that
	\[b_n=\sum_{k=0}^{n-1}(b_{k+1}-b_k)=\sum_{k=0}^{n-1}a_kp^k.\]
	Taking $n\to\infty$ recovers a power series in $p$.
\end{example}
\begin{example}
	The $2$-adic integer $u\in\ZZ_2$ given by
	\[u:=1+2+2^2+2^3+\cdots\]
	is actually just $-1$. To be explicit, we set
	\[u_n\equiv\sum_{k=0}^{n-1}2^k\pmod{2^n}\]
	and consider the sequence $\{u_k\}_{k\in\NN}$; notably, $u_n\equiv u_m\pmod{2^m}$ for $n>m$ by simply expanding out $u_n$ and $u_m$. Now, if we multiply $(1-2)u$, then we see that
	\[(1-2)u_n=(1-2)\left(\sum_{k=0}^{n-1}2^k\right)=1-2^n\equiv1\pmod{2^n}.\]
	Thus, $\{(1-2)u_n\}_{n\in\NN}=\{1\}_{n\in\NN}$, so $(1-2)u=1$. After rearranging, we see that $u=-1$.
\end{example}

\subsection{Complete Rings}
In the previous examples, we might have noticed that there are natural inclusions $k[x]\subseteq k[[x]]$ and $\ZZ\subseteq\ZZ_2$. More generally, we have the following.
\begin{lemma} \label{lem:naturalcompletionmap}
	Fix $R$ a ring and $\mathcal J$ a filtration. Then there is a natural map $R\to\widehat R_\mathcal J$ by $r\mapsto\{r\}_{n\in\NN}$. If $\mathcal J$ is the $I$-adic filtration for a proper ideal $I$, and $R$ is local or a domain, then this map is injective.
\end{lemma}
\begin{proof}
	For now, denote the filtration $\mathcal J$ by
	\[R=I_0\supseteq I_1\supseteq I_2\supseteq\cdots.\]
	Let $\iota:R\to\widehat R_\mathcal J$ by $r\mapsto\{r\}_{n\in\NN}$. To see that $\iota$ is well-defined, we note that
	\[r\equiv r\pmod {I_s}\]
	for any ideal $I_s$ in the filtration $\mathcal J$, so we are done. To see that $\iota$ is a ring homomorphism, note that we have small ring homomorphisms
	\[\pi_s:R\onto R/I_s\]
	by $r\mapsto[r]_{I_s}$, which will glue into a larger map
	\[\pi:R\to\prod_{s\in\NN}R/I_s\]
	by $r\mapsto\{r\}_{s\in\NN}$. In particular, we see that $\iota=\pi$, and we are promised that $\pi$ is a ring homomorphism by universal property, so we are done.

	Now take $I$ to be a proper ideal and fix $I_s:=I^s$ so that $\mathcal J$ is the $I$-adic filtration. We show that $\iota$ is injective. Indeed, if $\iota(r)=0$, then we have that
	\[r\equiv0\pmod{I^s}\]
	for all $s$. In particular,
	\[r\in\bigcap_{s\in\NN}I^s.\]
	Thus, if $R$ is local or a domain, we see that \autoref{cor:krullintersect} forces $r=0$, so indeed, $\ker\iota$ is trivial, making $\iota$ injective.
\end{proof}
Notably, the image of an ideal need not be an ideal; for example, the inclusion $\ZZ\into\QQ$ does not map the ideal $\ZZ\subseteq\ZZ$ to an ideal of $\QQ$.

The above inclusion gives rise to the following definition, which is our motivation for the word ``completion.''
\begin{definition}[Complete]
	Fix a filtration $\mathcal J$ and a ring $R$. Then the ring $R$ is \textit{complete with respect to $\mathcal J$} if and only if $\widehat R_\mathcal J=R$, in that the natural map $\iota:R\to\widehat R_\mathcal J$ is an isomorphism.
\end{definition}
Before actually showing that the completion is complete, we need to talk about which filtration we are complete with respect to.
\begin{notation}
	Fix $R$ a ring and $\mathcal J$ a filtration. Then, given an ideal $I\subseteq R$ of the filtration $\mathcal J$, we let $\widehat I$ denote the kernel of the projection map (onto the $I$th coordinate) $\widehat R_\mathcal J\onto R$.
\end{notation}
Note that $\widehat R_\mathcal J\onto R$ is indeed surjective because $\iota(r)$ maps to $r$ under this projection.
\begin{remark}
	One might wish that we had used the ideal generated by $I$ under the natural map $R\to\widehat R_\mathcal J$ instead. We will show these notions coincide when $R$ is Noetherian later.
\end{remark}
Anyways, here is our result.
\begin{lemma}
	Fix $R$ a ring and $\mathcal J$ a filtration denoted by
	\[R=I_0\supseteq I_1\supseteq I_2\supseteq\cdots.\]
	Then the completion $\widehat R:=\widehat R_\mathcal J$ is a complete ring with respect to the induced filtration $\widehat{\mathcal J}$ given by
	\[\widehat R=\widehat{I_0}\supseteq \widehat{I_1}\supseteq \widehat{I_2}\supseteq\cdots.\]
\end{lemma}
\begin{proof}
	We very quickly check that $n>m$ implies $I_n\subseteq I_m$, so the kernel of $\widehat R\to R/I_n$ is smaller than the kernel of $\widehat R\to R/I_m$, so $\widehat{I_n}\subseteq\widehat{I_m}$. So indeed, $\widehat{\mathcal J}$ is indeed a filtration.

	Now, for notational ease, set $S:=\widehat R$ and $\widehat S:=\widehat S_{\widehat{\mathcal J}}$ its completion so that  we are showing that the map $\widehat\iota:S\to\widehat S$ is an isomorphism. We already know that $\widehat\iota$ is a ring homomorphism by \autoref{lem:naturalcompletionmap}, so it suffices to show that $\widehat\iota$ is a bijection.

	Well, by definition of $\widehat{I_s}$, we see that
	\[\pi_s:\widehat R/\widehat{I_s}\to R/I_s\]
	by $\{r_n\}_{n\in\NN}\mapsto[r_s]_{I_s}$ is an isomorphism. On the other hand, the map $\iota:R\to\widehat R/\widehat{I_s}$ takes $r$ to $\{[r]\}_{n\in\NN}$. In particular,
	\[\pi_s(\iota(r))=[r]_{I_s},\]
	so $I_s\subseteq\ker(\pi_s\circ\iota)$, so $\pi_s\circ\iota$ will induce a map $R/I_s\to R/I_s$ which is the identity. Because $\pi_s$ is an isomorphism, we see that $\ker\iota$ must be $I_s$, so we get an induced map $\iota_s:R/I_s\to\widehat R/\widehat{I_s}$ which is inverse to $\pi_s$ and therefore an isomorphism.

	We now glue the $\iota_s$ together to show $\widehat\iota$ is a bijection. Notably, they glue together to give us a bijection
	\[\varphi:\prod_{s\in\NN}R/I_s\to\prod_{s\in\NN}\widehat R/\widehat{I_s}\]
	by $\varphi:\{[r_s]\}_{s\in\NN}\mapsto\{[\iota(r_s)]_{\widehat{I_s}}\}_{s\in\NN}$. In particular, restricting to $\widehat R$, we see that the $n$th component comes out to
	\[\iota(r_n)\stackrel*\equiv\{[r_s]\}_{s\in\NN}=\left(\widehat\iota\left(\{[r_s]\}_{s\in\NN}\right)\right)_n\pmod{\widehat{I_n}},\]
	where we have to check $\stackrel*\equiv$ by hand on coordinates: for $m<n$, we have $r_n\equiv r_m\pmod{I_m}$, so $[r_n]=[r_m]$ here. For $m=n$, there is nothing to say, and for $m>n$, we note that $\iota(r_n)-\{[r_s]\}_{s\in\NN}$ now has a zero component in the $I_n$ term, so it lives in the kernel of the map $\widehat R\to R/I_n$, so we get our equivalence$\pmod{\widehat{I_n}}$.

	In particular, the glued map $\varphi$ is equal to $\widehat\iota$. Because $\varphi$ was injective, it follows that $\widehat\iota$ is also injective. We now show that $\widehat\iota$ is surjective. Well, given an element of
	\[\widehat S\subseteq\prod_{s\in\NN}\widehat R/\widehat{I_s},\]
	we note that it will have exactly one pull-back along $\varphi$, and $\varphi|_{\widehat R}=\widehat\iota$ as shown above, so it suffices to show that this pull-back is an element of $\widehat R$.
	
	Well, fix our element $\{\hat r_s\}_{s\in\NN}\in\widehat S$, where $\hat r_s=\{r_{s,q}\}_{q\in\NN}\in\widehat R$. The inverse map of $\varphi$ is made by gluing together the inverse maps of $\iota_s$, but we know these maps as $\pi_s$. Thus,
	\[\varphi^{-1}\left(\{\hat r_s\}_{s\in\NN}\right)=\{\iota_s^{-1}(\hat r_s)\}_{s\in\NN}=\{\pi_s(\hat r_s)\}_{s\in\NN}=\{r_{s,s}\}_{s\in\NN}\]
	To show that this is a well-defined element of $\widehat R$, it remains to show that $n>m$ has $r_{n,n}\equiv r_{m,m}\pmod{I_m}$. Well, at the very least we know that
	\[\{r_{n,q}\}_{n\in\NN}-\{r_{m,q}\}_{m\in\NN}\in\widehat{I_m}\]
	because $\{\hat r_s\}_{s\in\NN}\in\widehat S$, so we project $\{r_{n,q}\}_{n\in\NN}-\{r_{m,q}\}_{m\in\NN}$ onto the $m$th coordinate to be forced to have
	\[r_{n,m}\equiv r_{m,m}\pmod{I_m}\]
	by definition of $\widehat{I_m}$. However, $r_{n,n}\equiv r_{n,m}\pmod{I_n}$ because $\hat r_n\in\widehat R$, so we are done.
	% \begin{lemma}
	% 	Fix $R$ a ring and $\mathcal J$ a filtration denoted by
	% 	\[R=I_0\supeteq I_1\supseteq I_2\supseteq\cdots.\]
	% 	Then the natural map $\iota:R\to\widehat R_\mathcal J$ induces an isomorphism $\iota_s:R/I_s\to\widehat R_\mathcal J/\widehat{I_s}$ for any $s\in\NN$.
	% \end{lemma}
	% \begin{proof}
	% 	Set $\pi$ to be the composite
	% 	\[R\to\widehat R_\mathcal J/\widehat{I_s}.\]
	% 	We need to show that $\pi$ is surjective and has kernel $I_s$.
	% 	\begin{itemize}
	% 		\item We show that $\pi$ is surjective. Indeed, fix some $\{r_n\}_{n\in\NN}\in\widehat R_\mathcal J$ whose coset we want to represent. Well, we simply fix $r:=r_{s-1}$. Notably,
	% 		\[r=r_{s-1}\equiv r_k\pmod{I_k}\]
	% 		for each $k<s$, so we match in the first $s$ coordinates. It remains to show that $\{r_n\}_{n\in\NN}-\iota(r)\in\widehat{I_s}$.
	% 	\end{itemize}
	% \end{proof}
	% Fix some $a$ in the completion of $\widehat R_{\widehat {\mathcal J}}$.
\end{proof}

\subsection{Completions, Topologically}
Let's have a little fun with our completions.
\begin{definition}[Limit]
	Fix a completion $\widehat R_\mathcal J$. Then we say that an element $a\in\widehat R_\mathcal J$ is the \textit{limit} of a sequence $\{a_k\}_{k\in\NN}$ if and only if, for each $n$, there exists $N$ such that $a_i-a\in I_n$ for each $i>N$.
\end{definition}
\begin{definition}[Cauchy sequence]
	Fix a completion $\widehat R_\mathcal J$. Then a sequence $\{a_k\}_{k\in\NN}$ is \textit{Cauchy} if and only if, for each $n$, there exists $N$ such that $a_i-a_j\in I_n$ for each $i,j>N$.
\end{definition}
So the following is why we used the term ``complete.''
\begin{lemma}
	A ring $R$ is complete with respect to the filtration $\mathcal J$ if and only if every Cauchy sequence in $\widehat R$ converges.
\end{lemma}
\begin{proof}
	Omitted.
\end{proof}
As an aside, we note that, if we have a sequence $a:=\{a_k\}_{k\in\NN}\in\widehat R_\mathcal J$, then we have $a_{i+1}-a_i\in\mf I_n$, so we can write
\[a=\sum_{j=1}^\infty(a_{j+1}-a_j)\]
to be an infinite convergent series; in particular, the partial sums converge.

\subsection{Properties Preserved by Completion}
Taking the completion also tends to look like localization.
\begin{proposition} \label{prop:completelocalize}
	Fix $R$ complete with respect to an $I$-adic filtration. Then any element $1-a$ with $a\notin I$ is a unit.
\end{proposition}
\begin{proof}
	We see that
	\[(1-a)^{-1}=1+a+a^2+\cdots\]
	will work, essentially by just summing an infinite geometric series.
\end{proof}
\begin{corollary}
	Fix $R$ a ring and $\mf m$ a maximal ideal. Then $\widehat R_\mf m$ is a local ring with maximal ideal $\mf m\widehat R_\mf m$.
\end{corollary}
\begin{proof}
	We need to check that $a\notin\mf m\widehat R_\mf m$ implies that $a$ is a unit. Well, we can write out
	\[a=\sum_{k=0}^nb_k\]
	where $b_k\in\mf m^k$. We can select $b$ so that $ab\equiv1\pmod{\mf m}$ because $R/\mf m$ is a field, so we see that $ab=1+c$ for some $c\in\mf m$. But now \autoref{prop:completelocalize} assures us that $1+c$ is a unit, so $a$ must be a unit also.
\end{proof}
In fact, we can see how similar completion looks like localization by how it focuses on $\mf m$.
\begin{lemma}
	We have that $R/\mf m^n\cong\widehat R/\mf m\widehat R_\mf m^n$.
\end{lemma}
\begin{proof}
	This is by definition of $\mf m\widehat R_\mf m^n$.
\end{proof}
And now we can glue these isomorphisms together.
\begin{corollary}
	We have that $\op{gr}_\mf mR\cong\op{gr}_{\mf m\widehat R_\mf m}\widehat R_\mf m$.
\end{corollary}
\begin{proof}
	Glue together the isomorphisms of the preceding result. The multiplicative structure matches.
\end{proof}
We can also say the following.
\begin{theorem}
	Fix $R$ a Noetherian ring and $\mf m\subseteq R$ an ideal. Then $\widehat R_\mf m$ is Noetherian.
\end{theorem}
\begin{proof}
	We need the following lemma.
	\begin{lemma}
		Fix $R$ a ring complete with respect to the filtration $\mathcal J$ denoted by
		\[R=I_0\supseteq I_1\supseteq I_2\supseteq\cdots.\]
		Further, if we have $I=(a_1,\ldots,a_n)$, then $(\op{in}I)=(\op{in}a_1,\ldots,\op{in}a_n)$.
	\end{lemma}
	\begin{proof}
		Suppose for the sake of contradiction that we have some element $f\notin(\op{in}a_1,\ldots,\op{in}a_n)$. But now, we can write
		\[f=\sum_{i=1}^na_ig_i\]
		for some elements $g_i$, so if $f\in I^d\setminus I^{d+1}$, then we have
		\[\op{in}f=\Bigg[\sum_{i=1}^na_kg_i\Bigg]_{\mf m^{d+1}}.\]
		Now, lifting the elements $g_i$ to the completion, we hit our contradiction because the above should be an equality.
	\end{proof}
	We now attack the proof of the theorem. Well, fix some ideal $I\subseteq\widehat R_\mf m$. Then we project $I$ into $\op{gr}_\mf mR$, which is equal to $\op{gr}_{\mf m\widehat R_\mf m}\widehat R_\mf m$, and here we see that $I$ must be finitely generated because $\op{gr}_\mf mR$ needs to be Noetherian as lifted from $R$, by the lemma. So we are done.
\end{proof}
We close by stating the following theorem.
\begin{definition}[Completion, modules]
	Fix $R$ a ring with filtration $\mathcal J$ denoted by
	\[R=I_0\supseteq I_1\supseteq I_2\supseteq\cdots.\]
	Further, given an $R$-module $M$, there is an induced filtration
	\[M=I_0M\supseteq I_1M\supseteq I_2M\supseteq\cdots.\]
	Doing the same construction as for $\widehat R$ gives rise to $\widehat R$.
\end{definition}
We can check that $\widehat M$ is an $\widehat R$-module.

We have the following, which says that completion is really looking like localization.
\begin{theorem}
	Fix $R$ a Noetherian ring with ideal $I$. Given a finitely generated module $M$, we have
	\[\widehat M_\mf m\cong\widehat R_\mf m\otimes_RM.\]
	In fact, $\widehat R_\mf m$ is a flat $R$-module.
\end{theorem}
\begin{proof}
	We will show this next time.
\end{proof}