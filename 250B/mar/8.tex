% !TEX root = ../notes.tex

Let's begin.

\subsection{A Criterion for Flatness}
Let's review the following result.
\begin{theorem} \label{thm:flatcondition}
	An $R$-module $M$ is flat if and only if $\op{Tor}_1^R(R/I,M)=0$ for all finitely generated ideals $I\subseteq R$. Equivalently, $\op{Tor}_1^R(R/I,M)=0$ if and only if the natural map $I\otimes M\to M$ is injective.
\end{theorem} % \todo{restatable}
\begin{proof}
	We proceed with the following steps.
	\begin{enumerate}
		\item From the long exact sequence of $\op{Tor}$, the injectivity of $I\otimes M\to M$ is equivalent to
		\[\op{Tor}_1^R(R/I,M)=0\]
		for all finitely generated ideals $I$. In particular, we use the short exact sequence
		\[0\to I\to R\to R/I\to 0\]
		and consider the long exact sequence we get upon applying $-\otimes_RM$.
		\item We extend to all ideals $I=(x_\alpha)_{\alpha\in\lambda}$. Well, if we have a nontrivial element $x\otimes m$ of the kernel of $I\otimes M\to M$, then we can write
		\[x=\sum_{i=1}^nr_ix_{\alpha_i}\otimes m,\]
		because $x$ only uses finitely many of the $x_\alpha$, so we see that the finitely generated (!) ideal $(x_{\alpha_1},\ldots,x_{\alpha_n})$ will also have a kernel.
		\item We reduce checking that, for all submodule $N'\into N$, we have that $M\otimes_RN'\into M\otimes_RN$, to the case where $N$ is finitely generated. Well, if there is a nontrivial kernel, then we write a nontrivial element of the kernel as
		\[\sum_{i=1}^nm_i\otimes n_i'\mapsto0,\]
		so the same trick lets us assume that both $M$ and $N'$ are finitely generated.

		\item We would like to check that $\op{Tor}_1^R(M,N)=0$ for finitely generated $N$. Well, let $n$ be the minimal number of generators for $N$. For $n=0$, we have $N=0$ and are done. For $n=1$, we use step 1. Then for our induction, we write
		\[0\to N''\to N\to N'\to0\]
		where $N''$ and $N'$ have fewer than $n$ generators. Then the long exact sequence tells us that
		\[\op{Tor}_1^R(M,N'')\to\op{Tor}_1^R(M,N)\to\op{Tor}_1^R(M,N)\]
		is exact, so the induction forces the left and right terms to vanish, so $\op{Tor}_1^R(M,N)=0$.
		\qedhere
	\end{enumerate}
\end{proof}
Let's see some example.
\begin{exe}
	Set $R:=k[x]/\left(x^2\right)$. We show that an $R$-module $M$ is flat if and only if $M$ is free.
\end{exe}
\begin{proof}
	We use \autoref{thm:flatcondition}. Certainly if $I=R$ or $I=0$ we are done. The only other ideal to check is $(x)$, so we just need to verify that
	\[(x)M\to M\]
	is injective. In particular, we need to check that $\ker(x)/\im(x)=0$, which amounts to verifying that $M$ is free because $x$ is the only element that could provide us with a kernel.
\end{proof}
\begin{remark}[Serganova]
	In fact, we can show that any $R$-module $M$ can be written as $M_0\oplus F$ where $M_0\cong\ker(x)/\im(x)$ and $F$ is free.
\end{remark}
\begin{exe}
	Fix $R$ a principal ideal domain. Then an $R$-module $M$ is flat if and only if $M$ is torsion-free.
\end{exe}
\begin{proof}
	All ideals take the form $(a)$, but $(a)\cong R$ because $R$ is a principal ideal domain, so we are merely verifying that the map $R\otimes_RM\to M$ is injective, which is true if and only if $M$ is torsion free.
\end{proof}
\begin{example}
	If a $\ZZ$-module $M$ is finitely generated and torsion free, then $M$ must actually be free. Also, $\QQ$ is torsion free and hence flat.
\end{example}

\subsection{Flatness Locally}
We note the following.
\begin{lemma}
	Fix $R$ a ring and $\mf p$ a prime. If $M$ is a flat $R$-module, then $M_\mf p$ is a flat $R_\mf p$-module.
\end{lemma}
\begin{proof}
	Use the $\op{Tor}$ condition for flatness and note that $R_\mf p$ being flat allows us to simply tensor in the projective resolution for $M$ to give a projective resolution for $M_\mf p$. Alternatively, simply note that $R_\mf p\otimes_RM$ is the tensor product of two flat modules.
\end{proof}
We might hope the converse holds. Indeed, it does.
\begin{proposition}
	Fix $R$ a ring and $M$ an $R$-module. If $M_\mf p$ is flat for all primes $\mf p$, then $M$ is also flat.
\end{proposition}
\begin{proof}
	Well, fix some inclusion $N\subseteq N'$ so that we want to show that
	\[M\otimes_R N'\to M\otimes_RN\]
	is also an inclusion. Well, we know that, upon localization, we have an inclusion
	\[M_\mf p\otimes_{R_\mf p}N'_\mf p\to M_\mf p\otimes_{R_\mf p}N_\mf p,\]
	so because this is an inclusion locally, it becomes an inclusion globally as we showed a while ago.
\end{proof}
So we are motivated to study how flat modules behave under localization.
\begin{prop}
	Fix $R$ a local ring with maximal ideal $\mf p$. Further, let $M$ be a finitely presented $R$-module. If $M$ is flat, then $M$ is free.
\end{prop}
\begin{proof}
	The idea is to use Nakayama's lemma. Because $M$ is finitely presented, we can build a short exact sequence
	\[0\to N\to F\to M\to 0,\]
	where $N$ is finitely generated, and $F$ is free. Upon tensoring with $R/\mf p$, we get the right-exact sequence
	\[N/\mf pN\to F/\mf pF\to M/\mf pM\to 0.\]
	Now, choose $F$ such that $\dim F/\mf pF=\dim M/\mf pM$, for otherwise we could use Nakayama's lemma to generate $F$ by fewer (namely, $\dim F/\mf pF$ many) elements.
	
	It follows that $N/\mf pN$ must vanish, so because $N$ is finitely generated, Nakayama's lemma promises that $N=0$. Thus, $F=M$, so $M$ is free, so we are done.
\end{proof}
\begin{remark}
	In fact, any projective module over a local ring is free. The case for finitely generated modules is on the homework. The homework also includes some examples of flatness checks.
\end{remark}

\subsection{Completion}
The point is to give certain very nice rings some very nice topologies. Here is our definition.
\begin{definition}[Completion, rings]
	Fix $R$ with a filtration $\mathcal J$ given by
	\[R=I_0\supseteq I_1\supseteq I_2\supseteq\cdots.\]
	Then we define the \textit{completion $\hat R_\mathcal J$} as a subring of $\prod_sR$ by
	\[\hat R_\mathcal J=\{(r_0,r_1,\ldots):r_i\equiv r_j\pmod{I_j}\text{ for }i>j\}.\]
\end{definition}
Most specifically, the most interesting filtrations for us will be the $I$-adic filtrations.

Let's see some examples.
\begin{example}
	Fix $R:=k[x]$ with $\mf m:=(x)$ a maximal ideal. Then $\hat R_\mf m$ consists of sequences of polynomials $\{p_n(x)\}_{n\in\NN}$ such that
	\[p_n\equiv p_m\pmod{x^m}\]
	for $n>m$. These sequences just define formal power series, where the last $n$ terms of the power series are determined by $p_n$, and the coherence above guarantees that this is well-defined.
\end{example}
\begin{example}
	Fix $R:=\ZZ$ with $\mf m:=(p)$. Then $\hat R_\mf m$ consists of sequences $\{b_n\}_{n\in\NN}$ which behave as ``formal power series'' in $p$ as
	\[a_0+a_1p+a_2p^2+\cdots,\]
	where $a_\bullet\in\ZZ/p\ZZ$ with $b_n\equiv\sum_{k<n}a_kp^k$. These are the $p$-adic integers.
\end{example}
\begin{example}
	The $2$-adic integer $u\in\ZZ_2$ given by
	\[u:=1+2+2^2+2^3+\cdots\]
	is actually just $-1$. Indeed, if we multiply $(1-2)u$, then we get $1$ after the mass cancellation, so we get $(1-2)u=1$, so $u=-1$.
\end{example}
We quickly note that, as in the above examples, we have a natural inclusion
\[\iota:R\to\hat R_\mathcal J\]
by simply taking $r\mapsto(r,r,r,\ldots)$. This gives rise to the following definition.
\begin{definition}[Complete]
	Fix a filtration $\mathcal J$ and a ring $R$. Then the ring $R$ is \textit{complete with respect to $\mathcal J$} if and only if $\hat R_\mathcal J=R$, in that the natural map $\iota:R\to\hat R_\mathcal J$ is an isomorphism.
\end{definition}
We have the following check to justify the name completion.
\begin{lemma}
	Fix $R$ a ring and $\mathcal J$ a filtration. Then $\hat R_\mathcal J$ is a complete ring with respect to the induced filtration $\hat{\mathcal J}$ given by
	\[\hat R=I_0\hat R\supseteq I_1\hat R\supseteq I_2\hat R\supseteq\cdots.\]
\end{lemma}
\begin{proof}
	We omit this proof.
	% Fix some $a$ in the completion of $\hat R_{\hat {\mathcal J}}$.
\end{proof}

Let's have a little fun with our completions.
\begin{definition}[Limit]
	Fix a completion $\hat R_\mathcal J$. Then we say that an element $a\in\hat R_\mathcal J$ is the \textit{limit} of a sequence $\{a_k\}_{k\in\NN}$ if and only if, for each $n$, there exists $N$ such that $a_i-a\in I_n$ for each $i>N$.
\end{definition}
\begin{definition}[Cauchy sequence]
	Fix a completion $\hat R_\mathcal J$. Then a sequence $\{a_k\}_{k\in\NN}$ is \textit{Cauchy} if and only if, for each $n$, there exists $N$ such that $a_i-a_j\in I_n$ for each $i,j>N$.
\end{definition}
So the following is why we used the term ``complete.''
\begin{lemma}
	A ring $R$ is complete with respect to the filtration $\mathcal J$ if and only if every Cauchy sequence in $\hat R$ converges.
\end{lemma}
\begin{proof}
	In one direction, suppose that
\end{proof}
As an aside, we note that, if we have a sequence $a:=\{a_k\}_{k\in\NN}\in\hat R_\mathcal J$, then we have $a_{i+1}-a_i\in\mf I_n$, so we can write
\[a=\sum_{j=1}^\infty(a_{j+1}-a_j)\]
to be an infinite convergent series; in particular, the partial sums converge.

\subsection{Properties Preserved by Completion}
Taking the completion also tends to look like localization.
\begin{proposition} \label{prop:completelocalize}
	Fix $R$ complete with respect to an $I$-adic filtration. Then any element $1-a$ with $a\notin I$ is a unit.
\end{proposition}
\begin{proof}
	We see that
	\[(1-a)^{-1}=1+a+a^2+\cdots\]
	will work, essentially by just summing an infinite geometric series.
\end{proof}
\begin{corollary}
	Fix $R$ a ring and $\mf m$ a maximal ideal. Then $\hat R_\mf m$ is a local ring with maximal ideal $\mf m\hat R_\mf m$.
\end{corollary}
\begin{proof}
	We need to check that $a\notin\mf m\hat R_\mf m$ implies that $a$ is a unit. Well, we can write out
	\[a=\sum_{k=0}^nb_k\]
	where $b_k\in\mf m^k$. We can select $b$ so that $ab\equiv1\pmod{\mf m}$ because $R/\mf m$ is a field, so we see that $ab=1+c$ for some $c\in\mf m$. But now \autoref{prop:completelocalize} assures us that $1+c$ is a unit, so $a$ must be a unit also.
\end{proof}
In fact, we can see how similar completion looks like localization by how it focuses on $\mf m$.
\begin{lemma}
	We have that $R/\mf m^n\cong\hat R/\mf m\hat R_\mf m^n$.
\end{lemma}
\begin{proof}
	This is by definition of $\mf m\hat R_\mf m^n$.
\end{proof}
And now we can glue these isomorphisms together.
\begin{corollary}
	We have that $\op{gr}_\mf mR\cong\op{gr}_{\mf m\hat R_\mf m}\hat R_\mf m$.
\end{corollary}
\begin{proof}
	Glue together the isomorphisms of the preceding result. The multiplicative structure matches.
\end{proof}
We can also say the following.
\begin{theorem}
	Fix $R$ a Noetherian ring and $\mf m\subseteq R$ an ideal. Then $\hat R_\mf m$ is Noetherian.
\end{theorem}
\begin{proof}
	We need the following lemma.
	\begin{lemma}
		Fix $R$ a ring complete with respect to the filtration $\mathcal J$ denoted by
		\[R=I_0\supseteq I_1\supseteq I_2\supseteq\cdots.\]
		Further, if we have $I=(a_1,\ldots,a_n)$, then $(\op{in}I)=(\op{in}a_1,\ldots,\op{in}a_n)$.
	\end{lemma}
	\begin{proof}
		Suppose for the sake of contradiction that we have some element $f\notin(\op{in}a_1,\ldots,\op{in}a_n)$. But now, we can write
		\[f=\sum_{i=1}^na_ig_i\]
		for some elements $g_i$, so if $f\in I^d\setminus I^{d+1}$, then we have
		\[\op{in}f=\Bigg[\sum_{i=1}^na_kg_i\Bigg]_{\mf m^{d+1}}.\]
		Now, lifting the elements $g_i$ to the completion, we hit our contradiction because the above should be an equality.
	\end{proof}
	We now attack the proof of the theorem. Well, fix some ideal $I\subseteq\hat R_\mf m$. Then we project $I$ into $\op{gr}_\mf mR$, which is equal to $\op{gr}_{\mf m\hat R_\mf m}\hat R_\mf m$, and here we see that $I$ must be finitely generated because $\op{gr}_\mf mR$ needs to be Noetherian as lifted from $R$, by the lemma. So we are done.
\end{proof}
We close by stating the following theorem.
\begin{definition}[Completion, modules]
	Fix $R$ a ring with filtration $\mathcal J$ denoted by
	\[R=I_0\supseteq I_1\supseteq I_2\supseteq\cdots.\]
	Further, given an $R$-module $M$, there is an induced filtration
	\[M=I_0M\supseteq I_1M\supseteq I_2M\supseteq\cdots.\]
	Doing the same construction as for $\hat R$ gives rise to $\hat R$.
\end{definition}
We can check that $\hat M$ is an $\hat R$-module.

We have the following, which says that completion is really looking like localization.
\begin{theorem}
	Fix $R$ a Noetherian ring with ideal $I$. Given a finitely generated module $M$, we have
	\[\hat M_\mf m\cong\hat R_\mf m\otimes_RM.\]
	In fact, $\hat R_\mf m$ is a flat $R$-module.
\end{theorem}
\begin{proof}
	We will show this next time.
\end{proof}