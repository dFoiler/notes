% !TEX root = ../notes.tex

Welcome back from spring break.

% \subsection{Review}
% We start with a little review. Here was our central definition.
% \krulldimdef*
% \noindent Further, if $\mf p\subseteq R$ is prime, we defined the codimension
% \[\codim\mf p:=\dim R/\mf p.\]
% As an example of something we showed, $\dim R=0$ if and only if $R$ is Artinian. We also showed that integral extensions preserve dimension.
% \intextensiondimension*
% \noindent And we also showed the principal ideal theorem.
% \pidtheorem*
% \noindent We also generalized this to minimal primes over ideals $I$ with $s$ generators. In particular, if $\mf p=(x_1,\ldots,x_r)$, then
% \[\codim\mf p\le r.\]
% We were even able to provide a converse.
% \pidthmconverse*
% \begin{example}
% 	Given a polynomial ring $k[x_1,\ldots,x_n]$, fix $\mf p:=(x_1,\ldots,x_k)$. Then
% 	\[\codim\mf p=k,\]
% 	which we could also see directly by constructing the chain
% 	\[(0)\subseteq(x_1)\subseteq\cdots\subseteq(x_1,\ldots,x_k)=\mf p.\]
% \end{example}

\subsection{The Rank--Nullity Theorem}
In light of \autoref{thm:dimislocal}, we can compute dimension by localizations, so we will focus on local rings.
\begin{lemma} \label{lem:dimiscodim}
	Fix a local ring $R$ with maximal ideal $\mf m$. Then $\dim R=\codim\mf m$.
\end{lemma}
\begin{proof}
	Certainly $\codim\mf m\le\dim R$ by \autoref{prop:upperbounddimension}, so we need to show $\dim R\le\codim\mf m$. Well, for any chain of distinct primes
	\[\mf p_0\subsetneq\mf p_1\subsetneq\cdots\subseteq\mf p_d\]
	in $R$, we claim that $d\le\codim\mf m$, which will finish by setting $d=\dim R$. Note that $\mf m$ is the unique maximal ideal, so $\mf p_d\subseteq\mf m$. As such, we can replace $\mf p_d$ at the end with $\mf m$ (namely, if $\mf p_{d-1}$ exists, then $\mf p_{d-1}\subsetneq\mf m$ still), so we have a chain of length $d$ descending from $\mf m$. As such, $d\le\codim\mf m$ by \autoref{lem:codimisdescend}.
\end{proof}
\begin{proposition} \label{prop:localdimension}
	Fix a local ring $R$ with maximal ideal $\mf m$. Then $\dim R$ is the minimal $d\in\NN$ such that there exist generators $f_1,\ldots,f_d$ so that
	\[\mf m^n\subseteq(f_1,\ldots,f_d)\subseteq\mf m\]
	for some $n$.
\end{proposition}
\begin{proof}
	By \autoref{lem:dimiscodim}, we have $\dim R=\codim\mf m$. Now, in one direction, if
	\[\mf m^n\subseteq(f_1,\ldots,f_d)\subseteq\mf m,\]
	for some $n$, we note that $\mf m$ is the unique minimal prime over $(f_1,\ldots,f_d)$: indeed, $\mf m/(f_1,\ldots,f_d)$ is a nilpotent ideal, so any prime between must be equal to $\mf m$.
	
	In the other direction, if $\mf m$ is a minimal prime over $(f_1,\ldots,f_d)$, then the ring
	\[R/(f_1,\ldots,f_d)\]
	is local, and its only prime is $\mf m/(f_1,\ldots,f_d)$ by the uniqueness, so $R/(f_1,\ldots,f_d)$ is Artinian, so the Jacobson radical
	\[\mf m/(f_1,\ldots,f_d)\]
	is still nilpotent, so we get
	\[\mf m^n\subseteq(f_1,\ldots,f_d)\subseteq\mf m,\]
	which is what we wanted.
\end{proof}
\begin{remark}
	There is a geometric meaning the discovered $(f_1,\ldots,f_d)$. Namely, if $X$ is an algebraic set, and $P\in X$ is a point, then we can set $R:=A(x)$ and $\mf p$ to correspond to $P$. Then the functions $\{f_1,\ldots,f_d\}$ will be ``local coordinates'' in a neighborhood of $P$. Roughly speaking, these elements $f_k$ have powers which go to $0$, so they merely work in an infinitesimal neighborhood of $P$.
\end{remark}
To set up our next result, we note that dimension is a somewhat general concept: for example, in $\mathrm{Set}$, dimension should be the cardinality. As an example, given a map $\varphi:X\to B$, we can pick any point $p\in B$ and find that
\[\dim X\le\dim B+\dim\varphi^{-1}(p).\] % \todo{wat}
In our algebraic story, we are interested in families of varieties $X\to B$, which corresponds to ring maps $A(B)\to A(X)$. Still looking locally at the fiber at a point, we have the following.
\begin{proposition} \label{prop:ranknullity}
	Fix two local rings $R$ and $S$ with maximal ideals $\mf m$ and $\mf n$, respectively. Now, given a map $\varphi:R\to S$ such that $\varphi(\mf m)\subseteq\mf n$, we have
	\[\dim S\le\dim R+\dim S/\mf nS\]
	In fact, if $S$ is a flat as an $R$-module, then we have equality.
\end{proposition}
This is still approximately our geometric picture because $S/\mf mS$ is the coordinate ring of the pre-image of the point $\mf m$.
\begin{proof}
	We use \autoref{prop:localdimension}. For brevity, set $r:=\dim R$ and $s:=\dim S/\mf S$. As such, \autoref{prop:localdimension} promises us elements $\{f_1,\ldots,f_r\}\subseteq\mf n$ and some $p\in\NN$ such that
	\[\mf m^p\subseteq(f_1,\ldots,f_r)\subseteq\mf m.\]
	Similarly, there exist $\{g_1,\ldots,g_s\}\subseteq\mf n$ so that
	\[\mf n^q\subseteq(g_1,\ldots,g_s)+\mf mS\]
	so that
	\[\mf n^{pq}\subseteq\mf m^pS+(g_1,\ldots,g_s)\subseteq(f_1,\ldots,f_r,g_1,\ldots,g_s)\subseteq\mf n\]
	by taking the $p$th power and applying the previous inequality. Thus, $\dim S=\codim\mf n\le r+s$, which is what we wanted.

	We now take $S$ to be flat. Set $\mf q\subseteq S$ to be a minimal prime over $\mf mS$, and we note $s=\dim\mf q$ because $\mf q$ is minimal over $\mf mS$. We do know that
	\[\dim S\ge\dim\mf q+\codim\mf q=s+\codim\mf q,\]
	so to get the result, we merely need to show $\codim\mf q\ge r$. For this, we take the following lemma.
	\begin{lemma}[Going down]
		Let $S$ be a flat $R$-algebra by $\varphi:R\to S$. Further, fix $\mf p'\subseteq\mf p$ are prime ideals in $R$. If we have $\mf q\subseteq S$ so that $\mf p=\varphi^{-1}\mf q$, then there is a $\mf q'$ such that $\mf p'=\varphi^{-1}\mf q'$.
	\end{lemma}
	Pictorially, we are building $\mf q'$ in the following diagram.
	% https://q.uiver.app/?q=WzAsNixbMCwwLCJTIl0sWzAsMSwiUiJdLFsxLDAsIlxcY29sb3J7cmVkfVxcbWYgcSciXSxbMSwxLCJcXG1mIHAnIl0sWzIsMSwiXFxtZiBwIl0sWzIsMCwiXFxtZiBxIl0sWzEsMCwiXFx2YXJwaGkiXSxbMiw1LCJcXHN1YnNldGVxIiwzLHsic3R5bGUiOnsiYm9keSI6eyJuYW1lIjoibm9uZSJ9LCJoZWFkIjp7Im5hbWUiOiJub25lIn19fV0sWzMsNCwiXFxzdWJzZXRlcSIsMyx7InN0eWxlIjp7ImJvZHkiOnsibmFtZSI6Im5vbmUifSwiaGVhZCI6eyJuYW1lIjoibm9uZSJ9fX1dLFsyLDMsIiIsMyx7InN0eWxlIjp7ImhlYWQiOnsibmFtZSI6Im5vbmUifX19XSxbNSw0LCIiLDMseyJzdHlsZSI6eyJoZWFkIjp7Im5hbWUiOiJub25lIn19fV1d&macro_url=https%3A%2F%2Fraw.githubusercontent.com%2FdFoiler%2Fnotes%2Fmaster%2Fnir.tex
	\[\begin{tikzcd}
		S & {\color{red}\mf q'} & {\mf q} \\
		R & {\mf p'} & {\mf p}
		\arrow["\varphi", from=2-1, to=1-1]
		\arrow["\subseteq"{marking}, draw=none, from=1-2, to=1-3]
		\arrow["\subseteq"{marking}, draw=none, from=2-2, to=2-3]
		\arrow[no head, from=1-2, to=2-2]
		\arrow[no head, from=1-3, to=2-3]
	\end{tikzcd}\]
	Anyways, here is the proof of the lemma.
	\begin{proof}
		For psychological reasons, we will replace $R$ with $R/\mf p'$ and $S$ with $S/\mf p'S$. We do note that $S/\mf p'S:=S\otimes_RR/\mf p'$ must be flat over the $R$-algebra $R/\mf p'$, as shown on the homework.

		Thus, we may assume that $R$ is a domain with $\mf p'=(0)$. Our diagram is now as follows.
		% https://q.uiver.app/?q=WzAsNixbMCwwLCJTIl0sWzAsMSwiUiJdLFsxLDAsIlxcY29sb3J7cmVkfVxcbWYgcSciXSxbMSwxLCIoMCkiXSxbMiwxLCJcXG1mIHAiXSxbMiwwLCJcXG1mIHEiXSxbMSwwLCJcXHZhcnBoaSJdLFsyLDUsIlxcc3Vic2V0ZXEiLDMseyJzdHlsZSI6eyJib2R5Ijp7Im5hbWUiOiJub25lIn0sImhlYWQiOnsibmFtZSI6Im5vbmUifX19XSxbMyw0LCJcXHN1YnNldGVxIiwzLHsic3R5bGUiOnsiYm9keSI6eyJuYW1lIjoibm9uZSJ9LCJoZWFkIjp7Im5hbWUiOiJub25lIn19fV0sWzIsMywiIiwzLHsic3R5bGUiOnsiaGVhZCI6eyJuYW1lIjoibm9uZSJ9fX1dLFs1LDQsIiIsMyx7InN0eWxlIjp7ImhlYWQiOnsibmFtZSI6Im5vbmUifX19XV0=&macro_url=https%3A%2F%2Fraw.githubusercontent.com%2FdFoiler%2Fnotes%2Fmaster%2Fnir.tex
		\[\begin{tikzcd}
			S & {\color{red}\mf q'} & {\mf q} \\
			R & {(0)} & {\mf p}
			\arrow["\varphi", from=2-1, to=1-1]
			\arrow["\subseteq"{marking}, draw=none, from=1-2, to=1-3]
			\arrow["\subseteq"{marking}, draw=none, from=2-2, to=2-3]
			\arrow[no head, from=1-2, to=2-2]
			\arrow[no head, from=1-3, to=2-3]
		\end{tikzcd}\]
		As such, we define $\mf q'$ to be a minimal prime over $(0)$ inside $\mf q$, which certainly exists by saying something about Zorn's lemma. It remains to show that
		\[\varphi^{-1}(\mf q)\stackrel?=(0).\]
		Well, $\mf q$ only has zero-divisors by minimality, but because $S$ is flat (here is where we use the condition!), we have that the function $\mu_r:s\mapsto rs$ is injective for nonzero $r$, so $\varphi^{-1}(\mf q)$ had better not contain any nonzero element of $R$. So $\varphi^{-1}(\mf q)=(0)$, finishing.
	\end{proof}
	Now, note that the lemma finishes the theorem because $\codim\mf q$ will be $\dim S/\mf q$, which upon going down will be at least $r$ by tracking a full chain of primes through.
\end{proof}
\begin{ex}
	Consider $S:=k[x,y,z,w]/(xw-yz)$ to be the ring of singular matrices, as an algebra over $R:=k[x,y]$. Further, set $\mf n:=(x,y,z,w)$ so that $\mf q=(x,y)$. We track through the dimensions of \autoref{prop:ranknullity} at $\mf n$.
\end{ex}
\begin{proof}
	Note that $\dim R_\mf m=2$ as a localization of a two-dimensional ring. Then we can compute
	\[\dim S_\mf n/\mf mS_\mf n=\dim k[z,w]_{(z,w)}=2,\]
	where we are using the fact that $(xw-yz)$ once localized will disappear as a condition. However, $\dim S=3$, so $\dim S_\mf n\le3$, which is strictly less than $\dim R_\mf m+\dim S_\mf n/\mf mS_\mf n=4$.
\end{proof}
\begin{remark}
	At a high level, we have codified the notion that flat means continuously varying fibers: the dimension of the fiber $S/\mf mS$ must be $\dim S-\dim R$ and therefore is constant!
\end{remark}
As a consequence, we show that dimension is preserved by completion.
\begin{corollary}
	Fix a Noetherian local ring $R$ with maximal ideal $\mf m$. Then $\dim R=\dim\widehat R$.
\end{corollary}
\begin{proof}
	We know that $\widehat R$ is flat as an $R$-algebra, via the natural map $R\to\widehat R$. Now, $\mf m$ goes to $\widehat{\mf m}=\mf m\widehat R$ under this map, so we conclude from \autoref{prop:ranknullity} that
	\[\dim\widehat R=\dim R+\dim\widehat R/\widehat{\mf m}.\]
	But $\widehat R/\widehat{\mf m}\cong R/\mf m$ is a field, so it has dimension $0$. This finishes.
\end{proof}
And here is another corollary.
\begin{proposition} \label{prop:polyringdimension}
	Fix a Noetherian ring $R$. Then $\dim R[x]=\dim R+1$.
\end{proposition}
\begin{proof}
	We start by showing $\dim R[x]\ge\dim R+1$. Well, taking $r:=\dim R$, we pick up a chain
	\[\mf p_0\subsetneq\mf p_1\subsetneq\cdots\subsetneq\mf p_r.\]
	Lifting these to $R[x]$, we get the chain
	\[\mf p_0R[x]\subsetneq\mf p_1R[x]\subsetneq\cdots\subsetneq\mf p_rR[x].\]
	But now we can add $\mf p_rR[x]+(x)$ to the end of this chain, which is prime by checking that the quotient $R[x]/(\mf p_rR[x]+(x))\cong R/\mf p_r$ is an integral domain. In particular, $\mf p_rR[x]$ contains no monic polynomials, so $\mf p_rR[x]+(x)$ is indeed strictly bigger.

	To finish, we show $\dim R[x]\le\dim R+1$. Well, pick up a maximal ideal $\mf n\subseteq R[x]$ such that $\mf n$ appears in the maximum chain, promising
	\[\codim\mf n=\dim R[x].\]
	As such, we pull $\mf n$ back to $R$ as $\mf m\subseteq R$. Localizing at $\mf m$, we see
	\[\dim R[x]=\dim R[x]_{\mf n}\le\dim R+\dim R[x]_\mf n/\mf mR[x]_\mf n,\]
	where we have used \autoref{prop:ranknullity}. But now $\dim R[x]_\mf n/\mf mR[x]_\mf n=1$ because it is the dimension of a polynomial ring in one variable over a field.
\end{proof}
\begin{corollary}
	For a field $k$, we have $k[x_1,\ldots,x_n]=n$.
\end{corollary}
\begin{proof}
	This follows from induction, starting with $\dim k=0$ and incrementing by \autoref{prop:polyringdimension}.
\end{proof}

\subsection{Dimension for Modules}
We now add a notion of Krull dimension for modules.
\begin{definition}[Krull dimension, modules]
	Given an $R$-module $M$, we define the \textit{dimension} of $M$ as $\dim M:=\dim R/\op{Ann}M$.
\end{definition}
This lets us intelligently talk about colength.
\begin{definition}[Colength]
	Fix an ideal $\mf q\subseteq R$. Then $\mf q$ \textit{has finite colength} on a module $M$ if and only if $M/\mf qM$ has finite length.
\end{definition}
We have the following check.
\begin{lemma} \label{lem:finitecolengthcheck}
	Fix a local ring $R$ with maximal ideal $\mf m$. Then $\mf q$ has finite colength if and only if there exists an integer $n$ so that
	\[\mf m^n\subseteq\mf q+\op{Ann}M.\]
\end{lemma}
\begin{proof}
	Note that $M/\mf m^nM$ has finite length by Nakayama's lemma, so $M/\mf qM$ having finite length needs to be measured as such somehow.
\end{proof}
As usual, our notion of size behaves in short exact sequences.
\begin{proposition}
	Fix a short exact sequence of $R$-modules
	\[0\to A\to B\to C\to 0.\]
	If $\mf q$ has finite colength on $A$ and $C$, then $\mf q$ has finite colength on $B$.
\end{proposition}
\begin{proof}
	Omitted.
\end{proof}
And here are a few more results.
\begin{proposition} \label{prop:generatorsdimension}
	Fix a local ring $R$ with maximal ideal $\mf m$ and an $R$-module $M$. Then $\dim M$ is equal to the minimal $d$ such that $(f_1,\ldots,f_d)$ has finite colength on $M$.
\end{proposition}
\begin{proof}
	Omitted. This follows from \autoref{lem:finitecolengthcheck}.
\end{proof}
\begin{proposition}
	Fix a local ring $R$ with maximal ideal $\mf m$ and an $R$-module $M$. Given $x\in\mf m$, we have
	\[\dim M/xM\ge\dim M-1.\]
\end{proposition}
\begin{proof}
	Use the previous proposition directly. Namely, if $d=\dim M/xM$, then \autoref{prop:generatorsdimension} promises us an ideal $(f_1,\ldots,f_d)$ which has finite colength on $M/xM$. But then
	\[(f_1,\ldots,f_d,x)\]
	has finite colength over $M$, so $\dim M\le\dim M/xM-1$ by \autoref{prop:generatorsdimension} again, which is what we wanted.
\end{proof}

\subsection{Regular Rings}
We close class by defining regular.
\begin{definition}[Regular]
	Fix a local ring $R$ with maximal ideal $\mf m$, of dimension $d$. Then $R$ is \textit{regular} if and only if there exist elements $\{f_1,\ldots,f_d\}\subseteq R$ such that
	\[\mf m=(f_1,\ldots,f_d).\]
\end{definition}
In particular, we know that $\mf m$ is at least a minimal prime over such an ideal; regularity asserts us equality.
\begin{remark}
	We have $\dim_{R/\mf m}\mf m/\mf m^2=d$ when $R$ is regular, essentially by expanding out the definition and applying Nakayama's lemma.
\end{remark}
\begin{example}
	Fix $R:=k[x_1,\ldots,x_d]_{\mf m}$, where $\mf m=(x_1,\ldots,x_d)$. Then $R$ is regular.
\end{example}
\begin{example}
	The completion $k\bb{x_1,\ldots,x_n}$ is regular.
\end{example}
\begin{example}
	Any local principal ideal domain $R$ has dimension $1$, and the maximal ideal also is generated by one element, so $R$ is regular.
\end{example}
And here is a last result.
\begin{proposition}
	A regular local ring is an integral domain.
\end{proposition}
\begin{proof}
	We omit this proof.
\end{proof}