\documentclass[../notes.tex]{subfiles}

\begin{document}

% !TEX root = ../notes.tex

Today's notes were transcribed from Miles's notes of the class. Thank you, Miles! We are finishing up completions today.

\subsection{Idempotents}
We recall the following definitions.
\begin{definition}[Idempotent]
	Fix a ring $R$. An element $e\in R$ is \textit{idempotent} if and only if $e^2=e$.
\end{definition}
\begin{remark}
	Equivalently, $e\in R$ is idempotent if and only if $e$ is a root of the polynomial $f(x)=x^2-x$.
\end{remark}
\begin{remark}
	If $e\in R$ is idempotent, then we can give $Re$ a ring structure, where the addition and multiplication are inherited from $R$. In particular, we are closed under addition as $re+se=(r+s)e$, and we are closed under multiplication as $re\cdot se=rse^2=(rs)e$.
	
	However, our identity element is now $e$, which works because $e\cdot re=re\cdot e=re^2=re$ for any $re\in Re$. The coherence (commutativity, associativity, distributivity, etc.) checks are all inherited directly from $R$.
\end{remark}
\begin{example} \label{ex:subfromoneidemp}
	If $e\in E$ is an idempotent, then $1-e$ is also an idempotent: we can directly compute
	\[(1-e)^2=1-e-e+e^2=1-e-e+e=1-e.\]
\end{example}
\begin{lemma} \label{lem:basicidempdecomp}
	Fix $e$ an idempotent of a ring $R$. Then $R\cong Re\times R(1-e)$.
\end{lemma}
\begin{proof}
	% When $e$ is an idempotent, we turn $Re$ into a ring using the inherited operations from $R$, but it is not a subring because $Re$
	This was on the homework.
\end{proof}

We will want to lift idempotents, but we will want to keep track of a little more data.
\begin{definition}[Orthogonal idempotents]
	Fix $R$ a ring. Then a set $E\subseteq R$ of idempotents of $R$ is \textit{orthogonal} if and only if $ee'=0$ for any two distinct $e,e'\in E$.
\end{definition}
\begin{example}
	Fix $e\in R$ an idempotent. Then the elements $\{e,1-e\}$ are orthogonal. In particular, these are orthogonal because
	\[e(1-e)=e-e^2=e-e=0.\]
\end{example}
Here are some reasons to care about orthogonal idempotents. For one, they give more idempotents.
\begin{lemma} \label{lem:sumidemps}
	Fix $R$ a ring and $E\subseteq R$ any finite set of orthogonal idempotents. Then
	\[\sum_{e\in E}e\]
	is another idempotent.
\end{lemma}
\begin{proof}
	This is a direct computation. For concreteness, enumerate $E=\{e_1,\ldots,e_n\}$. Then we can compute
	\[\left(\sum_{k=1}^ne_k\right)\left(\sum_{\ell=1}^ne_\ell\right)=\sum_{k,\ell=1}^ne_ke_\ell=\sum_{k=1}^ne_ke_k+\sum_{\substack{k,\ell=1\\k\ne\ell}}^ne_ke_\ell.\]
	In particular, we note that $e_k^2=e_k$ for each $k$ because $E$ is only made of idempotents. Further, $e_ke_\ell=0$ for any $k\ne\ell$ because $E$ is made of orthogonal idempotents. It follows that
	\[\left(\sum_{k=1}^ne_k\right)^2=\sum_{k=1}^ne_k,\]
	which is what we wanted.
\end{proof}
The above lemma is in fact detecting a larger ring decomposition, generalizing \autoref{lem:basicidempdecomp}.
\begin{lemma} \label{lem:idempringdecomp}
	Fix $R$ a ring and $E\subseteq R$ a finite set of orthogonal idempotents such that
	\[\sum_{e\in E}e=1.\]
	Then
	\[\bigoplus_{e\in E}Re\cong R\]
	by taking $(r_ee)_{e\in E}$ to $\sum_{e\in E}r_ee$.
\end{lemma}
\begin{proof}
	Let $\varphi:(r_ee)_{e\in E}\mapsto\sum_{e\in E}r_ee$ be the map in question. We have the following checks on $\varphi$; let $(r_ee)_{e\in E}$ and $(s_ee)_{e\in E}$ be elements of $\bigoplus_{e\in E}Re$.
	\begin{itemize}
		\item Additive: we note that
		\[\varphi\left((r_ee)_{e\in E}+(s_ee)_{e\in E}\right)=\varphi\left(((r_e+s_e)e)_{e\in E}\right)=\sum_{e\in E}(r_e+s_e)e=\varphi\left((r_e)_{e\in E}\right)+\varphi\left((s_e)_{e\in E}\right).\]
		\item Multiplicative: we note that
		\[\varphi\left((r_ee)_{e\in E}\cdot(s_ee)_{e\in E}\right)=\varphi\left((r_es_ee)_{e\in E}\right)=\sum_{e\in E}r_es_ee.\]
		However,
		\[\varphi\left((r_e)_{e\in E}\right)\varphi\left((s_e)_{e\in E}\right)=\left(\sum_{e\in E}r_ee\right)\left(\sum_{e'\in E}s_{e'}e'\right)=\sum_{e=e'\in E}r_es_{e'}e'+\sum_{\substack{e,e'\in E\\e\ne e'}}r_es_{e'}ee'.\]
		The left sum here evaluates term-wise as $r_es_ee^2=r_es_ee$, and the right sum here evaluates term-wise as $r_es_{e'}ee'=0$ because $E$ is made of orthogonal idempotents (!). So we are done.
		\item Identity: we note that
		\[\varphi\left((1e)_{e\in E}\right)=\sum_{e\in E}e=1\]
		by hypothesis.
		\item Injective: if an element $(r_ee)_{e\in E}\in\ker\varphi$, then we see that
		\[\sum_{e\in E}r_ee=0\]
		by definition. Now, fixing any $e'\in E$, we note that $r_eee'=0$ for any $e\ne e'$ because $E$ is made of orthogonal idempotents; however, when $e=e'$, we have $r_eee'=r_ee^2=r_ee$. As such, we see
		\[r_{e'}e'=e'\cdot\sum_{e\in E}r_ee=e'\cdot0=0,\]
		so $r_{e'}e'=0$ for any $e'\in E$. Thus, $(r_ee)_{e\in E}$ is the zero element.
		\item Surjective: for any $r\in R$, we note that
		\[\varphi\left((re)_{e\in E}\right)=\sum_{e\in E}re=r\sum_{e\in E}e=r\cdot1=r,\]
		so we are done.
	\end{itemize}
	The above checks finish showing that $\varphi$ is an isomorphism of rings.
\end{proof}
The condition that the orthogonal idempotents sum to $1$ might look limiting, but it really is not so bad.
\begin{lemma}
	Fix $R$ a ring and $E\subseteq R$ a finite set of idempotents. Then
	\[1-\sum_{e\in E}e\]
	is another idempotent, orthogonal from the rest of $E$.
\end{lemma}
\begin{proof}
	By \autoref{lem:sumidemps}, $\sum_{e\in E}e$ is an idempotent. By \autoref{ex:subfromoneidemp}, $1-\sum_{e\in E}e$ is an idempotent.

	It remains to show that this idempotent is orthogonal to $E$. Well, fixing any $e'\in E$, we note that
	\[e'\cdot\sum_{e\in E}e=\sum_{e\in E}ee'.\]
	If $e=e'$, then $ee'=(e')^2=e'$; else if $e\ne e'$, then $ee'=0$ because $E$ consists of orthogonal idempotents. So the sum collapses to
	\[e'\left(1-\sum_{e\in E}e\right)=e'-e'\sum_{e\in E}e=e'-e'=0.\]
	This finishes.
\end{proof}
As such, given any set of orthogonal idempotents, $E$, we can just add in the idempotent
\[1-\sum_{e\in E}e\]
to get a new larger set of orthogonal idempotents, from which we can use \autoref{lem:idempringdecomp}.

\subsection{Lifting Idempotents}
With that definition out of the way, here is our main statement.
\begin{proposition} \label{prop:liftidemp}
	Fix a Noetherian, local ring $R$ complete with respect to its maximal ideal $\mf m$. Further, take $A$ to be a finite $R$-algebra (not necessarily commutative!), and pick up a finite set of orthogonal idempotents
	\[\overline E\subseteq A/\mf mA.\]
	Then each idempotent $\overline e\in\overline E$ can be lifted to an idempotent $e\in E$ such that the set of lifts remains an orthogonal 
\end{proposition}
\begin{proof}
	We divide the proof into two cases. We start with the case where $A$ is a commutative ring. Our starting step is to reduce to the case where $R=A$. Indeed, note that $A$ is complete with respect to the ideal $\mf mA$ because
	\[\widehat A_{\mf mA}\cong\widehat R_\mf m\otimes_RA=R\otimes_RA=A,\]
	where we are using the tensor-product characterization of the completion; notably, $\widehat R_\mf m=R$ because $R$ is complete with respect to $\mf m$. Because $A$ is now complete with respect to an ideal $\mf mA$, we might as well ignore $R$. In particular, $A$ is local with maximal ideal $\mf mA$ and Noetherian as a finite algebra over a Noetherian ring.

	Now, suppose that we have an idempotent $\overline e\in A/\mf mA$, and let $e_0\in A$ be any representative. Set $f(x):=x^2-x$, and we will use Hensel lifting. In particular,
	\[f(e_0)=e_0^2-e_0\equiv\overline e^2-\overline e\equiv0\pmod{\mf m},\]
	and
	\[(f'(e_0))^2=(2e_0-1)^2=4\left(e_0^2-e_0\right)+1\equiv1\pmod{\mf m},\]
	so we may use Hensel's lemma to lift $\overline e$ to an element $e\in A$ such that $e\equiv e_0\equiv\overline e\pmod{\mf m}$ and $f(e)=e^2-e=0$.

	It remains to preserve orthogonality. For concreteness, enumerate $\overline E$ by $\{\overline{e_1},\ldots,\overline{e_n}\}\subseteq A/\mf mA$, which we lift to $\{e_1,\ldots,e_n\}\subseteq A$. Now, for $k\ne\ell$, we need to show that $e_ke_\ell=0$. Well,
	\[e_ke_\ell\equiv\overline{e_k}\cdot\overline{e_\ell}\equiv0\pmod{\mf m}\]
	because the $\overline{e_\bullet}$ are orthogonal. However, we can do better than this because we have idempotents: for any $d\in\NN$, we see that
	\[e_ke_\ell=e_k^de_\ell^d=(e_ke_\ell)^d\in\mf m^d.\]
	Notably, this is the point where we are crucially using the fact that $A$ is commutative: we need $e_k^de_\ell^d=(e_ke_\ell)d$. Anyways, it follows that
	\[e_ke_\ell\in\bigcap_{\mf m}\mf m^d.\]
	By the Krull intersection theorem (recall $A$ is a Noetherian local ring), we conclude that $e_ke_\ell=0$.

	We now add on the case where $A$ is not commutative, by reducing to the commutative case. We proceed by induction on $\#E$. For example, if $\#\overline E=1$, then we can still lift our idempotent from $A$ upwards as discussed above, and the ring $R[e]$ is commutative, so we can directly reduce to the commutative case.

	More generally, suppose that we have orthogonal idempotents $\{\overline{e_1},\ldots,\overline{e_n}\}\subseteq A/\mf mA$. By the inductive hypothesis, we can lift the first $n-1$ of these to orthogonal idempotents
	\[\{e_1,\ldots,e_{n-1}\}\subseteq A.\]
	Now, we add in the last idempotent by hand: set
	\[f:=1-\sum_{i=1}^{n-1}e_i\]
	so that $\{e_1,\ldots,e_{n-1},f\}$ is a set of orthogonal idempotents. With this ``auxiliary'' idempotent, we finish by taking any representative $e_n'\in A$ of $\overline{e_n}$ and then lift $fe_nf$ in the commutative case $R[e_1,\ldots,e_{n-1},fe_n'f]$. This lifted idempotent $e_n$ will work. % \todo{}
\end{proof}
Being able to lift idempotents gives us the following ring decomposition, akin to \autoref{lem:basicidempdecomp}.
\begin{lemma}
	Fix $R$ a Noetherian local ring complete with respect to its maximal ideal $\mf m$. Further, given a finite (commutative) $R$-algebra $A$, we have that $A$ has only finitely many maximal ideals $\{\mf m_1,\ldots,\mf m_n\}$, and
	\[A\cong\prod_{k=1}^nA_{\mf n_k}\]
\end{lemma}
Notably, $A_{\mf m_k}$ is a localization.
\begin{proof}
	The point is to use the Artinian decomposition from our discussion of modules of finite length and then lift by appealing to idempotents. Namely, $A/\mf mA$ is a finite-dimensional $R/\mf m$-vector space, so $A/\mf mA$ is an Artinian ring. In particular, we can write
	\[A/\mf mA\cong\prod_{i=1}^n\overline{A_i},\]
	where we are writing $A_i$ as our product of the various localizations of $A/\mf mA$. Because we have expressed $A/\mf mA$ as a product of rings, we can identify the inclusion maps
	\[\overline{A_i}\into A/\mf mA\]
	as taking $1$ to some idempotent of $A/\mf mA$, which we will call $\overline{e_i}$. For psychological reasons, we will identify $\overline{A_i}$ with its image (via the above inclusion map) in $A/\mf mA$ so that $\overline{A_i}=(A/\mf mA)\overline{e_i}$. % \todo{Why is this A/mA?}

	We now lift. Our idempotents $\{\overline{e_1},\ldots,\overline{e_n}\}$ are orthogonal, so we can use our lifting to give us a set of orthogonal idempotents
	\[\{e_1,\ldots,e_n\}\subseteq A\]
	such that $e_1\equiv\overline{e_1}\pmod{\mf m}$. Then these orthogonal idempotents give rise to the ring decomposition
	\[A=\prod_{i=1}^nA_i,\]
	where $A_i=Ae$.

	To finish, we need to show that $A_i$ is a localization of $A$ with respect to a maximal ideal. To be explicit, we claim that $A_i$ is local with maximal ideal $\mf n_i:=A_i\cap\mf mA$. For this, we note
	\[\frac{R}{\mf n_i\cap R}\subseteq\frac{A_i}{\mf n_iA_i}\]
	is an integral extension (in particular, finite), where $R/(\mf n_i\cap R)$ is a field (as an extension of $R/\mf m$). From here, we can realize $A_i$ as a localization by setting
	\[\mf m_i=\prod_{j=1}^{i-1}A_i\times\mf n_i\times\prod_{j=i+1}^nA_i.\]
	Thus, we have forced $A$ to be a product of the localizations of $A$, from which we conclude that we have found all of our maximal ideals.
\end{proof}

\subsection{The Cohen Structure Theorem}
We end our discussion of completions with a few works on the Cohen structure theorem. Here is the statement.
\begin{theorem}[Cohen structure] \label{thm:cohenstruct}
	Fix a Noetherian local ring $R$ complete with respect to its maximal ideal $\mf m$. Further, let $\kappa:=R/\mf m$ be the residue field; if $R$ contains a field, then
	\[R\cong\kappa\bb{x_1,\ldots,x_n}/I\]
	for some ideal $I\subseteq\kappa\bb{x_1,\ldots,x_n}$.
\end{theorem}
Note that the condition that $R$ contains its residue field $k$ is necessary.
\begin{nex}
	Take $R=\ZZ_p$ to be the ring of $p$-adic integers, which has residue field $k=R/\mf m=\ZZ_p/p\ZZ_p\cong\FF_p$. However, $R$ does not contain $\FF_p$ because $R$ has characteristic $0$.
\end{nex}
We'll show \autoref{thm:cohenstruct} in the case where $\kappa$ is perfect.
\begin{proof}[Proof of \autoref{thm:cohenstruct} when $\kappa$ is perfect]
	Note that, if $R$ contains any field $K$ which surjects onto $\kappa$, then we can give $\mf m$ generators $\{f_1,\ldots,f_n\}$ over $R$ and then surjecting
	\[\pi:K\bb{x_1,\ldots,x_n}\onto R\]
	by lifting $K\onto R/\mf m$ and then sending $x_\bullet\mapsto f_\bullet$. % \todo{wat}
	From here, we can mod out by the kernel to get an isomorphism of the form
	\[R\cong\frac{K\bb{x_1,\ldots,x_n}}{\ker\pi},\]
	which gives the desired map.

	So it suffices to show that, if $R$ contains any field, then $R$ contains a field isomorphic to $\kappa$. Well, by modding out by $\mf m$, we can at least be sure that the field which $R$ contains can be embedded into $\kappa$, so we can build the extension
	\[K\subseteq K(\{t_i:i\in I\})\subseteq\kappa,\]
	where $K\subseteq K(\{t_i:i\in I\})$ is a purely transcendental extension and $K(t_i:i\in I)\subseteq\kappa$ is a purely algebraic extension.

	Now, the transcendental elements $t_i\in\kappa=R/\mf m$ can be lifted arbitrarily to $R$, which induces an embedding $F\into R$, where $F$ is some very large field. As such, we let $K'$ be the maximal subfield of $R$ which contains $F$. We would like to show that $K'$ surjects onto $\kappa$ when taken modulo $\mf m$, which will finish by choosing our lifts carefully.

	Well, let $\alpha\in\kappa\setminus K'$ be an element of $\kappa$ such that $\kappa=K'(\alpha)$. However, we can lift the root $\alpha$ up to the root of some polynomial in $R$, which produces a strictly larger field than $K'$. This is a contradiction, so we must instead have $K'\onto R/\mf m$. This finishes.
\end{proof}
\begin{remark}
	We used separability at the end of the proof during our use of the Primitive element theorem.
\end{remark}

\end{document}