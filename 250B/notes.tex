% LTeX: enabled=false

\documentclass[openany]{book}
\usepackage[utf8]{inputenc}

\newcommand{\nirpdftitle}{250B Notes}
\usepackage{import}
\inputfrom{..}{nir}

\pagestyle{contentpage}

\title{250B: Commutative Algebra\\
\Large For the Morbidly Curious}
\author{Nir Elber}
\date{Spring 2022}
\rhead{\textit{250B: COMM. ALGEBRA}}

\begin{document}

\maketitle

\toctrue
\tableofcontents
\tocfalse

\newpage

\chapter{Introduction}

\epigraph{Hold tight to your geometric motivation as you learn the formal structures which have proved to be so effective in studying fundamental questions.}
{---Ravi Vakil}

\foreach \n in {18,20}
{
	\section{January \n}
	\input{jan/\n}
}

\chapter{Local Study}

\epigraph{That something so small could be so beautiful.}
{---Anthony Doerr}

\foreach \n in {25,27}
{
	\section{January \n}
	\input{jan/\n}
}

\foreach \n in {1,3,8}
{
	\section{February \n}
	\input{feb/\n}
}

\chapter{Monic Polynomials}

\epigraph{One is the loneliest number that you'll ever do}
{---Harry Nilsson}

\foreach \n in {10,15,17,22}
{
	\section{February \n}
	\input{feb/\n}
}

\chapter{Working in Chains}

\epigraph{But this is like trying to scale a glacier. It's hard to get your footing, and your fingertips get all red and frozen and torn up.}
{---Anne Lamott}

\foreach \n in {24}
{
	\section{February \n}
	\input{feb/\n}
}

\foreach \n in {1,3}
{
	\section{March \n}
	\input{mar/\n}
}

\foreach \n in {8homoalg}
{
	\section{March 8}
	\input{mar/\n}
}

\chapter{Completions}

\epigraph{Completion is a goal, but we hope it is never the end.}
{---Sarah Lewis}

\foreach \n in {8completions}
{
	\section{March 8}
	\input{mar/\n}
}

\foreach \n in {10}
{
	\section{March \n}
	\input{mar/\n}
}

\foreach \n in {15completions}
% \foreach \n in {10}
{
	\section{March 15}
	\input{mar/\n}
}

\chapter{Introduction to Dimension}

\epigraph{In this sense the algebraic geometers have never left paradise: There is no snake (that is, Peano curve) in the garden.}
{---David Eisenbud}

\foreach \n in {15dimension}
{
	\section{March 15}
	\input{mar/\n}
}

\foreach \n in {17,29,31}
{
	\section{March \n}
	\input{mar/\n}
}

\foreach \n in {5,7}
{
	\section{April \n}
	\input{apr/\n}
}

\chapter{Higher Dimensions}

\epigraph{To deal with a 14-dimensional space, visualize a 3-D space and say `fourteen' to yourself very loudly. Everyone does it.}
{---Geoffrey Hinton}

\foreach \n in {12,14,19,26}
{
	\section{April \n}
	\input{apr/\n}
}

% quote for later chapter: "To deal with a 14-dimensional space, visualize a 3-D space and say 'fourteen' to yourself very loudly. Everyone does it."
% Algebra is the offer made by the devil to the mathematician...All you need to do, is give me your soul: give up geometry --Michael Atiyah

% localization of noncommutative rings?
% can all ideals appear as Ann M?
% reference for support of tensor product of fin gen mods?
% Hilbert polynomial at negative indices?
% geometric view of Jacobson radical?

% is there a module over a non-noetherian ring with no associated primes?
% geometric view of idempotent splitting R = Re x R(1-e)

% notion for galois theory for integral extensions?
% doing linear algebra in noncommutative rings?

% when does the irrelevant ideal-adic filtration equal the graded filtration?

% does the Krull topology on the completion communicate with the Zariski topology

% non-noetherian rings with trivial dimension
% dimension in the noncommutative case?
% connection between completion and localization?

% realize the invertible module examples as fractional ideals
% invertible modules as line bundles?

% \todo{T2.125, L4.120, 5.2.2, 5.2.3, 5.2.4, 5.3?}

\nirprintindex

\end{document}