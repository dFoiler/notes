% !TEX root = ../notes.tex

Here we go.

\subsection{The Nullstellensatz, Special Case}
Today we prove Hilbert's Nullstellensatz. Here is the statement.
\begin{theorem}[Nullstellensatz] \label{thm:nullstellensatz}
	Fix $k$ an algebraically closed field.
	\begin{listalph}
		\item There is a bijection between algeraic sets $X\subseteq\AA^n(k)$ and radical ideals $J\subseteq k[x_1,\ldots,x_n]$ by taking
		\[X\mapsto I(X):=\{f\in k[x_1,\ldots,x_n]:f(p)=0\text{ for all }p\in X\},\]
		and
		\[J\mapsto Z(J):=\left\{p\in\AA^n(k):f(p)=0\text{ for all }p\in J\right\}.\]
		In particular, $I(Z(J))=J$ and $Z(I(X))=X$.
		\item Points $p$ of $X$ are in bijection with maximal ideals of $A(X):=k[x_1,\ldots,x_n]/I(X)$, which are in bijection with maximal ideals of $k[x_1,\ldots,x_n]$ containing $I(X)$.
	\end{listalph}
\end{theorem}
\begin{proof}[Proof of \autoref{thm:nullstellensatz} for uncountable fields]
	We start with a proof where $k$ is an uncountable field; in other words, one should read $k=\CC$ into the following proof. We start with part (b). We have the following lemma.
	\begin{lemma}
		Fix $k$ an uncountable field and $F/k$ a field extension with $\#[F:k]=\#\NN$. Then the extension $F/k$ is algebraic.
	\end{lemma}
	\begin{proof}
		The main point is that $\#k>\#[F:k]$. Indeed, we work by contrapositive. Suppose that $F/k$ is not algebraic so that we are promised some element $x\in F$ which is not algebraic over $k$, and we show $\#[F:k]\ge\#k$. But then $k(t)\subseteq F$. But then the set
		\[\left\{\frac1{x-a}:a\in k\right\}\]
		is an uncountable $k$-linearly independent set in $F$, forcing $\#[F:k]\ge\#k$.
	\end{proof}
	\begin{corollary}
		Fix $k$ an uncountable field. Then, for any maximal ideal $\mf m\subseteq k[x_1,\ldots,x_n]$, the field extension
		\[\frac{k[x_1,\ldots,x_n]}{\mf m}\supseteq k\]
		is algebraic.
	\end{corollary}
	\begin{proof}
		The degree of the extension is countable and hence algebraic.
	\end{proof}
	Thus, when $k$ is algebraic and $\mf m\subseteq k[x_1,\ldots,x_n]$, we have that
	\[\frac{k[x_1,\ldots,x_n]}{\mf m}\]
	is an algebraic extension of $k$, which gives part (b). In particular, one can track where each of the $x_\bullet$ go to in the isomorphism (say, $a_\bullet$) and use this to describe $\mf m$ (as $(x_i-a_i)_{i=1}^n$).

	Now we attack part (a).
	\begin{lemma} \label{lem:betterjacobson}
		Fix $k$ an algebraically closed field, and let $R:=k[x_1,\ldots,x_n]$. Then any prime ideal $\mf p\subseteq R$ is the intersection of the maximal ideals containing $\mf p$.
	\end{lemma}
	\begin{proof}
		If $\mf p$ is maximal, then there is nothing to say. Of course it is true that
		\[\mf p\subseteq\bigcap_{\mf m\supseteq\mf p}\mf m.\]
		The other inclusion is harder. For the other inclusion, we pick up $b\notin\mf p$, and we need to find some maximal ideal $\mf m$ containing $\mf p$ but not $b$.

		For this, we work in $R/\mf p\left[b^{-1}\right]$. We note that $R/\mf p\left[b^{-1}\right]$ is not a field: % \todo{}
		it is an algebraic extension of $k$, but $k$ is algebraically closed, so this would force $b^{-1}\in k$, which is false. Thus, $R/\mf p\left[b^{-1}\right]$ will have some maximal ideal which we can pull back to a maximal ideal $\mf m\subseteq R$ which contains $\mf p$ but not $b$.
	\end{proof}
	Now we show part (a). Because $I$ is a radical ideal, we can use \autoref{prop:radprimes} to write
	\[I=\bigcap_{\mf p\supseteq I}\mf p=\bigcap_{\mf p\supseteq I}\bigcap_{\mf m\supseteq\mf p}\mf m=\bigcap_{\mf m\supseteq I}\mf m,\]
	where we have used \autoref{lem:betterjacobson}. However, we remark that we have classified our maximal ideals! A maximal ideal containing $I$ is the same as an ideal $(x_1-a_1,\ldots,x_n-a_n)$ where $(a_1,\ldots,a_n)\in Z(I)$. Thus, $I=I(Z(I))$, and we are done. (That $Z(I(X))=X$ is easier, and we have showed it before.) % \todo{}
\end{proof}

\subsection{The Nullstellensatz, General Proof}
We now provide an alternative, more general proof.
\begin{proof}[General proof of \autoref{thm:nullstellensatz}]
	We have the following definition.
	\begin{definition}[Jacobson]
		A ring $R$ is \textit{Jacobson} if and only if any prime ideal is the intersection of some maximal ideals. In other words, by pulling back from $R/\mf p$, for each prime $\mf p$, we have $\rad R/\mf p=(0)$.
	\end{definition}
	\begin{example}
		The ring $\ZZ$ is Jacobson because all nonzero primes are maximal, and
		\[(0)=\bigcap_{p\ne0}(p).\]
	\end{example}
	\begin{example}
		For the same reason, $k[x]$ is Jacobson.
	\end{example}
	\begin{nex}
		A local domain which is not a field is not Jacobson; e.g., $\ZZ_2$ is not local. The issue is that being local implies that there is only one maximal ideal, but it is not $(0)$ because we are not in a field, but $(0)$ is some prime because we are in a domain.
	\end{nex}
	We will want to show that $k[x_1,\ldots,x_n]$ is Jacobson, akin to \autoref{lem:betterjacobson}.
	\begin{lemma}
		Fix $R$ a domain but not a field. Then $\rad R=(0)$ if and only if $R\left[b^{-1}\right]$ is not a field for any $b\in R\setminus\{0\}$.
	\end{lemma}
	\begin{proof}
		The main point is that ideals of $R\left[b^{-1}\right]$ are in one-to-one correspondence with ideals of $R$ which avoid $b$. To be explicit, if $R\left[b^{-1}\right]$ is a field for some $b\in R\setminus\{0\}$, then all maximal ideals will have to pull back from $(0)$ and hence contain $b$. In the other direction, if
		\[\bigcap_{\mf m}\mf m=0,\]
		then any maximal ideal avoiding $b$ will witness $R\left[b^{-1}\right]$ having a nonzero proper ideal.
	\end{proof}
	\begin{corollary} \label{cor:betterjacobson}
		Fix $R$ a ring. Then $R$ is Jacobson if and only if each prime but not maximal prmie $\mf p$ has $R/\mf p\left[b^{-1}\right]$ not a field for each $b\notin\mf p$.
	\end{corollary}
	\begin{proof}
		Use the lemma on $R/\mf p$.
	\end{proof}
	And here is our main result.
	\begin{theorem} \label{thm:gennullstellensatz}
		Fix $R$ a Jacobson ring and $S$ a finitely generated $R$-algebra.
		\begin{listalph}
			\item Then $S$ is a Jacobson ring.
			\item For each maximal ideal $\mf m\subseteq S$, we have $\mf m\cap R$ maximal in $R$, and
			\[\frac R{\mf m\cap R}\subseteq\frac S{\mf m}\]
			is a finite extension.
		\end{listalph}
	\end{theorem}
	\autoref{thm:nullstellensatz} will follow from this, essentially using the same argument from before. Explicitly, (b) above gives (b) for \autoref{thm:nullstellensatz}, and (a) above combined with the argument from the uncountable case will prove (a).
	\begin{proof}[Proof of \autoref{thm:gennullstellensatz}]
		By induction, it will suffice to show the case where $S$ is generated by a single element over $R$. Namely, if $n$ is generated by $x_1,\ldots,x_n$, induction can get us up to $R[x_1,\ldots,x_{n-1}]$, and then we add $x_n$ and mod out by the necessary statement. So let $S=R[t]/J$ for some $J\subseteq R[t]$.

		We begin with (a). The main point is to use \autoref{cor:betterjacobson}. Well, fix $\mf p$ a prime of $R$ which is not maximal. Now, the main point is that we have an integral extension
		\[\underbrace{\frac R{R\cap\mf p}}_{R'}\subseteq\underbrace{\frac S{\mf p}}_{S'}.\]
		Using the same $t$ and polynomial $I$, we see that $S'$ is still generated over $R'$ by a single element, so we write $S'=R'[t]/I$. We have two cases; it suffices to show that $S'\left[b^{-1}\right]$ is not a field for any $b\ne0$.
		\begin{itemize}
			\item Take $I=0$. Then, for any $b\ne0$ in $R'$, we are easily not going to have $R'[t]\left[b^{-1}\right]$ not a field because $R'[t]$ was not a field.
			\item Otherwise, take our $b\in S'$. Because $S'$ is finitely generated over $R'$, we note that $t$ must satisfy some polynomial equation
			\[a_nt^n+a_{n-1}t^{n-1}+\cdots+a_0=0,\]
			where $a_\bullet\in R$. Now, $R'\left[a_0^{-1}\right]$ is not a field because $R'$ is Jacobson (!). Continuing, we write
			\[c_nb^n+c_{n-1}b^{n-1}+\cdots+c_0=0,\]
			where $c_\bullet\in R$. But now we see that
			\[R'\left[c_m^{-1}a_0^{-1}\right]\subseteq S'\left[c_m^{-1}a_0^{-1},b^{-1}\right]\]
			will be an integral extension, so the right-hand ring cannot be a field because it would force the left-hand side to give a field by integrality.
		\end{itemize}
		Lastly, we note that part (b) is the case that we just proved in the case where $I$ is maximal.
	\end{proof}
	\noindent The above theorem finishes the proof, as discussed. So we are done.
\end{proof}
\begin{remark}
	The midterm will only include up to Nakayama's lemma because we have not done homework on the other content.
\end{remark}

\subsection{Example Problems}
Let's do some example problems, to review.
\begin{exe} \label{exe:det}
	Fix a field $k$. We work in $k^{n\times n}$. We show that the ideal
	\[\det\begin{bmatrix}
		x_{11} & \cdots & x_{1n} \\
		\vdots & \ddots & \vdots \\
		x_{n1} & \cdots & x_{nn}
	\end{bmatrix}.\]
	We show that $(\det X)$ is a prime ideal in $k[x_{ij}]$. Note that it suffices to show $\det X$ is irreducible.
\end{exe}
\begin{ex}
	In the case of $n=2$, we are showing $\det X=x_{11}x_{22}-x_{12}x_{21}$ is irreducible. Well, for any $x_{ij}$, if we could write
	\[\det X=f(X)g(X),\]
	then we must have $\deg_{x_{ij}}f=0$ or $\deg_{x_{ij}}g=0$. In particular, we have two cases.
	\begin{itemize}
		\item We might have $(x_{11}x_{22}-b)c$ for some $b$ and $c$. But then this forces $c=1$.
		\item We might have $(x_{11}-b)(x_{22}-c)$ for some $b$ and $c$. But then $cx_{11}$ would have to live in the polynomial, so $cx_{11}=0$, and similar for $bx_{22}$, causing everything to collapse.
	\end{itemize}
\end{ex}
\begin{proof}[Proof of \autoref{exe:det}]
	Use expansino by minors to write
	\[\det X=x_{11}\det X_{11}-q,\]
	where $X_{11}$ is $X$ without the top and left row. But if we tried to factor, then $x_{11}$ cannot live on its own, so we can factor
	\[\det X=X_{11}\left(x_{11}-\frac q{X_{11}}\right).\]
	But $X_{11}$ is irreducible by proceeding inductively, so we cannot have something like this. % \todo{what just happenedd}
\end{proof}
\begin{exe}
	Fix $R=k[x,y]$ and $M=k[x,y]/\left(x^2,xy\right)$.
	\begin{itemize}
		\item We compute $\op{Ass}M$.
		\item We compute $\op{Supp}M$.
		\item We compute $H_M(s)$.
	\end{itemize}
\end{exe}
\begin{proof}
	We start with $H_M(s)$. Let's tabulate.
	\begin{itemize}
		\item $H_M(0)=1$, with $1$.
		\item $H_M(1)=2$, with $x$ and $y$.
		\item $H_M(2)=1$ with $y^2$.
		\item In fact, $H_M(s)=1$ for $s>1$ with $y^s$ because all other monomials have $xy$ killed.
	\end{itemize}
	The point is that $M$ looks like $k[x,y]/(x)$, which is $x=0$.

	Now we compute $\op{Supp}M$. Because $M$ is a finitely generated module over a Noetherian ring, the support consists of the primes $\mf p\subseteq k[x,y]$ containing $\op{Ann}M=\left(x^2,xy\right)$. Well, any such prime must contain $(x)$, and then we can have any larger prime, which look like $(x,y-a)$ for any $a\in k$. These are the only primes containing $(x)$ by considering $k[x]/(x)$.

	Lastly, we compute $\op{Ass}M$. Well, we note $\op{Ann}y=(x)$ and $\op{Ann}x=(x,y)$. However, the other elements of the support take the form $(x,y-a)$, which will not be annihilators.
\end{proof}
\begin{remark}
	Professor Serganova recommends doing exercises 2.19, 2.22, and 4.11 from Eisenbud.
\end{remark}