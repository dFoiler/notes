% !TEX root = ../notes.tex

Here we go.

\subsection{The Nullstellensatz}
Today we prove Hilbert's Nullstellensatz. Here is the statement.
\begin{theorem}[Nullstellensatz] \label{thm:nullstellensatz}
	Fix $k$ an algebraically closed field.
	\begin{listalph}
		\item There are bijections between algebraic sets $X\subseteq\AA^n(k)$ and radical ideals $J\subseteq k[x_1,\ldots,x_n]$ by taking
		\[X\mapsto I(X):=\{f\in k[x_1,\ldots,x_n]:f(p)=0\text{ for all }p\in X\},\]
		and
		\[J\mapsto Z(J):=\left\{p\in\AA^n(k):f(p)=0\text{ for all }p\in J\right\}.\]
		In particular, $I(Z(J))=J$ and $Z(I(X))=X$.
		\item Points $p$ of an algebraic set $X\subseteq\AA^n(k)$ are in bijection with maximal ideals of $k[x_1,\ldots,x_n]/I(X)$, which are in bijection with maximal ideals of $k[x_1,\ldots,x_n]$ containing $I(X)$.
	\end{listalph}
\end{theorem}
Before jumping into the proof, we give some remarks on what we can show without too much effort. For example, back in \autoref{rem:othernullstellensatz}, we showed that $Z(I(X))=X$, so the harder direction is that $I(Z(J))=J$ for $J$ a radical ideal.

In fact, we note that $J\subseteq I(Z(J))$ is fairly easy as well: for each $f\in J$, we note that $f$ will vanish on any $a\in Z(J)$ by definition of $Z(J)$, so $f\in I(Z(J))$ follows. Thus, the hart part of (a) is showing
\[J\stackrel?\supseteq I(Z(J)).\]
We also remark that the last claim of (b) is merely ring theory.
\begin{lemma} \label{lem:pullmax}
	Fix $R$ a ring and $I\subseteq R$ an ideal. Then maximal ideals of $R/I$ are in bijection with maximal ideals of $R$ containing $I$.
\end{lemma}
\begin{proof}
	Let $\pi:R\onto R/I$ be the canonical projection. We send maximal ideals $\mf m$ of $R$ containing $I$ to the ideal $\pi(\mf m)\subseteq R/I$; conversely, we send maximal ideals $\mf m\subseteq R/I$ to the ideal $\pi^{-1}(\mf m)$. We have the following checks.
	\begin{itemize}
		\item Fix $J$ an ideal containing $I$. We claim $\pi^{-1}(\pi(J))=J$. To see this, we note $x\in\pi^{-1}(\pi(J))$ if and only if $\pi(x)\in\pi(J)$ if and only if $[x]_I=[y]_I$ for some $y\in J$ if and only if $x-y\in I\subseteq J$ if and only if $x\in J$.
		\item Similarly, fix an ideal $J\subseteq R/I$. We claim $\pi\left(\pi^{-1}(J)\right)=J$. To see this, we note $\pi(y)\in\pi\left(\pi^{-1}(J)\right)$ if and only if $y\in\pi^{-1}(J)$ if and only if $\pi(y)\in J$.
		\item Fix $\mf m\subseteq R/I$, and we show that $\pi^{-1}(\mf m)$ is a maximal ideal of $R$. Now, we know that $\pi^{-1}(\mf m)$ is proper because it is prime. Additionally, if $\pi^{-1}(\mf m)\subseteq J$ for some ideal $J$, then $\mf m\subseteq\pi(J)$, so $\pi(J)=\mf m$ or $\pi(J)=R/I$. In the former case, $\pi^{-1}(\mf m)=J$; in the latter case, $J=\pi^{-1}(R/I)=R$.
		\item Fix $\mf m\subseteq R$ a maximal ideal containing $I$, and we show that $\pi(\mf m)$ is a maximal ideal of $R/I$. Note $[1]_I\in\pi(\mf m)$ would imply that $1+x\in\mf m$ for some $x\in I$, so $1\in\mf m$ because $I\subseteq\mf m$. But $1\in\mf m$ is false, so we see that $\pi(\mf m)$ is proper.

		Now, $\pi(\mf m)\subseteq J$ for some ideal $J\subseteq R/I$ implies that $\mf m\subseteq\pi^{-1}(J)$, so $\pi^{-1}(J)=\mf m$ or $\pi^{-1}(J)=R$. Note that $[0]_I\in J$ implies $I\subseteq\pi^{-1}(J)$. So we may say that, in the former case, $J=\pi(\mf m)$; in the latter case, $J=\pi(R)=R/I$.
	\end{itemize}
	So we see that the described maps are mutually inverses and well-defined, so we are done.
\end{proof}
There is a little more that we can say about (b): it is not too hard to reduce it to the case where $X=\AA^n(k)$ and $I(X)=\emp$.
\begin{lemma} \label{lem:nzeasierb}
	Suppose that all maximal ideals of $k[x_1,\ldots,x_n]$ take the form $(x_1-a_1,\ldots,x_n-a_n)$ for $(a_1,\ldots,a_n)\in\AA^n(k)$. Then (b) of \autoref{thm:nullstellensatz} holds.
\end{lemma}
\begin{proof}
	By the previous lemma, we only have to show the first sentence of (b). Our bijection will be by
	\[(a_1,\ldots,a_n)\in X\longmapsto(x_1-a_1,\ldots,x_n-a_n)\subseteq k[x_1,\ldots,x_n].\]
	We have the following checks to show (b).
	\begin{itemize}
		\item We check that $(x_1-a_1,\ldots,x_n-a_n)$ is maximal. To see this, we claim that $(x_1-a_1,\ldots,x_n-a_n)$ is the kernel of the surjective map
		\[\varphi\in k[x_1,\ldots,x_n]\to k\]
		defined by lifting $\id_k:k\to k$ by $x_i\mapsto a_i$, which will be enough. To see this, note that certainly each $x_i-a_i$ will live in the kernel. Conversely, for any $f\in k[x_1,\ldots,x_n]$, we can apply the division algorithm to $f$ by each of the $x_i-a_i$ to write
		\[f(x_1,\ldots,x_n)=f(a_1,\ldots,a_n)+\sum_{i=1}^n(x_i-a_i)q_i(x).\]
		Formally, one should show this by induction on $n$, but we won't bother. The point is that $f\in\ker\varphi$ implies that $f(a_1,\ldots,a_n)\in\ker\varphi$, so $f(a_1,\ldots,a_n)=0$, so $f\in(x_1-a_1,\ldots,x_n-a_n)$.
		\item We check that $(x_1-a_1,\ldots,x_n-a_n)$ contains $I(X)$. Indeed, if $f\in I(X)$, then $f$ vanishes on $(a_1,\ldots,a_n)$, so $f$ lives in the kernel $\ker\varphi$ constructed above, so $f\in(x_1-a_1,\ldots,x_n-a_n)$.
		\item We show the map is injective. The key claim is that
		\[Z((x_1-a_1,\ldots,x_n-a_n))=\{(a_1,\ldots,a_n)\}.\]
		Indeed, if $(b_1,\ldots,b_n)$ lives in this vanishing set, then $b_i-a_i=0$ for each $i$, so $(a_1,\ldots,a_n)=(b_1,\ldots,b_n)$. Of course, each $x_i-a_i$ does vanish on $(a_1,\ldots,a_n)$, so we are done.

		So to finish, we note that $(x_1-a_1,\ldots,x_n-a_n)=(x_1-a_1',\ldots,x_n-a_n')$ implies that their vanishing sets match, so $(a_1,\ldots,a_n)=(a_1',\ldots,a_n')$, so we are done.
		\item We show the map is surjective. This requires some trickery. Suppose $\mf m$ is a maximal containing $I(X)$. Because $\mf m$ is maximal, we do know that
		\[\mf m=(x_1-a_1,\ldots,x_n-a_n)\]
		by hypothesis. So we see that $\mf m\supseteq I(X)$ implies that
		\[X=Z(I(X))\supseteq Z(\mf m)=\{(a_1,\ldots,a_n)\},\]
		where we have used \autoref{rem:othernullstellensatz} in the first equality. Thus, $\mf m$ is indeed of the required form, so we are done.
		\qedhere
	\end{itemize}
\end{proof}

\subsection{The Uncountable Case}
Let's start with an easier special case.
\begin{proof}[Proof of \autoref{thm:nullstellensatz} for uncountable fields]
	We prove \autoref{thm:nullstellensatz} where $k$ is an uncountable field; in other words, one should read $k=\CC$ into the following proof. We will actually start by showing (b) in the case where $X=\AA^n(k)$ and $I(X)=\emp$. The following will be the way we use that $k$ is uncountable.
	\begin{lemma} \label{lem:useuncountable}
		Fix $k$ an uncountable field and $F/k$ a field extension with $[F:k]<\#k$. Then the extension $F/k$ is algebraic.
	\end{lemma}
	\begin{proof}
		We show the contrapositive. Suppose that $F/k$ is not algebraic, and we show that $[F:k]\ge\#k$. Because $F/k$ is not algebraic, we are promised some element $x\in F$ which is not algebraic over $k$. But then $k(x)\subseteq F$ is a very large subfield, so we consider the set
		\[S:=\left\{\frac1{x-a}:a\in k\right\}.\]
		We quickly observe that these elements are legal: note that $x\ne a$ for each $a\in k$ because $k/k$ is an algebraic extension; thus, $\frac1{x-a}$ is a legal element of $F$.
		
		We claim that $S\subseteq k(x)$ is $k$-linearly independent, which will show that $[F:k]=\dim_kF\ge\dim_kk(x)\ge\#S=\#k$, which is what we want. Now, to show that $S$ is $k$-linearly independent, suppose that we have a relation
		\[\sum_{i=1}^nr_i\cdot\frac1{x-a_i}=0\]
		for some $r_1,\ldots,r_n\in k$ and distinct $a_1,\ldots,a_n\in k$. We need to show that $r_i=0$ for each $r_i$. For this, we note that
		\[0=\left(\prod_{i=1}^n(x-a_i)\right)\left(\sum_{i=1}^n\frac{r_i}{x-a_i}\right)=\sum_{i=1}^n\Bigg(r_i\prod_{\substack{j=1\\j\ne i}}^n(x-a_j)\Bigg).\tag{$*$}\label{eq:readytoplug}\]
		Now, though this equation is technically taking place in $k(x)$, we may pull it back to an equation in $k[x]$ (noting that $k[x]\into k(x)$ is injective).

		But with our equation holding in $k[x]$, we note that $k[x]\subseteq F$ is a free $k$-algebra,\footnote{More formally, note there is a morphism $k[T]\to k[x]$ extending $\id_k:k\to k$ by sending $T\mapsto x$. It is not hard to see that this is surjective, and it is injective because it has trivial kernel because $x$ is transcendental. So $k[x]\cong k[T]$.} so we may apply the universal property of $k[x]$ to note there is a morphism $k[x]\to k$ extending $\id_k:k\to k$ by sending $x\mapsto a_m$ for any $a_m$. Pushing \autoref{eq:readytoplug} through this morphism, we see
		\[\sum_{i=1}^n\Bigg(r_i\prod_{\substack{j=1\\j\ne i}}^n(a_m-a_j)\Bigg)=0.\]
		All terms of the sum will vanish except when $i=m$ because the product will feature a $(a_m-a_m)$ term otherwise. So we see
		\[r_m\prod_{\substack{j=1\\j\ne m}}^n(a_m-a_j)=0.\]
		Because the $a_i$ are all distinct, we see $a_m-a_j\ne0$ for each $m\ne j$, so the entire product is nonzero ($k$ is an integral domain), so $r_m=0$. This finishes.
	\end{proof}
	\begin{corollary} \label{cor:forcealgebraic}
		Fix $k$ an uncountable field. Then, for any maximal ideal $\mf m\subseteq k[x_1,\ldots,x_n]$, the field extension
		\[\frac{k[x_1,\ldots,x_n]}{\mf m}\supseteq k\]
		is algebraic.
	\end{corollary}
	\begin{proof}
		We quickly note that $\mf m\cap k=(0)$ because otherwise $\mf m$ would contain a unit; thus, the map $k\into k[x_1,\ldots,x_n]/\mf m$ is indeed injective, so we do have a sane field extension.
		
		Now, recall that any element $k[x_1,\ldots,x_n]$ can be written (uniquely) in the form
		\[\sum_{(d_1,\ldots,d_n)\in\NN^n}a_{(d_1,\ldots,d_n)}x_1^{d_1}\cdots x_n^{d_n},\]
		where all but finitely many of the $a_\bullet\in k$ vanish. Thus, the monomials $x_1^{d_1}\cdots x_n^{d_n}$ will generate $k[x_1,\ldots,x_n]$ and hence span $k[x_1,\ldots,x_n/\mf m$. In particular,
		\[\dim_k\frac{k[x_1,\ldots,x_n]}{\mf m}\le\#\left\{x_1^{d_1}\cdots x_n^{d_n}:(d_1,\ldots,d_n)\in\NN^n\right\}=\#\left(\NN^n\right).\]
		However, $\NN^n$ is countable, so $\dim_k\frac{k[x_1,\ldots,x_n]}{\mf m}\le\#\NN<\#k$, so \autoref{lem:useuncountable} assures us that the extension $\frac{k[x_1,\ldots,x_n]}{\mf m}\supseteq k$ is an algebraic extension.
	\end{proof}
	So now we can show the hypothesis of \autoref{lem:nzeasierb} without tears. As discussed, we need to show that all maximal ideals $\mf m\subseteq k[x_1,\ldots,x_n]$ take the form $(x_1-a_1,\ldots,x_n-a_n)$.
	
	Well, picking up some maximal ideal $\mf m\subseteq k[x_1,\ldots,x_n]$ is maximal, we have that
	\[\frac{k[x_1,\ldots,x_n]}{\mf m}\]
	is an algebraic extension of $k$ by \autoref{cor:forcealgebraic}, but $k$ is algebraically closed, so this field must equal $k$. In particular, we are promised an isomorphism
	\[\varphi:\frac{k[x_1,\ldots,x_n]}{\mf m}\cong k.\]
	We can lift this to a map
	\[\overline\varphi:k[x_1,\ldots,x_n]\to k\]
	with kernel $\mf m$. But $x_i-\varphi(x_i)$ must certainly live in the kernel of $\overline\varphi$, so
	\[(x_1-\varphi(x_1),\ldots,x_n-\varphi(x_n))\subseteq\mf m.\]
	But the left-hand ideal is maximal, so equality follows. So indeed, all maximal ideals of $k[x_1,\ldots,x_n]$ have the requested form.

	Now we attack part (a). In addition to (b), we will need the following technical result.
	\begin{lemma} \label{lem:betterjacobson}
		Fix $k$ an algebraically closed field, and let $R:=k[x_1,\ldots,x_n]$. Then any prime ideal $\mf p\subseteq R$ is the intersection of the maximal ideals containing $\mf p$.
	\end{lemma}
	\begin{proof}
		If $\mf p$ is maximal, then there is nothing to say. Thus, we may take $\mf p$ prime but not maximal so that $R/\mf p$ is a domain but not a field. In one direction, we note that
		\[\mf p=\bigcap_{\mf m\supseteq\mf p}\mf p\subseteq\bigcap_{\mf m\supseteq\mf p}\mf m.\]
		The other inclusion is harder. To show it, we proceed by contraposition: pick up $b\notin\mf p$, and we find some maximal ideal $\mf m$ containing $\mf p$ but not $b$ so that $b\notin\bigcap_{\mf m\supseteq\mf p}\mf m$.

		For this, we work in $R/\mf p\left[[b]_\mf p^{-1}\right]$. We claim that $R/\mf p\left[[b]_\mf p^{-1}\right]$ is not a field. If $R/\mf p\left[[b]_\mf p^{-1}\right]$ is a field, then because it has countable degree over $k$ (it is spanned by products of powers of $b^{-1}$ and monomials of $R$, of which there are countably many), we see that
		\[k\subseteq R/\mf p\left[[b]_\mf p^{-1}\right]\]
		is an algebraic extension by \autoref{cor:forcealgebraic}. But because $k$ is algebraically closed, this extension must collapse, implying that $[b]_\mf p^{-1}$ is algebraic over $k$. Because $k$ is a field, we may give $[b]_\mf p^{-1}$ a monic polynomial in $k[x]$, which we notate by
		\[\left([b]_\mf p^{-1}\right)^m+a_{m-1}\left([b]_\mf p^{-1}\right)^{m-1}+\cdots+a_1\left([b]_\mf p^{-1}\right)+a_0=0.\]
		However, this polynomial also shows that $[b]_\mf p^{-1}$ is the root of some monic polynomial in $(R/\mf p)[x]$, so $[b]_\mf p^{-1}$ is integral over $R/\mf p$. But now we note $R/\mf p$ is an integral domain and $[b]_\mf p\ne0$ implies that
		\[R/\mf p\subseteq(R/\mf p)\left[[b]_\mf p^{-1}\right]\]
		is an embedding (\autoref{ex:intdomainlocalinject}), so $R/\mf p$ is a field by \autoref{prop:integralfields}. But we presupposed that $R/\mf p$ is not a field, so we have hit a contradiction.

		So because $R/\mf p\left[b^{-1}\right]$ is not a field, we have the following movements.
		\begin{itemize}
			\item We will have some nonzero maximal ideal $\mf m\subseteq R/\mf p\left[[b]_\mf p^{-1}\right]$.
			\item Because we still know
			\[R/\mf p\subseteq(R/\mf p)\left[[b]_\mf p^{-1}\right]\]
			is an integral extension of domains, we can use \autoref{rem:goingdownmax} to pull $\mf m$ back to a maximal ideal $\mf m'$ of $R/\mf p$. Note $\mf m'$ will not contain $[b]_\mf p$ by \autoref{thm:localizedprimes}.
			\item Lastly, we can pull $\mf m'\subseteq R/\mf p$ to a maximal ideal $\mf m'+\mf p\subseteq R$ containing $\mf p$ by \autoref{lem:pullmax}. Because $[b]_\mf p\notin\mf m'$, we see $b\notin\mf m'+\mf p$ as well.
		\end{itemize}
		So we see that $\mf m'+\mf p\subseteq R$ is the maximal ideal we are looking for.
	\end{proof}
	Now we show part (a). Fix $J$ a radical ideal, and we will show $J\supseteq I(Z(J))$. We can use \autoref{prop:radprimes} to write
	\[J=\bigcap_{\mf p\supseteq J}\mf p\stackrel*=\bigcap_{\mf p\supseteq J}\bigcap_{\mf m\supseteq\mf p}\mf m=\bigcap_{\mf m\supseteq J}\mf m,\]
	where we have used \autoref{lem:betterjacobson} in $\stackrel*=$. Thus, fixing $f\in I(Z(J))$, it will suffice to show that $f\in\mf m$ for any $\mf m\supseteq J$.
	
	However, we classified our maximal ideals above! So we get to write $\mf m=(x_1-a_1,\ldots,x_n-a_n)$, which is the kernel of the ``evaluation at $(a_1,\ldots,a_n)$'' map by \autoref{lem:nzeasierb}. In particular, we note $\mf m\supseteq J$ tells us that
	\[Z(J)\supseteq Z(\mf m)=\{(a_1,\ldots,a_n)\},\]
	as computed in \autoref{lem:nzeasierb}. So $f\in I(Z(J))$ implies that $f$ vanishes on $Z(J)$, so $f$ vanishes on $(a_1,\ldots,a_n)$, so $f$ lives in the kernel of the ``evaluation at $(a_1,\ldots,a_n)$'' map, so $f\in\mf m$. This finishes.
\end{proof}

\subsection{The General Case}
We now provide an alternative, more general proof.
\begin{proof}[General proof of \autoref{thm:nullstellensatz}]
	We have the following definition.
	\begin{definition}[Jacobson]
		A ring $R$ is \textit{Jacobson} if and only if any prime ideal is the intersection of some maximal ideals. In other words, by pulling back from $R/\mf p$, for each prime $\mf p$, we have $\rad R/\mf p=(0)$.
	\end{definition}
	\begin{example}
		The ring $\ZZ$ is Jacobson because all nonzero primes are maximal, and
		\[(0)=\bigcap_{p\ne0}(p).\]
	\end{example}
	\begin{example}
		For the same reason, $k[x]$ is Jacobson.
	\end{example}
	\begin{nex}
		A local domain which is not a field is not Jacobson; e.g., $\ZZ_2$ is not local. The issue is that being local implies that there is only one maximal ideal, but it is not $(0)$ because we are not in a field, but $(0)$ is some prime because we are in a domain.
	\end{nex}
	We will want to show that $k[x_1,\ldots,x_n]$ is Jacobson, akin to \autoref{lem:betterjacobson}.
	\begin{lemma}
		Fix $R$ a domain but not a field. Then $\rad R=(0)$ if and only if $R\left[b^{-1}\right]$ is not a field for any $b\in R\setminus\{0\}$.
	\end{lemma}
	\begin{proof}
		The main point is that ideals of $R\left[b^{-1}\right]$ are in one-to-one correspondence with ideals of $R$ which avoid $b$. To be explicit, if $R\left[b^{-1}\right]$ is a field for some $b\in R\setminus\{0\}$, then all maximal ideals will have to pull back from $(0)$ and hence contain $b$. In the other direction, if
		\[\bigcap_{\mf m}\mf m=0,\]
		then any maximal ideal avoiding $b$ will witness $R\left[b^{-1}\right]$ having a nonzero proper ideal.
	\end{proof}
	\begin{corollary} \label{cor:betterjacobson}
		Fix $R$ a ring. Then $R$ is Jacobson if and only if each prime but not maximal prime $\mf p$ has $R/\mf p\left[b^{-1}\right]$ not a field for each $b\notin\mf p$.
	\end{corollary}
	\begin{proof}
		Use the lemma on $R/\mf p$.
	\end{proof}
	And here is our main result.
	\begin{theorem} \label{thm:gennullstellensatz}
		Fix $R$ a Jacobson ring and $S$ a finitely generated $R$-algebra.
		\begin{listalph}
			\item Then $S$ is a Jacobson ring.
			\item For each maximal ideal $\mf m\subseteq S$, we have $\mf m\cap R$ maximal in $R$, and
			\[\frac R{\mf m\cap R}\subseteq\frac S{\mf m}\]
			is a finite extension.
		\end{listalph}
	\end{theorem}
	\autoref{thm:nullstellensatz} will follow from this, essentially using the same argument from before. Explicitly, (b) above gives (b) for \autoref{thm:nullstellensatz}, and (a) above combined with the argument from the uncountable case will prove (a).
	\begin{proof}[Proof of \autoref{thm:gennullstellensatz}]
		By induction, it will suffice to show the case where $S$ is generated by a single element over $R$. Namely, if $n$ is generated by $x_1,\ldots,x_n$, induction can get us up to $R[x_1,\ldots,x_{n-1}]$, and then we add $x_n$ and mod out by the necessary statement. So let $S=R[t]/J$ for some $J\subseteq R[t]$.

		We begin with (a). The main point is to use \autoref{cor:betterjacobson}. Well, fix $\mf p$ a prime of $R$ which is not maximal. Now, the main point is that we have an integral extension
		\[\underbrace{\frac R{R\cap\mf p}}_{R'}\subseteq\underbrace{\frac S{\mf p}}_{S'}.\]
		Using the same $t$ and polynomial $I$, we see that $S'$ is still generated over $R'$ by a single element, so we write $S'=R'[t]/I$. We have two cases; it suffices to show that $S'\left[b^{-1}\right]$ is not a field for any $b\ne0$.
		\begin{itemize}
			\item Take $I=0$. Then, for any $b\ne0$ in $R'$, we are easily not going to have $R'[t]\left[b^{-1}\right]$ not a field because $R'[t]$ was not a field.
			\item Otherwise, take our $b\in S'$. Because $S'$ is finitely generated over $R'$, we note that $t$ must satisfy some polynomial equation
			\[a_nt^n+a_{n-1}t^{n-1}+\cdots+a_0=0,\]
			where $a_\bullet\in R$. Now, $R'\left[a_0^{-1}\right]$ is not a field because $R'$ is Jacobson (!). Continuing, we write
			\[c_nb^n+c_{n-1}b^{n-1}+\cdots+c_0=0,\]
			where $c_\bullet\in R$. But now we see that
			\[R'\left[c_m^{-1}a_0^{-1}\right]\subseteq S'\left[c_m^{-1}a_0^{-1},b^{-1}\right]\]
			will be an integral extension, so the right-hand ring cannot be a field because it would force the left-hand side to give a field by integrality.
		\end{itemize}
		Lastly, we note that part (b) is the case that we just proved in the case where $I$ is maximal.
	\end{proof}
	\noindent The above theorem finishes the proof, as discussed. So we are done.
\end{proof}
\begin{remark}
	The midterm will only include up to Nakayama's lemma because we have not done homework on the other content.
\end{remark}

\subsection{Example Problems}
Let's do some example problems, to review.
\begin{exe} \label{exe:det}
	Fix a field $k$. We work in $k^{n\times n}$. We show that the ideal
	\[\det\begin{bmatrix}
		x_{11} & \cdots & x_{1n} \\
		\vdots & \ddots & \vdots \\
		x_{n1} & \cdots & x_{nn}
	\end{bmatrix}=:\det X\]
	is a prime ideal in $k[x_{ij}]$. Note that it suffices to show $\det X$ is irreducible.
\end{exe}
\begin{ex}
	In the case of $n=2$, we are showing $\det X=x_{11}x_{22}-x_{12}x_{21}$ is irreducible. Well, for any $x_{ij}$, if we could write
	\[\det X=f(X)g(X),\]
	then we must have $\deg_{x_{ij}}f=0$ or $\deg_{x_{ij}}g=0$. In particular, we have two cases.
	\begin{itemize}
		\item We might have $(x_{11}x_{22}-b)c$ for some $b$ and $c$. But then this forces $c=1$.
		\item We might have $(x_{11}-b)(x_{22}-c)$ for some $b$ and $c$. But then $cx_{11}$ would have to live in the polynomial, so $cx_{11}=0$, and similar for $bx_{22}$, causing everything to collapse.
	\end{itemize}
\end{ex}
\begin{proof}[Proof of \autoref{exe:det}]
	Use expansion by minors to write
	\[\det X=x_{11}\det X_{11}-q,\]
	where $X_{11}$ is $X$ without the top and left row. By induction, we may assume that $\det X_{11}$ is irreducible.
	
	Now we attempt to factor $\det X=fg$. By looking at the degree of $x_{11}$, we see that exactly one of $f$ or $g$ will have the linear term $x_{11}$, which means we factor into one of the following two forms.
	\begin{itemize}
		\item We might have $\det X=(x_{11}+b)(\det X_{11}+c)$. This cannot be because $x_{11}\det X_{11}$ contains all terms with an $x_{11}$, so $c=0$ is forced. But then $\det X_{11}\mid\det X$, which does not make sense. For example, running the above argument again for $x_{12}$ shows that the analogously defined $X_{12}$ has $\det X_{12}\mid\det X$, but $\det X_{11}$ and $\det X_{12}$ are distinct irreducibles and hence coprime, which forces
		\[\deg\det X\ge\deg\det X_{11}+\deg\det X_{12},\]
		which does not make sense.
		\item We might have $\det X=(x_{11}X_{11}+b)c$. But by degree arguments, we see that $c$ is constant, which means that $c$ is a unit already.
	\end{itemize}
	In particular, no other factorizations are possible because they would require factoring $\det X_{11}$, which is irreducible.
	% But if we tried to factor, then $x_{11}$ cannot live on its own, so we can factor
	% \[\det X=X_{11}\left(x_{11}-\frac q{X_{11}}\right).\]
	% But $X_{11}$ is irreducible by proceeding inductively, so we cannot have something like this. % \todo{what just happenedd}
\end{proof}
\begin{exe}
	Fix $R=k[x,y]$ and $M=k[x,y]/\left(x^2,xy\right)$.
	\begin{itemize}
		\item We compute $\op{Ass}M$.
		\item We compute $\op{Supp}M$.
		\item We compute $H_M(s)$.
	\end{itemize}
\end{exe}
\begin{proof}
	Let's start with $H_M(s)$. Let's tabulate.
	\begin{itemize}
		\item $H_M(0)=1$, with $1$.
		\item $H_M(1)=2$, with $x$ and $y$.
		\item $H_M(2)=1$ with $y^2$.
		\item In fact, $H_M(s)=1$ for $s>1$ with $y^s$ because all other monomials have $xy$ killed.
	\end{itemize}
	The point is that $M$ looks like $k[x,y]/(x)$, which is $x=0$.

	Now we compute $\op{Supp}M$. Because $M$ is a finitely generated module over a Noetherian ring, the support consists of the primes $\mf p\subseteq k[x,y]$ containing $\op{Ann}M=\left(x^2,xy\right)$. Well, any such prime must contain $(x)$, and then we can have any larger prime, which look like $(x,y-a)$ for any $a\in k$. These are the only primes containing $(x)$ by considering $k[x]/(x)$.

	Lastly, we compute $\op{Ass}M$. Well, we note $\op{Ann}y=(x)$ and $\op{Ann}x=(x,y)$. However, the other elements of the support take the form $(x,y-a)$, which will not be annihilators.
\end{proof}
\begin{remark}
	Professor Serganova recommends doing exercises 2.19, 2.22, and 4.11 from Eisenbud.
\end{remark}