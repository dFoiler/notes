\documentclass[../notes.tex]{subfiles}

\begin{document}

% !TEX root = ../notes.tex

Here we go.

\subsection{The Nullstellensatz}
Today we prove Hilbert's Nullstellensatz. Here is the statement.
\begin{theorem}[Nullstellensatz] \label{thm:nullstellensatz}
	Fix $k$ an algebraically closed field.
	\begin{listalph}
		\item There are bijections between algebraic sets $X\subseteq\AA^n(k)$ and radical ideals $J\subseteq k[x_1,\ldots,x_n]$ by taking
		\[X\mapsto I(X):=\{f\in k[x_1,\ldots,x_n]:f(p)=0\text{ for all }p\in X\},\]
		and
		\[J\mapsto Z(J):=\left\{p\in\AA^n(k):f(p)=0\text{ for all }p\in J\right\}.\]
		In particular, $I(Z(J))=J$ and $Z(I(X))=X$.
		\item Points $p$ of an algebraic set $X\subseteq\AA^n(k)$ are in bijection with maximal ideals of $k[x_1,\ldots,x_n]/I(X)$, which are in bijection with maximal ideals of $k[x_1,\ldots,x_n]$ containing $I(X)$.
	\end{listalph}
\end{theorem}
Before jumping into the proof, we give some remarks on what we can show without too much effort. For example, back in \autoref{rem:othernullstellensatz}, we showed that $Z(I(X))=X$, so the harder direction is that $I(Z(J))=J$ for $J$ a radical ideal.

In fact, we note that $J\subseteq I(Z(J))$ is fairly easy as well: for each $f\in J$, we note that $f$ will vanish on any $a\in Z(J)$ by definition of $Z(J)$, so $f\in I(Z(J))$ follows. Thus, the hart part of (a) is showing
\[J\stackrel?\supseteq I(Z(J)).\]
We also remark that the last claim of (b) is merely ring theory.
\begin{lemma} \label{lem:pullmax}
	Fix $R$ a ring and $I\subseteq R$ an ideal. Then maximal ideals of $R/I$ are in bijection with maximal ideals of $R$ containing $I$.
\end{lemma}
\begin{proof}
	Let $\pi:R\onto R/I$ be the canonical projection. We send maximal ideals $\mf m$ of $R$ containing $I$ to the ideal $\pi(\mf m)\subseteq R/I$; conversely, we send maximal ideals $\mf m\subseteq R/I$ to the ideal $\pi^{-1}(\mf m)$. We have the following checks.
	\begin{itemize}
		\item Fix $J$ an ideal containing $I$. We claim $\pi^{-1}(\pi(J))=J$. To see this, we note $x\in\pi^{-1}(\pi(J))$ if and only if $\pi(x)\in\pi(J)$ if and only if $[x]_I=[y]_I$ for some $y\in J$ if and only if $x-y\in I\subseteq J$ if and only if $x\in J$.
		\item Similarly, fix an ideal $J\subseteq R/I$. We claim $\pi\left(\pi^{-1}(J)\right)=J$. To see this, we note $\pi(y)\in\pi\left(\pi^{-1}(J)\right)$ if and only if $y\in\pi^{-1}(J)$ if and only if $\pi(y)\in J$.
		\item Fix $\mf m\subseteq R/I$, and we show that $\pi^{-1}(\mf m)$ is a maximal ideal of $R$. Now, we know that $\pi^{-1}(\mf m)$ is proper because it is prime. Additionally, if $\pi^{-1}(\mf m)\subseteq J$ for some ideal $J$, then $\mf m\subseteq\pi(J)$, so $\pi(J)=\mf m$ or $\pi(J)=R/I$. In the former case, $\pi^{-1}(\mf m)=J$; in the latter case, $J=\pi^{-1}(R/I)=R$.
		\item Fix $\mf m\subseteq R$ a maximal ideal containing $I$, and we show that $\pi(\mf m)$ is a maximal ideal of $R/I$. Note $[1]_I\in\pi(\mf m)$ would imply that $1+x\in\mf m$ for some $x\in I$, so $1\in\mf m$ because $I\subseteq\mf m$. But $1\in\mf m$ is false, so we see that $\pi(\mf m)$ is proper.

		Now, $\pi(\mf m)\subseteq J$ for some ideal $J\subseteq R/I$ implies that $\mf m\subseteq\pi^{-1}(J)$, so $\pi^{-1}(J)=\mf m$ or $\pi^{-1}(J)=R$. Note that $[0]_I\in J$ implies $I\subseteq\pi^{-1}(J)$. So we may say that, in the former case, $J=\pi(\mf m)$; in the latter case, $J=\pi(R)=R/I$.
	\end{itemize}
	So we see that the described maps are mutually inverses and well-defined, so we are done.
\end{proof}
There is a little more that we can say about (b): it is not too hard to reduce it to the case where $X=\AA^n(k)$ and $I(X)=\emp$.
\begin{lemma} \label{lem:nzeasierb}
	Suppose that all maximal ideals of $k[x_1,\ldots,x_n]$ take the form $(x_1-a_1,\ldots,x_n-a_n)$ for $(a_1,\ldots,a_n)\in\AA^n(k)$. Then (b) of \autoref{thm:nullstellensatz} holds.
\end{lemma}
\begin{proof}
	By the previous lemma, we only have to show the first sentence of (b). Our bijection will be by
	\[(a_1,\ldots,a_n)\in X\longmapsto(x_1-a_1,\ldots,x_n-a_n)\subseteq k[x_1,\ldots,x_n].\]
	We have the following checks to show (b).
	\begin{itemize}
		\item We check that $(x_1-a_1,\ldots,x_n-a_n)$ is maximal. To see this, we claim that $(x_1-a_1,\ldots,x_n-a_n)$ is the kernel of the surjective map
		\[\varphi\in k[x_1,\ldots,x_n]\to k\]
		defined by lifting $\id_k:k\to k$ by $x_i\mapsto a_i$, which will be enough. To see this, note that certainly each $x_i-a_i$ will live in the kernel. Conversely, for any $f\in k[x_1,\ldots,x_n]$, we can apply the division algorithm to $f$ by each of the $x_i-a_i$ to write
		\[f(x_1,\ldots,x_n)=f(a_1,\ldots,a_n)+\sum_{i=1}^n(x_i-a_i)q_i(x).\]
		Formally, one should show this by induction on $n$, but we won't bother. The point is that $f\in\ker\varphi$ implies that $f(a_1,\ldots,a_n)\in\ker\varphi$, so $f(a_1,\ldots,a_n)=0$, so $f\in(x_1-a_1,\ldots,x_n-a_n)$.
		\item We check that $(x_1-a_1,\ldots,x_n-a_n)$ contains $I(X)$. Indeed, if $f\in I(X)$, then $f$ vanishes on $(a_1,\ldots,a_n)$, so $f$ lives in the kernel $\ker\varphi$ constructed above, so $f\in(x_1-a_1,\ldots,x_n-a_n)$.
		\item We show the map is injective. The key claim is that
		\[Z((x_1-a_1,\ldots,x_n-a_n))=\{(a_1,\ldots,a_n)\}.\]
		Indeed, if $(b_1,\ldots,b_n)$ lives in this vanishing set, then $b_i-a_i=0$ for each $i$, so $(a_1,\ldots,a_n)=(b_1,\ldots,b_n)$. Of course, each $x_i-a_i$ does vanish on $(a_1,\ldots,a_n)$, so we are done.

		So to finish, we note that $(x_1-a_1,\ldots,x_n-a_n)=(x_1-a_1',\ldots,x_n-a_n')$ implies that their vanishing sets match, so $(a_1,\ldots,a_n)=(a_1',\ldots,a_n')$, so we are done.
		\item We show the map is surjective. This requires some trickery. Suppose $\mf m$ is a maximal containing $I(X)$. Because $\mf m$ is maximal, we do know that
		\[\mf m=(x_1-a_1,\ldots,x_n-a_n)\]
		by hypothesis. So we see that $\mf m\supseteq I(X)$ implies that
		\[X=Z(I(X))\supseteq Z(\mf m)=\{(a_1,\ldots,a_n)\},\]
		where we have used \autoref{rem:othernullstellensatz} in the first equality. Thus, $\mf m$ is indeed of the required form, so we are done.
		\qedhere
	\end{itemize}
\end{proof}

\subsection{The Uncountable Case}
Let's start with an easier special case.
\begin{proof}[Proof of \autoref{thm:nullstellensatz} for uncountable fields]
	We prove \autoref{thm:nullstellensatz} where $k$ is an uncountable field; in other words, one should read $k=\CC$ into the following proof. We will actually start by showing (b) in the case where $X=\AA^n(k)$ and $I(X)=\emp$. The following will be the way we use that $k$ is uncountable.
	\begin{lemma} \label{lem:useuncountable}
		Fix $k$ an uncountable field and $F/k$ a field extension with $[F:k]<\#k$. Then the extension $F/k$ is algebraic.
	\end{lemma}
	\begin{proof}
		We show the contrapositive. Suppose that $F/k$ is not algebraic, and we show that $[F:k]\ge\#k$. Because $F/k$ is not algebraic, we are promised some element $x\in F$ which is not algebraic over $k$. But then $k(x)\subseteq F$ is a very large subfield, so we consider the set
		\[S:=\left\{\frac1{x-a}:a\in k\right\}.\]
		We quickly observe that these elements are legal: note that $x\ne a$ for each $a\in k$ because $k/k$ is an algebraic extension; thus, $\frac1{x-a}$ is a legal element of $F$.
		
		We claim that $S\subseteq k(x)$ is $k$-linearly independent, which will show that $[F:k]=\dim_kF\ge\dim_kk(x)\ge\#S=\#k$, which is what we want. Now, to show that $S$ is $k$-linearly independent, suppose that we have a relation
		\[\sum_{i=1}^nr_i\cdot\frac1{x-a_i}=0\]
		for some $r_1,\ldots,r_n\in k$ and distinct $a_1,\ldots,a_n\in k$. We need to show that $r_i=0$ for each $r_i$. For this, we note that
		\[0=\left(\prod_{i=1}^n(x-a_i)\right)\left(\sum_{i=1}^n\frac{r_i}{x-a_i}\right)=\sum_{i=1}^n\Bigg(r_i\prod_{\substack{j=1\\j\ne i}}^n(x-a_j)\Bigg).\tag{$*$}\label{eq:readytoplug}\]
		Now, though this equation is technically taking place in $k(x)$, we may pull it back to an equation in $k[x]$ (noting that $k[x]\into k(x)$ is injective).

		But with our equation holding in $k[x]$, we note that $k[x]\subseteq F$ is a free $k$-algebra,\footnote{More formally, note there is a morphism $k[T]\to k[x]$ extending $\id_k:k\to k$ by sending $T\mapsto x$. It is not hard to see that this is surjective, and it is injective because it has trivial kernel because $x$ is transcendental. So $k[x]\cong k[T]$.} so we may apply the universal property of $k[x]$ to note there is a morphism $k[x]\to k$ extending $\id_k:k\to k$ by sending $x\mapsto a_m$ for any $a_m$. Pushing \autoref{eq:readytoplug} through this morphism, we see
		\[\sum_{i=1}^n\Bigg(r_i\prod_{\substack{j=1\\j\ne i}}^n(a_m-a_j)\Bigg)=0.\]
		All terms of the sum will vanish except when $i=m$ because the product will feature a $(a_m-a_m)$ term otherwise. So we see
		\[r_m\prod_{\substack{j=1\\j\ne m}}^n(a_m-a_j)=0.\]
		Because the $a_i$ are all distinct, we see $a_m-a_j\ne0$ for each $m\ne j$, so the entire product is nonzero ($k$ is an integral domain), so $r_m=0$. This finishes.
	\end{proof}
	\begin{corollary} \label{cor:forcealgebraic}
		Fix $k$ an uncountable field. Then, for any maximal ideal $\mf m\subseteq k[x_1,\ldots,x_n]$, the field extension
		\[\frac{k[x_1,\ldots,x_n]}{\mf m}\supseteq k\]
		is algebraic.
	\end{corollary}
	\begin{proof}
		We quickly note that $\mf m\cap k=(0)$ because otherwise $\mf m$ would contain a unit; thus, the map $k\into k[x_1,\ldots,x_n]/\mf m$ is indeed injective, so we do have a sane field extension.
		
		Now, recall that any element $k[x_1,\ldots,x_n]$ can be written (uniquely) in the form
		\[\sum_{(d_1,\ldots,d_n)\in\NN^n}a_{(d_1,\ldots,d_n)}x_1^{d_1}\cdots x_n^{d_n},\]
		where all but finitely many of the $a_\bullet\in k$ vanish. Thus, the monomials $x_1^{d_1}\cdots x_n^{d_n}$ will generate $k[x_1,\ldots,x_n]$ and hence span $k[x_1,\ldots,x_n]/\mf m$. In particular,
		\[\dim_k\frac{k[x_1,\ldots,x_n]}{\mf m}\le\#\left\{x_1^{d_1}\cdots x_n^{d_n}:(d_1,\ldots,d_n)\in\NN^n\right\}=\#\left(\NN^n\right).\]
		However, $\NN^n$ is countable, so $\dim_k\frac{k[x_1,\ldots,x_n]}{\mf m}\le\#\NN<\#k$, so \autoref{lem:useuncountable} assures us that the extension $\frac{k[x_1,\ldots,x_n]}{\mf m}\supseteq k$ is an algebraic extension.
	\end{proof}
	So now we can show the hypothesis of \autoref{lem:nzeasierb} without tears. As discussed, we need to show that all maximal ideals $\mf m\subseteq k[x_1,\ldots,x_n]$ take the form $(x_1-a_1,\ldots,x_n-a_n)$.
	
	Well, picking up some maximal ideal $\mf m\subseteq k[x_1,\ldots,x_n]$ is maximal, we have that
	\[\frac{k[x_1,\ldots,x_n]}{\mf m}\]
	is an algebraic extension of $k$ by \autoref{cor:forcealgebraic}, but $k$ is algebraically closed, so this field must equal $k$. In particular, we are promised an isomorphism
	\[\varphi:\frac{k[x_1,\ldots,x_n]}{\mf m}\cong k.\]
	We can lift this to a map
	\[\overline\varphi:k[x_1,\ldots,x_n]\to k\]
	with kernel $\mf m$. But $x_i-\varphi(x_i)$ must certainly live in the kernel of $\overline\varphi$, so
	\[(x_1-\varphi(x_1),\ldots,x_n-\varphi(x_n))\subseteq\mf m.\]
	But the left-hand ideal is maximal, so equality follows. So indeed, all maximal ideals of $k[x_1,\ldots,x_n]$ have the requested form.

	Now we attack part (a). In addition to (b), we will need the following technical result.
	\begin{lemma} \label{lem:betterjacobson}
		Fix $k$ an algebraically closed field, and let $R:=k[x_1,\ldots,x_n]$. Then any prime ideal $\mf p\subseteq R$ is the intersection of the maximal ideals containing $\mf p$.
	\end{lemma}
	\begin{proof}
		If $\mf p$ is maximal, then there is nothing to say. Thus, we may take $\mf p$ prime but not maximal so that $R/\mf p$ is a domain but not a field. In one direction, we note that
		\[\mf p=\bigcap_{\mf m\supseteq\mf p}\mf p\subseteq\bigcap_{\mf m\supseteq\mf p}\mf m.\]
		The other inclusion is harder. To show it, we proceed by contraposition: pick up $b\notin\mf p$, and we find some maximal ideal $\mf m$ containing $\mf p$ but not $b$ so that $b\notin\bigcap_{\mf m\supseteq\mf p}\mf m$.

		For this, we work in $R/\mf p\left[[b]_\mf p^{-1}\right]$. We claim that $R/\mf p\left[[b]_\mf p^{-1}\right]$ is not a field. If $R/\mf p\left[[b]_\mf p^{-1}\right]$ is a field, then because it has countable degree over $k$ (it is spanned by products of powers of $b^{-1}$ and monomials of $R$, of which there are countably many), we see that
		\[k\subseteq R/\mf p\left[[b]_\mf p^{-1}\right]\]
		is an algebraic extension by \autoref{cor:forcealgebraic}. But because $k$ is algebraically closed, this extension must collapse, implying that $[b]_\mf p^{-1}$ is algebraic over $k$. Because $k$ is a field, we may give $[b]_\mf p^{-1}$ a monic polynomial in $k[x]$, which we notate by
		\[\left([b]_\mf p^{-1}\right)^m+a_{m-1}\left([b]_\mf p^{-1}\right)^{m-1}+\cdots+a_1\left([b]_\mf p^{-1}\right)+a_0=0.\]
		However, this polynomial also shows that $[b]_\mf p^{-1}$ is the root of some monic polynomial in $(R/\mf p)[x]$, so $[b]_\mf p^{-1}$ is integral over $R/\mf p$. But now we note $R/\mf p$ is an integral domain and $[b]_\mf p\ne0$ implies that
		\[R/\mf p\subseteq(R/\mf p)\left[[b]_\mf p^{-1}\right]\]
		is an embedding (\autoref{ex:intdomainlocalinject}), so $R/\mf p$ is a field by \autoref{prop:integralfields}. But we presupposed that $R/\mf p$ is not a field, so we have hit a contradiction.

		So because $R/\mf p\left[b^{-1}\right]$ is not a field, we have the following movements.
		\begin{itemize}
			\item We will have some nonzero maximal ideal $\mf m\subseteq R/\mf p\left[[b]_\mf p^{-1}\right]$.
			\item Because we still know
			\[R/\mf p\subseteq(R/\mf p)\left[[b]_\mf p^{-1}\right]\]
			is an integral extension of domains, we can use \autoref{rem:goingdownmax} to pull $\mf m$ back to a maximal ideal $\mf m'$ of $R/\mf p$. Note $\mf m'$ will not contain $[b]_\mf p$ by \autoref{thm:localizedprimes}.
			\item Lastly, we can pull $\mf m'\subseteq R/\mf p$ to a maximal ideal $\mf m'+\mf p\subseteq R$ containing $\mf p$ by \autoref{lem:pullmax}. Because $[b]_\mf p\notin\mf m'$, we see $b\notin\mf m'+\mf p$ as well.
		\end{itemize}
		So we see that $\mf m'+\mf p\subseteq R$ is the maximal ideal we are looking for.
	\end{proof}
	Now we show part (a). Fix $J$ a radical ideal, and we will show $J\supseteq I(Z(J))$. We can use \autoref{prop:radprimes} to write
	\[J=\bigcap_{\mf p\supseteq J}\mf p\stackrel*=\bigcap_{\mf p\supseteq J}\bigcap_{\mf m\supseteq\mf p}\mf m=\bigcap_{\mf m\supseteq J}\mf m,\]
	where we have used \autoref{lem:betterjacobson} in $\stackrel*=$. Thus, fixing $f\in I(Z(J))$, it will suffice to show that $f\in\mf m$ for any $\mf m\supseteq J$.
	
	However, we classified our maximal ideals above! So we get to write $\mf m=(x_1-a_1,\ldots,x_n-a_n)$, which is the kernel of the ``evaluation at $(a_1,\ldots,a_n)$'' map by \autoref{lem:nzeasierb}. In particular, we note $\mf m\supseteq J$ tells us that
	\[Z(J)\supseteq Z(\mf m)=\{(a_1,\ldots,a_n)\},\]
	as computed in \autoref{lem:nzeasierb}. So $f\in I(Z(J))$ implies that $f$ vanishes on $Z(J)$, so $f$ vanishes on $(a_1,\ldots,a_n)$, so $f$ lives in the kernel of the ``evaluation at $(a_1,\ldots,a_n)$'' map, so $f\in\mf m$. This finishes.
\end{proof}

\subsection{Rabinowitch's Trick}
We now provide an alternative, more general proof. For this, we pick up the following definition.
\begin{definition}[Jacobson]
	A ring $R$ is \textit{Jacobson} if and only if any prime ideal is the intersection of some maximal ideals.
\end{definition}
\begin{remark}[Nir] \label{rem:jacobsonviarad}
	We note that $J$ is Jacobson if and only if, for each prime $\mf p$,
	\[\bigcap_{\mf m\supseteq\mf p}\mf m=\mf p.\tag{$*$}\label{eq:explicitintersection}\]
	Namely, if the above holds, then $\mf p$ is the intersection of some maximal ideals; and if $\mf p$ is the intersection of the maximal ideals in some set $S$, then $\mf m\in S$ implies $\mf p\supseteq\mf p$, so
	\[\bigcap_{\mf m\in S}\mf m=\mf p\subseteq\bigcap_{\mf m\supseteq\mf p}\mf m\subseteq\bigcap_{\mf m\in S}\mf m.\]
	Further, \autoref{eq:explicitintersection} is equivalent to $\rad R/\mf p=(0)$: letting $\pi:R\onto R/\mf p$ be the canonical projection, \autoref{lem:pullmax} says that (maximal) ideals $R/\mf p$ are in bijection (maximal) ideals of $R$ containing $\mf p$ by $\pi$, so
	\[\pi^{-1}\left(\bigcap_{\overline{\mf m}\subseteq R/\mf p}\overline{\mf m}\right)=\bigcap_{\mf m\supseteq\mf p}\mf m\quad\stackrel\pi\implies\quad\bigcap_{\overline{\mf m}\subseteq R/\mf p}\overline{\mf m}=\pi\left(\bigcap_{\mf m\supseteq\mf p}\mf m\right).\]
\end{remark}
\begin{example}
	The ring $\ZZ$ is Jacobson because all nonzero primes are maximal ($\ZZ/p\ZZ$ is a field for prime $p>0$), and
	\[(0)=\bigcap_{p\ne0}(p).\]
\end{example}
\begin{example} \label{ex:pidisjacobson}
	Similarly, $R$ is Jacobson for any principal ideal domain. To start, $(0)=\bigcap_\mf m\mf m$ because $f\ne0$ will have $f\notin\mf m$ for any $\mf m$ if $f\in R^\times$; otherwise, we can place $f+1$ in a maximal ideal $\mf m$, and we see $1\notin\mf m$ requires $f\notin\mf m$ and so $f\notin\bigcap_\mf m\mf m$.

	Otherwise, fix $\mf p\subseteq R$ a nonzero prime ideal; because $R$ is a principal ideal domain, we can write $\mf p=(\pi)$ for some nonzero prime $\pi\in R$. Then the argument from \autoref{thm:pidimpliesufd} shows that all prime elements are maximal, so $\mf p$ is actually a maximal ideal.
\end{example}
\begin{nex}
	A local domain which is not a field is not Jacobson; e.g., $\ZZ_2$ is not Jacobson. The issue is that being local implies that there is only one maximal ideal, but it is not $(0)$ because we are not in a field. Thus, $(0)$ is some prime (because we are in a domain) which is not the intersection of some number of maximal ideals (of which there is only one).
\end{nex}
Our main goal is to show that $k[x_1,\ldots,x_n]$ is Jacobson, akin to \autoref{lem:betterjacobson}. In an attempt to generalize the argument given there, we have the following lemma.
\begin{lemma} \label{lem:rabinowitch}
	Fix $R$ a domain but not a field. Then $\rad R=(0)$ if and only if $R\left[b^{-1}\right]$ is not a field for any $b\in R\setminus\{0\}$.
\end{lemma}
\begin{proof}
	The main point is that prime ideals of $R\left[b^{-1}\right]$ are in one-to-one correspondence with prime ideals of $R$ which avoid $b$. We show our directions independently.
	\begin{itemize}
		\item Suppose that $R\left[b^{-1}\right]$ is a field for some $b\in R\setminus\{0\}$. Then, for any maximal ideal $\mf m\subseteq R$, we claim that $b\in\mf m$, which will finish. Well, $\mf mR\left[b^{-1}\right]$ is some ideal, and it is nonzero because $\mf m$ is nonzero; explicitly, the map $R\to R\left[b^{-1}\right]$ is injective by \autoref{ex:intdomainlocalinject}.

		But $R\left[b^{-1}\right]$ is a field and so all nonzero ideals must be all of $R\left[b^{-1}\right]$. Thus, $\frac11\in\mf mR\left[b^{-1}\right]$, so $\frac xu=\frac11$ and then $b^k\left(x-b^\ell\right)=0$ for some $b^k,b^\ell\in\left\{b^n:n\in\NN\right\}$. But $b\ne0$, so $b^k\ne0$, so
		\[x=b^\ell\in\mf m.\]
		Because $\mf m$ is proper, we conclude $\ell>0$, and because $\mf m$ is prime, we conclude $b\in\mf m$.

		\item Suppose that
		\[\bigcap_\mf m\mf m=(0).\]
		Then we claim that $R\left[b^{-1}\right]$ is not a field. We do this by exhibiting a nonzero proper ideal of $R\left[b^{-1}\right]$.
		
		Well, $b\ne0$, so the above intersection promises us some maximal ideal $\mf m$ such that $b\notin\mf m$. It follows that $\mf m\cap\{b^n:n\in\NN\}=\emp$, so \autoref{thm:localizedprimes} tells us that
		\[\mf mR\left[b^{-1}\right]\]
		is a prime ideal and hence proper. Further, as noted above, \autoref{ex:intdomainlocalinject} tells us $R\to R\left[b^{-1}\right]$ is injective, so the fact that $\mf m$ is nonzero ($R$ is not a field, so $(0)$ is not maximal) implies that $\mf mR\left[b^{-1}\right]$ is also nonzero
		\qedhere
	\end{itemize}
\end{proof}
This gives the following corollary.
\begin{corollary}[Rabinowitch] \label{cor:betterjacobson}
	Fix $R$ a ring. Then $R$ is Jacobson if and only if each non-maximal prime $\mf p$ has $R/\mf p\left[[b]_\mf p^{-1}\right]$ not a field for each $b\notin\mf p$. Equivalently, $R$ is Jacobson if and only if, for each prime $\mf p$, $(R/\mf p)\left[b^{-1}\right]$ for some $b\in(R/\mf p)\setminus\{0\}$ implies that $R/\mf p$ is a field.
\end{corollary}
\begin{proof}
	Fix $\mf p$ any prime so that we want to show $\rad R/\mf p=(0)$ by \autoref{rem:jacobsonviarad}. If $\mf p$ is maximal, then $R/\mf p$ is a field, so $(0)$ is the only maximal ideal, so $\rad R/\mf p=(0)$ follows.

	Otherwise, we have a non-maximal ideal $\mf p$. By \autoref{lem:rabinowitch}, we want to show that $\rad R/\mf p=(0)$. Well, by \autoref{lem:rabinowitch}, this is equivalent to $(R/\mf p)\left[[b]_\mf p^{-1}\right]$ not being a field for all $[b]_\mf p\ne0$ (i.e., for all $b\notin\mf p$).
\end{proof}

\subsection{The General Case}
And now we can more or less proceed as in the earlier proof of the Nullstellensatz. Here is our ``abstract'' version of the Nullstellensatz.
\begin{theorem}[General Nullstellensatz] \label{thm:gennullstellensatz}
	Fix $R$ a Jacobson ring and $S$ a finitely generated $R$-algebra by $\varphi:R\to S$ (which we do not assume to be injective).
	\begin{listalph}
		\item Then $S$ is a Jacobson ring.
		\item For each maximal ideal $\mf m\subseteq S$, we have $\mf m\cap R$ maximal in $R$, and
		\[\frac R{\mf m\cap R}\subseteq\frac S{\mf m}\]
		is a finite extension of fields. (Recall $\mf m\cap R:=\varphi^{-1}(\mf m)$.)
	\end{listalph}
\end{theorem}
\autoref{thm:nullstellensatz} will follow from this, essentially using the same argument from before. We will be more explicit afterwards.
\begin{proof}[Proof of \autoref{thm:gennullstellensatz}]
	By induction, it will suffice to show the case where $S$ is generated by a single element over $R$; we will be more explicit about this induction at the end.
	% Namely, if $n$ is generated by $x_1,\ldots,x_n$, induction can get us up to $R[x_1,\ldots,x_{n-1}]$, and then we add $x_n$ and mod out by the necessary statement.
	So for now, let $S\cong R[t]/J$ for some $J\subseteq R[t]$.

	We begin with (a). The main point is to use \autoref{cor:betterjacobson}. Well, fix $\mf p$ a prime of $S$. Now, we note that we have an extension of domains
	\[\underbrace{\frac R{\varphi^{-1}(\mf p)}}_{R':=}\stackrel\varphi\into\underbrace{\frac S{\mf p}}_{S':=}.\]
	Now, we still have a projection $R[t]\onto S\onto S'$ with kernel containing $\varphi^{-1}(\mf p)\subseteq R$, so we have a projection $R'[t]\onto S'$. In particular, we can write $S'\cong R'[t]/\mf P$ for some ideal $\mf P\subseteq R'[t]$. Note $\mf P$ is prime because $S'$ is a domain.
	
	Now, to use \autoref{cor:betterjacobson}, we need to show that, if $\mf p$ is prime but not maximal, then $(S/\mf p)\left[b^{-1}\right]=S'\left[b^{-1}\right]$ is not a field for any $b\in S'\setminus\{0\}$. By contraposition, we will actually suppose that we found $b\in S'\setminus\{0\}$ such that $S'\left[b^{-1}\right]$ is a field, then $S'$ is a field, which implies $\mf p$ is maximal. We proceed by casework on $\mf P$.
	\begin{enumerate}[label=(\roman*)]
		\item Suppose $\mf P=0$ so that $S'\cong R'[t]$. Then $R'[t]\left[b^{-1}\right]$ being a field will imply that $K(R')[t]\left[b^{-1}\right]$ is a field (e.g., clear denominators and then find the inverse in $R'[t]$), but now \autoref{cor:betterjacobson} says that $\big(K(R')[t]/(0)\big)\left[b^{-1}\right]$ being a field will force $(0)\subseteq K(R')[t]$ to be a maximal ideal because $K(R')[t]$ is Jacobson by \autoref{ex:pidisjacobson}. In other words, $K(R')[t]$ is a field, which simply does not make sense.

		It might be surprising that we hit a contradiction here, but this simply means that $R'[t]\left[b^{-1}\right]$ should never be a field.

		\item Otherwise, $\mf P\ne0$ so that $R'[t]\onto S'$ has some nontrivial kernel. Our idea, now, is to force our elements to be integral in the rudest way possible. We have two steps.
		\begin{itemize}
			\item We make $t$ integral. Fixing $a(t)\in\mf P\setminus\{0\}$, we note that $t\in S$ must vanish on $a(t)$, which we expand as out
			\[a_nt^n+a_{n-1}t^{n-1}+\cdots+a_1t+a_0=0,\]
			where $a_\bullet\in R'$ have $a_n\ne0$ in $R'$. The point of this equation is to make $t$ integral over some localization of $R'$: $\varphi(a_n)\in S'\left[b^{-1}\right]$ will also be nonzero in $S'\left[b^{-1}\right]$, so because $S'\left[b^{-1}\right]$ is a field, $\varphi(a_n)$ is a unit
			
			In particular, we see that we can extend $\varphi$ to a map $R'\left[a_n^{-1}\right][t]\to S'\left[b^{-1}\right]$, which turns $S'\left[b^{-1}\right]$ into an $R'\left[a_n^{-1}\right]$-algebra. In fact, the map $R'\left[a_n^{-1}\right]\to S'\left[b^{-1}\right]$ is still injective by \autoref{rem:localizebasering} because the map $R'\into S'\into S'\left[b^{-1}\right]$ is also injective.
			
			Now, when viewing $S'\left[b^{-1}\right]$ as an $R'\left[a_n^{-1}\right]$-algebra, we can write
			\[t^n+a_n^{-1}a_{n-1}t^{n-1}+\cdots+a_n^{-1}a_1t+a_n^{-1}a_0=0,\]
			thus making $t$ integral over $R'\left[a_n^{-1}\right]$.

			\item We make $b$ integral. Noting that all nonzero elements of $R'$ go to units in $K(S')$, we see that \autoref{rem:localizebasering} promises us an embedding $K(R')\subseteq K(S')$. Extending this by sending $t\mapsto t$, we have the familiar surjection $R'[t]\onto S'$ becoming $K(R')[t]\onto K(S')$, so $K(S')$ is $K(R')$-spanned by powers of $t$.
			
			However, the polynomial $a(t)$ from the previous point tells us that there is a relation among the elements $\left\{1,\ldots,t^n\right\}$, so we can inductively write any exponent $t^N$ for $N\ge n$ in terms of smaller-degree terms. Thus, $K(S')$ is $K(R')$-spanned by a finite number of elements and in particular is a finite field extension.

			This is all to say that the powers $\left\{1,b,b^2,\ldots\right\}\subseteq K(S')$ must get a $K(R')$-relation eventually, which we label by
			\[c_mb^m+\cdots+c_1b+c_0=0,\]
			where not all the terms vanish. By taking $m$ as small as possible, we note that we cannot have $c_0=0$, for otherwise we could divide out by $b$ (recall $b\in S'$ lives in an integral domain).
			
			Now, clearing denominators, we may assume that the $c_\bullet$ all live in $R'$, so we actually have created a relation for $b$. In fact, porting this equation over to $S\left[b^{-1}\right]$, we can multiply through by $\left(b^{-1}\right)^m$ to see
			\[c_0\left(b^{-1}\right)^m+\cdots+c_{m-1}b^{-1}+c_m=0.\]
			In particular, extending the embedding $R'\left[a_n^{-1}\right]\into S'$ to
			\[R'\left[(c_0a_n)^{-1}\right]\into S'\into S'\left[b^{-1}\right]\]
			(the former is an embedding by \autoref{rem:localizebasering}, and the latter is an embedding by \autoref{ex:intdomainlocalinject}), we see that $b^{-1}$ is now the solution of an equation in $R'\left[(c_0a_n)^{-1}\right]$ with unit leading coefficient, so $b^{-1}$ is now integral over $R'\left[(c_0a_n)^{-1}\right]$.
		\end{itemize}
		For technicality reasons, we are forced to admit that $t$ remains integral over $R'\left[(c_0a_n)^{-1}\right]$ using the same equation. So now we have an embedding of domains
		\[R'\left[(c_0a_n)^{-1}\right]\subseteq S'\left[b^{-1}\right],\]
		where $S'\left[b^{-1}\right]=R'[t]\left[b^{-1}\right]$ is generated by integral elements. So in fact the above is an integral extension of domains by \autoref{lem:betterfinite}.

		To finish, we see that $S'\left[b^{-1}\right]$ is a field implies that $R'\left[(c_0a_n)^{-1}\right]$ is a field (by \autoref{prop:integralfields}) implies that $R'$ is a field (because $R$ is Jacobson). Thus, the equation $a(t)$ for $t$ can be made monic without tears, so $t$ is integral over $S'$, so $S'$ is integral over $R'$ by \autoref{lem:betterfinite}. Thus, $S'$ is a field by \autoref{prop:integralfields}.
	\end{enumerate}
	Technically, the above two points do finish the proof of part (a), but we note that considering $\mf p\subseteq S$ to be a maximal ideal will give (b), as follows. We quickly remark that there certainly exists a $b\in S'\setminus\{0\}$ for which $S'\left[b^{-1}\right]$ because $S'=S/\mf p$ is a field, so we can simply set $b=1$.
	
	Now, we see that the map $R'[t]\onto S'$ is forced to have nontrivial kernel because we derived contradiction in (i), so we must live case (ii) of the above. Here, the arguments of (ii) show that $R'=R/\varphi^{-1}(\mf p)$ is a field, so $\varphi^{-1}(\mf P)$ is indeed maximal. Further, (ii) showed that $S'=K(S')$ is spanned finitely by $R'=K(R')$, so $R'\subseteq S'$ is a finite field extension.

	The above completes the proof in the case that $S$ is generated over $R$ by a single element. We now provide the induction. Indeed, suppose that $S$ is generated over $R$ by $r+1$ elements so that we are promised a surjection
	\[\pi:R[x_1,\ldots,x_{r+1}]\onto S.\]
	Now, we set $S_0:=k[x_1,\ldots,x_r]$ so that $S_0$ is generated over $R$ by $r$ elements, meaning we can use the inductive hypothesis when viewing $S_0$ as an $R$-algebra. And then further, $\pi$ above becomes a surjection $\pi:S_0[x_{r+1}]\onto S$; for the sake of notation, we let $\iota_0:R\into S_0$ be the embedding. We now attack our claims in sequence.
	\begin{listalph}
		\item By the inductive hypothesis, $R$ being Jacobson implies that $S_0$ is Jacobson, and the work in the one-variable case then shows that $S$ is Jacobson.
		\item Fix $\mf m\subseteq S$ a maximal ideal. By the one-variable case, we see that $\pi^{-1}(\mf m)$ is a maximal ideal, and the field extension $S_0/\pi^{-1}(\mf m)\subseteq S/\mf m$ is finite. Continuing, by the inductive hypothesis, pulling $\pi^{-1}(\mf m)\subseteq S_0$ to $\iota^{-1}\left(\pi^{-1}(\mf m)\right)\subseteq R$ remains a maximal ideal, and in fact
		\[\iota^{-1}\left(\pi^{-1}(\mf m)\right)=\varphi^{-1}(\mf m).\]
		Indeed, we see $r\in\varphi^{-1}(\mf m)$ if and only if $\pi(\iota(r))=\varphi(r)\in\mf m$. So $\varphi^{-1}(\mf m)\subseteq R$ is a maximal ideal where $R/\varphi^{-1}(\mf m)\subseteq S_0/\pi^{-1}(\mf m)$ is a finite extension. So we have the chain
		\[R/\varphi^{-1}(\mf m)\subseteq S_0/\pi^{-1}(\mf m)\subseteq S/\mf m\]
		of finite extensions, from which we conclude that $R/\varphi^{-1}(\mf m)\subseteq S/\mf m$ is a finite extension.
		\qedhere
	\end{listalph}
\end{proof}
\noindent Now we prove \autoref{thm:nullstellensatz}. All the logic is borrowed from the specific case, but we merely change the proof of the key lemmas \autoref{cor:forcealgebraic} and \autoref{lem:betterjacobson} to use \autoref{thm:gennullstellensatz}.
\begin{proof}[General proof of \autoref{thm:nullstellensatz}]
	We follow the argument from the special case. To start, note that $k$ is Jacobson because it is (stupidly) a principal ideal domain, so $k\subseteq k[x_1,\ldots,x_r]$ satisfies the hypotheses of \autoref{thm:gennullstellensatz}. We have the following.
	\begin{itemize}
		\item We see that \autoref{cor:forcealgebraic} holds for general algebraically closed fields $k$ by part (b) of \autoref{thm:gennullstellensatz}, which was what we needed for part (b) of \autoref{thm:nullstellensatz}.
		\item Additionally, we get \autoref{lem:betterjacobson} in the general case by part (a) of \autoref{thm:gennullstellensatz}, which when combined with (b) of \autoref{thm:nullstellensatz} is what we needed to prove part (a) of \autoref{thm:nullstellensatz}.
	\end{itemize}
	The above proves all of \autoref{thm:nullstellensatz}.
\end{proof}
\begin{remark}
	The midterm will only include up to Nakayama's lemma because we have not done homework on the content past then.
\end{remark}

\subsection{Example Problems}
Let's do some example problems, to review.
\begin{exe} \label{exe:det}
	Fix a field $k$. We work in $k^{n\times n}$. We show that the ideal
	\[\det\begin{bmatrix}
		x_{11} & \cdots & x_{1n} \\
		\vdots & \ddots & \vdots \\
		x_{n1} & \cdots & x_{nn}
	\end{bmatrix}=:\det X\]
	is a prime ideal in $k[x_{ij}]$. Note that it suffices to show $\det X$ is irreducible.
\end{exe}
\begin{ex} \label{ex:dettwo}
	In the case of $n=2$, we are showing $\det X=x_{11}x_{22}-x_{12}x_{21}$ is irreducible. Well, for any $x_{ij}$, if we could write
	\[\det X=f(X)g(X),\]
	then we must have $\deg_{x_{ij}}f=0$ or $\deg_{x_{ij}}g=0$. In particular, we have two cases.
	\begin{itemize}
		\item We might have $(x_{11}x_{22}-b)c$ for some $b$ and $c$. But then this forces $c=1$.
		\item We might have $(x_{11}-b)(x_{22}-c)$ for some $b$ and $c$. But then $cx_{11}$ would have to live in the polynomial, so $cx_{11}=0$, and similar for $bx_{22}$, causing everything to collapse.
	\end{itemize}
\end{ex}
\begin{proof}[Proof of \autoref{exe:det}]
	Use expansion by minors to write
	\[\det X=x_{11}\det X_{11}-q,\]
	where $X_{11}$ is $X$ without the top and left row. By induction, we may assume that $\det X_{11}$ is irreducible.
	
	Now we attempt to factor $\det X=fg$. By looking at the degree of $x_{11}$, we see that exactly one of $f$ or $g$ will have the linear term $x_{11}$. Similarly, because $\det X_{11}$ is irreducible, we cannot split it between $f$ and $g$, so it must wholesale appear in one of the factors. This gives us the following cases.
	\begin{itemize}
		\item We might have $\det X=(x_{11}+b)(\det X_{11}+c)$. Now, because $x_{11}\det X_{11}$ contains all terms with an $x_{11}$, so $c=0$ is forced. But then $\det X_{11}\mid\det X$, which does not make sense. For example, running the above argument again for $x_{12}$ shows that the analogously defined $X_{12}$ has $\det X_{12}\mid\det X$, but $\det X_{11}$ and $\det X_{12}$ are distinct irreducibles and hence coprime, which forces
		\[\deg\det X\ge\deg\det X_{11}+\deg\det X_{12},\]
		which does not make sense.
		\item We might have $\det X=(x_{11}\det X_{11}+b)c$. But by degree arguments, we see that $c$ is constant, which means that $c$ is a unit already.
	\end{itemize}
	In particular, no other factorizations are possible because they would require factoring $\det X_{11}$, which is irreducible.
	% But if we tried to factor, then $x_{11}$ cannot live on its own, so we can factor
	% \[\det X=X_{11}\left(x_{11}-\frac q{X_{11}}\right).\]
	% But $X_{11}$ is irreducible by proceeding inductively, so we cannot have something like this. % \todo{what just happenedd}
\end{proof}
\begin{exe}
	Fix $k$ be algebraically closed, and fix $R=k[x,y]$ and $M=k[x,y]/\left(x^2,xy\right)$.
	\begin{itemize}
		\item We compute $\op{Ass}M$.
		\item We compute $\op{Supp}M$.
		\item We compute $H_M(s)$.
	\end{itemize}
\end{exe}
\begin{proof}
	Let's start with $H_M(s)$. Let's tabulate.
	\begin{itemize}
		\item $H_M(0)=1$, with $1$.
		\item $H_M(1)=2$, with $x$ and $y$.
		\item $H_M(2)=1$ with $y^2$.
		\item In fact, $H_M(s)=1$ for $s>1$ with $y^s$ because all other monomials have $xy$ and therefore are killed.
	\end{itemize}

	Now we compute $\op{Supp}M$. Because $M$ is a finitely generated module, \autoref{prop:fingensupport} says the support consists of the primes $\mf p\subseteq k[x,y]$ containing $\op{Ann}M=\left(x^2,xy\right)$. Well, any such prime $\mf p$ must contain $x^2$ and therefore $x$ and therefore $(x)$, but of course $\mf p\supseteq(x)$ implies $\mf p\supseteq\left(x^2,xy\right)$. Thus,
	\[\op{Supp}M=\{\mf p:\mf p\supseteq(x)\}.\]
	We finish by concluding that the only primes containing $(x)$ are either $(x)$ or of the form $(x,y-a)$ for some $a\in k$, which we can see because any prime $\mf p$ containing $(x)$ can be projected on by
	\[k[y]\cong k[x,y]/(x)\onto k[x,y]/\mf m,\]
	and the only way to lift $\mf m$ to a prime of $k[y]$ is by $y-a$. (Notably, $k$ is algebraically closed.)

	Lastly, we compute $\op{Ass}M$. Well, $(x)\cap(x,y)^2=\left(x^2,xy\right)$ is a primary decomposition where no primary ideal can be removed ($(x)$ is $(x)$-primary, and $(x,y)^2$ is $(x,y)$-primary), so \autoref{thm:primdecompii} tells us that $\op{Ass}M=\{(x),(x,y)\}$.
\end{proof}
\begin{remark}
	Professor Serganova recommends doing exercises 2.19, 2.22, and 4.11 from Eisenbud.
\end{remark}

\end{document}