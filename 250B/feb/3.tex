% !TEX root = ../notes.tex

Today we are talking associated primes.

\subsection{Associated Primes}
Fix $M$ an $R$-module. Then given $m\in M$, recall that we can look at
\[\op{Ann}m=\{r\in R:rm=0\}.\]
Observe that $m\ne0$ promises $\op{Ann}m\ne R$ because $1_Rm\ne0$. In particular, these ideals are proper most of the time.

These annihilators give rise to the following definition.
\begin{definition}[Associated primes]
	Fix $M$ an $R$-module. Then a prime ideal $\mf p\in\op{Spec}R$ is \textit{associated to $M$} if and only if $\mf p=\op{Ann}m$ for some $m\in M$. We denote $\op{Ass}M\subseteq\op{Spec}R$ to be the set of all associated primes to $M$.
\end{definition}
As usual, let's see some examples.
\begin{example}
	We have that $\op{Ass}0=\emp$ because the annihilator of any element in $0$ is $R$, which is not prime.
\end{example}
\begin{exe}
	Fix $M:=\ZZ/n\ZZ$ as a $\ZZ$-module. Then we compute $\op{Ass}M$.
\end{exe}
\begin{proof}
	Let $(p)$ be a prime. Then $(p)$ is associated to $M$ if and only if there exists some $m\in M$ such that
	\[\op{Ann}m=(p).\]
	If $p\mid n$, then we note that $\op{Ann}\frac np=(p)$. Conversely, if $\op{Ann}m$ is prime, then it follows $p\mid n$ because any annihilator will have to contain $(n)$. % \todo{}
\end{proof}
\begin{example}
	Fix $\mf p$ a prime ideal and fix $M:=R/\mf p$. Certainly $\mf p\in\op{Ass}M/\mf p$ because of $[1]_\mf p$. Conversely, fix some $b\in R\setminus\mf p$, and we want to know what primes can arise as
	\[\op{Ann}([b]_\mf p)=\{a\in R:ab\in\mf p\}.\]
	But with $b\notin\mf p$, the primality of $\mf p$ forces $\op{Ann}([b]_\mf p)=\mf p$. So $\op{Ass}M=\{\mf p\}$.
\end{example}
In the spirit of the above example, we have the following proposition.
\begin{proposition} \label{prop:generateass}
	Fix $M$ an $R$-module. Suppose $I\subseteq R$ is an ideal maximal in
	\[\mathcal P:=\{\op{Ann}m:m\in M\setminus\{0\}\}.\]
	Then we claim $I$ is prime.
\end{proposition}
\begin{proof}
	Because $I\in\mathcal P$, we can say $I=\op{Ann}m$, so we have a natural map
	\[\varphi:R/I\into M\]
	by $\varphi:[r]_I\mapsto rm$.

	Now take $a\notin I$, and we show that $ab\notin I$ implies $b\in I$. Well, $ab\in I$ would imply that $ab\in\op{Ann}m$, or alternately, $a\in\op{Ann}bm$. But with $a\notin I$, the maximality of $I$ forces $bm=0$, so $b\in I$.
\end{proof}
We have the following corollary.
\begin{corollary}
	Fix $R$ a Noetherian ring and $M$ a nonzero ring. Then $\op{Ass}M$ is nonempty.
\end{corollary}
\begin{proof}
	Use the Noetherian condition to generate an ideal maximal in the sense of \autoref{prop:generateass}, which will finish.
\end{proof}

\subsection{Associateed Primes for Fun and Profit}
For the sake of comparison, let's talk about $\op{Supp}M$ now.
% Well, if $M$ is finitely generated, any $\mf p$ in the support of $M$ will have $\op{Ass}M\subseteq\mf p$.
\begin{example}
	Fix $R$ a domain. Then $\op{Ass}_RR=\{(0)\}$ by the integral domain condition. However, the support $\op{Supp}R=\op{Spec}R$, so the associated primes appear smaller.
\end{example}
Indeed, we will find that the associated primes will be smaller.

Localization was able to tell us about maps by building upwards: an element was $0$ if and only if zero on all localizations.
\begin{proposition}
	Fix $R$ Noetherian and $M,N$ as $R$-modules. The following are true; fix $m\in M$.
	\begin{listalph}
		\item We have $m=0$ if and only if $\frac m1=0$ in $M_\mf p$ for each associated prime $\mf p\in\op{Ass}M$.
		\item In fact, a submodule $N\subseteq M$ has $N=0$ if and only if $N_\mf p$ if and only if $N_\mf p=0$ for each $\mf p\in\op{Ass}M$.
		\item A map $\varphi:N\to M$ is injective/surjective/isomorphic if and only if $\varphi:N_\mf p\to M_\mf p$ is injective/surjective if and only if $\mf p\in\op{Ass}M$.
	\end{listalph}
\end{proposition}
\begin{proof}
	It suffices to only prove (a), and the other two follow. For (a), the forwards direction is easy because we know that $m=0$ if and only if $\frac m1=0$ in $M_\mf p$ for each prime $\mf p$.

	In the other direction, suppose $m\ne0$, and we need to find an associated prime $\mf p$ for which $\frac m1\ne0$ in $M_\mf p$. But we simply take the annihilator maximal among the annihilators containing $\op{Ann}m$ by \autoref{prop:generateass} to finish.
\end{proof}
Associated primes also behave in short exact sequences, somewhat.
\begin{lemma} \label{lem:assses}
	Suppose
	\[0\to A\to B\to C\to 0\]
	is a short exact sequence of $R$-modules. Then $\op{Ass}A\subseteq\op{Ass}B\subseteq\op{Ass}A\cup\op{Ass}C$.
\end{lemma}
\begin{proof}
	We have that $\op{Ass}A\subseteq\op{Ass}B$ because any annihilator in $A$ will end up being any annihilator in $B$ as well.

	It remains to show $\op{Ass}B\subseteq\op{Ass}A\cup\op{Ass}C$. Well, suppose $\mf p\in\op{Ass}B\setminus\op{Ass}A$, and we show that $\mf p\in\op{Ass}C$. Namely, we can find $b\in B$ such that
	\[\op{Ann}b=\mf p.\]
	Then we note that $Rb\cong R/\mf p$ by $[r]_\mf p\mapsto rb$. However, $Rb\cap A=0$ because otherwise $\mf p$ will be appear in $\op{Ass}A$ from the intersected submodule. It follows that
	\[Rb\subseteq C\]
	as an honest submodule (upon modding out by $A$), which gives us the result.
\end{proof}
\begin{corollary} \label{cor:dirsumass}
	Suppose $B=A\oplus C$ as $R$-modules. Then $\subseteq\op{Ass}B=\op{Ass}A\cup\op{Ass}C$.
\end{corollary}
\begin{proof}
	Use \autoref{lem:assses} on the split short exact sequences
	\[0\to A\to B\to C\to 0\]
	and
	\[0\to C\to B\to A\to 0\]
	to finish.
\end{proof}

\subsection{Finitely Many Associated Primes}
Let's build towards the statement that the associated primes are finite. Before this, let's do an example.
\begin{ex}
	We work with $\ZZ$-modules. Then $\op{Ass}\ZZ=\{(0)\}$ and $\op{Ass}\ZZ/p^\bullet\ZZ=\{(p)\}$. Then from here we can build all finitely generated abelian groups as a direct sum of these and get our associated primes by \autoref{cor:dirsumass}.
\end{ex}
Now here is our main lemma.
\begin{lemma} \label{lem:fingenfiltration}
	Fix $M$ a finitely generated module over a Noetherian ring $R$. Then $M$ has a finite filtration
	\[0=:M_0\subseteq M_1\subseteq\cdots\subseteq M_n=M\]
	such that each quotient $M_{k+1}/M_k\cong R/\mf p_k$ for some prime ideals $\{\mf p_k\}_{k=0}^{n-1}$.
\end{lemma}
\begin{proof}
	Because $R$ is Noetherian, $\op{Ass}M$ is nonempty. So find some $\mf p_0=\op{Ann}m_0$ and set $M_1:=Rm_1\cong R/\mf p_1$. Then we can continue this process to $M/M_1$ and pull back along the projection to get $M_2$. This gives us an ascending chain of $R$-submodules
	\[M_0\subseteq M_1\subseteq\cdots,\]
	which must eventually terminate because $M$ is Noetherian (as it is finitely generated over $R$).
\end{proof}
\begin{theorem}
	Fix $M$ a finitely generated module over a Noetherian ring $R$. The following are true.
	\begin{listalph}
		\item If $M$ is nonzero, then $\op{Ass}M$ is finite.
		\item Fix $U\subseteq R$ a multiplicatively closed subset. Then we have that
		\[\op{Ass}_{R\left[U^{-1}\right]}M\left[U^{-1}\right]=\{\mf p\left[U^{-1}\right]:\mf p\in\op{Ass}M,\mf p\cap U=\emp\}.\]

		(Recall that $\op{Spec}R\left[U^{-1}\right]$ consists of localizations of primes of $R$.)
		\item We have that
		\[\bigcup_{\mf p\in\op{Ass}M}\mf p=\bigcup_{m\in M\setminus\{0\}}\op{Ann}m.\]
		\item If $\mf p$ is a minimal prime ideal containing $\op{Ann}M$, then $\mf p\in\op{Ass}M$.
	\end{listalph}
\end{theorem}
\begin{proof}
	Here we go.
	\begin{listalph}
		\item Decompose the filtration for $M$ from \autoref{lem:fingenfiltration} and use \autoref{lem:assses} to upper-bound $\op{Ass}M$. Namely, if the factors of the filtration as $R/\mf p_k$ for primes $\{\mf p_k\}_{k=0}^{n-1}$, then
		\[\op{Ass}M\subseteq\{\mf p_0,\ldots,\mf p_{n-1}\}.\]
		\item In one direction, suppose that $\mf p\in\op{Ass}M$ and $\mf p\cap U=\emp$. We already know that $\mf p\left[U^{-1}\right]$ is prime. Because $\mf p\in\op{Ass}M$, we are promised some $m\in M$ such that
		\[R/\mf p\cong Rm\subseteq M,\]
		so localization provides an injection $R\left[U^{-1}\right]/\mf p\left[U^{-1}\right]\into M\left[U^{-1}\right]$, which can again be read off to be an associated prime. % \todo{it looks like this is the "real condition" for associated prime}

		In the other direction, suppose that $\mf p\left[U^{-1}\right]\in\op{Ass}_{R\left[U^{-1}\right]}M\left[U^{-1}\right]$, so we are promised some injection
		\[\varphi:R\left[U^{-1}\right]/\mf p\left[U^{-1}\right]\into M\left[U^{-1}\right].\]
		Here is the key trick: because everything here is finitely generated and in particular finitely presented, we get an isomorphism
		\[\op{Hom}_{R\left[U^{-1}\right]}\left(R\left[U^{-1}\right]/\mf p\left[U^{-1}\right],M\left[U^{-1}\right]\right)\cong\op{Hom}_R(R/\mf p)\left[U^{-1}\right].\]
		So we are promised some $\widetilde\varphi=\frac\psi u$ where $\psi:R/\mf p\to M$ and for some $u\in U$. But because $\varphi$ was an injection, $\widetilde\varphi$ will be an injection, so $\psi$ will be an injection, and we do get that $\mf p$ is an associated prime.
		\item In oe direction, suppose that $a\in\op{Ann}m$. Then we can choose $\mf p$ to be the ideal maximal among all annihilators connecting $\op{Ann}m$, which makes $\mf p$ an associated prime by \autoref{prop:generateass}. So $a\in\mf p$ here.
		\item We leave this as an exercise. The main idea is to localize at $\mf p$ so that $\op{Ass}M_\mf p$ is $\{\mf p_\mf p\}$.
		\qedhere
	\end{listalph}
\end{proof}
\begin{quot}
	I hope you see how powerful this idea is, of localization.
\end{quot}
\begin{corollary}
	Fix $M$ a finitely generated module over a Noetherian ring $R$, and fix any ideal $J\subseteq R$ and $m\in M$. Then $J\subseteq\op{Ann}m$ (for some $m\notin M$) or there exists $a\in J$ such that $am=0$ implies $m=0$ for each $m\in M$.
\end{corollary}
\begin{proof}
	The idea is to use (c) of the theorem. If every $a\in J$ annihilates some nonzero element of $M$, then
	\[J\subseteq\bigcup_{m\in M\setminus\{0\}}\op{Ann}m=\bigcup_{\mf p\in\op{Ass}M}\mf p.\]
	We claim that $J\subseteq\mf p$ for some $\mf p\in\op{Ass}M$; otherwise, we can find $x_i\in\mf p_j\setminus J$ for each $\mf p_i\ne\mf p_i$ while $x_i\in\mf p_i$, for which $x_1+x_2+\cdots+x_n$ will not lie in any prime ideal.
\end{proof}

\subsection{Motivating Primay Decomposition}
Let's give some motivational remarks for the primary decomposition. As an example, we consider $\ZZ$, where it happens that
\[(m)\cap(n)=(mn)=(m)(n).\]
This is a very nice property to have, with respect to proving unique prime factorization and such. Namely, to state unique prime factorization, we call an ideal ``primary'' if it is the power of some prime ideal, and then we see that any ideal $I$ is the intersection of finitely many ``primary'' ideals.

We will try to generalize this. Here is our definition of ``primary.''
\begin{definition}[\texorpdfstring{$\mf p$}{p}-primary]
	Fix $\mf p\in\op{Spec}R$ a prime ideal and $R$-modules $N\subseteq M$. Then $N$ is a \textit{$\mf p$-primary} submodule of $M$ if and only if
	\[\op{Ass}M/N=\{\mf p\}.\]
\end{definition}
\begin{example}
	The ideals $\left(p^\bullet\right)$ are $p$-primary in $\ZZ$.
\end{example}
\begin{example}
	Any prime ideal $\mf p$ is $\mf p$-primary in $R$ because $\op{Ass}(R/\mf p)=\{\mf p\}$.
\end{example}
Let's prove one nice lemma to finish off today.
\begin{lemma}
	Fix $N_1,\ldots,N_m$ a finite collection of $\mf p$-primary submodules of an $R$-module $M$. Then
	\[\bigcap_{k=1}^nN_k\]
	is also $\mf p$-primary.
\end{lemma}
\begin{proof}
	By induction, it suffices to show the result for $m=2$ so that we want to show $N_1\cap N_2$ is $\mf p$-primary. Now, we have the right-exact sequence
	\[0\to N_1\cap N_2\to M\to\frac M{N_1}\oplus\frac M{N_2},\]
	which tells us that $\frac M{N_1\cap N_2}$ is a submodule of $\frac M{N_1}\oplus\frac M{N_2}$. But then
	\[\op{Ass}\frac M{N_1\cap N_2}\subseteq\op{Ass}M/N_1\cup\op{Ass}M/N_2=\{\mf p\},\]
	so we are done.
\end{proof}
And we close by stating the theorem.
\begin{theorem}
	Any finitely generated module $M$ over a Noetherian ring $R$ is an intersection of finitely many primary submodules.
\end{theorem}