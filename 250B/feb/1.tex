% !TEX root = ../notes.tex

Hopefully we finish localizing today.

\subsection{A Little On Tensor Products}
Let's start with some review exercises.
\begin{prop}
	Fix $R$ a ring and $M$ an $R$-module and $I\subseteq R$ an $R$-ideal. This gives the $R$-module $R/I$, and we claim that
	\[(R/I)\otimes_RM\cong M/IM.\]
	canonically.
\end{prop}
\begin{proof}
	We will use a few facts about the tensor product here. To start off, we use the short exact sequence
	\[0\to I\to R\to R/I\to0\]
	and then tensor by $\otimes_RM$. This gives the right-exact sequence
	\[I\otimes_RM\to R\otimes_RM\to (R/I)\otimes_RM\to0.\]
	We know that $R\otimes_RM\cong M$ (canonically) by $r\otimes m\mapsto rm$, and then tracking the image of $I\otimes_RM$ through the isomorphism $R\otimes_RM\cong M$, we see that
	\[I\otimes_RM\cong\{rm:r\in I\text{ and }m\in M\}=IM.\]
	So we are promised the right-exact sequence
	\[IM\to M\to(R/I)\otimes_RM\to0,\]
	which gives the desired isomorphism.
\end{proof}
\begin{remark}[Nir]
	As usual, this isomorphism is functorial in $M$.
\end{remark}
\begin{corollary}
	Fix $R$ a ring and $I,J\subseteq R$ ideals. Then $(R/I)\otimes(R/J)\cong R/(I+J)$.
\end{corollary}
\begin{proof}
	From the above we can conclude
	\[(R/I)\otimes_R(R/J)\cong\frac{R/J}{I(R/J)}\cong\frac{R/J}{(I+J)/J}\cong\frac R{I+J}.\]
	This finishes.
\end{proof}
The above result could be used for fun and profit on the homework.
\begin{remark}
	Professor Serganova does not care too much about noncommutative rings.
\end{remark}

We also have the following ``change of constants'' results.
\begin{proposition}
	Fix $S$ an $R$-algebra. Then, given $S$-modules $A$ and $B$ and $C$, we have
	\[(A\otimes_RB)\otimes_SC\simeq A\otimes_R(B\otimes_SC).\]
\end{proposition}
\begin{proof}
	The isomorphism is by $(a\otimes b)\otimes c\mapsto a\otimes(b\otimes c)$.
\end{proof}
\begin{proposition}
	Fix $S$ an $R$-algebra. Then, given $R$-modules $M$ and $N$, we have
	\[S\otimes_R(M\otimes_RN)\cong(S\otimes_RM)\otimes_S(S\otimes_RN),\]
	where $S\otimes_RM$ is given an $S$-module structure by multiplying the left coordinate.
\end{proposition}
\begin{proof}
	The trick is to use associativity in clever ways. Indeed,
	\begin{align*}
		(S\otimes_RM)\otimes_S(S\otimes_RN) &\cong (M\otimes_RS)\otimes_S
		(S\otimes_RN) \\
		&\cong (M\otimes_R(S\otimes_S(S\otimes_RN))) \\
		&\cong (M\otimes_R((S\otimes_SS)\otimes_RN)) \\
		&\cong (M\otimes_R(S\otimes_RN)),
	\end{align*}
	which becomes $S\otimes_R(M\otimes_RN)$ after more association.
\end{proof}

\subsection{Artinian Rings}
We have the following definition, dual to the ascending chain condition for Noetherian modules.
\begin{definition}[Artinian module]
	An $R$-module $M$ is \textit{Artinian} if and only if any descending chain of $R$-submodules
	\[M\supseteq M_1\supseteq M_2\supseteq\cdots\]
	will stabilize.
\end{definition}
\begin{definition}
	The ring $R$ is \textit{Artinian} if and only if $R$ is an Artinian as an $R$-module.
\end{definition}
In other words, after recalling that $R$-submodules of $R$ are ideals, we see that being an Artinian ring is the same as having the descending chain on ideals.
\begin{example}
	Fix $k$ a field and $p(x)\in k[x]\setminus\{0\}$. Then $k[x]/p(x)$ is a finite-dimensional $k$-vector space (in fact, of dimension $\deg p$), which means that it is both Noetherian and Artinian because a chain of $k$-subspaces can be measured to stabilize by dimension.
\end{example}
\begin{example}
	More generally, any finite-dimensional $k$-algebra is an Artinian ring.
\end{example}
\begin{example}
	The ring $\ZZ/n\ZZ$ is finite and hence Artinian (and Noetherian).
\end{example}
Observe that all of our examples of Artinian rings are in fact Noetherian. In fact, we will show that all Artinian rings are Noetherian; in the process, we will be able to describe all Artinian rings.

\subsection{Composition Series}
The main character in our story on Artinian rings will be the ``module of finite length.''
\begin{definition}[Composition series]
	Fix an $R$-module $M$. Then a \textit{composition series} (or \textit{Jordan--H\"older series}) is a chain of distinct $R$-submodules
	\[M:=M_0\supsetneq M_1\supsetneq\cdots\supsetneq M_N:=\{0\}\]
	such that each quotient $M_i/M_{i+1}$ is a simple $R$-module. The $M_i/M_{i+1}$ are the \textit{composition factors}.
\end{definition}
\begin{lemma}
	If an $R$-module $M$ is both Artinian and Noetherian, then it has a composition series.
\end{lemma}
\begin{proof}
	Because $M$ is Noetherian, the set of all proper ideals will have a maximal element, which we call $M_1$. Observe that $M_1$ will then be both Artinian and Noetherian (as a submodule of $M$), so we can repeat the process to get a maximal submodule $M_2\subsetneq M_1$. Continuing, we get
	\[M\supsetneq M_1\supseteq M_2\supseteq\cdots.\]
	But this process must stop eventually because $M$ is Artinian, so we are done.
\end{proof}
\begin{nex}
	This process oes not work when $M$ is not Artinian. For example,
	\[\ZZ\supsetneq2\ZZ\supsetneq4\ZZ\supsetneq 8\ZZ\supseteq\cdots\]
	creates an infinite descending chain.
\end{nex}
Here is the main result on composition series.
\begin{theorem}[Jordan--H\"older]
	Fix $M$ an $R$-module which has a composition series. Then all composition series of $M$ have the same length and in fact the same multiset of composition factors.
\end{theorem}
\begin{proof}
	Omitted. This is Eisenbud and in fact essentially the same as the proof for groups if one has seen the corresponding proof for groups. % \todo{}
\end{proof}
The above theorem gives us the following definition.
\begin{defi}[Length]
	An $R$-module $M$ with a composition series is said to have \textit{length} $n$ if and only if all composition series have $n$ factors.
\end{defi}
\begin{definition}[Finite length]
	An $R$-module $M$ is \textit{of finite length} if and only if $M$ has a composition series.
\end{definition}
Quickly, note that the support of $M$ is particularly nice when $M$ has a composition series, which essentially comes from various facts we've already proven.
\begin{lemma}
	Fix $M$ an $R$-module with a finite composition series
	\[M:=M_0\supsetneq M_1\supsetneq\cdots\subsetneq M_0:=\{0\}.\]
	Then $\op{Supp}M$ is the composition factors.
\end{lemma}
\begin{proof}
	The point is to turn the composition series into a whole bunch of short exact sequence, from which we can read off the support. Namely, we have the short exact sequences
	\[0\to M_i\to M_{i+1}\to M_{i+1}/M_i\to0\]
	from which we can read off the support inductively.
\end{proof}
And here is a nice result which we get from this.
\begin{theorem}
	Fix $M$ a $R$-module of finite length. Then the following are true.
	\begin{listalph}
		\item We can glue the localization maps $M\to M_\mf p$ together to form an $R$-module isomorphism
		\[M\cong\bigoplus_{\mf p\in\op{Supp}M}M_\mf p.\]
		\item The multiplicity of a simple module $R/\mf m$ as a composition factor is the length of $M_\mf p$ as an $R_\mf p$-module.
	\end{listalph}
\end{theorem}
\begin{proof}
	We will be very brief. The details are in Eisenbud. Last time we showed that if a morphism $\varphi:M\to N$ induces isomorphisms $\varphi_\mf m:M_\mf m\to N_\mf m$ for each maximal ideal $\mf m\subseteq R$, then $\varphi$ is an isomorphism. Thus, it suffices to show the canonical map
	\[\varphi:M\to\bigoplus_{\mf p\in\op{Supp}M}M_\mf p\]
	induces isomorphisms under localization. Namely, localizing by some maximal ideal $\mf m$, we get a map
	\[\varphi_\mf m:M_\mf m\to\bigoplus_{\mf p\in\op{Supp}M}(M_\mf p)_\mf m.\]
	The main point, now, is to compute that
	\[(R/\mf p)_\mf q\cong\begin{cases}
		0 & \mf p\ne\mf q, \\
		R_\mf p/\mf pR_\mf p & \mf p=\mf q.
	\end{cases}\]
	For (b), the point is to localize a composition series to get the result, again using the above computatin.
\end{proof}

\subsection{Artinian Grab-Bag}
We are now able to give the following classification.
\begin{theorem}
	Fix $R$ a ring. Then the following are equivalent.
	\begin{listalph}
		\item $R$ is Noetherian, and all prime ideals are maximal.
		\item $R$ is Artinian.
	\end{listalph}
\end{theorem}
\begin{proof}
	We show the directions one at a time.
	\begin{itemize}
		\item Suppose that $R$ is neither Noetherian nor Artinian. These two conditions conspire to give us an ideal $J$ maximal with respect to the property that $R/J$ is not Artinian. We claim that $J$ must be prime, and we see that it will not be maximal because then $R/J$ would be simple. In particular, this will mean we have found a prime which is not maximal.

		Well, suppose that $a\notin J$. Consider the short exact sequence
		\[0\to\frac{J+(a)}J\to \frac RJ\to\frac R{J+(a)}\to0.\]
		Observe that, by maximality of $J$, $R/(J+(a))$ will be forced to have finite length. Because $R/J$ does not have finite length, we see that we are forced into $(J+(a))/J$ not being Artinian. However, note that we have an isomorphism
		\[\frac{J+(a)}J\cong\frac R{\{b\in R:ba\in J\}}.\]
		Now supposing for the sake of contradiction that $J$ were not prime, then we could find $ab\in J$ with $a\notin J$ and $b\notin J$, which would imply that $J\subsetneq\{b\in R:ba\in J\}$ and hence $R/\{b\in R:ba\in J\}$ would have finite length, which is a contradiction.
	\end{itemize}
	For the other direction, we bring up the following definition.
	\begin{definition}[Jacobson radical]
		Fix $R$ a ring. Then we define the \textit{Jacobson radical} $J\subseteq R$ to be
		\[J:=\bigcap_\mf m\mf m,\]
		where $\mf m$ ranges over all maximal ideals of $R$.
	\end{definition}
	We now proceed with the proof. Suppose that $R$ is Artinian. Note that $R$ has finitely many maximal ideals because each maximal ideal $\mf m$ induces a simple module $R/\mf m$ which would give infinitely many different composition factors. % \todo{what if they are just isomorphic but not inducing differnet elements of the composition series?}
	Then by the Chinese remainder theorem, we see that
	\[R\to\prod_\mf mR/\mf m.\]
	will be surjective; in fact, this has kernel $J$, so we get a short exact sequence
	\[0\to J\to R\to\prod_\mf mR/\mf m.\]
	Now, the key claim is as follows.
	\begin{lemma}
		Fix $R$ an Artinian ring. Then the Jacobson radical $J$ is nilpotent.
	\end{lemma}
	\begin{proof}
		Observe that we have a descending chain
		\[J\supseteq J^2\supseteq\cdots,\]
		which stabilizes because $R$ is Artinian. Suppose that $J^n=J^{n+1}=I$, and we hope $I=(0)$. By the stabilization, we see $I^2=I$.

		Now, if $I\ne(0)$, then we can find a minimal ideal $K\subseteq I$ such that $IK=K$ and $K\ne(0)$. (Note that $I=K$ would work but is perhaps not minimal.) This means we can find $a\in K$ such that $ab=a$ for some $b\in I$.

		But here is the key trick: we can write
		\[a(1-b)=0,\]
		but $1-b\notin J$, so $1-b\in R^\times$ because $b$ does not lie in any maximal ideal! So $a=0$, which is our contradiction.
	\end{proof}
	We now return to the proof. Observe that the chain
	\[R\supseteq J\supseteq J^2\supseteq\cdots\supseteq J^n=(0)\]
	has the nice property that $J^{i+1}/J^i$ is a finite direct sum of simple modules; this essentially comes from our formula for $R/J$ from the short exact sequence. % \todo{}
	This implies that we have given $R$ a composition series, so $R$ has finite length, so $R$ is Noetherian.

	It remains to check that all prime ideals of $R$ are maximal. Well, fix $\mf p$ some prime ideal, and we observe that $R/\mf p$ must be an integral domain and in fact it will be Artinian. So we need to show that all Artinian domains are fields. Indeed, in this case the short exact sequence
	\[0\to J\to R\to\prod_\mf mR/\mf m\to0\]
	becomes probelmatic. % \todo{}
\end{proof}

\subsection{Geometry of Artinian Rings}
While we're here, we provide some more nice facts.
\begin{proposition}
	Any Artinian ring is a product of local Artinian rings.
\end{proposition}
\begin{proof}
	This essentially comes down to modules of finite length being products of localizations over their support.
\end{proof}
We can even give a geometric view to what we are doing.
\begin{proposition}
	Fix $I\subseteq k[x_1,\ldots,x_n]$. Then the following are equivalent.
	\begin{itemize}
		\item The ring $R:=k[x_1,\ldots,x_n]/I$ is Artinian.
		\item The set $Z(I)\subseteq\AA^n(k)$ is finite.
		\item The ring $R$ is a finite-dimensional $k$-algebra.
	\end{itemize}
\end{proposition}
\begin{proof}
	Omitted. See Eisenbud.
\end{proof}
And we end our discussion with the following miscellaneous result.
\begin{proposition}
	Fix an ideal $I\subseteq R$. Then
	\[\op{rad}I=\bigcap_{I\subseteq\mf p}\mf p,\]
	where $\mf p$ ranges over all prime ideals.
\end{proposition}
\begin{proof}
	The main point is the following lemma.
	\begin{lemma}
		Fix $R$ a ring and $I\subseteq R$ an ideal and $U\subseteq R$ a multiplicatively closed subset such that $I\cap U=\emp$. Suppose $\mf p$ is maximal in the set of ideals satisfying $\mf p\cap U=\emp$ and $I\subseteq\mf p$. Then $\mf p$ is prime.
	\end{lemma}
	\begin{proof}
		We omit this proof. It is easiest stated by contraposition.
	\end{proof}
	We have two directions. In one direction, suppose that $a\in\rad I$ so that $a^n\in I$ for some $n\in\NN$. Then for any prime $\mf p$ containing $I$, we have $a^n\in\mf p$, so $a\in\mf p$ by primality.

	The other direction requires the lemma. Suppose that $a\notin\rad I$, and we will find a prime $\mf p\supseteq I$ such that $a\notin\mf p$. Indeed, we pick up a prime $\mf p$ which is maximal avoiding the set
	\[\langle a\rangle:=\{a^n:n\in\NN\}\]
	so that $\langle a\rangle$ is multiplicatively closed. Then our ideal $\mf p$ is prime and containing $I$ and avoiding $a$, so we are done.
\end{proof}
\begin{example}
	We have that $\op{rad}(0)=\{r\in R:r^n=0\text{ for some }n\in\NN\}$ will be the intersection of all prime ideals of $R$. In other words, an element $r\in R$ is \textit{nilpotent} if and only if $r\in\mf p$ for all primes $\mf p\in\op{Spec}R$.
\end{example}