% !TEX root = ../notes.tex

Hopefully we finish localizing today.

\subsection{A Little On Tensor Products}
Let's start with some review exercises.
\begin{prop} \label{prop:tensorquotient}
	Fix $R$ a ring and $M$ an $R$-module and $I\subseteq R$ an $R$-ideal. This gives the $R$-module $R/I$, and we claim that we have a canonical isomorphism
	\[(R/I)\otimes_RM\cong M/IM\]
	by $[r]_I\otimes m\mapsto[rm]_{IM}$.
\end{prop}
\begin{proof}
	We will use a few facts about the tensor product here. To start off, we use the short exact sequence
	\[0\to I\to R\to R/I\to0\]
	and then tensor by $\otimes_RM$. This gives the right-exact sequence
	\[I\otimes_RM\to R\otimes_RM\to (R/I)\otimes_RM\to0.\]
	We know that $R\otimes_RM\cong M$ (canonically) by $r\otimes m\mapsto rm$, and then tracking the image of $I\otimes_RM$ through the isomorphism $R\otimes_RM\cong M$, we see that
	\[I\otimes_RM=\{r\otimes m:r\in R\text{ and }m\in M\}\cong\{rm:r\in I\text{ and }m\in M\}=IM.\]
	So we are promised the following commutative diagram with (right) exact rows, where the dashed arrow is induced by the rest of the diagram.
	% https://q.uiver.app/?q=WzAsOCxbMCwwLCJJXFxvdGltZXNfUk0iXSxbMSwwLCJSXFxvdGltZXNfUk0iXSxbMiwwLCIoUi9JKVxcb3RpbWVzX1JNIl0sWzMsMCwiMCJdLFswLDEsIklNIl0sWzEsMSwiTSJdLFsyLDEsIk0vSU0iXSxbMywxLCIwIl0sWzAsNF0sWzAsMV0sWzQsNV0sWzEsNV0sWzEsMl0sWzUsNl0sWzIsM10sWzYsN10sWzIsNiwiIiwxLHsic3R5bGUiOnsiYm9keSI6eyJuYW1lIjoiZGFzaGVkIn19fV1d
	\[\begin{tikzcd}
		{I\otimes_RM} & {R\otimes_RM} & {(R/I)\otimes_RM} & 0 \\
		IM & M & {M/IM} & 0
		\arrow[from=1-1, to=2-1]
		\arrow[from=1-1, to=1-2]
		\arrow[from=2-1, to=2-2]
		\arrow[from=1-2, to=2-2]
		\arrow[from=1-2, to=1-3]
		\arrow[from=2-2, to=2-3]
		\arrow[from=1-3, to=1-4]
		\arrow[from=2-3, to=2-4]
		\arrow[dashed, from=1-3, to=2-3]
	\end{tikzcd}\]
	To be explicit, the induced arrow is created by pulling back $(R/I)\otimes_RM$ to $R\otimes_RM$, then pushing forward through to $M$ and then $M/IM$. Explicitly, we take
	\[[r]_I\otimes m\mapsfrom r\otimes m\mapsto rm\mapsto[rm]_{IM}.\]
	Being well-defined is by the commutativity and exactness of the diagram: if $r\equiv s\pmod I$, then $r\otimes m$ and $s\otimes m$, but $(r-s)\otimes m$ is in the kernel of $R\otimes_RM\to(R/I)\otimes_RM$, so $(r-s)m$ is in the kernel of $M\to M/IM$, so $[rm]_{IM}=[sm]_{IM}$.
	
	The fact that the left two vertical morphisms are isomorphisms forces the rightmost induced morphism to be an isomorphism. Formally, we should replace $I\otimes_RM$ and $IM$ with their images in $R\otimes_RM$ and $M$ to ensure that we have short exact sequences, and then we can finish by applying the snake lemma.
\end{proof}
\begin{remark}[Nir] \label{rem:tensorquotientfunctorial}
	As usual, this isomorphism is functorial in $M$ in the following sense: if we have $\varphi:M\to N$, then the following diagram commutes.
	% https://q.uiver.app/?q=WzAsNCxbMCwwLCIoUi9JKVxcb3RpbWVzX1JNIl0sWzEsMCwiKFIvSSlcXG90aW1lc19STiJdLFswLDEsIk0vSU0iXSxbMSwxLCJOL0lOIl0sWzAsMSwiXFx2YXJwaGkiXSxbMiwzLCJcXHZhcnBoaSJdLFswLDIsIlxcY29uZyIsMl0sWzEsMywiXFxjb25nIl1d
	\[\begin{tikzcd}
		{(R/I)\otimes_RM} & {(R/I)\otimes_RN} \\
		{M/IM} & {N/IN}
		\arrow["\varphi", from=1-1, to=1-2]
		\arrow["\varphi", from=2-1, to=2-2]
		\arrow["\cong"', from=1-1, to=2-1]
		\arrow["\cong", from=1-2, to=2-2]
	\end{tikzcd}\]
	Here, the $\varphi$ arrows are all induced. To see the commutativity, we track $[r]_I\otimes m\mapsto[r]_I\otimes\varphi(n)\mapsto[r\varphi(n)]_{IM}$ along the top, and similarly $[r]_I\otimes m\mapsto[r]_I\otimes\varphi(m)\mapsto[r\varphi(m)]_{IM}$ along the bottom.
\end{remark}
\begin{corollary}
	Fix $R$ a ring and $I,J\subseteq R$ ideals. Then $(R/I)\otimes_R(R/J)\cong R/(I+J)$.
\end{corollary}
\begin{proof}
	From the above we can compute
	\[(R/I)\otimes_R(R/J)\cong\frac{R/J}{I(R/J)}=\frac{R/J}{(I+J)/J}\cong\frac R{I+J}.\]
	Here, $I(R/J)=(I+J)/J$ is set-theoretic: $I(R/J)$ is $\{[x]_J:x\in I\}$, but in fact $[x]_J=[x+y]_J$ for any $y\in J$, so we can write this as $\{[x]_J:x\in I+J\}$. Additionally, $R/(I+J)\cong(R/J)/((I+J)/J)$ is by tracking the kernel of the (surjective) composite $R\onto R/J\onto(R/J)/((I+J)/J)$.
\end{proof}
The above result could be used for fun and profit on the homework.
\begin{remark}
	Professor Serganova does not care too much about noncommutative rings in this class.
\end{remark}

We also have the following ``change of constants'' results.
\begin{proposition} \label{prop:tensorassociate}
	Fix $S$ an $R$-algebra. Then, given an $R$-module $A$ as well as $S$-modules $B$ and $C$, we have
	\[(A\otimes_RB)\otimes_SC\cong A\otimes_R(B\otimes_SC).\]
\end{proposition}
\begin{proof}
	The isomorphism is by $(a\otimes b)\otimes c\mapsto a\otimes(b\otimes c)$. Doing this proof rigorously would induce a lot of pain, so we won't bother.
\end{proof}
\begin{proposition} \label{prop:tensordistrib}
	Fix $S$ an $R$-algebra. Then, given $R$-modules $M$ and $N$, we have
	\[S\otimes_R(M\otimes_RN)\cong(S\otimes_RM)\otimes_S(S\otimes_RN),\]
	where $S\otimes_RM$ is given an $S$-module structure by multiplying the left coordinate.
\end{proposition}
\begin{proof}
	The trick is to use associativity in clever ways. Indeed,
	\begin{align*}
		(S\otimes_RM)\otimes_S(S\otimes_RN) &\cong (M\otimes_RS)\otimes_S
		(S\otimes_RN) \\
		&\stackrel*\cong (M\otimes_R(S\otimes_S(S\otimes_RN))) \\
		&\stackrel*\cong (M\otimes_R((S\otimes_SS)\otimes_RN)) \\
		&\cong (M\otimes_R(S\otimes_RN)),
	\end{align*}
	which becomes $S\otimes_R(M\otimes_RN)$ after more association. Note we have used \autoref{prop:tensorassociate} (carefully!) on the isomorphisms denoted $\stackrel*\cong$.
\end{proof}
\begin{remark}[Nir] \label{rem:associatefunctorial}
	Tracking through the isomorphism, we see that $(s\otimes m)\otimes(t\otimes n)$ gets sent to $(m\otimes s)\otimes(t\otimes m)$ gets sent to $m\otimes(s\otimes(t\otimes n))$ gets sent to $m\otimes((s\otimes t)\otimes n)$ gets sent to $m\otimes(st\otimes n)$ gets sent to $st\otimes(m\otimes n)$.
\end{remark}
\begin{corollary} \label{cor:localizetensor}
	Fix $R$ a ring and $U\subseteq R$ a multiplicatively closed subset. Then, given $R$-modules $M$ and $N$, we have
	\[(M\otimes_RN)\left[U^{-1}\right]\cong M\left[U^{-1}\right]\otimes_{R\left[U^{-1}\right]}N\left[U^{-1}\right].\]
\end{corollary}
\begin{proof}
	After noting that $M\left[U^{-1}\right]\cong M\otimes_RR\left[U^{-1}\right]$, we see that we are trying to show
	\[(M\otimes_RN)\otimes_RR\left[U^{-1}\right]\cong(M\otimes_RR\left[U^{-1}\right])\otimes_R{\left[U^{-1}\right]}(N\otimes_RR\left[U^{-1}\right]),\]
	which is exactly \autoref{prop:tensordistrib}.
\end{proof}
\begin{remark}[Nir]
	Tracking through the isomorphism, we see that $\frac1um\otimes\frac1v\otimes n$ gets sent to $(m\otimes\frac1u)\otimes(n\otimes\frac1v)$ gets sent to $\frac1{uv}\otimes(m\otimes n)$ gets sent to $\frac1{uv}(m\otimes n)$.
\end{remark}
\begin{remark}[Nir] \label{rem:localizetensorfunctorial}
	Thus, we see that \autoref{cor:localizetensor} is functorial in $M$ and $N$. That is, if we have morphisms $\varphi_M:M\to M'$ and $\varphi_N:N\to N'$, the following diagram commutes.
	% https://q.uiver.app/?q=WzAsNCxbMCwwLCJNXFxsZWZ0W1Veey0xfVxccmlnaHRdXFxvdGltZXNfe1JcXGxlZnRbVV57LTF9XFxyaWdodF19TlxcbGVmdFtVXnstMX1cXHJpZ2h0XSJdLFsxLDAsIk0nXFxsZWZ0W1Veey0xfVxccmlnaHRdXFxvdGltZXNfe1JcXGxlZnRbVV57LTF9XFxyaWdodF19TidcXGxlZnRbVV57LTF9XFxyaWdodF0iXSxbMCwxLCIoTVxcb3RpbWVzX1JOKVxcbGVmdFtVXnstMX1cXHJpZ2h0XSJdLFsxLDEsIihNJ1xcb3RpbWVzX1JOJylcXGxlZnRbVV57LTF9XFxyaWdodF0iXSxbMCwyXSxbMSwzXSxbMCwxXSxbMiwzXV0=
	\[\begin{tikzcd}
		{M\left[U^{-1}\right]\otimes_{R\left[U^{-1}\right]}N\left[U^{-1}\right]} & {M'\left[U^{-1}\right]\otimes_{R\left[U^{-1}\right]}N'\left[U^{-1}\right]} \\
		{(M\otimes_RN)\left[U^{-1}\right]} & {(M'\otimes_RN')\left[U^{-1}\right]}
		\arrow[from=1-1, to=2-1]
		\arrow[from=1-2, to=2-2]
		\arrow[from=1-1, to=1-2]
		\arrow[from=2-1, to=2-2]
	\end{tikzcd}\]
	The vertical arrows are isomorphisms, and the horizontal arrows are induced. To see this commutes, along the top we take $\frac1um\otimes\frac1vn$ to $\frac1u\varphi_Mm\otimes\frac1v\varphi_Nn$ to $\frac1{uv}(\varphi_Mm\otimes\varphi_Nn)$. Then along the bottom we take $\frac1um\otimes\frac1vn$ to $\frac1{uv}(m\otimes n)$ to $\frac1{uv}(\varphi_Mm\otimes\varphi_Nn)$, which is the same.
\end{remark}

\subsection{Artinian Rings}
We have the following definition, dual to the ascending chain condition for Noetherian modules.
\begin{definition}[Artinian module]
	An $R$-module $M$ is \textit{Artinian} if and only if any descending chain of $R$-submodules
	\[M\supseteq M_1\supseteq M_2\supseteq\cdots\]
	will stabilize.
\end{definition}
\begin{definition}
	The ring $R$ is \textit{Artinian} if and only if $R$ is an Artinian as an $R$-module.
\end{definition}
In other words, after recalling that $R$-submodules of $R$ are ideals, we see that being an Artinian ring is the same as having the descending chain on ideals.
\begin{example}
	Fix $k$ a field and $p(x)\in k[x]\setminus\{0\}$. Then $k[x]/p(x)$ is a finite-dimensional $k$-vector space (in fact, of dimension $\deg p$), which means that it is both Noetherian and Artinian because a chain of $k$-subspaces can be measured to stabilize by dimension.
\end{example}
\begin{example}
	More generally, any finite-dimensional $k$-algebra is an Artinian ring.
\end{example}
\begin{example}
	The ring $\ZZ/n\ZZ$ is finite and hence Artinian (and Noetherian).
\end{example}
Observe that all of our examples of Artinian rings are in fact Noetherian. In fact, we will show that all Artinian rings are Noetherian; in the process, we will be able to describe all Artinian rings.

Here is a technical result which we will want to use later; it is dual to the Noetherian case in \autoref{prop:noetherianses}.
\begin{prop} \label{prop:artinianses}
	Fix a short exact sequence
	\[0\to A\to B\to C\to 0\]
	of $R$-modules. Then $B$ is Artinian if and only if $A$ and $C$ are Artinian.
\end{prop}
\begin{proof}
	We omit this proof; one can essentially copy the proof of the Noetherian case in \autoref{prop:noetherianses}. % \todo{}
\end{proof}

\subsection{Composition Series}
The main character in our story on Artinian rings will be the ``module of finite length.''
\begin{definition}[Composition series]
	Fix an $R$-module $M$. Then a \textit{composition series} (or \textit{Jordan--H\"older series}) is a chain of distinct $R$-submodules
	\[M:=M_0\supsetneq M_1\supsetneq\cdots\supsetneq M_N:=(0)\]
	such that each quotient $M_i/M_{i+1}$ is a nonzero simple $R$-module. The $M_i/M_{i+1}$ are the \textit{composition factors}.
\end{definition}
Composition series give rise to the notion of length.
\begin{defi}[Length]
	An $R$-module $M$ with a composition series is said to have \textit{length} $n$ if and only if the shortest composition series (of which there might be many) have $n$ factors.
\end{defi}
\begin{definition}[Finite length]
	An $R$-module $M$ is \textit{of finite length} if and only if $M$ has a composition series.
\end{definition}
Note that we can already see the Artinian condition playing with being of finite length.
\begin{lemma} \label{lem:artnoethimpliesfinlength}
	If an $R$-module $M$ is both Artinian and Noetherian, then $M$ is of finite length.
\end{lemma}
\begin{proof}
	If $M=(0)$, we can use the composition series made of only $M$. Otherwise, because $M$ is Noetherian, the set of all proper ideals will have a maximal element, which we call $M_1$.
	
	If $M_1=(0)$, then we have a finite composition series made of $M\supseteq M_0$. Otherwise, observe that $M_1$ will then be both Artinian and Noetherian (as a submodule of $M$), so we can repeat the process to get a maximal submodule $M_2\subsetneq M_1$.
	
	We can continue this process inductively, which gives us the descending chain
	\[M\supsetneq M_1\supsetneq M_2\supsetneq\cdots,\]
	where the quotients are simple modules. But this process must stop eventually because $M$ is Artinian, and the only way to stop is when $M_n=(0)$ for some $n$, so indeed, this is a composition series. So $M$ is of finite length.
\end{proof}
\begin{nex}
	This process does not work when $M$ is not Artinian. For example,
	\[\ZZ\supsetneq2\ZZ\supsetneq4\ZZ\supsetneq 8\ZZ\supsetneq\cdots\]
	creates an infinite descending chain.
\end{nex}
In fact, we can build composition series in short exact sequences, just like how the Noetherian and Artinian conditions build in short exact sequences.
\begin{prop} \label{prop:finlengthses}
	Fix a short exact sequence
	\[0\to A\to B\to C\to 0\]
	of $R$-modules. Then $B$ is of finite length if and only if $A$ and $C$ are of finite length. In fact, the length of $B$ upper-bounds the lengths of $A$ and $B$, and the length of $B$ is at most the sum of the lengths of $A$ and $C$.
\end{prop}
\begin{proof}
	We use the embedding $A\into B$ to view $A$ as an $R$-submodule of $B$, and we use the projection $B\onto C$ to view $C\cong B/A$ as a quotient. We now take the directions independently.
	\begin{itemize}
		\item Suppose that $B$ is of finite length; namely, we get a composition series
		\[B=:B_0\supsetneq B_1\supsetneq\cdots\supsetneq B_{n-1}\supsetneq B_0:=(0).\]
		We have two parts.
		\begin{itemize}
			\item We show that $A$ has finite length. Indeed, set $A_k:=B_k\cap A$ so that we get the descending chain
			\[A=A_0\supseteq A_1\supseteq\cdots\supseteq A_{n-1}\supseteq A_n=(0).\]
			Now, we can compute the quotients as\footnote{The kernel of the composition $A\cap B_k\into(A\cap B_k)+B_{k+1}\onto((A\cap B_k)+B_{k+1})/B_{k+1}$ is $A\cap B_{k+1}$. The map is surjective because any element of $((A\cap B_k)+B_{k+1})/B_{k+1}$ will have a representative in $A\cap B_k$.}
			\[\frac{A\cap B_k}{A\cap B_{k+1}}\cong\frac{(A\cap B_k)+B_{k+1}}{B_{k+1}},\]
			which we can see is a submodule of the simple module $B_k/B_{k+1}$. Thus, the quotients $A_k/A_{k+1}$ are either $0$ or simple, so after removing the $A_k$ which have $A_k=A_{k+1}$, we will have a composition series of length at most $n$. 
			\item We show that $C$ has finite length. Indeed, set $C_k:=(B_k+A)/A$ so that we get the descending chain
			\[C=C_0\supseteq C_1\supseteq\cdots\supseteq C_{n-1}\supseteq C_n=(0).\]
			We can compute the quotients as\footnote{The kernel of the composite of surjective maps $B_k+A\onto(B_k+A)/A\onto\frac{(B_k+A)/A}{(B_{k+1}+A)/A}$ is $B_{k+1}+A$.}
			\[\frac{(B_k+A)/A}{(B_{k+1}+A)/A}\cong\frac{B_k+A}{B_{k+1}+A}.\]
			But now we note that the map $B_k\into B_k+A\onto(B_k+A)/(B_{k+1}+A)$ is surjective and has kernel containing $B_{k+1}$, so there is a surjective map
			\[\frac{B_{k+1}}{B_k}\onto\frac{C_{k+1}}{C_k}.\]
			In particular, this kernel is a submodule of a simple module, so the quotient $C_{k+1}/C_k$ is either $B_{k+1}/B_k$ (and therefore simple) or $(0)$. So, removing the $C_k$ such that $C_k=C_{k+1}$ will remove the $(0)$s from the composition series will give $C$ a composition series of length at most $n$.
		\end{itemize}
		We remark that the above arguments showed that the length of $B$ upper-bounds the lengths of $A$ and $C$ by constructing a composition series with length at most the length $B$.

		\item Suppose that $A$ and $C$ both have finite length. In particular, we can conjure a composition series
		\[C=:C_0\supsetneq C_1\supsetneq\cdots\supsetneq C_{n-1}\supseteq C_n:=(0),\]
		and the idea is to pull this back along $\pi:B\onto C$, setting $B_k:=\pi^{-1}(C_k)$. In particular, we will get a descending chain (in fact strictly descending because $\pi$ is surjective) of submodules
		\[B=B_0\supsetneq B_1\supsetneq\cdots\supsetneq B_{n-1}\supseteq B_n=A,\tag{$*$}\label{eq:startcompseriesb}\]
		where $B_n=\pi^{-1}((0))=A$ by exactness. Furthermore, we see that $\pi$ restricts to a surjection $B_k\to C_k$, and upon modding out the image by $C_{k+1}$, we see that exactly $B_{k+1}$ will be in the kernel, implying that the quotient
		\[\frac{B_k}{B_{k+1}}\cong\frac{C_k}{C_{k+1}}\]
		will be simple. So indeed, \hyperref[eq:startcompseriesb]{($*$)} starts a composition series for $B$ with so far $n$ composition factors.

		However, we can then append \hyperref[eq:startcompseriesb]{($*$)} with the composition series of $A$, thus providing a composition series for $B$ with length equal to the sum of the lengths of $A$ and $C$. It follows from the definition that the length of $B$ is at most the sum of the lengths of $A$ and $C$.
		\qedhere
	\end{itemize}
\end{proof}
\begin{remark}[Nir]
	We will show below that the length of a module of finite length is unique among composition series. In this case, the second part of the argument shows that equality holds: the length of $B$ is equal to the sums of the lengths of $A$ and $C$.
\end{remark}
\begin{corollary} \label{cor:finlengthquotient}
	Fix a module $M$ and a chain of submodules
	\[M:=M_0\supseteq M_1\supseteq\cdots\supseteq M_N:=(0).\]
	If each quotient $M_i/M_{i+1}$ is of finite length, then $M$ is of finite length.
\end{corollary}
\begin{proof}
	We induct on $N$. When $N=1$, we have $M=M_0/M_1$, so there is nothing to say. Otherwise, by the induction, we may assume that $M_1$ is of finite length because of the chain of submodules
	\[M_1\supseteq\cdots\supseteq M_N:=\{0\}\]
	with $M_i/M_{i+1}$ always simple. But now we see we have the short exact sequence
	\[0\to M_1\to M\to M_0/M_1\to0,\]
	so because $M_1$ and $M_0/M_1$ both have finite length, $M=M_0$ will have finite length.
\end{proof}

\subsection{The Jordan--H\"older Theorem}
We will now check that the length of a submodule is well-defined. Here is a follow-up result from the argument of \autoref{prop:finlengthses}; we will use it as a technical lemma in the proof.
\begin{lemma} \label{lem:stepdownlength}
	Fix $A\subsetneq B$ a proper containment of $R$-modules, and suppose that $B$ has finite length so that $A$ also has finite length. Then the length of $A$ is strictly less than the length of $B$.
\end{lemma}
\begin{proof}
	We will show that, if the lengths of $A$ and $B$ are in fact equal, then $A=B$. As in the argument for \autoref{prop:finlengthses}, fix a composition series
	\[B=:B_0\supsetneq B_1\supsetneq\cdots\supsetneq B_{n-1}\supsetneq B_0:=(0),\]
	where $n$ is the length of $B$. This induces a descending chain
	\[A=A_0\supseteq A_1\supseteq\cdots\supseteq A_{n-1}\supseteq A_n=(0),\tag{$*$}\label{eq:compseriesa}\]
	where $A_k:=A\cap B_k$. This chain for $A$ would be a composition series, but some composition factors might vanish, and we obtained a composition series for $A$ by removing the equal terms from the series.
	
	However, if the length of $A$ were equal to the length of $B$, then in the process of removing redundancies from \hyperref[eq:compseriesa]{($*$)} must not do anything at all, for any removed redundancy would imply that the length of $A$ is strictly less than the length of $B$.
	
	It follows that we have
	\[\frac{A_k}{A_{k+1}}=\frac{A\cap B_k}{A\cap B_{k+1}}\cong\frac{(A\cap B_k)+B_{k+1}}{B_{k+1}}\]
	is equal to $B_k/B_{k+1}$ for each $k$. In particular, $(A\cap B_k)+B_{k+1}=B_k$ for each $k$.

	Now, we claim that $A$ contains $B_k$ by inducting downwards on $k$; this will finish because it will show $A$ contains $B=B_0$ and hence equals $B$. Now, the statement is true for $k=n$ because $A_n=B_n=(0)$. Then for the inductive step, we know $A\supseteq B_{k+1}$, so it follows
	\[B_k=(A\cap B_k)+B_{k+1}\subseteq A\]
	as well, finishing.
\end{proof}

Here is the main result on composition series.
\begin{theorem}[Jordan--H\"older] \label{thm:jordanholder}
	Fix $M$ an $R$-module which has a composition series. Then all composition series of $M$ have the same length. Namely, any strictly descending chain of submodules of $M$ can be refined into a composition series of length equal to the length of $M$.
\end{theorem}
\begin{proof}
	We follow the proof in Eisenbud. Fix $M$ of length $n$. The key claim is as follows.
	\begin{lemma} \label{lem:boundchain}
		Fix $M$ an $R$-module of length $n$. If we have a strictly descending chain
		\[M=M_0\supsetneq M_1\supsetneq M_2\supsetneq\cdots\supsetneq M_k,\]
		then $k\le n$.
	\end{lemma}
	\begin{proof}
		We induct on $n$. When $n=0$, the definition of a composition series forces $M=0$, so our strictly descending chain must consist of only $0$, so $k=0$.

		For the inductive step, we note that \autoref{lem:stepdownlength} forces the length of $M_1$ to be strictly less than the length of $M$, so the length of $M_1$ is at most $n-1$. So the inductive hypothesis tells us that the chain
		\[M_1\supsetneq M_2\supsetneq\cdots\supsetneq M_k\]
		forces $k-1\ge n-1$, so $k\ne n$ follows.
	\end{proof}
	It follows from \autoref{lem:boundchain} that all composition series have length at most $n$, but because $n$ is the length of the shortest composition series, we see that all composition series must have length exactly $n$.

	As for the second claim, suppose that we have a strictly descending chain
	\[M=M_0\supsetneq M_1\supsetneq M_2\supsetneq\cdots\supsetneq M_k.\]
	If this is not a composition series, we claim that we can make it longer. Indeed, if some term $M_\ell/M_{\ell+1}$ is not simple, then we can find a proper nonzero submodule $N'\subseteq M_\ell/M_{\ell+1}$, which we can pull back along $M_\ell\onto M_\ell/M_{\ell+1}$ to a submodule $N$ strictly contained between $M_\ell$ and $M_{\ell+1}$. In particular, $N'\ne0$ forces $N\ne M_{\ell+1}$, and $N'\ne M_\ell/M_{\ell+1}$ forces $N\ne M_{\ell}$.
	
	Thus, we have the strictly descending chain
	\[M=M_0\supsetneq M_1\supsetneq\cdots\supsetneq M_\ell\supsetneq N\supsetneq M_{\ell+1}\supsetneq\cdots\supsetneq M_k\]
	of length $k+1$. We can continue this process as long as we don't have a composition series, but because all composition series have length $n$, this means that we can only add terms so long as we have length less than $n$. Namely, we have showed that the all strictly descending chains can be refined to a composition series of length $n$.
\end{proof}

\subsection{Modules of Finite Length}
The Jordan--H\"older theorem gives us the following quick result about modules of finite length, which is arguably a classification of modules of finite length. (We will shortly be able to give better descriptions of modules of finite length.)
\begin{corollary}
	Fix $M$ an $R$-module. Then $M$ is of finite length if and only if $M$ is both Artinian and Noetherian.
\end{corollary}
\begin{proof}
	The backwards direction is \autoref{lem:artnoethimpliesfinlength}.
	
	For the forwards direction, suppose $M$ has a composition series with $n$ composition factors. We show that $M$ is Noetherian and Artinian separately. The point is that \autoref{lem:boundchain} basically says that we cannot have arbitrarily long strictly descending or ascending chains.
	\begin{itemize}
		\item We show that $M$ is Noetherian. For this, fix an ascending chain of submodules
		\[N_0\subseteq N_2\subseteq N_3\subseteq\cdots.\]
		Suppose for the sake of contradiction that this ascending chain never stabilizes. Removing equal terms from the chain, we may assume that all the submodules are distinct. But then we can create the descending chain
		\[M\supseteq N_{n+1}\supsetneq N_n\supsetneq N_{n-1}\supsetneq\cdots\supsetneq N_1\supsetneq N_0\]
		of length $n+1$. This violates \autoref{lem:boundchain}, which is our contradiction.
		\item We show that $M$ is Artinian. For this, fix a descending chain of submodules
		\[N_0\supseteq N_2\supseteq N_3\supseteq\cdots.\]
		Suppose for the sake of contradiction that this descending chain never stabilizes. Removing equal terms from the chain, we may assume that all the submodules are distinct. But then we can create the descending chain
		\[M\supseteq N_0\supsetneq N_1\supsetneq N_2\supsetneq\cdots\supsetneq N_n\supsetneq N_{n+1}\]
		of length $n+1$. This violates \autoref{lem:boundchain}, which is our contradiction.
		\qedhere
	\end{itemize}
\end{proof}

Quickly, note that the support of $M$ is particularly nice when $M$ has a composition series, which essentially comes from various facts we've already proven.
\begin{lemma} \label{lem:computefinitesupp}
	Fix $M$ an $R$-module with a finite composition series
	\[M:=M_0\supsetneq M_1\supsetneq\cdots\supsetneq M_n:=\{0\}.\]
	If the composition factors are $R/\mf p_k\cong M_{k-1}/M_k$ for $k\in\{1,\ldots,n\}$, then $\op{Supp}M=\{\mf p_1,\ldots,\mf p_n\}$.
\end{lemma}
\begin{proof}
	We induct on the length $n$ of $M$, using \autoref{prop:suppses} for the induction. If $n=0$, then the composition series has no composition factors, so $M=0$ and so $\op{Supp}M=\emp$ by \autoref{ex:emptysupport}, which matches.

	For the inductive step, we take $n>0$ and note that
	\[M_1\supsetneq\cdots\supsetneq M_n=\{0\}\]
	provides a composition series for $M_1$ of length $n-1$, where our composition factors are $R/\mf p_k\cong M_{k-1}/M_k$ for $k\in\{2,\ldots,n\}$. So the inductive hypothesis promises
	\[\op{Supp}M_1=\{\mf p_2,\ldots,\mf p_n\}.\]
	Now we use \autoref{prop:suppses}. Namely, we have the short exact sequences
	\[0\to M_1\to M_0\to M_0/M_1\to0\]
	which tells us that
	\[\op{Supp}M=\op{Supp}M_0=\op{Supp}(M_0/M_1)\cup\op{Supp}M_1\]
	by \autoref{prop:suppses}. But $\op{Supp}(M_0/M_1)=\op{Supp}R/\mf p_1=\{\mf p_1\}$ by \autoref{exe:classifysimple}. It follows that
	\[\op{Supp}M=\{\mf p_1,\mf p_2,\ldots,\mf p_n\},\]
	which is what we wanted.
\end{proof}
\begin{remark}[Nir] \label{rem:suppismaximal}
	As a corollary, we see that the support of a module of finite length is made of exclusively maximal ideals because each of the $\mf p_k$ are maximal by \autoref{exe:classifysimple}.
\end{remark}
And here is a nice result which we get from this.
\begin{theorem} \label{thm:finitelengthdecomp}
	Fix $M$ an $R$-module of finite length. Then the following are true.
	\begin{listalph}
		\item The multiplicity of a simple module $R/\mf p$ as a composition factor is the length of $M_\mf p$ as an $R_\mf p$-module. In particular, this is independent of the choice of composition series.
		\item We can glue the localization maps $M\to M_\mf p$ together to form an $R$-module isomorphism
		\[M\cong\bigoplus_{\mf p\in\op{Supp}M}M_\mf p.\]
	\end{listalph}
\end{theorem}
\begin{proof}
	The key is the following lemma.
	\begin{lemma} \label{lem:localizesimplemodule}
		Let $R$ be a ring and $\mf p,\mf q\subseteq R$ be maximal ideals. Then
		\[(R/\mf p)_\mf q=\begin{cases}
			R_\mf p/\mf p_\mf p & \mf p=\mf q, \\
			0 & \mf p\ne\mf q.
		\end{cases}\]
	\end{lemma}
	\begin{proof}
		Because localization is an exact functor, we have that
		\[(R/\mf p)_\mf q\simeq R_\mf q/\mf p_\mf q.\]
		Now, $R_\mf q$ is a local ring with maximal ideal $\mf q_\mf q$ by \autoref{prop:localizetolocal}, so we have two cases.
		\begin{itemize}
			\item If $\mf p=\mf q$, then we just have $R_\mf p/\mf p_\mf p$, as promised.
			\item If $\mf p\ne\mf q$, then $\mf q\not\subseteq\mf p$ because $\mf q$ is a maximal ideal, so we can find $u\in\mf q\setminus\mf p$. The point is that $\frac u1\in\mf m_\mf p$ is a unit because $u\in R\setminus\mf p$. It follows
			\[\mf m_\mf p=R_\mf p\]
			because $\mf m_\mf p$ contains a unit, so $M_\mf p=0$ has length $0$.
		\end{itemize}
		The above computations finish the proof.
	\end{proof}
	We now proceed with the proof.
	\begin{listalph}
		\item We glue together cases of \autoref{lem:localizesimplemodule}. Fix a composition series
		\[M\coloneqq M_0\supsetneq M_1\supsetneq\cdots\supsetneq M_n\coloneqq\{0\}.\]
		Choosing any simple module $R/\mf p$, we note that we can localize this composition series into
		\[M_\mf p=(M_0)_\mf p\supseteq(M_1)_\mf p\supseteq\cdots\supseteq(M_n)_\mf p=\{0\}.\]
		Now, for each $1\le k\le n$, we set $R/\mf p_k\cong M_k/M_{k-1}$ so that we can write
		\[(M_k)_{\mf p_k}/(M_{k-1})_{\mf p}\simeq(M_k/M_{k-1})_{\mf p}\cong(R/\mf p_k)_\mf p=\begin{cases}
			R_\mf p/\mf p_\mf p & \mf p_k=\mf p, \\
			0 & \mf p_k\ne\mf p,
		\end{cases}\]
		where we have used the fact that localization is exact in the first equality and \autoref{lem:localizesimplemodule} in the last equality. As such, let $k_1>k_2>\ldots>k_m$ denote the indices $k$ with $M_k/M_{k-1}\cong R/\mf p$ so that
		\[(M_{k_1})_\mf p\supsetneq(M_{k_2})_\mf p\supseteq\cdots\supseteq(M_{k_m})_\mf p\supseteq\{0\}\]
		is a composition series for $M_\mf p$. Indeed, we see that
		\[(M_{k_i})_\mf p/(M_{k_i-1})_\mf p=(M_{k_i-1})_\mf p/(M_{k_i-2})_\mf p=\cdots=M_{k_{i-1}+1}/M_{k_{i-1}}=0\]
		by the construction of $k_i>k_{i-1}$, so $(M_0)_\mf p=(M_{k_1})_\mf p$ and $(M_{k_i})_\mf p/(M_{k_{i+1}})_\mf p\cong R_\mf p/\mf p_\mf p$ is simple for each $i$.

		To finish, we note that $m$ is the length of $M_\mf p$ as an $R_\mf p$-module because we have witnessed such a composition series. But at the same time $m$ is the number of indices $k$ with $M_k/M_{k-1}\cong R/\mf p$, which is the multiplicity of $R/\mf p$ as a composition factor in a composition series of $M$. This finishes.

		\item Let $\varphi$ denote the map
		\[M\to\bigoplus_{\mf p\in\op{Supp}M}M_\mf p\]
		given by summing the localization maps. By \autoref{cor:liftinglocal}, it suffices to show that the localized map $\varphi_\mf m$ is an isomorphism for each maximal ideal $\mf m\subseteq R$. Note that $\op{Supp}M$ is finite by \autoref{lem:computefinitesupp}, so we may write
		\[\Bigg(\bigoplus_{\mf p\in\op{Supp}M}M_\mf p\Bigg)_\mf m=\Bigg(\bigoplus_{\mf p\in\op{Supp}M}M_\mf p\Bigg)\otimes_RR_\mf m\simeq\bigoplus_{\mf p\in\op{Supp}M}M_\mf p\otimes_RR_\mf m=\bigoplus_{\mf p\in\op{Supp}M}(M_\mf p)_\mf m.\]
		In particular, $\varphi_\mf m$ sends $m/s\in M_\mf m$ to $\varphi(m)/s$, which is $\frac1s(m/1)_{\mf p}\in\big(\bigoplus_\mf pM_\mf p)_\mf m$. Tracking this element through the above isomorphisms, we see that we go to $\big(\frac1s(m/1)\big)_{\mf p}$, so $\varphi_\mf m$ is the sum of the localization maps
		\[M_\mf m\to (M_\mf p)_\mf m\]
		as $\mf p$ varies over $\op{Supp}M$.\footnote{Here's a more abstract way to see this: localization is a tensor functor and hence additive, so writing $\varphi=\sum_\mf p\iota_\mf p$ where $\iota_\mf p\colon M\to M_\mf p$ gives $\varphi_\mf m=\sum_\mf p(\iota_\mf p)_\mf m$.} 
		
		We now evaluate $(M_\mf p)_\mf m$. We now have two cases.
		\begin{itemize}
			\item Take $\mf p\ne\mf m$. Note from (a) that $M_\mf p$ has a composition series as an $R_\mf p$-module where all the composition factors are isomorphic to $R_\mf p/\mf p_\mf p$ as $R_\mf p$-modules. However, pulling back to $R$, we see that $R_\mf p/\mf p_\mf p\cong R/\mf p$ as $R$-modules\footnote{This is induced by the natural map $R\to R_\mf p\onto R_\mf p/\mf p_\mf p$: if $\frac r1=\frac xs\in\mf p_\mf p$ for $x\in\mf p$ and $s\notin\mf p$, then there is $u\notin\mf p$ with $(us)r=ux\in\mf p$, so $r\in\mf p$.} which will give a composition series of $M_\mf p$ with composition factors consisting entirely of $R/\mf p$.

			On the other hand, the length of $(M_\mf p)_\mf m$ as an $R_\mf m$-module is the number of $R/\mf m$s in its composition series. Because $\mf m\ne\mf p$, we see that this is $0$, so $(M_\mf p)_\mf m$ has length $0$, implying
			\[(M_\mf p)_\mf m=0.\]
			\item Take $\mf m=\mf p$. Then in fact the localization map $M_\mf p\to(M_\mf p)_\mf p$ is an isomorphism. Indeed, we can construct its inverse as
			\[\arraycolsep=1.4pt\begin{array}{cccc}
				& M_\mf p &\to& (M_\mf p)_\mf p \\
				f\colon & (m/s) &\mapsto& {\textstyle\frac{m/s}1} \\
				g\colon & m/(us) &\mapsfrom& {\textstyle\frac{m/s}u}
			\end{array}\]
			where we can see $g(f(m/s))=g\big(\frac{m/s}1\big)=\frac ms$ and $f\big(g\big(\frac{m/s}u\big)\big)=\frac{m/(us)}1=\frac{(um)/(us)}u=\frac{m/s}u$.
		\end{itemize}
		In total, if $\mf m\notin\op{Supp}M$, then $M_\mf m=0$ and $\mf p\ne\mf m$ for each $\mf p\in\op{Supp}M$, so $\varphi_\mf m$ is the zero map from $0$ to $0$. Otherwise, if $\mf m\in\op{Supp}M$, then $\varphi_\mf m$ is zero everywhere except for exactly at $\mf p=\mf m$ where $\varphi_\mf m$ is an isomorphism; so $\varphi_\mf m$ is an isomorphism plus a bunch of zeroes, finishing.
	\end{listalph}
	The above finishes the proof.
\end{proof}

\subsection{Artinian Grab-Bag}
We are now able to give the following classification.
\begin{theorem} \label{thm:artingrabbag}
	Fix $R$ a ring. Then $R$ is Artinian if and only if $R$ is Noetherian and all its primes are maximal.
\end{theorem}
We split the proof into two parts.
\begin{proof}[Proof of the backwards direction in \autoref{thm:artingrabbag}]
	Suppose $R$ is neither Artinian nor Noetherian. It will suffice to show that not all prime ideals of $R$ are maximal.
	
	Being neither Artinian nor Noetherian conspire to give us an ideal $J$ maximal with respect to the property that $R/J$ is not Artinian: because $R$ is not Artinian, the collection
	\[\mathcal P:=\{\text{ideal }J\subseteq R:R/J\text{ is not Artinian}\}\]
	is nonempty (for $(0)\in\mathcal P$), and because $R$ is Noetherian, there will be a maximal element, which we call $\mf p$. Observe that $\mf p$ is not maximal, for then $R/\mf p$ would be a field and hence be Artinian.
	
	With this in mind, we claim that $\mf p$ must be prime. This will finish because $\mf p$ will be a prime which is not maximal. Well, suppose that $a\notin\mf p$. Consider the short exact sequence of $R$-modules
	\[0\to\frac{\mf p+(a)}{\mf p}\to \frac R{\mf p}\to\frac R{\mf p+(a)}\to0.\]
	We are going to profit from studying this short exact sequence by using \autoref{prop:artinianses}. In particular, $R/{\mf p}$ is not Artinian, so we cannot have both $R$-modules on its left and right be Artinian.

	Well, $\mf p\subsetneq\mf p+(a)$, so by maximality, $R/(\mf p+(a))$ will have to be Artinian. So instead $(\mf p+(a))/\mf p$ cannot be Artinian. But now we observe that we have the following isomorphism of $R$-modules.
	\begin{lemma}
		Fix $R$ a ring and $I\subseteq R$ an ideal and $a\in R$. Then we define $(I:a):=\{r\in R:ar\in I\}$, and we claim that $(I:a)$ is an ideal and
		\[\frac R{(I:a)}\cong\frac{I+(a)}I.\]
	\end{lemma}
	\begin{proof}
		Note that there is an $R$-module map $\varphi:R\to(a)+I$ by
		\[\varphi:x\mapsto ax.\]
		Indeed, $\varphi(r_1x_1+r_2x_2)=ar_1x_1+ar_2x_2=r_1\varphi(x_1)+r_2\varphi(x_2)$. Now, modding out the image by $I\subseteq(a)+I$, we get a map
		\[\widetilde\varphi:R\to\frac{(a)+I}{I}.\]
		We note that this map is surjective because any coset $[x]_I$ with $x\in(a)+I$ can have $x=ar+p$ where $r\in R$ and $p\in I$, meaning that $\widetilde\varphi(r)=[ar]_I=[x]_I$. Further, we can compute the kernel of $\widetilde\varphi$ as
		\[\{r\in R:ar\in I\}=(I:a).\]
		Thus, $(I:a)=\ker\widetilde\varphi$ is an ideal, and $\widetilde\varphi$ induces an isomorphism $R/(I:a)\to(I+(a))/I$, finishing.
	\end{proof}
	Now, because $(\mf p+(a))/\mf p$ is not Artinian, we see $R/(\mf p:a)$ cannot be Artinian. But certainly $\mf p\subseteq(\mf p:a)$ because each $x\in\mf p$ has $ax\in\mf p$, so we must have
	\[\mf p=(\mf p:a)\]
	by the maximality of $\mf p$. We now finish the proof. Suppose now that $ab\in\mf p$, and we claim that $b\in\mf p$. Well, $ab\in\mf p$ implies that $b\in(\mf p:a)=\mf p$. So we are done.
\end{proof}
\begin{proof}[Proof of the forwards direction of \autoref{thm:artingrabbag}]
	For the other direction, we note that we can show all primes are maximal without tears.
	\begin{lemma}
		Fix $R$ an Artinian ring. Then any prime ideal $\mf p\subseteq R$ is maximal.
	\end{lemma}
	\begin{proof}
		We follow the argument given \href{http://math.stanford.edu/~conrad/210BPage/handouts/math210b-Artinian.pdf}{here}. Well, given $\mf p$ a prime so that $R/\mf p$ is an integral domain, we show that $R/\mf p$ is actually a field. Well, we can pick up $[x]_\mf p\ne0$ represented by some $x\notin \mf p$, and we show that $[x]_\mf p$ is a unit. Note that we have the descending chain
		\[(x)\supseteq\left(x^2\right)\supseteq\left(x^3\right)\supseteq\cdots,\]
		which must eventually stabilize, so there is some $n\in\NN$ such that $\left(x^n\right)=\left(x^{n+1}\right)$, so there is $r\in R$ with $x^n=rx^{n+1}$. In particular,
		\[x^n(1-xr)=0.\]
		Working in $R/\mf p$, we see that $[x]_\mf p\ne0$, so the fact that $R/\mf p$ is an integral domain implies that
		\[[x]_\mf p\cdot[r]_\mf p=1,\]
		so indeed, $[x]_\mf p$ is a unit.
	\end{proof}
	So it remains to show that $R$ is Artinian implies that $R$ is Noetherian. We introduce the following definition.
	\begin{definition}[Jacobson radical]
		Fix $R$ a ring. Then we define the \textit{Jacobson radical} $J\subseteq R$ to be
		\[J:=\bigcap_\mf m\mf m,\]
		where $\mf m$ ranges over all maximal ideals of $R$.
	\end{definition}
	Note that the Jacobson radical is an ideal because ideals are closed under intersection. Alternatively, we can view $J$ as the kernel of the map
	\[R\to\prod_\mf mR/\mf m,\]
	for any ring $R$, where the product is over maximal ideals $\mf m\subseteq R$.

	In fact, in the case where $R$ is Artinian, the above map will be surjective. By the Chinese remainder theorem, it suffices to show that there are only finitely many maximal ideals of $R$.
	\begin{lemma} \label{lem:artinianfinmaxes}
		Fix $R$ an Artinian ring. Then $R$ has only finitely many maximal ideals.
	\end{lemma}
	\begin{proof}
		We follow the argument from \href{https://stacks.math.columbia.edu/tag/00J4}{here} because I think it is pretty close to what I understand Professor Serganova saying in class.
		
		The point here is that infinitely many maximal ideals will induce an infinite composition series. Indeed, suppose that we have some infinite collection $\{\mf m_k\}_{k=1}^\infty$ of maximal ideals, and we claim that the chain
		\[\mf m_1\supseteq\mf m_1\cap\mf m_2\supseteq\mf m_1\cap\mf m_2\cap\mf m_3\supseteq\cdots\]
		is an infinite composition series; this will verify that $R$ is not Artinian.
		
		But now, this chain is infinite, and to see that it is a composition series, we have to check that
		\[\frac{\mf m_1\cap\cdots\cap\mf m_k}{\mf m_1\cap\cdots\cap\mf m_k\cap\mf m_{k+1}}\]
		is simple for each $k\ge1$. Indeed, note that we have the following commutative diagram with exact rows, where the vertical morphisms are isomorphisms given by the Chinese remainder theorem.
		% https://q.uiver.app/?q=WzAsMTAsWzAsMCwiMCJdLFsxLDAsIlxcZGlzcGxheXN0eWxlXFxmcmFje1xcbWYgbV8xXFxjYXBcXGNkb3RzXFxjYXBcXG1mIG1fbn17XFxtZiBtXzFcXGNhcFxcY2RvdHNcXGNhcFxcbWYgbV9uXFxjYXBcXG1mIG1fe24rMX19Il0sWzIsMCwiXFxkaXNwbGF5c3R5bGVcXGZyYWMgUntcXG1mIG1fMVxcY2FwXFxjZG90c1xcY2FwXFxtZiBtX25cXGNhcFxcbWYgbV97bisxfX0iXSxbMywwLCJcXGRpc3BsYXlzdHlsZVxcZnJhYyBSe1xcbWYgbV8xXFxjYXBcXGNkb3RzXFxjYXBcXG1mIG1fbn0iXSxbNCwwLCIwIl0sWzQsMSwiMCJdLFszLDEsIlxcZGlzcGxheXN0eWxlXFxiaWdvcGx1c197az0xfV5uUi9cXG1mIG1fayJdLFsyLDEsIlxcZGlzcGxheXN0eWxlXFxiaWdvcGx1c197az0xfV57bisxfVIvXFxtZiBtX2siXSxbMCwxLCIwIl0sWzEsMSwiUi9cXG1mIG1fe24rMX0iXSxbMiw3XSxbMiwzXSxbMyw2XSxbNyw2XSxbMyw0XSxbNiw1XSxbOCw5XSxbOSw3XSxbMCwxXSxbMSwyXV0=&macro_url=https%3A%2F%2Fraw.githubusercontent.com%2FdFoiler%2Fnotes%2Fmaster%2Fnir.tex
		\[\begin{tikzcd}
			0 & {\displaystyle\frac{\mf m_1\cap\cdots\cap\mf m_n}{\mf m_1\cap\cdots\cap\mf m_n\cap\mf m_{n+1}}} & {\displaystyle\frac R{\mf m_1\cap\cdots\cap\mf m_n\cap\mf m_{n+1}}} & {\displaystyle\frac R{\mf m_1\cap\cdots\cap\mf m_n}} & 0 \\
			0 & {R/\mf m_{n+1}} & {\displaystyle\bigoplus_{k=1}^{n+1}R/\mf m_k} & {\displaystyle\bigoplus_{k=1}^nR/\mf m_k} & 0
			\arrow[from=1-3, to=2-3]
			\arrow[from=1-3, to=1-4]
			\arrow[from=1-4, to=2-4]
			\arrow[from=2-3, to=2-4]
			\arrow[from=1-4, to=1-5]
			\arrow[from=2-4, to=2-5]
			\arrow[from=2-1, to=2-2]
			\arrow[from=2-2, to=2-3]
			\arrow[from=1-1, to=1-2]
			\arrow[from=1-2, to=1-3]
		\end{tikzcd}\]
		In particular, the square commutes because $[r]_{\mf m_1\cap\cdots\mf m_n\cap\mf m_{n+1}}$ in the top-left will go to $([r]_{\mf m_1},\ldots,[r]_{\mf m_n})$ in the bottom-right, no matter which we path we choose. Thus, there is an induced isomorphism
		\[\frac{\mf m_1\cap\cdots\cap\mf m_n}{\mf m_1\cap\cdots\cap\mf m_n\cap\mf m_{n+1}}\cong\frac R{\mf m_{n+1}},\]
		so indeed this $R$-module is simple, say by \autoref{rem:classifysimple}.
	\end{proof}
	\begin{remark}
		Intuitively, there can only be finitely many maximal ideals $\mf m$ because each $R/\mf m$ will induce a composition factor, of which there are only finitely many because $R$ is Artinian. In the above proof, we have actually shown how to induce such a composition series using each of these composition factors.
	\end{remark}
	\begin{remark}[Nir]
		In fact, an Artinian ring will have only finitely many prime ideals, which we can see directly because all primes are maximal.
	\end{remark}

	We now proceed with the proof of \autoref{thm:artingrabbag}. The main idea is to try to make \autoref{lem:artinianfinmaxes} sharp by using the descending chain of submodules
	\[R\supsetneq\mf m_1\supsetneq\mf m_1\cap\mf m_2\supsetneq\cdots\supsetneq\bigcap_{k=1}^r\mf m_k,\]
	where $\{\mf m_k\}_{k=1}^r$ are our maximal ideals. However, it turns out that this descending chain may and can simply bottom out at the Jacobson radical $J$, which might be nonzero, and so we will not get an actual composition series. But at least we can (again) hope that $J$ is ``small enough'' so that continue this sequence somehow.
	\begin{remark}[Serganova] \label{rem:expandrmodj}
		Here is alternate motivation for the below claim: the payoff to \autoref{lem:artinianfinmaxes} is that the Chinese remainder theorem gives us right-exactness of the short exact sequence
		\[0\to J\to R\to\prod_\mf mR/\mf m\to 0.\]
		In particular,
		\[R/J\cong\prod_\mf mR/\mf m\]
		is a product of finitely many simple modules $R/\mf m$, so $R/J$ will be of finite length. (Note $R/J$ has only finitely many ideals because each $R/\mf m$ has only two ideals.) We would like to turn the fact that $R/J$ is of finite length into the fact that $R$ is of finite length, but we will need a smallness condition on $J$ to make this work.
	\end{remark}
	
	The key claim is as follows.
	\begin{lemma} \label{lem:jacobsonnilpotent}
		Fix $R$ an Artinian ring. Then the Jacobson radical $J$ is nilpotent.
	\end{lemma}
	\begin{proof}
		Observe that we have a descending chain
		\[J\supseteq J^2\supseteq J^3\supseteq\cdots,\]
		which stabilizes because $R$ is Artinian. So suppose that $J^N=J^{N+1}=I$ for some $N\ge1$, and we hope $I=(0)$. By the stabilization, we see $I^2=J^{2N}=J^N=I$.

		Now, if $I\ne(0)$, then we can find a minimal ideal $K\subseteq I$ such that $IK\ne(0)$ and $K\ne(0)$. (Note that $I=K$ would work---$I^2=I\ne(0)$---but is perhaps not minimal; we need the Artinian condition to get the minimal such ideal.) We start with some fact-collection on $K$. Note that $I(IK)=I^2K=IK\ne(0)$ and $IK\ne(0)$ while $IK\subseteq K$, so $K$'s minimality forces
		\[IK=K.\]
		Furthermore, because $K\ne(0)$, there exists $a\in K\setminus\{(0)\}$ such that $aI\ne(0)$. So $(a)I\ne(0)$ while $(a)\ne0$, so $(a)\subseteq K$ combined with $K$'s minimality (again) forces
		\[K=(a).\]
		Combining the above two facts, we are granted $b\in I$ such that $ba=a$.

		But here is the key trick: we can write $ba=a$ as
		\[a(1-b)=0.\]
		However, $b\in I$ implies $b\in J$ implies $1-b\notin J$, so $(1-b)$ is not in any maximal ideal. So it follows $(1-b)=R$, so $1-b\in R^\times$! Upon cancelling, we see $K=(a)=0$, which is our contradiction.
	\end{proof}
	We now return to the proof. We claim that $R$ is of finite length, which will imply that $R$ is Noetherian. Instead of using intersections of maximal ideals as in \autoref{lem:artinianfinmaxes}, our salvage will use products of maximal ideals, which grant us enough flexibility.
	
	Indeed, the main obstruction is verifying that some finite product of maximal ideals will actually vanish. But if, say, $J^N=(0)$ where $J\subseteq R$ is our Jacobson radical, then
	\[(\mf m_1\cdots\mf m_r)^N\subseteq\left(\bigcap_{k=1}^n\mf m_k\right)^N=J^N=(0).\]
	So some finite product of maximal ideals will vanish; for the sake of not mixing up our letters, let $\{\mf p\}_{k=1}^n$ be a sequence of (not necessarily distinct) maximal ideals so that $\mf p_1\cdots\mf p_n=(0)$. Then we work with the chain
	\[R\supseteq\mf p_1\supseteq\mf p_1\mf p_2\supseteq\cdots\supseteq\mf p_1\mf p_2\cdots\mf p_{n-1}\supseteq\mf p_1\mf p_2\cdots\mf p_n=(0).\]
	By \autoref{cor:finlengthquotient}, it suffices to check that each quotient
	\[M_k:=\frac{\mf p_1\cdots\mf p_k}{\mf p_1\cdots\mf p_{k+1}}\]
	is of finite length, for each $k\ge0$. (When $k=0$, the empty product gives $R$.) Now, $M_k$ is an $R$-module, but note that the $\mf p_{k+1}$-action kills an element, so in fact the ring morphism $R\to\op{End}(M_k)$ descends to a ring morphism $R/\mf p_{k+1}\to\op{End}(M_k)$.
	
	This is to say that $M_k$ is an $R/\mf p_{k+1}$-vector space. To show that $M_k$ is of finite length, we need to know that $M_k$ is finite-dimensional. Well, if $M_k$ were not finite-dimensional, then an infinite basis would provide an infinitely descending chain of $R/\mf p_{k+1}$-submodules
	\[\frac{\mf p_1\cdots\mf p_k}{\mf p_1\cdots\mf p_{k+1}}\supsetneq\frac{N_1}{\mf p_1\cdots\mf p_{k+1}}\supsetneq \frac{N_1}{\mf p_1\cdots\mf p_{k+1}}\supsetneq\cdots.\]
	Taking the pre-images of $R\to R/\mf p_1\cdots\mf p_{k+1}$, this lifts to an infinite descending chain
	\[\mf p_1\cdots\mf p_k\supsetneq N_1+\mf p_1\cdots\mf p_{k+1}\supsetneq N_2+\mf p_1\cdots\mf p_{k+1}\supsetneq\cdots,\]
	which violates the condition that $R$ is Artinian.\footnote{Technically we ought to check that these submodules are distinct. This is because the projection map $\varphi:R\to R/\mf p_1\cdots\mf p_{k+1}$ is surjective, so the pre-image of distinct sets will remain distinct.} This finishes.
	% But observe that we have the short exact sequences
	% \[0\to\frac{J^i}{J^{i+1}}\to\frac{R}{J^{i+1}}\to\frac R{J^i}\to0.\tag{$*$}\label{eq:jises}\]
	% So by \autoref{prop:finlengthses}, it suffices to check that $R/J^{i+1}$ is of finite length for each $i$.
	% We do this by induction. For $i=0$, we have already shown that $R/J$ is of finite length above. For the inductive step, take $i>0$. We know that $R/J^i$ is of finite length, but then \autoref{prop:finlengthses} applied to \hyperref[eq:jises]{($*$)} tells us that that $R/J^{i+1}$ will be of finite length as well. This finishes.
	% has the nice property that $J^{i+1}/J^i$ is a finite direct sum of simple modules; this essentially comes from our formula for $R/J$ from the short exact sequence. % \todo{}
	% This implies that we have given $R$ a composition series, so $R$ has finite length, so $R$ is Noetherian.
\end{proof}
\begin{remark}[Miles]
	Here is an alternate finish after \autoref{lem:jacobsonnilpotent}. The point is to extend the unfinished composition series
	\[R\supseteq\mf m_1\supseteq\mf m_1\cap\mf m_2\supseteq\cdots\supseteq J\]
	by $J\supseteq J^2\supseteq\cdots\supseteq J^N=(0)$. Namely, it remains to check that $J^k/J^{k+1}$ has finite length. Well, we use \autoref{prop:tensorquotient} and \autoref{rem:expandrmodj} to write
	\[\frac{J^k}{J^{k+1}}=\frac{\left(J^k\right)}{J\left(J^k\right)}\cong J^k\otimes_R\frac RJ\cong J^k\otimes_R\bigoplus_\mf mR/\mf m\cong\bigoplus_\mf m\left(J^k\otimes_RR/\mf m\right).\]
	So to finish, we need to show $J^k\otimes_RR/\mf m$ has finite length, for which it suffices to show $\mf m\otimes_RR/\mf m$ has finite length. But by \autoref{prop:tensorquotient}, $\mf m\otimes_RR/\mf m\cong\mf m/\mf m^2$, which is a finite-dimensional $R/\mf m$-vector space (when $R$ is Artinian!) as discussed at the end of the above proof.
\end{remark}

\subsection{Geometry of Artinian Rings}
While we're here, we provide some more nice facts.
\begin{proposition} \label{prop:artiniandecomp}
	Any Artinian ring is a product of finitely many local Artinian rings. In fact, if $R$ is an Artinian ring, then the natural map
	\[R\to\bigoplus_{\mf p\in\op{Spec}R}R_\mf p\]
	is an isomorphism.
\end{proposition}
\begin{proof}
	Note Artinian rings $R$ are both Artinian and Noetherian by \autoref{thm:artingrabbag}, so $R$ is an $R$-module of finite length by \autoref{lem:artnoethimpliesfinlength}. Thus, we can use \autoref{thm:finitelengthdecomp} to see that the natural map
	\[\varphi\colon R\to\bigoplus_{\mf p\in\op{Supp}R}R_\mf p\]
	is an isomorphism of $R$-modules. Note $\op{Supp}R=\op{Spec}R$ is finite by \autoref{thm:artingrabbag}, so we are just dealing with a summation of finitely many localization maps, which we know to be individually ring maps, so they also sum to a ring map. Thus, $\varphi$ is in fact an isomorphism of rings.
\end{proof}
We can even give a geometric view to what we are doing.
\begin{proposition} \label{prop:artinianisfinite}
	Fix $I\subseteq k[x_1,\ldots,x_n]$. Then the following are equivalent.
	\begin{listalph}
		\item The ring $R:=k[x_1,\ldots,x_n]/I$ is Artinian.
		\item The set $Z(I)\subseteq\AA^n(k)$ is finite.
		\item The ring $R$ is a finite-dimensional $k$-algebra.
	\end{listalph}
\end{proposition}
\begin{proof}
	We follow Eisenbud. We take our implications in sequence.
	\begin{itemize}
		\item We show (a) implies (b). Suppose that the ring $R:=k[x_1,\ldots,x_n]/I$ is Artinian. Then $R$ has finitely many maximal ideals by \autoref{lem:artinianfinmaxes}, which are in bijection to points in $Z(I)$, so $Z(I)$ is finite.

		\item We show (b) implies (c). Suppose that $Z(I)$ is finite. Then $R=k[x_1,\ldots,x_n]/I$ is in bijection with $k$-valued (polynomial) functions on $Z(I)$, but as $Z(I)$ is finite, we can build any function as a polynomial function by (say) Lagrange interpolation.

		Rigorously, two polynomials $f,g\in k[x_1,\ldots,x_n]$ has $[f]_I=[g]_I$ if and only if $[f-g]_I=[0]_I$ if and only if $f-g\in I$ if and only if $f-g$ vanishes on $Z(I)$ if and only if $f$ and $g$ agree on $Z(I)$. So $[f]_I$ can indeed be viewed as a function on $Z(I)$.
		
		Thus, $R$ is in bijection with $k$-valued functions on finitely many points, but this space is simply $k^{Z(I)}$, which is a finite-dimensional vector space. Adding in the ring structure to $R$ makes $R$ into a finite-dimensional $k$-algebra.

		\item We show (c) implies (a). Indeed, any strictly descending chain of submodules of a finite-dimensional $k$-vector space must terminate, so $R$ is Artinian as a $k$-vector space. Any $R$-submodule of $R$ will also be a $k$-vector space, so we see that any strictly descending chain of $R$-submodules of $R$ must terminate as well. Thus, $R$ is Artinian.
		\qedhere
	\end{itemize}
\end{proof}

\subsection{The Radical, Returned}
And we end our discussion with the following miscellaneous result.
\begin{proposition} \label{prop:radprimes}
	Fix an ideal $I\subseteq R$. Then
	\[\op{rad}I=\bigcap_{I\subseteq\mf p}\mf p,\]
	where $\mf p$ ranges over all prime ideals containing $I$.
\end{proposition}
\begin{proof}
	The main point is the following lemma.
	\begin{lemma}
		Fix $R$ a ring and $I\subseteq R$ an ideal and $U\subseteq R$ a multiplicatively closed subset such that $I\cap U=\emp$. Suppose $\mf p$ is maximal in the set of ideals satisfying $\mf p\cap U=\emp$ and $I\subseteq\mf p$. Then $\mf p$ is prime.
	\end{lemma}
	Before proving the lemma, we note that, under the hypotheses of the problem, such a maximal ideal $\mf p$ will exist, which we can conjure by Zorn's lemma from the set of all ideals satisfying $\mf p\cap U=\emp$ and $I\subseteq\mf p$.\footnote{This set is nonempty because $I\cap U=\emp$ and $I\subseteq I$. All ascending chains have an upper bound by taking the union along the chain.}
	\begin{proof}
		Suppose that $a,b\notin\mf p$, and it suffices to show that $ab\notin\mf p$. Well, $(a)+\mf p$ and $(b)+\mf p$ are both strictly larger than $\mf p$ while containing $I\subseteq\mf p$, so they must intersect $U$. Suppose $u\in((a)+\mf p)\cap U$ and $v\in((b)+\mf p)\cap U$. Then
		\[uv\in((a)+\mf p)((b)+\mf p)=(ab)+(a)\mf p+(b)\mf p+\mf p^2\subseteq(ab)+\mf p.\]
		Thus, $(ab)+\mf p$ intersects $U$ at $uv\in U$, so it follows $\mf p\ne(ab)+\mf p$ because $\mf p\cap U=\emp$. Thus, $ab\notin\mf p$, finishing.
	\end{proof}
	We now attack the proposition directly. In one direction, suppose that $a\in\rad I$ so that $a^n\in I$ for some $n\in\NN$. Then for any prime $\mf p$ containing $I$, we have $a^n\in\mf p$, so $a\in\mf p$ by primality of $\mf p$. It follows
	\[\rad I\subseteq\bigcap_{I\subseteq\mf p}\mf p.\]
	The other inclusion requires the lemma. Suppose that $a\notin\rad I$, and we will find a prime $\mf p\supseteq I$ such that $a\notin\mf p$. Indeed, we pick up an ideal $\mf p$ containing $I$ which is maximal avoiding the set
	\[\langle a\rangle:=\{a^n:n\in\NN\}.\]
	In particular, such an ideal $\mf p$ by the discussion preceding the lemma, and it is prime by the lemma. But $a\notin\mf p$ while $I\subseteq\mf p$, so it follows that
	\[a\notin\bigcap_{I\subseteq\mf p}\mf p,\]
	finishing.
\end{proof}
\begin{corollary}
	Fix $R$ a ring. Then $r\in R$ is nilpotent if and only if $r\in\mf p$ for each prime ideal $\mf p\subseteq R$.
\end{corollary}
\begin{proof}
	The set of nilpotent elements in $R$ is
	\[\op{rad}(0)=\{r\in R:r^n=0\text{ for some }n\in\NN\}.\]
	By \autoref{prop:radprimes}, this will be the intersection of all prime ideals of $R$. In other words, an element $r\in R$ is nilpotent if and only if $r\in\mf p$ for all primes $\mf p$, which is what we wanted.
\end{proof}