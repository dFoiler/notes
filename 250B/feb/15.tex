% !TEX root = ../notes.tex

Here we go.
\begin{convention}
	For today's lecture, an $S$-algebra $R$ should be thought of as providing an embedding $R\subseteq S$ (though we will not actually assume that this ring map is injective).
\end{convention}

\subsection{More on Integrality}
Last time we introduced the following proposition.
\integralitydef*
\noindent This gave rise to the following definitinon.
\integraldefi*
\noindent Being integral is intended to be a generalization of having a finite extension of fields. Along these lines, we get the following definition.
\finitedefi*
\noindent As with fields, we know that any finite field extension must be algebraic, so we might hope that an integral extension is also finite.
\begin{lemma}
	Every finite $R$-algebra $S$ is integral.
\end{lemma}
\begin{proof}
	We use the Cayley--Hamilton theorem. Namely, take our endomorphism $\varphi$ to be multiplication by one of the generators of $S$ as an $R$-module and then stitch these together.
\end{proof}
In fact, we can provide a converse.
\begin{lemma} \label{lem:betterfinite}
	Fix $S$ an $R$-algebra. Then $S$ is finite if and only if it is finitely generated as an $R$-algebra, where the generators are integral.
\end{lemma}
\begin{proof}
	We have two directions.
	\begin{itemize}
		\item In one direction, suppose that $S=R[s_1,\ldots,s_n]$, and we consider the chain
		\[R\subseteq R[s_1]\subseteq R[s_1,s_2]\subseteq\cdots\subseteq R[s_1,\ldots,s_n].\]
		Each extension is finite because the generators are integral, and we can build a finite set of generators by multiplying the sets together.
		\item In the other direction, take $S$ a finite $R$-algebra. Then all elements are integral over $R$, but we are only permitted finitely many generators, so we can just keep choosing until we are done.
		\qedhere
	\end{itemize}
\end{proof}
Sometimes we aren't integral, but we can always make one.
\begin{definition}[Integral closure]
	Fix $S$ an $R$-algebra. Then the \textit{integral closure} $S'$ is the set of all elements of $S$ which are integral over $R$.
\end{definition}
\begin{remark}
	The integral closure depends on the choie of $S$.
\end{remark}
\begin{proposition}
	Fix $S$ an $R$-algebra. Then the integral closure of $S$ is an $R$-subalgebra of $S$.
\end{proposition}
\begin{proof}
	The main idea is to use \autoref{lem:betterfinite}. We emulate the proof that the set of algebraic elements is a field extension. Namely, for any elements $s_1$ and $s_2$ which are integral over $R$, \autoref{lem:betterfinite} tells us that
	\[R[s_1,s_2]\]
	is a finite $R$-algebra, so all of its elements are integral. Thus, $s_1s_1$ and $s_1+s_2$ are integral, showing that $S$ is closed under addition and multiplication. We are also closed under the $R$-action because elements $r\in R$ in $S$ are integral by the monic polynomial $(x-r)\in R[x]$.
\end{proof}
We close our discussion by quickly discussing localization: localization commutes with the integral closure.
\begin{proposition}
	Fix $S$ an $R$-algebra with integral closure $S'$; further take $U\subseteq R$ a multiplicative subset. Then $S'\left[U^{-1}\right]$ is the integral closure of $R\left[U^{-1}\right]$ in $S\left[U^{-1}\right]$.
\end{proposition}
\begin{proof}
	The directino that all elements of $S'\left[U^{-1}\right]$ are integral over $R\left[U^{-1}\right]$ is not hard because multiplication by units will not affect integrality.
	
	In the other direction, fix some $\frac su\in S\left[U^{-1}\right]$ is integral over $R\left[U^{-1}\right]$ so that we hvae some polynomial
	\[\left(\frac su\right)^n+\frac{r_1}{u_1}\left(\frac su\right)^{n-1}+\cdots+\frac{r_n}{u_n}=0.\]
	Multiplying through by $s(u_1\cdots u_nu)^n$ will show that $s(u_1\cdots u_n)$ is integral over $R$ and hence lives in $S'$, which finishes.
\end{proof}

\subsection{Normality}
We have the following defintions.
\begin{definition}[Normal]
	Fix $R$ a domain with field of fractions $K(R)$. Then $R$ is \textit{normal} if and only if $R$ is integrally closed in $K(R)$.
\end{definition}
\begin{definition}[Normalization]
	Fix $R$ a domain with field of fractions $K(R)$. We can define the \textit{normalization} of $R$ to be the integral closure of $R$ in $K(R)$.
\end{definition}
Let's see some examples.
\begin{exe} \label{exe:zisnormal}
	Consider $R=\ZZ$ with $K(R)=\QQ$. Then we show that the integral closure of $\ZZ$ is $\ZZ$. In particular, $\ZZ$ is normal.
\end{exe}
\begin{proof}
	Of course elements of $\ZZ$ are integral over $\ZZ$. Suppose that $\frac pq\in\QQ$ is integral; without loss of generality, we may assume $\gcd(p,q)=1$. Now, we are promised some monic polynomial such that
	\[(p/q)^n+a_1(p/q)^{n-1}+\cdots+a_n=0\]
	so that all of the coefficients are in $\ZZ$. However, multiplying by $q^n$, we see that
	\[p^n=-\left(a_1p^{n-1}q+\cdots+a_nq^n\right).\]
	In particular, $q$ divides the right-hand side, so $q$ divides $p^n$, so $1=\gcd(p^n,q)=|q|$. In particular, 
\end{proof}
Essentially the same proof will work for any unique factorization domain.
\begin{proposition}
	Any unique factorization domain is normal.
\end{proposition}
\begin{proof}
	Copy the proof of \autoref{exe:zisnormal}.
\end{proof}
And here are more examples.
\begin{example}
	The ring $\ZZ[i]$ is normal and hence integrally closed in $\QQ(i)$.
\end{example}
\begin{nex}
	The ring $\ZZ\left[\sqrt5\right]$ is not normal. Note that the field of fractions is $\QQ(\sqrt5)$, so we note $\frac{1+\sqrt5}2\in\QQ(\sqrt5)$ is the root of the polynomial
	\[x^2-x-1\]
	by the quadratic formula. However, the integral closure is $\ZZ\left[\frac{1+\sqrt5}2\right]$, so this is essentially the only exception.
\end{nex}
\begin{example}
	The integral closure $\overline\ZZ$ of $\ZZ$ in $\CC$ is the ring of all the roots of monic polynomials; these are called the algebraic integers. For example, $\overline\ZZ\subseteq\overline\QQ$.
\end{example}

\subsection{Normality via Geometry}
There is also a context for normality in algebraic geometry.
\begin{exe}
	We compute the integral closure of the ring $R=k[x,y]/\left(y^2-x^3\right)$.
\end{exe}
\begin{proof}
	Here is our image.
	\begin{center}
		\begin{asy}
			unitsize(1cm);
			import graph;
			real y(real t)
			{
				return t;
			}
			real x(real t)
			{
				return cbrt(t*t);
			}
			draw(graph(x,y,-2,2));
		\end{asy}
	\end{center}
	Note that, working in the fraction field, $\left(\frac yx\right)^2=x$ because $y^2=x^3$, so $R$ is not normal because it does not include $\frac yx$.

	To compute our integral closure, we create a map $R\to k[t]$ by $y\mapsto t^3$ and $x\mapsto t^2$ (so that $t=y/x$), and we find $R$ embeds into $k[t]$. But because $k[t]$ is now integrally closed (it's a unique factorization domain), we see that the pull-back $R[y/x]$ will in fact be integrally closed, so this is our integral closure.
\end{proof}
\begin{example}
	Consider the ring $R=k[x,y]/\left(y^2-x^2(x+1)\right)$. Then $\left(\frac yx\right)^2=x+1$, so $R$ is not normal because it does not include $\frac yx$.
\end{example}
More generally, suppose that we have affine algebraic sets $X$ and $Y$ with an embedding $A(X)\to A(Y)$. This corresponds to a map $Y\to X$. Normality then means that the image of $Y$ in $X$ is ``Zariski dense'' so that there is no proper closed subset of $X$ which contains $Y$.

Speaking with more geometry, a map $Y\to X$ of affine varieties is proper (over $\CC$, say) essentially gives us the result that the pre-image of a compact set is compact.
\begin{remark}
	I did not follow the above discussin.
\end{remark}
We have the following proposition.
\begin{proposition}
	Fix $S$ an $R$-algebra with a monic polynomial $f\in R[x]$. If we can factor $f=gh$ for $g,h\in S[x]$. Then the coefficients of $g$ and $h$ are integral over $R$.
\end{proposition}
\begin{proof}
	Imagine adding some root $\alpha_1$ of $g$ to $S$ to get a bigger $R$-algebra named $R[\alpha_1]$. So, writing $g(x)=(x-\alpha_1)g_1(x)$, we see that we can divide out to get
	\[\frac{f(x)}{(x-\alpha_1)}=g_1(x)h(x).\]
	Inductively removing all roots $\alpha_1,\ldots,\alpha_m$ of $g$ and $\beta_1,\ldots,\beta_n$ of $h$, we see that
	\[f(x)=(x-\alpha_1)\cdots(x-\alpha_m)(x-\beta_1)\cdots(x-\beta_n).\]
	Here the leading coefficients match, so we do not inherit a leading term. However, upon expansion, we see that the coefficients of $g$ and $h$ will be elementary symmetric functions of the $\alpha_\bullet$ and $\beta_\bullet$, so in particular they will all be contained in the finite extension $R[\alpha_1,\ldots,\alpha_m,\beta_1,\ldots,\beta_n]$ and hence be integral.
\end{proof}
\begin{corollary}
	Fix $R$ a normal domain and $f(x)\in R[x]$ some monic polynomial. Then, if $f(x)$ is irreducible, then $f(x)$ is prime.
\end{corollary}
\begin{proof}
	Fix $f(x)\in R[x]$. Then $f$ will remain irreducible in $K(R)$, which comes from the above proposition. In particular, we are promised an embedding
	\[\frac{R[x]}{(f(x))}\into\frac{K[x]}{(f(x))},\]
	so $R[x]/(f(x))$ is a subring of a field and hence an integral domain.
\end{proof}
\begin{remark}
	This generalizes the result that, if $R$ is a unique factorization domain, then $R[x]$ is also a unique factorization domain.
\end{remark}

\subsection{Lifting Primes}
Speaking generally for a moment, suppose we have an $S$-algebra $R$. Then, if $\varphi:R\to S$ is our promised map, we note that we have a map $\op{Spec}S\to\op{Spec}R$ by $\varphi^{-1}$. In particular, when $\varphi^{-1}$ is an embeding $R\subseteq S$, we get that primes $\mf q$ of $S$ go to $\mf q\cap R$.

When we have integral extensions, we get some more control.
\begin{proposition}
	Fix $R\subseteq S$ an integral extension of rings. For any $\mf p\in\op{Spec}R$, there exists $\mf q\in\op{Spec}S$ such that $\mf q\cap R=\mf p$.
\end{proposition}
\begin{proof}
	Set $U:=R\setminus\mf p$, and we will localize at $U$. Because localization preserves embeddings, we get an embedding $R_\mf p=R\left[U^{-1}\right]\subseteq S\left[U^{-1}\right]$. It will suffice to show the statement for the localization because then we can pre-image back to the original statement.

	Now, by how primes work in localization, we know that
	\[\mf pS\left[U^{-1}\right]\cap R_\mf p=\mf p.\]
	Thus, because $\mf p$ is the unique maximal ideal of $R_\mf p$, it suffices to put $\mf pS\left[U^{-1}\right]$ in any larger ideal and then pull-back, as long as we don't get the full ring $R\left[U^{-1}\right]$.

	Well, any maximal ideal containing $\mf pS\left[U^{-1}\right]$ will do, so we have to show $\mf pS\left[U^{-1}\right]\cap R_\mf p=R_\mf p$. Well, suppose for the sake of contradiction this is true so that
	\[1=p_1s_1+\cdots+p_ns_n\]
	for some $p_1,\ldots,p_n\in\mf p$ and $s_1,\ldots,s_n\in S$. But then $M=R[s_1,\ldots,s_n]$ is a finitely generated $R$-module (by integrality) where $\mf pM=M$ (because of the above equation), which forces $M=0$ by Nakayama's lemma, which is a contradiction.
\end{proof}
In fact, we have the following.
\begin{corollary}
	Fix $R\subseteq S$ an integral extension. Further, if $I\subseteq R$ is an ideal with $SI\subseteq R\subseteq\mf p$ for some $\mf p\in\op{Spec}R$, then we can choose $\mf q$ with $\mf q\cap R=\mf p$ which contains $I$.
\end{corollary}
\begin{proof}
	One can work in the integral extension $R/I\subseteq S/SI$ and then use the previous proposition.
\end{proof}
In the case of domains, we have some communication with the field extensions. 
\begin{lemma}
	Fix $R\subseteq S$ an integral extension of domains. Then $K(S)$ is algebraic over $K(R)$.
\end{lemma}
\begin{proof}
	This follows from simply choosing finitely many integral generators of $S$ over $R$.
\end{proof}
This gives us the following lack of ``avoidance'' in integral domains.
\begin{prop}
	Fix $R\subseteq S$ an integral extension of domains and $I\ne0$ a nonzero ideal of $S$. Then $I\cap R\ne0$.
\end{prop}
\begin{proof}
	Suppose $b\in I$. By writing out the polynomial for $b$ over $K(R)$ and then multiplying out by all the denominators, we get some equation in $R$ of the form
	\[a_nb^n+\cdots+a_0=0.\]
	By forcing $n$ minimal, we get $a_0\ne0$ (here we use that these are domains), but then $a_0\in Sb\subseteq I$ as well as $a_0\in R$. This finishes.
\end{proof}
\begin{proposition}
	Fix $R\subseteq S$ an extension of integral domains. Then, $R$ is a field if and only if $S$ is a field.
\end{proposition}
\begin{proof}
	In one direction, if $R$ is a field, then take any $s\in S$ and write out its equation
	\[s^n+a_1s^{n-1}+\cdots+a_0=0.\]
	Again, we can force $a_0\ne0$, so $a_0\in R$ is a unit. By factoring out $s$ from the first $n$ terms, we get $s(\text{stuff})=-a_0\in R^\times$, so $s$ is a unit.

	In the oother direction, suppose for the sake of contradiction that $S$ is a field while $S$ is not. Then $R$ has some nonzero maximal ideal $\mf p$ which lifts to a nonzero maximal ideal $\mf P$ up in $S$. But the only ideals of $S$ are $(0)$ or $S$, neither of which can be the lift of $\mf P$.
\end{proof}