% !TEX root = ../notes.tex

Here we go.

\subsection{Unique Factorization Domains}

We start with the following result; it is due to Gauss.
\begin{theorem} \label{thm:ch}
	Fix $R$ a unique factorization domain. Then $R[x]$ is a unique factorization domain.
\end{theorem}
\begin{proof}
	The main character is as follows.
	\begin{definition}[Content]
		Fix $f(x)=a_0+\cdots+a_nx^n\in R[x]$. Then we define the \textit{content} of $f$ to be the ideal
		\[\op{cont}(f):=(a_0,\ldots,a_n)\subseteq R.\]
	\end{definition}
	Here is the main claim.
	\begin{lemma}
		Fix $f,g\in R[x]$. Then $\op{cont}(fg)\subseteq\op{cont}(f)\op{cont}(g)\subseteq\rad\op{cont}(fg)$.
	\end{lemma}
	\begin{proof}
		That $\op{cont}(fg)\subseteq\op{cont}(f)\op{cont}(g)$ follows from expanding out $fg$. To show the other direction, we recall from \autoref{prop:radprimes} that
		\[\rad\op{cont}(fg)=\bigcap_{\op{cont}(fg)\subseteq\mf p}\mf p.\]
		Now, for any $\mf p$ containing $\op{cont}(fg)$, we have to show that $\op{cont}(f)\op{cont}(g)\subseteq\mf p$. But then we note that $\overline f,\overline g\in(R/\mf p)[x]$ will have
		\[\overline f\cdot\overline g=0\]
		by definition of $\mf p$. Thus, either $\overline f\in\mf p$ or $\overline g\in\mf p$, so $\op{cont}(f)\subseteq\mf p$ or $\op{cont}(g)\subseteq\mf p$. This finishes the other inclusion.
	\end{proof}
	The key to continue is to work with the field of fractions $K:=\op{Frac}(R)$, noting that any polynomial $f\in R[x]$ will have a unique factorization in $K[x]$ (because, say, $K[x]$ is Euclidean).

	Namely, our next step is to classify irreducibles in $R[x]$.
	\begin{lemma}
		Fix everything as above. Then $f$ is irreducible in $R[x]$ implies that $f$ is irreducible in $K[x]$.
	\end{lemma}
	\begin{proof}
		We proceed by contraposition. If $f(x)$ is not irreducible (but still not zero and not a unit), then we can write $f=g_0h_0$ where $\deg g_0,\deg h_0<\deg f$. Clearing denominators, we can write
		\[rf=gh\]
		where $g,h\in R[x]$. So now we can talk about the content. If $r$ is a unit, then we see that $f$ is indeed not reducible. If $r$ is not a unit, then for each prime $p\mid f$, we see that $\op{cont}(f)\op{cont}(g)\subseteq\rad\op{cont}(fg)\subseteq(p)$, so either $g\in(p)$ or $h\in(p)$, and then we can cancel out by $p$. In particular, we can, after enough cancelling, force $r$ to be a unit, thus finishing.
	\end{proof}
	To finish checking that $R[x]$ is a unique factorization domain, we recall that it suffices to show all irreducibles are prime.
	\begin{lemma}
		Fix everything as above. If $f$ is irreducible, in $R[x]$, then $f$ is prime.
	\end{lemma}
	\begin{proof}
		Well, if $f$ is irreducible in $R[x]$, then it is irreducible and hence prime in $K[x]$. In particular, if $f\mid gh$ for $g,h\in R[x]$, then without loss of generality
		\[g=f(q/r)\]
		for some $q/r\in K[x]$ so that $q\in K[x]$. In particular, $fq=rg$, so arguing as above we can remove the primes dividing $r$ into $q$ because $f$ will not be divisible by any prime constant.
	\end{proof}
	The above lemma finishes the proof.
\end{proof}
\begin{corollary}
	The ring $k[x_1,\ldots,x_n]$ is a unique factorization domain.
\end{corollary}
\begin{proof}
	Induction on $n$.
\end{proof}
\begin{example}
	We show that $\left(y^2-x^3\right)\subseteq k[x,y]$ is prime. It suffices to show that $y^2-x^3$ is prime in $k[x,y]$, for which it suffices to show that $y^2-x^3$ is irreducible in $k(x)[y]$. But $y^2-x^3$ is a quadratic in $k(x)[y]$ and therefore irreducible because it has no roots: there is no $y=f(x)/g(x)$ such that $f(x)^2/g(x)^2=x^3$.
\end{example}
\begin{example}
	We show that $\left(y^2-x^3\right)\subseteq k[x,y]$ is prime a different way. Indeed, by sending $x\mapsto t^2$ and $y\mapsto t^3$, there is an embedding
	\[\frac{k[x,y]}{\left(y^2-x^3\right)}\into k\left[t^2,t^3\right]\]
	by a homework problem. So the quotient is a domain, so $\left(y^2-x^3\right)$ is prime.
\end{example}

\subsection{The Cayley--Hamilton Theorem}
Here is the main result we are going to prove.
\begin{theorem}
	Fix $R$ a ring and $A\in R^{n\times n}$ a matrix. Then define $p_A(x):=\det(xI-A)\in R[x]$. Then $p_A(A)=0\in R^{n\times n}$.
\end{theorem}
This statement is usually stated in linear algebra over a field, but it should hold for arbitrary rings.
\begin{proof}
	We need to pick up the following definition.
	\begin{definition}[Adjugate matrix]
		Fix $A\in R^{n\times n}$. Then we define the \textit{adjugate matrix} by
		\[C_{ij}:=(-1)^{ij}\det A_{i,j},\]
		where $A_{i,j}$ is the matrix $A$ without the $i$th row and without the $j$th column.
	\end{definition}
	\begin{example}
		Set
		\[A=\begin{cases}
			a_{11} & a_{12} \\
			a_{21} & a_{22}
		\end{cases}.\]
		Then
		\[C=\begin{cases}
			a_{22} & -a_{21} \\
			-a_{12} & a_{11}
		\end{cases}.\]
		We can verify by hand that $C^\intercal A=(\det A)I$.
	\end{example}
	The central idea behind the adjugate matrix is that it ``almost inverts'' $A$.
	\begin{lemma}
		Fix $A\in R^{n\times n}$ with adjugate matrix $C$. Then $C^\intercal A=(\det A)I$.
	\end{lemma}
	\begin{proof}
		Omitted.
	\end{proof}
	To show the result, we define
	\[A(x):=xI-A,\]
	which vanishes upon taking $x=A$. Then we see that
	\[C^\intercal(x)A(x)=\det(x1-A)I,\]
	which still vanishes upon taking $x=A$.
\end{proof}
Our application to commutative algebra is as follows.
\begin{theorem} \label{thm:betterch}
	Fix $M$ a finitely generated $R$-module with $n$ generators. Further, fix $\varphi\in\op{End}_R(M)$. Then there exists some monic polynomial
	\[p_\varphi=x^n+p_1x^{n-1}+\cdots+p_n\]
	of degree $n$ such that $p_\varphi(\varphi)$ is zero. In fact, if there is an ideal $I\subseteq R$ such that $IM=M$, then $p_k\in I^k$.
\end{theorem}
\begin{proof}
	Let $\{m_1,\ldots,m_n\}$ generate $M$ so that we can use the equations
	\[\varphi(m_i)=\sum_{k=1}^na_{ij}m_j\]
	to give a matrix form for $\varphi$. Then we set $p_\varphi$ to be the characteristic polynomial for $\varphi$, which finishes the first part by the Cayley--Hamilton theorem. For the second part, we note that $IM=M$ can force that $a_{ij}\in I$ for each $i,j$, which upon writing out the coefficients for the characteristic polynomial will finish.
\end{proof}
Let's now see some applications.
\begin{proposition} \label{prop:epiisiso}
	Let $M$ be a finitely generated $R$-module and $\psi\in\op{End}_R(M)$. Then if $\psi$ is surjective, then $\psi$ is an isomorphism.
\end{proposition}
\begin{proof}
	The key trick is to give $M$ an $R[t]$-module structure to $M$ by defining $R[t]\mapsto\op{End}_RM$ by sending $t\mapsto\psi$.
	In particular, using the above, we get some $p_{\id}$ such that $p_{\id}({\id})=0$.
	Further, because $\psi$ is surjective, we see that $(t)\cdot M=M$, so when we write out
	\[p_{\id}(x)=x^n+p_1x^{n-1}+\cdots+p_n,\]
	we see that $p_i\in\left(t^i\right)$ for each $i$. In particular, plugging in $x=\id$, we see that
	\[0=\id+t\cdot q(t)\]
	for some $q\in R[t]$. In particular, $t$ is invertible with inverse $q(t)$.
\end{proof}
\begin{remark}
	This need not be true even in vector spaces which are not finitely generated. For example, consider
	\[V:=\bigoplus_{i=1}^\infty kv_i\]
	for some vectors $\{v_i\}_{i=1}^\infty$. Then we have the surjective map defined by $v_1\mapsto0$ and $v_i\mapsto v_{i-1}$ for $i>1$, and this is not an isomorphism.
\end{remark}
\begin{remark}
	This definitely need not be true for injections giving isomorphisms. For example, $\ZZ\to2\ZZ\into\ZZ$ is injective but not an isomorphism.
\end{remark}
\begin{corollary}
	Fix $m$ and $n$ positive integers. Then if $R^n\cong R^m$ is an isomorphism of $R$-modules. Then $m=n$.
\end{corollary}
\begin{proof}
	Without loss of generality take $n\ge m$. Then we can construct a surjective map
	\[R^m\cong R^n\onto R^m,\]
	which must be an isomorphism by \autoref{prop:epiisiso}. However, the map $R^n\onto R^m$ by projection has a kernel whenever $n>m$, so the composite would have a kernel, which is our contradiction. So we must have $n=m$.
\end{proof}

\subsection{Nakayama's Lemma}
Recall the definition.
\begin{definition}[Jacobson radical]
	Fix a ring $R$. Then we define the \textit{Jacobson radical} by
	\[\rad R:=\bigcap_{\mf m}\mf m.\]
\end{definition}
Observe that $r\in\rad R$ if and only if $1-r\in R^\times$. In particular, $r\in\rad R$ if and only if $1-r$ is not in any maximal ideal if and only if $(1-r)=R$.

We have the following result.
\begin{theorem}[Nakayama's lemma]
	Fix $I\subseteq\op{rad}R$ and $M$ a finitely generated $R$-module. Then if $IM=M$, we have $M=0$.
\end{theorem}
\begin{proof}
	The main idea is in the following lemma.
	\begin{lemma}
		Fix everything as above. Then $IM=M$ implies that there is some $r\in I$ such that $(1-r)M=0$.
	\end{lemma}
	\begin{proof}
		The idea is, as usual, to use \autoref{thm:betterch}. We are promised some polynomial
		\[p_{\id}(x):=x^n+p_1x^{n-1}+\cdots+p_n,\]
		where $p_k\in I^k$. But plugging in $x=\id$ gives the result after rearranging for $\id^n=\id$.
	\end{proof}
	From this lemma the result directly follows because the promised $1-r$ is a unit.
\end{proof}
\begin{corollary}
	Fix $I\subseteq\op{rad}R$ and $M$ a finitely generated $R$-module with elements $m_1,\ldots,m_n\in M$. Then if the images $\overline{m_1},\ldots,\overline{m_n}$ generate $M/IM$, then the original elements generate $M$.
\end{corollary}
\begin{proof}
	Consider
	\[M':=Rm_1+\cdots+Rm_n.\]
	Then because the given elements generate $M/IM$, we note that $M/M'=I(M/M')$, so $M/M'=0$, so $M=M'$.
\end{proof}

Here is an application, to localization.
\begin{proposition} \label{prop:localtensorintdomain}
	Fix $R$ a local ring with $M$ and $N$ finitely generated $R$-modules. Then $M\otimes_RN=0$ if and only if $M=0$ or $N=0$.
\end{proposition}
\begin{proof}
	If $M=0$ or $N=0$, then of course $M\otimes_RN=0$.
	
	In the reverse direction, suppose $M\ne0$. Fix $\mf m$ the maximal ideal. Then, because $M$ is finitely generated by some elements, we get a surjective map $M\onto R/\mf m$. Tensoring, we see that
	\[M\otimes_RN\to(R/\mf m)\otimes_RN\to0\]
	is also surjective, but then $R/\mf m\otimes_RN=N/\mf mN$. However, $M\otimes_RN=0$, so $N/\mf mN=0$ for all maximal ideals, so $N=0$.
\end{proof}
\begin{corollary}
	Fix $M$ and $N$ finitely generated $R$-modules. Then $M\otimes_RN=0$ if and only if $\op{Ann}M+\op{Ann}N=R$.
\end{corollary}
\begin{proof}
	If $\op{Ann}M+\op{Ann}N=R$, then $M\otimes_RN=0$ by decomposing $1=a+b$ where $a\in\op{Ann}M$ and $b\in\op{Ann}N$.

	In the other direction, suppose that $I:=\op{Ann}M+\op{Ann}N\subsetneq R$. Putting $I$ in some maximal ideal $\mf m$, we note that we can localize to $R_\mf m$ and then reduce to \autoref{prop:localtensorintdomain}.
\end{proof}
\begin{corollary}
	Fix $M$ and $N$ finitely generated $R$-modules. Then $\op{Supp}(M\otimes_RN)=\op{Supp}M\cap\op{Supp}N$.
\end{corollary}
\begin{proof}
	Fix a prime $\mf p$. Then we see that
	\[(M\otimes_RN)_\mf p=M_\mf p\otimes_{R_\mf p}N_\mf p\]
	will vanish if and only if $M_\mf p=0$ or $N_\mf p=0$, which is precisely the statement after negation.
\end{proof}
We note that the condition that $M$ and $N$ are finitely generated is crucial.
\begin{nex}
	We note that $\QQ\otimes_\ZZ\ZZ/n\ZZ=0$ because $\QQ$ is divisible and $\ZZ/n\ZZ$ is torsion. But $\QQ\ne0$ and $\ZZ/n\ZZ\ne0$.
\end{nex}

\subsection{Integrality Preview}
We will spend the rest of class on the following theorem.
\begin{prop}
	Fix $R$ a ring and an $R$-algebra $S:=R[s]/I$ for some ideal $I$. We have the following.
	\begin{listalph}
		\item $S$ is finitely generated as an $R$-module if and only if $I$ contains a monic polynomial (i.e., there is some monic $p(x)\in R[x]$ such that $p(s)=0$).
		\item $S$ is a free, finitely generated $R$-module if $I=(p)$ for some monic polynomial $p$.
	\end{listalph}
\end{prop}
\begin{proof}
	Here we go.
	\begin{listalph}
		\item If $S$ is finitely generated as an $R$-module, we apply \autoref{thm:betterch} with $\varphi=\mu_s:m\mapsto sm$ to finish.

		In the other direction, suppose
		\[p(x):=x^n+p_1x^{n-1}+\cdots+p_n\]
		is a polynomial in $I$. Then $\{1,s,s^2,\ldots,s^{n-1}\}$ will generate $S$ over $R$: indeed, it suffices to check that each $s^k$ can be written as an $R$-linear combination of the $\{1,s,\ldots,s^{n-1}\}$, but we get this by induction after noting $p$ promises
		\[s^{n+\ell}=-\sum_{i=0}^{n-1}p_{n-i}s^{i+\ell}.\]

		\item If $S$ is a free, finitely generated $R$-module, we got some monic polynomial in $I$, so we find the nonzero polynomial of least degree. Because the generation is free, using a power basis means we can force this polynomial to be monic, which gives the result.

		The other direction is similar to the other direction above.
		\qedhere
	\end{listalph}
\end{proof}
We close with some definitions.
\begin{definition}[Finite]
	Fix $S$ an $R$-algebra. Then $S$ is \textit{finite} over $R$ if and only if $S$ is finitely geneated over $R$.
\end{definition}
\begin{definition}[Integral]
	Fix $S$ an $R$-algebra. Then $s\in S$ is \textit{integral over $R$} if and only if $s$ is a root of some monic polynomial over $R$. If all elements $s\in S$ are integral over $R$, then we say $S$ is \textit{integral over $R$}.
\end{definition}