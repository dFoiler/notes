% !TEX root = ../notes.tex

So it's the day after death.

\subsection{Midterm Review}
Let's start talking about the second problem on the midterm. Note that $X^2=0$ for a matrix $X\in\CC^{2\times2}$ if and only if the characteristic polynomial is $X^2$ if and only if
\[\det X=\tr X=0,\]
so the set of matrices with $X^2=0$ is generated by these two conditions. To see that these make a principal ideal, we check that
\[\frac{\CC[a,b,c,d]}{(a+d,ad-bc)}\cong\frac{\CC[a,b,c]}{(-d^2-bc)}\]
is an integral domain, which is not very hard.

\subsection{Filtration of Rings}
Today we are talking about the Artin--Rees lemma, which requires us talking about filtrations.
\begin{definition}[Filtration, rings]
	Fix $R$ a ring. Then a \textit{filtration} of $R$ is a sequence of ideals
	\[R=I_0= I_1\supseteq I_2\supseteq\cdots\]
	such that $I_iI_j\subseteq I_{i+j}$.
\end{definition}
\begin{example} \label{ex:graded}
	Fix $R=R_0\oplus R_1\oplus R_2\oplus\cdots$ a graded ring. Then we can set
	\[I_p:=\bigoplus_{i\ge p}R_i\]
	so that
	\[R=I_0\supseteq I_1\supseteq\cdots\]
	is a filtration.
\end{example}
\begin{defi}[\texorpdfstring{$I$}{I}-adic filtration]
	Fix $R$ a ring and $I\subseteq R$ an ideal. Then
	\[R\supseteq I\supseteq I^2\supseteq I^3\supseteq\cdots\]
	is a filtration. This is called the \textit{$I$-adic filtration}.
\end{defi}
\begin{example}
	More concretely, consider $R=k[x_1,\ldots,x_n]$ graded by degree. Then we set $I_m$ to be the union of $\{0\}$ the set of all polynomials with degree at least $m$. This manifests \autoref{ex:graded}, but it is also the $(x_1,\ldots,x_n)$-adic filtration.
\end{example}
Generally speaking, if we have a filtration
\[R=I_0\supseteq I_1\supseteq I_2\supseteq\cdots,\]
we might be interested in the ``bottom'' of this filtration
\[\bigcap_{i=0}^\infty I_i.\]
Surely this is an ideal, but it might not be $0$. Regardless, today we will be interested in the case where this is $0$.

\subsection{Associated Graded Rings}
Filtrations give rise to the following definition.
\begin{definition}[Associated graded ring]
	Fix a filtration $\mathcal J$ notated
	\[R= I_0\supseteq I_1\supseteq I_2\supseteq\cdots.\]
	Then we define $R_i:=I_i/I_{i+1}$ and define
	\[\op{gr}_{\mathcal J}R:=\bigoplus_{k\ge0}I_k/I_{k+1}\]
	to be the \textit{associated graded ring}. If $\mathcal J$ is the $I$-adic filtration, we denote this by $\op{gr}_I(R)$. If the filtration is obvious, we will omit the subscript entirely.
\end{definition}
A priori, the associated graded ring is only some very large module, but we can give it a ring structure as follows: if we have terms $[a]\in I_p/I_{p+1}$ and $[b]\in I_q/I_{q+1}$, then we can lift them to $a\in I_p$ and $b\in I_q$ so that $ab\in I_pI_q\subseteq I_{p+q}$, which is unique up to representative of $I_{p+q+1}$.

In particular, if $a\equiv a'\pmod{I_{p+1}}$ and $b\equiv b'\pmod{I_{q+1}}$, then
\[aa'\equiv bb'\pmod{I_{p+q+1}},\]
which is something we can check by hand by looking at $aa'+a(b'-b)+b'(a-a')-bb'$.
\begin{example}
	We work in $R:=k[[x]]$, which is local with maximal ideal $I:=(x)$. Then $I^n=\left(x^n\right)$ gives rise an $I$-adic filtration. We can compute
	\[I^n/I^{n+1}\cong\left\{ax^n:a\in k\right\}\cong kx^n\]
	because we are taking ($0$ or) a very long polynomial with minimal degree $x^n$ and then killing all higher degree terms. So our filtration reads as
	\[\op{gr}_IR=R/I\oplus I/I^2\oplus I^2/I^\oplus\cdots=k\oplus kx\oplus kx^2\oplus\cdots=k[x].\]
	We can check that the multiplication rule actually matches.
\end{example}
\begin{example}
	Fix $R=\ZZ$ and $I=(p)$ a prime ideal. Then $I^n/I^{n+1}=p^n\ZZ/p^{n+1}\ZZ=p^n(\ZZ/p\ZZ)$, so we can represent anyone in $\op{gr}_IR$ by
	\[a_0+a_1p+a_2p^2+\cdots\]
	where $a_0,a_1,\ldots\in\ZZ/p\ZZ$. Checking our ring structure, we can identify this with the ``finite-length'' $p$-adic integers $\ZZ_p$. What we get at the end is $\ZZ/p\ZZ[x]$, which requires some care with the ring structure: think of $a_1p\cdot b_1p$ as living in the $a_1b_1p^2$ coordinate, so this is essentially a polynomial ring.
\end{example}
We have the following warning.
\begin{warn}
	There is no natural ring homomorphism $R\to\op{gr}_IR$.
\end{warn}
However, there is a natural map of sets. Explicitly, for our filtration
\[R\supseteq I\supseteq I^2\supseteq\cdots,\]
we want to find an element of the associated graded ring. In analogy to picking up the ``initial'' nonzero homogeneous part of a polynomial, we pick up $f\in R$ and define
\[\op{in}f=f+I^{n+1},\]
where $n$ is the largest possible such that $f\in I^n$. Of course, there is something of a problem when $f$ lives in all of $I^n$, in which case we set $\op{in}f:=0$.

Let's think about how this plays with our ring structure. Taking $f,g\in R$, if $R$ is a domain, then we get that $\op{in}(fg)=\op{in}(f)\op{in}(g)$. However, if $f$ and $g$ are zero-divisors, then we might be in trouble when $fg=0$.

And now for some examples.
\begin{example}
	Fix $X\subseteq\AA^n(k)$ a Zariski closed set with $X=Z(J)$ such that $J\subseteq k[x_1,\ldots,x_n]=:R$ is an ideal. Taking $p\in X$ to correspond to a maximal ideal $\mf m\subseteq A(X)$, we claim that
	\[\op{gr}_IR\]
	is the ring corresponding to the ``tangent cone to $p$ at $X$.''
\end{example}
As an example, consider the curve $y^2=x^2(x+1)$, which splits at $0$. % \todo{make diagram}
At a point which is not $(0,0)$, we will have a line and therefore will expect to get a polynomial ring.

However, let's focus on what happens at $(0,0)$. Analytically, we find that
\[\frac{y^2}{x^2}=x+1.\]
Very close to $(0,0)$, we get that
\[\left(\frac{dy}{dx}\right)^2=1\]
so that the slope is $\pm1$.

Let's try to think more algebraically. We have the following lemma.
\begin{lemma}
	Work in the context of the above example. Then
	\[\op{gr}_I(R/J)=(\op{gr}_IR)/\op{in}J.\]
\end{lemma}
\begin{proof}
	This is on the homework.
\end{proof}
The point of this lemma is that $\op{gr}_IR$ we know to be a polynomial ring. With $I=(x,y)$ as in the example we are working out, we find that $\op{in}(x)^2=\op{in}(y)^2$ because our ideal $J$ is $y^2-x^2(x+1)$. Namely, our associated ring looks like functions generated by the lines $\op{in}x=\op{in}y$ and $\op{in}x=-\op{in}y$, which is what we expected.

In contrast, the cusp $y^2=x^3-x$ will give a double point, generated only at $\op{in}(x)^2$. Here, we will be generated by $(\op{in}y)^2$, which is what our cusp looks like intuitively.

\subsection{Filtration of Modules}
Consider the following construction.
\begin{defi}[Hilbert function, rings]
	Fix $R$ a local Noetherian ring where $I$ is the maximal ideal. Then we define
	\[\dim_{R/I}(\op{gr}_IR)_n=\dim_{R/I}\left(I^n/I^{n+1}\right)=H_R(n).\]
	Note that this definition is well-formed because $R/I$ is a field.
\end{defi}
We would like to generalize this to modules. We have the following series of definitions.
\begin{definition}[Filtration, modules]
	Given an $R$-module $M$, a \textit{filtration} is a descending chain
	\[M=M_0\supseteq M_1\supseteq M_2\supseteq\cdots.\]
	This is an \textit{$I$-filtration} if and only if $IM_n\subseteq M_{n+1}$.
\end{definition}
There is no multiplicative condition on the filtration because $M$ has no multiplication.
\begin{definition}[Associated graded module]
	Fix an $R$-module $M$, with a filtration $\mathcal J$, denoted by
	\[M=M_0\supseteq M_1\supseteq M_2\supseteq\cdots.\]
	Then we define
	\[\op{gr}_\mathcal JM:=M/M_1\oplus M_1/M_2\oplus\cdots.\]
\end{definition}
We remark that $\op{gr}_\mathcal JM$ is a graded $\op{gr}_IR$-module, which is not too hard to check by hand.
\begin{definition}[Stable]
	An $I$-filtration of an $R$-module $M$, denoted by
	\[M=M_0\supseteq M_1\supseteq M_2\supseteq\cdots\]
	is \textit{$I$-stable} if and only if $IM_j=M_{j+1}$ for sufficiently large $j$.
\end{definition}
It's a math class, so let's try to prove something today.
\begin{proposition}
	Fix $I\subseteq R$ an ideal. Further, take $M$ to be a finitely generated $R$-module, $\mathcal J$ to be a stable $I$-filtrations by finitely generated $R$-modules. Then $\op{gr}_\mathcal JM$ is a finitely generated $\op{gr}_IR$-module.
\end{proposition}
\begin{proof}
	We definition-chase. Let our filtration be
	\[M=M_0\supseteq M_1\supseteq M_2\supseteq\cdots.\]
	For sufficiently large $n$, we have that $I^kM_n=M_{n+k}$. Thus, it suffices to take generators for $M_0,M_1,\ldots,M_n$ to generate the entire associated graded module.
\end{proof}
This lets us construct our Hilbert function for modules.
\begin{defi}[Hilbert function, modules]
	Fix $R$ a local Noetherian ring where $I$ is the maximal ideal with $M$ a finitely generated $R$-module. Then we define
	\[H_M(n)=\dim_{R/I}\left(I^nM/I^{n+1}M\right).\]
	Note that this definition is well-formed because $M$ is finitely generated.
\end{defi}

\subsection{The Artin--Rees Lemma}
We are finally ready to provide our main result.
\begin{theorem}[Artin--Rees lemma]
	Fix $R$ a Noetherian ring and $I\subseteq R$ an ideal with $M$ a finitely generated $R$-module granted a stable $I$-filtration $\mathcal J$ denoted by
	\[M=M_0\supseteq M_1\supseteq M_2\supseteq\cdots.\]
	Then given a submodule $N\subseteq M$, the induced filtration by $N_k:=M_k\cap N$ is also a stable $I$-filtration.
\end{theorem}
\begin{proof}
	To prove this, we need to introduce the blow-up ring.
	\begin{definition}[Blow-up ring]
		Fix $R$ a ring and $I\subseteq R$ an ideal. Then we define the \textit{blow-up ring} $\op B_IR$ by
		\[\op B_IR:=R\oplus I\oplus I^2\oplus\cdots.\]
	\end{definition}
	Concretely, think about $\op B_IR$ as getting its ring structure from $k[t]$ by something like $k[It]$. This also gives us our grading. In particular, is that $\op B_IR/I\op B_IR\cong\op{gr}_IR$ after tracking everything through.
	\begin{ex}
		Fix $R:=k[x,y]$ and consider $(0,0)\in\AA^2(k)$ with associated maximal ideal $I:=(x,y)\subseteq R$. In this case, our blow-up ring looks like $k[x,y][tx,ty]$. To look at points, we need to look at the ``graded'' spectrum of $\op B_IR$. Here are some ways to do this.
		\begin{itemize}
			\item Look at $Z\subseteq\AA^2(k)\times\PP^1(k)$ to be points $(p,\ell)$ such that $p\in\ell$. We can project $Z\onto\AA^2(k)$ in the natural way. As long as $p\ne0$, there is exactly one pre-image. But if $p=0$, then our pre-image contains all the lines in $\PP^1(k)$! So we have created some ``blowing up'' at the origin.
			\item Alternatively, focus on $k[x,y][tx,ty]$. Set $u=tx$ and $v=ty$ so that we are essentially looking at the ring
			\[\frac{k[x,y,u,w]}{(xw-yv)},\]
			which correspond to the $2\times2$ singular matrices. Taking the quotient by the ``line action'' of matrices
			\[\begin{bmatrix}
				1 & 0 \\
				0 & \lambda
			\end{bmatrix}.\]
			Most of the time, this quotient process will give us $0$, but rarely we will have an entire line after doing the quotient.
		\end{itemize}
	\end{ex}
	We remark that there is also a notion of the blow-up module.
	\begin{definition}[Blow-up ring]
		Fix $R$ a ring and $I\subseteq R$ an ideal. Further, fix $\mathcal J$ an $I$-filtration. Then we define the \textit{blow-up module} $\op B_IM$ by
		\[\op B_IM:=M_0\oplus M_1\oplus M_2\oplus\cdots,\]
		which we can check to be a graded $\op B_IR$-module.
	\end{definition}
	In line with this, we have the following proposition.
	\begin{proposition}
		Fix $R$ a Noetherian ring and $I\subseteq R$ an ideal with $M$ a finitely generated $R$-module granted an $I$-filtration $\mathcal J$ denoted by
		\[M=M_0\supseteq M_1\supseteq M_2\supseteq\cdots.\]
		Then $\op B_\mathcal JM$ is finitely generated as a $\op B_IR$-module if and only if $\mathcal J$ is $I$-stable.
	\end{proposition}
	\begin{proof}
		We omit this proof. It is largely definition-chasing.
	\end{proof}
	We are now ready to attack the proof of the Artin--Rees lemma. Let $\mathcal J'$ be the induced filtration for $N$. From the definition, we see that $\op B_{\mathcal J'}N\subseteq\op B_\mathcal JM$ is a $\op B_IR$-submodule. Now, $\op B_{\mathcal J'}N$ is a submodule of the finitely generated module $\op B_\mathcal JM$ under the Noetherian ring $\op B_IR$, so we are done.
\end{proof}
Here is a nice application.
\begin{theorem}[Krull intersection]
	Fix $R$ a Noetherian ring with an ideal $I$ and finitely generated module $M$. Then
	\[N:=\bigcap_{s\ge0}I^sM\]
	satisfies that there is some $x\in I$ such that $(1-r)N=0$.
\end{theorem}
\begin{proof}
	By construction, we see that $IN=N$, which in particular holds because the standard $I$-filtration of $M$ is stable. Then we showed as a lemma to Nakayama's lemma back in \autoref{lem:nakayamalemlem} that there is an element $r\in I$ with $(1-r)N=0$.
\end{proof}
\begin{corollary}
	Fix $R$ a Noetherian ring with a proper ideal $I$. Further, if $R$ is local or a domain, then
	\[\bigcap_{s\ge0}I^s=0.\]
\end{corollary}
\begin{proof}
	Set
	\[J:=\bigcap_{s\ge0}I^s.\]
	By the proof of the theorem, we get $IJ=J$, which finishes by Nakayama's lemma. When $R$ is a domain, then the theorem gives us some $r\in I$ such that $(1-r)J=0$, but $R$ being a domain will force $J=0$ from this.
\end{proof}
\begin{remark}
	The condition that $R$ is Noetherian is necessary.
\end{remark}
We close with an exercise.
\begin{exe}
	Fix $R$ a local Noetherian ring. If $\op{gr}_IR$ is a domain, then $R$ is a domain.
\end{exe}
\begin{proof}
	The main idea is that $\op{in}f=0$ implies $f=0$, essentially by the corollary above.
\end{proof}