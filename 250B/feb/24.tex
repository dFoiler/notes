% !TEX root = ../notes.tex

So it's the day after death.

\subsection{Midterm Review}
Let's start talking about the second problem on the midterm.
\begin{exe}
	Identify matrices $X\in\CC^{2\times2}$ with $\AA^4(\CC)$ by
	\[\begin{bmatrix}
		a & b \\
		c & d
	\end{bmatrix}\mapsto(a,b,c,d).\]
	Let $Z:=\left\{X\in\CC^{2\times2}:X^2=0\right\}$. Then show that the ideal $I(Z)$ is prime.
\end{exe}
\begin{proof}
	We start by showing
	\[Z=\{(a,b,c,d):a+d=ad-bc=0\}.\]
	In one direction, we note that $X^2=0$ implies that all eigenvalues are $0$, so the characteristic polynomial of $X$ will be $X^2=0$, so we see that $\tr X=\det X=0$. Thus, $(a,b,c,d)\in I$ implies $a+d=ad-bc=0$.

	Conversely, if $(a,b,c,d)$ have $a+d=ad-bc=0$, then the associated matrix $X$ satisfies the equation
	\[x^2=x^2-(\tr X)x+\det X=0\]
	by \autoref{thm:ch}, so we conclude that $X^2=0$.

	Now, we see that $Z=Z(a+d,ad-bc)$, so by \autoref{thm:nullstellensatz},
	\[I(Z)=\rad(a+d,ad-bc).\]
	So we claim that $(a+d,ad-bc)$ is prime, which will show that it is radical and therefore showing $I(Z)=(a+d,ad-bc)$ is prime. To show $(a+d,ad-bc)$ is prime, we note that we have a map
	\[\CC[a,b,c,d]\to\CC[a,b,c]\]
	by sending $d\mapsto-a$. It is not too hard to check that $(a+d)$ is the kernel of this map, so we have an embedding
	\[\frac{\CC[a,b,c,d]}{(a+d)}\into\CC[a,b,c].\]
	Now, if we want to mod out the left by $(ad-bc)$, this goes to $\left(-a^2-bc\right)=\left(a^2+bc\right)$ on the right. In fact, the pre-image of $\left(a^2+bc\right)$ we can check will actually be $(a+d)$, so we get an embedding
	\[\frac{\CC[a,b,c,d]}{(a+d,ad-bc)}\into\frac{\CC[a,b,c]}{\left(a^2+bc\right)}.\]
	Now, to show that $(a+d,ad-bc)$ is prime, it suffices to show that the left-hand ring is an integral domain, for which it suffices to show that $\CC[a,b,c]/\left(a^2+bc\right)$ is an integral domain, for which it suffices to show that $a^2+bc$ is an irreducible element because $\CC[a,b,c]$ is a unique factorization domain. Well, by degree arguments with $a$, the only way to factor this would be as
	\[(a+f)(a+g)\qquad\text{or}\qquad\left(a^2+f\right)g,\]
	where $f$ and $g$ feature no $a$s. The former would force $ag$ and $af$, but $a^2+bc$ has no terms other than $a^2$ with an $a$. The latter would force $g=1$ because of the $a^2g$ term, so this factorization is trivial.
\end{proof}

\subsection{Filtrations of Rings}
Today we are talking about the Artin--Rees lemma, which requires us talking about filtrations. Here is our definition.
\begin{definition}[Filtration, rings]
	Fix $R$ a ring. Then a \textit{filtration} of $R$ is a sequence of ideals $\{I_p\}_{p\in\NN}$ forming the chain
	\[R=I_0\supseteq I_1\supseteq I_2\supseteq\cdots\]
	such that $I_pI_q\subseteq I_{p+q}$.
\end{definition}
While we're here, we record the following philosophy.
\begin{idea}
	Filtrations are useful to understand an object in smaller steps.
\end{idea}
Anyways, the condition $I_pI_q\subseteq I_{p+q}$ should remind us of grading, and indeed graded rings have nice filtrations.
\begin{exe}[``Graded'' filtration] \label{exe:gradedfiltration}
	Fix $R=R_0\oplus R_1\oplus R_2\oplus\cdots$ a graded ring. Then the ideals
	\[I_p:=\bigoplus_{i\ge p}R_i\]
	form a filtration.
\end{exe}
\begin{proof}
	We see that $R=I_0$ and $I_p\supseteq I_{p+1}$ is by construction, so we are allowed to write
	\[R=I_0\supseteq I_1\supseteq I_2\supseteq\cdots.\]
	Additionally, for any $f\in I_p$ and $g\in I_q$, then we can write out
	\[f=\sum_{i\ge p}a_i\qquad\text{and}\qquad g=\sum_{j\ge q}b_j\]
	with $a_i\in R_i$ and $b_j\in R_j$ so that, upon distributing,
	\[fg=\sum_{i\ge p,j\ge q}a_ib_j.\]
	Each term $a_ib_j$ lives in $R_iR_j\subseteq R_{i+j}\subseteq I_{p+q}$, so $fg\in I_{p+q}$.
\end{proof}
Here is our other chief example of filtration.
\begin{defihelper}[\texorpdfstring{$I$}{I}-adic filtration] \nirindex{I-adic filtration @\texorpdfstring{$I$}{I}-adic filtration}
	Fix $R$ a ring and $I\subseteq R$ an ideal. Then
	\[R=I^0\supseteq I^1\supseteq I^2\supseteq I^3\supseteq\cdots\]
	is a filtration. This is called the \textit{$I$-adic filtration}.
\end{defihelper}
As a brief justification, we see that $R=I^0$ by definition of $I^0$; $I^p\supseteq I^{p+1}$ is because $II^k\subseteq RI^k=I^k$; and lastly, $I^pI^q=I^{p+q}\subseteq I^{p+1}$.
\begin{exe}
	The graded filtration produced by grading $R=k[x_1,\ldots,x_n]$ by degree and using \autoref{exe:gradedfiltration} is the $(x_1,\ldots,x_n)$-adic filtration.
\end{exe}
\begin{proof}
	Let $R_d\subseteq k[x_1,\ldots,x_n]$ be the union of $\{0\}$ and the polynomials homogeneous of degree $d$. Then, fixing some nonnegative integer $p$, we are asserting that
	\[\bigoplus_{i\ge d}R_d\stackrel?=(x_1,\ldots,x_n)^d.\]
	But this is true essentially by tracking through what everything means. By definition,
	\[(x_1,\ldots,x_n)^d=\left(x_1^{d_1}\cdots x_n^{d_n}:d_1+\cdots+d_n=d\right).\]
	In particular, $(x_1,\ldots,x_n)^d$ is generated by elements in $R_d\subseteq\bigoplus_{i\ge d}R_d$.

	In the other direction, suppose that we have any $f\in\bigoplus_{i\ge d}R_d$. Then we can decompose
	\[f=\sum_{i\ge d}f_i,\]
	where $f_i\in R_i$. We claim that each $f_i$ lives in $(x_1,\ldots,x_n)^d$, which will finish by showing $f\in(x_1,\ldots,x_n)^d$. Well, by definition of $R_i$, we can write
	\[f_i(x_1,\ldots,x_n)=\sum_{d_1+\cdots+d_n=i}a_{(d_1,\ldots,d_n)}x_1^{d_1}\cdots x_n^{d_n}.\]
	Now, $d_1+\cdots+d_n=i\ge d$, so the monomial $x_1^{d_1}\cdots x_n^{d_n}$ is divisible by a polynomial of degree $d$ and therefore lives in $(x_1,\ldots,x_n)^d$.\footnote{Writing this out would be very annoying; here is one way: find the largest $m\ge0$ such that $d_1+\cdots+d_m<d$. Note $m<n$ because $d<i$. Then $x_1^{d_1}\cdots x_m^{d_m}x_{m+1}^{n-(d_1+\cdots+d_m)}$ divides $x_1^{d_1}\cdots x_n^{d_n}$.} So each monomial in the expansion of $f_i$ lives in $x_1^{d_1}\cdots x_n^{d_n}$, so $f_i$ lives in $x_1^{d_1}\cdots x_n^{d_n}$.
\end{proof}
\begin{remark}[Nir, Miles]
	Fix a graded ring $R$ and $I$ the irrelevant ideal. It is not in general true that the ``graded'' filtration from \autoref{exe:gradedfiltration} is the same as the $I$-adic filtration. For example, consider $R:=k\left[x^2\right]$ graded by degree; namely,
	\[R_{2d}=kx^{2d}\qquad\text{and}\qquad R_{2d+1}=0.\]
	Then we see that $I$ contains no nonzero linear polynomials, so $I^2$ will contain no nonzero quadratic polynomials, so the second term of the $I$-adic filtration has no quadratics. However, the graded filtration has the second term as $R_2\oplus R_3\oplus\cdots$, which definitely contains quadratics.
\end{remark}
To set up the discussion that follows, we note that, if we have a filtration
\[R=I_0\supseteq I_1\supseteq I_2\supseteq\cdots,\]
we might be interested in the ``bottom'' of this filtration
\[I:=\bigcap_{i=0}^\infty I_i.\]
It is not too hard to check that this is an ideal: $x,y\in I$ and $r,s\in R$ have $x,y\in I_i$ and therefore $rx+sy\in I_i$ for any $i$, so $rx+sy\in I$. Now, if we have a ``good'' filtration, we might hope that $I=0$ so that our filtration can actually see to the bottom of $R$. Of course, we will need some conditions on the filtration to guarantee this.

\subsection{Associated Graded Rings}
We saw that gradings give filtrations back in \autoref{exe:gradedfiltration}. We can partially go the other way as well.
\begin{definition}[Associated graded ring]
	Fix a ring $R$ and a filtration $\mathcal J$ notated
	\[R= I_0\supseteq I_1\supseteq I_2\supseteq\cdots.\]
	Then we set $R_i:=I_i/I_{i+1}$ and define
	\[\op{gr}_{\mathcal J}R:=\bigoplus_{p\ge0}I_p/I_{p+1}\]
	to be the \textit{associated graded ring}. If $\mathcal J$ is the $I$-adic filtration, we denote the associated graded ring by $\op{gr}_I(R)$. If the filtration is obvious, we will omit the subscript entirely.
\end{definition}
A priori, the associated graded ring is only some very large module, but we can give it a ring structure as follows: if we have terms $[a]\in I_p/I_{p+1}$ and $[b]\in I_q/I_{q+1}$, then we can lift them to some $a\in I_p$ and $b\in I_q$ so that $ab\in I_pI_q\subseteq I_{p+q}$, so we set
\[[a]\cdot[b]:=[ab]\in I_{p+q}/I_{p+q+1}.\]
We now run the following checks.
\begin{lemma}
	Fix $R$ a ring and filtration $\mathcal J$ notated
	\[R=I_0\supseteq I_1\supseteq I_2\supseteq\cdots.\]
	The above multiplication on $\op{gr}_\mathcal JR$ makes $\op{gr}_\mathcal JR$ into a graded ring in the natural way by $(\op{gr}_\mathcal JR)_p:=I_p/I_{p+1}$.
\end{lemma}
\begin{proof}
	We start by showing that the multiplication of our ``homogeneous'' elements is well-defined. If $a\equiv a'\pmod{I_{p+1}}$ both represent $[a]$ and $b\equiv b'\pmod{I_{q+1}}$ both represent $[b]$, then
	\[ab-a'b'=ab-ab'+ab'-a'b'=a(b-b')+(a-a')b'.\]
	Now, $a\in I_p$ and $b-b'\in I_{q+1}$, so $a(b-b')\in I_{p+q+1}$; similarly, $a-a'\in I_{p+1}$ and $b'\in I_q$, so $(a-a')b\in I_{p+q+1}$. Thus, the entire element lives in $I_{p+q+1}$, so $ab\equiv a'b'\pmod{I_{p+q+1}}$.
	
	Now we acknowledge that the above multiplication law extends distributively as
	\[\left(\sum_{p\ge0}[a_p]_{I_{p+1}}\right)\left(\sum_{q\ge0}[b_q]_{I_q}\right):=\sum_{n\ge0}\left(\sum_{p+q=n}[a_pb_q]_{I_{n+1}}\right).\]
	So we have indeed defined a multiplication on all of $\op{gr}_\mathcal JR$. We remark that we can see somewhat directly that one could imagine showing that the multiplication commutes (this is not so bad), associates (the point is to write the inner sum as $p+q+r=n$), and distributes (cry), but we will not write out these checks; the curious can port over the proof that multiplication in $R[x]$ forms a ring structure. % \todo{maybe do write out these checks?}

	It remains to show that the ring is actually graded in the natural way. Specifically, we need to show that
	\[(\op{gr}_\mathcal JR)_p(\op{gr}_\mathcal JR)_q\subseteq(\op{gr}_\mathcal JR)_{p+q}.\]
	But this is by definition of our multiplication: we see that $(\op{gr}_\mathcal JR)_p(\op{gr}_\mathcal JR)_q$ is generated by products
	\[[a]_{I_{p+1}}[b]_{I_{q+1}}=[ab]_{I_{p+q+1}}\in(\op{gr}_\mathcal JR)_{p+q},\]
	where $[a]_{I_{p+1}}\in(\op{gr}_\mathcal JR)_p$ and $[b]_{I_{q+1}}\in(\op{gr}_\mathcal JR)_q$.
\end{proof}
\begin{remark}
	Technically we should say that the element
	\[[1]_I+[0]_{I^2}+[0]_{I^3}+\cdots\]
	is our unit element. Indeed, we can compute
	\[([1]_I+[0]_{I^2}+[0]_{I^3}+\cdots)\cdot[a]_{I^n}=[a]_I^n\]
	by looking component-wise, and our identity will extend to all of $\op{gr}_\mathcal JR$ by how we defined our multiplication.
\end{remark}
\begin{remark}[Nir] \label{rem:iadicgradedmodule}
	In fact, $\op{gr}_IR$ is an $R/I$-module. Indeed, our map $R\to\op{End}_R(\op{gr}_IR)$ is created by stitching together the maps
	\[R\to\op{End}_R\left(I^p/I^{p+1}\right)\]
	by $r\mapsto([x]\mapsto[rx])$. However, we note that if $r\in I$ with $[x]\in I^p/I^{p+1}$, then $[rx]\in I^{p+1}/I^{p+1}$, so $[rx]=0$. In particular, the above map has $I$ in its kernel, so we actually get to stitch together the maps
	\[R/I\to\op{End}_R\left(I^p/I^{p+1}\right)\]
	to a map $R/I\to\op{End}_R(\op{gr}_IR)$.
\end{remark}
Anyways, let's see some examples.
\begin{exe}
	Fix $R:=k[[x]]$ and $I:=(x)$. We show that $\op{gr}_IR\cong k[x]$ as graded rings.
\end{exe}
\begin{proof}
	Here we are using the $I$-adic filtration given by $I^n=(x)^n=\left(x^n\right)$. In particular, given any
	\[f(x)=\sum_{d\ge n}a_dx^d\in\left(x^n\right),\]
	we see that $\sum_{d\ge n+1}a_dx^d\in\left(x^{n+1}\right)$, so we can give $f(x)\in I^n/I^{n+1}$ a fairly natural representative by
	\[f(x)=a_nx^n+\sum_{d\ge n+1}a_dx^d\equiv a_nx^n\pmod{I^{n+1}}.\]
	So, given $f(x)\in I^n/I^{n+1}$, we define $\varphi_n(f(x)):=a_nx^n$ so that $\varphi_n:I^n/I^{n+1}\to kx^n$. As such, we can assemble the $\varphi_n$ into a map
	\[\varphi:\bigoplus_{n\ge0}I^n/I^{n+1}\to\bigoplus_{n\ge0}kx^n\]
	component-wise. Now, we observe that the domain of $\varphi$ is $\op{gr}_IR$ and the codomain is $k[x]$, so it remains to show that $\varphi$ is an isomorphism of graded rings.

	We start by showing that $\varphi$ is a graded homomorphism. The grading part is fairly simple because $\varphi$ restricts to $\varphi_n:I^n/I^{n+1}\to kx^n$ on each component, so $\varphi$ does preserve the grading. Now, by the universal property of direct sums, it suffices to show that each $\varphi_n$ is a group homomorphism. Well, if we pick up
	\[f(x)=\sum_{d\ge n}a_dx^d\qquad\text{and}\qquad g(x)=\sum_{d\ge n}b_dx^d,\]
	and we compute
	\[\varphi_n\left([f]_{I^{n+1}}+[g]_{I^{n+1}}\right)=\varphi_n\left([f+g]_{I^{n+1}}\right)=a_nx^n+b_nx^n=\varphi_n\left([f]_{I^{n+1}}\right)+\varphi_n\left([g]_{I^{n+1}}\right).\]
	Continuing, we see that $\varphi$ preserves identity because
	\[\varphi(1)=\varphi\left(\sum_{n\ge0}[1_{n=0}]_{I^{n+1}}\right)=\sum_{n\ge0}1_{n=0}=1.\]
	Lastly, to check that $\varphi$ is multiplicative, we note that our multiplication was uniquely determined by what it did to homogeneous elements, so it suffices to show that $\varphi$ is multiplicative on homogeneous elements, for this will extend by distributivity. So pick up
	\[f_n(x)=\sum_{d\ge n}a_dx^d\qquad\text{and}\qquad g_m(x)=\sum_{e\ge m}b_ex^e\]
	so that $\varphi_n(f_n)=a_d$ and $\varphi_m(g_m)=b_mx^m$. Then $[f_n]_{I^{n+1}}=\left[a_nx^n\right]_{I^{n+1}}$ and $[g_m]_{I^{m+1}}=\left[b_mx^m\right]_{I^{m+1}}$ so that the well-definedness of our multiplication promises
	\[\varphi_{n+m}\left([f_n]_{I^{n+1}}\cdot[g_m]_{I^{m+1}}\right)=\varphi_{n+m}\left([a_nb_mx^{n+m}]_{I^{n+m+1}}\right)=a_nx^n\cdot b_mx^m=\varphi_n\left([f_n]_{I^{n+1}}\right)\cdot\varphi_m\left([g_m]_{I^{m+1}}\right),\]
	which is what we wanted. To be convinced that our distributivity hand-waving is legitimate, we note that we could write out
	\[\varphi\left(\sum_{p\ge0}[f_p]\cdot\sum_{q\ge0}[g_q]\right)=\varphi\left(\sum_{p+q=n}[f_pg_q]\right)=\sum_{p+q=n}\varphi_{p+q}([f_pg_q])=\sum_{p+q=n}\varphi_p([f_p])\varphi_q([g_q]),\]
	which we can then distribute backwards to $\varphi\left(\sum_p[f_p]\right)\varphi\left(\sum_q[g_q]\right)$.

	It remains to show that $\varphi$ is a bijection. For this, it suffices to show that $\varphi$ is an isomorphism of $R$-modules, for which we note that it suffices to check that $\varphi$ restricts to an isomorphism on each component $\varphi_n:I^n/I^{n+1}\to kx^n$. In fact, we already know that this is a group homomorphism (because $\varphi$ is additive), so we merely need to know that $\varphi_n$ is bijective.
	\begin{itemize}
		\item We show that $\varphi_n$ is surjective. Well, given $a_nx^n$, we see that $\varphi_n\left(\left[a_nx^n\right]_{I^{n+1}}\right)=a_nx^n$.
		\item We show that $\varphi_n$ is injective. Well, suppose that $f\in I^n$ has $\varphi_n\left([f]_{I^{n+1}}\right)=0x^n$. Then, by definition, our expansion
		\[f(x)=\sum_{d\ge n}a_dx^d\]
		has $a_n=0$. In particular, we can write $f(x)=\sum_{d\ge n+1}a_dx^d$ so that $f(x)\in I^{n+1}$, so $[f]_{I^{n+1}}=[0]_{I^{n+1}}$, which is what we wanted.
	\end{itemize}
	The above checks finish the proof that $\varphi$ is an isomorphism.
\end{proof}
% \begin{example}
% 	We work in $R:=k[[x]]$, which is local with maximal ideal $I:=(x)$. Then $I^n=\left(x^n\right)$ gives rise an $I$-adic filtration. We can compute
% 	\[I^n/I^{n+1}\cong\left\{ax^n:a\in k\right\}\cong kx^n\]
% 	because we are taking ($0$ or) a very long polynomial with minimal degree $x^n$ and then killing all higher degree terms. So our filtration reads as
% 	\[\op{gr}_IR=R/I\oplus I/I^2\oplus I^2/I^\oplus\cdots=k\oplus kx\oplus kx^2\oplus\cdots=k[x].\]
% 	We can check that the multiplication rule actually matches.
% \end{example}
We will be briefer with our next example because it is similar.
\begin{example} % \todo{add details}
	Fix $R=\ZZ$ and $I=(p)$ a prime ideal, where $p>0$ is a positive prime. Then, in $I^n/I^{n+1}=p^n\ZZ/p^{n+1}\ZZ$, all elements have a unique representative as $\left[p^na\right]_{p^{n+1}}$ for $a\in\ZZ/p\ZZ$, so we can represent anyone in $\op{gr}_IR$ by
	\[a_0+a_1p+a_2p^2+\cdots\]
	where $a_0,a_1,a_2\ldots\in\ZZ/p\ZZ$. In particular, we can see that multiplication of homogeneous elements behaves as
	\[\left[a_kp^k\right]_{p^{k+1}}\cdot\left[b_\ell p^\ell\right]_{p^{\ell+1}}=\left[a_kb_\ell p^{k+\ell}\right]_{p^{k+\ell+1}}.\]
	In particular, if we imagine taking $p\mapsto x$, the above is really the polynomial grading, so we see that extending $p\mapsto x$ to all of $\op{gr}_{(p)}\ZZ$ gives an isomorphism $\op{gr}_{(p)}\ZZ\cong(\ZZ/p\ZZ)[x]$.
\end{example}

\subsection{Initial Forms}
We begin with the following warning.
\begin{warn}
	There is no natural ring homomorphism $R\to\op{gr}_{\mathcal J}R$.
\end{warn}
This is sad because we would like to use the associated graded ring to understand the original ring, so not having a natural map significantly hinders our ability.

However, there is a natural map of sets.
\begin{definition}[Initial form]
	Fix $R$ a ring and $\mathcal J$ a filtration notated
	\[R=I_0\supseteq I_1\supseteq I_2\supseteq\cdots.\]
	Then, fix $f\in R$, and we define the \textit{initial form of $f$} $\op{in}f\in\op{gr}_IR$ as follows.
	\begin{itemize}
		\item If $f\in I_p$ for all $p$, then $\op{in}f=0$.
		\item If $f\in I_p$ but $f\notin I_{p+1}$, then $\op{in}f=[f]_{I_{p+1}}$.
	\end{itemize}
	Intuitively, $\op{in}$ extracts the smallest ``homogeneous'' part of $f$ to put in $\op{gr}_IR$; we return $0$ when this is impossible.
\end{definition}
This map of sets really does not have extra structure, though it feels like it comes close, which is reassuring.
\begin{remark}[Nir]
	To see that $\op{in}$ is not a ring homomorphism, we see that it is not additive. Consider $R=k[x]$ given the $(x)$-adic filtration.
	\begin{itemize}
		\item Take $f=x$ so that $\op{in}f=[x]_{x^2}$.
		\item Take $g=x^2$ so that $\op{in}g=\left[x^2\right]_{x^3}$.
		\item Then $f+g=x+x^2$ so that $\op{in}(f+g)=[x]_{x^2}$.
	\end{itemize}
	Notably, $\op{in}f+\op{in}g$ is not even homogeneous, but the image of $\op{in}$ always is. In general, we will have $\op{in}f+\op{in}g\in\{\op{in}f,\op{in}g,\op{in}(f+g)\}$ based off of if $f$ and $g$ live in the same part of the filtration; we will prove this shortly.
\end{remark}
In fact, $\op{in}$ need not even be multiplicative, though it is almost multiplicative.
\begin{proposition}
	Fix $R$ a ring and $\mathcal J$ a filtration
	\[R=I_0\supseteq I_1\supseteq I_2\supseteq\cdots.\]
	Then, given $f,g\in R$, we have $\op{in}f\cdot\op{in}g\in\{\op{in}(fg),0\}$. If $\op{gr}_\mathcal JR$ is an integral domain, then $\op{in}f\cdot\op{in}g=\op{in}(fg)$.
\end{proposition}
\begin{proof}
	Very quickly, suppose that one of $f$ or $g$ lives in $I_p$ for all $P$. Without loss of generality, say that $f\in I_p$ for all $I_p$, so $fg\in I_p\in I_p$ for all $p$, so
	\[\op{in}(fg)=0=\op{in}f=\cdot\op{in}f\cdot\op{in}g,\]
	dealing with the case of $\op{gr}_\mathcal JR$ an integral domain automatically as well.

	Otherwise, we have $f\in I_p\setminus I_{p+1}$ and $g\in I_q\setminus I_{q+1}$ for some $p$ and $q$ so that $\op{in}f=[f]_{I_{p+1}}$ and $\op{in}g=[g]_{I_{q+1}}$. Because $\mathcal J$ is a filtration, we see that
	\[fg\in I_pI_q\subseteq I_{p+q}.\]
	We now have two cases.
	\begin{itemize}
		\item If $fg\notin I_{p+q+1}$, then $\op{in}(fg)=[fg]_{I_{p+q+1}}=[f]_{I_{p+1}}\cdot[g]_{I_{q+1}}=\op{in}f\cdot\op{in}g$.
		\item If $fg\in I_{p+q+1}$, then $\op{in}f\cdot\op{in}g=[fg]_{I_{p+q+1}}=0$.
	\end{itemize}
	The above casework finishes the proof of the general case. When $\op{gr}_\mathcal JR$ is an integral domain, we note that $\op{in}f,\op{in}g\ne0$ disallows the second case, so we will always have $\op{in}f\cdot\op{in}g=\op{in}(fg)$.
\end{proof}

\subsection{Tangent Cone} \todo{Figure out what is happening here}
It turns out that the initial form has a nice geometric application.
\begin{example}
	Fix $X\subseteq\AA^n(k)$ a Zariski closed set with $X=Z(J)$ such that $J\subseteq k[x_1,\ldots,x_n]=:R$ is an ideal. Taking $p\in X$ to correspond to a maximal ideal $\mf m\subseteq A(X)$, we claim that
	\[\op{gr}_IR\]
	is the ring corresponding to the ``tangent cone to $p$ at $X$.''
\end{example}
As an example, consider the curve $y^2=x^2(x+1)$, which splits at $0$. Here is the image.
\begin{center}
	\begin{asy}
		unitsize(1cm);
		import graph;
		real x(real t)
		{
			return t*t-1;
		}
		real y(real t)
		{
			return t*(t*t-1);
		}
		draw(graph(x, y,-1.521,1.521));
		draw((-1.5,0) -- (2,0)); label("$x$", (2,0), E);
		draw((0,-2) -- (0,2)); label("$y$", (0,2), N);
	\end{asy}
\end{center}
At a point which is not $(0,0)$, we will have a line and therefore will expect to get a polynomial ring.

However, let's focus on what happens at $(0,0)$. Analytically, we find that
\[\frac{y^2}{x^2}=x+1.\]
Very close to $(0,0)$, we get that
\[\left(\frac{dy}{dx}\right)^2=1\]
so that the slope is $\pm1$.

Let's try to think more algebraically. We have the following lemma.
\begin{lemma}
	Work in the context of the above example. Then
	\[\op{gr}_I(R/J)=(\op{gr}_IR)/\op{in}J.\]
\end{lemma}
\begin{proof}
	This is on the homework.
\end{proof}
The point of this lemma is that $\op{gr}_IR$ we know to be a polynomial ring. With $I=(x,y)$ as in the example we are working out, we find that $\op{in}(x)^2=\op{in}(y)^2$ because our ideal $J$ is $y^2-x^2(x+1)$. Namely, our associated ring looks like functions generated by the lines $\op{in}x=\op{in}y$ and $\op{in}x=-\op{in}y$, which is what we expected.

In contrast, the cusp $y^2=x^3-x$ will give a double point, generated only at $\op{in}(x)^2$. Here, we will be generated by $(\op{in}y)^2$, which is what our cusp looks like intuitively.

\subsection{Filtrations of Modules}
Consider the following construction.
\begin{defi}[Hilbert function, rings] \label{def:hilbfuncring}
	Fix $R$ a local Noetherian ring with maximal ideal $\mf m$. Then we define the \textit{Hilbert function of $R$} as
	\[H_R(s):=\dim_{R/\mf m}(\op{gr}_\mf mR)_s=\dim_{R/\mf m}\left(\mf m^s/\mf m^{s+1}\right).\]
	Note that this definition is well-formed because $R/I$ is a field.
\end{defi}
We note that the definition of $H_R(s)$ is well-formed: $\mf m^s/\mf m^{s+1}$ is in fact an $R/\mf m$-module by \autoref{rem:iadicgradedmodule}, which is actually an $R/\mf m$-vector space because $R/\mf m$ is a field. As for finiteness, $\mf m^s$ is a finitely generated $R$-module (because $R$ is Noetherian), so $\mf m^s/\mf m^{s+1}$ is as well, so $\mf m^s/\mf m^{s+1}$ is a finite-dimensional $R/\mf m$-vector space.

The theory of the Hilbert function was actually stated for modules, so we would like to generalize this to modules. We have the following series of definitions.
\begin{definition}[Filtration, modules]
	Given an $R$-module $M$, a \textit{filtration} is a descending chain
	\[M=M_0\supseteq M_1\supseteq M_2\supseteq\cdots.\]
	Given an ideal $I$, the above is an \textit{$I$-filtration} if and only if $IM_q\subseteq M_{q+1}$. Note that this last condition is equivalent to $I^sM_q\subseteq M_{s+q}$ by an induction.
\end{definition}
Note there is no multiplicative condition on the filtration because $M$ has no multiplication.

As before, from filtrations we can build the associated graded module.
\begin{definition}[Associated graded module]
	Fix an $R$-module $M$ with a filtration $\mathcal J$, denoted by
	\[M=M_0\supseteq M_1\supseteq M_2\supseteq\cdots.\]
	Then we define
	\[\op{gr}_\mathcal JM:=\bigoplus_{s\ge0}M_s/M_{s+1}=M/M_1\oplus M_1/M_2\oplus\cdots.\]
\end{definition}
A priori, $\op{gr}_\mathcal JM$ merely has an $R$-module structure inherited as a direct sum, but when $\mathcal J$ is an $I$-filtration, then we do get a graded structure from our graded module.
\begin{lemma}
	Fix an $R$-module and $I\subseteq R$ an ideal. If $M$ is an $R$-module with an $I$-filtration $\mathcal J$ denoted by
	\[M=M_0\supseteq M_1\supseteq M_2\supseteq\cdots,\]
	then $\op{gr}_\mathcal JM$ is a $\op{gr}_IR$-module.
\end{lemma}
\begin{proof}
	We start by describing our action. Given $[a]_{I^{s+1}}\in I^s/I^{s+1}$ and $[b]_{M_{q+1}}\in M_q/M_{q+1}$, we see that
	\[ab\in I^sM_q\subseteq M_{s+q}.\]
	In fact, if we pick up another representative $[a]=[a']$ and $[b]=[b']$, then
	\[ab-a'b'=ab-ab'+ab'-a'b'=a(b-b')+(a-a')b'\in I^sM_{q+1}+I^{s+1}M_q\subseteq M_{s+q+1}.\]
	Thus, the representative $[ab]_{M_{s+q+1}}\in M_{s+q}/M_{s+q+1}$ is unique of the exact choice of representative for $a$ and $b$. We have the following checks on this action.
	\begin{itemize}
		\item Fix $[a]_{I^{s+1}}$; then the action on $M_q$ is $R$-linear: we compute
		\[[a]_{I^{s+1}}\cdot\left(r_1[b_1]_{M_{q+1}}+r_2[b_2]_{M_{q+1}}\right)=[ar_1b_1+ar_2b_2]_{M_{q+1}}=r_1\left([a]_{I^{s+1}}\cdot[b_1]_{M_{q+1}}\right)+r_2\left([a]_{I^{s+1}}\cdot[b_2]_{M_{q+1}}\right).\]
		In particular, we see that we have defined a function
		\[(\op{gr}_IR)_s\to\op{Hom}_R\left((\op{gr}_\mathcal JM)_q,(\op{gr}_\mathcal JM)_{q+s}\right)\]
		\item The function defined in the previous point is $R$-linear. Namely, we compute
		\[\left(r_1[a_1]_{I^{s+1}}+r_2[a_2]_{I^{s+1}}\right)\cdot[b]_{M_{q+1}}=[r_1a_2b+r_2a_2b]_{I^{s+1}}=[r_1a_1+r_2a_2]_{I^{s+1}}\cdot[b]_{M_{q+1}},\]
		so it follows that the action by $r_1[a_1]_{I^{s+1}}+r_2[a_2]_{I^{s+1}}$ is equal to the action by $[r_1a_1+r_2a_2]_{I^{s+1}}$.
	\end{itemize}
	So we have an $R$-module homomorphism
	\[(\op{gr}_IR)_s\to\op{Hom}_R\left((\op{gr}_\mathcal JM)_q,(\op{gr}_\mathcal JM)_{q+s}\right)\]
	is in fact an $R$-module homomorphism. By the tensor-hom adjunction, this induces a morphism
	\[(\op{gr}_IR)_s\otimes_R(\op{gr}_\mathcal JM)_q\to(\op{gr}_\mathcal JM)_{q+s}\into\op{gr}_\mathcal JM\]
	by $[a]_{I^{s+1}}\otimes[b]_{M_{q+1}}\mapsto[ab]_{M_{q+s+1}}$. We see that we can assemble the above morphisms into a large morphism
	\[\bigoplus_{s,q\ge0}\left((\op{gr}_IR)_s\otimes_R(\op{gr}_\mathcal JM)_q\right)\to\op{gr}_\mathcal JM\]
	by $\sum_{s,q}[a_s]_{I^{s+1}}\otimes[b_q]_{M_{q+1}}\mapsto\sum_{s,q}[a_sb_q]_{M_{q+s+1}}$. Because tensor products commute with tensor products, we get a morphism
	\[(\op{gr}_IR)\otimes_R(\op{gr}_\mathcal JM)\to\op{gr}_\mathcal JM\]
	by $\left(\sum_s[a_s]_{I^{s+1}}\right)\otimes\left(\sum_q[b_q]_{M_{q+1}}\right)\mapsto\sum_{s,q}[a_sb_q]_{M_{q+s+1}}$. Using the tensor-hom adjuction once more, we get an $R$-module homomorphism
	\[\op{gr}_IR\to\op{End}_R(\op{gr}_\mathcal JM),\]
	which verifies that we have an action by
	\[\left(\sum_{s\ge0}[a_s]_{I^{s+1}}\right)\cdot\left(\sum_{q\ge0}[b_q]_{M_{q+1}}\right)=\sum_{s,q\ge0}[a_sb_q]_{M_{q+s+1}}.\]
	In particular, we see that $(\op{gr}_\mathcal IR)_s\cdot(\op{gr}_\mathcal JM)_q\subseteq\op{gr}_\mathcal JM)_{q+s}$ by construction of our action: either we can check this directly above as $[a_s]_{I^{s+1}}\cdot[b_q]_{M_{q+1}}=[a_sb_q]_{M_{q+s+1}}$, or we can see it from the original construction. So our action in fact makes a graded module, as we wanted.
\end{proof}
It is somewhat natural to expect that the ``best'' $I$-filtration for a module $M$ is the filtration
\[M\supseteq IM\supseteq I^2M\supseteq\cdots.\]
This specific filtration will turn out to be overly restrictive for our purposes, so we have the following definition.
\begin{definition}[Stable]
	An $I$-filtration of an $R$-module $M$, denoted by
	\[M=M_0\supseteq M_1\supseteq M_2\supseteq\cdots,\]
	is \textit{$I$-stable} if and only if $M_{j+1}=IM_j$ for sufficiently large $j$.
\end{definition}
As a sign that we have done something good, it turns out that stability will communicate nicely with the module structure of $\op{gr}_\mathcal JM$.
\begin{proposition}
	Fix $I\subseteq R$ an ideal. Further, take $M$ to be a finitely generated $R$-module with a stable $I$-filtration $\mathcal J$ by finitely generated modules. Then $\op{gr}_\mathcal JM$ is a finitely generated $\op{gr}_IR$-module.
\end{proposition}
\begin{proof}
	We definition-chase. Let our filtration $\mathcal J$ be
	\[M=M_0\supseteq M_1\supseteq M_2\supseteq\cdots.\]
	Because $\mathcal J$ is $I$-stable, we are promised some $N$ such that $M_{J+p}=I^pM_N$ for $p\ge0$. As such, we choose generators for $M_1,\ldots,M_n$ over $R$ to generate $\op{gr}_\mathcal JM$ as a $\op{gr}_IR$-module.
	
	Thus, it suffices to take generators for $M_0,M_1,\ldots,M_n$ to generate the entire associated graded module.
\end{proof}
This lets us construct our Hilbert function for modules.
\begin{defi}[Hilbert function]
	Fix $R$ a local Noetherian ring where $I$ is the maximal ideal with $M$ a finitely generated $R$-module. Then we define
	\[H_M(n)=\dim_{R/I}\left(I^nM/I^{n+1}M\right).\]
	Note that this definition is well-formed because $M$ is finitely generated.
\end{defi}

\subsection{The Artin--Rees Lemma}
We are finally ready to provide our main result.
\begin{theorem}[Artin--Rees lemma]
	Fix $R$ a Noetherian ring and $I\subseteq R$ an ideal with $M$ a finitely generated $R$-module granted a stable $I$-filtration $\mathcal J$ denoted by
	\[M=M_0\supseteq M_1\supseteq M_2\supseteq\cdots.\]
	Then given a submodule $N\subseteq M$, the induced filtration by $N_k:=M_k\cap N$ is also a stable $I$-filtration.
\end{theorem}
\begin{proof}
	To prove this, we need to introduce the blow-up ring.
	\begin{definition}[Blow-up ring]
		Fix $R$ a ring and $I\subseteq R$ an ideal. Then we define the \textit{blow-up ring} $\op B_IR$ by
		\[\op B_IR:=R\oplus I\oplus I^2\oplus\cdots.\]
	\end{definition}
	Concretely, think about $\op B_IR$ as getting its ring structure from $k[t]$ by something like $k[It]$. This also gives us our grading. In particular, is that $\op B_IR/I\op B_IR\cong\op{gr}_IR$ after tracking everything through.
	\begin{ex}
		Fix $R:=k[x,y]$ and consider $(0,0)\in\AA^2(k)$ with associated maximal ideal $I:=(x,y)\subseteq R$. In this case, our blow-up ring looks like $k[x,y][tx,ty]$. To look at points, we need to look at the ``graded'' spectrum of $\op B_IR$. Here are some ways to do this.
		\begin{itemize}
			\item Look at $Z\subseteq\AA^2(k)\times\PP^1(k)$ to be points $(p,\ell)$ such that $p\in\ell$. We can project $Z\onto\AA^2(k)$ in the natural way. As long as $p\ne0$, there is exactly one pre-image. But if $p=0$, then our pre-image contains all the lines in $\PP^1(k)$! So we have created some ``blowing up'' at the origin.
			\item Alternatively, focus on $k[x,y][tx,ty]$. Set $u=tx$ and $v=ty$ so that we are essentially looking at the ring
			\[\frac{k[x,y,u,w]}{(xw-yv)},\]
			which correspond to the $2\times2$ singular matrices. Taking the quotient by the ``line action'' of matrices
			\[\begin{bmatrix}
				1 & 0 \\
				0 & \lambda
			\end{bmatrix}.\]
			Most of the time, this quotient process will give us $0$, but rarely we will have an entire line after doing the quotient.
		\end{itemize}
	\end{ex}
	We remark that there is also a notion of the blow-up module.
	\begin{definition}[Blow-up ring]
		Fix $R$ a ring and $I\subseteq R$ an ideal. Further, fix $\mathcal J$ an $I$-filtration. Then we define the \textit{blow-up module} $\op B_IM$ by
		\[\op B_IM:=M_0\oplus M_1\oplus M_2\oplus\cdots,\]
		which we can check to be a graded $\op B_IR$-module.
	\end{definition}
	In line with this, we have the following proposition.
	\begin{proposition}
		Fix $R$ a Noetherian ring and $I\subseteq R$ an ideal with $M$ a finitely generated $R$-module granted an $I$-filtration $\mathcal J$ denoted by
		\[M=M_0\supseteq M_1\supseteq M_2\supseteq\cdots.\]
		Then $\op B_\mathcal JM$ is finitely generated as a $\op B_IR$-module if and only if $\mathcal J$ is $I$-stable.
	\end{proposition}
	\begin{proof}
		We omit this proof. It is largely definition-chasing.
	\end{proof}
	We are now ready to attack the proof of the Artin--Rees lemma. Let $\mathcal J'$ be the induced filtration for $N$. From the definition, we see that $\op B_{\mathcal J'}N\subseteq\op B_\mathcal JM$ is a $\op B_IR$-submodule. Now, $\op B_{\mathcal J'}N$ is a submodule of the finitely generated module $\op B_\mathcal JM$ under the Noetherian ring $\op B_IR$, so we are done.
\end{proof}
Here is a nice application.
\begin{restatable}[Krull intersection]{theorem}{krullintersect}
	Fix $R$ a Noetherian ring with an ideal $I$ and finitely generated module $M$. Then
	\[N:=\bigcap_{s\ge0}I^sM\]
	satisfies that there is some $x\in I$ such that $(1-r)N=0$.
\end{restatable}
\begin{proof}
	By construction, we see that $IN=N$, which in particular holds because the standard $I$-filtration of $M$ is stable. Then we showed as a lemma to Nakayama's lemma back in \autoref{lem:nakayamalemlem} that there is an element $r\in I$ with $(1-r)N=0$.
\end{proof}
\begin{corollary}
	Fix $R$ a Noetherian ring with a proper ideal $I$. Further, if $R$ is local or a domain, then
	\[\bigcap_{s\ge0}I^s=0.\]
\end{corollary}
\begin{proof}
	Set
	\[J:=\bigcap_{s\ge0}I^s.\]
	By the proof of the theorem, we get $IJ=J$, which finishes by Nakayama's lemma. When $R$ is a domain, then the theorem gives us some $r\in I$ such that $(1-r)J=0$, but $R$ being a domain will force $J=0$ from this.
\end{proof}
\begin{remark}
	The condition that $R$ is Noetherian is necessary.
\end{remark}
We close with an exercise.
\begin{exe}
	Fix $R$ a local Noetherian ring. If $\op{gr}_IR$ is a domain, then $R$ is a domain.
\end{exe}
\begin{proof}
	The main idea is that $\op{in}f=0$ implies $f=0$, essentially by the corollary above.
\end{proof}