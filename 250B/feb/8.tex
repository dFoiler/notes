% !TEX root = ../notes.tex

Today we discuss primary decomposition.

\subsection{Minimal Primes}

Let's talk a little bit about minimal prime ideals. In particular, suppose that we have a striclty descending chain of prime ideals
\[\mf p_1\supseteq\mf p_2\supseteq\cdots.\]
Because prime ideals are closed under intersection in the chain, Zorn's lemma now promises us a minimal prime ideal.

More generally, we have the following definition.
\begin{definition}[Minimal prime]
	Fix $R$ a ring and $I\subseteq R$ an ideal. Then a prime $\mf p\subseteq R$ is a \textit{minimal prime ideal over $I$} if $\mf p$ is a minimal element of the set of prime ideals containing $I$.
\end{definition}
The above Zorn's lemma argument shows that these minimal primes actually exist.
\begin{proposition}
	Fix $R$ a ring and $I\subsetneq R$ a proper ideal. Then there is a minimal prime $\mf p$ over $I$.
\end{proposition}
\begin{proof}
	We will be a little more careful than the above discussion. Let $\mathcal P$ be the set of primes containing $I$, and we partial order $\mathcal P$ by inclusion. Note that $I$, being a proper ideal, is contained in some maximal ideal $\mf m$, so $\mf m\in\mathcal P$. Thus, $\mathcal P$ is nonempty.

	To apply Zorn's lemma to get out a minimal element, we need to show that any descending chain in $\mathcal P$ is bounded below. Well, suppose that we have a chain
	\[\mf p_1\supseteq\mf p_2\supseteq\mf p_3\supseteq\cdots\]
	of prime ideals containing $I$. Then we set
	\[\mf p:=\bigcap_{k=1}^\infty\mf p_k.\]
	We claim that $\mf p\in S$, which will finish by Zorn's lemma. Because $I\subseteq\mf p_k$ for each $\mf p_k$, we have $I\subseteq\mf p$. Because each $\mf p_k$ is an ideal, the intersection $\mf p$ will be an ideal.
	
	Lastly, to see that $\mf p$ is prime, suppose that $xy\in\mf p$ so that we need to show $x\in\mf p$ or $y\in\mf p$. If $x\in\mf p_k$ for each $\mf p_k$, then $x\in\mf p$, and we are done. Otherwise there exists $\mf p_N$ such that $x\notin\mf p_N$, but then for any $n\ge N$, we have $\mf p_n\subseteq\mf p_N$, so $x\notin\mf p_n$ as well. But because $xy\in\mf p_n$, we see that
	\[y\in\mf p_n\]
	for each $n\ge N$. Because $\mf p_N\subseteq\mf p_m$ for each $m\le N$, we see that in fact $y\in\mf p_k$ for each $k$. Thus $y\in\mf p$.
\end{proof}

In the Noetherian case, our minimal primes are somewhat controlled.
\begin{prop} \label{prop:finminprimes}
	Fix $R$ a Noetherian ring and $I\subseteq R$ an ideal. Then there are only finitely many minimal prime ideals over $I$.
\end{prop}
\begin{proof}[Proof 1]
	Note $I=R$ has no minimal prime ideals over $R$, so we only consider proper ideals. Suppose for the sake of contradiction we have a proper ideal $J$ for which there are infinitely many minimal prime ideals over $J$. Then, because $R$ is Noetherian, we may find a maximal such ideal, and we name it $I$.
	
	Note that if $I$ is prime, then $I$ is the unique minimal prime over $I$, for any minimal prime over $I$ which is contained in $I$ must equal $I$. Thus, $I$ cannot be prime, so there exist $a,b\in R$ such that $a,b\notin I$ while $ab\in I$. Now we look at
	\[I+(a)\qquad\text{and}\qquad I+(b).\]
	In particular, $I$ is a strict subset of both $I+(a)$ and $I+(b)$, so maximality of $I$ forces $I+(a)$ and $I+(b)$ to have only finitely many minimal prime ideals over them. Let $\mathcal P_a$ and $\mathcal P_b$ be the finite sets of minimal primes over $I+(a)$ and $I+(b)$, respectively.

	To finish, we claim that the set of minimal primes over $I$ is a subset $\mathcal P_a\cup\mathcal P_b$, which will be enough because it will show that there are only finitely many minimal primes over $I$, a contradiction. Indeed, suppose $\mf p$ is a minimal prime over $I$. Then $ab\in\mf p$, so $a\in\mf p$ or $b\in\mf p$; without loss of generality, we take
	\[I+(a)\subseteq\mf p.\]
	So we claim that $\mf p\in\mathcal P_a$. Indeed, if a prime $\mf q$ is contained in $\mf p$ while containing $I+(a)$, then $\mf q$ is a prime contained in $\mf p$ while containing $I$, so minimality of $\mf p$ implies $\mf p=\mf q$. This finishes.
\end{proof}
\begin{proof}[Proof 2]
	We can use the machinery of associated primes we have been building. Note that $R/I$ is a finitely generated $R$-module, and $x\in\op{Ann}R/I$ if and only if $x\cdot[r]_I=[0]_I$ for all $r\in R$ if and only if $xr\in I$ for all $r\in R$ if and only if $x\in I$ (by taking $r=1$). Thus,
	\[\op{Ann}R/I=I.\]
	Now, any prime minimal over $I$ will be a minimal prime containing $\op{Ann}R/I$, which by \autoref{prop:minassprimes} will be a prime associated to $R/I$. Thus, the set of minimal primes over $I$ is a subset of
	\[\op{Ass}R/I,\]
	which is finite by \autoref{thm:finass}.
\end{proof}

\subsection{Primary Decomposition for Geometers}
We note that \autoref{prop:finminprimes} has the following corollary.
\begin{corollary} \label{cor:easyprimarydecomp}
	Fix $R$ a Noetherian ring and $I\subseteq R$ an ideal. Then $\rad I$ is the intersection of finitely many prime ideals.
\end{corollary}
\begin{proof}
	By \autoref{prop:radprimes}, we write
	\[\rad I=\bigcap_{I\subseteq\mf p}\mf p.\]
	Letting $\mathcal P$ be the set of minimal primes over $I$, we see that each $\mf p$ over $I$ has some $\mf P_\mf p\in\mathcal P$ such that $\mf P_\mf p\subseteq\mf p$ (for otherwise $\mf p$ ought be minimal). Thus,
	\[\bigcap_{\mf p\in\mathcal P}\mf p\subseteq\bigcap_{I\subseteq\mf p}\mf p\subseteq\bigcap_{I\subseteq\mf p}\mf P_\mf p\subseteq\bigcap_{\mf p\in\mathcal P}\mf p,\]
	so equalities follow. Thus, $\rad I$ is equal to the intersection of the primes in $\mathcal P$, of which there are finitely many by \autoref{prop:finminprimes}.
\end{proof}
We close with a geometric interpretation of \autoref{cor:easyprimarydecomp}. We have the following definition.
\begin{defi}[Irreducible]
	Fix $X\subseteq\AA^n(k)$ an affine algebraic set. Then $X$ is \textit{irreducible} if and only if $I(X)\subseteq k[x_1,\ldots,x_n]$ is prime; i.e., $A(X)$ is an integral domain. We might call $X$ a \textit{variety} in this case.
\end{defi}
As an application, consider any algebraic set $X\subseteq\AA^n(k)$. Then we know that $I(X)$ is radical, so \autoref{cor:easyprimarydecomp} is saying that we can decompose
\[I(X)=\bigcap_{\mf p\in\mathcal P}\mf p,\]
where $\mathcal P$ is the finite collection of minimal primes over $I(X)$. Taking zero-sets everywhere, we see that\footnote{If $x\in Z(I\cap J)$ while $x\notin Z(J)$, then there is $g\in J$ such that $g(x)\ne0$; but $x\in Z(I\cap J)$ means that each $f\in I$ has $(fg)(x)=0$, forcing $f(x)=0$. Conversely, if $x\in Z(I)$ (without loss of generality), $x$ will also vanish on $I\cap J$.} $Z(I\cap J)=Z(I)\cup Z(J)$, so
\[X=Z(I(X))=Z\left(\bigcap_{\mf p\in\mathcal P}\mf p\right)=\bigcup_{\mf p\in\mathcal P}Z(\mf p).\]
Now, $I(Z(\mf p))=\rad\mf p$ by the Nullstellensatz, and $\rad\mf p=\mf p$ be primality, so the point of the above is that we have written an arbitrary algebraic set $X$ as a union of finitely many irreducible algebraic sets $Z(\mf p)$.
\begin{corollary}
	Fix $X\subseteq\AA^n(k)$ an algeraic set. Then $X$ can be written as the union of finitely many irreducible algebraic sets.
\end{corollary}
\begin{proof}
	This follows from the above discussin.
\end{proof}
And now let's see a physical example.
% a minimal prime $\mf p$ over $I(X)$, of which we know there are finitely many, so we call them
% \[\{\mf p_1,\ldots,\mf p_n\}.\]
% However, the intersection of all these primes will be the intersection of all the primes containing $I(X)$, which is the radical of $I(X)$, which is $I(X)$. Speaking geometrically, this means that any algebraic set is a finite union of maximal irreducible algebraic set.
\begin{exe}
	Fix $I:=\left(yx-x^3\right)\subseteq k[x_1,x_2]$, and we decompose $Z(I)\subseteq\AA^2(k)$ into irreducibles.
\end{exe}
\begin{proof}
	Well, $yx-x^3=0$ if and only if $x=0$ or $y-x^2=0$, so
	\[I=(x)\cap\left(y-x^2\right).\]
	So here is the image of $Z(I)$.
	\begin{center}
		\begin{asy}
			unitsize(1cm);
			import graph;
			real f(real x)
			{
				return x*x;
			}
			draw(graph(f,-1.5,1.5), red);
			label("$\left(y-x^2\right)$", (1,1), SE, red);
			draw((0,2.25)--(0,-0.5), blue);
			label("$(x)$", (0,1.8), E, blue);
		\end{asy}
	\end{center}
	Now we note $(x)$ and $\left(y-x^2\right)$ are prime ideals.
\end{proof}

\subsection{Primary Grab-Bag}
In \autoref{cor:easyprimarydecomp}, we saw that when $I$ was a radical ideal, we could write it as a finite intersection of prime ideals. Primary decomposition provides us with a general theory to do something similar for arbitrary modules. Let's start building towards that.

We pick up the following definitions.
\begin{definition}[\texorpdfstring{$\mf p$}{p}-primary]
	Fix $M$ a finitely generated module over $R$ a Noetherian ring. Then a submodle $N\subseteq M$ is \textit{$\mf p$-primary} if and only if $\op{Ass}M/N=\{\mf p\}$.
\end{definition}
\begin{definition}[\texorpdfstring{$\mf p$}{p}-coprimary]
	Fix $M$ a finitely generated module over $R$ a Noetherian ring. Then $M$ is \textit{$\mf p$-coprimary} if and only if $\op{Ass}M=\{\mf p\}$.
\end{definition}
In other words, $N\subseteq M$ is $\mf p$-primary if and only if $M/N$ is $\mf p$-coprimary.

We would like some more concrete conditions for being $\mf p$-primary.
\begin{proposition}
	Fix $M$ a finitely generated module over $R$ a Noetherian ring. Then the following are equivalent.
	\begin{listalph}
		\item $M$ is $\mf p$-coprimary.
		\item $\mf p$ is the unique minimal prime over $\op{Ann}M$, and $\mf p$ contains $\op{Ann}m$ for each $m\in M$. % \todo{I don't think we have to assume that p is unique?}
		\item $\mf p^n\subseteq\op{Ann}M$ for some positive integer $n$, and $\mf p$ contains $\op{Ann}m$ for each $m\in M$.
	\end{listalph}
\end{proposition}
\begin{proof}
	We take our implications one at a time.
	\begin{itemize}
		\item We show that (a) implies (b). We are given that $\op{Ass}M=\{\mf p\}$. But we know that any prime minimal containing $\op{Ann}M$ will be associated (by \autoref{prop:minassprimes}), so $\mf p$ must be a minimal prime over $\op{Ann}M$.

		Additionally, we recall from \autoref{prop:unionass} that
		\[\bigcup_{m\in M\setminus\{0\}}\op{Ann}m=\bigcup_{\mf q\in\op{Ass}M}\mf q,\]
		but $\op{Ass}M=\{\mf p\}$, so the result follows.

		\item Recall that $\rad\op{Ann}M$ is the intersection of all primes containing $\op{Ann}M$, so $\mf p=\rad\op{Ann}M$, so $\mf p^n\subseteq\op{Ann}M$ for some positive integer $n$.

		% Technically we must check that $\mf p$ is the unique minimal prime, so suppose 
		\item We show that (c) implies (a). If $\mf p^n\subseteq\mf q\subseteq\mf p$ for some prime $\mf q$ containing $\op{Ann}M$, then $\mf q=\mf p$. So $\mf p$ is the unique minimal prime over $\op{Ann}M$, and the hypothesis implies that there are no other minimal primes.
		% \todo{}
		\qedhere
	\end{itemize}
\end{proof}
\begin{corollary}
	An ideal $I\subseteq R$ is $\mf p$-primary if and only if $\mf p^n\subseteq I$ and, for all $a\notin\mf p$, we have $ab\in I$ implies $b\in\mf p$.
\end{corollary}
\begin{proof}
	This follows directly from the proposition.
\end{proof}

\subsection{Primary Decompositon}
And here is our main result.
\begin{theorem}[Primary decomposition, I]
	Fix $M$ a finitely generated module over a Noetherian ring $R$. Then every submodule $N\subseteq M$ is the intersection of finitely many primary submodules.
\end{theorem}
\begin{proof}
	The key to this result is to instead talk about irreducible decomposition.
	\begin{definition}[Irreducible]
		Fix $M$ a module over a ring $R$. Then a submodule $N\subseteq M$ is \textit{irreducible} if and only if $N=N_1\cap N_2$ implies $N=N_1$ or $N=N_2$.
	\end{definition}
	It will turn out that irreducible implies primary, but we do not know this.

	Here is the key claim.
	\begin{lemma}
		Fix $M$ a finitely generated module over a Noetherian ring $R$. Then every submodule $N\subseteq M$ is the intersection of finitely many irreducible submodules.
	\end{lemma}
	\begin{proof}
		Suppose for the sake of contradiction that the statement is false. Because $M$ is a Noetherian module, we can find a maximal such submodule $N$. Note that $N\ne M$ because $M$ is irreducible.

		However, $N$ is not irreducible, so we can write $N=N_1\cap N_2$ for $N\subsetneq N_1,N_2$. But by hypothesis on $N$, we can write $N_1$ and $N_2$ as intersections of irreducible modules, so $N$ is also the intersection of irreducible modules, so we are done.
	\end{proof}
	\begin{remark}[Nir]
		If you squint really hard, this is essentially the proof of existence of prime factorizatins.
	\end{remark}
	And now we can finish up.
	\begin{lemma}
		Fix $M$ a finitely generated module over a Noetherian ring $R$. Then any irreducible submodule $N\subsetneq M$ is primary.
	\end{lemma}
	\begin{proof}
		We know that $\op{Ass}M/N$ is nonempty because $M\ne N$, so we need to show that there is exactly one prime; we proceed by contrapositin. So suppose $\mf p,\mf q\in\op{Ass}M/N$ which are distinct. But then by \autoref{lem:betterassociated}, we get embeddings
		\[\iota_\mf p:R/\mf p\into M/N\qquad\text{and}\qquad \iota_\mf q:R/\mf p\into M/N.\]
		However, the annihilator of any nonzero element in $\im\iota_\mf p$ and $\im\iota_\mf q$ will be $\mf p$ and $\mf q$ respectively, so $\im\iota_\mf p\cap\im\iota_\mf q=0$. So we can write
		\[N=(\im\iota_\mf p+N)\cap(\im\iota_\mf q+N)\]
		to break that $N$ is irreducible. (These are strictly larger because of where $\im\iota_\mf p$ and $\im\iota_\mf q$ are supposed to map to.)
	\end{proof}
	The above two lemmas finish the theorem.
\end{proof}
Let's try to make primary decomposition a little more canonical.
\begin{theorem}[Primary decomposition, II]
	Fix $M$ a finitely generated module over a Noetherian ring $R$. Then write some submodule $N\subseteq M$ as
	\[N=\bigcap_{i=1}^nN_i\]
	such that $N_i$ is $\mf p_i$-primary. Then the following are true.
	\begin{listalph}
		\item We have $\op{Ass}M/N\subseteq\{\mf p_1,\ldots,\mf p_n\}$.
		\item If we cannot remove any $N_i$ from the decomposition, then equality in (a) holds.
		\item If $n$ is as small as possible, then each $\mf p_i$ is unique.
	\end{listalph}
\end{theorem}
\begin{proof}
	We show these one at a time. For psychological reasons, we note that we may swap $M$ and $N$ with $M/N$ and $(0)$ to assume that $N=(0)$ (without loss of generality).
	\begin{listalph}
		\item Note that we have the sequence
		\[M\into\bigoplus_{i=1}^nM/N_i\to 0\]
		with kernel $N=(0)$. It follows that $\op{Ass}M$ is contained in $\bigcup_{i=1}^n\op{Ass}M/N_i=\{\mf p_1,\ldots,\mf p_n\}$.
		\item We are given that, for each $j$, we have
		\[K_j:=\bigcap_{i\ne j}N_i\ne0.\]
		Now, $N_j\cap K=0$, so we can embed $K_j$ into $M/N_j$, so $\op{Ass}M/K_j=\{\mf p_j\}$, making $K_j$ into a $\mf p_j$-primary module. It follows that $\mf p_j\in\op{Ass}M$.
		\item If there are any redundancies, then we intersect all of the redundancies for some prime $\mf p$ to remove the redundancies, so there will be at most one $\mf p$-primary ideal for any prime $\mf p$ when $N$ is minimal.
		\qedhere
	\end{listalph}
\end{proof}
Primary decomposition also behaves with localization.
\begin{theorem}[Primary decomposition, III]
	Fix $M$ a finitely generated module over a Noetherian ring $R$. Then write some submodule $N\subseteq M$ as
	\[N=\bigcap_{i=1}^nN_i\]
	such that $N_i$ is $\mf p_i$-primary. If $U\subseteq R$ is some multiplicatively closed subset, then
	\[N\left[U^{-1}\right]=\bigcap_{\substack{i=1\\\mf p_i\cap U=\emp}}^n\mf p_i\left[U^{-1}\right].\]
\end{theorem}
\begin{proof}
	We omit this proof. The main point is that all of the primes we avoided are actually fully $M\left[U^{-1}\right]$, and localization preserves intersections.
\end{proof}

\subsection{Examples of Primary Decomposition}
Let's do some examples.
\begin{example}
	For any nonzero integer $n\in\ZZ\setminus\{0\}$, we can write
	\[(n)=\prod_{p\text{ prime}}\left(p^{\alpha_p}\right),\]
	for some exponents $\alpha_p$. This is our primary decomposition, and it is unique.
	% \todo{what if n=0 here?}
\end{example}
\begin{example}
	In general, if $R$ is a unique factorization domain, then any $f\in R$ will have
	\[f=up_1^{d_1}\cdots p_n^{d_n}\]
	for some unit $u$, primes $p_\bullet$, and exponents $d_\bullet$. Then
	\[(f)=\bigcap_{i=1}^n\left(p_i^{d_i}\right).\]
\end{example}
In fact, we have the following statement.
\begin{proposition}
	Fix $R$ a Noetherian domain. Then $R$ is a unique factorization domain if and only if every prime ideal over a principal ideal is principal.
\end{proposition}
\begin{proof}
	We omit this proof. The main idea to the backwards direction is to take a minimal prime ideal $(p)$ to sufficiently high power to read off the exponent.
\end{proof}

We remark that primary decomposition is not unique, in general, however.
\begin{example}
	Consider $R:=k[x,y]/\left(x^2,xy\right)$. As a $k$-module, this has basis $\{1,x,y,y^2,y^3\ldots\}$. Now, we see that we can write
	\[(x)\cap\left(y^n\right)=\left(xy^n\right)=(0)\]
	for any $n\in\NN$. This is a primary decomposition: in particular, $R/\left(y^n\right)$ is a local ring (notably, all the basis elements other than $1$ nilpotent, so $R$ minus their $k$-span is the unique maximal ideal), and our associated prime is $(x,y)R/\left(y^n\right)$. Additionally, $R/(x)$ will have unique maximal ideal $(x,y)/(x)$, for similar reasons.
\end{example}
The point is that the above example provides ``lots'' of different minimal primary decompositions.

\subsection{Graded Rings}
We will want to consider primary decomposition for graded rings because later in life we will want to talk about projective space.
\begin{proposition}
	Fix $R=R_0\oplus R_1\oplus\cdots$ a graded ring and $M$ a graded module over $R$. Suppose that we have $m\in M$ and $\mf p:=\op{Ann}m$ a prime ideal. Then $\mf p$ is a graded ideal of $R$.
\end{proposition}
In other words, associated primes of a graded module are graded.
\begin{proof}
	We have to show that
	\[\mf p=\bigoplus_{i=1}^\infty(R_i\cap\mf p).\]
	So we write, for some $f\in\mf p$, that
	\[f=\sum_{i=1}^p f_i,\]
	where $f_i\in R_{d_i}$, and $d_1<\cdots<d_n$. By induction, it will be enough to show that $f_1\in\mf p$ and then consider $f-f_1$ to continue. 
	
	Well, we may write
	\[m=\sum_{j\in\ZZ}^qm_j,\]
	where $m_j\in R_{e_j}$. Because $fm=0$, the term of lowest degree, which is $f_1m_1$, must itself vanish, so
	\[f_1m=\sum_{j=2}^pf_1m_j.\]
	Thus $\mf p\subseteq\op{Ann}f_1m$, so we set $I:=\op{Ann}f_1m$. If $\mf p=\op{Ann}f_1m$, then we are done because we have effectively removed our decomposition. Otherwise, we can find $g\in I\setminus\mf p$. But then $gf_1$ annihilates $m$, so $gf_1\in\mf p$, so $f_1\in\mf p$, and we are done.
\end{proof}