% !TEX root = ../notes.tex








So it begins.

\subsection{Unique Factorization}
\begin{warn}
	I am missing approximately the first third of class, so the remaining notes will likely be scattered as well.
\end{warn}
We have the following definition.
\begin{definition}[Irreducible, prime]
	Fix $R$ a ring and $r\in R$ an element.
	\begin{itemize}
		\item We say that $r\in R$ is \textit{irreducible} if and only if $r=ab$ for $a,b\in R$ implies that one of $a$ or $b$ is a unit.
		\item We say that $r\in R$ is \textit{prime} if and only if $r$ is not a unit, not zero, and $(r)$ is a prie ideal: $r\mid ab$ implies $r\mid a$ or $r\mid b$.
	\end{itemize}
\end{definition}
This gives rise to the following important definition.
\begin{definition}[Unique factorization domain]
	Fix $R$ an integral domain. Then $R$ is a \textit{unique factorization domain} if and only if all nonzero elements of $R$ have a unique factorization into irreducible elements.
\end{definition}
\begin{example}
	The ring $\ZZ$ is a unique factorization domain.
\end{example}
Note there are two things to check: that the factorization exists and that it exists.
\begin{example}
	Consider the subring $R:=k\left[x^2,xy,y^2\right]\subseteq k[x,y]$. Here $x^2,xy,y^2$ are all irreducibles because there are no linear polynomials, so the only possible decompositions must include a degree-$0$ (i.e., unit) polynomial. However, they are not prime:
	\[x^2\mid xy\cdot xy\]
	while $x^2$ does not divide $xy$.
\end{example}
The following condition will provide a more 
\begin{definition}[Ascending chain condition]
	Given a collection of sets $\mathcal S$, we say that $\mathcal S$ \textit{has the ascendinc chain condition} (ACC) if and only every chain of sets in $\mathcal S$ must eventually stablize.
\end{definition}
\begin{example}[ACC for principal ideals]
	A ring $R$ has the ascending chain condition for principal ideals if and only if every ascending chain of principal ideals
	\[(a_1)\subseteq (a_2)\subseteq(a_3)\subseteq\cdots\]
	has some $N$ such that $(a_N)=(a_n)$ for $n\ge N$.
\end{example}
Now, the fact that $\ZZ$ is a unique factorization domain roughly comes from the fact that $\ZZ$ is a principal ideal domain.
\begin{theorem}
	Fix $R$ a ring. Then $R$ is a principal ideal domain implies that $R$ is a unique factorizatino domain.
\end{theorem}
\begin{proof}
	We start by showing that $R$ has the ascending chain for principal ideals. Indeed, suppose that we have some ascending chain of principal ideals
	\[(a_1)\subseteq (a_2)\subseteq(a_3)\subseteq\cdots.\]
	Then the key idea is to look at the union of all these ideals, which will be an ideal by following the chain condition. However, $R$ is a principal ideal domain, so there exists $b\in R$ such that
	\[\bigcup_{k=1}^\infty(a_k)=(b).\]
	However, it follows $b\in(a_N)$ for some $N$, in which case $(a_n)=(a_N)$ for each $n\ge N$.

	Next we show that all prime elements are irreducible elements. The main idea is that prime ideals correspond to maximal ideals correspond to ``irreducible ideals.'' We will not write this out; we showed this last semester.
\end{proof}

\subsection{Digression on Gaussian Integers}
As an aside, the study of unique factorization came from Gauss's study of the Gaussian integers.
\begin{definition}[Gaussian integers]
	The \textit{Gaussian integers} are the ring
	\[\ZZ[i]:=\{a+bi:a,b\in\ZZ\}.\]
\end{definition}
One can in fact check that $\ZZ[i]$ is a principal ideal domain, which implies that $\ZZ[i]$ is a unique factorization domain. The correct way to check that $\ZZ[i]$ is a principal ideal domain is to show that it is Euclidean.
\begin{lemma}
	The ring $\ZZ[i]$ is Euclidean, where our norm is $\op N(a+bi):=a^2+b^2$. In other words, given $\alpha,\beta\in\ZZ[i]$, we need to show that there exists $q\in\ZZ[i]$ such that
	\[a=bq+r\]
	where $r=0$ or $\op N(r)<\op N(\beta)$.
\end{lemma}
\begin{proof}
	The main idea is to view $\ZZ[i]\subseteq\CC$ geometrically as in $\RR^2$. We may assume that $|\beta|\le|\alpha|$, and then it suffices to show that in this case we may find $q$ so that $a-bq$ has smaller norm than $a$. Well, for this it suffices to look at $a+b,a-b,a+ib,a-ib$; the proof that one of these works essentially boils down to the following image.
	\begin{center}
		\begin{asy}
			unitsize(0.9cm);
			pair a = (3,4);
			pair b = (1,2);
			draw((0,0) -- a, EndArrow);
			draw(a -- a+b, EndArrow);
			draw(a -- a-b, EndArrow);
			draw(a -- a+(-2,1), EndArrow);
			draw(a -- a-(-2,1), EndArrow);
			dot("$a$", a, ENE);
		\end{asy}
	\end{center}
	Note that at least one of the endpoints here has norm smaller than $a$.
\end{proof}
What about the primes? Well, there is the following theorem which will classify.
\begin{theorem}[Primes in \texorpdfstring{$\ZZ[i]$}{Z[i]}]
	An element $\pi:=a+bi\in\ZZ[i]$ is \textit{prime} if and only if $\op N(\pi)$ is a $1\pmod4$ prime, $(pi)=(1+i)$, or $(\pi)=(p)$ for some prime $p\in\ZZ$ such that $p\equiv3\pmod4$.
\end{theorem}
We will not fully prove this; it turns out to be quite hard, but we can say small things: for example, $3\pmod4$ primes $p$ remain prime in $\ZZ[i]$ because it is then impossible to solve
\[p=a^2+b^2\]
by checking$\pmod4$.
\begin{remark}
	This sort of analysis of ``sums of squares'' can be related to the much harder analysis of Fermat's last theorem, which asserts that the Diophantine equation
	\[x^n+y^n=z^n\]
	for $xyz\ne0$ integers such that $n>2$.
\end{remark}

\subsection{Noetherian Rings}
We have the following definition.
\begin{definition}[Noetherian ring]
	A ring $R$ is said to be \textit{Noetherian} if its ideals have the ascending chain condition.
\end{definition}
There are some equivalent conditions to this.
\begin{proposition}
	Fix $R$ a ring. The following conditions are equivalent.
	\begin{itemize}
		\item $R$ is Noetherian.
		\item Every ideal of $R$ is finitely generated.
	\end{itemize}
\end{proposition}
\begin{proof}
	We will show one direction, if $R$ has an infinitely generated ideal
	\[I:=(a_1,a_2,a_3,\ldots),\]
	then we have the non-stabilizing ascending chain
	\[(a_1)\subseteq(a_2)\subseteq(a_3)\subseteq\cdots.\]
	We leave the other direction as an exercise.
\end{proof}
A large class of rings turn out to be Noetherian, and in fact oftentimes Noetherian rings can build more Noetherian rings.
\begin{prop}
	Fix $R$ a Noetherian ring and $I\subseteq R$ an ideal. Then $R/I$ is also Noetherian.
\end{prop}
\begin{proof}
	Any chain of ideals in $R/I$ can be lifted to a chain in $R$ by taking pre-images along $R\onto R/I$. Then the chain must stabilize in $R$, so they will stabilize back down in $R/I$ as well.
\end{proof}
However, things are not always so nice. It is not true that $R_1\subseteq R_2$ with $R_2$ Noetherian implies that $R_1$ is Noetherian.\todo{example}
\begin{nex}
	
\end{nex}
Here is another way to generate Noetherian rings.
\begin{theorem}[Hilbert basis] \label{thm:hilbasis}
	If $R$ is a Noetherian ring, then $R[x]$ is also a Noetherian ring.
\end{theorem}
\begin{corollary}
	By induction, if $R$ is Noetherian, then $R[x_1,x_2,\ldots,x_n]$ is Notherian for any finite $n$.
\end{corollary}
\begin{warn}
	It is not true that $R[x_1,x_2,\ldots]$ is Noetherian: the ideal $(x_1,x_2,\ldots)$ is not finitely generated!
\end{warn}
\begin{proof}[Proof of \autoref{thm:hilbasis}]
	The idea is to use the degree of polynomials to measure size. Fix $I\subseteq R[x]$ an ideal, and we apply the following inductive process.
	\begin{itemize}
		\item Pick up $f_1\in I$ of minimal degree in $I$.
		\item If $I=(f_1)$ then stop. Otherwise find $f_2\in I\setminus(f_1)$ of minimal degree.
		\item In general, if $I\ne(f_1,\ldots,f_n)$, then pick up $f_{n+1}\in I\setminus(f_1,\ldots,f_n)$ of minimal degree.
	\end{itemize}
	Importantly, we do not know that there are only finitely many $f_\bullet$ yet.

	Now, look at the leading coefficients of the $f_\bullet$, which we name $a_\bullet$. However, the ideal
	\[(a_1,a_2,\ldots)\subseteq R\]
	must be finitely generated, so there is some finite $N$ such that
	\[(a_1,a_2,\ldots)=(a_1,a_2,\ldots,a_N).\]
	To finish, we claim that
	\[I\stackrel?=(f_1,f_2,\ldots,f_N).\]
	Well, suppose for the sake of contradiction that we can find some $g\in I\setminus(f_1,f_2,\ldots,f_N)$ of least degree. We must have $\deg g\deg f_\bullet$ for each $f_\bullet$, or else we contradict the construction of $f_\bullet$ as being least degree.

	To finish, let $a$ be the leading coefficient of $g$, which must be an element of $(a_1,a_2,\ldots)$\todo{why?}, but then we can look at
	\[g-\sum_{k=1}^Nc_ka_kx^{(\deg g)-(\deg f_k)}f_k,\]
	which will have degree smaller than $g$ while not being in $I$ either, which contradicts the minimality of the degree of $g$.
\end{proof}

\subsection{Modules}
To review, we pick up the following definition.
\begin{definition}[Module]
	Fix $R$ a ring. Then $M$ is an abelian group with an $R$-action. Explicitly, we have the following properties. Fix any $a,b\in R$ and $m,n\in M$.
	\begin{itemize}
		\item $1_Rm=m$.
		\item $a(bm)=(ab)m$.
		\item $(a+b)m=am+bm$.
		\item $a(m+n)=am+an$.
	\end{itemize}
\end{definition}
\begin{example}
	Any ideal $I\subseteq R$ is an $R$-module. In fact, ideals exactly correspond to the $R$-submodules of $R$.
\end{example}
\begin{example}
	Given any two $R$-module $M$ with a submodule $N\subseteq M$, we can form the quotient $N\subseteq M$.
\end{example}
Modules also have a notion of being Noetherian.
\begin{definition}[Noetherian module]
	We say that an $R$-module $M$ is \textit{Noetherian} if and only if all $R$-submodules of $M$ are finitely generated. Equivalently, the submodules of $M$ have the ascending chain condition.
\end{definition}
Modules have some interesting ways to create new Noetherian submodules. Here is one important way.
\begin{proposition}
	Fix a short exact sequence
	\[0\to A\to B\to C\to 0\]
	of $R$-modules. Then $B$ is Noetherian if and only if $A$ and $C$ are both Noetherian.
\end{proposition}
\begin{proof}
	We will not prove this here. The forwards direction is not too hard: $A$ being a submodule of a Noetherian module implies that all of its submodules still must be finitely generated. Then for $C$, one considers the quotient in the same way we did for rings.
\end{proof}
Because we like Noetherian rings, the following is a reassuring way to make Noetherian modules.
\begin{proposition}
	Every finitely generated $R$-module over a Noetherian ring $R$ is Noetherian.
\end{proposition}
\begin{proof}
	If $M$ is finitely generated, then there exists some $n\in\NN$ and surjective morphism
	\[\varphi:R^n\onto M.\]
	Now, because $R$ is Noetherian, $R^n$ will be Noetherian by an induction: the inductive step looks at the short exact sequence
	\[0\to R\to R^n\to R^{n-1}\to 0.\]
	Thus, $M$ is the quotient of a Noetherian ring $R^n$ and hence Noetherian.
\end{proof}
Here is the analogous result for algebras.
\begin{definition}[Algebra]
	A ring $S$ is an $R$-module if and only if there is an embedding $R\into S$.
\end{definition}
\begin{prop}
	Fix $R$ a Noetherian ring. Then any finitely generated $R$-algebra is Noetherian. Equivalently, we may think of an $R$-algebra as a ring with an $R$ action.
\end{prop}
\begin{proof}
	Saying that $S$ is a finitely generated $R$-algebra is the same as saying that there is a surjective morphism
	\[\varphi:R[x_1,\ldots,x_n]\onto S\]
	for some $n\in\NN$. But then $S$ is the quotient of a $R[x_1,\ldots,x_n]$, which is Noetherian by \autoref{thm:hilbasis}, so $S$ is Noetherian as well.
\end{proof}

\subsection{Invariant Theory}
In our discussion, fix $k$ a field of characteristic $0$, and let $G$ be a finite group or $\op{GL}_n(k)$ (say). Now, suppose that we have a map
\[G\to\op{GL}_n(k).\]
Then this gives $k[x_1,\ldots,x_n]$ a $G$-action by writing $gf(\vec x):=f(g^{-1}\vec x).$ The central question of invariant theory is then as follows.
\begin{ques}[Invariant theory]
	Fix everything as above. Then can we describe $k[x_1,\ldots,x_n]^G$?
\end{ques}
By checking the group action, it is not difficult to verify that $k[x_1,\ldots,x_n]^G$ is a subring of $k[x_1,\ldots,x_n]$. For brevity, we will write $R:=k[x_1,\ldots,x_n]$ and $S:=R^G$.

Here is a result of Hilbert.
\begin{theorem}[Hilbert]
	Fix everything as above. Then the ring $k[x_1,\ldots,x_n]^G$ is Noetherian.
\end{theorem}
\begin{proof}
	Please read this on your time; we skipped it in class. The main ingredient is the Hilbert basis theorem.
\end{proof}
The main example here is as follows.
\begin{example}
	Consider the action of $S_n$ on $R=k[x_1,\ldots,x_n]$ by permuting the coordinates: $\sigma\in S_n$ acts by $\sigma\cdot x_m:=x_{\sigma n}$. Then the polynomials in $S=R^G$ are the \textit{homogeneous} polynomials. The fundamental theorem of symmetric polynomials tells us that
	\[R^G=k[e_1,e_2,\ldots,e_n],\]
	where the $e_\bullet$ are \textit{elementary symmetric functions}. Explicitly,
	\[e_m:=\sum_{\substack{T\subseteq\{1,\ldots,n\}\\\#S=m}}\prod_{t\in T}x_t.\]
\end{example}
\begin{remark}
	Yes, I am in fact running out of letters. Thank you for asking.
\end{remark}
Here is more esoteric example.
\begin{example}
	Fix $R:=k[x,y]$ and $G=\{1,g\}\cong\ZZ/2\ZZ$. Then we define our $G$-action by
	\[g\cdot x=-x\qquad\text{and}\qquad g\cdot y=-y.\]
	Then $R^G$ consists of all polynomials $f(x,y)$ such that $f(x,y)=f(-x,-y)$; i.e., these are the polynomials in $R$ that all have even degree. It follows after some pushing that
	\[R^G=k\left[x^2,xy,y^2\right].\]
	To see that this ring is Noetherian, we note that there is a surjection
	\[\varphi:k[u,v,w]\to k\left[x^2,xy,y^2\right]\]
	taking $u\mapsto x^2$ and $v\mapsto xy$ and $w\mapsto y^2$. Thus, $R$ is the quotient of a Noetherian ring and hence Noetherian itself. In fact, we can check that $\ker\varphi=\left(uw-v^2\right)$.
\end{example}
Next class we will start talking about the Nullstellensatz, which has connections to algebraic geometry.