% !TEX root = ../notes.tex

We continue following the Eisenbud machine.

\subsection{Nullstellensatz}
This subsection is very important. And because it is important, its name is in German (which was the language of German).

To begin our discussion, we start with some geometry.
\begin{definition}[Affine space]
	Given a field $k$ and nonnegative integer $n$, we define $d$-dimensional \textit{affine space} over $k$ to be $\AA^d(k):=k^d.$
\end{definition}
Now, given affine space $\AA^n(k)$, we are interested in studying subsets which are solutions to some set of polynomial equations
\[f_1,\ldots,f_n\in k[x_1,\ldots,x_d].\]
This gives rise to the following definition.
\begin{definition}[Algebraic]
	A subset $X\subseteq\AA^n(k)$ is \textit{algebraic} if and only if it is the set of solutions to some system of polynomials equations $f_1,\ldots,f_n\in k[x_1,\ldots,x_d]$.
\end{definition}
\begin{example}
	The hyperbola
	\[\left\{(x,y)\in\RR^2:x^2-y^2-1=0\right\}\]
	is an algebraic set.
\end{example}
\begin{example}
	The set $\emp\subseteq\RR^1$ is algebraic becasue it is the set of solutions to the equation $x^2+1=0$ in $\RR$.
\end{example}
The above example is a little disheartening because it feels like $x^2+1$ really ought to have a solutionm, namely $i\in\CC$. So with this in mind, we make the following convention.
\begin{warn}
	In the following discussion, $k$ will always be an algebraically closed field.
\end{warn}

Now, the story so far is that we can take a set of polynomials and make algebraic sets. We can in fact go in the opposite direction.
\begin{definition}[\texorpdfstring{$I(X)$}{I(X)}]
	If $X\subseteq$ is an algebraic set, we define
	\[I(X):=\{f\in k[x_1,\ldots,x_n]:f(X)=0\}.\]
\end{definition}
Notice that every ideal $I$ can arise as some $I(X)$, for if $f^m(X)=0$, then $f(X)=0$, so $I$ will satisfy the property that $f^m\in I$ implies $f\in I$. With this in mind, we have the following construction.
\begin{definition}[Radical]
	Given an $R$-ideal $I$, we define the \textit{radical of $I$}
	\[\op{rad}I:=\left\{x\in R:x^n\in I\text{ for some }n\ge1\right\}.\]
	If $I=\op{rad}I$, then we call $I$ a \textit{radical ideal}.
\end{definition}
Often, $\op{rad}I$ is a larger ideal.
\begin{remark}
	Alternatively, an $R$-ideal $I$ is \textit{radical} if and only if $R/I$ is reduced; i.e., if and only if $R/I$ has no nonzero nilpotent elments. This is approximately as easy to show as the equivalent condition for prime ideals.
\end{remark}
With all of the machinery we have in place, we can now state Hilbert's Nullstellensatz.
\begin{theorem}[Nullstellensatz, I]
	Fix $k$ an algebraically closed field. Then there is a bijection between radical ideals and (affine) algebraic sets.
\end{theorem}
So far we have defined a map from algebraic sets to radical ideals by $X\mapsto I(X)$. The reverse map is as follows.
\begin{definition}[\texorpdfstring{$Z(I)$}{Z(I)}]
	Given a subset $S\subseteq k[x_1,\ldots,x_n]$, we define the \textbf{zero set} of $I$ by
	\[Z(S):=\{x\in\AA^n(k):f(x)=0\text{ for all }f\in I\}.\]
\end{definition}
Note that replacing $S$ with the ideal it generates $(S)$ makes no difference to $Z(S)$, so we will mostly consider $S$ to be an ideal.

With these maps in hand, we can restate the Nullstellensatz.
\begin{theorem}[Nullstellensatz, II]
	Fix $k$ an algebraically closed fields. Then for ideals $I\subseteq k[x_1,\ldots,x_n]$ by
	\[I(Z(I))=\op{rad}I.\]
\end{theorem}
\begin{remark}
	Yes, it is important that $k$ is algebraically closed here.
\end{remark}
\begin{example}
	We have that $I(Z(R))=R$ because $Z(R)=\emp$.
\end{example}

\subsection{Commends on Affine Space}
We would like to understand $\AA^n(k)$; one way to do this is by a topology, so here is one way we can do this.
\begin{definition}[Zariski topology, I]
	Given affine space $\AA^n(k)$, we define the \textit{Zariski topology} as having closed sets which are the algebraic sets.
\end{definition}
There are some things to check that we actually get a topology. Here are the checks.
\begin{itemize}
	\item The empty set is the set of solutions to the equation $1=0$.
	\item The full space is the set of solutions to the equation $0=0$.
	\item Arbitrary intersection of closed sets is closed: given algebraic sets $X(I)$ for ideal $I\subseteq\mathcal I$, we note
	\[\bigcap_{I\in\mathcal I}X(I)=X\left(\sum_{I\in\mathcal I}I\right).\]
	\item Finite unions of closed sets are closed: given algebraic sets $X(I_1),\ldots,X(I_n)$, we note
	\[\bigcup_{i=1}^nX(I_i)=X\left(\prod_{i=1}^nI_i\right).\]
\end{itemize}
We can also understand algebraic sets $X\subseteq\AA^n(k)$ by their ring of functions. Here is the way we understand this.
\begin{definition}
	Given an algebraic set $X\subseteq\AA^n(k)$, we define the \textit{coordinate ring} on $X$ as
	\[A(X):=k[x_1,\ldots,x_n]/I(X).\]
\end{definition}
Note that, because $I(X)$ is a radical ideal, the ring $A(X)$ will be reduced.

\subsection{Corollaries of the Nullstellensatz}
To convince us that the Nullstellensatz is important, here are some nice corollaries.
\begin{corollary}
	Suppose that a system of polynomial equations
	\[\begin{cases}
		f_1(x_1,\ldots,x_n) = 0, \\
		\quad\vdots \\
		f_r(x_1,\ldots,x_n) = 0
	\end{cases}\]
	has no solutions. Then there exists $p_1,\ldots,p_r\in k[x_1,\ldots,x_n]$ such that
	\[\sum_{i=1}^rp_if_i=1.\]
\end{corollary}
\begin{proof}
	There is not much to say for the reverse direction. In the forwards direction, the main point is that $Z((f_1,\ldots,f_r))=\emp$, so
	\[\op{rad}(f_1,\ldots,f_r)=I(Z((f_1,\ldots,f_n)))=I(\emp)=R,\]
	so $1\in\op{rad}(f_1,\ldots,f_r)$. Then it follows $1\in(f_1,\ldots,f_r)$, so there exists $p_1,\ldots,p_r\in k[x_1,\ldots,x_n]$ such that
	\[\sum_{i=1}^rp_if_i=1.\]
	This is what we wanted.
\end{proof}

To set up the next corollary, we note that any point $(a_1,\ldots,a_n)\in\AA^n$ makes a closed set corresponding to the ideal
\[I(a):=(x_1-a_1,\ldots,x_n-a_n).\]
In fact,\todo{record this proof from Nullstellensatz} points turn out to be in bijection with maximal ideals in $\AA^n(k)$, and the same will be true for any algebraic set $X$ by looking at the ring $A(X)$, which is a quoteint of $A(\AA^n(k))$. This is nice because instead of having to look at the geometry of $X$, we can instead focus on the algebra of $A(X)$.

For our last corollary, we consider maps $\varphi:X\to Y$. Well, in affine space, we only have access to polynomial functions, so our only morphisms will be some (direct) product of polynomials. But then we observe that precomposition gives a function
\[\varphi:A(X)\rightarrow A(Y).\]
We would like to view this more geometrically, so we might want to pull back the ``point'' $\mf m\subseteq A(X)$ to $A(Y)$ by pre-imaging along $\varphi$. However, there is a problem here: $\varphi^{-1}(\mf m)$ might not be maximal.

A key observation, then, is that prime ideals are preserved by pre-imaging. So we have the following definition.
\begin{definition}[Spectrum of a ring]
	Given a ring $R$, we define \textit{spectrum of $R$} by
	\[\op{spec}R:=\{\mf p\subseteq R:\mf p\text{ is a prime ideal}\}.\]
\end{definition}
In fact, $\op{spec}R$ also has a Zariski topology as follows.
\begin{definition}[Zariski topology, II]
	Given a ring $R$, we define the \textit{Zariski topology} to have closed sets
	\[X(I):=\{\mf p\in\op{spec}R:I\subseteq\mf p\}\]
	for $R$-ideals $I$.
\end{definition}
This topology has some bad properties.
\begin{itemize}
	\item The Zariski topology is usally not Hausdorff. Namely, it tends to be cofinite.
	\item $\AA^n(k)$ is compact under its Zariski topology.\todo{I missed this proof in class}

	The main point is an open cover looks something like
	\[\sum_{\alpha\in\lambda}I_\alpha=R,\]
	for ideals $I_\alpha$. But by tracking where $1$ lives, we may take a finite subcover here.
	\item Similarly, the ring $R$ is compact under its Zariski topology:
\end{itemize}

\subsection{Projective Space}
To define projective varities, we need to define projective space first.
\begin{definition}[Projective space]
	Fix $k$ a field. Then we define $n$-dimensional \textit{projective space} $\PP^n(k)$ to be the one-dimensional subspaces of $k^{n+1}$.
\end{definition}
Concretely, we will think about lines in homogeneous coordinates, in the form
\[(a_0:a_1:\ldots:a_n)\in\PP^n(k),\]
where multiplication by a constant $c\in k^\times$ gives the same line and hence the same point. Additionally, we will ban the point $(0:0:\ldots:0)$ from projective space because it is not associated to any line.

Note that we have a sort of embedding $\AA^n(k)\into\PP^n(k)$ by
\[(x_1,x_2,\ldots,x_n)\mapsto(x_1:x_2:\ldots:x_n:1).\]
Geometrically, we can imagine the plane $z=1$ (which is isomorphic to $\AA^2(k)$) in $k^3$: any point on $z=1$ will define a unique line. However, this is not all of the points, for there are still lots of points of the form $(x:y:0)$, which are ``points at infinity.'' Nevertheless, we can collect the remaining points into $\PP^{n-1}$, so we see
\[\PP^n(k)``=''\AA^n(k)\sqcup\PP^{n-1}(k).\]
\begin{remark}
	The above decomposition is not canonical: one has to choose which points to get to be infinity.
\end{remark}

Anyways, we interested in studying the algebraic sets of projective space, but because of the constant factors allowed to wiggle, we see that we really should only be looking at homogeneous equations. So we have the following definition.
\begin{definition}[Projective variety]
	A subset $X\subseteq\PP^n(k)$ is a \textit{projective variety} if and only if it is the solution set to some set of homogeneous polynomials equations.
\end{definition}
\begin{example}
	If we wanted to study $xy-1=0$ in $\AA^2(k)$, to move this into projective space we want to look at the solution set to all of $xy-z^2=0$ in $\PP^2(k)$, and then we can look at the embedded affine space by the $z=1$ idea above.
\end{example}

\subsection{Graded Rings}
We have the following definition.
\begin{definition}[Graded ring]
	A ring $R$ is \textit{graded} by the abelian groups $R_0,R_1,\ldots$ if and only if
	\[R\cong\bigoplus_{d=0}^\infty R_d\]
	as abelian groups and $R_iR_j\subseteq R_{i+j}$ for any $i,j\in\NN$.
\end{definition}
\begin{example}
	The ring $R=k[x_1,\ldots,x_n]$ is graded by setting $R_d$ to be the space of all homogeneous $n$-variable polynomials of degree $d$ (unioned with $\{0\}$).
\end{example}
With graded rings, it is natural to ask what other ring-theoretic constructions we can grade.
\begin{definition}[Graded ideal]
	Fix $R$ a graded ring. We say that an ideal $I$ is \textit{graded} if and only if
	\[I=\bigoplus_{d=0}^\infty(R_d\cap I).\]
\end{definition}
The point of this definition is that
\[\frac{k[x_1,\ldots,x_n]}I\]
will also be a graded ring, which is nice.

The ideal
\[R_1\oplus R_2\oplus R_3\oplus\cdots\]
consisting of polynomials with vanishing constant term is called the irrelevant ideal because we don't like it.

As a quick application, here is one reason to care about graded rings.
\begin{proposition}
	A graded ring $R=R_0\oplus R_1\oplus\cdots$ is Noetherian if and only if $R_0$ is Noetherian and $R$ is a finitely generated $R_0$-algebra.
\end{proposition}
\begin{proof}
	Omitted.\todo{}
\end{proof}

\subsection{Hilbert Polynomials}
Continuing the above discussion, let $R=R_0\oplus R_1\oplus\cdots$ be a graded ring. Here is another ring-theoretic construction which we can grade.
\begin{definition}[Graded module]
	Fix $R=R_0\oplus R_1\oplus\cdots$ a graded ring. Then an $R$-module $M$ is \textit{graded} if and only if we can write
	\[M\cong\oplus_{d\in\ZZ}M_d\]
	such that $R_iM_j\subseteq M_{i+j}$ for any $i\in\NN$ and $j\in\ZZ$.
\end{definition}
As usual, the ring that we care the most about is $R=k[x_0,\ldots,x_n]$, graded by degree. In this case, take $M$ to be a a finitely-generated graded $R$-module, and it follows that
\[\dim_kM_d<\infty\]
for each $d\in\ZZ$. This is true because $R$ is Noetherian, so $M$ is Noetherin, so $\bigoplus_{d_0\ge d}M_{d_0}$ is a finitely generated, so $M_d$ is finitely generated over $k$.

This gives us the following definition.
\begin{definition}[Hilbert function]
	Fix everything as above. Then we define the \textit{Hilbert function} of $M$ as
	\[H_M(d):=\dim_kM_d.\]
\end{definition}
\begin{example}
	Take $M=R$. We can check that, for $d\ge0$,
	\[H_m(d)=\binom{d+n}{n}.\]
	To see this, we note that $M_d=R_d$ is generated by degree-$d$ monomials in $n$ letters, which are in one-to-one correspondence to their exponents. So we are counting nonnegative integer solutions to
	\[a_0+\cdots+a_n=0,\]
	which is a combinatorics problem.
\end{example}
The above example found that $H_m(d)$ is a polynomial in $d$ of degree $r$. This happens in general.
\begin{theorem}
	Let $M$ be a finitely generated graded module over the ring $R:=k[x_0,\ldots,x_n]$, where $R$ has been graded by degree. Then there exists a polynomial $P_M(d)$ of degree $n-1$ which matches $H_M(d)$ for sufficiently large $d$.
\end{theorem}
\begin{proof}
	The proof is by induction on $n$, where we will apply dimension-shifting of the grading for the inductive step. Our base case is $n=-1$, which makes $M$ into a $k$-vector space, which means $H_M(d)$.

	We will need to dimension-shift our grading in the proof that follows, so we have the following definition.
	\begin{definition}[Twist]
		Given a graded $R$-module $M$, we define the $d$th \textit{twist} $M(d)$ of $M$ to be the same module but with grading given by
		\[M(d)_i:=M_{d+i}.\]
	\end{definition}
	Note that $H_{M(d)}(s)=H_M(s+d)$ by this shifting.

	Now, for the inductive step, the main point is to kill the $x_n$ coordinate in creative ways. Namely, $M/x_nM$ will be finitely generated by $k[x_0,\ldots,x_{n-1}]$ (the letter $x_n$ does not help), so it is ripe for our induction, so we start with exact sequence
	\[M\to M/x_rM\to 0.\]
	Now, to take this backwards, we would like to prepend this by $M\stackrel{x_r}\to$, but this is not legal because this map will change the grading, so instead we have to write down
	\[M(-1)\stackrel{x_r}\to M\to M/x_rM\to 0.\]
	And to finish our short exact sequence, we let $K$ be the kernel of the multiplication by $x_r$---which is also finitely generated over $k[x_0,\ldots,x_{n-1}]$ because the $x_n$ letter does not help us---and we get to write
	\[0\to K(-1)\to M(-1)\stackrel{x_r}\to M\to M/x_rM\to 0.\]
	Taking the Hilbert function everywhere, size points on the short exact sequence imply that
	\[H_{K(-1)}(d)-H_{M(-1)}(d)+H_M(d)-H_{M/x_rM}(d)=0.\]
	We can rewrite this as
	\[H_M(d)-H_M(d-1)=H_{M/x_rM}(d)-H_K(d-1),\]
	so we see that the first finite difference of $H_M$ agrees with $H_{M/x_rM}(d)-H_K(d-1)$, and the latter agrees with a polynomial of degree at most $n-1$ for sufficiently large $d$ by inductive hypothesis. So theory of finite differences tells us that $H_M(d)$ will be a polynomial of degree at most $n$.
\end{proof}
\begin{remark}
	Geometrically, most of the time $M$ will end up being the coordinate ring of a projective variety, in which case the degree of the above Hilbert ``polynomial'' is the ``degree'' of the projective variety. So heuristically, most of the time the degree of the Hilbert polynomial will not achieve its maximum.
\end{remark}
Let's do some examples.
\begin{exe}
	Take $M:=k[x,y,z]/\left(x^2,y^2,z^2\right)$. We compute the Hilbert function for $M$.
\end{exe}
\begin{proof}
	We have the following.
	\begin{itemize}
		\item We see $M_0$ is simply $k$, so it has dimension $1$.
		\item We see $M_1$ is generated by $x,y,z$, so it has dimension $3$.
		\item We see $M_2$ is generated by $xy,yz,zx,x^2,y^2$, so it has dimension $5$.
		\item For the general case, we note that there is a short exact sequence
		\[0\to R_{d-2}\stackrel{x^2+y^2+z^2}\to R_d\to R_d\to0.\]
		So we see that $\dim M_d=\dim R_d-\dim R_{d-2}=\binom{n+2}2-\binom n2=2n+1$.
	\end{itemize}
	Geometrically, we can see that this is a projective quadratic by the leading coefficient.\todo{check Eisenbud}
\end{proof}
\begin{exe}[Eisenbud 1.19]
	Define $M:=k[x,y,z]/\left(xz-y^2,yx-z^2,xw-yz\right)$. We compute the Hilbert function for $M$.
\end{exe}
\begin{proof}
	For brevity, we set $I:=\left(xz-y^2,yx-z^2,xw-yz\right)$. The key observation is that it happens that\todo{} $I$ is a free $k[x,w]$-module with basis $\{1,y,z\}$.

	Thus, we can view $M$ as a $T:=k[x,w]$-module, and checking the basis, we get that $M=T\oplus T(-1)\oplus T(-1)$ corresponding to our basis elements $\{1,y,z\}$, so it follows that the Hilbert function is $H_M(n)=3n+1$.
\end{proof}
We will start with localization next class.