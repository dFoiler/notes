% !TEX root = ../notes.tex

We localize more.

\subsection{Flat Modules}
Last time we left off with the right-exactness of the tensor product: a right-exact sequence of $R$-modules
\[A\to B\to C\to 0\]
becomes a right-exact sequence
\[M\otimes_RA\to M\otimes_RB\to M\otimes_RC\to 0\]
for any other $R$-module $M$. More formally, we have the following statement.
\begin{proposition}
	Fix $R$ a ring and $M$ an $R$-module. Then the functor $M\otimes_R-:\mathrm{Mod}_R\to\mathrm{Mod}_R$ is right-exact.
\end{proposition}
\begin{proof}
	This is a restatement of the discussion above.
\end{proof}
However, it is not true that a short exact sequence
\[0\to A\to B\to C\to 0\]
will always become a short exact sequence
\[0\to M\otimes_RA\to M\otimes_RB\to M\otimes_RC\to 0.\]
In fact, this is rather rare! Explicitly, the problem is that $M\otimes_RA\to M\otimes_RB$ might not be injectivem ruining exactness at the front, and this is the only obstruction by right-exactness.
\begin{example}
	We work in $\mathrm{Mod}_\ZZ$, and let $n$ be a positive integer. Then tensoring the short exact sequence
	\[0\to\ZZ\stackrel{\times n}\to\ZZ\to\ZZ/n\ZZ\to0\]
	with $\ZZ/n\ZZ$ will give the commutative diagram
	% https://q.uiver.app/?q=WzAsMTAsWzAsMCwiMCJdLFsxLDAsIlxcWlpcXG90aW1lc19cXFpaXFxaWi9uXFxaWiJdLFsyLDAsIlxcWlpcXG90aW1lc19cXFpaXFxaWi9uXFxaWiJdLFszLDAsIlxcWlovblxcWlpcXG90aW1lc19cXFpaXFxaWi9uXFxaWiJdLFs0LDAsIjAiXSxbMSwxLCJcXFpaL25cXFpaIl0sWzIsMSwiXFxaWi9uXFxaWiJdLFswLDEsIjAiXSxbMywxLCJcXFpaL25cXFpaXFxvdGltZXNfXFxaWlxcWlovblxcWloiXSxbNCwxLCIwIl0sWzAsMV0sWzEsMiwiXFx0aW1lcyBuIl0sWzIsM10sWzMsNF0sWzUsNiwiZiJdLFsxLDVdLFs3LDVdLFsyLDZdLFs2LDhdLFszLDhdLFs4LDldXQ==&macro_url=https%3A%2F%2Fraw.githubusercontent.com%2FdFoiler%2Fnotes%2Fmaster%2Fnir.tex
	\[\begin{tikzcd}
		0 & {\ZZ\otimes_\ZZ\ZZ/n\ZZ} & {\ZZ\otimes_\ZZ\ZZ/n\ZZ} & {\ZZ/n\ZZ\otimes_\ZZ\ZZ/n\ZZ} & 0 \\
		0 & {\ZZ/n\ZZ} & {\ZZ/n\ZZ} & {\ZZ/n\ZZ\otimes_\ZZ\ZZ/n\ZZ} & 0
		\arrow[from=1-1, to=1-2]
		\arrow["{\times n}", from=1-2, to=1-3]
		\arrow[from=1-3, to=1-4]
		\arrow[from=1-4, to=1-5]
		\arrow["f", from=2-2, to=2-3]
		\arrow[from=1-2, to=2-2]
		\arrow[from=2-1, to=2-2]
		\arrow[from=1-3, to=2-3]
		\arrow[from=2-3, to=2-4]
		\arrow[from=1-4, to=2-4]
		\arrow[from=2-4, to=2-5]
	\end{tikzcd}\]
	after tracking through the canonical isomorphisms $\ZZ\otimes_\ZZ M\cong M$. But we can see that $f$ here sends $[k]_n$ lifts to $1\otimes[k]_n$, which goes to $n\otimes[k]_n=1\otimes[0]_n$ and therefore is $[0]_n$ downstairs. So $f$ is the zero map and not injective for any $n>1$.
\end{example}
But sometimes left-exactness will be preserved, and this is a property worthy of a name.
\begin{definition}[Flat]
	Fix $R$ a ring. Then an $R$-module $M$ is \textit{flat} if and only if the functor $M\otimes_R-$ is exact.
\end{definition}
\begin{remark}
	As above, we note that $M\otimes_R-$ will always be right-exact, so $M$ will be flat if and only if it preserves the injectivity at the end of a short exact sequence. In other words, $A\into B$ induces $M\otimes_RA\into M\otimes_RB$.
\end{remark}
\begin{example}
	The ring $R$ is a flat module because $R\otimes_RM\cong M$ (canonically). Explicitly, the following diagram commutes because the map $M\cong R\otimes_RM$ is $m\mapsto1\otimes m$.
	% https://q.uiver.app/?q=WzAsNCxbMCwwLCJBIl0sWzEsMCwiQiJdLFswLDEsIlJcXG90aW1lc19SQSJdLFsxLDEsIlJcXG90aW1lc19SQiJdLFswLDEsIiIsMCx7InN0eWxlIjp7InRhaWwiOnsibmFtZSI6Imhvb2siLCJzaWRlIjoidG9wIn19fV0sWzIsMywiIiwwLHsic3R5bGUiOnsidGFpbCI6eyJuYW1lIjoiaG9vayIsInNpZGUiOiJ0b3AifX19XSxbMCwyXSxbMSwzXV0=&macro_url=https%3A%2F%2Fraw.githubusercontent.com%2FdFoiler%2Fnotes%2Fmaster%2Fnir.tex
	\[\begin{tikzcd}
		A & B \\
		{R\otimes_RA} & {R\otimes_RB}
		\arrow[from=1-1, to=1-2]
		\arrow[from=2-1, to=2-2]
		\arrow[from=1-1, to=2-1]
		\arrow[from=1-2, to=2-2]
	\end{tikzcd}\]
	It follows $R\otimes_RA\to R\otimes_RB$ is injective when $A\into B$ is injective because this map is the composite $R\otimes_RA\cong A\into B\cong R\otimes_RB$, which is injective as the composite of injective maps.
\end{example}
\begin{example}
	Any free $R$-module $R^n$ is also flat by using direct sums. In particular, if we have $A\to B$, then the following diagram commutes.
	% https://q.uiver.app/?q=WzAsNCxbMCwwLCJSXm5cXG90aW1lc19SQSJdLFsxLDAsIlJeblxcb3RpbWVzX1JCIl0sWzAsMSwiKFJcXG90aW1lcyBBKV5uIl0sWzEsMSwiKFJcXG90aW1lcyBCKV5uIl0sWzAsMV0sWzIsM10sWzAsMl0sWzEsM11d&macro_url=https%3A%2F%2Fraw.githubusercontent.com%2FdFoiler%2Fnotes%2Fmaster%2Fnir.tex
	\[\begin{tikzcd}
		{R^n\otimes_RA} & {R^n\otimes_RB} \\
		{(R\otimes_R A)^n} & {(R\otimes_R B)^n}
		\arrow[from=1-1, to=1-2]
		\arrow[from=2-1, to=2-2]
		\arrow[from=1-1, to=2-1]
		\arrow[from=1-2, to=2-2]
	\end{tikzcd}\]
	Indeed, the map $R^n\otimes_R A\to(R\otimes_R A)^n$ is by $(r_k)_{k=1}^n\otimes a\mapsto(r_k\otimes a)_{k=1}^n$, so the commutativity follows. But we see that $A\into B$ means the individual maps $R\otimes_RA\to R\otimes_RB$ are injective, so the bottom row is injective. Tracking the isomorphisms through, we see the top row is also forced to be injective.
\end{example}

\subsection{Localization via Flatness}
Now let's return to discussing localization, which turns out to play nicely with the tensor product.
\begin{proposition}
	Fix $R$ a ring and $U\subseteq R$ a multiplicatively closed subset. Then, for any $R$-module $M$, we have a canonical isomorphism
	\[M\otimes_RR\left[U^{-1}\right]\cong M\left[U^{-1}\right].\]
\end{proposition}
\begin{proof}
	We define our maps in both directions explcitly. To go $M\left[U^{-1}\right]\to M\otimes_R R\left[U^{-1}\right]$, we define
	\[m/u\mapsto1/u\otimes m.\]
	One has to check that this is well-defined and an $R$-module homomorphism, but this is not hard.

	In the other direction, we go $M\otimes_RR\left[U^{-1}\right]\to M\left[U^{-1}\right]$ by
	\[r/u\otimes m\mapsto rm/u,\]
	and we can check that this is well-defined an an $R$-modle homomorphism.
\end{proof}
The above is nice because it means we technically would only need to check that $R\left[U^{-1}\right]$ exists in order to define localization of general modules.

Now, we have the following result, which looks like it's about localization but is actually about flatness.
\begin{proposition}
	Fix $R$ a ring and $U\subseteq R$ a multiplicatively closed subset. Then localization is an exact functor: given a short exact sequence
	\[0\to A\to B\to C\to 0,\]
	then the short exact sequence
	\[0\to A\left[U^{-1}\right]\to B\left[U^{-1}\right]\to C\left[U^{-1}\right]\to 0.\]
\end{proposition}
\begin{proof}
	Because $M\left[U^{-1}\right]\cong M\otimes_RR\left[U^{-1}\right]$ is a tensor product functor, we only need to show exactness on the left. In particular, we need to show that the map
	\[A\otimes_RR\left[U^{-1}\right]\to B\otimes_RR\left[U^{-1}\right]\]
	is injective. Letting $\varphi:A\into B$ be the original map and $\overline\varphi$ be the induced map, then we need to check that $\ker\overline\varphi$ is trivial.
	Well, if $\overline\varphi\left(\frac au\right)=0$, then $\varphi\left(u\cdot\frac au\right)=0$,
	so we get $\frac au=0$ because $\ker\varphi$ is trivial.
\end{proof}
Flatness gives us the following result, which again looks like it's about localization but is really about flatness.
\begin{corollary}
	Fix $R$ a ring and $U\subseteq R$ a multiplicatively closed subset. Then, taking $M_1,\ldots,M_n\subseteq M$ finitely many $R$-modules of some $R$-module $M$, we get
	\[\bigcap_{i=1}^nM_i\left[U^{-1}\right]=\left(\bigcap_{i=1}^nM_i\right)\left[U^{-1}\right].\]
	Note these intersections make sense because the $M_i$ all live inside of $M$.
\end{corollary}
\begin{proof}
	The main point is that intersecitons can be realized as a kernel. Namely, consider the short exact sequence
	\[0\to\bigcap_{i=1}^nM_i\to M\to M\to\bigoplus_{i=1}^nM/M_i\to 0.\]
	It is not too hard to check manually that this sequence is in fact short exact. Now, upon localization, we get the short exact sequence
	\[0\to\left(\bigcap_{i=1}^nM_i\right)\left[U^{-1}\right]\to M\left[U^{-1}\right]\to\bigoplus_{i=1}^n(M/M_i)\left[U^{-1}\right]\to0,\]
	where we are using the fact that localization already commutes with direct sums (for tensor product reasons)\todo{did we use localization here?}. In fact, we may localize quotients as well because of tensor product reasons, so we get the short exact sequence
	\[0\to\left(\bigcap_{i=1}^nM_i\right)\left[U^{-1}\right]\to M\left[U^{-1}\right]\to\bigoplus_{i=1}^n\frac{M\left[U^{-1}\right]}{M_i\left[U^{-1}\right]}\to0.\]
	From here the proposition follows.
\end{proof}
We need to be careful because localization need not commute with infinite interseciotns.
\begin{example}
	Set $R:=k[x]$ and $U=R\setminus\{0\}$. Observe that
	\[\bigcap_{a\in k}(x-a)=(0),\]
	but $(x-a)\left[U^{-1}\right]=k(x)$ because $U$ is allowed to divide by $(x-a)$. In particular, the left-hand side is $k(x)$, and the right-hand side is $(0)$.
\end{example}

\subsection{Some Homological Algebra}
We start by discussing a particular adjuction. For the discussion here, fix a ring homomorphism $\psi:R\to S$. Now, if $M$ is an $S$-module, we can give $M$ an $R$-action by
\[r\cdot m:=\psi(r)m,\]
making the $R$-module $\op{Res}_RM$. In the other direction, if $N$ is an $R$-module, we can create an $S$-module $\op{Ind}_R^SN:=S\otimes_RN$, where we get an $S$-action by multiplying on the left coordinate.

Here is our claim.
\begin{proposition}
	Let $R$ and $S$ be rings with a map $\varphi:R\to S$. Then, given an $R$-module $M$ and an $S$-module $N$, we have a canonical isomorphism
	\[\op{Hom}_R(M,\op{Res}_RN)\cong\op{Hom}_S(S\otimes_RM,N).\]
	In other words, tensoring is left-adjoint to restriction.
\end{proposition}
\begin{proof}
	We simply construt the two maps manually.
	\begin{itemize}
		\item Fix $\varphi\in\op{Hom}_R(M,\op{Res}_RN)$. Then we define $\widetilde\varphi\in\op{Hom}_S(S\otimes_RM,S)$ by defining
		\[\widetilde(s\otimes m)=s\varphi(m).\]
		Again, one has to check that this is well-defined, but we will not do so here.

		\item In the other direction, we notice that we havve a map $M\to S\otimes_RM$ by $m\mapsto 1\otimes m$. But composing this map with $\widetilde\varphi$ will give a map $M\to N$ which should provide us with an inverse.\todo{excuse me}
		\qedhere
	\end{itemize}
\end{proof}

Next let's discuss base change. Again, fix a ring homomorphism $\varphi:R\to S$ to make $S$ into an $R$-algebra. Given two $R$-modules named $M$ and $N$, we can form $S$-modules
\[S\otimes_R\op{Hom}_R(M,N)\qquad\text{and}\qquad\op{Hom}_S(S\otimes_RM,S\otimes_RN).\]
In general, there need not be an isomorphism between these $S$-modules, but there is a canonical map from the left to the right.
\begin{lemma} \label{lem:basechange}
	Fix everything as above. Then there is a canonical map
	\[\alpha:S\otimes_R\op{Hom}_R(M,N)\to\op{Hom}_S(S\otimes_RM,S\otimes_RN).\]
\end{lemma}
\begin{proof}
	We start by considering the restriction $\op{Res}_R\op{Hom}_S(S\otimes_RM,S\otimes_RN)$, for which we notice that we have a canonical map
	\[\gamma:\op{Hom}_R(M,N)\to\op{Res}_R\op{Hom}_S(S\otimes_RM,S\otimes_RN),\]
	where the point to check is that $\op{Res}_R$ is a safe move. But then the adjunction above provides a canonical map
	\[\widetilde\gamma:S\otimes_R\op{Hom}_R(M,N)\to\op{Hom}_S(S\otimes_RM,S\otimes_RN),\]
	which finishes. \todo{excuse me}
\end{proof}
We would like this to be an isomorphism. One condition will be that $S$ is flat over $R$; for the other condition we have the following definition.
\begin{definition}[Finitely presented]
	An $R$-module $M$ is \textit{finitely presented} if and only if there are $M$ is finitely generated, and we can find $R^m$ an $R^n$ making the following right-exact sequence
	\[R^m\to R^n\to M\to 0.\]
	In other words, the kernel of $R^n\onto M$ is finitely generated.
\end{definition}
\begin{example}
	For example, if $R$ is Noetherian, then the kernel in $R^n$ will be a submodule and hence finitely generated.\todo{find a counterexanple}
\end{example}
Now, here is the culmination of base change.
\begin{proposition}
	Work in the set-up of \autoref{lem:basechange}. Then if $S$ is flat and $M$ is finitely presented, then the $\alpha$ from \autoref{lem:basechange} is an isomorphism.
\end{proposition}
\begin{proof}
	We begin by writing down the finite presentation
	\[R^m\to R^n\to M\to 0\]
	of $M$. Now, this gives us a left-exact sequence\todo{}
	\[0\to\op{Hom}_R(M,N)\to\op{Hom}_R(R^n,N)\to\op{Hom}_R(R^m,N),\]
	and by flatness of $S$, we get another left-exact sequence
	\[0\to S\otimes_R\op{Hom}_R(M,N)\to S\otimes_R\op{Hom}_R(R^n,N)\to\op{Hom}_R(R^m,N).\]
	Taking $\op{Hom}_S$s now, we can use the tensor-hom adjunction to give us
	\[0\to\op{Hom}_S(S\otimes_RM,S\otimes_RN)\to\op{Hom}_S(S^n\otimes_RN)\to\op{Hom}_S(S^m,S\otimes_RN).\]
	But now we have vertical isomorphisms with the previous short exact sequence in the last two entries, which induces an isomorphism $S\otimes_R\op{Hom}_R(M,N)\cong\op{Hom}_S(S\otimes_RM,S\otimes_N)$, which ends up being $\alpha$.
\end{proof}

\subsection{Support of a Module}
We have the following definition.
\begin{definition}[Support]
	Fix $R$ a ring and $M$ an $R$-module. Then we define the \textit{support} of $M$ to be
	\[\op{Supp}M:=\{\mf p\in\op{Spec}R:M_\mf p\ne0\}.\]
\end{definition}
There is an analogous notion of maximal support using $\op{mSpec}$ instead of $\op{Spec}$.

We can provide a more concrete condition for $M_\mf p=0$. Recall that we have a canonical map $\varphi:M\to M_\mf p$ by $m\mapsto m/1$. Now, $M_\mf p=0$ if and only if the kernel $\varphi$ is all of $M$. But $m\in M$ has $\frac m1=0$ if and only if there is $u\notin\mf p$ such that
\[um=0.\]
In other words, we are interested in when $\op{Ann}(m)\not\subseteq\mf p$.

Continuing, if $M$ is finitely generated, we can get a covnverse: it suffices to check $\op{Ann}(m)\not\subseteq\mf p$ merely on the generators, which we will call $m_1,\ldots,m_n$. So we are promised $u_1,\ldots,u_n\notin\mf p$ such that $u_\bullet m_\bullet=0$. However, this implies that the product
\[u:=\prod_{k=1}^nu_k\notin\mf p\]
will kill all of $M$ anyways. Thus, $M_\mf p=0$ if and only if $\op{Ann}(m)\not\subseteq\mf p$. So we get the following statement.
\begin{proposition}
	Fix $R$ a ring. If $M$ is a finitely generated $R$-module, then
	\[\op{Supp}M=\{\mf p\in\op{Spec}M:\op{Ann}M\subseteq\mf p\}.\]
\end{proposition}
In particular, this is a Zariski-closed subset of $\op{Spec}R$!\todo{can all ideals appear as Ann M?}
\begin{example}
	Let $M$ be a simple module, which means that it has no proper nonzero submodules. Then, by considering the map $r\mapsto re_M$, we get a short exact sequence
	\[0\to\op{Ann}M\to R\to M\to 0,\]
	where the right-exactness is by simplicity. Thus, we can write $M=R/I$, where $I$ is some maximal ideal (maximal by simplicity), so it follows $\op{Supp}M=\{I\}$.
\end{example}
\begin{example}
	Consider a ring $R$, in which case $\op{Ann}R=(0)$, so $\op{Supp}M=\op{Spec}R$.
\end{example}
\begin{example}
	If we set $M=(0)$, then everyone in $R$ will support $(0)$, so $\op{Ann}M=R$. It follows that $\op{Supp}(0)=\emp$.
\end{example}
The support also behaves well in short exact sequences.
\begin{proposition}
	Suppose we have a short exact sequence
	\[0\to A\to B\to C\to 0\]
	of $R$-modules. Then $\op{Supp}B=\op{Supp}A\cup\op{Supp}C$.
\end{proposition}
\begin{proof}
	The main point is that localization is an exact functor. Namely, if $\mf p$ is any prime of $R$, then we get a short exact sequence
	\[0\to A_\mf p\to B_\mf p\to C_\mf p\to0.\]
	In particular, $B_\mf p\ne0$ if and only if $A_\mf p\ne0$ or $C_\mf p\ne0$, from which the statement follows.
\end{proof}
Additinally, we can learn something from the module itself by studying the support.
\begin{proposition}
	Fix an $R$-module $M$. Then $M=0$ if and only if $M_\mf m=0$ for all maximal ideals $\mf m\subseteq R$. In particular, $M=0$ if and only if $\op{Supp}M=\op{Spec}R$.
\end{proposition}
\begin{proof}
	We have already discussed the forwards direction. In the other direction, supose that the $R$-module $M$ has $R_\mf m$ for every maximal ideal $\mf m\subseteq R$.

	Suppose for the sake of contradiction we can find $m\in M\setminus\{0\}$. Then $\op{Ann}(m)$ is an $R$-ideal, and it must have $1\notin\op{Ann}(m)$ because $m\ne0$. Thus, we can place $\op{Ann}(m)\subseteq\mf m$ inside of some maximal ideal, but then $\frac m1\in M_\mf m$ is nonzero by our dicussion before. This finishes.
\end{proof}
\begin{corollary}
	Fix $\varphi:M\to N$ an $R$-module homomorphism and $\mf m\subseteq R$ a maximal ideal. Then we are promised a localized map $\varphi_\mf m:M_\mf m\to N_\mf p$. Then $\varphi_\mf m$ is injective/surjective/isomorphic for all maximal ideals $\mf m$ if and only if $\varphi$ is as well.
\end{corollary}
\begin{proof}
	Being injective is equivalent to saying that $\ker\varphi_\mf m=(\ker\varphi)_\mf m$ is zero, and being surjective is equivalent to saying that $\coker\varphi_\mf m=(\coker\varphi)_\mf m$ is zero. So we are done by the previous proposition.
\end{proof}
We continue our fact-collection.
\begin{proposition}
	Fix $M$ and $N$ finitely generated $R$-modules. Then
	\[\op{Supp}(M\otimes_RN)=\op{Supp}M\cap\op{Supp}N.\]
\end{proposition}
\begin{proof}
	We leave this as an exercise.\todo{please why}
\end{proof}

\subsection{Tensoring Algebras}
For the next construction, we note that if $S$ and $T$ are $R$-algebras, then $S\otimes_RT$ is an $R$-algebra, where our multiplication is defined by
\[(s_1\otimes t_1)(s_2\otimes t_2)=(s_1s_2)\otimes(t_1t_2).\]
One can run through the checks that this will be an $R$-algebra.

Concretely, we have the following.
\begin{exe}
	If we have two free $k$-algebras $k[x_1,\ldots,x_n]$ and $k[y_1,\ldots,y_m]$, then we claim that
	\[[x_1,\ldots,x_n]\otimes_kk[y_1,\ldots,y_m]\]
	is freely generated by the elements of the form $x_\bullet\otimes1$ and $1\otimes y_\bullet$.
\end{exe}
\begin{remark}
	Geometrically, we can write this as $A\left(\AA^n(k)\right)\otimes_kA\left(\AA^m(k)\right)\cong A\left(\AA^n(k)\times\AA^m(k)\right)$, which makes more immediate sense.
\end{remark}
More generally, the following is true.
\begin{proposition}
	Fix affine algebraic sets $X$ and $Y$. Then $A(X\times Y)=A(X)\otimes_kA(Y)$.
\end{proposition}
\begin{proof}
	Omitted.
\end{proof}
\begin{remark}
	One can generalize this construction to fiber products.
\end{remark}
Next class we will finish up localization by discussing modules of finite length.