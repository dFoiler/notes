% !TEX root = ../notes.tex

Today we localize.

\subsection{Geometric Motivation}
Let's do examples from geometry.

Fix $X\subseteq\AA^n(k)$ an algebraic set. Then fix $U\subseteq X$ an open subset, and we want to define functions on $U$. Concretely, we might take $X=\AA^1(k)$ and $U=X\setminus\{0\}$. In this case, we have $A(X)=k[x]$, but we see that upon removing $0$ allows us to divide by $x$, giving
\[A(U)=k[x,1/x].\]
These turn out to be all the functions we care about.

An alternative way to do this construction is to simply add a new function $y$ to $A(X)$ and then mod out in the freest possile way by the requirement $xy=1$, giving
\[A(U)=\frac{k[x,y]}{(xy-1)}.\]
Namely, these are the functions out of the hyperbola $xy=1$ in the plane $\AA^2(k)$. This sort of construction is a little bit harder if we take $X=\AA^2(k)$ and $U=X\setminus\{(0,0)\}$. Namely, localization cannot describe these functions.

The point is that localization is one way we can talk about functions of spaces.

\subsection{Localization of Rings}
Let's build towards the definition of localization.
\begin{definition}[Multiplicatively closed]
	Fix $R$ a ring. Then a subset $U\subseteq R$ is \textit{multiplicatively closed} if any product of elements in $U$ also lives in $U$.
\end{definition}
Note that, by convention, the empty product $1$ will need to live in $U$. So by induction, it suffices for $1\in U$ and for $x,y\in U$ to imply $xy\in U$.
\begin{remark}
	We do permit $U=\{0,1\}$. This tends to not be very interesting.
\end{remark}
And here is our main character. In the discussion that follows, $R$ will be a ring and $U$ will always be multiplicatively closed.
\begin{definition}[Localization, rings]
	Fix $R$ a ring and $U\subseteq R$ multiplicatively closed. Then we define $R\left[U^{-1}\right]$ to be the set of ordered pairs notated $\frac ru$ (with $r\in R$ and $u\in U$) modded out by the equivalence relation
	\[\frac{r_1}{u_1}=\frac{r_2}{u_2}\iff\text{there exists }v\in U\text{ such that }v(u_2r_1-u_1r_2)=0.\]
\end{definition}
\begin{remark}
	One needs the $v$ in the definition above to make $\equiv$ transitive.
\end{remark}
We can turn $R\left[U^{-1}\right]$ into a ring by using the standard addition and multiplication operations of these numbers. Namely, we define
\[\frac{r_1}{u_1}+\frac{r_2}{u_2}:=\frac{u_2r_1+u_1r_2}{u_1u_2}\qquad\text{and}\qquad\frac{r_1}{u_1}\cdot\frac{r_2}{u_2}:=\frac{r_1r_2}{u_1u_2}.\]
It is an exercise to check that these operations do not depend on the exact representative of the equivalence relation $\equiv$ defined above, which we do not do here. One can also show that by hand that these operations do in fact form a ring.
\begin{remark}
	Observe that because $1\in U$, there is a canonical map $R\to R\left[U^{-1}\right]$ by $r\mapsto r/1$. This need not be injective; e.g., take $U=\{0,1\}.$
\end{remark}

We might also want to localize by sets which are not 
\begin{definition}[Multiplicative closure]
	Fix $R$ a ring. Then for any $U\subseteq R$, we define the \textit{multiplicative closure} $\overline U$ to be the set of all products of $U$.
\end{definition}
One can check that $\overline U$ is multiplicatively closed, so by convention we will define $R\left[U^{-1}\right]:=R\left[\overline U^{-1}\right]$ for arbitrary subsets.

Here are some standard examples of localization.
\begin{definition}[Field of fractions]
	If $R$ is an integral domain, then $U:=R\setminus\{0\}$ is multiplicatively closed (and in fact this is equivalent to $R$ being an integral domain). So we define the \textit{field of fractions}
	\[K(R):=R\left[(R\setminus\{0\})^{-1}\right].\]
\end{definition}
\begin{example}
	We have that $K(\ZZ)=\QQ$.
\end{example}
\begin{example}
	We have that $K(k[x])=k(x)$.
\end{example}
\begin{definition}[Localization at a prime]
	Fix $R$ a ring and $\mf p\subseteq R$ a prime ideal. Then by definition of primality, $R\setminus\mf p$ will be multiplicatively closed, so we define the \textit{localization at a prime}
	\[R_\mf p:=R\left[(R\setminus\mf p)^{-1}\right].\]
\end{definition}
In fact, $R_\mf p$ becomes a local ring, with its single maximal ring being
\[\mf p:=\{a/b\in R_\mf p:a\notin\mf p\text{ and }b\in\mf p\}.\]
It turns out that we can realize the field of fractions from this construction.
\begin{example}
	When $R$ is an integral domain, $(0)$ is prime, and $R_{(0)}=K(R)$.
\end{example}
\begin{example}
	We have that
	\[\ZZ_{(p)}:=\left\{\frac ab:a,b\in\ZZ\text{ and }p\nmid b\right\}.\]
\end{example}
The ideal $\mf p_\mf p$ gives rise to the following definition for these local rings.
\begin{definition}[Residue field]
	Fix $R$ a local ring with unique maximal ideal $\mf p$. Then we define the \textit{residue field} to be $R/\mf p$.
\end{definition}
\begin{example}
	We have that $\ZZ_{(p)}/p\ZZ_{(p)}\cong\ZZ/p\ZZ$. In particular, observe that the characteristic has changed.
\end{example}
Geometrically, we view primes $\mf p$ as living in the ``space'' $\op{Spec}R$. Then here $R_\mf p$ is intended to look like a ``neighborhood'' or ``germ'' at the point $\mf p$. Hence the name localization.

There is also a universal property for localization.
\begin{proposition}
	Fix $R$ a ring and $U\subseteq R$ a multiplicatively closed subset. Let $\varphi:R\to R\left[U^{-1}\right]$ be the canonical map. Now, suppose we are given a map $\psi:R\to S$ such that $\psi(U)\subseteq R^\times$. Then there is a unique morphism $\gamma$ making the diagram commute.
	% https://q.uiver.app/?q=WzAsMyxbMCwwLCJSIl0sWzEsMCwiUlxcbGVmdFtVXnstMX1cXHJpZ2h0XSJdLFsxLDEsIlMiXSxbMCwyLCJcXHBzaSIsMl0sWzAsMSwiXFx2YXJwaGkiXSxbMSwyLCJcXGdhbW1hIl1d
	\[\begin{tikzcd}
		R & {R\left[U^{-1}\right]} \\
		& S
		\arrow["\psi"', from=1-1, to=2-2]
		\arrow["\varphi", from=1-1, to=1-2]
		\arrow["\gamma", from=1-2, to=2-2]
	\end{tikzcd}\]
\end{proposition}
\begin{proof}
	Essentially this is because we can send fractions to fractions, and there is a unique way to do this. We omit the details.
\end{proof}

\subsection{Localization of Modules}
We can also localize modules, in essentially the same way.
\begin{definition}[Localization, modules]
	Fix $R$ a ring and $U\subseteq R$ multiplicatively closed. Then, given an $R$-module $M$, we define $M\left[U^{-1}\right]$ to be the set of ordered pairs notated $\frac mu$ (with $m\in M$ and $u\in U$) modded out by the equivalence relation
	\[\frac{m_1}{u_1}=\frac{m_2}{u_2}\iff\text{there exists }v\in U\text{ such that }v(u_2m_1-u_1m_2)=0.\]
\end{definition}
One can define addition by fractions in the same way, by hand, writing
\[\frac{m_1}{u_1}+\frac{m_2}{u_2}=\frac{u_1m_2+u_2m_1}{u_1u_2}.\]
Again, it is not too hard to check that this is well-defined and gives an abelian group law. Further, we even have an $R\left[U^{-1}\right]$ structure by
\[\frac rv\cdot\frac mu:=\frac{rm}{vu}.\]
Thus, localizing at $U$ will be able to define a functor from $R$-modules to $R\left[U^{-1}\right]$-modules.
\begin{remark}
	Again, we even have our canonical $R$-module homomorphism $M\to M\left[U^{-1}\right]$ by $m\mapsto m/1$. This need not be injective, and in fact we can describe the kernel: $\frac m1=\frac01$ if and only if there exists $u$ such that $um=0$. In other words, the kernel is the set
	\[U\cap\{r\in M:rm=0\}\ne\emp.\]
\end{remark}

\subsection{Localization of Ideals}
We would like to discuss what happens to ideals under localization. Recall that, given a morphism $\varphi:R\to S$, the pre-image of an ideal $I\subseteq S$ will be an ideal $\varphi^{-1}(I)\subseteq R$. In fact, we discussed above that prime ideals go to prime ideals; we can also show that this map preserves inclusions and intersections, which holds on the level that $\varphi$ is a function of sets.

Now, in our case, we are focusing on the canonical morphism $\varphi:R\to R\left[U^{-1}\right]$. We have the following proposition.
\begin{proposition}
	Fix $R$ a ring and $U\subseteq R$ a multiplicatively closed set, and let $\varphi:R\to R\left[U^{-1}\right]$ be the canonical map.
	\begin{listalph}
		\item Given any $R\left[U^{-1}\right]$-ideal $I$, pre-image followed by localization does nothing:
		\[I=\varphi^{-1}(I)\left[U^{-1}\right].\]
		It follows that the map from $R\left[U^{-1}\right]$-ideals to $R$-ideals by $I\mapsto\varphi^{-1}(I)$ is injective.
		\item Fix an $R$-ideal $J$. The following are equivalent.
		\begin{listroman}
			\item $J=\varphi^{-1}(I)$ for some $R\left[U^{-1}\right]$-ideal $I$
			\item $J=\varphi^{-1}\left(J\left[U^{-1}\right]\right)$
			\item There is no $u\in U$ such that $[u]_J\in R/J$ is not a zero-divisor.
		\end{listroman}
		\item The above show that $\varphi^{-1}$ provides a bijection between the prime ideals of $R$ which are distinct from $U$ and the prie ideals of $R\left[U^{-1}\right]$.
	\end{listalph}
\end{proposition}
\begin{proof}
	We note that (c) follows from (a) and (b) to establish the bijection, using part (iii).\todo{excuse me}

	Let's discuss (a). The inclusion that $\varphi^{-1}(I)\left[U^{-1}\right]\subseteq I$ is not difficult, so we focus on the inclusion $I\subseteq\varphi^{-1}(I)\left[U^{-1}\right]$. Well, given any $\frac ru\in I$, we see that $\frac r1\in I$, so $r\in\varphi^{-1}(I)$, so $\frac ru\in\varphi^{-1}(I)\left[U^{-1}\right]$.

	For (b), the equivalence of (i) and (ii) is from (a). It remains to show the equivalence of (iii). In one direction, suppose we can find $u\in U$ and $r\in R$ such that $au\in J$. Then it follows that $a\in J\left[U^{-1}\right]$, so $a\in J$ follows.\todo{excuse me}

	In the other direction, suppose that $r\in\varphi^{-1}(J\left[U^{-1}\right])$ so that $\frac r1=\frac ju$ for some $j\in J$. This implies $v(ru-j)=0$ for some $v\in V$, implying that $(vu)r\in J$, but because $u$ and $v$ are not zero-divisors, we see that $r\in J$ is forced.
\end{proof}
Here is a reason to care about the above proposition.
\begin{corollary}
	Any localization of a Noetherian ring $R$ is still a Noetherian ring.
\end{corollary}
\begin{proof}
	Show that any ideal $I\subseteq R\left[U^{-1}\right]$ is finitely generated by pulling it back to $R$, finding generators, and then pushing them forwards again.
\end{proof}

\subsection{The \textrm{Hom}-Functor}
Later in life we will discuss localization as a tensor product, but before then we must talk about the tensor product, so for now we will talk about the \textrm{Hom}-functor. Here is our definition.
\begin{definition}[\textrm{Hom}]
	Fix $R$ a ring. Then, for $R$-modules $M$ and $N$, we define $\op{Hom}_R(M,N)$ to be the abelian group of $R$-module homomorphisms $M\to N$.
\end{definition}
In fact, we can endow $\op{Hom}_R(M,N)$ with an $R$-module structure, essentially because our rings are commutative. Namely, we define
\[(r\varphi)(m):=r\cdot\varphi(m).\]
It is not too hard to verify that this does in fact define a ring action.
\begin{definition}[\textrm{End}]
	Fix $R$ a ring. Then we define the \textit{endomorphisms} of an $R$-module $M$ to be $\op{End}_R(M):=\op{Hom}_R(M,M)$.
\end{definition}
Note that $\op{End}_R(M)$ is in fact a (non-commutative) $R$-algebra, where our multiplication is given by composition.

Here are some basic facts.
\begin{enumerate}
	\item We have that $\op{Hom}_R(R,M)\cong M$ canonically by $\varphi\mapsto\varphi(1)$.
	\item Given two morphisms $\alpha:M_2\to M_1$ and $\beta:N_1\to N_2$, then we have a map $\op{Hom}_R(M_1,N_1)\to\op{Hom}_R(M_2,N_2)$ by $\varphi\mapsto\beta\circ\varphi\circ\alpha$. In fact, this is an $R$-module homomorphism.
	\item We have that
	\[\op{Hom}_R\left(\bigoplus_{\alpha\in I}M_\alpha,N\right)\cong\prod_{\alpha\in I}\op{Hom}_R(M_\alpha,N)\]
	for any collection of $R$-modules $\{M_\alpha\}_{\alpha\in I}$.
	\item In fact, $\op{Hom}$ is a left-exact functor. Namely, exact sequences
	\[0\to A\to B\to C\]
	yields the exact sequence
	\[0\to\op{Hom}_R(M,A)\to\op{Hom}_R(M,B)\to\op{Hom}_R(M,C).\]
	Similarly,
	\[\op{Hom}_R(C,M)\to\op{Hom}_R(B,M)\to\op{Hom}_R(A,M)\to0\]
	is exact. Note the reversal of direction of arrows here.
\end{enumerate}
\begin{remark}
	However, $\op{Hom}_R$ does not fully preserve short exact sequences. In the first, case we are saying that a morphism $\op{Hom}_R(M,C)$ might not be extendable to a map $\op{Hom}_R(M,B)$. By way of example, consider the short exact sequence of $\ZZ$-modules
	\[0\to2\ZZ\to\ZZ\to\ZZ/2\ZZ\to0.\]
	Then taking $\op{Hom}_\ZZ(\ZZ/2\ZZ,-)$ gives
	\[0\to0\to0\to\ZZ/2\ZZ\to0,\]
	which is not exact in the last term.
\end{remark}

\subsection{Tensor Product}
We should probably start by defining tensor products, which requires defining bilinear maps.
\begin{definition}[Bilinear]
	Fix $A,B,C$ as $R$-modules for some ring $R$. Then a map $\varphi:A\times B\to C$ is \textit{$R$-bilinear} if and only if is $R$-linear in both arguments. Namely, we require
	\[\varphi(r_1a_1+r_2a_2,b)=r_1\varphi(a_1,b)+r_2\varphi(a_2,b)\]
	and
	\[\varphi(a,r_1b_1+r_2b_2)=r_1\varphi(a,b_1)+r_2\varphi(a,b_2).\]
\end{definition}
This lets us define the tensor product to more or less be the object universal with respect to giving bilinear maps.
\begin{definition}[Tensor product]
	Fix $R$ a ring and $A$ and $B$ as $R$-modules. Then we define $A\otimes_RB$ to be the free module generated generated by $a\otimes b$ for $a\in A$ and $b\in B$ modulo the relation
	\[(a_1m_1+a_2m_2)\otimes(b_1n_1+b_2n_2)=a_1b_1(m_1\otimes n_1)+a_1b_2(m_1\otimes n_2)+a_2b_1(m_2\otimes n_1)+a_2b_2(m_2\otimes n_2).\]
\end{definition}
Elements of the tensor product $A\otimes B$ are in general not very easy to understand and in general they can be described as being some finite sum of elements $a\otimes b$ for various $a\in A$ and $b\in B$. In the case where $A$ and $B$ are vector spaces over a field, then these the tensor of two basis vectors will create a basis, but this is essentially the only general example.

Nevertheless, let us work with some examples.
\begin{example}
	We work in $\ZZ$-mod, and we compute $\ZZ/2\ZZ\otimes_\ZZ\ZZ/3\ZZ$. By the Chinese remainder theorem, it is not too hard to show that $1\otimes1$ generates the full product. However,
	\[2(1\otimes1)=2\otimes1=0\qquad\text{and}\qquad3(1\otimes1)=1\otimes3=0,\]
	so $1\otimes1=0$ follows. Thus, $\ZZ/2\ZZ\otimes_\ZZ\ZZ/3\ZZ=0$.
\end{example}
\begin{prop}
	Fix $a$ and $b$ integers. Then
	\[\ZZ/a\ZZ\otimes_\ZZ\ZZ/b\ZZ\cong\ZZ/\gcd(a,b)\ZZ.\]
\end{prop}
\begin{proof}
	The main point is that the smallest positive multiple of $1\otimes1$ which will go to $0$ will be the smallest positive element of $a\ZZ+b\ZZ$, which is $\gcd(a,b)$.
\end{proof}
As with $\op{Hom}_R$, the tensor product $\otimes_R$ has the following list of nice properties.
\begin{enumerate}
	\item We have that $M\cong R\otimes_RM$ by $m\mapsto m\otimes1$.
	\item Given morphisms $\alpha:M_1\to M_2$ and $\beta:N_1\to N_2$, we can define a map
	\[\alpha\otimes\beta:M_1\otimes_R N_1\to M_2\otimes_R N_2\]
	by extending $m\otimes n\mapsto\alpha m\otimes\beta n$ linearly to the full tensor product.
	\item We have that $M\otimes_R N\cong N\otimes_R M$ by $m\otimes n\mapsto n\otimes m$.
	\item We have that
	\[\left(\bigoplus_{\alpha\in I}M_\alpha\right)\otimes_RI\cong\bigoplus_{\alpha\in I}(M_\alpha\otimes_R I).\]
	\item The functor $-\otimes_RM$ is right-exact: given an exact sequence
	\[A\to B\to C\to 0,\]
	we have an exact sequence
	\[A\otimes_RM\to B\otimes_RM\to C\otimes_RM\to 0.\]
	Here the maps are the induced ones. To see this, note $B\otimes_RM\to C\otimes_RM$ is surjective by lifting some $m\otimes_Rc$ for $m\in M$ and $c\in C$ by taking some pre-image $b\in B$ for $c$ and then conjuring $m\times b\mapsto m\otimes c$.

	For exactness in the middle, we let $f:A\to B$ be the first map and consider the induced map
	\[M\otimes C\to\frac{M\otimes B}{M\otimes\im f}\]
	in order to split the short exact sequence or something.
\end{enumerate}
Next time we will show $M\left[U^{-1}\right]$ is canonically isomorphic to $R\left[U^{-1}\right]\otimes_RM$ to continue our discusion of localization.