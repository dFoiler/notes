% !TEX root = ../notes.tex














Here we go.

\subsection{Product Formula for Number Fields}
We have the following lemma.
\begin{lemma}
	Fix $L/K$ an extension of number fields and $v$ a place of $K.$ Then
	\[\prod_{\substack{w\in M_L\\w\mid v}}\lVert x\rVert_w=\lVert\op N_K^Lx\rVert_v\]
	for each $x\in L.$
\end{lemma}
\begin{proof}
	We simply write
	\[\lVert x\rVert_w=\lVert\op N_{K_v}^{L_w}x\rVert_w^{1/[L_w:K_v]}\]
	from last class, which becomes $\lVert\op N_{K_v}^{L_w}x\rVert_v$ by our discussion on extending norms. But now
	\[\prod_{w\mid v}\lVert x\rVert_w=\left\lVert\prod_{w\mid v}\op N_{K_v}^{L_w}x\right\rVert_v=\lVert\op N_K^Lx\rVert\]
	by the fact that local norms multiply to give global norms.
\end{proof}
This gives us the following theorem.
\begin{theorem}
	Fix $K$ a number field. Then
	\[\prod_{v\in M_K}\lVert x\rVert=1\]
	for each $x\in K^\times.$
\end{theorem}
\begin{proof}
	The key is to use the fact $K$ is lying over $\QQ.$ Indeed,
	\[\prod_{v\in M_K}\lVert x\rVert_v=\prod_{p\in M_\QQ}\prod_{\substack{v\in M_K\\v\mid p}}\lVert x\rVert_v,\]
	which by lemma is equal to
	\[\prod_{p\in M_\QQ}\lVert\op N_\QQ^Kx\rVert,\]
	which we know is equal to $1$ because we showed the product formula for $\QQ.$
\end{proof}
\begin{remark}
	We can extend this proof to global fields without too much pain, where we reduce to the case of $\FF_p[t]$ instead of reducing to $\QQ,$ but Professor Vojta is worried about inseparable extensions.
\end{remark}
\begin{remark}
	The product formula is somewhat akin to the fact that the sum of the zeroes and poles is zero for a compact Riemann surface.
\end{remark}

\subsection{Replete Ideals}
We have the following definition.
\begin{definition}[Picard group]
	Fix $K$ a number field and define $J(\mathcal O_K)$ to be the fractional ideals of $K$ and $P(\mathcal O_K)$ to be the principal fractional ideals. Then we define the \textit{Picard group of $K$} to be
	\[\op{Pic}(\mathcal O_K)=J(\mathcal O_K)/P(\mathcal O_K)\cong\op{Cl}_K.\]
	Sometimes we choose to write $\op{Pic}$ additively.
\end{definition}
The Picard group is not including the infinite places, which would be akin to working over a Riemann surface which is not compact.
\begin{definition}[Replete ideal]
	Fix $K$ a number field. A \textit{replete ideal} (i.e., an Arakelov divisor of $\op{Spec}\mathcal O_K$) is an element of the group
	\[\overline J(\mathcal O_K):=J(\mathcal O_K)\times\prod_{v\mid\infty}\RR^\times_{>0}.\]
	We will think $J(\mathcal O_K)\subseteq\overline J(\mathcal O_K)$ by the canonical embedding.
\end{definition}
Essentially we are just adding points to $\op{Spec}\mathcal O_K$ corresponding to the infinite places. We are allowed to take a Cartesian product in the above definition (instead of a direct sum) because there will only be finitely many infinite places anyways.
\begin{definition}[Notation for places]
	Given a number field $K,$ we continue to set $M_K$ to be the set of places of $K,$ but we define
	\[M_K^\infty:=\{v\in M_K:v\mid\infty\}\qquad\text{and}\qquad M_K^\circ=M_K\setminus M_K^\infty.\]
	We note that we have a natural embedding $M_K^\infty\into\overline J(\mathcal O_K).$
\end{definition}
In fact, we have a nice embedding taking $\mf p\in M_K^\infty$ and $\nu\in\RR$ to
\[\mf p^\nu:=\left\{(1),(e^{\nu1_{\mf q=\mf p}})_{\mf q\mid\infty}\right\}.\]
The point of saying is that we can write any element of $\overline J(\mathcal O_K)$ in the form
\[\prod_{\mf p\in M_K}\mf p^{\nu_\mf p}\]
where $\nu_\mf p$ is an integer when $\mf p$ is finite and real when $\mf p$ is infinite, and we also require that all but finitely many of the $\nu_\mf p$ vanish.
\begin{remark}
	Professor Vojta is unsure if we should instead define
	\[\mf p^\nu:=\left\{(1),(e^{{\color{red}-}\nu1_{\mf q=\mf p}})_{\mf q\mid\infty}\right\}.\]
	We might change this definition for convenience later.
\end{remark}
Anyways, this gives us the following definition.
\begin{definition}[Valuation at a place]
	For all $a\in K^\times$ and place $v\in M_K,$ we define
	\[v(\alpha):=\begin{cases}
		\nu_\mf p((a)) & v\in M_K^\circ\text{ belongs to }\mf p\nmid\infty, \\
		-\log|\tau a| & v\in M_K^\infty\text{ belongs to }\tau:K\into\CC.
	\end{cases}\]
\end{definition}
\begin{definition}[Norm at a place]
	Fix $v$ a place of a number field $K.$ We define $\op N(v)$ to be $\op N(\mf p)$ when $v$ is finite, $\op N(v)=e$ when $v$ is real, and $\op N(v)=e^2$ when $v$ is complex. This extends multiplicatively to a full map on $\overline J(\mathcal O_K).$
\end{definition}
This gives us $\lVert a\rVert_v=\op N(v)^{-\nu(a)}$ for each $a\in K^\times.$\todo{} Additionally, if $\mf p\in M_K$ lies over $p\in M_\QQ$ then $\op N(\mf p)=p^{f(\mf p/p)}.$ This comes out to
\[\op N(\mf p)=\lVert\pi\rVert_v^{-1},\]
where $\pi$ is a uniformizer of $K_v$ in $K$ when $\mf p$ is finite and $\pi$ is some element such that $|\tau\pi|=\frac1e$ where $\tau:K\into\CC$ corresponds to $v.$
\begin{definition}[Principal replete ideal]
	Fix $a\in K^\times.$ Then the \textit{principal replete ideal} of $a$ is defined as
	\[[a]:=\prod_{\mf p\in M_K}\mf p^{\nu_\mf p(a)}.\]
	The set of all principal replete ideals is $\overline P(\mathcal O_K.$)
\end{definition}
\begin{example}
	For $K=\QQ,$ we have $[2]=(2)^{\nu_2(2)}\cdot(\infty)^{\nu_\infty(2)}$ by unwinding all of our definitions. Then
	\[\op N([2])=2^1\cdot e^{-\log2}=2\cdot\frac12=1\]
	by tracking everything through.
\end{example}
We can see that, for a given $a\in K^\times,$ we have
\[\op N([a])=\prod{\mf p\in M_K}\op N(\mf p)^{\nu_\mf p(a)}=\prod{\mf p\in M_K}\lVert a\rVert_\mf p^{-1}=1\]
by using the product formula and tracking through the definitions. This gives the following.
\begin{definition}[\texorpdfstring{$\overline{\op{Pic}}$}{Pic}]
	We define
	\[\overline{\op{Pic}}(\mathcal O_K):=\overline J(\mathcal O_K)/\overline P(\mathcal O_K).\]
	Note that $\op N$ will descend to a homomorphism on $\overline{\op{Pic}}.$ We also define
	\[\overline{\op{Div}}(\mathcal O_K)=\left(\bigoplus_{\mf p\in M_K^\circ}\ZZ\right)\times\left(\bigoplus_{\mf p\in M_K^\infty}\RR\right)\]
	to be the additive version, and we denote its elements additively.
\end{definition}
Note that $\overline{J}(\mathcal O_K)$ and $\overline{\op{Div}}(\mathcal O_K)$ contain essentially the same data, where
\[\prod_{\mf p\in M_K}\mf p^{\nu_\mf p}\longleftrightarrow\sum_{\mf p\in M_K}\nu_\mf p\mf p.\]
Similarly, an element $a\in K^\times$ will go to $[a]$ of $\overline J(\mathcal O_K)$ and $\sum_\mf p\nu_\mf p(a)\mf p$ as above.

\subsection{A Little with Riemann Surfaces}
Let's go back to setting $X$ to be a compact Riemann surface, and as usual fix $K:=\mathcal K(X)$ with $B$ the integral closure of $\CC[t]$ in $K,$ where $t$ is some map $\pi:X\to\CP^1.$

Now, when $f\in K^\times$ is a nonzero map where $f:X\to\CC$ is a meromorphic map, we define
\[\lVert f\rVert_v:=e^{-\op{ord}_v(f)},\]
where $v$ is an place of $K$ trivial on $\CC$ and in particular belonging to a point of $X.$ Now, we see that the product formula will let us say
\[\sum_{p\in X}\op{ord}_p(f)=0,\]
so we define our divisor group by
\[\op{Div}(X):=\bigoplus_{p\in X}\ZZ.\]
More or less, this corresponds to $\overline{\op{Div}}.$