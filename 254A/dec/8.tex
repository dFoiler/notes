% !TEX root = ../notes.tex














The speed{run} continues.

\subsection{Riemann \texorpdfstring{$\zeta$}{Z}-function}
We quickly recall the definition we had
\[\zeta(s):=\sum_{n=1}^\infty\frac1{n^s}=\prod_p\left(1-\frac1{p^{-s}}\right)^{-1}\]
for $\op{Re}s>1.$ We also defined
\[\Gamma(s):=\int_0^\infty e^{-y}y^s\,\frac{dy}y,\]
for $\op{Re}s>0,$ and we had an analytic continuation of $\frac1{\Gamma(s)}$ to all of $\CC.$ We now define
\[\Xi(s):=\pi^{-3/s}\Gamma\left(\frac s2\right)\zeta(s),\]
for $\op{Re}s>1.$ We now have the following theorem.
\begin{theorem}
	The function $\Xi(s)$ extends to a meromorphic function on all of $\CC,$ with simple poles at $s=0$ and $s=1$ with residues $-1$ and $1$ respectively. In fact, $\Xi$ satisfies
	\[\Xi(s)=\Xi(1-s).\]
\end{theorem}
\begin{proof}
	We won't prove all of this, but we will prove a bit. So far we only have $\Xi$ only defined to the right of $\op{Re}s=1.$ We start with a trick: we extend $\zeta$ to $\op{Re}s>0$ so that $\Xi$ will also be defined over there.
	\begin{definition}
		A \textit{Dirichlet series} is a sum of the form $\sum_na_nn^{-s},$ where $a_n\in\CC.$
	\end{definition}
	\begin{lemma}
		Fix $\sum_na_nn^{-s}$ be a Dirichlet series and define, for $n\in\NN,$
		\[A_n=\sum_{k=1}^na_k.\]
		If we can find $c>0$ and $\sigma>0$ such that $|A_n|<cn^\sigma$ for each $n,$ then the full series will converge with $\op{Re}s>\sigma.$
	\end{lemma}
	\begin{proof}
		We refer to Lang's \textit{Algebraic Number Theory}, Chapter VIII, Theorem 2.
	\end{proof}
	And now we use the lemma.
	\begin{prop}
		The function $\zeta(s)$ has a meromorphic continuation to $\op{Re}s>0,$ with a simple pole at $s=1.$
	\end{prop}
	\begin{proof}
		We can use the lemma. Set
		\[\zeta_s(s):=\sum_{n=1}^\infty\frac{(-1)^{n+1}}{n^s}\]
		to be the alternating version of $\zeta.$ Then we see that $|A_n|\le1$ always, so $\zeta_2$ will converge on $\op{Re}s>\sigma$ for each $\sigma>0,$ so $\zeta_2$ will converge on $\op{Re}s>0$ by pushing $\sigma$ back to $0.$

		But now we can compute that
		\[\frac2{2^s}\zeta(s)+\zeta_2(s)=\sum_{n=1}^\infty\frac{2}{(2n)^s}+\sum_{n=1}^\infty\frac{(-1)^n}{n^s}=\sum_{n=1}^\infty\frac1{n^s}=\zeta(s),\]
		so
		\[\zeta(s)=\frac{\zeta_2(s)}{1-2^{1-s}}.\]
		Because both functions here agree on $\op{Re}s>1,$ the right-hand side will provide a continuation for $\zeta$ to $\op{Re}s>0.$	Tracking our poles, we see that $\zeta_2$ will have no poles here, but $1-2^{1-s}$ will have a zero at $s=1+\frac{2\pi ni}{\log2}$ for each $n.$

		We would like to get rid of these poles. For this, we define
		\[\zeta_3(s):=\zeta(s)-\frac3{3^s}\zeta(s)=\frac1{1^s}+\frac1{2^s}-\frac{2}{3^s}+\frac1{4^s}+\frac1{5^2}-\frac{2}{6^s}+\cdots.\]
		Again by lemma, $\zeta_3$ will continue to $\op{Re}s>0,$ and we now find that
		\[\zeta(s)=\frac{\zeta_3(s)}{1-3^{1-s}}\]
		for $\op{Re}s>1,$ so we have a continuation for $\zeta$ to $\op{Re}s>0.$ But tracking our poles like last time, now our poles will occur at $s=1+\frac{3\pi ni}{\log3}$ for each $n,$ but the intersection is
		\[\left\{1+\frac{3\pi ni}{\log3}:n\in\ZZ\right\}\cap\left\{1+\frac{3\pi ni}{\log3}:n\in\ZZ\right\}=\{1\}\]
		because $\frac{\log2}{\log3}$ is irrational. So our only possible pole is at $s=1,$ which we can verify does indeed exist by looking at $\zeta(s)$ as $s\to1^+.$
	\end{proof}
	From here, we have extended $\Xi$ to $\op{Re}s>0,$ and we can prove the functional equation (with some effort), which provides a continuation of $\Xi$ to all of $\CC.$
\end{proof}

\subsection{The Riemann Hypothesis}
It is known that, for each positive integer $n,$ we have
\[\zeta(1-n)=\frac{-B_n}{n!},\]
where $B_n$ are the Bernoulli numbers, given by the generating function
\[\frac{t}{1-e^{-t}}=\sum_{n=0}^\infty B_n\frac{t^n}{n!}.\]
In particular, $\zeta$ has zeroes at the negative even integers. In fact, $\zeta$ has no other zeroes outside of the strip $0<\op{Re}s<1.$ 
\begin{conj}
	The Riemann hypothesis is the conjecture that
	\[\zeta(s)=0\implies s\in-2\ZZ_{>0}\text{ or }\op{Re}s=\frac12.\]
\end{conj}
This might appear abstract, but it has many implications to analytic number theory. To state this, we have the following definitions.
\begin{definition}[Prime-counting functions]
	Fix $x>0.$ Then define $\pi(x)$ to be the number of primes less than $x$; define
	\[\op{Li}(x):=\int_0^x\frac{dx}{\log x}.\]
	Define
	\[\psi(x):=\sum_{p^m\le x}\log p.\]
\end{definition}
\begin{theorem}
	Fix $\varepsilon\ge0.$ The following are equivalent.
	\begin{listroman}
		\item The function $\zeta(s)$ has no zeroes $\op{Re}s>\frac12+\varepsilon.$
		\item We have that $\psi(x)=x+)\left(x^{1/2+\varepsilon}(\log x)^2\right).$
		\item We have that $\pi(x)=\op{Li}(x)+O\left(xe^{-a\sqrt{\log x}}\right),$ for some $a>0.$
	\end{listroman}
\end{theorem}
We remark that it is known that $\psi(x)=x+O\left(xe^{-c\sqrt{\log x}}\right),$ which is the Prime number theorem.

\subsection{Class Field Theory}
As we must, we begin with a little abstraction.
\begin{definition}[Restricted direct product]
	Fix $\{G_\alpha\}_{\alpha\in\lambda}$ be a collection of groups, and fix subgroups $\{H_\alpha\}_{\alpha\in\lambda}$ so that $H_\alpha\subseteq G_\alpha.$ Now we define the \textit{restricted direct product} defined as
	\[\sideset{}{'}\prod_{\alpha\in\lambda}G_\alpha:=\left\{(g_\alpha)_{\alpha\in\lambda}\in\prod_{\alpha\in\lambda}G_\alpha:g_\alpha\in H_\alpha\text{ all but finitely often}\right\},\]
	with operation inherited from the product.
\end{definition}
This lets us define the following two objects.
\begin{definition}[Id\'eles]
	Fix $K$ a global field. Then the group of \textit{id\'{e{l}es}} $I_K$ of $K$ is the restricted product
	\[\sideset{}{'}\prod_{\mf p\in M_K}K_\mf p^\times\]
	where our subgroups are $\mathcal O_\mf p^\times,$ which is the unit group of the valuation ring associated to the place $\mf p\in M_K^\circ.$ If $\mf p$ is real, then take $\RR^\times_{>0}$; if $\mf p$ is complex, take $\CC^\times.$
\end{definition}
To codify the above subgroups, we have the following definition.
\begin{definition}[Valuation rings]
	Fix $K$ a global field and $\mf p$ a place. Then
	\[U_\mf p:=\begin{cases}
		\mathcal O_\mf p^\times & \mf p\text{ finite}, \\
		\RR^\times_{>0} & \mf p\text{ real}, \\
		\CC^\times & \mf p\text{ complex}.
	\end{cases}\]
\end{definition}
We wold like to be able to kill some finite set of places, so given a set $S\subseteq M_K$ of places, we define
\[I^S_K:=\left\{(a_\mf p)_{\mf p}\in\prod_\mf p K_\mf p^\times:a_\mf p\in U_\mf p\text{ for }\mf p\notin S\right\}\]
so that
\[I^S_K=\prod_{\mf p\in S}K^\times_\mf p\times\prod_{\mf p\notin S}U_\mf p.\]
In particular, $S\subseteq S'$ implies $I^S_K\subseteq I^{S'}_K$ and $I_K=\bigcup I^S_K$ so that $I^S_K\subseteq I_K.$

What makes the id\'eles usable is that we have a diagonal map
\[K^\times\subseteq I_K\]
by taking $\alpha\mapsto(\alpha)_\mf p.$ We can see that, indeed, only finitely many places $\mf p$ will have $\alpha\notin U_\mf p.$
\begin{definition}[Principal id\'ele]
	An id\'ele is \textit{principal} if it is in $K^\times\subseteq I_K.$
\end{definition}
\begin{example}
	If $K$ is a number field, then
	\[K^\times\cap I_K^{M_K^\infty}=\mathcal O_K^\times\]
	More generally, if $M_K^\infty\subseteq S,$ then $K^\times\cap I_K^S=\mathcal O^\times_{K,S},$ which are the $S$-units of $\mathcal O_K.$
\end{example}
The above example leads us to the next definition.
\begin{definition}[General \texorpdfstring{$S$}{S}-integers]
	Fix $K$ a global field. Given a finite set $S$ of places, we set $K^S:=K^\times\cap I_K^S.$
\end{definition}
The addition of infinite places being able to add restrictions really only matters for real places $\rho,$ where we are requesting that the image under $\rho:K\into\RR^\times$ be positive.