\documentclass[../notes.tex]{subfiles}

\begin{document}

% !TEX root = ../notes.tex
















And so it ends.

\subsection{Id\'ele Class Group}
We continue our story of class field theory. Today we will be modest enough to take $K$ to be a number field.
\begin{definition}[Id\'ele class group]
	The \textit{id\'ele class group} is defined as the quotient $C_K:=I_K/K^\times,$ where $K^\times\subseteq I_K$ using the usual embedding.
\end{definition}
We would like to relate $C_K$ with our usual ideal class group $\op{Cl}_K.$
\begin{remark}
	We have that $C_K$ is uncountable.
\end{remark}
There is a homomorphism $I_K\to J_K,$ where $J_K$ are the fractional ideals by taking
\[(\alpha_\mf p)_{\mf p}\longmapsto\prod_{\mf p\nmid\infty}\mf p^{\nu_\mf p(\alpha_\mf p)},\]
which is a legal, finite product because all but finitely many of the $\alpha_\mf p$ have $\nu_\mf p(\alpha_\mf p)=0$ by definition of the restricted product. We note that the above map is surjective because look at it, and its kernel essentially consists of the infinite places:
\[I_K^{M_K^\infty}=\prod_{\mf p\mid\infty}K_\mf p^\times\times\prod_{\mf p\nmid\infty}\mathcal O_\mf p^\times.\]
Thus, $I_K/I_K^{M_K^\infty}\cong J_K$ and modding out by $K^\times$ on both sides gives us a surjective map $C_K\to\op{Cl}_K$ with kernel $K^\times I_K^{M_K^\infty}/K^\times.$ Putting this all together, we get the following exact sequence.
\[0\to K^\times\to K^\times I_K^{M_K^\infty}\to C_K\to\op{Cl}_K\to0\]
Of course, the above story does not have to be for just infinite places/
\begin{proposition}
	For any finite subset $S$ which contains $M_K^\infty,$ we have the following exact sequence.
	\[0\to K^\times\to K^\times I_K^S\to\op{Cl}_K^S\to0,\]
	where $\op{Cl}_K^S$ is the class group of our $S$-integers.
\end{proposition}
\begin{proof}
	Imitate the above discussion.
\end{proof}
We note that if $S$ is sufficiently large, namely to contain a generating set for $\op{Cl}_K,$ we will have that $\op{Cl}_K^S=0,$ so we get the short exact sequence
\[0\to K^\times\to K^\times I_K^S\to C_K\to0.\]
By drawing the following diagram, the five lemma tells us that $K^\times I_K^S\cong I_K.$
% https://q.uiver.app/?q=WzAsMTAsWzAsMCwiMCJdLFsxLDAsIkteXFx0aW1lcyJdLFsyLDAsIkteXFx0aW1lcyBJX0teUyJdLFszLDAsIkNfSyJdLFs0LDAsIjAiXSxbMCwxLCIwIl0sWzEsMSwiS15cXHRpbWVzIl0sWzIsMSwiS15cXHRpbWVzIElfSyJdLFszLDEsIkNfSyJdLFs0LDEsIjAiXSxbMCwxXSxbMSwyXSxbMiwzXSxbMyw0XSxbNSw2XSxbNiw3XSxbNyw4XSxbOCw5XSxbMSw2XSxbMiw3XSxbMyw4XV0=
\[\begin{tikzcd}
	0 & {K^\times} & {K^\times I_K^S} & {C_K} & 0 \\
	0 & {K^\times} & {K^\times I_K} & {C_K} & 0
	\arrow[from=1-1, to=1-2]
	\arrow[from=1-2, to=1-3]
	\arrow[from=1-3, to=1-4]
	\arrow[from=1-4, to=1-5]
	\arrow[from=2-1, to=2-2]
	\arrow[from=2-2, to=2-3]
	\arrow[from=2-3, to=2-4]
	\arrow[from=2-4, to=2-5]
	\arrow[from=1-2, to=2-2]
	\arrow[from=1-3, to=2-3]
	\arrow[from=1-4, to=2-4]
\end{tikzcd}\]
This is sufficiently cute.

\subsection{Topological Groups}
We have the following definition.
\begin{definition}[Topological group]
	A \textit{topological group} $G$ is a group $G$ together with a topology such that the maps
	\[(x,y)\mapsto xy\qquad\text{and}\qquad x\mapsto x^{-1}\]
	are continuous maps.
\end{definition}
\begin{example}
	Everyone's favorite fields give topological groups: $(\QQ,+),\RR^\times,\CC^\times,$ and so on.
\end{example}
We remark that if $G$ is a topological group, then given $g\in G,$ the maps $x\mapsto gx$ and $x\mapsto xg$ are both continuous by the first condition, so they are homeomorphisms $G\to G$ by noting the inverse maps are from $g^{-1}.$

We note that this implies we can create a topology on $G$ using only the open sets around the identity of $G.$ Namely, if $U$ is a nonempty open subset with $g\in U,$ then $g^{-1}U$ will be an open subset containing $e.$

We pick up the following facts.
\begin{proposition}
	Fix $G$ a topological group with $H\subseteq G$ a subgroup.
	\begin{listalph}
		\item If $H$ is open, then $H$ is closed.
		\item If $H$ is closed and of finite index, then $H$ is open.
		\item If $H_1\subseteq H_2\subseteq G$ and $H_1$ is open, then $H_2$ is also open.
	\end{listalph}
\end{proposition}
\begin{proof}
	For (a) and (b), we simply look at $G/H.$ For (a), all cosets in $G/H$ are open, so $H$ is closed by considering the union of all the cosets not equal to $H.$ For (b), all cosets in $G/H$ are closed, so $H$ is now open. For (c), tile $H_2$ by $H_1$s as in $H_2/H_1$ and then union them together.
\end{proof}
With this notation in mind, we observe that we can build a basic set of open sets around $1\in I_K$ by
\[\prod_{\mf p\in S}W_\mf p\times\prod_{\mf p\notin S}U_\mf p,\]
where $S\subseteq M_K$ is finite and $W_\mf p$ is an open neighborhood of $1$ in $K_\mf p^\times.$ We can also view this as the coarsest group topology that we can give $I_K$ while requiring the projections
\[I_K\to K_\mf p^\times\]
to be continuous while maintaining $I_K^S\subseteq I_K$ being open.

One can show that this topology makes $K^\times\subseteq I_K$ into a discrete subgroup and so closed. Additionally, we get that is $C_K$ a topological group by modding.

\subsection{Norms}
To continue using our embedding $K^\times\subseteq I_K,$ we pick up the following definition.
\begin{definition}[Absolute norm]
	Given some id\'ele $a:=(a_\mf p)_\mf p\in I_K,$ we define the \textit{absolute norm} as
	\[\op N(a)=\prod_\mf p\lvert\lVert a_\mf p\rVert_\mf p^{-1}.\]
\end{definition}
It happens that this is a continuous homomorphism $I_K\to\RR_{>0},$ it is surjective, and it has kernel $K^\times$ (by the product formula on $K^\times$). Thus, the absolute norm in fact descends to a map $C_K\onto\RR^\times_{>0}.$

The norm is a good way to measure size of the id\'ele class group, but we would like to make the id\'ele class group smaller to be able to handle it more properly. This gives us the following definition.
\begin{definition}[\texorpdfstring{$C_K^0$}{}]
	We define $C_K^0:=\ker C_K.$
\end{definition}
It is a theorem that $C_K^0$ has now been made small enough: namely, $C_K^0$ is compact.

There is also a relative norm to accompany our absolute norm.
\begin{definition}[Relative norm]
	Fix $L/K$ an extension of number fields. Then we define the \textit{relative norm} $\op N^L_K:I_L\to I_K$ by taking $\beta\in I_L$ to
	\[\left(\op N^L_K\beta\right)_\mf p:=\prod_{\mf q\mid\mf p}\op N^{L_\mf q}_{K_\mf p}B_\mf q.\]
\end{definition}
This norm is also functorial, making the following diagram commute.
% https://q.uiver.app/?q=WzAsNCxbMCwwLCJMXlxcdGltZXMiXSxbMCwxLCJLXlxcdGltZXMiXSxbMSwwLCJJX0wiXSxbMSwxLCJJX0siXSxbMCwyLCIiLDAseyJzdHlsZSI6eyJ0YWlsIjp7Im5hbWUiOiJob29rIiwic2lkZSI6InRvcCJ9fX1dLFsxLDMsIiIsMCx7InN0eWxlIjp7InRhaWwiOnsibmFtZSI6Imhvb2siLCJzaWRlIjoidG9wIn19fV0sWzAsMSwiXFxvcCBOXkxfSyIsMl0sWzIsMywiXFxvcCBOXkxfSyJdXQ==
\[\begin{tikzcd}
	{L^\times} & {I_L} \\
	{K^\times} & {I_K}
	\arrow[hook, from=1-1, to=1-2]
	\arrow[hook, from=2-1, to=2-2]
	\arrow["{\op N^L_K}"', from=1-1, to=2-1]
	\arrow["{\op N^L_K}", from=1-2, to=2-2]
\end{tikzcd}\]
Hooray.

\subsection{Global Class Field Theory}
We are now ready to state the main theorem of global class field theory.
\begin{theorem} \label{thm:gcft}
	Fix $K$ a number field. There exists a canonical bijection between finite abelian extensions $L/K$ and open subgroups $H\subseteq C_K$ of finite index. In fact, $\op{Gal}(L/K)\cong C_K/H.$
\end{theorem}
We would like to further extend this correspondence, in the way that Galois theory does.
\begin{theorem}
	Fix $K$ a number field. Then the correspondence in \autoref{thm:gcft} is inclusion-reversing.
	\begin{listalph}
		\item If the extensions $L$ and $L'$ correspond to the subgroups $H$ and $H',$ then $L\supseteq L'$ if and only if $H\subseteq H'.$
		\item Further, $LL'$ corresponds to $H\cap H',$ and $L\cap L'$ correspond to $HH'.$
	\end{listalph}
\end{theorem}
We also have some control over prime-splitting.
\begin{theorem}
	Fix $K$ a number field. Then the correspondence in \autoref{thm:gcft} controls prime-splitting.
	\begin{listalph}
		\item For all finite places $\mf p\in M_K^\circ,$ we have that $\mf p$ is unramified in $L$ if and only if $H\supseteq K^\times U_\mf p/K^\times.$
		\item For all places $\mf p\in M_K^\circ,$ then $\mf p$ splits completely in $L$ if and only if $H\supseteq K^\times K_\mf p^\times/K^\times.$ 
	\end{listalph}
\end{theorem}
Let's do some applications.
\begin{definition}[Hilbert class field]
	Fix $K$ a number field. Then the \textit{Hilbert class field} of $K$ is the extension $L/K$ which corresponds to $H=K^\times I_K^{M_K^\infty}/K^\times\subseteq C_K.$
\end{definition}
The Hilbert class field has some magic associated with it. Fix $K$ a number field and set $H:=K^\times I_K^{M_K^\infty}/K^\times$ corresponding to $L$ the Hilbert class field. We have the following.
\begin{listalph}
	\item We have that $C_K/H\cong I_K/K^\times I_K^{M_K^\infty}\cong J_K/K^\times=\op{Cl}_K.$ In particular, $\op{Gal}(L/K)\cong\op{Cl}_K.$
	\item Noting that $H\supseteq K^\times K_\mf p/K^\times$ for each infinite place $\mf p$ and $H\supseteq K^\times U_\mf p/K^\times$ for each finite place $\mf p,$ it follows that $L$ is unramified at all finite places and splits completely at all infinite places.
	\item In fact, we can see that $L$ is the largest abelian extension of $K$ unramified at all finite places.
\end{listalph}
As an aside, we remark that we can recover the Kronecker--Weber theorem from global class field theory by essentially showing $\op{Gal}\left(\QQ^{\op{ab}}/\QQ\right)\cong\widehat{\ZZ}.$

\end{document}