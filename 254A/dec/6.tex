% !TEX root = ../notes.tex


















As usual, we pick up the $AKLB$ picture with separable $L/K$ and $\lambda/\kappa.$

\subsection{Ramification Control: Different}
We have the following lemma.
\begin{lemma}
	Fix $\alpha\in L.$ Then the following are true.
	\begin{listalph}
		\item If $A[\alpha]$ is a full, finitely generated $A$-submodule of $L$ (which comes from $L=K(\alpha)$ and $\alpha\in B$), then we have that
		\[(A[\alpha])^*=\frac1{f'(\alpha)}A[\alpha].\]
		\item If $B=A[\alpha],$ then we claim that $\mathcal D_{B/A}$ is the principal ideal $\delta_{L/K}(\alpha)B.$
	\end{listalph}
	We have that $\mathcal D_{B/A}$ is generated by the $\delta_{L/K}(\alpha)$ for $\alpha\in L.$
\end{lemma}
\begin{proof}
	We sketch (b). Note that
	\[\frac{f(x)}{x-\alpha}=b_{n-1}x^{n-1}+\cdots+b_0\]
	with the $b_\bullet\in A.$ We can check that
	\[\op T^L_K\left(\alpha^i\frac{b_j}{f'(\alpha)}\right)=1_{i-j},\]
	so the dual basis for $\{\alpha^\bullet\}$ comes out to
	\[\frac{b_\bullet}{f'(\alpha)}.\]
	In particular, it follows that
	\[A[\alpha]^*=\frac1{f'(\alpha)}\sum_kb_kA,\]
	which we can check to tbe $\frac1{f'(\alpha)}A[\alpha].$ From here, (a) follows directly from (b), for reasons which are not clear to me.
\end{proof}
We have the following theorem.
\begin{theorem}
	We have that $\mathcal D_{B/A}$ is generated by the $\delta_{L/K}(\alpha)$ for $\alpha\in L.$
\end{theorem}
\begin{proof}
	We sketch. For each $\alpha\in B$ with minimal polynomial $f,$ we set $b=f'(\alpha).$ We will take $\alpha$ so that $L=K(\alpha).$ Now, we can compute the conductor
	\[\mf f=\mf f_{A[\alpha]}=\{x\in L:xB\subseteq A[\alpha]\},\]
	so for each $x\in L,$ we have $x\in\mf f$ if and only if $xB\subseteq A[\alpha]$ if and only if $b^{-1}xB\subseteq b^{-1}A[\alpha]=A[\alpha]^*$ (by lemma) if and only if $\op T^L_K(b^{-1}xyA[\alpha])\subseteq B$ for each $y\in B$ if and only if $\op T^L_K(b^{-1}xBA[\alpha])\subseteq A$ if and only if $\op T^L_K(b^{-1}xB)\subseteq A$ if and only if $b^{-1}x\in B^*$ if and only if $x\in b\mathcal D_{B/A}^{-1}.$ Thus,
	\[(\delta_{L/K}(\alpha))=(b)\in\mf f\mathcal D_{L/K}.\]
	Now, for each nonzero prime $\mf q\subseteq B$ lying over $\mf p\subseteq A,$ we localize: find $\alpha$ with $\mf q\nmid\mf f_{A[\alpha]}$ (check the completion in $\hat B$ and then find a nearby $\alpha\in B$) with $L=K(\alpha).$ We leaving verifying the existence of this $\alpha$ as an exercise. Everything works in the localization, so we can lift up the powers of $\mf q$ to the general case.\todo{}
\end{proof}

This gives us the following control over ramification.
\begin{theorem}
	We have the following. Fix $\mf q$ of $L$ lying over $\mf p$ of $K.$
	\begin{listalph}
		\item A prime $\mf q$ of $L$ is ramified over $K$ if and only if $\mf q\mid\mathcal D_{B/A}.$
		\item Fix $s:=\nu_\mf q(\mathcal D_{B/A})$ and $e:=e(\mf q/\mf p).$ Then
		\[\begin{cases}
			e\le s\le e-1+\nu_\mf q(e) & \mf q\text{ is wildly ramified}, \\
			e\le s\le e-1+\nu_\mf q(e) & \mf q\text{ is tamely ramified}.
		\end{cases}\]
	\end{listalph}
\end{theorem}
\begin{proof}
	We refer to the book. The main point is to localize at $\mf q$ and set $\alpha$ to be a uniformizer, finishing by the previous theorem.
\end{proof}

\subsection{Ramification Control: Discriminant}
We have the following definition.
\begin{definition}
	The \textit{discriminant ideal} $D_{B/A}$ of $A$ (living downstairs) is the ideal generated by the discriminants $d(\alpha_1,\ldots,\alpha_n)$ for $\alpha_\bullet\in B.$
\end{definition}
\begin{example}
	Taking $K$ to be a number field, $D_{K/\QQ}=(\op{disc}\mathcal O_K).$
\end{example}
It happens that the discriminant essentially comes from the different.
\begin{proposition}
	We have that $D_{B/A}=\op N^L_K\mathcal D_{L/K}.$
\end{proposition}
\begin{proof}
	The main point is that both sides commute with localizations by subsets $S\subseteq A,$ so we may take $A$ to be a local and hence principal ring. It follows that there is an integral basis for $B$ over $A,$ which we name $\{\alpha_1,\ldots,\alpha_n\}\subseteq B.$ Now, it follows that
	\[D_{B/A}=(\op{disc}(\alpha_1,\ldots,\alpha_n))\]
	because changing to any other basis only multiplies by a square of the determinant of the change-of-basis matrix.

	Now, $\mathcal C_{B/A}$ is generated by some dual basis $\alpha_1',\ldots,\alpha_n'.$ Noting that $B$ will only have finitely many primes while being a Dedekind ring, we see that $B$ is a principal ideal domain, so we can find some $b\in L$ such that $\mathcal C_{B/A}$ has a basis $b\alpha_1,\ldots,b\alpha_n$ over $A.$\todo{} It follows that
	\[(1)=\left(\det[\op T^L_K\alpha_i\alpha_j']\right)=\left((\op N^L_Kb)\cdot\det[\op T^L_K\alpha_i\alpha_j]\right)=\op N^L_Kb\cdot D_{L/K},\]
	I guess. Anyways, it's supposed to follow that $D_{L/K}=\op N^L_K\mathcal D_{L/K}$ by tracking through the $\delta$s.
\end{proof}
So we get the following corollary.
\begin{proposition}
	A prime $\mf p$ of $K$ ramifies up in $L$ if and only if $\mf p\mid D_{L/K}.$
\end{proposition}
\begin{proof}
	Norm the primes dividing $\mathcal D_{B/A}$ down to primes dividing $D_{L/K}.$
\end{proof}
In particular, we have that a prime $p$ of $\ZZ$ will ramify up in some number field $K$ if and only if $p\mid\op{disc}\mathcal O_K.$

We also have the following tower law.
\begin{proposition}
	For a tower $K\subseteq L\subseteq K,$ we have that
	\[D_{M/K}=D_{L/K}^{[M:L]}D_{M/L}.\]
\end{proposition}
\begin{proof}
	Taking the tower law for the different and norm everything downwards. The exponent of $[M:L]$ comes from {norm}ing elements of $L$ downwards.
\end{proof}
While we're here, we pick up the following statement.
\begin{theorem}[Hermite--Minkowski]
	There are only finitely many number fields of given discriminant $\op{disc}\mathcal O_K.$
\end{theorem}
\begin{proof}
	See the book.
\end{proof}

\subsection{Zeta and \texorpdfstring{$L$}{L}-functions}
Here is our main character.
\begin{definition}[Riemann \texorpdfstring{$\zeta$}{zeta}-function]
	We define
	\[\zeta(s):=\sum_{n=1}^\infty\frac1{n^s}\]
	for all $s\in\CC$ such that $\op{Re}s>1.$
\end{definition}
We note that this does indeed converge (absolutely!) for each $s\in\CC$ such that $\op{Re}s>1$ because
\[\sum_{n=1}^\infty\left|\frac1{n^s}\right|=\sum_{n=1}^\infty\frac1{n^{\op{Re}s}}<\infty.\]
We also have uniform absolute convergence on any set $\{s\in\CC:\op{Re}s>1+\delta\}$ for each $\delta>0.$

The reason we care about this function in number theory is the following.
\begin{proposition}[Euler product]
	We have that
	\[\zeta(s)=\prod_{p\text{ prime}}\frac1{1-p^{-s}}\]
	again for each $s\in\CC$ with $\op{Re}s>1.$
\end{proposition}
\begin{proof}
	The point is to use unique prime factorization. We have that
	\[\frac{1-p^{-s}}=\sum_{k=0}^\infty\frac1{p^{ks}}.\]
	Now, for any finite set $T$ of primes, we have that
	\[\prod_{p\in T}\frac1{1-p^{-s}}=\prod_{p\in T}\sum_{\nu_p=0}^\infty\frac1{p^{-s\nu_p}},\]
	which will expand out to a sum of $\frac1{n^{-s}}$ such that $s$ only contains primes in $T.$ Setting $T$ to be the primes up to $N,$ we can bound
	\[\left|\zeta(s)-\prod_{p\in T}\frac1{1-p^{-s}}\right|=\left|\sum_{p\mid n\implies p\notin T}\frac1{n^s}\right|<\sum_{n=N+1}^\infty\left|\frac1{n^s}\right|.\]
	Taking $N\to\infty$ causes this to go to $0,$ so we get the result we want.
\end{proof}
To continue our story, we have the following definition.
\begin{definition}[The \texorpdfstring{$\Gamma$}{Gamma} function]
	We define
	\[\Gamma(s):=\int_0^\infty e^{-t}t^s\frac{dt}t\]
	for $s\in\CC$ such that $\op{Re}s>0.$
\end{definition}
\begin{example}
	We have that $\Gamma\left(\frac12\right)=\sqrt\pi.$
\end{example}
The function $\Gamma$ can be meromorphically continued to all of $\CC$ as follows.
\begin{proposition}
	We have that $\Gamma(s+1)=s\Gamma(s)$ for each $s$ where $\Gamma$ happens to be defined.
\end{proposition}
\begin{proof}
	Apply integration by parts.
\end{proof}
\begin{corollary}
	We have that $\Gamma(n)=(n-1)!$ for each $n\in\ZZ^+.$
\end{corollary}
\begin{proof}
	Start with $\Gamma(1)=1$ and induct using the above proposition.
\end{proof}
Formally speaking, we get an analytic continuation of $\frac1{\Gamma(s)}$ to all of $\CC$ by simply applying the identity repeatedly, which is giving us our meromorphic continuation to all of $\CC$ with poles at exactly the non-positive integers $s\in\{0,-1,-2,\ldots\}.$