\documentclass[../notes.tex]{subfiles}

\begin{document}









Ok, I guess we can start.

\subsection{Logistics}
We're usign Neukirch. The reading for Friday is \S1.1. Office hours are Monday, Wednesday, Friday, 12PM--1PM and 3PM--4PM at 883 Evans. Email is \texttt{voyta@math.berkeley.edu}. There is a bcourses, but the course website is \href{https://math.berkeley.edu/~vojta/254a.html}{\texttt{math.berkeley.edu/\~vojta/254a.html}}.

There will be no final exam, but there will be weekly homeworks. We'll go over most of Chapter 1 (sans the last 2--3 sections), Chapter 2 (sans section 6 and some of 7, 9, 10), some of Chapter 3 (\S1--3), some of Chapter 7\footnote{Analytic!} (a little bit), and some of Chapter 6 (parts of \S1--2).

\subsection{Overview}
Here's our main character.
\begin{defi}[Number Field]
    A \textit{number field} $K$ is a finite field extension of $\QQ.$
\end{defi}
\begin{ex}
    For example, $\QQ(\sqrt2)$ is a number field.
\end{ex}
We often work with the picture that $\ZZ\subseteq\QQ$ and $\mathcal O_K\subseteq K,$ where $\mathcal O_K$ is the integral closure of $\ZZ$ in $K.$ (Namely, $\mathcal O_K$ consists of the roots of monic polynomials with integer coefficients in $\ZZ$ which live in $K.$) The trivial example is that $\ZZ$ is the integral closure of $\ZZ$ in $\QQ$ by the Rational root theorem.
\begin{warn}
    We might sloppily call the elements $\OO_K$ ``integers.'' To distinguish, we may often call $\ZZ$ the ``rational integers.''
\end{warn}

\subsubsection{Chapter 1}
In Chapter 1, we will interest ourselves in $\AA,$ which is the integral closure of $\ZZ$ in $\overline\QQ.$ The questions we ask here are what carries from $\ZZ$ to a particular $\OO_K.$
\[\begin{array}{c|c}
    \ZZ & \mathcal O_K \\\hline
    \text{finite units} & \mathcal O_K^\times\text{ is finitely generated} \\
    \text{PID} & \text{not a PID in general} \\
    \text{UFD} & \text{iff a PID} \\
    & \text{ideals have UPF}
\end{array}\]
In Chapter 1, we will also talk a little about quadratic reciprocity.

\subsubsection{Chapter 2}
In Chapter 2, we localize. The idea is to focus on a single prime, say $\mf p\subseteq\mathcal O_K.$ For example, recall that we're interested in solving Diophantine equations in number theory. In the number field case, we are interested in solutions to equations which might extend to number fields and their rings of integers.

For example, the equation
\[x^2+y^2=-1\]
has no solutions in $\QQ$ because it has local obstructions at the infinite place (namely, no solutions in $\RR$). Similarly,
\[x^2+y^2=7\]
has local obstructions in $\QQ_2$ (check$\pmod4$). The theory in Chapter 2 will let us generalize these arguments.

\subsubsection{Chapter 3}
In Chapter 3, we will study the different and discriminant. Very roughly, these are related to the question of how we might want to study ideal factorization via prime-splitting.

\subsubsection{Chapter 7}
In Chapter 7, we will do some analytic number theory. This is the study of the Riemann $\zeta$ function and friends.

\subsubsection{Chapter 6}
In Chapter 6, we will go over the statements of class field theory. Here we study abelian extension of number fields. If you want proofs, take 254B.

\subsection{Philosophy}
In number theory, we are interested in Diophantine equations and a few other things.
\begin{ex}
    We can solve $xy=24$ with $x,y\in\ZZ$ by using the prime factorization of $24.$
\end{ex}
\begin{ex}
    We can solve $xy=24$ with $x,y\in\QQ$ by parameterization: $(x,y)=(t,24/t)$ for $t\in\QQ^\times.$
\end{ex}
\begin{ex}
    We can solve $x^2-2y^2=1$ with $x,y\in\ZZ$ with more work.
    \begin{enumerate}
        \item We know that signs do not alter solutions. Further, the trick is that $(x,y)\mapsto(3x+4y,2x+3y)$ and $(x,y)\mapsto(3x-4y,-2x+3y)$ also maps solutions. Then we can generate all solutions from $(1,0)$ in this manner.
        
        In general, we can solve $x^2-dy^2=b$ for any $b\in\ZZ$ and $d\in\ZZ\setminus\QQ^{\times2}$ in this manner.
        \item We factor this as $(x+y\sqrt2)(x-y\sqrt2)=1$ and decide to work in $\ZZ[\sqrt2].$ Namely, $x+y\sqrt2\in\ZZ[\sqrt2]^\times$ (its multiplicative inverse is $x-y\sqrt2$), so we see we have to study the units of $\ZZ[\sqrt2].$ (We do have to be careful that $(1-\sqrt2)(1+\sqrt2)=-1$ gives a unit with norm $-1.$)
        
        In fact, we have some sort of converse: if $x+y\sqrt2\in\ZZ[\sqrt2]^\times,$ then let its multiplicative inverse by $x'+y'\sqrt2.$ Multiplying their components, we see
        \[1=(xx'+2yy')+(xy'+x'y)\sqrt2,\]
        forcing the individual components to be $1$ and $0$ respectively (note $\sqrt2$ is irrational). So $y/x=-y'/x',$ and they must both be reduced because $\gcd(x,y)=\gcd(x',y')=1$ from $xx'+2yy'=1.$ So $y=\pm y'$ and $x=\mp x'.$ From this it follows
        \[x^2-2y^2=(x+y\sqrt2)(x-y\sqrt2)=\mp(x+y\sqrt2)(\mp x\pm y\sqrt2)=\mp1.\]
        So indeed, studying units in $\ZZ[\sqrt2]$ will at least get us somewhere because they are almost solutions to $x^2-2y^2=1.$
    \end{enumerate}
\end{ex}
Here is something interesting about the second method here: we can let $\sigma:\QQ(\sqrt2)\to\QQ(\sqrt2)$ by the nontrivial element of $\op{Gal}(\QQ(\sqrt2)/\QQ).$ This lets us define
\[\op N_\QQ^K:\QQ(\sqrt2)^\times\to\QQ^\times\]
whose pre-image from $\{\pm1\}$ consists of our units, and whose pre-image from $1$ is actually our solutions. With some more pushing in this direction, we can recover $x+y\sqrt2=\pm(1+\sqrt2)^{2\bullet}$ as our solutions set to the actual problem.

\end{document}