The reading for Monday is \S1.1 and \S1.2

\subsection{Basic Definitions}
In this course, our rings will always be rings with identity; ring homomorphisms will always send units to units. Our rings will almost always be commutative; we'll see if not. We take the following definitions.
\begin{defi}[Entire/integral domain]
    A ring $R$ is \textit{entire} or \textit{an integral domain} if $1\ne0$ and has no zero divisors. In other words, $R\ne\{0\}$ and $ab=0$ implies $a=0$ or $b=0.$
\end{defi}
We note that $R$ is an integral domain if and only if it can be embedded into a field (namely, its fraction field).
\begin{defi}[Factorial/UFD]
    A ring $R$ is \textit{factorial} if and only if it is an entire and all nonzero elements have a factorization into irreducible elements, up to permutation and associates.
\end{defi}
\begin{ex}
    We have that $\ZZ$ is factorial by elementary number theory.
\end{ex}
\begin{defi}[Principal]
    A ring $R$ is \textit{principal} if and only if $1\ne0$ and every ideal is principal.
\end{defi}
Note we can have principal rings which are not entire, but PIDs are principal rings.
\begin{defi}
    A polynomial $p(x)\in R[x]$ is monic if and only if $p\ne0$ and its leading coefficient is $1.$
\end{defi}

\subsection{Integral Rings}
We start with Gauss's lemma (on polynomials).
\begin{lem}[Gauss] \label{lem:gauss}
    Fix $A$ a unique factorization domain with fraction field $K.$ Fix $f\in A[x]\setminus\{0\}.$ If $f=gh$ for $g,h\in K[x],$ then there is a $c\in K^\times$ such that $f=(cg)(h/c)$ with $cg,h/c\in A[x].$
\end{lem}
In other words, we can turn factorizations from $K[x]$ to $A[x].$
\begin{proof}
    This is an exercise but not in the homework; it's in Neukirch.
\end{proof}
We want to define integral rings, so here is.
\begin{defi}[Integral]
    Let $A\subseteq B$ be rings.
    \begin{enumerate}[label=(\alph*)]
        \item We have $b\in B$ is \textit{integral} over $A$ if $b$ is the root of some monic polynomial $f\in A[x]\setminus\{0\}.$ We will maybe call $f$ an integral dependence relation of $b.$
        \item We have $B$ is integral over $A$ if each $b\in B$ are integral over $A.$
        \item The integral closure of $A$ in $B$ is the set of all $b\in B$ such that $b$ is integral over $A.$
        \item If $A$ is an integral domain, the integral closure of $A$ is the integral closure of $A$ in its fraction field.
    \end{enumerate}
\end{defi}
We will show that the integral closure is a ring later. Note that the integral closure and the algebraic closure are potentially very different objects. The integral closure of $\ZZ$ in $\QQ$ is $\ZZ.$ Here is an example.
\begin{prop}
    The integral closure of $\ZZ$ in $\QQ(\sqrt2)$ is $\ZZ[\sqrt2].$
\end{prop}
\begin{proof}
    In one direction, take $\alpha\in\ZZ[\sqrt2]$ and write $\alpha=a+b\sqrt2$ for $a,b\in\ZZ.$ Then $\alpha$ is a root of the monic polynomial
    \[(x-a)^2-2b^2\in\ZZ[x].\]
    So indeed, $\alpha$ is integral over $\ZZ.$
    
    In the other direction, suppose $\alpha\in\QQ(\sqrt2)$ is integral over $\ZZ.$ We are promised a monic polynomial $f\in\ZZ[x]$ with $f(\alpha)=0.$ Additionally, $\alpha$ is algebraic, so we get $g\in\QQ[x]$ the minimal (monic, irreducible) polynomial of $\alpha.$
    
    Then $g\mid f,$ so we can use \autoref{lem:gauss} to get $c\in K^\times$ such that $cg\in\ZZ[x]$ with $f/(cg)\in\ZZ[x].$ But now $f$ and $g$ are monic, so $c\in\ZZ$ and $1/c\in\ZZ$ as the leading coefficient of $f/(cg),$ from which it follows $c=\pm1,$ so in fact $g\in\ZZ[x]$ all along.
    
    We now have two cases.
    \begin{enumerate}
        \item If $\alpha\in\QQ,$ then $g(x)=x-\alpha,$ so $\alpha\in\ZZ,$ and we are safe.
        \item Otherwise $\alpha=a+b\sqrt2$ with $b\ne0.$ Here $g(x)=(x-a)^2-2b^2$ is our minimal polynomial, so expanding coefficients tells us $2a\in\ZZ$ and $a^2-2b^2\in\ZZ.$
        
        We need to know $a,b\in\ZZ,$ which requires some pushing. We have $a\in\frac12\ZZ,$ so we would like a similar condition on $b.$ Note $8b^2=(2a)^2-4\left(a^2-2b^2\right)\in\ZZ,$ so $2b\in\ZZ$ by checking denominators.
        
        Now things collapse: if $a\notin\ZZ,$ then $2a$ is odd, so $4a^2$ is odd, so $4\left(a^2-2b^2\right)=4a^2-2\cdot(2b)^2$ is also odd, which is a contradiction because $a^2-2b^2\in\ZZ.$ Thus, $a\in\ZZ,$ which forces $b\in\ZZ,$ finishing.
        \qedhere
    \end{enumerate}
\end{proof}
\begin{remark}
    Life is not always so good. In $\QQ(\sqrt5),$ our integral closure is not $\ZZ[\sqrt5]$ because $\frac{1+\sqrt5}2$ is a root of $x^2-x-1\in\ZZ[x].$ However, it turns out that $\ZZ\left[\frac{1+\sqrt5}2\right]$ is our integral closure.
\end{remark}

\subsection{A Better Integrality}
Here are some definitions.
\begin{defi}[Algebraic number]
    An \textit{algebraic number} is an element of the algebraic closure $\overline\QQ$ of $\QQ$ in $\CC.$
\end{defi}
\begin{defi}[Algebraic integer]
    An \textit{algebraic integer} is an algebraic number which is integral over $\ZZ.$
\end{defi}
\begin{defi}[Rational integer]
    A \textit{rational integer} is an element of $\ZZ.$
\end{defi}
We want to talk about integral ring extensions, so we take the following definitions.
\begin{defi}[Finite]
    With $A\subseteq B$ rings, we define $B$ to be \textit{finite} over $A$ or is a \textit{finite $A$-algebra} if $B$ is a finitely generated $A$-module.
\end{defi}
\begin{ex}
    We see $\QQ[t]$ is a finitely generated $\QQ$-algebra (namely, generated by $t$), but it is not a finitely generated $\QQ$-module, so it is not finite over $\QQ.$
\end{ex}
\begin{ex}
    If $K\subseteq L$ are fields, then $L$ is finite over $K$ if and only if $[L:K]$ is finite.
\end{ex}
We want to show that the integral closure is actually a ring.
\begin{defi}[Faithful]
    An $A$-module $M$ is \textit{faithful} if $aM=0$ implies $a=0.$ In other words, the $A$-action does a good job.
\end{defi}
\begin{prop}
    Fix $A\subseteq B$ rings with $b\in B.$ These are equivalent.
    \begin{enumerate}[label=(\alph*)]
        \item $b$ is integral over $A.$
        \item $A[b]$ is finite over $A.$
        \item There is a faithful $A[b]$-module $M$ which is finitely generated as an $A$-module.
    \end{enumerate}
\end{prop}
The last condition is the weirdest and, sadly, the most useful.
\begin{remark}
    In fact, all $A[b]$-modules $M$ with $A[b]\subseteq M\subseteq B$ are faithful because $1\in M$ has $a\cdot1=0$ implies $a=0.$
\end{remark}
\begin{proof}
    We go in sequence.
    \begin{itemize}
        \item We start with (a) implies (b). If $b$ is a zero of a monic polynomial of degree $n$ in $A[x],$ then $A[b]$ is generated as an $A$-module by $\{1,b,\ldots,b^{n-1}\}$ because higher powers can be reduced via the monic polynomial.
        
        \item Next we show (b) implies (c). Here $A[b]$ works.
        
        \item Now we show (c) implies (a). We use the determinant trick. Suppose $\{m_1,\ldots,m_n\}$ generate $M$ over $A.$ Then for eack $k,$ we can write
        \[bm_k=\sum_{\ell=1}^nc_{k,\ell}m_\ell\]
        for some coefficients $c_{k,\ell}\in A.$ These $c_{k,\ell}$ give a matrix, which we name $C.$ Letting $C^*$ be the adjoint matrix, we have $CC^*=(\det C)I_n.$
        
        Now $f(x)=\det(xI_n-C)\in A[x]$ is monic, which we hope is our monic polynomial witnessing the integrality of $b.$ Defining $D:=bI_n-C,$ we see
        \[D\begin{bmatrix}
            m_1 \\
            \vdots \\
            m_n
        \end{bmatrix}=\begin{bmatrix}
            bm_1-\sum_{\ell=1}^nc_{1,\ell}m_\ell \\
            \vdots \\
            bm_n-\sum_{\ell=1}^nc_{1,\ell}m_\ell
        \end{bmatrix}=\begin{bmatrix}
            0 \\
            \vdots \\
            0
        \end{bmatrix}.\]
        In particular, $D^*D\langle m_1,\ldots,m_n\rangle=0,$ so $D^*D=(\det D)I_n$ implies $(\det D)\langle m_1,\ldots,m_n\rangle=0.$ However, $\det D=f(b),$ so in fact $f(b)(m_k)=0$ for all $k.$ And now we are in the home stretch: $M$ is generated by the $m_k,$ so $f(b)M=0,$ so $f(b)=0$ because $M$ is faithful.
        \qedhere
    \end{itemize}
\end{proof}