\documentclass[../notes.tex]{subfiles}

\begin{document}












Reading for Wednesday is \S1.3. There's also a handout on the discriminant of $x^n+ax+b.$

\subsection{Integral Closures}
Last time we (barely) showed the following.
\begin{prop} \label{prop:betterint}
    Let $A\subseteq B$ be rings and let $b\in B.$ Then the following are equivalent.
    \begin{enumerate}[label=(\alph*)]
        \item $b$ is integral over $A.$
        \item $A[b]$ is finite over $A.$
        \item There is a faithful $A[b]$-module $M$ which is finitely generated as an $A$-module.
    \end{enumerate}
\end{prop}
Let's begin to reap the rewards of this. We want to show that the integral closure of a ring in a larger ring is in fact a ring, so we move towards there.
\begin{lem}
    Let $A\subseteq B$ be rings with $b_1,b_2\in B.$ If $b_1$ and $b_2$ are both integral over $A,$ then $b_1+b_1$ and $b_1-b_2$ and $b_1b_2$ are also integral over $A.$
\end{lem}
\begin{proof}
    We imitate the proof of algebraic-ness for fields. Because $b_2$ is integral over $A,$ it will remain integral over $A[b_1]$ (simply inherit the same polynomial: $A[x]\subseteq A[b_1][x]$). Then $A[b_1,b_2]$ is finite over $A[b_1]$ by \autoref{prop:betterint}. The point is that we get the following tower of rings which are finite over each other so that $A[b_1,b_2]$ is finite over $A.$ (We leave working out this last assertion an exercise, though not a hard one.)
    \[\begin{tikzcd}
    	{A[b_1,b_2]} \\
    	{A[b_1]} \\
    	A
    	\arrow["{\text{finite}}", no head, from=3-1, to=2-1]
    	\arrow["{\text{finite}}", no head, from=2-1, to=1-1]
    \end{tikzcd}\]
    In particular, all elements of $A[b_1,b_2]$ are finite over $A,$ so all the listed objects in the conclusion of our lemma are in fact integral over $A.$ (Here we are taking $M=A[b_1,b_2]$ in \autoref{prop:betterint}.)
\end{proof}

So we get the following corollary for our efforts.
\begin{thm}
    Let $A\subseteq B$ be rings. Then the integral closure of $A$ in $B$ is a subring of $B$ containing $A.$
\end{thm}
\begin{proof}
    The previous lemma shows that the integral closure is in fact closed under our operations. To show that the integral closure contains $A,$ we know that any $a\in A$ is a root of the polynomial $x-a\in A[x].$
\end{proof}
\begin{remark}
    It is not necessary for the integral closure of $A$ in $B$ to be finite over $A.$ For example, $\overline\ZZ,$ which is the integral closure of $\ZZ$ in $\CC,$ is not finite over $\ZZ.$
\end{remark}

We now take the following definition.
\begin{defi}[Integrally closed]
    Let $A\subseteq B$ be rings.
    \begin{enumerate}[label=(\alph*)]
        \item If the integral closure of $A$ in $B$ equals $A,$ then we say that $A$ is \textit{integrally closed}.
        \item Take $A$ entire. Then $A$ is \textit{integrally closed} if it is integrally closed in its fraction field.
    \end{enumerate}
\end{defi}
Last time we defined integral closure, so here we are defining what it means for the lower ring to be integrally closed.

We hope that the integral closure is in fact integrally closed. Let's show this.
\begin{lem}
    Let $A\subseteq B\subseteq C$ be rings. If $B$ is integral over $A$ and $C$ is integral over $B,$ then $C$ is integral over $A.$
\end{lem}
\begin{proof}
    Let $c\in C,$ which we know is integral over $B.$ Then we are promised a monic polynomial in $B[x]$ such that
    \[c^n+\sum_{k=0}^{n-1}b_kc^k=0.\]
    By induction, we see that $A[b_1,\ldots,b_n]$ is finite over $A$ (add the elements one at a time), and in fact $A[b_1,\ldots,b_n,c]$ is finite over $A$ as well because of the above polynomial. So it follows that $c$ is integral over $A$ by \autoref{prop:betterint}.
\end{proof}
And this gives us our result.
\begin{prop}
    Let $A\subseteq B$ be rings. Then the integral closure of $A$ in $B$ is integrally closed in $B.$
\end{prop}
\begin{proof}
    Rename $B$ to $C$ and set $B$ to the integral closure of $A$ in $C.$
\end{proof}
\begin{prop}
    Let $A$ be an entire rings. Then the integral closure of $A$ is integrally closed.
\end{prop}
\begin{proof}
    Take $B=\op{Frac}(A)$ in the previous proposition.
\end{proof}
In fact, we have the following.
\begin{prop}
    Any factorial, entire ring $A$ is integrally closed.
\end{prop}
\begin{proof}
    This essentially follows from \autoref{lem:gauss}. Let $K$ be the fraction field of $A$ and $\alpha\in K$ such that $\alpha$ is integral over $A$ so that we want to show $\alpha\in A.$ Well, we are promised a monic polynomial $f\in A[x]$ such that $f(\alpha)=0.$
    
    Note that then $(x-\alpha)\mid f(x)$ in $K[x]$ and both $f$ and $x-\alpha$ are monic, and $f\in A[x],$ so we are forced into having $x-\alpha\in A[x].$ So it follows $\alpha\in A.$
\end{proof}

\subsection{The \texorpdfstring{$AKLB$}{} Setup}
We will have the following situation a lot.
% https://q.uiver.app/?q=WzAsNCxbMCwxLCJBIl0sWzEsMSwiSyJdLFswLDAsIkIiXSxbMSwwLCJMIl0sWzEsMywiIiwwLHsic3R5bGUiOnsiaGVhZCI6eyJuYW1lIjoibm9uZSJ9fX1dLFswLDEsIlxcc3Vic2V0ZXEiLDEseyJzdHlsZSI6eyJib2R5Ijp7Im5hbWUiOiJub25lIn0sImhlYWQiOnsibmFtZSI6Im5vbmUifX19XSxbMiwzLCJcXHN1YnNldGVxIiwxLHsic3R5bGUiOnsiYm9keSI6eyJuYW1lIjoibm9uZSJ9LCJoZWFkIjp7Im5hbWUiOiJub25lIn19fV0sWzAsMiwiIiwxLHsic3R5bGUiOnsiYm9keSI6eyJuYW1lIjoiZGFzaGVkIn0sImhlYWQiOnsibmFtZSI6Im5vbmUifX19XV0=
\[\begin{tikzcd}
	B & L \\
	A & K
	\arrow[no head, from=2-2, to=1-2]
	\arrow["\subseteq"{description}, draw=none, from=2-1, to=2-2]
	\arrow["\subseteq"{description}, draw=none, from=1-1, to=1-2]
	\arrow[dashed, no head, from=2-1, to=1-1]
\end{tikzcd}\]
Here, $A$ is integrally closed, $K$ is its fraction field, $L$ is a finite field extension of $K,$ and $B$ is the integral closure of $A$ in $L.$ In this case, note that because $A$ is integrally closed, $B\cap K=A.$ (All elements of $B$ are integral over $A.$)
\begin{remark}
    We define $L$ before $B$ because there are potentially ``lots'' of base rings 
\end{remark}

We hope that the $B\subseteq L$ is actually a fraction field containment.
\begin{prop}
    We have that $L$ is the fraction field of $B.$
\end{prop}
\begin{proof}
    Let $\alpha\in L.$ Then we show stronger: we show that there is a nonzero $a\in A$ such that $a\alpha\in B.$ (In other words, the denominator lives in $A.$) Fix $f(x)\in K[x]$ the monic irreducible polynomial promised for $\alpha$ over $K.$ For concreteness,
    \[f(x)=x^n+a_{n-1}x^{n-1}+\cdots+a_0x^0\]
    for constants $a_0,\ldots,a_{n-1}\in K.$ The point is that we can choose $a\in A$ equal to the product of the denominators of the $a_\bullet$ so that $a^n\cdot f(x)=0$ reads
    \[(a\alpha)^n+(aa_{n-1}(a\alpha)^{n-1}+(a^2a_{n-2})(a\alpha)^{n-2}+\cdots+(a^na_0)=0.\]
    So this is a monic polynomial for $a\alpha$ that lives in $A[x],$ so $a\alpha\in B$ because $B$ is the integral closure of $B$ in $L.$
\end{proof}
We take the following.
\begin{prop}
    Take the $AKLB$ setup. Assume that $A$ is factorial, and let $\alpha\in L.$ Then $\alpha\in B$ if and only if the irreducible polynomial for $\alpha$ in $K$ lies in $A[x].$
\end{prop}
\begin{proof}
    The backwards direction has nothing to show; the forward direction is by Gauss's lemma.\todo{}
\end{proof}

\subsection{Norm and Trace}
We like measuring sizes of things, but in this class we want our sizes to be Galois-invariant. Here is one way we do this.
\begin{defi}[Norm and trace]
    Fix $L/K$ a finite field extension, and let $\alpha\in L.$ Viewing $L$ as a $K$-vector space of dimension $[L:K]<\infty,$ we define $T_\alpha:x\mapsto\alpha x$ as a $K$-linear transformation $L\to L.$ This lets us define
    \[\op N^K_L(\alpha):=\det T_\alpha\qquad\text{and}\qquad\op T^K_L(\alpha)=\op{trace}T_\alpha.\]
    The function $\op N^K_L$ is called the \textit{norm}, and the function $\op T^K_L$ is called the \textit{trace}.
\end{defi}
\begin{remark}
    Importantly, we have the following properties.
    \begin{itemize}
        \item $\op N_K^L(\alpha)\in \alpha$ and $\op T^K_L(\alpha)\in K$ because we are viewing these as $K$-linear transformations.
        \item For $\alpha\ne0$ is equivalent to $T_\alpha$ is invertible, so $\alpha\ne0$ is equivalent to $\op N^K_L(\alpha)=0.$
        \item $\op N_L^K$ is multiplicative because $T_{\alpha\beta}=T_\alpha\circ T_\beta,$ and the determinant is multiplicative.
        \item $\op T_L^K$ is additive because $T_{\alpha+\beta}=T_\alpha+T_\beta,$ and the trace is additive.
    \end{itemize}
\end{remark}
We also have the following definition.
\begin{prop}
    Fix an algebraic closure $\overline K$ of $K,$ and let $\sigma_1,\ldots,\sigma_d$ be distinct embeddings of $L$ into $\overline K$ which fix $K.$ (Note $d=[L:K]^{\op{sep}}.$) Then we have
    \[\op N_K^L(\alpha)=\prod_{k=1}^n\sigma_k(\alpha)^{[L:K]^{\op{insep}}}\qquad\text{and}\qquad\op T_K^L(\alpha)=[L:K]^{\op{insep}}\sum_{k=1}^d\sigma_j(\alpha).\]
    Here, $[L:K]^{\op{insep}}$ is the inseparable degree.
\end{prop}
\begin{proof}
    This is Proposition 2.6 in Neukirch or Chapter IV, \S5 in Lang's \textit{Algebra}.
\end{proof}
The point of this is that the norm and the trace can detect integrality, albeit weakly.
\begin{prop}
    Take the $AKLB$ setup. If $\alpha\in B,$ then $\op N_L^K(\alpha)$ and $\op T_K^L(\alpha)$ are in $A.$
\end{prop}
\begin{proof}
    The point is that, for each embedding $\sigma:L\to\overline K$ fixing $K,$ we have that $\sigma\alpha$ is integral over $A.$ Then we use the above characterization.
\end{proof}
\begin{cor}
    If $M/L/K$ are finite extensions of fields, then $\op N_K^L\circ\op N^M_L=\op N^M_K$ and $\op T^L_K\circ\op T^M_L=\op T^M_K.$
\end{cor}
\begin{proof}
    Compose embeddings of $M\to L$ and of $L\to K$ to get all the embeddings of $M\to K.$
\end{proof}
\begin{cor}
    If $L=K(\alpha)$ for some $\alpha$ with monic irreducible polynomial given by
    \[f(x)=x^n-a_{n-1}x^{n-1}+a_{n-2}x^{n-2}+\cdots+(-1)^na_0,\]
    then $f(x)=\prod_{k=1}^d(x-\sigma_k(\alpha))^{[K(\alpha):K]^{\op{insep}}}$ so that $\op N^L_K\alpha=a_0$ and $\op T_K^L\alpha=a_{n-1}.$
\end{cor}
\begin{proof}
    Given in the statement.
\end{proof}
The above corollary is especially nice for quadratics.

\end{document}