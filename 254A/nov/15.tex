% !TEX root = ../notes.tex
















Stuff happens.

\subsection{Loose Ends}
A while ago we showed that, if $K$ is a complete discretely valued prime and $\mf p\subseteq A$ is the unique maximal ideal of the valuation ring, then there is a unique prime $\mf q$ over $\mf p$ in the extension $L/K$ of complete discretely valued fields.

We also remark that our definition of complete discretely valued fields admits the trivial valuation for any field. This is okay with us, for now.

\subsection{Unramified Extensions}
Neukirch also talks about tamely ramified extensions, but we will probably skip it, at least for now. We set $K$ to be a discretely valued field with absolute value $|\cdot|$ and valuation $\nu:K\onto\ZZ\cup\{\infty\}.$ We will give it valuation ring $A$ and $\mf p$ the valuation ideal and $\kappa:=A/\mf p$ the residue field. And to make things interesting, we take $L/K$ a finite extension, with $B,\mf q,\lambda$ fixed as in the previous sentence; set $w:L\to\QQ\cup\{\infty\}$ the corresponding valuation.
\begin{definition}[Unramified extension]
	Fix everything as above. We say that $L/K$ is \textit{unramified} if and only if $e(\mf q/\mf p)=1$ and $\lambda/\kappa$ is a separable extension.
\end{definition}
\begin{remark}
	We have that being unramified is equivalent to having $[L:K]=[\lambda:\kappa]$ because we already know $[L:K]=e(\mf q/\mf p)f(\mf q/\mf p),$ given that $\lambda/\kappa$ is separable.
\end{remark}
We have the following small results.
\begin{proposition}
	We have the following.
	\begin{itemize}
		\item Fix $M/L/K$ finite extensions. Then $M/K$ is unramified is equivalent to $M/L$ and $L/K$ being unramified.
		\item A finite Galois extension $L/K$ is unramified if and only if $I_{\mf q/\mf p}=(1).$
	\end{itemize}
\end{proposition}
\begin{proof}
	We take these one at at time.
	\begin{itemize}
		\item The ramification index condition comes directly from the tower law for $e.$ The separability follows from field theory.
		\item We have that $I_{\mf q/\mf p}=(1)$ if and only if $L/K$ unramified because $\#I=e(\mf q/\mf p),$ given that $\lambda/\kappa$ is separable. (Namely, $\#I=e(\mf q/\mf p)[L:K]_{\op{sep}}.$).
		\qedhere
	\end{itemize}
\end{proof}
Let's move into something more serious.
\begin{proposition} \label{prop:liftcdfs}
	Fix the following diagram of fields.
	% https://q.uiver.app/?q=WzAsNCxbMSwyLCJLIl0sWzAsMSwiTCJdLFsxLDAsIkwnIl0sWzIsMSwiSyciXSxbMiwxLCIiLDAseyJzdHlsZSI6eyJoZWFkIjp7Im5hbWUiOiJub25lIn19fV0sWzEsMCwiIiwwLHsic3R5bGUiOnsiaGVhZCI6eyJuYW1lIjoibm9uZSJ9fX1dLFsyLDMsIiIsMix7InN0eWxlIjp7ImhlYWQiOnsibmFtZSI6Im5vbmUifX19XSxbMywwLCIiLDIseyJzdHlsZSI6eyJoZWFkIjp7Im5hbWUiOiJub25lIn19fV1d
	\[\begin{tikzcd}
		& {L'} \\
		L && {K'} \\
		& K
		\arrow[no head, from=1-2, to=2-1]
		\arrow[no head, from=2-1, to=3-2]
		\arrow[no head, from=1-2, to=2-3]
		\arrow[no head, from=2-3, to=3-2]
	\end{tikzcd}\]
	Namely, take $K$ and $K'$ are complete discretely valued fields (such that $K'/K$ have compatible absolute values) with $L/K$ algebraic and $L'=LK'.$ Then if $L/K$ is finite and unramified, then $L'/K'$ is finite and unramified.
\end{proposition}
\begin{proof}
	The finiteness of $L'/K'$ follows from the finiteness of $L/K$ and some field theory I guess. So now take $L'/K'$ finite and unramified; fix $A',\mf p',\kappa'$ and $B',\mf q',\lambda'$ to be what they should be for $K'$ and $L'.$
	
	Now, fix $\overline\alpha$ a primitive element for the extension $\lambda/\kappa$; lifting it to some $\alpha\in B$ somehow, so fix $f$ as the irreducible polynomial for $\alpha.$ It follows that $f$ is monic because $K$ is complete, using the strong form of Gauss's lemma for complete nonarchimedean fields. Reducing, we take $\overline f\in\kappa[x]$ so that $\overline f(\overline\alpha)=0.$ Now,
	\[\deg\overline f=\deg f\le[L:K]=[\lambda:\kappa]=[\kappa(\overline\alpha):\kappa]\le\deg\overline f,\]
	where the last inequality holds because $\overline\alpha$ is a primitive element for $\lambda/\kappa,$ and $\overline f$ vanishes on $\overline\alpha$ (though we do not know yet if it is irreducible). So we get that $\deg f=[L:K]$ and so $L=K(\alpha)$; additionally, we see that $\overline f\in\kappa[x]$ is irreducible due to the equality case.

	We also see that $L'(\alpha)=LK'(\alpha)=K'(\alpha),$ so set $g\in A'[x]$ to be the irreducible polynomial for $\alpha$ over $K'.$ Now, $\overline g\mid\overline f,$ so $\overline g$ is separable. Additionally, $\overline g$ is irreducible, for otherwise when we factor it into irreducible factors in $\kappa'[x],$ we could lift it via Hensel to a factorization in $K'[x].$ So
	\[[\lambda':\kappa']\le[L':K']=\deg g=\deg\overline g=[\kappa'(\alpha):\kappa']\le[\lambda':\kappa'].\]
	So we get equalities all the way down, meaning that $[\lambda':\kappa']=[L':K'],$ and $\lambda'=\kappa'(\alpha)$ is generated by a separable element, so we do indeed see that $L'/K'$ is unramified.
\end{proof}
\begin{proposition}
	Fix $K$ a global field (or a function field in one variable with constant field inside of the valuation ring). Now take $K'/K$ an extension complete with respect to a valuation $\nu_\mf p:K\to\ZZ$ for some nonzero prime $\mf p$ of $A$ so that we can take $(K',|\cdot|_\mf p)$ a complete discretely valued field.

	Further, we take $L/K$ a finite extension with $B$ the integral closure of $A$ in $L$; take $\mf q$ a prime of $B$ lying over $\mf p$ with $\mf q/\mf p$ unramified. Then, $L'=K'L$ is unramified over $K'.$
	% https://q.uiver.app/?q=WzAsNixbMiwwLCJLIl0sWzIsMSwiTCJdLFsxLDAsIkEiXSxbMSwxLCJCIl0sWzAsMCwiXFxtZiBwIl0sWzAsMSwiXFxtZiBxIl0sWzQsMiwiXFxzdWJzZXRlcSIsMyx7InN0eWxlIjp7ImJvZHkiOnsibmFtZSI6Im5vbmUifSwiaGVhZCI6eyJuYW1lIjoibm9uZSJ9fX1dLFsyLDAsIlxcc3Vic2V0ZXEiLDMseyJzdHlsZSI6eyJib2R5Ijp7Im5hbWUiOiJub25lIn0sImhlYWQiOnsibmFtZSI6Im5vbmUifX19XSxbMywxLCJcXHN1YnNldGVxIiwzLHsic3R5bGUiOnsiYm9keSI6eyJuYW1lIjoibm9uZSJ9LCJoZWFkIjp7Im5hbWUiOiJub25lIn19fV0sWzUsMywiXFxzdWJzZXRlcSIsMyx7InN0eWxlIjp7ImJvZHkiOnsibmFtZSI6Im5vbmUifSwiaGVhZCI6eyJuYW1lIjoibm9uZSJ9fX1dLFs0LDUsIiIsMyx7InN0eWxlIjp7ImhlYWQiOnsibmFtZSI6Im5vbmUifX19XSxbMiwzLCIiLDMseyJzdHlsZSI6eyJoZWFkIjp7Im5hbWUiOiJub25lIn19fV0sWzAsMSwiIiwzLHsic3R5bGUiOnsiaGVhZCI6eyJuYW1lIjoibm9uZSJ9fX1dXQ==
	\[\begin{tikzcd}
		{\mf p} & A & K \\
		{\mf q} & B & L
		\arrow["\subseteq"{marking}, draw=none, from=1-1, to=1-2]
		\arrow["\subseteq"{marking}, draw=none, from=1-2, to=1-3]
		\arrow["\subseteq"{marking}, draw=none, from=2-2, to=2-3]
		\arrow["\subseteq"{marking}, draw=none, from=2-1, to=2-2]
		\arrow[no head, from=1-1, to=2-1]
		\arrow[no head, from=1-2, to=2-2]
		\arrow[no head, from=1-3, to=2-3]
	\end{tikzcd}\]
\end{proposition}
\begin{proof}
	We build the following diagram, where $\hat K$ and $\hat L$ are the completions of $|\cdot|_\mf p$ and $|\cdot|_\mf q$ respectively.
	% https://q.uiver.app/?q=WzAsNixbMCwyLCJMIl0sWzAsMywiSyJdLFsxLDIsIlxcaGF0IEsiXSxbMSwxLCJcXGhhdCBMIl0sWzIsMCwiTCciXSxbMiwxLCJLJyJdLFswLDEsIiIsMCx7InN0eWxlIjp7ImhlYWQiOnsibmFtZSI6Im5vbmUifX19XSxbMywyLCIiLDAseyJzdHlsZSI6eyJoZWFkIjp7Im5hbWUiOiJub25lIn19fV0sWzQsNSwiIiwwLHsic3R5bGUiOnsiaGVhZCI6eyJuYW1lIjoibm9uZSJ9fX1dLFs0LDMsIiIsMSx7InN0eWxlIjp7ImhlYWQiOnsibmFtZSI6Im5vbmUifX19XSxbNSwyLCIiLDEseyJzdHlsZSI6eyJoZWFkIjp7Im5hbWUiOiJub25lIn19fV0sWzMsMCwiIiwxLHsic3R5bGUiOnsiaGVhZCI6eyJuYW1lIjoibm9uZSJ9fX1dLFsyLDEsIiIsMSx7InN0eWxlIjp7ImhlYWQiOnsibmFtZSI6Im5vbmUifX19XV0=
	\[\begin{tikzcd}
		&& {L'} \\
		& {\hat L} & {K'} \\
		L & {\hat K} \\
		K
		\arrow[no head, from=3-1, to=4-1]
		\arrow[no head, from=2-2, to=3-2]
		\arrow[no head, from=1-3, to=2-3]
		\arrow[no head, from=1-3, to=2-2]
		\arrow[no head, from=2-3, to=3-2]
		\arrow[no head, from=2-2, to=3-1]
		\arrow[no head, from=3-2, to=4-1]
	\end{tikzcd}\]
	The point is that $\hat L/\hat K$ is unramified, essentially by just looking at our local picture, which makes $L'/K'$ unramified by \autoref{prop:liftcdfs}.
\end{proof}
The converse of this statement is also true.
\begin{prop}
	For an extension of global fields $L/K$ in the $AKLB$ set-up, then upon taking $\hat L$ the completion of $|\cdot|_\mf q$ for some prime $\mf q$ of $B$ above $\mf p$ in $A,$ we also take $\hat K$ the completion of $K$ with respect to $|\cdot|_\mf p,$ we claim that $\mf q$ is unramified over $\mf p.$
\end{prop}
\begin{proof}
	The residue fields of $\hat K$ and $\hat L$ become the residue fields for $L$ and $K,$ and the ramification indices match as well. So all the data matches as needed.
\end{proof}

\subsection{Bigger Unramified Extensions}
Let's return to complete fields.
\begin{corollary}
	Finite unramified extensions form a distinguished class of extensions, where we borrow the notion from Lang's \textit{Algebra}. In particular, the composite of finite unramified extensions is also finite unramified.
\end{corollary}
\begin{proof}
	Being a distinguished class essentially means preserved under towers and liftings, all of which we have discussed already (e.g., see \autoref{prop:liftcdfs}).
\end{proof}
This gives us the following definitin.
\begin{defi}[Unramified, II]
	Fix $L/K$ an algebraic extensions of complete discretely valued fields. Then $L/K$ is unramified if and only if all of its finite subextensions are unramified.
\end{defi}
Now that this makes sense even for finite unramified extensions because we know that subextensions are unramified when the bigger extension is unramified.

With this notion, we get to talk about the following.
\begin{corollary}
	Any finite algebraic extension $L/K$ of complete discretely valued fields has a largest unramified subextension, equal to the composite of all finite unramified extensions.
\end{corollary}
\begin{proof}
	Continually lift with \autoref{prop:liftcdfs} until we are done by transfinite induction.
\end{proof}
So this gives us the following definition.
\begin{defi}
	We define $K^{\op{unram}}$ to be the largest unramified extension of $\overline K/K,$ where $\overline K$ is the algebraic closure.
\end{defi}
Note that even infinite unramified subextensions of $\overline K$ will be contained in $K^{\op{unram}}$ because all their finite subextensions will be contained in $K^{\op{unram}}.$

Anyways, we hvae the following proposition.
\begin{proposition}
	Fix $L/K$ an algebraic extension with $T/K$ the largest unramified extension. Fixing $\kappa,\lambda,\tau$ the residue fields of $K,L,T$ as one would expect, we have that $\tau$ is the separable closure of $\kappa$ in $\lambda,$ as in the following diagram.
	% https://q.uiver.app/?q=WzAsMyxbMCwyLCJcXGthcHBhIl0sWzAsMSwiXFx0YXUiXSxbMCwwLCJcXGxhbWJkYSJdLFsyLDEsIlxcb3B7cHVyZWx5IGluc2VwfSIsMCx7InN0eWxlIjp7ImhlYWQiOnsibmFtZSI6Im5vbmUifX19XSxbMSwwLCJcXG9we3NlcH0iLDAseyJzdHlsZSI6eyJoZWFkIjp7Im5hbWUiOiJub25lIn19fV1d
	\[\begin{tikzcd}
		\lambda \\
		\tau \\
		\kappa
		\arrow["{\op{purely~insep}}", no head, from=1-1, to=2-1]
		\arrow["{\op{sep}}", no head, from=2-1, to=3-1]
	\end{tikzcd}\]
\end{proposition}
\begin{proof}
	Well, fix $\kappa'$ the separable closure of $\kappa$ in $\lambda.$ We see that $\tau\subseteq\kappa'$ because $\tau$ is generated by the residue fields of finite unramified subextensions of $L/K,$ and all of these residue fields are separable over $\kappa$ by definition of unramified. Explicitly, given $\overline\alpha\in\tau,$ we can lift $\overline\alpha$ to some $\alpha\in L,$ but then $K(\alpha)$ will be unramified over $K,$ so $\kappa(\overline\alpha)/\kappa$ will be a separable extension, so $\overline\alpha\in\kappa'.$

	For the other direction, we reverse the steps. Fix $\overline\alpha\in\kappa'$ and take $\overline f$ the irreducible polynomial for $\overline\alpha$ over $\kappa.$ Then, lifting $\overline f$ to a monic polynomial
	\[f\in A[x].\]
	Irreducibility of $\overline f$ induces irreducibility of $f,$ and Hensel's lemma promises a root $\alpha\in L$ which reduces to $\overline\alpha\in\lambda,$ due to the separability of $f$ (see Newton's method or something). But now we can check
	\[[\kappa(\alpha):\kappa]=\deg\overline f=\deg f=[K(\alpha):K]\]
	with $\kappa(\alpha)/\kappa$ unramified. It follows that $K(\alpha)$ is unramified over $K,$ so $\alpha\in T,$ so $\overline\alpha\in\tau.$ This finishes.
\end{proof}


% is the converse of this true?