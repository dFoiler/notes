% !TEX root = ../notes.tex
















Let's just get to the point.

\subsection{Global Fields}
Last time we ended by talking about global fields, which were defined as follows.
\begin{definition}[Global field]
	A \textit{global field} is either a finite extension of $\QQ$ (a number field) or a finite extension of $\FF_p[t]$ (a finitely generated extension of a finite field of transcendence degree $1$).
\end{definition}
In good number theory, all global fields will be considered at once. We remark that we might want to look at any finite extension of $F(t_1,\ldots,t_d)$ for arbitrary fields $F,$ but these are not global. Let's discuss why we restrict to $d=1$ in the above definition.

In what follows, fix $F$ a field and $K$ a finite extension of $F(t).$ Notably, we are not requiring $K$ to be global, but it might be easier psychologically to make $F$ finite so that this is the case. In analogy with the case of number fields, we would like to create a ring of integers in $K,$ but it is not well-defined here: both
\[F[t]\qquad\text{and}\qquad F[1/t]\]
are Dedekind rings with fraction field $F(t),$ so we have difficulties even in the case where $K=F[t].$ Regardless, being Dedekind appears to be good candidate for being a ring of integers.

So let's motivate our definition of a global field. For $F$ a field and $A$ a (non-field) Dedekind ring of finite type over $F$ with fraction field $K.$ Then the transcendence degree of $K$ over $F$ is equal to $1,$ by some result in algebraic geometry. It follows $K$ is a finite extension of $F(t),$ for any transcendental element $t\in F.$
\begin{definition}[Funtion field]
	Fix everything as in the previous paragraph. Then $K$ is a \textit{function field in one variable over }$F,$ where $F$ is our constant field for $K.$
\end{definition}
\begin{remark}
	A finite extension of $F(t_1,\ldots,t_d)$ is a ``function field in $d$ variables over $F.$'' Note that this does cover our $d=1$ case (in contrast to the above definition) because the integral closure of $F[t]$ in $K$ will be a Dedekind ring in $K$ and can serve the role of $A$ above. We will not elaborate on this here.
\end{remark}
So now we fix $K$ a number field or function field in one variable over an arbitrary field $F$ (and hence not necessarily global), and let $A=\mathcal O_K$ in the number field case or a Dedekind subring of $K$ in the function field case.

\subsection{Global \texorpdfstring{$AKLB$}{} Set-Up}
Now let's build an $AKLB$ set-up for our global fields. Fix $L/K$ a finite extension of fields, and we take $B$ to be the integral closure of $A$ in $L$; probably following from some theory we developed long ago, $B$ is finite over $A$ and Dedekind, giving the following diagram.
% https://q.uiver.app/?q=WzAsNCxbMCwxLCJBIl0sWzAsMCwiQiJdLFsxLDAsIkwiXSxbMSwxLCJLIl0sWzMsMiwiIiwwLHsic3R5bGUiOnsiaGVhZCI6eyJuYW1lIjoibm9uZSJ9fX1dLFswLDEsIiIsMCx7InN0eWxlIjp7ImhlYWQiOnsibmFtZSI6Im5vbmUifX19XSxbMSwyLCJcXHN1YnNldGVxIiwxLHsic3R5bGUiOnsiYm9keSI6eyJuYW1lIjoibm9uZSJ9LCJoZWFkIjp7Im5hbWUiOiJub25lIn19fV0sWzAsMywiXFxzdWJzZXRlcSIsMSx7InN0eWxlIjp7ImJvZHkiOnsibmFtZSI6Im5vbmUifSwiaGVhZCI6eyJuYW1lIjoibm9uZSJ9fX1dXQ==
\[\begin{tikzcd}
	B & L \\
	A & K
	\arrow[no head, from=2-2, to=1-2]
	\arrow[no head, from=2-1, to=1-1]
	\arrow["\subseteq"{description}, draw=none, from=1-1, to=1-2]
	\arrow["\subseteq"{description}, draw=none, from=2-1, to=2-2]
\end{tikzcd}\]
Further, we fix $\mf p$ a nonzero prime in $A,$ and let $\mf q_1,\ldots,\mf q_r$ be the primes of $\mf p$ lying over $\mf p,$ which we can add to the diagram as follows. (Namely, extensions of Dedekind rings have the usual prime ideal theory.)
% https://q.uiver.app/?q=WzAsNixbMSwxLCJBIl0sWzEsMCwiQiJdLFsyLDAsIkwiXSxbMiwxLCJLIl0sWzAsMSwiXFxtZiBwIl0sWzAsMCwiXFxtZiBxX1xcYnVsbGV0Il0sWzMsMiwiIiwwLHsic3R5bGUiOnsiaGVhZCI6eyJuYW1lIjoibm9uZSJ9fX1dLFswLDEsIiIsMCx7InN0eWxlIjp7ImhlYWQiOnsibmFtZSI6Im5vbmUifX19XSxbMSwyLCJcXHN1YnNldGVxIiwxLHsic3R5bGUiOnsiYm9keSI6eyJuYW1lIjoibm9uZSJ9LCJoZWFkIjp7Im5hbWUiOiJub25lIn19fV0sWzAsMywiXFxzdWJzZXRlcSIsMSx7InN0eWxlIjp7ImJvZHkiOnsibmFtZSI6Im5vbmUifSwiaGVhZCI6eyJuYW1lIjoibm9uZSJ9fX1dLFs0LDUsIiIsMSx7InN0eWxlIjp7ImhlYWQiOnsibmFtZSI6Im5vbmUifX19XSxbNSwxLCJcXHN1YnNldGVxIiwxLHsic3R5bGUiOnsiYm9keSI6eyJuYW1lIjoibm9uZSJ9LCJoZWFkIjp7Im5hbWUiOiJub25lIn19fV0sWzQsMCwiXFxzdWJzZXRlcSIsMSx7InN0eWxlIjp7ImJvZHkiOnsibmFtZSI6Im5vbmUifSwiaGVhZCI6eyJuYW1lIjoibm9uZSJ9fX1dXQ==&macro_url=https%3A%2F%2Fgist.githubusercontent.com%2FdFoiler%2F1e12fec404cad7e185260f0c9b68977d%2Fraw%2F909cc7837a29133fb63fb0e9300d15bfe7417fc5%2Fnir.sty
\[\begin{tikzcd}
	{\mf q_\bullet} & B & L \\
	{\mf p} & A & K
	\arrow[no head, from=2-3, to=1-3]
	\arrow[no head, from=2-2, to=1-2]
	\arrow["\subseteq"{description}, draw=none, from=1-2, to=1-3]
	\arrow["\subseteq"{description}, draw=none, from=2-2, to=2-3]
	\arrow[no head, from=2-1, to=1-1]
	\arrow["\subseteq"{description}, draw=none, from=1-1, to=1-2]
	\arrow["\subseteq"{description}, draw=none, from=2-1, to=2-2]
\end{tikzcd}\]
Now we focus locally. As usual, each nonzero prime $\mf p$ will induce a nonarchimedean absolute value on $K$ by $|x|_\mf p:=c^{-\nu_\mf p(x)}$ for some fixed $c>0.$ We set $\hat K$ to be the completion of $K$ with respect to $|\cdot|_\mf p$ and set
\[\hat A\cong\limit A/\mf p^\bullet\]
to be the valuation ring in $\hat K$ with $\hat{\mf p}$ is the maximal ideal of $\hat A.$ By checking the uniformizer of $\mf p,$ we see that $\mf p\hat A=\hat{\mf p}.$

We remark that if we pull back $\hat A\subseteq\hat K$ along $\iota:K\into\hat K$ to $K,$ then we get the localization $A_\mf p$; similarly, if we pull back $\hat{\mf p},$ then we get $\mf pA_\mf p.$ More generally, we can check that $\hat A/\hat{\mf p}^\bullet\cong A/\mf p^\bullet$ and the pull-back of $\hat{\mf p}^\bullet$ to $K$ is $\mf p^\bullet$ and $\hat{\mf p}^\bullet=\mf p^n\hat A.$

We would like to extend $|\cdot|_\mf p$ up to $L$ to continus our story. It happens that these extensions are in bijection with the primes $\mf q_\bullet$ above $\mf p,$ where $\mf q_\bullet$ induces a nonarchimedean valuation in the usual way. We will fix a chosen prime $\mf q$ over $\mf p$ so that $|\cdot|_\mf q:L\to\RR$ extends $|\cdot|_\mf p:K\to\RR.$

So we set $\hat L$ to be the completion of $(L,|\cdot|_\mf q),$ and we will take care of our book-keeping by setting $j:L\into\hat L$ and 
\[\hat B\cong\limit B/\mf q^\bullet\]
to be the valuation ring of $\hat L.$ Quickly, we see that the universal property of $\hat K,$ there is a unique map $\hat K\to\hat L$ of fields commuting (namely, $K\into L\into\hat L,$ and $\hat L$ is complete), which is injective automatically because this is a mapping of fields. So we get the following local diagram.
% https://q.uiver.app/?q=WzAsNCxbMSwxLCJcXGhhdCBLIl0sWzEsMCwiXFxoYXQgTCJdLFswLDAsIlxcaGF0IEIiXSxbMCwxLCJcXGhhdCBBIl0sWzIsMSwiXFxzdWJzZXRlcSIsMyx7InN0eWxlIjp7ImJvZHkiOnsibmFtZSI6Im5vbmUifSwiaGVhZCI6eyJuYW1lIjoibm9uZSJ9fX1dLFszLDAsIlxcc3Vic2V0ZXEiLDMseyJzdHlsZSI6eyJib2R5Ijp7Im5hbWUiOiJub25lIn0sImhlYWQiOnsibmFtZSI6Im5vbmUifX19XSxbMywyLCIiLDMseyJzdHlsZSI6eyJoZWFkIjp7Im5hbWUiOiJub25lIn19fV0sWzAsMSwiIiwzLHsic3R5bGUiOnsiaGVhZCI6eyJuYW1lIjoibm9uZSJ9fX1dXQ==&macro_url=https%3A%2F%2Fgist.githubusercontent.com%2FdFoiler%2F1e12fec404cad7e185260f0c9b68977d%2Fraw%2F909cc7837a29133fb63fb0e9300d15bfe7417fc5%2Fnir.sty
\[\begin{tikzcd}
	{\hat B} & {\hat L} \\
	{\hat A} & {\hat K}
	\arrow["\subseteq"{marking}, draw=none, from=1-1, to=1-2]
	\arrow["\subseteq"{marking}, draw=none, from=2-1, to=2-2]
	\arrow[no head, from=2-1, to=1-1]
	\arrow[no head, from=2-2, to=1-2]
\end{tikzcd}\]
For fun, we let $\hat{\mf q}$ to be the valuation ideal of $\hat B,$ and we get statements like $j^{-1}(\hat B)=B_\mf q$ from the preceding discussion.

We would like to make the above local diagram an actual local $AKLB$ set-up. As some book-keeping to keep track of the diagram, we have the following lemma.
\begin{lemma}
	We have that $\hat L=L\hat K,$ where the composition is taking place in the canonical embeddings in $L.$
\end{lemma}
\begin{proof}
	We can set
	\[L=K(\alpha_1,\ldots,\alpha_n).\]
	Then $L\hat K=\hat K(\alpha_1,\ldots,\alpha_n),$ and this is finite over $\hat K$ because the generators $\alpha_\bullet$ were finite over $K,$ so $L\hat K$ is complete. Additinoally, $L\hat K$ of course contains $L,$ so the universal property of $\hat L$ gives us a unique map
	\[\hat L\into L\hat K\]
	commuting with everything. This turns into an equality for reasons which are not clear to me, finishing. \todo{}
\end{proof}
The point of the above statement is to say that $[\hat L:\hat K]$ is a finite extension. We also remark that $\hat B$ is indeed the integral closure of $\hat A$ in $\hat L$ because $\alpha\in\hat L$ is integral over $\hat A$ is equivalent to $|\alpha|_\mf q\le1$ from a resul we showed a while ago. We will omit the check that $\hat B$ is finite over $\hat A.$

So indeed, our local diagram is an actual local $AKLB$ set-up. Now let's start moving towards some number theory. We see that
\[f_{\hat {\mf q}/\hat {\mf p}}=f_{\mf q/\mf p}\]
because these residue fields come from $\hat B/\hat{\mf q}\cong B/\mf q$ and $\hat A/\hat{\mf p}\cong A/\mf p.$ Additionally, $e_{\hat{\mf q}/\hat{\mf p}}=e_{\mf q/\mf p}$ because the factorization of $\mf p$ in $B$ will coincide with the factorization of $\hat{\mf p}$ in $\hat B$ after pushing everything through $L\into\hat L.$ More rigorously, we can see that $\nu_{\hat{\mf q}}\circ j=\nu_{\mf q}$ and $\nu_{\hat{\mf p}}\circ i=\nu_{\mf p},$ and the equality of the valuations gives equality of the ramifiactions.

The point of this is that we get the following version of the fundamental identity by simply pushing our ramification and inertial information through.
\begin{corollary}
	Fix everything as above. We have that
	\[[L:K]=\sum_{k=1}^r[\hat L_k:\hat K],\]
	where $\hat L_k$ is the completion of $L$ with respect to $|\cdot|_{\mf q_k}.$
\end{corollary}
Hooray.

\subsection{A Little Global Function Fields}
Let's work more closely with global fields.
\begin{definition}[Global function field]
	A \textit{global function field} is a finite extension of $\FF_p(t).$
\end{definition}
\begin{proposition}
	Fix $K$ a global function field with $\FF_q$ its largest finite subfield, which exists because I said so. Then, for any $t\in K\setminus\FF_q,$ we have that $K$ is a finite extension of $\FF_q(t).$ Further, all places of $K$ are trivial on $\FF_q$ and come from one of the following $AKLB$ set-ups.
	% https://q.uiver.app/?q=WzAsOCxbMCwxLCJcXEZGX3FbdF0iXSxbMCwwLCJBIl0sWzEsMCwiSyJdLFsxLDEsIlxcRkZfcSh0KSJdLFszLDAsIkEiXSxbNCwwLCJLIl0sWzMsMSwiXFxGRl9xWzEvdF0iXSxbNCwxLCJcXEZGX3EodCkiXSxbMSwyLCJcXHN1YnNldGVxIiwxLHsic3R5bGUiOnsiYm9keSI6eyJuYW1lIjoibm9uZSJ9LCJoZWFkIjp7Im5hbWUiOiJub25lIn19fV0sWzAsMywiXFxzdWJzZXRlcSIsMSx7InN0eWxlIjp7ImJvZHkiOnsibmFtZSI6Im5vbmUifSwiaGVhZCI6eyJuYW1lIjoibm9uZSJ9fX1dLFsxLDAsIiIsMSx7InN0eWxlIjp7ImhlYWQiOnsibmFtZSI6Im5vbmUifX19XSxbMiwzLCIiLDEseyJzdHlsZSI6eyJoZWFkIjp7Im5hbWUiOiJub25lIn19fV0sWzQsNiwiIiwxLHsic3R5bGUiOnsiaGVhZCI6eyJuYW1lIjoibm9uZSJ9fX1dLFs1LDcsIiIsMSx7InN0eWxlIjp7ImhlYWQiOnsibmFtZSI6Im5vbmUifX19XSxbNCw1LCJcXHN1YnNldGVxIiwxLHsic3R5bGUiOnsiYm9keSI6eyJuYW1lIjoibm9uZSJ9LCJoZWFkIjp7Im5hbWUiOiJub25lIn19fV0sWzYsNywiXFxzdWJzZXRlcSIsMSx7InN0eWxlIjp7ImJvZHkiOnsibmFtZSI6Im5vbmUifSwiaGVhZCI6eyJuYW1lIjoibm9uZSJ9fX1dXQ==&macro_url=https%3A%2F%2Fgist.githubusercontent.com%2FdFoiler%2F1e12fec404cad7e185260f0c9b68977d%2Fraw%2F909cc7837a29133fb63fb0e9300d15bfe7417fc5%2Fnir.sty
	\[\begin{tikzcd}
		A & K && A & K \\
		{\FF_q[t]} & {\FF_q(t)} && {\FF_q[1/t]} & {\FF_q(t)}
		\arrow["\subseteq"{description}, draw=none, from=1-1, to=1-2]
		\arrow["\subseteq"{description}, draw=none, from=2-1, to=2-2]
		\arrow[no head, from=1-1, to=2-1]
		\arrow[no head, from=1-2, to=2-2]
		\arrow[no head, from=1-4, to=2-4]
		\arrow[no head, from=1-5, to=2-5]
		\arrow["\subseteq"{description}, draw=none, from=1-4, to=1-5]
		\arrow["\subseteq"{description}, draw=none, from=2-4, to=2-5]
	\end{tikzcd}\]
\end{proposition}
\begin{proof}
	To see that the valuation is trivial on $\FF_q,$ we note that all elements of $\FF_q$ are torsion. The rest comes from algebraic geometry.
\end{proof}