% !TEX root = ../notes.tex
















It is finally November.

\subsection{Quick Remark}
We quickly remark from last time that, given $A$ a Dedekind ring with $\mf p$ a nonzero prime, then the canonical map $A\into\hat A$ into the metric completion, this will factor uniquely through the map to the local ring, as per the following diagram.
% https://q.uiver.app/?q=WzAsNCxbMCwwLCJBIl0sWzEsMCwiQV9cXG1mIHAiXSxbMCwxLCJcXGhhdCBBIl0sWzEsMSwiXFxoYXQgQV9cXG1mIHAiXSxbMCwyXSxbMCwxXSxbMSwyLCIhIiwxLHsic3R5bGUiOnsiYm9keSI6eyJuYW1lIjoiZGFzaGVkIn19fV0sWzMsMiwiISIsMSx7InN0eWxlIjp7ImJvZHkiOnsibmFtZSI6ImRhc2hlZCJ9fX1dXQ==&macro_url=https%3A%2F%2Fgist.githubusercontent.com%2FdFoiler%2F1e12fec404cad7e185260f0c9b68977d%2Fraw%2F909cc7837a29133fb63fb0e9300d15bfe7417fc5%2Fnir.sty
\[\begin{tikzcd}
	A & {A_\mf p} \\
	{\hat A} & {\hat A_\mf p}
	\arrow[from=1-1, to=2-1]
	\arrow[from=1-1, to=1-2]
	\arrow["{!}"{description}, dashed, from=1-2, to=2-1]
	\arrow["{!}"{description}, dashed, from=2-2, to=2-1]
\end{tikzcd}\]
Namely, we have a canonical morphism $\hat A_\mf p\to\hat A$ by the universal property.

\subsection{Equivalent Absolute Values}
We have the following definition. Recall that absolute values induce a distance metric and hence a topology. So we have the following notion.
\begin{definition}[Equivalent]
	Fix $K$ a field. Two absolute values $|\cdot|$ and $|\cdot|'$ on $K$ are \textit{equivalent} if they induce the same topology on $K.$
\end{definition}
We have the following lemma.
\begin{lemma} \label{lem:eqabsvals}
	Fix $|\cdot|_1$ and $|\cdot|_2$ nontrivial absolute values on $K.$ Then if
	\[\{x\in K:|x|_1<1\}\subseteq\{x\in K:|x|_2<1\},\]
	then there is $s\in\RR^+$ such that $|x|_1=|x|_2^s.$
\end{lemma}
\begin{proof}
	Certainly we get this for $x=0,$ where any $s$ will suffice. Otherwise, find some $x\in K$ with $|x|_1<1,$ which exists because $|\cdot|_1$ is nontrivial, so there is a nonzero element with absolute value not $1,$ and we can take reciprocals to get our element of absolute value less than $1.$ We note that $|x|_2<1$ as well by hypothesis.

	The point of this $x$ is to induce our $s$ by writing
	\[s:=\frac{-\log|x|_1}{-\log|x|_2}.\]
	In particular, by construction $s>0$ and $|x|_1=|x|_2^s.$ Now, for any other $y\in K^\times,$ we note that all $n,m\in\ZZ$ with $m>0$ have
	\[\left|\frac{x^n}{y^m}\right|_\bullet<1\]
	is equivalent to
	\[n\log|x|_\bullet-m\log|y|_\bullet<0,\]
	is equivalent to
	\[\frac nm>\frac{-\log|y|_\bullet}{-\log|x|_\bullet},\]
	for either absolute value. Thus, it follows that
	\[\frac{-\log|y|_1}{-\log|x|<1}<\frac nm\implies\frac{-\log|y|_2}{-\log|x|_2}<\frac nm\]
	by hypothesis on our valuations, so we see that
	\[\frac{-\log|y|_1}{-\log|x|<1}\le\frac{-\log|y|_2}{-\log|x|_2},\]
	and taking ratios tells us that $-\log|y|_2\le\frac1s(-\log|y|_1),$ so we get $|y|_2\le|y|_1^s.$ Running the same argument through with $1/y$ tells us that $|y|_1\le|y|_2^s,$ so we get the equality here.
\end{proof}
\begin{remark}
	Note that the conclusion is symmetric even though the hypothesis is asymmetric.
\end{remark}
The main result is as follows.
\begin{proposition}
	Two absolute values $|\cdot|_1$ and $|\cdot|_2$ are equivalent if and only if there is $s\in\RR^+$ such that $|\cdot|_1=|\cdot|_2^s.$
\end{proposition}
\begin{proof}
	To start, the absolute value is trivial if and only if it induces the discrete topology on $K,$ which we won't show in detail here. The forward direction is essentially because our distance are always $1,$ and the backwards direction is because nontrivial valuations will have a nontrivial ball somewhere.

	So now take $|\cdot|_1$ and $|\cdot|_2$ nontrivial. Now, if $|\cdot|_1=|\cdot|_2^s$ for $s\in\RR^+,$ then of course our balls are the same by some topological open ball argument. Conversely, if $|\cdot|_1$ and $|\cdot|_2$ are equivalent, then we note that $x^n\to0$ as $n\to\infty$ is equivalent to $|x|<1,$ so
	\[\{x\in K:|x|_1<1\}=\{x\in K:x^n\to0\}=\{x:|x|_2<1\}\]
	because approaching zero is a purely topological concept (!). Now we are able to finish by \autoref{lem:eqabsvals}.
\end{proof}
We also have the following corollary.
\begin{corollary} \label{lem:eqivcor}
	Fix $|\cdot|_1$ and $|\cdot|_2$ are absolute values. Then if $|\cdot|_1$ is nontrivial and
	\[\{x\in K:|x|_1<1\}\subseteq\{x\in K:|x|_2<1\},\]
	then our absolute values are equivalent.
\end{corollary}
\begin{proof}
	This follows from the proof of the previous proposition.
\end{proof}

\subsection{Theorems on Valuations}
Our notion of equivalence is good enough to deserve a name.
\begin{definition}[Place]
	Fix $K$ a field. Then a \textit{place} of a field $K$ is a nontrivial equivalence class of absolute values on $K.$
\end{definition}
We have the following theorem, which essentially says that these absolute values measure numbers about as orthogonally as one could ask for.
\begin{theorem}[Strong approximation]
	Suppose that $|\cdot|_1,\ldots,|\cdot|_n$ are pairwise nonequivalent, nontrivial absolute values on a field $K.$ Further, fix $a_1,\ldots,a_n\in K$ and some $\varepsilon>0.$ Then there exists $x\in K$ such that
	\[|x-a_\bullet|_\bullet<\varepsilon\]
	for each $a_\bullet.$
\end{theorem}
\begin{proof}
	We have the following claim.
	\begin{lemma}
		There is some $z\in K$ such that $|z|_1>1$ and $|z|_\bullet<1$ for all other absolute values $|\cdot|_\bullet.$
	\end{lemma}
	\begin{proof}
		We induct. For $n=1,$ anything with absolute value bigger than $1$ will do, which exists because these absolute values are nontrivial. Then for $n=2,$ we note that reading \autoref{lem:eqivcor} backwards requires there to be $\alpha$ with $|\alpha|_1<1$ and $|\alpha|_2\ge1$ and $\beta$ with $|\alpha|_1\ge1$ and $|\beta|_2<1.$ Then we can check that $\beta/\alpha$ will induce the correct inequalities.

		Otherwise take $n>2,$ and we are promised $z$ with $|z|_1>1$ and $|z|_k<1$ for each $1<k<n.$ We have the following cases. Again using \autoref{lem:eqivcor} we can find $|y|_1>1$ and $|y|_n<1,$ which is from the $n=2$ case.
		\begin{itemize}
			\item If $|z|_n<1$ already, we are finished.
			\item If $|z|_n=1,$ then we can take $z^my$ for some sufficiently large $m.$ Namely, the power is intended to kill the $|y|_\bullet$ between $1$ and $n.$
			\item Otherwise $|z|_n>1.$ Now we set
			\[t_m:=\frac{z^m}{1+z^m}.\]
			Te point is that $z_m\to1$ in $|\cdot|_1$ and $|\cdot|_n$ by running the manipulation through, and it goes to $0$ elsewhere. So $t_my$ will work for some sufficiently large $m.$
			\qedhere
		\end{itemize}
	\end{proof}
	Now from the lemma we can find $z_k\in K$ such that $|z_k|_k>1$ and $|z_k|_\ell<1$ for each $k,\ell.$ In particular, the sequence
	\[\frac{z_k^m}{1+z_k^m}\]
	goes to $1$ with respect to $|\cdot|_k$ and goes to $0$ with respect to the other $|\cdot|_\bullet.$ So we can set
	\[z:=\sum_{k=1}^na_k\cdot\frac{z_k^m}{1+z_k^m}\]
	as long as $m$ is large enough to overcome $\varepsilon.$
\end{proof}

We close this subsection with the following statement.
\begin{theorem}
	All nontrivial absolute values on $\QQ$ are equivalent to either $|\cdot|_\infty$ or $|\cdot|_p$ for some prime $p.$
\end{theorem}
\begin{proof}
	This is in the book. Professor Vojta doesn't look this proof, but I do. It's a fun proof.
\end{proof}
\begin{remark}[Nir]
	Of course the above holds in more general global fields.
\end{remark}

\subsection{Units in Discrete Valuation Rings}
In this subsection, fix $A$ a discrete valuation ring with $\mf p$ its unique maximal ideal. We let $U:=A^\times=A\setminus\mf p$ be the unit group, and we let
\[U^{(n)}:=1+\mf p^n\]
be the a subgroup of $U.$
\begin{remark}
	Our definition gives the descending chains
	\[U^{(0)}\supseteq U^{(1)}\supseteq\cdots\]
\end{remark}

We have the following statement.
\begin{proposition}
	We have the following.
	\begin{enumerate}[label=(\alph*)]
		\item $U/U^{(n)}\cong(A/\mf p^n)^\times$ for each $n\in\NN.$
		\item $U^{(n)}/U^{(n+1)}\cong A/\mf p$ for each positive integer $n.$
	\end{enumerate}
\end{proposition}
\begin{proof}
	We take these one at a time.
	\begin{enumerate}
		\item We note the induced map
		\[U\to \left(A/\mf p^n\right)^\times\]
		is onto because $U=A\setminus\mf p.$ (To be explicit, this map is induced as $u\mapsto u+\mf p^n.$) And we also note that we have kernel $U^{(n)}$ essentially by construction of $U^{(n)}.$
		\item Fix $\pi\in\mf p\setminus\mf p^2.$ Then we note that we have a map $\varphi:U^{(n)}\mapsto A/\mf p$ by taking
		\[\varphi:1+\pi^na\mapsto a+\mf p.\]
		We see $\varphi$ is well-defined because $\mf p^n=(\pi^n).$ Then $\varphi$ is onto by just taking any coset representative backwards. And the kernel happens when $a\in\mf p=(\pi),$ which pulls back to the original element living in $1+\mf p^{n+1}=U^{(n+1)}.$
		
		Lastly, we can check by hand that this is homomorphic:
		\[\varphi\big((1+\pi^na)(1+\pi^nb)\big)=\varphi\left(1+\pi^n\left(a+b+\pi^nab\right)\right)=(a+b)+\mf p=(a+\mf p)+(b+\mf p)=\varphi(1+\pi^na)+\varphi(1+\pi^nb).\]
		This finishes.
		\qedhere
	\end{enumerate}
\end{proof}