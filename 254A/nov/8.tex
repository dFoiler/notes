% !TEX root = ../notes.tex















Here we go.

\subsection{Extending Absolute Values}
We are extending absolute values today.
\begin{theorem}
	Fix $(K,|\cdot|)$ a complete valued field. Fix $L/K$ an algebraic extension. Then there exists a unique absolute value $|\cdot|_L:L\to\RR_{\ge0}$ extending the absolute value on $K,$ given by
	\[|\alpha|_L=\left|\op N_K^{K(\alpha)}\alpha\right|^{1/[K(\alpha):K]}\]
	for all $\alpha\in L^\times.$
\end{theorem}
\begin{proof}
	We proceed in cases.
	\begin{itemize}
		\item Take $K$ archimedean. If $L=K,$ there is nothing to say here. Otherwise, by our classification of complete archimedean fields, $K=\RR$ and $L=\CC,$ so we get the statement by Ostrowski's theorem classifying the places of $\CC.$ There is some care required to check that equivalence of absolute values becomes full equality, but it can be done.
		\item Take $K$ nonarchimedean and nontrivial with valuation ring $A.$ We split the proof in half.
		\begin{enumerate}[label=(\alph*)]
			\item We show that the $|\cdot|_L$ suggested does indeed satisfy the constraints.\footnote{This even works if $|\cdot|$ on $K$ is trivial, which would imply that $|\cdot|_L$ is also trivial by construction.} Namely, $|\alpha|_L=|\alpha|$ for each $\alpha\in K$ by pushing everything through. And otherwise we note that, for any $L'$ containing $\alpha,$ we have
			\[|\alpha|_L=\left|\op N^{L'}_K\alpha\right|^{1/[L':K]}\tag{$*$}\]
			by checking things with norms, so we get some well-defined-ness.

			We now need to check that $|\alpha|_L$ is in fact an absolute value. Well, $|\alpha|_L\ge0$ with equality if and only if $\alpha=0$ comes from the corresponding statement about norms. Then
			\[|\alpha\beta|_L=|\alpha|_L\cdot|\beta|_L\]
			holds by using $(*)$ on $K(\alpha,\beta)$ and the multiplicativity of the norm.

			Lastly, we need to check the strong triangle inequality, which requires some trickery. Without loss of generality, take $|\alpha|_L\le|\beta|_L,$ and we want to show that
			\[|\alpha+\beta|_L\stackrel?\le|\beta|_L.\]
			By applying division, it suffices to show that $|\alpha/\beta+1|\le1$ given that $|\alpha/\beta|\le1.$ But $|\alpha/\beta|\le1$ implies that $\alpha/\beta$ is integral over $A,$ so $\alpha/\beta+1$ is integral over $A,$ so $|\alpha/\beta+1|\le1.$ This finishes the proof of existence.

			\item Take $|\cdot|$ a nontrivial absolute value, and we show uniqueness of the extension. Suppose that $|\cdot|'_L$ be another absolute value on $L$ extending $|\cdot|.$ We set
			\[B:=\{\alpha\in L:|\alpha|_L\le1\}\qquad\text{and}\qquad B':=\{\alpha\in L:|\alpha|'_L\le1\}.\]
			We claim that $B\subseteq B'.$ Indeed, otherwise there is $\alpha\in B\setminus B',$ which is an $\alpha\in L$ with $|\alpha|_L\le1$ and $|\alpha|'_L\ge1.$ But now look at the irreducible polynomial for $\alpha$ as
			\[f(x)=\sum_{k=0}^na_nx^n\in K[x].\]
			We note that we get $A[x]$ because $|\op N_K^{K(\alpha)}\alpha|\le1,$ which is the last thing we showed last class. But now $|\alpha^{-1}|'_L<1$ even though
			\[1=-a_{n-1}\alpha^{-1}-a_{n-2}\alpha^{-2}-\cdots-a_0\alpha^{-n}\]
			after doing some rearranging with $f,$ which is monic. It follows that the size of the left-hand side is $1$ while the size of the right-hand side is at least $1$ by the strong triangle inequality, which is a contradiction.

			Now, recall from our discussion about places that $B\subseteq B'$ will imply that $|\cdot|'_L=|\cdot|^s_L$ for some $s>0.$ But now, because $|\cdot|$ is nontrivial on $K,$ we can plug in some value of $\alpha\in K$ giving $|\alpha|_L\ne1,$ which forces $s=1.$ This finishes.
		\end{enumerate}
		\item If $|\cdot|$ is trivial on $K,$ then we leave this as an exercise. The extension should be the trivial absolute value.
		\qedhere
	\end{itemize}
\end{proof}
The above gets us that algebraic extensions of complete valued fields are at least valued. However, they need not be complete.

\subsection{Normed Vector Spaces}
We have the following definition.
\begin{definition}[Normed vector spaces]
	Fix $V$ a $K$-vector space, where $(K,|\cdot|)$ is a valued field. Then we have the following definitions. A \textit{norm} on $V$ is a function $\lVert\cdot\rVert:V\to\RR$ such that the following hold.
	\begin{enumerate}[label=(\roman*)]
		\item We have $\lVert v\rVert\ge0$ for any $v\in V$ with equality if and only if $v=0.$
		\item We have $\lVert cv\rVert=|c|\cdot\lVert v\rVert,$ for any $c\in K$ and $v\in V.$
		\item We have $\lVert v+w\rVert\le\lVert v\rVert+\lVert w+\rVert$ for each $v,w\in V.$
	\end{enumerate}
\end{definition}
\begin{example}
	Any vector space will have a norm by taking the supremum of the coefficients under the coordinates of a particular basis.
\end{example}
We note that a norm will induce a metric on $V$ in the usual way.
\begin{definition}[Equivalence of norms]
	Fix $\lVert\cdot\rVert_1$ and $\lVert\cdot\rVert_2$ to norms on $V.$ They are \textit{equivalent} if and only if there are constants $\rho_1,\rho_2>0$ such that
	\[\rho_1\lVert v\rVert_1\le\lVert v\rVert_2\le\rho_2\lVert v\rVert_2\]
	for each $v\in V.$
\end{definition}
It is not hard to see that equivalence is in fact an equivalence relation.

We have the following lemma.
\begin{lemma}
	Fix $V$ a finite-dimensional $K$-vector space, where $(K,|\cdot|)$ is a complete valued field. Then we have the following.
	\begin{enumerate}[label=(\alph*)]
		\item Any two norms on $V$ are equivalent.
		\item $V$ is a complete metric space under the induced metric.
	\end{enumerate}
\end{lemma}
\begin{remark}
	Even if $K$ is not complete, neither (a) nor (b) need be true. For example, extensions of number fields do not have unique extensions.
\end{remark}
\begin{proof}
	We sketch. Fix $\{v_k\}_{k=1}^n$ a basis of $V$ over $K.$ Then we define our initial norm as
	\[\left\lVert\sum_{k=1}^na_kv_k\right\rVert:=\max_{1\le k\le n}\{a_k\},\]
	which we won't check to be a norm. We can check that $V$ is complete with respect to this norm by looking at coordinate-wise convergence in a Cauchy sequence, using the fact that $K$ is complete.

	It remains to show that $\lVert\cdot\rVert$ is the only equivalence class. For the archimedean case, there are compactness arguments. For the nonarchimedean case, see the book.
\end{proof}
Anyways, we bring up this machinery because of the following corollary.
\begin{corollary}
	Fix $(K,|\cdot|)$ a complete valued field and $L/K$ a finite extension. Then $L$ is complete with respect to the (unique) extension $|\cdot|_L$ described previously.
\end{corollary}
\begin{proof}
	This comes from the above, viewing the absolute value on $L$ as a norm on $L$ as a $K$-vector space.
\end{proof}
The finite extension hypothesis is necessary.
\begin{example}
	The algebraic closure $\overline{\QQ_p}$ of $\QQ_p$ is not complete, but the metric completion $\CC_p$ of $\overline{\QQ_p}$ is algebraically closed, so there is a field with all the nice properties we want.
\end{example}

\subsection{Local \texorpdfstring{$AKLB$}{} Set-Up}
So now we can build a local $AKLB$ set-up: fix $(K,|\cdot|)$ a nontrivial complete nonarchimedean valued field with valuation ring $A.$ Then any finite extension $L$ will have a unique extension $|\cdot|_L$ of $|\cdot|,$ and we can let $B$ be the corresponding valuation ring.

Now, $A$ and $B$ are local rings, so (say) $A$ is Dedekind if and only if $A$ is a discrete valuation ring if and only if there is a discrete valuation on $K$ (by extension) with valuation ring $A.$ But then this new valuation induces some absolute value $|\cdot|'$ with the same $A=\{\alpha\in K:|\alpha|'\le1\}$ set, so it is equivalent to the absolute value on $K$ we started with.

The point of this computation is that we need to check if the valuation induced by $|\cdot|$ on $K$ is a discrete valuation, which will make the valuation on $L$ discrete by its construction (it will output into $\frac1{[L:K]}\im(|\cdot|),$ which maintains being discrete), so in this case $B$ will also be Dedekind in this case.

So we get the following picture.
% https://q.uiver.app/?q=WzAsNCxbMSwxLCJLIl0sWzAsMSwiQSJdLFswLDAsIkIiXSxbMSwwLCJMIl0sWzMsMCwiIiwwLHsic3R5bGUiOnsiaGVhZCI6eyJuYW1lIjoibm9uZSJ9fX1dLFsyLDEsIiIsMCx7InN0eWxlIjp7ImhlYWQiOnsibmFtZSI6Im5vbmUifX19XSxbMiwzLCJcXHN1YnNldGVxIiwxLHsic3R5bGUiOnsiYm9keSI6eyJuYW1lIjoibm9uZSJ9LCJoZWFkIjp7Im5hbWUiOiJub25lIn19fV0sWzEsMCwiXFxzdWJzZXRlcSIsMSx7InN0eWxlIjp7ImJvZHkiOnsibmFtZSI6Im5vbmUifSwiaGVhZCI6eyJuYW1lIjoibm9uZSJ9fX1dXQ==
\[\begin{tikzcd}
	B & L \\
	A & K
	\arrow[no head, from=1-2, to=2-2]
	\arrow[no head, from=1-1, to=2-1]
	\arrow["\subseteq"{description}, draw=none, from=1-1, to=1-2]
	\arrow["\subseteq"{description}, draw=none, from=2-1, to=2-2]
\end{tikzcd}\]
Technically we should check that $B$ is finite over $A,$ but we will not do this here.

Further, we let $\mf p\subseteq A$ and $\mf q\subseteq B$ be their maximal ideals. Because $|\cdot|_L$ extends $|\cdot|,$ we see that checking definitions gives $\mf q$ over $\mf p.$ And in fact $\mf q$ is the unique prime ideal of $B$ lying over $\mf p$ because there is only one prime in $B.$ So the fundamental identity gives
\[[L:K]=e(\mf q/\mf p)f(\mf q/\mf p).\]
So our local theory is nice.

\subsection{Global Fields}
Let's start being a little less local. We have the following definition.
\begin{definition}[Global field]
	A \textit{global field} is a finite extension of $\QQ$ or $\FF_p(t).$ In other words, a global field is a number field or a ``global function field''---function field over a curve over a finite field.
\end{definition}
We will not care so much about the latter case because algebraic geometry is not a prerequisite for this course.