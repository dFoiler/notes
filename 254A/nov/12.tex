\documentclass[../notes.tex]{subfiles}

\begin{document}

% !TEX root = ../notes.tex

















Ok.

\subsection{Some Loose Ends}
Last time we claimed the following loose ends; let's talk about them.
\begin{lemma}
	A global function field $K$ has a largest finite subfield.
\end{lemma}
\begin{proof}
	We note that a sub-extension of a finitely generated field extension is finitely generated, so we get the result by noting that $\overline{\FF_p}\cap K$ must be finitely generated over $\FF_p$ and hence be our largest finite subfield.
\end{proof}
\begin{lemma}
	Fix $K$ a global function field and $\mf p$ a nonzero prime of the valuation ring $A.$ Then we can show that all extensions of $|\cdot|_\mf p:K\to\RR$ to some field extension $L$ are induced as $|\cdot|_\mf q$ from primes $\mf q$ in $L$ over $\mf p.$
\end{lemma}
\begin{proof}
	This will go on the homework.
\end{proof}
This goes into the $AKLB$ set-up from last time.
\begin{lemma}
	Fix $(K,|\cdot|)$ a complete nonarchimedean valued field with valuation ring $A.$ If $L/K$ is a finite field extension with $|\cdot|:L\to\RR$ the unique extension of $|\cdot|:K\to\RR,$ and we set $B$ to be the integral closure of $A$ in $L.$ Then $B$ is finitely generated over $A.$
\end{lemma}
\begin{proof}
	The main point is to forcefully localize. We know that $L$ is complete with respect to $|\cdot|$ already, and we showed some time ago in \autoref{cor:corsquared} that $B$ is in fact the valuation ring of $L.$

	Now, fix $\mf p$ the valuation ideal of $A$ and $\kappa:=A/\mf p$ the residue field, and we fix some uniformizer $\pi\in\mf p\setminus\mf p^2.$ Now, $\lambda=B/\mf pB$ is not necessarily a field, but it is a $\kappa$-vector space (it is an $A$-module automatically, but the $\mf p$-action has been nullified), so find some
	\[\{b_i\}_{i\in I}\subseteq B\]
	which give a $\kappa$-basis for $\lambda$ upon reduction. We have the following checks.
	\begin{itemize}
		\item We show that $I$ is finite. The main point is to show that $\{b_i\}_{i\in I}$ is linearly independent over $K$: indeed, any nontrivial linear relation
		\[\sum_{i\in I}a_ib_i=0\]
		with $a_\bullet\in K$ can be turned into one where not all of the $a_\bullet$ are in $\mf p$ (by dividing out by $\pi$ enough times), which turns into a nontrivial relation in $B/\mf pB$ upon reduction. But this cannot be because the $b_\bullet$ were a basis of $B/\mf pB$ and hence linearly independent.
		
		The above linear independence in $L/K$ shows that $\#I\le[L:K]$ and in particular is finite.

		\item We now show that the $b_\bullet$ span. Appropriately, we may take $E$ to be
		\[E:=\bigoplus_{i\in I}Ab_i\]
		to be generated by the $b_i.$ We note that each $x\in B$ can be reduced$\pmod{\mf p}$ to some element in $B/\mf pB,$ which will have a representative in $E$ by the spanning. Namely, we can write
		\[x=c_0+x_1\pi,\]
		for some $c_0\in E.$ Iterating this process for $x_1$ and so on, we get a series
		\[x=c_0+c_1\pi+c_2\pi^2+\cdots,\]
		where $c_\bullet\in E.$ Now, because $A$ is complete, when we expand out the $c_\bullet$ in terms of their $b_\bullet$ components, we will get a power series in $\pi$ in $A,$ which will have to converge in $A$ by that completeness. In particular, $x\in E.$

		So indeed, $B\subseteq E,$ so $B=E.$
	\end{itemize}
	Thus, $B$ finitely generated over $A,$ which finishes.
\end{proof}

\subsection{Local Fields}
Let's now return to our story. We have the following claim.
\begin{proposition}
	Fix $K$ a global function field, for concreteness viewed as a finite extension of $\FF_p(t).$ Then all places of $K$ come from extensions of places of $\FF_p(t).$ All places of $\FF_p(t)$ come from prime ideals of $\FF_p[t]$ or $\FF_p[1/t].$
\end{proposition}
\begin{proof}
	Omitted; is is essentially death by algebraic geometry.
\end{proof}
\begin{remark}[Nir]
	I think the place of $\FF_p[1/t]$ we need to worry about is the one which comes from the prime ideal $(1/t).$
\end{remark}
\begin{remark}
	The above does induce the set of all places of $K$ and hence is independent of our choice of $t$ in ``for concreteness'' phrase. So it goes.
\end{remark}
\begin{remark}
	This statement is not true for more general function fields; e.g., we can take $K=\CC(t,u)$ as a one-variable function field over $\CC(t)$ or $\CC(u),$ which induce disjoint sets of places.
\end{remark}
We now have the following definition.
\begin{definition}[Local fields]
	A \textit{local field} is the metric completion of a global field at some place. Note that local fields are complete automatically.
\end{definition}
\begin{example}
	By Ostrowski's theorem, we can enumerate our local fields as one of the following.
	\begin{itemize}
		\item Archimedean localizations of number fields: $\RR$ or $\CC.$
		\item Nonarchimedean localizations of number fields: finite extensions of $\QQ_p.$
		\item Nonarchimedean localizations of global function fields: finite extensions of $\FF_p(t).$
	\end{itemize}
\end{example}
\begin{warn}
	Some authors actively exclude $\RR$ and $\CC$ from the definition because everyone else in the club is nonarchimedean.
\end{warn}
For example, {Serre}'s \textit{Local Fields} often does not say the term ``local field'' at all.

To extend our definition, we have the following definition.
\begin{definition}
	A \textit{complete discretely valued field} is a complete valued field $(K,|\cdot|)$ such that $x\mapsto-\log|x|$ has discrete image. In particular, such a discretely valued field is automatically nonarchimedean because complete archimedean fields are either $\RR$ or $\CC.$
\end{definition}
\begin{example}
	All nonarchimedean local fields are complete discretely valued fields.
\end{example}
\begin{example}
	Any finite extension of $F((t))$ for any field $F$ is a complete discretely valued field.
\end{example}
\begin{example}
	There are some infinite extensions of $\QQ_p$ which are complete discretely valued fields.
\end{example}

\subsection{Topological Field Theory}
We have the following statement.
\begin{proposition}
	Fix $K$ a finite extension of $\QQ_p$ or $F((t))$ for a ground field $F.$ Then $K^\times$ is homeomorphic to $\ZZ\times\kappa\times U^{(1)},$ where $\kappa$ is the residue field of $K,$ $A$ is the valuation ring, and $U^{(1)}$ is the kernel of $A^\times\to\kappa^\times.$
\end{proposition}
\begin{proof}
	Pick a uniformizer $\pi$ for the discrete valuation ring $A.$ Then we see that $K^\times\cong A^\times\times\ZZ$ by the mapping $u\pi^n\mapsto(u,n).$ Indeed, we have exhibited is a isomorphism of the underlying topological groups, and then $A^\times$ is both open and closed in $K^\times$\todo{}, which induces the isomorphism after slicing $K^\times$ by norm.

	Continuing, we have a short exact sequence
	\[1\to U^{(1)}\to A^\times\to\kappa^\times\to1.\]
	We have two cases.
	\begin{itemize}
		\item If $\kappa$ is finite with $q$ elements, then $x^{q-1}-1$ will fully split in $\kappa[x],$ so Hensel's lemma promises that $x^{q-1}-1$ will fully split in $A.$ This induces $q-1$st roots in $K,$ which will successfully split the short exact sequence.
		\item Otherwise $\kappa$ is infinite. In this case we must have $K$ be a finite extension of $F((t))$ for some ground field $F.$ Now, we see $F^\times\subseteq A^\times.$ Additionally, $F\into K,$ so the Cohen structure theorem\footnote{Professor Vojta does not expect us to know what this is.} implies that $A\cong\kappa[[t']],$ so $\kappa\into K$ again has a lift backwards. This splits the short exact sequence.
	\end{itemize}
	Lastly, we note that $U^{(1)}=1+\mf p,$ where $\mf p$ is the valuation ideal of $A,$ so $U^{(1)}$ is open in $A^\times$ and hence also closed. So we are partitioning $A^\times\cong\kappa^\times\times U^{(1)},$ finishing.
\end{proof}
We remark that, in the case that $K$ is a finite extension of $\QQ_p,$ we have group homomorphisms $\mf p^n\to U^{(n)}$ and backwards: forwards is by taking an exponential, and backwards is by taking a log. This is by doing some power series; we won't be more explicit than this.

\end{document}