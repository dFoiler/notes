% !TEX root = ../notes.tex















\subsection{Advertisement: Riemann Surface}
We continue to take $X$ to be a connected, compact Riemann surface. We also define
\[L:=\mathcal K(X):=\op{Hom}_{\text{meromorphic}}(X,\CC),\]
which is in fact a field. We also recall that $K:=\mathcal K(\CP^1)=\CC(t),$ where $t$ corresponds to $\id:\CC\to\CC,$ induced by the canonical map $\CC\into\CP^1.$

Now, we set
\[A:=\CC[t]:=\{\text{holomorphic maps }\CC\to\CC\text{ extending to meromorphic }\CP^1\to\CC\},\]
and we set $B$ to be the integral closure of $A$ in $L.$ Given the canonical embedding $X\onto\CP^1,$ we get an $AKLB$ set-up as follows.
% https://q.uiver.app/?q=WzAsNixbMCwwLCJCIl0sWzAsMSwiQSJdLFsxLDAsIkwiXSxbMSwxLCJLIl0sWzIsMCwiWCJdLFsyLDEsIlxcQ1BeMSJdLFszLDIsIiIsMCx7InN0eWxlIjp7ImhlYWQiOnsibmFtZSI6Im5vbmUifX19XSxbMSwwLCIiLDAseyJzdHlsZSI6eyJoZWFkIjp7Im5hbWUiOiJub25lIn19fV0sWzAsMiwiXFxzdWJzZXRlcSIsMSx7InN0eWxlIjp7ImJvZHkiOnsibmFtZSI6Im5vbmUifSwiaGVhZCI6eyJuYW1lIjoibm9uZSJ9fX1dLFsxLDMsIlxcc3Vic2V0ZXEiLDEseyJzdHlsZSI6eyJib2R5Ijp7Im5hbWUiOiJub25lIn0sImhlYWQiOnsibmFtZSI6Im5vbmUifX19XSxbNCw1LCJcXHBpIiwwLHsic3R5bGUiOnsiaGVhZCI6eyJuYW1lIjoiZXBpIn19fV1d
\[\begin{tikzcd}
	B & L & X \\
	A & K & {\CP^1}
	\arrow[no head, from=2-2, to=1-2]
	\arrow[no head, from=2-1, to=1-1]
	\arrow["\subseteq"{description}, draw=none, from=1-1, to=1-2]
	\arrow["\subseteq"{description}, draw=none, from=2-1, to=2-2]
	\arrow["\pi", two heads, from=1-3, to=2-3]
\end{tikzcd}\]
Note that there is also a ``dual'' $AKLB$ set-up by setting $A'=\CC[1/t]$ and $B'$ the integral closure of $A'$ in $L.$ It is a theorem that these $AKLB$ set-ups give all nontrivial valuations on $L$ which are trivial on $\CC.$ Roughly speaking, this is because we spent so much time working in general function fields in one variable.

Note that, for each $a\in\CP^1\setminus\{\infty\},$ we can view $a\in\CC$ so that $t-a\in A=\CC[t],$ which will lift to an element of $B$ which is holomorphic on $\pi^{-1}(\CC),$ vanishing at all points of $\pi^{-1}(a).$ For some given $p\in\pi^{-1}(a),$ we let the order of vanishing be $e_p>0,$ and it is true that
\[\sum_{pin\pi^{-1}(a)}e_p=\deg\pi=[L:K]\]
by essentially doing complex analysis. Additionally, the residue field at our $a\in\CP^1$ for given $p\in\pi^{-1}(a)$ is in fact $\CC,$ for reasons I don't understand, so the inertial degree $f_p$is $1.$ So we have verified our fundamental identity by hand.

Additionally, by complex analysis/algebraic geometry, it is true that any $x\in\pi^{-1}(\CC)$ has
\[\mf q_x:=\{f\in B:f(x)=0\}\]
is a prime ideal of $B$ (which is not hard to see by checking primality by hand), which provides a bijection between $\pi^{-1}(\CC)$ and the nonzero primes of $B$ (which is hard to see). Regardless, it is true that
\[(t-a)B=\prod_{p\in\pi^{-1}(a)}\mf q_p^{e_p}.\]
And here, the ramification index $e_p$ is actually ramifiaction in the sense of a Riemann surface. Again using our trivial inertial indices, we see that, again, we have
\[[L:K]=\sum_{p\in\pi^{-1}(a)}e_pf_p,\]
which we computed by hand again.

\subsection{Setting Up Riemann--Roch}
Let's start building the machinery for Riemann--Roch. In short, we will be interested in sets of the form
\[\{f\in\mathcal K(X):f\text{ has poles only on }S\text{ of order }\nu_s\in\ZZ\text{ for each }s\in S\}.\]
The reason we are allowing poles, though with some control, is because holomorphic functions on these sorts of spaces are quite boring. Note that we are allowing the order to be negative, which corresponds to order of vanishing. In other words, given a tuple $\{n_Q\}\in\ZZ^{\oplus X}$ of integers, we are looking at
\[L(\{n_Q\}_{Q\in X})=\{f\in\mathcal K(X):\nu_Q(f)\ge-n_Q\text{ for each }Q\in X\}.\]
This set turns into a $\CC$-vector space (which we can see by hand) and is finite-dimensional (which is harder to see---roughly speaking, poles induce constraints on the functions).
\begin{example}
	Using $X=\CP^1$ with $n_2=1$ and $n_5=3$ and $n_\infty=10$ and $n_8=9$ and $n_p=0$ elsewhere, out set consists of functions which have poles of order less than equal to $1$ at $2,$ less than or equal to $3$ at $5,$ a pole of at most $10$ at $\infty,$ a zero of order at least $9$ at $8,$ and holomorphic everywhere else.

	In particular, any $f$ satisfying these constraints has $f(z-1)(z-5)^3(z-8)^{-9}$ will extend to a holomorphic function (on $\CC$) with a pole of order at most $10+1+3-9=5$ at $\infty.$ So we see we it suffices to describe a fifth-degree polynomial, for which there are six dimensions needed.
\end{example}
The point of Riemann--Roch is to approximate the dimensions of these sets we are looking at.

Here is a small example.
\begin{proposition}
	Given a tuple $\{n_Q\}_{Q\in X}\in\ZZ^{\oplus X}$ has
	\[\sum_{Q\in X}n_Q<0,\]
	then
	\[L(\{n_Q\}_{Q\in X})=\{0\}.\]
\end{proposition}
\begin{proof}
	This is because we have a product formula for $\CC(t)$: given $f\ne0,$ we have
	\[\sum_{Q\in X}\nu_Q(f)=0,\]
	which holds by complex analysis or our product formula for number fields or similar. The point is that it is impossible to satisfy the given constraints when we have
	\[\sum_{Q\in X}\nu_Q(f)\le\sum_{Q\in X}n_Q<0.\]
	This finishes.
\end{proof}

\subsection{Places of Number Fields}
Let's quickly return to number fields.
\begin{definition}[In/finite places]
	A place $\nu$ of a number field is \textit{finite} if and only if nonarchimedean or \textit{infinite} if and only if nonarchimedean. Finite places are denoted by $\nu\nmid\infty,$ and infinite places are denoted by $\nu\mid\infty,$ essentially meaning that $\nu$ ``lies over'' the infinite place $\infty$ of $\QQ.$
\end{definition}
\begin{definition}[Real/complex places]
	An infinite place $\nu$ of a number field $K$ is \textit{real} if and only if $K_\nu=\RR$ and \textit{complex} if and only if $K_\nu=\CC.$
\end{definition}
We quickly remark that finite places are neither real nor complex places. Let's start our discussion of extensions.
\begin{definition}[Infinite inertia/ramification]
	Fix $L/K$ an extension of number fields. Given $v$ an infinite place of $K$ and $w$ an infinite place of $L$ extending $v.$ Now we set $\kappa(v):=K_v\in\{\RR,\CC\}$ and define
	\[f_{w/v}:=[\kappa(w):\kappa(v)]=[L_w:K_v]\]
	and $e_{w/v}:=1$ so that $[L_w:K_v]=e_{w/v}f_{w/v},$ in accordance with the case with finite places.
\end{definition}
\begin{warn}
	Here Neukirch is forcing $\CC/\RR$ to be unramified, but other authors prefer this to be ramified.
\end{warn}
\begin{definition}[Norms of places]
	Fix $v$ a place of a global field $K,$ and we define a \textit{norm} $\lVert\cdot\rVert_v$ as follows.
	\begin{listalph}
		\item If $v$ is finite, then let $\mf p\in\op{Spec}\mathcal O_K$ be the corresponding valuation ideal. We define $\op N(v):=\op N(\mf p)$ to be the absolute norm of $\mf p.$ Then by convention we will take
		\[\lVert x\rVert_v:=\op N(v)^{-\nu_\mf p(x)}\]
		to be an absolute value representing our place $v.$ (Namely, $\op N(v)=p^{f_v}$ with $v(p)=e_v$ so that $\lVert p\rVert_v=p^{-e_vf_v}=|p|_p^{e_vf_v}$ so that $\lVert x\rVert_v=|x|_p^{[K_v:\QQ_p]}.$)
		\item For real place $v$ coming from $\rho:K\into\RR,$ we set $\lVert x\rVert_v=|\rho x|.$
		\item For complex places $v$ coming from $\sigma:K\into\CC,$ we set $\lVert x\rVert_v=|\sigma x|^2=|\overline\sigma x|^2.$
	\end{listalph}
\end{definition}
In the above set-up with $v$ finite, we note that, setting $p:=\op{char}\kappa(v)\in\mf p,$ we have that $K_v$ is a finite extension of $\QQ_p$ of degree $e_vf_v.$ Additinoally, by construction, we see that
\[\lVert x\rVert_\nu=|x|_p^{[K_\nu:\QQ_p]}\]
for any place $\nu,$ by construction. Namely, when $\nu$ is complex, we want to square here, as we did in the above definition.
\begin{warn}
	When $v$ is complex, $\lVert\cdot\rVert_\nu$ violates the triangle inequality, so it is not an absolute value. For example, $\lVert2\rVert>\lVert1\rVert+\lVert1\rVert.$
\end{warn}