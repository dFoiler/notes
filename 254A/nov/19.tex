% !TEX root = ../notes.tex














I was not present for class today because I was stuck in Walgreens getting my flu shot. The following notes were transcribed from Austin Lei's notes, so please thank him for the content and blame me for any typos or other errors.

\subsection{Lifting Places}
As usual, we fix $K$ a number field or function field in one variable over a fixed ground field $F.$ (Notably, $K$ need not be global---$\QQ(t)$ is permissable.) As in our $AKLB$ set-up for function fields, we will assume that the integer ring $A$ contains the constant field $F$ when $K$ is a function field. In particular, all places are trivial on $F.$

So let's start building our $AKLB$ set-up here. Fix $L/K$ a finite extension (not necessarily separable) and $v$ a (possible archimedean) place of $K,$ and find some place $w$ of $L$ extending $v.$
\begin{defi}[Extending places]
	Fix $L/K$ as above. Then we say that $w$ \textit{divides} $v,$ and we will notate this by $w\mid v.$
\end{defi}
Anyways, we are able to set up the following diagram of field extensions.
% https://q.uiver.app/?q=WzAsNCxbMSwyLCJLIl0sWzAsMSwiTCJdLFsxLDAsIkxfdyJdLFsyLDEsIktfdiJdLFsyLDEsIiIsMCx7InN0eWxlIjp7ImhlYWQiOnsibmFtZSI6Im5vbmUifX19XSxbMSwwLCIiLDAseyJzdHlsZSI6eyJoZWFkIjp7Im5hbWUiOiJub25lIn19fV0sWzIsMywiIiwyLHsic3R5bGUiOnsiaGVhZCI6eyJuYW1lIjoibm9uZSJ9fX1dLFszLDAsIiIsMix7InN0eWxlIjp7ImhlYWQiOnsibmFtZSI6Im5vbmUifX19XV0=
\[\begin{tikzcd}
	& {L_w} \\
	L && {K_v} \\
	& K
	\arrow[no head, from=1-2, to=2-1]
	\arrow[no head, from=2-1, to=3-2]
	\arrow[no head, from=1-2, to=2-3]
	\arrow[no head, from=2-3, to=3-2]
\end{tikzcd}\]
We showed a while ago that $L_w$ is indeed the composite $LK_v,$ or at least isomorphic to it.

Continuing our set-up, we fix $A$ the integer ring in $K.$ When $K$ is a number field, this means $\mathcal O_K,$ and when $K$ is a function field in one variable over a constant field $F,$ we will set $K=F(t)$ for any $F$-transcendental element $t\in K$ and then fix $A$ to be the integral closure of $F[t]$ or $F[1/t]$ in $K.$

Now let's discuss our place $v$ a bit more finely.
\begin{itemize}
	\item In the case where $v$ is nonarchimedean, we are assuming that $v$ is being induced by a nonzero prime ideal of our integer ring $A.$ This is a fact (generalized Ostrowski's theorem) for number fields, and I think it's also true more generally for our function fields.
	\item If $v$ is archimedean, then we are restricting ourselves to the case where $K$ is a number field. Because the only complete archimedean fields are $\RR$ and $\CC,$ we must have $K_v\cong\RR$ or $K_v\cong\CC,$ where the absolute value associated to $v$ comes from $\RR$ or $\CC,$ respectively.
\end{itemize}
This second case can be dealt with quickly.
\begin{prop}
	Fix $L/K$ as above and $v$ an archimedean place. Then $v$ (and $w$) is induced by restricting $|\cdot|_\RR$ or $|\cdot|_\CC$ from an embedding $K\into\RR$ or $K\into\CC.$
\end{prop}
\begin{proof}
	As described above, we have $K_v\cong\RR$ or $K_v\cong\CC.$ Choosing appropriately, we essentially get an embedding
	\[K\into K_v,\]
	so the behavior of $v$ on $K$ is simply its behavior on $K_v$ (which is either $\RR$ or $\CC$) properly restricted. This is what we wanted.
\end{proof}
We would like to generalize this for nonarchimedean places. In particular, we have the following.
\begin{theorem}[Extension]
	Fix $L/K$ as above and $v$ a place (not necessarily archimedean). Then we have the following.
	\begin{listalph}
		\item We can induce $w$ as $\overline v\circ\tau$ where $\tau:L\into\overline{K_v}$ is some embedding into the algebraic closure of $K_v.$ Here, $\overline v$ is the unique extension of $v$ from $K_v$ to the algebraic extension $\overline{K_v}/K_v.$
		\item Two embeddings $\tau_1,\tau_2:L\into\overline{K_v}$ induce the same place $w$ if and only if they are conjugate. Explicitly, $\tau_1$ and $\tau_2$ are conjugate if and only if there is some $\sigma\in\op{Gal}(\overline{K_v}/K_v)$ such that $\tau_1=\sigma\tau_2.$
	\end{listalph}
\end{theorem}
\begin{proof}
	We take these one at a time.
	\begin{listalph}
		\item Fix $w$ an extension of $v$ to $L$ so that we have a canonical map $K_v\into L_w$ by, say, the universal property of $K_v.$ (Namely, $K\subseteq L\into L_w$ where the metrics cohere, and $L_w$ is complete.)

		Now, $L_w=LK_v$ is a finite extensions of $K_v,$ and in particular it will be an algebraic extensions, so fix some embedding $\tau:L_w\into\overline{K_v}$ fixing $K_v.$ This gives the following diagram.
		% https://q.uiver.app/?q=WzAsNSxbMSwzLCJLIl0sWzAsMiwiTCJdLFsyLDIsIktfdiJdLFsxLDEsIkxfdyJdLFsyLDAsIlxcb3ZlcmxpbmV7S192fSJdLFszLDQsIlxcdGF1IiwwLHsic3R5bGUiOnsiYm9keSI6eyJuYW1lIjoiZGFzaGVkIn19fV0sWzMsMSwiIiwyLHsic3R5bGUiOnsiaGVhZCI6eyJuYW1lIjoibm9uZSJ9fX1dLFsxLDAsIiIsMix7InN0eWxlIjp7ImhlYWQiOnsibmFtZSI6Im5vbmUifX19XSxbNCwyLCIiLDAseyJzdHlsZSI6eyJoZWFkIjp7Im5hbWUiOiJub25lIn19fV0sWzMsMiwiIiwwLHsic3R5bGUiOnsiaGVhZCI6eyJuYW1lIjoibm9uZSJ9fX1dLFsyLDAsIiIsMCx7InN0eWxlIjp7ImhlYWQiOnsibmFtZSI6Im5vbmUifX19XV0=
		\[\begin{tikzcd}
			&& {\overline{K_v}} \\
			& {L_w} \\
			L && {K_v} \\
			& K
			\arrow["\tau", dashed, from=2-2, to=1-3]
			\arrow[no head, from=2-2, to=3-1]
			\arrow[no head, from=3-1, to=4-2]
			\arrow[no head, from=1-3, to=3-3]
			\arrow[no head, from=2-2, to=3-3]
			\arrow[no head, from=3-3, to=4-2]
		\end{tikzcd}\]
		Now, by the uniqueness of extension of the absolute value of complete discretely valued fields, the unique valuation on $\overline{K_v}$ must cohere back with the absolute value in $L_w.$ Namely, for any $\alpha\in L_w,$ we have $|\alpha|_w=|\tau\alpha|_{\overline v},$ where $|\cdot|_{\overline v}:\overline{K_v}\to\RR.$
		
		So we see that our chosen embedding $\tau:L_w\into\overline{K_v}$ provides us with some $\tau|_L:L\into\overline{K_v}$ such that the $w=\overline v\circ\tau.$

		\item In one direction, suppose that $\tau_1,\tau_2:L\into\overline{K_v}$ are conjugate so that there is $\sigma\in\op{Gal}(\overline{K_v}/K_v)$ such that $\tau_1=\sigma\tau_2\sigma^{-1}.$ But we know that
		\[|\sigma\alpha|_{\overline v}=|\alpha|_{\overline v}\]
		because automorphisms will preserve the absolute value.\footnote{I think this is because $K_v$ is complete: viewing $\overline{K_v}$ as a colimit of finite extensions, there is exactly one way to extend $v$ to any finite extensions of $K_v,$ so the two valuations $\alpha\mapsto|\alpha|_{\overline v}$ and $\alpha\mapsto|\sigma\alpha|_{\overline v}$ must coincide.} So indeed, we find that
		\[|\tau_1\alpha|_{\overline v}=\left|\sigma\tau_2\alpha\right|_{\overline v}.\]

		In the other direction, suppose that $\tau_1,\tau_2:L\into\overline{K_v}$ induce the same place of $L.$ Now, we build the following diagram.
		% https://q.uiver.app/?q=WzAsNCxbMCwwLCJcXG92ZXJsaW5le0tfdn0iXSxbMCwxLCJcXHRhdV8xKEwpIl0sWzEsMSwiXFx0YXVfMihMKSJdLFsxLDAsIlxcb3ZlcmxpbmV7S192fSJdLFsxLDAsIlxccm90YXRlYm94ezkwfXtcXHN1YnNldGVxfSIsMSx7InN0eWxlIjp7ImJvZHkiOnsibmFtZSI6Im5vbmUifSwiaGVhZCI6eyJuYW1lIjoibm9uZSJ9fX1dLFsyLDMsIlxccm90YXRlYm94ezkwfXtcXHN1YnNldGVxfSIsMSx7InN0eWxlIjp7ImJvZHkiOnsibmFtZSI6Im5vbmUifSwiaGVhZCI6eyJuYW1lIjoibm9uZSJ9fX1dLFsxLDIsIlxcdGF1XzJcXGNpcmNcXHRhdV8xXnstMX0iLDJdXQ==&macro_url=https%3A%2F%2Fgist.githubusercontent.com%2FdFoiler%2F1e12fec404cad7e185260f0c9b68977d%2Fraw%2F909cc7837a29133fb63fb0e9300d15bfe7417fc5%2Fnir.sty
		\[\begin{tikzcd}
			{\overline{K_v}} & {\overline{K_v}} \\
			{\tau_1(L)} & {\tau_2(L)}
			\arrow["{\rotatebox{90}{$\subseteq$}}"{description}, draw=none, from=2-1, to=1-1]
			\arrow["{\rotatebox{90}{$\subseteq$}}"{description}, draw=none, from=2-2, to=1-2]
			\arrow["{\tau_2\circ\tau_1^{-1}}"', from=2-1, to=2-2]
		\end{tikzcd}\]
		In particular, we see that we are interested in the map $\sigma:=\tau_2\circ\tau_1^{-1}.$ Because our embeddings are preserving our absolute values, we see that $\sigma$ is a continuous map, where the topologies are induced by $\overline{K_v}.$ So we can extend $\sigma$ to a continuous map
		\[\sigma:\overline{\tau_1(L)}\to\overline{\tau_2(L)},\]
		where the overline denotes the topological closures. But $\overline{\tau_\bullet(L)}$ is essentially isomorphic to the metric completion of $L$ with respect to the induced placce, so we get that $\overline{\tau_\bullet(L)}=\tau_\bullet(L)K_v.$ Continuing up to the algebraic closure, we can extend $\sigma$ to a map $\overline{K_v}\to\overline{K_v}.$

		Continuing, $\sigma$ is trivial on $K$ because the original embeddings $\tau_\bullet$ cohere on $K,$ so because $\sigma$ is continuous, $\sigma$ will be trivial on $K_v$ because $K$ is desnse in $K_v.$ So $\sigma,$ trivial on $K_v,$ is really an element of $\op{Gal}(\overline{K_v}/K_v).$ So indeed, $\tau_1=\sigma\tau_2,$ finishing.
		\qedhere
	\end{listalph}
\end{proof}

\subsection{Applcations of Lifting}
This gives the following nice version of Dedekind--Kummer.
\begin{proposition}[Dedekind--Kummer]
	Fix $L/K$ as before and $v$ a place of $K.$ Further assume that there exists $\alpha\in L$ such that $L=K(\alpha),$ and fix $f\in K[x]$ the monic irreducible polynomial for $\alpha.$ Then, factoring $f$ into $K_v[x]$ as
	\[f=\prod_{k=1}^rf_k^{e_k},\]
	there is a natural bijection between places $w$ over $v$ and monic irreducible polynomials in $K_v[x].$
\end{proposition}
\begin{proof}
	The bijection is actually only natural up to our $\alpha$ chosen in advance. Indeed, there is a bijection
	\[\{\tau:L\into\overline{K_v}\text{ fixing }K\}\to\{\beta\in\overline{K_v}:f(\beta)=0\}\]
	by taking $\tau\mapsto\tau\alpha.$ Indeed, because $L=K(\alpha),$ the behavior of $\tau$ on $\alpha$ will uniquely determine the entire embedding, and $\tau$ is allowed to send $\alpha$ to exactly any of the roots.

	Now, two embeddings $\tau_1$ and $\tau_2$ induce the same place $w$ over $v$ if and only if they are conjugate over $K_v.$ But the corresponding roots $\tau_1\alpha$ and $\tau_2\alpha$ are also going to be in the same irreducible factor $f_\bullet$ of $f$ if and only if there is an automorphism $\sigma\in\op{Gal}(\overline{K_v}/K_v)$ such that $\sigma\tau_1\alpha=\tau_2\alpha,$ which still implies (and is equivalent to) $\tau_1=\sigma\circ\tau_2,$ so they are conjguate.

	So modding out the set of embeddings $\tau:L\into\overline{K_v}$ fixing $K$ gives a set in bijection with the places $w$ over $v.$ And on the other side this modding creates equivalence classes of roots which correspond to the monic irreducibles in the factorization. This is what we wanted.
\end{proof}
\begin{remark}
	This statement is intended to generalize the Dedekind--Kummer theorem, which applied to (finite) prime-splitting and only worked when the (finite) prime was coprime to the conductor of $A[\alpha].$ I am under the impression that factorization in $K_v$ for finite places $v$ is done by factoring modulo the prime and then lifting the factorization via Hensel, so roughly the same hard computations (e.g., factoring over a finite field) are required.
\end{remark}
\begin{example}
	Fix $K$ a number field, and we study the places extending $\infty$ of $\QQ.$ Fixing some $\alpha$ such that $K=\QQ(\alpha)$ with monic irreducible $f,$ we see that the factorization of $f$ in $\QQ_\infty=\RR$ will be into one quadratic for each pair of complex conjugate roots of $f$ and one linear factor for each real root. Each complex conjugate pair does indeed induce exactly one embedding $K\into\CC,$ and each linear root will induce exactly one embedding $K\into\RR.$
\end{example}
Anyways, let's get back to theory. We have the following generalization of the fundamental identity to places.
\begin{proposition}
	Fix $L/K$ as before with $v$ a place, and further assume that $L/K$ is separable. Then
	\[[L:K]=\sum_{w\mid v}[L_w:K_v].\]
\end{proposition}
\begin{proof}
	Because $L/K$ is finite and separable, we may find $\alpha\in L$ such that $L=K(\alpha)$ and apply our upgrade of Dedekind--Kummer above. Now, taking $f$ the irreducible monic polynomial for $\alpha,$ we see that $L/K$ being separable implies $f$ will fully split into distinct linear factors in $\overline K.$ So $f$ will fully split into distinct linear factors in $\overline{K_v}$ (by tracking an embedding $\overline K\into\overline{K_v}$ induced by the universal property of $\overline K$).

	In particular, factoring $f=\prod_{k=1}^rf_k$ in $K_v[x]$ has all exponents equal to $1,$ so
	\[[L:K]=\deg f=\sum_{k=1}^r\deg f_k=\sum_{k=1}^r[L_w:K_v],\]
	where $\deg f_k=[L_w:K_v]$ because $L_w=K_vL=K_vK(\tau\alpha)=K_v(\tau\alpha).$ Here, the $\tau$ has $w$ chosen as the place in the bijection of the extension of Dedekind--Kummer above so that $\deg f_k$ will match up with $[K_v(\tau\alpha):K_v].$ Anyways, this finishes.
\end{proof}
Also, we can now finally classify extensions of nonarchimedean extensions.
\begin{proposition}
	Fix $L/K$ as before and $v$ a nonarchimedean place corresponding to a prime $\mf p$ of $A.$ Then any extension $w$ of $v$ to $L$ is induced by some prime $\mf q$ of $B$ lying over $\mf p.$
\end{proposition}
\begin{proof}
	As in the extension theorem, we may induce $w$ by some embedding $\tau:L\into\overline{K_v}$ fixing $K.$ Now, the needed prime ideal is
	\[\mf q:=\left\{\beta\in B:|\tau\beta|_{\overline v}<1\right\}.\]
	We will discuss the primality of $\mf q$ momentarily, but we do note that this is indeed the valuation ideal for a place, so it should work.

	Now, $\mf q$ is the pull-back of the valuation ideal in $\overline{K_v},$ which is certainly prime there, and pull-backs of primes are prime. We are kind of implicitly using that $\tau(B)$ will be in the valuation ring of $\overline{K_v},$ which is true because any $\beta\in B$ is integral over $A,$ so $\tau\beta$ is integral over $\tau(A),$ and $\tau(A)$ is certainly contained in the valuation ring of $K_v.$
\end{proof}