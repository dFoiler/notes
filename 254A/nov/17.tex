% !TEX root = ../notes.tex















Here we go.

\subsection{Unramified Extensions}
For today, we will continue to take $K$ a complete discretely valued field, with $A$ its valuation ring, $\mf p$ its valuation ideal, and $\kappa$ its residue field. Then we will also have $L/K$ an algebraic extension (not necessarily finite), with $B,\mf q,\lambda$ as one might expect.
\begin{proposition}
	Fix $\lambda_0$ a separable extension of $\kappa.$ Then there exists an unramified algebraic extension $L/K$ with residue field isomorphic to $\lambda_0.$ In fact, this $L$ is unique: given $L_1$ and $L_2$ satisfying, there is a unique pair $(f,\varphi)$ of isomorphisms $f:L_1\to L_2$ and $\varphi:\lambda_1\to\lambda_2,$ commuting.
\end{proposition}
\begin{proof}
	We proceed in steps.
	\begin{itemize}
		\item We start by taking $[\lambda_0:\kappa]<\infty,$ and show that $L/K$ exists. Then there is some $\overline\alpha\in\lambda_0$ such that $\lambda_0\cong\kappa(\overline\alpha),$ which we we lift by hand to $L:=K(\alpha).$ Now we see that $[L:K]=[\lambda_0:\kappa]$ using our arguments from last class, so $L/K$ is indeed unramified.
		\item We still take $[\lambda_0:\kappa]<\infty,$ and now we show the uniqueness. Now suppose $L_2$ and $L_2$ are two extensions satisfying the needed result. Well, take $\overline\alpha\in\lambda_0$ with $\lambda_0=\kappa(\overline\alpha).$

		Then we can lift $\overline\alpha$ to $\alpha_1\in L_1$ and $\alpha_2\in L_2$ such that $L_1=K(\alpha_1)$ and $L_2=K(\alpha_2)$ with irreducible polynomial $p.$ Now we see that
		\[L_1\cong\frac{K[x]}{(p)}\cong L_2\]
		are $K$-algebra isomorphisms; further, the residue fields under the compatible isomorphism have
		\[\lambda_1\cong\frac{\kappa[x]}{(\overline p)}\cong\lambda_2.\]
		This isomorphisms are unique because $\alpha_1$ must get sent to $\alpha_2$ (they are both roots of $p$ reducing to $\overline\alpha$) and $\overline{\alpha_1}$ must get sent to $\overline{\alpha_2}$ (through $\lambda_0,$ I think) for reasons which are unclear to me. \todo{}
		\item For the case of infinite algebraic extensions, throw Zorn's lemma at the problem.
		\qedhere
	\end{itemize}
\end{proof}
\begin{example}
	We have that $\QQ_p^{\op{unram}}$ has residue field $\overline{\FF_p}$ has the same value group as $\QQ_p.$ We have that $\FF_p((t))^{\op{unram}}$ has residue field $\overline{\FF_p}$ has the same value group as $\FF_p((t)).$
\end{example}
We have the following corollary.
\begin{corollary}
	Take $\kappa$ finite and $n$ a positive integer. Then there exists exactly one unramified extension of $K$ of degree $n.$
\end{corollary}
\begin{proof}
	The corresponding unramified extension will have residue field of size $(\#\kappa)^n,$ of which there is only one field.
\end{proof}
\begin{example}
	The above corollary includes all nonarchimedean local fields.
\end{example}

\subsection{Totally Ramified Extensions}
We are interested in finite extensions with separable residue fields, such as with number fields. We have the following lemma.
\begin{lem}[Krasner's]
	Fix $K$ a complete discrete valued field. Fix $\alpha,\beta\in\overline K$ such that $\alpha$ is separable over $K$ and $|\beta-\alpha|<|\alpha'-\alpha|$ for each Galois conjugate $\alpha'$ of $\alpha$ not equal to $\alpha.$ Then it follows $\alpha\in K(\beta).$
\end{lem}
\begin{remark}
	In the condition, we say that $\alpha$ belongs to $\beta.$
\end{remark}
\begin{proof}
	By the strong triangle inequality, we observe that $|\alpha'-\beta|=|\alpha-\beta|$ for each Galois conjugate $\alpha'\ne\alpha,$ so the strong triangle inequality again gives $|\beta-\alpha|<|\beta-\alpha'|.$

	Now, fix $\sigma\in\op{Gal}(\overline K/K(\beta)).$ Then
	\[|\beta-\sigma\alpha|=|\sigma(\beta-\alpha)|=|\beta-\alpha|<|\beta-\alpha'|\]
	for each Galois conjugate $\alpha'\ne\alpha.$ (That $|\sigma(\beta-\alpha)|=|\beta-\alpha|$ holds because the absolute value must extend properly and uniquely over isomorphic algebraic extensions, and $K(\alpha-\beta)\cong K(\sigma(\alpha-\beta)).$) Thus, $\sigma\alpha=\alpha$ is forced. So because $\alpha$ is separable over $K(\beta),$ we get $\alpha\in K(\beta)$ by Galois theory.
\end{proof}
So now take $L/K$ a finite extension with $\lambda/\kappa$ separable. Setting $T$ to be the largest unramified intermediate extension of $L/K.$ Then it happens that the residue field $\tau$ of $T$ will be equal to $\lambda,$ making $L/T$ a totally ramified extension, as in the following definition.
\begin{definition}[Total ramified]
	An extension of complete discretely valued fields $L/K$ is \textit{totally ramified} if and only if $f(\mf q/\mf p)=1$ and $e(\mf q/\mf p)=[L:K].$
\end{definition}
Indeed, the given extension is totally unramified by checking the tower law, I think: $T$ is supposed to absorb all of the inertial degree, for if there is any inertial degree in $L/T,$ then we could find an unramified extension of the required degree to kill that inertial degree.

So now we are interested in totally ramified extensions. Reset $L/K$ to be a totally ramified extension, and take $\pi\in L$ to be a uniformizer for $B.$ Let $v$ and $w$ be the (discrete) valuations of $K$ and $L,$ and we will force $v(K^\times)=\ZZ$ and $w|_K=v$ to restrict properly, implying $v(L^\times)=\frac1e\ZZ.$ Now, fix
\[f=a_0+a_1x+a_2x^2+\cdots+a_{n-1}x^{n-1}+x^n\in A[x]\]
to be the monic irreducible for $\pi.$ We note that each $a_\bullet$ lives in $\mf p$ because these coefficients are elementary symmetric polynomials in the conjugates of $\pi,$ which must all live in $\mf q,$ so $a_\bullet\in\mf q\cap L=\mf p.$

Further, we see that $a_0\notin\mf p^2$ because $v(a_0)=v(\op N_K^{K(\pi)}\pi)=[K(\pi):K]/e,$ but $[K(\pi):K]\le[L:K]=e$ while $v(a_0)\in\ZZ,$ which gives $L=K(\pi)$ and $v(a_0)=1.$ In particular, $f$ is an Eisenstein polynomial, I guess.

Anyways, we have the following corollary.
\begin{corollary}[Krasner]
	Fix $f\in K[x]$ to be a separable, monic irreducible polynomial. Then any $g\in K[x]$ coefficient-wise sufficiently close to $f$ will have the following.
	\begin{itemize}
		\item $g$ is irreducible and separable.
		\item For each root $\alpha\in\overline K$ of $f,$ there is a root $\beta\in\overline K$ of $g$ such that $\beta\in K(\alpha),$ a condition which comes from Krasner's lemma. In fact, $K(\alpha)=K(\beta).$
	\end{itemize}
\end{corollary}
\begin{proof}
	By multiplying $f$ through with a sufficiently large constant, we may force $f\in A[x]$ so that sufficiently close $g$ will have $g\in A[x].$ Fix $\alpha$ a root of $f.$ Then $f'(\alpha)\ne0$ because $f$ is separable, so any $g$ sufficiently close will have
	\[|g(\alpha)|<|f'(\alpha)|^2=|g'(\alpha)|^2.\]
	In particular, the sufficiently close can push $|g(\alpha)|$ arbitrarily small and $f'(\alpha)$ componentwise equal to $g'(\alpha)$ in sizes.

	Thus, Hensel's lemma promises a root $\beta\in K(\alpha)$ kind of close to $\alpha.$ In particular, for sufficiently close to $g,$
	\[|g(\alpha)|<|f'(\alpha)\cdot\min\{|\alpha'-\alpha|\},\]
	where the minimum is taken under all Galois conjugates $\alpha'\ne\alpha.$ So Krasner's lemma gives $\alpha\in K(\beta),$ giving $K(\alpha)=K(\beta).$

	To finish, we note that $g$ is irreducible because $f$ is irreducible---they are both monic of the same degree, so $K(\alpha)=K(\beta)$ forces $g$ irreducible. And $g$ is separable by sending $g$ close enough for the above roots promised by Hensel to be necessarily distinct.
\end{proof}
Anyways, we have the following definition, which we will use later.
\begin{definition}[\texorpdfstring{$p$}{p}-adic Field]
	A $p$-adic field is a finite extension of $\QQ_p.$
\end{definition}
In particular the above corollary gives the following.
\begin{corollary}
	Fix $K$ a $p$-adic field. Then the following are true.
	\begin{itemize}
		\item Given an $e>0,$ there are only finitely many totally ramified extensions of $K$ of degree $e$ up to isomorphism.
		\item Given an $n>0,$ there are only finitely many extensions of $K$ of degree $n.$
	\end{itemize}
\end{corollary}
\begin{proof}
	We leave the proof as an exercise. The main point is that (a) comes from compactness of $\mf p^{e-1}\times(\mf p\setminus\mf p^2),$ and (b) follows because any extension can be decomposed into an unramified extension and a totally ramified extension, both of which we know there are finitely many options.
\end{proof}