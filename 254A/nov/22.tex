% !TEX root = ../notes.tex













For today, we continue to let $K$ be a number field or a function field in one variable over a fixed constant field $F.$ Then we have $v$ a place of $K$ upon which $v$ is trivial. We also take $L/K$ to be a finite extension, where $w$ extends $v.$ This gives the following diagram.
% https://q.uiver.app/?q=WzAsNCxbMSwyLCJLIl0sWzAsMSwiTCJdLFsyLDEsIktfdiJdLFsxLDAsIkxfdyJdLFszLDEsIiIsMCx7InN0eWxlIjp7ImhlYWQiOnsibmFtZSI6Im5vbmUifX19XSxbMSwwLCIiLDAseyJzdHlsZSI6eyJoZWFkIjp7Im5hbWUiOiJub25lIn19fV0sWzMsMiwiIiwyLHsic3R5bGUiOnsiaGVhZCI6eyJuYW1lIjoibm9uZSJ9fX1dLFsyLDAsIiIsMix7InN0eWxlIjp7ImhlYWQiOnsibmFtZSI6Im5vbmUifX19XV0=
\[\begin{tikzcd}
	& {L_w} \\
	L && {K_v} \\
	& K
	\arrow[no head, from=1-2, to=2-1]
	\arrow[no head, from=2-1, to=3-2]
	\arrow[no head, from=1-2, to=2-3]
	\arrow[no head, from=2-3, to=3-2]
\end{tikzcd}\]
We see that $LK_v=L_w,$ so the natural map $L\otimes_KK_v\to L_w$ is an isomorphism because $L\cap K_v=K,$ I think.\todo{}

\subsection{More Extensions of Valuations}
Last time we took $L/K$ to be a separable and achieved
\[[L:K]=\sum_{w\mid v}[L_w:K_v].\]
Today we show the following.
\begin{proposition}
	Take $L/K$ to be separable. Then the map
	\[L\otimes_KK_v\to\prod_{w\mid v}L_w\]
	obtained by combining the $L\otimes_KK_v\to L_w$ isomorphisms is in fact an isomorphism of $K_v$-algebras.
\end{proposition}
\begin{proof}
	By separability, we may write $L=K(\alpha),$ and we let $f\in K[x]$ be the irreducible polynomial for $\alpha.$ We can fully factor $f$ as
	\[f=\prod_{k=1}^rf_r,\]
	where $f_r$ are also distinct monic irreducibles in $K_v[x].$ (They are distinct by separability.) Now, the isomorphism $L\cong K[x]/(f)$ becomes
	\[L\otimes_KK_v\cong\frac{K_v[x]}{(f)}\cong\prod_{k=1}^r\frac{K_v[x]}{(f_r)}\cong\prod_{w\mid v}L_w.\]
	Tracking the isomorphisms through, we are explicitly taking $\alpha\otimes1$ to $x+(f)$ to $(x+(f_k))_{k=1}^r$ to $(\beta_1,\ldots,\beta_r),$ where the $\beta_\bullet$ are a root of $f_\bullet$ in $L_w,$ where the correspondence is explicit from last class.
\end{proof}

\subsection{Local Norms and Traces}
Take $L/K$ separable still. Technically, we only need
\[L\otimes_KK_v\to\prod_{w\mid v}L_w\]
to be an isomorphism, but separability is nice for psychological reasons.

Now fix some $\alpha\in L.$ We see that multiplication by $\alpha$ becomes a $K$-linear map $L\to L,$ and so it will induce a $K_v$-linear map $L\otimes_KK_v\to L\otimes_KK_v,$ and in fact we are preserving the multiplication structure in the product
\[L\otimes_KK_v\cong\prod_{w\mid v}L_w.\]
In particular, multiplication by $\alpha$ corresponds to multiplication by the various $\beta_\bullet,$ induced as in the proposition above. This gives us the following definition.
\begin{definition}[Local norm and trace]
	Fix everything as in the previous paragraph. Then
	\[\op N_K^L\alpha:=\prod_{w\mid v}\op N_{K_v}^{L_w}\beta_\bullet.\]
	We also define
	\[\op T_K^L\alpha=\sum_{w\mid v}\op T_{K_v}^{L_w}\alpha\]
	in essentially the same way.
\end{definition}
These correspond to the usual definition of the norm and trace by using the definition of norm that is the determinant of the multiplication map. We discussed above how this multiplication map behaves via the isomorphisms, so the above does indeed work out, I think.

\subsection{\texorpdfstring{$AKLB$}{AKLB} Set-Up: Function Field Edition}
We are going to start chapter 3 today. Our main point is that function fields are easier than number fields because we have access to algebraic geometry for function fields. So one strategy is to try to push these function field ideas back to number fields to prove things.

So let's study function fields. Take $X$ to be a compact Riemann surface, like a torus or something. Fix $\pi:X\to\CC$ be some nonconstant meromorphic function on $X.$ Viewing $\CC$ as a subset of the Riemann sphere $\CP^1$ (as $\CC\cup\{\infty\}$). If $\pi$ has a pole at some point $p\in X,$ then $\frac1\pi$ has a removable singularity at $p,$ which just turns into a zero. The point is that $\pi$ extends to a holomorphic map
\[\pi:X\to\CP^1.\]
We have the following statement.
\begin{proposition}
	We define
	\[K:=\{\text{meromorphic functions }f:\CP^1\to\CC\}.\]
	Here $K$ is a field, and in fact it is $\CC(z),$ where $\CC$ is identified with our constant functions, and $z$ is identified with reversing the natural embedding $\CC\into\CP^1.$
\end{proposition}
\begin{proof}
	Fix $f:\CC\to\CC$ some meromorphic function. Then we may multiply out the roots via some finite polynomial $\prod_\bullet(z-\zeta_\bullet)^{n_\bullet}$ which extends to a holomorphic function on $\CC.$ In particular, our new function as a pole at $\infty$ of finite order (holomorphic functions only have finitely many roots), so it is a polynomial $g\in\CC[z]$ so that
	\[f=\frac g{\prod_\bullet(z-\zeta_\bullet)^{n_\bullet}}\in\CC(z).\]
	Of course, the converse is also true: anyone in $\CC(z)$ will give a meromorphic function $\CP^1\to\CC.$
\end{proof}
\begin{remark}
	By doing the extending process described above, we get
	\[K=\{\text{holomorphic functions}\CP^1\to\CP^1\}.\]
\end{remark}
We claimed earlier that we had the following; let's prove it now.
\begin{proposition}
	The places of $\CC(z)$ which are trivial on $\CC$ are in canonical bijection with
	\[(\op{Spec}\CC[z]\setminus\{0\})\cup(\op{Spec}\CC[1/z]\setminus\{0\}).\]
\end{proposition}
\begin{proof}
	Let's prove this. In particular, any absolute value $|\cdot|$ on $\CC[z]$ trivial on $\CC$ will necessarily have $|z|\le1$ or $|1/z|\le1,$ so either $\CC[z]$ or $\CC[1/z]$ is in the valuation ring. We also remark that $|\cdot|$ is trivial on $\CC$ and hence trivial on $\ZZ,$ so $|\cdot|$ is nonarchimedean.
	\begin{remark}
		The above union is not disjoint because $\CC[z]$ and $\CC[1/z]$ will both localize to $\CC[z,1/z].$
	\end{remark}
	We now just take our cases one at a time.
	\begin{itemize}
		\item If $\CC[z]$ is in the valuation ring $A$ of $|\cdot|,$ then $\{\alpha\in K:|\alpha|<1\}\subseteq A$ pulls back to a prime ideal $\mf p$ of $\CC[z],$ notably not everything because $\CC$ is excluded.
		
		We would like this prime ideal to be nonzero; well, $|\cdot|$ is nontrivial, so there is some $\alpha\in\CC(z)^\times$ suc that $|\alpha|\ne1.$ Without loss of generality, we take $|\alpha|<1.$ But now setting $\alpha=f/g$ with $f,g\in\CC[z],$ we find that $|g|\le1$ because $|z|\le1$ and $|\cdot|$ is trivial on $\CC,$ and so $|f|=|\alpha|\cdot|g|<1.$ So $f\in\mf p$ is in our prime ideal.
		\item The case where $\CC[1/z]$ is in the valuation ring is analogous.
		\qedhere
	\end{itemize}
\end{proof}
We remark that our copies of $\CC$ cover the Riemann sphere $\CP^1$ by stereographic projection. There's a nice circular picture, or we can just draw the picture with $\CC$ as a line. I'm too lazy to live-TeX this diagram.
% \begin{center}
% 	\begin{asy}
% 		unitsize(2cm);
% 		draw((0,0)--(1,0));
% 		draw((0,-1/3)--(1,-1/3));
% 		draw((0,-2/3)--(1,-2/3));
% 		label("$\mathbb C$", )
% 	\end{asy}
% \end{center}

Now let's return to $X.$ We set
\[L=\mathcal K(X):=\{\text{meromorphic functions }f:X\to\CC\}.\]
This is a field because look at it. Now, if we take any $\pi:X\to\CC$ meromorphic and lift it up to $\pi:X\to\CP^1,$ we see that we get a function induced by $\pi$ which sends $\CC(z)=\mathcal K(\CP^1)$ into $\mathcal K(X)$ by pre-composition. In fact, this is a ring homomorphism of fields, so it becomes a field homomorphism, so it becomes injective for free.

The point is that, when we let $A=\CC[z]$ to be the meromorphic functions $\CP^1\to\CC$ (where we are using the correct copy of $\CC$ for our chosen $\pi$), then we can look at the integral closure $B$ in $L$ is the set of holomorphic maps $\pi^{-1}\CC\to\CC$ which extend fully to $X\to\CC.$ It also happens that $L/K$ is finite, which gives us an $AKLB$ set-up, as follows.
% https://q.uiver.app/?q=WzAsNCxbMCwxLCJBIl0sWzAsMCwiQiJdLFsxLDAsIkw9XFxtYXRoY2FsIEsoWCkiXSxbMSwxLCJLPVxcQ0MoeikiXSxbMiwzLCIiLDAseyJzdHlsZSI6eyJoZWFkIjp7Im5hbWUiOiJub25lIn19fV0sWzEsMCwiIiwwLHsic3R5bGUiOnsiaGVhZCI6eyJuYW1lIjoibm9uZSJ9fX1dLFsxLDIsIlxcc3Vic2V0ZXEiLDEseyJzdHlsZSI6eyJib2R5Ijp7Im5hbWUiOiJub25lIn0sImhlYWQiOnsibmFtZSI6Im5vbmUifX19XSxbMCwzLCJcXHN1YnNldGVxIiwxLHsic3R5bGUiOnsiYm9keSI6eyJuYW1lIjoibm9uZSJ9LCJoZWFkIjp7Im5hbWUiOiJub25lIn19fV1d
\[\begin{tikzcd}
	B & {L=\mathcal K(X)} \\
	A & {K=\CC(z)}
	\arrow[no head, from=1-2, to=2-2]
	\arrow[no head, from=1-1, to=2-1]
	\arrow["\subseteq"{description}, draw=none, from=1-1, to=1-2]
	\arrow["\subseteq"{description}, draw=none, from=2-1, to=2-2]
\end{tikzcd}\]
We will not prove actually prove that we are getting an $AKLB$ set-up because it would take us a bit too far afield.