% !TEX root = ../notes.tex















The fun will soon continue.

\subsection{Hensel's Lemma}
As usual, we continue setting $(K,|\cdot|)$ is a nonarchimedean valued field, and $A$ is its valuation ring. We continue showing Hensel's lemma.
\begin{lemma}[Hensel]
	Fix $f\in A[x],$ and suppose that we have $\alpha_0\in A$ such that $|f(\alpha_0)|<|f'(\alpha_0)|^2.$ Then the sequence defined recursively by
	\[\alpha_{n+1}:=\alpha_n-\frac{f(\alpha_n)}{f'(\alpha_n)}\]
	for each $n\in\NN$ will converge to a root $\alpha$ of $f$ such that
	\[|\alpha-\alpha_0|\le\frac{|f(\alpha_0)|}{|f'(\alpha_0)|}<|f'(\alpha_0)|,\]
	and in fact $\alpha$ is the only such root with $|\alpha-\alpha_0|<|f'(\alpha_0)|.$
\end{lemma}
\begin{proof}[Proof continued from last class]
	Last time we showed the following, for any $k\in\NN.$
	\begin{enumerate}[label=(\alph*)]
		\item $|\alpha_k-\alpha_0|\le\frac{|f(\alpha_)|}{|f'(\alpha_0)|}\le1,$
		\item $|f'(\alpha_k)|=|f'(\alpha_0)|,$
		\item $|\alpha_{k+1}-\alpha_k|\le c^{2^{i}},$ and
		\item $|f(\alpha_k)|le c^{2^i}|f'(\alpha_0)|^2,$ where $c<1.$
	\end{enumerate}
	Our old (c) is now (d), and (c) was proven above as an intermediate step. In particular, (c) implies that $\{\alpha_k\}_{k\in\NN}$ is a Cauchy sequence: we can bound, for any $k>\ell\ge0,$
	\[|\alpha_k-\alpha_\ell|=\left|\sum_{n=\ell+1}^k\alpha_n-\alpha_{n-1}\right|\le\max_{\ell+1\le n\le k}|\alpha_n-\alpha_{n-1}|\to0.\]
	So we may conjure $\alpha\in A$ the limit of this sequence, which satisfies $(*)$ because of (a) above and continuity of the function $x\mapsto|x-\alpha_0|.$ We will leave the check of uniqueness to the handout.
\end{proof}
We have the following special case.
\begin{corollary}
	Fix $f\in A[x].$ If $\alpha_0\in A$ has $|f'(\alpha_0)|=1$ and $|f(\alpha_0)|<1,$ then there is a unique root $\alpha$ of $f$ (described by the above recursion) with $|\alpha-\alpha_0|<1.$ In other words, if $\kappa$ is the residue field of $A,$ then we are asking for $\alpha_0\in\kappa$ such that $f(\alpha_0)=0$ while $f'(\alpha_0)\ne0.$
\end{corollary}
\begin{proof}
	This follows directly from Hensel's lemma.
\end{proof}
The book has the following form of Hensel's lemma, for the purposes of factoring.
\begin{definition}[Primitive]
	Fix $A$ a unique factorization domain. Then $f\in A[x]$ is \textit{primitive} if not all of its coefficients are divisible by any prime. If $A$ is a discrete valuation ring with $\mf m$ the unique prime ideal, then we are asserting $f\in A[x]\setminus\mf m[x].$
\end{definition}
\begin{lemma}[Hensel, II]
	Fix $(K,|\cdot|)$ and $A$ as usual with $\mf p:=\{\alpha\in A:|\alpha|<1\}$ the unique prime ideal of $A$ and $\kappa:=A/\mf p$ the residue field. Now, given $f\in A[x]$ a primitive polynomial and a factorization
	\[\overline f=\overline g\cdot\overline h\]
	in $\kappa[x]$ such that $\gcd(\overline g,\overline h)=(1),$ then we can lift $\overline g$ and $\overline h$ to a factorization $f=gh$ with $g,h\in A[x]$ such that $\deg g=\deg\overline g.$
\end{lemma}
\begin{proof}
	This is kind of technical, so we just refer to the book.
\end{proof}
\begin{remark}
	To relate this to the original version of Hensel's lemma, we can take $\overline g:=x-\overline{\alpha_0}$ to be linear in the above so that $\overline f(\overline{\alpha_0})=0.$ Then the condition that $\overline f'(\overline{\alpha_0})\ne0$ essentially means that $\overline f$ does not have a double root at $\alpha_0,$ which corresponds with $\gcd(\overline g,\overline f/\overline g)=(1).$ So the book Hensel's lemma will give us a linear
	\[g:=x-\alpha,\]
	yielding a root $\alpha$ of $f.$
\end{remark}

\subsection{Applications of Hensel's Lemma}
Anyways, we have the following corollary.
\begin{corollary}
	Fix $f\in\kappa[x]$ an irreducible polynomial, denoted by
	\[f(x)=\sum_{k=0}^na_kx^k,\]
	where $a_n\ne0.$ We define $|f|:=\max_k\{|a_k|\},$ and we claim $|f|=\max\{|a_0|,|a_n|\}.$
\end{corollary}
\begin{proof}
	We do this in cases.
	\begin{enumerate}[label=(\roman*)]
		\item If $a_0=0,$ then $f=a_1x$ for $f$ to be irreducible. This immediately gives the result.
		\item Take $a_0\ne0.$ By factoring out constants, we may assume that $|f|=1$ so that $f\in A[x]$ and is primitive. Now, find the smallest $r$ such that $|a_r|=1$ so that
		\[f(x)\equiv \underbrace{x^r}_{\overline g}\underbrace{\sum_{k=r}^{n-r}a_kx^{k-r}}_{\overline h}\pmod{\mf p}.\]
		Now we apply the second version of Hensel's lemma with $\overline g$ and $\overline h$ as above, where $\gcd(\overline g,\overline h)=(1)$ because $\overline h$ has a nonzero constant term and hence is not divisible by $x.$ So we get $g,h\in A[x]$ with $\deg g=r$ such that
		\[f=gh.\]
		But now $f$ is irreducible, so one of $g$ or $h$ is constant, so $r\in\{0,n\}.$ It follows $|a_r|=1$ is either reading as $|a_0|=1$ or $|a_n|=1,$ so $\max\{|a_0|,|a_n|\}=1,$ as needed.
		\qedhere
	\end{enumerate}
\end{proof}
And here is a corollary to our corollary.
\begin{corollary}
	Fix everything as in the previous corollary. Graphic $(k,\nu(a_k))$ for $0\le k\le n,$ we have that none of the points are below the line from $(0,\nu(a_0))$ to $(n,\nu(a_n)).$
\end{corollary}
\begin{proof}
	This is more or less given in the previous corollary, but we won't give more details here.
\end{proof}
\begin{remark}
	This leads to the theory of the Newton polygon, which we won't discuss here.
\end{remark}
Here is another corollary.
\begin{corollary}[Gauss's lemma, nonarchimedean case]
	Fix $(R,|\cdot|)$ a (not necessarily complete) nonarchimedean valued ring. Then, given $f,g\in R[x],$ we have that $|fg|=|f|\cdot|g|,$ where we defined $|f|$ and $|g|$ in the previous corollary.
\end{corollary}
\begin{proof}
	Fix
	\[f(x)=\sum_{k=0}^Ma_kx^k\qquad\text{and}\qquad g(x)=\sum_{\ell=0}^Nb_kx^k.\]
	Then we have that
	\[f(x)g(x)=\sum_{n=0}^{N+M}c_n x^n,\]
	where
	\[c_n=\sum_{k+\ell=n}a_kb_\ell,\]
	where we extend $a_k$ and $b_\ell$ with $0$s as is necessary. Now, we recall $|f|=\max_k\{|a_k|\}$ and $|g|=\max_\ell\{|b_\ell|\},$ and we note that
	\[|c_n|\le\max_{k,\ell\in\NN}\{|a_kb_\ell|\}=\max_{k\in\NN}\{|a_k|\}\cdot\max_{\ell\in\NN}\{|b_\ell|\}=|f|\cdot|g|,\]
	so $|fg|\le|f|\cdot|g|.$ For the other direction, we find the least $k_0$ and $\ell_0$ with $|a_{k_0}|=|f|$ and $|b_{\ell_0}|=|g|.$ Then we see that
	\[c_{k_0+\ell_0}=\sum_{k+\ell=k_0+\ell_0}a_kb_\ell.\]
	Now, the term $a_{k_0}b_{\ell_0}$ will have size $|f|\cdot|g|.$ Each other term $a_kb_\ell$ will either have $k<k_0$ or $\ell<\ell_0,$ implying that $|a_kb_\ell|<|a_{k_0}b_{\ell_0}|=|f|\cdot|g|.$ Thus, $|c_{k_0+\ell_0}|\ge|f|\cdot|g|,$ so $|fg|\ge|f|\cdot|g|,$ as needed.
\end{proof}
At this point, the corollaries are stacking up on each other.
\begin{corollary} \label{cor:corsquared}
	Fix $\alpha\in\overline K$ the algebraic closure of a complete nonarchimedean field $K.$ Then $\alpha$ is integral over $A$ if and only if $|\op N_K^{K(\alpha)}\alpha|\le1.$
\end{corollary}
\begin{proof}
	In one direction, take $\alpha$ integral so that $f\in A[x]$ has $f$ monic and $f(\alpha)=0.$ We also set $g$ to be the irreducible polynomial for $\alpha$ over $K.$ Then we see that $g\mid f$ in $K[x]$ and $|f|=1$ ($f$ is monic), with $|g|,|f/g|\ge1$ (they are monic), so Gauss's lemma bounds to $|g|=|f/g|=1.$ Thus,
	\[\left|\op N_K^{K(\alpha)}\alpha\right|=|\pm g(0)|\le1.\]
	For the other direction, still take $g$ the monic irreducible polynomial for $\alpha$ over $K.$ We see that $g(0)=\pm N_K^{K(\alpha)}(\alpha),$ so
	\[|g|=\max\{|1|,|g(0)|\}=1.\]
	Thus, $g\in A[x],$ so $\alpha$ is indeed integral over $A.$ This finishes.
\end{proof}