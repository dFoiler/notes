% !TEX root = ../notes.tex
















It is another day. The reading for Friday is chapter II, \S4.

Quickly, we note that two equivalent absolute values $|\cdot|_1$ and $|\cdot|_2$ on $K$ so that $|\cdot|_1=|\cdot|_2^s$ for some $s\in\RR^+,$ then the completions $\hat K_1$ and $\hat K_2$ are isomorphic in such a way that the $s$ is preserved. Essentially this holds because it will hold on the $K$ hiding inside $\hat K_1$ and $\hat K_2.$

\subsection{Completions of Archimedean Places}
We recall Ostrowski's theorem.
\begin{theorem}[Ostrowski]
	Fix $(K,|\cdot|)$ a complete archimedean field. Then $K$ is isomorphic to either $\RR$ or $\CC$ such that $|\cdot|$ becomes the normal absolute value on $\RR$ or $\CC.$
\end{theorem}
\begin{proof}
	This is in the book. We won't give it here even though it is cool.
\end{proof}
This gives the following corollaries.
\begin{corollary}
	We have that any (not necessarily complete) archimedean valued field $(K,|\cdot|)$ has completion $\hat K$ isomorphic to either $\RR$ or $\CC$ such that the absolute value on $K$ is equivalent to the pull-back from either $\RR$ or $\CC.$
\end{corollary}
\begin{corollary}
	If $(K,|\cdot|)$ is an archimedean valued field, then $(K,|\cdot|^s)$ is as well, for $0<s\le1.$
\end{corollary}
We note that we are taking $s\le1$ because $s>1$ might break the triangle inequality.

Now we specialize to the case where $K$ is a number field. It has (distinct) real embeddings $\rho_1,\ldots,\rho_r$ and (distinct) complex embeddings $\rho_1,\overline\rho_1,\ldots,\rho_s,\overline\rho_s,$ where $r+2s=n=[K;\QQ].$ We note that each embedding $K\into\CC$ will give rise to an archimedean place on $K$ by taking $x\in K$ to
\[|x|_\tau:=|\tau x|.\]
When $\tau$ is real, this places are distinct, but when $\tau$ is complex, we note that $|x|_\tau=|x|_{\overline\tau}$ implies that we only have $r+s$ total embeddings. We show this below.
\begin{proposition}
	Fix everything as above. The embeddings
	\[|\cdot|_{\rho_1},\quad\ldots,\quad|\cdot|_{\rho_r},\quad|\cdot|_{\sigma_1},\quad\ldots,\quad|\cdot|_{\sigma_s}\]
	are pairwise nonequivalent.
\end{proposition}
\begin{proof}
	Suppose that $\tau_1,\tau_2\in\op{Hom}(K,\CC)$ where $\tau_1\ne\tau_2$ and $\tau_1\ne\overline{\tau_2}.$ Then we want to show that $|\cdot|_{\tau_1}\ne|\cdot|_{\tau_2}.$ Well, fix $\alpha\in K$ some primitive element over $\QQ$ so that $\tau_1(\alpha)\ne\tau_2(\alpha)$ and $\tau_1(\alpha)\ne\overline{\tau_2(\alpha)}$ because having the same image on $\alpha$ would extend to all of $K=\QQ(\alpha).$

	But from this we claim that
	\[|\tau_1\alpha|\ne|\tau_2\alpha|\qquad\text{or}\qquad|\tau_1(\alpha+1)|\ne|\tau_2(\alpha+1)|.\]
	Indeed, if these were equal, then we can create the following triangles.
	\begin{center}
		\begin{asy}
			unitsize(2cm);
			draw((-1.5,0)--(1.5,0));
			draw((0,-1.5)--(0,1.5));
			dot("$-1$", (-1,0), S);
			pair alpha = dir(40);
			dot((0,0));
			draw(alpha--(0,0)--(-1,0)--cycle);
			dot("$\tau_1\alpha$", alpha, NE);
			dot("$\overline{\tau_1\alpha}$", dir(-40), SE);
			draw(dir(-40)--(0,0)--(-1,0)--cycle);
		\end{asy}
	\end{center}
	Namely, from these triangles we are able to force $\tau_2\alpha$ to be either $\tau_1\alpha$ or its conjugate by side-side-side congruence, where the side between $0$ and $1$ is fixed.

	So without loss of generality (possibly replacing $\alpha$ with $\alpha+1$), we may take $|\tau_1\alpha|<|\tau_2\alpha|.$ Now we choose $r\in\QQ$ such that $|\tau_1\alpha|<r<|\tau_2\alpha|$ so that
	\[\left|\frac{\alpha}r\right|_{\tau_1}<1\qquad\text{but}\qquad\left|\frac{\alpha}r\right|_{\tau_2}>1.\]
	It follows that $(\alpha/r)^\bullet$ will go to $0$ under $|\cdot|_{\tau_1}$ but not in $|\cdot|_{\tau_2},$ so their induced topologies are indeed different.
\end{proof}

\subsection{Completions of Nonarchimedean Valuations}
Fix $(K,|\cdot|)$ be some complete nonarchimedean valued field, and we fix $A$ its valuation ring. For our discussion, we need Hensel's lemma, which we take from the handout.
\begin{lemma}[Hensel]
	Fix $f\in A[x],$ and suppose that we have $\alpha_0\in A$ such that $|f(\alpha_0)|<|f'(\alpha_0)|^2.$ Then the sequence defined recursively by
	\[\alpha_{n+1}:=\alpha_n-\frac{f(\alpha_n)}{f'(\alpha_n)}\]
	for each $n\in\NN$ will converge to a root $\alpha$ of $f$ such that
	\[|\alpha-\alpha_0|\le\frac{|f(\alpha_0)|}{|f'(\alpha_0)|}<|f'(\alpha_0)|,\]
	and in fact $\alpha$ is the only such root with $|\alpha-\alpha_0|<|f'(\alpha_0)|.$
\end{lemma}
\begin{remark}
	We note that the right-hand side $\frac{|f(\alpha_0)|}{|f'(\alpha_0)|}$ is less than $1$ because it is less than $f'(\alpha_0),$ which has magnitude no more than $1$ because it is in $A.$
\end{remark}
\begin{proof}
	Quickly, we note that each $\alpha\in K$ which has $|\alpha-\alpha_0|<|f'(\alpha_0)|$ will actually have $|f'(\alpha)|=|f'(\alpha_0)|.$ Indeed, because $\alpha_0\in A,$ we see $f'(\alpha_0)\in A,$ so it has magnitude at most $1.$ And then by doing a Taylor expansion, we can find $\beta\in A$ such that
	\[f'(\alpha)=f'(\alpha_0)+\beta(\alpha-\alpha_0),\tag{$*$}\]
	where $\beta$ simply accumulates all of the higher-order Taylor terms; to be explicit, $(x-\alpha_0)$ will divide $f'(x)-f'(\alpha_0)$ formally because it vanishes as a polynomial at $x=\alpha_0,$ so we can find $g$ such that
	\[f'(x)-f'(\alpha_0)=g(x)(x-\alpha_0)\]
	and then plug in $\alpha.$ So using the triangle inequality on $(*)$ tells us that
	\[|f'(\alpha)|-|f'(\alpha_0)|\le|\alpha-\alpha_0|<|f'(\alpha_0)|,\]
	so it follows $|f'(\alpha)|=|f'(\alpha_0)|$ because all triangles are isosceles.

	We now attack the proof directly. Set
	\[c:=\frac{|f(\alpha_0)|}{|f'(\alpha_0)|^2},\]
	which has $c<1$ by hypothesis. We claim the following.
	\begin{lemma}
		Fix everything as above. For any $k\in\NN,$ we have the following.
		\begin{enumerate}[label=(\alph*)]
			\item $|\alpha_k-\alpha_0|\le\frac{|f(\alpha_0)|}{|f'(\alpha_0)|}<1,$
			\item $|f'(\alpha_k)|=|f'(\alpha_0)|,$ and
			\item $|f(\alpha_k)|\le c^{2^i}|f'(\alpha_0)|.$
		\end{enumerate}
	\end{lemma}
	\begin{proof}
		For $k=0,$ there is nothing to say. So we jump into the inductive step; take the above true for $k,$ and we show $k+1.$ We note that, by definition of $\alpha_{k+1},$
		\[|\alpha_{k+1}-\alpha_k|=\left|\frac{f(\alpha_k)}{f'(\alpha_k)}\right|\stackrel{\text{(c)}}\le c|f'(\alpha_0)|=\frac{|f(\alpha_0)|}{|f'(\alpha_0)|}<1,\]
		where we have used (c), where marked. This gives (a).

		For (b), we start by seeing
		\[|\alpha_{k+1}-\alpha_0|\le\stackrel{\text{(1)}}\le\frac{|f(\alpha_0)|}{|f'(\alpha_0)|}=c|f'(\alpha_0)|=c|f'(\alpha_0)|<|f'(\alpha_0)|,\]
		from which $|f'(\alpha_{k+1})|=|f'(\alpha_k)|$ is supposed to follow.

		Lastly, for (c), we use a Taylor expansion again to write
		\[f(\alpha_{k+1})=f(\alpha_k)+f'(\alpha_k)(\alpha_{k+1}-\alpha_k)+\gamma(\alpha_{k+1}-\alpha_k)^2\]
		for some $\gamma\in A$ which eats all of the higher-order terms. Plugging in the definition of $\alpha_{k+1}$ causes the first two terms to vanish, so we have that
		\[|f(\alpha_{k+1})|\le|\alpha_{k+1}-\alpha_k|^2=\left|\frac{f(\alpha_k)}{f'(\alpha_k)}\right|^2.\]
		By (2), the right-hand side is $\left|\frac{f(\alpha_k)}{f'(\alpha_0)}\right|^2,$ which is $\left(c^{2^k}|f'(\alpha_0)|\right)^2$ by the inductive hypothesis on (3), which collapses down to (3) after plugging in for the bound on $|f'(\alpha_0)|.$ This finishes.
		\qedhere
	\end{proof}
	We will finish this proof next time.
\end{proof}