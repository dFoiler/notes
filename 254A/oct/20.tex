\documentclass[../notes.tex]{subfiles}

\begin{document}

% !TEX root = ../notes.tex














The fun continues.

\subsection{Splitting in Cyclotomic Fields}
As usual, for each $n\in\ZZ^+,$ we set $\zeta_n\coloneqq e^{2\pi i/n}.$ We recall the following proposition.
\begin{proposition}
	Fix $n$ a positive integer and $p$ a prime, and write $n=p^\nu m$ (where $\nu=0$ is permitted) with $p\nmid m.$ Fixing $\mf q$ a prime of $K\coloneqq \QQ(\zeta_n)$ lying over $(p),$ and set $f_p$ the multiplicative order of $p$ in $(\ZZ/m\ZZ)^\times.$ Then
	\[f(\mf q/p)=f_p\qquad\text{and}\qquad e(\mf q/p)=\varphi\left(p^\nu\right).\]
\end{proposition}
\begin{proof}
	We do some cases.
	\begin{itemize}
		\item Suppose that $p\nmid n.$ We can use Dedekind--Kummer freely because our extension is monogenic, so we note that $x^n-1$ has $p$ distinct roots in $\overline{\FF_p},$ so it follows that our ramification index is $1$ everywhere. Fix $\mf q$ some prime over $\mf p.$

		Now, by homework, there is a unique $\sigma\in\op{Gal}(K/\QQ)$ such that for each $\alpha,$
		\[\sigma\alpha\equiv\alpha^p\pmod{\mf q},\]
		which is our Frobenius element. By taking $a=\zeta_n,$ we see that $\sigma$ is simply $\overline p\in(\ZZ/m\ZZ)^\times.$ Now, the inertial index $f(\mf q/\mf p)$ is equal to the degree $\mathcal O_K/\mf q$ over $\FF_p,$ so by looking at the size of the Galois group (here we are using that the Galois group is cyclic generated by this $\sigma$), it is the least $f$ for which
		\[\alpha^{p^f}\equiv\alpha\pmod{\mf q}\]
		for each $\alpha.$ But now pushing this through into our Galois group, we are asserting that $f$ is minimal for $\sigma^{f}=\id,$ which is equivalent to $f$ minimal for $\overline p^f=1.$ So indeed, $f(\mf q/\mf p)$ is the multiplicative order of $p\pmod m.$

		\item Let's do the general case now. Fix $\zeta'\coloneqq \zeta_{p^\nu}.$ We have the following diagram.
		% https://q.uiver.app/?q=WzAsNCxbMSwwLCJcXFFRKFxcemV0YV9uKSJdLFswLDEsIlxcUVEoXFx6ZXRhX20pIl0sWzIsMSwiXFxRUShcXHpldGEnKSJdLFsxLDIsIlxcUVEiXSxbMywyLCIiLDAseyJzdHlsZSI6eyJoZWFkIjp7Im5hbWUiOiJub25lIn19fV0sWzIsMCwiIiwwLHsic3R5bGUiOnsiaGVhZCI6eyJuYW1lIjoibm9uZSJ9fX1dLFsxLDAsIiIsMix7InN0eWxlIjp7ImhlYWQiOnsibmFtZSI6Im5vbmUifX19XSxbMywxLCIiLDIseyJzdHlsZSI6eyJoZWFkIjp7Im5hbWUiOiJub25lIn19fV1d
		\[\begin{tikzcd}
			& {\QQ(\zeta_n)} \\
			{\QQ(\zeta_m)} && {\QQ(\zeta')} \\
			& \QQ
			\arrow[no head, from=3-2, to=2-3]
			\arrow[no head, from=2-3, to=1-2]
			\arrow[no head, from=2-1, to=1-2]
			\arrow[no head, from=3-2, to=2-1]
		\end{tikzcd}\]
		Indeed, because $\gcd(m,p^\nu)=1,$ we have that $\QQ(\zeta_m)\cap\QQ(\zeta')=\QQ$ and $\QQ(\zeta_m)\QQ(\zeta')=\QQ(\zeta_n).$
		
		However, the main point is that the irreducible polynomial of $\zeta_m$ over $\QQ(\zeta')$ is the same as the irreducible polynomial over $\QQ,$ using our Galois extension lifting argument.

		Now, because $p\nmid m,$ primes over $p$ in $\QQ(\zeta_m)$ are unramified, essentially using the special case above. By the Galois lifting, the irreducible polynomial of $\QQ(\zeta_n)$ over $\QQ(\zeta')$ is the same as the irreducible polynomial for $\QQ(\zeta_m)$ over $\QQ,$ so this comes from Dedekind--Kummer.

		So checking our tower, we see that $e(\mf p/K)=e(\mf p/\QQ(\zeta')),$ which is $\varphi\left(p^\nu\right)$ because $(p)$ totally ramifies here. (For example, check the irreducible polynomial again.)

		It remains to compute the inertial degree $f(\mf p/K).$ The point is that the unique prime over $p$ in $\QQ(\zeta')$ is totally ramified over $\QQ(\zeta_n),$ so the inertial index is $1$ here. Thus, for each prime $\mf q$ lying over $\mf p$ in $K,$ we have
		\[e(\mf q/\QQ(\zeta_n))=[\QQ(\zeta'):\QQ]=[K:\QQ(\zeta_m)]=\varphi(m),\]
		so bounding with the fundamental identity\todo{} gives
		\[f(\mf q/\QQ)=f((\mf q\cap\QQ(\zeta_m))/\QQ)=\op{ord}_p(f)\]
		by the special case above. 
		\qedhere
	\end{itemize}
\end{proof}

\subsection{Totally Real Subfields}
This is not in the book, but it should be fun. We have the following definition.

Here is the set-up. Fix $n$ a positive integer, and fix $K\coloneqq \QQ(\zeta_n).$ Now, for each embedding $\tau:K\into\CC,$ there is an involution of $K$ by
\[\tau^{-1}\circ(z\mapsto\overline z)\circ\tau.\]
Because complex conjugation is an automorphism this involution is actually independent of $\tau.$ (The main point here is that we may embed $K$ into $\CC$ however we please so that the exact choice of $\zeta_n$ is $\QQ(\zeta_n).$) Explicitly, $\tau:\zeta_n\mapsto\zeta_n^a,$ but even still the involution takes $\zeta_n\mapsto\zeta_n^{-1}$ up in $\CC,$ which does not care for our exact choice of $K$ or $\tau.$

So complex conjugation is nicely canonical, and it corresponds to the element $-1$ in $(\ZZ/n\ZZ)^\times$ of the Galois group. For $n>2,$ there is a (canonical!) subgroup $\{\pm1\}$ of our Galois group of order $2,$ so taking the fixed field gives $E$ where $[K:E]=2$ and $[E:\QQ]=\varphi(n)/2.$ Here is the tower.
% https://q.uiver.app/?q=WzAsMyxbMCwwLCJLIl0sWzAsMSwiRSJdLFswLDIsIlxcUVEiXSxbMCwxLCJcXHtcXHBtMVxcfSIsMCx7InN0eWxlIjp7ImhlYWQiOnsibmFtZSI6Im5vbmUifX19XSxbMCwyLCIoXFxaWi9uXFxaWileXFx0aW1lcyIsMix7ImN1cnZlIjoyLCJzdHlsZSI6eyJoZWFkIjp7Im5hbWUiOiJub25lIn19fV1d
\[\begin{tikzcd}
	K \\
	E \\
	\QQ
	\arrow["{\{\pm1\}}", no head, from=1-1, to=2-1]
	\arrow["{(\ZZ/n\ZZ)^\times}"', curve={height=12pt}, no head, from=1-1, to=3-1]
\end{tikzcd}\]
So we have the following definition.
\begin{definition}[Totally real subfield]
	Fix everything as above. Then $E$ is called the \textit{totally real subfield} of $K.$
\end{definition}
We note that the unit groups of $\mathcal O_K^\times$ and $\mathcal O_E^\times$ have the same rank by checking Dirichlet's unit theorem, but they are not equal, for example, because $\zeta\in\mathcal O_K^\times\setminus\mathcal O_E^\times.$

\subsection{Valuations}
Let's return to the book. We have the following definition.
\begin{definition}[\texorpdfstring{$p$}{}-adics]
	We define
	\[\ZZ_p\coloneqq \limit\ZZ/p^\bullet\ZZ\]
	in the usual way as a limit of abelian groups where the maps $\ZZ/p^n\ZZ\onto\ZZ/p^m\ZZ$ be the standard projection for $n\ge m$ by $1\mapsto1.$
\end{definition}
\begin{remark}
	We note that $\ZZ_p$ is not $\ZZ/p\ZZ,$ nor is it $\ZZ_{(p)}.$
\end{remark}
It is not obvious, though it is true, that $\ZZ_p$ is a ring. As for why, we work in a little more generality.
\begin{prop}
	Fix $\{A_k\}_{k\in\NN}$ some sequence of groups (rings), and let $\varphi_k:A_k\to A_{k-1}$ be (ring) homomorphisms for each $k>0.$ Then
	\[\limit A_\bullet\subseteq\prod_{k\in\NN}A_k\]
	by
	\[\limit A_\bullet=\left\{\{a_k\}_{k\in\NN}\in\prod_{k\in\NN}A_k:\varphi(a_k)=a_{k-1}\text{ for each }k>0\right\}.\]
	And as usual, $\limit A_\bullet$ satisfies the universal property of the limit: for each $B$ with maps $\psi_\bullet:B\to A_\bullet$ commuting with the $\varphi_\bullet,$ there is a unique map $\psi:B\to\limit A_\bullet$ making the following diagram commute.
	% https://q.uiver.app/?q=WzAsNCxbMSwwLCJCIl0sWzEsMSwiXFxsaW1pdCBBX1xcYnVsbGV0Il0sWzAsMiwiQV9tIl0sWzIsMiwiQV9uIl0sWzMsMl0sWzEsMl0sWzEsM10sWzAsMiwiXFxwc2lfbSIsMix7ImN1cnZlIjoyfV0sWzAsMywiXFxwc2lfbiIsMCx7ImN1cnZlIjotMn1dLFswLDEsIlxccHNpIiwxLHsic3R5bGUiOnsiYm9keSI6eyJuYW1lIjoiZGFzaGVkIn19fV1d
	\[\begin{tikzcd}
		& B \\
		& {\limit A_\bullet} \\
		{A_m} && {A_n}
		\arrow[from=3-3, to=3-1]
		\arrow[from=2-2, to=3-1]
		\arrow[from=2-2, to=3-3]
		\arrow["{\psi_m}"', curve={height=12pt}, from=1-2, to=3-1]
		\arrow["{\psi_n}", curve={height=-12pt}, from=1-2, to=3-3]
		\arrow["\psi"{description}, dashed, from=1-2, to=2-2]
	\end{tikzcd}\]
\end{prop}
We won't show this here; this is more algebra than number theory.
\begin{example}
	We can realize elements of $\ZZ_p$ as infinite tuples
	\[\{a_k\}_{k\in\ZZ^+}\in\prod_{k\in\ZZ^+}\ZZ/p^k\ZZ\]
	such that $a_{k+1}\equiv a_k\pmod{p^k}$ for each $k\in\ZZ^+.$ For example $\{1234\}_{k\in\ZZ^+}$ is one such element, as is
	\[\left\{0,\frac{p+1}2,\frac{p^2+1}2,\frac{p^3+1}2,\ldots\right\}\]
	for $p$ odd. This last element corresponds to $1/2.$
\end{example}

\end{document}