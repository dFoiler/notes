% !TEX root = ../notes.tex














Today, $A$ will be a Dedekind domain except when we explicitly say otherwise, and $K$ is its fraction field.

\subsection{Localizing Away from Primes}
The book has the following definition.
\begin{definition}[Localization away from primes]
    Fix $X\subseteq\op{Spec}A$ a subset with finite complement. Then we define
    \[\mathcal O(X):=\left\{\frac fg:f,g\in A\text{ and }\nu_\mf p(g)=0\text{ for each }\mf p\in X\setminus\mf p\right\}.\]
\end{definition}
Note that we are allowing $(0)\in X,$ but we will actively pretend it does not exist, making ``for all $\mf p\in X$'' mean ``for all $\mf p\in X\setminus\{(0)\}.$''

We make a few short remarks.
\begin{remark}
    We note that $\mathcal O(X)\subseteq K$ is a subring of $K,$ and $A\subseteq\mathcal O(X)$ as well. In fact, we note that $\mathcal O(X)$ is a localization of
    \[S:=\{g:\nu_\mf p(g)=0\text{ for all }\mf p\in X\}\]
    so that $\mathcal O(X)=S^{-1}A.$ Then, because $A$ is Dedekind, we see that $\mathcal O(X)$ is Dedekind.
\end{remark}
\begin{warn}
    It is not in general true that $\op{Spec}\mathcal O(X)=X\cup\{(0)\}.$
\end{warn}
To fix the problem in the warning, we take the following definition.
\begin{definition}[Localizing, again]
    Fix $X\subseteq\op{Spec}A$ a subset with finite complement. Then we define
    \[\mathcal O(X)_{\text{Vojta}}:=\bigcap_{\mf p\in X}A_\mf p\subseteq K,\]
    which is $\{x\in K^\times:\nu_\mf p(x)\ge0\text{ for each }\mf p\in X\}\cup\{0\}.$
\end{definition}
We now have the following proposition to reconcile the two definitions.
\begin{proposition}
    Suppose that all (nonzero) primes $\mf p\in\op{Spec}A\setminus X$ are torsion in the class group. Then $\mathcal O(X)_{\text{Vojta}}=\mathcal O(X).$
\end{proposition}
\begin{proof}
    Professor Vojta hasn't worked this out yet, so it will be on the homework.
\end{proof}
\begin{remark}
    It is also true that $\mathcal O(X)_{\text{Vojta}}=\mathcal O(X)$ if $A$ is of finite type over a field using some results from algebraic geometry. Note that the Picard group need not be finite: elliptic curves over (say) the rationals may have infinite elements.
\end{remark}
Here is one reason we are bringing up localizing: it behaves nicely with our short exact sequence.
\begin{proposition}
    We have the following canonical exact sequence.
    \[1\to A^\times\to\mathcal O(X)^\times\to\bigoplus_{\mf p\notin X}(K^\times/A_\mf p^\times)\to\op{Cl}(A)\to\op{Cl}(\mathcal O(X))\to0.\]
\end{proposition}
\begin{proof}
    We check exactness one at a time.
    \begin{itemize}
        \item Exactness at $A^\times$ holds because $A\subseteq\mathcal O(X).$
        \item Exactness at $\mathcal O(X)^\times$: fix $\alpha\in\mathcal O(X)^\times$ with
        \[\alpha\in\ker\left(\mathcal O(X)^\times\to\bigoplus_{\mf p\notin X}K^\times/A_\mf p^\times\right).\]
        This is equivalent to $\alpha,\alpha^{-1}\in A_\mf p$ for each $\mf p\notin X,$\todo{wut} which is equivalent to $\alpha,\alpha^{-1}\in A_\mf p$ for each $\mf p\in\op{Spec}A$ because the primes in $X$ survive anyways, and lastly this is equivalent to $\alpha,\alpha^{-1}\in A$ by checking the $\nu_\mf p$ coordinates.
        \item Exactness as $\bigoplus_{\mf p\notin X}K^\times/A_\mf p^\times$: observe that $\nu_\mf p:K^\times/A_\mf p^\times\to\ZZ$ is an isomorphism for each of the $\mf p\notin X$ because $A_\mf p$ definitionally have $\nu_\mf p$ vanish. Now suppose
        \[n:=(n_\mf p)_{\mf p\notin X}\in\ker\left(\bigoplus_{\mf p\notin X}K^\times/A_\mf p^\times\to\op{Cl}(A)\right).\]
        Because we are taking $n$ to the product of its primes' ideal class, $n$ is in the kernel if and only if we can find some $\alpha\in K^\times$ such that
        \[\prod_{\mf p\notin X}\mf p^{n_\mf p}=\alpha A.\]
        Professor Vojta got confused, so we will come back to this proof later. \todo{}
        \item Exactness at $\op{Cl}(A)$: Suppose $c\in\op{Cl}(A)$ lives in $\ker\big(\op{Cl}(A)\to\op{Cl}(\mathcal O(X))\big).$ This is equivalent to $c$ being represented by a fractional ideal
        \[\prod_{\mf p\in\op{Spec}A\setminus\op{Spec}\mathcal O(X)}\mf p^{n_\mf p}\]
        by checking our primes individually. \todo{wut} So this is equivalent to $c$ living in the image of $\bigoplus_{\mf p\notin X}K^\times/A_\mf p^\times\to\op{Cl}(A).$
        \item Exactness at $\op{Cl}(\mathcal O(X))$: essentially, all primes $\mf q$ of $\mathcal O(X)$ can be pulled back to $\mf p\mathcal O(X)$ for some prime $\mf p\in A.$ It follows that the map $J_A\to J_{\mathcal O(X)}$ of ideals is surjective.
        \qedhere
    \end{itemize}
\end{proof}

\subsection{\texorpdfstring{$S$}{}-integers}
Let's see localization do something. Fix $K$ a number field, $A=\mathcal O_K$ its ring of integers, and let $S$ be a finite set of nonzero primes. We have the following definition.
\begin{definition}[\texorpdfstring{$S$}{}-integers]
    Fix $K$ a number field. Then we define the \textit{ring of $S$-integers of $K$} to be $\mathcal O_{K,S}:=\mathcal O(\op{Spec}\mathcal O_K\setminus S).$ THe group of $S$-units is $\mathcal O_{K,S}^\times.$
\end{definition}
Let's show a few things.
\begin{proposition}
    Fix everything as above. The class group $\op{Cl}(\mathcal O_{K,S})$ is finite, and $\mathcal O_{K,S}^\times$ has rank $\#S+r+s-1$ and torsion $\mu(K).$
\end{proposition}
\begin{proof}
    This follows by staring at the exact sequence from earlier. For example, $\op{Cl}(\mathcal O_{K,S})$ is finite because
    \[\op{Cl}A\to\op{Cl}(\mathcal O_{K,S})\]
    is a surjective map from a finite set. Further, we can compute the rank of $\mathcal O_{K,S}^\times$ as
    \[\op{rank}\mathcal O_K^\times+\op{rank}\bigoplus_{\mf p\in S}K^\times/(\mathcal O_K)_\mf p^\times\]
    because all rank from $\mathcal O_{K,S}^\times\to\bigoplus_{\mf p\in X}K^\times/A_\mf p^\times$ must vanish when going into $\op{CL}A$ because $\op{Cl}A$ is finite and hence fully torsion. So we can compute the rank of $\mathcal O_{K,S}^\times$ is $r+s-1+\#S.$
\end{proof}
So we have the following general idea.
\begin{idea}
    In general, $\mathcal O_{K,S}$ has many similar properties as $\mathcal O_K,$ but with fewer primes to worry about.
\end{idea}

\subsection{Extensions of Dedekind Domains}
We now take the $AKLB$ setup, where $A$ is a Dedekind ring, $K$ is its fraction field, $L/K$ is a finite extension, and $B$ is the integral closure of $A$ in $L.$
% https://q.uiver.app/?q=WzAsNCxbMCwxLCJBIl0sWzAsMCwiQiJdLFsxLDAsIkwiXSxbMSwxLCJLIl0sWzMsMiwiIiwwLHsic3R5bGUiOnsiaGVhZCI6eyJuYW1lIjoibm9uZSJ9fX1dLFswLDEsIiIsMCx7InN0eWxlIjp7ImJvZHkiOnsibmFtZSI6ImRhc2hlZCJ9LCJoZWFkIjp7Im5hbWUiOiJub25lIn19fV0sWzEsMiwiXFxzdWJzZXRlcSIsMSx7InN0eWxlIjp7ImJvZHkiOnsibmFtZSI6Im5vbmUifSwiaGVhZCI6eyJuYW1lIjoibm9uZSJ9fX1dLFswLDMsIlxcc3Vic2V0ZXEiLDEseyJzdHlsZSI6eyJib2R5Ijp7Im5hbWUiOiJub25lIn0sImhlYWQiOnsibmFtZSI6Im5vbmUifX19XV0=
\[\begin{tikzcd}
	B & L \\
	A & K
	\arrow[no head, from=2-2, to=1-2]
	\arrow[dashed, no head, from=2-1, to=1-1]
	\arrow["\subseteq"{description}, draw=none, from=1-1, to=1-2]
	\arrow["\subseteq"{description}, draw=none, from=2-1, to=2-2]
\end{tikzcd}\]
We will also assume that $B$ is finite over $A,$ which is true in the cases we care about (i.e., $A=\mathcal O_K$ for a number field $K$ or if $A$ is of finite type over a field or if $A$ is a localization of one of those). We note that $B$ is Dedekind by some theorem by Krull--Akizuki.

We also take the following definition.
\begin{definition}[Ramification, inertial index]
    Work in the $AKLB$ setup, and let $\mf p\in\op{Spec}A\setminus\{(0)\}.$ Then $\mf pB$ is a nonzero ideal of $B$ with prime factorization
    \[\mf pB=\prod_{k=1}^r\mf q_k^{e_k}.\]
    We say that $\mf q_k$ \textit{lies over} $\mf p.$ Then we have the following definitions.
    \begin{itemize}
        \item The \textit{ramification degree} $e(\mf q_k/\mf p)$ is $e_k$ in the above factorization.
        \item The \textit{inertial index} $f(\mf q_k/\mf p)$ is the dimension of the field extension $[B/\mf q_k:A/\mf p].$
    \end{itemize}
    We note that the claimed field extension exists because $\mf q_k\cap A\subseteq\mf p$ and some isomorphism theorem.
\end{definition}
There are some claims about to be proven (namely, that the factorization is nonempty) as well as that $\mf q_k\cap A=\mf p.$