% !TEX root = ../notes.tex














We do more on cyclotomic fields today. In today's class, we will fix $n$ a positive integer, and $\zeta=\zeta_n=e^{2\pi i/n}\in\CC$ (more generally, any primitive $n$th roof of unity will do).

\subsection{Talking \texorpdfstring{$\mathcal O_{\QQ(\zeta)}$}{}}
Recall that last time we showed the following.
\begin{proposition}
	Fix $n=\ell^\nu$ with $\ell$ prime and $\nu\in\ZZ^+.$ Setting $\lambda:=1-\zeta$ and $K=\QQ(\zeta),$ then we have the following.
	\begin{itemize}
		\item $\mathcal O_K=\ZZ[\zeta]$
		\item $\ell\mathcal O_K=(\lambda)^{\varphi(n)}$
		\item $(\lambda)$ is prime of absolute norm $\ell$
		\item $d_K=\ell^{\nu-1}(\nu\ell-\nu-1)$
	\end{itemize}
\end{proposition}
We now move to general positive integers.
\begin{proposition}
	For any positive integer $n,$ we set $L=\QQ(\zeta)$ so that $\mathcal O_L=\ZZ[\zeta]>$
\end{proposition}
\begin{proof}
	The book uses an intermediate lemma that two Galois extensions with rings of integers with coprime discriminant lets us compute the ring of integers of the composite field by direct multiplication. We will not do this here.

	We do strong induction on $n.$ The base case is if $n$ is a prime power, which we already took care of above. (If $n=1,$ this taking $L=\QQ$ where $\mathcal O_L=\ZZ.$) Otherwise, we take $n$ with at least two distinct prime factors. Suppose for the sake of contradiction that $\mathcal O_L\ne\ZZ[\zeta],$ and we take
	\[p\mid[\mathcal O_L:\ZZ[\zeta]]\]
	and $\ell$ a prime divisor of $n$ not equal to $p.$ Now factor $n=\ell^\nu m$ with $\ell\nmid m,$ and set $\zeta':=\zeta_{\ell^\nu}.$ This gives the following tower of fields.
	% https://q.uiver.app/?q=WzAsNCxbMSwwLCJMIl0sWzAsMSwiSzo9XFxRUShcXHpldGFfbSkiXSxbMiwxLCJcXFFRKFxcemV0YScpIl0sWzEsMiwiXFxRUSJdLFszLDEsIlxcdmFycGhpKG0pIiwwLHsic3R5bGUiOnsiaGVhZCI6eyJuYW1lIjoibm9uZSJ9fX1dLFsxLDAsIiIsMCx7InN0eWxlIjp7ImhlYWQiOnsibmFtZSI6Im5vbmUifX19XSxbMiwwLCIiLDIseyJzdHlsZSI6eyJoZWFkIjp7Im5hbWUiOiJub25lIn19fV0sWzMsMiwiXFx2YXJwaGkoXFxlbGxeXFxudSkiLDIseyJzdHlsZSI6eyJoZWFkIjp7Im5hbWUiOiJub25lIn19fV1d
	\[\begin{tikzcd}
		& L \\
		{K:=\QQ(\zeta_m)} && {\QQ(\zeta')} \\
		& \QQ
		\arrow["{\varphi(m)}", no head, from=3-2, to=2-1]
		\arrow[no head, from=2-1, to=1-2]
		\arrow[no head, from=2-3, to=1-2]
		\arrow["{\varphi(\ell^\nu)}"', no head, from=3-2, to=2-3]
	\end{tikzcd}\]
	We note that $L=K(\zeta')$ because we can write $\zeta=(\zeta')^r(\zeta_m)^s$ for some $r,s\in\ZZ$ because $\ell^\nu$ and $m$ are relatively prime, letting us use Bezout's lemma here. Additionally, we notice that
	\[[L:K]=\frac{\varphi(n)}{\varphi(m)}=\varphi\left(\ell^\nu\right)=[\QQ(\zeta'):\QQ],\]
	so the extension $L/K$ is a ``lifting'' of $\QQ(\zeta')/\QQ.$ In particular, this degree comparison forces that $\zeta'$ has the same irreducible polynomial over $K$ as over $\QQ.$

	Now, by the inductive hypothesis, take $\mathcal O_K=\ZZ[\zeta_m],$ and let $S=\ZZ\setminus(p).$ We have two parts to the rest of the proof.
	\begin{itemize}
		\item We quickly claim that $S^{-1}\mathcal O_L\ne S^{-1}\ZZ[\zeta].$ In other words, our inequality of rings is preserved under this localization. Well, we have the following diagram of abelian groups with exact row and column.
		% https://q.uiver.app/?q=WzAsNyxbMCwyLCIwIl0sWzEsMiwiXFxaWltcXHpldGFdIl0sWzIsMiwiXFxtYXRoY2FsIE9fTCJdLFszLDIsIlxcbWF0aGNhbCBPX0wvXFxaWltcXHpldGFdIl0sWzQsMiwiMCJdLFszLDAsIjAiXSxbMywxLCJcXFpaL3BcXFpaIl0sWzAsMV0sWzEsMl0sWzIsM10sWzMsNF0sWzUsNl0sWzYsM11d
		\[\begin{tikzcd}
			&&& 0 \\
			&&& {\ZZ/p\ZZ} \\
			0 & {\ZZ[\zeta]} & {\mathcal O_L} & {\mathcal O_L/\ZZ[\zeta]} & 0
			\arrow[from=3-1, to=3-2]
			\arrow[from=3-2, to=3-3]
			\arrow[from=3-3, to=3-4]
			\arrow[from=3-4, to=3-5]
			\arrow[from=1-4, to=2-4]
			\arrow[from=2-4, to=3-4]
		\end{tikzcd}\]
		Namely, $p\mid\#(\mathcal O_L/\ZZ[\zeta]),$ so there's an element of order $p$ in the group by Cauchy's theorem. Localizing by $S$ everywhere, we maintain exactness, giving the following diagram.
		% https://q.uiver.app/?q=WzAsNyxbMCwyLCIwIl0sWzEsMiwiU157LTF9XFxaWltcXHpldGFdIl0sWzIsMiwiU157LTF9XFxtYXRoY2FsIE9fTCJdLFszLDIsIlNeey0xfVxcbWF0aGNhbCBPX0wvXFxaWltcXHpldGFdIl0sWzQsMiwiMCJdLFszLDAsIjAiXSxbMywxLCJTXnstMX1cXFpaL3BcXFpaIl0sWzAsMV0sWzEsMl0sWzIsM10sWzMsNF0sWzUsNl0sWzYsM11d
		\[\begin{tikzcd}
			&&& 0 \\
			&&& {S^{-1}\ZZ/p\ZZ} \\
			0 & {S^{-1}\ZZ[\zeta]} & {S^{-1}\mathcal O_L} & {S^{-1}\mathcal O_L/\ZZ[\zeta]} & 0
			\arrow[from=3-1, to=3-2]
			\arrow[from=3-2, to=3-3]
			\arrow[from=3-3, to=3-4]
			\arrow[from=3-4, to=3-5]
			\arrow[from=1-4, to=2-4]
			\arrow[from=2-4, to=3-4]
		\end{tikzcd}\]
		Here, we do have that $S^{-1}\ZZ/p\ZZ\ne0$ because embedding $S$ into $\ZZ/p\ZZ$ must send it to nonzero elements,\todo{wut} so we are merely inverting out the nonzero elements of the field, so there are nonzero elements to mod out here. It follows that $S^{-1}\mathcal O_L\ne S^{-1}\ZZ[\zeta]$ by the exactness of the bottom row because the quotient object of the short exact sequence is nonzero.

		\item But now we claim the opposite, that $S^{-1}\mathcal O_L=S^{-1}\ZZ[\zeta].$ Indeed, we have that
		\[\mathcal O_K=\ZZ[\zeta_m]=\ZZ[\zeta^{n/m}],\]
		but now because $\zeta'$ has the same irreducible polynomial over $K$ as over $\QQ,$ we find that $\zeta'$ gives a power basis for $L$ over $K,$ and the discriminant is again a power of $\ell$ up to sign\todo{it is?}, but this is a unit in $S^{-1}\mathcal O_K$ and hence in $S^{-1}\ZZ$ (!).
		
		In particular, $S^{-1}\mathcal O_K$ only has finitely many maximal ideals (it only has the ones lying above $(p$) and Dedekind, it is a principal ideal domain, and in particular, we see that $S^{-1}\mathcal O_L$ has an integral basis over $S^{-1}\mathcal O_K,$\todo{why} which we call $\{\omega_k\}_{k=1}^d,$ where $d:=\varphi\left(\ell^\nu\right).$ Letting $M$ be the matrix expressing the $(\zeta')^\bullet$ in terms of the $\omega_\bullet,$ we see that
		\[d(1,\zeta',\ldots,(\zeta')^{d-1})=(\det M)^2d(\omega_\bullet).\]
		But now the left hand-side is a unit in $S^{-1}\mathcal O_K,$ and both factors are in $S^{-1}\mathcal O_K,$ we must have that both factors are themselves units, so in particular $\det M$ is a unit in $S^{-1}\mathcal O_K.$ Now it follows that
		\[S^{-1}\mathcal O_L=S^{-1}\mathcal O_K[\zeta']\]
		by change of basis with $M,$ and
		\[S^{-1}\mathcal O_K[\zeta']=S^{-1}\ZZ[\zeta]\]
		because $\ZZ[\zeta]=\ZZ[\zeta_m,\zeta']=\mathcal O_K[\zeta'].$ This finishes the proof.
		\qedhere
	\end{itemize}
\end{proof}

\subsection{Combining Cyclotomic Fields}
Here are some extra definitions.
\begin{definition}[Roots of unity]
	Fix $k$ a field with $n$ a positive integer. Then we set
	\[\mu_n(k):=\{x\in k:x^n=1\}.\]
	With no argument, we set $\mu_n=\mu_n(\overline k).$
\end{definition}
Observe that as long as the characteristic of $k$ does not divide $n,$ then $\mu_n$ will have $n$ elements by looking for a primitive root of unity; otherwise, it might have fewer.

We recall we have the following lemmas.
\begin{lemma}
	We have that $\QQ(\zeta_n)\QQ(\zeta_m)=\QQ(\zeta_{\lcm(m,n)}).$
\end{lemma}
\begin{proof}
	Indeed, $\zeta_n,\zeta_m\in\QQ(\zeta_{\lcm(m,n)})$ shows that $\QQ(\zeta_n)\QQ(\zeta_m)\subseteq\QQ(\zeta_{\lcm(m,n)})$; and in the other direction, note that $\frac{\lcm(n,m)}n$ and $\frac{\lcm(n,m)}m$ are relatively prime, so we can find integers $r$ and $s$ so that
	\[r\cdot\frac{\lcm(n,m)}n+s\cdot\frac{\lcm(n,m)}m=1,\]
	from which it follows $\zeta_{\lcm(m,n)}=\zeta_m^r\zeta_n^s.$
\end{proof}
\begin{lemma}
	We have that $\QQ(\zeta_n)\cap\QQ(\zeta_m)=\QQ(\zeta_{\gcd(m,n)}).$
\end{lemma}
\begin{proof}
	Certainly $\zeta_n,\zeta_m\in\QQ(\zeta_{\gcd(m,n)}),$ so $\QQ(\zeta_n)\cap\QQ(\zeta_m)\subseteq\QQ(\zeta_{\gcd(m,n)}).$ In the other direction, we use Galois theory. Fix $N:=\lcm(m,n)$ and $g:=\gcd(m,n).$ We have the following Galois diagram.
	% https://q.uiver.app/?q=WzAsNCxbMSwyLCJcXFFRIl0sWzAsMSwiXFxRUShcXHpldGFfbSkiXSxbMiwxLCJcXFFRKFxcemV0YV9uKSJdLFsxLDAsIlxcUVEoXFx6ZXRhX04pIl0sWzAsMSwiIiwwLHsic3R5bGUiOnsiaGVhZCI6eyJuYW1lIjoibm9uZSJ9fX1dLFsxLDMsIkgiLDAseyJzdHlsZSI6eyJoZWFkIjp7Im5hbWUiOiJub25lIn19fV0sWzIsMywiSyIsMix7InN0eWxlIjp7ImhlYWQiOnsibmFtZSI6Im5vbmUifX19XSxbMCwyLCIiLDIseyJzdHlsZSI6eyJoZWFkIjp7Im5hbWUiOiJub25lIn19fV0sWzAsMywiRyIsMSx7InN0eWxlIjp7ImhlYWQiOnsibmFtZSI6Im5vbmUifX19XV0=
	\[\begin{tikzcd}
		& {\QQ(\zeta_N)} \\
		{\QQ(\zeta_m)} && {\QQ(\zeta_n)} \\
		& \QQ
		\arrow[no head, from=3-2, to=2-1]
		\arrow["H", no head, from=2-1, to=1-2]
		\arrow["K"', no head, from=2-3, to=1-2]
		\arrow[no head, from=3-2, to=2-3]
		\arrow["G"{description}, no head, from=3-2, to=1-2]
	\end{tikzcd}\]
	In particular, we set $H$ and $K$ to be the corresponding subgroups of $G:=\op{Gal}(\QQ(\zeta_n)/\QQ)$ which fix $\QQ(\zeta_m)$ and $\QQ(\zeta_n)$ respectively. Namely, $G\cong(\ZZ/N\ZZ)^\times,$ and $H$ is the kernel of $(\ZZ/N\ZZ)^\times\to(\ZZ/m\ZZ)^\times$ by projection, or in other words
	\[H=\left\{a\in(\ZZ/N\ZZ)^\times:a\equiv1\pmod m\right\},\]
	and similar for $K$ as $\left\{a\in(\ZZ/N\ZZ)^\times:a\equiv1\pmod n\right\}.$ We would like to show that $\QQ(\zeta_{\gcd(m,n)})\subseteq\QQ(\zeta_n)\cap(\zeta_m),$ which the Galois correspondence says is the same as showing the reverse $HK\subseteq\op{Gal}(\QQ(\zeta_n)/\QQ(\zeta_g)).$

	Well, if we factor $N=\prod\ell^{\nu_\ell},$ then
	\[(\ZZ/N\ZZ)^\times\cong\prod(\ZZ/\ell^{\nu_\ell}\ZZ)^\times,\]
	and $H$ and $K$ will correspond to some product of these subgroups, as $HK$ will be. This finishes, where the details are left as exercise.\todo{wut}
\end{proof}

And we'll close off by working towards the following result.
\begin{proposition}
	Fix $n$ a positive integer and $p$ a prime, and write $n=p^\nu m$ (where $\nu=0$ is permitted) with $p\nmid m.$ Fixing $\mf q$ a prime of $K:=\QQ(\zeta_n)$ lying over $(p),$ and set $f_p$ the multiplicative order of $p$ in $(\ZZ/m\ZZ)^\times.$ Then
	\[f(\mf q/p)=f_p\qquad\text{and}\qquad e(\mf q/p)=\varphi\left(p^\nu\right).\]
\end{proposition}
\begin{proof}
	We'll save this proof for Wednesday.
\end{proof}
Here is a corollary of this result.
\begin{corollary}
	Fix $n$ a positive integer. Then odd primes $p$ ramify in $\QQ(\zeta_n)$ if and only if $p\mid n.$ Further, $(2)$ ramifies if and only if $4\mid n.$
\end{corollary}
\begin{proof}
	By the previous result, we are testing for $\varphi\left(p^\nu\right)\ge1.$ If $p$ is odd, this is equivalent to $\nu\ge1$ is equivalent to $p\mid n.$ If $p=2,$ then this is equivalent to $\nu\ge2$ is equivalent to $4\mid n.$
\end{proof}
Note the second part roughly comes from the fact that $\QQ(\zeta_m)=\QQ(\zeta_{2m})$ when $m$ is odd.