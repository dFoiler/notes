% !TEX root = ../notes.tex













Here we go.

\subsection{Quadratic Reciprocity Example}
Let's start with some applications of quadratic reciprocity.
\begin{example}
	We can check if $101$ is a square$\pmod{223}.$ Well, $101\equiv1\pmod4,$ so we have
	\[\left(\frac{101}{223}\right)=\left(\frac{223}{101}\right)=\left(\frac{21}{101}\right).\]
	Now, factoring $21=3\cdot7,$ we can compute
	\[\left(\frac{21}{101}\right)=\left(\frac3{101}\right)\left(\frac7{101}\right)=\left(\frac{101}3\right)\left(\frac{101}7\right)=\left(\frac23\right)\left(\frac37\right).\]
	because $101\equiv1\pmod4.$ We can compute by hand that $\left(\frac23\right)=-1$ and $\left(\frac37\right)=-1,$ so indeed, $\left(\frac{101}{223}\right)=1.$
\end{example}

\subsection{Decomposition Groups}
Fix $AKLB$ as usual, and for this subsection will assume that $L/K$ is a Galois extension with $G:=\op{Gal}(L/K).$ The point is that $G$ acts on $B$ fixing $A,$ which will give us some structure. Here, fix $\mf p$ a nonzero prime of $A,$ and we let
\[\mf p=\prod_{k=1}^r\mf q_k^{e_k}\]
our factorization of $\mf pB$ up in $B.$ And as usual, take $f_k:=f(\mf q_k/\mf p).$ The key claim is as follows.
\begin{lemma}
	Fix everything as above. Then $G$ acts transitively on the $\{\mf q_k\}_{k=1}^r.$
\end{lemma}
\begin{proof}
	That there is a $G$-action comes down to the fact that, for each $\sigma\in G,$ $\sigma\mf q_\bullet$ must be some prime ($\sigma$ is an automorphism, so this is not hard to check), and $\sigma\mf q_\bullet$ must live over $\mf p$ because $\sigma\mf q_\bullet\cap A=\sigma(\mf a_\bullet\cap A)=\sigma\mf p=\mf p.$

	Showing that the action is transitive requires some trickery. Suppose for the sake of contradiction that $\mf q_k$ and $\mf q_\ell$ have different orbits under the $G$-action. The trick is to find $\alpha\in B$ such that
	\[\alpha\equiv1\pmod{\sigma\mf q_k}\qquad\text{and}\qquad\alpha\equiv0\pmod{q_\ell}\]
	for each $\sigma\in G$; this exists by the Chinese remainder theorem. But now $\op N^L_K\alpha\notin\mf q_k\cap A=\mf p$ from the left while $\op N^L_K\alpha\notin\mf q_\ell\cap A=\mf p$ from the right. This is a contradiction.
\end{proof}
And here is the main result.
\begin{corollary}
	The $e_k$ and $f_k$ do not depend on $k.$
\end{corollary}
\begin{proof}
	Apply $\sigma$ to the prime factorization of $\mf p$ to get that the $e_k$ are independent of $k.$ To get that the $f_k$ are independent of $k,$ we note that $\mf q_k=\sigma\mf q_\ell$ has the isomorphism
	\[A/\mf q_k\to A/\mf q_\ell\]
	by applying $\sigma.$
\end{proof}
The above tells us that
\[n:=[L:K]=\sum_{k=1}^re(\mf q_k/\mf p)f(\mf q_k/\mf p)=ref,\]
where $e=e_k$ and $f=f_k$ for each $k.$ Namely, $\#G=ref.$

To keep better track of our data, we have the following definitions.
\begin{definition}[Decomposition group]
	Fix everything as above, and fix some $\mf q$ over $\mf p.$ Then we define the \textit{decomposition group} of $\mf q$ is
	\[G_\mf q=\{\sigma\in G:\sigma\mf q:=\mf q\},\]
	the stabilizer of $\mf q$ under the $G$-action.
\end{definition}
We have some small remarks.
\begin{itemize}
	\item We remark that $[G:G_\mf q]$ is the size of the orbit of $\mf q$ by group theory, which is exactly the number of primes $r.$ It follows $\#G_\mf q=\#G/r=ef$ from earlier.
	\item Note that if we place $\mf q$ with some $\sigma\mf q,$ then we have $G_{\sigma\mf q}=\sigma G_\mf q\sigma^{-1},$ again by group theory.
\end{itemize}
And Galois theory also provides us with a decomposition field.
\begin{definition}[Decomposition field]
	Fix everything as before. Then the \textit{decomposition field} of $\mf q$ is $Z_\mf q$ is the fixed field of $G_\mf q.$
\end{definition}
Here is the diagram.
% https://q.uiver.app/?q=WzAsMTAsWzIsMCwiTCJdLFsyLDIsIksiXSxbMywxLCJaX3EiXSxbMSwwLCJCIl0sWzEsMSwiQyJdLFsxLDIsIkEiXSxbMiwxXSxbMCwwLCJcXG1mIHEiXSxbMCwxLCJcXG1mIHFcXGNhcCBDIl0sWzAsMiwiXFxtZiBwIl0sWzEsMCwibiIsMix7InN0eWxlIjp7ImhlYWQiOnsibmFtZSI6Im5vbmUifX19XSxbMSwyLCJyIiwyLHsic3R5bGUiOnsiaGVhZCI6eyJuYW1lIjoibm9uZSJ9fX1dLFsyLDAsImVmIiwyLHsic3R5bGUiOnsiaGVhZCI6eyJuYW1lIjoibm9uZSJ9fX1dLFszLDAsIlxcc3Vic2V0ZXEiLDMseyJzdHlsZSI6eyJib2R5Ijp7Im5hbWUiOiJub25lIn0sImhlYWQiOnsibmFtZSI6Im5vbmUifX19XSxbNCw2LCJcXHN1YnNldGVxIiwzLHsic3R5bGUiOnsiYm9keSI6eyJuYW1lIjoibm9uZSJ9LCJoZWFkIjp7Im5hbWUiOiJub25lIn19fV0sWzUsMSwiXFxzdWJzZXRlcSIsMix7InN0eWxlIjp7ImJvZHkiOnsibmFtZSI6Im5vbmUifSwiaGVhZCI6eyJuYW1lIjoibm9uZSJ9fX1dLFs4LDksIiIsMix7InN0eWxlIjp7ImhlYWQiOnsibmFtZSI6Im5vbmUifX19XSxbNyw4LCIiLDIseyJzdHlsZSI6eyJoZWFkIjp7Im5hbWUiOiJub25lIn19fV0sWzMsNCwiIiwzLHsic3R5bGUiOnsiaGVhZCI6eyJuYW1lIjoibm9uZSJ9fX1dLFs0LDUsIiIsMyx7InN0eWxlIjp7ImhlYWQiOnsibmFtZSI6Im5vbmUifX19XV0=
\[\begin{tikzcd}
	{\mf q} & B & L \\
	{\mf q_Z:=\mf q\cap C} & C & {} & {Z_q} \\
	{\mf p} & A & K
	\arrow["n"', no head, from=3-3, to=1-3]
	\arrow["r"', no head, from=3-3, to=2-4]
	\arrow["ef"', no head, from=2-4, to=1-3]
	\arrow["\subseteq"{marking}, draw=none, from=1-2, to=1-3]
	\arrow["\subseteq"{marking}, draw=none, from=2-2, to=2-3]
	\arrow["\subseteq"{marking}, draw=none, from=3-2, to=3-3]
	\arrow[no head, from=2-1, to=3-1]
	\arrow[no head, from=1-1, to=2-1]
	\arrow[no head, from=1-2, to=2-2]
	\arrow[no head, from=2-2, to=3-2]
\end{tikzcd}\]
So what is $\mf q_Z$? Well, because $\op{Gal}(L/Z_\mf q)=G_\mf q,$ all the primes over $\mf q_Z,$ of which $\mf q$ is one, need to live in the orbit of $\mf q,$ which is $\{\mf q\}.$ So the only prime above $\mf q_Z$ is $\mf q,$ and our $r$ here is $1.$

We can also check that $\#G_\mf q=e(\mf q/\mf q_Z)f(\mf q/\mf q_Z)=ef$ because they are both the size of the decomposition group for $\mf q$ for $L/Z_q$ and $L/K$ respectively. But multiplicativity forces
\[e(\mf q/\mf q_Z)\le e\qquad\text{and}\qquad f(\mf q/\mf q_Z)\le e.\]
So we are forced into $e(\mf q_Z/\mf q)=e$ and $f(\mf q_Z/\mf q)=f,$ which forces $e(\mf q_Z/\mf p)=f(\mf q_Z/\mf p)=1.$ The point is the following.
\begin{proposition}
	Fix everything as above. Then we have the following.
	\begin{enumerate}
		\item $\mf q_Z$ is nonsplit in $B.$
		\item $e(\mf q/\mf q_Z)=e$ and $f(\mf q/\mf q_Z)=f.$
		\item $\mf q_Z$ has ramification index and inertial degree $1$ over $\mf p.$
	\end{enumerate}
\end{proposition}
\begin{proof}
	This follows from the above discussion.
\end{proof}

Here is another reason why the decomposition group is good: they help control our residue fields.
\begin{proposition}
	Fix evreything as above. We have that $B/\mf q$ is a normal field extension of $A/\mf p,$ and there is a canonical surjective group homomorphism
	\[G_\mf q\onto\op{Gal}\left(\frac{B/\mf q}{A/\mf p}\right).\]
\end{proposition}
Even though we are writig $\op{Gal},$ we are not actually requiring the field extension to be Galois. (Namely, we ought work in the maximal separable subextension of $B/\mf q$ over $A/\mf p.$)
\begin{proof}
	We note that $f(\mf q_Z/\mf p)=1$ forces $C/\mf q_Z\cong A/\mf p,$ so we are allowed to focus on $\mf q_Z$ instead, by taking $K\leftarrow Z_\mf q$ and $G\leftarrow G_\mf q.$ Namely, the corresponding residue fields we care about remain unchanged.

	Now, set $\lambda:=B/\mf q$ and $\kappa:=A/\mf p,$ for convenience. We have two claims.
	\begin{itemize}
		\item We show that $\lambda/\kappa$ is normal by showing that all irreducible polynomials of $\kappa[x]$ which have a root in $\lambda$ will fully factor in $\lambda.$
	
		Indeed, take some irreducible polynomial $\overline g\in\kappa[x]$ with a root of $\overline\theta\in B.$ Pulling back $\overline\theta$ to some $\theta\in B,$ we can give it minimal polynomial $f\in A[x].$ But now $f$ has a root in $L,$ and because $L/K$ is normal (!), $f$ will fully split up in $L[x],$ so $\overline f$ will also fully split down in $\lambda[x].$
	
		Now, because $\overline\theta$ is a root of $\overline f,$ the fact that $\overline g$ was irreducible will force $\overline g\mid\overline f,$ but $\overline f$ fully splits into linear factors, so $\overline g$ also fully splits into linear factors. This finishes the proof that $\lambda/\kappa$ is normal.
		\item We define our map $G_\mf q\to\op{Hom}_\kappa(\kappa(\overline\theta),\lambda)$ by
		\[\varphi:\sigma\mapsto(\overline\theta\mapsto\overline\theta),\]
		upon fixing some primitive element for the separable closure of $\lambda$ over $\kappa.$ (In particular, we will see that $\op{Hom}_\kappa(\kappa(\overline\theta),\lambda)=\op{Gal}(\lambda^{\op{sep}}/\kappa).$) We can check that this is well-defined: if $\overline{\alpha_1}=\overline{\alpha_2}$ for $\alpha_1,\alpha_2\in K(\theta)\cap B,$ then this is equivalent to $\alpha_1\equiv\alpha_2\pmod{\mf q},$ so $\alpha_1-\alpha_2\in\mf q,$ so
		\[\sigma(\alpha_1-\alpha_2)\in\sigma\mf q=\mf q\]
		because $\sigma\in G_\mf q,$ from which $\overline{\sigma(\alpha_1)}=\overline{\sigma(\alpha_2)}$ follows.

		We note that $\sigma\mapsto\overline\sigma$ is canonical because $\overline\sigma$ always takes $\overline\alpha$ to $\overline{\sigma(\alpha)},$ no matter what choice of $\overline\theta$ we made. So no choices were ``required'' for the proof.

		It remains to show $\sigma\mapsto\overline\sigma$ is surjective. So let $\overline\tau$ be any element of $\op{Gal}(\lambda/\kappa).$ Again using our primitive element $\overline\theta,$ we set $\overline g$ its minimal irreducible polynomial and plug into the normality argument above. Using that notation, $\overline\theta$ is the root of $\overline f,$ as is $\overline\tau(\overline\theta),$ and pulling everything back we have $\tau\in G_\mf q$ such that $\tau(\theta)$ has $\overline{\tau(\theta)}=\overline\tau(\overline\theta).$ But then see $\tau$ goes to $\overline\tau$ because they have the same behavior on $\overline\theta,$ finishing.
		\qedhere
	\end{itemize}
\end{proof}