\documentclass[../notes.tex]{subfiles}

\begin{document}

% !TEX root = ../notes.tex

















For today, our $AKLB$ setup will require $L/K$ to be a Galois extension, with $G\coloneqq \op{Gal}(L/K).$ We also fix $\mf p$ a nonzero prime of $A$ with $\mf q$ over $\mf p.$ As usual, we take $G_\mf q$ the decomposition group (stabilizer) of $\mf q$ under the $G$-action.

For brevity, we will also define $\lambda\coloneqq B/\mf q$ and $\kappa\coloneqq A/\mf p.$

\subsection{Inertia Groups}
Recall that we defined a surjective group homomorphism $\varphi:G_\mf q\onto\op{Gal}(\lambda/\kappa)$ such that
\[\varphi(\sigma)(\overline\alpha)=\overline{\sigma\alpha},\]
where $\sigma\in G_\mf q$ and $\alpha\in B.$

We want the above to be an isomorphism, so let's make it an isomorphism.
\begin{definition}[Inertia group]
	The group \textit{inertia group} $I_\mf q$ of $\mf q$ over $K$ is the kernel of our map $G_\mf q\onto\op{Gal}(\lambda/\kappa).$ In other words, $\sigma\in I_\mf q$ if and only if
	\[\sigma\theta\equiv\theta\pmod{\mf q}\]
	for each $\theta\in B.$ This also lets us define $T_\mf q,$ the \textit{inertial field}.
\end{definition}
We remark that we could also define
\[I_\mf q=\{\sigma\in G:\sigma\theta\equiv\theta\pmod{\mf q}\text{ for all }\theta\in B\}.\]
Indeed, the only thing to check here is that $\sigma\in I_\mf q$ implies $\sigma\in G_\mf q,$ which is true because $\sigma\in G_\mf q$ only means that $\theta\equiv0$ implies $\sigma\theta\equiv0\pmod{\mf q}.$

So the chain of subgroups $I_\mf q\subseteq G_\mf q\subseteq G$ gives the following diagram.
% https://q.uiver.app/?q=WzAsMTIsWzEsMywiQSJdLFsxLDIsIkNfWiJdLFsxLDEsIkNfVCJdLFsxLDAsIkIiXSxbMiwzLCJLIl0sWzIsMiwiWl9cXG1mIHEiXSxbMiwxLCJUX1xcbWYgcSJdLFsyLDAsIkwiXSxbMCwzLCJcXG1mIHAiXSxbMCwyLCJcXG1mIHFfWiJdLFswLDEsIlxcbWYgcV9UIl0sWzAsMCwiXFxtZiBxIl0sWzgsOSwiIiwwLHsic3R5bGUiOnsiaGVhZCI6eyJuYW1lIjoibm9uZSJ9fX1dLFs5LDEwLCIiLDAseyJzdHlsZSI6eyJoZWFkIjp7Im5hbWUiOiJub25lIn19fV0sWzEwLDExLCIiLDAseyJzdHlsZSI6eyJoZWFkIjp7Im5hbWUiOiJub25lIn19fV0sWzAsMSwiIiwwLHsic3R5bGUiOnsiaGVhZCI6eyJuYW1lIjoibm9uZSJ9fX1dLFsxLDIsIiIsMCx7InN0eWxlIjp7ImhlYWQiOnsibmFtZSI6Im5vbmUifX19XSxbMiwzLCIiLDAseyJzdHlsZSI6eyJoZWFkIjp7Im5hbWUiOiJub25lIn19fV0sWzQsNSwiIiwwLHsic3R5bGUiOnsiaGVhZCI6eyJuYW1lIjoibm9uZSJ9fX1dLFs1LDYsIiIsMCx7InN0eWxlIjp7ImhlYWQiOnsibmFtZSI6Im5vbmUifX19XSxbNiw3LCIiLDAseyJzdHlsZSI6eyJoZWFkIjp7Im5hbWUiOiJub25lIn19fV1d
\[\begin{tikzcd}
	{\mf q} & B & L \\
	{\mf q_T} & {C_T} & {T_\mf q} \\
	{\mf q_Z} & {C_Z} & {Z_\mf q} \\
	{\mf p} & A & K
	\arrow[no head, from=4-1, to=3-1]
	\arrow[no head, from=3-1, to=2-1]
	\arrow[no head, from=2-1, to=1-1]
	\arrow[no head, from=4-2, to=3-2]
	\arrow[no head, from=3-2, to=2-2]
	\arrow[no head, from=2-2, to=1-2]
	\arrow[no head, from=4-3, to=3-3]
	\arrow[no head, from=3-3, to=2-3]
	\arrow[no head, from=2-3, to=1-3]
\end{tikzcd}\]
So what do we expect based on the numbers?
\begin{proposition}
	We have that $T_\mf q$ is Galois over $Z_\mf q,$ and $\op{Gal}(T_\mf q/Z_\mf q)\cong\op{Gal}(\lambda/\kappa)$ and $\op{Gal}(L/T_\mf q)\cong I_\mf q.$
\end{proposition}
\begin{proof}
	This is essentially Galois theory. Namely, $I_\mf q$ is normal in $G_\mf q$ because it is a kernel, so $T_\mf q$ is normal over $Z_\mf q.$ Then $\op{Gal}(T_\mf q/Z_\mf q)\cong\op{Gal}(\lambda/\kappa)$ and $\op{Gal}(L/T_\mf q)\cong I_\mf q$ by tracking the quotient of the Galois groups through in the diagram.
\end{proof}
\begin{prop}[Split, inert, ramify]
	Suppose that $\lambda$ is separable over $\kappa$ (which is reasonable because most cases we care about will have $\kappa$ finite). Then $\#I_\mf q=[L:T_\mf q]=e$ and $[G_\mf q:I_\mf q]=[T_\mf q:Z_\mf q]=f.$ It follows
	\[e(\mf q/\mf q_T)=e(\mf q/\mf p)\qquad\text{and}\qquad e{(\mf q_T/\mf p)}=1,\]
	and 
	\[f(\mf q/\mf q_T)=1\qquad\text{and}\qquad f(\mf q_T/\mf p)=f(\mf q/\mf p).\]
\end{prop}
\begin{proof}
	We see $\#I_\mf q=[L:T_\mf q]$ and $[G_\mf q:I_\mf q]=[T_\mf q:Z_\mf q]$ by Galois theory. The other claims are left as an exercise.
\end{proof}
\begin{corollary}
	Still taking $\lambda$ separable over $\kappa,$ we have $\mf q$ unramified over $K$ if and only if $\#I_\mf q=1.$
\end{corollary}
\begin{proof}
	This follows by tracking our indices through in the above discussion.
\end{proof}
So here is our diagram.
% https://q.uiver.app/?q=WzAsMTIsWzEsMywiQSJdLFsxLDIsIkNfWiJdLFsxLDEsIkNfVCJdLFsxLDAsIkIiXSxbMiwzLCJLIl0sWzIsMiwiWl9cXG1mIHEiXSxbMiwxLCJUX1xcbWYgcSJdLFsyLDAsIkwiXSxbMCwzLCJcXG1mIHAiXSxbMCwyLCJcXG1mIHFfWiJdLFswLDEsIlxcbWYgcV9UIl0sWzAsMCwiXFxtZiBxIl0sWzgsOSwiZT0xLGY9MSIsMCx7InN0eWxlIjp7ImhlYWQiOnsibmFtZSI6Im5vbmUifX19XSxbOSwxMCwiZT0xLGY9ZiIsMCx7InN0eWxlIjp7ImhlYWQiOnsibmFtZSI6Im5vbmUifX19XSxbMTAsMTEsImU9ZSxmPTEiLDAseyJzdHlsZSI6eyJoZWFkIjp7Im5hbWUiOiJub25lIn19fV0sWzAsMSwiIiwwLHsic3R5bGUiOnsiaGVhZCI6eyJuYW1lIjoibm9uZSJ9fX1dLFsxLDIsIiIsMCx7InN0eWxlIjp7ImhlYWQiOnsibmFtZSI6Im5vbmUifX19XSxbMiwzLCIiLDAseyJzdHlsZSI6eyJoZWFkIjp7Im5hbWUiOiJub25lIn19fV0sWzQsNSwiIiwwLHsic3R5bGUiOnsiaGVhZCI6eyJuYW1lIjoibm9uZSJ9fX1dLFs1LDYsIiIsMCx7InN0eWxlIjp7ImhlYWQiOnsibmFtZSI6Im5vbmUifX19XSxbNiw3LCIiLDAseyJzdHlsZSI6eyJoZWFkIjp7Im5hbWUiOiJub25lIn19fV1d
\[\begin{tikzcd}
	{\mf q} & B & L \\
	{\mf q_T} & {C_T} & {T_\mf q} \\
	{\mf q_Z} & {C_Z} & {Z_\mf q} \\
	{\mf p} & A & K
	\arrow["{e=1,f=1}", no head, from=4-1, to=3-1]
	\arrow["{e=1,f=f}", no head, from=3-1, to=2-1]
	\arrow["{e=e,f=1}", no head, from=2-1, to=1-1]
	\arrow[no head, from=4-2, to=3-2]
	\arrow[no head, from=3-2, to=2-2]
	\arrow[no head, from=2-2, to=1-2]
	\arrow[no head, from=4-3, to=3-3]
	\arrow[no head, from=3-3, to=2-3]
	\arrow[no head, from=2-3, to=1-3]
\end{tikzcd}\]
We remark that we also have the following definition.
\begin{definition}[Higher ramification]
	There are \textit{higher ramifiaction} groups
	\[I_\mf q=I_{\mf q,0}\supseteq I_{\mf q,1}\supseteq I_{\mf q,2}.\]
\end{definition}
These are useful to keep track of the following.
\begin{definition}[Tame and wild ramification]
	We say that $\mf q$ is \textit{tamely ramified} if and only if $\op{char}\kappa=0$ or $\op{char}\kappa\nmid\#I_\mf q$ with $\lambda/\kappa$ separable. Then $\mf q$ is \textit{wildly ramified} otherwise.
\end{definition}
Namely, $I_{\mf q,1}$ helps capture wild ramification.

\subsection{Cyclotomic Fields}
In the discussion that follows, fix $n$ a positive integer, and let $\zeta$ be a primitive $n$th root of unity. Then we recall from Galois theory that
\[[\QQ(\zeta):\QQ]=\varphi(n)\qquad\text{and}\qquad\op{Gal}(\QQ(\zeta)/\QQ)\cong(\ZZ/n\ZZ)^\times,\]
where the isomorphism on the right is by taking $a\in(\ZZ/n\ZZ)^\times$ to $a\mapsto(\zeta\mapsto\zeta^n).$

Removing our picture from view, we fix $K=\QQ(\zeta).$ Our goal for now is to compute $\mathcal O_K.$ As a spoiler, we want it to be $\ZZ[\zeta].$ Let's start with prime powers.
\begin{lemma}
	Fix $n\coloneqq \ell^\nu$ a prime-power, where $\ell$ is prime and $\nu$ a positive integer. Fixing $K\coloneqq \QQ(\zeta)$ and $\lambda\coloneqq 1-\zeta,$ and we claim the following.
	\begin{enumerate}[label=(\alph*)]
		\item The ideal $(\lambda)$ is prime with absolute norm $\ell.$
		\item We have $\ell\mathcal O_K=(\lambda)^d,$ where $d=\varphi(n).$
		\item The power basis $\{\zeta^k\}_{k=0}^{d-1}$ of $K/\QQ$ has discriminant $\pm\ell^s,$ where $s=\ell^{\nu-1}(\nu\ell-\nu-1).$
	\end{enumerate}
\end{lemma}
\begin{proof}
	Let $f$ be the irreducible polynomial of $\zeta,$ which is
	\[f(x)=\frac{x^n-1}{x^{n/p}-1}=1+x^{n/p}+x^{2n/p}+\cdots+x^{n-n/p},\]
	which is
	\[f(x)=x^{\ell^{\nu-1}(\ell-1)}+x^{\ell^{\nu-1}(\ell-2)}+\cdots+x^{\ell^{\nu-1}}+1.\]
	Now, the irreducible polynomial of $\lambda$ is $(-1)^df(1-x),$ where the $(-1)^d$ adjusts for sign. In particular, we can compute that our degree is $d=\ell^{\nu-1}(\ell-1),$ and the constant term is $(-1)^d\ell.$

	It follows that the product of the conjugates of $\lambda$ is $(-1)^d\ell,$ so $(\lambda)$ has absolute norm $\ell,$ implying that $(\lambda)$ is a prime and lies over $\ell.$ Further, its conjugates are of the form, for some $a\in(\ZZ/n\ZZ)^\times,$
	\[\left(1-\zeta^a\right)=(1-\zeta)\left(1+\cdots+\zeta^{a-1}\right),\]
	which is divisible by $\lambda,$ but this forces $\left(1-\zeta^a\right)=(1-\zeta)$ because both factors need to have the same absolute norm (we can apply an automorphism to get one from the other). Thus, we see that
	\[(\ell)=\prod_{a\in(\ZZ/n\ZZ)^\times}(1-\zeta^a)=(\lambda)^d,\]
	so in fact $\ell$ fully splits, with inertial index $1,$ ramification index $d,$ and only a single prime above it.

	It remains to check (c). Well, we have to compute
	\[\prod_{1\le k<\ell\le d-1}\left(\zeta^k-\zeta^\ell\right)^2=\pm\prod_{\substack{k,\ell=1\\k\ne\ell}}^{d-1}\left(\zeta^k-\zeta^\ell\right).\]
	But now $\zeta^k-\zeta^\ell$ is a power of $\lambda$ (it might not equal to $\lambda$ because it might have $k-\ell$ not coprime to $\ell$). Then this will eventually fully collapse down to $\pm\ell^s$ by tracking all of this through, where $s$ as in the statement.
\end{proof}
And now we can show $\mathcal O_K=\ZZ[\zeta]$ in this case.
\begin{lemma}
	Again fix $n\coloneqq \ell^\nu$ a prime-power as before, with $K\coloneqq \QQ(\zeta).$ Then $\mathcal O_K=\ZZ[\zeta].$
\end{lemma}
\begin{proof}
	The index of $[\mathcal O_K:\ZZ[\zeta]]$ must be a power of $\ell$ because it needs to divide the discriminat of the power bases $\zeta^\bullet.$ By our computation of the inertial index, we see that
	\[\mathcal O_K/\lambda\mathcal O_K\cong\ZZ/\ell\ZZ,\]
	so in particular, the map $\ZZ\to\mathcal O_K/\lambda\mathcal O_K$ is surjective (with kernel $\ell\ZZ$). Then we have the computation
	\[\mathcal O_K=\ZZ+\lambda\mathcal O_K=\ZZ[\zeta]+\lambda\mathcal O_K,\]
	but now we can re-substitute (!) back into $\mathcal O_K$ as many times as we wish to see that $\mathcal O_K=\ZZ[\zeta]+\ell^\bullet\mathcal O_K$ for as high a power of $\ell^\bullet$ as we please. But by our disctiminat computation, $\ell^s\in\ZZ[\zeta],$ so at this point we conclude that $\mathcal O_K=\ZZ[\zeta].$
\end{proof}
Next time we will take $n$ arbitrary, using the prime-power case and some strong induction.

\end{document}