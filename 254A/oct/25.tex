\documentclass[../notes.tex]{subfiles}

\begin{document}

% !TEX root = ../notes.tex

















So we're still talking about localization.

\subsection{\texorpdfstring{$p$}{}-adic Absolute Value}
For $a\in\QQ^\times,$ recall that we define $\nu_p(a)$ equal to the exponent of the prime ideal $(p)$ in the (ideal!) prime factorization of the fractional ideal $(a).$ More concretely, if $a=p^m\cdot\frac bc$ with $p\nmid b,c,$ then $\nu_p(a)=m.$ With this in mind, we see that
\[\ZZ_{(p)}=\left\{a\in\QQ^\times:\nu_p(a)\ge0\right\}.\]
In particular,
\[\nu_p:\QQ^\times\to\ZZ\]
is a surjective homomorphism.

By definition, we will often let $\nu_p(0)=+\infty$ so that $\ZZ_{(p)}$ becomes a discrete valuation ring, where $\nu_p$ is our valuation. We have the following basic properties to check about being a valuation.
\begin{proposition} \label{prop:basicvps}
	Fix $p$ prime and $a,b\in\QQ.$ Then the following are true.
	\begin{enumerate}[label=(\alph*)]
		\item $\nu_p(a)=+\infty$ if and only if $a=0.$
		\item $\nu_p(ab)=\nu_p(a)+\nu_p(b).$
		\item $\nu_p(a+b)\ge\min\{\nu_p(a),\nu_p(b)\}.$
	\end{enumerate}
\end{proposition}
\begin{proof}
	Some of these follow from the above discussion; the rest are exercises.\todo{}
\end{proof}
\begin{remark}
	The above arguments work for a general prime $\mf p$ of a Dedekind ring.
\end{remark}
The point of \autoref{prop:basicvps} is that we may define the following.
\begin{defi}[\texorpdfstring{$p$}{}-adic absolute value]
	We define the \textit{$p$-adic absolute value} for $a\in\QQ$ by
	\[|a|_p:=p^{-\nu_p(a)}.\]
	We will also let $|\cdot|_\infty$ be the usual absolute value on $\QQ.$
\end{defi}
Using \autoref{prop:basicvps}, we have the following.
\begin{corollary}
	For $a,b\in\QQ$ and $p$ prime, the following are true.
	\begin{enumerate}[label=(\alph*)]
		\item $|a|_p\ge0$ with equality if and only if $a=0.$
		\item $|ab|_p=|a|_p\cdot|b|_p.$
		\item $|a+b|_p\le\max\{|a|_p,|b|_p\}.$
	\end{enumerate}
\end{corollary}
\begin{proof}
	These follow directly from \autoref{prop:basicvps}.
\end{proof}
We also have the following cute result.
\begin{proposition}
	All triangles in $\QQ$ with metric given by $|\cdot|_p$ are isosceles.
\end{proposition}
\begin{proof}
	Fix vertices $a,b,c\in\QQ$ so that we want to show that two of
	\[|a-b|_p,\qquad|b-c|_p,\qquad|c-a|_p\]
	are equal. Well, suppose for the sake of contradiction these are unequal, and set $x:=|a-b|_p$ to be the largest; then set $y:=b-c.$ We note also that $|\pm1|_p=1$ because $|\pm1|_p\cdot|\pm1|_p=|\pm1|_p,$ so it follows $|-b|_p<|a|_p.$ Thus,
	\[|x|_p=|(x+y)+(-y)|_p\le\max\{|x+y|_p,|-y|_p\}<|x|_p,\]
	which is our contradiction.
\end{proof}
Here is an important result on our valuations.
\begin{theorem}[Product formula on \texorpdfstring{$\QQ$}{}]
	Fix $\alpha\in\QQ^\times.$ Then $|\alpha|_p=1$ for all but finitely many $p\in\{\infty\}\cup\{\text{primes}\},$ and
	\[\prod_{p\in\{\infty\}\cup\{\text{primes}\}}|\alpha|_p.\]
\end{theorem}
\begin{proof}
	Letting $\alpha=\frac nm$ with $n,m\in\ZZ\setminus\{0\},$ we note that $p\nmid n,m$ implies that $\alpha=p^0\cdot\frac nm$ with $p\nmid n,m,$ so it follows that $\nu_p(\alpha)=0,$ so $|\alpha|_p=1.$

	To show the product formula, we note that multiplicativity of $|\cdot|_p$ implies that it suffices to show the formula for $\alpha\in\{-1\}\cup\{\text{primes}\}.$ Well, if $\alpha=-1,$ then $|\alpha|_p=1$ for each $p.$ Otherwise, if $\alpha=\ell$ is some prime, then we see that
	\[|\alpha|_p=\begin{cases}
		\ell & \text{ if }p=\infty, \\
		1/\ell & \text{ if }p=\ell, \\
		1 & \text{otherwise},
	\end{cases}\]
	where the last follows from writing $\ell=p^0\cdot\frac\ell1.$ Multiplying over the primes finishes the proof.
\end{proof}

\subsection{Function Field Analogy}
Fix $k$ an algebraically closed field, take $A:=k[x]$ the polynomial ring and $K:=k(x)$ its field of fractions. Then we recall from a course in commutative algebra that the maximal ideals of $A$ all look like
\[(x-\alpha)\]
for some $\alpha\in k,$ which follows from $k$ being algebraically closed.

Now, it is true that $A$ is a Dedekind ring. To try to tell the story above, we let $f\in k(x)^\times$ and define
\[\op{ord}_{(x-\alpha)}(f):=\text{order of vanishing of }f\text{ at }\alpha,\]
for each maximal ideal $(x-\alpha).$ We also need to deal with an ``infinite prime,'' which is
\[\op{ord}_\infty(f):=\deg f,\]
for $f\in k(x)^\times,$ where $\deg f:=\deg g-\deg h$ when $f=g/h$ for some $g,h\in k[x].$ Also, we will by convention take
\[\op{ord}_\mf p(0):=+\infty\]
for each $\mf p$ described above (either $\infty$ or $(x-\alpha)$).

For notational convenience, we define
\[M_K:=\{\infty\}\cup\{(x-\alpha):\alpha\in k\},\]
which is the set of places of $K.$ Then we have the following mirror of \autoref{prop:basicvps}.
\begin{proposition} \label{prop:basicvps2}
	We have the following, for $\mf p\in M_K$ and $f,g\in k(x).$
	\begin{enumerate}[label=(\alph*)]
		\item $\op{ord}_\mf p(f)\in\ZZ\cup\{\infty\},$ where $\op{ord}_\mf p(f)=\infty$ if and only if $f=0.$
		\item $\op{ord}_\mf p(fg)=\op{ord}_\mf p(f)+\op{ord}_\mf p(g).$
		\item $\op{ord}_\mf p(f+g)\ge\min\{\op{ord}_\mf p(f),\op{ord}_\mf p(g)\}.$
	\end{enumerate}
\end{proposition}
\begin{proof}
	Omitted as an exercise I guess.\todo{}
\end{proof}
Now, for some fixed $c>1,$ we define
\[|f|_\mf p:=c^{-\op{ord}_\mf p(f)}.\]
This turns into the following statement.
\begin{corollary}
	We have the following,f or $\mf p\in M_K$ and $f,g\in k(x).$
	\begin{enumerate}[label=(\alph*)]
		\item $|f|_p\ge0$ with equality if and only if $f=0.$
		\item $|fg|_\mf p=|f|_\mf p\cdot|g|_\mf p.$
		\item $|f+g|_\mf p\le\max\{|f|_\mf p,|g|_\mf p\}.$
	\end{enumerate}
\end{corollary}
\begin{proof}
	These follow from plugging in our definition of $|\cdot|_\mf p$ into \autoref{prop:basicvps2}.
\end{proof}
We seem to be getting this structure a lot. Let's abstract it.
\begin{definition}[Non-archimedean absolute value]
	A \textit{non-archimedean absolute value} on a ring $R$ is a function $|\cdot|:R\to\RR\cup\{\infty\}$ satisfying the following for $x,y\in R.$
	\begin{enumerate}
		\item $|x|\ge0$ with equality if and only if $x=0.$
		\item $|xy|=|x|\cdot|y|.$
		\item $|x+y|\le\max\{|x|,|y|\}.$
	\end{enumerate}
\end{definition}
\begin{remark}
	The ``non-archimedean'' adjective means that the size $|n|$ of various $n\in\ZZ$ is bounded above. This is not true for the standard valuation on $\QQ,$ for example.
\end{remark}
It turns out that we also have a product formula for $K.$
\begin{theorem}[Product formula for \texorpdfstring{$K$}{}]
	For each $f\in k(x)^\times,$ then $|f|_\mf p=1$ for each $\mf p\in M_K.$ Then
	\[\prod_{\mf p\in M_K}|f|_\mf p=1.\]
\end{theorem}
\begin{proof}
	This is similar to the proof in $\QQ$ and hence omitted.
\end{proof}
\begin{example}
	With $k=\CC,$ we have that $K$ is the field of all meromorphic functions on the Riemann sphere, $\CC\cup\{\infty\}.$ Also note that $\CC\cup\{\infty\}$ is in bijection with our elements of $M_K,$ where $\op{ord}_\mf p(f)=\op{ord}_c(f),$ where $\mf p=(x-c),$ where $c$ is allowed to be $\infty.$ Here the product formula turns into the statement
	\[\sum_{z\in\CC\cup\{\infty\}}\op{ord}_z(f)=0.\]
\end{example}

\subsection{Constructing Local Completions}
Fix $A$ a Dedekind ring with $K$ its field of fractions, as usual. Then we set $\mf p$ to be a nonzero prime $A$-ideal. For each $f\in K,$ we may define
\[\nu_\mf p(f)=\op{ord}_\mf p(f)\]
to be the exponent of $\mf p$ in the prime factorization of $(f).$ As usual, we have the following statement.
\begin{proposition} \label{prop:basicvps3}
	We have the following, for $\mf p$ a prime ideal and $f,g\in K.$
	\begin{enumerate}[label=(\alph*)]
		\item $\nu_\mf p(f)\in\ZZ\cup\{\infty\},$ where $\nu_\mf p(f)=\infty$ if and only if $f=0.$
		\item $\nu_\mf p(fg)=\nu_\mf p(f)+\nu_\mf p(g).$
		\item $\nu_\mf p(f+g)\ge\min\{\nu_\mf p(f),\nu_\mf p(g)\}.$
	\end{enumerate}
\end{proposition}
\begin{proof}
	Omitted, as usual.
\end{proof}
Then we can set $c>1$ some real number and define
\[|f|_\mf p:=c^{-\nu_\mf p(f)}.\]
So we get the following corollary, as usual.
\begin{corollary}
	We have that $|\cdot|_\mf p$ is a non-archimedean absolute value on $K.$
\end{corollary}
\begin{proof}
	This follows from running through the checks on \autoref{prop:basicvps3}.
\end{proof}
The notion of ``archimedean'' makes us think about distances, so we note that it is true that
\[d(f,g):=|f-g|_\mf p\]
does turn $K$ into a metric space.
\begin{warn}
	For the rest of this discussion, fix $\mf p$ and work with $|\cdot|:=|\cdot|_\mf p.$
\end{warn}
More generally, the conditions we need to get a metric space are as follows.
\begin{definition}[Absolute value]
	An \textit{absolute value} on a ring $R$ is a function $|\cdot|:R\to\RR$ that satisfy the following for $x,y\in R.$
	\begin{enumerate}[label=(\alph*)]
		\item $|x|\ge0$ with equality if and only if $x=0.$
		\item $|xy|=|x|\cdot|y|.$
		\item $|x+y|\le|x|+|y|.$
	\end{enumerate}
	Then we say that $|\cdot|$ is \textit{non-archimedean} if we satisfy $|x+y|\le\max\{|x|,|y|\}$ and \textit{archimedean} otherwise.
\end{definition}
It is not too hard to check that
\[d(x,y):=|x-y|\]
does induce a metric on $R$ as described above. The main point is that the triangle inequality for $d$ comes from $|x+y|\le|x|+|y|.$

We also have the following definition to talk about the underlying algebraic structure.
\begin{definition}[Valued rings, fields]
	A \textit{valued ring} consists of the data a ring $R$ as well as a valuation $|\cdot|$ on $R.$ A \textit{valued field} is a valued ring where the underlying ring $R$ is a field. Then a valued ring/field is a \textit{non-archimedean} valued ring/field if and only if the underlying valuation is non-archimedean.
\end{definition}
\begin{definition}[Complete valued rings, fields]
	A valued ring/field is \textit{complete} if and only if it is complete with respect to the distance metric induced by the underlying valuation.
\end{definition}
We close by picking up the following definition and theorem.
\begin{definition}[Morphisms of valued rings]
	A \textit{morphism} $\varphi:(R,|\cdot|)\to(R',|\cdot|')$ of valued rings is a ring homomorphism $R\to R'$ such that $|r|=|\varphi(r)|'$ for each $r\in R.$ This morphism $\varphi$ is an \textit{embedding} if $\varphi$ is injective.
\end{definition}
\begin{theorem}
	Every nontrivial valued ring $(R,|\cdot|)$ can be embedded into a complete valued ring $(R',|\cdot|'),$ where $R$ embeds as a dense subset. This embedding is unique up to unique isomorphism. In fact, if $R$ is a field, then $R'$ is also a field.
\end{theorem}
We'll prove this next time.

\end{document}