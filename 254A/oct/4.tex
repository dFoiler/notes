\documentclass[../notes.tex]{subfiles}

\begin{document}

% !TEX root = ../notes.tex














Today we continue with ramification theory.

\subsection{Review}
Last time we had some trouble showing exactness of
\[1\to A^\times\to\mathcal O(X)\to\bigoplus_{\mf p\notin X}\left(K^\times/A_\mf p^\times\right)\to\op{Cl}(A)\to\op{Cl}(\mathcal O(X))\to0\]
at $\bigoplus_{\mf p\notin X}\left(K^\times/A_\mf p^\times\right)$ and at $\op{Cl}(\mathcal O(X)).$
\begin{itemize}
	\item For exactness at $\bigoplus_{\mf p\notin X}\left(K^\times/A_\mf p^\times\right),$ there is an example making this not exact\footnote{There is an example for which $X\ne\op{Spec}A$ while $\mathcal O(X)=A.$}. However, it is exact in the situation given in the last homework problem for this week.
	\item Similarly, exactness at $\op{Cl}(\mathcal O(X))$ is on the homework.
\end{itemize}
We remark that the above sequence is also exact if $A$ is of finite type over a field, using some algebraic geometry.

\subsection{Fundamental Identity}
For the rest of today, we take the $AKLB$ setup, where $A$ is a Dedekind domain, $K$ its fraction field, $L/K$ a finite extension (not necessarily separable), and $B$ is the integral closure of $A$ in $L.$

We continue the definition of ramification degree and inertial index from last time.
\begin{defi} \label{def:raminert}
	Fix $\mf p$ a nonzero prime of $A.$ Then we may factor
	\[\mf pB=\prod_{k=1}^r\mf q_k^{e_k},\]
	where $\mf q_\bullet$ are primes in $B,$ and $e_k>0$ for each $k.$
	\begin{itemize}
		\item The \textit{ramification degree}  $e(\mf q_k/\mf p)=e_k$ as above.
		\item We say $\mf q_\bullet$ \textit{lies over} $\mf p.$
		\item Let $\kappa=A/\mf p$ and $\lambda_k=B/\mf q_k,$ which are fields because primes are maximal. Then $A\into B$ will induce a map $\kappa\into\lambda_k,$ which is injective because its a homomorphism of fields, and it is well-defined because the kernel is $A\cap\mf q_\bullet=\mf p$ as below.
		
		So $\kappa$ embeds into $\lambda_k,$ so we may define
		\[f(\mf q_k/\mf p):=[\lambda_k:\kappa].\]
	\end{itemize}
\end{defi}
We remark that $\mf q$ lies above $\mf p$ if and only if $\mf q\cap A=\mf p.$ Certainly each $\mf q_\bullet$ has $\mf q_\bullet\cap A=\mf p$\todo{}; but further, $\mf q\cap A=\mf p$ implies $\mf q\supseteq\mf pB$ implies $\mf q\mid\mf pB.$

With these, we have the following.
\begin{proposition}[Fundamental Identity]
	Fix everything as in \autoref{def:raminert}. Then
	\[\sum_{k=1}^re_kf_k=[L:K].\]
	For example, this implies $\mf pB\ne B$ because some primes must exist.
\end{proposition}
\begin{proof}
	This proceeds in steps.
	\begin{enumerate}
		\item We start by localizing. Fix $S:=A\setminus\mf p$ with $A_\mf p:=S^{-1}A$ and set $B_\mf p:=S^{-1}B$ as well. Because localization commutes with integral closure, we see that $B_\mf p$ is the integral closure of $A_\mf p$ in $L.$ Further, $A_\mf p$ is Dedekind (it's a Dedekind domain, localized) with fraction field $K,$ and $B_\mf p$ is finite over $A_\mf p$ because it was before as well.\todo{}

		So we have the usual picture, as follows.
		% https://q.uiver.app/?q=WzAsNCxbMCwxLCJBX3tcXG1hdGhmcmFrIHB9Il0sWzAsMCwiQl9cXG1hdGhmcmFrIHAiXSxbMSwwLCJLIl0sWzEsMSwiTCJdLFszLDIsIiIsMCx7InN0eWxlIjp7ImhlYWQiOnsibmFtZSI6Im5vbmUifX19XSxbMCwxLCIiLDAseyJzdHlsZSI6eyJib2R5Ijp7Im5hbWUiOiJkYXNoZWQifSwiaGVhZCI6eyJuYW1lIjoibm9uZSJ9fX1dLFswLDMsIlxcc3Vic2V0ZXEiLDEseyJzdHlsZSI6eyJib2R5Ijp7Im5hbWUiOiJub25lIn0sImhlYWQiOnsibmFtZSI6Im5vbmUifX19XSxbMSwyLCJcXHN1YnNldGVxIiwxLHsic3R5bGUiOnsiYm9keSI6eyJuYW1lIjoibm9uZSJ9LCJoZWFkIjp7Im5hbWUiOiJub25lIn19fV1d
		\[\begin{tikzcd}
			{B_\mathfrak p} & K \\
			{A_{\mathfrak p}} & L
			\arrow[no head, from=2-2, to=1-2]
			\arrow[dashed, no head, from=2-1, to=1-1]
			\arrow["\subseteq"{description}, draw=none, from=2-1, to=2-2]
			\arrow["\subseteq"{description}, draw=none, from=1-1, to=1-2]
		\end{tikzcd}\]
		Now, from our localization theory, we see that
		\[\mf p_\mf pB_\mf p=(S^{-1}\mf p)\underbrace{(S^{-1}B)}_{\text{trivial}}=\prod_k(S^{-1}\mf q_k)^{e_k}.\]
		Further, we see that $S^{-1}\mf q_k$ is a prime ideal of $B_\mf p$ for each $k$ (because $\mf q_\bullet\cap A=\mf p,$ we have $\mf q_\bullet\cap S=\emp$), and the $e_k$ match from before because we just factored $\mf p.$

		And because localization is an exact sequence, we see that
		\[0\to\mf q_k\to B\to\lambda_k\to0\]
		localizes down to
		\[0\to S^{-1}\mf q_k\to B_\mf p\to S^{-1}\lambda_k\to0\]
		is also exact. (Here, $S^{-1}\lambda_k$ has to mod out by $\mf q_k$ first, but $0\notin S$ makes this safe because $\mf q_k\cap S=\emp,$ so nothing gets killed on modding.) Similarly,
		\[0\to\mf p_\mf p\to A_\mf p\to S^{-1}\kappa\to0\]
		is exact. The point is that our residue fields are also unchanged\todo{}, so our inertial indices are also unchanged.

		\item It remains to work locally. Fix $(A,\mf p)$ a local field with residue field $\kappa,$ and fix $n:=[L:K].$ By the Chinese remainder theorem, we see that
		\[B/\mf pB\cong\prod_{k=1}^r(B/\mf q_k^{e_k}).\]
		We start with the right-hand side. We saw sometime ago that $\mf q_k^{e+1}/\mf q_k^e\cong B/\mf q_k$ (say, as groups), so we can compute indices
		\[\dim_\kappa(B/\mf q_k^{e_k})=e_k\dim_\kappa B/\mf q_k=e_k\dim_\kappa B/\lambda_k=e_kf_k.\]

		Now we study the left-hand side of our Chinese remainder theorem. We recall that $A$ being Dedekind with finitely many primes (!) makes it a principal ideal domain, which gives $B$ an integral basis over $A,$\todo{wut} implying $\dim_\kappa B/\mf pB=[L:K].$ Indeed,
		\[B/\mf pB\cong B\otimes_A(A/\mf p)\cong A^n\otimes_A(A/\mf p)\cong\kappa^{[L:K]}.\]
		This finishes.
		\qedhere
	\end{enumerate}
\end{proof}

\subsection{Dedekind--Kummer Theorem}
We take the following definition.
\begin{defi}[Conductor]
	Fix the $AKLB$ in the usual picture, and suppose $\theta$ is a primitive element of $L/K$ such that $\theta\in B$ (by clearing denominators). Then $A[\theta]$ is a full submodule of $L$ over $A$ contained in $B,$ and we define the \textit{conductor} of $A[\theta]$ to be
	\[\{\alpha\in B:\alpha B\subseteq A[\theta]\}.\]
\end{defi}
We note that the conductor is an $A[\theta]$-ideal: given $r_1,r_2\in A[\theta]$ and $\alpha_1,\alpha_2\in B,$ we have that
\[(r_1\alpha_1+r_2\alpha_2)B=r_1(\alpha_1B)+r_2(\alpha_2B)\subseteq A[\theta].\]
However, the main point to the conductor is that the conductor is $B$ if and only if $B=A[\theta].$ The point here is to try to figure out when we can find a power basis for our integral basis.

We also note that the conductor is nonzero because $B$ is finite over $A,$ meaning that we just have to check that $\alpha$ times each generator of $B$ is safely in $A[\theta],$ and we can construct something.

We have the following proposition.
\begin{proposition}[Dedekind--Kummer]
	Work in the $AKLB$ set-up. Fix $\theta$ as above with minimal polynomial $p(x)\in A[x],$ and let $\mf f$ be the conductor of $A[\theta].$

	Now, let $\mf p$ be a nonzero prime of $A,$ and fix $\overline p\in(A/\mf p)[x]$ to be the image of $p.$ Further, assume $\mf p$ is coprime to $\mf f$ so that $B=A[\theta].$ Now, if we factor into irreducible polynomials by
	\[\overline p=\prod_{k=1}^r\overline{p_k}^{e_k}\]
	in $(A/\mf p)[x],$ where $p_k\in A[x]$ is monic, then
	\[\mf q_k:=\mf pB+p_k(\theta)B\]
	describes the factorization of $\mf pB$ in $B,$ where $e(\mf q_k/\mf p)=e_k$ and $f(\mf q_k/\mf p)=\deg\overline{p_k}.$
\end{proposition}
\begin{proof}
	We again have two steps: deal with the local case, and then expand out. We start with the local case.
	\begin{enumerate}
		\item We start by taking $(A,\mf p,\kappa)$ to be a local ring. We claim
		\[B/\mf pB\cong\kappa[x]/\left(\overline p(x)\right)\]
		as $A$-modules; indeed, because $\mf p$ is coprime to $\mf f,$ we should have $\mf f=B$ ($\mf p$ is the only prime lying around)\todo{}, so
		\[B=A[\theta]\cong A[x]/\big(p(x)\big)\to\kappa[x]/\left(\overline p(x)\right)\]
		is some surjective chain of homomorphisms. The kernel here is $\mf pA[x]/(p(x)),$ which shows the claim.
		\qedhere
	\end{enumerate}
\end{proof}
We will continue the proof next class.

\end{document}