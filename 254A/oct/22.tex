\documentclass[../notes.tex]{subfiles}

\begin{document}

% !TEX root = ../notes.tex
















The fun, as they say, never stops.

\subsection{Talking \texorpdfstring{$\ZZ_p$}{}}
Fix $p$ a (nonzero) prime number. Last time we defined
\[\limit\ZZ/p^\bullet\ZZ\subseteq\prod_{k\in\NN}\ZZ/p^k\ZZ.\]
Namely, we can notate elements of $\ZZ_p$ by $(a_k)_{k\in\NN},$ where $a_{k+1}\equiv a_k\pmod{p^k}.$ Viewing $\ZZ_p$ as a product like this provides us with (surjective) morphisms
\[\pi_k:\ZZ_p\onto\ZZ/p^k\ZZ.\]
We can write this (purely formally) as
\[\alpha\coloneqq b_0+b_1p+b_2p^2+\cdots\]
where the $b_\bullet\in\ZZ.$ Here, we set $b_k\coloneqq \frac{a_{k+1}-a_k}p,$ where everything is viewed as an integer. (Nuekirch defines the $p$-adics using these formal power series.) We also note that, when $b_\bullet\in\left[0,p\right),$ this representation is unique.

Anyways, by the universal property, we note that the canonical surjections $\psi_k:\ZZ\onto\ZZ/p^k\ZZ$ (for each $k\in\NN$) provide a unique ring homomorphism $\ZZ\to\ZZ_p.$\footnote{We could note that $\ZZ$ is initial in ring here, but this does not give us the commutativity with the $\psi_k.$} It turns out that this map $\ZZ\to\ZZ_p$ is injective: any integer in the kernel must be $0\pmod{p^k}$ for each $k,$ which forces the kernel to be zero. As such, we have that the map $\ZZ\into\ZZ_p$ lets us view $\ZZ$ as a subring of $\ZZ_p.$

We also remark that any element $a\in\ZZ_p$ is a unit when $a\not\equiv0\pmod p$ because in this case we can invert $a\pmod{p^k}$ for each $k\in\NN,$ letting us construct out inverse. As such, all elements of $\ZZ\setminus p\ZZ$ are units in $\ZZ_p,$ which lets us extend $\ZZ\into\ZZ_p$ to the localization
\[\ZZ_{(p)}\to\ZZ_p\]
by the universal property. Recall that $\ZZ_{(p)}$ allows denominators outside of $(p).$

\subsection{Solutions in Modular Systems}
Why do we care about $\ZZ_p$? Here are some reasons.
\begin{itemize}
	\item $\ZZ_p$ and its specially its fraction field $\QQ_p$ let us do analysis, for example by Newton's method/Hensel's lemma.
	\item Similarly, investigating how $\ZZ_p$ interact, we can study solutions to polynomial equations in generic modular systems.
\end{itemize}
Indeed, the second point is codified by the following.
\begin{proposition}
	Fix $F$ some polynomial in $\ZZ[x_1,\ldots,x_k],$ and take $m$ and $n$ coprime positive integers. Then there is a canonical bijection between
	\[\left\{x\in(\ZZ/mn\ZZ)^k:F(x)=0\right\}\leftrightarrow\left\{x\in(\ZZ/m\ZZ)^k:F(x)=0\right\}\times\left\{x\in(\ZZ/n\ZZ)^\times:F(x)=0\right\}.\]
\end{proposition}
\begin{proof}
	Use the Chinese remainder theorem to construct the bijection.
\end{proof}
\begin{corollary}
	Fix $F$ some polynomial in $\ZZ[x_1,\ldots,x_k].$ Then $F$ has solutions in every modulus if and only if it has solutions$\pmod{p^\nu}$ for each $\nu\in\NN.$
\end{corollary}
\begin{proof}
	If it has solutions$\pmod{p^\nu}$ for each $\nu\in\NN,$ then we can use the Chinese remainder theorem to get each$\pmod m.$ The other direction is easier because$\pmod{p^\nu}$ is some modulus.
\end{proof}
\begin{prop}
	Fix $F$ some polynomial in $\ZZ[x_1,\ldots,x_k].$ We have that $F(x)=0$ has solutions in every modulus if and only if it has solutions in all $\ZZ_p^k.$
\end{prop}
\begin{proof}
	It suffices to show that $F$ has a solution$\pmod{p^\nu}$ for each $\nu$ if and only if it has a solution in $\ZZ_p.$ In one direction, if we have a solution in $(a_1,\ldots,a_k)\in\ZZ_p^k,$ then we can take
	\[(a_1\pmod{p^\nu},a_2\pmod{p^\nu},\ldots,a_k\pmod{p^\nu})\]
	for our solution$\pmod{p^\nu}.$

	The other direction is harder because generic solutions might not lift. For each $\nu\in\NN,$ set $\Sigma_\nu\subseteq(\ZZ/p^\nu\ZZ)^k$ be the set of solutions$\pmod{p^\nu}.$ By assumption, each $\Sigma_\bullet$ is nonempty. We also note that $\mu\ge\nu$ induces a map $\Sigma_\mu\to\Sigma_\nu$ by reduction. As such, we set
	\[\Sigma'_\nu=\bigcap_{\mu\ge\nu}\im(\sigma_\mu\to\Sigma_\nu)\]
	to be the set of elements in $\Sigma_\nu$ with lifts from each $\Sigma_\mu.$
	
	Now, the key trick is to see that $\Sigma'_\nu$ is nonempty because the $\im(\sigma_\mu\to\Sigma_\nu)$ is a decreasing sequence of nonempty subsets of $\left(\ZZ/p^\nu\ZZ\right)^k,$ so it must stabilize, and it cannot stabilize to a nonempty set because this would imply some $\Sigma_\mu$ is empty.

	We also note that the restricted map $\Sigma_\mu'\to\Sigma_\nu$ will always go into $\Sigma_\nu'$ because anything with an infinite lift from $\Sigma_\mu'$ will have an infinite lift. Further, the map $\Sigma_\mu'\to\Sigma_\nu'$ will be surjective because anything with an infinite lift from $\Sigma_\nu'$ will induce an infinite lift from its lift in $\Sigma_\mu'.$

	So now we have a system of surjective maps
	\[\cdots\onto\Sigma_3'\onto\Sigma_2'\onto\Sigma_1'\onto\Sigma_0',\]
	and we can choose a commuting element from $\limit\Sigma_\bullet',$ which is the needed solution in $\ZZ_p^k.$
\end{proof}
\begin{remark}
	We can also extend the above logic to work with finite systems of polynomial equations, at the cost of a minor headache.
\end{remark}

\subsection{Basic Properties of \texorpdfstring{$\ZZ_p$}{}}
Let's run through some basic properties.
\begin{proposition}
	We have that $\ZZ_p$ is an integral domain.
\end{proposition}
\begin{proof}
	By our construction of $\ZZ_p$ as a limit object, we know that it is a ring with identity; and we know that it is nonzero because, say, it surjects onto $\ZZ/p\ZZ.$

	So, of course, the hard part is showing that $\ZZ_p$ has no nontrivial zero-divisors. For this, we quickly recall that we have a valuation $\nu_p:\QQ\to\ZZ\cup\{\infty\}$ given by
	\[\nu_p(q)=\max\left\{a:p^a\mid q\right\}.\]
	(We could also state this in terms of ideal factorizations, but we won't.) Now fix some $\alpha\in\ZZ_p$ with
	\[\alpha=(a_0,a_1,\ldots),\]
	and $a_\bullet\in\ZZ$ for each $a_\bullet.$ If we force $a_\bullet\ne0$ for each $a_\bullet$ (say, set it to $p^\bullet$ when necessary), then we see that $\alpha=0$ if and only if the sequence $\nu_p(a_\bullet)$ is unbounded: if $\alpha$ is nonzero, then eventually one of the $a_k$ will not be divisible by $p^k,$ so each subsequent $a_\bullet$ will remain not divisible by $p^k$; conversely, if $\alpha=0,$ then $\nu_p(a_k)\ge k$ for each $k,$ giving unboundedness.

	We now attack zero-divisors directly. Suppose $\alpha\beta=0$ for $\alpha=(a_0,a_1,\ldots)\in\ZZ_p$ and $\beta=(b_0,b_1,\ldots)\in\ZZ_p,$ with $a_\bullet$ and $b_\bullet$ notated as above. Now, if $\alpha\ne0$ and $\beta\ne0,$ then $\nu_p(a_\bullet)$ and $\nu_p(\bullet)$ are bounded, so
	\[\nu_p(a_\bullet b_\bullet)=\nu_p(a_\bullet)+\nu_p(b_\bullet)\]
	is also bounded, so $\alpha\beta=(a_0b_0,a_1b_1,\ldots)$ is nonzero.
\end{proof}
\begin{proposition}
	We have that $\ZZ_p^\times=\pi_1^{-1}(\FF_p^\times).$
\end{proposition}
\begin{proof}
	In one direction, the fact that $\pi_1$ is a ring homomorphism implies $\pi_1\left(\ZZ_p^\times\right)\subseteq\FF_p^\times.$

	In the other direction, suppose $\alpha=(a_0,a_1,\ldots)$ is in $\pi_1^{-1}\left(\FF_p^\times\right)$ so that $p\nmid a_1$ and $p\nmid a_k$ for each $k\ge1.$ Thus, for each $k,$ we may find $a_k^{-1}\in\ZZ/p^k\ZZ,$ and these inverses will commute with each other to give
	\[\beta=\left(a_0^{-1},a_1^{-1},\ldots\right)\in\ZZ_p.\]
	Then we can compute that $\alpha\beta=(1,1,\ldots)=1\in\ZZ_p,$ so $\alpha\in\ZZ_p^\times.$
\end{proof}
\begin{proposition}
	We have that $\ZZ_p$ is a discrete valuation ring.
\end{proposition}
\begin{proof}
	Fix $\alpha\in\ZZ_p$ with $\alpha\ne0.$ Then we claim $\alpha=\left(p^\bullet\right)$ for some $p^\bullet.$ Indeed, write $\alpha=(a_0,a_1,\ldots),$ and we see that $\nu_p(a_\bullet)$ is a bounded sequence. Let $M$ be some bound, from which it follows $k,\ell\ge M$ has
	\[\nu_p(a_k)=\nu_p(\ell)\]
	because here $\nu_p(a_k)$ can only be modified by a power of $p^k$ and therefore cannot change, lest it jump above $M.$ In particular, our sequence eventually stabilizes, so we set $m$ to this limit so that
	\[p^{\min\{m,n\}}\mid a_n\]
	for each $n\in\NN,$ even for small $n$ because there we must have $0\pmod{p^n}.$ Then we can set
	\[\beta=(a_m/p^m,a_{m+1}/p^m,\ldots)\in\ZZ_p,\]
	with $p^m\beta=\alpha.$ We also note that $p\nmid a_m/p^m$ lest we have higher powers later\todo{eh}, so $\beta$ is in fact a unit, so it follows $(\alpha)=\left(p^m\right),$ which finishes this claim.

	With this in mind, we define $\nu_p(\alpha)$ equal to the integer such that $(\alpha)=\left(p^{\nu_p(\alpha)}\right).$ Now, to get a discrete valuation ring, we note that all nonzero ideals $I,$ set $m$ to the minimum of $\nu_p(\alpha)$ for each $\alpha\in I,$ and then we have that
	\[I=\bigcup_{\alpha\in I}(\alpha)=\bigcup_{\alpha\in I}\left(p^{\nu_p(\alpha)}\right)=\left(p^m\right),\]
	so indeed, all ideals of $\ZZ_p$ are principal. We also note that $\ZZ_p^\times=\pi_1^{-1}(\FF_p^\times)=\ZZ_p\setminus p\ZZ_p$ shows that $\ZZ_p$ is local. We also note $\ZZ_p$ is not a field because $p\ZZ_p$ are not units. Thus, we may conclude that $\ZZ_p$ is a discrete valuation ring.
\end{proof}

\end{document}