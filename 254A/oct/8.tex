\documentclass[../notes.tex]{subfiles}

\begin{document}

% !TEX root = ../notes.tex














\subsection{Some Examples}
Let's do some computations. It'll be fun.
\begin{example}
	Take $d$ a squarefree integer not equal to $1,$ and set $K=\QQ(\sqrt d).$ If for primes $p\nmid d,$ we set $\mathcal O_K=\ZZ[\theta],$ where $\theta:=\sqrt d.$ It follows from our computation of $\mathcal O_K$ that
	\[\mf f\mid(2),\]
	so we can factor $(p)$ using Dedekind--Kummer. Explicitly, $(p)$ remains prime if and only if $x^2-d$ remains irreducible if and only if $x^2\equiv d\pmod p$ has a solution.
\end{example}
And we can actually do class group computations now!
\begin{proposition}
	We compute the class group of $K:=\QQ(\sqrt{10}).$
\end{proposition}
\begin{proof}
	Here $d_K=40,n=r=2,s=0,$ so our Minkowski bound is
	\[\frac{n!}{n^n}\left(\frac4\pi\right)^s\sqrt{d_K}=\frac24\sqrt{40}=\sqrt{10}<4.\]
	So every ideal class has a representative with an ideal of norm less than $4.$ This leaves available norms of $2$ or $3,$ which correspond to primes lying over $(2)$ or $(3)$ in $\ZZ.$

	Well, note that with $\theta:=\sqrt{10},$ our conductor $\mf f=(1),$ so we can use Dedekind--Kummer somewhat freely.
	\begin{itemize}
		\item We have $x^2-10\equiv x\cdot x\pmod2,$ so $(2)=(2,\sqrt{10})^2.$ So we see that $(2)$ ramifies.\footnote{This is wild.} Define $\mf p:=(2,\sqrt{10}).$
		\item We have $x^2-10\equiv(x-1)(x+1)\pmod3,$ so $(3)=(3,\sqrt{10}-1)(3,\sqrt{10}+1).$ So we see that $(3)$ splits. Define $\mf q:=(3,\sqrt{10}+1)$ and $\overline{\mf q}:=(3,\sqrt{10}-1).$
		\item For fun, we can also check that $x^2-10$ has no roots$\pmod7,$ so $(7)$ is inert.
	\end{itemize}
	So every ideal class has a representative among $\{\mf p,\mf q,\overline{\mf q}\}.$ We have the following observations.
	\begin{enumerate}[label=(\roman*)]
		\item However, $[\mf p]^2=[(2)]=[(1)].$ Further, $[\mf q][\overline{\mf q}]=[(3)]=[(1)],$ so $\mf q=\overline{\mf q}^{-1}.$

		\item Additionally, we can check that $\mf p$ is not principal. Indeed, if $\mf p=(\alpha),$ then $\op N_\QQ^K(\alpha)=2,$ so $\alpha=a+b\sqrt{10}$ forces
		\[a^2-10b^2=\pm2,\]
		which is not possible by checking$\pmod5.$ Thus, $[\mf p]$ is an element of order $2$ in $\op{Cl}_K.$
	
		\item We can check in a similar way that $\mf q,\mf q\ne[(1)]$ because the argument above would ask for
		\[a^2-10b^2=\pm3,\]
		which still has no solutions$\pmod5.$
	
		\item To check if $\mf p\mf q$ is principal, the argument above requires
		\[a^2-10b^2=\pm4,\]
		which has solutions $a\in\{\pm4\}$ and $b\in\{\pm1\}.$ But we can compute
		\[\mf p\mf q=(2,\sqrt{10}),(3,1+\sqrt{10})=(6,2+2\sqrt{10},3\sqrt{10},10+\sqrt{10}),\]
		which contains $4+\sqrt{10},$ so a norm argument forces $\mf p\mf q=(4+\sqrt{10}).$ Thus, $[\mf q]=[\mf p^{-1}]=[\mf p],$ and so it follows $[\overline{\mf q}]=[\mf p].$
	\end{enumerate}
	So we see that $\op{Cl}_K=\{[(1)],[\mf p]\}\cong\ZZ/2\ZZ.$
\end{proof}

\subsection{Describing Factorization}
We have the following definition.
\begin{definition}[Factoriation types]
	Fix $A,K,L,B$ as in the $AKLB$ set-up. Fix $\mf p$ a nonzero prime of $A,$ and factor $\mf p$ by
	\[\mf p=\prod_{k=1}^r\mf q_k^{e_k}.\]
	Then we have the following.
	\begin{itemize}
		\item $\mf p$ is \textit{inert} if and only if $r=e_1=1.$ (In particular, $f_1=[L:K].$)
		\item $\mf p$ is \textit{unramified} if and only if $e_\bullet=1$ always and $B/\mf q_k$ is separable over $A/\mf p.$ (In global fields, this extra condition does not matter.)
		\item $\mf p$ is \textit{ramified} if and only if is is not unramified.
		\item $\mf p$ is \textit{totally ramified} if and only if $r=f_1=1.$ (In particular, $e_1=[L:K].$)
		\item $\mf p$ is \textit{split completely} if and only if $r=[L:K].$ (In particular, $e_k=f_k=1$ for each $k.$)
	\end{itemize}
\end{definition}
\begin{example}
	Back in $\QQ(\sqrt{10}),$ we saw $(2)$ was totally ramified, $(3)$ was totally split, and $(7)$ was inert.
\end{example}
We note that if $A=\ZZ$ with $K=\QQ$ so that $L$ is a number field, then for all $\mf p$ lying over a nonzero prime $(p)\in\op{Spec}\ZZ,$ then
\[\op N\mf q=[A/\mf q:\FF_p]=p^{f(\mf q/(p))},\]
so norm computations are fairly nice.

We also have the following fact.
\begin{proposition}
	Consider the chain of extensions as follows, where $\mf Q$ lies above $\mf q$ lies above $\mf p.$
	% https://q.uiver.app/?q=WzAsOSxbMCwyLCJcXG1mIHAiXSxbMCwxLCJcXG1mIHEiXSxbMCwwLCJcXG1mIFEiXSxbMSwwLCJDIl0sWzEsMSwiQiJdLFsxLDIsIkEiXSxbMiwyLCJLIl0sWzIsMSwiTCJdLFsyLDAsIk0iXSxbMiwzLCJcXHN1YnNldGVxIiwxLHsic3R5bGUiOnsiYm9keSI6eyJuYW1lIjoibm9uZSJ9LCJoZWFkIjp7Im5hbWUiOiJub25lIn19fV0sWzEsNCwiXFxzdWJzZXRlcSIsMSx7InN0eWxlIjp7ImJvZHkiOnsibmFtZSI6Im5vbmUifSwiaGVhZCI6eyJuYW1lIjoibm9uZSJ9fX1dLFswLDUsIlxcc3Vic2V0ZXEiLDEseyJzdHlsZSI6eyJib2R5Ijp7Im5hbWUiOiJub25lIn0sImhlYWQiOnsibmFtZSI6Im5vbmUifX19XSxbMyw4LCJcXHN1YnNldGVxIiwxLHsic3R5bGUiOnsiYm9keSI6eyJuYW1lIjoibm9uZSJ9LCJoZWFkIjp7Im5hbWUiOiJub25lIn19fV0sWzQsNywiXFxzdWJzZXRlcSIsMSx7InN0eWxlIjp7ImJvZHkiOnsibmFtZSI6Im5vbmUifSwiaGVhZCI6eyJuYW1lIjoibm9uZSJ9fX1dLFs1LDYsIlxcc3Vic2V0ZXEiLDEseyJzdHlsZSI6eyJib2R5Ijp7Im5hbWUiOiJub25lIn0sImhlYWQiOnsibmFtZSI6Im5vbmUifX19XSxbMCwxLCIiLDEseyJzdHlsZSI6eyJoZWFkIjp7Im5hbWUiOiJub25lIn19fV0sWzEsMiwiIiwxLHsic3R5bGUiOnsiaGVhZCI6eyJuYW1lIjoibm9uZSJ9fX1dLFs0LDMsIiIsMSx7InN0eWxlIjp7ImhlYWQiOnsibmFtZSI6Im5vbmUifX19XSxbNSw0LCIiLDEseyJzdHlsZSI6eyJoZWFkIjp7Im5hbWUiOiJub25lIn19fV0sWzcsOCwiIiwxLHsic3R5bGUiOnsiaGVhZCI6eyJuYW1lIjoibm9uZSJ9fX1dLFs2LDcsIiIsMSx7InN0eWxlIjp7ImhlYWQiOnsibmFtZSI6Im5vbmUifX19XV0=
	\[\begin{tikzcd}
		{\mf Q} & C & M \\
		{\mf q} & B & L \\
		{\mf p} & A & K
		\arrow["\subseteq"{description}, draw=none, from=1-1, to=1-2]
		\arrow["\subseteq"{description}, draw=none, from=2-1, to=2-2]
		\arrow["\subseteq"{description}, draw=none, from=3-1, to=3-2]
		\arrow["\subseteq"{description}, draw=none, from=1-2, to=1-3]
		\arrow["\subseteq"{description}, draw=none, from=2-2, to=2-3]
		\arrow["\subseteq"{description}, draw=none, from=3-2, to=3-3]
		\arrow[no head, from=3-1, to=2-1]
		\arrow[no head, from=2-1, to=1-1]
		\arrow[no head, from=2-2, to=1-2]
		\arrow[no head, from=3-2, to=2-2]
		\arrow[no head, from=2-3, to=1-3]
		\arrow[no head, from=3-3, to=2-3]
	\end{tikzcd}\]
	Then
	\[e(\mf Q/\mf p)=e(\mf Q/\mf q)e(\mf q/\mf p)\qquad\text{and}\qquad f(\mf Q/\mf p)=f(\mf Q/\mf q)f(\mf q/\mf p).\]
\end{proposition}
\begin{proof}
	The ramification statement holds by ``plugging in'' the factorization of $\mf p$ from $\mf q$ up to $\mf Q$ because $\mf Q$ will lie over exactly one prime in $B.$

	The inertial degree statement holds because it is merely asserting
	\[[C/\mf Q:A/\mf p]=[C/\mf Q:B/\mf q][B/\mf q:A/\mf p],\]
	which is true in any tower of fields.
\end{proof}

\subsection{Discriminants}
We have the following definition.
\begin{definition}
	Fix the $AKLB$ setup, as usual. Then the \textit{discriminant} of $B$ over $A$ is the ideal
	\[\mathcal D_{B/A}:=\left\langle d(\omega_1,\ldots,\omega_n):\{\omega_k\}_{k=1}^n\text{ is a basis of }L/K\text{ in }B\right\rangle.\]
\end{definition}
We note that if $B/A$ has an integral basis for $B$ over $A,$ then all the $d(\omega_\bullet)$ will be generated by the discriminant of our integral basis. For example, this occurs when $A$ is a discrete valuation ring.\todo{why} Similarly, because all number rings have an integral basis, we have
\[\mathcal D_{\mathcal O_K/\ZZ}=(\op{disc}\mathcal O_K)\]
for each number field $K.$

Naturally, it turns out that discriminants work well localization.
\begin{proposition}
	Fix the $AKLB$ setup and $S$ a multiplicative subset of $A$ not containing $0.$ Then
	\[\mathcal D_{S^{-1}B/S^{-1}A}=S^{-1}\mathcal D_{B/A}.\]
\end{proposition}
\begin{proof}
	The fact that
	\[\mathcal D_{S^{-1}B/S^{-1}A}\supseteq S^{-1}\mathcal D_{B/A}\]
	holds because any generator in $S^{-1}\mathcal D_{B/A}$ works as a generator for $S^{-1}B/S^{-1}A.$

	For the other inclusion, let $\omega_1,\ldots,\omega_n$ be an $S^{-1}B$-basis for $L/K.$ Then multiplying through by the denominators in $B,$ we see that there are $s_\bullet$ such that $\{s_k\omega_k\}$ form a basis for $L$ over $K$ in $B.$ Then
	\[d(s_1\omega_1,\ldots,s_n\omega_n)=(s_1\cdots s_n)^{n}d(\omega_1,\ldots,\omega_n),\]
	so the extra generators do not help us.
\end{proof}

The reason that we care about discriminants is the following.
\begin{proposition}
	Fix $\mf p$ a nonzero prime of $A.$ Then $\mf p$ ramifies in $B$ if and only if $\mf p\supseteq\mathcal D_{B/A}.$
\end{proposition}
\begin{proof}
	Localization at $\mf p$ does not affect ramification information or the status if $\mf p$ dividing the discriminant, so we may suppose that $\mf p$ is the only prime of $A.$ But now, $B=A[\theta]$ for some $\theta$ for reasons which are unclear, so powers of $\theta$ make an integral basis. Letting $p$ be the minimal polynomial of $\theta,$ we see
	\[\mathcal D_{B/A}=\langle\op{disc}\theta^\bullet\rangle=\langle\op{disc}p(x)\rangle.\]
	Now, $\mf p$ ramifies if and only if $e_\bullet>1$ somewhere or $(B/\mf q_\bullet)/(A/\mf p)$ is inseparable. This is equivalent to $\overline p$ having a repeated factor in its factorization$\pmod{\mf p}$ or some factor is inseparable, and this laster statement is equivalent to some factor having multiple roots. \todo{review}

	So in total, this is equivalent to $\overline p$ having multiple roots, which is equivalent to its discriminat$\pmod{\mf p}$ vanishing, which is equivalent to $\mf p$ dividing into $\mathcal D_{B/A}.$
\end{proof}

\end{document}