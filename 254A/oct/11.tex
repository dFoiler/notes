\documentclass[../notes.tex]{subfiles}

\begin{document}

% !TEX root = ../notes.tex














Maybe we do quadratic reciprocity today.

\subsection{Housekeeping}
Question 3 from homework 6, which asked localization to commute with the conductor, actually applies in the general $AKLB$ set-up. Namely, the assumption that $L/K$ is separable is only needed to have $B$ finite over $A.$ So the proof of Dedekind--Kummer from in class is valid of the $AKLB$ set-up.

Questions 1 and 2 of homework 6, people should have used the fact that the equivalence relation on $S^{-1}A$ was minimal with respect to $\frac as=\frac{as'}{ss'}.$ In particular, we have the following corollary.
\begin{corollary}
	Fix $A$ a ring and $S$ a multiplicative subset. Then for functions $f:S\times A\to B,$ if $f(s,a)=f(ss',s'a)$ for each $s,s'\in S$ and $a\in A,$ then there is a unique function $f:S^{-1}A\to B$ commuting.
\end{corollary}

Last time we ended with the following.
\begin{proposition}
	Fix the $AKLB$ set-up. A nonzero prime $\mf p$ ramifies if and only if $\mf p\mid\mathcal D_{B/A}.$
\end{proposition}
We showed this under the assumption $B_\mf p=A_\mf p[\theta]$ for some $\theta\in B$; we'll need a little more theory (of completions) to prove the result in general.

We do have the following corollary, which we can prove without that theory.
\begin{proposition}
	If $L/K$ is separable, then only finitely many primes ramify.
\end{proposition}
\begin{proof}
	We note that $\mathcal D_{B/A}$ is nonzero, so there are only finitely many prime divisors of $\mathcal D_{B/A}.$ Further, we can somewhat control the conductor: let $\theta$ be a primitive element for $L/K$ in $B,$ and only finitely many primes will divide the conductor for $A[\theta]$ (the conductor is nonzero). So we do have $A_\mf p[\theta]=B_\mf p$ for all but finitely many primes $\mf p,$ and the argument works.
\end{proof}
The above result need not true when $L/K$ fails to be separable.
\begin{nex}
	Let $A=\overline{\FF_p}[t]$ and $K$ its fraction field. Then $L=\overline{\FF_p}(t^{1/p})$ and $B=\overline{\FF_p}[t^{1/p}]$ (indeed, $B$ is integrally closed and integral over $A$). Now, for any $\alpha\in\overline{\FF_p}$ implies
	\[(t-\alpha)B=(t^{1/p}-\alpha^{1/p})^p\]
	is ramified for each $\alpha\in\overline{\FF_p},$ so there are infinitely many ramified primes.
\end{nex}

\subsection{Quadratic Reciprocity}
We start with the following definition.
\begin{defi}[Legendre symbol]
	Fix $p\in\ZZ$ an odd, rational prime. Then, given $a\in\ZZ$ not divisible by $p,$ we define the \textit{Legendre symbol}
	\[\left(\frac ap\right):=\begin{cases}
		1 & a\text{ is a nonzero square}\pmod p, \\
		-1 & \text{else}.
	\end{cases}\]
\end{defi}
We have the following.
\begin{prop}[Euler's criterion]
	Fix $p\in\ZZ$ an odd, rational prime and $a\in\ZZ$ not divisible by $p.$ Then
	\[\left(\frac ap\right)\equiv a^{(p-1/2}\pmod p.\]
\end{prop}
\begin{proof}
	Recall that $\FF_p^\times$ is cyclic (it is a finite multiplicative subgroup of a field), so fix $g$ a generator so that $a\equiv g^r$ for some $r\in\ZZ.$

	Now, $a^{(p-1)/2}\equiv1$ if and only if $\zeta^{r(p-1)/2}\equiv1$ if and only if $p-1\mid\frac{p-1}2\cdot r$ if and only if $2\mid r$ if and only if $a$ is a square.\footnote{Showing that $2\nmid r$ implies that $a$ is not a square is somewhat annoying, but it is not too hard.} And to finish, we note that at least
	\[a^{(p-1)/2}\in\{\pm1\pmod p\}\]
	because it squares to $a^{p-1}\equiv1,$ and $x^2-1$ only has roots $\{\pm1\}.$
\end{proof}
\begin{corollary}
	Fix $p$ an odd, rational prime. Then, for each $a,b\in\ZZ$ not divisible by $p,$ we have
	\[\left(\frac{ab}p\right)=\left(\frac ap\right)\left(\frac bp\right).\]
\end{corollary}
\begin{proof}
	Plug into Euler's criterion to get
	\[\left(\frac{ab}p\right)\equiv\left(\frac ap\right)\left(\frac bp\right)\pmod p.\]
	Now, $p>2$ implies that we must have equality because $1\not\equiv-1\pmod2.$
\end{proof}
And here is quadratic reciprocity.
\begin{theorem}
	Fix $p$ an odd rational prime. Then we have the following.
	\begin{enumerate}[label=(\alph*)]
		\item For an odd rational prime $\ell\ne p,$ we have
		\[\left(\frac\ell p\right)\left(\frac p\ell\right)=(-1)^{\frac{p-1}2\cdot\frac{\ell-1}2}.\]
		\item We have
		\[\left(\frac{-1}p\right)\equiv(-1)^{\frac{p-1}2}.\]
		\item We have
		\[\left(\frac 2p\right)=(-1)^{\frac{p^2-1}8}.\]
	\end{enumerate}
\end{theorem}
\begin{proof}
	We show (a) but not the others.
	\begin{enumerate}[label=(\alph*)]
		\item Let $\zeta$ be a primitive $\ell$th root of unity (in $\CC$), and we work in $\ZZ[\zeta].$ The key is to try and induce an element in the quadratic subfield of $\ZZ[\zeta]$; for this, we set
		\[\tau:=\sum_{a\in\FF_\ell^\times}\left(\frac a\ell\right)\zeta^a.\]
		We claim $\tau^2=\left(\frac{-1}\ell\right)\ell.$ Indeed, we have
		\begin{align*}
			\left(\frac{-1}\ell\right)\tau^2 &= \sum_{a,b\in\FF_\ell^\times}\left(\frac{-1}\ell\right)\left(\frac a\ell\right)\left(\frac b\ell\right)\zeta^{a+b} \\
			&= \sum_{a,b\in\FF_\ell^\times}\left(\frac {a}\ell\right)\left(\frac {-b}\ell\right)\zeta^{a-(-b)} \\
			&= \sum_{a,b\in\FF_\ell^\times}\left(\frac {a}\ell\right)\left(\frac b\ell\right)\zeta^{a-b} \\
			&= \sum_{a,b\in\FF_\ell^\times}\left(\frac {ab^{-1}}\ell\right)\zeta^{a-b} \\
			&= \sum_{b,c\in\FF_\ell^\times}\left(\frac c\ell\right)\zeta^{bc-b} \\
			&= \sum_{c\in\FF_\ell^\times\setminus\{1\}}\left(\frac c\ell\right)\sum_{b\in\FF_\ell^\times}(\zeta^{c-1})^b+\sum_{b\in\FF_\ell^\times}1 \\
			&\stackrel*= \sum_{c\in\FF_\ell^\times\setminus\{1\}}\left(\frac c\ell\right)(-1)+(\ell-1)
			&= \ell,
		\end{align*}
		where $\stackrel*=$ used the fact that $\zeta^{c-1}$ will be a primitive $\ell$th root of unity.

		Continuing, we have the equality
		\[\tau^p=\tau\left(\tau^2\right)^{(p-1)/2}=\tau\left(\left(\frac{-1}\ell\right)\ell\right)^{(p-1)/2}\equiv\tau(-1)^{\frac{\ell-1}2\cdot\frac{p-1}2}\left(\frac\ell p\right).\]
		On the other hand, applying the Frobenius automorphism tells us
		\[\tau^p\equiv\sum_{a\in\FF_\ell^\times}\left(\frac a\ell\right)\zeta^{ap}=\left(\frac p\ell\right)\sum_{a\in\FF_\ell^\times}\left(\frac{pa}\ell\right)\zeta^{ap}=\left(\frac p\ell\right)\tau.\]
		So in total, we divide out by $\tau$ (which is a unit$\pmod p$ because $\tau^2\in\{\pm\ell\}$ is a unit)
		\[(-1)^{\frac{\ell-1}2\cdot\frac{p-1}2}\left(\frac\ell p\right)\equiv\left(\frac p\ell\right)\pmod p,\]
		which finishes because $1\not\equiv-1\pmod p,$ even in $p\ZZ[\zeta].$
		\qedhere
	\end{enumerate}
\end{proof}

\end{document}