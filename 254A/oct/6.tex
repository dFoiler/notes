% !TEX root = ../notes.tex















As usual, we take the $AKLB$ set-up.

\subsection{More on Dedekind--Kummer}
Last time we were proving the following statement, which we have now amended slightly.
\begin{theorem}[Dedekind--Kummer] \label{thm:dk}
	Suppose that $L/K$ has a primitive element $\theta\in B$ where $\mf f$ is the conductor of $A[\theta].$ Further, take $\mf p$ to be a nonzero prime of $A$ such that $\mf pB$ is coprime to $\mf f.$

	Now, let $p\in A[x]$ be the minimal polynomial for $\theta$ over $K,$ and factor it in $(A/\mf p)[x]$ as
	\[\overline p=\prod_{k=1}^r\overline{p_k}^{e_k},\]
	where $p_\bullet$ are distinct monic polynomials for which $\overline{p_\bullet}$ is irreducible in $(A/\mf p)[x].$ Then for each $k,$ we can set $\mf q_k:=\mf pB+p_k(\theta)B$ so that
	\[\mf pB=\prod_{k=1}^r\mf q_k^{e_k},\]
	where $f(\mf q_k/\mf p)=\deg\overline{p_k}.$
\end{theorem}
Before going into this proof, we take the following lemmas.
\begin{lemma}
	In the $AKLB$ setup, let $\mf q$ be a nonzero prime of $B.$ Then $\mf q\cap A\ne0.$
\end{lemma}
\begin{proof}
	Find any $\alpha\in\mf q\setminus\{0\}.$ Then $\op N_K^L(\alpha)/\alpha\in B$ because it is the product of conjugates of $\alpha,$ and $\alpha$ being integral forces the entire product to be integral. So now $\op N_K^L(\alpha)$ is a nonzero element of $\mf q\cap A.$
\end{proof}
\begin{lemma} \label{lem:specprod}
	Fix $A_1,\ldots,A_r$ rings. Then
	\[\op{Spec}\left(\prod_{k=1}^rA_k\right)=\bigsqcup_{k=1}^r\pi_k^*\op{Spec}A_k,\]
	where $\pi_\ell$ is the projection $\prod_{k=1}^rA_k\onto A_\ell.$ Further, the map $\pi_\ell^*:\op{Spec}A_k\to\op{Spec}\prod_{k=1}^rA_k.$
\end{lemma}
\begin{proof}
	By induction, we only have to look at $r=2.$ (There is nothing to show for $r=0$ or $r=1.$) Now, fix any $\mf p\in\op{Spec}(A_1\times A_2),$ and the trick is that
	\[(1,0)\cdot(0,1)=(0,0)\in\mf p.\]
	Namely, either $(1,0)\in\mf p$ or $(0,1)\in\mf p$; without loss of generality, we take $(1,0)\in\mf p.$ Note that $(0,1)\notin\mf p,$ for this would imply that $A_1\times A_2\subseteq\mf p.$
	
	Thus, $(a,b)\in\mf p$ if and only if $(a,b)(0,1)\in\mf p$ if and only if $(0,b)\in\mf p,$ so we can write $\mf p=A_1\times\mf p'$ for some $\mf p',$ and we can verify by hand that $\mf p'$ is a prime ideal.
\end{proof}
We now go into the proof.
\begin{proof}[Proof of \autoref{thm:dk}]
	We have two steps.
	\begin{enumerate}
		\item We start with the local case, where $A$ is a local ring, $\mf p$ its unique prime, and $\kappa=A/\mf p.$ By assumption, $\mf f$ is not divisible by any nonzero prime of $B$ lying over $\mf p,$ and in fact these are all the primes of $B$ because each prime in $B$ must lie over some prime in $A,$ and the only prime available is $\mf p.$

		Thus, $\mf f$ is not divisible by any prime, so it follows $\mf f=(1),$ so $B=A[\theta].$ We now have a nice power basis. We now claim that
		\[B/\mf pB\cong\kappa[x]/\left(\overline p(x)\right).\]
		The point here is that we have the sequence of maps
		\[B=A[\theta]\cong\frac{A[x]}{(p(x))}\onto\frac{(A/\mf p)[x]}{(p(x))}.\]
		Then the kernel of this map consists of the elements of $A[\theta]$ whose coefficients were in $\mf p,$ which is exactly $\mf pB.$ So the claim follows.

		Continuing, we have by the Chinese remainder theorem that
		\[\frac{\kappa[x]}{(\overline p(x))}\cong\prod_{k=1}^r\frac{\kappa[x]}{\left(\overline p_k(x)^{e_k}\right)}.\]
		Now, the point of \autoref{lem:specprod} is that the primes of $B$ lying over $\mf p$ are the same as the primes of $B/\mf pB$\todo{wut}, which is simply the above product, so the primes of $B$ can be identified with the disjoint union
		\[\bigsqcup_{k=1}^r\op{Spec}\kappa[x]/\left(\overline p_k^{e_k})\right).\]
		However, any prime $\mf p$ of $\kappa[x]/\left(\overline p_k^{e_k}\right)$ must contain the zero element $\overline p_k^{e_k}+\left(\overline p_k^{e_k}\right),$ so any prime must be $\overline p_k^{e_k}.$ Now going back up to $\op{Spec}\kappa[x]/(\overline p),$ we find that its prime are the $\overline p_\bullet,$ which gives
		\[\op{Spec}B/\mf pB=\{p_k(\theta):1\le k\le r\},\]
		so the primes over $\mf p$ are of the form $p_k(\theta)+\mf pB=:\mf q_k.$

		Furthermore, we have the diagram for each $k.$
		% https://q.uiver.app/?q=WzAsNixbMCwwLCJcXHRoZXRhIl0sWzEsMCwieCtcXGxlZnQoXFxvdmVybGluZSBwX2tee2Vfa31cXHJpZ2h0KSJdLFswLDEsIkIiXSxbMSwxLCJcXGthcHBhW3hdL1xcbGVmdChcXG92ZXJsaW5lIHBfa157ZV9rfVxccmlnaHQpIl0sWzEsMiwiQS9cXG1mIHAiXSxbMCwyLCJBIl0sWzUsNF0sWzIsM10sWzAsMV0sWzAsMl0sWzUsMiwiIiwxLHsic3R5bGUiOnsidGFpbCI6eyJuYW1lIjoiaG9vayIsInNpZGUiOiJ0b3AifX19XSxbMSwzXSxbNCwzLCIiLDEseyJzdHlsZSI6eyJ0YWlsIjp7Im5hbWUiOiJob29rIiwic2lkZSI6InRvcCJ9fX1dXQ==
		\[\begin{tikzcd}
			\theta & {x+\left(\overline p_k^{e_k}\right)} \\
			B & {\kappa[x]/\left(\overline p_k^{e_k}\right)} \\
			A & {A/\mf p}
			\arrow[from=3-1, to=3-2]
			\arrow[from=2-1, to=2-2]
			\arrow[from=1-1, to=1-2]
			\arrow[from=1-1, to=2-1]
			\arrow[hook, from=3-1, to=2-1]
			\arrow[from=1-2, to=2-2]
			\arrow[hook, from=3-2, to=2-2]
		\end{tikzcd}\]
		The point here is that $B/\mf q_k\cong\kappa[x]/(\overline p_k),$ which is a field due to the unique factorization in $\kappa[x],$ so $\mf q_k$ is maximal and in particular a nonzero prime of $B.$ Further, we can see that $\kappa\to\kappa[x]/\left(\overline p_k\right)$ is injective.\todo{what is happening} The point is that
		\[f(\mf q_k/\mf p)=[B/\mf q_k:A/\mf p]=[\kappa[x]/\left(\overline p_k\right):\kappa]=\deg\overline p_k.\]
		We also note that the $\mf q_k$ are distinct because the $\overline p_k$ are distinct\todo{wut}. In fact the $\mf q_\bullet$ are all the primes of $B$ because all primes must lie over some prime of $A,$ so all primes of $B$ must lie over $\mf p\subseteq A,$ so each prime must be a prime in $B/\mf pB,$ which we classified as our $\mf q_\bullet.$

		And lastly we have to cover the ramification indices. The point is that
		\[\mf pB=\prod_{k=1}^r\mf q_k^{d_k}\]
		must be our factorization because the $\mf q_\bullet$ are the only available primes. We note that $e_k\ge d_k$ because lots of distribution gives
		\[\prod_{k=1}^r\mf q_k^{e_k}=\prod_{k=1}^r(\mf pB+p_k(\theta)B)^{e_k}\subseteq\mf pB+\left(\prod_{k=1}^rp_i(\theta)^{e_k}\right)B=\mf pB+p(\theta)B,\]
		which is $\mf pB$ because $p(\theta)=0.$ Thus, 
		\[\prod_{k=1}^r\mf q_k^{d_k}\mf pB~\bigg|~\prod_{k=1}^r\mf q_k^{e_k},\]
		which tells us $d_k\le e_k$ for each $k,$ which is what we wanted.

		To finish, we note that
		\[n=\sum_{k=1}^re(\mf q_k/\mf p)f(\mf q_k/\mf p)=\sum_{k=1}^rd_kf_k\]
		while
		\[\sum_{k=1}^re_kf_k=\deg\overline p=\deg p=n,\]
		so the termwise $e_k\ge d_k$ inequalities forces $e_k=d_k$ everywhere. This finishes the local case.

		\item Now we show the general case. Fix $A$ and $B$ as before, and set $\mf q_k:=\mf pB+p_k(\theta)B$ as in the statement of the problem. Letting $\mf q_k':=\mf pB_\mf p+p_i(\theta)B_\mf p,$ we note that the local case above tells us that the $\mf q_\bullet'$ are the primes of $B_\mf p$ lying over $\mf pA_\mf p,$ but it is not clear if $\mf q_\bullet$ are the primes of $B$ lying over $\mf p.$ Well,
		\[\mf q_kB_\mf p=\mf pBB_\mf p+p_i(\theta)BB\mf p=\mf pB_\mf p+p_k(\theta)B_\mf p=\mf q_k'.\]
		Now, $\mf q_k\supseteq\mf pB,$ so all the primes in the factorization of $\mf q_k$ must consist of primes lying over $\mf p$ as well; we let our factorization be
		\[\mf q_k=\prod_\ell\widetilde{\mf q_{k\ell}}^{\widetilde{e_{k\ell}}}.\]
		Localizing, we set $S:=A/\mf p$ so that
		\[\mf q_k'=\mf q_kB_\mf p=S^{-1}\mf q_k=\prod_\ell\left(S^{-1}\widetilde{\mf q_{k\ell}}\right)^{\widetilde{e_{k\ell}}}\]
		is a prime factorization of $\mf q_k'.$ But $\mf q_k'$ is prime, so this factorization must collapse into a single prime.\todo{wut}

		Now, we can get the $e_k$ and the $f_k$ from the local case, and the $\mf q_\bullet$ are the only primes above $\mf p$ because $\mf q_\bullet B_\mf p=S^{-1}\mf p$ is still prime ni $B_\mf p,$ so it must hit one of the $\mf q_\bullet'.$ So our factorization remains valid up in $B.$\todo{wut}
		\qedhere
	\end{enumerate}
\end{proof}