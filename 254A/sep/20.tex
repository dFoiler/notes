\documentclass[../notes.tex]{subfiles}

\begin{document}

% !TEX root = ../notes.tex

















For today's class, we take $K$ to be a number field.

\subsection{Small Elements of Ideals}
Recall the following statement from last lecture.
\begin{proposition}
    Fix $\mf a$ an integral $\mathcal O_K$-ideal. Then $j(\mf a)$ is a full lattice of $K_\RR$ with covolume $\sqrt{|d_K|}\cdot[\mathcal O_K:\mf a].$
\end{proposition}
This will give us the following theorem.
\begin{theorem} \label{thm:minkapp}
    Fix $\mf a$ a nonzero $\mathcal O_K$-ideal. Then the following are true.
    \begin{enumerate}[label=(\alph*)]
        \item Fix $\{c_\tau\}_{\tau\in\op{Hom}(K,\CC)}$ be a collection of positive real numbers such that $c_\tau=c_{\overline\tau}$ for each $\tau.$ Further suppose their product is
        \[\prod_\tau c_\tau>A[\mathcal O_K:\mf a],\]
        where $A=\left(\frac2\pi\right)^s\sqrt{|d_K|}.$ Then there is a nonzero $a\in\mf a\setminus\{0\}$ such that $|\tau a|<c_\tau$ for each $\tau.$
        \item The above remains true if we replace both inequalities above with non-strict ones.
    \end{enumerate}
\end{theorem}
Note that the $c_{\overline\tau}=c_\tau$ condition exists because this is how $\tau a$ also behaves.
\begin{proof}
    We show these one at a time.
    \begin{enumerate}[label=(\alph*)]
        \item Fix
        \[X:=\left\{(z_\tau)\in K_\RR:|z_\tau|<c_\tau\text{ for each }\tau\right\}.\]
        We can check that $X$ is convex and centrally symmetric, so we would like to throw Minkowski's theorem at it. Well, computing the volume of $X$ as various boxes in $\RR$ and various disks in $\CC,$ and then adding in the $2^s$ from the canonical measure, we get
        \[\op{vol}X=2^r\cdot2^s\cdot\pi^s\prod_{\tau}c_\tau.\]
        Using our bounds, this is larger than
        \[\op{vol}X>2^r\left(2\pi\cdot\frac{2}\pi\right)^s\sqrt{|d_K|}\cdot[\mathcal O_K:\mf a].=\underbrace{2^r\cdot4^s}_{2^n}\cdot\op{covol}j(\mf a),\]
        so Minkowski's theorem does the job.
        \item This is on the homework. Essentially $X$ above becomes compact.
        \qedhere
    \end{enumerate}
\end{proof}
And now we get the following major step to finiteness of the class group.
\begin{corollary}
    Every nonzero $\mathcal O_K$-ideal $\mf a$ contains a nonzero element $a\in\mf a\setminus\{0\}$ such that
    \[[|\op N_\QQ^K(a)|<\left(\frac2\pi\right)^s\sqrt{|d_K|}\cdot[\mathcal O_K:\mf a].\]
\end{corollary}
\begin{proof}
    Use \autoref{thm:minkapp} part (b): choose any good $c_\tau$ with product equal to $A[\mathcal O_K:\mf a]$ (where here we have $A:=\left(\frac2\pi\right)^s\sqrt{|d_K|}$ as usual) so that the nonzero element $a\in\mf a\setminus\{0\}$ we are promised has
    \[|\op N_\QQ^K(a)|=\prod_\tau|\tau a|<\prod_\tau c_\tau=A[\mathcal O_K:\mf a],\]
    which is what we wanted.
\end{proof}
We remark that, for $a\in\mf a$ from the above corollary, we have that $(a)\subseteq\mf a$ so that $\mf a\mid(a),$ making $a\mf a^{-1}$ an $\mathcal O_K$-ideal. In order to get finiteness of the class group, we would like to bound $[\mathcal O_K:a\mf a^{-1}]$ because this would place all ideal classes in a box.

For this, we need the following tool.
\subsection{Norms of Ideals}
We have the following lemma.
\begin{lemma}
    For all nonzero $a\in\mathcal O_K\setminus\{0\},$ we have that
    \[[\mathcal O_K:(a)]=\left|\op N_\QQ^K(a)\right|.\]
\end{lemma}
\begin{proof}
    Fix $\omega_1,\ldots,\omega_n$ be an integral $\ZZ$-basis for $\mathcal O_K.$ Then
    \[\{a\omega_1,\ldots,a\omega_n\}\]
    is also an integral basis for $(a).$ Comparing discriminants, we have that
    \[d((a))=\det\begin{bmatrix}
        \tau_1(a\omega_1) & \cdots & \tau_1(a\omega_n) \\
        \vdots & \ddots & \vdots \\
        \tau_n(a\omega_1) & \cdots & \tau_n(a\omega_n)
    \end{bmatrix}^2=\left(\prod_\tau\tau(a)^2\right)\det\begin{bmatrix}
        \tau_1(\omega_1) & \cdots & \tau_1(\omega_n) \\
        \vdots & \ddots & \vdots \\
        \tau_n(\omega_1) & \cdots & \tau_n(\omega_n)
    \end{bmatrix}^2,\]
    by factoring out $\tau_\bullet(a)$ from each column. However, this right-hand side is $|\op N_\QQ^K(a)|^2d(\mathcal O_K),$ so it suffices to recall that
    \[d((a))=[\mathcal O_K:(a)]^2d(\mathcal O_K)\]
    from much earlier, finishing.
\end{proof}
With this in mind, we generalize norms of elements to norms of ideals as follows.
\begin{definition}[Absolute norm]
    The \textit{absolute norm} of a nonzero $\mathcal O_K$-ideal $I$ is $\op N(I):=[\mathcal O_K:I].$
\end{definition}
This norm turns out to behave as we want.
\begin{proposition}
    Given nonzero $\mathcal O_K$-ideals $\mf a$ and $\mf b,$ we have $\op N(\mf a)\op N(\mf b)=\op N(\mf a\mf b).$
\end{proposition}
\begin{proof}
    We use the Chinese remainder theorem; we have the following cases.
    \begin{enumerate}
        \item If $\mf a$ and $\mf b$ are relatively prime (i.e., $\mf a+\mf b=\mathcal O_K$), then $\mf a\cap\mf b=\mf a\mf b,$ and we finish by the Chinese remainder theorem:
        \[\frac{\mathcal O_K}{\mf a\mf b}=\frac{\mathcal O_K}{\mf a\cap\mf b}\cong\frac{\mathcal O_K}{\mf a}\times\frac{\mathcal O_K}{\mf b},\]
        which gives what we wanted.
        \item By unique prime factorization, it remains to deal with prime-powers. Take $\mf a=\mf p^k$ and $\mf b=\mf p^\ell$ for some prime $\mf p.$ By induction, we may take $\ell=1$ and then strip off powers of $\mf p$ from $\mf b$ one at a time.

        So we have to show that
        \[\op N\left(\mf p^{k+1}\right)=\op N(\mf p^K)\op N(\mf p).\]
        However, $\mathcal O_K/\mf p^{k+1}$ is principal (this was on the homework), so we can find $\pi\in\mathcal O_K$ (namely, $\pi\in\mf p^k\setminus\mf p^{k+1}$) such that $(\pi\mod\mf p^{k+1})=\mf p^k/\mf p^{k+1}.$ It follows that the map
        \[x\mapsto ax+\mf p^{k+1}\]
        is surjective because $\mathcal O_K/\mf p^{k+1}$ is principal, with kernel containing $\mf p$ (because $a\in\mf p^k$) but not containing $1,$ so we have induced an isomorphism
        \[\mathcal O_K/\mf p\cong\mf p^{k+1}/\mf p^k.\]
        So we can compute the indices
        \[[\mathcal O_K:\mf p]=\left[\mf p^{k+1}:\mf p^k\right]=\frac{\left[\mathcal O_K:\mf p^{k+1}\right]}{\left[\mathcal O_K:\mf p^k\right]},\]
        which is what we wanted.
        \qedhere
    \end{enumerate}
\end{proof}

\subsection{Finiteness of the Class Group}
Now we have the following corollary, finishing up.
\begin{corollary}
    Every ideal class in $K$ contains an integral ideal of index at most
    \[A_K:=\left(\frac2\pi\right)^2\sqrt{|d_K|}.\]
\end{corollary}
\begin{remark}
    We can improve the constant to $M_K:=\frac{n!}{n^n}\left(\frac4\pi\right)^s\sqrt{|d_k|}$ with some effort.
\end{remark}
\begin{proof}
    Fix an ideal class $C$ with $\mf a\in C^{-1}$ an integral ideal, and find some $a\in\mf a$ such that
    \[\left|\op N_\QQ^K(a)\right|\le A[\mathcal O_K:\mf a].\]
    Now, $a\mf a^{-1}=(a)\mf a^{-1}$ is an integral ideal in the original ideal class $C$ because $a\in\mf a.$ We note that We can bound this norm by
    \[\left[\mathcal O_K:a\mf a^{-1}\right]=\op N(a\mf a^{-1})=\frac{\op N((a))}{\op N(\mf a)}=\frac{\left|\op N_\QQ^K(a)\right|}{[\mathcal O_K:\mf a]}\le A.\]
    Note that we had to use multiplicativity of the ideal norm to write $\op N(a\mf a^{-1})\op N(\mf a^{-1})=\op N((a)).$
\end{proof}
We are almost at the finish line.
\begin{lemma}
    Suppose we have a constant $B=B_K$ depending only $K$ such that every ideal class of $K$ contains an integral ideal of norm bounded by $B.$ Then $\op{Cl}_K$ is finite.
\end{lemma}
\begin{proof}
    We have to show that there only finite many ideals of norm bounded by $B$ so that there are only finitely many candidates for distinct ideal classes. For this, it suffices to show that there are finitely many ideals of a particular norm $m$ with $0<m\le B.$

    Well, fix $\mf a$ an integral domain of index $m.$ Then it follows $\mf a\supseteq m\mathcal O_K,$\todo{excuse me?} so $\mf a$ is an additive subgroup of $\mathcal O_K/m\mathcal O_K\cong(\ZZ/m\ZZ)^n.$ But there are only finitely many elements of $(\ZZ/m\ZZ)^n,$ so only finitely many additive subgroups, so only finitely many options for $\mf a.$
\end{proof}
And here we are.
\begin{theorem}
    We have that $\op{Cl}_K$ is finite.
\end{theorem}
\begin{proof}
    This follows from the above discussion.
\end{proof}

This lets us have the following definition.
\begin{definition}[Class number]
    The \textit{class number} of a number field $K$ is $h_K:=\#\op{Cl}_K.$
\end{definition}
\begin{example}
    For $K=\QQ(\sqrt{-7}),$ we have that $h_K=1.$ Here we have $n=2,=0,s=1,$ and $d_K=-7.$ It follows that every ideal class contains an integral ideal with norm at most
    \[M_K=\frac{2!}{2^2}\left(\frac4\pi\right)^1\sqrt{|-7|}=\frac2\pi\sqrt7<2.\]
    However, there is only ideal with index $1,$ so our only ideal class is the principal one.
\end{example}

\end{document}