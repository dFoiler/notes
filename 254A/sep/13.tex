\documentclass[../notes.tex]{subfiles}

\begin{document}

% !TEX root = ../notes.tex












I'm using Visual Studio Code today. I've also done some upgrades to my preamble. Let's see how I like it.

\subsection{Unique Prime Factorization, Continued}
Today, we will have $A$ be a Dedekind domain with $K$ its fraction field. Recall that we were proving the following statement.
\begin{thm} \label{thm:upf2}
    Every nonzero ideal $I\subseteq A$ admits a factorization
    \[I=\prod_{k=1}^r\mf p_k\]
    into prime ideals $\{\mf p_k\}_{k=1}^r,$ which is unique up to order.
\end{thm}
We left off proving the existence. Last time we showed the following.
\begin{lem}
    If $\mf p$ is a nonzero prime ideal and $I$ a nonzero ideal, then $I\mf p^{-1}\supsetneq I.$
\end{lem}
As a corollary, we have the following.
\begin{cor}
    We have that $\mf p^{-1}\mf p=A$ for each $\mf p$ a nonzero prime.
\end{cor}
\begin{proof}
    We see that $\mf p\subsetneq\mf p\mf p^{-1}\subseteq A,$ where we have applied the previous lemma. However, $\mf p$ is maximal, so we must have $\mf p\mf p^{-1}=A.$
\end{proof}
We now jump into the proof.
\begin{proof}[Proof of Existence in \autoref{thm:upf2}]
    Suppose for the sake of contradiction not. Then we can consider the set of all nonzero prime ideals which do not have a prime factorization and note that this nonempty collection must have a maximal element $\mf m.$

    Note that $\mf m\ne A$ because $A$ has the empty factorization, so we can actually put $\mf m$ inside of a maximal (prime) ideal $\mf p\subseteq\mf m.$ In fact, $\mf m\ne\mf p$ because otherwise we would have the single-prime factorization. We note that
    \[\mf m\mf p^{-1}\subseteq\mf p\mf p^{-1}=A\]
    while $\mf m\subsetneq \mf m\mf p^{-1}.$ So by maximality of $\mf m,$ we see that $\mf m\mf p^{-1}$ has a prime factorization, but then we can multiply both sides of the factorization by $\mf p$ to get a factorization for $\mf m.$ This is a contradiction, finishing the proof.
\end{proof}
\begin{proof}[Proof of Uniqueness in \autoref{thm:upf2}]
    This is as usual. Suppose that we have
    \[\prod_{k=1}^r\mf p_k=\prod_{\ell=1}^s\mf q_\ell\]
    for primes $\mf p_\bullet$ and $\mf q_\bullet.$ Then we see that $\mf p_1$ must contain one of the primes on the other side, say $\mf q_1,$ which forces $\mf p_1=\mf q_1,$ so we can cancel both sides by this prime and finish by induction.
\end{proof}
\begin{remark}
    This is roughly why we name ideals ``ideals'': they are giving us this lovely unique prime factorization.
\end{remark}
This gives us the following corollary.
\begin{cor}
    The monoid of nonzero ideals of $A$ is a free monoid generated by nonzero prime ideals.
\end{cor}
\begin{proof}
    This follows directly from \autoref{thm:upf2}.
\end{proof}
We would like to turn this monoid into a group.

\subsection{Fractional Ideals}
We have the following definition.
\begin{defi}
    A fractional ideal $I$ of $A$ is a nonzero, finitely generated $A$-submodule of $K.$ (If $K$ is a number field, we might say that a fractional ideal of $K$ is a fractional ideal of $\mathcal O_K.$)
\end{defi}
We have the following claim.
\begin{prop} \label{prop:fraclem}
    Let $I$ be an $A$-submodule of $K.$ The following are equivalent.
    \begin{enumerate}[label=(\alph*)]
        \item $I$ is a fractional ideal of $A.$
        \item $I$ is nonzero, and there is a nonzero $c\in A$ such that $cI\subseteq A.$
        \item $I$ is nonzero, and there is a $c\in K^\times$ such that $cI\subseteq A.$
    \end{enumerate}
\end{prop}
\begin{proof}
    We take these one at a time. We see that (a) implies (b) by noting $I$ is certainly nonzero, and in fact, $I$ is finitely generated (as a fractional ideal) by, say, $\{x_k\}_{k=1}^m.$ Then we can multiply $I$ by the product of the denominators in $x_k$ to get inside of $A.$

    We see that (b) implies (c) with no work.

    Lastly, we show that (c) implies (a). Well, fix our promised $c.$ Because $cI$ is an $A$-ideal, it follows that $cI$ is finitely generated because $A$ is Noetherian. Thus, $I$ is finitely generated by dividing out each generator by $c.$
\end{proof}
This gives the following corollary.
\begin{cor}
    All nonzero ideals of $A$ are fractional ideals.
\end{cor}
\begin{proof}
    Take $c=1$ in (b) of \autoref{prop:fraclem}.
\end{proof}
\begin{remark}
    To avoid confusion, we might say ``integral ideals'' for $A$-ideals.
\end{remark}
Carrying unique prime factorization over to fractional ideals, we have the following theorem.
\begin{thm}[Unique factorization of fractional ideals]
    Fix $\{\mf p_\alpha\}_{\alpha\in\lambda}$ be the prime ideals of $A.$ Then every fractional ideal $I$ of $A$ has a unique factorization into primes
    \[I=\prod_{\alpha\in\lambda}\mf p_\alpha^{e_\alpha},\]
    where the $e_\alpha\in\ZZ$ and all but finitely many of the $e_\alpha$ are zero. In fact, $I$ is integral if and only if $e_\alpha\ge0$ for each $\alpha.$
\end{thm}
\begin{proof}
    We leave this as an exercise.
\end{proof}
So we get our group after all.
\begin{thm}
    The set of fractional ideals of $A$ forms an abelian group under multiplication, isomorphic to
    \[\bigoplus_{\mf p}\ZZ.\]
    If $A=\mathcal O_K$ for a number field $K,$ then we denote this group by $J_K.$
\end{thm}
This gives the following definition.
\begin{defi}
    Fix $\mf a$ and $\mf b$ fractional ideals of $A.$ Then we write $\mf a\mid\mf b$ if and only if $\mf a^{-1}\mf b\subseteq A.$
\end{defi}
In fact, we see that $\mf a\mid\mf b$ if and only if $\mf aI=\mf b$ for some integral ideal $I.$ We even get the following.
\begin{lem}
    Fix $\mf a$ and $\mf b$ fractional ideals. We have that $\mf a\mid\mf b$ if and only if $\mf a\supseteq\mf b.$
\end{lem}
\begin{proof}
    By definition, we see that $\mf a\mid\mf b$ if and only if $\mf a^{-1}\mf b\subseteq A$ if and only if $\mf b\subseteq A\mf a=\mf a.$
\end{proof}
We are interested in the simplest fractional ideals, the principal ones.
\begin{defi}[Principal]
    Fix $x\in K^\times.$ Then we define $(x):=xA$ to be the fractional ideal generated by $x.$ We say that a fractional ideal is \textit{principal}, if it takes this form, and denote $P_A$ (or $P_K$ if $K$ is a number field) to be the set of fractional ideals.
\end{defi}
This gives us the following definition.
\begin{defi}[Class group]
    We define the \textit{ideal class group} $\op{Cl}_K:=J_K/P_K$ for a number field $K.$
\end{defi}
Note that we are requiring $K$ to be a number field here because it turns out to behave better in this case. Namely, we will show, soon, that it is finite.

Anyways, we have another definition.
\begin{defi}
    Fix $\mf a$ and $\mf b$ fractional ideals of $A.$ Then we define
    \[(\mf b:\mf a):=\{x\in K:x\mf a\subseteq\mf b\}.\]
    For example, $(A:\mf p)=\mf p^{-1}$ for all nonzero primes $\mf p.$
\end{defi}
Be careful: this notation does not mean index! This gives us the following.
\begin{lem}
    We have that $(\mf b:\mf a)=\mf b\mf a^{-1}.$
\end{lem}
\begin{proof}
    This computation is by force. Note that $x\in\mf b\mf a^{-1}$ if and only if $(x)\subseteq\mf b\mf a^{-1}$ if and only if $\mf b\mf a^{-1}\mid(x)$ if and only if $\mf b\mid(x)\mf a$ if and only if $x\mf a\subseteq\mf b$ if and only if $x\in(\mf b:\mf a).$
\end{proof}

While we're here, we pick up the Chinese remainder theorem.
\begin{thm}[Chinese remainder]
    Let $A$ be any commutative ring with ideals $I_1,\ldots,I_n$ such that $I_k+I_\ell=A$ for each $k\ne\ell.$ Then
    \[\frac{A}{I_1I_2\cdots I_n}=\prod_{k=1}^nA/I_k.\]
\end{thm}
\begin{proof}
    Omitted.
\end{proof}
Note that $\prod$ is distinct from $\bigoplus.$ This doesn't make much of a difference here because we are going to be projecting into this proof, not including.
\begin{remark}
    If $A$ is a Dedekind ring, then $I$ and $J$ are relatively prime ideals if and only if they are coprime in the sense of prime factorization. Indeed, $I+J\ne A$ if and only if there exists $\mf p\supseteq I+J$ if and only if $\mf p\supseteq I$ and $\mf p\supseteq J$ if and only if $\mf p\mid I$ and $\mf p\mid J.$
\end{remark}

\end{document}