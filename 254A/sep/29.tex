% !TEX root = ../notes.tex

















We continue with out story on localization.

\subsection{More Localization}
From last time, we recall the following.
\begin{prop}
    Fix $A$ a Dedekind domain and $S\subseteq A$ a multiplicative subset not containing $0.$ Then $S^{-1}A$ is a Dedekind domain.
\end{prop}
So we have the following.
\begin{corollary}
    Fix $S$ and $A$ as above with $\mf a$ a nonzero $A$-ideal. Then if
    \[\mf a=\prod_{\mf p}\mf p^{\alpha_\mf p},\]
    we have
    \[S^{-1}\mf a=\prod_{\mf p\cap S=\emp}\left(S^{-1}\mf p\right)^{\alpha_\mf p}.\]
\end{corollary}
\begin{proof}
    Because localization commutes with products,
    \[S^{-1}\mf a=S^{-1}\left(\prod_{\mf p}\mf p^{\alpha_\mf p}\right)=\prod_{\mf p}\left(S^{-1}\mf p\right)^{\alpha_\mf p}.\]
    If $S\cap\mf p\ne\emp,$ then $\mf p$ contains a unit of $S^{-1}A,$ so $\mf p$ dies. Otherwise, $S\cap\mf p=\emp$ retains primality, so we get the factorization
    \[S^{-1}\mf a=\prod_{\mf p\cap S=\emp}\left(S^{-1}\mf p\right)^{\alpha_\mf p}.\]
    We can also check that the factors are distinct by checking how localization behaves with the spectrum.
\end{proof}

\subsection{Local Rings}
We have the following definition.
\begin{definition}[Local]
    A ring $R$ is \textit{local} if it has a unique maximal ideal. We might say $(A,\mf m)$ is a local ring, where $\mf m$ is the unique maximal ideal of $A.$ We might also include the data of the residue field $k:=A/\mf m.$
\end{definition}
We remark that the units of a local ring $(A,\mf m)$ are the elements of $A\setminus\mf m.$ Certainly no element of a proper ideal can be a unit (multiplying by an element of $A$ would stay in the proper ideal); conversely, if $u\in A\setminus\mf m,$ then $(u)$ cannot be placed inside of a maximal ideal, so $(u)$ must not be proper, so $(u)=R,$ so $u$ is a unit.

Note that some care is needed for our definitions: localizing does not make one a local ring. 
\begin{nex}
    We see $A=\ZZ$ and $S=\{1\}$ or $S=\langle2\rangle$ do not give local rings.
\end{nex}
Regardless, here are some examples.
\begin{example}
    Take $A=\ZZ$ and $S=2\ZZ+1.$ Then $S^{-1}A$ consists of all fractions with odd denominator; here $(2)$ is the only nonzero prime ideal.
\end{example}
\begin{example}
    More generally, fix $A$ a ring wth $\mf p$ prime. Then $S=A\setminus\mf p$ is multiplicative, and we can define
    \[A_\mf p:=S^{-1}A.\]
    Here $\mf p$ is the only prime of $S^{-1}A.$
\end{example}
We note that, in the notation of uor second example, the first example is $\ZZ_{(2)}.$

Anyways, we have the following.
\begin{proposition}
    Fix $A$ a Dedekind domain with only finitely many primes; then $A$ is principal.
\end{proposition}
\begin{proof}
    From homework, we know that every ideal class has an ideal avoiding some finite set of primes. But then we can have all ideal classes avoid all of the primes at once, of which the only available ideal representative is $(1).$ So all ideal classes are $(1),$ so all ideals are principal.
\end{proof}
In particular, local rings found from localization are principal. For example, $A_\mf p$ from above is principal.

\subsection{Discrete Valuation Rings}
We have the following definition.
\begin{definition}[Discrete valuation ring]
    A \textit{discrete valuation ring} is a principal, local, entire ring, which is not a field.
\end{definition}
This is the most algebraic way to define a discrete valuation ring, but it is not the most concrete.
\begin{proposition}
    The following are equivalent.
    \begin{enumerate}[label=(\alph*)]
        \item $A$ is a principal, local, entire ring which is not a field.
        \item $A$ is a Noetherian, local, entire ring whose maximal ideal is nonzero and principal.
        \item $A$ is a local, Dedekind ring which is not a field.
    \end{enumerate}
\end{proposition}
\begin{proof}
    All of these are local, entire, Noetherian, and not a field for free. So the main work is showing that being principal is the same as maximum ideal nonzero and principal is the same as integrally closed plus dimension $1.$

    The hardest part is getting integrally closed from the other assertions. As usual, we will not do this here.
\end{proof}

We should probably talk about what it means to be a valuation.
\begin{proposition} \label{prop:forwards}
    Fix $(A,\mf m)$ a discrete valuation ring with $K$ its fraction field, and find some $\pi$ with $\mf m=(\pi).$ Then there is a unique isomorphism
    \[K^\times/A^\times\to\ZZ\]
    by $\pi\mapsto1,$ and in fact the choice of $\pi$ is irrelevant. The isomorphism is given by $\alpha\in K^\times\longmapsto n$ such that $(\alpha)=\mf m^{n}.$
\end{proposition}
\begin{proof}
    Fix $\pi$ as above. Note that $A$ being principal implies that $A$ has unique factorization, so let's classify our primes. Well, if $\rho$ is an irreducible, then $\rho\notin A^\times=A\mf m,$ so $\rho\in\mf m,$ but $(\rho)$ is prime, so $(\rho)$ is maximal, so $(\rho)=\mf m=(\pi),$ so $\rho=u\pi$ for some $u\in A^\times.$

    Thus, every nonzero $a\in A$ can be uniquely written as
    \[a=u\pi^n\]
    for some $u\in A^\times$ and $n\in\NN.$ This extends uniquely to a homomorphism $K^\times\onto\ZZ$ with kernel $A^\times,$ so we get our isomorphism $K^\times/A^\times\to\ZZ.$
\end{proof}
This gives the following more standard definition of a discrete valuation ring.
\begin{definition}[Valuation]
    Fix $A$ an entire ring with fraction field $K.$ Then a \textit{valuation} on $K$ $\nu:K^\times\to G$ is a group homomorphism to a totally ordered group $G.$ It is a valuation on $A$ if it satisfies the following.
    \begin{itemize}
        \item $\nu(a)\ge0$ for $a\in A\setminus\{0\}.$
        \item For convenience, we will append $\nu(0)=\infty>g$ for any $g\in G.$
        \item $\nu(a+b)\ge\min\{\nu(a),\nu(b)\}.$ (Note $a+b=0$ is legal by the above convention.)
    \end{itemize}
    A valuation $\nu:K^\times\to G$ is \textit{discrete} if and only if it has image isomorphic to $\ZZ.$
\end{definition}
The definition of a totally ordered group is exactly what we expect: $x\le y$ is equivalent to $xz\le yz$ if and only if $zx\le zy.$ Note that we permit nonabelian $G$ above, but in practice we will only use abelian groups.

This lets us define the following.
\begin{definition}[Valuation ring]
    The \textit{valuation ring} of a valuation $\nu:K^\times\to G$ is the set
    \[A:=\left\{\alpha\in K:\nu(\alpha)\ge0\right\}.\]
    This is a ring because it is closed under addition and multiplication by the requirements on $\nu.$ In fact, $A$ is local ring with maximal ideal
    \[\mf m:=\{\alpha\in K:\nu(\alpha)>0\}.\]
\end{definition}
In particular, we get the following.
\begin{proposition}
    A ring $A$ is a discrete valuation ring if and only if it is a valuation ring of a discrete valuation.
\end{proposition}
\begin{proof}
    The forwards direction is \autoref{prop:forwards}. In the backwards direction, we already know that $A$ is local, entire, and not a field. Then it is principal because its maximal ideal is principal (which we don't show here) and then do something funny.
\end{proof}
We'll close with the following definition.
\begin{definition}
    Fix $A$ a Dedekind domain and $K$ its field of fractions. Fixing $\mf p$ a nonzero prime ideal in $A,$ we define $\nu_\mf p:K^\times\to\ZZ$ to send $x\in K^\times$ to $\alpha_\mf p$ in the factorization
    \[(x)=\prod_\mf q\mf q^{\alpha_\mf q}.\]
\end{definition}
And now we bring things full-circle.
\begin{prop}
    It happens that $\nu_\mf p$ is always a discrete valuation, with valuation ring $A_\mf p.$ It follows $A_\mf p$ is a discrete valuation ring and that
    \[A=\bigcap_\mf pA_\mf p.\]
\end{prop}
\begin{proof}
    Omitted because of course.
\end{proof}