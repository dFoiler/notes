\documentclass[../notes.tex]{subfiles}

\begin{document}










\subsection{The ``Usual Picture''}
We recall the following definition, but we revise it this time.
\begin{defi}[Integrally closed]
    A ring is \textit{integrally closed} if and only if it is entire and integrally closed in its field of fractions.
\end{defi}
The point here is that we are forcing our integrally closed rings to be entire.
\begin{defi}[The ``usual picture,'' or the \texorpdfstring{$AKLB$}{} set-up]
    Fix $\mathcal O_K$ an integrally closed ring and $K$ its field of fractions. Further, fix $L$ a finite extension of $K,$ and we set $\mathcal O_L$ to be the integral closure of $\mathcal O_K$ in $L.$ We also add the condition that $\mathcal O_K$ is Noetherian.
    % https://q.uiver.app/?q=WzAsNCxbMCwxLCJBIl0sWzEsMSwiSyJdLFsxLDAsIkwiXSxbMCwwLCJCIl0sWzEsMiwiIiwwLHsic3R5bGUiOnsiaGVhZCI6eyJuYW1lIjoibm9uZSJ9fX1dLFswLDMsIiIsMCx7InN0eWxlIjp7ImhlYWQiOnsibmFtZSI6Im5vbmUifX19XSxbMCwxLCJcXHN1YnNldGVxIiwxLHsic3R5bGUiOnsiYm9keSI6eyJuYW1lIjoibm9uZSJ9LCJoZWFkIjp7Im5hbWUiOiJub25lIn19fV0sWzMsMiwiXFxzdWJzZXRlcSIsMSx7InN0eWxlIjp7ImJvZHkiOnsibmFtZSI6Im5vbmUifSwiaGVhZCI6eyJuYW1lIjoibm9uZSJ9fX1dXQ==
    \[\begin{tikzcd}
    	\mathcal O_L & L \\
    	\mathcal O_K & K
    	\arrow[no head, from=2-2, to=1-2]
    	\arrow[no head, from=2-1, to=1-1]
    	\arrow["\subseteq"{description}, draw=none, from=2-1, to=2-2]
    	\arrow["\subseteq"{description}, draw=none, from=1-1, to=1-2]
    \end{tikzcd}\]
    We note that we showed on Monday that $L=\op{Frac}(B).$
\end{defi}
Lots of fields we like satisfy the $AKLB$ set-up.
\begin{ex}
    Number fields satisfy the usual picture, where $K/L$ is a finite field extension, and $\mathcal O_K$ and $\mathcal O_L$ are the integral closures of $\ZZ$ in $K$ and $L,$ respectively.
\end{ex}
\begin{ex}
    Function fields $K$ finite extensions $\FF_p(t),$ where $\mathcal O_K$ is the integral closure of $\FF_p[t]$ in $K.$
\end{ex}
\begin{ex}
    We can also localize $K$ and $\mathcal O_K$ by some multiplicative set and still satisfy the usual picture. Metric completions of these localizations also work.
\end{ex}
We're goig to live in the $AKLB$ set-up for the rest of class.

\subsection{Integral Bases}
We would like to build an integral basis for $\mathcal O_L$ as an $\mathcal O_K$-module, which is an integral basis.
\begin{lem}
    There exists a basis of $L/K$ contained in $\mathcal O_L.$
\end{lem}
\begin{proof}
    Let $\omega_1,\ldots,\omega_n$ be some basis for $L/K,$ where $n:=[L:K].$ We showed on Monday that, for each $\omega_k\in L,$ there is some $a_k\in\mathcal O_K$ such that $a_k\omega_k\in B.$ So we can ``multiply out'' our denominators to get a basis $a_1\omega_1,\ldots,a_n\omega_n\subseteq\mathcal O_L.$
\end{proof}
Now, for the rest of the class, we take $A$ finite over $\ZZ,$ or of finite type as an algebra over some field. We want to show that $\mathcal O_L$ is finite over $\mathcal A,$ or is a localization.
\begin{thm}
    We show $B$ is finite over $A.$
\end{thm}
\begin{proof}
    We do this case-by-case.
    \begin{enumerate}
        \item Fix $K$ a number field with $\mathcal O_K$ the integral closure of $\ZZ$ in $K.$ We take $\omega_1,\ldots,\omega_n$ a basis for $L$ over $K,$ and we fix $d:=\op{disc}(\{\omega_1,\ldots,\omega_n\})$ so that $d\in\mathcal O_K\setminus\{0\}.$
        
        Because $d\ne0,$ we see that $\mathcal O_L\cong d\mathcal O_L$ as $\mathcal O_K$-modules and that $d\mathcal O_L\subseteq\mathcal O_L.$ We claim that
        \[d\mathcal O_L\subseteq\bigoplus_{k=1}^n\mathcal O_K\omega_k.\]
        Take any $\alpha\in\mathcal O_L$ so that we may write
        \[\alpha=a_1\omega_1+\cdots+a_n\omega_n,\qquad a_k\in K.\tag{$(*)$}\]
        Then we see that the $a_\bullet$ are the solution to the system of equations
        \[\sum_{\ell=1}^n\underbrace{\op T_K^L(\omega_k\omega_\ell)}_{\in K}x_\ell=\op T_K^L(\omega_k\alpha),\qquad k=1,\ldots,n\]
        by taking traces of $(*).$ So we use the determinant trick again. By Cram\'er's rule,\todo{wut} this will have solutions which live in $d^{-1}\mathcal O_K.$ So the claim  follows because we can write
        \[d\alpha=\sum_{k=1}^n(da_k)\omega_k\in\bigoplus_{k=1}^n\mathcal O_K\omega_k.\]
        
        Now, because $\mathcal O_K$ is Noetherian, and the above shows that $d\mathcal O_L$ is a $\mathcal O_K$-submodule, it follows that $d\mathcal O_L$ is finitely generated over $\mathcal O_K.$ So because $\mathcal O_K\cong d\mathcal O_K,$ it follows that $\mathcal O_K$ is also finitely generated over $\mathcal O_K.$
        
        \item Here $\mathcal O_K$ is finitely generated as an algebra over a field. This turns into ``finiteness of the integral closure,'' for which we reference Eisenbud to kill.
        
        \item If $\mathcal O_K$ is finitely generated as an algebra over $\ZZ$ with $p:=\op{char}K\ne0,$ then $\mathcal O_K$ is finitely generated as an algebra over $\FF_p,$ which goes to the previous case.\todo{wut}
        
        \item If $\mathcal O_K$ is a localization of one of the rings we looked at earlier, then we are still done because integral closure commutes with localization, which is something we can check by hand.
    \end{enumerate}
    There is some worry that we have not covered all cases where $\mathcal O_K$ is of finite type over $\ZZ$; positive characteristic for $K$ was done in the third case, but $\op{char}K=0$ assumed we were a number field. This turns out to be okay when we assume that $\mathcal O_K$ is finite over $\ZZ.$
\end{proof}
So we take the following definition.
\begin{defi}[Full]
    We say that a $\mathcal O_K$-submodule of $L$ is \textit{full} if it is not contained in any proper $K$-linear subspace of $L.$ In other words, it spans $L$ as a $K$-vector space.
\end{defi}
So we have the following proposition.
\begin{prop}
    Fix $M$ a nonzero finitely-generated $\mathcal O_L$-submodule of $L.$ Then $M$ is finitely generated as an $\mathcal O_K$-module and $M$ is full.
\end{prop}
\begin{proof}
    Because $M$ is finitely generated as an $\mathcal O_L$-module and $\mathcal O_L$ is finitely generated over $\mathcal O_K,$ we see that $M$ is finitely generated over $\mathcal O_K.$
    
    For the other side, we note that $\mathcal O_L$ is full as a $\mathcal O_L$-module, so it follows that $M$ is full; essentially, take an integral basis of $L/K$ and multiply all elements by some $m\in M.$
\end{proof}
So now when we continue with the class, we fix $M$ to be a full, finitely generated $\mathcal O_K$-submodule of $L.$
\begin{remark}
    This is more general the book because sometimes we will care about subrings of $\mathcal O_L$ and want integral bases for them too.
\end{remark}
Anyways, we have the following definition.
\begin{defi}[Integral basis]
    An \textit{integral basis} for $M\subseteq L$ over $\mathcal O_K$ is a set of elements $\{\omega_1,\ldots,\omega_n\}\subseteq M$ such that the $\omega_\bullet$ are a basis for $M$ as a free $\mathcal O_K$-module.
\end{defi}
These don't always exist, but we like it when they do.
\begin{prop}
    The following are equivalent.
    \begin{itemize}
        \item $M$ has an integral basis over $\mathcal O_K.$
        \item $M$ is a free $\mathcal O_K$-module.
        \item $M$ is a free $\mathcal O_K$-module of rank $n:=[L:K].$
    \end{itemize}
\end{prop}
\begin{proof}
    This is in the handout. It is, in Vojta's words, ``essentially trivial.''
\end{proof}
\begin{prop}
    If $\mathcal O_K$ is principal, then $M$ has an integral basis.
\end{prop}
\begin{proof}
    Note $M$ has no torsion because $M\subseteq L.$\todo{wut} Then we get our integral basis because $M$ is a finitely generated module over a principal ideal domain.
\end{proof}
In particular, if $\mathcal O_K=\ZZ,$ then $M$ has an integral basis, so rings of integers of number fields have an integral basis.

\end{document}