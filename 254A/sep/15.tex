\documentclass[../notes.tex]{subfiles}

\begin{document}

% !TEX root = ../notes.tex
















\subsection{Lattices}
Fix $V$ a vector space over $\RR$ of dimension $n.$ We have the following definitions.
\begin{definition}[Lattice]
    A \textit{lattice} $\Lambda\subseteq V$ is a free, finitely generated additive subgroup. In other words, we can write
    \[\Lambda=\bigoplus_{k=1}^m\ZZ v_k\]
    for $\{v_k\}_{k=1}^m$ linearly independent vectors in $V.$ We call $\{v_k\}_{k=1}^m$ a \textit{basis} for $\Lambda.$
\end{definition}
\begin{definition}[Complete]
    A lattice $\Lambda\subseteq V$ is \textit{complete} or \textit{full} if its basis spans $V.$
\end{definition}
There is also the following geometric definition of a lattice.
\begin{proposition}
    Lattices are equivalent to discrete subgroups of $V.$
\end{proposition}
\begin{proof}
    This is surprisingly technical, so we will omit it.
\end{proof}
We would like to talk geometrically about $V/\Lambda$ for lattices $\Lambda.$
\begin{definition}[Fundamental mesh]
    Given a complete lattice $\Lambda$ with basis $\{v_k\}_{k=1}^n,$ we define the set
    \[V/\Lambda:=\left\{\sum_{k=1}^n x_k v_k:x_k\in[0,1)\right\},\]
    which is the \textit{fundamental mesh} of $\Lambda.$
\end{definition}
Note that the fundamental mesh is not defined independent of a basis.
% As an example, let's actually draw out a lattice.
% \begin{center}
%     \begin{asy}
%         unitsize(1.3cm);
%         draw((0,0) -- (1,0));
%     \end{asy}
% \end{center}
% This a lattice.

We would like to talk about the relative size of $V/\Lambda,$ so we fix an additive Haar measure $\mu$ on $V$; i.e., we require that $\mu$ be translation-invariant. Equivalently, for any isomorphism $\varphi:V\to\RR^n,$ there exists a constant $c$ such that $\mu(S)=c\op{Vol}(\varphi(S))$ for any measurable $S.$ This gives the following definition.
\begin{definition}[Covolume]
    We define the \textit{covolume} of $\Lambda$ as $\mu(V/\Lambda).$
\end{definition}
Note that this is well-defined, even for different bases of $\Lambda$ because there is a change of basis matrix between our bases, which has determinant $1$ and hence yields the same $\mu(V/\Lambda).$

\subsection{Minkowski Theory}
We take the following definitions.
\begin{definition}[Convex]
    We say that a subset $X$ of $V$ is \textit{convex} if and only if $X$ contains the full line segment $\overline{AB}$ for any $A,B\in X.$
\end{definition}
\begin{definition}[Centrally symmetric]
    We say that a subset $X$ of $V$ is \textit{centrally symmetric} if and only if $v\in X$ implies that $-v\in X.$
\end{definition}
All of the best sets are convex and centrally symmetric: circles, ellipses squares, not pentagons for some reason, etc.

The point of these definitions is the following.
\begin{theorem}[Minkowski]
    Let $\Lambda$ be a complete lattice of $V,$ and let $X$ be a convex, centrally symmetric subset of $V.$ If one of the following is true, then $X$ contains a nonzero lattice point of $\Lambda.$
    \begin{enumerate}[label=(\alph*)]
        \item $\op{vol}X>2^n\op{covol}(V/\Lambda)$
        \item $\op{vol}X\ge2^n\op{covol}(V/\Lambda)$ and $X$ is compact.
    \end{enumerate}
\end{theorem}
\begin{proof}
    We show (a); (b)) follows from taking some arbitrary intersections.
    \begin{enumerate}[label=(\alph*)]
        \item The key fact is that $\op{vol}\frac12X=\frac1{2^n}\op{vol}X>\op{covol}\Lambda.$

        Fix $D$ some fundamental mesh for $\Lambda.$ By tiling the plane by $D,$ we see that
        \[\frac12X\subseteq\bigcup_{v\in\Lambda}(x+D)=V\]
        so that
        \[\frac12X=\bigcup_{v\in\Lambda}(v+D)\cap\frac12X.\]
        Taking measures, we see that
        \[\op{covol}\Lambda<\op{vol}\frac12X\le\sum_{v\in\Lambda}\op{vol}\left((v+D)\cap\frac12X\right).\]
        Our measure is translation-invariant, so we may slide each $(v+D)\cap\tfrac12X$ back along $v$ to $D\cap(\tfrac12X-v),$ implying
        \[\op{covol}\Lambda<\sum_{v\in\Lambda}D\cap(\tfrac12X-v).\]
        Now, if these $D\cap(\tfrac12X-v)$ are all disjoint, then their summed measure is less than or equal to $\op{vol}D=\op{covol}\Lambda,$ which is false!

        So we have that two translates $D\cap(\tfrac12X-v_1)$ and $D\cap(\tfrac12X-v_2)$ which intersect for two distinct vectors $v_1,v_2.$ Say $v$ is in the intersection so that $v+v_1,v+v_2\in\tfrac12X,$ implying $2v+2v_1,-2v-2v_2\in X$ because $X$ is centrally symmetric (!), so
        \[v_1-v_2=\frac{(2v+2v_1)+(-2v-2v_2)}2\in X\]
        because $X$ is convex (!). To finish, we see that $v_1-v_2\in\Lambda$ and is nonzero because $v_1\ne v_2.$
        \item Homework.
        \qedhere
    \end{enumerate}
\end{proof}
\begin{remark}
    Pay careful attention to where each condition on $X$ was used. They did each show up.
\end{remark}

\subsection{Geometry of Numbers}
Fix $K$ a number field, $n:=[K:\QQ],$ and $\mf a$ some fractional ideal of $K.$ Note that $\mf a$ has a $\ZZ$-basis of length $n$ because $c\mf a$ is an $\mathcal O_K$-ideal for some $c\in K^\times,$ and integral ideals have integral bases.

The point of this is that $\mf a$ corresponds to some complete lattice in $K\into\RR^n$; note this induces isomorphisms $K\cong\QQ^n$ and $\mf a\cong\ZZ^n.$ In a good algebraic world, we would be able to set $V:=K\otimes_\QQ\RR\cong\RR^n$ and show that $K\into V$ is a complete lattice of $V.$ This is not easy to do.

The more concrete way to do this is to consider the $n$ embeddings $\tau_1,\ldots,\tau_n$ of $K\into\CC$ which fix $\QQ.$ This gives us a map $K\into\CC^n$
\[k\longmapsto(\tau_1k,\ldots,\tau_nk).\]
Of course, $\dim_\RR\CC^n=2n$ is too big, but we can refine this. We do two things.
\begin{itemize}
    \item We start by taking $\rho_1,\ldots,\rho_r$ to be the $\tau_\bullet$ for which $\im\rho_\bullet\subseteq\RR$ (i.e., the ``real'' embeddings). This immediately kills a dimension for each $\rho_\bullet.$
    \item As for the ``complex'' embeddings, we note that each $\tau_k$ for which $\im\tau_k\not\subseteq\RR$ has a pair embedding $\overline{\tau_k}$ for which $\im\overline{\tau_k}\not\subseteq\RR.$ So all of our complex embeddings comes in these pairs; we take only one of them. Explicitly, fix $\sigma_1,\ldots,\sigma_s$ be one of the complex embeddings in each conjugate pair.
\end{itemize}
This gives us the map $j:K\into\RR^r\times\CC^s$ by
\[j:k\longmapsto(\rho_1k,\ldots,\rho_rk,\sigma_1k,\ldots,\sigma_sk).\]
We can compute that the $\RR$-dimension of $\RR^r\times\CC^s$ is $r+2s=n$ by counting our embeddings. We might abuse notation and write $j:K\into\RR^n.$
\begin{definition}[\texorpdfstring{$K_\CC$ and $K_\RR$}{}]
    Fix everything above. Then we write fix $K_\CC=\CC^n$ and $K_\RR=\RR^r\times\CC^s,$ where the \textit{canonical measure} on $K_\RR$ is $2^s$ times the standard measure on $\RR^n$ pulled back to $K_\RR.$
\end{definition}
It is not clear what this $2^s$ is doing, but it is very important.

\end{document}