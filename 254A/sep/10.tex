




\subsection{Dedekind Rings}
Recall the definition.
\begin{defi}[Dedekind]
    A \textit{Dedekind ring} is an entire, integrally closed, Noetherian ring of Krull dimension at most $1.$
\end{defi}
\begin{warn}
    Some authors require Krull dimension equal to $1$ in the above definition, preventing fields from being Dedekind.
\end{warn}
\begin{ex}
    We know $\ZZ$ is Dedekind. Also $k[t]$ for fields $k$ is Dedekind.
\end{ex}
\begin{ex}
    The localization $\ZZ_\mf p$ for a nonzero prime ideal $\mf p$ is still Dedekind.
\end{ex}
\begin{nex}
    The ring $k[x,y]$ for $k$ nor $\ZZ[x]$ are not Dedekind because they have Krull dimension $2.$ In general, $A[t]$ for any Dedekind ring $A$ (which is not a field) is not Dedekind because its Krull dimension is too big.
\end{nex}
Of course, we're really doing number theory in this class, so let's do some number theory.
\begin{thm}
    Fix $K$ a number field. Then $\mathcal O_K$ is a Dedekind ring.
\end{thm}
\begin{proof}
    We check these one at a time.
    \begin{itemize}
        \item We see $\mathcal O_K$ is integrally closed because it is the integral closure of $\ZZ,$ which must be integrally closed.
        \item We see $\mathcal O_K$ is Noetherian because $\mathcal O_K$ is finite over $\ZZ$ (shown earlier), so $\mathcal O_K$ is of finite type over $\ZZ,$ so we are done because we can realize $\mathcal O_K$ is smaller than some polynomial ring over $\ZZ,$ finishing by the Hilbert basis theorem.
        \item We lastly check that $\mathcal O_K$ has Krull dimension $1.$
        
        Let $\mf p$ be a nonzero prime ideal, which we want to show is maximal.
        % Then we see that $\mf p\cap\ZZ$ is a prime ideal: it is not all of $\ZZ$ because $1\notin\mf p,$ and $ab\in\mf p\cap\ZZ$ implies $a\in\mf p$ or $b\in\mf p,$ so $a\in\mf p\cap\ZZ$ or $b\in\mf p\cap\ZZ$; lastly, $\mf p\cap\ZZ$ is nonzero because it has the norm of some nonzero element in $\mf p.$
        % Now let $\mf p\cap\ZZ=(p)$ for a nonzero prime integer $p\in\ZZ.$
        We note that the index $[\mathcal O_K:\mf p]$ is finite by \autoref{prop:212}\footnote{Both $\mathcal O_K$ and $\mf p$ are finitely generated $\mathcal O_K$-modules, so they are full. It follows that the discriminant of these is well-defined and nonzero, from which \autoref{prop:212} gives the finiteness.}, but $\mathcal O_K/\mf p$ is at least an integral domain, so being finite forces $\mathcal O_K/\mf p$ to be a field, implying that $\mf p$ is a maximal ideal, finishing.
        \qedhere
    \end{itemize}
\end{proof}
\begin{remark}
    In general, in the $ALKB$ set-up, if $A$ is Dedekind, then $B$ is also Dedekind. This is in Eisenbud.
\end{remark}
We take the following definition.
\begin{defi}[Ideal operations]
    Fix $A$ an entire ring with fraction field $K.$ Further fix $I$ and $J$ be $A$-submodules of $K.$ Then we define
    \[I+J:=\{a+b:a\in I,b\in J\}\]
    and
    \[IJ=\left\{\sum_{k=1}^na_kb_k:n\in\ZZ,\{a_k\}_{k=1}^n\subseteq I,\{b_k\}_{k=1}^n\subseteq J\right\}.\]
    In other words, $I+J$ is the set of all sums, and $IJ$ is the submodule generated by products.
\end{defi}
Note that we have called the above definition ``ideal operations'' but have defined them more generally for $A$-submodules of $K.$
\begin{lem}[Prime avoidance]
    Fix $A$ a ring. Fix $I_1,\ldots,I_r$ and $\mf p$ be $A$-ideals with $\mf p$ prime. If $I_1\cdots I_r\subseteq\mf p,$ then there is an $I_k$ with $I_k\subseteq\mf p.$
\end{lem}
\begin{proof}
    The idea is to show the contrapositive: if $I_k\not\subseteq\mf p$ for each $k,$ then each $I_k$ has some $x_k\in I_k\setminus\mf p.$ Then we see that $x_1\cdots x_r\notin\mf p$ (otherwise, there would be some $x_k\in\mf p$), so $x_1\cdots x_r\in I_1\cdots I_r\setminus\mf p,$ finishing.
\end{proof}
For the rest of class, we take $A$ a Dedekind ring and $K$ its fraction field.
\begin{thm}[Unique prime facotrization of ideals]
    Every nonzero $A$-ideal $I$ has a prime factorization
    \[I=\prod_{k=1}^r\mf p_k,\]
    where the $\mf p_\bullet.$ The primes $\{\mf p_k\}_{k=1}^n$ is unique up to permutation.
\end{thm}
\begin{remark}
    This proof is similar to the proof that all PIDs are UFDs.
\end{remark}
We pick up the following lemma, which essentially sets up our ``induction.''
\begin{lem}
    Every nonzero $A$-ideal $I$ contains a product of noonzero prime ideals.
\end{lem}
\begin{proof}
    Suppose for the sake of contradiction that this is false; then the set of coutnerexamples is nonempty collection of ideals and hence contains a maximal element because $A$ is Dedekind and hence Noetherian. Let $I$ be such a maximal element.
    
    Certainly $I\ne A$ because then it contains the empty product of primes; also, $I$ itself is not prime, for then $I$ would contain itself. So $I,$ being not prime, promises elements $a,b\in A\setminus I$ and $ab\in I.$ However, we know set
    \[\mf a=I+(a)\qquad\text{and}\qquad\mf b=I+(b).\]
    By maximality, $\mf a$ and $\mf b$ each contain a product of primes. But $I$ also contains $\mf a\mf b=I+aI+bI+(ab)\subseteq I,$ so $I$ contains the product inside of $\mf a$ times the product inside of $\mf a.$ This finishes.
\end{proof}
With this in mind, we would next like to be able to divide out by ideals.
\begin{defi}[Inverse ideals]
    Fix $I$ a nonzero $A$-ideal. Then we define
    \[I^{-1}=\{x\in K:xI\subseteq A\}.\]
    We can check by hand that $I^{-1}$ is an $A$-submodule of $K,$ containing $A.$
\end{defi}
This gives us the following lemma.
\begin{lem}
    Fix $\mf p$ a a nonzero prime ideal and $I$ a nonzero $A$-ideal. Then $I\mf p^{-1}\supsetneq I.$
\end{lem}
\begin{proof}
    We start by showing that $\mf p^{-1}\ne A,$ which is equivalent to $\mf p^{-1}\supsetneq A.$ Indeed, pick up $a\in\mf p\setminus\{0\},$ and use the previous lemma to find primes $\{\mf p_k\}_{k=1}^r$ whose product is contained in $a.$ In fact, by well-ordering, we may take $r$ as small as possible. It follows that
    \[\mf p_1\cdots\mf p_r\subseteq\mf p,\]
    so we have that $\mf p_k\subseteq\mf p$ for one of the $\mf p_k$; without loss of generality, take $\mf p_r=\mf p_1,$ and we see that $\mf p_r=\mf p$ by maximality. Because $r$ is minimal, we have that $\mf p_1\cdots\mf p_{r-1}\not\subseteq(a),$ so we are promised $b\in(a)\setminus\mf p_1\cdots\mf p_{r-1}.$ Now, $b^{-1}a\notin A,$ but $b\mf p=b\mf p_1\subseteq\mf p_1\cdots\mf p_r=(a)$ implies that $a^{-1}b\mf p\subseteq A.$ So $a^{-1}b\in A\setminus\mf p^{-1},$ finishing.
    
    We now return to actually proving the lemma. We pick up our promised $x\in\mf p^{-1}\setminus A.$ Suppose for the sake of contradiction $I\mf p^{-1}\subseteq I.$ Then $xI\subseteq\mf p^{-1}I\subseteq I.$ Then we see that $xI$ is a faithful $A[x]$-submodule over $K$ and is finitely generated over $A$ because $I$ is faithful and finitely generated over $A.$ It follows that $x$ is integral over $A$ (it's generating a finitely generated $A$-module), forcing $x\in A,$ which is a contradiction.
\end{proof}
\begin{remark}
    We just used the fact that $A$ is integrally closed above!
\end{remark}
We'll continue with the proof of existence next class.