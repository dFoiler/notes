% !TEX root = ../notes.tex

















The Minkowski grind continues. Fix $K$ a number field, as usual.

\subsection{Setting up the Diagram}
We want to study $\mathcal O_K^\times.$ We use the following commutative diagram.
% https://q.uiver.app/?q=WzAsNixbMCwwLCJLXlxcdGltZXMiXSxbMSwwLCJLX3tcXG1hdGhiYiBDfV5cXHRpbWVzIl0sWzIsMCwiXFxwcm9kX1xcdGF1XFxtYXRoYmIgUiJdLFswLDEsIlxcbWF0aGJiIFFeXFx0aW1lcyJdLFsxLDEsIlxcbWF0aGJiIENeXFx0aW1lcyJdLFsyLDEsIlxcbWF0aGJiIFIiXSxbMCwxLCJqIl0sWzEsMiwiXFxlbGwiXSxbMCwzLCJcXG9wZXJhdG9ybmFtZSBOX3tcXG1hdGhiYiBRfV5LIiwyXSxbMSw0LCJcXG9wZXJhdG9ybmFtZSBOIiwyXSxbMyw0LCIiLDEseyJzdHlsZSI6eyJ0YWlsIjp7Im5hbWUiOiJob29rIiwic2lkZSI6InRvcCJ9fX1dLFsyLDUsIlxcb3BlcmF0b3JuYW1lIFQiXSxbNCw1LCJ6XFxtYXBzdG8tXFxsb2d8enwiXV0=
\[\begin{tikzcd}
	{K^\times} & {K_{\mathbb C}^\times} & {\prod_\tau\mathbb R} \\
	{\mathbb Q^\times} & {\mathbb C^\times} & {\mathbb R}
	\arrow["j", from=1-1, to=1-2]
	\arrow["\ell", from=1-2, to=1-3]
	\arrow["{\operatorname N_{\mathbb Q}^K}"', from=1-1, to=2-1]
	\arrow["{\operatorname N}"', from=1-2, to=2-2]
	\arrow[hook, from=2-1, to=2-2]
	\arrow["{\operatorname{Tr}}", from=1-3, to=2-3]
	\arrow["{z\mapsto-\log|z|}", from=2-2, to=2-3]
\end{tikzcd}\]
Here, $j$ is the restriction of $K\into K_\mathcal C,$ by
\[h=(\tau_1,\ldots,\tau_n)=(\rho_1,\ldots,\rho_r,\sigma_1,\overline{\sigma_1},\ldots,\sigma_s,\overline{\sigma_2}).\]
The new map here is $\ell,$ which takes $z\mapsto-\log|z|$ to each coordinate.
\begin{remark}
    As abuse of notation, given $f:A\to B$ and $g:C\to D,$ we write $f\times g:A\times C\to B\times D.$ Professor Vojta is making a pretty big deal of this.
\end{remark}
Continuing, $\op N$ takes a tuple in $K_\CC^\times$ to the product of the coordinates, so that $K^\times\to\QQ^\times\into\CC^\times$ commutes. Lastly, the trace $\op{Tr}:\prod_\tau\RR\to\RR$ takes the sum of coordinates.

Lots of these groups turn out to have an involution, some of which are interesting. We name all of these $F$ by abuse of notation.
\begin{itemize}
    \item On $K^\times,\QQ^\times,$ and $\RR,$ we choose the identity map.
    \item On $\CC^\times,$ we choose conjugation $z\mapsto\overline z.$
    \item On $K_\CC^\times,$ we conjugate the $\rho$ components and conjugate and swap the complex components. In other words, we take
    \[\begin{tikzcd}
        {\rho_1} & \cdots & {\rho_r} & {\sigma_1} & {\overline{\sigma_1}} & \cdots & {\sigma_s} & {\overline{\sigma_s}} \\
        \\
        {\rho_1} & \cdots & {\rho_r} & {\sigma_1} & {\overline{\sigma_1}} & \cdots & {\sigma_s} & {\overline{\sigma_s}}
        \arrow[from=1-1, to=3-1]
        \arrow[from=1-3, to=3-3]
        \arrow[from=1-5, to=3-4]
        \arrow[from=1-4, to=3-5]
        \arrow[from=1-8, to=3-7]
        \arrow[from=1-7, to=3-8]
    \end{tikzcd}\]
    where each vertical line downwards is complex conjugation.
    \item On $\prod_\tau\RR,$ we do nothing on the $\rho$ components and swap the $\sigma$ components. In other words, we take
    \[\begin{tikzcd}
        {\rho_1} & \cdots & {\rho_r} & {\sigma_1} & {\overline{\sigma_1}} & \cdots & {\sigma_s} & {\overline{\sigma_s}} \\
        \\
        {\rho_1} & \cdots & {\rho_r} & {\sigma_1} & {\overline{\sigma_1}} & \cdots & {\sigma_s} & {\overline{\sigma_s}}
        \arrow[from=1-1, to=3-1]
        \arrow[from=1-3, to=3-3]
        \arrow[from=1-5, to=3-4]
        \arrow[from=1-4, to=3-5]
        \arrow[from=1-8, to=3-7]
        \arrow[from=1-7, to=3-8]
    \end{tikzcd}\]
    where each vertical line downwards is the identity. (Technically, we could make this complex conjugation, which does nothing on $\RR.$)
\end{itemize}
We can check that each arrow in our commutative diagram preserves the $F$-invariants; in fact, each of these maps commutes with $F.$
\begin{itemize}
    \item The embedding $\QQ^\times\into\CC^\times$ does nothing.
    \item The map $\CC^\times\to\RR,$ complex conjugation does nothing to $z\mapsto-\log|z|.$
    \item The map $K^\times\to K_\CC^\times,$ we see that $F$ does nothing to the $\rho$ coordinates, and the output of $j$ has the $\sigma_\bullet$ and $\overline{\sigma_\bullet}$ coordinates conjugated.
    \item The map $K_\CC^\times\to\prod_\tau\RR$ doesn't care about the conjugation happening in $K_\CC^\times,$ but the swapping in $K_\CC^\times$ is preserved in $\prod_\tau\RR$ by construction of $F$ on $\prod_\tau\RR.$
    \item The vertical map $K^\times\to\QQ^\times$ is okay because $F$ acts trivially on both.
    \item The map $K_\CC^\times\to\CC^\times$ is okay because, when taking the product of all elements, we can conjugate before in $K_\CC^\times$ or after in $\CC^\times.$ (The swapping does not matter because we are just taking the product of everything.)
    \item Lastly, the map $\prod_\tau\RR\to\RR$ we see that $F$ merely permutes the coordinates of $\prod_\tau\RR,$ which $\op{Tr}$ does not care about when taking the sum of everyone. So we are still safe.
\end{itemize}

\subsection{Expanding the Diagram}
So, taking the $F$ invariants $X\mapsto X^F$ of our commutative diagram gives another commutative diagram, as follows.
% https://q.uiver.app/?q=WzAsNixbMCwwLCJLXlxcdGltZXMiXSxbMSwwLCJLX3tcXG1hdGhiYiBSfV5cXHRpbWVzIl0sWzIsMCwiKFxccHJvZF9cXHRhdVxcbWF0aGJiIFIpXkYiXSxbMCwxLCJcXG1hdGhiYiBRXlxcdGltZXMiXSxbMSwxLCJcXG1hdGhiYiBSXlxcdGltZXMiXSxbMiwxLCJcXG1hdGhiYiBSIl0sWzAsMSwiaiJdLFsxLDIsIlxcZWxsIl0sWzAsMywiXFxvcGVyYXRvcm5hbWUgTl97XFxtYXRoYmIgUX1eSyIsMl0sWzEsNCwiXFxvcGVyYXRvcm5hbWUgTiIsMl0sWzMsNCwiIiwxLHsic3R5bGUiOnsidGFpbCI6eyJuYW1lIjoiaG9vayIsInNpZGUiOiJ0b3AifX19XSxbMiw1LCJcXG9wZXJhdG9ybmFtZXtUcn0iXSxbNCw1LCJ6XFxtYXBzdG8tXFxsb2d8enwiXV0=
\[\begin{tikzcd}
	{K^\times} & {K_{\mathbb R}^\times} & {(\prod_\tau\mathbb R)^F} \\
	{\mathbb Q^\times} & {\mathbb R^\times} & {\mathbb R}
	\arrow["j", from=1-1, to=1-2]
	\arrow["\ell", from=1-2, to=1-3]
	\arrow["{\operatorname N_{\mathbb Q}^K}"', from=1-1, to=2-1]
	\arrow["{\operatorname N}"', from=1-2, to=2-2]
	\arrow[hook, from=2-1, to=2-2]
	\arrow["{\operatorname{Tr}}", from=1-3, to=2-3]
	\arrow["{z\mapsto-\log|z|}", from=2-2, to=2-3]
\end{tikzcd}\]
Namely, we recall that the $F$-invariants of $K_\CC$ were in (natural) bijection with $K_\RR,$\footnote{In fact, arguably we should define $K_\RR$ as the $F$-invariants of $K_\CC.$} and the $F$-invariants of $\CC$ are real numbers. We remark that
\[\left(\prod_\tau\RR\right)^F=\{(t_1,\ldots,t_r,u_1,u_1,\ldots,u_s,u_s):t_\bullet,u_\bullet\in\RR\}\]
is isomorphic to $\RR^{r+s}$ by
\[(t_1,\ldots,t_r,u_1,u_1,\ldots,u_s,u_s)\mapsto (t_1,\ldots,t_r,2u_1,\ldots,2u_s).\]
Note that we are doubling here so that the above map commutes with the trace (i.e., the sum of the coordinates is preserved). To be explicit, we write that the composite $K_\RR^\times\to(\prod_\tau\RR)^F\to\RR$ is
\[(x_{\rho_1},\ldots,x_{\rho_r},x_{\sigma_1},x_{\overline{\sigma_1}},\ldots,x_{\overline{\sigma_s}},x_{\overline{\sigma_s}})\mapsto\left(-\log|x_{\rho_1}|,\ldots,-\log|x_{\rho_r}|,-\log|x_{\sigma_1}|^2,\ldots,-\log|x_{\sigma_s}|^2\right).\]

Now, to focus on $\mathcal O_K,$ we restrict $K^\times$ to $\mathcal O_K^\times$ as follows.
% https://q.uiver.app/?q=WzAsOSxbMCwxLCJLXlxcdGltZXMiXSxbMSwxLCJLX3tcXG1hdGhiYiBSfV5cXHRpbWVzIl0sWzIsMSwiKFxccHJvZF9cXHRhdVxcbWF0aGJiIFIpXkYiXSxbMCwyLCJcXG1hdGhiYiBRXlxcdGltZXMiXSxbMSwyLCJcXG1hdGhiYiBSXlxcdGltZXMiXSxbMiwyLCJcXG1hdGhiYiBSIl0sWzAsMCwiXFxtYXRoY2FsIE9fSyJdLFsxLDAsIlMiXSxbMiwwLCJIIl0sWzAsMSwiaiJdLFsxLDIsIlxcZWxsIl0sWzAsMywiXFxvcGVyYXRvcm5hbWUgTl97XFxtYXRoYmIgUX1eSyIsMl0sWzEsNCwiXFxvcGVyYXRvcm5hbWUgTiIsMl0sWzMsNCwiIiwxLHsic3R5bGUiOnsidGFpbCI6eyJuYW1lIjoiaG9vayIsInNpZGUiOiJ0b3AifX19XSxbMiw1LCJcXG9wZXJhdG9ybmFtZXtUcn0iXSxbNCw1LCJ6XFxtYXBzdG8tXFxsb2d8enwiXSxbNiwwLCJcXHN1YnNldGVxIiwxLHsic3R5bGUiOnsiYm9keSI6eyJuYW1lIjoibm9uZSJ9LCJoZWFkIjp7Im5hbWUiOiJub25lIn19fV0sWzcsMSwiXFxzdWJzZXRlcSIsMSx7InN0eWxlIjp7ImJvZHkiOnsibmFtZSI6Im5vbmUifSwiaGVhZCI6eyJuYW1lIjoibm9uZSJ9fX1dLFs4LDIsIlxcc3Vic2V0ZXEiLDEseyJzdHlsZSI6eyJib2R5Ijp7Im5hbWUiOiJub25lIn0sImhlYWQiOnsibmFtZSI6Im5vbmUifX19XSxbNiw4LCJcXGxhbWJkYSIsMSx7ImN1cnZlIjotMiwic3R5bGUiOnsiYm9keSI6eyJuYW1lIjoiZGFzaGVkIn19fV0sWzYsN10sWzcsOF1d
\[\begin{tikzcd}
	{\mathcal O_K} & S & H \\
	{K^\times} & {K_{\mathbb R}^\times} & {(\prod_\tau\mathbb R)^F} \\
	{\mathbb Q^\times} & {\mathbb R^\times} & {\mathbb R}
	\arrow["j", from=2-1, to=2-2]
	\arrow["\ell", from=2-2, to=2-3]
	\arrow["{\operatorname N_{\mathbb Q}^K}"', from=2-1, to=3-1]
	\arrow["{\operatorname N}"', from=2-2, to=3-2]
	\arrow[hook, from=3-1, to=3-2]
	\arrow["{\operatorname{Tr}}", from=2-3, to=3-3]
	\arrow["{z\mapsto-\log|z|}", from=3-2, to=3-3]
	\arrow["\subseteq"{description}, draw=none, from=1-1, to=2-1]
	\arrow["\subseteq"{description}, draw=none, from=1-2, to=2-2]
	\arrow["\subseteq"{description}, draw=none, from=1-3, to=2-3]
	\arrow["\lambda"{description}, curve={height=-12pt}, dashed, from=1-1, to=1-3]
	\arrow[from=1-1, to=1-2]
	\arrow[from=1-2, to=1-3]
\end{tikzcd}\]
Here, we define
\begin{align*}
    S :&= \{y\in K_\RR^\times:\op N(y)=\pm1\}. \\
    H :&= \left\{(x_\tau)\in\left(\prod_\tau\RR\right)^F:\op{Tr}(x_\tau)=0\right\}.
\end{align*}
We might call $S$ the ``norm $\pm1$ hypersurface'' and $H$ the ``trace $0$ hyperplane.'' And indeed, $j:K^\times\to K_\RR^\times$ does indeed take $\mathcal O_K^\times$ to the elements of norm $1$ in $K_\RR^\times,$ so we can restrict $j$ to $\mathcal O_K\to S.$ Similarly, $S$ upon taking logs will map straight to the origin.

\subsection{Doing Number Theory}
It will happen that $\mathcal O_K^\times$ maps into a lattice $\Gamma:=\im\lambda.$ So let's actually start doing some number theory.
\begin{definition}[Roots of unity]
    Fix $\mu_K$ to be the roots of unity in $K.$ Note that $\mu_K\subseteq\mathcal O_K^\times.$
\end{definition}
\begin{lemma}
    We claim that $\ker\lambda=\mu_K.$
\end{lemma}
\begin{proof}
    In one direction, suppose $\zeta\in\mu_K.$ Then we as remark have $\zeta\in\mathcal O_K$ and in fact $\zeta^{-1}\in\mathcal O_K.$ However, $H$ is torsion-free, so we may write
    \[\lambda(\zeta)=\frac1m\lambda(\zeta^m)=\frac1m\lambda(1)=0.\]
    (Here, $\lambda(1)=0$ because surely the identity lives in the kernel.)
    
    In the other direction, suppose $\zeta\in\ker\lambda.$ Well, then we have that $|\tau\zeta|=1$ for each $\tau\in\op{Hom}(K,\CC),$ which turns out to be a problem. Indeed, we have the following.
    \begin{lemma}
        We claim that
        \[\{\alpha\in\mathcal O_K:|\tau\alpha|\le c_\tau\text{ for all }\tau\}\]
        is a finite set for any sequence $\{c_\tau\}$ of positive real numbers.
    \end{lemma}
    \begin{proof}
        Note that $j(\mathcal O_K)$ is a discrete subgroup of $K_\RR$ as when we studied additive Minkowski theory, so $j(\mathcal O_K)$ is a lattice and therefore contains only finitely many elements in the given bounded set.
    \end{proof}
    It follows that
    \[\ker\lambda\subseteq\{\alpha\in\mathcal O_K:|\tau\alpha|\le1\text{ for all }\tau\}\]
    is finite, so $\ker\lambda$ is finite, so all its elements have finite order.
\end{proof}
\begin{remark}
    We note that the above proof also tells us that $\mu_K$ is finite, for free.
\end{remark}
So now that we have the above, we get the following short exact sequence.
\[1\to\mu_K\to\mathcal O_K\to\Gamma\to1\]
Next time we will show that $\Gamma$ is a complete lattice in $H.$ For now, we'll show it's a lattice.
\begin{lemma}
    We have that $\Gamma$ is discrete in $H.$
\end{lemma}
\begin{proof}
    Our lemma earlier showed that
    \[\{\alpha\in\mathcal O_K:|\tau\alpha|\le c\text{ for all }\tau\}\]
    is finite for any given $c>0.$ Transporting this through $\lambda,$ we get that
    \[\{(x_1,\ldots,x_{r+s})\in\RR^{r+s}:|x_k|<c\text{ for all }k\}\]
    is still finite because the kernel we just exhibited was finite.
\end{proof}
\begin{lemma}
    For any given $a\in\ZZ,$ there are only finitely many $\alpha\in\mathcal O_K$ such that $\op N_\QQ^K(\alpha)=a,$ up to multiplication by a unit.
\end{lemma}
Here we need the last caveat because there might be infinitely many units.
\begin{proof}
    We note that, given $\alpha,\beta\in\mathcal O_K^\times,$ we see $\alpha/\beta$ is a unit if and only if $(\alpha)=(\beta),$ so we get the result by noting there are only finitely many ideals of $\mathcal O_K$ with norm $|a|.$ In other words, we are counting $\alpha$ with $\op N_\QQ^K(\alpha)=a,$ up to unit, by checking the ideal they generate, of which we know there are finitely many.
\end{proof}