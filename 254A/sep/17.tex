% !TEX root = ../notes.tex

















For today's lecture, $K$ is a number field with $\mathcal O_K$ its ring of integers.

\subsection{Motivating the Minkowski Measure}
Recall that we have the map $K\into\CC^n=:K_\CC$ by $k\mapsto(\tau_1k,\ldots,\tau_nk)$ for each of our embeddings $\tau_\bullet$ as well as $j:K\into K_\RR$ by
\[k\longmapsto(\rho_1k,\ldots,\rho_rk,\sigma_1k,\ldots,\sigma_sk).\]
By Galois theory, the points fixed by all the embeddings are those in $K.$ Anyways, here is our diagram.
% https://q.uiver.app/?q=WzAsMyxbMCwwLCJLIl0sWzIsMCwiS19cXENDIl0sWzEsMiwiS19cXFJSIl0sWzAsMiwiKFxccmhvX1xcYnVsbGV0LFxcc2lnbWFfXFxidWxsZXQpIiwwLHsic3R5bGUiOnsidGFpbCI6eyJuYW1lIjoiaG9vayIsInNpZGUiOiJ0b3AifX19XSxbMiwxLCIiLDAseyJzdHlsZSI6eyJ0YWlsIjp7Im5hbWUiOiJob29rIiwic2lkZSI6InRvcCJ9fX1dLFswLDEsIihcXHRhdV9cXGJ1bGxldCkiLDIseyJzdHlsZSI6eyJ0YWlsIjp7Im5hbWUiOiJob29rIiwic2lkZSI6InRvcCJ9fX1dXQ==
\[\begin{tikzcd}
	K && {K_\CC} \\
	\\
	& {K_\RR}
	\arrow["{(\rho_\bullet,\sigma_\bullet)}", hook, from=1-1, to=3-2]
	\arrow[hook, from=3-2, to=1-3]
	\arrow["{(\tau_\bullet)}"', hook, from=1-1, to=1-3]
\end{tikzcd}\]
We recall that the canonical measure on $K_\RR$ is the one obtained by pulling the measure on $\RR^n$ back to $K_\RR$ and multiplying by $2$ on each complex coordinate. We also define $K_\CC=\CC^n$ with a ``conjugation'' symmetry $F:K_\CC\to K_\CC$ to do the following.
% https://q.uiver.app/?q=WzAsMTYsWzAsMCwiXFxyaG9fMSJdLFsxLDAsIlxcY2RvdHMiXSxbMiwwLCJcXHJob19yIl0sWzMsMCwiXFxzaWdtYV8xIl0sWzQsMCwiXFxvdmVybGluZXtcXHNpZ21hXzF9Il0sWzUsMCwiXFxjZG90cyJdLFs2LDAsIlxcc2lnbWFfcyJdLFs3LDAsIlxcb3ZlcmxpbmV7XFxzaWdtYV9zfSJdLFswLDIsIlxccmhvXzEiXSxbMSwyLCJcXGNkb3RzIl0sWzIsMiwiXFxyaG9fciJdLFszLDIsIlxcc2lnbWFfMSJdLFs0LDIsIlxcb3ZlcmxpbmV7XFxzaWdtYV8xfSJdLFs1LDIsIlxcY2RvdHMiXSxbNiwyLCJcXHNpZ21hX3MiXSxbNywyLCJcXG92ZXJsaW5le1xcc2lnbWFfc30iXSxbMCw4XSxbMiwxMF0sWzQsMTFdLFszLDEyXSxbNywxNF0sWzYsMTVdXQ==
\[\begin{tikzcd}
	{\rho_1} & \cdots & {\rho_r} & {\sigma_1} & {\overline{\sigma_1}} & \cdots & {\sigma_s} & {\overline{\sigma_s}} \\
	\\
	{\rho_1} & \cdots & {\rho_r} & {\sigma_1} & {\overline{\sigma_1}} & \cdots & {\sigma_s} & {\overline{\sigma_s}}
	\arrow[from=1-1, to=3-1]
	\arrow[from=1-3, to=3-3]
	\arrow[from=1-5, to=3-4]
	\arrow[from=1-4, to=3-5]
	\arrow[from=1-8, to=3-7]
	\arrow[from=1-7, to=3-8]
\end{tikzcd}\]
Each arrow here is complex conjugation; the point is that we want to swap the pairs of complex embeddings because they ``should'' be conjugate on $K.$

Let's try to motivate the multiplications by $2$ in our canonical measure.
\begin{lemma} \label{lem:detisvol}
    Fix $V$ a real vector space of dimension $n$ with $\mu$ a Haar measure $\langle\cdot,\cdot\rangle$ a non-degenerate inner product. Further, fix $\alpha_1,\ldots,\alpha_n$ a basis for $V/\RR$ such that the measure of the ``unit cube''
    \[\mu\left(\left\{\sum_{k=1}^n t_k\alpha_k:t_\bullet\in[0,1]\right\}\right)=\left|\det\begin{bmatrix}
        \langle\alpha_1,\alpha_1\rangle & \cdots & \langle\alpha_1,\alpha_n\rangle \\
        \vdots & \ddots & \vdots \\
        \langle\alpha_n,\alpha_1\rangle & \cdots & \langle\alpha_n,\alpha_n\rangle
    \end{bmatrix}\right|^{1/2}.\tag{$*$}\]
    Then $(*)$ holds for all $\alpha_1,\ldots,\alpha_n$ whether or not these are a basis.
\end{lemma}
\begin{proof}
    Let $\beta_1,\ldots,\beta_n$ be arbitrary elements of $V.$ If not a basis, then both sides of $(*)$ vanish. Otherwise, we fix $B$ such that
    \[\begin{bmatrix}\alpha_1 \\ \vdots \\ \alpha_n\end{bmatrix} = B\begin{bmatrix}\beta_1 \\ \vdots \\ \beta_n \end{bmatrix}.\]
    Then, staring at $(*),$ we see that changing bases from the $\alpha_\bullet$ to $\beta_\bullet$ multiplies the right-hand side by $|\det B|.$ On the other side, we see that
    \[B\begin{bmatrix}
        \langle\alpha_1,\alpha_1\rangle & \cdots & \langle\alpha_1,\alpha_n\rangle \\
        \vdots & \ddots & \vdots \\
        \langle\alpha_n,\alpha_1\rangle & \cdots & \langle\alpha_n,\alpha_n\rangle
    \end{bmatrix}B^\intercal=\begin{bmatrix}
        \langle\beta_1,\beta_1\rangle & \cdots & \langle\beta_1,\beta_n\rangle \\
        \vdots & \ddots & \vdots \\
        \langle\beta_n,\beta_1\rangle & \cdots & \langle\beta_n,\beta_n\rangle
    \end{bmatrix},\]
    so the right-hand side is also multiplied by $\sqrt{(\det B)^2}.$
\end{proof}
\begin{example}
    Take $V=\RR^n$ with $\mu$ the standard measure and $\alpha_1,\ldots,\alpha_n$ our standard basis. Then both sides of $(*)$ are $1,$ so $(*)$ will always hold.
\end{example}
\autoref{lem:detisvol} justifies the multiplication by $2$ in our canonical measure. Indeed, looking at the standard inner product
\[\langle (z_1,\ldots,z_n),(w_1,\ldots,w_n)\rangle=\sum_{k=1}^nz_k\overline{w_k}.\]
So now we embed $K_\RR$ into $K_\CC,$ which is by
\[(x_1,\ldots,x_r,z_1,\ldots,z_s)\longmapsto(x_1,\ldots,x_r,z_1,\overline{z_1},\ldots,z_s,\overline{z_s}).\]
For concreteness, we look at a single complex coordinate: here we take $\CC\into\CC^2$ by $z\mapsto(z,\overline z).$ Then the inner product over at $\CC^2$ looks like
\[\langle(z,\overline z),(w,\overline w)\rangle=z\overline w+\overline zw=2\op{Re}(z\overline w).\]
In particular, $z=x+iy$ and $w=u+iv$ gives $2\op{Re}(z\overline w)=2(xu+yv),$ which is twice the standard inner product we want when taking $x+iy\mapsto(x,y)$ as our embedding of $\CC\to\RR^2.$

In total, we see that the pull-back of the inner product on $K_\CC\cong\CC^n$ to $K_\RR$ by defining $K_\RR$ to be the $F$-invariants in $K_\CC,$ we get the standard inner product on $\RR^n$ where each of the complex embeddings gets multiplied by $2.$ This is our canonical measure.
\begin{remark}
    Note that we are using inner products as our interface with our measure instead of trying to pull back the measure from $K_\CC$ directly. This is because the measure of $K_\RR$ in $K_\CC$ is zero.
\end{remark}

\subsection{More Geometry of Numbers}
Now let's do some theory.
\begin{proposition}
    Fix $j:K\into K_\RR$ as above. Further, let $\mf a$ be a nonzero $\mathcal O_K$-ideal. Then $j(\mf a)$ is a complete lattice of $K_\RR$ with covolume $\sqrt{|d_K|}\cdot[\mathcal O_K:\mf a],$ with respect to the canonical measure of $K_\RR.$
\end{proposition}
\begin{proof}
    Fix $\alpha_1,\ldots,\alpha_n$ an integral basis for $\mf a$ over $\ZZ.$ Then we fix $M$ to be the matrix defining the discriminant of $\mf a$; because the discriminant of $\mf a$ is nonzero, this matrix $M$ has nonzero determinant, so $j(\mf a)$ will be a full lattice of $K_\RR.$

    It follows
    \begin{align*}
        \op{covol}j(\mf a) &= \mu\left\{\sum_{k=1}^n t_k j(\alpha_k):t_\bullet\in[0,1]\right\} \\
        &\stackrel{(*)}=\left|\det\begin{bmatrix}
            \langle j\alpha_1,j\alpha_1\rangle & \cdots & \langle j\alpha_1,j\alpha_n\rangle \\
            \vdots & \ddots & \vdots \\
            \langle j\alpha_n,j\alpha_1\rangle & \cdots & \langle j\alpha_n,j\alpha_n\rangle
        \end{bmatrix}\right|^{1/2} \\
        &= \left|\det\left[\sum_{m=1}^n\tau_m\alpha_k\cdot\overline{\tau_m\alpha_\ell}\right]_{k,\ell}\right|^{1/2} \\
        &= |\det(M^\intercal \overline M)|^{1/2} \\
        &= |d(\mf a)|^{1/2} \\
        &= \sqrt{|d_k|}\cdot[\mathcal O_K:\mf a].
    \end{align*}
    There are lots of details to fill in here, but I can't be bothered to try to figure out what's going on.
\end{proof}
\begin{remark}
    The above holds more generally: $\op{covol}j(\mf a)=|d(\mf a)|^{1/2}$ for any full $\ZZ$-submodule $\mf a$ of $\mathcal O_K$ as well as any full $\ZZ$-submodule of $K$ because any full $\ZZ$-submodule we can lift to a submodule of $\mathcal O_K$ by multiplying by some integer $a$ first.
\end{remark}
On Monday, we will use this to show finiteness of the class group.