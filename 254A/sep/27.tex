% !TEX root = ../notes.tex












Today we localize.

\subsection{Setting Up Localization}
Throughout today's class, we take $A$ a (commutative) ring (with identity) where $S$ is a multiplicative subset. What does multiplicative mean?
\begin{definition}[Multiplicative]
    A subset $S\subseteq A$ is \textit{multiplicative} if and only if $1\in S$ and $x,y\in S$ implies $xy\in S.$ In other words, $S$ is a submonoid of $(A,\times).$
\end{definition}
We have the following.
\begin{proposition}
    Let $\sim$ be a relation on $A\times S$ defined by
    \[(a_1,s_1)\sim(a_2,s_2)\iff\exists s':s'(s_1a_2-s_2a_2)=0.\]
    Then we have the following.
    \begin{enumerate}[label=(\alph*)]
        \item $\sim$ is an equivalence relation on $A\times S.$
        \item $\sim$ is the smallest equivalence relation on $A\times S$ such that for which all $(a,s)\in A\times S$ have $(a,s)\sim(s_0a,s_0s)$ for any $s_0\in S.$
    \end{enumerate}
\end{proposition}
\begin{proof}
    The proof of (a) is annoying and will be omitted.

    We now do (b). Suppose $\approx$ is the smallest equivalence relation satisfying the condition of (b). We can check by hand that
    \[(a,s)\approx(s'a,s's)\]
    so that $\approx\subseteq\sim.$ Conversely, if $(a_1,s_1)\sim(a_2,s_2),$ then there exists some $s'$ such that $s's_1a_2=s's_2a_1.$ It follows
    \[(a_1,s_1)\approx(s's_2a_1,s's_2s_1)=(s's_1a_2,s's_1s_2)\approx(a_2,s_2),\]
    so $\sim\subseteq\approx.$
\end{proof}
\begin{remark}
    The extra $s'$ in the equivalence relation exists because general rings might have zero-divisors, and this is needed to make our relation transitive.
\end{remark}
So we have the following definition.
\begin{definition}[Localized ring]
    Define $S^{-1}A$ to be $(A\times S)/\sim.$
\end{definition}
Then we have the following.
\begin{proposition}
    The usual addition and multiplication rules for fractions on $S^{-1}A$ make $S^{-1}A$ into a commutative ring such that $A\to S^{-1}A$ by $a\mapsto a/1$ is a ring homomorphism.
\end{proposition}
\begin{proof}
    We omit this because it is boring. The main point is that (b) from the above proposition gives us a more direct way to check the addition and multiplication rules are well-defined.
\end{proof}
We also have the following universal property.
\begin{proposition}
    Any homomorphism $\varphi:A\to B$ for which $\varphi(S)\subseteq B$ factors uniquely through $A\to S^{-1}A.$ Namely, the induced arrow in the following diagram exists and is unique.
    % https://q.uiver.app/?q=WzAsMyxbMCwwLCJBIl0sWzIsMCwiQiJdLFsxLDIsIlNeey0xfUEiXSxbMCwxLCJcXHZhcnBoaSJdLFswLDJdLFsyLDEsIiIsMix7InN0eWxlIjp7ImJvZHkiOnsibmFtZSI6ImRhc2hlZCJ9fX1dXQ==
    \[\begin{tikzcd}
        A && B \\
        \\
        & {S^{-1}A}
        \arrow["\varphi", from=1-1, to=1-3]
        \arrow[from=1-1, to=3-2]
        \arrow[dashed, from=3-2, to=1-3]
    \end{tikzcd}\]
\end{proposition}
\begin{proof}
    Check by hand.
\end{proof}
There is also a notion of a localized module.
\begin{definition}
    If $M$ is an $A$-module, then essentially the same relation on $M\times S$ lets us define $S^{-1}M.$
\end{definition}
And we have the following.
\begin{proposition}
    The module $S^{-1}M$ is an $S^{-1}A$-module, and $S^{-1}M\cong M\otimes_AS^{-1}A.$ In fact, $M\mapsto S^{-1}M$ is an exact functor: if
    \[0\to M'\to M\to M''\to 0\]
    is an exact sequence of $A$-modules, then
    \[0\to S^{-1}M'\to S^{-1}M\to S^{-1}M''\to0\]
    is an exact sequence of $S^{-1}A$-modules.
\end{proposition}
\begin{proof}
    We leave this as an exercise because, as usual, it is not very interesting.
\end{proof}

Let's do some examples.
\begin{example}
    If $A$ is entire, then $S=A\setminus\{0\}$ is multiplicative, so we have $A\to S^{-1}A$ is really just $A$ mapping into its fraction field.
\end{example}
\begin{example}
    We have that $S^{-1}A=0$ if and only if $0\in S.$ In one direction, if $1/1=0/1,$ then we see there is $s'\in S$ such that $s'(1-0)=0,$ so $s'=0\in S.$ In the other direction, if $0\in S,$ then $a/s=0/1$ because $0\cdot(1a-0s)=0.$
\end{example}
\begin{example}
    If $A$ is entire with fraction field $K,$ then $0\notin S$ has $A\into S^{-1}A$ an injective map, and in fact $S^{-1}A$ is a subring of $K.$
\end{example}

\subsection{Some Theory}
We have the following proposition.
\begin{proposition}
    Fix $\varphi:A\to S^{-1}A$ the canonical map. Then we have the following.
    \begin{enumerate}[label=(\alph*)]
        \item $\ker\varphi=\{a\in A:\op{Ann}(a)\cap S\ne\emp\}.$
        \item Let $\mf a$ and $\mf b$ be ideals of $A.$ Then $S^{-1}\mf a$ and $S^{-1}\mf b$ (defined as $S^{-1}A$-modules because $\mf a$ and $\mf b$ are $A$-modules) are $S^{-1}A$-ideals. Further,
        \[S^{-1}(\mf a\mf b)=S^{-1}\mf aS^{-1}\mf b.\]
        \item If $\mf a$ is an $A$-ideal, then $\varphi^{-1}\left(S^{-1}\mf a\right)=\{a\in A:aS\cap\mf a\ne\emp\}.$ Expanding, this is equivalent to $a/s\in S^{-1}\mf a$ if and only if $s'a\in\mf a$ for some $s'\in S.$
        \item If $\mf a'$ is an $S^{-1}A$-ideal, then $S^{-1}\left(\varphi^{-1}\mf a'\right)=\mf a'.$
        \item if $\mf a$ is an $A$-ideal, then $S^{-1}A/S^{-1}\mf a\cong(A/\mf a)\otimes_AS^{-1}A\cong\overline S^{-1}\mf a,$ where $\overline S$ is the image of $S$ under $A\onto A/\mf a.$
    \end{enumerate}
\end{proposition}
\begin{proof}
    This is omitted because it's too long.
\end{proof}
This gives the following corollaries.
\begin{corollary}
    Given $\mf a$ an $A$-ideal, the canonical map $\mf a\to S^{-1}\mf a$ is surjective.
\end{corollary}
\begin{proof}
    This essentially follows from (d) above.
\end{proof}
\begin{corollary}
    Given $\mf a'$ an $S^{-1}A$-ideal, the canonical map $\mf a'\to\varphi^{-1}\mf a'$ is injective and preserves inclusions and therefore strict inclusions. It follows $\mf a'\subseteq\mf b'$ if and only if $\varphi^{-1}\mf a\subseteq\mf b.$
\end{corollary}
\begin{proof}
    Omitted because what even is mathematics.
\end{proof}
So we have the following.
\begin{proposition}
    If $A$ is Noetherian, then $S^{-1}A$ is Noetherian.
\end{proposition}
\begin{proof}
    This is because the pull-back $\varphi^{-1}$ preserves strict inclusions, so $S^{-1}A$ may have no non-stabilizing infinite ascending chains.
\end{proof}

And let's talk briefly about being integrally closed.
\begin{proposition}
    Fix $A\subseteq B\subseteq C$ entire rings with $0\notin S.$ Then we have the following.
    \begin{enumerate}
        \item If $\alpha\in B$ is integral over $A,$ then it is integral over $S^{-1}A.$
        \item If $\alpha\in B$ is integral over $S^{-1}A,$ then there is some $s\in S$ such that $s\alpha$ is integral over $A.$
        \item If $B$ is the integral closure of $A$ in $C,$ then $S^{-1}B$ is the integral closure of $S^{-1}A$ in $S^{-1}C.$
        \item If $A$ is integrally closed, then $S^{-1}A$ is also integrally closed.
    \end{enumerate}
\end{proposition}
\begin{proof}
    As usual, the proof is omitted.
\end{proof}
\begin{remark}
    This more or less generalizes the story of what happened with $\ZZ$ and $\QQ$ way back
\end{remark}

\subsection{Spec Smack}
We have the following definition.
\begin{definition}[Spectrum]
    We define the \textit{spectrum} of $A,$ notated $\op{Spec}A,$ to be the set of primes in $A,$ including $0.$
\end{definition}
\begin{example}
    In $\ZZ,$ we have $\op{Spec}A=\{(p):p\text{ prime}\}.$
\end{example}
One of the nice things about the spectrum is that a ring homomorphism $\varphi:A\to B$ will take a prime $\mf p\in\op{Spec}B$ to another prime $\varphi^{-1}\mf p\in\op{Spec}A.$ This means we have an induced map
\[\varphi^*:\op{Spec}B\to\op{Spec}A.\]
In our case, we have the following.
\begin{proposition}
    The canonical map $\varphi:A\to S^{-1}A$ has the induced $\varphi^*$ injective. We will often sloppily identify $\op{Spec}S^{-1}A$ with its image under $\varphi^*$ in $\op{Spec}A.$
\end{proposition}
\begin{remark}
    If $0\in S,$ then this actually works because $\op{Spec}S^{-1}A=\op{Spec}0=\emp.$
\end{remark}
\begin{proof}
    As usual, omitted.
\end{proof}
We continue.
\begin{proposition}
    Fix $\mf p\in\op{Spec}A.$ Then we have the following.
    \begin{enumerate}[label=(\alph*)]
        \item We have that $S^{-1}\mf p=(1)$ if and only if $\mf p\cap S\ne\emp.$
        \item If $\mf p\cap S=\emp,$ then $a/s\in S^{-1}\mf p$ if and only if $a\in\mf p.$
        \item If $\mf p\cap S=\emp,$ then $S^{-1}\mf p$ is prime. It follows $\op{Spec}S^{-1}A=\{\mf p\in\op{Spec}A:\mf p\cap S=\emp\}.$
    \end{enumerate}
\end{proposition}
\begin{proof}
    We outline in sequence.
    \begin{enumerate}[label=(\alph*)]
        \item This follows because $1/1\in S^{-1}\mf p$ if and only if $s/s\in S^{-1}\mf p$ so that $s\in\mf p$ and $s\in S.$
        \item This follows because $a/s\in S^{-1}\mf p$ if and only if $a\in\varphi^{-1}(S^{-1}\mf p),$ which is equivalent to there existing $s'\in S$ such that $s'a\in\mf p,$ which is equivalent to $a\in\mf p$ because $S\cap\mf p=\emp.$
        \qedhere
    \end{enumerate}
\end{proof}
So we have the following.
\begin{proposition}
    Fix $A$ a Dedekind ring and $0\notin S.$ Then $S^{-1}A$ is also Dedekind.
\end{proposition}
\begin{proof}
    We can see that $S^{-1}A$ is entire by hand; we already checked that it is Noetherian and integrally closed given that $A$ is.

    Lastly, $S^{-1}A$ has dimension $1$ because of the classification of its primes. In particular, if $S^{-1}A$ has non-maximal primes, then we could pull these back to non-maximal primes of $A$ by $\varphi^{-1}$ because $\varphi^{-1}$ preserves strict inclusion.
\end{proof}