\subsection{Integral Bases}
We continue to work in the $AKLB$ set-up and that $B$ is finite over $A.$
\begin{defi}[Number field]
    A \textit{number field} is a finite field extension of $\QQ.$
\end{defi}
\begin{defi}[Ring of integers]
    The ring of integers $\mathcal O_K$ of a number field $K$ is the integral closure of $\ZZ$ in $K.$ It is finite over $\ZZ$ as we showed last time because integral basis of $K$ is an integral bases of $\mathcal O_K$ over $\ZZ.$
\end{defi}
\begin{defi}[Discriminant]
    The \textit{discriminant} $d_K$ of $K$ is the discriminant of $\mathcal O_K$ over $\ZZ.$ It is a well-defined nonzero rational integer.
\end{defi}
\begin{defi}
    Fix $M$ a full, finitely generated $A$-submodule of $L.$ If $M$ has an integral basis (over $A$), then $\op{disc}M$ is the discriminant of that integral basis.
\end{defi}
\begin{remark}
    By \autoref{lem:lem0}, this is well-defined up to multiplication by the square of a unit in $A.$ In particular, when $A=\ZZ,$ this is well-defined.
\end{remark}
\begin{prop} \label{prop:212}
    Fix $K$ a number field and $M\subseteq M'$ be full, finitely generated $\ZZ$-submodules of $K.$ Then $[M':M]<\infty$ and
    \[d(M)=[M':M]^2d(M').\]
\end{prop}
\begin{proof}
    From \autoref{lem:lem0}, the point is that $[M':M]$ is the determinant of a change-of-basis matrix from an integral basis of $M$ to an integral basis of $M'.$ Indeed, fix $\{\omega_k\}_{k=1}^n$ and $\{\omega_k\}_{k=1}^n$ be integral bases of $M$ and $M'$ respectively, where $n=[K:\QQ].$ Fix $N$ as the change-of-basis matrix satisfying
    \[\begin{bmatrix}\omega_1 \\ \vdots \\ \omega_n\end{bmatrix}=N\begin{bmatrix}\omega_1' \\ \vdots \\ \omega_n'\end{bmatrix}.\]
    Then \autoref{lem:lem0} implies that
    \[d(M)=(\det N)^2d(M').\]
    So we need to show that $|\det N|=[M':M].$ We can show this by integer Gaussian elimination on $N,$ which makes $N$ diagonal. \todo{blegh}
\end{proof}
\begin{cor}
    Fix $\theta\in\mathcal O_K$ such that the minimal polynomial of $\theta,$ which we name $f(x),$ over $\QQ$ has squarefree discriminant. Then $\mathcal O_{\QQ(\theta)}=\ZZ[\theta],$ and $d_{\QQ(\theta)}=\op{disc}f(x).$
\end{cor}
\begin{remark}
    This is not always possible. For example, we cannot do this with any element of $\QQ(\sqrt2)$ because $d_{\QQ(\sqrt2)}$ is $8$ and not squarefree.
\end{remark}
\begin{proof}
    Note that $\op{disc}\ZZ[\theta]=\op{disc}f(x)$ is squarefree, but
    \[[\mathcal O_K:\ZZ[\theta]]^2\mid\op{disc}\ZZ[\theta],\]
    so we are forced to have $[\mathcal O_K:\ZZ[\theta]]=1.$ So indeed, $\mathcal O_K=\ZZ[\theta].$ This works because $\ZZ[\theta]$ is full and lives in $\mathcal O_K.$
\end{proof}
We can compute the ring of integers of a number field by looking for primitive integral elements whose minimal polynomial has squarefree discriminant. In general, we can try to find a chain of rings
\[\ZZ[\theta]=B_0\subseteq B_1\subseteq\cdots\subseteq B_\ell=\mathcal O_K.\]
\begin{ex}
    We compute the ring of integers of $\QQ(\sqrt D)$ where $D$ is squarefree and not $1.$ Fixing $\theta=\sqrt D,$ we see that $\ZZ[\sqrt D]$ has discriminant $4D,$ so the only square divisor to worry about is $4.$ It follows
    \[[\mathcal O_K:\ZZ[\sqrt D]]\in\{1,2\}.\]
    Further, we see that $[\mathcal O_K:\ZZ[\sqrt D]]=2$ if and only if $\mathcal O_K$ contains half of some element from $\ZZ[\sqrt D]$ (what else could we have?) if and only if one of $\frac12,\frac{\sqrt D},\frac{1+\sqrt D}2$ lives in $\mathcal O_K.$ Of course, $1/2,\sqrt D/2$ are not integral, so $[\mathcal O_K:\ZZ[\sqrt D]]=2$ is equivalent to $\frac{1+\sqrt D}2\in\mathcal O_K.$ This gives rise to the classification.
\end{ex}

\subsection{Dedekind}
Here is our notion of Noetherian.
\begin{prop}
    Fix $A$ a commutative ring. The following are equivalent.
    \begin{enumerate}[label=(\alph*)]
        \item All ideals are finitely generated.
        \item $A$ has the ascending chain condition: all ascending chains $I_0\subseteq I_1\subseteq I_2\subseteq\cdots$ must stabilize.
        \item Every nonempty collection of ideals contains a maximal element.
    \end{enumerate}
\end{prop}
\begin{proof}
     This is in any book on algebraic number theory, commutative algebra, or maybe even ring theory, so we outline.
     \begin{itemize}
         \item (a) implies (b) because the union of all the ideals is finitely generated.
         \item (b) implies (c) by set theory: showing the contrapositive, we can build our infinite strictly ascending chain because at no point can we ever hit a maximal ideal.
         \item (c) implies (a) because any ideal $I$ gives rise to a collection of its sub-ideals which are finitely generated. A maximal element in this partially ordered set must be $I.$
         \qedhere
     \end{itemize}
\end{proof}
\begin{defi}[Noetherian]
    A ring $A$ is \textit{Noetherian} if it satisfies one of the above equivalent conditions.
\end{defi}
\begin{defi}[Dedekind]
    A ring $A$ is \textit{Dedekind} if it integrally closed (and hence entire), Noetherian, and of Krull dimension $\le1$ (i.e., every nonzero prime is maximal).
\end{defi}