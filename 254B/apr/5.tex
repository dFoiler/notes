% !TEX root = ../notes.tex

\documentclass[../notes.tex]{subfiles}

\begin{document}

Today we would like to state the Main theorem of complex multiplication.

\subsection{The Reflex Norm}
We need to discuss the reflex norm. To describe our definition, fix a CM type $(E,\Phi)$, and let $E^*$ be the reflex field. Recall that we may view $\Phi$ as a subset of $\op{Hom}(E,\ov\QQ)$. Note that
\[E\otimes_\QQ K\cong\prod_{\sigma\in\op{Hom}(E,\ov\QQ)}K_\sigma\]
for any field $K$ containing all embeddings of $E$ into $\ov\QQ$. Then Galois descent provides an $E\otimes_\QQ E^*$ module $V_\Phi$ such that
\[V_\Phi\otimes_{E^*}K\cong\prod_{\varphi\in\Phi}K_\varphi\]
simply by definition of $E^*$ as being fixed by automorphisms $\sigma\colon\ov\QQ\to\ov\QQ$ permuting $\Phi$.
\begin{definition}[reflex norm]
	Fix a CM type $(E,\Phi)$, and define $V_\Phi$ as above. Then we define $\op N_\Phi\colon(E^\times)^*\to E^\times$ by
	\[\op N_\Phi(\alpha)\coloneqq\det(\alpha~|~V_\Phi).\]
	In fact, for any $K$ containing $E^*$, one can define $\op N_{K,\Phi}\colon K^\times\to K^\times$ by
	\[\op N_{K,\Phi}(\alpha)\coloneqq\det(\alpha~|~V_\Phi\otimes_{E^*}K).\]
\end{definition}
\begin{remark}
	Because of transitivity of norms, we see that
	\[{\op N_{K,\Phi}}={\op N_\Phi}\circ{\op N_{K/E^*}}.\]
\end{remark}
This definition work well with \Cref{thm:st-proved}.
\begin{proposition}
	Fix a field $K$ containing the all images of $E$ in $\ov\QQ$. Then any $a\in K^\times$ has
	\[\op N_{K,\Phi}(a)=\prod_{\varphi\in\Phi}\varphi^{-1}\left(\op N_{K/\varphi(E)}a\right)\]
\end{proposition}
\begin{proof}
	Omitted. See \cite[Proposition~1.26]{milne-cm}. The point is to expand out the definitions and stratify along $\Phi$.
\end{proof}
\begin{remark}
	By tensoring with local fields suitably, we see that $\op N_{K,\Phi}$ provides a map $\AA_K^\times\to\AA_E^\times$ and also a map on the fractional ideals.
\end{remark}

\subsection{The Main Theorem}
Now, for our set-up, let $A$ be an abelian $\ov\QQ$-variety with CM type $(E,\Phi)$. Let $E^*$ be the reflex field. From here, note that $\sigma\in\op{Gal}(\ov\QQ/\QQ)$ lets us define
\[A^\sigma\coloneqq A\otimes_{\ov\QQ}\ov\QQ,\]
which has CM type given by $\sigma\Phi$; thus, if $\sigma\in\op{Gal}(\ov\QQ/E^*)$, then $\sigma\Phi=\Phi$ by definition of $E^*$, meaning that our CM type is preserved! This observation will simplify matters, though it is possible to work with more general $\sigma$ if one is willing to put in more work.

Continuing, note that pointwise application of $\sigma$ provides a map $\sigma\colon A\to A^\sigma$, and this isomorphism is compatible with the $E$-action on both sides. Continuing, we are granted an isogney $\alpha\colon A\to A^\sigma$ which is compatible with the $E$-action and unique up to multiplication by $E^\times$; this is because $A$ and $A^\sigma$ are both CM abelian varieties with the same CM type.\footnote{An easy way to see the uniqueness up to $E^\times$ is to use the Albert classification: it suffices to show that $\beta\in\op{End}^0(A)$ commuting with the $E$-action must be in $E$, which can be seen by looking at the cases individually.} We would like to understand our Galois representations, so we define
\[\widehat T(A)\coloneqq\prod_\ell T_\ell A\qquad\text{and}\qquad\widehat V(A)\coloneqq\widehat T(A)\otimes_\ZZ\QQ.\]
So we get our isomorphism $\widehat V(\sigma)\colon\widehat V(A)\to\widehat V(A^\sigma)$ and similarly get some $\widehat V(\alpha)$. These maps are both going to be $E\otimes\AA_f$ linearly, where $\AA_f\coloneqq\AA_{E,f}$ is denoting the finite ad\'eles. Comparing our two morphisms, we get some $\eta(\sigma)\in\AA_{E,f}^\times$ such that
\[\alpha(\eta(\sigma)x)=\sigma(x)\]
for all $x\in\widehat V(A)$; this is simply because we have provided two isomorphisms between Galois representations, which must be unique up to some multiplication. In total, we have gotten a group homomorphism
\[\eta\colon\op{Gal}(\ov\QQ/E^*)\to\AA_{E,f}^\times/E^\times.\]
We now get the feeling that global class field theory should come up. Because the target is abelian, the above map actually factors through the abelianization, so it factors through $\op{Gal}(E^{*,\mathrm{ab}},E)\to\AA_{f,E}^\times/E^\times$. On the other hand, we know from the next subsection that there is a global Artin map
\[\AA_{E^*,f}^\times/E^*\to\op{Gal}(E^{*,\mathrm{ab}}/E^*).\]
The statement of our Main theorem is then the following.
\begin{theorem}[Main] \label{thm:main}
	Fix everything as above. Then the following diagram commutes.
	% https://q.uiver.app/#q=WzAsNCxbMCwwLCJcXG9we0dhbH0oXFxvdlxcUVEvRV4qKSJdLFswLDEsIlxcb3B7R2FsfShFXnsqLFxcbWF0aHJte2FifX0vRV4qKSJdLFsxLDAsIlxcQUFfe0UsZn1eXFx0aW1lcy9FXlxcdGltZXMiXSxbMSwxLCJcXEFBX3tFXiosZn0vKEVeKileXFx0aW1lcyJdLFswLDIsIlxcZXRhIl0sWzMsMSwiXFxvcHtBcnR9Il0sWzAsMV0sWzMsMiwiXFxvcCBOX1xcUGhpIiwyXV0=&macro_url=https%3A%2F%2Fraw.githubusercontent.com%2FdFoiler%2Fnotes%2Fmaster%2Fnir.tex
	\[\begin{tikzcd}
		{\op{Gal}(\ov\QQ/E^*)} & {\AA_{E,f}^\times/E^\times} \\
		{\op{Gal}(E^{*,\mathrm{ab}}/E^*)} & {\AA_{E^*,f}/(E^*)^\times}
		\arrow["\eta", from=1-1, to=1-2]
		\arrow["{\op{Art}}", from=2-2, to=2-1]
		\arrow[from=1-1, to=2-1]
		\arrow["{\op N_\Phi}"', from=2-2, to=1-2]
	\end{tikzcd}\]
\end{theorem}
In particular, we are granted essentially total understanding of the Galois action on the Tate module.
\begin{remark}
	Later, we will use this fine understanding of the Galois representation in order to compute the $L$-function of a CM abelian variety.
\end{remark}
\begin{remark} \label{rem:polarization-main-theorem}
	Fix a polarization $\lambda\colon A\to A^\lor$ such that $(\cdot)^\dagger$ is complex conjugation on $E$. Then one has a Weil pairing $\psi\colon\widehat V(A)\times\widehat V(A)\to\AA_f(1)$ given by gluing together the local Weil pairings. We now define $\psi^\sigma$ on $A^\sigma$ by $\psi^\sigma(\sigma x,\sigma y)\coloneqq\sigma(\psi(x,y))$, which by definition of $\AA_f(1)$ is just $\chi(\sigma)\psi(x,y)$ where $\chi$ is the cyclotomic character. Applying \Cref{thm:main} to our situation, we get some $s$ such that
	\begin{align*}
		\psi^\sigma(\sigma x,\sigma y) &= \psi^\sigma(\alpha(\op N_\Phi(s)x),\alpha(\op N_\Phi(s)y)) \\
		&= \psi^\sigma(\op N_\Phi(s)\ov{\op N_\Phi(s)}\alpha x,\alpha y) \\
		&= \op N_\Phi(s)\ov{\op N_\Phi(s)}\psi^\sigma(\alpha x,\alpha y).
	\end{align*}
	(Note that we get complex conjugation on the $\op N_\Phi(s)$ because $(\cdot)^\dagger$ is complex conjugation.) So we are able to compare $\psi^\sigma$ with $\chi_{\mathrm{cyc}}$ by comparing our two expressions.
\end{remark}

\subsection{A Little Global Class Field Theory}
We quickly review the statement of global class field theory. Fix a number field $K$, and let $K^{\mathrm{ab}}$ be its abelian closure.
\begin{definition}[Artin map]
	Fix a number field $K$. Then there is a canonical homomorphism
	\[\op{Art}_K\colon\AA_K^\times/K^\times\to\op{Gal}(K^{\mathrm{ab}}/K)\]
	satisfying the following: for any finite place $v$, of $K$ and $w\mid v$, the following diagram commutes.
	% https://q.uiver.app/#q=WzAsNSxbMCwwLCJLX3YiXSxbMCwxLCJcXEFBX0teXFx0aW1lcy9LXlxcdGltZXMiXSxbMSwxLCJcXG9we0dhbH0oS157XFxtYXRocm17YWJ9fS9LKSJdLFsyLDEsIlxcb3B7R2FsfShML0spIl0sWzIsMCwiXFxvcHtHYWx9KExfdy9LX3YpIl0sWzAsMSwiaV92IiwyXSxbMSwyLCJcXG9we0FydH1fSyIsMl0sWzIsMywiIiwyLHsic3R5bGUiOnsiaGVhZCI6eyJuYW1lIjoiZXBpIn19fV0sWzQsMywiIiwwLHsic3R5bGUiOnsidGFpbCI6eyJuYW1lIjoiaG9vayIsInNpZGUiOiJ0b3AifX19XSxbMCw0LCJcXG9we0FydH1fdiJdXQ==&macro_url=https%3A%2F%2Fraw.githubusercontent.com%2FdFoiler%2Fnotes%2Fmaster%2Fnir.tex
	\[\begin{tikzcd}
		{K_v} && {\op{Gal}(L_w/K_v)} \\
		{\AA_K^\times/K^\times} & {\op{Gal}(K^{\mathrm{ab}}/K)} & {\op{Gal}(L/K)}
		\arrow["{i_v}"', from=1-1, to=2-1]
		\arrow["{\op{Art}_K}"', from=2-1, to=2-2]
		\arrow[two heads, from=2-2, to=2-3]
		\arrow[hook, from=1-3, to=2-3]
		\arrow["{\op{Art}_{L_w/K_v}}", from=1-1, to=1-3]
	\end{tikzcd}\]
	Here, $\op{Art}_v$ is the local Artin map; it is also an isomorphism.
\end{definition}
\begin{remark}
	Let's describe some properties of the local Artin map.
	\begin{itemize}
		\item If $L_w/K_v$ is unramified and nonarchimedean, then
		\[\op{Art}_{L_w/K_v}(\alpha)=\op{Frob}_{L_w/K_v}^{-v(\alpha)},\]
		where the $-$ in the exponent is a rather annoying convention.
		\item If $L_w/K_v$ is the extension $\CC/\RR$, then we are looking at the sign map $\RR^\times\to\op{Gal}(\CC/\RR)$.
	\end{itemize}
\end{remark}
\begin{remark}
	One can take a quotient suitably to provide an Artin isomorphism
	\[\op{Art}_{L/K}\colon\frac{\AA_K^\times}{K^\times\op N(\AA_L^\times)}\to\op{Gal}(L/K).\]
\end{remark}
\begin{remark}
	If $K$ is CM, one may basically ignore the infinite places because they all start out as $\CC$.
\end{remark}

\end{document}