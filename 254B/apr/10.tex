% !TEX root = ../notes.tex

\documentclass[../notes.tex]{subfiles}

\begin{document}

Today we continue towards our discussion of $L$-functions.

\subsection{\texorpdfstring{$L$}{L}-functions for Abelian Varieties}
We begin by checking that we have actually defined a Hecke character.
\begin{lemma}
	Fix everything as previously discussed. Then $\alpha^\tau(K^\times)=1$.
\end{lemma}
\begin{proof}
	Quickly, for $s\in\AA_K^\times$, let $s_f\in\AA_{K,f}^\times$ be the finite part, and we recall that
	\[\lambda(s)\op N_\Phi^{-1}\left(\op{N}_{K/E^*}s_f\right)=\lambda(s)\op N_{K,\Phi}(s_f)=\rho\left(\op{Art}_K^{-1}(s)\right)\]
	from \Cref{thm:main}. Now, the right-hand side is trivial for $s\in K^\times$, and we are able to compute that the left-hand norm is
	\[\op N_{K,\Phi}^{-1}(s_f)=\op N_{K,\Phi,\infty}^{-1}(s)\]
	basically by definition of $s$, so we are able to conclude.
\end{proof}
To continue, we note that our Hecke character is actually algebraic.
\begin{definition}[algebraic]
	Fix a number field $K$. A Hecke character $\chi\colon\AA_K^\times/K^\times\to\CC^\times$ is \textit{algebraic} if and only if its archi\-medean part $\chi_\infty\coloneqq\chi|_{K_\infty^\times}$ is of the form
	\[\chi_\infty(x_\infty)=\prod_{v\text{ real}}x_v^{n_v}\cdot\prod_{v\text{ complex}}x_v^{n_v}\ov x_v^{n_{\ov v}}\]
	for integers $n_\bullet\in\ZZ$.
\end{definition}
\begin{remark}
	Approximately speaking, we are asking for this to come from morphism $\op{Res}_{K/\QQ}\mathbb G_m\to\mathbb G_m$. In particular, a priori, $\chi_\infty$ can have exponents which are any integers, so we are placing a fairly strong algebraic limitation.
\end{remark}
Unwinding the definition of $\op N_{K,\Phi,\infty}$ reveals that $\op N_{K,\Phi,\infty}^{-1}\lambda$ is an algebraic Hecke character.

We are now able to define our $L$-function on the level of the Hecke character.
\begin{definition}[conductor]
	Fix a number field $K$. The \textit{conductor} $\mf m$ of a Hecke character $\chi\colon\AA_K^\times/K^\times\to\CC^\times$ is a finite ideal $\mf m=\prod_\mf p\mf p^{m_\mf p}$ chosen to be the smallest possible so that $\chi$ is trivial on $\prod_\mf p\left(1+\mf p^{m_\mf p}\right)$.
\end{definition}
Note that $\mf m$ conductor exists by continuity of $\chi$.
\begin{definition}
	Fix a number field $K$. A Hecke character $\chi\colon\AA_K^\times/K^\times\to\CC^\times$ of conductor $\mf m$ has associated $L$-function given by\todo{Outside these factors?}
	\[L(\chi,s)\coloneqq\prod_{\mf p\nmid\mf m}\frac1{1-\chi_\mf p(\varpi_\mf p)\op{N}_{K/\QQ}(\mf p)^{-s}},\]
	where $\varpi_\mf p\in\mf p$ is a uniformizer.
\end{definition}
We now recall the following result on these $L$-functions.
\begin{theorem}[Hecke, Tate's thesis]
	Fix a number field $K$ and a Hecke character $\chi\colon\AA_K^\times/K^\times\to\CC^\times$. Then $L(s,\chi)$ admits a functional equation and a meromorphic continuation to all $\CC$.
\end{theorem}
On the other hand, we can build an $L$-function for $A$.
\begin{definition}
	Fix an abelian variety $A$ over a number field $K$. Then the \textit{$L$-function of $A$} is
	\[L(A,s)\coloneqq\prod_{\mf p}\frac1{\det\left(1-\op{Frob}_\mf p(\op N_{K/\QQ}\mf p)^{-s}~|~V_\ell A\right)},\]
	where the Euler factor written is correct when $A$ has good reduction at $\mf p$, but at bad reduction we must look at the part of $V_\ell A$ fixed by inertia.
\end{definition}
It turns out that for $\op{Re}s$ large enough, the Euler product will converge; this is essentially by the Weil conjectures. In more words, we may only look at primes of good reduction (there are only finitely many primes of bad reduction), and the eigenvalues of $\op{Frob}_\mf p$ have magnitude $\left|\op N\mf p\right|^{1/2}$, so we should expect convergence after $\op{Re}s>3/2$.

We now have the following result.
\begin{theorem} \label{thm:l-func-cm-av}
	Fix an abelian variety $A$ over a number field $K$, and assume that $A$ has complex multiplication by the CM algebra $E$. Then
	\[L(A,s)=\prod_{\tau\colon E\to\CC}L(\alpha^\tau,s).\]
	In particular, $L(A,s)$ admits a functional equation and meromorphic continuation.
\end{theorem}
Basically, what is happening is that the Galois representation attached to $A$ is abelian, so we should be able to decompose it into characters. The theorem will follow from the following result.
\begin{proposition} \label{prop:frob-action-on-cm-av}
	Fix an abelian variety $A$ over a number field $K$, and assume that $A$ has complex multiplication by the CM algebra $E$. Fix a prime $\mf p$ of $K$. We will basically have two steps.
	\begin{enumerate}
		\item If $A$ has good reduction at $\mf p$, then the restricted character $\chi_\mf p\coloneqq\lambda|_{K_\mf p^\times}$ is trivial on $\OO_{K_\mf p}^\times$. (In fact, the converse is true, which we will show next lecture.)
		\item $\lambda_\mf p(\varpi_\mf p)\in\OO_E$ acts on $\mc A_{\kappa(\mf p)}$ as $\op{Frob}_\mf p$.
	\end{enumerate}
\end{proposition}
We now prove \Cref{thm:l-func-cm-av}.
\begin{proof}[Proof of \Cref{thm:l-func-cm-av} assuming \Cref{prop:frob-action-on-cm-av}]
	We compare the Euler factors by hand. Note $V_\ell A$ is a rank-$1$ module over $E\otimes_\QQ\QQ_\ell$, so
	\[\det\left(1-{\op{Frob}_\mf p}T~|~V_\ell A\right)=\op{N}_{E/\QQ}(1-\mathrm{Frob}_\mf pT).\]
	Now, \Cref{prop:frob-action-on-cm-av} implies that this equals
	\[\op N_{E/\QQ}(1-\lambda_\mf p(\varpi_\mf p)T)=\prod_{\tau\colon E\to\CC}\left(1-\alpha_\mf p^\tau(\varpi_\mf p)T\right),\]
	as desired.
\end{proof}
We now move towards a proof of \Cref{prop:frob-action-on-cm-av}.
\begin{proof}[Proof of \Cref{prop:frob-action-on-cm-av}]
	We begin with (a). Fix $\mf p$, and choose $\ell$ not divisible by $\mf p$. By local class field theory, the inertia subgroup $I_\mf p\subseteq\op{Gal}(K^{\mathrm{ab}}/K)$ is the image of $\OO_{K_\mf p}^\times$ under the Artin map, which means that
	\[\rho(\op{Art}_K(s))_\ell=1\]
	for $s\in\OO_{K_\mf p}^\times\subseteq\AA_K^\times$. So $\op{Art}_K(s)$ acts trivially on $T_\ell A$, but then we see that $\lambda(s)^{-1}\op N_{K,\Phi}^{-1}(s)$ vanishes on $T_\ell A$. Similarly, $\op N_{K,\Phi}^{-1}(s)_\ell=1$ because $\mf p$ does not divide $\ell$, so we are forced to conclude that $\lambda(s)$ acts trivially, as desired.

	We now turn to (b). Here, the point is that $\lambda_\mf p(\varpi_\mf p)$ acts on $T_\ell A=T_\ell\mc A_{\kappa(\mf p)}$ as $\rho\left(\op{Art}_{K_\mf p}(\varpi_\mf p)^{-1}\right)$, which is $\rho({\op{Frob}_\mf p})$, as desired.
\end{proof}

\end{document}