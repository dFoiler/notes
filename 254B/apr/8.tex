% !TEX root = ../notes.tex

\documentclass[../notes.tex]{subfiles}

\begin{document}

We begin to talk about $L$-functions.

\subsection{Hecke Characters from Abelian Varieties}
Fix an abelian variety $A$ over a number field $K\subseteq\ov\QQ$ with complex multiplication by $E\subseteq\op{End}^0(A)$. For simplicity, we will assume that $E^*\subseteq K$. Recall we built a Galois representation
\[\rho_A\colon\op{Gal}(\ov K/K)\to\op{Aut}_{\AA_{E,f}}(\widehat VA),\]
but because $A$ has complex multiplication, this right-hand side is $\op{Aut}_{E,f}^\times$, so in fact $\rho$ will factor as
\[\ov\rho_A\colon\op{Gal}(K^{\mathrm{ab}}/K)\to\AA_{E,f}^\times.\]
It turns out that \Cref{thm:main} implies that
\[\rho(\op{Art}_K(s))=\op N_\Phi(\op N_{K/E^*}(s))\cdot\lambda_s^{-1}\]
for some unique $\lambda_s\in E^\times$. Indeed, we know that 
\[\begin{tikzcd}
	{\op{Gal}(\ov\QQ/E^*)} & {\AA_{E,f}^\times/E^\times} \\
	{\op{Gal}(E^{*,\mathrm{ab}}/E^*)} & {\AA_{E^*,f}/(E^*)^\times}
	\arrow["\eta", from=1-1, to=1-2]
	\arrow["{\op{Art}}", from=2-2, to=2-1]
	\arrow[from=1-1, to=2-1]
	\arrow["{\op N_\Phi}"', from=2-2, to=1-2]
\end{tikzcd}\]
commutes, so we combine this with the functoriality of the global Artin map, which says that
% https://q.uiver.app/#q=WzAsNCxbMCwwLCJcXEFBX0teXFx0aW1lcy9LXlxcdGltZXMiXSxbMCwxLCJcXEFBX3tFXip9XlxcdGltZXMvKEVeKileXFx0aW1lcyJdLFsxLDAsIlxcb3B7R2FsfShcXG92XFxRUS9LKV57XFxtYXRocm17YWJ9fSJdLFsxLDEsIlxcb3B7R2FsfShcXG92XFxRUS9FXiopXntcXG1hdGhybXthYn19Il0sWzAsMiwiXFxvcHtBcnR9X0siXSxbMSwzLCJcXG9we0FydH1fe0VeKn0iXSxbMCwxLCJcXG9wIE5fe0svRV4qfSIsMl0sWzIsMywiXFxvcHtyZXN9Il1d&macro_url=https%3A%2F%2Fraw.githubusercontent.com%2FdFoiler%2Fnotes%2Fmaster%2Fnir.tex
\[\begin{tikzcd}
	{\AA_K^\times/K^\times} & {\op{Gal}(\ov\QQ/K)^{\mathrm{ab}}} \\
	{\AA_{E^*}^\times/(E^*)^\times} & {\op{Gal}(\ov\QQ/E^*)^{\mathrm{ab}}}
	\arrow["{\op{Art}_K}", from=1-1, to=1-2]
	\arrow["{\op{Art}_{E^*}}", from=2-1, to=2-2]
	\arrow["{\op N_{K/E^*}}"', from=1-1, to=2-1]
	\arrow["{\op{res}}", from=1-2, to=2-2]
\end{tikzcd}\]
commutes. Combining the two diagrams is able to produce our result.\todo{}

To continue, we have the following result.
\begin{proposition} \label{prop:get-hecke-char}
	The map $\lambda_\bullet\colon\AA_{K,f}^\times\to E^\times$ is continuous, where $E$ has been given the discrete topology.
\end{proposition}
We are going to take a roundabout way to this result. We begin with the following result, which will also be a key input to our proof of the Weil conjectures for abelian varieties.
\begin{remark}
	 Fix a polarized abelian variety $(A,\varphi)$. Then for any $m\ge3$, it turns out that
	\[\op{Aut}((A,\varphi))\to\op{Aut}A[m]\]
	is injective. 
\end{remark}
To prove the remark, we will want 
\begin{proposition}
	Fix an abelian $k$-variety $A$. Further, fix any endomorphism $\alpha$ such that $\alpha^\dagger\circ\alpha=[n]$ for some nonzero $n\in\ZZ$. Then the following are true.
	\begin{listalph}
		\item Then $\QQ(\alpha)\subseteq\op{End}^0(A)$ is semisimple.
		\item The multiset $\{\omega_i\}$ of roots of the characteristic polynomial all have absolute value $\sqrt {\left|n\right|}$.
		\item The multiset $\{\omega_i\}$ is stable under $\omega\mapsto n/\omega$.
	\end{listalph}
\end{proposition}
\begin{proof}
	We begin with (a). Here, we find $\alpha^\dagger=n\circ\alpha^{-1}$, so $\QQ(\alpha)$ is preserved under $(\cdot)^\dagger$. The point is that $x\mapsto\tr(xx^\dagger)$ can now be defined to be a positive-definite quadratic form on $\QQ(\alpha)$.
	
	We now show that $\QQ(\alpha)$ is semisimple. Let $\mf a\subseteq\QQ(\alpha)$ be an ideal, then having our quadratic form lets us define $\mf a^\perp$ and see that $\QQ(\alpha)=\mf a\oplus\mf a^\perp$ because everything in sight is finite-dimensional. Thus, we can decompose $\QQ(\alpha)$ into simple algebras inductively, meaning that $\QQ(\alpha)$ is semisimple.

	We now handle (b) and (c). We can write
	\[\QQ(\alpha)=K_1\times K_2\times\cdots\times K_m\]
	for some $m$. Because $(\cdot)^\dagger$ is positive-definite, it cannot swap any of these fields, so in fact $(\cdot)^\dagger$ must preserve all these fields, meaning that each is either totally real or has complex multiplication (by some argument from the Albert classification). Now, we see that the $\omega_j$ are the images of $\alpha$ via the various embeddings
	\[\QQ(\alpha)\to K_i\to\CC,\]
	but $(\cdot)^\dagger$ becomes complex conjugation in $\CC$, so we see that the image of $\alpha$ will have magnitude $\sqrt{\left|n\right|}$ by direct computation passing through $\alpha^\dagger\alpha=[n]$. Additionally, we see that we can exchange $\alpha$ with $\alpha^\dagger$ to send $\omega_i$ to $n/\omega_i$, which yields (c).
\end{proof}
\begin{remark}
	One recovers the Riemann hypothesis part of the Weil conjectures for abelian varieties by applying this result to the fact that
	\[{\op{Frob}^\dagger}\circ{\op{Frob}}=[q].\]
	One gets the other parts of the Weil conjectures by formally unraveling everything into the other parts; for example, (b) will give rise to the functional equation.
\end{remark}
We next pick up \cite[Theorem~21.5]{mumford}.
\begin{theorem} \label{thm:pav-finite-aut}
	Fix a polarized abelian variety $(A,\varphi)$. Then for any $m\ge3$, the map
	\[\op{Aut}((A,\varphi))\to\op{Aut}A[m]\]
	is injective.
\end{theorem}
\begin{proof}
	Suppose that $\alpha$ is an automorphism of $\op{Aut}((A,\varphi))$. Then $\alpha^\dagger\circ\alpha=1$ because we are an isomorphism of the polarized abelian variety. (This is a matter of writing down the corresponding commuting square for an isomorphism of polarized abelian varieties.) It follows that all eigenvalues of $\alpha$ are algebraic integers with norm $1$, so they are all roots of unity.

	We are now ready to complete the proof. It is enough to show that the map has trivial kernel. But then we see that $\alpha=1+Mx$ for some $x$ where $M>3$, so $\omega_i=1+Mx_i$ for some $x_i$ where $M>3$, from which some algebraic number theory is able to enforce that $\omega_i=1$ for each $i$.
\end{proof}
\begin{remark}
	The point of this is to show that the isomorphism class of our polarized abelian varieties is finite. The reason we must have the word ``polarized'' is that it is possible to provide abelian varieties with infinitely many automorphisms; for example, take an abelian variety with complex multiplication by $\QQ(\sqrt{-1},\sqrt{-2})$, which then will have endomorphisms in an order of $\QQ(\sqrt2)$, which has infinitely many units.
\end{remark}
We are now ready to prove \Cref{prop:get-hecke-char}.
\begin{proof}[Proof of \Cref{prop:get-hecke-char}]
	Because everything in sight is a group, it is enough to show that $s\to1$ in $\AA_{K,f}^\times$ implies that $\lambda_s=1$. Well, for $m$ large enough and $s$ close enough to $1$, we can achieve $\lambda_s\in\OO_E^\times$ (namely, force $s$ to be a unit at all finite places), $\lambda_s\equiv1\pmod M$ for large $M$ (which is finitely many congruence conditions), and $\lambda_s\in\op{End}(A_{\ov\QQ})$.\todo{How?} In particular, we know that $\lambda_s^{\pm1}$ lives in $\OO_E\cap\op{Aut}A_{\ov\QQ}$ and acts trivially on $A_{\ov\QQ}[M]$.

	Continuing, we are given that there is a polarization $\varphi\colon A_{\ov\QQ}\to A_{\ov\QQ}^\lor$ which is compatible with the $E$-action, and we are able to descend this polarization to its field of definition as $\varphi\colon A_L\to A_L^\lor$. The point is that we will be able to see that $\lambda_s$ preserves $\varphi$, from which $\lambda_s=1$ will follow from \Cref{thm:pav-finite-aut}. To see that we preserve $\varphi$, we recall the ideas and notations from \Cref{rem:polarization-main-theorem}, which grants a rational number $c$ such that
	\[c\psi(x,y)=\psi^\sigma\left(\lambda_s^{-1}x,\lambda_s^{-1}y\right)\]
	With $s$ close enough to $1$, we will get $\psi^\sigma=\psi$ because $\sigma=\op{Art}s$ (so $s$ close enough to $1$ will make $\sigma$ trivial over the field of definition $L$). Continuing, unwinding definitions, we see
	\[c^{-1}\psi(x,y)=\psi(\lambda_sx,\lambda_sy)=\psi(\lambda_s\ov{\lambda_s}x,y),\]
	from which the non-degeneracy of the Weil pairing forces $\lambda_s\ov{\lambda_s}\in\QQ$. Thus, for degree reasons (and positivity reasons), we see that $\lambda_s\cdot\ov{\lambda_s}$ is a positive integer but also invertible, so it must be $1$, so indeed $\lambda_s$ preserves $\varphi$ by staring at the above computation.
\end{proof}
We are now ready to define Hecke characters and check that we've built one.
\begin{definition}[Hecke character]
	Fix a number field $K$. A \textit{Hecke character} is a continuous homomorphism $\chi\colon\AA_K^\times/K^\times\to\CC^\times$. If $\im\chi\subseteq S^1$, we say that $\chi$ is \textit{unitary}.
\end{definition}
\begin{remark}
	For any Hecke character $\chi\colon\AA_K^\times/K^\times\to\CC^\times$, one has a unique decomposition $\chi=\chi_0\left|\cdot\right|^\sigma$ for some $\sigma\in\RR$ where $\chi_0$ is unitary and $\left|\cdot\right|$ is the norm. Indeed, the main point is to define $\sigma$ as $\left|\chi\right|$; then the image of $\chi\left|\cdot\right|^{-\sigma}$ lands in $S^1$.
\end{remark}
Thus, we see that we need $\lambda_\bullet$ to be trivial on $K^\times$.

Let's describe this construction. Fix everything as before, and choose an embedding $\tau\colon E\into\CC$. Then one can define a map $\alpha^\tau$ via the composite
\[\AA_K^\times\to\prod_{v\mid\infty}E_v^\times\onto E_\tau\stackrel\tau\to\CC^\times.\]
Here, the first map is given by $s\mapsto\op N_{K,\Phi,\infty}^{-1}(s)\lambda(s)$; here, $\op N_{K,\Phi,\infty}$ is given by taking the infinite components of the local reflex norms $\op N_{K,\Phi}\colon\AA_K^\times\to\AA_E^\times$. The continuity of $\alpha^\tau$ is basically by definition (everything involved in the definition is continuous), so it remains to check that $\alpha^\tau$ vanishes on $K^\times$.

\end{document}