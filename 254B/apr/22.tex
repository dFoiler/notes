% !TEX root = ../notes.tex

\documentclass[../notes.tex]{subfiles}

\begin{document}

Today we continue discussing the main theorem of complex multiplication.

\subsection{Continuing the Reduction Step}
We continue with the notations and notions from last lecture.
\begin{lemma}
	Fix some $\sigma\in\op{Gal}(E^{*,\mathrm{ab}}/E^*)$, and select $s\in\AA_{E^*,f}^\times/(E^*)^\times$ so that $\op{Art}(s)=\sigma$. Further, choose an isogeny $\alpha\colon A\to A^\sigma$, and we recall that we have some $\eta\colon\widehat V A\to\widehat V A$ such that $\widehat V\alpha\circ\eta=\widehat V\sigma$. Then we claim that
	\[\frac{\eta(\sigma)}{\op N_\Phi(s)}\in\frac{T(\AA_f)}{T(\QQ)}.\]
\end{lemma}
The point of this lemma is to reduce everything to the finite setting over $\QQ$, allowing us to transition to an ideal-theoretic statement.
\begin{proof}
	Fix a polarization $\lambda$ giving rise to the Weil pairing $\psi\colon\widehat V(A)\times\widehat V(A)\to\AA_f(1)$. Then recall from \Cref{rem:polarization-main-theorem} that
	\begin{align*}
		\chi_{\mathrm{cyclo}}(\sigma)\psi(x,y) &= \psi^\sigma(\sigma x,\sigma y) \\
		&= \psi^\sigma(\alpha(\eta(x)),\alpha(\eta(y))) \\
		&= \psi^\sigma(\alpha x,\alpha y) \\
		&= \eta(\sigma)\ov{\eta(\sigma)}\psi(\alpha x,\alpha y).
	\end{align*}
	Now, note that $\psi$ and $\psi^\sigma\circ\alpha^2$ are both Weil pairings compatible with the $E$-action (by an explicit check: everything in sight commutes with the $E$-action), so the classification of Riemann forms over $\CC$ allows us to say that they are of the form $\tr_{E/\QQ}(\xi x\ov y)$ where $\xi$ is totally negative. The point is that any two Weil pairings will differ by a totally positive element in $F=E^+$, so
	\[\eta(\sigma)\ov{\eta(\sigma)}=\chi_{\mathrm{cyclo}}(\sigma)c\]
	for some totally positive $c\in F$.

	On the other hand, compatibility of global class field theory requires
	\begin{align*}
		\op N_\Phi(s)\ov{\op N_\phi(s)} &= \op{Nm}_{\AA_{E^\times,f}/\AA_f}(s) \\
		&= \op{Art}_\QQ(\sigma|_{\QQ^{\mathrm{ab}}}) \\
		&= \chi_{\mathrm{cyclo}}(\sigma)
	\end{align*}
	up to multiplication by $\QQ^\times$. Our norm must be positive, so the ``multiplication by $\QQ^\times$'' must upgrade to ``multiplication by $\QQ^+$.''\todo{Why?}

	Now, define $t\coloneqq\eta(\sigma)/\op N_\Phi(s)$ so that $t\ov t$ must be a totally positive element in $E^+$ too (notably, the cyclotomic character cancels out), so checking the Hasse norm principle allows us to conclude that we have some $e\in E$ such that $e\ov e=t\ov t$. Then $(t/e)\ov{(t/e)}=1$, so we are able to conclude $t\pmod{E^\times}$ lives in $T(\AA_f)/T(\QQ)$.\todo{Why?}
\end{proof}

\subsection{Ideal-Theoretic Class Field Theory}
We are working with CM fields, so we will take our number fields to be totally imaginary.
\begin{definition}[modulus]
	Fix a totally imaginary number field $K$. Then a \textit{modulus} is a formal product of the form
	\[\mf m\coloneqq\prod_{\mf p\in V(K)}\mf p^{\mf m(\mf p)}\]
	where $\mf m(\mf p)\ge0$ always, $\mf m(\mf p)=0$ for infinite places, and $\mf m(\mf p)>0$ for only finitely many $\mf p$.
\end{definition}
Here is some more notation we will want to state ideal-theoretic global class field theory.
\begin{definition}[ray class group]
	Fix a modulus $\mf m$ of a totally imaginary number field $K$. We let $S(\mf m)\coloneqq\{\mf p:\mf m(\mf p)\ne0\}$ be the \textit{support} of $\mf m$, and we define $I^{S(\mf m)}$ to be the subgroup of fractional ideals freely generated by $S(\mf m)$. Then we define
	\[{\op{Cl}^{\mf m}_K}\coloneqq I^{S(\mf m)}/K_{\mf m,1},\]
	where $K_{\mf m,1}\coloneqq\{\alpha\in K^\times:\alpha\equiv1\pmod{\mf p^{\mf m(\mf p)}}\text{ for }\mf p\in S(\mf m)\}$.
\end{definition}
\begin{definition}
	Fix a modulus $\mf m$ of a totally imaginary number field $K$. Then we define
	\begin{align*}
		\AA_{K,\mf m}^\times &\coloneqq \prod_{v\nmid\mf m}(K_v^\times,\OO_v^\times)\times\prod_{v\mid\mf m}\left(1+\mf p_v^{\mf m(\mf p_v)}\OO_v\right), \\
		U_{\mf m} &\coloneqq\prod_{v\mid\mf m}\left(1+\mf p_v^{\mf m(\mf p_v)}\OO_V\right)\times\prod_{v\nmid\mf m\infty}\OO_v^\times, \\
		W_{\mf m} &\coloneqq \prod_{\substack{v\nmid\mf m\\v\mid\infty}}K_v^\times U_{\mf m}, \\
		C_{\mf m} &\coloneqq \AA_{E,\mf m}^\times/K_{\mf m,1}W_\mf m.
	\end{align*}
\end{definition}
\begin{remark}
	One can see that the $U_{\mf m}$s form an open neighborhood basis of $1$ in $\AA_{f,E}^\times/E^\times$, so it forms an open neighborhood basis of $1$ in $T(\AA_f)/T(\QQ)$ upon intersection. Now, $\eta$ and $\op N_\Phi$ are both continuous, essentially by their definition, so we are granted a modulus $\mf m$ such that $\op N_\Phi$ actors as $\mathrm{Cl}^{\mf m}_{E^*}\to C_{\mf m}(E)$.
\end{remark}
We now state a version of \Cref{thm:st-proved} which will help us with our ideal-theoretic main theorem of complex multiplication.
\begin{theorem} \label{thm:ideal-theoretic-main}
	Fix an abelian variety $A$ over $\ov\QQ$ with CM type $(E,\Phi)$. Let $E^*$ be the reflex field, and fix $\sigma\in\op{Gal}(\ov\QQ/E^*)$, and choose a nonnegative integer $m$. We will assume that $\op{End}A\cap E=\OO_E$. Then the following are true.
	\begin{listalph}
		\item There is an ideal $\mf a(\sigma)\subseteq\OO_E$ and isogeny $\alpha\colon A\to A^\sigma$ such that $\alpha(x)=\sigma(x)$ for $x\in A[m]$ and $\alpha$ is an $\mathfrak a(\sigma)$-multiplication. In fact, the ideal class $[\mf a(\sigma)]$ in $\op{Cl}^{m}_E$ is uniquely determined by $\sigma$.
		\item For sufficiently large modulus $\mf m$ of $E^*$, the class $[\mf a(\sigma)]$ only depends on the action $\sigma$ on the ray class field $L_{\mf m}$ of $\mf m$, and $[\mf a(\sigma)]=[\op N_\Phi(\mf a^*)]$ where $\mf a^*\in C_{\mf m}$ corresponds to $\sigma$ via the reciprocity map $\op{Gal}(L_{\mf m}/E)\cong\op{Cl}^{\mf m}_E$.
	\end{listalph}
\end{theorem}
Wait what does $\mf a(\sigma)$-multiplication mean?
\begin{definition}
	Fix an abelian variety $A$ with complex multiplication by $E$ such that $\OO_E\subseteq\op{End}A$. Fix an ideal $\mf a$ of $\OO_E$. A surjective homomorphism $\lambda\colon A\to B$ is an $\mathfrak a$-multiplication if and only if each $a\in\mf a$ has the map $a\colon A\to A$ factor through $B$, and $\lambda$ is in fact universal with respect to this factoring. (Namely, any other $\lambda'\colon A\to B'$ similarly factoring has a unique map $B'\to B$ commuting with everything in sight.)
\end{definition}
\begin{remark}
	If $E$ is not a field but instead merely a CM algebra, then we must make $\mf a$ into a lattice instead of an ideal.
\end{remark}
\begin{remark}
	For any lattice $\mf a\subseteq E$, there is some $(B,\lambda)$ satisfying the required universal property. Indeed, simply take $B\coloneqq A/\ker\mf a$, where
	\[\ker\mf a\coloneqq\bigcap_{a\in\mf a}\ker a,\]
	which we note is actually a finite intersection because $\mf a$ is finitely generated.
\end{remark}
\begin{remark}
	One expects to have $\op{Art}(s)=\sigma$ yielding $\eta(\sigma)$ corresponding to the class $[\mf a(\sigma)]^{-1}$, which will be able to provide the required result.
\end{remark}

\end{document}