% !TEX root = ../notes.tex

\documentclass[../notes.tex]{subfiles}

\begin{document}

\section{April 26}

Today we finish up with Iskovskikh's examples. There will be make-up lectures on Wednesday and Friday next week, at the same time and place.

\subsection{Finishing Iskovskikh's Examples}
We want to show that $X(\AA_\QQ)^{\mathrm{Br}}=\emp$. We don't have many tools for this: we have to just pick up some point $(x_v)_v\in X(\AA_\QQ)$, for which we claim
\[\sum_{v\in V_\QQ}\op{inv}_vx_v^*[A]=\frac12\]
always, which produces our Brauer--Manin obstruction. Here, note that $x_v^*[A]$ sometimes might require passing to a different presentation of $A$ should $A$ not actually be defined at this point. In particular, we must deal with $v$ on cases.
\begin{itemize}
	\item We take $p\notin\{2,\infty\}$. Here, let $\alpha$ denote the $x$-coordinate of our point $x_p\in X(\QQ_p)$. We have the following cases on $v_p(\alpha)$.
	\begin{itemize}
		\item If $v_p(\alpha)<0$, then $3/\alpha^2-1\in\ZZ_p^\times$, so we work with $[C]$ so that $x_p^*[A]=x_p^*[C]$ can be represented by $H\left(3/\alpha^2-1,-1\right)$, which lives in $\op{Br}\ZZ_p$. However, one can show that the map
		\[\op{Br}\ZZ_p\to\op{Br}\FF_p\]
		going down to the residue field is a bijection (we technically showed something like this over the course of our discussion of class field theory), so we conclude that $x_p^*[A]=0$ because $\op{Br}\FF_p=0$.
		\item If $v_p(\alpha)\ge0$, then one of $3-\alpha^2$ or $\alpha^2-2$ lives in $\ZZ_p^\times$. Indeed, certainly it each be in $\ZZ_p$, but the sum of these two elements produces $1$, so one must be a unit. As such, we still get to run the above argument using either $A$ or $-B$ as necessary.
	\end{itemize}
	\item We take $p=\infty$. Letting $\pi\colon X\to\PP^1_{x,w}$ be our projection, we claim that $\pi(x_\infty)\ne[1:0]$. Indeed, this would imply that we have
	\[y^2+z^2=\left(3\cdot0^2-1^2\right)\left(1^2-2\cdot0^2\right)=-1<0,\]
	which is impossible over $\QQ_p=\RR$. In particular, we see $2\le\alpha^2\le3$ in order for the above equation to have a solution. Thus, we see $3-\alpha^2>0$ or $\alpha^2-2>0$, so we can work with the class $x_\infty^*A$ or $x_v^\infty B$, respectively, from which it follows that we again have the zero class because positive real numbers are norms from $\CC$.
	\item Lastly, we take $p=2$. As before, we have three cases.
	\begin{itemize}
		\item If $v_2(\alpha)>0$, then $3-\alpha^2\equiv-1\pmod4$, so $3-\alpha^2$ fails to be a norm from $\QQ_2(i)\to\QQ_2$, so we conclude that the quaternion class of $x_2^*A=H^2\left(3-\alpha^2,-1\right)$ that fails to vanish in $\op{Br}\QQ_2$.
		\item If $v_2(\alpha)=0$, then $\alpha^2-2\equiv-1\pmod4$, so we can run the above argument with $B$.
		\item Lastly, if $v_2(\alpha)<0$, then $3/x^2-1\equiv-1\pmod4$, so we can run the above argument with $C$.
	\end{itemize}
\end{itemize}
In total, we conclude that
\[\sum_{v\in V_\QQ}\op{inv}_vx_v^*[A]\ne0.\]
Namely, we can see that it should be $1/2$ because it does double to $0$, as we can see from the $p=2$ case (which is the only nonzero contribution).
\begin{remark}
	The above example appears quite ad-hoc. Professor Olsson does not have a conceptual explanation for what is going on.
\end{remark}

\subsection{Descent Obstructions}
In general, fix a (say) proper $K$-variety $X$. What is going on with the Brauer--Manin obstruction is that a point $x\in X(\Omega)$ for a field $\Omega$ produces pullback maps
\[H^2(X,\mathbb G_m)\stackrel{x^*}\to H^2(\Omega,\mathbb G_m),\]
where the point is that we can say something more concrete about $H^2(\Omega,\mathbb G_m)$. The Brauer--Manin obstruction then tries to put some compatibility conditions on these pullbacks.

More generally, we may fix an algebraic $K$-group $G$. This becomes an \'etale sheaf on $X$ by taking the \'etale open set $U\to X$ to $\op{Mor}_K(U,G)$.
\begin{example}
	Taking $G=\op{GL}_n$ and $U=\Spec R$, we see
	\[\op{Mor}_K(U,G)=\op{GL}_n(R).\]
\end{example}
Now, for our descent obstruction, we work with $H^1(X,G)$ and tell the above story.\footnote{We did not work with $H^1(X,\mathbb G_m)$ earlier because the $H^2(\Omega,\mathbb G_m)$ all vanish.} Let's see an explicit example: take $G=\ZZ/2\ZZ$, and let $X$ be a curve. Here, $H^1(X,\ZZ/2\ZZ)$ corresponds to unramified $2$-to-$1$ maps of curves $D\to X$, and these have a $\ZZ/2\ZZ$-action by our double-cover. Now, given a point $x\in X(\Omega)$ for some field $\Omega$, we produce a pullback map
\[H^1(X,\ZZ/2\ZZ)\to H^1(\Omega,\ZZ/2\ZZ)=\op{Hom}(\op{Gal}(\Omega^{\mathrm{sep}}/\Omega),\ZZ/2\ZZ),\]
where we now see hope because the target is nonzero. In particular, we can partition points in $X(K)$ based on what class they hit in the target, which is better because we can lift such $K$-points up a point in the curve $D$ which is doing our covering.

\end{document}