% !TEX root = ../notes.tex

\documentclass[../notes.tex]{subfiles}

\begin{document}

Today we discuss the fact that an abelian variety with complex multiplication has potentially good reduction everywhere.

\subsection{Potentially Good Reduction Everywhere}
The following definition is our main character.
\begin{definition}[potentially good reduction]
	Fix an abelian variety $A$ over a number field $L$. For a prime $\mf p$ of $K$, we say that $A$ has \textit{potentially good reduction at $\mf p$} if and only if there is a some prime $\mf P$ over $\mf p$ from a finite extension $L_\mf P$ of $K_\mf p$ such that $A_{L_\mf P}$ has good reduction at $\mf P$.
\end{definition}
\begin{remark}
	We already know that $A$ has good reduction at all but finitely many primes of $K$. So if $A$ has potentially good reduction at all primes, we can find a suitably large finite extension $L/K$ such that $A_L$ has good reduction everywhere. Indeed, simply take a single extension $L$ which is okay for one prime $\mf P$ over each prime $\mf p$ of $K$ which originally had bad reduction. Then one may make $L$ larger without losing our good reduction at those primes, but then we can replace $L$ with its Galois closure, and then the primes are permuted transitively by the Galois group, so we will get good reduction over every prime $\mf P'$ over a prime $\mf p$ of $K$ which originally has bad reduction.
\end{remark}
Here is our main result for today.
\begin{proposition} \label{prop:cm-av-good-reduction}
	Fix an abelian variety $A$ over a number field $K$ with complex multiplication. Then $A$ has potentially good reduction everywhere.
\end{proposition}
For this, we will use the following criterion for good reduction.
\begin{theorem}[N\'eron--Ogg--Shafarevich criterion]
	Fix a discrete valuation ring $(R,\mf p,\kappa)$ with fraction field $K$. Then an abelian variety $A$ over $K$ has good reduction if and only if the inertia subgroup $I\subseteq\op{Gal}(\ov K/K)$ acts trivially on $T_\ell A$ for some $\ell$ not dividing $\op{char}\kappa$.
\end{theorem}
\begin{proof}
	We will only sketch the proof because we don't want to get bogged down with the theory of affine algebraic groups.

	For the converse direction, let $\mc A$ (over $R$) be the N\'eron denote the N\'eron model of $A$ over $K$. Then the N\'eron mapping property implies that
	\[\mc A(K^{\mathrm{unr}})[\ell^\bullet]\cong\mc A\left(\OO_{K^{\mathrm{unr}}}\right)[\ell^\bullet]\onto\mc A(\kappa)[\ell^\bullet].\]
	Note that the last map is an isomorphism by Hensel's lemma, namely by our smoothness. This now implies the forward direction: good reduction means that we are proper in the target, so the end becomes $T_\ell A$, but inertia acts trivially on the left, so it must act trivially on the right.

	We now focus on the harder converse direction. Because inertia acts trivially on $T_\ell A$, our left-hand side is just $A(\ov K)[\ell^\bullet]$. (A priori, this would only be the submodule of $A(\ov K)[\ell^\bullet]$ fixed by inertia because we are only looking at the unramified part.) This is somehow ``too big'' for $\mc A$ to be anything other than an abelian variety. Let's explain this. Note $\mc A_\kappa$ is a smooth commutative finite type group scheme over $\kappa$, so it lives in a short exact sequence
	\[0\to\mc A_\kappa^\circ\to\mc A_\kappa\to\mc A_\kappa/\mc A_\kappa^\circ\to0,\]
	where the target is finite, and $\mc A_\kappa^\circ$ lives in some short exact sequence
	\[1\to U\to\mc A_\kappa^\circ\to G\to0,\]
	where $U$ is unipotent and $G$ is semi-abelian (i.e., an extension of an abelian variety $B$ by a torus $T$). (This last clause follows by some structure theory of algebraic groups.) Notably, we see that $\dim A=\dim\mc A_\kappa=\dim U+\dim T+\dim B$. We now examine the torsion everywhere.
	\begin{itemize}
		\item $\#B(\ov\kappa)[\ell^n]$ is $\ell^{n\cdot2\dim B}$.
		\item $\#T(\ov\kappa)[\ell^n]$ is $\ell^{n\dim T}$ because over the algebraically closed field, this will split into $\mathbb G_m^{\dim T}$, which has torsion given by $\mu_{\ell^n}^{\dim T}$.
		\item $\#U(\ov\kappa)[\ell^n]$ is one because unipotent groups are torsion-free.
	\end{itemize}
	Now, sending $n\to\infty$ forces that
	\[2\dim A=2\dim B+\dim T,\]
	so $\dim T=0$ and $\dim B=\dim A$, so $\dim U=0$ as well. Thus, $\mc A_\kappa$ is proper, which can then be lifted to show that $\mc A$ is proper and hence an abelian scheme.\todo{How?}
\end{proof}
We are now ready to prove \Cref{prop:cm-av-good-reduction}.
\begin{proof}[Proof of \Cref{prop:cm-av-good-reduction}]
	We may extend $K$ immediately so that the endomorphisms promised by complex multiplication are all defined. We are going to use a little local class field theory and the fact that the Galois representation $\rho_\ell\colon\op{Gal}(\ov K/K)\to\op{Aut}(T_\ell A)$ is abelian.
	
	Fix some prime $\mf p$ of $K$ which we would like to show that $A$ has potentially good reduction at $\mf p$ (and choose $\ell$ not divisible by $\mf p$). Then we note that
	\[\op{Gal}(\ov K_\mf p/K_\mf p)\subseteq\op{Gal}(\ov K/K)\to\op{Aut}T_\ell A\]
	has abelian image and hence must factor as
	\[\op{Gal}(K_\mf p^{\mathrm{ab}}/K_\mf p)\to\op{Aut}T_\ell A.\]
	Let $I_\mf p$ be the corresponding inertia subgroup so that we want $\rho_\ell(I_\mf p)$ to be trivial after some extension.

	Now, by Local class field theory, $I_\mf p$ contains $\OO_{K_\mf p}^\times$ and hence contains a finite-index subgroup of the form $1+\mf p\OO_{K_\mf p}$. Further, $\op{Aut}T_\ell A$ has an $\ell$-adic topology, and we see that it has $1+\ell\op{End}T_\ell A$ as a finite-index subgroup, which is a pro-$\ell$ group. Now, taking the pre-image of $\op{Aut}(T_\ell A)$'s finite-index neighborhood of the identity and then intersecting with $1+\mf p\OO_{K_\mf p}$ produces a map from a finite-index pro-$p$ subgroup of $I_\mf p$ to a finite-index pro-$\ell$ subgroup of $\op{Aut}T_\ell A$. But such a thing must have finite image, so $\rho_\ell(I_\mf p)$ must still have finite image by going back up the finite index subgroups. However, we can kill this finite image by passing to a finite extension of $K$, so we are done. (Namely, the pre-image of the identity is an open finite index subgroup of $I_\mf p$, so we just extend $K$ enough so that the new inertia subgroup goes in there.)
\end{proof}

\subsection{Honda--Tate Theory}
As a fun application of some of the theory we've built so far will be to classify isogeny classes of abelian varieties over finite fields. Let's state our theorem, which requires the notion of a ``$q$-Weil number.''
\begin{definition}[Weil numbers]
	Fix a prime-power $q$. Then a \textit{$q$-Weil number} is an algebraic integer $\pi$ such that $\left|\sigma(\pi)\right|^2=q$ for any embedding $\sigma\colon\QQ(\pi)\to\CC$. Two $q$-Weil numbers $\pi$ and $\pi'$ are \textit{conjugate}, written $\pi\sim\pi'$, if and only if there is an isomorphism $\QQ(\pi)\to\QQ(\pi')$ sending $\pi\mapsto\pi'$. (In other words, $\pi$ and $\pi'$ have the same minimal polynomial, which is equivalent to $\pi$ and $\pi'$ being Galois conjugates.)
\end{definition}
\begin{remark}
	Let's explain where this notion is coming from. Well, fix an abelian $\FF_q$-variety $A$. Then we know that
	\[{\op{Frob}_A^\dagger}\circ{\op{Frob}_A}=[q],\]
	so $\pi_A\coloneqq\op{Frob}_A$ has that $\QQ(\pi_A)$ is semisimple (and hence a field when $A$ is simple), so the Albert classification explaining how to embed this into $\CC$ tells us that $\pi_A$ is a $q$-Weil number.
\end{remark}
Here is our result.
\begin{theorem}[Honda--Tate] \label{thm:ht}
	Fix a prime power $q$. There is a bijection between isogeny classes of simple abelian $\FF_q$-variety $A$ and conjugacy classes of $q$-Weil numbers $\pi$ given by sending $A\mapsto\op{Frob}_A$.
\end{theorem}
The injectivity of the map $A\mapsto\op{Frob}_A$ is due to Tate. We will not prove this, but here is the precise statement which Tate proved.
\begin{theorem}[Tate] \label{thm:tate}
	Fix a prime power $q$ and a prime $\ell$ not dividing $q$. Then the Tate functor $T_\ell$ is fully faithful.
\end{theorem}
We already know that $T_\ell$ is faithful, so the main content is showing that this functor is full. This turns out to be rather difficult, though it is not too far outside the scope of the current course. The main point here is that we will be able to construct morphisms of abelian varieties only by providing morphisms of the Tate modules.
\begin{corollary}
	Fix a prime power $q$ and a prime $\ell$ not dividing $q$. Then the following are equivalent.
	\begin{listalph}
		\item $A$ and $B$ are $\FF_q$-isogenous.
		\item $V_\ell A\cong V_\ell B$ (as Galois representations) for some prime $\ell$ not dividing $q$.
		\item $V_\ell A\cong V_\ell B$ (as Galois representations) for all primes $\ell$ not dividing $q$.
		\item $P_A(t)=P_B(t)$, where the $P_A$ and $P_B$ are the characteristic polynomials of the Frobenius.
	\end{listalph}
\end{corollary}
\begin{proof}
	We already know that (a) implies (c) (the isogeny provides the isomorphism of the Tate modules), which implies (b) (with no content), which implies (d) by taking the characteristic polynomial on both sides and seeing that the isomorphism forces them to agree.

	We now show the harder implications. To see that (d) implies (c), we note that $\op{Frob}$ is semisimple, so having $P_A=P_B$ implies that $V_\ell(A)=V_\ell(B)$, where the equality even preserves the Frobenius action, and this Frobenius action is the same as the total Galois action because we are over a finite field. Explicitly, $P_A=P_B$ implies that $\op{Frob}_A$ and $\op{Frob}_B$ are conjugate on the Tate module (base-changed to $\ov\QQ$) because they have the same eigenvalues; this then descends to an isomorphism to $V_\ell A\cong V_\ell B$ preserving Frobenius by Hilbert's theorem 90 by Galois descent for representations. (Namely, any obstruction to descent would be a $1$-cocycle in a vanishing cohomology group.)\footnote{Here is another argument: $P_A=P_B$ implies that one can explicitly write down what $V_\ell A$ and $V_\ell B$ should be and then show that they are isomorphic.}

	It remains to show that (c) implies (a), which will follow from \Cref{thm:tate}. Namely, having two isomorphic Galois representations provides inverse maps on the level of Tate modules, which can then be lifted to inverse maps of the abelian varieties (up to multiplication by an integer, which is an isogeny), which is what we wanted.
\end{proof}
\begin{remark}
	Without much more work, we can upgrade this to state that the following are equivalent.
	\begin{listalph}
		\item There is an isogeny of $A$ onto an abelian subvariety of $B$.
		\item $V_\ell A$ is a Galois sub-representation of $V_\ell B$.
		\item $P_A$ divides $P_B$.
	\end{listalph}
\end{remark}
We now see that the equivalence of (a) and (d) implies that the map sending $A$ to the conjugacy class of $q$-Weil numbers given by $\op{Frob}_A$ will be injective, which is the injectivity required in \Cref{thm:ht}. Let's be more explicit about this: if $\pi_A=\pi_B$, then they have the same minimal polynomial, so one of $P_A$ will have to divide $P_B$ (using the remark), so one of $A$ or $B$ is isogenous to an abelian subvariety of the other, but then simplicity forces full isomorphism.

\end{document}