% !TEX root = ../notes.tex

\documentclass[../notes.tex]{subfiles}

\begin{document}

\section{April 24}

We continue with Iskovskikh's examples.

\subsection{Iskovskikh's Examples}
For brevity, we today set $F(x,w)\coloneqq\left(3w^2-x^2\right)\left(x^2-2w^2\right)$. Let's be more explicit about our construction from last class. Namely, to define $s\in\Gamma\left(\PP^1_{x,w},\op{Sym}^2\mc E\right)$, it suffices to define our section on the affine lines $\AA^1_x$ and $\AA^1_w$. Over $\AA^1_x\times\PP^2_{y,z,v}$, we see that $s$ becomes $y^2+z^2-F(x,1)v^2$, where we are choosing global sections $y$, $z$, and $v$ to trivialize
\[\mc E=\OO_{\PP^1}\oplus\OO_{\PP^1}\oplus\OO_{\PP^1}(2).\]
We note that $U$ will thus embed into $X|_{\AA^1_x}$. On the other side, we see that $s=y^2+z^2-F(1,w)(v')^2$ on $\AA^1_w\times\PP^2_{y,z,v'}$ by a similar argument.

We now run some checks.
\begin{itemize}
	\item We must check that $X$ is a smooth surface. This is a matter of writing everything out on the standard affine open subschemes.
	\item The map $\pi\colon X\to\PP^1_{x,w}$ has exactly four singular fibers. Indeed, fix some $[\alpha:\beta]\in\PP^1_{x,w}$. Then the fiber we are looking at is cut out by
	\[y^2+z^2-F(\alpha,\beta)v^2\]
	in $\PP^2_{y,z,v}$. If $F(\alpha,\beta)\ne0$, then we are defining a degree-$2$ curve rational curve in $\PP^2$ given by $y^2+z^2-F(\alpha,\beta)v^2$; we see that this is nonsingular (for example by checking affine-locally). However, if $F(\alpha,\beta)=0$, then we are looking at $y^2+z^2=0$ in $\PP^2_{y,z,v}$, which we can see is not smooth because it is the union of two lines intersecting at the origin.
	\item We have $X\left(\AA_\QQ\right)\ne\emp$. Running the above construction with $\ZZ$ as our base scheme will produce a scheme $\mc X$ over $\PP^1_\ZZ$. We claim that $\mc X_{\FF_p}$ has an $\mathbb R$-point and a smooth point for each prime $p$, which by Hensel's lemma will produce a $\ZZ_p$-point for each $p$, thus giving an $\mathbb A_\QQ$-point.

	Well, for $p\notin\{2,3\}$, we see that $\mc X_{\FF_p}\to\PP^1_{\FF_p}$ has a smooth fiber geometrically isomorphic to $\PP^1_{\FF_p}$ by the above computation. However, the Brauer group of $\FF_p$ vanishes, so because this fiber is geometrically isomorphic to $\PP^1_{\FF_p}$ means that this fiber must be isomorphic to $\PP^1_{\FF_p}$ on the nose; namely, any Brauer--Severi scheme over $\FF_p$ must be $\PP^1_{\FF_p}$.\footnote{One could in theory write down the point explicitly, but it would probably require some casework.}

	Continuing, for $\RR$, we note that our equation
	\[y^2+z^2=\left(3-x^2\right)\left(x^2-2\right),\]
	so it suffices to choose $x$ for which $2<x^2<3$.

	For $p=2$ and $p=3$, one can find the necessary points by hand and check that they are smooth. For $p=3$, essentially the same argument as above will hold, but $p=2$ needs some care.

	\item We have $X\left(\AA_\QQ\right)^{\mathrm{Br}}=\emp$. Set $K\coloneqq K(X)$, and let $A$ be the quaternion algebra $H\left(3-x^2,-1\right)$. In particular, as shown on the homework, taking quaternion algebras produces a $\ZZ$-bilinear antisymmetric map
	\[\frac{K^\times}{K^{\times2}}\times\frac{K^\times}{K^{\times2}}\stackrel H\to H^2\left(\op{Gal}(K^{\mathrm{sep}}/K),K^{\mathrm{sep}\times}\right).\]
	As an aside, we pick up the following fact: if $K(i)/K$ is a degree-$2$ extension, then we have isomorphisms
	\[\frac{K^\times}{\op N(K(i)^\times)}=\widehat H^0(\op{Gal}(K(i)/K),K(i)^\times)\cong H^2(\op{Gal}(K(i)/K),K(i)^\times)\subseteq H^2\left(\op{Gal}(K^{\mathrm{sep}}/K),K^{\mathrm{sep}\times}\right).\]
	One can check, as done on the homework that the class $a$ on the left goes to the quaternion algebra $H(a,1)$ on the right.

	With this in mind, we also write down
	\[B\coloneqq H\left(x^2-2,-1\right)\qquad\text{and}\qquad C\coloneqq\left(3/x^2-1,-1\right).\]
	Using bilinearity of $H$, one can compute $A+B=\left(y^2+z^2,-1\right)$, but $y^2+z^2$ is a norm of $y+zi\in K(i)$, so $A+B=0$, so $A=-B$. Analogously, we note that $C$ differs from $A$ only by a square, so $A=C$.

	Now, $A$ will define an Azumaya algebra over all points $V_A$ in $X$ except where $3-x^2$ has a zero or pole in $\PP^1_{x,w}$. Similarly, $B$ defines an Azumaya algebra over all points $V_B$ except where $x^2-2$ has zeroes or poles, and $C$ defines an Azumaya algebra over all points $V_C$ where $3/x^2-1$ has zeroes or poles. But one can see that $V_A\cap V_B\cap V_C=\emp$, so $A$ suitably extends to all codimension-$1$ points (as either $-B$ or $C$), meaning that $A$ will produce a bona fide algebra Azumaya algebra on $X$ by \eqref{eq:grothendieck-purity}.

	The last thing to check that is that any local point $(x_v)_v\in X(\AA_\QQ)$ will have
	\[\sum_{v\in V_\QQ}\op{inv}_vx_v^*[A]=\frac12\]
	always, which produces our Brauer--Manin obstruction.
\end{itemize}

\end{document}