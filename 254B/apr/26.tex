% !TEX root = ../notes.tex

\documentclass[../notes.tex]{subfiles}

\begin{document}

Today we complete the proof, for real this time.

\subsection{Completing the Proof}
We are going to combine the Shimura--Taniyama formula with \Cref{thm:ideal-theoretic-main} to conclude our proof. Fix an abelian variety $A$ over a number field $K$ with CM type $(E,\Phi)$, and suppose that $K$ is Galois and contains all Galois conjugates of $E$. Fix a prime $p$, and let $\mf p$ be a prime of $E^*$ above $p$, and let $\mf P$ be a prime of $K$ above $\mf p$. We will further assume that $K_{\mf P}/\QQ_p$ is unramified and that $\OO_E\subseteq\op{End}A$.
\begin{corollary}
	Fix everything as above.
	\begin{listalph}
		\item There exists an $\mathfrak a$-multiplication $\alpha\colon A\to A^\sigma$ (defined over a finite extension of $K$) where $\sigma\in\op{Gal}(K/E^*)$ reduces to the Frobenius automorphism of $\kappa(\mf P)/\kappa(\mf p)$.
		\item In fact, $\mf a=\op N_\Phi(\mf p)$.
	\end{listalph}
\end{corollary}
\begin{proof}
	From \Cref{thm:ideal-theoretic-main}, we get some $f\colon A\to A^\sigma$ which is a $\mf b$-multiplication, so the same is true after passing to the reductions $A_0$ and $A_0^\sigma$ over $\kappa(\mf P)$, for example by considering the construction of $\mf b$-multiplications as a tensor product. Now, because we have $\mf b$-multiplications, we see
	\[\op{Hom}_E(A,A^\sigma)=\mf b^{-1}=\mf b^{-1}=\op{Hom}_E(A,A^\sigma),\]
	so our Frobenius $A_0\to A_0^\sigma$ lifts to $\alpha\colon A\to A^\sigma$ (where we are implicitly using the N\'eron mapping property).

	Now, for (b), the point is that we will be able to take powers to recover the Frobenius. Namely, we know from \Cref{thm:st} that there is $\pi\in\OO_E$ which is a lift of the endomorphism $x\mapsto x^{\#\kappa(\mf P)}$ on the reduction $A_0$. Now, we know that $(\pi)$ is $\op N_{K,\Phi}(\mf P)$ be a valuation computation (and everything in sight being unramified). Continuing, we compute
	\[\op N_{K,\Phi}(\mf P)=\op N_\Phi\left(\op{N}_{K/E^*}\mf P\right)=\op N_\Phi(\mf p)^{f(\mf P/\mf p)}.\]
	Now, we note that we can write $\pi=\alpha\cdot\sigma\alpha\cdots\sigma^{f(\mf P/\mf p)-1}\alpha$, where $\sigma$ is the relative $\#\kappa(\mf p)$-power Frobenius, but this twisting does not adjust which ideal we are going to live in, so $(\pi)=\mf a^{f(\mf P/\mf p)}$. The equality in (b) follows.
\end{proof}
We now show the main theorem. Reciprocity tells us that $\sigma$ corresponding to the Frobenius element corresponds to $\mf p\in\op{Cl}^{\mf m}(E^*)$. Thus, the above result shows the result for all Frobenius corresponding to $\mf p$ in the case where $\mf p$ is unramified in $K/E^*$ and where $\mf p$ is unramified in $E^*/\QQ$. However, such $\mf p$ have their Frobenius elements are dense in the Galois group $\op{Gal}(\ov\QQ/E^*)$, so we are okay because everything in sight is continuous.
\begin{remark}
	To recover the ad\'elic statement, one finds that $\eta(\sigma)$ in $\op{Cl}^{\mf m}(E^*)$ is $\mf a(\sigma)^{-1}$ by unwinding the definition of the corresponding $\alpha\colon A\to A^\sigma$ in the ad\'elic language.
\end{remark}

\subsection{A Little on the Andr\'e--Oort Conjecture}
Here is our result.
\begin{theorem}
	Fix an irreducible polynomial $P\in\CC[j,j']$. If $P$ uses both variables, and $P$ is not divisible by $P_N$ (which is the defining equation for the subscheme $Y_0(N)\subseteq\AA^1\times\AA^1$ of pairs $(E_1,E_2)$ for which there is a cyclic $N$-isogeny $E_1\to E_2$), then there are only finitely many pairs $(j_n,j_n')$ corresponding to points with complex multiplication such that $P(j_n,j_n')=0$.
\end{theorem}
Geometrically, we should imagine $P$ as cutting out an irreducible curve in $\AA^2$, which is being viewed as a coarse moduli space for elliptic curves. Essentially, we are saying that if $C(\CC)$ has infinitely many CM points, then either $C$ is $X_0(N)$, vertical, or horizontal.
\begin{remark}
	The Andr\'e--Oort conjecture is about this story for general Shimura variety, which was recently proved.
\end{remark}
Anyway, here is our proof.
\begin{proof}
	Suppose for the sake of contradiction that we have an infinite sequence of points $(j_n,j_n')$ on which $P$ vanishes.
	\begin{enumerate}
		\item We reduce to the case where $P$ has rational coefficients, and $P$ is irreducible over $\QQ$. Well, the points $(j_n,j_n')$ all live in $\ov\QQ$ because these points have complex multiplication, so $P$ being irreducible with all these roots requires $P$ to have coefficients in $\ov\QQ$. However, $P$ has only finitely many coefficients, so say they live in a number field $F$. By replacing $P$ with an irreducible factor of
		\[\prod_{\sigma\colon F\to\ov\QQ}\sigma(P)\in\QQ[j,j']\]
		divisible by $P$, we maintain all of our roots but now live in our reduced case.

		\item We set some notation. Let $E_n$ and $E_n'$ be the elliptic curves with $j$-invariant $j_n$ and $j_n'$. Then we set $\OO_n\coloneqq\op{End}((E_n)_{\ov\QQ})$ and $K_n\coloneqq\op{Frac}\OO_n$ and $d_n\coloneqq\op{disc}K_n$ and $D_n\coloneqq\op{disc}D_n$, which is $f_n^2d_n$ for some $f_n$. We also set $h_n\coloneqq\#\op{Cl}\OO_n$.

		Now, for $n$ very large, we claim that $K_n=K_n'$ and $D_n'/D_n$ lives in some finite set. We will basically show that there are not so many possibilities with $K_n\ne K_n'$, so for the moment, we drop the $n$ from our notation. Set $L\coloneqq KK'$ and $M\coloneqq L(j)\cap L(j')$. Then we have the following tower of fields.
		% https://q.uiver.app/#q=WzAsNSxbMSwzLCJMIl0sWzEsMiwiTSJdLFswLDEsIkwoaikiXSxbMiwxLCJMKGonKSJdLFsxLDAsIkwoaixqJykiXSxbMCwxLCIiLDAseyJzdHlsZSI6eyJoZWFkIjp7Im5hbWUiOiJub25lIn19fV0sWzEsMywiIiwwLHsic3R5bGUiOnsiaGVhZCI6eyJuYW1lIjoibm9uZSJ9fX1dLFszLDQsIiIsMCx7InN0eWxlIjp7ImhlYWQiOnsibmFtZSI6Im5vbmUifX19XSxbMiw0LCIiLDIseyJzdHlsZSI6eyJoZWFkIjp7Im5hbWUiOiJub25lIn19fV0sWzEsMiwiIiwyLHsic3R5bGUiOnsiaGVhZCI6eyJuYW1lIjoibm9uZSJ9fX1dXQ==&macro_url=https%3A%2F%2Fraw.githubusercontent.com%2FdFoiler%2Fnotes%2Fmaster%2Fnir.tex
		\[\begin{tikzcd}
			& {L(j,j')} \\
			{L(j)} && {L(j')} \\
			& M \\
			& L
			\arrow[no head, from=2-1, to=1-2]
			\arrow[no head, from=2-3, to=1-2]
			\arrow[no head, from=3-2, to=2-1]
			\arrow[no head, from=3-2, to=2-3]
			\arrow[no head, from=4-2, to=3-2]
		\end{tikzcd}\]
		Now, the degrees in the square are all bounded in degree by $P$, but the degree of $L(j)/L$ by some explicit class field theory is either $h$ or $h/2$. All of this is able to imply that $D$ and $D'$ are all bounded, which proves our claim. Namely, $h$ is proportional to $\sqrt D$ by the Brauer--Siegel theorem, but Gauss genus theory tells us that the $2$-torsion of the class group is $2$ to the power of the number of primes dividing $D$. As such, one can relate $h(\OO)$ to $h(\OO_K)$ to achieve the bounding.

		From here, we are able to take $K=K'$ or the remainder of our argument. It remains to bound $D'/D$. Well, on one hand, $[K(j,j'):K(j')]$ is bounded above by $\deg P$, but on the other hand, it is bounded above by
		\[\frac{\op{lcm}(f,f')}{f}\prod_{p\mid\lcm(f,f')/f}\left(1-\left(\frac dp\right)\frac1p\right)\]
		where $f$ and $f'$ are chosen so that $\OO=\ZZ+f\OO_K$ and $\OO'=\ZZ+f'\OO_{K'}$. This tells us that $f/f'$ is bounded, so we are done.

		\item For the remainder of our proof, we will assume that all the $K_\bullet$s and $K'_\bullet$s are all the same. We go ahead and throw out $j$-invariants in the same Galois orbit. Now, we define the notation $E_n\coloneqq\CC/\OO_n$ with $\tau_n\coloneqq\frac12(D_n+\sqrt{D_n})$. In particular, one can show that $\log\left|j_n\right|\approx\Im\tau_n\approx\left|D_n\right|^{1/2}$.

		Now, we claim that $j_n'\to\infty$ as $n\to\infty$. Well, we choose a fundamental domain $\mc F$, which is compact, so the $\tau_n$s must converge somewhere if they are unbounded. But then one can show
		\[\left|D_n\right|^{1/2}\sim\log\left|j_n\right|\sim-\log\left|j_n'-j_\infty'\right|\ll\OO(\log\left|D_n'\right|),\]
		which is a problem.

		\item To complete the proof, one passes to a subsequence to get inside $Y_0(N)$.
		\qedhere
	\end{enumerate}
\end{proof}

\end{document}