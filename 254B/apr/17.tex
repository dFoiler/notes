% !TEX root = ../notes.tex

\documentclass[../notes.tex]{subfiles}

\begin{document}

Today we complete the proof of the surjectivity of Honda--Tate theory.

\subsection{Finishing Honda--Tate Theory}
We continue our construction of the required CM field.
\begin{proposition}
	Fix a $q$-Weil number $\pi$, and set $K\coloneqq\QQ(\pi)$. Let $D$ be a $K$-division algebra satisfying the local conditions of \Cref{thm:div-alg-of-simple-av-fq}(c). Then there is a CM field $L$ such that $D\otimes_KL$ splits at all places of $L$, and $[L:K]=\sqrt{[D:K]}$.
\end{proposition}
\begin{proof}
	Once again, we leave the case where $K$ is totally real to homework. As a quick sketch, one takes $L\coloneqq K(\sqrt{-p})$ where $p\coloneqq\op{char}\FF_q$.

	Otherwise, $K$ is CM with totally real subfield $K^+\coloneqq\QQ(\pi+q/\pi)$. Set $d\coloneqq\sqrt{[D:K]}$ to be our reduced degree. Now, there exists a totally real extension $L^+$ of $K^+$ of degree $d$ such that each place $v_0$ of $K^+$ above $p$ remains inert in $L^+$: simply construct an irreducible polynomial (with real roots) which remains irreducible over all our finitely many places, which comes down to some explicit construction. Then $L\coloneqq KL^+$ is CM over $L^+$ and has the correct degree.

	It remains to check that $D\otimes_KL$ splits at all places of $L$. Note $D$ already splits at all places not above $p$, so we just need to check that we split at the places above $p$. Well, for each $w\in V(L)$ above $v\in V(K)$ above $p$, we see
	\[\op{inv}_w(D\otimes_KL)=[L_w:K_v]\op{inv}_v(D),\]
	which vanishes because $\op{inv}_v(D)$ vanishes once multiplied by $d$ by some facts of central simply algebras.
\end{proof}
\begin{remark}
	We quickly recall \cite[Corollary~IV.3.7]{milne-cft}. Under the assumption $[L:K]=\sqrt{[D:K]}$, then one knows that $D\otimes_KL$ splitting everywhere locally implies splitting globally (by the fundamental exact sequence), which is equivalent to having a $K$-algebra embedding $L\subseteq D$.
\end{remark}
We now produce our abelian variety.
\begin{proposition}
	Fix a $q$-Weil number $\pi$, and set $p\coloneqq\op{char}\FF_q$ and $K\coloneqq\QQ(\pi)$. Then there is an abelian scheme $\mc A$ over $\OO_{K'}$ where $K'$ is a finite extension of $\QQ_p$ such that $\mc A_{K'}$ admits CM by the $L$ constructed in the previous proposition and $\mc A_\kappa$ (where $\kappa$ is the residue field) has Frobenius conjugate to $\pi^N$ for some positive integer $N$.
\end{proposition}
\begin{proof}
	By the complex theory, it is enough to construct the required CM type $(L,\Phi)$. Then we can take $K'$ large enough so that $\mc A$ has good reduction everywhere, and we will use the Shimura--Taniyama formula to check what's going on with the Frobenius. Well, let's recall from \Cref{thm:st} that $\Phi\subseteq\op{Hom}(L,\ov\QQ_p)$, so we define $\Phi_w\coloneqq\Phi\cap H_w$, and we would like
	\[\frac{\op{ord}_w(\pi_{\mc A_\kappa})}{\op{ord}_w(\#\kappa)}\stackrel?=\frac{\#\Phi_w}{\#H_w}.\]
	To continue, we want the following lemma, which explains why we need $\pi^N$ in our construction if we are only ever going to use the above condition from $\Phi$.
	\begin{lemma}
		Fix a Weil $q$-number $\pi$ and $q'$-Weil number $\pi'$, where $p\coloneqq\op{char}\FF_q=\op{char}\FF_{q'}$. If $K$ is a field containing $\QQ(\pi,\pi')$ and
		\[\frac{\op{ord}_w(\pi)}{\op{ord}_w(q)}=\frac{\op{ord}_w(\pi')}{\op{ord}_w(q')}\]
		for any place $w\in V(K)$ above $p$, then $(\pi')^{a'}=\pi^a$ for some positive integers $a$ and $a'$.
	\end{lemma}
	\begin{proof}
		By taking powers of $\pi$ and $\pi'$ (which continue to be Weil numbers), we may assume that $\pi\ov\pi=\pi'\ov{\pi'}$, meaning $q=q'$. We now want to show that $\pi/\pi'$ is a root of unity (so that they will become equal after taking more powers). But we know that
		\[\left|\tau\pi\right|=\left|\tau\pi'\right|\]
		for any embedding $\tau\colon\QQ(\pi)\to\CC$, so it will be enough to check that $\pi/\pi'$ is an algebraic integer. Well, for any place $w$ not above $p$, we know that $w(\pi)=w(\pi')=0$ because $\pi\ov\pi=\pi'\ov{\pi'}$ are powers of $p$. And for any place $w$ of $p$, the hypothesis tells us that $w(\pi)=w(\pi')$ still. Thus, we are able to conclude that $\pi/\pi'$ is an algebraic integer, all of whose archimedean norms are $1$, so it is a root of unity.
	\end{proof}
	We now continue with the proof with the above lemma in mind. Let's quickly explain how to construct $\Phi$ so that
	\[\#\Phi_w\stackrel?=H_w\cdot\frac{\op{ord}_w(\pi_{\mc A_\kappa})}{\op{ord}_w(\#\kappa)}=[L_w:K_v]\cdot[K_v:\QQ_p]\cdot\frac{\op{ord}_v(\pi)}{\op{ord}_v(q)}.\]
	Note the product of the central and right factors on the rightmost side is $\op{inv}_v(D)$, which we know becomes an integer after multiplying by $[L_w:K_v]$ by construction of $L$ at this place. So we may choose $\Phi_w\subseteq H_w$ somewhat randomly to have the right number of elements. The only extra constraint on $\Phi$ is to have $\Phi\sqcup \ov\Phi$ to be the full $\op{Hom}(L,\ov\QQ_p)$, which amounts to requiring
	\[\#\Phi_w+\#\Phi_{\ov w}\stackrel?=\#H_w=\#H_{\ov w}\]
	after rearranging our chosen $\Phi_w$ appropriately. But comparing what we are requiring about $\#\Phi_w$, we see we are asking for $\op{ord}_v(\pi)+\op{ord}_{\ov v}(\pi)=\op{ord}_v(q)$, which is true because $\op{ord}_{\ov v}(\pi)=\op{ord}_v(\ov\pi)$.

	So in total, we have constructed a special CM type $(L,\Phi)$, which produces an abelian variety over some number field with the correct CM type by our Galois descent arguments from much earlier, and then the theory of N\'eron models provides us with our CM abelian scheme $\mc A$ with CM type $(L,\Phi)$. Then \Cref{thm:st} grants
	\[\frac{\op{ord}_w(\pi_{\mc A_\kappa})}{\op{ord}_w(\#\kappa)}=\frac{\#\Phi_w}{\#H_w}=\frac{\op{ord}_w(\pi)}{\op{ord}_w(q)}\]
	for any prime $w\in V(L)$ above $p$, and then the lemma tells us that $\pi$ is realized up to a power as the Frobenius $\pi_{\mc A_\kappa}$. Note we can base-change $\mc A$ a little further in order to replace $\pi_{\mc A_\kappa}$ with a higher power, so we are done.
\end{proof}
We are now ready to prove the surjectivity of \Cref{thm:ht}. Thus far, for our $q$-Weil number $\pi$, we have produced an abelian variety $A$ over a large finite field $\kappa$ such that $\pi^N=\pi_A$. Note that we must have $\#\kappa=q^N$ because $\pi_A\ov{\pi_A}=\left|\#\kappa\right|$. To complete the proof, we use Weil restriction, and we will leave some details to the homework.
\begin{definition}[Weil restriction]
	Fix a finite field extension $L/K$. Given an $L$-group $G$, we define the \textit{Weil restriction} $\op{Res}_{L/K}G$ on $R$-points (for $R\in\op{Alg}_K$) by
	\[\op{Res}_{L/K}G(R)\coloneqq G(R\otimes_KL).\]
\end{definition}
\begin{remark}
	On the homework, we will show that
	\[V_\ell(\op{Res}_{L/K}A)\cong\op{Ind}_{\op{Gal}(\ov K/K)}^{\op{Gal}(\ov L/L)}V_\ell A\]
	for any finite extension $L/K$ of fields.
\end{remark}
Using the previous remark, we set $B\coloneqq\op{Res}_{\kappa/\FF_q}A$ and see that the action of $\op{Frob}_{A,\FF_q}^N$ on $\op{Res}_{\kappa/\FF_q}A$ is going to be $\op{Frob}_{A,\kappa}$, which then splits up as conjugation by cosets on the induction on each piece of the induction $V_\ell B=\op{Ind}V_\ell A$,\todo{What?} so we see that $\pi^N_{B}=\pi_A$ and $P_{B}(t)=P_A\left(t^N\right)$, meaning $\pi$ is a root of $P_B$, so $\pi_B$ is conjugate to $\pi$, completing our surjectivity construction.

\end{document}