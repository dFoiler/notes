% !TEX root = ../notes.tex

\documentclass[../notes.tex]{subfiles}

\begin{document}

\section{April 24}

We provide an example today.

\subsection{Writing Down the Brauer Group}
The goal for today is to construct a smooth projective $\QQ$-surface $X$ such that $X(\AA_\QQ)\ne\emp$ but $X(\AA_\QQ)^{\mathrm{Br}}=\emp$. In other words, the local-to-global principle fails and is detected by the Brauer--Manin obstruction.

Quickly, we recall our definition of the Brauer group. Precisely, we are looking at the isomorphism classes of Azumaya algebras over $X$, where we mod out by the equivalence relation trivializing $\mathcal End(\mathcal E)$ for locally free sheaves $\mc E$ over $X$ of finite rank. In other words, we say that $\mc A\sim\mc A'$ if and only if there are vector bundles $\mc E$ and $\mc E'$ such that
\[\mc A\otimes\mathcal End(\mc E)\cong\mc A'\otimes\mathcal End(\mc E').\]
Notably, an Azumaya algebra $\mc A$ of rank $n^2$ has a simple transitive $\op{PGL}_n$-action on the \'etale sheaf
\[\mc I_\mc A(U)\coloneqq\{\text{isomorphisms }\mc A|_U\cong M_n(\OO_U)\},\]
so we are producing an element of $H^1(X_{\mathrm{et}},{\op{PGL}_n})$. However, we have an exact sequence
\[1\to\mathbb G_m\to{\op{GL}_n}\to{\op{PGL}_n}\to1,\]
of \'etale sheaves, so we get a map $H^1(X,\mathrm{PGL}_n)\to H^2(X,\mathbb G_m)$, so there is an inclusion $\op{Br}(X)\to H^2(X,\mathbb G_m)$. Indeed, we have the exact sequence
\[H^1(X,\mathrm{GL}_n)\to H^1(X,\mathrm{PGL}_n)\to H^2(X,\mathbb G_m),\]
and one can check that the previously defined equivalence relation is exactly dictated by the ``locally free'' objects coming from $H^1(X,\mathrm{GL}_n)$.
\begin{theorem}
	For reasonable $X$ (for example, it has an ample line bundle), then the inclusion $\op{Br}(X)\to H^2(X,\mathbb G_m)$ is an isomorphism.
\end{theorem}
There are counterexamples, though we do not have a good understanding in general. For example, we do not know if we may merely assume that $X$ is a smooth variety.

Let's talk a bit about ``reducing to fields.'' Here is the motivation.
\begin{remark}
	Suppose that $X$ is a smooth, integral, projective $k$-scheme, where $k$ is a field of characteristic $0$. It is then a fact that there is an injection $\op{Br}(X)\to\op{Br}(K(X))$ with image we can explicitly describe; namely, we can describe them using residue maps.
\end{remark}
As such, for our purposes, we will be interested in $R$ being a complete discrete valuation ring (in characteristic $0$) with residue field $k$, and we define $K\coloneqq\op{Frac}R$. We now recall that
\begin{align*}
	\op{Br}K &= H^2(\op{Gal}(K^{\mathrm{sep}}/K),K^{\mathrm{sep}\times}) \\
	&= H^2(\op{Gal}(K^{\mathrm{unr}}/K),K^{\mathrm{unr}\times}) \\
	&\stackrel v\to H^2(\op{Gal}(K^{\mathrm{unr}}/K),\ZZ) \\
	&\from H^1(\op{Gal}(K^{\mathrm{unr}}/K),\QQ/\ZZ) \\
	&\cong H^1(\op{Gal}(k^{\mathrm{sep}}/k),\QQ/\ZZ)
\end{align*}
from class field theory that these are all isomorphisms. For example, the leftward arrow $\from$ is an isomorphism by staring at the exact sequence
\[0\to\ZZ\to\QQ\to\QQ/\ZZ\to0,\]
where the point is that the divisible group $\QQ$ is not going to have any cohomology. Now, in our set-up, it is a theorem of Grothendieck that
\begin{equation}
	0\to\op{Br}X\to\op{Br}(K(X))\to\bigoplus_{x\in X^{(1)}}H^1(\op{Gal}(k(x)^{\mathrm{sep}}/k(x)),\QQ/\ZZ), \label{eq:grothendieck-purity}
\end{equation}
where $X^{(1)}$ means codimension $1$. (In other words, we are summing over divisors.) To understand where the right map is coming from, we note that any class in $\op{Br}K(X)$ can restrict to the corresponding discrete valuation ring in some small neighborhood, complete it, and run through the above set-up.\footnote{One should check that a given algebra $A\in\op{Br}K(X)$ will have residue $0$ at only finitely many codimension-$1$ points. This is a little involved to check; the point is that any $A\in\op{Br}K(X)$ will extend in definition to a nonempty open set $U$, and any codimension-$1$ scheme intersecting $U$ will have vanishing residue. Thus, we only need to be worried about the finitely many codimension-$1$ points living in $X\setminus U$.} Here are some corollaries of this.
\begin{itemize}
	\item We can more or less smooth over the Brauer group. Explicitly, for an open subscheme $U\subseteq X$ such that $X\setminus U$ has codimension at least $2$, we have that the restriction map $\op{Br}(X)\to\op{Br}(U)$ is an isomorphism. Namely, our codimension-$1$ points do not change upon restriction to such a $U$.
	\item Azumaya algebras more or less glue together: if we have an element of $\op{Br}K(X)$ which merely vanishes locally in codimension $1$, we know that it must arise from $\op{Br}X$.
\end{itemize}

\subsection{Iskorskih's Examples}
We are now ready for our example. Define the $\QQ$-scheme $U$ in $\PP^2$ as cut out by the equation
\[y^2+z^2=\left(3-x^2\right)\left(x^2-2\right),\]
and let $j\colon U\to X$ be a smooth projective model for this surface. (Namely, any $U$ is birational to a smooth projective surface, which is our desired $X$.) For brevity, we set $K\coloneqq K(X)$, and we choose $A\coloneqq H\left(3-x^2,-1\right)$ to be the generalized quaternion algebra.

There are now many things to check. To begin, note that $X$ is a conic bundle over $\PP^1$. Namely, note that there is a map $U\to\AA^1_x$ by projecting onto the $x$ coordinate, and also $U$ embeds into $\AA^1_x\times\AA^2_{y,z}$ in the obvious way. To make $X$ a conic bundle, we would like to write $X\subseteq\PP^2$ as cut out by a degree-$2$ homogeneous polynomial. As a bundle over $\PP^1$, we are actually asking for an embedding $X\into\PP\mc E$ where $\mc E$ is a vector bundle over $\PP^1$, and we want to ``cut out'' $X$ by a section $s\in\Gamma\left(\PP^1,\op{Sym}^2\mc E\right)$. Now, we take
\[\mc E\coloneqq\underbrace{\OO_{\PP^1}}_{\mc L_0}\oplus\underbrace{\OO_{\PP^1}}_{\mc L_1}\oplus\underbrace{\OO_{\PP^1}(2)}_{\mc L_2}.\]
We now define $s_0=1\in\Gamma\left(\PP^1,\mc L_0^{\otimes2}\right)$ and $s_1=1\in\Gamma\left(\PP^1,\mc L_1^{\otimes2}\right)$ and $s_2=-\left(3w^2-x^2\right)\left(x^2-2w^2\right)\in\Gamma\left(\PP^1,\OO_{\PP^1}(4)\right)$. In total, our section $s$ will be given by $s\coloneqq s_0+s_1+s_2$. The point is that if we restrict to the affine open chart given by $z=1$, then $X$ collapses down to $U$.

\end{document}