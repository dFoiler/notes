% !TEX root = ../notes.tex

\documentclass[../notes.tex]{subfiles}

\begin{document}

Today we begin our discussion of duality in earnest. Homework will be posted next week.

\subsection{The Picard Scheme of an Abelian Variety}
Note that an abelian $k$-variety $A$ has a $k$-rational point $0_A\in A(k)$ and is smooth, projective, and so on. Thus, $\Pic^\circ_{A/k}A$ exists. Because $A$ is smooth, this scheme is projective. We would like this to agree with our construction of $\Pic^0A$ from earlier.
\begin{theorem} \label{thm:pic-zero}
	Fix an abelian $k$-variety $A$. Then $\Pic^\circ_{A/k}(k)=\Pic^0(A)$.
\end{theorem}
Namely, our goal is to make sense of the following definition.
\begin{definition}[dual abelian variety]
	For an abelian $k$-variety $A$, we set $A^\lor\coloneqq\Pic^\circ_{A/k}$ to be the \textit{dual abelian variety}.
\end{definition}
We know that $\Pic^\circ_{A/k}$ is a connected (and hence irreducible) group scheme, but we do not yet know if it is smooth; \Cref{thm:pic-zero} will help with this. For example, we do know that $A^\lor_{\mathrm{red}}$ is in fact an abelian variety.

It will help to have the following notion.
\begin{definition}[Poincar\'e line bundle]
	Fix a $k$-scheme $X$ for which $\Pic_{X/k}$ exists. Then there is a universal \textit{Poincar\'e (rigidified) line bundle} $(\mc P,\alpha_{\mc P})$ on $X\times_k\Pic_{X/k}$ where $\alpha\colon\mc P|_{\{x\}\times\Pic_{X/k}}\cong\OO_{\Pic_{X/k}}$. Namely, $(\mc P,\alpha)$ corresponds to ${\id_{\Pic_{X/k}}}\in h_{\Pic_{X/k}}({\Pic_{X/k}})$.
\end{definition}
\begin{remark}
	Unwinding via the Yoneda lemma, any $T$-point $\varphi\colon T\to\Pic_{X/k}$ corresponds to the rigidified line bundle $(\mc L,\alpha)=\varphi^*(\mc P,\alpha_{\mc P})$. For example, if $k'/k$ is a field extension, then a $k'$-point $\lambda\in\Pic_{X/k}$ corresponds to the rigidified line bundle $\mc P|_{X\times\{\lambda\}}$.
\end{remark}
It will be useful to have some notion of equivalence.
\begin{definition}[algebraically equivalent]
	Fix line bundles $\mc M$ and $\mc N$ over a $k$-scheme $X$, where $k$ is algebraically closed. Then $\mc M$ and $\mc N$ are \textit{algebraically equivalent} if and only if there is a connected $k$-variety $T$ and a line bundle $\mc L$ over $X_{k}\times T$ and $t_1,t_2\in T(k)$ such that
	\[\mc M\cong\mc L|_{X\times\{t_1\}}\qquad\text{and}\qquad\mc N\cong\mc L|_{X\times\{t_2\}}.\]
\end{definition}
\begin{remark}
	One may restrict $T$ to just being a curve by finding a curve between $t_1$ and $t_2$.
\end{remark}
\begin{remark}
	Rational equivalence basically amounts to taking $T=\PP^1_k$.
\end{remark}
Algebraic equivalence is in fact a weaker condition.
\begin{lemma} \label{lem:use-alg-equiv}
	Fix a line bundle $\mc L'$ on a $k$-scheme $X$ (with marked point $e\in X(k)$) coming from some $\lambda\in\Pic_{X/k}(k)$ (where we assume $\Pic_{X/k}$ exists). Then $\lambda\in\Pic^\circ_{X/k}(k)$ if and only if $\mc L'_{\ov k}$ and $\OO_{X_{\ov k}}$ are algebraically equivalent.
\end{lemma}
\begin{proof}
	In the forward direction, we take $T\coloneqq\left(\Pic_{X/k}^\circ\right)_{\ov k,\mathrm{red}}$, which is a connected variety. (For connectivity, we see) Then the universal line bundle $\mc P$ restricts to $\mc L'$ on $X\times\{\lambda\}$ (by definition of $\lambda$) and restricts to $\OO_X$ on $X\times\{e\}$ (by definition of $\mc P$).

	In the reverse direction, pick up our $k$-scheme $T$ and the provided line bundle $\mc L$ over $X\times T$ with points $t_1,t_2\in T$, and let $\{U_i\}_{i\in I}$ be a trivializing open cover, and we assume that the $U_i$ are connected. Notably, being a trivializing open cover means that we have equipped ourselves with morphisms $U_i\to\Pic_{X/k}$. Now, we know
	\[\mc L|_{X_{\ov k}\times\{t_1\}}\cong\mc L_{\ov k}'\qquad\text{and}\qquad\mc L|_{X_{\ov k}\times\{t_2\}}\cong\OO_{X_{\ov k}}.\]
	Now, the marked point $t_2$ lives in some $U_i$, and this $U_i$ goes to $\Pic^\circ_{X/k}$ by the above trivialization, so because $T$ is connected, actually is all maps to $\Pic^\circ_{X/k}$. Thus, we can specialize to $t_1$ to get $\mc L$ in $\Pic^\circ_{X/k}$.\todo{What?}
\end{proof}
This allows us to prove part of \Cref{thm:pic-zero}.
\begin{lemma}
	Fix an abelian $k$-variety $A$. Then $\Pic^\circ_{A/k}(k)\subseteq\Pic^0(A)$.
\end{lemma}
\begin{proof}
	Let $\mc P$ be the universal line bundle on $A\times A^\lor_{\mathrm{red}}$, which is legal because we're only ever going to work with $k$-points anyway. Now, pick up some $\mc L\in\Pic^\circ_{A/k}(k)$, and we need to show that
	\[m^*\mc L\cong\op{pr}_1^*\mc L\otimes\op{pr}_2^\circ\mc L.\]
	By pullback, it suffices to show this for $\mc P$, so define
	\[\mc M\coloneqq (m\otimes{\id_{A^\lor}})^*\mc P\otimes(\op{pr}_1\otimes{\id_{A^\lor}})^*\mc P^{-1}\otimes(\op{pr}_2\otimes{\id_{A^\lor}})^*\mc P^{-1},\]
	which we want to show is trivial. Well, the above is a line bundle on $A\times A\times A^\lor$, so we use \Cref{thm:cube}. For this, note $\mc P|_{\{0_A\}\times A^\lor_{\mathrm{red}}}\cong\OO_{A^\lor_{\mathrm{red}}}$ by construction of the universal line bundle, and $\mc P|_{A\times\{0_{A^\lor}\}}\cong\OO_A$ again by construction. Now,
	\[\mc M|_{\{0_A\}\times A\times A^\lor_{\mathrm{red}}}\cong\op{pr}_1^*\left(\mc P|_{\{0_A\}\times A^\lor_{\mathrm{red}}}\right)\]
	because the first and last terms cancel, and the above line bundle trivializes as discussed; the argument is similar for $A\times\{0_A\}\times A^\lor$. Lastly, we see
	\[\mc M|_{A\times A\times\{0_{A^\lor_{\mathrm{red}}}\}}=\left(m^*\otimes({\op{pr}_1^*})^{-1}\otimes({\op{pr}_2^*})^{-1}\right)(\mc P|_{A\times\{0_A\}})\]
	vanishes because now we're just over $A\times A$.
\end{proof}
Before showing the other inclusion, we make some remarks. Well, given some line bundle $\mc L$, we build $\varphi_\mc L\colon A(\ov k)\to\Pic^0(A_{\ov k})$, which we claim actually factors through $\Pic^\circ_{A/k}(\ov k)$. Indeed, for $x\in A(\ov k)$, we want to know that $t_x^*\mc L\otimes\mc L^{-1}$ is algebraically equivalent to $\OO_{A_{\ov k}}$ by \Cref{lem:use-alg-equiv}. But then $A_{\ov k}\times A_{\ov k}$ itself witnesses the algebraic equivalence because you can write down a line bundle which specializes to both $t_x^*\mc L\otimes\mc L^{-1}$ and $t_e^*\mc L\otimes\mc L^{-1}$ (the latter of which is trivial).

So we are going to want to show the following.
\begin{restatable}{proposition}{amplesurjprop} \label{prop:ample-gives-surjective}
	Fix an abelian $k$-variety $A$ and an ample line bundle $\mc L$ on $A$. Then the map $\varphi_\mc L\colon A(\ov k)\to\Pic^0(A_{\ov k})$ is surjective.
\end{restatable}
\noindent It will help to prove the following lemma.
\begin{lemma} \label{lem:line-bundle-cohom-dies}
	Fix an abelian $k$-variety $A$ and a nontrivial line bundle $\mc L\in\Pic^0(A)$. Then $H^i(A,\mc L)=0$ for all $i$.
\end{lemma}
\begin{proof}
	We begin by showing $H^0(A,\mc L)=0$. Well, if this is not the case, then we may find an effective divisor $D\subseteq A$ such that $\mc L\cong\OO_A(D)$ by viewing $\Gamma(A,\mc L)$ as parameterizing linear systems. Now, we compute
	\[\OO_A=0_A^*\mc L=([1]_A\times[-1]_A)^*m^*\mc L=([1]_A\times[-1]_A)(\op{pr}_1^*\mc L\otimes\op{pr}_2^*\mc L)=\mc L\otimes[-1]^*\mc L.\]
	But then we are being told that $D+[-1]^*D$ is rationally equivalent to $0$, which forces $\mc L$ to be trivial.

	For the second part, we use the K\"unneth formula. Let $k$ be the smallest positive integer where $H^k(A,\mc L)$ is nonzero; note $k>0$ by the above paragraph. Now, we note that we have the commutative diagram.
	% https://q.uiver.app/#q=WzAsMyxbMCwwLCJIXmsoQSxcXG1jIEwpIl0sWzEsMCwiSF5rKEFcXHRpbWVzIEEsbV4qXFxtYyBMKSJdLFsxLDEsIkheayhBLFxcbWMgTCkiXSxbMCwxLCJtXioiXSxbMSwyLCIoMF9BXFx0aW1lc1xcaWRfQSleKiJdLFswLDIsIiIsMix7ImxldmVsIjoyLCJzdHlsZSI6eyJoZWFkIjp7Im5hbWUiOiJub25lIn19fV1d&macro_url=https%3A%2F%2Fraw.githubusercontent.com%2FdFoiler%2Fnotes%2Fmaster%2Fnir.tex
	\[\begin{tikzcd}
		{H^k(A,\mc L)} & {H^k(A\times A,m^*\mc L)} \\
		& {H^k(A,\mc L)}
		\arrow["{m^*}", from=1-1, to=1-2]
		\arrow["{(0_A\times\id_A)^*}", from=1-2, to=2-2]
		\arrow[Rightarrow, no head, from=1-1, to=2-2]
	\end{tikzcd}\]
	Thus, $H^k(A\times A,m^*\mc L)$ is nonzero, but the K\"unneth formula tells us that
	\[H^k(A\times A,\op{pr}_1^*\mc L\otimes\op{pr}_2^*\mc L)=\bigoplus_{i+j=k}H^i(A,\mc L)\otimes H^j(A,\mc L).\]
	The left-hand side is nonzero, but then some term on the right-hand side must be nonzero, which is a contradiction because we cannot have $i=0$ or $j=0$.
\end{proof}

\end{document}