% !TEX root = ../notes.tex

\documentclass[../notes.tex]{subfiles}

\begin{document}

\section{February 10}

Today we finish proving that $H^2(\op{Gal}(K^{\mathrm{sep}}/K),K^{\mathrm{sep}\times})\cong\QQ/\ZZ$.

\subsection{Cohomology of Ramified Extensions}
Quickly, we pick up the following cohomological tool.
\begin{proposition}[Restriction--inflation] \label{prop:res-inf}
	Fix a normal subgroup $H$ of a group $G$. Given a $G$-module $M$ such that $H^1(H,M)=H^2(H,M)=0$, we have
	\[H^2\left(G/H,M^H\right)\cong H^2(G,M).\]
\end{proposition}
\begin{proof}
	We make a few remarks.
	\begin{itemize}
		\item The functor $(-)^H\colon\mathrm{Mod}_G\to\mathrm{Mod}_{G/H}$ preserves injectives. Indeed, this functor has an exact left adjoint: namely, we want an exact functor $L\colon\mathrm{Mod}_{G/H}\to\mathrm{Mod}_G$ such that any $M\in\mathrm{Mod}_G$ and $S\in\mathrm{Mod}_{G/H}$ has
		\[\op{Hom}_G(LS,M)\simeq\op{Hom}_{G/H}\left(S,M^H\right).\]
		Well, we simply define $LS$ to be $S$ viewed as a $G$-module via $G\to G/H$. Namely, a $G$-module morphism from $S\to M$ must be fixed by $H$ because $S$ is a trivial $H$-module, so we have really defined a morphism $S\to M^H$ as $G/H$-modules. Also, observe that $L$ is exact because exactness can be checked in $\mathrm{Ab}$, and we have done nothing to the underlying abelian groups.

		We now show that $L$ tells us we preserve injectives. Well, let $I$ be an injective $G$-module, and fix some embedding of $S\subseteq S'$ of $H$-modules. Given some morphism $S\to I^H$, we want to fill in the following arrow.
		% https://q.uiver.app/?q=WzAsNCxbMCwwLCIwIl0sWzEsMCwiUyJdLFsyLDAsIlMnIl0sWzIsMSwiSV9IIl0sWzEsM10sWzIsMywiIiwyLHsic3R5bGUiOnsiYm9keSI6eyJuYW1lIjoiZGFzaGVkIn19fV0sWzAsMV0sWzEsMl1d&macro_url=https%3A%2F%2Fraw.githubusercontent.com%2FdFoiler%2Fnotes%2Fmaster%2Fnir.tex
		\[\begin{tikzcd}
			0 & S & {S'} \\
			&& {I_H}
			\arrow[from=1-2, to=2-3]
			\arrow[dashed, from=1-3, to=2-3]
			\arrow[from=1-1, to=1-2]
			\arrow[from=1-2, to=1-3]
		\end{tikzcd}\]
		Hitting this with our exact adjunction, it is equivalent to fill in the following arrow.
		% https://q.uiver.app/?q=WzAsNCxbMCwwLCIwIl0sWzEsMCwiTFMiXSxbMiwwLCJMUyciXSxbMiwxLCJJIl0sWzEsM10sWzIsMywiIiwyLHsic3R5bGUiOnsiYm9keSI6eyJuYW1lIjoiZGFzaGVkIn19fV0sWzAsMV0sWzEsMl1d&macro_url=https%3A%2F%2Fraw.githubusercontent.com%2FdFoiler%2Fnotes%2Fmaster%2Fnir.tex
		\[\begin{tikzcd}
			0 & LS & {LS'} \\
			&& I
			\arrow[from=1-2, to=2-3]
			\arrow[dashed, from=1-3, to=2-3]
			\arrow[from=1-1, to=1-2]
			\arrow[from=1-2, to=1-3]
		\end{tikzcd}\]
		However, $I$ is injective, so such an arrow exists.

		\item We now compute cohomology. We are granted a left-exact sequence as follows.
		\[0\to M\to I^0\to I^1\to I^2\to\cdots.\]
		These injective $G$-modules are also injective $H$-modules (just write down the diagram), so we can compute group cohomology in $G$ or $H$ by taking cohomology of the above resolution. Namely,
		\begin{align*}
			H^\bullet(H,M) &= H^\bullet\left(I^{0H}\to I^{1H}\to\cdots\right), 
			\\
			H^\bullet(G,M) &= H^\bullet\left(I^{0G}\to I^{1G}\to\cdots\right).
		\end{align*}
		Now, because $H^1(H,M)=H^2(H,M)$, we know that
		\[0\to M^H\to I^{0H}\to I^{1H}\to I^{2H}\to I^{3H}\]
		is exact, and the previous point tells us that this is the beginning of an injective resolution in $\mathrm{Mod}_{G/H}$. Now computing $G/H$-invariants, we see that
		\[H^2\left(G/H,M^H\right)=\frac{\ker\left((I^{2H})^{G/H}\to (I^{3H})^{G/H}\right)}{\im\left((I^{1H})^{G/H}\to (I^{2H})^{G/H}\right)}=H^2(G,M),\]
		which is what we wanted.
		\qedhere
	\end{itemize}
\end{proof}
\begin{remark}
	In the background, this result really comes from a spectral sequence.
\end{remark}
\begin{corollary}
	Fix a local field $K$, and let $I\subseteq\op{Gal}(K^{\mathrm{sep}}/K)$ be the kernel of the restriction map $\op{Gal}(K^{\mathrm{sep}}/K)\to\op{Gal}(K^{\mathrm{unr}}/K)$. If $H^2(I,K^{\mathrm{sep}\times})=0$, then $H^2(\op{Gal}(K^{\mathrm{sep}}/K),K^{\mathrm{sep}\times})\cong\QQ/\ZZ$.
\end{corollary}
\begin{proof}
	Combine \Cref{prop:res-inf} with \Cref{cor:h2-unr}.
\end{proof}
We now turn to totally ramified extensions $L/K$. Speaking philosophically, $H^2(\op{Gal}(L/K),L^\times)$ is a class field theory question, a question about Brauer groups (one can simply translate everything into central simple algebras), or a geometry question via our Brauer--Severi varieties. Let's do geometry.
\begin{lemma}
	Fix a totally ramified extension of local fields $L/K$. Given an $\mathcal O_K$-scheme $P_{\mathcal O_K}$ such that $P_{\mathcal O_L}\cong\PP^n_{\mathcal O_L}$, we also have $P_{\mathcal O_K}\cong\PP^n_{\mathcal O_K}$.
\end{lemma}
\begin{proof}
	We provide a sketch.
	\begin{enumerate}
		\item To begin, one can show there is a closed embedding $P\into\PP^M_K$ for some $M>0$. Roughly speaking, one can pick up a line bundle $\mc L_{P_L}$ inducing the isomorphism $P_L\cong\PP^n_L$, but the cocycle condition allows us to know it takes values in the roots of unity, so taking a large enough power means we induce an embedding to projective $K$-space. (To work with infinite extensions, we note that specifying such a morphism only needs a finite amount of polynomial data, so it's okay to pass to the colimit.) By abuse of notation, we say $\mc L_{P_K}$ is the line bundle yielding our embedding.
		\item We claim that it is enough to show $P_K(K)$ is nonempty. Indeed, we want to show that there is a line bundle $\mc M$ on $P_K$ such that $\mc M^{\otimes\deg\mc L_{P_K}}\cong\mc L_{P_K}$, which is enough because $\mc M$ will induce an isomorphism $P_K\to\PP^M_k$, which is good enough.

		Well, we would like to chose $\mc M_L$ coming from $\PP^n_L$ such that $\varepsilon\colon\mc M^{\otimes\deg\mc L_{P_K}}\cong\mc L_{P_L}$ and want it to be compatible with the Galois action, but this need not be the case. Namely, we would like for this morphism to be unique in some sense and therefore compatible with the Galois action. To get rid of the extraneous automorphisms, we fix $x\in P_K(K)$ and consider pairs of line bundles $(\mc U,\rho)$ where $\rho\colon k_{\mc U}(x)\cong K$.

		Notably, isomorphisms between such pairs are unique when they exist, but this category of pairs up to isomorphism is still just $\op{Pic}P_K$, even with the tensor product. Reformulating our problem, we are trying to find a line bundle $\mc M^{\otimes\deg\mc L_{P_K}}\cong\mc L_{P_L}$ with the $\rho$, and this data will now be automatically compatible with the Galois action.
		\item Note that having some $P_{\OO_K}\subseteq\PP^N_{\OO_K}$ with $P_{\OO_L}\cong\PP^n_{\OO_L}$ also grants us a $B$-point $\Spec\OO_L\to P_{\OO_L}$ by the valuative criterion. This story of our $B$-point with residue field $k$ will give us a $k$-point coming from $A$ as well because $A$ and $B$ have the same residue field.

		As such, unwrapping the algebraic geometry, we have a morphism $A\to k$ and a morphism $\widehat\OO_{P_A,x}\to k$, and we would like to lift this to $\widehat\OO_{P_A,x}\to A$, which will give us the desired $A$-point to finish. Well, map $\widehat\OO_{P_A,x}$ to $\widehat\OO_{P_k,x}$ and then left elements of $\mf m/\mf m^2$ to lift back to $\widehat\OO_{P_A,x}$. This will define a map $A\bb{x_\bullet}$ to $\widehat\OO_{P_A,x}$, which will finish the proof.
		\qedhere
	\end{enumerate}
\end{proof}

\end{document}