% !TEX root = ../notes.tex

\documentclass[../notes.tex]{subfiles}

\begin{document}

\section{February 8}

We quickly remark on some sources. Our discussion of cohomology roughly follows \cite{serre-local-fields} and \cite{milne-cft}. For a discussion of the Brauer group, we are roughly following Poonen.

\subsection{Cohomology of Unramified Extensions}
Today we will be discussing the following result; throughout, $L/K$ is a Galois extension of local fields with Galois group $G$.
\begin{theorem} \label{thm:lcft}
	Fix a local field $K$.
	\begin{listalph}
		\item For any finite Galois extension $L/K$ with $G\coloneqq\op{Gal}(L/K)$, we have $H^2(G,L^\times)\cong\frac1{\#G}\ZZ/\ZZ$.
		\item Taking the direct limit, we have $H^2(G_K,(K^{\mathrm{sep}})^\times)\cong\QQ/\ZZ$.
	\end{listalph}
\end{theorem}
Let's do the archimedean case first.
\begin{lemma}
	We compute $H^2(\op{Gal}(\CC/\RR),\CC^\times)\cong\ZZ/2\ZZ$.
\end{lemma}
\begin{proof}
	Write $G\coloneqq\op{Gal}(\CC/\RR)=\{1,\sigma\}$, where $\sigma$ is complex conjugation. Now, recall from \Cref{prop:cyclic-cohomology} that we may compute
	\[H^2(G,\CC^\times)\cong\widehat H^0(G,\CC^\times)=\frac{\left(\CC^\times\right)^G}{\op N_G\CC^\times}=\frac{\RR^\times}{\left\{|z|^2:z\in\CC^\times\right\}}=\frac{\RR^\times}{\RR^+},\]
	and this last group is indeed $\ZZ/2\ZZ$.
\end{proof}
\begin{remark}
	Using \Cref{prop:cyclic-cohomology} and \Cref{thm:h90}, we see that
	\[\widehat H^i(\op{Gal}(\CC/\RR),\CC^\times)\cong\begin{cases}
		0 & \text{if }i\text{ is odd}, \\
		\ZZ/2\ZZ & \text{if }i\text{ is even}.
	\end{cases}\]
\end{remark}
We now move towards \Cref{thm:lcft}. Here is our first case.
\begin{remark}
	For any extension $L/K$ of local fields with residue field extension $\lambda/\kappa$, the subextension fixed by Frobenius is $L^{\mathrm{unr}}/K$ whose residue field extension remains $\lambda/\kappa$. But now $L^{\mathrm{unr}}/K$ is an unramified extension, and $L/L^{\mathrm{unr}}$ is totally ramified.
\end{remark}
With the above remark in mind, our approach will be the following.
\begin{enumerate}
	\item We will begin with $L/K$ unramified and show $H^2(G,L^\times)=\frac1{\#G}\ZZ/\ZZ$. The intuition here is that our cohomological contribution will come from these unramified extensions.
	\item Next, we will show that $L/K$ being totally ramified yields $H^2(G,L^\times)=0$.
	\item Lastly, we will combine the above two cases accordingly.
\end{enumerate}
Let's go at it. Let's set some notation. Quickly, recall the structure of $K^\times$: let $\pi_K\in\mf p_K$ be a uniformizer for $K$ so that $K^\times\cong\pi_K^\ZZ\times\OO_K^\times$. However, we can express $\OO_K^\times$ by
\[\OO_K^\times=\limit\left(\OO_K/\mf p_K^n\right)^\times.\]
Now, we can think about $\OO_K^\times$ as decomposed as
\[1\to\frac{1+\mf p_K^{n-1}}{1+\mf p_K^n}\to\left(A/\mf p_K^n\right)^\times\to\left(A/\mf p_K^{n-1}\right)^\times\to1.\]
But now we see that the group on the left here is isomorphic to $(\OO_K/\mf p_K,+)$ by $a\mapsto 1+a\mf p_K^{n-1}$; one should check this works. There is a similar description for $L$.
\begin{lemma} \label{lem:unramified-units-cohom}
	Let $L/K$ be a finite unramified Galois extension of local fields with Galois group $G$. Then $\widehat H^i(G,\OO_L^\times)=0$.
\end{lemma}
\begin{proof}
	We compute with Tate cohomology. Because $G$ is generated by the Frobenius, it is cyclic, so there are two computations.
	\begin{enumerate}
		\item We show $H^1(G,\OO_L^\times)=0$. This is easier: indeed, \Cref{thm:h90} tells us that $H^1(G,L^\times)=0$, and $\OO_L^\times$ is a direct summand of $L^\times$, so we are done.

		\item We show $\widehat H^0(G,\OO_L^\times)=0$. By definition of Tate cohomology, it's enough to show that the norm map
		\[\op N^L_K\colon\OO_L^\times\to\OO_K^\times\]
		is surjective. Because $\op N^L_K$ is continuous, so it suffices to show that it has dense image, so we show
		\[\op N^L_K\colon\left(\frac{\OO_L}{\mf p_L^n}\right)^\times\to\left(\frac{\OO_K}{\mf p_K^n}\right)^\times\]
		is surjective for all $n$. (This is well-defined because $\mf p_L=\mf p_K\OO_L$ because $L/K$ is unramified!) We show this by induction. Well, for $n=1$, we are showing that the norm map in an extension of finite fields is surjective. We can do this by hand: for an extension of finite fields $\FF_{q^r}/\FF_q$, let $g\in\FF_{q'}^\times$ generate so that
		\[N(g)=\prod_{i=0}^{r-1}g^{p^i}=g^{\left(q^r-1\right)/(q-1)}\]
		has order $q-1$ and is in $\FF_q^\times$ and is thus a generator.

		Then for the inductive step, we draw the following morphism of short exact sequences.
		% https://q.uiver.app/?q=WzAsMTAsWzAsMCwiMSJdLFsxLDAsIlxcZGlzcGxheXN0eWxlXFxmcmFjezErXFxtZiBwX0xebn17MStcXG1mIHBfTF57bisxfX0iXSxbMiwwLCJcXGRpc3BsYXlzdHlsZVxcbGVmdChcXGZyYWN7XFxPT19MfXtcXG1mIHBfTF57bisxfX1cXHJpZ2h0KV5cXHRpbWVzIl0sWzMsMCwiXFxkaXNwbGF5c3R5bGVcXGxlZnQoXFxmcmFje1xcT09fTH17XFxtZiBwX0xee259fVxccmlnaHQpXlxcdGltZXMiXSxbNCwwLCIxIl0sWzEsMSwiXFxkaXNwbGF5c3R5bGVcXGZyYWN7MStcXG1mIHBfS15ufXsxK1xcbWYgcF9LXntuKzF9fSJdLFsyLDEsIlxcZGlzcGxheXN0eWxlXFxsZWZ0KFxcZnJhY3tcXE9PX0t9e1xcbWYgcF9LXntuKzF9fVxccmlnaHQpXlxcdGltZXMiXSxbMywxLCJcXGRpc3BsYXlzdHlsZVxcbGVmdChcXGZyYWN7XFxPT19LfXtcXG1mIHBfS157bn19XFxyaWdodCleXFx0aW1lcyJdLFswLDEsIjEiXSxbNCwxLCIxIl0sWzAsMV0sWzEsMl0sWzIsM10sWzMsNF0sWzgsNV0sWzUsNl0sWzYsN10sWzcsOV0sWzEsNV0sWzIsNl0sWzMsN11d&macro_url=https%3A%2F%2Fraw.githubusercontent.com%2FdFoiler%2Fnotes%2Fmaster%2Fnir.tex
		\[\begin{tikzcd}
			1 & {\displaystyle\frac{1+\mf p_L^n}{1+\mf p_L^{n+1}}} & {\displaystyle\left(\frac{\OO_L}{\mf p_L^{n+1}}\right)^\times} & {\displaystyle\left(\frac{\OO_L}{\mf p_L^{n}}\right)^\times} & 1 \\
			1 & {\displaystyle\frac{1+\mf p_K^n}{1+\mf p_K^{n+1}}} & {\displaystyle\left(\frac{\OO_K}{\mf p_K^{n+1}}\right)^\times} & {\displaystyle\left(\frac{\OO_K}{\mf p_K^{n}}\right)^\times} & 1
			\arrow[from=1-1, to=1-2]
			\arrow[from=1-2, to=1-3]
			\arrow[from=1-3, to=1-4]
			\arrow[from=1-4, to=1-5]
			\arrow[from=2-1, to=2-2]
			\arrow[from=2-2, to=2-3]
			\arrow[from=2-3, to=2-4]
			\arrow[from=2-4, to=2-5]
			\arrow[from=1-2, to=2-2]
			\arrow[from=1-3, to=2-3]
			\arrow[from=1-4, to=2-4]
		\end{tikzcd}\]
		Here, the vertical maps are all $\op N^L_K$. By induction, the right map is surjective, so by the Snake lemma, it suffices to show that the left map is surjective. Well, computing this map, we use the fact that $\pi_K$ is a uniformizer for $L$ to write
		\[\op N^L_K\left(1+a\pi_K^n\right)=\prod_{\sigma\in G}\left(1+\sigma(a)\pi_K^n\right)\equiv1+\op T^L_K(a)\pi_K^n\pmod{1+\mf p_L^n}.\]
		Thus, it suffices to show that the trace map is surjective in an extension of finite fields $\FF_{q^r}/\FF_q$. Equivalently, we want to show that $\widehat H^0\left(\op{Gal}(\FF_{q^r}/\FF_q),\FF_{q^r}\right)$ vanishes, which is true because $\FF_{q^r}$ is an induced module.
		\qedhere
	\end{enumerate}
\end{proof}
\begin{lemma}
	Let $L/K$ be a finite unramified Galois extension of local fields with Galois group $G$. Then $H^2(G,L^\times)\cong\frac1{\#G}\ZZ/\ZZ$.
\end{lemma}
\begin{proof}
	Because $L^\times\cong\pi_K^\ZZ\times\OO_L^\times$, we use \Cref{lem:unramified-units-cohom} to yield
	\[H^2(G,L^\times)\cong H^2(G,\pi_K^\ZZ)\times H^2(G,\OO_L^\times)\cong H^2(G,\ZZ),\]
	where we are using the fact that $\pi_K$ is fixed by $G$. Thus, we want to compute
	\[H^2(G,\ZZ)\cong\widehat H^0(G,\ZZ)=\frac{\ZZ}{N_G\ZZ}=\frac\ZZ{\#G\ZZ},\]
	which is what we wanted.
\end{proof}
\begin{corollary} \label{cor:h2-unr}
	Fix a local field $K$. Then $H^2(\op{Gal}(K^{\mathrm{unr}}/K),(K^{\mathrm{unr}})^\times)=\QQ/\ZZ$.
\end{corollary}
\begin{proof}
	Take direct limits of the above lemma. It is not too hard to check that everything works out here in our transition maps.
\end{proof}

\end{document}