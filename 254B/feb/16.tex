% !TEX root = ../notes.tex

\documentclass[../notes.tex]{subfiles}

\begin{document}

Today we continue discussing degree.

\subsection{More on Degree}
Let's just get going.
\begin{lemma} \label{lem:deg-by-rank}
	Fix a proper integral $k$-scheme $X$ with generic point $\eta$. For a line bundle $\mc L$ on $X$ and a coherent sheaf $\mc F$ on $X$, we have
	\[d_\ell(\mc F)=(\op{rank}\mc F_\eta)(\deg\mc L).\]
\end{lemma}
\begin{proof}
	This is a standard ``d\`evissage'' argument. Because $\chi$ is additive in short exact sequences, it is enough to check that there is a coherent sheaf of ideals $\mc I\subseteq\OO_X$ fitting in the exact sequence
	\[0\to\mc I^{\op{rank}\mc F_\eta}\to\mc F\to\mc Q\to0,\]
	where $\mc Q$ is a torsion sheaf where $\op{supp}\mc Q$ is a closed subscheme of $X$ of positive codimension, and $\op{supp}\OO_X/\mc I$ is also a closed subscheme of $X$ of positive codimension. Indeed, this will imply that
	\[d_\ell(\mc F)=\op{rank}\mc F_\eta\cdot d_\ell(\mc I)=\op{rank}\mc F_\eta\cdot\deg\mc L\]
	by staring at our short exact sequences.

	So it remains to find $\mc I$. Well, $\mc F$ is coherent with rank $r$, so a spreading out argument promises that we can find an open subscheme $U\subseteq X$ such that $\mc F|_U=\OO_U^{\oplus r}$. Then we can view $X\setminus U$ as a divisor and take the line bundle associated to it given by $\mc I$. Because $U$ is dense, the quotient $\mc J$ will end up being torsion, which is enough.
\end{proof}
\Cref{prop:deg-of-pullback} will follow from this.
\degpullback*
\begin{proof}
	Exactness of $f$ allows us to see
	\[H^i\left(X,f^*\mc L^{\otimes n}\right)=H^i\left(Y,f_*f^*\mc L^{\otimes n}\right).\]
	Now, the adjunction formula tells us that this is $H^i(Y,f_*\OO_X\otimes\mc L^{\otimes n})$, so unwinding our characteristic polynomial reveals that
	\[\deg f^*\mc L=d_\mc L(f_*\OO_X)=(\deg f)(\deg\mc L),\]
	where the last equality has used \Cref{lem:deg-by-rank}.
\end{proof}
\begin{remark}
	One can weaken $f$ from being finite to dominant by passing to an open subscheme where we are finite.
\end{remark}
This allows us to understand $[n]_A$.
\begin{theorem} \label{thm:mul-n-av}
	Fix an abelian $k$-variety $A$. For any nonzero integer $n$, the map $[n]_A\colon A\to A$ is an isogeny of degree $n^{2\dim A}$.
	\begin{listalph}
		\item $[n]_A$ is separable if and only if $\op{char}k\nmid n$.
		\item If $p\coloneqq\op{char}k$, then the inseparable degree of $[p]_A$ is at least $p^{\dim A}$.
	\end{listalph}
\end{theorem}
\begin{proof}
	The degree computation is immediate from \Cref{prop:deg-of-pullback} and the computation $[n]_A^*\mc L=\mc L^{\otimes n^2}$ for an ample symmetric line bundle $\mc L$.

	For (a), we note that $[n]_A$ is separable if and only if it is \'etale (indeed, $[n]_A$ is already flat by miracle flatness), so it is enough to check smoothness. But being a group scheme means that we may as well check smoothness only at $0_A\in A$. Well, an induction on $n$ shows that $d[n]_A|_{0_A}\colon\op{Lie}A\to\op{Lie}A$ is multiplication-by-$n$,\footnote{The main thing to check is that $dm(t_1,t_2)=t_1+t_2$. This is computed in \cite[p.~40]{mumford}.} and this map is invertible if and only if $\op{char}k\nmid n$.

	Now, for (b), we note that $d[p]|_{0_A}\colon\op{Lie}A\to\op{Lie}A$ is the zero map. However, $[p]\colon A\to A$ produces a map by pullback in the opposite direction given by $[p]^*\Omega^1_A\to\Omega^1_A$. This map on the stalk at $0$ is dual to the map on $\op{Lie}A$, which is the zero map, so homogeneity now requires that $[p]^*\Omega^1_A\to\Omega^1_A$ is fully the zero map. In other words, for any $f\in K(A)$, we have $[p]^*df=0$ in $\Omega^1_{K(A)/k}$, which upon unwinding definitions (in the differentials) implies
	\[[p]^*f\in K(A)^p.\]
	The moral of the story is that $[p]^*K(A)\to K(A)$ factors through $k\cdot K(A)^p$. But $K(A)^p$ has transcendence degree $\dim A$ over $k$, so this extension has inseparable degree at least $p^{\dim A}$.
\end{proof}
\begin{corollary} \label{cor:tors-av}
	Fix an abelian $k$-variety $A$.
	\begin{listalph}
		\item If $\op{char}k\nmid n$, then $A[n](\ov k)\cong(\ZZ/n\ZZ)^{2\dim A}$.
		\item If $n=p^\nu$ where $p\coloneqq\op{char}k>0$, then $A[p](\ov k)\cong\left(\ZZ/p^\nu\ZZ\right)^i$ for some $i\le\dim A$.
	\end{listalph}
\end{corollary}
\begin{proof}
	For (a), the point is that $[n]_A$ being separable implies that $A(\ov k)[n]=\deg[n]_A$, so we know that $A(\ov k)[n]$ at least has the correct size by \Cref{thm:mul-n-av}. Now, for $n=\ell^\nu$ a prime power, one can induct on the power and use the fact that it has a quotient of the form $\ZZ/\ell^{\nu-1}\ZZ$ given by multiplication-by-$\ell$, so the sharper result holds. If $n$ is not a prime power, then we decompose into prime powers to conclude.
	
	The argument for (b) is similar. Note $\deg[p]=p^{2\dim A}$ still, but we have at least $\dim A$ stuck in inseparable degree, so
	\[\#A(\ov k)[p]=\deg_{\mathrm{sep}}[p]=\frac{p^{2\dim A}}{\deg_{\mathrm{insep}}[p]}=p^i\]
	for some $0\le i\le g$. But the group is $p$-torsion, so we get $(\ZZ/p\ZZ)^i$, and the same induction on $n$ achieves the result for $p^\nu$ in general. More explicitly, we write out the exact sequence
	\[0\to A(\ov k)[p]\to A(\ov k)[p^\nu]\stackrel p\to A(\ov k)\left[p^{\nu-1}\right]\to0,\]
	which forces the middle by induction.
\end{proof}
\begin{remark}
	The $i$ in the above result is usually called the ``$p$-rank'' of $A$. It is an isomorphism invariant, so for example it can produce a stratification of the moduli space. As an example of this being interesting, it is known that having maximal $p$-rank implies that $A$ is ``ordinary,'' which relates to the Frobenius action.
\end{remark}
This permits the following definition.
\begin{definition}[Tate module]
	Fix an abelian $k$-variety $A$ and a prime $\ell$ coprime to $\op{char}k$. Then
	\[T_\ell A\coloneqq\limit A[\ell^\bullet].\]
\end{definition}
The point is that $T_\ell A=\ZZ_\ell^{2\dim A}$ by taking limits over \Cref{cor:tors-av}.
\begin{remark}
	To define a Tate module for $\ell=\op{char}k$, one needs to define a $p$-divisible group.
\end{remark}

\subsection{The Picard Scheme}
We will need a little moduli theory but not too much. In particular, we need the Picard functor.
\begin{definition}[Picard]
	Fix a $k$-scheme $T$. Then the \textit{Picard functor} takes $k$-schemes $T$ to $\op{Pic}_{X/k}(T)$ of isomorphism classes of line bundles on $X\times_kT$. Given a $k$-rational point $x\in X(k)$, this is in bijection with ``rigidified'' line bundles $(\mc L,\alpha)$ on $X\times_kT$, where $\alpha\colon\mc L|_{\{x\}\times T}\cong\OO_T$ is a choice of trivialization.
\end{definition}
Here is the theorem.
\begin{theorem}[Grothendieck]
	The functor $\op{Pic}_{X/k}$ is representable by a separated $k$-scheme locally of finite type. In fact, $\op{Pic}^\circ_{X/k}$ is quasi-projective and is projective if $X$ is smooth.
\end{theorem}
We will not need to know any part of this proof, but we do need to use that this scheme exists.

\end{document}