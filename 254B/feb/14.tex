% !TEX root = ../notes.tex

\documentclass[../notes.tex]{subfiles}

\begin{document}

We began class by completing the proof of \Cref{thm:get-ample}, which I have edited into directly.

\subsection{Finishing Up Ample Line Bundles}
From last class, we needed the following lemma.
\begin{lemma} \label{lem:translate-curve}
	Fix an irreducible projective curve $C$ sitting inside an abelian $k$-variety $A$. Given an effective divisor $E$ with $E\cap C=\emp$, we will have $(x-y)+E=E$ for any $x,y\in C$.
\end{lemma}
\begin{proof}
	Fix $\mc L\coloneqq\OO_A(E)$; the hypothesis is that $\mc L|_C=\OO_C$. Note there is a (restricted) multiplication map $m\colon C\times A\to A$, so we may look at the line bundle $m^*\mc L$ on $C\times A$. For example, for $a\in A$, we may compute
	\[\chi\left(m^*\mc L|_{C\times\{x\}}\right)=\chi\left(t_x^*\mc L|_C\right).\]
	On the other hand, the Euler characteristic needs to be constant in our family, so we can compute this at $x=0_A$ as $\chi(t_0^*\mc L|_C)=\chi(\OO_C)$. From here, Riemann--Roch implies $\deg t_x^*\mc L|_C=\deg\OO_C=0$. But $E$ being an effective divisor requires that $t_x^*\mc L|_C$ to fully trivialize, so either $(x+E)$ cannot intersect $C$ at all. Thus, for any $x,y\in C$ and $z\in E$, one has $z\in(z-y+C)\cap E$, so actually $z-y+C\subseteq E$, so $z-y+x\in E$, so $z\in(y-x)+E$. Looping over all $z\in E$ completes the proof.
\end{proof}
Here is a nice application.
\begin{corollary} \label{cor:n-isog}
	Fix an abelian $k$-variety $A$. For any nonzero integer $n$, the map $[n]_A\colon A\to A$ is an isogeny.
\end{corollary}
\begin{proof}
	Because the dimension of the target and source are the same, it is enough to check that $[n]_A$ is surjective or finite kernel; see \cite[Proposition~7.1]{milne-av}. The point is that the dimension of the fiber needs plus the dimension of the image needs to be the dimension of the target.

	As such, we will show that $[n]_A$ has finite kernel. Well, fix an ample line bundle $\mc L$ on $A$, which exists by \Cref{cor:ab-var-proj}. In fact, we may replace $\mc L$ by $\mc L\otimes[-1]^*\mc L$, which is still ample because pulling back by an automorphism $[-1]$ preserves being ample. So $[-1]^*\mc L=\mc L$, and then we can compute $[n]^*\mc L=\mc L^{\otimes n^2}$.

	Let $A[n]$ be the kernel of $[n]\colon A\to A$. We want to show that $A[n]$ is finite, and because $A$ is quasicompact, it will be enough to show that $A[n]$ is zero-dimensional. Now, $A[n]^\circ$ is an abelian variety, so we want to show that $A[n]^\circ=\{0_A\}$. But
	\[[n]^*\mc L|_{A[n]^\circ}\cong\OO_{A[n]^\circ},\]
	so the trivial line bundle on $\OO_{A[n]^\circ}$ is ample, forcing $A[n]^\circ$ to be zero-dimensional.
\end{proof}

\subsection{Degree}
We will want to understand the degree of isogenies. Let's go ahead and give the general definition of degree.
\begin{definition}[degree]
	Fix a dominant morphism of $f\colon X\to Y$ of integral $k$-schemes such that $\dim X=\dim Y$. Then $\deg f\coloneqq[K(X):K(Y)]$ is the \textit{degree} of $f$; we define the \textit{separable degree} and \textit{inseparable degree} accordingly. We say that $f$ is \textit{separable} if and only if $f\colon K(Y)\to K(X)$ is separable.
\end{definition}
Here is another way to think about degree.
\begin{definition}[degree]
	Fix a proper $k$-variety $X$ and a line bundle $\mc L$ on $X$. For a coherent sheaf $\mc F$ on $X$, we define $P_L\colon\ZZ\to\ZZ$ by
	\[P_\mc L(\mc F,n)\coloneqq\chi\left(\mc F\otimes\mc L^{\otimes n}\right).\]
	It turns out that $P_L$ is a polynomial of degree bounded above by $\dim X$, by \cite[Theorem~19.6.1]{rising-sea}. Then the \textit{degree} of $\mc F$ with respect to $\mc L$ is the number $d_\mc L(\mc F)$ making the leading term of $P_L(\mc F,n)$, in the sense that
	\[\lim_{n\to\infty}\frac{P_\mc L(\mc F,n)}{d_\mc L(\mc F)n^{\dim X}/(\dim X)!}.\]
	Then we define the \textit{degree} as $\deg\mc L\coloneqq d_\mc L(\OO_X)$.
\end{definition}
Let's see how these align.
\begin{proposition} \label{prop:deg-of-pullback}
	Fix a dominant morphism of $f\colon X\to Y$ of proper integral $k$-schemes such that $\dim X=\dim Y$. Then
	\[(\deg f)(\deg\mc L)=\deg f^*\mc L.\]
\end{proposition}
\begin{example}
	Fix an abelian $k$-variety $A$. Then we claim $\deg[n]_A=n^{2\dim A}$. As in \Cref{cor:n-isog}, choose an ample line bundle $\mc L$ with $\mc L=[-1]^*\mc L$. Being ample implies $\deg\mc L>0$: by taking powers, we may assume that $\mc L$ is very ample, and then one can do an explicit computation. (Alternatively, do intersection theory to realize the degree as an intersection number, which is positive.) But we showed
	\[[n]^*\mc L=\mc L^{\otimes n^2},\]
	so the result follows from \Cref{prop:deg-of-pullback}.
\end{example}

\end{document}