% !TEX root = ../notes.tex

\documentclass[../notes.tex]{subfiles}

\begin{document}

The homework is due today.

\subsection{Ample Line Bundles on Abelian Varieties}
Today we will show that abelian varieties are projective. The point is to exhibit an ample line bundle, so we want to understand ample line bundles.

As a corollary to \Cref{thm:cube}, we have the following result.
\begin{theorem}[of the Square] \label{thm:square}
	Fix a line bundle $\mc L$ on an abelian $k$-variety $A$. For any $a\in A(\ov k)$, let $t_a\colon A\to A$ denote the translation. Then for any $x,y\in A(\ov k)$, we have
	\[t_{x+y}^*\mc L\cong t_x^*\mc L\otimes t_y^*\mc L.\]
	Thus, the map $\varphi_\mc L\colon A(\ov k)\to\Pic(A_{\ov k})$ given by $x\mapsto t_x^*\mc L\otimes\mc L^{-1}$ is a group homomorphism.
\end{theorem}
\begin{proof}
	For the first claim, take $f\colon\{x\}\to A$ and $g\colon\{y\}\to A$ and $h\colon\{0_A\}\to A$ and apply \Cref{cor:cube-av}. For the second claim, we simply expand $\varphi_\mc L(x+y)=\varphi_\mc L(x)\otimes\varphi_\mc L(y)$ directly.
\end{proof}
\begin{remark}
	In fact, $\varphi_\bullet\colon\Pic A\to\op{Hom}(A(\ov k),\Pic(A_{\ov k}))$ is a group homomorphism, which we can see by expanding out the definitions directly.
\end{remark}
So we may make the following definition.
\begin{definition}
	Fix an abelian $k$-variety $A$. Then we define the subgroup $\Pic^0(A)\subseteq\Pic(A)$ as $\ker\varphi_\bullet$. In other words, $\varphi\in\Pic^0(A)$ if and only if $\varphi_\mc L$ is trivial.
\end{definition}
\begin{example}
	For an elliptic curve $A$, one can identify $A$ with $\Pic^0(A)$, so we get an exact sequence
	\[0\to\Pic^0(A)\to\Pic(A)\to\ZZ\to0,\]
	where the last map is the degree map.%\todo{}
\end{example}
Let's describe $\Pic^0$ in a better way.
\begin{lemma} \label{lem:better-pic0}
	Fix an abelian $k$-variety $A$. Then $\mc L\in\Pic^0(A)$ if and only if $m^*\mc L\cong\op{pr}_1^*\mc L\otimes\op{pr}_2^*\mc L$.
\end{lemma}
\begin{proof}
	We have two implications to show.
	\begin{itemize}
		\item Suppose $m^*\mc L\cong\op{pr}_1^*\mc L\otimes\op{pr}_2^*\mc L$. We must show that $\varphi_\mc L$ is trivial. Well, fix some point $x\in A(\ov k)$, and let $i_x\colon\{x\}\to A$ be the closed embedding. By definition, one sees
		\[t_x^*\mc L=i^*_xm^*\mc L=i^*\left(\op{pr}_1^*\mc L\otimes\op{pr}_2^*\mc L\right)=\OO_A\otimes\mc L\]
		by some projections. %\todo{}
		So $\varphi_\mc L(x)=\OO_A$ is trivial.
		\item Suppose $\mc L\in\Pic^0(A)$. Define $\mc M\coloneqq m^*\mc L\otimes\op{pr}_1^*\mc L^{-1}\otimes\op{pr}_2^*\mc L^{-1}$, which we want to show is trivial. Now, for each $x\in A(\ov k)$, we see that
		\[M|_{A\times\{x\}}\cong\OO_A,\]
		so \Cref{prop:seesaw} means that $\mc M$ must pull back to a trivial line bundle on both factors, so $\mc M$ is actually trivial.
		\qedhere
	\end{itemize}
\end{proof}
We now pick up some notation.
\begin{definition}
	Fix an abelian $k$-variety $A$. For a line bundle $\mc L$ on $A$, we define $K(\mc L)\coloneqq\ker\varphi_\mc L$, which by definition is
	\[\{x\in A(\ov k):t_x^*\mc L\cong\mc L\}.\]
\end{definition}
\begin{remark}
	Once we upgrade $\Pic$ to a scheme, we can view $K(\mc L)$ as the closed subscheme $\ker\varphi_\mc L$. For today, it will be enough to realize that $K(\mc L)$ is Zariski closed and then view it as a reduced closed subscheme of $A$.
\end{remark}
\begin{remark}
	One has $K(\mc L)=A$ if and only if $\mc L\in\Pic^0(A)$.
\end{remark}
Let's check that $K(L)$ is Zariski closed.
\begin{lemma}
	Fix an abelian $k$-variety $A$. For a line bundle $\mc L$ on $A$, the subset $K(\mc L)\subseteq A$ is Zariski closed.
\end{lemma}
\begin{proof}
	By definition, we see
	\[K(\mc L)=\left\{x\in A(\ov k):m^*\mc L\otimes\op{pr}_2^*\mc L^{-1}|_{A\times\{x\}}\cong\OO_A\right\}.\]
	However, a computation with \Cref{prop:seesaw} shows that $K(\mc L)$ is closed.
\end{proof}
\begin{lemma}
	Fix an abelian $k$-variety $A$. For a line bundle $\mc L$ on $A$, we have $K\left(\mc L^{-1}\right)=K(\mc L)$.
\end{lemma}
\begin{proof}
	Direct from the definition.
\end{proof}
Notably, the above lemma tells us that we cannot tell if a line bundle is ample just from looking at $K(\mc L)$: if $\mc L$ is ample, then $\mc L^{-1}$ is almost never ample. So we will need some notion of effectivity in the following result.
\begin{theorem} \label{thm:get-ample}
	Fix an abelian $k$-variety $A$. For an effective divisor $D$ on $A$, set $\mc L\coloneqq\OO(D)$. Then the following are equivalent.
	\begin{listalph}
		\item $\mc L$ is ample.
		\item $K(\mc L)$ is finite.
		\item $H(D)\coloneqq\{\text{closed }x\in A:x+D=D\}$ is finite.
		\item The linear system $\left|2D\right|\coloneqq\Gamma(X,\OO_A(2D))/k^\times$ (or equivalently, the collection of effective divisors linearly equivalent to $2D$) is base-point free, and the map $A\to\PP^{\left|2D\right|}_k$ is finite.
	\end{listalph}
\end{theorem}
Note $x+D=D$ is literal equality, not linear equivalence of divisors. Also, the addition by $x$ is a translation.
\begin{proof}
	The equivalence of (a) and (d) is algebraic geometry not arising from abelian varieties.
	\begin{itemize}
		\item We show (a) implies (b). Certainly $K(\mc L)$ is a closed $k$-subgroup of $A$. In particular, $B\coloneqq K(\mc L)^\circ$ will be connected (hence geometrically integral), reduced (hence smooth), and proper, so $B$ is an abelian variety. But by definition of $B$, we know $t_x^*\mc L|_B\cong\mc L|_B$, so $\mc L|_B\in\Pic^0(B)$, so \Cref{lem:better-pic0} implies
		\[m^*\mc L|_B\cong\op{pr}_1^*\mc L|_B\otimes\op{pr}_2^*\mc L|_B\]
		as line bundles on $B\times B$. Now $([1],[-1])\colon B\to B\times B$ has both $[1]\colon B\to B$ and $[-1]\colon B\to B$ being isomorphisms, by $m\circ([-1],[1])=[0]$, so pulling back along $([1],[-1])$ implies
		\[\OO_B\cong\mc L|_B\otimes[-1]^*\mc L|_B.\]
		But then $\mc L|_B$ is ample, and $[-1]$ is an isomorphism, so $[-1]^*\mc L|_B$ is ample, so $\OO_B$ is ample. But then $B$ must have dimension $0$, meaning that $B$ is finite, so $K(\mc L)$ is also finite.
		\item We show (b) implies (c). Indeed, $x\in H(D)$ with $x+D=D$ then implies $t_x^*\mc L\cong\mc L$, so $H(D)\subseteq K(\mc L)$, which is enough.
		\item We sketch (d) implies (a). It suffices to show that $\mc L^{\otimes2}$ is ample. We claim that the pullback of an ample line bundle by a finite morphism is ample. Well, $\mc L^{\otimes2}$ is ample if and only if
		\[H^i\left(X,\mc F\otimes\mc L^{\otimes2n}\right)=0\]
		for all coherent sheaves $\mc F$ and indices $i>0$. (The forward implication is just by a cohomology computation, noting that $\mc L$ to a sufficiently high power will simply induce an embedding to projective space, allowing us to compute the cohomology in projective space.)
		\item We show (c) implies (d). Quickly, we note that $(x+D)+(-x+D)\sim2D$ by translating around, via \Cref{thm:square}.

		Now, to be base-point free, we want to show that each point $p\in A$ has some section $x$ such that $(x+D)+(-x+D)$ fails to vanish on $y$; equivalently, we are asking for $x\notin(-y+D)$ and $x\notin y+D$. But $-y+D$ and $y+D$ are both of codimension $1$ in $A$, so these two divisors cannot cover $A$.

		Lastly, we need to show that the associated map $\varphi\colon A\to\PP^N_k$ is finite. Well, because $A$ is proper, it follows that $\varphi$ is proper, so it suffices to show that $\varphi$ is quasifinite. Well, suppose for the sake of contradiction that we have a closed point $y\in\PP^N_k$ with infinite fiber; surely the fiber is quasicompact, so the fiber must actually have positive dimension. Namely, there will be an irreducible proper curve $C\subseteq A$ such that $\varphi(C)$ is a point; notably, proper curves are projective, so we may as well say that $C$ is projective.

		Well, for any effective divisor $E\in\left|2D\right|$, we either have $E\cap C=C$ or $E\cap C=\emp$. For this, we will use \Cref{lem:translate-curve}, proven next class. Indeed, setting $E'\coloneqq x+D$ for some $x\in C$, one must have $(x+D)\cup(-x+D)\cap C\ne\emp$, so any $x,y\in C$ will have $(x-y+x)+D=x+D$, meaning $x-y\in H(D)$. But we have put too many points in $H(D)$, so we have achieved our contradiction.
		\qedhere
	\end{itemize}
\end{proof}
\begin{corollary} \label{cor:ab-var-proj}
	Fix an abelian $k$-variety $A$. Then $A$ is a projective $k$-scheme.
\end{corollary}
\begin{proof}
	It suffices to produce an ample line bundle $\mc L$. By \Cref{thm:get-ample}, it suffices to produce an effective divisor $D$ such that $H(D)\coloneqq\{\text{closed }x\in A:x+D=D\}$ is finite.
	
	For our construction, let $U\subseteq A$ be an affine open neighborhood of $e$. Then $D\coloneqq A\setminus U$ is an effective divisor (it is a fact that $D$ is pure of codimension $1$!). We will show our finiteness in two claims.
	\begin{itemize}
		\item We claim $H(D)\subseteq U$. Indeed, if $x\in H(D)$, then $x+D=D$, so $x+U=U$ by taking complements, so $x\in U$.
		\item We claim $H(D)\subseteq A$ is closed. Indeed, note there is a map $m\colon A\times D\to A$, and $H(D)$ by definition is $\op{pr}_A\left(m^{-1}(D)\right)$. But $D\subseteq A$ is closed, and $A$ is proper, so $H(D)$ is thus closed.
	\end{itemize}
	The above two claims imply that $H(D)$ is finite: giving $H(D)$ the reduced closed subscheme structure, we see that $H(D)\subseteq A$ is a proper $k$-variety, but it is contained in the affine $k$-variety $U$, so $H(D)$ must be zero-dimensional. (For example, to show finiteness, we may as well assume irreducibility, but then if we have positive dimension, then we will get non-constant global sections from $U$, so $\dim H(D)=0$ is forced.)
\end{proof}

\end{document}