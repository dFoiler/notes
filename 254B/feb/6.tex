% !TEX root = ../notes.tex

\documentclass[../notes.tex]{subfiles}

\begin{document}

\section{February 6}

Last class we discussed $H^1(G,\op{PGL}_n(L))$. We continue talking about $H^1$.

\subsection{Automorphisms of Projective Space}
Roughly speaking, the key point in our discussion of $H^1(G,\op{PGL}_n(L))$ was our application of the Skolem--Noether theorem to show $\op{Aut}_L(M_n(L))\cong\op{PGL}_n(L)$. In general, one can play a similar game whenever you have some object with the correct automorphisms.

Thus, we also note $\op{Aut}\PP^{n-1}_L=\op{PGL}_n(L)$. Indeed, for any automorphism $\alpha$, we can draw the following square.
% https://q.uiver.app/?q=WzAsNCxbMCwwLCJcXFBQXntuLTF9X0wiXSxbMSwwLCJcXFBQXntuLTF9X0wiXSxbMCwxLCJcXFNwZWMgTCJdLFsxLDEsIlxcU3BlYyBMIl0sWzIsMywiZyJdLFswLDEsImciXSxbMSwzXSxbMCwyXV0=&macro_url=https%3A%2F%2Fraw.githubusercontent.com%2FdFoiler%2Fnotes%2Fmaster%2Fnir.tex
\[\begin{tikzcd}
	{\PP^{n-1}_L} & {\PP^{n-1}_L} \\
	{\Spec L} & {\Spec L}
	\arrow["\alpha", from=2-1, to=2-2]
	\arrow["\alpha", from=1-1, to=1-2]
	\arrow[from=1-2, to=2-2]
	\arrow[from=1-1, to=2-1]
\end{tikzcd}\]
Notably, we can see from this square that $\alpha^*\OO_{\PP^{n-1}_L}(1)$ is ample\footnote{The ample line bundles in $\op{Pic}\PP^{n-1}_L$ are precisely the ones with global sections, and $\alpha^*\colon\op{Pic}\PP^{n-1}_L\to\op{Pic}\PP^{n-1}_L$ must send a line bundle with global sections to a line bundle with global sections.} and needs to generate $\op{Pic}\PP^{n-1}_L$ because $\alpha^*$ is an isomorphism, so we conclude that there is an isomorphism $\alpha^\flat\colon\alpha^*\OO_{\PP^{n-1}_L}(1)\to\OO_{\PP^{n-1}_L}$. (This isomorphism is not canonical!) Thus, taking global sections, we are getting a map
\[\Gamma(\PP^{n-1}_L,\OO_{\PP^{n-1}_L}(1))\cong\Gamma(\PP^{n-1}_L,\OO_{\PP^{n-1}_L}(1)).\]
However, both of these are isomorphic to $Lx_0\oplus\cdots L_{x_{n-1}}$, so the data of $(\alpha,\alpha^\flat)$ precisely describes an automorphism $L^n\to L^n$. If you mod out by the information of $\alpha^\flat$, it turns out that you exactly describe an element of $\op{PGL}_n(L)$ instead of $\op{GL}_n(L)$.

It turns out that one can do approximately the same story we gave last class to show that there is a bijection between $K$-schemes $P$ such that $P\times_{\Spec K}\Spec L\cong\PP^{n-1}_L$ and $H^1(G,\op{PGL}_n(L))$. Proving this is a little harder than last time because it is less obvious that a cocycle will come from $K$-scheme.

Nonetheless, we note that we now have two identifications of $H^1(G,\op{PGL}_n(L))$, so we should be able to take a central $K$-algebra $A$ such that $A\otimes_KL\cong M_n(L)$ and produce a $K$-scheme $P$. These are called the Brauer--Severi schemes.
\begin{example}
	Fix the field extension $\CC/\RR$ and let $\HH$ denote the quaternions, which is the nontrivial element of our $H^1$. Then it turns out that the corresponding $K$-scheme $P$ is $V\left(x^2+y^2+z^2\right)\subseteq\PP^2_k$. Notably, the line bundle $\OO_{\PP^2_k}(1)$ will pull back to $\OO_{\PP^1_k}(2)$ because it needs to pull back something with global sections, and then we can also check the dimension of these global sections to complete.
\end{example}
\begin{remark}
	One can show that the Brauer--Severi schemes are always projective and embed into $\PP^n_K$.\todo{} In fact, they have a $K$-point if and only if they are projective!
\end{remark}

\subsection{Moving to \texorpdfstring{$H^2$}{ H2}}
As usual, let $L/K$ be a Galois extension with Galois group $G$. Recall from last class that we had a short exact sequence
\[1\to L^\times\to\op{GL}_n(L)\to\op{PGL}_n(L)\to1,\]
which gave rise (via cocycles!) to a map $\delta_n\colon H^1(G,\op{PGL}_n(L))\to H^2(G,L^\times)$. It turns out that this fits into an exact sequence (of pointed sets)
\[H^1(G,\op{GL}_n(L))\to H^1(G,\op{PGL}_n(L))\stackrel{\delta_n}\to H^2\left(G,L^\times\right),\]
which is a check that we omit.
\begin{lemma} \label{lem:delta-n-is-iso}
	Fix everything as above.
	\begin{listalph}
		\item $H^1(G,\op{GL}_n(L))=1$.
		\item If $n=[L:K]$, then $\delta_n$ is surjective.
	\end{listalph}
\end{lemma}
\begin{proof}
	Here we go.
	\begin{listalph}
		\item We know from our discussion of Hilbert's theorem 90 that $H^1(G,\op{GL}_n(L))$ is in natural bijection to isomorphism classes $n$-dimensional $L$-vector spaces with a given semilinear $G$-action. However, this category $\mathrm{Mod}(L/K)$ we showed (in \Cref{thm:easy-descent}) is just the $K$-vector spaces of dimension $n$, so there is only one up to isomorphism, completing the proof.
		\item This requires a trick. Fix a $2$-cocycle $f\colon G^2\to L^\times$. Working explicitly, we want $\rho\colon G\to\op{GL}_n(L)$ such that
		\[f(g,g')=\rho_g\cdot g\rho_{g'}\cdot\rho_{gg'}^{-1},\]
		where we have identified $L^\times$ with its image in $\op{GL}_n(L)$. Note that such a $\rho$ grants us a $1$-cocycle $\overline\rho\colon G\to\op{PGL}_n(L)$ by modding out by $L^\times$ everywhere.
		
		Well, we use an induced module: set $V\coloneqq\op{Mor}(G,L)$, which we note has basis given by $e_s(g)\coloneqq1_{s=g}(g)$ because $G$ is finite. We may thus define $\rho_g\colon V\to V$ given by $\rho_g\colon e_s\mapsto f(g,s)e_s$. To finish, one can show that this $\rho_\bullet$ satisfies the needed equality.
		\qedhere
	\end{listalph}
\end{proof}
\begin{corollary} \label{cor:h2-is-algebras}
	If $n=[L:K]$, then there is a natural identification with central $K$-algebras $A$ such that $A\otimes_K\cong M_n(L)$ and elements of $H^2(G,L^\times)$.
\end{corollary}
\begin{proof}
	It suffices to show that our $\delta_n$ is an isomorphism. This follows directly from \Cref{lem:delta-n-is-iso}.
\end{proof}
\begin{remark}
	In fact, we note that we can fully go backward from a $2$-cocycle to its constructed $1$-cocycle in $H^1(G,\op{PGL}_n(L))$, and then we know how to turn that data into a central $K$-algebra $A$ with $A\otimes_KL\cong M_n(L)$.
\end{remark}
We are now ready to define the Brauer group.
\begin{definition}[Brauer group]
	Fix a Galois extension $L/K$, we define the \textit{Brauer group} $\op{Br}(L/K)$ as the set of isomorphism classes of central $K$-algebras $A$ such that $A\otimes_KL\cong M_n(L)$.
\end{definition}
We can extend this construction as follows: \Cref{cor:h2-is-algebras} grants us a natural isomorphism
\[H^2(G,L^\times)\to\op{Br}(L/K).\]
Now, define
\[H^2\left(\op{Gal}(K^{\mathrm{sep}}/K),(K^{\mathrm{sep}})^\times\right)\coloneqq\colimit_{K\subseteq L\subseteq K^{\mathrm{set}}} H^2(\op{Gal}(L/K),L^\times).\]
On the other side, define $\op{Br}K$ as the central $K$-algebras $A$ such that $A\otimes_KK^{\mathrm{sep}}\cong M_n(K^{\mathrm{sep}})$ for some $n$, but we mod out by the equivalence $A\sim B$ if and only if $M_n(A)\cong M_m(B)$ for some $n$ and $m$. Then one can show that the $\delta_n$s induce an isomorphism
\[\op{Br}K\cong H^2\left(\op{Gal}(K^{\mathrm{sep}}/K),(K^{\mathrm{sep}})^\times\right),\]
which allows us to stop paying attention to the field $L$.
\begin{remark}
	One can show that division rings are also in natural bijection with our algebras, giving us yet another identification.
\end{remark}

\end{document}