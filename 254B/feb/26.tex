% !TEX root = ../notes.tex

\documentclass[../notes.tex]{subfiles}

\begin{document}

We continue.

\subsection{Cohomology Rings as Hopf Algebras}
Last class we stated the following result.
\avcohomringprop*
\noindent In fact, we will use the classification of Hopf algebras to show that both sides here are Hopf algebras and that they are isomorphic. For example,
\[H_A\coloneqq\bigoplus_{i=0}^gH^i(A,\OO_A)\]
is a graded $k$-algebra with product given by the cup product. To see this cup product, one can define it by
\[H_A\otimes_kH_A\stackrel{\Delta^*}\to H_{A\times A}\to H_A\]
where the first map is the K\"unneth formula, and the last map is given by pullback along the diagonal $\Delta\colon A\to A\times A$. In fact, there is some extra structure of a cocommutative coalgebra. Indeed, there is a map
\[H_A\stackrel{m^*}\to H_{A\times A}\cong H_A\otimes_kH_A,\]
where again the second map is the K\"unneth formula. We also have an inversion $[-1]^*\colon H_A\to H_A$. All of this structure can be put into a Hopf algebra.
\begin{definition}[Hopf algebra]
	Fix a field $k$. Then a \textit{Hopf algebra} is a graded $k$-vector space equipped with a product $m\colon H\otimes H\to H$, a comultiplication $\Delta\colon H\to H\otimes H$, an inversion $s\colon H\to H$, an identity $\varepsilon H\to k$, and a coidentity $\delta\colon k\to H$, satisfying the following.
	\begin{itemize}
		\item $(m,\delta)$ makes $H$ into a $k$-algebra.
		\item $(\Delta,\varepsilon)$ makes $H$ into a $k$-coalgebra, meaning that the following diagrams commute.
		% https://q.uiver.app/#q=WzAsNCxbMCwwLCJIIl0sWzEsMCwiSFxcb3RpbWVzIEgiXSxbMCwxLCJIXFxvdGltZXMgSCJdLFsxLDEsIkhcXG90aW1lcyBIXFxvdGltZXMgSCJdLFswLDEsIlxcRGVsdGEiXSxbMCwyLCJcXERlbHRhIiwyXSxbMiwzLCJcXERlbHRhXFxvdGltZXNcXGlkIl0sWzEsMywie1xcaWR9XFxvdGltZXNcXERlbHRhIl1d&macro_url=https%3A%2F%2Fraw.githubusercontent.com%2FdFoiler%2Fnotes%2Fmaster%2Fnir.tex
		\[\begin{tikzcd}
			H & {H\otimes H} \\
			{H\otimes H} & {H\otimes H\otimes H}
			\arrow["\Delta", from=1-1, to=1-2]
			\arrow["\Delta"', from=1-1, to=2-1]
			\arrow["\Delta\otimes\id", from=2-1, to=2-2]
			\arrow["{{\id}\otimes\Delta}", from=1-2, to=2-2]
		\end{tikzcd}\]
		\item The maps $\Delta$ and $m$ are algebra and coalgebra homomorphisms, respectively.
		\item The map $s$
	\end{itemize}
\end{definition}
\begin{remark}
	We can see that commutative Hopf $k$-algebras $A$ are equivalent to affine group $k$-sche\-mes. Indeed, one can just unwind the definition of an affine group $k$-scheme to see that they are just schemes of the form $\Spec A$ where $A$ is a commutative Hopf $k$-algebra.
\end{remark}
We will also want some notion of commutativity in our graded setting.
\begin{definition}[graded commutative]
	A $k$-algebra $H$ is \textit{graded commutative} if and only if any homogeneous elements $a,b\in H$ have
	\[ab=(-1)^{(\deg a)(\deg b)}ba.\]
\end{definition}
\begin{example}
	Fix an abelian $k$-variety $A$. Then our work above tells us that $H_A$ is a finite dimensional graded commutative Hopf $k$-algebra. In fact, we see that $H^0=k$ by taking global sections. We also note that we can compute
	\[m^*(h)=(1\otimes h)+(h\otimes 1)+\sum_{i>j>0}(h_i\otimes h_j)\]
	for some unknown $h_i$ and $h_j$. One can see this because $m$ maps $1\otimes h$ and $h\otimes1$ to $h$.\todo{Why?}
\end{example}
The above data will be enough for our classification result.
\begin{lemma}
	Fix a perfect field $k$. Suppose that $H$ is a graded commutative Hopf $k$-algebra such that $H^0=k$ and $H^r=0$ for $r>g$ and any $h\in H$ has
	\[m^*(h)=(1\otimes h)+(h\otimes 1)+\sum_{i>j>0}(h_i\otimes h_j)\]
	for some unknown $h_i$ and $h_j$. Then $\dim H^1\le g$; in fact, if $\dim H^1=g$, then $H\cong\bigland H^1$ as graded commutative Hopf $k$-algebras.
\end{lemma}
\begin{proof}[Sketch]
	One can show (and it is due to Borel) that such an $H$ is generated by finitely many homogeneous elements, generated essentially freely by these elements (i.e., the only relations are given by the graded commutativity and nilpotent), so let these generators be $x_1,\ldots,x_m$. Notably,
	\[\deg\prod_{i=1}^mx_i=\sum_{i=1}^m\deg x_i\le g,\]
	where the inequality at the end is because the product must be nonzero, so we see that $\dim H_1\le g$ because $\dim H_1$ is upper-bounded by the number of $x_i$ with $\deg x_i=1$. But if we have $\dim H_1=g$, then the above degree computation must achieve equality, so all the generators must have degree exactly $1$, and there must be $g$ of them. Furthermore, we claim that $x_j^2=0$ for each generator $x_j$, which holds because $x_j^2\ne0$ means that the product $x_j\prod_ix_i$ is still nonzero but has degree larger than $g$, which is a contradiction.
\end{proof}
One can then feed the above lemma into \Cref{prop:cohom-of-av} to show that $\dim H^1(A,\OO_A)\le g$, which is enough for our purposes because the quasifinite surjection $A\onto A^\lor$ promises that $\dim A^\lor\ge\dim A$. So in fact we get the isomorphism claimed in \Cref{prop:cohom-of-av}. This in turn completes the proof of \Cref{thm:dual-av-smooth}.

\subsection{Polarizations}
We now discuss some special isogenies.
\begin{definition}[polarization]
	Fix an abelian $k$-variety $A$. An isogeny $\lambda\colon A\to A^\lor$ is a \textit{polarization} if and only if $\lambda_{\ov k}=\varphi_{\mc L}$ for some ample line bundle $\mc L$ on $A_{\ov k}$. A polarization $\lambda$ is \textit{principal} if and only if $\deg\lambda=1$; i.e., $\lambda$ is an isomorphism.
\end{definition}
\begin{remark}
	Each line bundle $\mc L'\in\Pic^\circ_{A/k}(A)$ will have $\varphi_{\mc L}=\varphi_{\mc L\otimes\mc L'}$, and in fact the converse still holds by unwinding the definition of $\Pic^0$. As such, we can think about polarizations as being a subset of
	\[\frac{\Pic(A_{\ov k})}{\Pic^0(A_{\ov k})}.\]
\end{remark}
\begin{definition}[N\'eron--Severi group]
	Fix an abelian $k$-variety $A$. Then the \textit{N\'eron--Severi group} is
	\[\op{NS}(A)\coloneqq\frac{\Pic_{A/k}(A_{\ov k})}{\Pic_{A/k}^\circ(A_{\ov k})}.\]
\end{definition}
Approximately speaking, the N\'eron--Serveri group measures polarizations.
\begin{remark}
	Take $k=\CC$. Then the exponential short exact sequence $0\to\ZZ\to\OO_A\to\OO_A^\times\to1$ produces a long exact sequence
	\[H^1(A,\ZZ)\to H^1(A,\OO_A)\to H^1(A,\OO_A^\times)\to H^2(A,\ZZ).\]
	Now, $H^1(A,\OO_A)/H^1(A,\ZZ)\cong\Pic^0(A)$ as discussed earlier, so we note that we have the exact sequence
	\[H^1(A,\ZZ)\to H^1(A,\OO_A)\to\underbrace{H^1(A,\OO_A^\times)}_{\Pic A}\to\op{NS}(A)\to0.\]
	Thus, $\op{NS}(A)$ is a finitely generated $\ZZ$-module because it embeds into $H^2(A,\ZZ)$.
\end{remark}
\begin{remark}
	More generally, $\op{NS}(A)$ is a free $\ZZ$-module of finite rank for any abelian $k$-variety $A$, for any field $k$. The point is that viewing $\op{NS}(A)$ as polarizations will embed into
	\[\op{Hom}_{\ov k}(A_{\ov k},A^\lor_{\ov k})\]
	by $\varphi_\bullet$, and the target is a free $\ZZ$-module of finite rank because one can show that any prime $\ell$ with $\op{char}k\nmid\ell$ builds an injection
	\[T_\ell\colon\op{Hom}_{\ov k}(A_{\ov k},A^\lor_{\ov k})\to\op{Hom}(T_\ell(A_{\ov k}),T_\ell(A^\lor_{\ov k})),\]
	and the target is a free $\ZZ_\ell$-module of finite rank, and the proof of this inclusion is able to show that the source is thus free of finite rank (over $\ZZ$!).
\end{remark}
\begin{remark}
	Just because $\lambda$ is a polarization does not mean that there is a line bundle $\mc L$ on $A$ such that $\lambda=\varphi_{\mc L}$. Take $k$ to be perfect so that we can use Galois descent by $G\coloneqq\op{Gal}(\ov k/k)$. By definition of $\op{NS}(A)$, we have an exact sequence
	\[0\to A^\lor(\ov k)\to\Pic_{A/k}(\ov k)\to\op{NS}(A)\to0,\]
	so we get a long exact sequence
	\[0\to A^\lor (k)\to\Pic_{A/k}(k)\to\op{NS}(A)^G\to H^1(G,A^\lor(\ov k)).\]
	As such, we are asking if every $\lambda\in\op{NS}(A)^G$ comes from $\op{Pic}_{A/k}(k)$, which might be false if $H^1(G,A^\lor(\ov k))$ fails to vanish. However, it turns out that this is not the case if $k$ is finite.
\end{remark}
\begin{remark}
	Fix a projective $k$-curve $X$ and some $k$-rational point $x_0\in X(k)$. One can show that $J(X)\coloneqq\op{Pic}_{X/k}^\circ$ is a smooth group scheme and hence an abelian variety. Now, each $d>0$ produces a map $X^d\to J(X)$ by sending $(x_1,\ldots,x_d)$ to the line bundle $\OO_X(x_1)\otimes\cdots\otimes\OO_X(x_d)\otimes\OO_X(-x_0)^{\otimes d}$. In particular, it turns out that the image of $X^{g-1}\to J(X)$ gives rise to an ample line bundle $\mc L$ and hence a polarization $\varphi_{\mc L}$. In fact, this is a principal polarization.
\end{remark}

\end{document}