% !TEX root = ../notes.tex

\documentclass[../notes.tex]{subfiles}

\begin{document}

We now move into a discussion of quotients, so we will want to understand some descent. Homework will be posted over the weekend.

\subsection{Cartier Duals}
Our end goal is the following result.
\begin{restatable}[{\cite[Theorem~15.1]{mumford}}]{theorem}{dualmapthm} \label{thm:dual-map}
	Fix an isogeny $f\colon A\to B$ of abelian $k$-varieties. Then there is a dual isogeny $f^\lor\colon B^\lor\to A^\lor$ defined by sending $(\mc L,\alpha)\in\Pic^\circ_{B/k}(T)$ to $(f^\mc L,f^*\alpha)\in\Pic^\circ_{A/k}(T)$. In fact, $\ker f^\lor=(\ker f)^\lor$.
\end{restatable}
\begin{remark}
	For definition of $f^\lor$ to make sense, $f$ merely needs to be a homomorphism.
\end{remark}
Wait, how does one define $(\ker f)^\lor$? Well, we will use the Cartier dual \cite[\S14]{mumford}.
\begin{definition}
	Fix a finite commutative group $k$-scheme $G$ given by the commutative (and cocommutative) Hopf $k$-algebra $H$. Then we define the dual $H^\lor\coloneqq\op{Hom}_k(H,k)$, which is still a Hopf $k$-algebra, so the \textit{Cartier dual $G^\lor$} of $G$ is the finite commutative group $k$-scheme
	\[G^\lor\coloneqq\Spec H^\lor.\]
\end{definition}
\begin{remark}
	Let's explain how $H^\lor$ is a Hopf $k$-algebra. For example, the unit is a map $k\to H$ dualizes to a map $H^\lor\to k$, which is the counit; similarly, the multiplication is a map $H\otimes_kH\to H$, which dualizes to a map $H^\lor\to H^\lor\otimes H^\lor$, which is the comultiplication.
\end{remark}
\begin{remark}
	We can see on the level of Hopf algebras that $G^{\lor\lor}=G$.
\end{remark}
More generally, one can discuss the Hopf algebra of morphisms.
\begin{defihelper}[$\mathrm{Hom}$ scheme] \nirindex{Hom scheme@$\mathrm{Hom}$ scheme}
	Fix commutative groups $S$-schemes $G$ and $H$. Then we define the functor $\underline{\op{Hom}_S(G,H)}\colon\mathrm{Sch}_S\to\mathrm{Ab}$ by
	\[\underline{\op{Hom}_S(G,H)}(T)\coloneqq\op{Hom}_T(G_T,H_T)\]
	is in fact represented by an $S$-scheme frequently.
\end{defihelper}
Let's see an example of this, which provides another way to think about the Cartier dual.
\begin{proposition}
	Fix a finite commutative group $k$-scheme $G$. Then $G^\lor$ represents $\underline{\op{Hom}_k(G,\mathbb G_m)}$.
\end{proposition}
\begin{proof}
	It suffices to check the result on affine schemes. Namely, given a $k$-algebra $R$, we need a (natural) isomorphism between $G^\lor(R)$ and $\op{Hom}_R(G_R,\mathbb G_{m,R})$. On one hand, we compute
	\[G^\lor(R)=\op{Hom}_k(H^\lor,R)=\op{Hom}_R(H^\lor\otimes_kR,R).\]
	Notably, this is a subset of $R$-linear maps $H^\lor\otimes_kR\to R$, which is $H_R$ after taking another dual. Well, $\varphi\in H_R$ if and only if $\varphi(1)=1$ and $\varphi(ab)=\varphi(a)\varphi(b)$. Explicitly, letting $\Delta\colon H\to H\otimes_kH$ denote the comultiplication and letting $\varepsilon\colon H\to k$ denote the counit, we find that we are asking for $\Delta_R(\varphi)=\varphi\otimes\varphi$ and $\varepsilon_R(\varphi)=1$.

	On the other hand,
	\[\op{Hom}_R(G_R,\mathbb G_{m,R})=\op{Hom}_{\mathrm{Hopf}_R}\left(R[t,t^{-1}],H_R\right).\]
	Thus, we see that we are in bijection with invertible elements of $H_R$ such that the relevant map preserves the Hopf algebra structure. In particular, preserving the comultiplication map $t\mapsto t\otimes t$ is asking for $\varphi\in H_R$ to be invertible as well as $\Delta_R(\varphi)=\varphi\otimes\varphi$.

	So in total, we need to relate units in $H_R$ to having $\varepsilon_R(\varphi)=1$, which is a general fact. Certainly $\varepsilon_R(\varphi)=1$ implies that $\varphi$ is a unit by using the comultiplication.  In the reverse direction, we note $1\cdot1=1$ rewrites as $(\varepsilon\otimes\varepsilon)\circ\Delta=\varepsilon$, meaning that $\varepsilon_R(\varphi)^2=\varepsilon_R(\varphi)$ always, so $\varphi$ being invertible requires $\varepsilon_R(\varphi)=1$.
\end{proof}
\begin{remark}
	Fix a finite commutative group $k$-scheme $G$ with $G=\Spec H$ for Hopf $k$-algebra $H$. Then $k[G]\cong H^\lor$, where we send $g\in G$ to the map $H\to k$ corresponding to evaluation at $g$. (Notably, we are viewing $H$ as global sections of $G$, so evaluation makes sense.)
\end{remark}
\begin{example}
	Take $G=(\ZZ/n\ZZ)_k$, which we note is an \'etale reduced group scheme with $n$ (closed) points. We claim that $G^\lor=\mu_n$. Set $H$ to be the Hopf $k$-algebra corresponding to $G$. Using the previous remark, we find that
	\[H^\lor=k[G]=\frac{k[x]}{\left(x^n-1\right)}\]
	at least as $k$-algebras. It remains to check that comultiplication structure is the same on both. On $\mu_n$, the comultiplication structure is given by $x\mapsto(x\otimes x)$, so we just have to track it through on the dual. Well, for global sections $f,g\in H$, we evaluate
	\[(fg)([1]_n)=f([1]_n)g([1]_n)=(f\otimes g)([1]_n\otimes[1]_n),\]
	so we have the correct comultiplication.
\end{example}

\subsection{fpqc Descent}
We take a short intermission to discuss fpqc descent. There are lots of references; for example, see \cite[\S6]{conrad-av}. We will work with relatively light hypotheses.
\begin{definition}[fpqc]
	A morphism $f\colon X\to Y$ of schemes is \textit{fpqc} if and only if it is faithfully flat and quasicompact.
\end{definition}
\begin{remark}
	Somehow we are generalizing the discussion for gluing on Zariski opens.
\end{remark}
Let's discuss what gluing looks like. Fix a map $f\colon S_0\to S$ which is fpqc; then we set $S_1\coloneqq S_0\times_SS_0$ and $S_2\coloneqq S_0\times_SS_0\times_SS_0$, and we let $p_{12},p_{23},p_{13}\colon S_2\to S_1$ and $p_1,p_2\colon S_1\to S_0$ be the projections. We would like to discuss when we can lift quasicoherent sheaves.
\begin{definition}[descent datum]
	Fix everything as above. Given a quasicoherent sheaf $\mc F$ on $S_0$, a \textit{descent datum} on $\mc F$ is an isomorphism $\theta\colon p_1^*\mc F\to p_2^*\mc F$ of quasicoherent sheaves on $S_1$ satisfying the ``cocycle condition'' that
	\[p_{13}^*\theta=p_{23}^*\theta\circ p_{12}^*\theta.\]
	A morphism of descent datum $h\colon(\mc F,\theta)\to(\mc G,\psi)$ is a morphism of the quasicoherent sheaves commuting with the isomorphisms. In other words, the following diagram commutes.
	% https://q.uiver.app/#q=WzAsNCxbMCwwLCJwXzFeKlxcbWMgRiJdLFsxLDAsInBfMV4qXFxtYyBHIl0sWzAsMSwicF8yXipcXG1jIEYiXSxbMSwxLCJwXzJeKlxcbWMgRyJdLFswLDEsInBfMV4qaCJdLFsyLDMsInBfMl4qaCJdLFswLDIsIlxcdGhldGEiLDJdLFsxLDMsIlxccHNpIl1d&macro_url=https%3A%2F%2Fraw.githubusercontent.com%2FdFoiler%2Fnotes%2Fmaster%2Fnir.tex
	\[\begin{tikzcd}
		{p_1^*\mc F} & {p_1^*\mc G} \\
		{p_2^*\mc F} & {p_2^*\mc G}
		\arrow["{p_1^*h}", from=1-1, to=1-2]
		\arrow["{p_2^*h}", from=2-1, to=2-2]
		\arrow["\theta"', from=1-1, to=2-1]
		\arrow["\psi", from=1-2, to=2-2]
	\end{tikzcd}\]
\end{definition}
More explicitly, the equality to define the descent datum is asking for the following diagram to commute.
% https://q.uiver.app/#q=WzAsNixbMCwwLCJwX3sxMn1eKnBfMV4qXFxtYyBGIl0sWzAsMSwicF97MTN9XipwXzFeKlxcbWMgRiJdLFsxLDAsInBfezEyfV4qcF8yXipcXG1jIEYiXSxbMiwwLCJwX3syM31eKnBfMl4qXFxtYyBGIl0sWzMsMCwicF97MjN9XipwXzFeKiJdLFszLDEsInBfezEzfV4qcF8yXipcXG1jIEYiXSxbMCwyLCJwX3sxMn1eKlxcdGhldGEiXSxbMiwzLCIiLDAseyJsZXZlbCI6Miwic3R5bGUiOnsiaGVhZCI6eyJuYW1lIjoibm9uZSJ9fX1dLFszLDQsInBfezIzfV4qXFx0aGV0YSJdLFswLDEsIiIsMix7ImxldmVsIjoyLCJzdHlsZSI6eyJoZWFkIjp7Im5hbWUiOiJub25lIn19fV0sWzQsNSwiIiwwLHsibGV2ZWwiOjIsInN0eWxlIjp7ImhlYWQiOnsibmFtZSI6Im5vbmUifX19XSxbMSw1LCJwX3sxM31eKlxcdGhldGEiLDJdXQ==&macro_url=https%3A%2F%2Fraw.githubusercontent.com%2FdFoiler%2Fnotes%2Fmaster%2Fnir.tex
\[\begin{tikzcd}
	{p_{12}^*p_1^*\mc F} & {p_{12}^*p_2^*\mc F} & {p_{23}^*p_2^*\mc F} & {p_{23}^*p_1^*} \\
	{p_{13}^*p_1^*\mc F} &&& {p_{13}^*p_2^*\mc F}
	\arrow["{p_{12}^*\theta}", from=1-1, to=1-2]
	\arrow[Rightarrow, no head, from=1-2, to=1-3]
	\arrow["{p_{23}^*\theta}", from=1-3, to=1-4]
	\arrow[Rightarrow, no head, from=1-1, to=2-1]
	\arrow[Rightarrow, no head, from=1-4, to=2-4]
	\arrow["{p_{13}^*\theta}"', from=2-1, to=2-4]
\end{tikzcd}\]
The equalities listed above are really natural isomorphisms induced by equalities of projections; for example, $p_1\circ p_{12}=p_1\circ p_{13}$.

Here is our result.
\begin{theorem} \label{thm:fpqc-descent}
	Fix a map $f\colon S_0\to S$ which is fpqc. Then $\mathrm{QCoh}(S)$ is equivalent to the category of descent data $(\mc F,\theta)$.
\end{theorem}
\begin{proof}
	The forward map takes a quasicoherent sheaf $\mc F$ on $S$ to the pair $(f^*\mc F,\theta_{\mc F})$ where $\theta_{\mc F}$ is the composite
	\[p_1^*f^*\mc F=(f\circ p_1)^*\mc F=(f\circ p_2)^*\mc F=p_2^*f^*\mc F.\]
	It remains to discuss the inverse functor. The proof reduces to the affine case, where we are talking about modules, and one can attempt to recover the original model from the descent datum by taking some kernel.
\end{proof}
One can even discuss descent datum on schemes.
\begin{definition}[descent datum]
	Fix an $S_0$-scheme $X$. Build $S_1$ and $S_2$ and the projections as above. Then a \textit{descent datum} is an isomorphism $\theta\colon X\times_{S_0,p_1}S_1\cong X\times_{S_0,p_2}S_1$ such that
	\[p_{13}^*\theta=p_{23}^*\theta\circ p_{12}^*\theta.\]
\end{definition}
\begin{remark}
	In general, we do not expect to be able to actually get a scheme from descent datum, but we will be okay for affine schemes because these are basically understood by their global sections.
\end{remark}

\end{document}