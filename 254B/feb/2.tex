% !TEX root = ../notes.tex

\documentclass[../notes.tex]{subfiles}

\begin{document}

Office hours next week will move to 2PM--4PM on Wednesday. I am pretty hopelessly behind catching up on adding details to these notes, but I will do my best to catch up over the weekend. Next week we start algebraic geometry.

\subsection{The Shimura--Taniyama Formula}
Fix an abelian variety $A$ over a number field $K$. We want to ``reduce $A$ modulo'' a prime $\mf P\in\Spec\OO_K$.
\begin{definition}[good reduction]
	Fix an abelian variety $A$ over a number field $K$. Given a prime $\mf P$ of $K$, we say that $A$ has \textit{good reduction at $\mf P$} if and only if there is an abelian scheme $\mc A$ over $\OO_{K_\mf P}$ such that $\mc A_K=A$. By abuse of notation, we let $A_\mf P$ denote $\mc A_{\OO_K/\mf P}$.
\end{definition}
\begin{remark}
	The theory of N\'eron models implies that the model $\mc A$ over $\OO_{K_\mf P}$ is unique. We will discuss this more later.
\end{remark}
\begin{remark}
	The theory of N\'eron models also tells us that
	\[\op{End}_K(A)\op{End}_{\OO_{K_\mf P}}(\mc A)\subseteq\op{End}(\mc A_\mf P).\]
	The last inclusion assumes complex multiplication of $A$.
\end{remark}
\begin{remark}
	It turns out that one can always extend $K$ to have good reduction.
\end{remark}
\begin{definition}[Frobenius]
	Fix a finite field $\FF_q$. Given an $\mathbb F_q$-variety $X$, we define the \textit{Frobenius morphism} $F_X\colon X\to X$ to be the identity on points and the $q$-power map on the sheaves $\OO_X\to\OO_X$.
\end{definition}
\begin{remark}
	On points, one can compute that the Frobenius map $F\colon\AA^n_{\FF_q}\to\AA^n_{\FF_q}$ maps $(x_1,\ldots,x_n)\in\AA^n_{\FF_q}(\ov{\FF_q})$ to $(x_1^q,\ldots,x_n^1)\in\AA^n_{\FF_q}(\ov{\FF_q})$ because we are merely composing with the $q$-power map.
\end{remark}
\begin{definition}[Tate module]
	Fix an abelian variety $A$ over a number field $K$ and a prime $\ell$. Then we define the \textit{Tate module} as
	\[T_\ell A\coloneqq\limit A\left[\ell^\bullet\right].\]
\end{definition}
And now here is our result.
\begin{theorem}[Shimura--Taniyama] \label{thm:st}
	Fix an abelian variety $A$ over a number field $K$ of CM type $(E,\Phi)$ such that $K$ contains all Galois conjugates of $E$ (namely, $E$ is a field) and $E\subseteq\op{End}^0_K(A)$. If $\mf P$ is a prime of good reduction, then the following hold.
	\begin{listalph}
		\item There is an element $\pi\in\OO_E$ such that $\pi\in\op{End}^0_K(A)$ is the Frobenius $F_A$.
		\item The ideal $(\pi)\subseteq\OO_E$ is given by
		\[\prod_{\varphi\in\Phi}\varphi^{-1}\left(\op N_{K/\varphi(E)}\mf P\right).\]
	\end{listalph}
\end{theorem}
\begin{warn}
	Today, we will prove \Cref{thm:st} under the additional assumptions that $K_\mf P/\QQ_p$ is unramified, where $p$ lies under $\mf P$, and that $\op{End}^0_K(A)\cap E=\OO_E$.
\end{warn}
Here is another statement of \Cref{thm:st}.
\begin{theorem}[Shimura--Taniyama] \label{thm:st-2}
	Fix an abelian variety $A$ over a number field $K$ of CM type $(E,\Phi)$, where $E\subseteq\op{End}^0_K(A)$ is a field. If $\mf P$ is a prime of good reduction, then the following hold.
	\begin{listalph}
		\item There is an element $\pi\in\OO_E$ such that $\pi\in\op{End}^0_K(A)$ is the Frobenius $F_A$.
		\item For each place $\mf p$ of $E$ lying over $p$, we have
		\[\frac{\ord_v(\pi)}{\ord_v(q)}=\frac{\#(\Phi\cap H_v)}{\#H_v},\]
		where $H_v\coloneqq\op{Hom}(E,\ov{\QQ}_p)=\bigsqcup_{v\mid p}\op{Hom}(E_v,\ov{\QQ}_p)$.
	\end{listalph}
\end{theorem}
Let's see an application.
\begin{corollary}
	Fix an abelian variety $A$ over a number field $K$ of CM type $(E,\Phi)$, where $E\subseteq\op{End}^0_K(A)$, and let $\mf P$ be a prime of good reduction.
	\begin{listalph}
		\item Let $P$ denote the characteristic polynomial of $F_{A_\mf P}$ acting on $H_1(A(\CC),\QQ)$. We have $P\in\ZZ[x]$.
		\item The $q$-adic valuation of the eigenvalues of $F_{A_\mf P}$ given by
		\[\left\{\frac{\#(\Phi\cap H_v)}{\#H_v}\right\}_{v\mid p},\]
		with multiplicities given by $H_v\coloneqq\op{Hom}(E,\ov{\QQ}_p)$ as before.
	\end{listalph}
\end{corollary}
\begin{proof}
	For (a), use \Cref{thm:st} so that $\pi\in\OO_E$ is the needed Frobenius element. Then the characteristic polynomial of $\pi$ acting on $H_1(A(\CC),\QQ)$ is simply $\pi$ acting on $E$, so our characteristic polynomial has integer coefficients because $\pi\in\OO_E$ is integral.

	For (b), we note over $\QQ_p$ we note that our characteristic polynomial is
	\[\prod_{v\mid p}\prod_{\sigma\in\op{Hom}(E_v,\ov{\QQ}_p)}(x-\sigma(\pi)),\]
	but looping over all $\sigma$ will have the same valuation as $\ord_v(\pi)/\ord_v(q)$, so normalizing with the valuation of $q$ as $1$ achieves the result directly from (b) of \Cref{thm:st-2}.
\end{proof}
\begin{remark}
	Part (a) does not need \Cref{thm:st}; this is true without even having complex multiplication at all.
\end{remark}
While we're here, let's see some examples.
\begin{example}
	Fix an elliptic curve $A$ with complex multiplication by an imaginary quadratic field $E/\QQ$, and let $\Phi$ be the CM type. Fix a prime $p$. There are two cases.
	\begin{itemize}
		\item Ordinary: we can have $p=\mf p_1\mf p_2$ up in $E$. Then $\#H_{\mf p_1}=\#H_{\mf p_2}$, so the eigenvalues of the Frobenius will be $0$ and $1$ by looking at \Cref{thm:st-2}.
		\item Supersingular: we can have $p$ inert or ramified so that $\#H_v=2$, but then $\Phi\cap H_v$ will always have a single intersection with $\Phi$, so our eigenvalues have valuation $1/2$ and $1/2$.
	\end{itemize}
\end{example}
\begin{example}
	Fix an abelian surface $A$ with complex multiplication by $E\coloneqq\QQ(\zeta_5)$. It turns out that all CM types are isomorphic to each other, so we will denote a random one by $\Phi$. We have the following cases for an unramified prime $p$.
	\begin{itemize}
		\item If $p$ splits completely, then $\#H_v=1$ for any $v\mid p$, so the $q$-valuation of the eigenvalues 
		\item If $p$ fails to split completely, then the $q$-valuations turn out to all be $1/2$. Quickly, one finds that all primes must be inert in the extension $E/\QQ(\sqrt5)$, and $c(H_v)=H_v$, so half of the elements will be in $H_v$ and half not.
	\end{itemize}
\end{example}
\begin{remark}
	On the homework, we will compute the $q$-adic valuation of the Frobenius eigenvalues of $J(C)$ from \cref{subsec:jac-cm}.
\end{remark}
\begin{remark}
	On the homework, we will compute an example of an abelian surface $A$ with complex multiplication such that its $q$-valuations have Frobenius eigenvalues of $q$-valuation $\{0,1/2,1/2,1\}$.
\end{remark}
\begin{remark}
	A presence of a Weil pairing on Tate modules explain why our eigenvalues of Frobenius appear ``symmetric'' (as in $\{0,1/2,1/2,1\}$).
\end{remark}
Anyway, let's sketch an argument for \Cref{thm:st-2}; we will do it in detail later in the class.
\begin{proof}[Sketch of (a) in \Cref{thm:st-2}]
	For (a), we note that the action of $F_{A_\mf P}$ on $\OO_E$ commutes with the action of the larger $\op{End}^0_{K_\mf P}(A_\mf P)$, so it follows that it must live in $\OO_E$ by an argument on semisimple modules. Namely, one does something with the Tate modules: one has $T_\ell A\otimes_{\ZZ_\ell}\QQ_\ell$ is a rank $1$ module over $\OO_E\otimes_{\ZZ_\ell}\QQ_\ell$, so they must be the same.
\end{proof}

\end{document}