% !TEX root = ../notes.tex

\documentclass[../notes.tex]{subfiles}

\begin{document}

Today we will prove the Theorem of the Cube.

\subsection{Proof of The Theorem of the Cube}
We prove \Cref{thm:cube}. The strategy is to reduce to the case of curves.
\cubethm*
\noindent We have two steps. To begin, we reduce to the case where $X$ is a curve. We will want the following tools.
\begin{theorem}[Chow's lemma] \label{thm:chow}
	Fix a proper $A$-scheme $\pi\colon X\to\Spec A$. Then there is an $A$-scheme map $\mu\colon X'\to X$ such that $\mu$ is surjective and projective, $X'$ is projective, and there is an open dense subscheme $U\subseteq X$ such that $\mu\colon\mu^{-1}U\to U$ is an isomorphism.
\end{theorem}
\begin{proof}
	See \cite[Vakil~19.9.2]{rising-sea}.
\end{proof}
\begin{theorem}[Bertini] \label{thm:bertini}
	Fix an infinite field $k$ and a geometrically integral projective $k$-scheme $X\subseteq\PP^N_k$. Then there is a hyperplane $H\subseteq\PP^N_k$ such that $H\cap X$ is geometrically integral. In fact, the collection of $H\in\left(\PP^N_k\right)^\lor$ with $H\cap X$ geometrically integral is Zariski dense.
\end{theorem}
\begin{remark}
	One can add adjectives to $X$ and then to the conclusion, like smoothness.
\end{remark}
\begin{remark} \label{rem:finite-field-bertini}
	One can allow finite fields by working with hypersurfaces instead of hyperplanes; see \cite{charles-poonen-2017}.
\end{remark}
This allows us to prove the following geometric fact.
\begin{lemma}
	Fix a proper, geometrically integral $k$-variety $X$. For any two closed points $x_0,x_1\in X$, there is a closed $1$-dimensional $k$-subvariety $C\subseteq X$ containing $x_0$ and $x_1$.
\end{lemma}
\begin{proof}
	By Chow's lemma (\Cref{thm:chow}), we may assume that $X$ is projective, basically by pulling back along our map $\mu\colon X'\to X$; getting back to $X$, one needs to project back along $\mu$.

	Explicitly, one can use \Cref{thm:bertini} to the blow-up $\op{Bl}_{\{x_0,x_1\}}X\to X$. Then $x_0$ and $x_1$ become codimen\-sion-$1$ closed subvarieties of the blow-up, so we can get them to intersect with a hypersurface (see \Cref{rem:finite-field-bertini}). So we may induct downwards.\todo{}
\end{proof}
\begin{remark}
	This statement is still true for any finite set of points.
\end{remark}
Let's now do the reduction.
\begin{proof}[Reduction to the curve case]
	It is enough to show that $\mc L|_{\{x\}\times Y\times\{z\}}$ is trivial for all $(x,z)$. Indeed, by \Cref{prop:seesaw}, one finds that $\mc L=\op{pr}_{13}^*\mc M$ for some line bundle $\mc M$ on $X\times Z$. Then the hypothesis tells us that $\mc L|_{X\times\{y_0\}\times Z}$ is trivial (replace $X$ with a curve connecting $x$ with $x_0$), so $\mc M$ will trivialize, so $\mc L$ will trivialize.
\end{proof}
\begin{warn}
	I did not really follow the below proof during class.
\end{warn}
\begin{proof}[Proof in the curve case]
	Fix $g\coloneqq g(X)$. Then we claim that there is a divisor $E\subseteq X$ of degree $g$ such that $\Gamma(X,\Omega_X(-E))=0$. Well, we are looking for global differentials on $X$ which vanish on $E$, so we choose points one at a time.

	Now, define $\mc M\coloneqq\op{pr}_1^*\OO_X(E)\otimes\mc L$, and let $W$ denote the support of $R^1\op{pr}_{23*}\mc M$, which is a closed subscheme of $Y\times Z$ by definition. Now, for all $y\in Y$, we know that $\mc L|_{X\times\{y\}\times\{z_0\}}$ is trivial, so $\mc M|_{\{x\}\times\{y\}\times\{z_0\}}\cong\OO_E$. But then
	\[H^1(X\times\{y\}\times\{z_0\},M|_{X\times\{y\}\times\{z_0\}})=H^1(X,\OO_X(E))\cong H^0(X,\Omega_X(-E)),\]
	where the last isomorphism is by Serre duality. But now $H^0(X,\Omega_X(-E))=0$ by construction of $E$, so the point is that $W\cap(Y\times\{z_0\})$ is empty.

	Now, because $Y$ is proper, we see $\op{pr}_Z(W)\subseteq Z$ is closed and avoiding $z_0$, so we can find an open $Z'\subseteq Z$ around $z_0$ such that $W\cap(Y\times Z')=0$. As such, we claim that $\mc L|_{X\times Y\times Z'}$ trivializes, which will be enough by \Cref{prop:seesaw}. Now, $R^1\op{pr}_{23*}\mc M$ is locally free of rank $1$ on $Y\times Z'$: it is enough to check that the Euler is constantly $1$, but being locally constant allows us to compute it on $z_0$, so
	\[\chi(M|_{X\times\{y\}\times\{z_0\}})=\chi(\mc M|_{X\times\{y\}\times\{z_0\}})=\chi(\OO_X(E)),\]
	and we know $\chi(\OO_X(E))=1$ by a Riemann--Roch computation.

	Being a line bundle now produces a divisor $D\subseteq X\times Y\times Z'$. Namely, on an affine open cover $\{U_i\}$ on $Y\times Z$, one has isomorphisms $\alpha_i\colon\OO_{U_i}\to\mc N|_{U_i}$, and we let $D_i$ denote the zero set of $\alpha_i(1)$ in $X\times U_i$, and we can glue these $D_i$ together. Namely, on the intersections, one can check gluing data from $\mc N$. The point is that $\OO(D)|_{X\times\{y\}\times\{z\}}\cong\mc M|_{X\times\{y\}\times\{z\}}$ for all $(y,z)\in Y\times Z$, essentially by construction.

	Quickly, we claim that $D=E\times Y\times Z$. Well, find some $p\in X$ not in the support of $E$, and we will show that $D\cap(\{p\}\times Y\times Z)$ is empty, which will imply the claim because then we will find that $D$ is the needed sum of points in $E$'s support times $Y\times Z$. So we will be able to complete the proof by restricting computing $\mc L$ on $D$ by its restriction to $X\times\{y_0\}\times\{z_0\}$, which we know to be trivial already.

	Well, to show the claim, we (sub)claim
	\[(D\cap(\{p\}\times Y\times Z))\cap((\{p\}\times Y\times\{z_0\})\cup(\{p\}\times\{y_0\}\times Z))\stackrel?=\emp.\]
	Well, $\mc L$ trivializes on $X\times\{y_0\}\times Z$ and $X\times Y\times\{z_0\}$ already, so $\mc M$ on this restriction is $\OO_X(E)$, so this intersection must then be empty.

	We now upgrade using that $Y$ is proper. The projection $\op{pr}_Z(D\cap(\{p\}\times Y\times Z))$ is a closed subset of $Z$, so $D\cap(\{p\}\times Y\times Z)$ must just be $\{p\}\times Y\times Z''$ for a codimension-$1$ subscheme $Z''\subseteq Z$. But the previous subclaim now requires everything to be empty.

	We now complete the proof. Right now we know that $\OO(D)|_{X\times\{y\}\times\{z\}}$ must be $\op{pr}_1^*\OO(E)|_{X\times\{y\}\times\{z\}}$ by the claim of the previous paragraphs. But $\OO(D)$ is just $\mc M$, so we are being told that $\mc L|_{X\times\{y\}\times\{z\}}$ is trivial, so \Cref{prop:seesaw} along with the trivialization of $\mc L|_{\{x_0\}\times Y\times Z}$ completes the argument that $\mc L$ is trivial.
\end{proof}

\end{document}