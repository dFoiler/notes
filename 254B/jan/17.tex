% !TEX root = ../notes.tex

\documentclass[../notes.tex]{subfiles}

\begin{document}

Let's get going.
\begin{warn}
	The proofs in this first chapter of the course will be somewhat sketchy. We will later go back and prove things in more generality using the machinery of algebraic geometry (instead of the theory of complex manifolds).
\end{warn}

\subsection{Course Notes}
Here are some course notes.
\begin{itemize}
	\item The professor for this course is Yunqing Tang. Her research is in arithmetic geometry. Office hours will begin next week.
	\item This course is on complex multiplication of abelian varieties.
	\item There will be homework, and it completely determines the grade. There will be (on average) biweekly homeworks, which can be found and turned in on bCourses.
	\item There is a syllabus on the bCourses: \href{https://bcourses.berkeley.edu/courses/1532318/}{\texttt{https://bcourses.berkeley.edu/courses/1532318/}}. The syllabus has many references, on abelian varieties, complex multiplication, and class field theory.
	\item There is a schedule page on the bCourses, though it does not refer to every possible reference.
	\item It is encouraged to seek out examples, such as by emailing Professor Yunqing Tang. For example, elliptic curves are important, but their theory is often significantly simpler than the general theory.
	\item Our main goal is to discuss the main theorem of complex multiplication. We will give some version of it in the first part of the class, and then we will give a second version later after a more thorough discussion of abelian varieties.
	\item Much of the language will be scheme-theoretic, so it is highly recommended having some algebraic geometry background on the level of Math 256A.
\end{itemize}

\subsection{Complex Tori}
Let's just jump on in. The most basic example of an abelian variety is an elliptic curve, so that is where we will begin.
\begin{definition}[elliptic curve]
	Fix a field $k$. Then an \textit{elliptic curve} is a pair $(E,e)$ of a smooth proper $k$-curve $E$ of genus $1$ and a marked point $e\in E(k)$.
\end{definition}
\begin{remark}
	One can replace ``proper'' with ``projective'' here without tears.
\end{remark}
\begin{example}
	Take $k\coloneqq\CC$. It turns out that an elliptic curve $(E,e)$ then makes $E(\CC)$ into a Riemann surface of genus $1$: smooth makes this a manifold, proper makes it compact, and the genus is preserved. But then $E(\CC)$ will have universal cover given by $\CC$ (in reality, we're looking at some kind of torus), and the projection map identifies $E(\CC)$ with $\CC/\Lambda$ for a lattice $\Lambda\subseteq\CC$. By translating, we may as well move the marked point $e\in E(\CC)$ to $0\in\CC/\Lambda$.
\end{example}
The above examples motivates us to look at higher-dimensional quotients, as follows.
\begin{definition}[complex torus]
	A \textit{complex torus} is a quotient of the form $V/\Lambda$ where $V$ is a finite-dimen\-sional $\CC$-vector space, and $\Lambda\subseteq V$ is a lattice of full rank.
\end{definition}
\begin{remark}
	In the sequel, it may be helpful to note that a complex vector space $V$ is just a real vector space $V$ together with an $\mathbb R$-linear map $J\colon V\to V$ such that $J^2=\id_V$. Namely, given a complex vector space $V$, we can build $J$ by the action of $i$. Conversely, given a real vector space $V$ with $J\colon V\to V$ such that $J^2=-\id_V$, we note that we have a map $\CC\to\op{End}_\RR(V)$ by $i\mapsto J$ because $\CC\cong\RR[x]/\left(x^2+1\right)$; as such, $V$ becomes a complex vector space restricting to the underlying real vector space. These constructions are inverse to each other by tracking back through that the action of $i$ is given by $J$.
\end{remark}
It turns out that a complex torus need not be an abelian variety, but one does have the following result to get projectivity from \cite[I.3, p.~33]{mumford}.
\begin{theorem} \label{thm:torus-to-ab-var}
	Fix a complex torus $X\coloneqq V/\Lambda$. Then the following are equivalent.
	\begin{listroman}
		\item $X$ can be embedded into a complex projective space.
		\item $X$ is the analytification of an algebraic $\CC$-variety.
		\item There exists a positive-definite Hermitian form $H$ on $V$ such that $H$ sends $\Lambda$ to $\ZZ$.
	\end{listroman}
\end{theorem}
\begin{proof}
	We will discuss this more later in the course.
\end{proof}
\begin{remark}
	Later on, we will understand the positive-definite Hermitian form as a polarization.
\end{remark}
Satisfying any of these equivalent conditions turns out to produce an abelian variety.
\begin{definition}[abelian variety] \label{def:ab-var-c}
	An \textit{abelian variety} is a $\CC$-variety $A$ which is a complex torus satisfying one of the equivalent conditions of \Cref{thm:torus-to-ab-var}. In practice, we will choose to define an abelian variety as a complex torus satisfying (iii).
\end{definition}
This definition is rather unsatisfying because it only works over the base field $\CC$, but it is good enough for now.
\begin{remark}
	It turns out that there is a unique algebraic structure on the variety, so there is no worry about this being vague.
\end{remark}
\Cref{thm:torus-to-ab-var} involves Hermitian forms, so we will want to get a better handle on these.
\begin{lemma} \label{lem:im-of-hermitian}
	Fix a finite-dimensional complex vector space $V$. Then there is a bijection between Hermitian forms $H$ on $V$ and skew-symmetric forms $\psi$ on the underlying real vector space of $V$ such that
	\[\psi(iv,iw)=\psi(v,w).\]
\end{lemma}
\begin{proof}
	We begin by describing our maps.
	\begin{itemize}
		\item In the forward direction, send $H\colon V\times V\to\CC$ to its imaginary part $\psi\coloneqq\im H$. Then we have a map $\psi\colon V\times V\to\RR$, and here are our checks on it.
		\begin{itemize}
			\item Skew-symmetric: note that $\psi(v,v)=\Im H(v,v)=0$ because $H(v,v)\in\RR$ because $H$ is Hermitian.
			\item Bilinear: note that $\psi(cv,w)=\Im H(cv,w)=c\Im H(v,w)=\Im H(v,cw)=\psi(v,cw)$ and
			\[\psi(v_1+v_2,w)=\Im H(v_1+v_2,w)=\Im H(v_1,w)+\Im H(v_2,w)=\psi(v_1,w)+\psi(v_2,w)\]
			and similarly $\psi(v,w_1+w_2)=\psi(v,w_1)+\psi(v,w_2)$.
			\item Note that $\psi(iv,iw)=\Im H(iv,iw)=\Im i(-i)H(v,w)=\Im H(v,w)=\psi(v,w)$.
		\end{itemize}
		\item For the backward direction, send $\psi$ to the form $H(v,w)\coloneqq\psi(iv,w)+i\psi(v,w)$. Here are our checks.
		\begin{itemize}
			\item Conjugate symmetry: note $\psi(v,w)=-\psi(v,w)$ implies that $\Im H(v,w)=-\Im H(w,v)$. Then we must show that $\Re H(v,w)=\Re H(w,v)$, or $\psi(iv,w)=\psi(iw,v)$. Well,
			\[\psi(iw,v)=-\psi(v,iw)=\psi\left(i^2v,iw\right)=\psi(iv,w)\]
			\item Bilinear: note
			\begin{align*}
				H(v_1+v_2,w) &= \psi(i(v_1+v_2),w)+i\psi(v_1+v_2,w) \\
				&= \psi(iv_1,w)+i\psi(v_1,w)+\psi(iv_2,w)+i\psi(v_2,w) \\
				&= H(v_1,w)+H(v_2,w).
			\end{align*}
			Also, for $c\in\RR$, we see that $H(cv,w)=\psi(icv,w)+i\psi(cv,w)=c(\psi(iv,w)+i\psi(v,w))=cH(v,w)$. So it remains to check that $H(iv,w)=iH(v,w)$. Well,
			\[H(iv,w)=\psi\left(i^2v,w\right)+i\psi(iv,w)=-\psi(v,w)+i\psi(iv,w)=iH(v,w).\]
		\end{itemize}
	\end{itemize}
	We now show that the constructions are inverse.
	\begin{itemize}
		\item Given $\psi$, we constructed $H_\psi$, and we see that $\Im H_\psi=\psi$ by construction.
		\item Given $H$, we set $\psi\coloneqq\Im H$. Then we must show that the constructed $H_\psi$ is equal to $H$. Note that $\Im H_\psi=\psi=\Im H$ by construction, and
		\[\Re H_\psi(v,w)=\psi(iv,w)=\Im H(iv,w)=\Im iH(v,w)=\Re H(v,w),\]
		so the result follows.
		\qedhere
	\end{itemize}
\end{proof}
\begin{remark} \label{rem:im-of-hermitian-pos-def}
	We remark that $H$ is a positive-definite Hermitian form if and only if the form $(v,w)\mapsto\Re H(v,w)$ is a positive-definite symmetric form. In terms of the above construction, this corresponds to the map $(v,w)\mapsto\psi(iv,w)$ being positive-definite; i.e., $\psi(iv,v)\ge0$ for all $v$ and equal to $0$ if and only if $v=0$.
\end{remark}
The moral of \Cref{lem:im-of-hermitian} is that we are allowed to only pay attention to the imaginary part. It is worth having a name for this.
\begin{definition}[Riemann form]
	Fix a lattice $\Lambda$ of full rank in a finite-dimensional complex vector space $V$. Then a skew-symmetric form $\psi\colon\Lambda\times\Lambda\to\ZZ$ is a \textit{Riemann form} if and only if $\psi_\RR\colon V\times V\to\RR$ defined by $\psi_\RR(x,y)\coloneqq\psi(ix,y)$ produces a symmetric positive-definite.
\end{definition}
\begin{remark}
	Quickly, we claim that $\psi_\RR$ is symmetric and positive-definite if and only if $\psi(iv,iw)=\psi(v,w)$ always and $(v,v)\mapsto\psi(iv,v)$ is positive-definite. Indeed, $\psi_\RR$ is the real part of the Hermitian form constructed in \Cref{lem:im-of-hermitian}, and we can track through symmetry in the proof and positive-definiteness from \Cref{rem:im-of-hermitian-pos-def}.
\end{remark}

\subsection{CM Fields}
We want to give some examples of what ``complex multiplication'' means. This begins with a discussion of CM fields.
\begin{lemma} \label{lem:how-to-cm}
	Fix a number field $E/\QQ$. Then the following are equivalent.
	\begin{listroman}
		\item There is a quadratic subextension $E^+\subseteq E$ such that $E^+/\QQ$ is totally real, and $E/E^+$ is totally imaginary.
		\item There exists a nontrivial field involution $c\colon E\to E$ such that $\sigma(c(\alpha))=\overline{\sigma(\alpha)}$ for any $\sigma\colon E\to\CC$ and $\alpha\in E$.
		\item There exists a unique nontrivial field involution $c\colon E\to E$ such that $\sigma(c(\alpha))=\overline{\sigma(\alpha)}$ for any $\sigma\colon E\to\CC$ and $\alpha\in E$.
		\item There exists a totally real subfield $E^+\subseteq E$ such that $E=E^+(\alpha)$ where $\alpha^2\in E^+$ is ``totally negative'' (i.e., it maps to a negative real element for every complex embedding $E^+\to\CC$).
	\end{listroman}
\end{lemma}
\begin{proof}
	We show our implications in sequence.
	\begin{itemize}
		\item We show (i) implies (iv). By completing the square in the quadratic extension $E^+/E$, we may select $\alpha\in E^+\setminus E$ such that $\alpha^2\in E^+$. Being quadratic implies that $E=E^+(\alpha)$.

		It remains to check that $\alpha$ is totally negative. Fix an embedding $\sigma\colon E\to\CC$, and let $\overline\sigma\colon E\to\CC$ be the complex conjugate embedding. Because $E$ is totally imaginary, we note $\sigma\ne\overline\sigma$, but $\sigma|_{E^+}=\overline\sigma|_{E^+}$ because $E^+$ is totally real, so we must then have $\sigma(\alpha)\ne\overline{\sigma(\alpha)}$. On the other hand, $\alpha^2\in E^+$ implies that
		\[\sigma(\alpha)^2=\overline{\sigma(\alpha)}^2\in\RR,\]
		so $\sigma(\alpha)=-\overline{\sigma(\alpha)}$. Thus, $\sigma(\alpha)$ must be imaginary, so $\sigma(\alpha)^2<0$.

		\item We show (ii) implies (i). Set $E^+\coloneqq E^c$; because $c^2=\id_E$, we see that $E/E^+$ is quadratic. To see that $E^+$ is totally real, we note that any embedding $\sigma\colon E^+\to\CC$ can be extended to $\widetilde\sigma\colon E\to\CC$. Now, for any $\alpha\in E^+$, we see that
		\[\overline{\sigma(\alpha)}=\overline{\widetilde\sigma(\alpha)}=\widetilde\sigma(c(\alpha))=\widetilde\sigma(\alpha)=\sigma(\alpha),\]
		so $\sigma(\alpha)\in\RR$. Thus, $\sigma$ actually outputs to $\RR$.

		Lastly, we must see that $E$ is totally imaginary. Suppose that $\sigma\colon E\to\CC$ is a complex embedding, and we show that the image is not contained in $\RR$. Indeed, if $\sigma(\alpha)\in\RR$, then
		\[\sigma(\alpha)=\overline{\sigma(\alpha)}=\sigma(c(\alpha)),\]
		so $\alpha\in E^+$. Thus, $\sigma(\alpha)\notin\RR$ for any $\alpha\in E\setminus E^+$.

		\item We show (ii) and (iii) are equivalent; of course (iii) implies (ii). To see that (ii) implies (iii), suppose that $c_1$ and $c_2$ are such field automorphisms $E\to E$. Then for any embedding $\sigma\colon E\to\CC$, we see that $\sigma(c_1(\alpha))=\sigma(c_2(\alpha))$ for any $\alpha\in E$, so $c_1=c_2$ follows.

		\item We show (iv) implies (ii). Define $c\in\op{Gal}(E^+/E)$ by $c(\alpha)\coloneqq-\alpha$. Then $c$ is an automorphism with $c^2=\id_E$. Also, for any embedding $\sigma\colon E\to\CC$, we know that $\sigma(a)\in\RR$ for any $a\in E^+$, and $\sigma(\alpha)^2<0$ by total negativity, so $\sigma(\alpha)$ is purely imaginary. Thus, for any $a+b\alpha\in E$, we see
		\[\sigma(c(a+b\alpha))=\sigma(a-b\alpha)=\sigma(a)-\sigma(b)\sigma(\alpha)=\overline{\sigma(a)+\sigma(b)\sigma(\alpha)}=\overline{\sigma(a+b\alpha)},\]
		as needed.
		\qedhere
	\end{itemize}
\end{proof}
\begin{remark} \label{rem:complex-conj-of-cm}
	The proof of (iv) implies (ii) has shown that if $E$ has been embedded into $\CC$ already, then $c$ is literally complex conjugation.
\end{remark}
This produces the following definition.
\begin{definition}[CM field]
	A number field $E/\QQ$ is a \textit{CM field} if and only if $E$ satisfies one of the equivalent conditions of \Cref{lem:how-to-cm}. We call the involution $c\colon E\to E$ the \textit{complex conjugation} of $E$.
\end{definition}
\begin{remark}
	The field $E$ need not be Galois.
\end{remark}
\begin{remark}
	It turns out that $E^+=E^c$ and is the maximal totally real subfield. Certainly $E^+\subseteq E$ is totally real. Conversely, suppose $F\subseteq E$ is a totally real subfield. We will show that $c$ fixes $F$, which then implies $F\subseteq E^c$. Well, for any $\alpha\in F$, we pick up any embedding $\sigma\colon E\to\CC$, and we see that
	\[\sigma(c(\alpha))=\overline{\sigma(\alpha)}=\sigma(\alpha),\]
	so $\alpha=c(\alpha)$ follows.
\end{remark}
Being CM is a fairly nice adjective.
\begin{lemma} \label{lem:comp-cm}
	Fix CM fields $E_1,\ldots,E_n\subseteq\overline\QQ$. Then the composite field $E_1\cdots E_n$ is CM.
\end{lemma}
\begin{proof}
	By induction, we may take $n=2$; define $E\coloneqq E_1E_2$ for brevity. Let $c_1\colon E_1\to E_1$ and $c_2\colon E_2\to E_2$ be the complex conjugations, which we would like to extend to a complex conjugation map $c\colon E\to E$. Well, a generic element of $E$ can be written as $\alpha=\sum_{i=1}^da_{1i}a_{2i}$ where $a_{1i}\in E_1$ and $a_{2i}\in E_2$, so we define
	\[c(\alpha)\coloneqq\sum_{i=1}^dc_1(a_{1i})c_2(a_{2i}).\]
	We ought to check that $c$ is well-defined. Suppose that $\sum_{i=1}^da_{1i}a_{2i}=\sum_{i=1}^da_{1i}'a_{2i}'$, and choose an embedding $\sigma\colon E_1E_2\to\CC$. Then $\sigma$ will restrict to embeddings $\sigma_1\colon E_1\to\CC$ and $\sigma_2\colon E_2\to\CC$, and we see that
	\[\sigma\Bigg(\sum_{i=1}^dc_1(a_{1i})c_2(a_{2i})\Bigg)=\sum_{i=1}^d\sigma_1(c_1(a_{1i}))\sigma_2(c_2(a_{2i}))=\overline{\sigma\Bigg(\sum_{i=1}^da_{1i}a_{2i}\Bigg)}\]
	and similar holds when we add primes. So the injectivity of $\sigma$ provides that $c$ is well-defined.

	Now, the above has actually automatically shown that $\sigma(c(\alpha))=\overline{\sigma(\alpha)}$ for any complex embedding $\sigma\colon E_1E_2\to\CC$ and $\alpha\in E_1E_2$. It remains to show that $c^2=\id_E$ and that $c$ is a nontrivial field homomorphism. To see that $c$ is a field homomorphism, we  note $c=\sigma^{-1}\circ\iota\circ\sigma\circ c$, where $\iota\colon\CC\to\CC$ is complex conjugation. To see that $c$ is nontrivial, we note that it extends $c_1\colon E_1\to E_1$, which is nontrivial. Lastly, to see that $c^2=\id_E$, choose $\sigma\colon E_1E_2\to\CC$, and we note that $\sigma\circ c^2=\iota^2\circ\sigma=\sigma$, so $c^2=\id_E$ is forced.
\end{proof}
\begin{corollary}
	Fix a CM field $E$. Then its Galois closure $M$ in $\overline\QQ$ is CM.
\end{corollary}
\begin{proof}
	Without loss of generality, choose an embedding $\ov\QQ\subseteq\CC$. Let $\sigma_1,\ldots,\sigma_n\colon E\to\CC$ denote the complex embeddings of $E$, and we note that the Galois closure of $E$ is the composite
	\[\sigma_1(E)\cdots\sigma_n(E).\]
	By \Cref{lem:comp-cm}, it thus suffices to show that $\sigma(E)$ is a CM field for any embedding $\sigma\colon E\to\CC$.
	
	Well, let $c\colon E\to E$ denote the complex conjugation of $E$; we note that this agrees with the complex conjugation in $\CC$ by \Cref{rem:complex-conj-of-cm}.
	%For this, we note that the literal complex conjugation embedding $\iota\colon E\to\CC$ must equal $c$: using \Cref{lem:how-to-cm} to write $E=E^+(\alpha)$, we note that $\iota$ and $c$ both fix $E^+$ and send $\iota(\alpha)=-\alpha=c(\alpha)$ by the proof, so $\iota=c$ follows.
	Then to show that $\sigma(E)$ is a CM field, we note that we have a complex conjugation $c_\sigma\colon\sigma(E)\to\sigma(E)$ by
	\[c_\sigma(\sigma(\alpha))\coloneqq\sigma(c(\alpha)).\]
	This is also $\overline{\sigma(\alpha)}$, which establishes that $c_\sigma$ is a nontrivial field involution. (Being nontrivial follows because $E$ is totally imaginary.) Lastly, for any complex embedding $\tau\colon\sigma(E)\to\CC$, we must show that $\tau(c_\sigma(\sigma(\alpha)))=\overline{\tau(\sigma(\alpha))}$. However, we simply note that $(\tau\circ\sigma)\colon E\to\CC$ is another embedding, and
	\[\tau(c_\sigma(\sigma(\alpha)))=(\tau\circ\sigma)(c(\alpha))=\overline{\tau(\sigma(\alpha))},\]
	as desired.
\end{proof}
Having CM fields allow us to define CM types.
\begin{definition}[CM type]
	Fix a CM field $E$ with complex conjugation $c$. Then a \textit{CM type} on $E$ is a subset $\Phi\subseteq\op{Hom}(E,\CC)$ such that
	\[\op{Hom}(E,\CC)=\Phi\sqcup c\Phi.\]
	We call the pair $(E,\Phi)$ a \textit{CM pair}.
\end{definition}
\begin{remark}
	When $E/\QQ$ is imaginary quadratic (which is what happens for elliptic curves), one does not really have a choice in CM type. But for higher degrees, which exist for higher-dimensional abelian varieties, there is indeed structure we want to keep track of.
\end{remark}
This allows us to write down an abelian variety.
\begin{exe}
	Fix a CM pair $(E,\Phi)$, and set $n\coloneqq\frac12[E:\QQ]$. For a lattice $\mf a\subseteq E$, set $\Lambda\coloneqq\mf a$, and use $\Phi$ to produce an embedding $\mf a\to\CC^\Phi$ by $\alpha\mapsto(\sigma(\alpha))_{\sigma\in\Phi}$. Then $\CC^\Phi/\mf a$ is an abelian variety.
\end{exe}
\begin{proof}
	Quickly, we show that $\mf a$ is a lattice of full rank in $\CC^\Phi$. Fix an integral basis $\{\alpha_1,\ldots,\alpha_{2n}\}$ of $\mf a$. Now, by viewing $\CC^\Phi$ as $\RR^{2n}$ by taking real and imaginary parts, we see that the determinant of the map $\OO_E\otimes_\ZZ\RR\to\RR^{2n}$ is, up to sign and a factor of $2$, equal to
	\[\det\begin{bmatrix}
		\sigma_1(\alpha_1) & \cdots & \sigma_{1}(\alpha_{2n}) \\
		\vdots & \ddots & \vdots \\
		\sigma_{2n}(\alpha_1) & \cdots & \sigma_{2n}(\alpha_{2n})
	\end{bmatrix},\]
	which is the discriminant of the $\alpha_\bullet$, which is nonzero. (Here, we enumerate $\Phi=\{\sigma_1,\ldots,\sigma_n\}$ and then $\sigma_{n+i}\coloneqq\overline{\sigma_i}$ for $i\in\{1,\ldots,n\}$.) This is sufficient because then $\OO_E$ is a lattice of rank $2n$ in $\RR^{2n}$. So we do indeed have a complex torus.

	To provide the abelian variety structure, it suffices to provide the $\psi$ of \Cref{lem:im-of-hermitian}. We will choose $\xi\in\mf a$ judiciously and then set
	\[\psi(x,y)\coloneqq\op{Tr}_{E/\QQ}(\xi x c(y)).\]
	For concreteness, we go ahead and embed $E$ into $\CC$ so that $c$ is literally complex conjugation by \Cref{rem:complex-conj-of-cm}. As such, we will write $c(y)$ as $\overline y$. Now, to choose $\xi$, we note that a weak approximation argument grants $\xi_0\in\mf a$ such that $\op{Im}\sigma(\xi_0)>0$ for each $\sigma\in\Phi$; such a thing exists by a strong approximation argument. Then set $\xi\coloneqq\xi_0-\ov{\xi_0}$ so that $\ov\xi=-\xi$ while still having
	\[\Im\sigma(\xi)=\Im\sigma(\xi_0)-\Im\sigma(\ov{\xi_0})=\Im\sigma(\xi_0)+\Im\sigma(\xi_0)>0.\]
	We are now ready to conduct our checks.
	\begin{itemize}
		\item Bilinear: the map $(x,y)\mapsto(\xi x,\overline y)$ is $\ZZ$-linear in both coordinates, and the map $(x,y)\mapsto\op{Tr}_{E/\QQ}(xy)$ is bilinear in both coordinates, so the composite $(x,y)\mapsto\psi(x,y)$ is also bilinear in both coordinates.
		\item Skew-symmetric: we must show that $\psi(x,x)=0$ for any $x\in\OO_E$. Now, it will be helpful to expand
		\[\psi(x,x)=\op{Tr}_{E/\QQ}(\xi x\ov x)=\sum_{i=1}^n(\sigma_i(\xi x\ov x)+\overline{\sigma_i}(\xi x\ov x)).\]
		Now, we note that $\overline{\sigma_i}(\xi x\ov x)=\overline{\sigma_i(\xi x\ov x)}=\sigma_i(\ov\xi\cdot x\ov x)=-\sigma_i(\xi x\ov x)$, so each term of this sum vanishes.
		\item Upon tensoring with $\RR$ to produce $\psi_\RR$, we must show that $\psi_\RR(ix,iy)=\psi_\RR(x,y)$. By scaling $x$ and $y$, we may assume that $x,y\in\OO_E$. We also note that $\xi$ is purely imaginary, so by scaling $ix$ and $iy$, it suffices to show that
		\[\psi(x,y)=\frac1{\left|\xi\right|^2}\psi(\xi x,\xi y).\]
		However, this is immediate from the linearity of the trace.
		\item Positive-definite: we must show that $\psi_\RR(ix,x)\ge0$ for each $x$ and is zero if and only if $x=0$. We may as well check this for $x\in\OO_E$, and a direct expansion produces
		\[\psi(ix,x)=\sum_{i=1}^n(\sigma_i(\xi ix\ov x)+\overline{\sigma_i}(\xi ix\ov x)),\]
		where one makes sense of $i$ by some kind of $\RR$-linearity. Expanding somewhat naively, we see
		\[\psi(ix,x)=\sum_{i=1}^n(\sigma_i(i\xi)+\sigma_i(-i\ov x))\sigma_i(x\ov x)=\sum_{i=1}^n2\sigma_i(i\xi)\sigma_i(x\ov x).\]
		Now, each term of the sum is nonnegative because $\Im\sigma_i(\xi)>0$ already, so the total sum can only vanish provided that all the individual terms vanish. For example, this requires that $\sigma_i(x\ov x)=0$ for all $i$, so $x\ov x=0$, so $x=0$ or $\ov x=0$, so $x=0$ is forced.
		\qedhere
	\end{itemize}
\end{proof}
\begin{remark}
	In general, one can replace $E$ by a CM algebra and replace $\OO_E$ by certain fractional ideals. This will turn out to provide all isomorphism classes of abelian varieties with CM.
\end{remark}
Next class we will define an abelian variety when not over $\CC$.

\end{document}