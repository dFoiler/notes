% !TEX root = ../notes.tex

\documentclass[../notes.tex]{subfiles}

\begin{document}

Let's get going.

\subsection{Course Notes}
Here are some course notes.
\begin{itemize}
	\item The professor for this course is Yunqing Tang. Her research is in arithmetic geometry. Office hours will begin next week.
	\item This course is on complex multiplication of abelian varieties.
	\item There will be homework, and it completely determines the grade. There will be (on average) biweekly homeworks, which can be found and turned in on bCourses.
	\item There is a syllabus on the bCourses: \href{https://bcourses.berkeley.edu/courses/1532318/}{\texttt{https://bcourses.berkeley.edu/courses/1532318/}}. The syllabus has many references, on abelian varieties, complex multiplication, and class field theory.
	\item There is a schedule page on the bCourses, though it does not refer to every possible reference.
	\item It is encouraged to seek out examples, such as by emailing Professor Yunqing Tang. For example, elliptic curves are important, but their theory is often significantly simpler than the general theory.
	\item Our main goal is to discuss the main theorem of complex multiplication. We will give some version of it in the first part of the class, and then we will give a second version later after a more thorough discussion of abelian varieties.
	\item Much of the language will be scheme-theoretic, so it is highly recommended having some algebraic geometry background on the level of Math 256A.
\end{itemize}

\subsection{Complex Tori}
Let's just jump on in. The most basic example of an abelian variety is an elliptic curve, so that is where we will begin.
\begin{definition}[elliptic curve]
	Fix a field $k$. Then an \textit{elliptic curve} is a pair $(E,e)$ of a smooth proper $k$-curve $E$ of genus $1$ and a marked point $e\in E(k)$.
\end{definition}
\begin{remark}
	One can replace ``proper'' with ``projective'' here without tears.
\end{remark}
\begin{example}
	Take $k\coloneqq\CC$. It turns out that an elliptic curve $(E,e)$ then makes $E(\CC)$ into a Riemann surface of genus $1$: smooth makes this a manifold, proper makes it compact, and the genus is preserved. But then $E(\CC)$ will have universal cover given by $\CC$ (in reality, we're looking at some kind of torus), and the projection map identifies $E(\CC)$ with $\CC/\Lambda$ for a lattice $\Lambda\subseteq\CC$. By translating, we may as well move the marked point $e\in E(\CC)$ to $0\in\CC/\Lambda$.
\end{example}
The above examples motivates us to look at higher-dimensional quotients, as follows.
\begin{definition}[complex torus]
	A \textit{complex torus} is a quotient of the form $V/\Lambda$ where $V$ is a finite-dimen\-sional $\CC$-vector space, and $\Lambda\subseteq V$ is a lattice of full rank.
\end{definition}
It turns out that a complex torus need not be an abelian variety, but one does have the following result to get projectivity from \cite[I.3, p.~33]{mumford}.
\begin{theorem} \label{thm:torus-to-ab-var}
	Fix a complex torus $X\coloneqq V/\Lambda$. Then the following are equivalent.
	\begin{listroman}
		\item $X$ can be embedded into a complex projective space.
		\item $X$ is the analytification of an algebraic $\CC$-variety.
		\item There exists a positive-definite Hermitian form $H$ on $V$ such that $H$ sends $\Lambda$ to $\ZZ$.
	\end{listroman}
\end{theorem}
\begin{proof}
	We will prove this later in the course.
\end{proof}
\begin{remark}
	Later on, we will understand the positive-definite Hermitian form as a polarization.
\end{remark}
Satisfying any of these equivalent conditions turns out to produce an abelian variety.
\begin{definition}[abelian variety]
	An \textit{abelian variety} is a $\CC$-variety $A$ which is a complex torus satisfying one of the equivalent conditions of \Cref{thm:torus-to-ab-var}.
\end{definition}
This definition is rather unsatisfying because it only works over the base field $\CC$, but it is good enough for now.
\begin{remark}
	It turns out that there is a unique algebraic structure on the variety, so there is no worry about this being vague.
\end{remark}
\Cref{thm:torus-to-ab-var} involves Hermitian forms, so we will want to get a better handle on these.
\begin{lemma} \label{lem:im-of-hermitian}
	Fix a finite-dimensional complex vector space $V$. Then there is a bijection between Hermitian forms $H$ on $V$ and skew-symmetric forms $\psi$ on the underlying real vector space of $V$ such that
	\[\psi(iv,iw)=\psi(v,w).\]
\end{lemma}
\begin{proof}
	In the forward direction, send $H$ to its imaginary part. For the backward direction, send $\psi$ to the form $H(v,w)\coloneqq\psi(iv,w)+i\psi(v,w)$.
\end{proof}
The moral of \Cref{lem:im-of-hermitian} is that we are allowed to only pay attention to the imaginary part. It is worth having a name for this.
\begin{definition}[Riemann form]
	Fix a lattice $\Lambda$ of full rank in a finite-dimensional complex vector space $V$. Then a skew-symmetric form $\psi\colon\Lambda\times\Lambda\to\ZZ$ is a \textit{Riemann form} if and only if $\psi_\RR\colon V\times V\to\RR$ produces a positive-definite Hermitian form via the construction of \Cref{lem:im-of-hermitian}.
\end{definition}

\subsection{CM Fields}
We want to give some examples of what ``complex multiplication'' means. This begins with a discussion of CM fields.
\begin{lemma} \label{lem:how-to-cm}
	Fix a number field $E/\QQ$. Then the following are equivalent.
	\begin{listroman}
		\item There is a quadratic subextension $E^+\subseteq E$ such that $E^+/\QQ$ is totally real, and $E/E^+$ is totally imaginary.
		\item There exists a nontrivial field involution $c\colon E\to E$ such that any $c$ commutes with each complex embedding $\sigma\colon E\to\CC$.
		\item There exists a unique nontrivial field involution $c\colon E\to E$ such that any $c$ commutes with each complex embedding $\sigma\colon E\to\CC$.
		\item There exists a totally real subfield $E^+\subseteq E$ such that $E=E^+(\alpha)$ where $\alpha^2\in E^+$ is ``totally negative'' (i.e., it maps to a negative real element for every complex embedding $E^+\to\CC$).
	\end{listroman}
\end{lemma}
\begin{proof}
	Omitted; see \cite[Section~I.1]{milne-cm}. We will discuss this later in the course.
\end{proof}
\begin{remark}
	It turns out that $E^+=E^c$ and is the maximal totally real subfield.
\end{remark}
This produces the following definition.
\begin{definition}[CM field]
	A number field $E/\QQ$ is a \textit{CM field} if and only if $E$ satisfies one of the equivalent conditions of \Cref{lem:how-to-cm}. We call the involution $c\colon E\to E$ the \textit{complex conjugation} of $E$.
\end{definition}
\begin{remark}
	The field $E$ need not be Galois.
\end{remark}
Being CM is a fairly nice adjective.
\begin{lemma}
	Fix CM fields $E_1,\ldots,E_n\subseteq\overline\QQ$. Then the composite field $E_1\cdots E_n$ is CM.
\end{lemma}
\begin{corollary}
	Fix a CM field $E$. Then its Galois closure $M$ in $\overline\QQ$ is CM.
\end{corollary}
Having CM fields allow us to define CM types.
\begin{definition}[CM type]
	Fix a CM field $E$ with complex conjugation $c$. Then a \textit{CM type} on $E$ is a subset $\Phi\subseteq\op{Hom}(E,\CC)$ such that
	\[\op{Hom}(E,\CC)=\Phi\sqcup c\Phi.\]
	We call the pair $(E,\Phi)$ a \textit{CM pair}.
\end{definition}
\begin{remark}
	When $E/\QQ$ is imaginary quadratic (which is what happens for elliptic curves), one does not really have a choice in CM type. But for higher degrees, which exist for higher-dimensional abelian varieties, there is indeed structure we want to keep track of.
\end{remark}
This allows us to write down an abelian variety.
\begin{example}
	Fix a CM pair $(E,\Phi)$, and set $n\coloneqq\frac12[E:\QQ]$. Then set $\Lambda\coloneqq\OO_E$, and use $\Phi$ to produce an embedding $\OO_E\to\CC^n$. Then $\CC^n/\OO_E$ is a complex torus, and it will turn out to be an abelian variety.
	
	Indeed, it suffices to provide the $\psi$ of \Cref{lem:im-of-hermitian}. To see this, we note that a weak approximation argument grants $\xi\in\OO_E$ such that $\op{Im}\varphi(\xi)>0$ for each $\varphi\in\Phi$. Then $\xi$ allows us to define $\psi\colon\OO_E\times\OO_E\to\ZZ$ by
	\[\psi(x,y)\coloneqq\op{Tr}_{E/\QQ}(\xi x c(y)).\]
	We will verify that this works on the homework. As an example, take $n=1$ so that $E/\QQ$ is imaginary quadratic and already embedded into $\CC$. Then write $\OO_E=\ZZ\oplus\tau\ZZ$ for some $\tau\in\CC$, and we note
	\[\op{Tr}_{E/\QQ}(\xi xc(y))=\xi x\overline y+\overline\xi\overline xy=\xi(x\ov y-\ov xy),\]
	from which one can see that this is positive-definite and Hermitian.
\end{example}
\begin{remark}
	In general, one can replace $E$ by a CM algebra and replace $\OO_E$ by certain fractional ideals. This will turn out to provide all isomorphism classes of abelian varieties with CM.
\end{remark}
Next class we will actually define an abelian variety.

\end{document}