% !TEX root = ../notes.tex

\documentclass[../notes.tex]{subfiles}

\begin{document}

Here we go. Today we will define an abelian variety in general, but we will stay focused on the analytic theory.

\subsection{Defining Abelian Varieties}
Abelian varieties are special kinds of group objects.
\begin{definition}[group scheme]
	Fix a base scheme $S$. Then a \textit{group $S$-scheme} is a group object $G$ in the category $\mathrm{Sch}_S$ of $S$-schemes. In other words, there exist $S$-morphisms $m\colon G\times_S G\to G$ (for multiplication) and $i\colon G\to G$ (for inversion) and $e\colon S\to G$ (for identity) making the following diagrams commute.
	\begin{itemize}
		\item Associativity:
		% https://q.uiver.app/#q=WzAsNCxbMCwwLCJHXFx0aW1lcyBHXFx0aW1lcyBHIl0sWzEsMCwiR1xcdGltZXMgRyJdLFswLDEsIkdcXHRpbWVzIEciXSxbMSwxLCJHIl0sWzAsMSwibVxcdGltZXNcXGlkX0ciXSxbMCwyLCJ7XFxpZF9HfVxcdGltZXMgbSIsMl0sWzIsMywibSJdLFsxLDMsIm0iXV0=&macro_url=https%3A%2F%2Fraw.githubusercontent.com%2FdFoiler%2Fnotes%2Fmaster%2Fnir.tex
		\[\begin{tikzcd}
			{G\times_S G\times_S G} & {G\times_S G} \\
			{G\times_S G} & G
			\arrow["{m\times\id_G}", from=1-1, to=1-2]
			\arrow["{{\id_G}\times m}"', from=1-1, to=2-1]
			\arrow["m", from=2-1, to=2-2]
			\arrow["m", from=1-2, to=2-2]
		\end{tikzcd}\]
		\item Identity: 
		% https://q.uiver.app/#q=WzAsNixbMCwxLCJHIl0sWzEsMCwiR1xcdGltZXNfU1MiXSxbMSwyLCJTXFx0aW1lc19TRyJdLFsyLDAsIkdcXHRpbWVzX1NHIl0sWzIsMiwiR1xcdGltZXNfU0ciXSxbMywxLCJHIl0sWzAsMV0sWzAsMl0sWzEsMywie1xcaWRfR31cXHRpbWVzIGUiXSxbMiw0LCJlXFx0aW1lc1xcaWRfRyJdLFszLDUsIiBtIl0sWzQsNSwibSJdLFswLDUsIiIsMSx7ImxldmVsIjoyLCJzdHlsZSI6eyJoZWFkIjp7Im5hbWUiOiJub25lIn19fV1d&macro_url=https%3A%2F%2Fraw.githubusercontent.com%2FdFoiler%2Fnotes%2Fmaster%2Fnir.tex
		\[\begin{tikzcd}
			& {G\times_SS} & {G\times_SG} \\
			G &&& G \\
			& {S\times_SG} & {G\times_SG}
			\arrow[from=2-1, to=1-2]
			\arrow[from=2-1, to=3-2]
			\arrow["{{\id_G}\times e}", from=1-2, to=1-3]
			\arrow["{e\times\id_G}", from=3-2, to=3-3]
			\arrow["{ m}", from=1-3, to=2-4]
			\arrow["m", from=3-3, to=2-4]
			\arrow[Rightarrow, no head, from=2-1, to=2-4]
		\end{tikzcd}\]
		\item Inversion: % https://q.uiver.app/#q=WzAsNSxbMSwwLCJHXFx0aW1lcyBHIl0sWzEsMiwiR1xcdGltZXMgRyJdLFswLDEsIkciXSxbMSwxLCJTIl0sWzIsMSwiRyJdLFsyLDAsIntcXGlkX0d9XFx0aW1lcyBpIl0sWzIsMSwiaVxcdGltZXN7XFxpZF9HfSIsMl0sWzIsM10sWzMsNCwiZSJdLFswLDQsIm0iXSxbMSw0LCJtIiwyXV0=&macro_url=https%3A%2F%2Fraw.githubusercontent.com%2FdFoiler%2Fnotes%2Fmaster%2Fnir.tex
		\[\begin{tikzcd}
			& {G\times_S G} \\
			G & S & G \\
			& {G\times_S G}
			\arrow["{{\id_G}\times i}", from=2-1, to=1-2]
			\arrow["{i\times{\id_G}}"', from=2-1, to=3-2]
			\arrow[from=2-1, to=2-2]
			\arrow["e", from=2-2, to=2-3]
			\arrow["m", from=1-2, to=2-3]
			\arrow["m"', from=3-2, to=2-3]
		\end{tikzcd}\]
	\end{itemize}
\end{definition}
\begin{remark}
	Equality of morphisms of $k$-varieties can be checked on geometric points, so we could just check the above commutativity on $G(\ov k)$.
\end{remark}
In particular, we want to be a variety.
\begin{definition}[group variety]
	Fix a base field $k$. Then a \textit{group $k$-variety} is a group scheme which is also a $k$-variety (i.e., reduced and separated).
\end{definition}
\begin{remark}
	By way of analogy, we also note that a Lie group is a group object in the category $\mathrm{Man}$ of smooth manifolds.
\end{remark}
Abelian varieties are special kinds of group varieties.
\begin{definition}[abelian variety]
	Fix a field $k$. Then an \textit{abelian $k$-variety} is a group $k$-variety which is smooth, connected, and proper.
\end{definition}
Here, smoothness is something like requiring that we are a manifold, and proper is something like requiring that we are projective. (It turns out that the conditions imply that $A$ is projective, though this is not obvious.)
\begin{remark}
	One can even replace ``$k$-variety'' with ``$k$-scheme'' because being smooth over a scheme implies being regular, which implies reduced.
\end{remark}
\begin{remark}
	It turns out that being geometrically integral is equivalent to being connected, by some argument involving the connected component.
\end{remark}
\begin{remark} \label{rem:ab-var-is-ab}
	It turns out that being proper implies that the group law on $A$ is abelian, which we have notably not included in the hypotheses.
\end{remark}
While we're here, we go ahead and define abelian schemes; these will be desirable because we may (perhaps) want to define varieties via equations in a ring which is not a field (like $\ZZ$) and then reduce to a field (like $\FF_p$) later.
\begin{definition}[abelian scheme]
	Fix a base scheme $S$. An \textit{abelian $S$-scheme} is a group $S$-scheme $A$ which is proper and smooth over $S$ such that the structure map $\pi\colon A\to S$ has connected geometric fibers. (This last condition means that any geometric point $\ov s\to S$ makes $A_{\ov s}$ connected.)
\end{definition}
\begin{remark}
	Here, smoothness can be verified by something like a Jacobian criterion, analogous to smoothness for embedded manifolds.
\end{remark}
\begin{remark}
	Notably, by the hypotheses, the geometric fibers $A_{\ov s}$ are abelian varieties.
\end{remark}

\subsection{Working over \texorpdfstring{$\CC$}{C}}
We now return to working over $k=\CC$. We quickly compare with \Cref{def:ab-var-c}: being an abelian variety over $\CC$ as defined in the previous subsection implies that $A(\CC)$ is a smooth complex analytic manifold which is connected and compact, simply by reading off the adjectives. Now, this means that $A(\CC)$ is connected and compact, so we have a connected compact complex Lie group $A(\CC)$, which one can show is always of the form $V/\Lambda$ where $V$ is a finite-dimensional $\CC$-vector space and $\Lambda\subseteq\CC$ is a lattice of full rank, as sketched in \Cref{rem:ab-var-is-torus}. From there, being algebraic does imply one of the equivalent conditions of \Cref{thm:torus-to-ab-var}, and the converse is similar.

Anyway, for a taste of the analytic theory, we show the following for $k=\CC$.
\begin{proposition}
	Fix an abelian $k$-variety $A$. Then the group law for $A$ is commutative.
\end{proposition}
\begin{proof}[Sketch over $k=\CC$]
	Consider the tangent space at the identity $e\in A$, which we will label $T_eA$. Now, for $e\in A(\CC)$, we have a holomorphic map $c_x\colon A(\CC)\to A(\CC)$ given by conjugation $y\mapsto xyx^{-1}$, and then this induces a linear map $dc_x\colon T_eA\to T_eA$. This construction $x\mapsto dc_x$ produces a holomorphic map
	\[A(\CC)\to\op{GL}(T_eA),\]
	but $A(\CC)$ is compact, and $\op{GL}(T_eA)$ is an open submanifold, so the above map must be constant. Noting that $de_x=\id_{T_eA}$, we see that actually $dc_x=\id_{T_eA}$, which implies that $c_x$ must be the identity, so the group law is commutative.
\end{proof}
\begin{remark} \label{rem:ab-var-is-torus}
	Continuing, there is a group homomorphism $\exp\colon T_eA\to A(\CC)$, which one can show is a covering space map. So $A(\CC)$ must then be a compact quotient of $T_eA$, and actually it is a quotient by something discrete, meaning that $A(\CC)\cong V/\Lambda$ as above.
\end{remark}
Here are some nice corollaries of realizing abelian varieties as complex tori.
\begin{corollary} \label{cor:mul-by-n-iso}
	Fix an abelian $\CC$-variety $A$ of dimension $g$. For any positive integer $n$, the multiplication-by-$n$ map $[n]\colon A(\CC)\to A(\CC)$ is a surjective group homomorphism, and its kernel is isomorphic to $(\ZZ/n\ZZ)^{2g}$.
\end{corollary}
\begin{proof}
	Note $[n]$ is a group homomorphism because $A(\CC)$ is abelian. For the other claims, write $A=V/\Lambda$ for $V$ a $g$-dimensional $\CC$-vector space.
\end{proof}
\begin{corollary}
	Fix an abelian $\CC$-variety $A$ of dimension $g$. Then
	\[\pi_1(A(\CC))\cong H_1(A(\CC),\ZZ)\cong\Lambda\cong\ZZ^{2g}.\]
\end{corollary}
\begin{proof}
	Again, write $A=V/\Lambda$ for $V$ a $g$-dimensional $\CC$-vector space. Then $V$ is the universal covering space for $V$, so $\pi_1(A(\CC))\cong\Lambda$, from which the rest of the isomorphisms follow quickly.
\end{proof}

\subsection{Isogenies}
While we're here, we define isogenies, which are ``squishy'' isomorphisms.
\begin{definition}[isogenies]
	Fix abelian $k$-varieties $A$ and $B$. A $k$-morphism $f\colon A\to B$ is a surjective homomorphism with finite kernel.
\end{definition}
\begin{example}
	For any positive integer $n$, the map $[n]\colon A\to A$ is an isogeny. We will prove this in general later, but over $\CC$, it follows from \Cref{cor:mul-by-n-iso}.
\end{example}
We would like to describe isogenies (over $\CC$) from the perspective of the complex tori. So we pick up the following proposition.
\begin{proposition}
	Fix complex tori $V/\Lambda$ and $V'/\Lambda'$. Then holomorphic maps $V/\Lambda\to V'/\Lambda'$ fixing $0$ are in bijection with $\CC$-linear maps $V\to V'$ sending $\Lambda\to\Lambda'$.
\end{proposition}
\begin{proof}[Sketch]
	The backward map is clear. For the forward map, note that a holomorphic map $V/\Lambda\to V'/\Lambda'$, lift to the universal cover to produce a holomorphic map $V\to V'$. To show that this map is linear, one shows that the derivative is constant, which follows from the compactness.
\end{proof}
\begin{remark}
	Basically, we can see that being an isogeny means that the underlying linear map will be a surjective linear map with finite kernel; in particular, $\dim_\CC V=\dim_\CC V'$. This motivates us thinking about isogenies as ``squishy'' isomorphisms.
\end{remark}

\end{document}