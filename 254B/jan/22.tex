% !TEX root = ../notes.tex

\documentclass[../notes.tex]{subfiles}

\begin{document}

Today we will talk more about the analytic theory.

\subsection{More on Isogenies}
We begin by picking up a piece of language.
\begin{definition}[isogenous]
	Fix abelian $k$-varieties $A$ and $B$. We say that $A$ and $B$ are \textit{isogenous}, written $A\sim B$, if and only if there is an isogeny $A\to B$.
\end{definition}
\begin{remark}
	It turns out that having an isogeny is an equivalence relation (in particular, it is symmetric!), so we will not care about the direction of being ``isogenous.''
\end{remark}
We can decompose abelian varieties based on their isogeny class.
\begin{theorem}[Poincar\'e reducibility] \label{thm:poincare-red}
	Fix an abelian $k$-variety $A$, and let $B\subseteq A$ be an abelian subvariety. Then there exists another abelian subvariety $B'\subseteq A$ such that $B\cap B'$ is a finite scheme, and
	\[B+B'=\{b+b':b\in B,b'\in B'\}\]
	is equal to $A$. In other words, the canonical map $B\times_k B'\to A$ given by summing is an isogeny.
\end{theorem}
\begin{proof}
	This is \cite[p.~160]{mumford} or \cite[Theorem~2.12]{milne-cm}. Read the proof for homework, and on the homework, we are asked for an example of $B\subseteq A$ such that $B\cap B'$ is nontrivial for any $B'\subseteq A$ satisfying the conclusion.
\end{proof}
In light of this decomposition, we can take the following definition.
\begin{definition}[simple]
	An abelian $k$-variety $A$ is \textit{$k$-simple} if and only if all abelian subvarieties of $A$ are either $\{e\}$ or $A$.
\end{definition}
\begin{remark}
	It is possible to have an abelian variety be simple over $k$ but not over $\ov k$.
\end{remark}
\begin{corollary}
	Fix an abelian $k$-variety $A$. Then there are simple abelian $k$-varieties $A_1,\ldots,A_n$ such that
	\[A\sim\prod_{i=1}^nA_i.\]
\end{corollary}
\begin{proof}
	Apply \Cref{thm:poincare-red} inductively. The process ends because we need to have
	\[\dim A=\sum_{i=1}^n\dim A_i,\]
	and all dimensions are required to be finite positive integers.
\end{proof}
For uniqueness of this decomposition, we will want to talk about morphisms between simple abelian varieties. It will be helpful to have some language for this.
\begin{definition}
	Fix abelian $k$-varieties $A$ and $B$. Then $\op{Hom}_k(A,B)$ is the abelian group of homomorphisms $A\to B$, and $\op{Hom}_k^0(A,B)\coloneqq\op{Hom}_k(A,B)\otimes_\ZZ\QQ$. Similarly, we define
	\[\op{End}_k(A)\coloneqq\op{Hom}_k(A,A)\qquad\text{and}\qquad\op{End}^0_k(A)\coloneqq\op{Hom}_k^0(A,A).\]
\end{definition}
\begin{remark}
	One can show that $\op{Hom}^0_k(A,B)$ and $\op{End}^0_k(A)$ only depend on the isogeny class of $A$ and $B$.
\end{remark}
\begin{corollary} \label{cor:end-a}
	Fix an abelian $k$-variety $A$. Then $\op{End}^0_k(A)$ is a division $\QQ$-algebra.
\end{corollary}
\begin{proof}
	The main point is the existence of inverses. Fix a nonzero endomorphism $\varphi\colon A\to A$. Then $\varphi\ne0$ means that $\im\varphi$ is nontrivial abelian subvariety of $A$, so $\im\varphi=A$. Similarly, the connected component of $\ker\varphi$ is an abelian variety, but it cannot be all of $A$, so it must be $\{e\}$. So $\varphi$ has finite kernel, killed by say $[n]$, allowing us to build an ``inverse'' endomorphism.
\end{proof}
\begin{remark}
	Using a decomposition of $A$ into simple abelian varieties, one can add in \Cref{cor:end-a} as an ingredient to show that $\op{End}^0_k(A)$ is a product of matrix algebras of division algebras. In particular, $\op{End}^0_k(A)$ is semisimple.
\end{remark}

\subsection{Complex Multiplication}
We are now ready to define complex multiplication for abelian varieties.
\begin{definition}[complex multiplication]
	Fix an abelian $k$-variety $A$. Then $A$ has \textit{complex multiplication} (or is \textit{CM}) if and only if there is a CM algebra $E$ (i.e., $E$ is a finite product of CM fields) such that $[E:\QQ]=2\dim A$, and there is an embedding $E\into\op{End}^0_k(A)$.
\end{definition}
Namely, $A$ has ``multiplication'' by some CM fields.
\begin{remark}
	It will turn out that this definition holds true for all abelian varieties over finite fields.
\end{remark}
\begin{remark}
	Suppose $A$ is a simple abelian $k$-variety. Then $A$ being CM is equivalent to $\op{End}^0_k(A)$ being isomorphic to a CM field of degree $2\dim A$. Certainly this condition is implied by being CM. In the other direction, over $\CC$, one sees that $\op{End}^0(A)$ acts faithfully on $H_1(A(\CC),\QQ)$ by \Cref{prop:map-of-complex-torus}. Thus, $\op{End}^0_k(A)$ is a division algebra of degree dividing $2\dim A$.
	
	Now, letting the center of $D\coloneqq\op{End}^0_k(A)$ by $F$, it turns out that the largest field contained in $D$ has degree (over $\QQ$) is $[D:F]^{1/2}[F:\QQ]$. To get this to be at most $2\dim A$, we must have $F=D$ by a degree argument. (See \cite[Section~I.1]{milne-cm} for the required facts on semisimple algebras.)
\end{remark}
\begin{remark}
	One can remove the requirement of being over $\CC$ in the above argument by working with the ``Tate module'' $H^1_{\text{\'et}}(A,\QQ_\ell)$ for $\ell\ne\op{char}k$ instead of $H^1(A(\CC),\QQ)$. Concretely, the Tate module is
	\[T_\ell A\coloneqq\limit_n A\left[\ell^n\right].\]
	We will work more with Tate modules later in this course.
\end{remark}
Here are some examples.
\begin{example}
	Fix an imaginary quadratic field $E$. Then $\CC/\OO_E$ is a CM abelian $\CC$-variety with complex multiplication by $E$; in particular, \Cref{prop:map-of-complex-torus} tells us that the endomorphism ring is $\OO_E$, so we get $E$ upon taking $-\otimes_\ZZ\QQ$. If $E_1$ and $E_2$ are distinct quadratic imaginary fields, then taking products reveals that $(\CC/\OO_{E_1})\times(\CC/\OO_{E_2})$ has complex multiplication by $E_1\times E_2$.
\end{example}
\begin{example}
	Fix an imaginary quadratic field $E$. Then $(\CC/\OO_E)^2$ has endomorphism algebra given by
	\[\op{End}^0_\CC\left((\CC/\OO_E)^2\right)\cong M_2(E).\]
	Here, there is a lot of choice in the CM algebra embedding into $M_2(E)$. Notably, for any $D\in\ZZ$, we see
	\[\begin{bmatrix}
		0 & D \\
		1 & 0
	\end{bmatrix}^2=DI,\]
	so $\QQ(\sqrt D)$ embeds into $M_2(\QQ)$ without tears.
\end{example}
\begin{remark}
	One might be interested in understanding what abelian varieties look like in general, which leads to the notion of a moduli space. It turns out that abelian varieties with complex multiplication forms an interesting subset of the full moduli space of abelian varieties.
\end{remark}
Let's now specialize more directly to $\CC$. We pick up the following definition.
\begin{definition}[CM type]
	Fix a CM field $E$, and let $(A,i)$ be an abelian variety with complex multiplication by $E$. Then $E$ acts faithfully on $H_1(A(\CC),\QQ)$. Hodge theory tells us that we can decompose
	\[H^1(A(\CC),\CC)=H^{01}\oplus H^{10},\]
	where $H^{10}=\ov{H^{01}}$; here $H^{01}=H^0(A(\CC),\Omega^1)$ is the space of global sections $1$-forms on $A(\CC)$. Dualizing, Poincar\'e duality tells us
	\[H_1(A(\CC),\CC)=\op{Lie}A(\CC)\oplus\ov{\op{Lie}A(\CC)},\]
	and in fact $E$ acts on $\op{Lie}A(\CC)$. Decomposing $\op{Lie}A(\CC)$ as an $E$-representation as $\bigoplus_{\varphi\in\Phi}\CC_\varphi$ where $\Phi\subseteq\op{Hom}(E,\CC)$. Then $\Phi$ is the CM type.
\end{definition}

\end{document}