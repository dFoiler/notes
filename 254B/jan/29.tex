% !TEX root = ../notes.tex

\documentclass[../notes.tex]{subfiles}

\begin{document}

Homework has been posted. It looks hard. We have two weeks to do it.

\subsection{The Rosati Involution}
Here is our definition.
\begin{definition}[Rosati involution]
	Fix an abelian $\CC$-variety $A=V/\Lambda$, and let $\psi\colon\Lambda\times\Lambda\to\ZZ$ be a Riemann form on $A$. Then we define the \textit{Rosati involution} $(-)^\dag\colon\op{End}^0(A)\to\op{End}^0(A)$ as follows: for each $\alpha\in\op{End}^0(A)$, we define $\alpha^\dag$ such that
	\[\psi(\alpha x,y)=\psi(x,\alpha^\dag y)\]
	for all $x,y\in\Lambda$.
\end{definition}
\begin{remark}
	Later on, we will view $(-)^\dag$ from the lens of dual abelian varieties, as follows. Note that $\psi$ provides an identification of $\Lambda$ with its dual lattice $\Lambda^\lor$, and then $\alpha^\dag$ is defined so that the following diagram commutes.
	% https://q.uiver.app/#q=WzAsNCxbMCwwLCIoXFxMYW1iZGFcXG90aW1lc19cXFpaXFxRUSkiXSxbMSwwLCIoXFxMYW1iZGFeXFxsb3JcXG90aW1lc19cXFpaXFxRUSkiXSxbMSwxLCIoXFxMYW1iZGFeXFxsb3JcXG90aW1lc19cXFpaXFxRUSkiXSxbMCwxLCIoXFxMYW1iZGFcXG90aW1lc19cXFpaXFxRUSkiXSxbMCwzLCJcXGFscGhhXlxcZGFnIiwyXSxbMSwyLCJcXGFscGhhXlxcbG9yIl0sWzAsMSwiXFxwc2kiXSxbMywyLCJcXHBzaSJdXQ==&macro_url=https%3A%2F%2Fraw.githubusercontent.com%2FdFoiler%2Fnotes%2Fmaster%2Fnir.tex
	\[\begin{tikzcd}
		{(\Lambda\otimes_\ZZ\QQ)} & {(\Lambda^\lor\otimes_\ZZ\QQ)} \\
		{(\Lambda\otimes_\ZZ\QQ)} & {(\Lambda^\lor\otimes_\ZZ\QQ)}
		\arrow["{\alpha^\dag}"', from=1-1, to=2-1]
		\arrow["{\alpha^\lor}", from=1-2, to=2-2]
		\arrow["\psi", from=1-1, to=1-2]
		\arrow["\psi", from=2-1, to=2-2]
	\end{tikzcd}\]
	Namely, this shows that $\alpha^\dag$ exists and is unique. Later on, we will have an analogous definition where $\Lambda$s above are replaced with $A$ itself (and $\Lambda^\lor$ is replaced with the dual abelian variety $A^\lor$).
\end{remark}
\begin{remark}
	One can check that $(\alpha^\dag)^\dag=\alpha$ via the above diagram.
\end{remark}
Here is the main result.
\begin{proposition}
	Fix an abelian $\CC$-variety $A$ with Riemann form $\psi\colon\Lambda\times\Lambda\to\ZZ$. The Rosati involution is positive: for all nonzero $\alpha\in\op{End}^0(A)$, we have
	\[\op{Tr}(\alpha^\dag\alpha)>0.\]
	Here, the trace map is defined by the trace in $\op{End}^0(A)\subseteq\op{End}(H_1(A,\QQ))$.
\end{proposition}
\begin{proof}
	Fix a $\QQ$-basis $B$ of $H_1(A,\QQ)=\Lambda\otimes_\ZZ\QQ$. Then, by definition, we see that
	\[\op{Tr}(\alpha^\dag\alpha)=\sum_{x\in B}\psi_\RR(ix,\alpha^\dag\alpha x)=\sum_{x\in B}\psi_\RR(\alpha ix,\alpha x),\]
	which is a sum of positive numbers because $\psi_\RR$ is positive-definite by definition.
\end{proof}
\begin{remark}
	There is a unique positive involution on any CM algebra $E$, namely its complex conjugation $c$. Thus, if $A$ is a simple abelian variety with complex multiplication by $E=\op{End}^0(A)$, we must have $\alpha^\dag=c(\alpha)$, so
	\[\psi(\alpha x,y)=\psi(x,c(\alpha)y).\]
	In general, if $A$ is not simple, then one can show that there is a CM algebra $E\subseteq\op{End}^0(A)$ of the correct degree and preserved by $(-)^\dag$.
\end{remark}
We now note that we have the following lemma.
\begin{lemma}
	Fix an abelian variety $A=V/\Lambda$ with complex multiplication by $E\subseteq\op{End}^0(A)$ fixed by the Rosati involution. Further, fix a non-degenerate skew-symmetric $E$-linear form $\psi\colon(\Lambda\otimes_\ZZ\QQ)^2\to\QQ$ such that $\psi(\alpha x,y)=\psi(x,c(\alpha y))$ for all $\alpha\in E$. Then
	\[\psi(x,y)=\op{Tr}_{E/\QQ}(\xi xc(y))\]
	for all $x,y\in E$, were $\xi\in E$ and $c(\xi)-\xi$.
\end{lemma}
\begin{proof}
	Do some linear algebra.
\end{proof}
And we may now give a classification of (polarized) abelian varieties.
\begin{theorem}
	Fix a CM algebra $E$. We parameterize polarized abelian varieties with complex multiplication by $E$, up to isomorphism.
\end{theorem}
\begin{proof}
	Here, an isomorphism $(A,i,\psi)\cong(A',i',\psi')$ is an isomorphism $f\colon A\to A'$ such that the diagram
	% https://q.uiver.app/#q=WzAsNCxbMCwwLCJBIl0sWzEsMCwiQSciXSxbMCwxLCJBIl0sWzEsMSwiQSciXSxbMCwxLCJmIl0sWzIsMywiZiJdLFswLDIsImkoXFxhbHBoYSkiLDJdLFsxLDMsImknKFxcYWxwaGEpIl1d&macro_url=https%3A%2F%2Fraw.githubusercontent.com%2FdFoiler%2Fnotes%2Fmaster%2Fnir.tex
	\[\begin{tikzcd}
		A & {A'} \\
		A & {A'}
		\arrow["f", from=1-1, to=1-2]
		\arrow["f", from=2-1, to=2-2]
		\arrow["{i(\alpha)}"', from=1-1, to=2-1]
		\arrow["{i'(\alpha)}", from=1-2, to=2-2]
	\end{tikzcd}\]
	commutes for every $\alpha\in E$, and the diagram
	% https://q.uiver.app/#q=WzAsNCxbMCwwLCJIXzEoQSxcXFpaKVxcdGltZXMgSF8xKEEsXFxaWikiXSxbMSwwLCJcXFpaIl0sWzEsMSwiXFxaWiJdLFswLDEsIkhfMShBJyxcXFpaKVxcdGltZXMgSF8xKEEnLFxcWlopIl0sWzAsMSwiXFxwc2kiXSxbMywyLCJcXHBzaSciXSxbMCwzLCJmIiwyXSxbMSwyLCIiLDEseyJsZXZlbCI6Miwic3R5bGUiOnsiaGVhZCI6eyJuYW1lIjoibm9uZSJ9fX1dXQ==&macro_url=https%3A%2F%2Fraw.githubusercontent.com%2FdFoiler%2Fnotes%2Fmaster%2Fnir.tex
	\[\begin{tikzcd}
		{H_1(A,\ZZ)\times H_1(A,\ZZ)} & \ZZ \\
		{H_1(A',\ZZ)\times H_1(A',\ZZ)} & \ZZ
		\arrow["\psi", from=1-1, to=1-2]
		\arrow["{\psi'}", from=2-1, to=2-2]
		\arrow["f"', from=1-1, to=2-1]
		\arrow[Rightarrow, no head, from=1-2, to=2-2]
	\end{tikzcd}\]
	also commutes.

	We now describe our constructions. Given $(A,i,\psi)$, we build $(E,\Phi,\mf a)$ as before, where $\mf a$ is constructor by taking the $\op{End}(A)$-orbit of a chosen vector $v\in H_1(A,\QQ)$, and then we pick $\xi\in E^\times$ with $c(\xi)=-\xi$ from the above lemma. Notably, the choice of $v$ is only defined up to multiplication by $E^\times$: replacing $v$ with $a^{-1}v$ will adjust $\mf a$ to $a\mf a$, and we can see that $\xi\mapsto\xi/(a(c(a)))$.
\end{proof}
% \begin{remark}
% 	On the homework, one might need to use the fact that Jacobians are principally polarized in order to compute the CM type.
% \end{remark}

\subsection{The Field of Definition: Abelian Varieties}
We will now show that abelian varieties with complex multiplication are defined over $\ov\QQ$.
\begin{remark}
	One can show that $\op{End}^0(A)$ is still defined over the reflex field. The same thing holds for Hodge cycles (from the perspective of the Shimura variety).
\end{remark}
Anyway, our result will follow from the following, by taking $k=\ov\QQ$.
\begin{proposition} \label{prop:base-change-av}
	Fix an algebraically closed field $k\subseteq\CC$. Then consider the base-change functor $(-)_\CC$ taking abelian varieties defined over $k$ to abelian varieties defined over $\CC$. Then $(-)_\CC$ is fully faithful and contains all CM abelian varieties in its (essential) image.
\end{proposition}
\begin{proof}
	The key observation is that we have an injection $A(k)\subseteq A(\CC)$ (because $\CC/k$ is a field extension), and we have an isomorphism $A(k)_{\mathrm{tors}}=A(\CC)_{\mathrm{tors}}$. Indeed, for any nonzero integer $n$, we see that $A(k)[n]=A[n](k)$, but $A[n](k)$ just consists of the solutions in $k$ to some set of polynomial equations. So the solutions over $k$ and over $\CC$ will be the same because both these fields are algebraically closed.

	Anyway, here are our checks. Fix abelian $k$-varieties $A$ and $A'$.
	\begin{itemize}
		\item Faithful: fix $f,g\colon A\to A'$ such that $f_\CC=g_\CC$. Then we see that $f_\CC$ and $g_\CC$ are the same over $A(\CC)_{\mathrm{tors}}$, so $f$ and $g$ are the same over $A(k)_{\mathrm{tors}}$. Thus, it is enough to check that $A(k)_{\mathrm{tors}}$ is Zariski dense in $A(k)$. Well, the Zariski closure $B\coloneqq\ov{A(k)_{\mathrm{tors}}}$ is a smooth proper group subvariety of $A(k)$: smoothness is from $\op{char}k=0$ and $k=\ov k$, properness is because it is a closed subscheme of $A$, and being reduced follows by construction because we took the Zariski closure. So $B^\circ$ is an abelian subvariety with $B^\circ(k)[p]=A(k)[p]$ for all primes $p>\#\pi_0(B)$: having an element of order $p$ outside $B^\circ$ would force there to be at least $p$ connected components (one for each multiple of this element), so this can only happen for $p<\#\pi_0(B)$. Thus, we see $\dim A=\dim B^\circ$, so we must have $B^\circ=A$ because $A$ is irreducible.

		\item Full: we use some descent theory. Fix a map $f\colon A_\CC\to A'_\CC$, which we must show is the base-change of a map $A\to A'$. Quickly, note that $k=\CC^{\op{Gal}(\CC/k)}$ by some infinite Galois theory (or alternatively, a more direct argument via Zorn's lemma). Notably, for $\tau\in\op{Gal}(\CC/k)$, there is a map $\tau(f)\colon A_\CC\to A'_\CC$ given by applying $\tau$ to the coefficients of $f$ viewed affine-locally; on $\CC$-points, one sees that $\tau(f)$ is the composite $\left(\tau\circ f\circ\tau^{-1}\right)\colon A(\CC)\to A'(\CC)$.

		Now, some descent theory shows that $f$ is defined over $k$ if and only if $f=\tau(f)$ for all $\tau\in\op{Gal}(\CC/k)$; approximately speaking, one can just see that the coordinates of $f$ must all in fact be defined over $k$. Well, the point is that $\tau|_k=\id_k$, so $f$ and $\tau(f)$ agree on $A(k)$ and hence on $A(\CC)$.

		\item Essential image: we will do this next class. Fix a CM abelian $\CC$-variety $A$. By a spreading out argument that we will give next class (see \Cref{prop:spread-out-av}), there is a finitely generated $k$-algebra $R\subseteq\CC$ such that we have an abelian scheme $\mc A$ over $S\coloneqq\Spec R$ specializing to $A$.

		Now, $\OO\coloneqq\op{End}_\CC(A)$ is finitely generated over $\ZZ$, so ensuring that these endomorphisms are all defined over $R$ (perhaps by localizing more), we may assume that $\OO\subseteq\op{End}_R(\mc A)$. In particular, $\mc A$ has complex multiplication. Choosing a geometric point of $R$ given by $\Spec k\to R$ and pulling back $\mc A$ makes an abelian variety $B$ over $k$.

		Quickly, note that the CM type of $B$ is just the $\Phi\subseteq\op{Hom}(E,\CC)$ appearing in the $E$-representation $\op{Lie}B$, which is simply $\op{Lie}A$. So $B_\CC$ is at least isogenous with $A$, so there is a finite kernel $G_\CC\subseteq B_\CC$ such that $B_\CC/G_\CC\subseteq A$. But $G$ is a finite group scheme, so it must be fully contained $B[n]$ for some $n$, so we can realize the quotient group scheme $B/G$ back over $k$, and $B/G$ is the required scheme.\footnote{Perhaps one should check that the quotient $B/G$ makes sense as an abelian variety, but it all works out, so we won't bother.}
		\qedhere
	\end{itemize}
\end{proof}
\begin{remark}
	Fix abelian varieties $A$ and $B$ defined over $\ov\QQ$. Then \Cref{prop:base-change-av} also tells us that a homomorphism $\varphi\colon A_\CC\to B_\CC$ is defined over $\ov\QQ$.
\end{remark}

\end{document}