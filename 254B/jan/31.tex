% !TEX root = ../notes.tex

\documentclass[../notes.tex]{subfiles}

\begin{document}

We began class by finishing an argument of last class, so I have edited the argument there.

\subsection{Spreading Out Abelian Varieties}
We quickly discuss a result on spreading out abelian varieties.
\begin{proposition} \label{prop:spread-out-ft}
	Fix a $K$-variety $A$ of finite type, and let $k\subseteq K$ be the prime field. Then there exists a finitely generated $k$-algebra $R$ and an $R$-scheme $\mc A$ such that $\mc A_K=A$.
\end{proposition}
\begin{proof}
	This follows from what it means to be finite type.
\end{proof}
\begin{proposition} \label{prop:spread-out-av}
	Fix an abelian $K$-variety $A$ of finite type, and let $k\subseteq K$ be the prime field. Then there exists a finitely generated $k$-algebra $R$ and an abelian $R$-scheme $\mc A$ such that $\mc A_K=A$.
\end{proposition}
\begin{proof}
	We get some $R$ and $\mc A$ by \Cref{prop:spread-out-ft}. We now spread out one condition on $\mc A$ at a time.
	\begin{itemize}
		\item Writing out equations, we may assume that the group law is well-defined by adding in enough denominators, essentially localizing $R$.
		\item For projectivity, we note that $A$ is projective, and we can basically use the same equations to realize $\mc A$ as a closed subscheme of projective $R$-space.\todo{Get proper directly?}
		\item For smoothness, we pass to the smooth locus of $\Spec R$, which is nonempty because we are already smooth on the generic fiber. (Notably, we are smooth on, say, the identity section.)
		\item Lastly, for geometrically connected, we note that having a connected fiber is equivalent to the map $\OO_{\Spec R}\to\pi_*\OO_{\mc A}$ being an isomorphism on stalks. (Namely, we are asking for the local rings to fail to be products of $R$.) This is an open condition, so we may again shrink $\Spec R$ enough to accommodate.
	\end{itemize}
	For a reference, Milne has an article on abelian varieties, where this argument is Remark~20.9.
\end{proof}

\subsection{The Field of Definition: Endomorphisms}
Quickly, we note that we can define a CM type as a collection $\Phi\subseteq\op{Hom}(E,\ov\QQ)$ because $E$ is finite \'etale over $\QQ$ anyway. Notably, CM types of abelian varieties also still make sense because an abelian $\QQ$-variety $A$ will have its Lie algebra $\op{Lie}A$ (now defined as the Zariski tangent space) continues to have the needed $E$-action, and we can decompose this is a representation into a $\ov\QQ$-vector space.

Anyway, we now define the reflex field.
\begin{definition}[reflex field]
	Fix a CM type $(E,\Phi)$. Then the \textit{reflex field} is the subfield $E^*\subseteq\ov\QQ$ fixed by
	\[\{\sigma\in\op{Gal}(\ov\QQ/\QQ):\sigma\Phi=\Phi\},\]
	where $\Phi$ is viewed as a subset of $\op{Hom}(E,\CC)$.
\end{definition}
\begin{remark}
	If $E$ is a field, then $E^*$ is contained in the Galois closure of $E$ (in $\ov\QQ$).
\end{remark}
\begin{lemma}[{\cite[Proposition~1.16, 1.18]{milne-cm}}] \label{lem:how-to-reflex}
	Fix a CM type $(E,\Phi)$.
	\begin{listalph}
		\item $E^*$ is generated by the elements
		\[\sum_{\varphi\in\Phi}\varphi(\alpha),\]
		where $\alpha\in E$.
		\item $E^*$ a CM field.
		\item If $(E,\Phi)=\prod_{i=1}^m(E_i,\Phi_i)$, then $E^*=E_1^*\cdots E_m^*$.
		\item If $(E',\Phi')$ is an extension of $(E,\Phi)$, then $(E')^*=E^*$.
	\end{listalph}
\end{lemma}
\begin{proof}
	Omitted. One does a little Galois theory to achieve the result.
\end{proof}
\begin{example}
	If $(E,\Phi)$ is a primitive CM type, then $E=E^*$.
\end{example}
And now we can provide our definition field for endomorphisms.
\begin{proposition}
	Fix an abelian $k$-variety, where $k\subseteq\CC$. Further, suppose $A_{\ov k}$ is a CM abelian variety with CM type $(E,\Phi)$.
	\begin{listalph}
		\item If $E\subseteq\op{End}^0_k(A)$, then $E^*\subseteq k$.
		\item If $E^*\subseteq k$, and $A_{\ov k}$ is simple, then $E\subseteq\op{End}^0_k(A)$.
	\end{listalph}
\end{proposition}
\begin{proof}
	We prove one part at a time.
	\begin{listalph}
		\item We use (a) of \Cref{lem:how-to-reflex}. Quickly, we note that\todo{Coherence of Lie?}
		\[\op{Lie}A\otimes_k\CC=\op{Lie}A_\CC=\bigoplus_{\varphi\in\Phi}\CC_\varphi.\]
		Thus, for each $\alpha\in E$, we see that the trace of $\alpha$ acting on $\op{Lie}A$ is $\sum_{\varphi\in\Phi}\varphi(\alpha)$, but being defined over $k$ requires that these endomorphisms have trace living in $k$. So the result follows.
		\item Being simple enforces $E=\op{End}^0_{\ov k}(A_{\ov k})$. Now, $\op{Gal}(\ov k/k)$ acts on $\op{End}^0_{\ov k}(A_{\ov k})$, so notably we want it to act trivially on $E\subseteq\op{End}^0_k(A)$ by some descent argument. Now, for each $\sigma\in\op{Gal}(\ov k/k)$, we produce the following commutative diagram.
		% https://q.uiver.app/#q=WzAsNCxbMCwwLCJcXG9we0xpZX1BX3tcXG92IGt9Il0sWzEsMCwiXFxvcHtMaWV9QV97XFxvdiBrfSJdLFswLDEsIlxcZGlzcGxheXN0eWxlXFxiaWdvcGx1c197XFx2YXJwaGlcXGluXFxQaGl9XFxvdiBrX1xcdmFycGhpIl0sWzEsMSwiXFxkaXNwbGF5c3R5bGVcXGJpZ29wbHVzX3tcXHZhcnBoaVxcaW5cXFBoaX1cXG92IGtfXFx2YXJwaGkiXSxbMCwxLCJcXHNpZ21hIl0sWzIsM10sWzAsMiwiIiwxLHsibGV2ZWwiOjIsInN0eWxlIjp7ImhlYWQiOnsibmFtZSI6Im5vbmUifX19XSxbMSwzLCIiLDEseyJsZXZlbCI6Miwic3R5bGUiOnsiaGVhZCI6eyJuYW1lIjoibm9uZSJ9fX1dXQ==&macro_url=https%3A%2F%2Fraw.githubusercontent.com%2FdFoiler%2Fnotes%2Fmaster%2Fnir.tex
		\[\begin{tikzcd}
			{\op{Lie}A_{\ov k}} & {\op{Lie}A_{\ov k}} \\
			{\displaystyle\bigoplus_{\varphi\in\Phi}\ov k_\varphi} & {\displaystyle\bigoplus_{\varphi\in\Phi}\ov k_\varphi}
			\arrow["\sigma", from=1-1, to=1-2]
			\arrow[from=2-1, to=2-2]
			\arrow[Rightarrow, no head, from=1-1, to=2-1]
			\arrow[Rightarrow, no head, from=1-2, to=2-2]
		\end{tikzcd}\]
		In particular, $\sigma$ carries the CM type to the CM type by isomorphism, so we can carry the isomorphism characterization over $\CC$ back to $\ov k$ and then argue that $\sigma$ ought to be an isomorphism and in particular fix $E$, which is what we wanted.\todo{What?}
		\qedhere
	\end{listalph}
\end{proof}
\begin{remark}
	This tells us that having CM makes our endomorphisms defined over $\ov\QQ$.
\end{remark}

\end{document}