% !TEX root = ../notes.tex

\documentclass[../notes.tex]{subfiles}

\begin{document}

\section{January 27}

Today, we continue talking around group cohomology.

\subsection{Tate Cohomology}
It will be convenient to connect group cohomology and group cohomology. Take $G$ to be a finite group. Fix some projective resolution $P_\bullet$ of $\ZZ$. Then we have exact sequences
\[\cdots P_2\otimes M\to P_1\otimes M\to P_0\otimes M\to M_G\to0\]
and
\[0\to M^G\to\op{Hom}(P_0,M)\to\op{Hom}(P_1,M)\to\op{Hom}(P_2,M)\to\cdots.\]
But with $G$ finite, we have a norm map $N_G\colon M_G\to M^G$, so we can splice these together to give one long sequence
\[\cdots P_2\otimes M\to P_1\otimes M\to P_0\otimes M\to\op{Hom}_\ZZ(P_0,M)\to\op{Hom}_\ZZ(P_1,M)\to\op{Hom}_\ZZ(P_2,M)\to\cdots,\]
where the map $P_0\otimes M\to\op{Hom}_\ZZ(P_0,M)$ is given by $P_0\otimes M\to M_G\to M^G\to\op{Hom}_\ZZ(P_0,M)$. We now define Tate cohomology is the cohomology of this complex, where degree-$0$ is at $\op{Hom}_\ZZ(P_0,M)$. Explicitly, we have the following.
\begin{definition}[Tate cohomology]
	Fix a finite group $G$. Given a $G$-module $M$, we define the \textit{Tate cohomology} as follows, for some $i\in\ZZ$.
	\[\widehat H^i(G,M)\coloneqq\begin{cases}
		H^i(G,M) & \text{if }i\ge1, \\
		H_{-i-1}(G,M) & \text{if }i\le-2, \\
		\ker N_G & \text{if }i=-1, \\
		M^G/N_G(M_G) & \text{if }i=0,
	\end{cases}\]
	where $N_G$ is the norm map $N_G\colon M_G\to M^G$.
\end{definition}
Let's see the computations at $i=-1$ and $i=0$ more explicitly.
\begin{itemize}
	\item At $i=-1$, we are computing
	\[\frac{\ker(P_0\otimes M\to M_G\to M^G)}{\im(P_1\otimes M\to P_0\otimes M)}.\]
	However, the image $P_1\otimes M\to P_0\otimes M$ is exactly the kernel of the surjection $P_0\otimes M\onto M_G$, so we are just computing the kernel along $M_G\to M^G$. Indeed, letting $I$ denote the image of $P_1\otimes M\to P_0\otimes M$, we get a morphism of exact sequences as follows.
	% https://q.uiver.app/?q=WzAsMTAsWzAsMCwiMCJdLFsxLDAsIkkiXSxbMiwwLCJQXzBcXG90aW1lcyBNIl0sWzMsMCwiTV9HIl0sWzQsMCwiMCJdLFswLDEsIjAiXSxbMSwxLCIwIl0sWzIsMSwiTV5HIl0sWzMsMSwiTV5HIl0sWzQsMSwiMCJdLFszLDgsIk5fRyJdLFswLDFdLFsxLDJdLFsyLDNdLFszLDRdLFs4LDldLFs3LDgsIiIsMCx7ImxldmVsIjoyLCJzdHlsZSI6eyJoZWFkIjp7Im5hbWUiOiJub25lIn19fV0sWzIsN10sWzUsNl0sWzYsN10sWzEsNl1d&macro_url=https%3A%2F%2Fraw.githubusercontent.com%2FdFoiler%2Fnotes%2Fmaster%2Fnir.tex
	\[\begin{tikzcd}
		0 & I & {P_0\otimes M} & {M_G} & 0 \\
		0 & 0 & {M^G} & {M^G} & 0
		\arrow["{N_G}", from=1-4, to=2-4]
		\arrow[from=1-1, to=1-2]
		\arrow[from=1-2, to=1-3]
		\arrow[from=1-3, to=1-4]
		\arrow[from=1-4, to=1-5]
		\arrow[from=2-4, to=2-5]
		\arrow[Rightarrow, no head, from=2-3, to=2-4]
		\arrow[from=1-3, to=2-3]
		\arrow[from=2-1, to=2-2]
		\arrow[from=2-2, to=2-3]
		\arrow[from=1-2, to=2-2]
	\end{tikzcd}\]
	Taking kernels, the snake lemma grants us an exact sequence
	\[0\to I\to\ker(P_0\otimes M\to M^G)\to\ker(M_G\to M^G)\to0,\]
	so the claim follows.
	\item At $i=0$, the computation is similar.
\end{itemize}
\begin{remark}
	We can now see how norms might be important in the future.
\end{remark}

\subsection{Cohomology of Cyclic Groups}
In this subsection, let $G=\langle\sigma\rangle$ be a cyclic group of order $n$. We saw in \Cref{ex:cyclic-group-module} that
\[\ZZ[G]=\frac{\ZZ[x]}{\left(x^n-1\right)},\]
so for example $\ZZ[G]$ is commutative. In our case, we can right down a particularly nice (augmented) free resolution of $\ZZ$ as
\[\cdots\to\ZZ[G]\stackrel T\to\ZZ[G]\stackrel N\to\ZZ[G]\stackrel T\to\ZZ[G]\to\ZZ\to0,\]
where $\ZZ[G]\to\ZZ$ is the usual augmentation map and $T\coloneqq(\sigma-1)$ and $N\coloneqq N_G$. Indeed, let's see that this is exact.
\begin{itemize}
	\item Note $\ZZ[G]\to\ZZ$ is of course surjective, so we are exact at $\ZZ$.
	\item Next, we see that the kernel of the map $\ZZ[G]\to\ZZ$ consists of the terms of degree $0$, which are $\ZZ$-generated by elements of the form $\left(\sigma^i-\sigma^j\right)$ for indices $i$ and $j$, but this means that we are $\ZZ[G]$-generated by $(\sigma-1)$.
	\item Continuing, the kernel of the map $T\colon\ZZ[G]\to\ZZ[G]$ is given by the elements of the form $\sum_{i=0}^{n-1}a_i\sigma^i$ which when multiplied by $T$ vanish. Explicitly, we see
	\[T\Bigg(\sum_{i=0}^{n-1}a_i\sigma^i\Bigg)=\sum_{i=0}^{n-1}\left(a_{i-1}-a_i\right)\sigma^i,\]
	where indices are taken$\pmod n$. Thus, this vanishes if and only if $a_i$ is constant, so we see that we are in the kernel if and only if we take the form
	\[\sum_{i=0}^{n-1}a\sigma^i=aN_G\]
	for some $a\in\ZZ[G]$. So the kernel here is indeed the image of the map $N\colon\ZZ[G]\to\ZZ[G]$.
	\item Lastly, we can compute the kernel of the map $N\colon\ZZ[G]\to\ZZ[G]$ as the image of the map $T\colon\ZZ[G]\to\ZZ[G]$. We omit this computation.
\end{itemize}
The point is that we can compute group homology via the sequence
\[\cdots\to M\stackrel T\to M\stackrel N\to M\stackrel T\to M,\]
and we can compute the group cohomology via the sequence
\[M\stackrel T\to M\stackrel N\to M\to\cdots.\]
Splicing these together gives us Tate cohomology, which works properly because the map $M_G\to M^G$ is precisely the norm. In particular, we get the following nice result.
\begin{proposition} \label{prop:cyclic-cohomology}
	Let $G=\langle\sigma\rangle$ be a cyclic group of order $n$. For any $G$-module $M$, the groups $\widehat H^i(G,M)$ are $2$-periodic in $i\in\ZZ$.
\end{proposition}
\begin{remark}
	Let's take a moment to figure out where we want to go. Fix a cyclic extension $L/K$ of number fields, where $G$ is the Galois group. For example, we wanted a statement like ``if $a\in K^\times$ is a norm in $K_v$ for each $v$, then $a$ is a norm in $K$.'' This conclusion on $a$ means we want $a$ to vanish in
	\[\frac{K^\times}{\op N^L_K\left(L^\times\right)}=\frac{\left(L^\times\right)^G}{N_G\left(L^\times\right)}=\widehat H^0\left(G,L^\times\right).\]
	Combining with our place data, we wanted some sort of statement like
	\[\widehat H^0\left(G,L^\times\right)\to\prod_v\widehat H^0\left(G_v,L_v^\times\right)\]
	to be true. Roughly speaking, this will reduce to some kind of cohomology on the id\'eles.
\end{remark}

\end{document}