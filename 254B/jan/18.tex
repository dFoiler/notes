% !TEX root = ../notes.tex

\documentclass[../notes.tex]{subfiles}

\begin{document}

\section{January 18}

Here we go.

\subsection{House-Keeping}
This is a second semester of algebraic number theory, but we are not really learning algebraic number theory. Instead, we will focus on rational points on varieties. Some notes.
\begin{itemize}
	\item There is a \href{https://bcourses.berkeley.edu/courses/1521007}{\texttt{bCourses}}, which has the syllabus.
	\item Ideally, we will require a graduate-level first course in algebraic number theory. Notably, we will not assume class field theory. We will also require algebraic geometry, at the level of chapter II of \cite{hartshorne}. Roughly speaking, the first half of the course will focus on algebraic number theory, and the second half will certainly use scheme theory.

	It might be helpful to know about cohomology in advance. We will need group cohomology to begin and more general derived functors later.
	\item Homework will be assigned about every two weeks. Don't stress too much about it. However, there will be no homework drops.
	\item There will be a term paper, about 10 pages. The idea is to pick a topic you like and then talk about it.
	\item Grades will be fine as long as you don't completely vanish.
	\item If you are sick, do not come to class.
\end{itemize}

\subsection{Course Overview}
Here are the topics for the class.

\subsubsection{Quadratic Forms}
We will begin with quadratic forms, which are essentially genus-$0$ curves. Explicitly, we are asking the following question.
\begin{ques} \label{ques:quad-form}
	Fix a field $K$ and a quadratic form $Q\in K[x_0,\ldots,x_n]$, which is a homogeneous polynomial of degree $2$; we are interested if $Q$ has nontrivial zeroes. In other words, we want to know if the projective variety $V(Q)\subseteq\PP^n_K$ has a $K$-point.
\end{ques}
\begin{example}
	Set $K=\QQ$ and $Q=x_0^2+x_1^2+x_2^2$. Then $Q$ has no nontrivial zeroes. Indeed, it has no nontrivial zeroes over $\RR$, and $\QQ\subseteq\RR$.
\end{example}
\begin{remark}
	We are describing these quadratic forms as ``genus-$0$ curves'' because the variety $V(Q)$ is isomorphic to $\PP^1_{\overline K}$ over $\overline K$.
\end{remark}
We will approach \Cref{ques:quad-form} from the perspective of the local-to-global principle. Indeed, we will show the following.
\begin{theorem} \label{thm:local-global-quad-form}
	Let $Q$ be a quadratic form over a number field $K$. Then $V(Q)$ has a $K$-point if and only if $V(Q)$ has a $K_v$ point for all places $v$ of $K$.
\end{theorem}
The above result \Cref{thm:local-global-quad-form} is very special to quadratic forms, and the analogous statement fails for, say, elliptic curves.

The reason we are interested in quadratic forms is that these computations lead naturally to class field theory.
\begin{example}
	Fix a number field $K$, and let $Q=x_0^2-ax_1^2$ be a quadratic form, where $a\in K^\times$. Roughly speaking, \Cref{thm:local-global-quad-form} now asserts that $a\in K$ is a square if and only if $a$ is a square in each localization $K_v$, which is tied to the Hasse norm theorem.
\end{example}
Here are some references.
\begin{itemize}
	\item Serre's \cite{course-arithmetic} is good, though Serre avoids class field theory by focusing on $K=\QQ$. We will not want to avoid these ideas, however, because we want to see a need for cohomology.
	\item Milne's \cite{milne-cft} is good, though we will of course not do all of it.
	\item Lam also has a book \cite{lam-quad-forms} on quadratic forms.
\end{itemize}
References for this portion of the course include 

\subsubsection{Elliptic Curves}
After discussing genus-$0$ curves, we will say something about elliptic curves. The goal is to prove the following result, which is the Mordell--Weil theorem.
\begin{theorem}
	Let $E$ be an elliptic curve over a number field $K$. Then $E(K)$ is a finitely generated abelian group.
\end{theorem}
Here are some references.
\begin{itemize}
	\item Silverman's \cite{silverman} is the standard resource, but it avoids algebraic geometry.
\end{itemize}
We might also spend a lecture saying words about higher-dimensional abelian varieties, but it is a lot harder.

\subsubsection{Brauer--Manin Obstructions}
These refer to special obstructions to the local-to-global principle, as seen in \Cref{thm:local-global-quad-form}. Poonen has a reasonable text on this. All of this is already potentially too much, so we will stop here.

\subsection{Quadratic Forms}
Let's do some math. For most of our discussion here, we fix $K$ to be a field with $\op{char}K\ne2$.
\begin{definition}[quadratic form]
	Fix a field $K$ with $\op{char}K\ne2$. Then a \textit{quadratic form} $Q$ on a finite-dimensional $K$-vector space $V$ is a map $Q\colon V\to K$ satisfying the following conditions.
	\begin{itemize}
		\item Quadratic: $Q(av)=a^2Q(v)$ for all $a\in K$ and $v\in V$.
		\item Bilinear: the function $B\colon V^2\to K$ defined by $B(v,w)\coloneqq\frac12(Q(v+w)-Q(v)-Q(w))$ is $K$-bilinear. Note $B$ is symmetric automatically.
	\end{itemize}
\end{definition}
\begin{remark}
	One can view the quadratic form $Q$ as cutting out a projective variety in $\PP V$.
\end{remark}
\begin{remark}
	Given a quadratic form $Q$ on $V$ giving the bilinear form $B$, we note
	\[B(v,v)=\frac12(Q(2v)-2Q(v))=Q(v),\]
	so we can recover the quadratic form from the bilinear form. This establishes a bijection between quadratic forms and bilinear forms.
\end{remark}
We now associate a special symmetric matrix $B^*$ to a bilinear form $B\colon V\times V\to K$. A bilinear form $B\colon V^2\to K$ gives a map $B\colon V\otimes_KV\to K$, which gives a map $B^*\colon V\to V^\lor$ by the tensor--hom adjunction. (Explicitly, $B^*\colon v\mapsto B(v,\cdot)$.) Giving $V$ a basis $\{e_i\}_{i=1}^n$ and $V^\lor$ the dual basis $\{e_i^\lor\}_{i=1}^n$, we may represent $B^*$ as the matrix $A=(a_{ij})_{1\le i,j\le n}$. Explicitly, we see
\[B(e_i,\cdot)=B^*(e_i)=\sum_{j=1}^na_{ij}e_j^\lor,\]
so $B(e_i,e_j)=a_{ij}=e_i^\intercal B^*e_j$. As such, we see that $a_{ij}=a_{ji}$ because $B$ is symmetric, so $B^*$ is symmetric.

More generally, for vectors $v=\sum_ix_ie_i$ and $w=\sum_jy_je_j$, we see
\[B(v,w)=\sum_{i=1}^n\sum_{j=1}^nx_iy_jB(e_i,e_j)=\sum_{i=1}^n\sum_{j=1}^n(x_ie_i^\intercal)B^*(y_je_j)=v^\intercal B^*w,\]
and so
\[Q(v)=B(v,v)=v^\intercal B^*v=\sum_{i=1}^n\sum_{j=1}^na_{ij}x_ix_j.\]
This justifies us viewing $Q$ as being a homogeneous polynomial of degree $2$.
\begin{definition}[non-degenerate]
	A quadratic form $Q$ on a finite-dimensional $K$-vector space $V$ is \textit{non-degenerate} if and only if the induced bilinear form $B\colon V\otimes_KV\to K$ induces an isomorphism $B^*\colon V\to V^\lor$.
\end{definition}
\begin{remark} \label{rem:non-degenerate-bilin}
	Because $\dim V=\dim V^\lor$, we see $Q$ is non-degenerate if and only if $B^*\colon V\to V^\lor$ is injective, which is equivalent to asserting $B(v,\cdot)\colon V\to K$ is the zero map if and only if $v=0$.
\end{remark}
Given our quadratic form $Q$ on $K$, we note there is a map
\[\bigwedge^nV\xrightarrow{\det B^*}\bigwedge^nV^\lor=\left(\bigwedge^nV\right)^\lor\]
of $1$-dimensional $K$-vector spaces, where $n=\dim V$. Equivalently, we get a map
\[\left(\bigwedge^nV\right)^{\otimes2}\to K,\]
which is still of $1$-dimensional vector spaces and is essentially given by $B^*$. This morphism produces an element of $K$, but we can visually see that adjusting the basis of $V$ adjusts this constant by a square in $K$.

More directly, letting $\{e'_i\}_{i=1}^n$ be a new basis of $V$, we can compute the new matrix by computing $B(e'_i,e'_j)$. Let $e'_i=\sum_{k=1}^ns_{ik}e_k$ so that $S=(s_{ij})_{1\le i,j\le n}$ is the change-of-basis matrix. Then
\[B(e'_i,e'_j)=\sum_{k=1}^n\sum_{\ell=1}^ns_{ik}s_{j\ell}B(e_i,e_j)=\sum_{k=1}^n\sum_{\ell=1}^ns_{ik}a_{ij}s_{j\ell}=(S^\intercal AS)_{ij},\]
so $S^\intercal AS$ is our new matrix, meaning we have adjusted or determinant by the square $(\det S)^2$.

So here is our definition.
\begin{definition}[discriminant]
	Fix a quadratic form $Q$ on a finite-dimensional $K$-vector space. Then the \textit{discriminant} is $\det B^*\in K/\left(K^{\times2}\right)$, where $B^*\colon V\to V^\lor$ is the associated linear transformation. Note that $Q$ is non-degenerate if and only if $\op{disc}Q\ne\{0\}$.
\end{definition}
The goal of this part of the course is the following result, which we will write down more precisely.
\begin{theorem}[Hasse--Minkowski] \label{thm:hasse-mink}
	Let $K$ be a number field, and let $Q$ be a quadratic form on the $K$-vector space $V$. Then $Q$ has a nontrivial $0$ in $V$ if and only if $Q$ has a nontrivial zero in $V\otimes_KK_v$ for all places $v$ of $K$.
\end{theorem}
We are going to black-box a few cohomological tools in the course of proving \Cref{thm:hasse-mink}. Later we will go back and prove them.

\end{document}