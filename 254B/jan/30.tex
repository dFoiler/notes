% !TEX root = ../notes.tex

\documentclass[../notes.tex]{subfiles}

\begin{document}

\section{January 30}

We continue discussing group cohomology.

\subsection{Cocycles}
We discuss cocycles, which will be an explicit way to discuss group cohomology.
\begin{remark}
	These notions come from algebraic topology, where a group $G$ gives rise to a space $EG$, which is constructed as functions $\op{Mor}([n+1],G)$ at degree $n$ satisfying certain conditions. One can use this to build a space which is contractible and has a free $G$-action; then $BG\coloneqq EG/G$ is the classifying space, the point of which is that $\pi_1(BG)=G$ and no other nontrivial homotopy groups. If you write everything out, you can get cocycles from this construction.
\end{remark}
So let's write things out. For $n\ge0$, define the $G$-module $P_n\coloneqq\ZZ\left[G^{n+1}\right]$, and define the differential $d\colon P_n\to P_{n-1}$ by
\[d(g_0,\ldots,g_n)\coloneqq\sum_{i=0}^n(-1)^i(g_0,\ldots,g_{i-1},g_{i+1},\ldots,g_n).\]
One can check by hand that $d^2=0$, so we get a complex
\[\cdots\to P_3\to P_2\to P_1\to P_0\to0.\]
Here are some checks.
\begin{itemize}
	\item Note that each $P_n$ is a free $\ZZ[G]$-module, generated by the elements of the form $(1,g_1,\ldots,g_n)$. Indeed, we can write
	\[\ZZ[G]\cdot(1,g_1,\ldots,g_n)=\bigoplus_{g\in G}\ZZ[(g,gg_1,\ldots,gg_n)],\]
	so looping over all basis elements completes this. As such, $P_n\cong\ZZ[G]^n$ for each $n\ge0$.
	\item We would like to turn this into a resolution of $\ZZ$. Well, there is the usual augmentation map $\varepsilon\colon P_0\to\ZZ$ given by $g\mapsto1$. Additionally, the composite $P_1\to P_0\to\ZZ$ is the zero map: for each basis element $(g_0,g_1)$, we see
	\[\varepsilon d(g_0,g_1)=\varepsilon(g_1-g_0)=0.\]
	\item We now claim that $\varepsilon\colon P_\bullet\to\ZZ$ is an (augmented) free resolution. We know that it's free, so it remains to check our exactness. Note we already have surjectivity $P_0\to\ZZ$, so we need to show that $H^n(P_\bullet)=0$ for $n\ge1$.
	
	Now, we want an isomorphism of some cohomology groups, so we would like to find a chain homotopy between ${\id}$ and zero. Explicitly, we would like to find group homomorphisms $h_n\colon P_n\to P_{n+1}$ fitting into the diagram
	% https://q.uiver.app/?q=WzAsMTAsWzMsMCwiUF8wIl0sWzQsMCwiXFxaWiJdLFs0LDEsIlxcWloiXSxbMywxLCJQXzAiXSxbMiwwLCJQXzEiXSxbMiwxLCJQXzEiXSxbMSwwLCJQXzIiXSxbMSwxLCJQXzIiXSxbMCwwLCJcXGNkb3RzIl0sWzAsMSwiXFxjZG90cyJdLFs4LDYsImQiXSxbNiw0LCJkIl0sWzAsMSwiXFx2YXJlcHNpbG9uIl0sWzksNywiZCIsMl0sWzcsNSwiZCIsMl0sWzUsMywiZCIsMl0sWzMsMiwiXFx2YXJlcHNpbG9uIiwyXSxbNCwwLCJkIl0sWzEsMywiaF97LTF9IiwxXSxbMCw1LCJoXzEiLDFdLFs0LDcsImhfMiIsMV1d&macro_url=https%3A%2F%2Fraw.githubusercontent.com%2FdFoiler%2Fnotes%2Fmaster%2Fnir.tex
	\[\begin{tikzcd}
		\cdots & {P_2} & {P_1} & {P_0} & \ZZ \\
		\cdots & {P_2} & {P_1} & {P_0} & \ZZ
		\arrow["d", from=1-1, to=1-2]
		\arrow["d", from=1-2, to=1-3]
		\arrow["\varepsilon", from=1-4, to=1-5]
		\arrow["d"', from=2-1, to=2-2]
		\arrow["d"', from=2-2, to=2-3]
		\arrow["d"', from=2-3, to=2-4]
		\arrow["\varepsilon"', from=2-4, to=2-5]
		\arrow["d", from=1-3, to=1-4]
		\arrow["{h_{-1}}"{description}, from=1-5, to=2-4]
		\arrow["{h_1}"{description}, from=1-4, to=2-3]
		\arrow["{h_2}"{description}, from=1-3, to=2-2]
	\end{tikzcd}\]
	so that $dh_n+h_{n-1}d=\op{id}$. The point here is that, for $n\ge1$, we see $z\in\ker(P_n\to P_{n-1})$ implies that $dh_n(z)+h_{n-1}(dz)=z$, but then $dh_n(z)=z$, so $z$ is in the image of the map $P_{n+1}\to P_n$. The exactness will then follow.

	For $n\ge-1$, we define $h_n\colon P_n\to P_{n+1}$ by
	\[h_n(g_0,\ldots,g_n)\coloneqq(1,g_0,\ldots,g_n).\]
	To check this works, we compute
	\begin{align*}
		dh_n(g_0,\ldots,g_n)+h_{n-1}d(g_0,\ldots,g_n) &= d(1,g_0,\ldots,g_n)+h_{n-1}\Bigg(\sum_{i=0}^n(-1)^i(g_0,\ldots,g_{i-1},g_{i+1},\ldots,g_n)\Bigg) \\
		&= \Bigg((g_0,\ldots,g_n)-\sum_{i=0}^n(-1)^i(1,g_0,\ldots,g_{i-1},g_{i+1},\ldots,g_n)\Bigg) \\
		&\qquad+\Bigg(\sum_{i=0}^n(-1)^i(1,g_0,\ldots,g_{i-1},g_{i+1},\ldots,g_n)\Bigg) \\
		&= (g_0,\ldots,g_n),
	\end{align*}
	which completes the computation.
\end{itemize}
Thus, we see we have a free resolution of $\ZZ$, so we can compute group cohomology as previously discussed in \Cref{rem:compute-cohom}. Explicitly, for a $G$-module $M$, we define
\[\widetilde C^n(G,M)\coloneqq\op{Hom}_{\ZZ[G]}(P_n,M)\subseteq\op{Mor}_G\left(G^{n+1},M\right),\]
and the differential sends $f\in\widetilde C^n(G,M)$ to $f\circ d$, which is
\[(df)(g_0,\ldots,g_n,g_{n+1})=\sum_{i=0}^n(-1)^if(g_0,\ldots,g_{i-1},g_{i+1},\ldots,g_n).\]
Indeed, we can see visually that this has constructed a $G$-module morphism.

The $G$-module $\widetilde C(G,M)$ has defined what are called ``homogeneous cocycles.'' However, recall that $P_n$ is a free $\ZZ[G]$-module generated by the elements of the form $(1,g_1,\ldots,g_n)$, so we can think of $\op{Hom}_{\ZZ[G]}\left(P_n,M\right)$ as functions $G^n\to M$, with no $G$-equivariance. However, our isomorphism $P_n\cong\ZZ[G]^n$ was moderately non-canonical, so our differential has changed somewhat. It is standard convention to define $P_n$ as instead generated by
\[(1,g_1,g_1g_2,g_1g_2g_3,\ldots,g_1\cdots g_n),\]
which makes our differential
\[(df)(g_1,\ldots,g_{n+1})=g_1f(g_2,\ldots,g_{n+1})+\sum_{i=1}^n(-1)^if(g_1,\ldots,g_ig_{i+1},\ldots,g_n)+(-1)^{n+1}f(g_1,\ldots,g_n).\]
This defines ``inhomogeneous cocycles,'' which we define as $C^n(G,M)$.
\begin{example}
	We discuss $H^1$. The differential $d\colon C^0(G,M)\to C^1(G,M)$ sends an element $m$ to the function $g\mapsto(g-1)m$. Further, the differential $d\colon C^1(G,M)\to C^2(G,M)$ is given by
	\[(df)(g_1,g_2)=g_1f(g_2)-f(g_1g_2)+f(g_1).\]
	In total, $H^1(G,M)$ is isomorphic to
	\[\frac{\{f:f(g_1g_2)=g_1f(g_2)+f(g_1)\}}{\{f:f(g)=(g-1)m\text{ for some }m\in M\}}.\]
	For example, if the $G$-action is trivial, the kernel of this differential is just the homomorphisms $G\to M$, so $H^1(G,M)=\op{Hom}(G,M)$.
\end{example}

\end{document}