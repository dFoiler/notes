% !TEX root = ../notes.tex

\documentclass[../notes.tex]{subfiles}

\begin{document}

Here we go. Office hours begin today.

\subsection{Classification of CM Abelian Varieties}
Here is our definition. The point is that we would like to ``recover'' the complex multiplication of a field of CM type acting on a CM abelian variety.
\begin{definition}[CM type]
	Fix a CM field $E$, and let $(A,i)$ be an abelian variety with complex multiplication by $E$ by $i\colon E\to\op{End}^0(A)$. Then $E$ acts faithfully on $H_1(A(\CC),\QQ)$. Hodge theory tells us that we can decompose
	\[H^1(A(\CC),\CC)=H^{01}\oplus H^{10},\]
	where $H^{10}=\ov{H^{01}}$; here $H^{01}=H^0(A(\CC),\Omega^1)$ is the space of global sections $1$-forms on $A(\CC)$. Dualizing, we see
	\[H_1(A(\CC),\CC)=\op{Lie}A(\CC)\oplus\ov{\op{Lie}A(\CC)},\]
	and in fact $E$ acts on $\op{Lie}A(\CC)$. Decomposing $\op{Lie}A(\CC)$ as an $E$-representation as $\bigoplus_{\varphi\in\Phi}\CC_\varphi$ where $\Phi\subseteq\op{Hom}(E,\CC)$. (This decomposes into $1$-dimensional representations because $E^\times$ is commutative.) Then $\Phi$ is the CM type.
\end{definition}
\begin{remark}
	The point of using the Hodge decomposition is to note that $\op{Hom}(E,\CC)=\Phi\sqcup\ov\Phi$ by taking the conjugation of the action. Thus, $(E,\Phi)$ is fact a CM type.
\end{remark}
\begin{example}
	Fix a CM type $(E,\Phi)$, and set $A\coloneqq\CC^\Phi/\OO_E$. Then the CM type of $A$ can be recovered as $\Phi$.
\end{example}
We are going to classify isogeny and isomorphism classes of these abelian varieties. The following definition will be useful.
\begin{definition}
	Fix CM types $(E,\Phi)$ and $(E',\Phi')$. An \textit{isomorphism of CM types} is an isomorphism $\alpha\colon E\to E'$ such that
	\[\Phi=\{\varphi'\circ\alpha:\varphi'\in\Phi'\}.\]
\end{definition}
Here is the point of this definition.
\begin{proposition} \label{prop:classify-cm-ab-isog}
	Fix a CM algebra $E$. Then the set of pairs $(A,i)$ of abelian varieties with complex multiplication by $i$ (up to isogeny commuting with $i$)\todo{Correct this notion of isogeny} is in bijection with CM types $(E,\Phi)$ up to isomorphism.
\end{proposition}
\begin{proof}
	Here, an isogeny $\varphi\colon(A,i)\to(A',i')$ commuting with the complex multiplication is simply an isogeny $\varphi\colon A\to A'$ such that $i'\colon E\to\op{End}^0(A')$ is the composite $E\stackrel i\to\op{End}^0(A)\stackrel\varphi\to\op{End}^0(A')$.
	
	We have already defined functors in both directions. To review, we send $(E,\Phi)$ to $\CC^\Phi/\OO_E$, and the other direction takes $(A,i)$ to $(E,\Phi)$ where $\Phi$ is the CM type given above. Here are some checks.
	\begin{itemize}
		\item The construction of $(E,\Phi)\mapsto\CC^\Phi/\OO_E$ is well-defined. Well, suppose we have an isomorphism of CM types $\alpha\colon(E,\Phi)\to(E,\Phi')$. Then we get a commutative diagram as follows.
		% https://q.uiver.app/#q=WzAsNixbMCwwLCJcXENDXlxcUGhpIl0sWzEsMSwiRSJdLFsyLDEsIkUnIl0sWzMsMSwiXFxPT19FIl0sWzAsMSwiXFxPT19FIl0sWzMsMCwiXFxDQ157XFxQaGknfSJdLFsxLDIsIlxcYWxwaGEiXSxbNCwxXSxbMywyXSxbNCwwLCJcXFBoaSJdLFszLDUsIlxcUGhpJyIsMl0sWzAsNV1d&macro_url=https%3A%2F%2Fraw.githubusercontent.com%2FdFoiler%2Fnotes%2Fmaster%2Fnir.tex
		\[\begin{tikzcd}
			{\CC^\Phi} &&& {\CC^{\Phi'}} \\
			{\OO_E} & E & {E} & {\OO_{E}}
			\arrow["\alpha", from=2-2, to=2-3]
			\arrow[from=2-1, to=2-2]
			\arrow[from=2-4, to=2-3]
			\arrow["\Phi", from=2-1, to=1-1]
			\arrow["{\Phi'}"', from=2-4, to=1-4]
			\arrow[from=1-1, to=1-4]
		\end{tikzcd}\]
		In total, we produce an isomorphism of vector spaces $\CC^\Phi\to\CC^{\Phi'}$ carrying $\OO_E$ to $\OO_{E}$, which is the needed isogeny; one can check that this isogeny commutes with the actions everywhere as well.

		\item We describe all $(A,i)$ with CM type $(E,\Phi)$. Recall that $A=(\op{Lie}A)/\Lambda$ for some lattice $\Lambda\subseteq\op{Lie}A$ of full rank; however, having CM type implies that we can just view $\op{Lie}A$ as $\CC^\Phi$, so $A=\CC^\Phi/\Lambda$.

		Now, $E$ must have an action on $\Lambda\otimes_\ZZ\QQ$. Fixing a nonzero vector $v$, we see that its orbit by $E$ will be all of $\Lambda\otimes_\ZZ\QQ$; now, having a lattice $\Lambda\subseteq E$ implies that $\Lambda$ and $\OO_E$ are commensurable, which establishes an isogeny with $\OO_E/\Lambda$, as needed.
	\end{itemize}
	We won't bother to check that these functors are inverses of each other.
\end{proof}
\begin{remark}
	We will eventually discuss the moduli space $\mc A_g$ of principally polarized $g$-dimensional abelian varieties. Then one can require that $\op{End}(A)$ contains $\OO_E$ for some CM field $E$ as well as $[E:\QQ]=2\dim A$, and this will make finitely many points. (In fact, we produce a Shimura variety of PEL type by adding in $\Phi$, which corresponds to a signature.) Dropping the condition that $[E:\QQ]=2\dim A$ could still desire a positive-dimensional subset of $\mc A_g$; in particular, we cannot expect that ``just'' (finite) combinatorics will be able to parameterize such abelian varieties.
\end{remark}
\begin{remark} \label{rem:fully-classify-cm-ab}
	We continue with a classification of the $(A,i)$ with CM type $(E,\Phi)$. Letting $\OO\subseteq E$ be the largest subring such that $\OO\cdot\Lambda\subseteq\Lambda$, it turns out that $\op{End}(A)=\OO$ by \Cref{prop:map-of-complex-torus}. Thus, $\Lambda$ is an $\mathcal O$-fractinal ideal.
\end{remark}
\begin{corollary}
	Fix a CM algebra $E$ and an order $\OO\subseteq E$. Then the isomorphism classes of CM abelian varieties $(A,i)$ with complex multiplication by $\OO\subseteq E$ (namely, such that $i\colon\OO\to\op{End}(A)$) is in bijection with equivalence classes of triples $(E,\Phi,\mf a)$ where $\Phi$ is a CM type of $E$, and $\mf a\subseteq\OO$ is a fractional ideal. The equivalence class of triples is given by $(E,\Phi,\mf a)\sim(E,\Phi',\mf a)$ if and only if there is an isomorphism $\alpha\colon E\to E$ carrying $\Phi$ to $\Phi'=\Phi\circ\alpha$ and $\alpha(\mf a)=c\mf a'$ for some $c\in E^\times$.
\end{corollary}
\begin{proof}
	Use the functors of \Cref{prop:classify-cm-ab-isog}, but now we use \Cref{rem:fully-classify-cm-ab} at the end of the proof.
\end{proof}
\begin{example}
	With $\OO=\OO_E$, we see that our abelian varieties are now in bijection with $\op{Cl}_E$.
\end{example}
\begin{remark}
	Later in life, we will want to add a polarization to results such as \Cref{prop:classify-cm-ab-isog}. Additionally, we are somehow studying ``geometric points'' in the moduli space; there is a separate question of asking over what fields these points in the moduli space can be found over.
\end{remark}

\subsection{Classifying Simple CM Abelian Varieties}
We would like to upgrade \Cref{prop:classify-cm-ab-isog} to restrict to simple abelian varieties. This requires the notion of a ``primitive'' CM type.
\begin{defihelper}[restriction, extension of CM types] \nirindex{restriction} \nirindex{extension}
	Fix an extension $E_0\subseteq E$ of CM algebras.
	\begin{itemize}
		\item Given a CM type $\Phi_0$ on $E_0$, we define its \textit{extension} to $E$ as
		\[\Phi\coloneqq\{\varphi\in\op{Hom}(E,\CC):\varphi|_{E_0}\in\Phi_0\}.\]
		\item Suppose $(E,\Phi)$ is a CM type which is an extension of a CM type $(E_0,\Phi_0)$. Then we can recover the \textit{restriction} to $E_0$ as
		\[\Phi|_{E_0}\coloneqq\{\varphi|_{E_0}:\varphi\in\Phi\}.\]
	\end{itemize}
\end{defihelper}
\begin{remark}
	In fact, $\Phi|_{E_0}$ will succeed in being a CM type if and only if it is an extension. This explains the hypothesis in the definition.
\end{remark}
\begin{definition}[primitive]
	Fix a CM algebra $E$. A CM type $\Phi$ on $E$ is \textit{primitive} if and only if $\Phi$ is not the extension of any CM type $(E_0,\Phi_0)$ for $E_0\subseteq E$.
\end{definition}
Here is a quick sanity check.
\begin{lemma}
	Fix a CM type $(E,\Phi)$, where $E$ is a field. Then there is a unique primitive CM type $(E_0,\Phi_0)$ extending to $(E,\Phi)$.
\end{lemma}
\begin{proof}
	Omitted. The reference is \cite[Proposition~1.9]{milne-cm}. Basically, one may assume that $E$ is Galois, and then one can restrict downwards via some kind of fixed field.
\end{proof}
And here is our result.
\begin{restatable}{proposition}{simplecmclassification}
	Fix a CM field $E$. Then there is a bijection between simple abelian varieties $A$ with complex multiplication by $E$ (up to isogeny) and primitive CM types $(E,\Phi)$ up to isomorphism.
\end{restatable}
We will prove this next class.

\end{document}