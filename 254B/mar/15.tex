% !TEX root = ../notes.tex

\documentclass[../notes.tex]{subfiles}

\begin{document}

Here we go.

\subsection{Degree via Tate Modules}
Here is our next result: characteristic polynomials can be computed on the Tate module.
\begin{theorem} \label{thm:tate-char-poly}
	Fix an endomorphism $f\in\op{End}^0(A)$.
	\begin{listalph}
		\item We have $\deg f=\det V_\ell f$, where $V_\ell$ is the functor $A\mapsto (T_\ell A)\otimes_\ZZ\QQ_\ell$. Thus, the characteristic polynomial $P_\ell(x)$ of $V_\ell f$ satisfies, for any $n\in\ZZ$,
		\[P_\ell(n)=\deg([n]_A-f).\]
		\item The characteristic polynomial $P(x)$ of $T_\ell f$ has integral coefficients.
	\end{listalph}
\end{theorem}
For (b), note that (a) actually tells us that $P(n)=\deg([n]_A-f)$ for all integers $n$.

It will be helpful to have the following lemma.
\begin{lemma}
	Fix an isogeny $f\colon A\to B$ of abelian $k$-varieties. Then the sequence
	\[0\to T_\ell A\to T_\ell B\to(\ker f)(k^{\mathrm{sep}})_\ell\to0\]
	is exact, where, where $(\cdot)_\ell$ denotes taking the $\ell$-primary part.
\end{lemma}
\begin{proof}
	See \cite[10.5--10.6]{egm-av}. Let's sketch the idea. The point is to use cohomology, so we begin by writing
	\[T_\ell A=\limit A[\ell^\bullet](k^{\mathrm{sep}})=\limit\op{Hom}\left(\ZZ/\ell^\bullet\ZZ,A(k^{\mathrm{sep}})\right)=\op{Hom}\left(\QQ_\ell/\ZZ_\ell,A(k^{\mathrm{sep}})\right).\]
	Now, setting $N\coloneqq\ker f$, we have an exact sequence
	\[0\to N\to A\stackrel f\to B\to0\]
	of fppf sheaves, which gives an exact sequence
	\[0\to N(k^{\mathrm{sep}})\to A(k^{\mathrm{sep}})\stackrel f\to B(k^{\mathrm{sep}})\to0,\]
	which is exact on the right by the surjectivity of $f$. Now, applying the functor $\op{Hom}(\QQ_\ell/\ZZ_\ell,-)$, we note that $N(k^{\mathrm{sep}})$ is finite anyway, so it will vanish under the functor, leaving us with the exact sequence
	\[0\to T_\ell A\to T_\ell B\to\op{Ext}^1(\QQ_\ell/\ZZ_\ell,N(k^{\mathrm{sep}}))\to\op{Ext}^1(\QQ_\ell/\ZZ_\ell,A(k^{\mathrm{sep}})).\]
	Notably, if $k$ is perfect, then $k^{\mathrm{sep}}=\ov k$, so $A(\ov k)$ being divisible makes the last term vanish; in the general case, one still gets that the map into that term vanishes because we are looking at $\ell$-torsion, where $A(k^{\mathrm{sep}})$ is going to be sufficiently divisible.

	So it remains to compute the $\op{Ext}^1$ term. To begin, note that
	\[\op{Ext}^1(\QQ_\ell/\ZZ_\ell,N(k^{\mathrm{sep}}))=\op{Ext}^1(\QQ_\ell/\ZZ_\ell,N(k^{\mathrm{sep}})_\ell)\]
	because $\QQ_\ell/\ZZ_\ell$ works to kill out anything other than $\ell$-torsion. (Namely, multiplication by something coprime to $\ell$ is an isomorphism on $\QQ_\ell/\ZZ_\ell$ but will kill out what happens in $N$.) Now, to compute this last term, we take the exact sequence
	\[0\to\ZZ_\ell\to\QQ_\ell\to\QQ_\ell/\ZZ_\ell\to0\]
	and apply $\op{Hom}(-,N(k^{\mathrm{sep}})_\ell)$ to get
	\[\op{Ext}^1(\QQ_\ell/\ZZ_\ell,N(k^{\mathrm{sep}})_\ell)=\op{Hom}(\ZZ_\ell,N(k^{\mathrm{sep}})_\ell),\]
	after some argument in the long exact sequence, which is what we wanted.
\end{proof}
We are now ready to prove \Cref{thm:tate-char-poly}.
\begin{proof}[Proof of \Cref{thm:tate-char-poly}]
	We focus on (a) for now; the second claim is immediate from the first and the definition of the characteristic polynomial, so we focus on the first claim. It suffices to prove the results for bona fide endomorphisms $f\colon A\to A$. Indeed, once we have the result here, scaling produces the result for $\QQ$, and then a density argument for $\QQ\subseteq\QQ_\ell$ achieves the full result for $\QQ_\ell$. Then we see that
	\[\left|\deg f\right|_\ell=\left|\#(\ker f)_\ell(k^{\mathrm{sep}})\right|_\ell.\]
	Now, the above lemma tells us that this is $\left|\det T_\ell f\right|_\ell$. This equality extends to $f\in\op{End}(A)\otimes\QQ_\ell$ via the aforementioned density argument.

	We now apply \Cref{prop:classify-norm-trace-form}. Now, write $\op{End}(A)\otimes_\ZZ\QQ_\ell$ as a product of simple $\QQ_\ell$-algebras $\prod_iD_i$ (notably, this algebra is semisimple because it is the base-change of a semisimple algebra). Now, $\deg$ and $\det$ agree on $\ell$-adic valuation as above, so the classification \Cref{prop:classify-norm-trace-form} forces them to be actually equal. More formally, one should write $f\mapsto\left|\deg f\right|_\ell$ as some product
	\[\prod_i\left(\op{Nm}_{K_i/\QQ_\ell}\circ\op{Nm}_{D_i/K_i}^\circ\right)^{v_i}\]
	where $K_i=Z(D_i)$ and the $v_\bullet$s are some integers. Doing similar for $\det$ and then plugging in various $f\in\prod_iD_i$ (forming a basis) reveals the desired equality.

	We now show (b). Define $P(n)\coloneqq\deg([n]_A-f)$, which we know is a polynomial (because $\deg$ is polynomial \Cref{thm:deg-is-poly}), meaning we can view $P$ as an element of $\QQ[x]$; thus, $P_\ell=P$ has rational coefficients. It remains to show that the coefficients are integral. Well, because $\op{End}(A)$ is free of finite rank over $\ZZ$, our $f$ is going to have some monic minimal polynomial $q\in\ZZ[x]$.\footnote{This can be checked by base-changing to $\CC$ and thinking about the minimal polynomial for a morphism on the level of lattices.} Thus, $q(V_\ell f)=0$, so the roots of $P_\ell$ must all be algebraic integers, meaning that $P(x)\in\ov\ZZ[x]$, meaning that $P(x)\in\ZZ[x]$, as desired.
\end{proof}

\subsection{Weil Pairing}
We now use duality for fun and profit.
\begin{definition}[Tate twist]
	Fix a field $k$ and a prime $\ell$ not divisible by $\op{char}k$. We define the \textit{Tate twist} $\ZZ_\ell(1)$ as $T_\ell(\mathbb G_m)$. Notably, $\ZZ_\ell(1)$ is a free $\ZZ_\ell$-module of rank $1$ with Galois action from $\op{Gal}(k^{\mathrm{sep}}/k)$ acting via the cyclotomic character, where the point is that
	\[T_\ell(\mathbb G_m)=\limit\mu_{\ell^\bullet}.\]
	More generally, given a free $\ZZ_\ell$-module $M$ of finite rank, we define $M(n)\coloneqq M\otimes_{\ZZ_\ell}\ZZ_\ell(1)^{\otimes n}$, where $n$ is an integer.
\end{definition}
To motive our Weil pairing, we note that $A^\lor[\ell^\bullet]\cong A[\ell^\bullet]^\lor$ by \Cref{thm:dual-map}, where the second dual is a Cartier dual. Thus, using our Cartier duality, we induce a map
\[A[\ell^\bullet]\times A^\lor[\ell^\bullet]\to\mu_{\ell^\bullet}.\]
We would like to take limits over $\ell^\bullet$, but to do this, we need the following diagram to commute.
% https://q.uiver.app/#q=WzAsNCxbMCwwLCJBW1xcZWxsXm5dXFx0aW1lcyBBXlxcbG9yW1xcZWxsXm5dIl0sWzEsMCwiXFxtdV97XFxlbGxebn0iXSxbMCwxLCJBXFxsZWZ0W1xcZWxsXntuKzF9XFxyaWdodF1cXHRpbWVzIEFeXFxsb3JcXGxlZnRbXFxlbGxee24rMX1cXHJpZ2h0XSJdLFsxLDEsIlxcbXVfe1xcZWxsXntuKzF9fSJdLFsyLDAsIihcXGVsbCxcXGVsbCkiXSxbMywxLCJcXGVsbCIsMl0sWzAsMV0sWzIsM11d&macro_url=https%3A%2F%2Fraw.githubusercontent.com%2FdFoiler%2Fnotes%2Fmaster%2Fnir.tex
\[\begin{tikzcd}
	{A[\ell^n]\times A^\lor[\ell^n]} & {\mu_{\ell^n}} \\
	{A\left[\ell^{n+1}\right]\times A^\lor\left[\ell^{n+1}\right]} & {\mu_{\ell^{n+1}}}
	\arrow["{(\ell,\ell)}", from=2-1, to=1-1]
	\arrow["\ell"', from=2-2, to=1-2]
	\arrow[from=1-1, to=1-2]
	\arrow[from=2-1, to=2-2]
\end{tikzcd}\]
To check the commutativity here, we need to recall the isomorphism $A^\lor[\ell^\bullet]\cong A[\ell^\bullet]^\lor$.

Namely, given an endomorphism $f\colon A\to A$, we need to recall why $(\ker f)^\lor\cong\ker f^\lor$. Well, fix a $\ov k$-point $x$ of $\ker f$ and some line bundle $\mc L$ on $\ker f$. Then we may choose some $\beta\colon f^*\mc L\to\OO_A$ (unique up to scalar), and we note that we have the following large diagram.
% https://q.uiver.app/#q=WzAsNSxbMCwwLCJ0X3heKmZeKlxcbWMgTCJdLFsxLDAsInRfeF4qXFxPT19BIl0sWzAsMSwiZl4qdF97Zih4KX1eKlxcbWMgTCJdLFswLDIsImZeKlxcbWMgTCJdLFsxLDIsIlxcT09fQSJdLFsxLDRdLFszLDQsInRfeF4qXFxiZXRhIiwyXSxbMCwyLCIiLDIseyJsZXZlbCI6Miwic3R5bGUiOnsiaGVhZCI6eyJuYW1lIjoibm9uZSJ9fX1dLFsyLDMsIiIsMix7ImxldmVsIjoyLCJzdHlsZSI6eyJoZWFkIjp7Im5hbWUiOiJub25lIn19fV0sWzAsMSwidF94XipcXGJldGEiXV0=&macro_url=https%3A%2F%2Fraw.githubusercontent.com%2FdFoiler%2Fnotes%2Fmaster%2Fnir.tex
\[\begin{tikzcd}
	{t_x^*f^*\mc L} & {t_x^*\OO_A} \\
	{f^*t_{f(x)}^*\mc L} \\
	{f^*\mc L} & {\OO_A}
	\arrow[from=1-2, to=3-2]
	\arrow["{t_x^*\beta}"', from=3-1, to=3-2]
	\arrow[Rightarrow, no head, from=1-1, to=2-1]
	\arrow[Rightarrow, no head, from=2-1, to=3-1]
	\arrow["{t_x^*\beta}", from=1-1, to=1-2]
\end{tikzcd}\]
Then our Weil pairing $e_f\colon(\ker f)(\ov k)\times(\ker f^\lor)(\ov k)\to\mathbb G_m$ is just given by
\[e_f(x,\mf L)\coloneqq t_x^*\beta\circ\beta^{-1},\]
which is an isomorphism $\OO_A\to\OO_A$ and hence provides a global section and hence an element of $\ov k^\times$, as needed. Note that this does not change if we adjust $\beta$ by a scalar, so it notably does not depend on the choice of $\beta$.

Let's now do our computation.
\begin{lemma}
	Fix an abelian $k$-variety $A$ and positive integers $n$ and $m$. Given $\mc L\in A^\lor[m](\ov k)$ and $x\in A[mn](\ov k)$, we have $e_{mn}(x,\mc L)=e_m(nx,\mc L)$.
\end{lemma}
\begin{proof}
	We do the explicit computation. Pick an isomorphism $\beta\colon[m]^*\mc L\to\OO_A$, which induces the isomorphism $[n]^*\beta\colon[mn]^*\mc L\to\OO_A$. Now, we compute
	\[e_{mn}(x,\mc L)=t_x^*([n]^*\beta)\circ([n]^*\beta)^{-1}=[n]^*\left(t_{nx}^*\beta\circ\beta^{-1}\right)=[n]^*e_m(nx,\mc L)=e_m(nx,\mc L).\]
	Here, this last equality comes about because we're just pulling back a full isomorphism $\OO_A\to\OO_A$, which does not change the produced global section.
\end{proof}
\begin{remark}
	On the homework, we will show that a homomorphism $f\colon A\to B$ reveals
	\[e_{\ell^\infty}((T_\ell f)x,y)=e_{\ell^\infty}(x,T_\ell(f^\lor)y),\]
	again by a reasonably explicit computation.
\end{remark}
As a corollary, we compute
\[e_{\ell^{n}}(\ell x,\ell\mc L)=e_{\ell^{n+1}}(x,\ell\mc L)=e_{\ell^{n+1}}(x,\mc L)^{\ell},\]
where the first equality is by the lemma, and the second equality is by using the explicit description for the pairing provided above. (More precisely, we can see that taking a power of $\ell$ induces a power at the end.) So we may take limits to produce the following definition.
\begin{definition}[Weil pairing]
	Fix an abelian $k$-variety $A$. Then we define the \textit{Weil pairing} as $e_\bullet\colon T_\ell A\times T_\ell A^\lor\to\ZZ_\ell(1)$ defined above.
\end{definition}
\begin{remark}
	A choice of polarization $A\to A^\lor$ grants us a skew-symmetric form
	\[T_\ell A\times T_\ell A\to\ZZ_\ell(1)\]
	induced by the Weil pairing.
\end{remark}

\end{document}