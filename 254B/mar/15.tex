% !TEX root = ../notes.tex

\documentclass[../notes.tex]{subfiles}

\begin{document}

\section{March 15}

Ok, let me continue then.

\subsection{More on Dual Isogenies}
We have the following results on dual isogenies.
\begin{lemma} \label{lem:cancel-right}
	Fix smooth, proper, projective, geometrically connected $k$-curves $C$ and $C'$ and $D$ and $D'$. Given non-constant morphisms $f,g\colon C\to D$ and more non-constant morphisms $p\colon C'\to C$ such that
	\[f\circ p=g\circ p.\]
	Then $f=g$.
\end{lemma}
\begin{proof}
	To begin, note that $p$ is dominant because it is non-constant onto a one-dimensional scheme, but it is also proper, so it is both a continuous and closed map; in other words, the topology on $C$ is merely given by the quotient topology from $C'$, so we conclude that $f=g$ on topological spaces. For clarity, call this map $h$.

	It remains to check equality on the level of sheaves. Here, we are looking at the composite
	\[h^{-1}\OO_D\rightrightarrows\OO_C\into p_*\OO_{C'},\]
	where the map on the right is injective because $p$ is dominant, and our scheme is reduced.
\end{proof}
\begin{remark} \label{rem:cancel-left}
	We can also cancel on the other side, in a special case. Fix isogenies of elliptic curves $f,g\colon(E,e)\to(E',e')$ such that there is an isogeny $q\colon(E',e')\to(E'',e'')$ with $q\circ f=q\circ g$. Then we see that $q\circ(f-g)$ maps everything to $e''$, so $(f-g)$ maps to $\ker q\subseteq E'$. However, $\ker q$ is a closed subscheme of $E'$ not equal to $E'$, so it is zero-dimensional. But if $f-g$ is non-constant, then it is dominant and cannot land in such a closed subscheme, so we conclude that $f-g$ must be constantly $e'$, finishing.
\end{remark}
\begin{proposition} \label{lem:dual-add}
	Fix isogenies $f,g\colon(E,e)\to(E',e')$. Then $f^t+g^t=(f+g)^t$.
\end{proposition}
\begin{proof}
	Here, the addition of the morphisms $f+g$ is defined as the composite of
	\[E\xrightarrow{(f,g)}E'\times E'\xrightarrow mE'.\]
	By taking the base-change to the algebraic closure, we may check our equality of morphisms on closed points. So one hand, for some $x'\in E'(k)$, we see that
	\[\OO_E\left((e)-(f+g)^t(x')\right)=(f+g)^*\OO_{E'}((e')-(x'))=(f,g)^*m^*\OO_{E'}((e')-(x')),\]
	by definition of $(f+g)^t$. On the other hand, $f^t(x')+g^t(x')$ corresponds to the line bundle
	\[f^*\OO_{E'}((e')-(x'))\otimes g^*\OO_{E'}((e')-(x')).\]
	We would like to use \Cref{cor:square}, so we rewrite the above line bundle as
	\[(f,g)^*p_1^*\OO_{E'}((e')-(x'))\otimes(f,g)^*p_2^*\OO_{E'}((e')-(x')),\]
	so by comparing our two line bundles, it is enough to show that
	\[m_*\OO_{E'}((e')-(x'))\cong p_1^*\OO_{E'}((e')-(x'))\otimes p_2^*\OO_{E'}((e')-(x')),\]
	which holds by \Cref{cor:square}.
\end{proof}
\begin{proposition}
	Fix isogenies $f\colon(E,e)\to(E',e')$ and $g\colon(E',e')\to(E'',e'')$. Then $(g\circ f)^t=f^t\circ g^t$.
\end{proposition}
\begin{proof}
	Tracking through the definition of the dual isogeny, we see we are looking at the following large diagram.
	% https://q.uiver.app/?q=WzAsNixbMCwwLCJFJyciXSxbMSwwLCJFJyJdLFsyLDAsIkUiXSxbMCwxLCJcXFBpY197RScnfV4wIl0sWzEsMSwiXFxQaWNfe0UnfV4wIl0sWzIsMSwiXFxQaWNfRV4wIl0sWzMsNCwiZ14qIl0sWzQsNSwiZl4qIl0sWzAsMSwiZ150IiwyXSxbMSwyLCJmXnQiLDJdLFswLDIsIihnXFxjaXJjIGYpXnQiLDAseyJjdXJ2ZSI6LTJ9XSxbMCwzXSxbMSw0XSxbMiw1XSxbMyw1LCIoZ1xcY2lyYyBmKV4qIiwyLHsiY3VydmUiOjJ9XV0=&macro_url=https%3A%2F%2Fraw.githubusercontent.com%2FdFoiler%2Fnotes%2Fmaster%2Fnir.tex
	\[\begin{tikzcd}
		{E''} & {E'} & E \\
		{\Pic_{E''}^0} & {\Pic_{E'}^0} & {\Pic_E^0}
		\arrow["{g^*}", from=2-1, to=2-2]
		\arrow["{f^*}", from=2-2, to=2-3]
		\arrow["{g^t}"', from=1-1, to=1-2]
		\arrow["{f^t}"', from=1-2, to=1-3]
		\arrow["{(g\circ f)^t}", curve={height=-12pt}, from=1-1, to=1-3]
		\arrow[from=1-1, to=2-1]
		\arrow[from=1-2, to=2-2]
		\arrow[from=1-3, to=2-3]
		\arrow["{(g\circ f)^*}"', curve={height=12pt}, from=2-1, to=2-3]
	\end{tikzcd}\]
	Here, the bottom triangle commutes by properties of the pullback (we might only know this for the non-scheme-theoretic $\Pic^0$, but then we can just check the equality on closed points), so by definition of $(g\circ f)^t$, the above triangle also commutes, which is what we wanted.
\end{proof}
Let's now build towards $(f^t)^t=f$.
\begin{lemma} \label{lem:dual-mult}
	Fix an elliptic $k$-curve $(E,e)$. Then $[m]^t=[m]$ for any $m\in\ZZ$.
\end{lemma}
\begin{proof}
	We induct. For $m=1$, we can track around the usual diagram to see that ${\id_E}^t={\id_E}$, which finishes. For the inductive step, we use \Cref{lem:dual-add} to see
	\[[m\pm 1]^t=([m]\pm[1])^t=[m]^t\pm[1]^t=[m]\pm[1]=[m\pm1].\]
	As such, we see that we may induct up and down from the base case of $m=1$ to get any $m\in\ZZ$.
\end{proof}
\begin{lemma} \label{lem:mult-not-constant}
	Fix an elliptic $k$-curve $(E,e)$. For any $m>0$, the map $[m]\colon E\to E$ is non-constant.
\end{lemma}
\begin{proof}
	We begin by showing that it's enough to show that $\deg[2]>1$. Indeed, for any $m$, if $[m]$ is constant, then
	\[[n]=[m]+[n-m]=[n-m],\]
	so the maps are periodic$\pmod m$, so the degree of the maps is bounded. However, if $\deg[2]>1$, then $\deg\left[2^k\right]\to\infty$ as $k\to\infty$.

	We now work in characteristic not equal to $2$ or $3$, for concreteness. Then one can write $E\colon y^2=x(x-1)(x-\lambda)$ over the algebraic closure, so we have a map $\pi\colon E\to\PP^1_k$ given by $(x,y)\mapsto x$. Notably, there is an involution $\iota\colon E\to E$ given by $(x,y)\mapsto(x,-y)$ such that $\pi\circ\iota=\pi$, so we note
	\[E[2]=\{p\in E(k):2P=e\}=\{p\in E(k):\iota(P)=P\}.\]
	However, the orbit of $P$ under $\iota$ is exactly the pre-image of $\pi(P)\in\PP^1_k$, so above we are asking for $p\in E[2]$ if and only if $\pi^{-1}(\{\pi p\})$ is a single point.

	To continue, we note that the orbit of $\pi$ has degree $2$ because it corresponds to the field extension $k(x)\into k(x)[y]/\left(y^2-x(x-1)(x-\lambda)\right)$. As such, we can use the Riemann--Hurwitz formula to compute
	\[2g(E)-2=(\deg\pi)(2g(\PP^1_k)-2)+\sum_{p\in\PP^1_k}(e_p-1),\]
	where $e_p$ is the ramification index. (Notably, the above formula doesn't quite work in characteristic $2$.) This will complete the proof upon rearranging: we see that only four points will live in $E[2]$.
\end{proof}
\begin{corollary} \label{cor:deg-mult}
	Fix an elliptic $k$-curve $(E,e)$. Then $\deg[m]=m^2$ for any $m\in\ZZ$.
\end{corollary}
\begin{proof}
	By \Cref{prop:f-then-dual-is-mult} and \Cref{lem:dual-mult}, we see that
	\[[\deg[m]]=[m]\circ[m]^t=[m]\circ[m]=\left[m^2\right],\]
	so we finish. It follows from \Cref{lem:mult-not-constant} that we may say $m^2=\deg[m]$ because multiplication by $\left|\deg[m]-m^2\right|\ge0$ is constant and thus must just have $\left|\deg[m]-m^2\right|=0$.
\end{proof}
\begin{lemma} \label{lem:dual-degree}
	Fix an isogeny $f\colon(E,e)\to(E',e')$ of elliptic $k$-curves. Then $\deg f^t=\deg f$.
\end{lemma}
\begin{proof}
	We use \Cref{prop:f-then-dual-is-mult}. On one hand, we see
	\[\deg\left(f\circ f^t\right)=\deg f\cdot\deg f^t\]
	(one can see that degree is multiplicative like this by comparing the field extensions $K(E')\subseteq K(E)\subseteq K(E')$), but on the other hand, we see
	\[\deg\left([\deg f]\right)=[\deg f]^2\]
	by \Cref{cor:deg-mult}. Comparing our degrees finishes.
\end{proof}
\begin{proposition}
	Fix an isogeny $f\colon(E,e)\to(E',e')$ of elliptic $k$-curves. Then $\left(f^t\right)^t=f$.
\end{proposition}
\begin{proof}
	By combining \Cref{lem:dual-degree} and \Cref{prop:f-then-dual-is-mult}, we see that
	\[f\circ f^t=[\deg f]=\left[\deg f^t\right]=\left(f^t\right)^t\circ f^t.\]
	Cancelling on the right with \Cref{lem:cancel-right} completes the proof.
\end{proof}

\end{document}