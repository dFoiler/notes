% !TEX root = ../notes.tex

\documentclass[../notes.tex]{subfiles}

\begin{document}

\section{March 17}

We now move more directly towards the Mordell--Weil theorem.

\subsection{The Mordell--Weil Theorem}
Here is our statement.
\begin{restatable}[Mordell--Weil]{theorem}{mwthm} \label{thm:mw}
	Fix a number field $K$. Given an elliptic $K$-curve $(E,e)$, then the group $E(K)$ is finitely generated.
\end{restatable}
\noindent Our proof will take two steps. We will first show the following result.
\begin{restatable}[Weak Mordell--Weil]{theorem}{weakmwthm} \label{thm:weak-mw}
	Fix a number field $K$. Given an elliptic $K$-curve $(E,e)$, then the group $E(K)/rE(K)$ is finite for any $r\ge1$.
\end{restatable}
\noindent Then our second step will use some theory of heights to recover \Cref{thm:mw}.

So let's go at \Cref{thm:weak-mw}. It will be no surprise that our approach is cohomological. Let's describe the idea. Note $[m]\colon E\to E$ is a non-constant map by \Cref{lem:mult-not-constant}, so it is surjective (as a scheme map, so on geometric points for example), so one has an exact sequence
\[0\to E[m](\overline K)\to E(\overline K)\stackrel m\to E(\overline K)\to0.\]
Formally, $E[m]$ is the kernel of $[m]$, which we can view as the pre-image of $e$ under $[m]$, which is the following fiber product.
% https://q.uiver.app/?q=WzAsNCxbMCwwLCJFW21dIl0sWzEsMCwiZSJdLFswLDEsIkUiXSxbMSwxLCJFIl0sWzIsMywiW21dIl0sWzEsM10sWzAsMV0sWzAsMl0sWzAsMywiIiwxLHsic3R5bGUiOnsibmFtZSI6ImNvcm5lciJ9fV1d&macro_url=https%3A%2F%2Fraw.githubusercontent.com%2FdFoiler%2Fnotes%2Fmaster%2Fnir.tex
\[\begin{tikzcd}
	{E[m]} & e \\
	E & E
	\arrow["{[m]}", from=2-1, to=2-2]
	\arrow[from=1-2, to=2-2]
	\arrow[from=1-1, to=1-2]
	\arrow[from=1-1, to=2-1]
	\arrow["\lrcorner"{anchor=center, pos=0.125}, draw=none, from=1-1, to=2-2]
\end{tikzcd}\]
Thus, viewing everything as a module over $G\coloneqq\op{Gal}(\overline K/K)$, we get an inclusion
\[\frac{E(K)}{mE(K)}\to H^1\left(G,E[m](\overline K)\right).\]
As such, the game is to control the image in this map.

Let's spend a moment discussing our $H^1$. Roughly speaking, some algebra is able to show that, as long as $m$ is not divisible by $\op{char}K$ (which is true because $K$ is a number field), then $E[m](K)\cong(\ZZ/m\ZZ)^2$ after some base-change of $K$ to pick up all the $m$-torsion point. Then
\[H^1\left(G,(\ZZ/m\ZZ)^2\right)=\op{Hom}\left(G,(\ZZ/m\ZZ)^2\right)=\op{Hom}\left(G^{\mathrm{ab}},\ZZ/m\ZZ\right)^2,\]
so elements here are in bijection with pairs $(L,L')$ of cyclic extensions of degree $m$ over $K$. In particular, we expect this $H^1$ to be quite infinite.

Being more explicit now, suppose we have some $P\in E(K)$. Then we lift it to some $Q\in E(\overline K)$ such that $P=mQ$, and the corresponding element in $H^1(G,E[m](\overline K))$ is $\sigma\mapsto(\sigma-1)Q$ for any $\sigma$. However, we can do a little better by thinking about the pre-image of $P$ along $[m]\colon E\to E$ as fitting in the fiber product as follows.
% https://q.uiver.app/?q=WzAsNCxbMCwwLCJbbV1eey0xfVAiXSxbMSwwLCJQIl0sWzAsMSwiRSJdLFsxLDEsIkUiXSxbMiwzLCJbbV0iXSxbMSwzXSxbMCwxXSxbMCwyXSxbMCwzLCIiLDEseyJzdHlsZSI6eyJuYW1lIjoiY29ybmVyIn19XV0=&macro_url=https%3A%2F%2Fraw.githubusercontent.com%2FdFoiler%2Fnotes%2Fmaster%2Fnir.tex
\[\begin{tikzcd}
	{[m]^{-1}P} & P \\
	E & E
	\arrow["{[m]}", from=2-1, to=2-2]
	\arrow[from=1-2, to=2-2]
	\arrow[from=1-1, to=1-2]
	\arrow[from=1-1, to=2-1]
	\arrow["\lrcorner"{anchor=center, pos=0.125}, draw=none, from=1-1, to=2-2]
\end{tikzcd}\]
And notably, the class of $P$ in $H^1(G,E[m](\overline K))$ vanishes if and only if $T_P\coloneqq[m]^{-1}(P)$ has $K$-points.

Now, taking the equation for our elliptic curve $E$, one can clear denominators to make $E$ actually into a scheme over $\OO_K[1/(mN)]$ for $N$ large enough. In fact, one can extend the addition, identity, smoothness, properness to $E$ now as a curve over $\Spec\OO_K[1/(mN)]$. In fact, by the valuative criterion for properness will still extend to a point over $\Spec\OO_K[1/(mN)]$.
% https://q.uiver.app/?q=WzAsNCxbMSwwLCJFIl0sWzAsMCwiXFxTcGVjIEsiXSxbMSwxLCJcXFNwZWNcXE9PX0tbMS8obU4pXSJdLFswLDEsIlxcU3BlY1xcT09fS1sxLyhtTildIl0sWzMsMCwiIiwwLHsic3R5bGUiOnsiYm9keSI6eyJuYW1lIjoiZGFzaGVkIn19fV0sWzEsMF0sWzEsM10sWzMsMl0sWzIsMF1d&macro_url=https%3A%2F%2Fraw.githubusercontent.com%2FdFoiler%2Fnotes%2Fmaster%2Fnir.tex
\[\begin{tikzcd}
	{\Spec K} & E \\
	{\Spec\OO_K[1/(mN)]} & {\Spec\OO_K[1/(mN)]}
	\arrow[dashed, from=2-1, to=1-2]
	\arrow[from=1-1, to=1-2]
	\arrow[from=1-1, to=2-1]
	\arrow[from=2-1, to=2-2]
	\arrow[from=2-2, to=1-2]
\end{tikzcd}\]
Notably, one hopes that we can now control $E[m]$ via this sort of spreading out.

To finish up for today, suppose we have a scheme $S$ and a line bundle $\mc L$ over $S$ equipped with an isomorphism $\sigma\colon\mc L^{m}\to\OO_S$. By base-changing a little, we assume that $S$ is a scheme over $\ZZ[1/m,\zeta_m]$. Then we can consider
\[\Spec_S\left(\OO_S\oplus\mc L\oplus\mc L^2\oplus\cdots\oplus\mc L^{m-1}\right),\]
which is intended to look like $\OO_S[x]/\left(x^p-\mc L\right)$. The point is that the above scheme comes equipped with a $\ZZ/m\ZZ$-action by having a fixed generator $\gamma\in\ZZ/m\ZZ$ act by
\[\gamma\cdot(\ell_0,\ldots,\ell_{m-1})=\left(\ell_0,\zeta_m\ell_1,\ldots,\zeta_m^{m-1}\ell_{m-1}\right).\]
It turns out that we can go backward: given a pair $(\mc L,\sigma)$ as above, we can take this to $\mc L$ to produce an element in $\Pic(S)[m]$, and one can ask for the kernel of this map, but it's just given by the set of ways to assign isomorphisms $\OO_S^m\cong\OO_S$, but it is just the set $\OO_S^\times/\OO_S^{\times m}$.

\end{document}