% !TEX root = ../notes.tex

\documentclass[../notes.tex]{subfiles}

\begin{document}

\section{March 22}

Last class we finished proving the weak Mordell--Weil theorem (\Cref{thm:weak-mw}). Today we begin developing the theory of heights to prove the Mordell--Weil theorem (\Cref{thm:mw}).
\begin{remark}
	We do not expect $E(K)$ to be finitely generated when $K$ is merely a local field. Roughly speaking, if $K=\CC$, then we can describe our elliptic curve as $\CC/(\ZZ+\ZZ\tau)\cong\CC^\times/\exp(2\pi i\tau)^\ZZ$, which has infinitely many points. If $K=\QQ_p$, some similar story is possible; namely, sometimes one can parameterize the points as $K^\times/q^\ZZ$ where $q\in\mf m_K\OO_K$.
\end{remark}

\subsection{Heights}
Roughly speaking, we want to be generated by the ``smallest'' points on our elliptic curve, but this requires building a notion of ``smallest.'' This is the purpose of heights.
\begin{definition}[height]
	Fix an abelian group $A$. Then a function $h\colon A\to\RR$ is a \textit{height function} if and only if it satisfies the following properties.
	\begin{listalph}
		\item Additivity: for fixed $Q\in A$, there exists a constant $c_Q$ such that
		\[h(P+Q)\le 2h(P)+c_Q\]
		for any $P\in A$.
		\item Quadratic: there is $m\ge2$ and a constant $c_A$ such that
		\[h(mP)\ge m^2h(P)-c_A\]
		for any $P\in A$.
		\item Bounded: the set $\{P\in A:h(P)\le c\}$ is finite for each $c$.
	\end{listalph}
\end{definition}
Here is our result.
\begin{proposition}
	Fix an abelian group $A$ equipped with a height function $h\colon A\to\RR$. If $A/mA$ is finite for all $m\ge2$, then $A$ is finitely generated.
\end{proposition}
\begin{proof}
	Because our height function is quadratic, we can find representatives for $A/mA$ for the $m$ satisfying our quadratic condition; let the representatives be $Q_1,\ldots,Q_r$, and to help us later, we let $c_Q$ to be the maximum of $c_{Q_i}$ over all $Q_i$. Now, for any $P_0\in A$, we may write
	\[P_0=mP_1+Q_{i_1}\]
	for some $P_1$; repeating this process inductively, we see that we get
	\[P_k=mP_{k+1}+Q_{i_{k+1}}\]
	for each $k$. This is an issue because it looks like the height of $P$ is getting smaller and smaller unless this vanishes. In particular, for each $k\ge1$, we see
	\begin{align*}
		h(P_k) &\le \frac1{m^2}\left(h(mP_k)+c_A\right) \\
		&\le \frac1{m^2}\left(h(P_{k-1}-Q_{k-1})+c_A\right) \\
		&\le \frac1{m^2}\left(2h(P_{k-1})+c_Q+c_A\right).
	\end{align*}
	Working inductively, we see that
	\[h(P_n)\le\left(\frac2{m^2}\right)^nh(P_0)+\left(\frac1{m^2}+\frac2{m^4}+\cdots+\frac{2^{n-1}}{m^{2n}}\right)(c_Q+c_A)\]
	for any $n$. We now bound the geometric sum as
	\[h(P_n)<\left(\frac2{m^2}\right)^nh(P_0)+\frac{c_A+c_Q}{m^2-2}\le2^{-n}h(P)+\frac{c_A+c_Q}2.\]
	Thus, for $n$, large enough, we see that each $P_n$ has height bounded by $1+\frac{c_A+c_Q}2$, so we may go ahead and claim that
	\[\left\{P\in A:h(P)\le1+\frac{c_A+c_Q}2\right\}\cup\{Q_1,\ldots,Q_r\}\]
	will generate $A$. Indeed, for any $P_0\in A$, we run the above process, we see that $P_n$ has height bounded by $1+\frac{c_A+c_Q}2$ for $n$ large enough, so then we have
	\[P_0=Q_{i_1}+mQ_{i_2}+m^2Q_{i_3}+\cdots+m^nQ_{i_n}+m^nP_n,\]
	which completes the proof.
\end{proof}
\begin{remark}
	The rationals $\QQ^\times$ has a height function given by $\frac ab\mapsto\log(|b|)+\log(|a|)$ (where $b$ is chosen to be minimal), but $\QQ^\times$ is not finitely generated because $\QQ^\times/\QQ^{\times m}$ isn't finite for any $m\ge2$!
\end{remark}
So to prove \Cref{thm:mw}, it suffices to build a height function.
\begin{definition}
	Fix a number field $K$. Define the function $H_K\colon\PP^n_K\to\RR_{>0}$ by
	\[H_K([x_0:\ldots:x_n])=\prod_{v\in M_K}\left(\max\left\{|x_0|_v,\ldots,|x_n|_v\right\}^{n_v}\right)^{1/[K:\QQ]}\]
	where $M_K$ is the set of places of $K$ defined to extend the standard absolute values, and $n_v\coloneqq[K_v:\QQ_p]$ where $v$ lies over $p$. Then we define $h\coloneqq\log H_K$ to be a function $\PP^n_K\to\RR$.
\end{definition}
\begin{remark}
	Let's take a moment to explain this definition. Suppose we have a $K$-point in $\PP^n_K$. This amounts to a morphism $K\to\PP^n_\ZZ$, which is equivalent data to a line bundle $\mc L$ on $K$ with generating sections $(x_0,\ldots,x_n)$. By clearing denominators in the $x_\bullet$, this provides a surjection $\OO_K^{\oplus(n+1)}\to\mc L$. In total, we are being given two different projective modules $L$ and $\OO_K$ sitting in $L\otimes_{\OO_K}K$, and then our height $h$ is essentially measuring the difference between these.
\end{remark}
\begin{lemma}
	Fix a number field $K$. The function $H\coloneqq H_K$ is well-defined.
\end{lemma}
\begin{proof}
	To begin, we remark that the product defining $H$ is a finite product because any $x_\bullet\in K$ is going to be a unit in all but finitely many $v$, so only finitely many factors of the product are not equal to $1$. So we can at least evaluate $H$ on a vector $(x_0,\ldots,x_n)$.
	
	Now, given $\lambda\in K^\times$, we must check that $H(x_0,\ldots,x_n)=H(\lambda x_0,\ldots,\lambda x_n)$. Well,
	\begin{align*}
		H(\lambda x_0,\ldots,\lambda x_n) &= \prod_{v\in M_K}\left(\max\left\{|\lambda x_0|_v,\ldots,|\lambda x_n|_v\right\}^{n_v}\right)^{1/[K:\QQ]} \\
		&= \prod_{v\in M_K}|\lambda|_v^{n_v/[K:\QQ]}\cdot\prod_{v\in M_K}\left(\max\left\{|\lambda x_0|_v,\ldots,|\lambda x_n|_v\right\}^{n_v}\right)^{1/[K:\QQ]}.
	\end{align*}
	The product formula tells us that the $\lambda$ term vanishes, so we are done.
\end{proof}
\begin{lemma}
	Fix an extension of number fields $L/K$. Then the diagram
	% https://q.uiver.app/?q=WzAsMyxbMCwwLCJcXFBQXm5fSyJdLFsxLDAsIlxcUFBebl9MIl0sWzEsMSwiXFxSUiJdLFswLDFdLFsxLDIsIkhfTCJdLFswLDIsIkhfSyIsMl1d&macro_url=https%3A%2F%2Fraw.githubusercontent.com%2FdFoiler%2Fnotes%2Fmaster%2Fnir.tex
	\[\begin{tikzcd}
		{\PP^n_K} & {\PP^n_L} \\
		& \RR
		\arrow[from=1-1, to=1-2]
		\arrow["{H_L}", from=1-2, to=2-2]
		\arrow["{H_K}"', from=1-1, to=2-2]
	\end{tikzcd}\]
	commutes, where the map $\PP^n_k\subseteq\PP^n_L$ is induced by $K\subseteq L$.
\end{lemma}
\begin{proof}
	Track around the definitions and the definitions of our absolute values.
\end{proof}
\begin{lemma}
	Fix a number field $K$. Then $\im H_K\subseteq[1,\infty)$.
\end{lemma}
\begin{proof}
	Any vector $[x_0:\ldots:x_n]\in\PP^n_K$ can be scaled so that one of the $x_i$ is equal to $1$ (by placing it in a distinguished affine open subscheme), but then
	\[\max\{|x_0|_v,\ldots,|x_n|_v\}\ge1\]
	for each $v\in M_K$, so the entire produce must be bounded below by $1$, so $H_K([x_0:\ldots:x_n])\ge1$ follows.
\end{proof}

\end{document}