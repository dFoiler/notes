% !TEX root = ../notes.tex

\documentclass[../notes.tex]{subfiles}

\begin{document}

\section{March 6}

Good morning everyone.

\subsection{Elliptic Curves as Cubics}
Fix an elliptic $k$-curve $(E,e)$. We want to view $(E,e)$ as a planar curve of degree $3$. Here is our claim.
\begin{proposition}
	Fix an elliptic $k$-curve $(E,e)$. Then the line bundle $\OO_E(3e)$ determines a closed embedding $E\into\PP^2_k$ of degree $3$. In particular, we factor through $V(F)$ for some homogeneous cubic polynomial $F$.
\end{proposition}
\begin{proof}
	We show our checks in sequence.
	\begin{itemize}
		\item Note that $h^0(E,\OO_E(3e))=3$ by \Cref{ex:use-rr}, so we will induce a projective morphism $E\to\PP^2_k$ by choosing our three basis vectors. Equivalently, we can choose these three basis vectors as a surjective map $\OO_E^3\onto\OO_E$. In some sense, it is more natural to think about our projective morphism as $E\to\PP\Gamma(E,\OO_E(3e))$ sending $p\in E$ to the quotient map $\Gamma(E,\OO_E(3e))\to\OO_E(3e)_p/\mf m_p$. In particular, this quotient map uniquely determines an element of $\PP\Gamma(E,\OO_E(3e))$ due to the choice of basis of the one-dimensional $k$-vector space $\OO_E(3e)_p/\mf m_p$.

		The quick way to show that we are very ample is to note that any two points $p,q\in E$ grant
		\[h^0(E,\OO_E(3e-p-q))=1=h^0(E,\OO_E(3e))-2,\]
		so the corresponding projective morphism separates points and tangent vectors and therefore induces a closed embedding.

		Technically we ought to show that $\OO_E(3e)$ is very ample. To show that we are generated by global sections, we need surjectivity of the corresponding map on points given by
		\[\Gamma(E,\OO_E(3e))\otimes_kk(p)\to\OO_E(3e)_p/\mf m_p.\]
		Really, we need the map to be nonzero because the target is one-dimensional. Note that by base-changing $E$ to $E_{k(p)}$, we can assume that $p$ is a $k$-point. As such, we note that we have the exact sequence
		\[0\to\OO_E(3e-p)\to\OO_E(3e)\to\OO_E(3e)_p/\mf m_p\to0\]
		by tensoring up the exact sequence $0\to\mc I_p\to\OO_E\to k(p)\to0$ with the locally free and hence flat sheaf $\OO_E$. Taking global sections produces the exact sequence
		\[0\to\Gamma(E,\OO_E(3e-p))\to\Gamma(E,\OO_E(3e))\to\OO_E(3e)_p/\mf m_p,\]
		but \Cref{ex:use-rr} tells us that $\dim_k\Gamma(E,\OO_E(3e-p))=2<3=\dim_k\Gamma(E,\OO_E(3e))$, so the kernel is not full, so our map is nonzero, which is what we wanted.

		\item We now check that our projective morphism is a closed embedding. For this, we must check that we separate points and tangent vectors. Because $E$ is a proper scheme (it's projective), to separate points, it is enough to check that two points go to different places in our projective space. Well, checking where two points $p,q\in E$ land, we are claiming that we are producing the same quotient map
		\[\Gamma(E,\OO_E(3e))\onto\OO_E(3e)_p/\mf m_p\qquad\text{and}\qquad\Gamma(E,\OO_E(3e))\onto\OO_E(3e)_q/\mf m_q.\]
		By a base-change of $E$, we may again assume that our points are $k$-rational. Now, above we computed the kernel of this map, so we would be requiring
		\[\Gamma(E,\OO_E(3e-p))\cap\Gamma(E,\OO_E(3e-q))=\Gamma(E,\OO_E(3e-p-q))\]
		to be $2$-dimensional, but in fact this is $1$-dimensional by \Cref{ex:use-rr}, so there is nothing to say here.

		Now, to separate tangent vectors, we want to see that the map
		\[\Gamma(E,\OO_E(3e))\to\Gamma(E,\OO_E(3e)/(\mc I_p\mc I_q))\]
		is surjective, but again our dimensions jump appropriately by \Cref{ex:use-rr}, so we must be surjective.

		\item Choose a basis for $V\coloneqq\Gamma(E,\OO_E(3e))$ named $\{u,v,w\}$, so $\PP\Gamma(E,\OO_E(3e))\cong\PP^2_k$ with basis given by $\{u,v,w\}$. We now note that we have the inclusions
		\[\Gamma(E,\OO_E)=\Gamma(E,\OO_E(e))\subsetneq\Gamma(E,\OO_E(2e))\subsetneq\Gamma(E,\OO_E(3e))\]
		by \Cref{ex:use-rr}. As such, we let $\{z\}$ denote a basis of $\Gamma(E,\OO_E)$, and then we extend it to a basis $\{z,x\}$ of $\Gamma(E,\OO_E(2e))$, and again we extend it to a basis $\{z,x,y\}$ of $\Gamma(E,\OO_E(3e))$. Going up further require some more care.
		\begin{itemize}
			\item We see $\Gamma(E,\OO_E(4e))$ has basis $\{z,x,y,x^2\}$, which are linearly independent because they have different valuations at $e$.
			\item Similarly, we see $\Gamma(E,\OO_E(5e))$ has basis $\left\{z,x,y,x^2,xy\right\}$.
			\item However, $\Gamma(E,\OO_E(6e))$ has basis $\left\{z,x,y,x^2,xy,y^2,x^3\right\}$, but we have dimension $6$, so there must be a relation now.
		\end{itemize}
		Thus, we get to write down a relation between $\left\{z,x,y,x^2,xy,y^2,x^3\right\}$, which after multiplying through by the ``scalar'' $z$ enough times grants us a homogeneous polynomial $F\in k[x,y,z]$ of degree $3$ dictating this relation.

		As such, for each $p\in X$, we see that $F\in\Gamma(E,\OO_E(6e))$ will vanish in $\OO_E(6e)_p/\mf m_p$, so it follows from the construction of the map $E\to\PP V$ that the image lands in $V(F)$.
		\qedhere
	\end{itemize}
\end{proof}

\end{document}