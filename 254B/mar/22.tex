% !TEX root = ../notes.tex

\documentclass[../notes.tex]{subfiles}

\begin{document}

Today we continue discussing endomorphism algebras.

\subsection{More on the Albert Classification}
We begin by discussing \Cref{thm:albert}. For our notation, $D$ is a simple algebra with center $K$ and positive involution $(\cdot)^\dagger$ so that $K^+=K^\dagger$. We also set $d\coloneqq\sqrt{[D:K]}$ and $e\coloneqq[K:\QQ]$ and $e_0\coloneqq[K^+:\QQ]$.
\begin{remark}
	In the case of a simple abelian $k$-variety $A$ of dimension $g$ with $D\coloneqq\op{End}^0(A)$, then we get the following data.
	\[\begin{array}{c|cc|cc|c}
		\text{Type} & [K^+:\QQ] & \sqrt{[D:K]} & {\op{char}}=0 & {\op{char}k}>0 & \dim_\QQ\op{NS}(A_{\ov k})_\QQ/\dim_\QQ\op{End}(A_{\ov k})_\QQ\\\hline
		\text{I}	&  e_0 & 1 &      e\mid g & e\mid g & 1 \\
		\text{II}	&  e_0 & 2 &     2e\mid g & 2e\mid g & 3/4 \\
		\text{III}	&  e_0 & 2 &     2e\mid g & e\mid g & 1/3 \\
		\text{IV}	& 2e_0 & d & e_0d^2\mid g & e_0d\mid g & 1/2
	\end{array}\]
	We will explain where these restrictions come from later. Do note that we do not if all possible simple algebras $D$ with positive involution $(\cdot)^\dagger$ come from simple abelian varieties. In characteristic $0$, we know exactly what occurs, due to Albert and Shimura. (Shimura, notably, used the geometry of the moduli space $\mc A_g$.)
\end{remark}
\begin{remark}
	Let's explain what's going on with the N\'eron--Severi group. This is occurring when $k$ is algebraically closed, and we pick a polarization $\lambda\colon A\to A^\lor$. Now, we have our embedding
	\[\op{NS}(A)\otimes_\ZZ\QQ\to\op{Hom}(A,A^\lor)\otimes_\ZZ\QQ\to\op{End}^0(A),\]
	where $\op{NS}(A)$ consists of the symmetric homomorphisms. Notably, being a symmetric homomorphism $\lambda'\colon A\to A^\lor$ means that $f=f^\dagger$, where $f\in\op{End}^0(A)$ is the isogeny such that $\lambda'=\lambda\circ f$. Indeed, we are asking for $\lambda\circ f=f^\lor\circ\lambda$ in order to be symmetric, which amounts to $f=f^\dagger$. So the point is that
	\[\op{NS}(A)\otimes_\ZZ\QQ\cong\op{End}^0(A)^\dagger.\]
\end{remark}
Let's see an example of a dimension restriction.
\begin{lemma}
	Fix a simple abelian $k$-variety $A$, where $\op{char}k=0$. Then set notation as above with $D\coloneqq\op{End}^0(A)$. Then $d^2e\mid2g$.
\end{lemma}
\begin{proof}
	In characteristic zero, the Lefschetz principle allows us to assume that $k\subseteq\CC$. Then we are granted that $\op{End}^0(A)\subseteq\op{End}^0(A_\CC)$ will act faithfully on $H^1(A(\CC),\QQ)$, meaning that $d^2e\mid 2g$ in order for the dimensions to check on.
\end{proof}
This provides the dimension restrictions in types II--IV.
\begin{lemma} \label{lem:general-degree-albert-bound}
	Fix a simple abelian $k$-variety $A$ for any field $k$. Then set notation as above with $D\coloneqq\op{End}^0(A)$. Then $de\mid2g$.
\end{lemma}
\begin{proof}
	The point is that $\deg\colon\op{End}^0(A)\to\QQ$ is a polynomial of degree $2g$, and in fact we showed earlier that $\deg$ is a norm form. But \Cref{prop:classify-norm-trace-form} tells us that
	\[\deg=\left({\op N_{K/\QQ}}\circ{\op N^\circ_{D/K}}\right)^i\]
	for some integer $i$. Computing the degree of the polynomials everywhere, we get that $dei=2g$ for some integer $i$, which is what we needed.
\end{proof}
This provides the dimension restrictions in types III--IV.
\begin{proposition}
	Fix a simple abelian $k$-variety $A$ for any field $k$. Then set notation as above with $D\coloneqq\op{End}^0(A)$. Further, suppose that $L$ is a subfield of $D$ fixed by $(\cdot)^\dagger$. Then $[L:\QQ]\mid g$.
\end{proposition}
\begin{proof}
	The point is that $L\subseteq\op{NS}(A_{\ov k})$ as discussed before. Now, choose a polarization $\lambda\colon A\to A^\lor$ so that $\lambda=\varphi_\mc L$; we also define $f\colon\op{NS}(A_{\ov k})_\QQ\to\QQ$ by
	\[f(\varphi_\mc M)\coloneqq\frac{\chi(\mc M)}{\chi(\mc L)},\]
	so \Cref{prop:deg-is-euler-char} tells us $f^2$ is a norm form. Namely, we know that $f(ab)=\pm f(a)f(b)$; an argument on the coefficients of our polynomial is able to show that we either have $f(ab)=+f(a)f(b)$ always or $f(ab)=-f(a)f(b)$ always. However, taking $\mc M=\mc L$, we see that the sign $+$ is forced, so $f$ is a norm form on $L$ of degree $g$! Arguing as in \Cref{lem:general-degree-albert-bound} completes.
\end{proof}
In the Type I case, one is able to take $L=K(\alpha)$ for suitable choice of $\alpha$ makes $K(\alpha)/K$ a degree-$2$ extension, providing the needed bounds for Type I. We won't discuss this in more detail.

Anyway, let's provide some examples.
\begin{example}
	Fix an elliptic curve $E$ so that $g=1$. We work in characteristic $0$. We see we may only have Type I with $e=1$ or Type IV with $d=e_0=1$, meaning that $\op{End}^0(E)$ is an imaginary quadratic extension of $\QQ$ so that $E$ has complex multiplication.
\end{example}
\begin{example}
	Fix an elliptic curve $E$ so that $g=1$ in characteristic $p>0$.
	\begin{itemize}
		\item It looks like we might be able to have Type I, which forces $e=1$; however, this does not happen over $\FF_q$ or even $\ov{\FF_q}$ by \Cref{rem:tate-conjecture}. (This does happen over $\FF_p(t)$.)
		
		\item We can still have Type IV, meaning that $e_0=d=1$, so $\op{End}^0(E)$ is an imaginary quadratic field; one can achieve this by finding an elliptic curve over $\QQ$ with ordinary reduction at some prime $p$ so that the Frobenius endomorphism $\op{Frob}$ fails to be in $\ZZ$.
		
		\item Lastly, it is possible to have Type III, which means that $e=e_0=1$, but we still must have $d\mid2g$, and $d=1$ is already considered above, so we actually have $d=2$ here. This means that $D\coloneqq\op{End}^0(\QQ)$ is a central simple $\QQ$-algebra, and \Cref{thm:albert} requires it to be $\HH$ at $\infty$. For finite $\ell\ne p$, we also see
		\[D\otimes_\QQ\QQ_\ell\subseteq\op{End}_{\op{Gal}(\ov k/k)}(T_\ell A\otimes_\ZZ\QQ_\ell).\]
		But both sides here have dimension $4$, so we must have $D\otimes_\QQ\QQ_\ell\cong M_2(\QQ_\ell)$, meaning that $D$ splits at all these finite primes $\ell$. The fundamental exact sequence now forces ramification at $p$.
	\end{itemize}
\end{example}
\begin{example}
	Fix an abelian surface $A$ so that $g=2$. We work in characteristic $0$.
	\begin{itemize}
		\item Type I is possible; an open subset of the moduli space has $e=1$ so that $\op{End}^0(A)=\QQ$, but it is still possible to have $e=2$ so that $\op{End}^0(A)$ is a real quadratic field.
		\item Type II is possible, but this forces $e=1$ so that $D$ is a quaternion $\QQ$-algebra. \Cref{thm:albert} requires $D$ to split at $\infty$; every quaternion algebra appears.
		\item Type III forces $e=1$, and Shimura shows that this never happens.
		\item For Type IV, one can have $e_0=2$ so that $A$ is an abelian surface with complex multiplication. Otherwise, $e_0=1$, this does not happen when $k$ is algebraically closed: we may take $k=\CC$, but then $\op{End}^0(A)$ contains a product of two imaginary quadratic fields, which forces $A$ to be isogenous to a product of elliptic curves, meaning that $A$ it not simple. However, it is possible that $e_0=1$ in general; for example, the Jacobian of $y^8=x(x-1)$ modulo the Jacobian of $y^4=x(x-1)$ over $\QQ$ will work.
	\end{itemize}
\end{example}

\end{document}