% !TEX root = ../notes.tex

\documentclass[../notes.tex]{subfiles}

\begin{document}

Here we go.

\subsection{Torsion as Finite Flat Group Scheme}
The finite group $k$-schemes of interest to us are of the form $A[n]$ where $n$ is an integer. \Cref{rem:decompose-a-torsion} tells us that the particularly bad case is $A[p^\nu]_{\ov k}$ where $p\coloneqq\op{char}k$ is positive. We know that this will have no \'etale-\'etale part, so it remains to find the remaining parts. Let's use \Cref{rem:classify-pieces-of-group-scheme} to take care of some of these.
\begin{itemize}
	\item We know that there will be some \'etale-local part of the form $(\ZZ/p^m\ZZ)^r$ (one needs to induct on $m$). Here, $r$ is the $p$-rank.
	\item By duality, we have some local-\'etale part of the form $\mu_{p^m}^s$ (again, one needs to induct on $m$).
\end{itemize}
We would like for $r=s$. This requires the following result.
\begin{proposition}
	Fix abelian $k$-varieties $A$ and $B$ of $p$-rank $r_A$ and $r_B$. If $A$ and $B$ are isogenous, then $r_A=r_B$.
\end{proposition}
\begin{proof}
	Let $f\colon A\to B$ be an isogeny, and let $n$ be the order of $\ker f$. Now, we see that $f$ restricts to a group homomorphism $A[p^m](\ov k)\to B[p^m](\ov k)$ with kernel of size at most $n$, so in light of our kernel having order $n$, we see that
	\[p^{mr_A}\le np^{mr_B}\]
	for all integers $m$. Sending $m\to\infty$ forces $r_A\le r_B$; by symmetry, we get the other inequality, so we are done.
\end{proof}
So we see that $r=s$ because $A$ and $A^\lor$ are isogenous.

\subsection{Local Finite Flat Group Schemes}
It remains to study the local-local piece. This is harder. We pick up the following definition.
\begin{definition}[height one]
	Fix a field $k$ of characteristic $p>0$. A finite commutative local $k$-group scheme $G$ is of \textit{height one} if and only if $x^p=0$ for all $x\in\mf m$, where $\mf m$ is the maximal ideal at $e_G\in G$.
\end{definition}
The point of being height one is that its Lie algebra. To understand the Lie algebra, we need to discuss differentials.
\begin{definition}[differential]
	Fix an $S$-group scheme $G$ of finite type. Then $\Omega^1_{G/S}$ is the \textit{sheaf of differentials} defined so that
	\[\op{Hom}_{\OO_G}\left(\Omega^1_{G/S},\mc F\right)=\op{Der}_S(\OO_G,\mc F)\]
	for any quasicoherent sheaf $\mc F$ on $S$. Here, $\op{Der}_S(\OO_G,\mc F)$ refers to the $\OO_S$-differentials $\delta\colon\OO_G\to\mc F$, which are additive maps vanishing on $f^{-1}\OO_S\subseteq\OO_G$ and satisfying the Leibniz rule.
\end{definition}
And here is our Lie algebra.
\begin{definition}[Lie algebra]
	Fix an $S$-group scheme $G$ of finite type. Then the \textit{Lie algebra} is the set of left-invariant differentials in $\op{Der}_S(\OO_G,\OO_G)$, which is canonically identified with $\op{Hom}_{\OO_G}(\Omega_{G/S},\OO_G)$.
\end{definition}
\begin{remark}
	Fix an $S$-group scheme $G$ of finite type. Then there is a natural isomorphism $\op{Lie}G\cong T_eG$ given by sending a differential $D$ to its restricted vector $D|_e$. We refer to \cite[pp.~92--94]{mumford} for the proof; the idea is to construct an inverse map by using right translates of $D|_e$ to build $D$.
\end{remark}
Anyway, here is our ``classification'' result, which at least gives us the coordinate ring.
\begin{lemma}
	Fix a field $k$ of characteristic $p>0$. Fix a finite local $k$-group scheme $G$ of height $1$ with coordinate ring $R$. Then
	\[R\cong\frac{k[x_1,\ldots,x_n]}{\left(x_1^p,\ldots,x_n^p\right)}\]
	for some $n$. In particular, $\dim_kR$ is a power of $p$.
\end{lemma}
\begin{proof}
	Fix $x_1,\ldots,x_n\in\mf m$ which form a $k$-basis of $\mf m/\mf m^2$. Because $G$ is local, this extends to a surjection $k[x_1,\ldots,x_r]\to R$. Being height one tells us that we now get a surjection
	\[\frac{k[x_1,\ldots,x_n]}{\left(x_1^p,\ldots,x_n^p\right)}\to R.\]
	We would like this to be an isomorphism. For this, we will want to show that no monomial with powers less than $p$ vanishes in $R$. (This is enough because any polynomial relation among the variables can multiply through by various $x_\bullet$s in order to derive that a monomial must equal zero; we are crucially using that $x_\bullet^p=0$ already.)

	We now use the Lie algebra. Let $D_1,\ldots,D_n\in\op{Lie}G$ be differentials providing a dual basis for $\ov x_1,\ldots,\ov x_n\in\mf m/\mf m^2$. Lifting this back up to $R$ tells us that
	\[D_j\prod_{i=0}x_i^{n_i}\ne0\]
	where $0\le n_i<p$ and $n_j\ne0$. But now any monomial being zero must have all exponents equal zero by applying the various $D_\bullet$s, so it remains to see that $1\ne0$.
\end{proof}
\begin{remark}
	Without the hypothesis on height, one needs to allow modding out by terms of the form $x_i^{p^\bullet}$.
\end{remark}
Let's continue discussing the Lie algebra.
\begin{definition}[Lie bracket]
	Fix an $S$-group scheme $G$ of finite type. Then there is a \textit{Lie bracket} given by
	\[[D_1,D_2]\coloneqq D_1D_2-D_2D_1\]
	for any derivations $D_1,D_2\in\op{Lie}G$.
\end{definition}
\begin{remark}
	If $G$ is a $k$-group scheme of finite type, then if $p\coloneqq\op{char}k$ is positive, then
	\[D^{\circ p}=\underbrace{D\circ\cdots\circ D}_p\]
	still lives in $\op{Lie}G$. Certainly this is additive and $G$-linear and vanishes on $k$, so it remains to check the Leibniz rule. The point is that one can expand out the Leibniz rule $p$ times as
	\[D^{\circ p}(ab)=\sum_{i+j=p}\binom piD^{\circ i}(a)D^{\circ j}(b),\]
	but with $p=\op{char}k$, all terms except the ending ones vanish, giving the Leibniz rule.
\end{remark}
The above remark motivates the following definition.
\begin{definition}
	Fix a field $k$ of positive characteristic $p>0$. Then a \textit{$p$-Lie algebra} is a Lie algebra $\mf g$ equipped with bracket $[\cdot,\cdot]$ as well as an endomorphism $(-)^{(p)}\colon\mf g\to\mf g$ satisfying the following.
	\begin{itemize}
		\item $(\lambda x)^{(p)}=\lambda^px^{(p)}$.
		\item The adjoint map $(\op{ad}x)\colon y\mapsto[x,y]$ satisfies $\op{ad}x^{(p)}=(\op{ad}x)^{(p)}$.
		\item One has $(x+y)^{(p)}=x^{(p)}+y^{(p)}+F_p(\op{ad}x,\op{ad}y)y$, where $F_p$ is some non-commutative polynomial which we will not write down.
	\end{itemize}
\end{definition}
We feel okay not writing down the polynomial $F_p$ because, in our setting, everything is commutative, so the Lie bracket vanishes, and it will be enough to remark that the relevant term always vanishes.

At long last, we note that we have the following result, explaining our remark earlier that height one means that it is enough to study the Lie algebra.
\begin{theorem} \label{thm:classify-height-one}
	The category of finite local $k$-group schemes of height one is equivalent to the category of $p$-Lie algebras over $k$.
\end{theorem}
\begin{proof}
	See \cite[III.14]{mumford}. Morally, the point is to recover the group $G$ from its $p$-Lie algebra $\mf g$. Well, one simply takes the universal enveloping algebra and quotients out by some extra relations arising from being a $p$-Lie algebra.
\end{proof}
For our application, we will want the following morphism.
\begin{definition}[relative Frobenius]
	Fix a group $k$-scheme $G$ of finite type, and let $F_G\colon G\to G$ be the absolute Frobenius given by taking $p$th powers. Then we define the \textit{relative Frobenius} $F_{G/\Spec k}\colon G\to G^{(1)}$ as the map of $k$-schemes making the following diagram commute, where the square is a pullback.
	% https://q.uiver.app/#q=WzAsNSxbMSwxLCJHXnsoMSl9Il0sWzIsMSwiRyJdLFsxLDIsImsiXSxbMiwyLCJrIl0sWzAsMCwiRyJdLFsyLDMsIkZfayJdLFswLDFdLFswLDJdLFsxLDNdLFs0LDIsIiIsMSx7ImN1cnZlIjoyfV0sWzQsMSwiRl9HIiwwLHsiY3VydmUiOi0yfV0sWzQsMCwiRl57KDEpfSIsMV1d&macro_url=https%3A%2F%2Fraw.githubusercontent.com%2FdFoiler%2Fnotes%2Fmaster%2Fnir.tex
	\[\begin{tikzcd}
		G \\
		& {G^{(1)}} & G \\
		& k & k
		\arrow["{F_k}", from=3-2, to=3-3]
		\arrow[from=2-2, to=2-3]
		\arrow[from=2-2, to=3-2]
		\arrow[from=2-3, to=3-3]
		\arrow[curve={height=12pt}, from=1-1, to=3-2]
		\arrow["{F_G}", curve={height=-12pt}, from=1-1, to=2-3]
		\arrow["{F^{(1)}}"{description}, from=1-1, to=2-2]
	\end{tikzcd}\]
\end{definition}
\begin{example}
	Take $G\coloneqq\mathbb G_{a,k}$. Then $\alpha_p=\ker F^{(1)}$.
\end{example}
\begin{remark}
	We note that $F^{(1)}$ is a group homomorphism by just writing out the relevant diagrams and noting that uniqueness of everything must make our diagrams commute. In fact, $\ker F^{(1)}$ is a finite local $k$-group scheme of height $1$! Indeed, the point is that $F^{(1)}$ is purely inseparable (by construction), making $\ker F^{(1)}$ local, and then we know
	\[\Gamma(\ker F^{(1)},\OO_{\ker F^{(1)}})=\frac{\OO_{G,e}}{\left\{x^p:x\in\mf m_{G,e}\right\}}.\]
\end{remark}
\begin{corollary}
	Fix a commutative finite group $k$-scheme $G$ of height $1$. Then the map $[p]\colon G\to G$ is the zero map.
\end{corollary}
\begin{proof}
	Note that $p$ vanishes on $\op{Lie}G$, from which the result follows from using the inverse functor of \Cref{thm:classify-height-one}.\todo{Why is Lie faithful?}
\end{proof}
\begin{corollary}
	Fix a commutative finite group $k$-scheme $G$ of order $m$. Then $[m]\colon G\to G$ is the zero map.
\end{corollary}
\begin{proof}
	It suffices to check the result on $\ov k$. Group theory will give the result for any \'etale part of $G$, so we may assume that $G$ is local and in particular has order $p^n$. Now, we note that we can build the composite of relative Frobenius maps
	\[G\to G^{(1)}\to G^{(2)}\to\cdots\to G^{(n)}.\]
	This produces injections $\ker F^{(1)}\subseteq\ker F^{(2)}\subseteq\cdots$ until $\ker F^{(n)}=G$. (Namely, one can see that if any two kernels are the same, then they must stabilize, but if they are all supposed to be distinct up until $G$, so we get this result.) But each quotient becomes a finite group $k$-scheme killed by $[p]$, so $\ker F^{(n)}$ will be killed by $[p^n]$, and we are done.\todo{What?}
\end{proof}

\end{document}