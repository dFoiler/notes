% !TEX root = ../notes.tex

\documentclass[../notes.tex]{subfiles}

\begin{document}

Here we go.

\subsection{More on the Weil Pairing}
We quickly provide a more explicit description of the Weil pairing. As before, choose geometric points $x\in A[n](\ov k)$ and $\mc L\in A^\lor[n](\ov k)$. Smoothness of our abelian $k$-variety $A$ allows us to realize any line bundle $\mc L\in A^\lor$ as $\OO_A(D)$ for some Weil divisor $D$. Note that $\OO_A(D)\subseteq\mc K_A$, where $\mc K_A$ is the sheaf of rational functions on $A$.

Now, computation of the Weil pairing requires us to choose an isomorphism $\beta\colon[n]^*\mc L\to\OO_A$. Well, we note that $[n]^*\mc L$ embeds into $[n]^*\mc K_A$, which is isomorphic to $\mc K_A$, so we choose the rational function $g\coloneqq[n]^*i\circ\beta^{-1}(1)$ so that $\op{div}g^{-1}=[n]^{-1}D$\todo{Why?} by tracking through what a pole or zero could be. Then we can compute
\[e_n(x,\mc L)=t_x^*\beta\circ\beta^{-1}=\frac{g(z+x)}{g(z)},\]
which is a number independent of the choice of $z$.

We now use this computation for some fun and profit.
\begin{proposition}
	Fix an abelian $k$-variety $A$. For any line bundle $\mc L$ on $A$, the composite map
	\[T_\ell A\times T_\ell A\xrightarrow{{\id}\times\varphi_\mc L}T_\ell A\times T_\ell A^\lor\xrightarrow{e_{\ell^\infty}}\ZZ_\ell(1)\]
	is skew-symmetric. In particular, if $\varphi_\mc L$ is a polarization, then this pairing is symplectic.
\end{proposition}
\begin{proof}
	We already know that the pairing is non-degenerate by a direct computation,\todo{Why?} so it remains to show that $(x,x)$ goes to $0$. Namely, we want to show that
	\[e_{\ell^\infty}(x,T_\ell(\varphi_\mc L)x)=0\]
	for each $x$. It suffices to show this result for all $x\in A[\ell^n]$ for any $n$ by taking the limit as $n\to\infty$.

	For this, we use our prior explicit description of the Weil pairing. Well, write $\mc L=\OO_A(D)$ for some Weil divisor $D$, and we compute
	\[\varphi_\mc L(x)=t_x^*\mc L\otimes\mc L^{-1}=\OO_A(t_{-x}D-D).\]
	We now choose $g$ as in the explicit description as above so that $\op{div}g^{-1}=[\ell^n]^{-1}(t_{-x}D-D)$. Now, to show that our Weil pairing vanishes, we want to show that $g(z+x)=g(z)$ for any given $z$. Well, for any $y\in A(\ov k)$ such that $\ell^ny=x$ (which exists by divisibility), we note
	\[\op{div}g^{-1}=t_{-y}\left([\ell^n]^{-1}D\right)-[\ell^n]^{-1}D,\]
	so telescoping implies
	\[\op{div}\prod_{i=0}^{\ell^n-1}t_{iy}^*\left(g^{-1}\right)=t_{-x}\left([\ell^n]^{-1}D\right)-[\ell^n]^{-1}D,\]
	but this vanishes because $x$ is $\ell^n$-torsion. Thus, our left-hand side is a function $h$ with no zeroes or poles, so it must be a constant function. For example, $h(z+y)=h(z)$, which unwinds to $g(z+x)=g(z)$ by another telescoping argument.
\end{proof}
\begin{remark}
	In fact, the above argument works for any polarization $\varphi\colon A\to A^\lor$. The passage to the algebraic closure is not so bad because.
\end{remark}
We now pick up the following result.
\begin{theorem}
	Fix a homomorphism $\varphi\colon A\to A^\lor$ of abelian $k$-varieties. Then the following are equivalent.
	\begin{listroman}
		\item $\varphi$ is symmetric.
		\item The pairing $E^\varphi$ induced by
		\[T_\ell A\times T_\ell A\xrightarrow{{\id}\times\varphi}T_\ell A\times T_\ell A^\lor\to\ZZ_\ell(1)\]
		is skew-symmetric.
		\item $2\varphi=\varphi_\mc L$ for some line bundle $\mc L$.
		\item If $k$ is algebraically closed, then $\varphi=\varphi_\mc L$ for some line bundle $\mc L$.
	\end{listroman}
\end{theorem}
% \begin{remark}
% 	Over an algebraically closed field, (iii) is equivalent to having $2\varphi=\varphi_\mc L$ for some line bundle $\mc L$. See \cite[p.~214]{mumford}.
% \end{remark}
\begin{proof}
	We will only prove part of this. In particular, we will not show the implication (iii) implies (iv); see \cite[p.~214]{mumford}. Our argument will use \Cref{rem:functorial-weil-pairing}. Additionally, we note that the diagram
	% https://q.uiver.app/#q=WzAsNCxbMCwwLCJBIl0sWzEsMCwiQiJdLFswLDEsIkFeXFxsb3IiXSxbMSwxLCJCXlxcbG9yIl0sWzAsMSwiZiJdLFsxLDMsIlxcdmFycGhpX1xcbWMgTCJdLFswLDIsIlxcdmFycGhpX3tmXipcXG1jIEx9IiwyXSxbMywyLCJmXlxcbG9yIl1d&macro_url=https%3A%2F%2Fraw.githubusercontent.com%2FdFoiler%2Fnotes%2Fmaster%2Fnir.tex
	\begin{equation}
		\begin{tikzcd}
			A & B \\
			{A^\lor} & {B^\lor}
			\arrow["f", from=1-1, to=1-2]
			\arrow["{\varphi_\mc L}", from=1-2, to=2-2]
			\arrow["{\varphi_{f^*\mc L}}"', from=1-1, to=2-1]
			\arrow["{f^\lor}", from=2-2, to=2-1]
		\end{tikzcd} \label{eq:around-with-duals}
	\end{equation}
	commutes. Unwinding definitions now implies that
	\[E^{f^*\mc L}(x,y)=E^{\mc L}(T_\ell f(x),T_\ell f(y)).\]
	We also quickly recall that $(A\times B)^\lor\cong A^\lor\times B^\lor$ essentially by restriction of line bundles; this result is on the homework. In particular, $A\times A^\lor$ is self-dual.

	Now, let $\mc P$ denote the Poincar\'e line bundle. We now execute the following computation.
	\begin{lemma}
		Let $\mc P$ be the Poincar\'e line bundle of an abelian $k$-variety $A$. Then
		\[E^\mc P((x,x^\lor),(y,y^\lor))=e_{\ell^\infty}(x,y^\lor)-e_{\ell^\infty}(y,x^\lor).\]
		Here, $x,y\in T_\ell A$ and $x^\lor,y^\lor\in T_\ell A^\lor$.
	\end{lemma}
	\begin{proof}
		By bilinearity and skew-symmetry, it is enough to verify the equalities on the pairs $((x,0),(y,0))$ and $((x,0),(0,y^\lor))$.
		\begin{itemize}
			\item We verify on $((x,0),(y,0))$, where our pairing should vanish. Pulling back along $({\id},0)\colon A\to A\times A^\lor$, we see $\mc P$ trivializes (it's from $\Pic$), so
			\[E^\mc P((x,0),(y,0))=E^{\OO_A}(x,y)=e_{\ell^\infty}(x,0)=0.\]
			\item We verify on $((x,0),(0,y^\lor))$. We pull back along $(0,{\id})\times A^\lor\to A\times A^\lor$, where the main point is to figure out where $\mc P$ goes. Well, one has the composite map
			\[A\times A^\lor\to(A\times A^\lor)^\lor=A^\lor\times A,\]
			where the last map is given by restriction of line bundles. In particular, the pair $(x,x^\lor)$ goes to the line bundle $t_{(x,x^\lor)}^*(\mc P\otimes\mc P^{-1})$, which then goes to $(t_x^*x^\lor,x)$ by a computation with the Poincar\'e line bundle, but $x^\lor\in A^\lor$ is translation invariant, so we are just going to $(x^\lor,x)$. So we can compute that
			\[E^\mc P((x,0),(0,y^\lor))=e_{\ell^\infty}((x,0),(y^\lor,0))=e_{\ell^\infty}(x,y^\lor),\]
			as desired.
			\qedhere
		\end{itemize}
	\end{proof}
	We now proceed with the proof. We already know that (iv) implies (i) and (ii) from earlier statements involving polarizations. To see that (i) implies (iii), we set $\mc L\coloneqq({\id}\times\varphi)^*\mc P$. Using \eqref{eq:around-with-duals}, we see
	\begin{align*}
		\varphi_\mc L(x) &= (1\times\varphi)^\lor\circ\varphi_\mc P\circ(1\times\varphi)(x) \\
		&= (1\times\varphi)^\lor(\varphi(x),x) \\
		&\stackrel*= (1\times\varphi)(\varphi(x),x) \\
		&= 2\varphi(x),
	\end{align*}
	where we have used symmetry of $\varphi$ at $\stackrel*=$. (Note $\varphi_\mc P$ swaps coordinates as shown in the previous proof.)

	We now show (ii) implies (iii). Continuing with the same $\mc L$ as in the previous paragraph, we use the lemma to see
	\begin{align*}
		E^{\mc L}(x,y) &= E^{\mc P}(T_\ell(1\times\varphi)(x),T_\ell(1\times\varphi)(y)) \\
		&= e_{\ell^\infty}(x,T_\ell\varphi(y))-e_{\ell^\infty}(y,T_\ell\varphi(x)) \\
		&= E^\varphi(x,y)-E^\varphi(y,x) \\
		&= 2E^\varphi(x,y),
	\end{align*}
	where the last equality has used skew-symmetry. Non-degeneracy of our pairing now forces $2\varphi=\varphi_\mc L$, so we are done.
\end{proof}
\begin{remark}
	This result shows that $\op{NS}(A)=\op{NS}(A_{\ov k})$ is exactly the $\ZZ$-submodule of symmetric homomorphisms $A\to A^\lor$.
\end{remark}

\subsection{The Rosati Involution}
Here is our definition.
\begin{definition}[Rosati involution]
	Fix a polarization $\lambda\colon A\to A^\lor$ of abelian $k$-variety $A$. Then the \textit{Rosati involution} $(-)^\dagger$ on $\op{End}^0(A)$ sends $\varphi$ to the map $\varphi^\dagger$ making the following diagram commute.
	% https://q.uiver.app/#q=WzAsNCxbMCwwLCJBIl0sWzEsMCwiQSJdLFswLDEsIkFeXFxsb3IiXSxbMSwxLCJBXlxcbG9yIl0sWzMsMiwiXFx2YXJwaGleXFxsb3IiLDJdLFsxLDAsIlxcdmFycGhpXlxcZGFnZ2VyIiwyXSxbMSwzLCJcXGxhbWJkYSJdLFswLDIsIlxcbGFtYmRhIiwyXV0=&macro_url=https%3A%2F%2Fraw.githubusercontent.com%2FdFoiler%2Fnotes%2Fmaster%2Fnir.tex
	\[\begin{tikzcd}
		A & A \\
		{A^\lor} & {A^\lor}
		\arrow["{\varphi^\lor}"', from=2-2, to=2-1]
		\arrow["{\varphi^\dagger}"', from=1-2, to=1-1]
		\arrow["\lambda", from=1-2, to=2-2]
		\arrow["\lambda"', from=1-1, to=2-1]
	\end{tikzcd}\]
	Explicitly,
	\[\varphi^\dagger\coloneqq\lambda^{-1}\circ\varphi^\lor\circ\lambda.\]
\end{definition}
\begin{remark}
	We are working with $\op{End}^0(A)$ so that we can write down $\lambda^{-1}$ in general. However, if $A$ is principally polarized, then this inverse already exists in $\op{End}(A)$, so we can still write down the Rosati involution even on $\op{End}(A)$.
\end{remark}
\begin{remark}
	The Rosati involution depends on $\lambda$, but this dependence is not too bad. Namely, if $\lambda_1$ and $\lambda_2$ are two polarizations (in particular, isogenies), then we get an isogeny such that $\lambda_1=\lambda_2\circ f$, so
	\[\lambda_1^{-1}\circ\varphi\circ\lambda_1=f^{-1}\circ\lambda_2^{-1}\circ\varphi\circ\lambda_2\circ f,\]
	so we at least have the same conjugacy class in $\op{End}^0(A)$.
\end{remark}

\end{document}