% !TEX root = ../notes.tex

\documentclass[../notes.tex]{subfiles}

\begin{document}

Here we go.

\subsection{Degree of Isogenies}
Today we are going to discuss degrees of isogenies. The point is that we are going to show that the degree map $\deg\colon\op{End}A\to\ZZ$ is polynomial. Let's define what this means
\begin{definition}
	Fix a $k$-vector space $V$. Then a function $f\colon V\to k$ is a \textit{homogeneous polynomial of degree $n$} if and only if $f|_W$ is a homogeneous polynomial of degree $n$ for any finite-dimensional subspace $W\subseteq V$. In other words, for any finite set of linearly independent vectors $\{v_1,\ldots,v_n\}$, the function
	\[(a_1,\ldots,a_n)\mapsto f(a_1v_1+\cdots+a_nv_n)\]
	is a homogeneous polynomial of degree $n$. (The point here is that change of basis does not adjust the fact that $f$ is a homogeneous polynomial of degree $n$.)
\end{definition}
\begin{remark} \label{rem:reduce-poly-map-to-two}
	An induction tells us that it suffices to check the result for sets of linearly independent vectors of size $2$.
\end{remark}
So we are actually going to show that $\deg\op{End}A\to\ZZ$ is a polynomial map of degree $2g$. For isogenies, we know how to make sense of degree, but we should probably make a convention if not an isogeny.
\begin{definition}
	Given a homomorphism $f\colon A\to B$ of abelian $k$-varieties, we define $\deg f=0$ if $f$ is not an isogeny. For a general map $\frac1nf\in\op{End}^0(A)$ where $g\coloneqq\dim A$, we define
	\[\deg\left(\frac1nf\right)\coloneqq\frac{\deg f}{n^{2g}}.\]
\end{definition}
\begin{remark}
	One can check that $\frac1mf=\frac1ng$ implies that
	\[\frac{\deg f}{m^{2g}}=\frac{\deg g}{n^{2g}}.\]
	Indeed, by \Cref{thm:mul-n-av}, we see that $\deg[n]=n^{2g}$ for any integer $n$, so $[m]\circ f=[n]\circ g$ yields the above equality after rearranging.
\end{remark}
So in fact we will aim to show that $\deg\colon\op{End}^0(A)\to\QQ$ is a homogeneous polynomial of degree $2g$.

Here is a starting lemma.
\begin{lemma} \label{lem:deg-by-euler-char}
	Fix an isogeny $g\colon A\to B$ of abelian $k$-varieties. For all line bundles $\mc L$ on $B$, one has
	\[\chi(g^*\mc L)=(\deg g)\chi(\mc L).\]
\end{lemma}
\begin{proof}
	See \cite[Theorem~12.2]{mumford}. Note that this result is similar to \Cref{prop:deg-of-pullback}. We will have more context for this result when we discuss Riemann--Roch for abelian varieties in more detail. The main point is to reduce to the elliptic curve case, where one can use the Riemann--Hurwitz formula; notable, the map $g$ is a group homomorphism and hence unramified.
\end{proof}
And here is our result.
\begin{theorem}
	Fix a simple abelian $k$-variety $A$ of dimension $g$. Then $\deg\colon\op{End}^0(A)\to\QQ$ is a homogeneous polynomial of degree $2g$.
\end{theorem}
\begin{proof}
	Once we know that we have a polynomial, the fact that $\deg(nf)=n^{2g}\deg f$ will enforce homogeneity. So it suffices to show that we are just polynomial, so by \Cref{rem:reduce-poly-map-to-two}, it suffices to show that the map $\deg(nf_1+f_2)$ is a polynomial map in $n$, where $f_1,f_2\colon A\to A$ are isogenies.

	Choose an ample line bundle $\mc L$ on $A$. By Serre's criterion for ampleness, we may replace $\mc L$ with a power of itself so that $\mc L$ has no higher cohomology. But $\mc L$ must be globally generated (it's ample), so $\mc L$ has some global sections, so $\chi(\mc L)\ne0$. Now, \Cref{lem:deg-by-euler-char} tells us
	\[\deg(nf_1+f_2)=\frac{\chi((nf_1+f_2)^*\mc L)}{\chi(\mc L)},\]
	so it remains to work with the numerator. We would like to evaluate $\mc L_n\coloneqq (nf_1+f_2)^*\mc L$ inductively, for which we must use \Cref{thm:cube}. In particular, \Cref{thm:cube} tells us that
	\[\mc L_{n+2}=(f_1+f_1+(nf_1+f_2))^*\mc L\cong\mc L_{n+1}\otimes\mc L_{n+1}\otimes(2f_1)^*\mc L\otimes\mc L_n^{-1}\otimes L_n^{-1}\otimes f_1^*\mc L^{\otimes-2}.\]
	An induction is now able to show that
	\[\mc L_n=\mc M^{\otimes n(n-1)/2}\otimes\mc N^{\otimes n}\otimes Q\]
	for some line bundles $\mc M$ and $\mc N$ and $\mc Q$ (which do not depend on $n$).

	Now, the same argument which shows that the Hilbert polynomial is a polynomial shows that the map
	\[(n_1,\ldots,n_r)\mapsto\chi\left(\mc L_1^{\otimes n_1}\otimes\cdots\otimes\mc L_r^{\otimes n_r}\right)\]
	is always a polynomial in $\QQ[n_1,\ldots,n_r]$, so we are done.
\end{proof}

\subsection{Riemann--Roch for Abelian Varieties}
Let's study the Euler characteristic of line bundles in more detail.
\begin{proposition} \label{prop:riemann-roch-av}
	Fix a line bundle $\mc L$ on an abelian $k$-variety $A$.
	\begin{listalph}
		\item Then $n\mapsto\chi\left(\mc L^{\otimes n}\right)$ is a homogeneous polynomial of degree $2g$.
		\item If $\mc L=\OO_A(D)$ for a divisor $D$, then $\chi(\mc L)=(D,\ldots,D)/g!$, where $(D,\ldots,D)$ is the intersection number.
		\item The degree of the map $\varphi_\mc L\colon A\to A^\lor$ is $\chi(\mc L)^2$.
	\end{listalph}
\end{proposition}
\begin{remark} \label{rem:euler-char-of-ample-nonzero}
	Note that (c) of the above tells us that indeed $\mc L$ being ample means $\varphi_\mc L$ is an isogeny, so $\chi(\mc L)^2$ must be nonzero, so $\chi(\mc L)$ is nonzero.
\end{remark}
\begin{proof}
	We may assume that $k$ is algebraically closed because we are just computing dimensions of cohomology, which is preserved by flat base change (such as a field extension).

	For (a), we proceed in steps. The point is to reduce to the case of $\mc L$ being symmetric or in $A^\lor$, which can be attacked separately.
	\begin{enumerate}
		\item We claim that $\mc L_1\otimes\mc L_2^{-1}\in A^\lor$ implies that $\chi(\mc L_1)=\chi(\mc L_2)$. Indeed, $\mc L_1\otimes\mc L_2^{-1}$ being in $A^\lor$ implies that $\mc L_1$ and $\mc L_2$ are algebraically equivalent, so they arise as restrictions of a larger line bundle $\mc L$ on $A\times S$. However, the Euler characteristic $\chi$ is locally constant, so we conclude $\chi(\mc L_1)=\chi(\mc L_2)$.

		\item We claim that any line bundle $\mc L$ on $A$ has line bundles $\mc L_1$ and $\mc L_2$ such that $\mc L=\mc L_1\otimes\mc L_2$ such that $\mc L_1$ is symmetric and $\mc L_2\in A^\lor$.

		The main point is the construction of $\mc L_2$. We would like to set $\mc L_1$ to be $\mc L\otimes[-1]^*\mc L$, and take $\mc L_2$ to be $\mc L\otimes[-1]^*\mc L^{-1}$, but this does not actually multiply to $\mc L$. So we will want to take some square-roots, which requires a more careful argument.
		
		To begin, we claim that $\mc L\otimes[-1]^*\mc L^{-1}\in A^\lor$. Indeed, it suffices to show that the line bundle is trans\-lation-invariant, so we compute
		\[t_x\left(\mc L\otimes[-1]^*\mc L^{-1}\right)\otimes\mc L^{-1}\otimes[-1]^*\mc L=t_x^*\mc L\otimes\mc L^{-1}\otimes[-1]^*\left(t_{-x}^*\mc L^{-1}\otimes\mc L\right).\]
		Now, $t_x^*\mc L\otimes\mc L^{-1}$ is certainly in $A^\lor$ because it is just in the image of $\varphi_\mc L$, and pulling back along $[-1]^*$ stays in $A^\lor$ because this map is just $[-1]_{A^\lor}\colon A^\lor\to A^\lor$. In fact, $[-1]$ corresponds to inverting the line bundle, so our line bundle now looks like
		\[t_x^*\mc L\otimes\mc L^{-1}\otimes\left(t_{-x}^*\mc L\otimes\mc L^{-1}\right),\]
		which vanishes by \Cref{thm:square}.

		We now use the fact that we are over $\ov k$, so we may find $\mc L_2\in A^\lor(\ov k)$ with $\mc L_2^{\otimes2}=\mc L\otimes[-1]^*\mc L^{-1}$. Now, we define $\mc L_1\coloneqq\mc L\otimes\mc L_2^{-1}$ so that
		\[[-1]^*\mc L_1=[-1]^*\mc L\otimes[-1]^*\mc L_2^{-1}=[-1]^*\mc L\otimes\mc L_2^{\otimes2}\otimes\mc L_2^{-1}=[-1]^*\mc L\otimes\mc L\otimes[-1]^*\mc L^{-1}\otimes\mc L_2^{-1}=\mc L_1,\]
		so $\mc L_1$ is in fact symmetric, as needed.

		\item We now prove (a). As remarked in the previous proof, this function is certainly polynomial, so it is enough to compute the degree. The result is true for $\mc L\in A^\lor$ by an inductive argument via $m^*\mc L=\op{pr}_1^*\mc L\otimes\op{pr}_2^*\mc L$, so it remains to handle the case where $\mc L$ is symmetric. Here, an induction with \Cref{thm:cube} shows that
		\[\chi\left(\mc L^{\otimes m^2n}\right)=\chi\left([m]^*\mc L^{\otimes n}\right)\stackrel*=\deg[m]\cdot\chi\left(\mc L^{\otimes n}\right)=m^{2g}\chi(\mc L^{\otimes n}),\]
		where $\stackrel*=$ has used \Cref{lem:deg-by-euler-char}. This completes the homogeneity check.
	\end{enumerate}
	We now hand-wave the proof of (b) and leave the details for \cite[Theorem~III.16]{mumford}. Any line bundle can be written as the difference of two very ample line bundles, so it is enough to check the result for very ample line bundles. If $\mc L$ is very ample, then intersection theory provides the result: a choice of generic global sections of $\mc L$ as $\sigma_0,\ldots,\sigma_g$ so that they have no common zeroes and the $\op{div}\sigma_\bullet$ intersect transversally; as such,
	\[D^g=(\op{div}\sigma_0,\ldots,\op{div}\sigma_g)\]
	is literally the number of points in the intersection of the $\op{div}\sigma_\bullet$s. Now, our choice of global sections induces a closed embedding $\varphi\colon A\to\PP^g$, and the above intersection number is the pre-image of the point $[1:0:\cdots:0]$, so we see that $D^g=\deg\varphi$. On the other hand, $\chi(\mc L)=(\deg\varphi)\chi(\OO_{\PP^g}(1))$, which completes the proof upon a computation.
\end{proof}

\end{document}