% !TEX root = ../notes.tex

\documentclass[../notes.tex]{subfiles}

\begin{document}

Today we use the Rosati involution for fun and profit.

\subsection{Positivity of the Rosati Involution}
Manipulation with definitions verify the following.
\begin{proposition}
	Fix a polarization $\lambda\colon A\to A^\lor$ of an abelian $k$-variety $A$.
	\begin{listalph}
		\item $(-)^\dagger$ is linear.
		\item $(-)^\dagger$ anti-commutes: $(\varphi\circ\psi)^\dagger=\psi^\dagger\circ\varphi^\dagger$.
		\item For the Weil pairing, $E^\lambda(T_\ell\varphi(x),y)=E^\lambda(x,T_\ell\varphi^\dagger(y))$.
	\end{listalph}
\end{proposition}
\begin{proof}
	For (a), note that composition is linear. For (b), proceed directly from the definitions and use the duality of $(-)^\lor$. Lastly, for (c), use \Cref{rem:functorial-weil-pairing} and then pass through $\lambda$ everywhere as needed.
\end{proof}
Now here is our main result on the Rosati involution.
\begin{theorem}[Positivity] \label{thm:positivity-rosati}
	Fix a polarization $\lambda\colon A\to A^\lor$ of an abelian $k$-variety $A$. Then for any nonzero $\varphi\in\op{End}^0(A)$, one has that
	\[\tr(\varphi\circ\varphi^\dagger)=\tr(\varphi^\dagger\circ\varphi)>0.\]
\end{theorem}
Here, $\tr$ refers to the reduced trace on the semisimple algebra $\op{End}^0(A)$. More precisely, for a semisimple $\QQ$-algebra $D$, let $K$ be its center, and then base-change to $\ov K$ and compute the trace as a matrix algebra because $D\otimes_K\ov K$ is a matrix algebra. Working with the characteristic polynomial allows us to compute the trace on the level of $V_\ell A$ via \Cref{thm:tate-char-poly}, or equivalently via the characteristic polynomial.
\begin{proof}
	By base-changing our morphisms (which does not adjust the reduced trace here), we may assume that $k$ is algebraically closed. As such, $\lambda=\varphi_\mc L$ for an ample line bundle $\mc L$; taking powers of $\mc L$ adjusts $\varphi_\mc L$ by multiplication by this power, but this does not change the Rosati involution, so we may actually assume that $\mc L$ is very ample, so say $\mc L=\OO_A(D)$ where $D$ is an effective Weil divisor (in fact, a hyperplane intersection).

	Now, let $g\coloneqq\dim A$, and the main claim is that
	\[\tr(\varphi\circ\varphi^\dagger)\stackrel?=2g\cdot\frac{\left(D^{g-1},\varphi^{-1}D\right)}{\left(D^g\right)}.\]
	This will complete the proof because $\varphi^{-1}(D)$ continues to be an effective Weil divisor, and then we are just computing some intersection numbers, which is positive.

	So it remains to prove the claim. We will use \Cref{prop:riemann-roch-av}. Note that $\varphi_{\varphi^*\mc L^{-1}\otimes\mc L^{\otimes n}}\colon A\to A^\lor$ is an isogeny, and this map is equal to $[n]\circ\varphi_\mc L-\varphi_{\varphi^*\mc L}$. The moral is that we can compute the degree
	\begin{align*}
		\deg\left([n]\circ\varphi_\mc L-\varphi_{\varphi^*\mc L}\right) &= \deg\left([n]\circ\varphi_\mc L-\varphi^\lor\circ\varphi_\mc L\circ\varphi\right) \\
		&= \deg\left(\varphi_\mc L\circ\left([n]-\varphi_\mc L^{-1}\circ\varphi^\lor\circ\varphi_\mc L\circ\varphi\right)\right) \\
		&= \deg\varphi_\mc L\cdot\deg\left([n]\varphi^\dagger\circ\varphi\right).
	\end{align*}
	This last quantity is now the characteristic polynomial $P(n)$ of $\varphi^\dagger\circ\varphi$. Thus,
	\[P(n)=\frac{\deg\left([n]\circ\varphi_\mc L-\varphi_{\varphi^*\mc L}\right)}{\deg\varphi_\mc L},\]
	which by \Cref{prop:deg-is-euler-char} is
	\[P(n)=\frac{\chi\left(\varphi^*\mc L^{-1}\otimes\mc L^{\otimes n}\right)}{\chi(\mc L)^2}.\]
	Now, \Cref{prop:riemann-roch-av} implies
	\[P(n)=\left(\frac{(nD-\varphi^{-1}D)^g}{(D^g)}\right)^2.\]
	We would like the term after the leading term of this polynomial, which by linearity looks like
	\[P(n)=\frac1{(D^g)^2}\left(n^g(D^g)-gn^{g-1}\left(D^{g-1},\varphi^{-1}D\right)+\cdots\right)^2,\]
	whose term after the leading term is exactly what we claimed it would be.
\end{proof}

\subsection{The Albert Classification}
% We would like to know which $\QQ$-algebras appear as endomorphism algebras of abelian varieties. By an argument similar to \Cref{cor:general-hom-of-ab-var}, it suffices to consider the case where $A$ is simple, which implies $\op{End}^0(A)$ is a division algebra.
We now see that the positivity of the Rosati involution now gives us some tools to classify algebras.
\begin{lemma}
	Fix a division $\QQ$-algebra $D$ equipped with a positive anti-involution $(-)^\dagger$ on $D$. Further, set $K\coloneqq Z(D)$ and $K^+\coloneqq\left\{x\in K:x=x^\dagger\right\}$ with $e\coloneqq[K:\QQ]$ and $e^+\coloneqq\left[K^+:\QQ\right]$. Then $K_0$ is totally real, and either $K=K^+$ or $K/K^+$ is a totally imaginary quadratic extension.
\end{lemma}
\begin{proof}
	We begin by trying to prove that $K^+$ is totally real. Well, we can write
	\[K_0\otimes\RR=\RR^r\otimes\CC^s\]
	for some nonnegative integers $r,s\ge0$. Notably, the quadratic form $x\mapsto\tr(xx^\dagger)$ is a quadratic form $q(x)$ on $K^+$ (note $x^\dagger=x$ here), which extends by continuity to a quadratic form $q_\RR$ on $\RR^r\times\CC^s$. Now, $q$ itself was defined over $\QQ$, so its null space will be defined over $\QQ$, but positivity of $(-)^\dagger$ tells us that this null space must vanish. So actually $q_\RR$ is positive-definite, but then there can be no copies of $\CC$ in $K^+\otimes\RR$ because one can always solve these quadratic equations over the complex numbers.

	We now complete the proof. Notably, by definition $\left[K:K^+\right]\in\{1,2\}$: indeed, $K^+$ is defined as being a subfield of $K$ fixed by a group of order $2$ (namely, generated by the automorphism $\alpha\mapsto\alpha^\dagger$). It remains to show that $K/K^+$ is a totally imaginary quadratic extension if nontrivial. Well, if nontrivial, we can write $K=K^+(\sqrt\alpha)$ for some $\alpha\in K^+$. Now, $(\sqrt\alpha)^2=\alpha$ must be fixed by $(-)^\dagger$, but $\sqrt\alpha$ is not, so we must have $(\sqrt\alpha)^\dagger=-\sqrt\alpha$.
	
	Continuing, suppose for the sake of contradiction that we have a real place $i\colon K\to\RR$. Then $i^\dagger\colon K\to\RR$ continues to be a real place but now has $i(\sqrt\alpha)=-i^\dagger(\sqrt\alpha)$. To derive contradiction, we work with the pieces $\RR\times\RR$ inside $K\otimes\RR$ corresponding to $i$ and $i^\dagger$, where we see that
	\[\tr\left((x,y)\cdot(x,y)^\dagger\right)=\tr((x,y),(y,x))=2xy\]
	now fails to be positive-definite.
\end{proof}
There is now a full classification of division algebras with positive anti-involution.
\begin{theorem}[Albert] \label{thm:albert}
	Fix a division $\QQ$-algebra $D$ equipped with a positive anti-involution $(-)^\dagger$ on $D$. Further, set $K\coloneqq Z(D)$ and $K^+\coloneqq\left\{x\in K:x=x^\dagger\right\}$ with $e\coloneqq[K:\QQ]$ and $e^+\coloneqq\left[K^+:\QQ\right]$. Then $(D,K,K^+)$ satisfies one of the following.
	\begin{itemize}
		\item Type I: $D=K=K^+$, and $(-)^\dagger=\id_D$.
		\item Type II: $K=K^+$, but $D$ is a quaternion $K$-algebra $D\otimes_\QQ\RR\cong\prod_{i\colon K\into\RR}M_2(\RR)$ split as the matrix algebra at all archimedean places of $K$ (which are necessarily real), where $(-)^\dagger$ (up to isomorphism) is given by transposition of matrices. Explicitly, we have $D\otimes_\QQ\RR\cong\prod_{i\colon K\into\RR}M_2(\RR)$.
		\item Type III: $K=K^+$, but $D$ is a quaternion $K$-algebra $D\otimes_\QQ\RR\cong\prod_{i\colon K\into\RR}\HH$ ramified at all places, and $(-)^\dagger$ is the standard involution on the quaternions.
		\item Type IV: $K/K^+$ is a totally imaginary extension with complex conjugation $c$, and $D$ is a division $K$-algebra such that $\op{inv}_v(D)+\op{inv}_{c(v)}D=0$ if $v\ne c(v)$, and $\op{inv}_v(D)=0$ if $v=c(v)$; $(-)^\dagger$ is conjugate transpose.
	\end{itemize}
\end{theorem}
\begin{remark}
	In the final case, one finds that
	\[D\otimes_\QQ\RR\cong\prod_{\substack{i\colon K\to\CC\\\text{up to }c}}M_d(\CC)\]
	where $d\coloneqq\sqrt{[D:K]}$ is the reduced degree.
\end{remark}

\end{document}