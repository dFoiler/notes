% !TEX root = ../notes.tex

\documentclass[../notes.tex]{subfiles}

\begin{document}

Homework has been posted. It is due shortly before spring break. There will be another homework assigned over spring break.
\begin{remark}
	Any surjective group homomorphism $f\colon A\to B$ of abelian $k$-varieties will be fpqc automatically: quasicompactness has no content, and flatness follows by Miracle flatness.
\end{remark}

\subsection{Construction of the Dual Abelian Variety}
For completeness, we provide a construction of $\Pic^\circ_{A/k}$; see \cite[{\S}II.8, {\S}III.13]{mumford}. The point is to use the surjection $\varphi_\mc L\colon A\to\Pic^\circ_{A/k}$ (which we know exists on the functor of points), so one can recover $\Pic^\circ_{A/k}$ as a quotient group scheme by $K(\mc L)\coloneqq\ker\varphi_\mc L$. For example, in characteristic $0$, our finite group scheme must be smooth, so we should use the reduced scheme structure.

Quickly, we provide a moduli interpretation for $A^\lor$.
\begin{definition}[Poincar\'e line bundle]
	Fix an abelian $k$-variety $A$. The \textit{Poincar\'e line bundle} $\mc P$ on $A\times A^\lor$ is the line bundle satisfying $\mc P|_{0_A\times A^\lor}$ and $\mc P|_{A\times\lambda}\cong\lambda$ for any (rigidified) line bundle $\lambda\in A^\lor$.
\end{definition}
\begin{remark}
	The line bundle $\mc P$ is unique by \Cref{prop:seesaw}.
\end{remark}
To see that it exists, for given very ample line bundle $\mc L$ on $A$, define
\[\mc M\coloneqq m^*\mc L\otimes\op{pr}_1^*\mc L^{-1}\otimes\op{pr}_2^*\mc L^{-1}\]
on $A\times A$. Notably, $\mc M|_{0_A\times A}\cong\OO_A$ and $\mc M|_{A\times\{x\}}=\varphi_\mc L(x)$ by some computation, so we expect to have $({\id_A}\times\varphi_\mc L)^*\mc P=\mc M$. So we will be able to construct $\mc P$ by some suitable descent.

Let's now give $K(\mc L)$ a scheme structure: we let it be the maximal subscheme of $A$ such that $\mc M|_{K(\mc L)\times A}$ is trivial; we won't bother to check that this exists. It turns out that this is the correct scheme structure for $\Pic^\circ_{A/k}$ by some checking. So to finish our construction of the Poincar\'e line bundle as providing descent data on
\[A\times A\times A\times K(\mc L)=(A\times A)\times_{A\times A^\lor}(A\times A)\to A\times A^\lor.\]
The descent data now amounts to providing an isomorphism $\mc M\cong(1\times t_x)^*\mc M$ for $x\in K(\mc L)$, which can be done by staring at the group law.

\subsection{Symmetry of Duality}
We defined $A^\lor$ as a dual, so one should expect that $A$ and $A^{\lor\lor}$ are canonically isogenous. In general, a line bundle $\mc Q$ (living in the connected component) on $A\times B$ produces a homomorphism $\kappa_\mc Q\colon B\to A^\lor$ by. For example, $\mc P$ on $A\times A^\lor$ corresponds to $\id\colon A^\lor\to A^\lor$. However, swapping coordinates produces an isomorphism $\sigma\colon A\times A^\lor\to A^\lor\times A$ will then produce a morphism $\kappa_{\sigma^*\mc P}\colon A\to (A^\lor)^\lor$, which we claim is an isomorphism.
\begin{proposition} \label{prop:dual-dual-av}
	Fix an abelian $k$-variety $A$ with Poincar\'e line bundle $\mc P$. Then swapping coordinates produces an isomorphism $\sigma\colon A\times A^\lor\to A^\lor\times A$ will then produce a canonical isomorphism $\kappa_{\sigma^*\mc P}\colon A\to (A^\lor)^\lor$. In fact, the following diagram commutes.
	% https://q.uiver.app/#q=WzAsMyxbMCwwLCJBIl0sWzEsMCwiKEFeXFxsb3IpXlxcbG9yIl0sWzAsMSwiQSJdLFswLDIsIlxcdmFycGhpX1xcbWMgTCIsMl0sWzEsMiwiXFx2YXJwaGlfe1xcbWMgTH1eXFxsb3IiXSxbMCwxLCJcXGthcHBhX3tcXHNpZ21hXipcXG1jIFB9Il1d&macro_url=https%3A%2F%2Fraw.githubusercontent.com%2FdFoiler%2Fnotes%2Fmaster%2Fnir.tex
	\[\begin{tikzcd}
		A & {(A^\lor)^\lor} \\
		A
		\arrow["{\varphi_\mc L}"', from=1-1, to=2-1]
		\arrow["{\varphi_{\mc L}^\lor}", from=1-2, to=2-1]
		\arrow["{\kappa_{\sigma^*\mc P}}", from=1-1, to=1-2]
	\end{tikzcd}\]
\end{proposition}
\begin{proof}
	To see that $\kappa_{\sigma^*\mc P}$ is an isomorphism, we let $\mc L$ be ample so that $\varphi_\mc L$ and $\varphi_\mc L^\lor$ is an isogeny, and both of these covers of $A^\lor$ have the same kernel by \Cref{thm:dual-map}. Now, $\kappa_{\sigma^*\mc P}$ is an isogeny because everything in sight is an isogeny (for example, everything has the same dimension, and finite kernel is forced because its composite with $\varphi_\mc L^{\lor}$ has finite kernel), and we are able to conclude that $\kappa_{\sigma^*\mc P}$ is an isomorphism because it has degree $1$ (indeed, $\deg\varphi_\mc L=\deg\varphi_\mc L^\lor$).

	We now show the commutativity of the given triangle by hand on closed points. In one direction, $\varphi_\mc L(x)=t_x^*\mc L\otimes\mc L^{-1}$ for $x\in A(\ov k)$. In other direction, we begin by computing
	\[\kappa_{\sigma^*\mc P}(x)=\mc P|_{\{x\}\times A^\lor}\]
	by definition of $\kappa$, and
	\[\varphi_\mc L^\lor\left(\kappa_{\sigma^*\mc P}(x)\right)=\varphi_\mc L^\lor\left(\mc P|_{\{x\}\times A^\lor}\right)=({\id}\times\varphi_\mc L)^*\mc P|_{\{x\}\times A}=\left(m^*\mc L\otimes\op{pr}_1^*\mc L^{-1}\otimes\op{pr}_2^*\mc L^{-1}\right)|_{\{x\}\times A},\]
	which agrees with the other side.
\end{proof}
\begin{remark}
	For an abelian $k$-variety $A$, we will let $\mc P_A$ be its Poincar\'e line bundle in this remark. Then it turns out that $\mc P_{A^\lor}$ on $A^\lor\times(A^\lor)^\lor$ is $\sigma^*\mc P_A$ pulled back along $\kappa_{\sigma^*\mc P_A}^{-1}\colon A^\lor\times(A^\lor)^\lor\to A^\lor\times A$. To see this, I think one can use a moduli interpretation or the commutativity of the above diagram for some uniqueness.
\end{remark}
\Cref{prop:dual-dual-av} motivates the following definition.
\begin{definition}
	Fix an abelian $k$-variety $A$. A homomorphism $\lambda\colon A\to A^\lor$ is \textit{symmetric} if and only if $\lambda^\lor=\lambda$ up to the identification of $A$ with $A^{\lor\lor}$.
\end{definition}
\begin{example}
	A polarization $\varphi_\mc L$ is symmetric by \Cref{prop:dual-dual-av} (perhaps needing to check on $\ov k$-points due to the definition of polarization).
\end{example}
\begin{remark}
	Fix a morphism of abelian $k$-varieties $f\colon A\to B$. Given a line bundle $\mc L$ on $B$, tracking through moduli interpretations produces the following commutative diagram.
	% https://q.uiver.app/#q=WzAsNCxbMCwwLCJBIl0sWzEsMCwiQiJdLFswLDEsIkFeXFxsb3IiXSxbMSwxLCJCXlxcbG9yIl0sWzAsMSwiZiJdLFsyLDMsImZeXFxsb3IiXSxbMCwyLCJcXHZhcnBoaV97Zl4qXFxtYyBMfSIsMl0sWzEsMywiXFx2YXJwaGlfXFxtYyBMIl1d&macro_url=https%3A%2F%2Fraw.githubusercontent.com%2FdFoiler%2Fnotes%2Fmaster%2Fnir.tex
	\[\begin{tikzcd}
		A & B \\
		{A^\lor} & {B^\lor}
		\arrow["f", from=1-1, to=1-2]
		\arrow["{f^\lor}", from=2-1, to=2-2]
		\arrow["{\varphi_{f^*\mc L}}"', from=1-1, to=2-1]
		\arrow["{\varphi_\mc L}", from=1-2, to=2-2]
	\end{tikzcd}\]
\end{remark}
This symmetry allows us to construct the ``dual'' isogeny.
\begin{theorem} \label{thm:inverse-isog}
	Fix an isogeny $f\colon A\to B$ of abelian $k$-varieties. Then there is an isogney $g\colon B\to A$ such that $g\circ f=[\deg f]_A$.
\end{theorem}
\begin{remark}
	It also turns out that $f\circ g=[\deg f]_B$ and so $\deg g=\deg f$ (where $f$ and $g$ are as above). This is essentially by doing cancellation on isogenies via quotients.
\end{remark}
The proof of \Cref{thm:inverse-isog} is surprisingly technical in its group theory. For example, one needs the following result.
\begin{theorem}[Deligne]
	Fix a commutative finite flat $k$-group scheme $G$ of order $m\coloneqq\dim\Gamma(G,\OO_G)$. Then $G$ is killed by $[m]_G$.
\end{theorem}
\begin{proof}
	Omitted. We will show this later.
\end{proof}
We now prove \Cref{thm:inverse-isog}.
\begin{proof}[Proof of \Cref{thm:inverse-isog}]
	Note $f$ is fpqc, so descent tells us that a $k$-scheme $X$ makes $X(B)$ an equalizer of $\op{pr}_1,\op{pr}_2\colon X(A)\to X(A\times_B A)$ by viewing $X$ as a quasicoherent sheaf. (See \cite[Theorem~6.2.14]{conrad-av}.) For example, the composite
	\[A\times\ker f\cong A\times_BA\stackrel{\op{pr}_\bullet}\to A\stackrel{[m]}\to A\]
	vanishes for each projection, where $m\coloneqq\deg f=\dim\ker f$. So $[m]\circ{\op{pr}_1}=[m]\circ{\op{pr}_2}$, so descent tells us that $[m]$ factors through $f$, which is what promises the existence of $g$.
\end{proof}

\end{document}