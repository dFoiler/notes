% !TEX root = ../notes.tex

\documentclass[../notes.tex]{subfiles}

\begin{document}

\section{March 1}

Let's talk about curves. Our language will follow \cite[Chapter~II, IV]{hartshorne}. One can in theory just follow the classical language of \cite{silverman}.

\subsection{Introducing Curves}
The definition of a curve in \cite{hartshorne} is as follows.
\begin{definition}[curve]
	Fix an algebraically closed field $k$. Then a \textit{$k$-curve} $X$ is a $1$-dimensional, integral, smooth, projective $k$-scheme.
\end{definition}
\begin{example}
	Fix an algebraically closed field $k$ and a homogeneous polynomial $f\in k[x,y,z]$. Given that $\del F/\del x$ and $\del F/\del y$ and $\del F/\del z$ and $F$ do not all simultaneously vanish, then $V(F)\subseteq\PP^2_k$ is a field.
\end{example}
We would like to relax the requirement that $k$ is algebraically closed.
\begin{definition}[geometrically integral]
	An $S$-scheme $X$ is \textit{geometrically integral} if and only if $X\times_ST$ is integral for any $S$-scheme $T$.
\end{definition}
\begin{definition}[curve]
	Fix a field $k$. Then a \textit{$k$-curve} is a $1$-dimensional, geometrically integral, smooth, projective $k$-scheme.
\end{definition}
\begin{remark}
	Equivalently, we can require our curves to just be curves over an algebraically closed field over base-change to an algebraic closure. Roughly speaking, these properties are preserved by base-change and also local on the target with respect to flat base-change, so one can go back and forth.
\end{remark}
\begin{remark}
	As an aside, note that smoothness implies ``locally integral,'' meaning that there is an open cover of integral domains. (One can check this locally.) Thus, connectedness here is equivalent to irreducible because we are already integral.
\end{remark}
Many of the proofs we do will work by first taking a base-change to an algebraic closure and appealing to \cite{hartshorne}.

\subsection{Divisors}
Throughout, we fix a regular $k$-scheme $X$.
\begin{definition}[divisor]
	The \textit{divisor group} on a regular $k$-scheme $X$, denoted $\op{Div}(X)$, is the free abelian group on the closed points of $X$.
\end{definition}
Note that being one-dimensional and integral implies that $X$ has only closed points and a single generic point. We would like to define degree, but one must be a little careful because we are trying to relax algebraically closed hypotheses.
\begin{example}
	Note $\left(x^2+1\right)$ is a closed point of $\AA^1_\RR=\Spec\RR[x]$. However, after base-changing by $\CC$, we get the following diagram.
	% https://q.uiver.app/?q=WzAsNixbMCwxLCJcXGRpc3BsYXlzdHlsZVxcU3BlY1xcZnJhY3tcXFJSW3hdfXtcXGxlZnQoeF4yKzFcXHJpZ2h0KX0iXSxbMSwxLCJcXFNwZWNcXFJSW3hdIl0sWzIsMSwiXFxTcGVjXFxSUiJdLFsyLDAsIlxcU3BlY1xcQ0MiXSxbMSwwLCJcXFNwZWNcXENDW3hdIl0sWzAsMCwiXFxkaXNwbGF5c3R5bGVcXFNwZWNcXGZyYWN7XFxDQ1t4XX17XFxsZWZ0KHheMisxXFxyaWdodCl9Il0sWzUsNF0sWzQsM10sWzAsMV0sWzEsMl0sWzMsMl0sWzQsMV0sWzUsMF1d&macro_url=https%3A%2F%2Fraw.githubusercontent.com%2FdFoiler%2Fnotes%2Fmaster%2Fnir.tex
	\[\begin{tikzcd}
		{\displaystyle\Spec\frac{\CC[x]}{\left(x^2+1\right)}} & {\Spec\CC[x]} & \Spec\CC \\
		{\displaystyle\Spec\frac{\RR[x]}{\left(x^2+1\right)}} & {\Spec\RR[x]} & \Spec\RR
		\arrow[from=1-1, to=1-2]
		\arrow[from=1-2, to=1-3]
		\arrow[from=2-1, to=2-2]
		\arrow[from=2-2, to=2-3]
		\arrow[from=1-3, to=2-3]
		\arrow[from=1-2, to=2-2]
		\arrow[from=1-1, to=2-1]
	\end{tikzcd}\]
	The point here is that $\Spec\CC[x]/\left(x^2+1\right)$ is two copies of $\CC$! As such, we morally should count the divisor $\left(x^2+1\right)$ as ``containing'' two closed points. Of course, the issue here is that the residue field of $\left(x^2+1\right)$ is a degree-$2$ extension of $\RR$.
\end{example}
\begin{definition}[degree]
	Fix a finite type, regular $k$-scheme $X$. The \textit{degree} of a divisor
	\[\sum_{p\in X}[k(p):k]n_pp\]
	is $\sum_{p\in X}n_p$. Note this defines a homomorphism $\op{Div}(X)\to\ZZ$.
\end{definition}
\begin{remark}
	We are assuming that, for a closed point $p$, the extension $k(p)/k$ is finite. Roughly speaking, one can see this affine-locally: $k(p)$ is the quotient of some finitely generated $k$-algebra $k[x_1,\ldots,x_n]$ by a maximal ideal, which by some kind of Hilbert's Nullstellensatz will be a finite extension of $k$.
\end{remark}
\begin{remark}
	The point is that, for a field $k$, we have a homomorphism $\op{Div}(X)\to\op{Div}(X_{\overline k})$ by sending a point $p\in X$ to the points in the pre-image of the base-change map $X_{\overline k}\to X$ as a subscheme, and our definition shows that the following diagram commutes.
	% https://q.uiver.app/?q=WzAsMyxbMCwwLCJcXG9we0Rpdn0oWCkiXSxbMCwxLCJcXG9we0Rpdn0oWF97XFxvdmVybGluZSBrfSkiXSxbMSwxLCJcXFpaIl0sWzAsMiwiXFxkZWciXSxbMSwyLCJcXGRlZyIsMl0sWzAsMV1d&macro_url=https%3A%2F%2Fraw.githubusercontent.com%2FdFoiler%2Fnotes%2Fmaster%2Fnir.tex
	\[\begin{tikzcd}
		{\op{Div}(X)} \\
		{\op{Div}(X_{\overline k})} & \ZZ
		\arrow["\deg", from=1-1, to=2-2]
		\arrow["\deg"', from=2-1, to=2-2]
		\arrow[from=1-1, to=2-1]
	\end{tikzcd}\]
	Indeed, the claim is that, for a closed point $p\in X$, the number of points $q\in X_{\overline k}$ (counted with multiplicity) which map down to $p\in X$ is the degree of the extension $k(p)/k$. I think one can just check this affine-locally.
\end{remark}
\begin{example}
	Let's try a purely inseparable extension. Take $X=\AA^1_k=\Spec k[x]$ where $k=\FF_p(t)$ for some prime $p$. Then we have the closed point given by $\left(x^p-t\right)$, and it has degree $p$. Here, our base-change diagram is as follows.
	% https://q.uiver.app/?q=WzAsNixbMSwxLCJcXFNwZWNcXEZGX3AodClbeF0iXSxbMiwxLCJcXFNwZWNcXEZGX3AodCkiXSxbMiwwLCJcXFNwZWNcXG92ZXJsaW5le1xcRkZfcCh0KX0iXSxbMSwwLCJcXFNwZWNcXG92ZXJsaW5le1xcRkZfcCh0KX1beF0iXSxbMCwxLCJcXGRpc3BsYXlzdHlsZVxcU3BlY1xcZnJhY3tcXEZGX3AodClbeF19e1xcbGVmdCh4XnAtdFxccmlnaHQpfSJdLFswLDAsIlxcZGlzcGxheXN0eWxlXFxTcGVjXFxmcmFje1xcb3ZlcmxpbmV7XFxGRl9wKHQpfVt4XX17XFxsZWZ0KHgtdF57MS9wfVxccmlnaHQpXnB9Il0sWzUsNF0sWzQsMF0sWzUsM10sWzMsMCwiXFxwaSJdLFszLDJdLFsyLDFdLFswLDFdXQ==&macro_url=https%3A%2F%2Fraw.githubusercontent.com%2FdFoiler%2Fnotes%2Fmaster%2Fnir.tex
	\[\begin{tikzcd}
		{\displaystyle\Spec\frac{\overline{\FF_p(t)}[x]}{\left(x-t^{1/p}\right)^p}} & {\Spec\overline{\FF_p(t)}[x]} & {\Spec\overline{\FF_p(t)}} \\
		{\displaystyle\Spec\frac{\FF_p(t)[x]}{\left(x^p-t\right)}} & {\Spec\FF_p(t)[x]} & {\Spec\FF_p(t)}
		\arrow[from=1-1, to=2-1]
		\arrow[from=2-1, to=2-2]
		\arrow[from=1-1, to=1-2]
		\arrow["\pi", from=1-2, to=2-2]
		\arrow[from=1-2, to=1-3]
		\arrow[from=1-3, to=2-3]
		\arrow[from=2-2, to=2-3]
	\end{tikzcd}\]
	In particular, $\pi^{-1}$ of our divisor $\left(x^p-t\right)$ goes to $p$ copies of $\left(x-t^{1/p}\right)$. (Namely, one can look at the corresponding quasicoherent ideal sheaf of our closed embedding.)
\end{example}

\subsection{Divisor Classes}
We note that elements of $K(X)$ produce divisors as well.
\begin{definition}[principal]
	Fix a $k$-curve $X$. Given $f\in K(X)^\times$, we define the \textit{principal divisor} by
	\[\op{div}f\coloneqq\sum_{p\in X}\op{ord}_p(f),\]
	where $\op{ord}_p(f)$ is the valuation of $f$ in the discrete valuation ring $\OO_{X,p}$.
\end{definition}
Note that the locations where $f$ vanishes is some closed subscheme of $X$ not equal to $X$ and therefore dimension $0$ and therefore finite. Arguing similarly to the locations $f$ of negative valuation, we see that $\op{div}f$ does in fact have finite support and will provide us with a divisor.
\begin{lemma} \label{lem:deg-principal}
	Fix a $k$-curve $C$. Given any $f\in K(C)$, we have $\deg\op{div}f=0$.
\end{lemma}
\begin{proof}
	We simply base-change to $\overline k$ and then appeal to \cite{hartshorne}. Indeed, observe that the following diagram commutes.
	% https://q.uiver.app/?q=WzAsNixbMCwwLCJLKFgpXlxcdGltZXMiXSxbMCwxLCJLKFhfe1xcb3ZlcmxpbmUga30pXlxcdGltZXMiXSxbMSwwLCJcXG9we0Rpdn1YIl0sWzEsMSwiXFxvcHtEaXZ9WF97XFxvdmVybGluZSBrfSJdLFsyLDEsIlxcWloiXSxbMiwwLCJcXFpaIl0sWzAsMiwiXFxvcHtkaXZ9Il0sWzIsM10sWzMsNCwiXFxkZWciXSxbMiw1LCJcXGRlZyJdLFs1LDQsIiIsMSx7ImxldmVsIjoyLCJzdHlsZSI6eyJoZWFkIjp7Im5hbWUiOiJub25lIn19fV0sWzEsMywiXFxvcHtkaXZ9Il0sWzAsMV1d&macro_url=https%3A%2F%2Fraw.githubusercontent.com%2FdFoiler%2Fnotes%2Fmaster%2Fnir.tex
	\[\begin{tikzcd}
		{K(X)^\times} & {\op{Div}X} & \ZZ \\
		{K(X_{\overline k})^\times} & {\op{Div}X_{\overline k}} & \ZZ
		\arrow["{\op{div}}", from=1-1, to=1-2]
		\arrow[from=1-2, to=2-2]
		\arrow["\deg", from=2-2, to=2-3]
		\arrow["\deg", from=1-2, to=1-3]
		\arrow[Rightarrow, no head, from=1-3, to=2-3]
		\arrow["{\op{div}}", from=2-1, to=2-2]
		\arrow[from=1-1, to=2-1]
	\end{tikzcd}\]
	However, the bottom composite is the zero map by \cite{hartshorne}, so the top composite is also the zero map.
\end{proof}
As such, we have a class group.
\begin{definition}[divisor class]
	Fix a $k$-curve $X$. The quotient
	\[\op{Cl}X\coloneqq\frac{\op{Div}X}{\left\{\op{div}f:f\in K(X)^\times\right\}}\]
	is called the \textit{divisor class group} of $X$. Note that \Cref{lem:deg-principal} implies that we have a well-defined degree map.
\end{definition}
\begin{remark}
	We quickly recall that $\op{Cl}X\cong\op{Pic}X$ by sending a divisor $D$ to the line bundle
	\[\OO_X(D)(U)\coloneqq\left\{f\in K(X)^\times:f|_U+D\ge0\right\}.\]
	As such, one can roughly tell this entire story in terms of line bundles, which is perhaps more intuitive in some aspects.
\end{remark}


\end{document}