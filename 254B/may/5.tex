% !TEX root = ../notes.tex

\documentclass[../notes.tex]{subfiles}

\begin{document}

\section{May 3}

We continue with the descent obstruction.

\subsection{Twisting a Torsor}
Throughout, $K$ is a field, and $X$ is a smooth projective $K$-variety. Fix a $G$-torsor $P$; explicitly, $P$ is an $X$-scheme equipped with a $G$-action which \'etale locally looks like $P_U\cong G\times_kU$ for \'etale open sets $U\to X$.

We want to twist one torsor by another. Namely, consider $\op{Aut}_G(P)$, which is the \'etale sheaf on $X$ sending the \'etale open set $U\to X$ to the automorphisms $P_U\to U$ compatible with the $G$-action.
\begin{example}
	One has $\op{Aut}_G(G_X)\cong G_X$ by sending $g\in G_X$ to the right-multiplication map $G_X\to G_X$ given by $h\mapsto hg$.
\end{example}
However, this isomorphism depends on a choice of isomorphism $\sigma\colon P\to G_X$. If we let $\varphi_\sigma\colon\op{Aut}_G(P)\to G_X$ be the corresponding isomorphism given by the above example, then we note that a different choice of isomorphism $\sigma'\colon P\to G_X$ will have
\[\sigma'=r_{g_0}\circ\sigma\]
because $\sigma'\circ\sigma^{-1}$ is an automorphism of $G_X$. From here, one can compute $\varphi_{\sigma'}=g_0\circ\varphi_\sigma\circ g_0^{-1}$.

This problem propagates with torsors. Namely, with any $G$-torsor $P$, it will \'etale locally look like a bunch of $P|_{U_i}\cong G_{U_i}$, but the choices of these isomorphisms being non-canonical (as above) means that we want $\op{Aut}_GP$ to be a twisted form $\mc G_P$ of $G_X$ on the \'etale open sets. In fact, tracking everything through, we are essentially describing a $1$-cocycle in the inner automorphisms of $G$.

From last time, we recall that a given $G$-torsor $P$ corresponding to a class $\alpha\in H^1(X,G)$ can be decomposed into
\[X(K)=\bigsqcup_{\gamma\in H^1(K,G)}\{x\in X(K):x^*\alpha=\gamma\}.\]
For our construction, we take our $G$-torsor $P$ and some $\zeta\in H^1(K,G)$ corresponding to the $G$-torsor $Q$ on $\Spec K$ and then define
\[Q\land^GP\coloneqq(Q\times_kP)/G\]
where our $G$-action is defined by $g*(q,p)=\left(g^{-1}q,gp\right)$. Notably, this locally looks like $G\times G$ modded out by the relation $(gh,h')=(h,gh')$, which is just $G$, so we in fact have a $\mc G_P$-torsor.
\begin{remark}
	One can check that $P\land^GP$ is trivial if and only if $Q_X\cong P$.
\end{remark}
The point of this discussion is that we may define $P^\zeta\coloneqq Q\land^GP$ with canonical map $\pi^\zeta\colon P^\zeta\to X$. As such, we now note that $x\in X(K)$ has $x^P=\zeta$ if and only if $x^*P^\zeta$ is trivial, which we can see is equivalent to $x^*P^\zeta$ having a $K$-point, which is lastly equivalent to $x\in\pi^\zeta(P^\zeta(K))$. So we have achieved
\[X(K)=\bigsqcup_{\gamma\in H^1(K,G)}\pi^\gamma(P^\gamma(K)).\]
The game, now, is to try to understand $K$-points of $X$ by trying to understand the $P$s instead.

\subsection{An Example}
We will try to find rational solutions to $y^2=\left(x^2+1\right)\left(x^4+1\right)$. To make this compact, we set
\[X\coloneqq\Proj\frac{K[x,y,z]}{\left(y^2-\left(x^2+z^2\right)\left(x^4+z^4\right)\right)},\]
where we are living inside weighted projective space.

The goal is to tile ourselves by elliptic curves. Set $G=\ZZ/2\ZZ$, and we will look for $G$-torsors. For this, we note there is a map $P\to X$ given as $P\subseteq\AA^3_{x,y,w}$ cut out by the equations $y^2=\left(x^2+1\right)\left(x^4+1\right)$ and $w^2=x^4+1$, which we note is a $G$-torsor via the action of $w\mapsto-w$. To understand the twists, we note that we are looking at
\[H^1(\QQ,\ZZ/2\ZZ)=\op{Hom}(\op{Gal}(\overline\QQ/\QQ),\ZZ/2\ZZ),\]
and such a homomorphism will correspond to a quadratic extension $\QQ(\sqrt c)$ of $\QQ$. As such, we can compute our twist as $P^c$ cut out by the equations $y^2=\left(x^2+1\right)\left(x^4+1\right)$ and $cw^2=x^4+1$.

It turns out that some $c$ will produce elliptic curves. For example, $P^1$ is isomorphic to the elliptic curve $y^2=x^3-x$, and $P^2$ is isomorphic to the elliptic curve $y^2=x^3-4x$. It turns out that $\pi^1\colon P^1\to X$ and $\pi^2\colon P^2\to X$ will cover $X(\QQ)$, so we can solve the elliptic curves to finish. In total, each elliptic curve gives $4$ points, so we achieve
\[U(\QQ)=\{(0,1),(0,-1),(1,2),(1,-2),(-1,-2),(-1,2)\},\]
as achieved. (We lost a few points due to points at infinity.)

\subsection{The State of the Art}
Fix a number field $K$. Then we set
\[X(\AA_K)^{\mathrm{\acute et},\mathrm{Br}}\coloneqq\bigsqcup_{\text{twisted finite groups }\mc G/K}\bigsqcup_{P\in H^1(X,\mc G)}P(\AA_K)^{\mathrm{Br}}\]
as a subset of $X(\AA_K)$; namely, for each of the $P\in H^1(X,\mc G)$ from the descent obstruction, we can apply the Brauer--Manin obstruction.
\begin{theorem}[Poonen]
	Fix a number field $K$. Then $X(\AA_K)^{\mathrm{\acute et},\mathrm{Br}}$ is not equal to $X(K)$ in general.
\end{theorem}

\end{document}