% !TEX root = ../notes.tex

\documentclass[../notes.tex]{subfiles}

\begin{document}

\section{February 23}
Here we go.

\subsection{Bundles on \texorpdfstring{$\PP^1$}{ P1}}
We continue letting $X$ be a curve over $\CC$.
\begin{example}
	The simplest stack is $\mathrm BG$ is the quotient stack $[\mathrm{pt}/G]$. Thus, maps $S\to\mathrm BG$ are the principal $G$-bundles on $S$.
\end{example}
\begin{example}
	The stack $\op{Pic}X\coloneqq\op{Bun}_{\op{GL}_1}(X)$ classifies line bundles on $X$. (Indeed, one is simply trying to glue scalars.) For example, when $X$ is a curve, then $\op{Pic}X$ admits a surjective degree map, so it has infinitely many automorphisms. Even the degree-zero component $\op{Pic}^0X$ may not quite be a variety because line bundles may have automorphisms.
\end{example}
\begin{example}
	With $G=\op{GL}_n$, we find that $\op{Bun}_G(X)$ classifies vector bundles on $X$ of rank $n$.
\end{example}
\begin{example}
	With $G=\op{SL}_n$, we find that $\op{Bun}_G(X)$ classifies vector bundles with trivial determinant. Similarly, if $G=\op{PGL}_n$, then we are classifying vector bundles up to equivalence by tensoring with line bundles.
\end{example}
\begin{example}
	We classify $\op{GL}_2$-bundles on the complex curve $\PP^1$ as direct sums of two line bundles.
\end{example}
\begin{proof}
	Note that any bundle $E$ admits some meromorphic section on $X=\PP^1$, so $\op{Hom}(\OO(m),E)\ne0$ for sufficiently small $m$, and this map is an inclusion for generic such maps. Thus, we get a short exact sequence
	\[0\to\OO(m)\to E\to\OO(n)\to0,\]
	where $m$ is small, and $\OO(n)$ is the line bundle representing the quotient. (The quotient is a line bundle for generic inclusions $\OO(m)\to E$.) Recall that the transition function for $\OO(m)$ is given by $z^m$ (used to transition between charts). Thus, the short exact sequence given above shows that the transition function for $E$ is given by
	\[\begin{bmatrix}
		z^m & f(z) \\ 0 & z^n
	\end{bmatrix},\]
	where $f(z)\in k[z,1/z]$. By changing the choice of trivialization, one can multiply this matrix on the left or right by some upper-triangular matrices, and one can use such moves to force $f(z)$ to have monomials concentrated in degrees between $m$ and $n$.

	One can now study the actual isomorphism class of taking each affine open for $E$. In particular, letting $r$ be the maximal integer which admits an inclusion $\OO(r)\to E$, which we see lives between $m$ and $n$. By massaging with these transition maps, one is able to show that $r\le(m+n)/2$ with some degree arguments. Eventually, one finds that the corresponding short exact sequence
	\[0\to\OO(r)\to E\to\OO(r')\to0,\]
	with $r'\coloneqq m+n-r$, which by using a similar argument to before forces this short exact sequence to genuinely split. While we're here, we remark that these $r$ and $r'$ have been chosen canonically from $E$, and so the decomposition of $E$ into $\OO(r)\oplus\OO(r')$ has unique $r$ and $r'$.
\end{proof}
\begin{remark}
	Here is another way to see the class of short exact sequence of $E$: the short exact sequence is some class in
	\[\op{Ext}^1(\OO(n),\OO(m))\cong\mathrm H^1\left(X;\OO(m-n)\right),\]
	which by Serre duality is
	\[\mathrm H^0(X;\OO(n-m-2))\]
	because the canonical bundle is $\OO(-2)$. But this space gives the homogeneous polynomials of degree $n-m-2$, which has dimension $\max\{n-m-1,0\}$.
\end{remark}
\begin{remark}
	One can use this to find connected components of $\op{Bun}_{\op{GL}_2}(\PP^1)$, which are labeled by their degree $m+n$ where the bundles are $\OO(m)\oplus\OO(n)$. Further, the points inside a connected component can be labeled by the invariant $n-m$ when $n\ge m$, so the points are $P_0,P_1,\ldots,P_{n-m}$.
	
	It turns out that $P_i$ specializes to $P_j$ whenever $j\ge i$, which one sees by suitably choosing degenerations of families of bundles. For example, consider the family of bundles $E_t$ with the clutching map $\begin{bmatrix}
		\ov z^{-1} & t \\ 0 & z
	\end{bmatrix}$. Then $E_t\cong\OO\oplus\OO$ for $t\ne0$ (by explicitly finding a global trivialization), but $E_0\cong\OO(1)\oplus\OO(-1)$
\end{remark}
One can upgrade this discussion as follows.
\begin{theorem}[Grothendieck]
	A vector bundle of rank $N$ on $\PP^1$ can be uniquely written as
	\[\bigoplus_{i=1}^N\OO(m_i),\]
	where $\{m_i\}$ is a decreasing sequence of integers.
\end{theorem}
\begin{proof}
	We proceed by induction. The case of $N=1$ is classical, so we focus on the inductive step. Let $E$ be a vector bundle of rank $N+1$. As before, choose maximal $m_0$ so that there is an embedding $\OO(m_0)\to E$, and generic such choices makes the quotient into another vector bundle $E'$. Thus, by the induction, we have a short exact sequence
	\[0\to\OO(m_0)\to E\to\OO(m_1)\oplus\cdots\oplus\OO(m_N)\to0.\]
	Now, for each $i$, we can let $E'$ be the kernel of $E$ projecting onto the sum of the $\OO(m_j)$s with $j\ne i$ and $j>0$, so there is a short exact sequence
	\[0\to\OO(m_0)\to E'\to\OO(m_i)\to0.\]
	The argument on transition functions from before on $N=2$ shows that $m_0\ge(m_0+m_i)/2$, so $m_0\ge m_i$ for all $i$ follows. A calculation with $\op{Ext}$ groups as before shows that the short exact sequence for $E$ must split, so we are done! (As before, to show uniqueness of the sequence $\{m_i\}$, we simply note that the maximal term $m_0$ is uniquely constructed above.)
\end{proof}
\begin{remark}
	This argument is more or less due to Birkhoff, though reinterpreted by Grothendieck.
\end{remark}
More generally, one has the following.
\begin{theorem}
	Fix a reductive group $G$ with maximal torus $T$. Then a $G$-bundle on $\PP^1$ admits a $T$-structure, and there is a bijection given by induction
	\[[\mathrm{Bun}_T/W]\to\op{Bun}_G\]
\end{theorem}
\begin{remark}
	The moral is that bundles can be labeled by dominant integral weights for the dual group.
\end{remark}

\subsection{Double Quotients}
We will want the following motivational tool.
\begin{theorem}[Harder]
	Fix a reductive group $G$ over an algebraically closed field $k$. Then any $G$-bundle on an affine curve is trivial.
\end{theorem}
\begin{proof}
	Omitted. This is pretty hard!
\end{proof}
The moral is that one expects to be able to glue bundles over just two charts on a projective curve. For us, we will use the affine chart $X\setminus\{x\}$ and the formal disk $D_x$ around $x$, which have intersection $D_x^\times\coloneqq D_x\setminus\{x\}$.

Notably, the transition function on $D_x^\times$ is some element $g\in G(K)$, where $K=k[D_x^\times]$; note that $K$ is isomorphic to $k((t))$, where $t$ is a choice of uniformizer.\footnote{Technically, any such $g$ spreads out to an open neighborhood of $x$, which then allows us to glue.} Thus, elements $g\in G(K)$ parameterize bundles; note that there is no required cocycle condition because there are only two charts.

However, two $G$-bundles will be isomorphic if we decide to take a quotient on the right side by $G(\OO)$ and on the left side by $G(R)$, where $R=k[X\setminus\{x\}]$. It follows that
\[\op{Bun}_G(X)=G(R)\backslash G(K)/G(\OO).\]
We are now motivated to make the following definition.
\begin{definition}[affine Grassmannian]
	Fix an affine algebraic group $G$. Then we define the \textit{affine Grassmannian} to be
	\[\mathrm{Gr}_G\coloneqq G(K)/G(\OO),\]
	which we note is just $G(k((t)))/G(k[[t]])$. In particular, $\op{Gr}_G$ does not depend on the choice of curve.
\end{definition}
We can tell the same story for any finite set of points $S\subseteq X$, from which we find that
\[\op{Bun}_G(X)=G(R)\backslash\prod_{x\in S}G(K_x)/\prod_{x\in S}G(\OO_x).\]
This does not quite work when $G$ is no longer assumed to be semisimple because $G$ does not trivialize on $X\setminus\{x\}$. In fact, $G$ has no reason to trivialize on any given set of points: even line bundles have no reason to trivialize away from a given set of points.

The right thing to do is to simply take the colimit over all points in $X(k)$. Then we still know that any $G$-bundle can be trivialized away from finite set of points in $X(k)$, and we get the following.
\begin{proposition}
	Fix a curve $X$ over an algebraically closed field $k$, and let $G$ be an affine algebraic group over $k$. Then
	\[\op{Bun}_G(X)(k)=G(F)\backslash G(\AA_F)/G(\OO_\AA),\]
	where $F=k(X)$, $\AA=\prod_{x\in X(k)}(K_x,\OO_x)$, and $\OO_\AA=\prod_{x\in X(k)}\OO_x$.
\end{proposition}
\begin{remark}
	The restricted direct product appears as a colimit of those ``meromorphic power series'' which are regular almost everywhere.
\end{remark}
\begin{remark}
	The same discussion works even if $k$ is not algebraically closed. In this case, the restricted product should take place over the Galois orbits of $X(\ov k)$.
\end{remark}
\begin{remark}
	This realizes $\op{Bun}_G(X)(k)$ as an arithmetic quotient. Namely, let $F$ be a number field or  field of functions of a curve. Then let $V(F)$ be its set of valuations (constant on the base field when $F$ is over a curve), and let $F_v$ be the completion for each $v\in V(F)$, and let $\OO_v$ be the ring of integers in $F_v$ when $v$ is nonarchimedean. Then the arithmetic quotient is
	\[G(F)\backslash G(\AA)/G(\OO_\AA).\]
	The case where $F$ is a number field is of interest to number theorists. When $F=\QQ$, one finds that this double quotient is $G(\ZZ)\backslash G(\RR)$ when one has weak approximation, meaning that $\prod_p(G(\QQ_p),G(\ZZ_p))=G(\QQ)\prod_pG(\ZZ_p)$. For example, modular forms arise from the case of $G=\op{SL}_2$.
\end{remark}

\subsection{Higgs Fields}
Let's use the double quotient to write down some tangent spaces. Let $E_\gamma$ be the line bundle defined by $\gamma\in G(k((t)))$. If we are in the case that $k=\CC$ so that $G$ is a Lie group with Lie algebra $\mf g$, then taking the quotient shows that
\[T_{E_\gamma}\op{Bun}_G(X)=\frac{\mf g((t))}{\left(\op{Ad}_{\gamma}^{-1}\mf g(X\setminus\{x\})+\mf g[[t]]\right)}.\]
This is the space of \v{C}ech $1$-cocycles for the ``adjoint'' vector bundle $\op{ad}E_\gamma$ modulo coboundaries for our given open cover, so we see that $E\coloneqq E_\gamma$ ought to have
\[T_E\op{Bun}_G(X)\cong\mathrm H^1(X;\op{ad}E),\]
which by Serre duality is $\mathrm H^0(X;K_X\otimes(\op{ad}E)^*)^*$. If one fixes an invariant inner product on $\mf g$, then we may identify $\op{ad}E$ with its dual. Removing the dual on the outside produces Higgs fields.
\begin{definition}[Higgs field]
	Fix a complex curve $X$, and let $G$ be an affine algebraic group. Then the space of \textit{Higgs fields} is
	\[\mathrm H^0(X;K_X\otimes\op{ad}E).\]
	We let $\mc M_G(X)$ be the \textit{Hitchin stack} classifying pairs $(E,\varphi)$, where $E$ is a principal $G$-bundle, and $\varphi$ is a Higgs field.
\end{definition}
\begin{remark}
	This argument also works over an algebraically closed field $k$ of sufficiently large characteristic.
\end{remark}
The above discussion motivates us to study $1$-forms on $X$ valued in $\op{ad}E$. For simplicity, we assume that a generic bundle has automorphism group $Z(G)$, which is true as long as $G$ is semisimple and the genus of $X$ exceeds $1$.
\begin{notation}
	Fix a complex curve $X$, and let $G$ be an affine algebraic group. Let $\op{Bun}_G^0(X)$ be the open subset of regularly stable $G$-bundles.
\end{notation}
\begin{remark}
	Any regularly stable bundle has automorphism group $Z(G)$.
\end{remark}
\begin{example}
	If $G=\op{GL}_n$ (or $\op{PGL}_n$ or $\op{SL}_n$), then ``stable'' means that for all sub-bundles $F\subseteq E$, we have
	\[\frac{\deg F}{\op{rank}F}<\frac{\deg E}{\op{rank}E}.\]
\end{example}
\begin{remark}
	Take $G$ semisimple. For a regular stable $E$, one finds that $\mathrm H^0(X;\op{ad}E)=\op{Lie}\op{Aut}E$, which vanishes because $Z(G)$ is finite. Thus, Serre duality implies $\mathrm H^1(X;K_X\otimes\op{ad}E)=0$, so
	\[\dim\mathrm H^0(X;K_X\otimes\op{ad}E)=\chi(K_X\otimes\op{ad}E).\]
	Because $\op{ad}E$ is some degree-zero bundle, one finds that the right-hand side is $\chi(K_X)\dim\mf g$. (Indeed, one can reduce to the case of line bundles by producing a filtration of $\op{ad}E$ by degree-zero vector bundles.) Thus, our dimension is $\dim\mf g(g(X)-1)$.
\end{remark}
Thus, we see that $\op{Bun}_G^0(X)$ is a smooth variety of dimension $\dim\mf g(g(X)-1)$.
\begin{remark}
	If $G$ is reductive, then a similar calculation shows that
	\[\dim\op{Bun}_G^0(X)=\dim\mf g(g(X)-1)+\dim Z(\mf g).\]
	For example, $G=\op{GL}_n$ gives $n^2(g-1)+1$.
\end{remark}
This has application to our Hitchin stack $\mc M_G(X)$ because $T^*\op{Bun}_G^0(X)$ is an open subscheme of the Hitchin stack $\mc M_G(X)$. In fact, $T^*\op{Bun}_G^0(X)$ is a symplectic variety because cotangent spaces have a canonical symplectic structure.

\subsection{Hamiltonian Reduction}
Let's recall something about Hamiltonian reduction. If a group $H$ with Lie algebra $\mf h$ acts on a variety $Y$, then $H$ acts no $T^*Y$ in a Hamiltonian way. There is also a moment map $\mu\colon T^*Y\to\mc h^*$ which is dual to the canonical map $\mf h\to\op{Vect}Y$ given by the infinitesimal action. (Namely, the Lie algebra should act infinitesimally by vector fields.)

Now, if $H$ acts freely, then
\[\mu^{-1}(0)/H\cong T^*(Y/H).\]
In fact, $H$-stable functions $f$ on $T^*Y$ automatically descend to $T^*(Y/H)$. The left-hand object is called the Hamiltonian reduction. In this way, one sees that $T^*\op{Bun}_G(X)$ is the Hamiltonian reduction of $T^*G(K)$ by $G(R)\times G(\OO)$.
\begin{remark}
	Here is something called the Poisson bracket: given a nondegenerate symplectic form $\omega$ on $M$, meaning that $d\omega=0$, then $\omega^{-1}$ defines a non-degenerate symplectic form on $\land^2TM$. Accordingly, we may define
	\[\{f,g\}\coloneqq\left\langle\omega^{-1},df\otimes dg\right\rangle,\]
	where $f$ and $g$ live on $\OO(M)$. For example, on $M=k^{2n}$ with coordinates $\{x_i,p_i\}$, there is an explicit way to write this out in terms of some derivatives.
\end{remark}

\end{document}