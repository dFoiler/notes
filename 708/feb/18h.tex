% !TEX root = ../notes.tex

\documentclass[../notes.tex]{subfiles}

\begin{document}

\section{February 18}
Here we go.

\subsection{Principal Bundles}
Let $X$ be a smooth irreducible projective curve over a field $k$, and let $G$ be a split, connected reductive group over $k$.
\begin{example}
	We will usually take $k=\CC$ and $G$ to be some kind of linear group.
\end{example}
\begin{definition}[principal bundle]
	A \textit{principal $G$-bundle} is a variety $P$ over $k$ equipped with a map $\pi\colon P\to X$ and a right $G$-action which preserves $\pi$ and locally on $X$ is isomorphic to $G\times U\to U$.
\end{definition}
\begin{remark}
	There was some disagreement about what ``locally'' means. In general, we should work \'etale-locally, but it is enough to work Zariski-locally if $k=\CC$.
\end{remark}
\begin{remark}
	Concretely, we may define a $G$-bundle by choosing an (\'etale) atlas $\{U_i\}$ on $X$ and some ``clutching maps'' $g_{ij}\colon U_{ij}\to G$ satisfying the cocycle conditions $g_{ii}=1$ and $g_{ij}g_{ji}=1$ and $g_{ij}g_{jk}g_{ki}=1$. Indeed, given the variety $P$, we take the atlas $\{U_i\}$ to be a trivializing cover for $P$, and $g_{ij}$ is induced by the composite
	\[G\times U_i\cong\pi^{-1}(U_i)\supseteq\pi^{-1}(U_ij)\subseteq\pi^{-1}(U_j)\cong G\times U_j.\]
	Conversely, given the atlas $\{U_i\}$, we define $P$ by taking a disjoint union of the varieties $\{G\times U_i\}_i$ and then gluing along the $g_{ij}$s; the resulting $P$ is in fact a variety by some faithfully flat descent.
\end{remark}
\begin{remark}
	Let $\OO_{X,G}$ be the sheaf of functions on $X$ valued in $G$. The previous remark explains that the data of a principal $G$-bundle comes from some \v{C}ech $1$-cocycle valued in $\OO_{X,G}$, and one can check adjusting such a cocycle by a coboundary does nothing to the isomorphism class of the actual bundle. In fact, one finds that cocycle classes are in bijection with isomorphism classes of bundles, so $\mathrm H^1_{\mathrm{\acute et}}(X;\OO_{X,G})$ classifies our bundles. Note that this $\mathrm H^1_{\mathrm{\acute et}}(X;\OO_{X,G})$ is not even a group in general!
\end{remark}
Our bundles can be classified by stacks.
\begin{definition}
	There is a stack $\op{Bun}_G(X)$ which classifies $G$-bundles on $X$: namely, given a test scheme $T$, maps $T\to\op{Bun}_G(X)$ are exactly given by $G$-bundles on $X\times T$.
\end{definition}
\begin{example}
	If $G=\op{GL}_1$, then principal bundles are line bundles, so $\op{Bun}_{\op{GL}_1}(X)=\op{Pic}X$.
\end{example}
\begin{remark}
	It turns out that $\op{Bun}_G(X)$ is a smooth algebraic stack.
\end{remark}
\begin{example}
	Here is an example of a stack: given an algebraic group $H$ acting on a scheme $Y$ (not necessarily freely), the quotient stack $[Y/H]$ is defined by its functor of points. Indeed, maps from a test scheme $T$ to $[Y/H]$ are given by a pair $(E,\varphi)$ of a principal $H$-bundle on $S$ and a map $\varphi\colon E\to Y$ which commutes with the $H$-action. The point is that a test scheme $T$  should fit into a commutative diagram as follows.
	% https://q.uiver.app/#q=WzAsNCxbMSwwLCJZIl0sWzEsMSwiW1kvSF0iXSxbMCwxLCJTIl0sWzAsMCwiRSJdLFszLDBdLFszLDJdLFsyLDFdLFswLDFdLFszLDEsIiIsMSx7InN0eWxlIjp7Im5hbWUiOiJjb3JuZXIifX1dXQ==&macro_url=https%3A%2F%2Fraw.githubusercontent.com%2FdFoiler%2Fnotes%2Fmaster%2Fnir.tex
	\[\begin{tikzcd}[cramped]
		E & Y \\
		S & {[Y/H]}
		\arrow[from=1-1, to=1-2]
		\arrow[from=1-1, to=2-1]
		\arrow["\lrcorner"{anchor=center, pos=0.125}, draw=none, from=1-1, to=2-2]
		\arrow[from=1-2, to=2-2]
		\arrow[from=2-1, to=2-2]
	\end{tikzcd}\]
\end{example}
However, it is possible for $\op{Bun}_G(X)$ to have stabilizers of arbitrary dimension, so it is not a quotient stack. Instead, it turns out to be a nested union of quotient stacks.

\end{document}