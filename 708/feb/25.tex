% !TEX root = ../notes.tex

\documentclass[../notes.tex]{subfiles}

\begin{document}

\section{February 25}
Today we talk about the classical Hitchin integrable system.

\subsection{The Classical Hitchin Integrable Systems}
For today, we continue to let $X$ be an irreducible smooth projective curve over $\CC$ of genus $g\ge2$, and $G$ is a connected reductive group over $\CC$.
\begin{definition}[Hitchin base]
	The \textit{Hitchin base} of $X$ for $\op{PGL}_n$ is the vector space
	\[\mc B\coloneqq\bigoplus_{i=1}^{n-1}\mathrm H^0\left(X;K_X^{\otimes (i+1)}\right).\]
\end{definition}
\begin{remark}
	If $m\ge2$, then
	\[\dim\mathrm H^0\left(X;K_X^{\otimes m}\right)=(2m-1)(g-1)\]
	by the Riemann--Roch theorem, so
	\[\dim\mc B=\sum_{m=2}^n(2m-1)(g-1)=\left(n^2-1\right)(g-1).\]
	Thus, $\mc B$ has the same dimension as $\op{Bun}_{\op{PGL}_n}(X)$.
\end{remark}
\begin{definition}[Hithin map]
	Suppose $G=\op{PGL}_n$. The \textit{Hitchin map} is the map $p\colon T^*\op{Bun}_G^0(X)\to\mc B$ is the map
	\[\rho(E,\varphi)\coloneqq\left(\tr\land^2\varphi,-\tr\land^3\varphi,\ldots,(-1)^n\tr\land^n\varphi\right).\]
	To make sense of this, we note that $\varphi\in\mathrm H^0(X;K_X\otimes\mf{sl}(E))$ has trace zero. Notably, taking each further wedge power increases the level of the differentials.
\end{definition}
\begin{theorem}[Hitchin] \label{thm:hitchin}
	Suppose $G=\op{PGL}_n$. Then the map $p$ is generically a Lagrangian fibration, meaning that the generic fibers are Lagrangian.
\end{theorem}
\begin{proof}
	We will do this later!
\end{proof}
\begin{remark}
	For example, a dimension argument is able to show that $p$ is dominant.
\end{remark}
\begin{remark}
	Concretely, this means that coordinate functions on $\mc B$ will pull back to algebraically independent functions, and their Poisson brackets vanish. Explicitly, by choosing a basis $\{b_j\}$ of $\mc B$ (compatible with the direct sum decomposition of $\mc B$), we see that
	\[p(E,\varphi)=\sum_jH_j(E,\varphi)b_j,\]
	where $H_j\colon T^*\op{Bun}_G^0(X)\to\CC$ are some algebraically independent coordinate functions, and $\{H_i,H_j\}=0$ for all $i$ and $j$. This is exactly the data of an integrable system.
\end{remark}
\begin{remark}
	For example, we can see that the fiber over $0$ is the nilpotent cone. Thus, $p$ is not smooth! However, $p$ turns out to be flat.
\end{remark}
One can define the Hitchin map in other situations. For example, $\op{GL}_n$ should add in the trace.
\begin{definition}[Hitchin map]
	The \textit{Hitchin base} of $X$ for $\op{GL}_n$ is the vector space
	\[\mc B\coloneqq\bigoplus_{i=0}^{n-1}\mathrm H^0\left(X;K_X^{\otimes (i+1)}\right).\]
	In this case, the \textit{Hitchin map} $p\colon T^*\op{Bun}_G^0(X)\to\mc B$ is the map
	\[\rho(E,\varphi)\coloneqq\left(-\tr\varphi,\tr\land^2\varphi,-\tr\land^3\varphi,\ldots,(-1)^n\tr\land^n\varphi\right).\]
\end{definition}
\begin{remark}
	Here, we find that $\dim\mc B=\left(n^2-1\right)(g-1)=n^2(g-1)+1$, which is $\dim\op{Bun}_G^0(X)$.
\end{remark}
We turn to general groups. To define our Hitchin base, we must generalize our functions $\tr\land^\bullet\varphi$. The importance of these functions is that they are functions on $\mf g$ which are $G$-invariant (with the adjoint action). This is desirable because we want to define a function on pairs $(E,\varphi)$, but changing charts may adjust the local coordinates of $E$ by some conjugation by elements of $G$. Thus, whatever function on $\varphi$ we choose needs to be invariant!

We are thus motivated to study $\CC[\mf g]^G$, for which we recall the following result.
\begin{theorem}[Chevalley]
	Fix a connected reductive group $G$ over $\CC$ with Lie algebra $\mf g$ and rank $r$. Then there are homogeneous polynomials $\{_1,\ldots,Q_r\}$ for which
	\[\CC[\mf g]^G=\CC[Q_1,\ldots,Q_r].\]
	In fact, the tuple of degrees $\{\deg Q_i\}_i$ are independent of the precise choices of $Q_\bullet$s.
\end{theorem}
\begin{example}
	For $G=\op{GL}_n$, the degrees are $\{1,\ldots,n\}$, and one can take $Q_i\coloneqq(-1)^i\tr\land^i\varphi$ for each $\varphi\in\mf g$.
\end{example}
\begin{example}
	For $G=\op{SL}_n$, the degrees are $\{2,\ldots,n\}$. One can take $Q_i\coloneqq(-1)^i\tr\land^i\varphi$ again, where $i\in\{2,\ldots,n\}$.
\end{example}
\begin{remark}
	It turns out that the degrees $\{d_i\}$ of the group $G$ arise via the decomposition
	\[\mf g=\bigoplus_{i=1}^rL_{2(d_i-1)}\]
	as a representation of the principal $\mf{sl}_2$-triple. (The principal $\mf{sl}_2$-triple is given by setting $e$ to be the sum of the simple $e_i$s, which is a regular nilpotent element.)
\end{remark}
\begin{definition}[Hitchin map]
	Let the degrees of $G$ be $\{d_i\}_{i=1}^r$. The \textit{Hitchin base} for $G$ is
	\[\mc B_{G,X}\coloneqq\bigoplus_{i=1}^r\mathrm H^0\left(X;K_X^{\otimes d_i}\right).\]
	When no confusion is possible, we write $\mc B$ for $\mc B_{G,X}$. We define the \textit{Hitchin map} $p\colon T^*\op{Bun}_G^0(X)\to\mc B$ given by
	\[p(E,\varphi)\coloneqq(Q_1(\varphi),\ldots,Q_r(\varphi)).\]
\end{definition}
\begin{remark}
	The space $\mc B_{G,X}$ admits a ``canonical'' definition as the space of sections of the weighted vector bundle $\mf h/W\otimes K_X$.
\end{remark}
\begin{remark}
	By the same Riemann--Roch argument, one finds that
	\[\dim\mc B=\sum_{i=1}^r(2d_i-1)(g-1)+\#\{i:d_i=1\}.\]
	The latter term is the rank $\dim Z(G)$, so the entire dimension is $\dim\op{Bun}_G^0(X)$.
\end{remark}
We are now ready to restate \Cref{thm:hitchin}.
\begin{theorem} \label{thm:hitchin-general}
	Fix an irreducible smooth projective curve $X$ over $\CC$ of genus $g\ge2$, and let $G$ be a connected reductive group over $\CC$. Then the Hitchin map $p\colon T^*\op{Bun}^0_G(X)\to\mc B$ is generically a Lagrangian fibration.
\end{theorem}
We will prove that the relevant $H_\bullet$s are Poisson-commutative for all semisimple $G$, but we will only show the independence for $G=\op{GL}_n$.
\begin{proof}[Proof of commutativity]
	For the Poisson commutativity, we use Hamiltonian reduction. Choose a local coordinate $t$ at some $x\in X$. Then our discussion of double quotients granted
	\[\op{Bun}_G(X)=G(R)\backslash G(\CC((t)))/G(\CC[[t]]),\]
	where $R=\CC[X\setminus\{x\}]$. (We have identified $\CC[[t]]$ with $\widehat\OO_x$.) Then the pre-image of $\op{Bun}_G^0(X)$ is some $G(\CC((t)))^0$.

	Now, the group $H\coloneqq G(R)\times G(\CC[[t]])/Z(G)$ acts freely on $Y\coloneqq G(\CC[[t]])^0$, and the quotient is $\op{Bun}_G^0(X)$. Because this action is free, we conclude that
	\[T^*(Y/H)=T^*Y/H,\]
	where the latter is the Hamiltonian reduction $\mu^{-1}(\{0\})$, where $\mu\colon T^*\to\mf h^*$ is the moment map. Now, if $\{F_i\}$ are $H$-invariant Poisson-commutative functions on $T^*Y$, then they descend to $\mu^{-1}(\{0\})$ by restriction and then to $\mu^{-1}(\{0\})/H$ by the invariance, so they give functions in $T^*(Y/H)$. In fact, these descended functions continue to be Poisson-commutative. The moral of this reduction is that it is comparatively easier to find functions on $T^*Y$.

	We are thus on the hunt for Poisson-commutative functions on
	\[T^*G(\CC((t)))^0=\mf g((t))^*\times G((t))^0.\]
	The idea is to use left translation to produce invariant functions. Let's start by computing this dual. Note that $\mf g((t))=\mf g\otimes\CC((t))$, and there is a canonical invariant pairing $\CC((t))\times\CC((t))\,dt\to\CC$ given by the residue, so $\mf g((t))^*=\mf g\otimes\CC((t))\,dt$.

	Let's now define our functions. By definition, a point in $T^*G(\CC((t)))$ is represented by a pair $(g,\varphi)$, where $g\in G((t))$, and $\varphi\in\mf g\otimes\CC((t))\,dt$. But this $\varphi$ is a Higgs field in the punctured formal disk $D_x^\times$, so we can use our prior discussion to get well-defined maps
	\[\varphi\mapsto(Q_1(\varphi),\ldots,Q_r(\varphi)),\]
	where $Q_i(\varphi)\in\CC((t))\,(dt)^{\otimes d_i}$. Taking Laurent coefficients of $Q_i$ produces many functions $Q_{i,n}$ on $T^*G((t))$. They are conjugation-invariant (because $Q_i$ is), and they are invariant by left translation (because of how we parallelized our cotangent space), so they are invariant under the two-sided action by $G(R)\times G(\CC[[t]])$.
	
	Let's check that the $Q_{i,n}$s are Poisson-commutative. In general, for any Lie group $A$ with Lie algebra $\mf a$, we are finding a Poisson structure on $\mf a^*$ given by the Lie bracket: choosing a basis $\{a_i\}$ of $\mf a$ so that $[a_i,a_j]=\sum_{k}c_{ij}^ka_k$, one defines
	\[\{f,g\}\coloneqq\sum_{i,j,k}c_{ij}^k\frac{\del f}{\del a_i}\frac{\del g}{\del a_j}a_k\]
	for any $f,g\colon\mf a^*\to\CC$. In particular, if $f$ is invariant, then $\{a_j,f\}=0$ for all $j$ by a direct calculation, so $\{-,f\}$ vanishes in general. The same sort of argument works in our situation because the functions $Q_{i,n}$ are suitable invariant.
	
	Thus, we receive functions which descend to $Y/H$ on $\op{Bun}_G^0(X)$. Poisson-commutativity will now follow as soon as we show that these functions descend to the required Hitchin map $p$. Well, Hamiltonian reduction produces a diagram
	% https://q.uiver.app/#q=WzAsNSxbMSwwLCJUXipHKFxcQ0MoKHQpKSleMCJdLFsyLDAsIlxcZGlzcGxheXN0eWxlXFxiaWdvcGx1c197aT0xfV5yXFxDQygodCkpXFwsZHRee1xcb3RpbWVzIGRfaX0iXSxbMCwwLCJcXG11XnstMX0oXFx7MFxcfSkiXSxbMCwxLCJUXipcXG9we0J1bn1fR14wKFgpIl0sWzIsMSwiXFxkaXNwbGF5c3R5bGVcXGJpZ29wbHVzX3tpPTF9XnJcXG1hdGhybSBIXjBcXGxlZnQoWDtLX1hee1xcb3RpbWVzIGRfaX1cXHJpZ2h0KSJdLFswLDEsIlFfMSxcXGxkb3RzLFFfciJdLFsyLDNdLFszLDRdLFsyLDBdLFs0LDFdXQ==&macro_url=https%3A%2F%2Fraw.githubusercontent.com%2FdFoiler%2Fnotes%2Fmaster%2Fnir.tex
	\[\begin{tikzcd}[cramped]
		{\mu^{-1}(\{0\})} & {T^*G(\CC((t)))^0} & {\displaystyle\bigoplus_{i=1}^r\CC((t))\,dt^{\otimes d_i}} \\
		{T^*\op{Bun}_G^0(X)} && {\displaystyle\bigoplus_{i=1}^r\mathrm H^0\left(X;K_X^{\otimes d_i}\right)}
		\arrow[from=1-1, to=1-2]
		\arrow[from=1-1, to=2-1]
		\arrow["{Q_1,\ldots,Q_r}", from=1-2, to=1-3]
		\arrow[from=2-1, to=2-3]
		\arrow[from=2-3, to=1-3]
	\end{tikzcd}\]
	where the right vertical map is given by expanding the global differentials at $x$. It is not too hard to check that this diagram commutes, so the $p^*b_j$s are linear combinations of the $Q_{i,n}$s, so they are Poisson commutative.
\end{proof}
\begin{remark}
	The expansion of $p^*b_j$ in terms of the $Q_{i,n}$s is far from unique!
\end{remark}

\subsection{Spectral Curve}
It remains to show the algebraic independence in \Cref{thm:hitchin-general}, which we only do for $G=\op{GL}_n$. For this, we will use the notion of the ``spectral curve.''
\begin{definition}[spectral curve]
	Take $G=\op{GL}_n$. For $b\in\mc B$, write $b=(b_1,\ldots,b_n)$, where we have $b_i\in\mathrm H^0(X;K_X^{\otimes i})$ for each $i$, and consider the factorized polynomial
	\[\lambda^n+b_1\lambda^{n-1}+\cdots+b_n=\prod_{i=1}^n(\lambda-\lambda_i).\]
	Thus, for each $x\in X$, we receive a set of roots $\{\lambda_1,\ldots,\lambda_n\}$in $T_x^*X=K_{X,x}$. As $x\in X$ varies, this cuts out the \textit{spectral curve} $C$ of $X$. Equivalently, $C\subseteq T^*X$ cut out by the above polynomial (which notably outputs to $(T^*X)^{\otimes n}$).
\end{definition}
\begin{remark}
	The curve $C$ has no reason to be connected or regular, but it is projective.
\end{remark}
\begin{remark}
	This is called the spectral curve because if $b=p(E,\varphi)$, then it turns out that $C$ gives the spectrum of $\varphi$.
\end{remark}
Here are some facts about this spectral curve.
\begin{theorem}[Hitchin]
	For generic $b\in\mc B$, the fiber $C_b$ is smooth and irreducible.
\end{theorem}
\begin{proof}
	The fiber being smooth and irreducible is an open condition, so it is enough to exhibit a single $b\in\mc B$. We will want the following result.
	\begin{lemma}
		Fix a smooth irreducible curve $X$ over $\CC$ of genus $g\ge2$.
		\begin{listalph}
			\item If $\mc L$ is a line bundle on $X$ of degree at least $2g$, hen a generic sectino of $\mc L$ has only simple zeroes.
			\item A generic section of $\mc K_X^{\otimes n}$ has only simple zeroes for $n\ge2$.
		\end{listalph}
	\end{lemma}
	\begin{proof}
		Quickly, note that (a) implies (b) by a degree calculation. It remains to show (a). Well, let $Y\subseteq\mathrm H^0(X;\mc L)$ be the subset of sections which have a double root, so we want to show that $Y$ is positive codimension. By the Riemann--Roch theorem, we know that any line bundle $\mc M$ of degree $m\ge2g-2$ has
		\[\dim\mathrm H^0(X;\mc M)=\begin{cases}
			m-g+1 & \text{if }\mc M\ne K_X, \\
			g & \text{if }\mc M=K_X.
		\end{cases}\]
		Now, for each $s\in Y$ with a double root at $z\in X$, we can view $s$ as a section of $\mc L\otimes\OO(-2z)$. But $\deg(\mc L\otimes\OO(-2z))\ge2g-2$. Thus, if $\mc L\otimes\OO(-2z)\ne\mc K_X$ (which happens for all but finitely many $z$ because this requires $\OO(2z)$ to equal a fixed line bundle, which is impossible for divisor reasons), then the space of global sections has dimension $\deg\mc L-g-1$. Looping over all $z$s shows that
		\[\dim Y\le d-g<\dim\mathrm H^0(X;\mc L),\]
		so we are done.
	\end{proof}
	We work with $b$ of the form $(0,\ldots,0,s)$, where $s\in\mathrm H^0\left(X;\mc K_X^{\otimes n}\right)$ is chosen generically to have no simple zeroes. Thus, our spectral curve is defined by the equation
	\[\lambda^n=s(x),\]
	which one can check is smooth and irreducible. Indeed, it is irreducible because it is 
\end{proof}
\begin{remark}
	In fact, this argument can also show that the fibers $C_b$ generically have genus $n^2(g-1)+1$. Indeed, one can calculate this genus via the Hurwitz formula with the projection $C_b\to X$. Indeed, the section $s$ admits $2n(g-1)$ simple zeroes, so
	\[\chi(C_b)=n\chi(X)-(n-1)\cdot 2n(g-1),\]
	which rearranges into the required claim.
\end{remark}

\end{document}