% !TEX root = ../notes.tex

\documentclass[../notes.tex]{subfiles}

\begin{document}

\section{February 11}
Today, we give some applications of crystalline cohomology.

\subsection{The Ax--Katz Theorem}
Crystalline cohomology comes with a natural Frobenius.
\begin{example}
	For any smooth scheme $X$ over $k$, there is a Frobenius $F\colon X\to X^{(1)}$, which produces an endomorphism
	\[F^*\colon\mathrm R\Gamma_{\mathrm{cris}}\left(X^{(1)}/W(k)\right)\to\mathrm R\Gamma_{\mathrm{cris}}(X/W(k)).\]
	For example, if $X$ is defined over $\FF_{q^r}$, then $X=X^{(r)}$, then we have an endomorphism $(F^*)^r$ of the complex $\mathrm R\Gamma_{\mathrm{cris}}(X/W(k))$.
	% By comparing this with de Rham cohomology, we see that $\OO_X$ being acyclic for Zariski cohomology implies that $\#X(\FF_q)\equiv1\pmod p$.
\end{example}
\begin{remark} \label{rem:lefschetz-trace}
	If $X$ is smooth and proper over $\FF_q$ where $q=p^r$, then $\mathrm H^n_{\mathrm{cris}}(X/W(\FF_q))$ is finitely generated over $W(\FF_q)$, and it turns out that
	\[\#X(\FF_q)=\sum_{i\ge0}(-1)^i\tr\left((F^*)^r;\mathrm H^i_{\mathrm{cris}}(X/W(\FF_q))\left[\frac1p\right]\right).\]
	One has to check that crystalline cohomology is a Weil cohomology theory, which is true. It is not too difficult to check this formula directly when $X$ is zero-dimensional, meaning that it is a finite disjoint union of points.
\end{remark}
\begin{theorem}[Ax--Katz] \label{thm:ax-katz}
	Let $X$ be a smooth proper geometrically connected variety over $\FF_q$ for which $\mathrm H^i(X;\OO_X)=0$ for all $i>0$. Then
	\[\#X(\FF_q)\equiv1\pmod q.\]
\end{theorem}
\begin{proof}
	The idea is to use \Cref{rem:lefschetz-trace}. Note that $\mathrm H^0_{\mathrm{cris}}(X/W(\FF_q))$ is the one-dimensional space with trivial Frobenius action, so it is enough to show that
	\[\tr\left((F^*)^r;\mathrm H^i_{\mathrm{cris}}(X/W(\FF_q))\left[\frac1p\right]\right)\equiv0\pmod q\]
	for all $i\ge1$. With an inversion of $p$, one will not be able to do better than$\pmod p$ information, so we see that we have to use the integral structure.

	For a given module $M$ over $W(\FF_q)$, we let $M_{\mathrm{tf}}$ denotes the quotient by torsion so that $M_{\mathrm{tf}}$ is torsion-free. It is now enough to show that the map
	\[F^*\colon\mathrm H^i_{\mathrm{cris}}(X^{(1)}/W(\FF_q))_{\mathrm{tf}}\to\mathrm H^i_{\mathrm{cris}}(X/W(\FF_q))_{\mathrm{tf}}\]
	is divisible by $p$. Indeed, the trace after $(-)[1/p]$ is the same as the trace on the torsion-free part (because then inverting $p$ does nothing), so we can just iterate this divisibility $r$ times to prove the claim. We now reduce$\pmod p$, for which we recall that
	\[\mathrm R\Gamma_{\mathrm{cris}}(X/W(\FF_q))\otimes^{\mathbb L}_{W(\FF_q)}\FF_q\cong\mathrm R\Gamma_{\mathrm{dR}}(X/\FF_q).\]
	Now, the same argument as in Universal coefficient theorem yields an exact sequence
	\[0\to\mathrm H^n_{\mathrm{cris}}(X/W(\FF_q))/p\to\mathrm H^n_{\mathrm{dR}}(X/\FF_q)\to\mathrm H^{n+1}_{\mathrm{cris}}(X/W(\FF_q))[p]\to0.\]
	By functoriality of the short exact sequence, if we could show that Frobenius was zero on de Rham cohomology (i.e., the map $F^*\colon\mathrm H^n_{\mathrm{dR}}(X^{(1)}/\FF_q)\to\mathrm H^n_{\mathrm{dR}}(X/\FF_q)$ vanishes), then it would vanish on $\mathrm H^n_{\mathrm{cris}}(X/W(\FF_q))/p$ and thus vanish on $\mathrm H^n_{\mathrm{cris}}(X/W(\FF_q))_{\mathrm{tf}}/p$, which was the required claim.

	We are thus reduced to computing some Frobenius action on the de Rham complex. Let's recall how this Frobenius action is defined: by pulling back along $F^*$, there are natural maps
	% https://q.uiver.app/#q=WzAsOCxbMCwwLCJGXnstMX1cXE9PX3tYXnsoMSl9fSJdLFsxLDAsIkZeey0xfVxcT21lZ2FeMV97WF57KDEpfX0iXSxbMiwwLCJGXnstMX1cXE9tZWdhXjJfe1heeygxKX19Il0sWzAsMSwiXFxPT19YIl0sWzEsMSwiXFxPbWVnYV4xX1giXSxbMiwxLCJcXE9tZWdhXjJfWCJdLFszLDAsIlxcY2RvdHMiXSxbMywxLCJcXGNkb3RzIl0sWzAsMywiRl4qIiwyXSxbMSw0LCJGXioiLDJdLFsyLDUsIkZeKiIsMl0sWzAsMV0sWzEsMl0sWzIsNl0sWzMsNF0sWzQsNV0sWzUsN11d&macro_url=https%3A%2F%2Fraw.githubusercontent.com%2FdFoiler%2Fnotes%2Fmaster%2Fnir.tex
	\[\begin{tikzcd}[cramped]
		{F^{-1}\OO_{X^{(1)}}} & {F^{-1}\Omega^1_{X^{(1)}}} & {F^{-1}\Omega^2_{X^{(1)}}} & \cdots \\
		{\OO_X} & {\Omega^1_X} & {\Omega^2_X} & \cdots
		\arrow[from=1-1, to=1-2]
		\arrow["{F^*}"', from=1-1, to=2-1]
		\arrow[from=1-2, to=1-3]
		\arrow["{F^*}"', from=1-2, to=2-2]
		\arrow[from=1-3, to=1-4]
		\arrow["{F^*}"', from=1-3, to=2-3]
		\arrow[from=2-1, to=2-2]
		\arrow[from=2-2, to=2-3]
		\arrow[from=2-3, to=2-4]
	\end{tikzcd}\]
	which then descend to cohomology. But note that $F^*(f\,dg)=f^p\,d(g^p)=0$, so $F^*$ vanishes on all the higher terms. Thus, the above morphism $F^*$ of complexes factors through the complex $\OO_{X}\to0\to0\to\cdots$. Thus, after taking hypercohomology, we see that $F^*$ factors through $\mathrm H^n\left(X;\OO_{X}\right)$, which vanishes by assumption!
\end{proof}
\begin{remark}
	We can remove the properness assumption, and one still finds that $F^*$ finds on de Rham cohomology.
\end{remark}
\begin{remark}[Chevalley--Warning]
	If $X\subseteq\PP^N_{\FF_q}$ is a hypersurface of degree at most $N$, then the hypothesis is satisfied. It is a classical result of Chevalley--Warning that $\#X(\FF_q)\equiv1\pmod p$. A cohomological argument could proceed by using the Atiyah--Bott formula
	\[\#X(\FF_q)=\sum_{i\ge0}(-1)^i\tr\left((F^*)^r;\mathrm H^n(X;\OO_X)\right)\pmod p\]
	and then argue as above. To access$\pmod q$ information, we need a different cohomology theory.
\end{remark}
\begin{remark}
	If one wants to prove that the Frobenius acting on crystalline cohomology was instead divisible by $p^2$, then there starts to be contributions of the cohomology of $\Omega_X$, via some discussion of Hodge--Tate weights.
\end{remark}

\subsection{A Weak Hodge Decomposition}
Here is another application.
\begin{notation}
	Given a formal scheme $\widetilde X$ over $W(k)$, we define
	\[\widetilde X^{(1)}\coloneqq\widetilde X\times_{W(k)}W(k),\]
	where the internal map $W(k)\to W(k)$ is the natural Frobenius on $W(k)$.
\end{notation}
\begin{proposition} \label{prop:lift-gives-quasi-iso}
	Fix a smooth scheme $X$ over a perfect field $k$ of characteristic $p$. Suppose that we can lift $X$ to a formal scheme $\widetilde X$ over $W(k)$ as well as a lift $\widetilde F\colon\widetilde X\to\widetilde X^{(1)}$ of $F$. Then the de Rham complex
	\[\OO_{\widetilde X}\stackrel d\to\Omega^1_{\widetilde X}\stackrel d\to\Omega^2_{\widetilde X}\stackrel d\to\cdots\]
	is quasi-isomorphic to
	\[\OO_{\widetilde X^{(1)}}\stackrel {pd}\to\Omega^1_{\widetilde X^{(1)}}\stackrel {pd}\to\Omega^2_{\widetilde X^{(1)}}\stackrel {pd}\to\cdots.\]
\end{proposition}
\begin{remark}
	If $X$ is affine, then such lifts $\widetilde X$ and $\widetilde F$ always exist.
\end{remark}
As an application, we get a version of the Hodge decomposition.
\begin{corollary} \label{cor:get-weak-hodge}
	Fix a smooth scheme $X$ over a perfect field $k$ of characteristic $p$. Suppose that we can lift $X$ to a formal scheme $\widetilde X$ over $W(k)$ as well as a lift $\widetilde F\colon\widetilde X\to\widetilde X^{(1)}$ of $F$. Then the de Rham complex
	\[\OO_X\stackrel d\to\Omega^1_X\stackrel d\to\Omega^2_X\to\cdots\]
	is quasi-isomorphic to
	\[\OO_{X^{(1)}}\stackrel0\to\Omega^1_{X^{(1)}}\to\Omega^2_{X^{(2)}}\to\cdots.\]
	Thus,
	\[\mathrm H^n_{\mathrm{dR}}(X/k)=\bigoplus_{i+j=n}^n\mathrm H^i\left(X^{(1)};\Omega^j_{X^{(1)}}\right).\]
\end{corollary}
\begin{proof}
	To show the quasi-isomorphism, merely reduce \Cref{prop:lift-gives-quasi-iso} modulo $p$, and then we know that crystalline cohomology becomes de Rham cohomology. To prove the last claim, note that the hypercohomology of the first complex is the de Rham cohomology, and hypercohomology of the second complex produces the right-hand side. Indeed, to take hypercohomology, one needs to resolve the second complex, but the ``columns'' in our resolution do not need to communicate with each other, so
	\[\mathbb H^n\left(\OO_{X^{(1)}}\stackrel0\to\Omega^1_{X^{(1)}}\to\Omega^2_{X^{(2)}}\to\cdots\right)=\bigoplus_{i+j=n}^n\mathrm H^i\left(X^{(1)};\Omega^j_{X^{(1)}}\right)\]
	follows after we keep track of our degrees.
\end{proof}
\begin{proof}[Proof of \Cref{prop:lift-gives-quasi-iso}]
	We are interested in filling in the following diagram.
	% https://q.uiver.app/#q=WzAsOCxbMCwwLCJcXE9PX3tcXHdpZGV0aWxkZSBYfSJdLFsxLDAsIlxcT21lZ2FeMV97XFx3aWRldGlsZGUgWH0iXSxbMiwwLCJcXE9tZWdhXjJfe1xcd2lkZXRpbGRlIFh9Il0sWzAsMSwiXFxPT197XFx3aWRldGlsZGUgWF57KDEpfX0iXSxbMSwxLCJcXE9tZWdhXjFfe1xcd2lkZXRpbGRlIFheeygxKX19Il0sWzIsMSwiXFxPbWVnYV4yX3tcXHdpZGV0aWxkZSBYXnsoMSl9fSJdLFszLDAsIlxcY2RvdHMiXSxbMywxLCJcXGNkb3RzIl0sWzMsMCwiXFx3aWRldGlsZGUgRl4qIl0sWzQsMSwiIiwwLHsic3R5bGUiOnsiYm9keSI6eyJuYW1lIjoiZGFzaGVkIn19fV0sWzUsMiwiIiwwLHsic3R5bGUiOnsiYm9keSI6eyJuYW1lIjoiZGFzaGVkIn19fV0sWzAsMV0sWzEsMl0sWzIsNl0sWzMsNF0sWzQsNV0sWzUsN11d&macro_url=https%3A%2F%2Fraw.githubusercontent.com%2FdFoiler%2Fnotes%2Fmaster%2Fnir.tex
	\[\begin{tikzcd}[cramped]
		{\OO_{\widetilde X}} & {\Omega^1_{\widetilde X}} & {\Omega^2_{\widetilde X}} & \cdots \\
		{\OO_{\widetilde X^{(1)}}} & {\Omega^1_{\widetilde X^{(1)}}} & {\Omega^2_{\widetilde X^{(1)}}} & \cdots
		\arrow[from=1-1, to=1-2]
		\arrow[from=1-2, to=1-3]
		\arrow[from=1-3, to=1-4]
		\arrow["{\widetilde F^*}", from=2-1, to=1-1]
		\arrow[from=2-1, to=2-2]
		\arrow[dashed, from=2-2, to=1-2]
		\arrow[from=2-2, to=2-3]
		\arrow[dashed, from=2-3, to=1-3]
		\arrow[from=2-3, to=2-4]
	\end{tikzcd}\]
	Staring at the diagram, we see that the map $\Omega^i_{\widetilde X^{(1)}}\to\Omega^i_{\widetilde X}$ had better be $p^{-i}F^*$. The proof of \Cref{thm:ax-katz} checked that $p^{-1}F^*$ makes sense for $i>0$, and by discussing Frobenius on higher wedges of differentials finds that even $p^{-i}F^*$ will always make sense.

	We have thus provided our morphism of complexes, which we want to be a quasi-isomorphism. Because the terms are $p$-adically complete, it turns out that it is enough to check that it is a quasi-isomorphism$\pmod p$.\footnote{This is a variant of Nakayama's lemma. For example, a sample statement is that a morphism $f\colon M\to N$ of torsion-free $p$-adically complete $\ZZ_p$-modules (meaning that $M\cong\lim M/p^\bullet$) can be checked to be an isomorphism$\pmod p$. After deriving, it turns out that we don't have to check that it is torsion-free.} It turns out that one can proceed as in the proof of \Cref{thm:cartier-iso}.
\end{proof}
This decomposition remains true with weaker assumptions, which will be our next goal.
\begin{theorem}[Berthelot--Ogus]
	Fix a smooth scheme $X$ over a perfect field $k$ which can be lifted to a formal scheme $\widetilde X$ over $W(k)$. Suppose further that $\dim X<p$. Then
	\[\OO_{\widetilde X}\stackrel d\to\Omega^1_{\widetilde X}\stackrel d\to\Omega^2_{\widetilde X}\stackrel d\to\cdots\stackrel d\to\Omega^{\dim X}_{\widetilde X}\]
	is quasi-isomorphic to
	\[\OO_{\widetilde X^{(1)}}\stackrel {pd}\to\Omega^1_{\widetilde X^{(1)}}\stackrel {pd}\to\Omega^2_{\widetilde X^{(1)}}\stackrel {pd}\to\cdots\stackrel{pd}\to\Omega^{\dim X}_{\widetilde X^{(1)}}.\]
\end{theorem}
\begin{remark} \label{rem:better-hodge}
	The same argument as in \Cref{cor:get-weak-hodge} is able to show that
	\[\mathrm H^n_{\mathrm{dR}}(X/k)\approx\bigoplus_{i+j=n}\mathrm H^i(X^{(1)};\Omega^j_{X^{(1)}}).\]
	(We will not be precise about what $\approx$ means now.) It turns out that this decomposition remains true if $X$ can merely be lifted to $W_2(k)$, and it even depends on this choice of lift.
\end{remark}
More precisely, our goal will be to prove the above remark.

\subsection{\texorpdfstring{$\mc D$}{ D}-modules by Crystals}
In order to access our Hodge decomposition, we are going to realize $\mc D_X$-modules as quasicoherent sheaves on some stack $X^{\mathrm{dR}}$; if $X$ is affine, then $X^{\mathrm{dR}}$ will be able to be realized as the quotient of some formal scheme by a formal group scheme. Notably, quasicoherent sheaves can then just be thought of as certain equivariant sheaves on a formal scheme.

To understand where $X^{\mathrm{dR}}$ may come from, we return to characteristic $0$.
\begin{lemma} \label{lem:trivialize-fiber}
	Fix a smooth scheme $F$ over a field $F$ of characteristic zero. For a $\mc D_X$-module $M$ and a point $x\in X(F)$, there is a natural isomorphism
	\[M|_{\widehat{\OO}_{X,x}}\cong M_x\widehat{\otimes}_F\widehat{\OO}_{X,x}.\]
	Here, the left-hand side is $\lim M\otimes_{\OO_X}\OO_{X,x}/\mf m_x^\bullet$, and $M_x=M\otimes_{\OO_X}\OO_{X,x}/\mf m_x$.
\end{lemma}
\begin{remark}
	The moral is that there is a canonical trivialization ``$M_x$'' of $M$ in a formal neighborhood of a point.
\end{remark}
\begin{proof}
	We will produce an isomorphism $M_x\to\left(M|_{\widehat{\OO}_{X,x}}\right)^{\nabla=0}$, where the target means that we are looking for sections with flat connection. Accordingly, choose $s\in M_x$, which can be lifted to some $\widetilde s\in M(U)$. Now, $\widetilde s$ has no reason to be flat, so we have to fix it. Accordingly, we use Taylor's formula to expand out $U$ in local coordinates $(t_1,\ldots,t_d)$ and then define
	\[\sum_{i_1,\ldots,i_d\ge0}(-1)^{i_1+\cdots+i_d}\frac{t_1^{i_1}\cdots t_d^{i_d}}{i_1!\cdots i_d!}\nabla_{\del_{t_1}}^{i_1}\circ\cdots\circ\nabla_{\del_{t_d}}^{i_d}(\widetilde s),\]
	and one can hit this with $\nabla$ to check that it is a flat section. Further, note that $i_1=\cdots=i_d=0$ reproduces $s$, so this section continues to lift $s$. We have thus defined our map, and one can check that it produces the required isomorphism by Nakayama's lemma.
\end{proof}
\begin{example}
	Let's write this out for $d=1$. Then in the first degree, we have replaced $\widetilde s$ with $t-\nabla_{\del_t}\widetilde s$; hitting this with $\nabla_{\del_t}$, we receive $-t\nabla_{\del_t}^2(\widetilde s)$. This is still nonzero, and we then further correct by $\frac12t^2\nabla^2_{\del_t}(\widetilde s)$ and continue.
\end{example}
We may want to extend these canonical trivializations into a full trivialization of $M$, but then we have to explain how to relate the various points.
\begin{notation}
	Fix a smooth scheme $X$ over a field $F$ of characteristic zero. Then we define $(X\times_FX)^\land_\Delta$ to be the union of the closed subschemes of $X\times X$ cut out by the powers of the quasicoherent ideal sheaf $\mc I_\Delta$ (of the diagonal). We can define other powers similarly.
\end{notation}
\begin{remark}
	Intuitively, $(X\times_FX)^\land_\Delta$ is the union of all nilpotent thickenings of $\Delta$.
\end{remark}
\begin{theorem} \label{thm:d-mod-to-crystal}
	Fix a smooth scheme $X$ over a field $F$ of characteristic zero. Define the formal scheme $(X\times_FX)^\land_\Delta$ to be the formal completion at the diagonal, which is the union of the nilpotent thickenings of the diagonal $\Delta$. Then the data of a $\mc D_X$-module $M$ is equivalent to the data of a quasicoherent sheaf $M$ on $X$ and an isomorphism $\alpha\colon\op{pr}_1^*M\to\op{pr}_2^*M$ on $(X\times_FX)^\land_\Delta$ making the diagram of sheaves on $(X\times X\times X)_\Delta^\land$
	% https://q.uiver.app/#q=WzAsMyxbMCwwLCJcXG9we3ByfV8xXipNIl0sWzEsMCwiXFxvcHtwcn1fMl4qTSJdLFsxLDEsIlxcb3B7cHJ9XzNeKk0iXSxbMCwxLCJcXG9we3ByfV97MTJ9XipcXGFscGhhIl0sWzEsMiwiXFxvcHtwcn1fezIzfV4qXFxhbHBoYSJdLFswLDIsIlxcb3B7cHJ9X3sxM31eKlxcYWxwaGEiLDJdXQ==&macro_url=https%3A%2F%2Fraw.githubusercontent.com%2FdFoiler%2Fnotes%2Fmaster%2Fnir.tex
	\[\begin{tikzcd}[cramped]
		{\op{pr}_1^*M} & {\op{pr}_2^*M} \\
		& {\op{pr}_3^*M}
		\arrow["{\op{pr}_{12}^*\alpha}", from=1-1, to=1-2]
		\arrow["{\op{pr}_{13}^*\alpha}"', from=1-1, to=2-2]
		\arrow["{\op{pr}_{23}^*\alpha}", from=1-2, to=2-2]
	\end{tikzcd}\]
	commute.
\end{theorem}
\begin{proof}[Sketch]
	We describe how to turn a $\mc D_X$-module $M$ into the desired data. Namely, we need to provide the data of $\alpha$. For example, to define $\alpha$ along the fiber $\op{pr}_1^{-1}(x)\cong\op{Spf}\widehat{\OO}_{X,x}$ needs to provide an isomorphism
	\[\op{pr}_1^*M|_{\op{pr}_1^{-1}x}\to\op{pr}_2^*M|_{\op{pr}_1^{-1}x}.\]
	The left-hand side is just $M_x\otimes_F\widehat{\OO}_{X,x}$ because we start at the point $x$ (which gives $M_x$) which is then expanded along the formal neighborhood. On the other hand, the right-hand side is $M|_{\widehat{\OO}_{X,x}}$ because the pullback is able to remember some horizontal information.\footnote{This discussion is a little confusing. It becomes easier to think about in the case of $X=\AA^1$, where we can think of $\Delta$ as the genuine diagonal of a square.} It turns out that the maps of \Cref{lem:trivialize-fiber} now suitably glue over all points.
\end{proof}
\begin{remark}
	The moral of $\alpha$ is that it lets us identify infinitesimally nearby fibers. Once one has identified ``nearby'' fibers, then one can expect to be able to take derivatives, which gives the $\mc D_X$-module structure.
\end{remark}
% \begin{remark}
% 	Given a map $s\colon\Spec R\to X$
% \end{remark}
\begin{remark}
	The data $(M,\alpha)$ is sometimes called a ``crystal.''
\end{remark}
\begin{example} \label{ex:de-rham-stack-a1}
	Consider $X=\AA^1_F$ (or any other algebraic group). Then we find that $\mc D_{\AA^1}$-modules can be seen to be the same as quasicoherent sheaves on $\AA^1$ with $\widehat{\mathbb G}_a$-equivariant structure. Well, we are asking for a suitable identification along the two projections
	\[\left(\AA^1\times\AA^1\right)_\Delta^\land\rightrightarrows\AA^1.\]
	However, the left-hand formal scheme is $\AA^1\times\widehat{\mathbb G}_a$ (given by sending $(x,y)$ to $(x,x-y)$), and one can check that the cocycle condition corresponds to an equivariant structure. It follows that we are looking at quasicoherent sheaves on $\AA^1/\widehat{\mathbb G}_a$. Unwinding, we remark that a $\mc D_{\AA^1}$-module $M$ goes to the same $R[t]$-module $M$ with $\widehat{\mathbb G}_a$-(co)action $M\to M\widehat{\otimes}\OO(\widehat{\mathbb G}_a)$ given by
	\[m\mapsto\sum_{n\ge0}\del_t^nm\otimes\frac{t^n}{n!}.\]
\end{example}

\end{document}