% !TEX root = ../notes.tex

\documentclass[../notes.tex]{subfiles}

\begin{document}

\section{February 4}
Today we continue our discussion of crystalline cohomology.

\subsection{Example of Crystalline Cohomology}
We continue working over a perfect field $k$ of characteristic $p$. Let's run an example.
\begin{remark}
	If our formal scheme $\widetilde X$ is smooth and affine, then $\mathrm R\Gamma_{\mathrm{dR}}(X/W(k))$ can be computed directly as the complex
	\[0\to\OO(\widetilde X)\stackrel d\to\Omega^1(\widetilde X)\stackrel d\to\Omega^2(\widetilde X)\stackrel d\to\cdots.\]
\end{remark}
\begin{example}
	Take $X=\AA^1_k$, which we can deform to a formal scheme $\widetilde X=\widehat{\AA}^1_k$. Then
	\[\OO(\widetilde X)=\lim W_n[t],\]
	which we may refer to as $W\langle t\rangle$. Here, $W\langle t\rangle$ contains those power series whose terms are more and more divisible by $p$. Similarly, we can compute that $\Omega^1(\widetilde X)=W\langle t\rangle\,dt$. For example, we now see that $W$ has characteristic zero, so the kernel of the differential $d\colon W\langle t\rangle\to W\langle t\rangle\,dt$ is exactly the constants. On the other hand, \Cref{thm:crystalline-props} tells us that $\mathrm H^1_{\mathrm{dR}}(\widetilde X/W(k))$ is required to be quite interesting because the de Rham complex of $\AA^1_k$ is interesting.
\end{example}
\begin{remark}
	One can extend \Cref{thm:crystalline-props} slightly: even if some $k$-scheme $X$ can only be deformed to a smooth scheme $X_n$ over $W_n(k)$, then we still have
	\[\mathrm R\Gamma_{\mathrm{cris}}(X/W(k))\otimes^{\mathbb L}_{W(k)}W_n(k)\cong\mathrm R\Gamma_{\mathrm{dR}}(X_n/W_n(k)).\]
\end{remark}
The above example falls under the paradigm where the formal scheme comes from a genuine scheme. Here are some motivational remarks about this case.
\begin{example}
	Let's consider the case where the formal scheme comes from a genuine scheme. Given a smooth scheme $Y$ over $W(k)$ with formal completion $\widehat Y$, then $\mathrm R\Gamma_{\mathrm{dR}}(\widehat Y/W(k))$ is the derived $p$-adic completion of $\mathrm R\Gamma_{\mathrm{dR}}(Y/W(k))$. Explicitly, one takes a complex representing $\mathrm R\Gamma_{\mathrm{dR}}(Y/W(k))$ and takes a $p$-completion of each module. (This turns out to be a well-defined operation on the derived category.)
\end{example}
\begin{example}
	We continue the previous example. Suppose further that $Y$ is smooth and proper. Then it turns out that $\mathrm R\Gamma_{\mathrm{dR}}(Y/W(k))$ can be represented by finitely generated projective $W(k)$-modules, so the $p$-completion does nothing, so
	\[\mathrm R\Gamma_{\mathrm{dR}}(\widehat Y/W(k))\cong\mathrm R\Gamma_{\mathrm{dR}}(Y/W(k)).\]
\end{example}

\subsection{The Mysterious Functor}
Crystalline cohomology produces the following strange functoriality.
\begin{corollary} \label{cor:key-crystalline}
	Fix a smooth formal scheme $\widetilde X$ over $W(k)$. Given an endomorphism $f$ of $\widetilde X$ such that $f$ is the identity$\pmod p$, then $f^*$ is the identity on the cohomology groups $\mathrm H^n_{\mathrm{dR}}(\widetilde X/W(k))$.
\end{corollary}
\begin{proof}[Proof using \Cref{thm:crystalline-props}]
	By \Cref{thm:crystalline-props}, one may pass to crystalline cohomology of the reduction $X\coloneqq X\times_kW(k)$, where the result has no content.
\end{proof}
\Cref{thm:crystalline-props} admits the following version in characteristic zero, where we heuristically replace the deformation $W(k)\to k$ with $\CC[[t]]\to\CC$.
\begin{proposition}
	Given a smooth formal scheme $\widetilde X$ over $\CC[[t]]$ with reduction $X$ over $\CC$. Then
	\[\mathrm R\Gamma_{\mathrm{dR}}(\widetilde X/\CC[[t]])\cong\mathrm R\Gamma_{\mathrm{dR}}(X/\CC)\otimes_\CC\CC[[t]].\]
\end{proposition}
\begin{proof}[Sketch]
	The idea of the proof is to pass through the larger de Rham cohomology $\mathrm R\Gamma_{\mathrm{dR}}(\widetilde X/\CC)$; by definition, this is the inverse limit of the de Rham cohomology of the reductions $\widetilde X_n$. By the Poincar\'e lemma, it turns out that $\mathrm R\Gamma_{\mathrm{dR}}(\widetilde X/\CC)$ is quasi-isomorphic to $\mathrm R\Gamma_{\mathrm{dR}}(X/\CC)$. (Approximately speaking, this is saying that a tubular neighborhood of $X$ is homotopic to $X$.) The result now follows by extending scalars.
\end{proof}
\begin{remark}
	This proof does not work in our mixed characteristic situation $W(k)\to k$ because there is no direct way to geometrically link $\widetilde X$ and $X$ via a Poincar\'e lemma.
\end{remark}
To prove \Cref{thm:crystalline-props}, it will turn out that the key ideas can be used to instead prove \Cref{cor:key-crystalline}.
\begin{proof}[Proof of \Cref{cor:key-crystalline} when $p>2$]
	We attempt to give a version of Poincar\'e's lemma. The idea there is to use Cartan's formula, which arises by viewing differentials as living in flows of vector fields.
	\begin{enumerate}
		\item Our analog of this will be
		\[\log f\coloneqq\sum_{n\ge1}(-1)^{n+1}\frac{(f-{\id})^{\circ n}}n,\]
		viewed as a $W$-linear endomorphism of $\OO_{\widetilde X}$. To see that this makes sense, note that ${f-\id}=pA$ for some endomorphism $A$, so it is at least true that the terms of the series are divisible by larger and larger powers of $p$. Thus, the series descends to a compatible sequence of endomorphisms for each $\OO_{\widetilde X_n}$ and hence an endomorphism of $\OO_{\widetilde X}$.
		\item It further turns out that $\log f$ is a derivation of $\OO_{\widetilde X}$, meaning that
		\[(\log f)(ab)=(\log f)(a)b+a(\log f)(b).\]
		This is a formal consequence of the fact that $f$ is a ring map, and $\log f$ takes products to sums. (Indeed, one can already find this in characteristic zero.)
		\item To continue, we take $p>2$ for simplicity. Indeed, then one can see $f=\exp(\log f)$, meaning that
		\[f\stackrel?=\sum_{n\ge0}\frac{(\log f)^{\circ n}}{n!}.\]
		The right-hand sum makes sense because $\log f$ is divisible by $p$, which then implies that the sum converges. Thus, we may write $\log f=pD$ for some derivation $D$. The equality now follows from some formal calculation of the series.
		\item We now pass to the de Rham complex. We would like to check that the endomorphism $f^*\colon\Omega^\bullet_{\widetilde X}\to\Omega^\bullet_{\widetilde X}$ is a quasi-isomorphism. Well, there is a ``Lie derivative'' $L_D$ for which $f^*$ acts on $\Omega^i_{\widetilde X}$ by
		\[\sum_{n\ge0}\frac{p^n}{n!}L^{\circ n}_D.\]
		Let's explain what this $L_D$ is, which we may define affine-locally. For any $W(k)$-algebra $A$, a derivation $W\colon A\to A$ induces an endomorphism $L_D$ on differentials as follows. Indeed, ${\id}+\varepsilon D$ is an automorphism of the thickening $A[\varepsilon]/\left(\varepsilon^2\right)$, so ${\id}+\varepsilon D$ acts on $\Omega^i_{A[\varepsilon]/\left(\varepsilon^2\right)}$. Certainly it must reduce to identity modulo $\varepsilon$, so we conclude that $({\id}+\varepsilon D)^*={\id}+\varepsilon L_D$ for some endomorphism $L_D$ on differentials!
		\item The previous step now tells us that $f^*={\id}+L_D\circ G$ for some $G$. We will be done as soon as we can check that the endomorphism $L_D$ of the de Rham complex is homotopic to zero. In other words, we are on the hunt for maps $\iota_D\colon\Omega^i_{\widetilde X}\to\Omega^{i-1}_{\widetilde X}$ fitting into the diagram
		% https://q.uiver.app/#q=WzAsMTAsWzEsMCwiXFxPT197XFx3aWRldGlsZGUgWH0iXSxbMiwwLCJcXE9tZWdhXjFfe1xcd2lkZXRpbGRlIFh9Il0sWzMsMCwiXFxPbWVnYV4yX3tcXHdpZGV0aWxkZSBYfSJdLFsxLDEsIlxcT09fe1xcd2lkZXRpbGRlIFh9Il0sWzIsMSwiXFxPbWVnYV4xX3tcXHdpZGV0aWxkZSBYfSJdLFszLDEsIlxcT21lZ2FeMl97XFx3aWRldGlsZGUgWH0iXSxbNCwwLCJcXGNkb3RzIl0sWzQsMSwiXFxjZG90cyJdLFswLDAsIjAiXSxbMCwxLCIwIl0sWzgsMF0sWzAsMV0sWzEsMl0sWzIsNl0sWzksM10sWzMsNF0sWzQsNV0sWzUsN10sWzAsMywiTF9EIiwxXSxbMSw0LCJMX0QiLDFdLFsyLDUsIkxfRCIsMV0sWzIsNCwiXFxpb3RhX0QiLDFdLFsxLDMsIlxcaW90YV9EIiwxXV0=&macro_url=https%3A%2F%2Fraw.githubusercontent.com%2FdFoiler%2Fnotes%2Fmaster%2Fnir.tex
		\[\begin{tikzcd}[cramped]
			0 & {\OO_{\widetilde X}} & {\Omega^1_{\widetilde X}} & {\Omega^2_{\widetilde X}} & \cdots \\
			0 & {\OO_{\widetilde X}} & {\Omega^1_{\widetilde X}} & {\Omega^2_{\widetilde X}} & \cdots
			\arrow[from=1-1, to=1-2]
			\arrow[from=1-2, to=1-3]
			\arrow["{L_D}"{description}, from=1-2, to=2-2]
			\arrow[from=1-3, to=1-4]
			\arrow["{\iota_D}"{description}, from=1-3, to=2-2]
			\arrow["{L_D}"{description}, from=1-3, to=2-3]
			\arrow[from=1-4, to=1-5]
			\arrow["{\iota_D}"{description}, from=1-4, to=2-3]
			\arrow["{L_D}"{description}, from=1-4, to=2-4]
			\arrow[from=2-1, to=2-2]
			\arrow[from=2-2, to=2-3]
			\arrow[from=2-3, to=2-4]
			\arrow[from=2-4, to=2-5]
		\end{tikzcd}\]
		so that $L_D=d\iota_D+\iota_DD$. This is precisely Cartan's formula, which admits an algebraic proof. (Namely, one takes $\iota_D$ to be a contraction whose vector field is $D$.) To show this, one can show directly that $d\iota_D+\iota_DD$ is a derivation, and then one can check that there is an equality on functions and $1$-forms.
		\qedhere
	\end{enumerate}
\end{proof}
\begin{remark}
	The construction of $L_D$ may appear ad-hoc. The intuition here is that our objects are not only functorial under endomorphisms but also under derivations.
\end{remark}
It turns out that all the interesting algebra to prove \Cref{thm:crystalline-props} is already contained in the proof of \Cref{cor:key-crystalline}, though this should not be obvious yet.

\subsection{The Cartier Isomorphism}
Recall that we wanted to prove \Cref{thm:cartier-iso}.
\cartieriso*
\begin{proof}
	Let's start with $i=0$. Note that there is a map
	\[\OO_{X^{(1)}}\to\ker\left(d\colon F_*\OO_X\to F_*\Omega^1_X\right)\]
	given by sending a function $f$ to $f^p$. We would like to check that this is an isomorphism. Well, it is enough to check this on an open cover, but because $X$ is smooth, we may pass \'etale-locally along some $f\colon U\to\AA^n_k$ (where $U\subseteq X$ is some open subset) because the de Rham complex is immune to such deformations: indeed,
	\[\left(F_*\Omega^0_X\right)|_{U^{(1)}}\cong f^{(1)*}\left(F_*\Omega^0_{\AA^n_k}\right).\]
	Thus, we may pass to $X=\AA^n$.

	We now work in general. By working locally, we may assume that $X$ is affine and equal to some $\Spec A$. Then we would like an isomorphism
	\[\mathrm H^i_{\mathrm{dR}}(X/k)\stackrel?\cong\Omega^i_{A^{(1)}/k}\]
	of $A$-modules. Well, consider $\mathrm R\Gamma_{\mathrm{cris}}(A/W_2(k))\coloneqq\mathrm R\Gamma_{\mathrm{cris}}(A/W(k))\otimes_{W(k)}W_2(k)$, which we know to be isomorphic to $\mathrm R\Gamma_{\mathrm{dR}}(\widetilde X/W_2(k))$ for any deformation $\widetilde X$ of $X$ to $W_2(k)$. Accordingly, we receive a distinguished triangle
	\[\mathrm R\Gamma_{\mathrm{dR}}(A/k)\to\mathrm R\Gamma_{\mathrm{cris}}(A/W_2(k))\to\mathrm R\Gamma_{\mathrm{dR}}(A/k)\]
	induced by the lifting $k\to W_2(k)\to k$, where the map $k\to W_2(k)$ is multiplication by $p$. Thus, we receive a long exact sequence
	\[\cdots\to\mathrm H^i_{\mathrm{dR}}(A/k)\to\mathrm H^i_{\mathrm{dR}}(A/k)\stackrel{\beta_i}\to\mathrm H^{i+1}_{\mathrm{dR}}(A/k)\to\cdots.\]
	It turns out that the graded map $\bigoplus_i\beta_i$ on the de Rham cohomology ring $\bigoplus_{i\ge0}\mathrm H^i_{\mathrm{dR}}(X/k)$ is a derivation; this is just some exercise in the homological algebra. Explicitly, one has to check that
	\[\beta_{i+j}(x_i\land x_j)=\beta_i(x_i)\land x_j+(-1)^ix_i\land\beta_j(x_j)\]
	for any $x_i$ and $x_j$ of degrees $i$ and $j$, respectively. It further turns out that $\beta_{i+1}\circ\beta_i=0$.

	Now that we have found a derivation, we can make differentials appear. The universal property of differentials produce morphisms
	% https://q.uiver.app/#q=WzAsMTAsWzEsMCwiXFxPT197QV57KDEpfX0iXSxbMiwwLCJcXE9tZWdhXjFfe0FeeygxKX19Il0sWzEsMSwiXFxtYXRocm0gSF4wX3tcXG1hdGhybXtkUn19KEEvaykiXSxbMiwxLCJcXG1hdGhybSBIXjFfe1xcbWF0aHJte2RSfX0oQS9rKSJdLFswLDAsIjAiXSxbMCwxLCIwIl0sWzMsMCwiXFxPbWVnYV4yX3tBXnsoMSl9fSJdLFszLDEsIlxcbWF0aHJtIEheMl97XFxtYXRocm17ZFJ9fShBL2spIl0sWzQsMCwiXFxjZG90cyJdLFs0LDEsIlxcY2RvdHMiXSxbNCwwXSxbMCwxLCJkIl0sWzUsMl0sWzIsMywiXFxiZXRhIl0sWzEsNiwiZCJdLFs2LDgsImQiXSxbMyw3LCJcXGJldGEiXSxbNyw5LCJcXGJldGEiXSxbMCwyLCJmXzAiXSxbMSwzLCJmXzEiXSxbNiw3LCJmXzIiXV0=&macro_url=https%3A%2F%2Fraw.githubusercontent.com%2FdFoiler%2Fnotes%2Fmaster%2Fnir.tex
	\[\begin{tikzcd}[cramped]
		0 & {\OO_{A^{(1)}}} & {\Omega^1_{A^{(1)}}} & {\Omega^2_{A^{(1)}}} & \cdots \\
		0 & {\mathrm H^0_{\mathrm{dR}}(A/k)} & {\mathrm H^1_{\mathrm{dR}}(A/k)} & {\mathrm H^2_{\mathrm{dR}}(A/k)} & \cdots
		\arrow[from=1-1, to=1-2]
		\arrow["d", from=1-2, to=1-3]
		\arrow["{c_0}", from=1-2, to=2-2]
		\arrow["d", from=1-3, to=1-4]
		\arrow["{c_1}", from=1-3, to=2-3]
		\arrow["d", from=1-4, to=1-5]
		\arrow["{c_2}", from=1-4, to=2-4]
		\arrow[from=2-1, to=2-2]
		\arrow["\beta", from=2-2, to=2-3]
		\arrow["\beta", from=2-3, to=2-4]
		\arrow["\beta", from=2-4, to=2-5]
	\end{tikzcd}\]
	where $f_0$ is induced by the degree-zero argument, and all the relevant morphisms intertwine $\beta$ and $d$ and produce maps of graded rings. For example, one finds that we need $c_1(df)=\beta_0(c_0(f))$ for functions $f$. One now checks that this in fact an isomorphism on the level of modules, which is checked \'etale-locally and then passed to affine spaces.
\end{proof}
\begin{remark} \label{rem:construct-cartier}
	We can afford to be a little more explicit about our construction of the maps $c_\bullet$. Fix a lift $\widetilde A$ over $W_2(k)$ as well as a lift of the Frobenius $\widetilde F$. Given $\omega\in\Omega^i_{A^{(1)}/k}$, we pull it back along $\widetilde F$ to get a differential divisible by $p$ in $\mathrm H^i_{\mathrm{dR}}(\widetilde A/W_2(k))$. Then one can check that the class $\widetilde F^*(\omega)/p$ is well-defined in $\mathrm H^i_{\mathrm{dR}}(A/k)$.
\end{remark}
\begin{remark}
	Crystalline cohomology is not technically necessary because we could choose a lift $\widetilde X$ by hand using \Cref{rem:construct-cartier}. However, one needs to check that the constructed map is independent of the lift. This is not impossible (such lifts are well-understood in some cohomology group by deformation theory), but it is a little difficult.
\end{remark}

\subsection{\texorpdfstring{$\mc D$}{ D}-modules}
We close our class by saying something about $\mc D$-modules.
\begin{defihelper}[$D$-module] \nirindex{D-module@$\mc D$-module}
	Fix a smooth scheme $X$ over a ring $R$. Then we define the quasicoherent sheaf $\mc D_{X/R}$ of associative algebras on $X$ defined explicitly as
	\[\OO_X\{\del_v:v\in T_X\},\]
	where the relations $\del_v$ satisfies the relations $\del_{v_1+v_2}=\del_{v_1}+\del_{v_2}$, $\del_{fv}=f\del_v$, $[\del_v,f]=L_v(f)$, and $[\del_v,\del_w]=\del_{[v,w]}$. Here, $T_X$ is the tangent bundle. A \textit{$\mc D$-module} is a quasicoherent sheaf which is a module for $\mc D_{X/R}$.
\end{defihelper}
\begin{remark}
	Precisely, we have given a definition of an associative ring on affine opens, which then glue together on affines as a sheaf on a base.
\end{remark}

\end{document}