% !TEX root = ../notes.tex

\documentclass[../notes.tex]{subfiles}

\begin{document}

\section{February 17}
Today, we discuss the Hodge decomposition.

\subsection{Divided Powers}
We are interested in sketching the following result, fulfilling our promise in \Cref{rem:better-hodge}.
\begin{theorem}[Drinfeld, Bhatt--Lurie]
	Fix a smooth variety $X$ over a perfect field $k$ of positive characteristic $p$. Given a smooth lift $\widetilde X$ over $W_2(k)$, there is a natural $\ZZ/p\ZZ$-grading
	\[\mathrm H^n_{\mathrm{dR}}(X/k)=\bigoplus_{i\in\ZZ/p\ZZ}\mathrm H^n_{\mathrm{dR}}(X/k)_i\]
	such that $\mathrm H^n_{\mathrm{dR}}(X/k)_i\cong\mathrm H^{n-i}\left(X^{(1)};\Omega^i_{X^{(1)}}\right)$ for $i\in\{0,1,\ldots,p-1\}$.
\end{theorem}
\begin{remark}
	This is not exactly a Hodge decomposition because this is only a $\ZZ/p\ZZ$-grading. Additionally, the Hodge decomposition usually only exists when $X$ is further proper.
\end{remark}
\begin{remark}
	It is possible to find a smooth proper surface without the lift to $W_2(k)$ and also so that
	\[\dim\mathrm H^n_{\mathrm{dR}}(X/k)\ne\sum_{0\le i<p}\mathrm H^{n-i}\left(X^{(1)};\Omega_{X^{(1)}}\right).\]
	Such varieties even appear in nature.
\end{remark}
It is difficult to find a grading directly on de Rham cohomology. Instead, we will pass through the de Rham stack. We may want to retell some of our story from last class with crystals, but there are problems with this construction in positive characteristic. For example, the power series defining the coaction in \Cref{ex:de-rham-stack-a1} have large denominators.

To fix this problem, we simply add in the required symbols by hand.
\begin{definition}
	We define the affine group scheme $\mathbb G_a^\sharp$ over $\FF_p$ to be given by the ring
	\[\ZZ\left[t,t/1,t^2/2,t^3/3!,\ldots\right]\otimes_\ZZ\FF_p.\]
	The group structure $\mathbb G_a^\sharp\times\mathbb G_a^\sharp\to\mathbb G_a^\sharp$ is given by sending $t_1\otimes t_2$ to $\frac1{n!}(t_1+t_2)^n=\sum_{i+j=n}\frac{t_1^i}{i!}\cdot\frac{t_2^j}{j!}$.
\end{definition}
\begin{remark}
	Explicitly, the ring is given by
	\[\frac{\FF_p[t,u_1,u_2,u_3\ldots]}{(u_1-t,2tu_2-u_1,3tu_3-u_2,\ldots)}.\]
	It turns out to be enough to merely add in the divided powers coming from powers of $p$.
\end{remark}
\begin{remark}
	One can also take the formal group scheme $\widehat{\mathbb G}_a^\sharp$ over $\FF_p$ to be given by the union of the affine group schemes given by the quotient rings
	\[\frac{\ZZ\left[t,t/1,t^2/2,t^3/3!,\ldots\right]}{\left(t^m/m!\right)_{m\ge n}}\otimes_\ZZ\FF_p.\]
	Alternatively, this is the formal scheme attached to the completed ring $\FF_p[t][[t^n/n!]]_{n\ge0}$.
\end{remark}
\begin{remark}
	The previous remark provides a natural map $\mathbb G_a^\sharp\to\mathbb G_a$. This is some map of group schemes, and we see that its kernel is given by
	\[\frac{\FF_p\left[t,t^p/p!,t^{p^2}/(p^2)!,\ldots\right]}{(t)},\]
	which continues to be some infinite-dimensional group scheme. Similarly, the kernel of the natural map $\widehat{\mathbb G}^\sharp_a\to\mathbb G_a$ is the formal scheme attached to the ring $\OO(\widehat{\mathbb G}_a^\sharp)/(t)$.
\end{remark}
Here is our analog of \Cref{ex:de-rham-stack-a1}.
\begin{lemma} \label{lem:d-mod-to-qcoh}
	Fix a ring $R$. Then the category of $\mc D_{\AA^1_R/R}$-modules is equivalent to the category of quasicoherent sheaves on $\AA^1_R$ with an equivariant action by $\widehat{\mathbb G}_a^\sharp$.
\end{lemma}
\begin{proof}
	The functor in the forward direction was already given by \Cref{ex:de-rham-stack-a1}. We will not check that it is an equivalence.
\end{proof}
\begin{example} \label{ex:motivate-p-curvature}
	Because the natural map $\widehat{\mathbb G}^\sharp_a\to\mathbb G_a$ admits a kernel, we see that its kernel will act by endomorphisms on a $\mc D$-module $M$. (Indeed, this kernel does not interact with the rest of the $\AA^1$-action.)
\end{example}

\subsection{The \texorpdfstring{$p$}{ p}-curvature}
It is remarkable that we have found so many additional endomorphisms. Let's explain this. We have already seen something like this before: recall that the natural map $\mc D_X\to\mathcal End_k(\OO_X)$ admitted a kernel in positive characteristic.
\begin{lemma}
	Fix a smooth scheme $X$ over a field $k$ of positive characteristic $p$. For any derivation $v\colon\OO_X\to\OO_X$, the $p$-fold composition $v^{\circ p}$ is also a derivation.
\end{lemma}
\begin{proof}
	Direct calculation.
\end{proof}
\begin{notation}
	Fix a scheme $X$ over a field $k$ of positive characteristic $p$. Given a vector field $v$ on some open subset $U\subseteq X$, we let $v^{[p]}$ denote the vector field corresponding to the $p$-fold derivation $v^{\circ p}$.
\end{notation}
\begin{example}
	With $X=\AA^1$, we see that $\left(\del_t\right)^{[p]}=0$ because hitting any $t^n$ with $p$ derivatives will kill it. Similarly, we see that $\left(t\del_t\right)^{[p]}=t\del_t$ because hitting $t^n$ with $t\del_t$ a total of $p$ times goes to $n^pt^n=nt^n$.
\end{example}
The point is that the element $\del_v^{\circ p}-\del_{v^{[p]}}$ acts by zero as a derivation on $\OO_X$ because $\del_v^{\circ p}$ and $\del_{v^{[p]}}$ admit the same action on $\OO_X$ (by construction of $v^{[p]}$). In particular, it lives in the kernel of the natural map $\mc D_X\to\mathcal End_k(\OO_X)$! It turns out that these elements generate.
\begin{proposition} \label{prop:generate-ker-d-mod}
	Fix a smooth variety $X$ over a perfect field $k$ of positive characteristic $p$.
	\begin{listalph}
		\item Then $F_*\mc D_X$ is a sheaf of $\OO_{X^{(1)}}$-algebras, and its center is the symmetric algebra generated by the elements
		\[\del_v^p-\del_{v^{[p]}}\]
		as $v$ varies over $T_{X^{(1)}}$.
		\item The kernel of $\mc D_X\to\mathcal End_k(\OO_X)$ is generated by the homogeneous elements generated by $\del_v^p-\del_{v^{[p]}}$ in positive degree.
	\end{listalph}
\end{proposition}
\begin{proof}
	Omitted.
\end{proof}
Thus, we have found many endomorphisms.
\begin{defihelper}[$p$-curvature] \nirindex{p-curvature@$p$-curvature}
	Fix a smooth variety $X$ over a perfect field $k$ of positive characteristic $p$. For a vector field $v$ and a $\mc D$-module $M$, the \textit{$p$-curvature} of $M$ is the action of
	\[\psi_v\coloneqq\del_v^{\circ p}-\del_v.\]
\end{defihelper}
\begin{remark}
	The action of $\psi_v$ is an endomorphism as a $\mc D_X$-module, which more or less follows from part (a) of \Cref{prop:generate-ker-d-mod}. (Note that the $F_*$ does not actually do much to the commutativity condition we are trying to check.) This action turns out to be the same as the one found in \Cref{ex:motivate-p-curvature}.
\end{remark}

\subsection{Quasi-nilpotent Modules}
We are now allowed to make the following definition, which will let us remove some formal schemes from view.
\begin{definition}[quasi-nilpotent]
	Fix a smooth variety $X$ over a perfect field $k$ of positive characteristic $p$. Then a $\mc D_X$-module $M$ is \textit{quasi-nilpotent} if and only if, for all vector fields $v$ on $U\subseteq X$, the action of $\psi_v$ on $M(U)$ is locally nilpotent: for any $m\in M(U)$, there is $N$ for which $\psi_v^{\circ N}(m)=0$. More generally, if $X$ is a smooth scheme over a $\ZZ/p^n\ZZ$-algebra $R$, then a $\mc D$-module $M$ is \textit{quasinilpotent} if and only if $M_{R/p}$ is nilpotent as a $\mc D$-module over $X_{R/p}$.
\end{definition}
\begin{example}
	Because the action by the $\psi_v$s is zero on $\OO_X$, we see that $\OO_X$ is quasi-nilpotent.
\end{example}
\begin{nex}
	The action of $\mc D_X$ on $\mc D_X$ is faithful, so $\mc D_X$ is not quasi-nilpotent.
\end{nex}
Let's now upgrade \Cref{lem:d-mod-to-qcoh}.
\begin{lemma}
	The category of quasi-nilpotent $\mc D$-modules on $\AA^1_{\FF_p}$ is equivalent to the category of quasicoherent sheaves on $\AA^1_{\FF_p}$ with an equivariant action by $\mathbb G_a^{\sharp}$.
\end{lemma}
\begin{proof}
	Via \Cref{lem:d-mod-to-qcoh}, we only have to characterize the image of the constructed functor. Indeed, the functor defined by \Cref{ex:de-rham-stack-a1} finds that the sum
	\[\sum_{n\ge0}\del^n_tm\otimes\frac{t^n}{n!}\]
	is always finite because $M$ is quasi-nilpotent.
\end{proof}
\begin{remark}
	For later use, it is worth noting that the de Rham complex continues to be computed by quasi-nilpotent sheaves: as in \Cref{lem:d-mod-to-de-rham}, we find that
	\[\Omega^\bullet_{X/R}\cong\mathrm{RHom}_{\mathrm{Mod}(\mc D_X)^{\mathrm{quasi-nilpotent}}}(\OO_X,\OO_X).\] 
\end{remark}
We additionally have a notion of crystals!
\begin{definition}
	Fix a closed, regular embedding $Y\into X$ of smooth varieties over a field $k$. Then we define the \textit{de Rham stack} $X^\sharp_Y$ as the ``divided power envelope''
	\[X^\sharp_Y=\op{Spec}_X\OO_X\left[\frac{f^n}{n!}\right]_{n\ge1,f\in\mc I_Y},\]
	where $\mc I_Y$ is the ideal sheaf.
\end{definition}
\begin{remark}
	Explicitly, if $Y$ is cut out by a regular sequence $\{f_1,\ldots,f_r\}$, then the ring can be described as the tensor product of $\OO_X$ with the free divided power algebra on the letters $x_1,\ldots,x_r$, where $x_i$ corresponds to $f_i$. It is not obvious that this ring does not depend on the choice of regular sequence!
\end{remark}
As promised, here is our analog of \Cref{thm:d-mod-to-crystal}.
\begin{theorem}
	Fix a smooth variety $X$ over a perfect field $k$ of positive characteristic $p$. Then the category of quasi-nilpotent $\mc D_X$-modules is equivalent to the category of quasicoherent sheaves $M$ on $X$ equipped with an isomorphism $\op{pr}_1^*M\cong\op{pr}_2^*M$ (and satisfying a suitable cocycle condition), where $\op{pr}_1$ and $\op{pr}_2$ are the canonical projections $(X\times X)^\sharp_\Delta\to X$.
\end{theorem}
\begin{proof}
	Omitted.
\end{proof}

\subsection{The de Rham Stack}
We are now ready to define the de Rham stack.
\begin{definition}[quotient stack]
	A \textit{quotient stack}, denoted $[X/G]$, is a pair $(G,X)$, where $G$ is a group scheme acting on a scheme $X$. Two quotient stacks $[X/G]$ and $[X'/G']$ are \textit{strongly equivalent} if and only if there is a flat surjection $G'\to G$ of group schemes and a map $X'\to X$ so that $X'\to X$ is a torsor for $\ker f$. Two quotient stacks $[X/G]$ and $[X'/G']$ are \textit{equivalent} if and only if they can be connected by a finite sequence of strong equivalences.
\end{definition}
\begin{notation}
	Given a quotient stack $[X/G]$, we define $\op{QCoh}([X/G])$ to consist of the $G$-equivariant sheaves on $X$.
\end{notation}
\begin{remark}
	One can check that $\op{QCoh}([X/G])$ does not depend on the equivalence class of $[X/G]$.
\end{remark}
Here is a first definition of our de Rham stack, for affine schemes.
\begin{definition}[de Rham stack]
	Fix an affine scheme $X\subseteq\AA^n_k$ over a perfect field $k$ of positive characteristic $p$. Then we define the \textit{de Rham stack} $X^{\mathrm{dR}}$ to be the quotient stack
	\[\left[\left(\AA^n\right)^\sharp_X/\left(\mathbb G_a^\sharp\right)^n\right].\]
	It turns out that this construction does not depend on the choice of embedding to $\AA^n_k$.
\end{definition}
\begin{remark}
	It turns out that the quasicoherent sheaves on $X^{\mathrm{dR}}$ correspond exactly to crystals and thus to quasi-nilpotent $\mc D_X$-modules. We have already seen this for $\AA^1$ in \Cref{ex:de-rham-stack-a1}.
\end{remark}
To make a definition in general, note that one can locally fit $X$ into a pullback square as follows.
% https://q.uiver.app/#q=WzAsNCxbMCwwLCJYIl0sWzEsMCwiXFxBQV5uIl0sWzAsMSwiXFxTcGVjIGsiXSxbMSwxLCJcXEFBXm0iXSxbMSwzXSxbMCwxLCIiLDAseyJzdHlsZSI6eyJ0YWlsIjp7Im5hbWUiOiJob29rIiwic2lkZSI6InRvcCJ9fX1dLFsyLDMsIiIsMix7InN0eWxlIjp7InRhaWwiOnsibmFtZSI6Imhvb2siLCJzaWRlIjoidG9wIn19fV0sWzAsMl1d&macro_url=https%3A%2F%2Fraw.githubusercontent.com%2FdFoiler%2Fnotes%2Fmaster%2Fnir.tex
\[\begin{tikzcd}[cramped]
	X & {\AA^n} \\
	{\Spec k} & {\AA^m}
	\arrow[hook, from=1-1, to=1-2]
	\arrow[from=1-1, to=2-1]
	\arrow[from=1-2, to=2-2]
	\arrow[hook, from=2-1, to=2-2]
\end{tikzcd}\]
(One may not be able to do this globally if $X$ is not a complete intersection.) But now we may as well base-change by $(\AA^m)^\sharp_0$, which produces the following pullback square.
% https://q.uiver.app/#q=WzAsNCxbMCwwLCJYXntcXG1hdGhybXtkUn19Il0sWzEsMCwiKFxcQUFebilee1xcbWF0aHJte2RSfX0iXSxbMCwxLCJcXFNwZWMgayJdLFsxLDEsIihcXEFBXm0pXntcXG1hdGhybXtkUn19Il0sWzEsM10sWzAsMV0sWzIsM10sWzAsMl1d&macro_url=https%3A%2F%2Fraw.githubusercontent.com%2FdFoiler%2Fnotes%2Fmaster%2Fnir.tex
\[\begin{tikzcd}[cramped]
	{X^{\mathrm{dR}}} & {(\AA^n)^{\mathrm{dR}}} \\
	{\Spec k} & {(\AA^m)^{\mathrm{dR}}}
	\arrow[from=1-1, to=1-2]
	\arrow[from=1-1, to=2-1]
	\arrow[from=1-2, to=2-2]
	\arrow[from=2-1, to=2-2]
\end{tikzcd}\]
Here, note $(\Spec k)^{\mathrm{dR}}=\Spec k$ because $\mc D$-modules on a point are just vector spaces. But now we see that $(\AA^n)^{\mathrm{dR}}=\left(\AA^1\right)^{\mathrm{dR},\times}$, so it becomes interesting to understand what the polynomial map $\AA^n\to\AA^m$ becomes after these isomorphisms.
\begin{remark}
	Note that $(\AA^1)^{\mathrm{dR}}$ is a ring stack: one can just base-change the addition and multiplication up from $\AA^1$ to the de Rham stack. In fact, this is a $k$-algebra stack, meaning that each $a\in k$ produces a multiplication $\mu_a\colon(\AA^1)^{\mathrm{dR}}\to(\AA^1)^{\mathrm{dR}}$.
\end{remark}
The moral is that the functor $X\mapsto X^{\mathrm{dR}}$ for every $X$ is understood just from gluing and taking closed subschemes from the $k$-algebra stack $\left(\AA^1\right)^{\mathrm{dR}}$. Indeed, one can make sense of the action of ``polynomials,'' which is enough to cut out $X$.

\end{document}