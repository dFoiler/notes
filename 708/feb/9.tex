% !TEX root = ../notes.tex

\documentclass[../notes.tex]{subfiles}

\begin{document}

\section{February 9}
Today we say more about the construction of crystalline cohomology.

\subsection{\texorpdfstring{$\mc D$}{ D}-modules}
We are going to use $\mc D$-modules to glue our de Rham complexes together.
\begin{defihelper}[$D$-module] \nirindex{D-module@$\mc D$-module}
	Fix a smooth scheme $X$ over a ring $R$. Then we define the quasicoherent sheaf $\mc D_{X/R}$ of associative algebras on $X$ defined explicitly as
	\[\OO_X\{\del_v:v\in T_X\},\]
	where the relations $\del_v$ satisfies the relations $\del_{v_1+v_2}=\del_{v_1}+\del_{v_2}$, $\del_{fv}=f\del_v$, $[\del_v,f]=L_v(f)$, and $[\del_v,\del_w]=\del_{[v,w]}$. Here, $T_X$ is the tangent bundle, which is the dual of $\Omega^1_{X/R}$ (equivalently, the sheaf of derivations $\OO_X\to\OO_X$). A \textit{$\mc D$-module} is a quasicoherent sheaf for which the $\OO_X$-action extends to the action of a module for $\mc D_{X/R}$.
\end{defihelper}
\begin{remark}
	Precisely, we have given a definition of an associative ring on affine opens, which then glue together on affines as a sheaf on a base.
\end{remark}
\begin{remark}
	There is a natural action of $\mc D_{X/R}$ on $\OO_X$, where functions act by multiplication, and the derivations act by taking derivatives. If $R$ contains $\QQ$, then this natural map $\mc D_{X/R}\to\op{End}_R(\OO_X)$ is injective.
\end{remark}
\begin{example}
	In positive characteristic $p$, the action map $\mc D_{X/R}\to\op{End}_R(\OO_X)$ need not be injective. Indeed, with $X=\AA^1_{\FF_p}$ over $\FF_p$, we see that the iterated derivation $\del_t^p$ kills all polynomials (because $p!=0$).
\end{example}
\begin{remark}
	The various relations make it so that a $\mc D$-module is exactly a vector bundle with a flat connection $\nabla\colon M\to M\otimes\Omega^1_{X/R}$.
\end{remark}
\begin{remark}
	The algebra $\mc D_{X/R}$ admits a filtration by the subgroups $\mc D_{X/R}^{\le n}$ of differentials at most $n$. It turns out that $\op{gr}\mc D_{X/R}=\op{Sym}^\bullet T_X$, which is probably a variant of the PBW theorem.
\end{remark}
Here is why $\mc D$-modules are relevant to us.
\begin{lemma} \label{lem:d-mod-to-de-rham}
	Fix a smooth scheme $X$ over a ring $R$. Then the de Rham complex $\Omega^\bullet_{X/R}$ is quasi-isomor\-phic to $\mathrm {RHom}_{\mc D_{X/R}}(\OO_X,\OO_X)$ in $D(\op{Sh}(X))$.
\end{lemma}
\begin{proof}
	We are going to find a locally projective resolution for $\OO_X$ as a $\mc D_{X/R}$-module. To start, note that there is a projection
	\[\mc D_{X/R}\to\OO_X\]
	given by sending $\delta\mapsto\delta(1)$. This is surjective because functions act by left multiplication. By definition, all the derivations $\del_v$ are in the kernel, and these are exactly what goes to zero, so we continue our sequence by writing
	\[\mc D_{X/R}\otimes_{\OO_X}T_{X/R}\to\mc D_{X/R}\to\OO_X\to0,\]
	where the left map sends $\delta\otimes v\mapsto\delta\del_v$. Then $\del_v\otimes w-\del_w\otimes v$ is in the kernel, and we can find that we should extend our complex by
	\[\mc D_{X/R}\otimes\OO_X\land^2T_{X/R}\to\mc D_{X/R}\otimes_{\OO_X}T_{X/R}\to\mc D_{X/R}\to\OO_X\to0,\]
	where the left map sends $\delta\otimes(v_1\land v_2)$ to $\delta\del_{v_1}\otimes v_2-\delta\del_{v_2}\otimes\del_{v_1}-\delta\otimes[v_1,v_2]$. This suggests that we are looking at a Koszul complex
	\[\cdots\to\mc D_{X/R}\otimes\OO_X\land^2T_{X/R}\to\mc D_{X/R}\otimes_{\OO_X}T_{X/R}\to\mc D_{X/R}\to\OO_X\to0,\]
	and one can check that it is acyclic by passing to the associated graded everywhere.

	We now compute $\op{RHom}$ by hitting our complex with $\op{Hom}_{\mc D_X}(-,\OO_X)$. Then the point is that there is an isomorphism $\op{Hom}_{\mc D_X}(\mc D_X\otimes E,\OO_X)=E^\lor$ in the obvious way, so taking $\op{Hom}_{\mc D_X}(-,\OO_X)$ produces the complex
	\[\OO_X\stackrel d\to\Omega^1_{X/R}\stackrel d\to\Omega^2_{X/R}\stackrel d\to\cdots,\]
	which is exactly the de Rham complex.
\end{proof}
\begin{remark}
	One can replace the target $\OO_X$ with any quasicoherent sheaf to obtain a similar result, which allows us to compute de Rham cohomology of an arbitrary quasicoherent sheaf.
\end{remark}

\subsection{The Category of Crystals}
We are now ready to construct crystalline cohomology.
\begin{theorem} \label{thm:construct-crystal}
	Suppose $p>2$. Fix any smooth variety $X$ over a perfect field $k$ of characteristic $p$. For each $n\ge1$, then there is an abelian tensor category $\op{Cris}(X/W_n(k))$ of crystals with unit object $\OO_X^{\mathrm{cris}}$, with the following property: for any lift $X_n$ of $X$ to $W_n(k)$, there is an isomorphism
	\[\op{Cris}(X/W_n(k))\cong\mathrm{Mod}(\mc D_{X_n/W_n})\]
	sending $\OO_X^{\mathrm{cris}}$ to $\OO_X$.
\end{theorem}
\begin{remark}
	This certainly implies \Cref{thm:crystalline-props} simply by defining
	\[\mathrm R\Gamma_{\mathrm{cris}}(X/W_n(k))\coloneqq\mathrm{RHom}_{\op{Cris}}(X/W_n(k))(\OO_X^{\mathrm{cris}},\OO_X^{\mathrm{cris}}).\]
	Namely, we may view the target as living in $\mc D_{X_n/W_n}$-modules for any lift $X_n$, which is then related to de Rham cohomology via \Cref{lem:d-mod-to-de-rham}. One can construct $\mathrm R\Gamma_{\mathrm{cris}}(X/W(k))$ by taking a suitable limit over $n$.
\end{remark}
\begin{remark}
	The category quasicoherent sheaves on $X_n$ (as a tensor category) recovers $X_n$ as a tensor category, so it is remarkable that the category of $\mc D$-modules does not!
\end{remark}
\begin{remark}
	The sheaf of algebras $\mc D_{X_n/W_n}$ does depend on the lift to $X_n$, so it is remarkable that we are able to construct $\op{Cris}(X/W_n(k))$ at all!
\end{remark}
We have been using a bit of deformation theory (for motivation) throughout, but let's state what we need more explicitly.
\begin{theorem}[Deformation] \label{thm:deformation}
	Fix a commutative ring $A$ and a nilpotent thickening $A\to A/I$ where $I^2=0$, and choose a smooth scheme $Y$ over $A/I$.
	\begin{listalph}
		\item There is a (natural) class $\mathrm{ob}_{Y,A}\in\mathrm H^2(Y;T_{Y/(A/I)}\otimes_{A/I}I)$ which vanishes if and only if there is a smooth lift $\widetilde Y$ over $A$ for which $\widetilde Y_{A/I}\cong Y$.
		\item Given a choice of lift $\widetilde Y$ with a choice of isomorphism $\rho\colon\widetilde Y_{A/I}\to Y$, then $\mathrm H^1(Y;T_{Y/(A/I)}\otimes I)$ parameterizes all lifts.
		\item Given a choice of lift $\widetilde Y$ with a choice of isomorphism $\rho\colon\widetilde Y_{A/I}\to Y$, then the group of automorphisms $f\colon\widetilde Y\to\widetilde Y$ which reduce to the identity modulo $I$ is isomorphic to $\mathrm H^0(Y;T_{Y/(A/I)}\otimes_{A/I}I)$.
	\end{listalph}
\end{theorem}
\begin{example}
	By (a), we see that one can always lift smooth curves.
\end{example}
\begin{remark}
	One can directly prove the commutativity of the group of automorphisms (roughly) as follows: $f$ amounts to a deformation which looks like ${\id}+\varepsilon$ (e.g., on the sheaf of rings), which all commute with each other because $I^2=0$.
\end{remark}
Let's explain how we use \Cref{thm:deformation}: if $X$ is a smooth variety over a field $k$, then $X$ is separated, so it admits an affine open cover $\{U_i\}_{i\in\Lambda}$ so that each intersection $U_i\cap U_j$ is still affine. By the first two parts of \Cref{thm:deformation} (applied iteratively), we are granted a unique smooth lift $\widetilde U_\alpha$ over $W_n(k)$ (because higher cohomology of affine schemes will vanish). By the uniqueness, we are granted isomorphisms
\[f_{ij}\colon\OO_{\widetilde U_i}|_{U_{ij}}\cong\OO_{\widetilde U_j}|_{U_{ij}},\]
but the third part of \Cref{thm:deformation} tells us that there are many choices for such an automorphism. Nonetheless, for any triple $(i,i',i'')$, we see that the composite $f_{i''i}f_{i'i''}f_{ii'}$ is the identity$\pmod p$. But this does not mean that our sheaves glue! This explains why we cannot lift $X$.

However, we will be able to lift $\mc D$-modules to a category of $\mc D$-modules with $W_n$-coefficients without ever having to lift $X$ to $X_n$. The key is the following, which asserts that such ``automorphisms which are the identity$\pmod p$'' are inner for $\mc D$-modules.
\begin{lemma} \label{lem:aut-of-d-mod-is-inner}
	Suppose $p>2$, and fix a smooth affine scheme $Y_n$ over $W_n$. Then for any automorphism $f$ of $Y_n$ which is the identity$\pmod p$, there is a $D_f\in\mc D_{Y_n/W_n}(Y_n)^\times$ which is $1\pmod p$, and
	\[f^*\delta=D_f\delta D_f^{-1}\]
	for any $\delta\in\mc D_{Y_n/W_n}(Y_n)$.
\end{lemma}
\begin{proof}
	We may find $\Delta_f$ to satisfy
	\[p\Delta_f=\log f^*,\]
	where $\log f^*$ refers to the power series $\sum_{r\ge1}(-1)^{r+1}(f-{\id})^{\circ r}/r$. (This exists because $p>2$.) This logarithm is some endomorphism on $\OO_{Y_n}$, and it is even a derivation of $\OO_{Y_{n-1}}$, which one can check directly. We now define $D_f\coloneqq\exp(p\del_{\Delta_f})$, and one can check the identity directly from the identity $f^*=\exp(p\Delta_f)$.
\end{proof}
\begin{remark}
	The moral of \Cref{lem:aut-of-d-mod-is-inner} is that any $\mc D_{Y_n}$-module $M$ admits a natural isomorphism $M\cong(f^*)_*M$, where the right-hand module means that $\mc D_{Y_n}$ acts on $M$ through the automorphism $f^*$. Indeed, simply send $m$ to $D_fm$.
\end{remark}
\begin{remark}
	\Cref{lem:aut-of-d-mod-is-inner} even has a formulation in characteristic zero, which admits basically the same proof.
\end{remark}
We are now ready to prove \Cref{thm:construct-crystal}.
\begin{proof}[Proof of \Cref{thm:construct-crystal}]
	Fix an affine open cover $\{U_i\}_{i\in\Lambda}$ on $X$ with $f_{ij}$s as before. The category of $\mc D_{X/k}$-modules can be thought of as the category of big tuples $\{M_i\}_i$, where $M_i$ is a module for $\mc D_{X/k}(U_i)$, and we are equipped with isomorphisms between $M_i|_{U_{ij}}$ and $M_j|_{U_{ij}}$ that satisfy some cocycle condition. In light of the discussion preceding \Cref{lem:aut-of-d-mod-is-inner}, our problem is that we do not know how to fix the cocycle condition.

	We now imitate this definition. Define our category $\op{Cris}(X/W_n(k))$ of crystals as being the category of tuples $\{M_i\}_i$ of modules, where $M_i$ is a module over $\mc D_{\widetilde U_i/W_n}(\widetilde U_i)$, equipped with isomorphisms
	\[\alpha_{ij}\colon f^*_{ij}M_i|_{U_{ij}}\to M_j|_{U_{ij}}\]
	such that the composites $\alpha_{i''i}\alpha_{i'i''}\alpha_{ii'}$ produces an automorphism $f_{i''i}^*f_{i'i''}^*f_{ii'}^*M_i\to M_i$, where this is an automorphism where we have identified the left module with $M_i$ via \Cref{lem:aut-of-d-mod-is-inner}.

	It may appear that our construction depends rather poorly on the choice of cover. However, this is not the case: one can simply show that the uniqueness of our constructions means that our construction does not depend on refinements of open covers, so it does not actually depend on the choice of open cover. (It may look like this is so because the identifications depend on the choice of $D_f$ in \Cref{lem:aut-of-d-mod-is-inner}, but this construction was in fact canonical in the proof.) Professor Petrov seems to think that one will need to check some cocycle condition on intersections of four open subsets.

	We now run our checks.
	\begin{itemize}
		\item For example, we note that the $\OO_{\widetilde U_i}$s glue to an object $\OO_X^{\mathrm{cris}}$ because all data is only every constructed$\pmod p$, and the cocycle condition is satisfied by the particular construction of $D_f$.
		\item Given a choice of lift $X_n$, we may actually lift everything to $X_n$, which shows that our crystals agree with $\mc D$-modules on $X_n$.
		\qedhere
	\end{itemize}
\end{proof}
\begin{remark}
	One can relax the hypothesis that $X$ is separated by further covering the $U_{ij}$s by affines.
\end{remark}
\begin{remark}
	These are called ``crystals'' because they are rigid and they grow. Namely, we constructed them locally, but they were rigid enough to be able to be glued to a global category.
\end{remark}
\begin{remark}
	Additionally, morphisms of varieties are finite type and separated, so it is not too hard to construct functoriality for our category of crystals. Explicitly, a morphism $f\colon X\to Y$ produces a morphism
	\[f^*\colon\mathrm R\Gamma_{\mathrm{cris}}(Y/W(k))\to\mathrm R\Gamma(X/W(k)).\]
\end{remark}

\end{document}