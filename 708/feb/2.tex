% !TEX root = ../notes.tex

\documentclass[../notes.tex]{subfiles}

\begin{document}

\section{February 2}
Here we go.

\subsection{Algebraic de Rham Cohomology}
Let's begin by describing what we mean by de Rham cohomology. We will consider a smooth variety $X$ over an algebraically closed field $F$.
\begin{definition}[smooth]
	We say that a variety $X$ over a field $F$ is \textit{smooth} if and only if $\Omega_{X/F}$ is a vector bundle of rank $\dim X$ on each connected component. Here, on an affine open subset $U\subseteq X$, recall that $\Omega_{X/F}(U)$ is spanned by symbols of the form $f\,dg$, where the symbol $d$ is (as usual) $F$-linear and satisfies the Leibniz rule.
\end{definition}
\begin{definition}[algebraic de Rham cohomology]
	Fix a smooth variety $X$ over a field $F$. Then one can iterate the $F$-linear map $d\colon\OO_X\to\Omega_{X/F}$ to a map $d\colon\Omega_{X/F}^i\to\Omega_{X/F}^{i+1}$ for each $i$, where $\Omega_{X/F}^i\coloneqq\land^i\Omega_{X/F}$. We now define the \textit{de Rham complex} to the complex
	\[\Omega_{X/F}^\bullet\colon 0\to\OO_X\stackrel d\to\Omega^1_{X/F}\stackrel d\to\cdots,\]
	and we define the \textit{de Rham cohomology} $\mathrm H^n_{\mathrm{dR}}(X/F)$ to be the $n$th hypercohomology of $\Omega_{X/F}^\bullet$. Here, hypercohomology means the total cohomology of some produced acyclic double complex which resolves the complex (e.g., a \v{C}ech resolution). Note that this hypercohomology is merely a vector space over $F$.
\end{definition}
\begin{example}
	The map $d\colon\Omega^1_{X/F}\to\Omega^2_{X/F}$ is given by $d(f\,dg)=df\land dg$.
\end{example}
\begin{example}
	Suppose that $X$ is affine. Then vector bundles are already acyclic, so the hypercohomology does nothing. Thus,
	\[\mathrm H^n_{\mathrm{dR}}(X/F)=\mathrm H^n\left(X;0\to\OO_X\stackrel d\to\Omega^1_{X/F}\stackrel d\to\cdots\right).\]
	As usual, this is $\ker\left(d|_{\Omega^n}\right)/\im\left(d|_{\Omega^{n-1}}\right)$.
\end{example}
\begin{remark}
	If $X$ is affine and $i>\dim X$, then $\Omega^i_{X/F}$ vanishes, so the algebraic de Rham cohomology also vanishes.
\end{remark}
\begin{remark}
	A different definition is required for non-smooth $X$. Roughly speaking, one should embed into a smooth variety and take cohomology there.
\end{remark}
Here is one way to convince ourselves that this is a reasonable cohomology theory.
\begin{theorem}[Grothendieck] \label{thm:alg-dr-to-betti}
	Suppose that $X$ is a smooth variety over $\CC$. Then there is a canonical isomorphism
	\[\mathrm H^n_{\mathrm B}(X(\CC);\CC)\to\mathrm H^n_{\mathrm{dR}}(X/\CC).\]
	Here, the left-hand side is Betti cohomology (also called singular cohomology).
\end{theorem}
\begin{proof}[Sketch]
	We argue in the case that $X$ is affine. Then $X(\CC)$ already has a notion of $\Omega^{i,\mathrm{an}}_{X/\CC}$ given by the holomorphic forms. Algebraic forms embed into holomorphic ones, which produces a morphism
	% https://q.uiver.app/#q=WzAsMTAsWzAsMCwiMCJdLFsxLDAsIlxcT09fWChYKSJdLFsyLDAsIlxcT21lZ2FeMV97WC9cXENDfShYKSJdLFswLDEsIjAiXSxbMSwxLCJcXE9PX1hee1xcbWF0aHJte2FufX0oWCkiXSxbMiwxLCJcXE9tZWdhX3tYL1xcQ0N9XnsxLFxcbWF0aHJte2FufX0oWCkiXSxbMywwLCJcXE9tZWdhXjJfe1gvXFxDQ30oWCkiXSxbMywxLCJcXE9tZWdhX3tYL1xcQ0N9XnsyLFxcbWF0aHJte2FufX0oWCkiXSxbNCwwLCJcXGNkb3RzIl0sWzQsMSwiXFxjZG90cyJdLFswLDFdLFsxLDJdLFsyLDZdLFszLDRdLFs0LDVdLFs1LDddLFs3LDldLFs2LDhdLFsxLDQsIiIsMSx7InN0eWxlIjp7InRhaWwiOnsibmFtZSI6Imhvb2siLCJzaWRlIjoidG9wIn19fV0sWzIsNSwiIiwxLHsic3R5bGUiOnsidGFpbCI6eyJuYW1lIjoiaG9vayIsInNpZGUiOiJ0b3AifX19XSxbNiw3LCIiLDEseyJzdHlsZSI6eyJ0YWlsIjp7Im5hbWUiOiJob29rIiwic2lkZSI6InRvcCJ9fX1dXQ==&macro_url=https%3A%2F%2Fraw.githubusercontent.com%2FdFoiler%2Fnotes%2Fmaster%2Fnir.tex
	\[\begin{tikzcd}[cramped]
		0 & {\OO_X(X)} & {\Omega^1_{X/\CC}(X)} & {\Omega^2_{X/\CC}(X)} & \cdots \\
		0 & {\OO_X^{\mathrm{an}}(X)} & {\Omega_{X/\CC}^{1,\mathrm{an}}(X)} & {\Omega_{X/\CC}^{2,\mathrm{an}}(X)} & \cdots
		\arrow[from=1-1, to=1-2]
		\arrow[from=1-2, to=1-3]
		\arrow[hook, from=1-2, to=2-2]
		\arrow[from=1-3, to=1-4]
		\arrow[hook, from=1-3, to=2-3]
		\arrow[from=1-4, to=1-5]
		\arrow[hook, from=1-4, to=2-4]
		\arrow[from=2-1, to=2-2]
		\arrow[from=2-2, to=2-3]
		\arrow[from=2-3, to=2-4]
		\arrow[from=2-4, to=2-5]
	\end{tikzcd}\]
	of complexes. It then turns out that this is an isomorphism on cohomology, so we reduce to comparing analytic de Rham cohomology with singular cohomology.

	This is now a problem of analysis. One can pass from holomorphic differentials to smooth differentials via a similar process, which produces another morphism
	% https://q.uiver.app/#q=WzAsMTAsWzAsMCwiMCJdLFsxLDAsIlxcT09fWF57XFxtYXRocm17YW59fShYKSJdLFsyLDAsIlxcT21lZ2Ffe1gvXFxDQ31eezEsXFxtYXRocm17YW59fShYKSJdLFszLDAsIlxcT21lZ2Ffe1gvXFxDQ31eezIsXFxtYXRocm17YW59fShYKSJdLFs0LDAsIlxcY2RvdHMiXSxbMCwxLCIwIl0sWzEsMSwiQ15cXGluZnR5KFgoXFxDQyksXFxDQykiXSxbMiwxLCJcXE9tZWdhXjFfe0NeXFxpbmZ0eX0oWChcXENDKSkiXSxbMywxLCJcXE9tZWdhXjJfe0NeXFxpbmZ0eX0oWChcXENDKSkiXSxbNCwxLCJcXGNkb3RzIl0sWzAsMV0sWzEsMl0sWzIsM10sWzMsNF0sWzUsNl0sWzYsN10sWzcsOF0sWzgsOV0sWzEsNiwiIiwxLHsic3R5bGUiOnsidGFpbCI6eyJuYW1lIjoiaG9vayIsInNpZGUiOiJ0b3AifX19XSxbMiw3LCIiLDEseyJzdHlsZSI6eyJ0YWlsIjp7Im5hbWUiOiJob29rIiwic2lkZSI6InRvcCJ9fX1dLFszLDgsIiIsMSx7InN0eWxlIjp7InRhaWwiOnsibmFtZSI6Imhvb2siLCJzaWRlIjoidG9wIn19fV1d&macro_url=https%3A%2F%2Fraw.githubusercontent.com%2FdFoiler%2Fnotes%2Fmaster%2Fnir.tex
	\[\begin{tikzcd}[cramped]
		0 & {\OO_X^{\mathrm{an}}(X)} & {\Omega_{X/\CC}^{1,\mathrm{an}}(X)} & {\Omega_{X/\CC}^{2,\mathrm{an}}(X)} & \cdots \\
		0 & {C^\infty(X(\CC),\CC)} & {\Omega^1_{C^\infty}(X(\CC))} & {\Omega^2_{C^\infty}(X(\CC))} & \cdots
		\arrow[from=1-1, to=1-2]
		\arrow[from=1-2, to=1-3]
		\arrow[hook, from=1-2, to=2-2]
		\arrow[from=1-3, to=1-4]
		\arrow[hook, from=1-3, to=2-3]
		\arrow[from=1-4, to=1-5]
		\arrow[hook, from=1-4, to=2-4]
		\arrow[from=2-1, to=2-2]
		\arrow[from=2-2, to=2-3]
		\arrow[from=2-3, to=2-4]
		\arrow[from=2-4, to=2-5]
	\end{tikzcd}\]
	of complexes, which is also an isomorphism on complexes. We are now reduced to the setting of de Rham's theorem for real manifolds.
\end{proof}
\begin{example}
	Consider $X\coloneqq\AA^1_\CC\setminus\{0\}=\Spec k[t,1/t]$.
	\begin{itemize}
		\item Our differential map $d\colon\CC[t,1/t]\to\CC[t,1/t]\,dt$ sends $t^n$ to $nt^{n-1}\,dt$. Thus, $\mathrm H^0_{\mathrm{dR}}(X)$ is one-dimen\-sional given by the constants, and $\mathrm H^1_{\mathrm{dR}}(X)$ is one-dimensional spanned by $dt/t$.
		\item The point above works also for holomorphic differentials. The interesting bit is in degree $1$, where the point is that there is no global antiderivative for $dx/x$.
		\item On the other hand, $X(\CC)=\CC\setminus\{0\}$ is homotopy equivalent to the circle, so we expect its singular cohomology to be supported in degrees $0$ and $1$, where it should be one-dimensional.
	\end{itemize}
\end{example}
\begin{corollary}[Artin vanishing]
	If $X$ is an affine algebraic complex smooth variety, then $\mathrm H^n(X(\CC);\CC)=0$ for $n>\dim X$.
\end{corollary}
\begin{proof}
	The algebraic de Rham cohomology complex vanishes above $\dim X$.
\end{proof}
\begin{corollary}
	Fix a smooth variety $X$ over $\CC$. Then $\mathrm H^n_{\mathrm{dR}}(X/\CC)$ is finite-dimensional.
\end{corollary}
\begin{proof}
	Pass to singular cohomology.
\end{proof}
\begin{remark}
	This corollary still admits algebraic proofs in characteristic zero by working with holonomic $\mc D$-modules. Pavel Etingof claims that there is an algebraic proof using the fact that the direct image of a holonomic $\mc D$-module is a holonomic $\mc D$-module.
\end{remark}
We would like to point out that our de Rham cohomology is algebraic but still interesting.
\begin{remark}
	Suppose that $X$ is smooth over $\QQ$. Base-changing by a field is exact, so
	\[\mathrm H^n_{\mathrm{dR}}(X/\QQ)_\CC\cong\mathrm H^n_{\mathrm{dR}}(X_\CC/\CC).\]
	However, \Cref{thm:alg-dr-to-betti} grants an isomorphism to $\mathrm H^n(X(\CC);\CC)\cong\mathrm H^n_{\mathrm B}(X(\CC);\ZZ)_\CC$. Notably, we then find a lattice and a rational structure over in some complex vector space, but the comparison between the two is quite interesting mathematically (and amounts to the study of periods).
\end{remark}
\begin{example}
	In the case that $X=\AA^1_\QQ\setminus\{0\}$, the comparison between $\mathrm H^1_{\mathrm{dR}}(X/\QQ)_\CC$ and $\mathrm H^1_{\mathrm B}(X(\CC);\ZZ)$ is mediated by a constant $2\pi i$. Indeed, once unwinds the de Rham theorem, this amounts to the statement that a contour integral of $dx/x$ going once around the origin is $2\pi i$.
\end{example}

\subsection{Frobenius Structure}
We now pass to positive characteristic. Let $k$ be a perfect field of positive characteristic $p$, and we may still consider a smooth variety $X$.
\begin{remark}
	If $k$ is perfect, then $\Omega^1_{X/k}=\Omega^1_{X/\FF_p}$ by doing some thinking about inseparable extensions. The moral is that
	\[y^{1/p}\,dy=d\left((y^{1/p})^p\right),\]
	so the coefficients can be brought down when everything is a $p$th power.
\end{remark}
This cohomology is rather strangely behaved.
\begin{example}
	Take $X\coloneqq\AA^1_k$. The de Rham cohomology still lives in degrees zero and one, so we would like to study the kernel and cokernel of the $k$-linear map $d\colon k[t]\to k[t]\,dt$ given by $t^n\mapsto nt^{n-1}$.
	\begin{itemize}
		\item We see that $\mathrm H^0_{\mathrm{dR}}(\AA^1_k/k)=\ker d$ is spanned by $t^{pi}$ for each $i$.
		\item We see that $\mathrm H^1_{\mathrm{dR}}(\AA^1_k/k)=\im d$ is infinite-dimensional because the differentials $t^{mp-1}\,dt$ fail to be in the image. In fact, these classes form a basis.
	\end{itemize}
\end{example}
Let's try to view these infinite-dimensional groups as a feature instead of a bug. Indeed, it turns out that the de Rham complex has some extra structure. The de Rham complex
\[0\to\OO_X\stackrel d\to\Omega^1_{X/k}\stackrel d\to\Omega^2_{X/k}\stackrel d\to\cdots\]
is merely made of sheaves of $k$-vector spaces over $X$. In characteristic zero, this is all the structure present, but in characteristic $p$, we have more structure.
\begin{notation}
	Fix a variety $X$ over a field $k$ of characteristic $p$. For a sheaf $\mc F$ of $\OO_X$-modules, we define
	\[\mc F^p\coloneqq\{f^p:f\in\OO_X\}\]
	to locally be given by the $p$th powers.
\end{notation}
The moral is that $d(f^p)=0$ always, so the de Rham complex is in fact $\OO_X^p$-linear! Let's attempt to codify this.
\begin{definition}[relative Frobenius]
	Fix a scheme $X$ over a field $k$ of characteristic $p$. Then there is an \textit{absolute Frobenius} $F_{\mathrm{abs}}\colon X\to X$ which is the identity on topological spaces and the $p$th power on sheaves. This is a morphism of schemes but not of $k$-schemes (in general). The \textit{relative Frobenius} $F\colon X\to X^{(1)}$ is the morphism fitting into the following diagram.
	% https://q.uiver.app/#q=WzAsNSxbMSwyLCJrIl0sWzIsMiwiayJdLFsyLDEsIlgiXSxbMSwxLCJYXnsocCl9Il0sWzAsMCwiWCJdLFsyLDFdLFswLDEsIkZfe1xcbWF0aHJte2Fic319Il0sWzQsMiwiRl97XFxtYXRocm17YWJzfX0iLDAseyJjdXJ2ZSI6LTJ9XSxbNCwwLCIiLDEseyJjdXJ2ZSI6Mn1dLFszLDBdLFszLDJdLFs0LDMsIiIsMSx7InN0eWxlIjp7ImJvZHkiOnsibmFtZSI6ImRhc2hlZCJ9fX1dLFszLDEsIiIsMSx7InN0eWxlIjp7Im5hbWUiOiJjb3JuZXIifX1dXQ==&macro_url=https%3A%2F%2Fraw.githubusercontent.com%2FdFoiler%2Fnotes%2Fmaster%2Fnir.tex
	\[\begin{tikzcd}[cramped]
		X \\
		& {X^{(1)}} & X \\
		& k & k
		\arrow[dashed, from=1-1, to=2-2]
		\arrow["{F_{\mathrm{abs}}}", curve={height=-12pt}, from=1-1, to=2-3]
		\arrow[curve={height=12pt}, from=1-1, to=3-2]
		\arrow[from=2-2, to=2-3]
		\arrow[from=2-2, to=3-2]
		\arrow["\lrcorner"{anchor=center, pos=0.125}, draw=none, from=2-2, to=3-3]
		\arrow[from=2-3, to=3-3]
		\arrow["{F_{\mathrm{abs}}}", from=3-2, to=3-3]
	\end{tikzcd}\]
\end{definition}
\begin{remark}
	Note that $X^{(p)}$ is isomorphic to $X$ as a scheme but not as a $k$-scheme! However, we now benefit because the relative Frobenius $F$ is morphism of $k$-schemes.
\end{remark}
\begin{remark}
	The relative Frobenius $F\colon X\to X^{(1)}$ is finite flat of degree $p^{\dim X}$
\end{remark}
\begin{example}
	If $X=\Spec k[t_1,\ldots,t_n]$, then $X^{(1)}=\Spec k\left[t_1^p,\ldots,t_n^p\right]$. Thus, we see that the embedding
	\[k\left[t_1^p,\ldots,t_n^p\right]\subseteq k[t_1,\ldots,t_n]\]
	is indeed finite flat of degree $p^n$.
\end{example}
We now see that
\[0\to F_*\OO_X\stackrel d\to F_*\Omega^1_{X/k}\stackrel d\to F_*\Omega^2_{X/F}\to\cdots\]
is a complex of quasicoherent sheaves on $X^{(1)}$. In fact, because $F$ is finite flat, these are all vector bundles: $F_*\OO_X$ has rank $p^{\dim X}$ and $F_*\Omega^i_{X/k}$ has rank $p^{\dim X}\binom{\dim X}i$. Because $\OO_{X^{(1)}}=(F_*\OO_X)^p$, we see that this complex is in fact $\OO_{X^{(1)}}$-linear.
\begin{example}
	Take $X=\Spec k[t]$. Then $X^{(1)}\coloneqq\Spec k\left[t^p\right]$, and $d\colon k[t]\to k[t]\,dt$ is $k\left[t^p\right]$-linear! Thus, $\mathrm H^i_{\mathrm{dR}}(X/k)$ was required to be given by $k\left[t^p\right]$-modules, which explains why we received vector spaces of infinite dimension.
\end{example}
Note that passing through $F_*$ is not going to adjust the underlying $k$-vector spaces, so
\[\mathrm H^n_{\mathrm{dR}}(X/k)=\mathbb H^n_{\mathrm{Zar}}\left(X^{(1)};0\to F_*\OO_X\stackrel d\to F_*\Omega^1_{X/k}\stackrel d\to F_*\Omega^2_{X/k}\stackrel d\to\cdots\right).\]
To see why this has globalized the $\OO_X^p$-linearity, we need the Cartier isomorphism.
\begin{theorem}[Cartier isomorphism] \label{thm:cartier-iso}
	Fix a smooth variety $X$ over a perfect field $k$. Then there is a canonical isomorphism
	\[\mc H^i(F_*\Omega^\bullet_X)\cong\Omega^i_{X^{(1)}}.\]
	Here, the left-hand side is a coherent $\OO_{X^{(1)}}$-module.
\end{theorem}
\begin{remark}
	This is a reason why characteristic $p$ may be more convenient than characteristic $0$: one could still try to understand $\mc H^i(\Omega^\bullet_{X/k})$ when $\op{char}k=0$, but this has no easy answer.
\end{remark}
\begin{example}
	Consider $X=\AA^1_k$. Then $\mc H^1$ is given by the module
	\[\frac{k[t]\,dt}{d(k[t])},\]
	which our formalism now remembers is a $k\left[t^p\right]$-module. And indeed, we can show that this is isomorphic to $k\left[t^p\right]\cdot t^{p-1}\,dt$. Setting $s\coloneqq t^p$, we know that $\Omega^1_{X^{(1)}/k}$ is given by the module $k[s]\,ds$, so our isomorphism of modules is given by sending $ds$ to $t^{p-1}\,dt$. One can even check that this isomorphism is canonical in the sense that it will not change under automorphisms of $\AA^1$.
\end{example}
We will prove \Cref{thm:cartier-iso} later after a detour.

\subsection{Crystalline Cohomology}
We continue with our perfect field $k$ of positive characteristic $p$. Our story so far has taken a variety $X$ over a field $k$, and then we have produced some (total) complex $\mathrm R\Gamma_{\mathrm{dR}}(X/k)$ in the derived category $D(\mathrm{Vec}_k)$. Crystalline cohomology will allow us to produce an answer in characteristic $0$ instead of characteristic $p$. The idea is to ``choose'' a lift to characteristic $p$ and then check that the answer is independent of the lift.

The correct formalism for this lifting is that of a ``formal scheme.''
\begin{definition}[Witt ring]
	Fix a perfect field $k$ of characteristic $p$. Then there is a ring $W(k)$ satisfying that
	\begin{itemize}
		\item $W(k)$ is $p$-torsion-free,
		\item $W(k)/p\cong k$, and
		\item $W(k)$ is the limit of the $W(k)/p^n$ as $n\to\infty$.
	\end{itemize}
	This ring $W(k)$ turns out to be unique up to unique isomorphism. We may write $W_n(k)\coloneqq W(k)/p^n$.
\end{definition}
\begin{example}
	One can see that $W(\FF_p)=\ZZ_p$ and $W(\ov\FF_p)$ is its unramified closure.
\end{example}
\begin{remark}
	There is a completely explicit construction of $W(k)$, but it is rather involved: given a $p$-torsion-free ring $R$, we identify $W(R)\coloneqq R^\NN$ but with ring structure chosen so that
	\[(a_0,a_1,a_2,\ldots)\mapsto a_0^{p^n}+pa_1^{p^{n-1}}+\cdots+p^na_n\]
	is a ring homomorphism $W(R)\to R$. It turns out that this ring structure is given by some polynomials (called ``ghost coordinates''), so we are allowed to define $W(k)$. From a higher level, it turns out that $W(k)$ is the unique deformation of $k$, which exists because $\Omega^1_{k/\FF_p}=0$.
\end{remark}
\begin{definition}[formal scheme]
	Fix a perfect field $k$ of characteristic $p$. A \textit{$p$-adic formal scheme} $X$ is a collection of schemes $X_n$ over $W_n(k)$ equipped with isomorphisms
	\[X_{n+1}\times_{W_{n+1}(k)}W_n(k)\to X_n.\]
	The structure sheaf $\widehat{\OO}_X$ is the inverse limit of the $\OO_{X_n}$s.
\end{definition}
\begin{example}
	Given a scheme $X$ over $W(k)$, we can produce a formal scheme $\widehat Y$ with $\widehat Y_n\coloneqq Y\times_{W(k)}W_n(k)$ and the induced internal isomorphisms.
\end{example}
\begin{remark}
	We can even define $\widehat\Omega X^1_{\widetilde X}$.
\end{remark}
We can now describe crystalline cohomology.
\begin{theorem}
	Fix a perfect field $k$ of positive characteristic $p$. Then there is a functor sending smooth $k$-varieties $X$ to a complex $\mathrm R\Gamma_{\mathrm{cris}}(X/W(k))$ in the derived category $D(\mathrm{Mod}_{W(k)})$ satisfying the following.
	\begin{listalph}
		\item There is a quasi-isomorphism $\mathrm R\Gamma_{\mathrm{cris}}(X/W(k))\otimes^{\mathbb L}_{W(k)}k\cong\mathrm R\Gamma_{\mathrm{dR}}(X/k)$.
		\item If $\widetilde X$ is a smooth formal scheme over $W(k)$ (meaning that $\widetilde X_n$ is smooth over $W_n(k)$ for all $n$), then
		\[\mathrm R\Gamma_{\mathrm{cris}}(X_1/W(k))\cong\mathrm R\Gamma_{\mathrm{Zar}}\left(X;\widehat\OO_{\widetilde X}\stackrel d\to\widehat\Omega^1_{\widetilde X}\stackrel d\to\cdots\right)\]
	\end{listalph}
\end{theorem}
\begin{remark}
	Here, (a) immediately tells us that the cohomology of $\mathrm R\Gamma_{\mathrm{cris}}(X/W(k))$ is not expected to be finitely generated.
\end{remark}
\begin{remark}
	There is something remarkable here, which is that choosing two different lifts of $X$ to a smooth formal scheme produces the same cohomology!
\end{remark}
\begin{remark}
	It turns out that flatness is equivalent to smoothness in this context.
\end{remark}

\end{document}