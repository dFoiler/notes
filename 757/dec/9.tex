% !TEX root = ../notes.tex

\documentclass[../notes.tex]{subfiles}

\begin{document}

\section{December 4}
Today we will prove localization.

\subsection{Localization for \texorpdfstring{$\PP^n$}{ Pn}}
As a warm-up, let's show the following.
\begin{proposition} \label{prop:localize-pn}
	We show that $\PP^n$ is $D$-affine. In particular, there is an equivalence
	\[\Gamma\colon\mathrm{Mod}(\mc D_{\PP^n})\to\mathrm{Mod}(\Gamma(\mc D_{\PP^n})).\]
\end{proposition}
\begin{proof}
	As in our sketch of \Cref{thm:twist-local}, there are two things to check: we need to check that $\Gamma$ is exact and $\Gamma(M)\ne0$ for nonzero $M$.
	\begin{itemize}
		\item We check that $\Gamma$ sends nonzero elements to nonzero elements. Given a surjection $\pi\colon\AA^{n+1}\setminus\{0\}\to\PP^n$, there is a functor $\pi^*\colon\mathrm{Mod}(\mc D_{\PP^n})\to\mathrm{Mod}(\mc D_{\AA^n\setminus\{0\}})$, and one finds that
		\[\Gamma(\pi^* M)=\bigoplus_i\Gamma(M(i)),\]
		where $M(i)$ denotes the twist $M\otimes\OO(i)$.

		Now, it is a fact that if $\mc F$ is a nonzero quasicoherent sheaf on some $\PP^n$, then $\Gamma(\mc F(i))\ne0$ for $i$ large enough. Thus, $\Gamma(M(i))\ne0$ for $i$ large enough, for which we will use a group action. One can view $\pi$ as $\mathbb G_m$-bundle, meaning that it acts on $\AA^{n+1}\setminus\{0\}$ with quotient $\PP^n$. Taking the differential gives a Lie algebra action, so we fix a generator $e\in\op{Lie}\mathbb G_m$, which up to scalar acts by
		\[\sum_ix_i\del_i.\]
		Namely, $e$ should act by $i\cdot\id$ on each $\Gamma(M(i))$ in the decomposition of $\Gamma(\pi^*M)$, meaning that $\Gamma(\pi^*M)_i=\Gamma(M(i))$. Accordingly, $\Gamma(\pi^*M)_i$ vanishing implies that $\Gamma(\pi^*M)_{i+1}$ vanishes, unless $i+1=0$. Indeed, for $v\in\Gamma(\pi^*M)_{i+1}$, we have
		\[(i+1)v=\sum_ix_i\underbrace{\del_iv}_{\in\Gamma(\pi^*M)_i},\]
		so $\Gamma(\pi^*M)_i$ vanishing will imply that $\Gamma(\pi^*M)_{i+1}$ vanishes. This action then forces $\Gamma(M)\ne0$, as required.

		\item We check that $\Gamma$ is exact. Note $\Gamma$ is always left-exact, so it suffices to show that it sends surjections to surjections. Thus, we select a surjection $M\onto M'$, and we want to exhibit a surjection on global sections.
		
		We use the fact that $\mathrm H^d(\mc F(i))=0$ for any coherent sheaf $\mc F$ and any $d>0$ as long as $i$ has been chosen large enough. Accordingly, choose a section $s'\in\Gamma(M')$. Then for $i$ large enough, the map $s'\colon\OO(i)\to M'(i)$ admits a section $\OO(i)\to M(i)$. (To see this, pass from $M'$ to the sheaf generated by $s$.) In other words, we are granted some monomial $x_{i_1}\cdots x_{i_d}$ for which $x_{i_1}\cdots x_{i_d}s$ lifts to $M(i)$, and we will use the $\mathbb G_m$-action to get rid of the monomial by descending induction. Indeed, write
		\[e=\sum_i\del_ix_i-(n+1).\]
		Then for $v\in\Gamma(\pi^*M)_i$, we can plug in $v$ to $ev=iv$ to show that $x_iv$ living in the image implies that $v$ does.
		\qedhere
	\end{itemize}
\end{proof}
\begin{remark}
	The same argument shows that
	\[\Gamma\colon\mathrm{Mod}(\mc D_{\PP^n,c})\to\mathrm{Mod}(\Gamma(\mc D_{\PP^n}))\]
	is exact except for $c\notin-n-\ZZ_{\ge1}$ and is conservative (sending nonzero elements to nonzero elements) for $c\notin-\ZZ_{\ge1}$.
\end{remark}

\subsection{Generalizing to \texorpdfstring{$G/B$}{ G/B}}
Let's explain how to generalize this argument to the flag variety $G/B$. To start, we will replace $\AA^{n+1}\setminus\{0\}\onto\PP^n$ with $G/U\onto G/B$. Let's explain which features are preserved: $G/U$ is quasi-affine, embedding as a dense open subset into an affine scheme $\overline{G/U}$. The big problem is that $\overline{G/U}$ is not smooth in general, so we will need to understand this affine scheme in more detail. One finds that
\[\CC[G/U]=\bigoplus_{\lambda\in\Lambda^+}V_\lambda,\]
where the multiplication sends $V_\lambda\otimes V_\mu\to V_{\lambda+\mu}$. This will be analogous to the ``weight decomposition'' which we used to shift $i$s around in the proof of \Cref{prop:localize-pn}.

To generalize the element $e$, we need a formal Fourier transform.
\begin{remark}
	In the case of $\AA^n$, we note that $\mc D$-modules on a vector space can be described as
	\[\frac{\CC\langle V,V'\rangle}{(\varphi\varphi'-\varphi'\varphi,vv'-v'v,[v\varphi]-\varphi(v))_{v,v'\in V,\varphi,\varphi'\in V^*}}.\]
	Thus, we can squint and see a Fourier transform $\Phi$ which simply switches the roles of $V$ and $V^*$. For example, after fixing a basis $\{x_1,\ldots,x_n\}$, one finds that $\Phi(x_i)=\del_i$.
\end{remark}
It turns out that the previous remark globalizes to any vector bundle $\mc V$ over a scheme $S$, producing a Fourier transform $\Phi\colon\mc D(\mc V)\to\mc D(\mc V^*)$.

For example, let's suppose that we want to show that $\Gamma$ is conservative. Then one finds
\[\Gamma(\pi^*M)=\bigoplus_\lambda\Gamma(M(\lambda)),\]
where $M(\lambda)$ means $M\otimes\OO(\lambda)$. To explain how we can shift these $\alpha$s around, we fix a root $\alpha$, and we let $P_\alpha$ be the minimal parabolic containing $B$ and $U_\alpha$. In this situation, we get a morphism
% https://q.uiver.app/#q=WzAsNixbMSwxLCJHL0IiXSxbMiwxLCJHL1BfXFxhbHBoYSJdLFswLDEsIlxcUFBeMSJdLFsxLDAsIkcvVSJdLFsyLDAsIkcvXFxvcHtyYWR9UF9cXGFscGhhIl0sWzAsMCwiXFxBQV4yXFxzZXRtaW51c1xcezBcXH0iXSxbMiwwXSxbMCwxXSxbNCwxXSxbMywwXSxbMyw0XSxbNSwyXSxbNSwzXV0=&macro_url=https%3A%2F%2Fraw.githubusercontent.com%2FdFoiler%2Fnotes%2Fmaster%2Fnir.tex
\[\begin{tikzcd}[cramped]
	{\AA^2\setminus\{0\}} & {G/U} & {G/[P_\alpha,P_\alpha]} \\
	{\PP^1} & {G/B} & {G/P_\alpha}
	\arrow[from=1-1, to=1-2]
	\arrow[from=1-1, to=2-1]
	\arrow[from=1-2, to=1-3]
	\arrow[from=1-2, to=2-2]
	\arrow[from=1-3, to=2-3]
	\arrow[from=2-1, to=2-2]
	\arrow[from=2-2, to=2-3]
\end{tikzcd}\]
of fiber bundle sequences, and one can formally complete the top row into $\overline{G/U}_\alpha\to G/[P_\alpha,P_\alpha]$ to be an $\mathbb A^2$-bundle. Notably, passing from $G/U$ to $\overline{G/U}_\alpha$ has merely added things in codimension $2$, so the global sections do not move. But now we are in a vector bundle situation, so get a Fourier transform $\Phi_{s_\alpha}$ on $\Gamma(\overline{G/U}_\alpha)$.
\begin{lemma}
	Fix everything as above.
	\begin{listalph}
		\item We have $\Phi_{s_\alpha}(x)=x$ for $x\in\mf g$.
		\item The operator $\Phi_{s_\alpha}$ acts on $\mf t$ by the $\cdot$-action of $s_\alpha$.
		\item The $\Phi_{s_\alpha}$s generate an action of $W$ on $\Gamma(\mc D_{G/U})$, denoted $\Phi_\bullet\colon W\to\op{End}\Gamma(\mc D_{G/U})$.
	\end{listalph}
\end{lemma}
\begin{lemma}
	Fix everything as above. Then
	\[\Gamma(\mc D_{G/U})^G=U\mf t.\]
\end{lemma}
\begin{proof}[Sketch]
	Set $D\coloneqq\Gamma(\mc D_{G/U})$ for brevity. Then $D$ embeds into $\op{End}\CC[G/U]$, where $\CC[G/U]=\bigoplus_\lambda V_\lambda$ as before. For $d\in D^G$, one finds that $d$ acts on $V_\lambda$ by some eigenvalue $c_\lambda$. In fact, because $d$ comes from a differential operator, one can check that the mapping $\lambda\mapsto c_\lambda$ is a polynomial function, which means that it comes from $U\mf t$.
\end{proof}
We are now ready to construct our elements $e$ analogous to the $\PP^n$ case: for $V_\lambda\subseteq\CC[G/U]$, there is some weight $\lambda'$ for which $V_\lambda^\lor=V_{\lambda'}$, and then we define
\[e_\lambda\coloneqq\sum_if_i\Phi_{w_0}(f^i),\]
where $\{f_i\}$ is a basis of $V_\lambda$ and $\{f^i\}$ is the dual basis of $V_{\lambda'}$. (Here, $w_0\in W$ is the long element.)
\begin{lemma}
	Fix everything as above. Then
	\[e_\lambda=\prod_\alpha\prod_{i=1}^{\langle\alpha^\lor,\lambda\rangle}(\alpha^\lor+\langle\alpha^\lor,\rho\rangle-1).\]
\end{lemma}
One can now prove that $\Gamma$ is conservative as in \Cref{prop:localize-pn}. Namely, for nonzero $M$, one knows that $\pi^*M$ is nonzero on a quasi-affine scheme, so it has some nonzero global sections, so $\Gamma(\pi^*M)_\lambda$ is nonzero for some dominant weight $\lambda\in\Lambda^+$. Fixing $\lambda'$ as before, one is able to use $e_{\lambda'}$ to bring the weight of our nonzero vector $v$ down: one has $e_{\lambda'}v\ne0$, so $\Phi_{w_0}(f^i)v\ne0$ for some $i$.

The proof of exactness uses a similar trick, thereby completing a sketch for \Cref{thm:localize-global-section}, and the proof of \Cref{thm:twist-local}(a) is similar. We close the course with some indications of applications.
\begin{remark}
	It turns out that the inverse equivalence of
	\[\Gamma\colon\mathrm{Mod}(\mc D_{G/B})\to\mathrm{Mod}(U_0\mf g)\]
	is given by $\mc L\colon M\mapsto\mc D_{G/B}\otimes_{U_0\mf g}M$. As an example application, the left-hand side allows us to take a fiber over a point $x\in G/B$; it turns out that there is a corresponding $\mc D$-module $\delta_x$ for which $\Gamma(\delta_x)$ is a Verma module. Now, for some $U_0\mf g$-module $M$, one has
	\[\mc L(M)_x=\delta_x\otimes_{\mc D}\mc D\otimes_{U_0\mf g}M,\]
	which is the tensor product of a Verma module with $M$. In particular, the weight spaces are slightly more controlled.
\end{remark}
\begin{remark}
	Let's explain how one might compute the characters of the discrete series. Fix a real group $G$ with maximal compact $K$, and we suppose that $\op{rank}G=\op{rank}K$. Further, fix a discrete series representation $L$. Then it turns out that $\op{supp}\mc L_\lambda(L)$ is a closed $K$-orbit in $G/B$, so the stabilizer is basically a Borel in $K$, meaning that $\op{supp}\mc L_\lambda(L)$ is basically a copy of the flag variety. One eventually finds that $\op{gr}\mc L_\lambda(L)$ is a line bundle induced from $U\mf t$ living on the conormal bundle of the orbit.
\end{remark}
\begin{remark}
	Let's say something about Kazhdan--Lusztig theory. We are interested in computing an isomorphism $\ZZ[W]\cong K_0(\OO_0)$, which will eventually be done via understanding a natural action of $W$ on $K_0(\OO_0)$. This action arises from an action of the braid group on the derived category $D^b(\OO_0)$. There are many ways to describe this action.
	\begin{itemize}
		\item One can have the braid group act by correspondences along $(G/B)\from(G/B)^2_w\to(G/B)$, where $(G/B)^2_w$ is the $G$-orbit of $w$.
		\item One can describe this action by a tensor product with Harish-Chandra bimodules.
		\item There is also a description with localization by tracking along a composite
		\[D^b(U_0\mf g)\cong D^b(\mc D_0)\stackrel{-\otimes\OO(w\cdot0)}\cong D^b(\mc D_{w\cdot0})\cong D^b(U_0).\]
	\end{itemize}
\end{remark}

\end{document}