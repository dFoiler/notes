% !TEX root = ../notes.tex

\documentclass[../notes.tex]{subfiles}

\begin{document}

\section{December 4}
Today we will start proving localization.

\subsection{Twisted Differential Operators}
We may want a localization theorem for general $U_\lambda\mf g$ instead of merely $U_0\mf g$. To this end, we introduce twisted differential operators.
\begin{definition}[twisted differential operator]
	Fix a line bundle $\mc L$ on a smooth variety $X$. Then we define the sheaf $\mc D_{\mc L}$ of \textit{twisted differential operators} on $\mc L$ as follows. On a trivializing open cover $\{V_i\}$ of $X$ for $\mc L$, we define
	\[\mc D|_{\mc L}|_{V_i}=\mc D|_{V_i},\]
	with compatibility maps $\mc D|_{\mc L}|_{V_i}|_{V_{ij}}\to\mc D|_{\mc L}|_{V_j}|_{V_{ij}}$ given by $d\mapsto f_{ij}^{-1}df_{ij}$, where $f_{ij}\in\OO_{V_{ij}}^\times$ is the unit from $\mc L|_{V_i}|_{V_{ij}}\to\mc L|_{V_j}|_{V_{ij}}$.
\end{definition}
\begin{example}
	The given map $d\mapsto f_{ij}^{-1}df_{ij}$ acts by identity on the sections of $\mc L$. It sends a vector field $v$ to $v+\langle v,df_{ij}/f_{ij}\rangle$.
\end{example}
\begin{remark}
	If $\omega$ is a closed $1$-form, then one can check directly that the maps
	\[\begin{cases}
		f\mapsto f & \text{for }f\in\OO, \\
		v\mapsto v+\langle v,\omega\rangle & \text{for }v\in\op{Vect}X,
	\end{cases}\]
	extends to a unique automorphism of $\mc D$.
\end{remark}
\begin{definition}[twisted differential operator]
	Fix a line bundle $\mc L$ on a smooth variety $X$. For a scalar $c$, we define the sheaf $\mc D_{\mc L^c}$ of \textit{twisted differential operators} on $\mc L$ as before but using the automorphism
	\[\begin{cases}
		f\mapsto f & \text{for }f\in\OO, \\
		v\mapsto v+c\langle v,df_{ij}/f_{ij}\rangle & \text{for }v\in\op{Vect}X,
	\end{cases}\]
	between open subsets of a cover.
\end{definition}
\begin{remark}
	There is an equivalence of categories $\mathrm{Mod}(\mc D)\to\mathrm{Mod}(\mc D_{\mc L})$ given by $\mc M\mapsto\mc M\otimes_\OO\mc L$, and in general, this mapping defines an equivalence
	\[\mathrm{Mod}(\mc D_{\mc L^c})\to\mathrm{Mod}(\mc D_{\mc L^{c+1}}).\]
	Accordingly, we didn't get anything new from $\mc D_{\mc L}$, but the freedom of the extra scalar $c$ may give us new modules.
\end{remark}
Let's explain how we can rid of the line bundle $\mc L$ with some geometry. Taking the total space of the line bundle produces a principal $\mathbb G_m$-bundle $\pi\colon\widetilde X\to X$. Then there is an equivalence
\[\mathrm{Mod}(\mc D_X)\simeq\mathrm{Mod}^{\CC^\times}(\mc D_{\widetilde X})\]
given by $\mc M\mapsto\pi^*\mc M$. This follows from some functoriality of $\mc D$-modules; the inverse functor is given by $\widetilde{\mc M}\mapsto(\pi_*\widetilde{\mc M})^{\mathbb G_m}$.

More generally, one finds that
\[\mathrm{Mod}(\mc D_{X,\mc L^c})\simeq\mathrm{Mod}^{\CC^\times,c}(\mc D_{\widetilde X}),\]
where the right-hand side consists of weakly $\CC^\times$-equivariant $\mc D_{\widetilde X}$-modules with some specified log monodromy $c$. In other words, for $\mc M\in\mathrm{Mod}(\mc D_{X,\mc L^c})$, we note $\pi^*\mc M$ is still a $\CC^\times$-equivariant quasicoherent sheaf, and it admits a compatible $\mc D_{\widetilde X}$-structure, but the two actions from $t\del_t\in\op{Lie}\mathbb G_m$ will disagree by $c$.
\begin{example}
	With $X$ as a point and $\mc M=0$, then we receive a vector bundle with flat connection $\nabla=d-c\,dt/t$.
\end{example}

\subsection{Twisted Localization}
In our setting of interest, we consider the quotient $G/U\onto G/B$ as a principal $T$-bundle. Because $T$ is some power of $\mathbb G_m$, we may apply the above framework.
\begin{theorem}[Localization] \label{thm:twist-local}
	Fix a complex reductive group $G$, and consider the $T$-bundle $G/U\onto G/B$. For $\lambda\in\mf t^*$, we consider the ring sheaf of twisted differential operators $\mc D_{G/B,\lambda}$.
	\begin{listalph}
		\item For all weights $\lambda$, taking global sections induces a ring isomorphism $U_{\ov\lambda}\mf g\to\Gamma(\mc D_{G/B,\lambda})$, where $\ov\lambda$ is any weight in the same $(W,\cdot)$-orbit as $\lambda$.
		\item If $\langle\lambda+\rho,\alpha^\lor\rangle\notin\ZZ_{\le0}$ for all positive coroots $\alpha^\lor$, then taking global sections produces an equivalence
		\[\Gamma_\lambda\colon\mathrm{Mod}(\mc D_{G/B,\lambda})\to\mathrm{Mod}(U_\lambda\mf g).\]
	\end{listalph}
\end{theorem}
\begin{proof}[Sketch]
	For (a), one notes that the Springer resolution $\pi\colon T^*(G/B)\to\mc N$ sends the structure sheaf to the structure sheaf. Working sufficiently locally basically allows one to ignore $\lambda$.

	We now move on to (b). There are two main things to check: we need to check that $\Gamma_\lambda$ is exact and sends nonzero modules to nonzero modules. (Exactness of $\Gamma_\lambda$ requires the hypothesis on $\lambda$.) From here, one notes that $\mc D_\lambda$ is a projective generator of $\mathrm{Mod}(\mc D_\lambda)$, and
	\[\Gamma_\lambda(\mc M)=\mathrm{Hom}_{\mc D_\lambda}(\mc D_\lambda,\mc M),\]
	so abstract nonsense tells us that $\Gamma_\lambda$ is an equivalence to the category of $\op{End}_{\mc D_\lambda}(\mc D_\lambda)\opp$-modules, but (a) has shown that $\op{End}_{\mc D_\lambda}(\mc D_\lambda)\opp=\Gamma_\lambda(\mc D_\lambda)=U_\lambda\mf g$.
\end{proof}
One should think of the hypothesis in (b) as a genericity hypothesis.
\begin{definition}[regular]
	Fix a complex Lie algebra $\mf g$. A weight $\lambda$ is \textit{regular} if and only if the stabilizer of the $(W,\cdot)$-action is trivial.
\end{definition}
\begin{remark}
	For any regular $\lambda$, we will be able to show that some weight in the orbit $W\cdot\lambda$ makes the conclusion of (b) in \Cref{thm:twist-local} is true.
\end{remark}
\begin{corollary}
	Fix regular central characters $\ov\lambda$ and $\ov\mu$ for which $\lambda$ and $\mu$ are dominant weights. If $\mu-\lambda$ is integral, then there is an equivalence of categories
	\[\mathrm{Mod}(U_{\ov\lambda}\mf g)\cong\mathrm{Mod}(U_{\ov\mu}\mf g).\]
\end{corollary}
\begin{proof}
	Pass to $\mc D$-modules via \Cref{thm:twist-local}.
\end{proof}
\begin{remark}
	There are also algebraic proofs which use translation functors.
\end{remark}
\begin{remark}
	A version of this corollary is true if merely $\lambda$ is regular.
\end{remark}
Let's now give some applications.
\begin{corollary}
	Fix a complex reductive group $G$. For any regular $\lambda$, taking global sections produces an equivalence
	\[\mathrm{Mod}^K(\mc D_{G/B})\to\mathrm{Mod}(\mf g,K)_\lambda,\]
	where the right-hand side consists of those $(\mf g,K)$-modules with infinitesimal character $\lambda$.
\end{corollary}
Here is an application to category $\OO_{\ov\lambda}$, which consists of those $U_{\ov\lambda}\mf g$-modules on which $\mf n$ acts locally nilpotently. Here, $N$ is the unipotent radical of $B$.
\begin{corollary}
	Fix a complex reductive group $G$. For any regular $\lambda$, taking global sections produces an equivalence
	\[\mathrm{Mod}^N(\mc D_{G/B,\lambda})\cong\OO_{\ov\lambda}.\]
\end{corollary}
\begin{corollary}
	Let $G$ be a complex reductive group, which we consider as a real reductive group by restricting scalars. Then the category of Harish-Chandra $(\mf g\oplus\mf g,G)$-bimodules with trivial central characters is equivalent to the subcategory of $\OO_0$ generated by those Verma modules where $\mf t$ acts diagonally.
\end{corollary}
\begin{proof}
	We see that $G_\CC=G\times G$, so localization tells us to consider $G$-equivariant $\mc D_{G/B\times G/B}$-modules. Because $(G/B\times G/B)/G=B\backslash G/B$, this is the same as considering $B$-equivariant $\mc D_{G/B}$-modules. This turns out to be equivalent to the subcategory of $\OO_0$ consisting of those Verma modules where $\mf t$ acts diagonally.
\end{proof}
\begin{remark}
	One can find that this equivalence is given by $M\mapsto M\otimes_{U\mf g}M_0$ for some $M_0$.
\end{remark}
\begin{corollary}
	Fix a complex reductive group $G$. Then the irreducible $(\mf g,K)$-modules with trivial central character are in bijection with pairs $(O,\psi)$, where $O$ is a $K$-orbit in $G/B$, and $\psi$ is an irreducible representation for $\pi_0(\op{Stab}(x))$, where $x\in O$.
\end{corollary}
\begin{remark}
	One may call such a $\psi$ an ``equivariant local system'' because it induces a representation of $\pi_1(K/\op{Stab}(x))$ via the connecting homomorphism
	\[\pi_1(K/\op{Stab}(x))\to\pi_0(\op{Stab}(x)).\]
\end{remark}
\begin{remark}
	Let's explain how to compute this stabilizer. Let $\theta$ be the Cartan involution. Then the stabilizer is $G^\theta\cap B_x=(B_x\cap B_x^\theta)^\theta$. The unipotent parts will not affect the connected component of the stabilizer, so we let $T_x$ be the torus of this subgroup, and we see that we are interested in the connected components of $T_x^\theta$.
\end{remark}
\begin{remark}
	One can use this corollary to exhibit a bijection between the irreducible $(\mf g,K)$-modules with trivial central character and $W=B\backslash G/B$.
\end{remark}
Now that we have enumerated our irreducibles by $W$, we may be interested in giving some character formulae. Let's give a rough sketch for how this is done. Recall that the character $M$ of a module in $\OO_0$ is given by
\[\op{ch}M=\sum_{\lambda\in\mf t^*}\dim M[\lambda]\cdot e^\lambda.\]
The moral is that there are some ``(co)standard'' objects whose characters are easy to compute: for $\OO_0$, one uses the (dual) Verma modules $M_\lambda$, which have
\[\op{ch}M_\lambda=e^\lambda\prod_{\alpha\in\Phi^+}\frac1{1-e^{-\alpha}}.\]
Then one can find explicit resolutions of our irreducibles using these representations, which produces our characters. Roughly speaking, one finds that the $K_0(\OO_0)$ admits two bases.
\begin{itemize}
	\item One can use Verma modules to produce a ring isomorphism $\ZZ[W]\to K_0(\OO_0)$ given by $w\mapsto M_{w\cdot0}$.
	\item One can use irreducibles to produce other generators of $K_0(\OO_0)$, known as the Kazhdan--Lusztig basis.
\end{itemize}
Thus, we are reduced to a linear algebra problem, trying to pass from one basis to the other. This is a difficult result!
\begin{remark}
	There is a similar story for $(\mf g,K)$-modules, due to Lusztig and Vogan. In fact, there is a uniform definition of ``costandard.'' In short, they consist of some extensions of equivariant bundles with flat connection from an orbit.
\end{remark}

\end{document}