% !TEX root = ../notes.tex

\documentclass[../notes.tex]{subfiles}

\begin{document}

\section{December 2}
Today we begin talking about localization.

\subsection{Introducing \texorpdfstring{$\mc D$}{ D}-Modules}
For motivation, recall that irreducible admissible $(\mf g,K)$-modules admit an infinitesimal character $\chi_\lambda$ (for some $\lambda\in\mf t^*/(W,\cdot)$, indexed via the Harish-Chandra isomorphism), so we may in general attempt to understand modules over
\[U_\lambda\mf g\coloneqq U\mf g\otimes_{Z\mf g}\CC_\lambda.\]
Our study of the nilpotent cone showed that $\op{gr}U_\lambda\mf g=\CC[\mc N]$, where $\mc N\subseteq\mf g^*$ is the nilpotent cone. Notably, $\mc N$ is a Poisson variety, so we may view $U_\lambda\mf g$ as a quantization: the Poisson bracket $\{\cdot,\cdot\}$ of $\CC[\mc N]$ turns out to be given by
\[\{\varphi,\psi\}=[\widetilde\varphi,\widetilde\psi]\pmod{U_\lambda^{<d+e-1}},\]
where $\widetilde\varphi$ and $\widetilde\psi$ are lifts to $U_\lambda\mf g$ of degrees $d$ and $e$, respectively.

The structure of $\mc N$ is fairly complicated, but it can be understood via geometry. In particular, $\widetilde N$ admits a resolution
\[\pi\colon T^*(G/B)\to\mc N.\]
To define this map, recall $T^*(G/B)$ can be identified with pairs $(x,B)$, where $B\in G/B$ and $x\in\op{rad}\op{Lie}B$; as such, $\pi$ is given by $(x,B)\mapsto x$. This resolution $\pi$ is indeed birational and proper, and $\pi$ preserves the Poisson structures present. (Namely, $T^*(G/B)$ is symplectic, which gives rise to a nice Poisson structure.)

We would now like to quantize the previous paragraph to hopefully understand $U_\lambda\mf g$. In general, a cotangent bundle $T^*X$ for a smooth algebraic variety $X$ can be quantized by considering the (algebraic) differential operators on $X$.
\begin{definition}[differential operators]
	Fix a smooth algebraic variety $X$. Then we define $\mc D_M$ to be the ring sheaf of differential operators on $X$. Explicitly, on local coordinates $(x_1,\ldots,x_n)$, the sections look like
	\[\sum_If_Id_{x_I}^{I},\]
	where $I=(i_1,\ldots,i_n)$ is some sequence and $f_I\in\OO$.
\end{definition}
% Namely, let $\mc D_M$ be the ring sheaf of differential operators on $X$, which 
\begin{remark}
	One can alternatively construct $\mc D_X$ as embedded in the sheaf $\op{End}\OO$ as admitting a filtration $\bigcup_{i\ge0}\mc D_X^{\le i}$, where $\mc D_X^{\le0}=\mc O$ and $d\in\mc D_X^{\le i}$ if and only if $[d,f]\in\mc D^{\le i-1}$ for all $f\in\mc O$.
\end{remark}
\begin{remark}
	In characteristic $0$, it turns out that $\mc D_X$ is generated by $\OO$ and the sheaf of vector fields. As such, one can view $\mc D_X$ as a sort of universal enveloping algebra from the differentials.
\end{remark}
\begin{example}
	One can check that
	\[\mc D_{\AA^n}=\frac{\CC\langle x_1,\ldots,x_n,d_1,\ldots,d_n\rangle}{([d_i,x_j]=1_{i=j},[x_i,x_j]=0,[d_i,d_j]=0)_{ij}}.\]
	Here, the numerator is the free non-commutative ring on the $x_\bullet$s and $d_\bullet$s.
\end{example}
It now turns out that $\op{gr}\mc D=\OO_{T^*X}$, so we may indeed view $\op{gr}\mc D$ as a quantization of $T^*X$.

We are thus motivated to study $\mc D$-modules.
\begin{defihelper}[$\mc D$-module] \nirindex{D-module@$\mc D$-module}
	Fix a smooth algebraic variety $X$. Then a (left) \textit{$\mc D$-module} is a sheaf of (left) modules over $\mc D_X$. The category $\mathrm{Mod}(\mc D_X)$.
\end{defihelper}
\begin{example}
	Fix an algebraic vector bundle $(\mc E,\nabla)$ on $X$ with a flat connection. Then $\mc E$ upgrades into a $\mc D$-module by having a vector field $z$ act by
	\[z\colon\sigma\mapsto\nabla_z(\sigma).\]
\end{example}
\begin{example}
	Given an open embedding $j\colon U\into X$, there is a functor $j_*\colon\mathrm{Mod}(\mc D_U)\to\mathrm{Mod}(\mc D_X)$ defined as expected.
\end{example}

\subsection{Holonomic \texorpdfstring{$\mc D$}{ D}-Modules}
As with Harish-Chandra bimodules, we have a notion of support.
\begin{definition}[singular support]
	Fix a smooth algebraic variety $X$, and let $\mc M$ be a coherent $\mc D$-module, meaning that $\mc M(U)$ is finitely generated over $\mc D(U)$ on each affine $U\subseteq X$. A \textit{good filtration} for $\mc M$ is an ascending filtration $\{\mc M_{\le i}\}_{i\in\ZZ}$ satisfying the following.
	\begin{itemize}
		\item Exhaustive: the filtration $\{\mc M_{\le i}\}_{i\in\ZZ}$ is exhaustive and satisfies $\bigcap_{i}\mc M_{\le i}=0$.
		\item Compatible with $\mc D$: $\mc D_{\le i}\mc M_{\le j}\subseteq\mc M_{\le i+j}$ for all $i$ and $j$.
		\item Finite type: $\op{gr}\mc M$ is finitely generated over $\op{gr}\mc D$.
	\end{itemize}
	Then the \textit{singular support} of $\mc M$ is $\op{supp}\op{gr}\mc M$.
\end{definition}
\begin{example}
	Here is one way to construct good filtrations: start with a finitely generated $\OO$-sub\-module $V\subseteq\mc M$ of generators for $\mc M$ over $\mc D$, and then simply define
	\[\mc M_{\le i}\coloneqq\mc D_{\le i}V.\]
\end{example}
\begin{remark}
	Even though $\op{gr}\mc M$ depends on the choice of (good) filtration, it turns out that the (set-theoretic) singular support does not. (Even the scheme-theoretic support may depend on the filtration!)
\end{remark}
It turns out that the support is always pretty large.
\begin{proposition}[Bernstein's inequality]
	Fix a smooth algebraic variety $X$. For any $\mc D$-module $\mc M$, we have
	\[\dim\op{supp}\mc M\ge\dim X.\]
\end{proposition}
\begin{example}
	For example, on $X=T^*\AA^1$, so it suffices to check that $\dim\op{supp}\mc M>0$. However, $\dim\op{supp}\mc M=0$ means that $\mc M$ is given by some finite-dimensional module over
	\[\mc D=\frac{\CC\langle x,\del\rangle}{([\del,x]=1)}.\]
	But then $\tr{\id}=\dim\mc M$ must vanish!
\end{example}
The modules with small support turn out to be interesting.
\begin{definition}[holonomic]
	Fix a $\mc D$-module $\mc M$ on a smooth algebraic variety $M$. Then $\mc M$ is \textit{holonomic} if and only if
	\[\dim\op{supp}\mc M=\dim M.\]
\end{definition}
\begin{example}
	For $(\mc E,\nabla)$ on some $X$, because $\mc E$ is a vector bundle, one can choose the filtration so that $\op{gr}\mc E$ is supported in degree $0$. It follows that the singular support is just the zero section of $X\to T^*X$, so $(\mc E,\nabla)$ is holonomic!
\end{example}
\begin{theorem}
	Fix a smooth algebraic variety $M$. Then the irreducible holonomic $\mc D$-modules on $X$ are in bijection with equivalence classes of pairs $(Y,(\mc E,\nabla))$, where $Y\subseteq X$ is a smooth irreducible locally closed subvariety, and $(\mc E,\nabla)$ is an irreducible vector bundle with flat connection on $Y$.
\end{theorem}
\begin{proof}
	We will say nothing interesting about the proof, but we do say something about how this construction goes. In the case where $Y\subseteq X$ is open, one can consider the $\mc D$-module $j_*(\mc E,\nabla)$, where $j\colon Y\into X$ is the inclusion. One then shows that $j_*(\mc E,\nabla)$ admits a unique irreducible subquotient.

	In general, one has a chain $Y\subseteq U\subseteq X$, where $i\colon Y\into U$ is closed, and $j\colon U\into X$ is open. It turns out that $(\mc E,\nabla)$ can be extended to a $\mc D$-module on $U$ which is supported on $Y$ (via some $i_*$ construction). Then we apply $j_*$ and find the unique irreducible submodule.
\end{proof}

\subsection{The Localization Theorem}
We can now return to our motivation.
\begin{theorem}[Localization]
	Fix a complex reductive group $G$ with Borel subgroup $B$. Then taking global sections produces an equivalence
	\[\Gamma\colon\mathrm{Mod}(\mc D_{G/B})\to\mathrm{Mod}(U_0\mf g),\]
	where the $0$ denotes the infinitesimal character of the trivial representation. In other words, $U_0\mf g=U\mf g/Z\mf g$.
\end{theorem}
\begin{remark}
	There is also a generalization for $U_\lambda\mf g$, but we need to modify our $\mf g$-modules slightly.
\end{remark}
The moral is that $\mc D$-modules do in fact provide the correct quantization of $T^*(G/B)$, in the same way that $U_\lambda\mf g$ is a quantization for $\mc N$. The above theorem follows from the following two theorems.
\begin{theorem} \label{thm:localize-global-section}
	Fix a complex reductive group $G$ with parabolic subgroup $P$. Then taking global sections produces an equivalence
	\[\Gamma\colon\mathrm{Mod}(\mc D_{G/P})\to\mathrm{Mod}(\Gamma(\mc D_{G/P})).\]
\end{theorem}
\begin{theorem}
	Fix a complex reductive group $G$ with Borel subgroup $B$. Then there is a ring isomorphism
	\[U_0\mf g\cong\Gamma(\mc D_{G/B}).\]
\end{theorem}
\begin{remark}
	Roughly speaking, the map $\mc U_0\mf g\to\Gamma(\mc D_{G/B})$ is a quantization of the springer resolution $T^*(G/B)\to\mc N$.
\end{remark}
\begin{example}
	Take $G=\op{SL}(2)$ so that $G/B=\PP^1$. Then there is a unique (up to isomorphism) $\mc D$-module $\delta_0$ supported at $0$. In short, $\delta_0$ has basis given by $\{t^{-n}\}_{n>0}$, where $t\del_t$ is an eigenvector with eigenvalue $-1$. From $\delta_0$, it turns out that one gets the irreducible Verma module $M_{-2}$; note that $M_{-2}$ has trivial central character, as explained by the Harish-Chandra isomorphism, so $M_{-2}$ is in fact a $U_0\mf g$-module.
\end{example}
To upgrade to $(\mf g,K)$-modules, it will be helpful to have groups act on our $\mc D$-modules.
\begin{definition}[equivariant]
	Fix an algebraic group $H$ acting a smooth variety $X$. Let $a\colon H\times X\to X$ be the action map. A sheaf $\mc F$ on $X$ is \textit{$H$-equivariant} if and only if there is an isomorphism $a^*\mc F\to\op{pr}_2^*\mc F$ satisfying a cocycle condition on $H\times H\times X$. An equivariant sheaf of modules is also required to commute with the module structure.
\end{definition}
\begin{remark}
	Alternatively, one may state this as having $H$ suitably acting on sections. Thus, we may say that an $H$-equivariant $\mc D$-module $\mc M$ is one in which the $H$-action on the sections respects the $\mc D$-module structure, and we require that the two actions of $\op{Lie}H$ coincide. Here, one actions of $\op{Lie}H$ is given by
	\[\op{Lie}H\to\op{Vect}X\subseteq\mc D_X\to\op{End}\mc M,\]
	and the other is given by differentiating the $H$-action on $\mc M$.
\end{remark}
\begin{example}
	If $X$ is a point, then an $H$-equivariant $\mc D$-module is a representation of $\pi_0H$ because the action of $\op{Lie}H$ must be trivial.
\end{example}
\begin{corollary}
	Fix a complex reductive group $G$. Then there is an equivalence
	\[\Gamma\colon\mathrm{Mod}^K(\mc D_{G/B})\to\mathrm{Mod}(\mf g,K)_0.\]
	Here, the left-hand side is the category of $K$-equivariant $\mc D$-modules on $G/B$. The right-hand side is the category of $(\mf g,K)$-modules with trivial central character.
\end{corollary}
\begin{example}
	Suppose $H$ acts on $X$ with finitely many orbits. (For example, $K$ acts on $G/B$ with finitely many orbits.) It turns out that irreducible $H$-equivariant $\mc D$-modules are in bijection with pairs $(\OO,\psi)$ where $\OO$ is an orbit $Hx$, and $\psi$ is a representation of $\pi_0(\op{Stab}(x))$. This allows us to classify irreducibles in $\mathrm{Mod}(\mf g,K)_0$.
\end{example}

\end{document}