% !TEX root = notes.tex

\documentclass[notes.tex]{subfiles}

\begin{document}

\chapter{Functional Analysis} \label{chap:functional}
In this appendix, we introduce the small amount of functional analysis we will need in order to get going with infinite-dimensional vector spaces. In other words, we need to set up the theory of Fr\'echet spaces. Throughout this appendix, $\FF$ denotes one of the fields $\RR$ or $\CC$. Our exposition is largely stolen from \cite{conway-functional}.

\section{Locally Convex Spaces}
We begin with the following definition.
\begin{definition}[topological vector space]
	Fix a topological field $k$. Then a \textit{topological vector space} is a vector space $V$ over $k$ equipped with a topology so that addition map $+\colon V\times V\to V$ and scalar multiplication map $\cdot\colon k\times V\to V$ are both continuous. In these notes, all topological vector spaces will be assumed to be Hausdorff.
\end{definition}
A Fr\'echet space will be a complete topological vector space admitting two notable definitions: having its topology is defined by a countable family of seminorms, or being locally convex and metrizable. As such, let's quickly recall the definition of a seminorm.
\begin{definition}[seminorm]
	Fix a vector space $V$ over $\FF$. Then a \textit{seminorm} is a function $p\colon V\to\RR$ satisfying the following.
	\begin{itemize}
		\item Subadditive: we have $p(x+y)\le p(x)+p(y)$ for any $x,y\in V$.
		\item Homogeneous: we have $p(\lambda x)=\left|\lambda\right|p(x)$ for any $x\in V$ and $\lambda\in\FF$.
	\end{itemize}
\end{definition}
\begin{remark}
	The homogeneity implies that $p(0)=0$, which is sometimes included in the definition. Similarly, the subadditivity now implies that $2p(x)=p(x)+p(-x)$ is at least $p(0)=0$, so $p$ is automatically nonnegative; this is also sometimes included in the definition.
\end{remark}
It is worthwhile to have a few ways to check continuity.
\begin{lemma} \label{lem:seminorm-continuity-grab-bag}
	Fix a topological vector space $V$ over $\FF$, and let $p\colon V\to\RR$ be a seminorm. Then the following are equivalent.
	\begin{listroman}
		\item $p$ is continuous.
		\item $\{v:p(v)<1\}$ is open.
		\item $p$ is continuous at $0$.
	\end{listroman}
\end{lemma}
\begin{proof}
	This is \cite[Proposition~1.3]{conway-functional}. Of course (i) implies (ii). We show the remaining implications independently.
	\begin{itemize}
		\item We show (ii) implies (iii). For any net $\{x_i\}$ converging to $0$, we want to show that $\{p(x_i)\}\to0$, which is true because any $x_i$ in the open neighborhood $\varepsilon U$ of $0$ has $p(x_i)<\varepsilon$.
		\item We show (iii) implies (i). Note that any net $\{x_i\}$ converging to some $x$ has
		\[\left|p(x_i)-p(x)\right|\le p(x_i-x),\]
		and $p(x_i-x)\to0$ because $x_i-x\to0$.
		\qedhere
	\end{itemize}
\end{proof}
We now start talking about convex sets, but we will relate our definitions back to seminorms.
\begin{definition}[convex]
	Fix a vector space $V$ over $\FF$. A subset $A\subseteq V$ is \textit{convex} if and only if any two $a,b\in A$ has
	\[ta+(1-t)b\in A\]
	for any $t\in[0,1]$.
\end{definition}
\begin{example} \label{ex:seminorm-to-convex}
	Let $p\colon V\to\RR$ be a seminorm. Then we claim that $A\coloneqq\left\{v\in V:p(v)<1\right\}$ is convex. Indeed, for $a,b\in A$ and $t\in[0,1]$, we see that
	\[p(ta+(1-t)b)=tp(a)+(1-t)p(b),\]
	which is still less than $1$, so $ta+(1-t)b\in A$.
\end{example}
\begin{example}[convex hull] \label{ex:convex-hull}
	For any subset $A\subseteq V$, we may define the convex hull
	\[\op{conv}(A)\coloneqq\left\{\sum_{i=1}^nt_ia_i:\{a_i\}_i\subseteq A,\{t_i\}_i\in[0,1],t_1+\cdots+t_n=1\right\}.\]
	Note that $\op{conv}(A)$ is convex: for two points $\sum_it_ia_i$ and $\sum_js_jb_j$ and $t\in[0,1]$, the sum $\sum_itt_ia_i+\sum_j(1-t)s_jb_j$ still has $\sum_itt_i+\sum_j(1-t)s_j=t+(1-t)=1$. In fact, if $B$ is convex and contains $A$, then $\op{conv}(A)\subseteq B$ because the sums $\sum_it_ia_i$ can be checked to be in $B$ by induction.
\end{example}
Convex sets on their own turn out to not be good enough for our purposes, so we will need extra adjectives.
\begin{definition}[balanced]
	Fix a vector space $V$ over $\FF$. A subset $A\subseteq V$ is \textit{balanced} if and only if $\lambda A\subseteq A$ for all $\lambda\in\FF$ such that $\left|\lambda\right|\le1$.
\end{definition}
\begin{example} \label{ex:seminorm-to-balanced}
	Let $p\colon V\to\RR$ be a seminorm. Then we claim that $A\coloneqq\left\{v\in V:p(v)<1\right\}$ is balanced. Indeed, for $a\in A$ and $\lambda$ with $\left|\lambda\right|\le1$, we see that $p(\lambda a)=\left|\lambda\right|p(a)$, which is still less than $1$, so $\lambda a\in A$.
\end{example}
\begin{example} \label{ex:balanced-hull}
	For any subset $A\subseteq V$, the subset
	\[\op{bal}(A)\coloneqq\bigcup_{\left|\lambda\right|\le1}\lambda A\]
	is balanced. Indeed, for any $\mu$ with $\left|\mu\right|\le1$, we see that $\mu\op{bal}(A)=\bigcup_\lambda\mu\lambda A$ is contained in $\op{bal}(A)$ because $\left|\lambda\mu\right|\le1$ whenever $\left|\lambda\right|\le1$. Of course, we always have $A\subseteq\op{bal}(A)$, and one can see that any balanced set containing $A$ must contain each $\lambda A$ and hence contain $\op{bal}(A)$.
\end{example}
It turns out that passing to balanced convex sets is not too big of a burden.
\begin{lemma} \label{lem:convex-to-convex-balanced}
	Fix a topological vector space $V$ over $\FF$. Any convex open neighborhood of $0$ contains a balanced convex open neighborhood of $0$.
\end{lemma}
\begin{proof}
	Let $U$ be a convex open neighborhood of $0$. The point is to use the continuity of scalar multiplication: the continuity of
	\[\cdot\colon\FF\times V\to V\]
	provides a basic open neighborhood $B(0,\varepsilon)\times U'$ of $(0,0)$ of $\FF\times V$ such that $B(0,\varepsilon)U'\subseteq U$. We claim that $\op{conv}(B(0,\varepsilon)U')$ is the desired open neighborhood. Here are our checks; set $U''\coloneqq B(0,\varepsilon)U'$ for brevity.
	\begin{itemize}
		\item Because $U$ is already convex, \Cref{ex:convex-hull} explains that $U''\subseteq U$ implies that $\op{conv}(U'')\subseteq U$.
		\item Convex: note $\op{conv}(U'')$ is convex by \Cref{ex:convex-hull}.
		\item Open: because scalar multiplication by a nonzero number is a homeomorphism, we see that $U''\coloneqq B(0,\varepsilon)U'$, which is
		\[\bigcup_{0<\left|\lambda\right|<\varepsilon}\lambda U',\]
		is open. Then once $U'$ is open, we see that $\op{conv}(U')$ can be written as a union
		\[\bigcup_{n\ge1}\Bigg(\bigcup_{t_1+\cdots+t_n=1}t_1U''+\cdots+t_nU''\Bigg),\]
		which is open because the sum of two open subsets is open (indeed, the sum of open sets is a union of translates of just one of the open sets).
		\item Balanced: the previous step realized $U''$ as a union $\bigcup_{0<\left|\lambda\right|<\varepsilon}\lambda U'$, which can be shown to be balanced exactly as in \Cref{ex:balanced-hull}.
		\qedhere
	\end{itemize}
\end{proof}
\begin{remark} \label{rem:balanced-convex-basis}
	Here is a sample application: suppose that $0$ has a neighborhood basis of convex sets. Then \Cref{lem:convex-to-convex-balanced} implies that $0$ also admits a neighborhood basis of balanced convex sets: for each $U$ in the neighborhood basis, the lemma produces a smaller open subset which is still convex but now also balanced. Similarly, admitting a countable neighborhood basis of convex sets can be upgraded to admitting a countable neighborhood basis of balanced convex sets.
\end{remark}
The previous remark allows us to make the following definition.
\begin{definition}[locally convex]
	A topological vector space $V$ is \textit{locally convex} if and only if $0$ admits a neighborhood basis of convex sets.
\end{definition}
\begin{remark} \label{rem:better-locally-convex}
	By translation, it is equivalent to require $V$ to have a basis of convex sets. (Namely, if $\mc U$ is the neighborhood basis of $0$, then $\bigcup_{v\in V}v+\mc U$ is the basis of $V$.) By \Cref{rem:balanced-convex-basis}, we can also upgrade these notions to having bases of balanced convex sets.
\end{remark}
It is notable that the previous two definitions avoid the mention of any topology. In order to continue not doing any topology, we pick up the following definition, which provides a linear algebraic stand-in for ``contains an open neighborhood of the origin.''
\begin{definition}[absorbing]
	Fix a vector space $V$ over $\FF$. A subset $A\subseteq V$ is \textit{absorbing} if and only if any $v\in V$ admits some $\varepsilon>0$ such that $tv\in A$ for all $t\in[0,\varepsilon)$.
\end{definition}
For example, we see that $0$ is contained in any absorbing subset.
\begin{remark}
	Of course, if $A$ is absorbing, and $A\subseteq B$, then $B$ is absorbing: for each $v$, the $\varepsilon$ which worked for $A$ continues to work for $B$.
\end{remark}
\begin{example} \label{ex:open-to-absorbing}
	Let's explain the remark given before the definition. If $V$ is a topological vector space over $\FF$, then we claim that any open neighborhood $U$ of $0$ is absorbing. This will follow by the continuity of scalar multiplication: for any $v\in V$, the map $\RR\to V$ given by $t\mapsto tv$ is continuous. Thus, because $0\in U$, there must be $\varepsilon>0$ such that $(-\varepsilon,\varepsilon)v\subseteq U$.
\end{example}
\begin{example} \label{ex:seminorm-to-absorbing}
	Let $p\colon V\to\RR$ be a seminorm, and set $A\coloneqq\{v\in V:p(v)<1\}$. Then we claim that $(-a)+A$ is absorbing for all $a\in A$; for example, setting $a=0$ will imply that $A$ is absorbing. Now, for any $v\in V$, we need to show that $a+tv\in A$ for small $t$. Well, $p(a+tv)\le p(a)+\left|t\right|p(v)$, so taking any $t$ with $p(v)\left|t\right|<(1-p(a))$ will do. (In particular, any $t$ will work if $p(v)=0$.)
\end{example}
We are now ready to construct some seminorms.
\begin{notation}
	Fix a vector space $V$ over $\FF$. For any absorbing subset $A\subseteq V$, we define $\norm\cdot_A\colon V\to\RR_{\ge0}$ by
	\[\norm v_A=\inf\left\{t\ge0:v\in tA\right\}.\]
\end{notation}
Here are some basic facts about this construction.
\begin{remark} \label{rem:mink-seminorm-defined}
	Because $A$ is absorbing, we see that any $v\in V$ does in fact have some $t>0$ for which $(1/t)v\in A$ and hence $v\in tA$, so the infimum is a real number.
\end{remark}
\begin{remark} \label{rem:better-mink-seminorm}
	Suppose further that $A$ is convex. Then we claim that $v\in tA$ whenever $t>\norm v_A$; note if $v=0$, then $\norm 0_A=0$, so there is nothing to do. The main point is to note that $sv\in A$ implies that $s'v\in A$ for any $s'\in[0,s]$ by convexity. Thus, if $t>\norm v_A$, then we know there is $s<t$ such that $v\in sA$, so $(1/s)v\in A$ while $1/t<1/s$, so $(1/t)v\in A$, so $v\in tA$.
\end{remark}
Let's put all our adjectives together, finally explaining the relationship between seminorms and convex sets.
\begin{proposition} \label{prop:convex-to-seminorm}
	Fix a vector space $V$ over $\FF$ and a subset $A\subseteq V$.
	\begin{listalph}
		\item There is a seminorm $p\colon V\to\RR$ such that $A=\{v\in V:p(v)<1\}$ if and only if $A$ is nonempty, convex, balanced, and $(-a)+A$ is absorbing for all $a\in A$.
		\item If $A$ is merely convex, balanced, and absorbing, then there is a seminorm $p\colon V\to\RR$ such that $\{v:p(v)<1\}\subseteq A$.
	\end{listalph}
\end{proposition}
\begin{proof}
	This is \cite[Proposition~1.14]{conway-functional}. The forward direction of (a) follows from combining \Cref{ex:seminorm-to-convex,ex:seminorm-to-balanced,ex:seminorm-to-absorbing}. It remains to show the reverse direction of (a) and (b). For both of these, we will take $p\coloneqq\norm\cdot_A$, which is a well-defined function by \Cref{rem:mink-seminorm-defined} (once we know that $A$ in (a) is absorbing). We will run our checks in a few pieces.
	\begin{itemize}
		\item In (a), we show that $A$ is absorbing. It is enough to check that $0\in A$. Well, being nonempty, there is some $a\in A$. Because $A$ is balanced, we see $-a\in A$, and because $A$ is convex, it follows that $0\in A$.
		\item If $A$ is absorbing and balanced, we check that $p(\lambda v)=\left|\lambda\right|p(v)$ for $v\in V$ and $\lambda\in\FF$. Well, $\lambda v\in tA$ if and only if $v\in\frac t\lambda A$, which is equivalent to $v\in\frac t{\left|\lambda\right|}A$ because $A$ is balanced! It follows that
		\[\{t\ge0:\lambda v\in tA\}=\left|\lambda\right|\left\{\frac t{\left|\lambda\right|}:v\in\frac t{\left|\lambda\right|}A\right\},\]
		so the check follows.
		\item If $A$ is convex and absorbing, then we check that $p(v+w)\le p(v)+p(w)$ for $v,w\in V$. The geometric input is that $tA+sA\subseteq(t+s)A$ for any $t,s>0$; this follows by convexity because
		\[tA+sA=(t+s)\left(\frac t{t+s}A+\frac s{t+s}A\right)\]
		is contained in $(t+s)A$ by convexity. Now, for the check, we note that having $p(v)<t$ and $p(w)<s$ implies that $v\in tA$ and $w\in sA$ by \Cref{rem:better-mink-seminorm}, so $v+w\in(t+s)A$, so $p(v+w)\le t+s$. Sending $t\to p(v)$ and $s\to p(w)$ completes the check.
		\item We complete the proof of (b). The above checks show that $p$ is a seminorm, so it remains to check that $\{v:p(v)<1\}\subseteq A$. This follows from $A$ being balanced: if $p(v)<1$, then there is $t<1$ such that $v\in tA$, and $tA\subseteq A$ because $A$ is balanced.
		\item We complete the proof of (a). The previous check shows that $\{v:p(v)<1\}\subseteq A$, so it remains to check the other inclusion. Well, for any $a\in A$, we see that $(-a)+A$ is absorbing, so $a+ta\in A$ for small $t>0$. It follows that $a\in(1+t)^{-1}A$, so $p(a)<(1+t)^{-1}<1$ follows.
		\qedhere
	\end{itemize}
\end{proof}
\begin{corollary} \label{cor:locally-convex-by-seminorm}
	Fix a topological vector space $V$ over $\FF$. The following are equivalent.
	\begin{listroman}
		\item $V$ is locally convex.
		\item The topology on $V$ is induced by a family of seminorms.
	\end{listroman}
\end{corollary}
\begin{proof}
	We show the implications separately.
	\begin{itemize}
		\item Suppose that $V$ is locally convex, so $0$ admits a neighborhood basis $\mc U$ of balanced convex sets by \Cref{rem:better-locally-convex}. By \Cref{ex:open-to-absorbing}, we see that each $U\in\mc U$ has $(-a)+U$ absorbing for all $a\in U$, so \Cref{prop:convex-to-seminorm} provides a seminorm $p_U\colon V\to\RR$ such that $U=\{v:p_U(v)<1\}$. Note that $p_U$ is continuous by \Cref{lem:seminorm-continuity-grab-bag}.

		Lastly, we should check that the topology given by the seminorms $\{p_U\}$ is the correct one. Well, this topology has basis given by finite intersections of sets of the form
		\[\{v\in V:p_U(v)\in(a,b)\},\]
		where $(a,b)\subseteq\RR$. The continuity of the $p_U$s implies that any such subset is open in $V$. Conversely, any open neighborhood of $0$ in $V$ contains some $U\in\mc U$ and therefore contains $\{v\in V:p(v)\in(-1,1)\}$, so a comparison of the neighborhood bases (via translation) implies that the open neighborhood of $0$ is still open.

		\item Suppose that $V$ has its topology generated by a family of seminorms $\{p_i\}$. Well, because $p_i(0)=0$ for each $i$, an open neighborhood basis of $0$ can be given by finite intersections of sets of the form $p_i^{-1}((-\varepsilon,\varepsilon))$. Of course, this is just
		\[\varepsilon\{v\in V:p_i(v)<1\},\]
		which we note is convex by \Cref{ex:seminorm-to-convex}. Thus, $0$ admits a neighborhood basis of convex sets.
		\qedhere
	\end{itemize}
\end{proof}
Now that we have an understanding of locally convex spaces, we may define Fr\'echet spaces.
\begin{definition}[complete]
	Fix a topological vector space $V$. Then a \textit{Cauchy net} in $V$ is a net $\{v_i\}$ such that each neighborhood $U$ of $0$ has some $N$ for which $v_i-v_j\in U$ for all $i,j>N$. We say that $V$ is \textit{complete} if every Cauchy net converges.
\end{definition}
\begin{remark}
	If $V$ is metrizable, then we may as well work with Cauchy sequences and being sequentially complete.
\end{remark}
\begin{definition}[Fr\'echet]
	A topological vector space $X$ is \textit{Fr\'echet} if and only if it is locally convex, metr\-izable, and complete.
\end{definition}
The bizarre addition here is metrizable. This condition fits in with the other ones as follows.
\begin{proposition} \label{prop:locally-convex-to-metric}
	Fix a locally convex topological vector space $V$ over $\FF$. Then the following are equivalent.
	\begin{listroman}
		\item $V$ has its topology induced by a translation-invariant metric.
		\item $V$ is metrizable.
		\item $V$ has a countable neighborhood basis of $0$.
		\item The topology on $V$ is induced by a countable family of seminorms.
	\end{listroman}
\end{proposition}
\begin{proof}
	The implication (i) to (ii) has no content, and (ii) to (iii) follows by taking the neighborhood basis of open subsets given by $\{v:d(v,0)<1/n\}$ for positive integers $n$. Next, (iii) implies (iv) by the proof of the forward direction of \Cref{cor:locally-convex-by-seminorm}, which built one seminorm for each balanced convex subset in the neighborhood basis of $0$.

	Lastly, we have to show that (iv) implies (i). Well, given the countable family of seminorms $\{p_i\}_{i\ge1}$, we define the function $d\colon V\times V\to\RR$ by
	\[d(x,y)\coloneqq\sum_{i=1}^\infty\frac1{2^i}\cdot\frac{p_i(x-y)}{1+p_i(x-y)}.\]
	Here are our checks on $d$.
	\begin{itemize}
		\item We check that $d$ is a metric. The summation always converges because $\frac{p_i(x-y)}{1+p_i(x-y)}\le1$ always. Continuing, $d(x,x)=0$ follows because $p_i(0)=0$ for all $i$, and the triangle inequality follows from the subadditivity of each of the $p_i$s.

		It remains to check the positivity of $d$. Well, if $x\ne y$, then because $V$ is Hausdorff, we see that $p_i(x-y)>0$ for some $p_i$. (Otherwise, the constant net $\{x-y\}$ would converge to both $0$ and $x-y$.) Thus, $d(x,y)>0$.

		\item We check that $d$ is translation-invariant. Well, for any $a\in V$, we see that $d(x+a,y+a)$ is a function of $(x+a)-(y+a)=(x-y)$ and will equal $d(x,y)$.

		\item Lastly, we check that $d$ induces the topology on $V$. It is enough to check that these two topologies have the same convergent nets. Well, a net $\{x_i\}$ converges to some $x\in V$ if and only if $p_\bullet(x_i-x)\to0$ for all seminorms $p_\bullet$. This surely implies that $d(x_i,x)\to0$, and conversely, $d(x_i,x)\to0$ will require that $p_\bullet(x_i-x)\to0$ for each $p_\bullet$.
		\qedhere
	\end{itemize}
\end{proof}

\section{Cauchy Sequences}
Because we will have occasion to work with more general topological spaces than metric spaces, we go ahead and pick up some facts about Cauchy sequences and completeness.
% \begin{definition}[uniformly continuous]
% 	Fix a linear map $f\colon V\to V'$ of topological vector spaces. Then $f$ is \textit{uniformly continuous} if and only if every open neighborhood $U'\subseteq V'$ of $0$ admits an open neighborhood $U\subseteq V$ of $0$ such that $x-y\in U$ implies $f(x)-f(y)\in U'$.
% \end{definition}
\begin{lemma} \label{lem:uniform-cont-is-cauchy-cont}
	Fix a continuous linear map $f\colon V\to V'$ of topological vector spaces. Then $f$ sends Cauchy nets to Cauchy nets.
\end{lemma}
\begin{proof}
	Suppose that $\{a_i\}_{i\in I}$ is a Cauchy net of $V$, and we would like to show that $\{f(a_i)\}_{i\in I}$ is a Cauchy net of $V'$. Well, for any open neighborhood $U'\subseteq V'$ of $0$, we are granted an open neighborhood $U\subseteq V$ of $0$ for which $x\in U$ implies that $f(x)\in U$. In particular, there is $N\in I$ for which $i,j>N$ implies that $a_i-a_j\in U$ and thus $f(a_i)-f(a_j)\in U'$, as required.
\end{proof}
\begin{definition}[Cauchy sequence]
	Fix a topological vector space $V$. Then a \textit{Cauchy sequence} is a Cauchy net where the index set is a $\NN$.
\end{definition}
\begin{proposition} \label{prop:extend-cont}
	Fix a sequentially dense linear subspace $W\subseteq V$, and choose a sequentially continuous linear map $f\colon W\to V'$ of topological vector spaces.
	\begin{listalph}
		\item There is at most one sequentially continuous linear map $g\colon V\to V'$ such that $g|_W=f$.
		\item If $V'$ is sequentially complete, then there is a sequentially continuous linear map $g\colon V\to V'$ such that $g|_W=f$.
	\end{listalph}
\end{proposition}
\begin{proof}
	We start with (a). By linearity of restriction, it is enough to show that any continuous linear map $g\colon V\to V'$ for which $g|_W=0$ has $g=0$. Well, for any $v\in V$, the density of $W\subseteq V$ grants a Cauchy sequence $\{w_i\}$ in $W$ for which $\{w_i\}\to v$. Thus, the continuity of $g$ requires that $\{g(w_i)\}\to g(v)$, so $g(v)=0$.

	It remains to show (b), which is trickier. We begin by defining $g$, and then we will run many checks on it. Because $W\subseteq V$ is sequentially complete, each $v\in V$ admits a Cauchy sequence $\{w_i\}$ of $W$ such that $\{w_i\}\to v$. Then \Cref{lem:uniform-cont-is-cauchy-cont} implies that $\{f(w_i)\}$ is still a Cauchy sequence in $V'$, so we define
	\[g(v)\coloneqq\lim_{i\to\infty}f(w_i).\]
	(The limit exists because $V'$ is sequentially complete!) It remains to run many checks on $g$.
	\begin{itemize}
		\item Well-defined: if $\{w_i\}$ and $\{w_i'\}$ are two Cauchy sequences approaching a given $v\in V$, then we need to check that $\lim f(w_i)=\lim f(w_i')$. Well, we see that $\{w_i-w_i'\}\to0$, so $\{f(w_i)-f(w_i')\}\to f(0)$ by continuity of $f$, so the claim follows.

		\item Linear: given scalars $a^1$ and $a^2$ and $v^1,v^2\in V$, we should check that $g\left(a^1v^1+a^2v^2\right)=a^1$. Well, we simply pick up Cauchy sequences $\left\{w^1_i\right\}$ approaching $v^1$ and $\left\{w^2_i\right\}$ approaching $v^2$. Then $\left\{a^1w^1_i+a^2w^2_i\right\}$ approaches $a^1v^1+a^2v^2$, so we can pass everything through the definition of $g$.

		\item Continuous: given a convergent sequence $\{v_i\}\to v$ in $V$, we would like to check that $\{g(v_i)\}\to g(v)$. By linearity of $g$, we may assume that $v=0$ (by subtracting out $v$ everywhere). Now, for any open neighborhood $U'\subseteq V'$ of $0$, we would like to show that $g(v_i)\in U'$ for large $i$. For technical reasons, we choose an open neighborhood $U'_0\subseteq U'$ of $0$ such that $U'_0+U'_0\subseteq U'$, which is possible by the continuity of addition. The point of doing this is that the closure of $U'_0$ is contained in $U'$: for any Cauchy net $\{u_i'\}$ in $U'_0$ converging to some $u'$, we see that $u'-u'_i\in U'_0$ for any $i$ large enough, so $u'\in U_0'+U'_0$.

		Now, by continuity of $f$, we are granted an open neighborhood $U\subseteq V$ of $0$ such that $x\in U\cap W$ implies $f(x)\in U'_0$. For any $i$ large enough, we see that $v_i\in U_0$. Then for each $i$, we fix a Cauchy sequence $\{w_{ij}\}$ of $W$ approaching $v_i$; for any $j$ large enough, we see $w_{ij}\in U_0$ as well. Thus, $\{f(w_{ij})\}\to g(v_i)$ by definition of $g$, so $g(v_i)$ is in the closure of $U'_0$, so $g(v_i)\in U'$, as required.
		\qedhere
	\end{itemize}
\end{proof}
\begin{remark}
	By replacing all sequences with nets in the proof, we can upgrade the statement to show that continuous linear maps extend uniquely from dense linear subspaces when the target is complete.
\end{remark}
% \begin{remark} \label{rem:seq-cont-to-cont}
% 	If $V$ is first countable, then $g\colon V\to V'$ being sequentially continuous implies that $g$ is continuous. Indeed, given any net $\{x_i\}_{i\in I}\to x$, we note that there is a subsequence $\{x_j\}_{j\in\NN}$ of the net converging to $x$: for a descending countable base $\{U_j\}_{j\in\NN}$ of the neighborhoods of $x$, one just chooses $x_j\in U_j$. Then $\{g(x_j)\}\to g(x)$ by assumption on $g$. But note $\{g(x_i)\}_{i\in I}$ is a Cauchy net by \Cref{lem:uniform-cont-is-cauchy-cont}, so it converges to at most one point, which we see must be $g(x)$.
% \end{remark}
% \begin{lemma}
% 	Fix a topological vector space $V$. Then the following are equivalent.
% 	\begin{listalph}
% 		\item $V$ is complete: every Cauchy net in $V$ admits a limit.
% 		\item For any continuous linear map $\varphi\colon V\to W$, the image $\im\varphi$ is closed in $W$.
% 	\end{listalph}
% \end{lemma}
% \begin{proof}
% \end{proof}

\section{The Open Mapping Theorem}
In this section, we review the proof of the Open mapping theorem in order to extend the usual proof (for Banach spaces) to the setting of Fr\'echet spaces.

As usual, our proof will have to rely on the Baire category theorem. Before introducing any strange terminology, let's start with a statement on just metric spaces.
\begin{lemma} \label{lem:baire-cat-start}
	Let $X$ be a nonempty complete metric space. Then a countable intersection of dense open subsets is dense.
\end{lemma}
\begin{proof}
	Let $\{U_i\}_{i\in\NN}$ be our collection of dense open subsets. We would like to show that their intersection $\bigcap_{i\in\NN}U_i$ intersects any open subset $V$ of $X$. The idea is to recursively choose nearby elements in $U_i\cap V$ for each $i$, and then use completeness of $X$ to finish the proof. We proceed in steps.
	\begin{enumerate}
		\item We build a sequence of points $\{x_n\}_{n\in\NN}$ recursively, as follows. To start us off, we note $V\cap U_0$ is nonempty and open (by density of $U_0$), so we are granted a point $x_0\in U_0$ and $\varepsilon_0$ such that $B(x_0,\varepsilon_0)\subseteq V\cap U_0$. For the recursion, we suppose that we are given such an open neighborhood $B(x_n,\varepsilon_n)$, and then because $U_{n+1}$ is open and dense, we are provided a point $x_{n+1}$ in the intersection and some $\varepsilon_{n+1}<\varepsilon_n/3$ such that
		\[B(x_{n+1},\varepsilon_{n+1})\subseteq B(x_n,\varepsilon_n)\cap U_{n+1}.\]

		\item We claim that the sequence $\{x_n\}_{n\in\NN}$ is (rapidly) Cauchy. Indeed, note $\varepsilon_{n+1}<\varepsilon_n/2$ for each $n$, so $\varepsilon_n<2^{-n}$ follows by an induction. Thus, $d(x_n,x_{n+1})<2^{-n}$ for each $n$, so our sequence is rapidly Cauchy. To finish checking that it is Cauchy, we note that whenever $i<j$, we have
		\[d(x_i,x_j)\le\sum_{k=i}^{j-1}\underbrace{d(x_k,x_{k+1})}_{<2^{-k}},\]
		which is upper-bounded by $2^{-i+1}$.

		\item Now, we let $x$ be a limit point of $\{x_n\}_{n\in\NN}$. (This is where we used completeness!) Because our sequence is eventually in $B(x_n,\varepsilon_n)$ for any given $n$, we see that $x\in B(x_n,\varepsilon_n)$ for each $n$. Thus, $x\in V\cap U_0$ by the first step of the construction, and $x\in U_n$ for each $n\ge1$ by the recursive step of the construction.
		\qedhere
	\end{enumerate}
\end{proof}
The previous lemma now upgrades to the Baire category theorem.
\begin{theorem}[Baire category] \label{thm:baire-cat}
	Let $X$ be a nonempty complete metric space. Let $\{U_i\}_{i\in\NN}$ be a countable collection of dense open subsets. Then the intersection $\bigcap_{i\in\NN}U_i$ is not contained in a countable union of nowhere dense subsets.
\end{theorem}
\begin{proof}
	Suppose for the sake of contradiction that we have
	\[\bigcap_{i\in\NN}U_i\subseteq\bigcup_{j\in\NN}A_j,\]
	where each $A_j$ is nowhere dense. It thus follows that
	\[\bigcap_{i\in\NN}U_i\cap\bigcap_{j\in\NN}X\setminus\overline{A_j}\]
	is empty. We claim that $X\setminus\overline{A_j}$ is open and dense, which yields the desired contradiction by \Cref{lem:baire-cat-start}. Certainly $X\setminus\overline{A_j}$ is open; for density, note that $\overline{A_j}$ contains no open subset, which means that the complement intersects any open subset.
\end{proof}
\begin{corollary} \label{cor:baire-cat-whole-space}
	Let $X$ be a nonempty complete metric space. Then $X$ is not the countable union of nowhere dense subsets.
\end{corollary}
\begin{proof}
	This follows from taking $U_i=X$ for each $i$ in \Cref{thm:baire-cat}.
\end{proof}
We now proceed with the Open mapping theorem. We will isolate the application of the Baire category theorem to the following lemma: \Cref{cor:baire-cat-whole-space} shows that the hypothesis is satisfied.
\begin{lemma} \label{lem:omt-use-baire}
	Let $f\colon X\to Y$ be a linear map of locally convex topological vector spaces. Suppose that $\im f$ is not the union of nowhere dense subsets. Then $\overline{f(U)}$ contains an open neighborhood of $0$ for each open neighborhood $U$ of $0$.
\end{lemma}
\begin{proof}
	This more or less follows from unwinding the hypothesis. Because $V$ is locally convex, we may shrink $U$ to make it convex and balanced (by \Cref{lem:convex-to-convex-balanced}); we will only use this at the end of the proof. The hypothesis is applied as follows: because $U$ is open, it is absorbing (by \Cref{ex:open-to-absorbing}), so $V=\bigcup_{i\in\NN}iU$, so
	\[\im f=\bigcup_{i>0}iU.\]
	Thus, the hypothesis implies that one of the $iU$ fails to be nowhere dense; because multiplication by $i>0$ is a homeomorphism, we see that $U$ is also fails to be nowhere dense, so it contains an open subset $V$.

	We now upgrade $V$ into an open neighborhood of $0$. Well, simply set $V'\coloneqq\frac12(V-V)$. Then $f(V')$ is contained in $\frac12(U-U)$ by linearity, which is $\frac12(U+U)$ because $U$ is balanced, which is contained in $U$ because $U$ is convex.
\end{proof}
It remains to use the hypothesis that $X$ is complete, which is done in the following lemma.
\begin{lemma} \label{lem:omt-use-complete}
	Let $f\colon X\to Y$ be a continuous linear map of metrizable locally convex topological spaces. Suppose that $X$ is complete and that $\overline{f(U)}$ contains an open neighborhood of $0$ for each open neighborhood $U$ of $0$. Then $f$ is open.
\end{lemma}
\begin{proof}
	The idea is to use the completeness of $X$ to construct points of $U$ which go to a required open neighborhood. We proceed in steps.
	\begin{enumerate}
		\item We are going to show that $f(U)$ contains an open neighborhood of $0$ for each open neighborhood $U$ of $0$, so let's spend a moment to explain why this is enough. For each open subset $U'\subseteq X$ and $x\in U'$, we note that $f(U'-x)$ contains an open neighborhood $V_x$ of the origin. Thus, $f(U')$ contains the open neighborhood $f(x)+V_x$, so $f(U')$ equals
		\[\bigcup_{x\in U'}f(x)+V_x,\]
		which is open because it is a union of open subsets.

		\item We unwind the hypothesis on $f$. By shrinking our open neighborhood $U$ of $0$, we may assume that $U$ is convex and balanced (by \Cref{lem:convex-to-convex-balanced}), so there is a seminorm $p$ on $X$ for which $U$ is $B(0,1)$ for some translation-invariant metric $d$ on $X$, chosen via \Cref{prop:locally-convex-to-metric}.
		% \[U=\{x\in X:p(x)<1\}\]
		% by \Cref{prop:convex-to-seminorm} (which applies because the absorbing hypothesis follows from \Cref{ex:open-to-absorbing}).
		Similarly, by hypothesis on $f$, we know that $\overline{f(U)}$ contains some open neighborhood $V$ of $0$, which we may again shrink until it is $B(0,2)$ for some translation-invariant metric $d$ on $Y$.
		% \[V=\{y\in Y:q(y)<2\}\]
		% for some seminorm $q$ on $Y$.
		It will be worthwhile to remove the closure from this statement. Well, for any $y\in Y$, we see that $y\in d(y,0)V$ and so $y\in\overline{q(y)\cdot f(U)}$, so any $\varepsilon>0$ has some $x\in X$ for which $d(x,0)<d(y,0)$ and $d(y,f(x))<\varepsilon$.

		\item We will actually show that $V\subseteq f(U)$, so choose some $y\in V$. The completeness of $X$ will be used via a limiting process to produce an element of $U$ mapping to $y$. To start us off, fix some $\varepsilon>0$ (to be fixed at the end of the proof), and we take $x_0\coloneqq0$. Now, if we are given $x_0+\cdots+x_n$, we may select $x_{n+1}$ so that $d(x_{n+1},0)<d(y-f(x_0+\cdots+x_n),0)$ and
		\[d(y-f(x_0+\cdots+x_n),f(x_{n+1}))<\varepsilon/2^{n+1}\]
		by the above paragraph.

		\item We complete the proof. Now, by construction, $d(x_n,0)<\varepsilon/2^{n-1}$ for all $n\ge2$, so the sequence of partial sums is rapidly Cauchy. As in the proof of \Cref{lem:baire-cat-start}, it follows that these partial sums converge to some $x\in X$.
		
		We claim that this $x$ is the desired element. To start, we see that $f(x_0+\cdots+x_n)\to y$ as $n\to\infty$ by construction, so $f(x)=y$ by continuity of $f$!

		It remains to check that $x\in U$. Well, $d(x,0)$ is bounded by
		\[\sum_{n=0}^\infty d(x_n,0)<d(x_0,0)+d(x_1,0)+\sum_{n=2}^\infty\frac\varepsilon{2^{n-1}},\]
		and $d(x_1,0)<d(y,0)<2$ and $\sum_{n=2}^\infty\frac\varepsilon{2^{n-1}}=\varepsilon$, so $x\in U$ for $\varepsilon$ small enough.
		\qedhere
	\end{enumerate}
\end{proof}
\begin{theorem}[Open mapping] \label{thm:omt}
	Let $f\colon X\to Y$ be a continuous linear map of Fr\'echet spaces. If $f$ is surjective, then $f$ is open.
\end{theorem}
\begin{proof}
	By \Cref{cor:baire-cat-whole-space}, we see that $X$ is not a countable union of nowhere dense subsets. The result now follows from combining \Cref{lem:omt-use-baire,lem:omt-use-complete}.
\end{proof}
\begin{corollary}
	Let $f\colon X\to Y$ be a bijective continuous linear map of Fr\'echet spaces. Then $f$ has continuous inverse.
\end{corollary}
\begin{proof}
	Let $g$ be the inverse map. Checking that $g$ is continuous is equivalent to checking that $f$ is open, which follows from \Cref{thm:omt}.
\end{proof}

\section{The Hahn--Banach Theorems}
In this section, we review the proof of the Hahn--Banach theorem. This section will be filled with plenty of nonsense. Ultimately, we are interested in extending continuous linear functionals on Fr\'echet spaces, but along the way, we will show that linear functionals separate convex sets.

For one of our applications, we will require a weaker notion than a seminorm.
\begin{definition}
	Fix a vector space $V$ over $\FF$. A function $p\colon V\to\RR_{\ge0}$ is a half-seminorm if and only if it is sublinear and satisfies $p(tx)=tp(x)$ for $t>0$ and $x\in V$.
\end{definition}
The lack of input for $p(-x)$ prevents $p$ from being a seminorm.

As with Banach spaces, we check if a linear functional is continuous by checking if it is bounded, but the definition of bounded needs to be adjusted.
\begin{definition}[bounded]
	Fix a topological vector space $V$ over $\FF$. A linear functional $\ell\colon V\to\FF$ is \textit{bounded} if and only if there is an open neighborhood $U$ of $0$ and a constant $c>0$ such that $\left|\ell(x)\right|\le c$ for all $x\in U$.
\end{definition}
\begin{lemma} \label{lem:linear-func-cont-grab-bag}
	Fix a topological vector space $V$ over $\FF$ and a linear functional $\ell$ on $V$. Then the following are equivalent.
	\begin{listroman}
		\item $\ell$ is continuous.
		\item $\ell$ is bounded.
		\item $\ell$ is continuous at $0$.
	\end{listroman}
\end{lemma}
\begin{proof}
	We use \Cref{lem:seminorm-continuity-grab-bag}. The implication from (i) to (ii) is direct; the proof that (iii) implies (i) is identical to the proof in \Cref{lem:seminorm-continuity-grab-bag}. To show (ii) implies (iii), we note $\left|\ell\right|$ is a seminorm, and by considering nets, we see that it is enough to check that $\left|\ell\right|$ is continuous at $0$, which follows from \Cref{lem:seminorm-continuity-grab-bag}(ii) and the fact that $\ell$ is bounded.
\end{proof}
\begin{corollary} \label{cor:functional-check-continuity}
	Fix a topological vector space $V$ over $\FF$, and let $p\colon V\to\RR_{\ge0}$ be a continuous half-seminorm. Given a linear functional $\ell\colon V\to\FF$, if $-p(-v)\le\ell(v)\le p(v)$ for all $v$, then $\ell$ is continuous.
\end{corollary}
\begin{proof}
	Because $-p(-v)\le\ell(v)\le p(v)$ for all $v$, we see that the open subset $U\coloneqq\{v\in V:p(v)<1\}$ bounds $\ell$ (with the constant $1$), so we are done by \Cref{lem:linear-func-cont-grab-bag}.
\end{proof}
Thus, it will be important to be able to extend linear functionals along with an upper bound against a seminorm. By considerations with Zorn's lemma, we will find that the hard part is extending the linear functional one step, which is the content of the next lemma.
\begin{lemma} \label{lem:hahn-banach-one-step}
	Fix a vector space $V$ over $\RR$, a half-seminorm $p$ on $V$, and a linear functional $\ell$ on a subspace $W\subseteq V$ such that $-p(-w)\le\ell(w)\le p(w)$ for all $w\in W$. Given any $v'\in V$, there is an extension $\ell'$ to a linear functional on a subspace $W'$ containing $v'$ such that $-p(-v)\le\ell'(v)\le p(v)$ for all $v$.
\end{lemma}
\begin{proof}
	If $v'\in W$ already, then there is nothing to do. Otherwise, for any real number $c$, we see that we may extend $\ell$ to a linear functional $\ell'$ on $W'\coloneqq W+\RR v'$ by setting $\ell'(v')\coloneqq c$. Namely, we have
	\[\ell'(w+tv')=\ell(w)+tc\]
	for any $w\in W$ and $t\in\RR$.
	
	We would like to show that we can choose $c$ so that $-p(-v)\le\ell'(v)\le p(v)$ for all $v$ of the form $w+tv'$. This requires a little trickery. By scaling, it is enough to only check with $t\in\{\pm1\}$ (because $t=0$ follows by hypothesis). Thus, we need both $\ell(w)+c\le p(w+v')$ and $\ell(w)-c\le p(w-v')$ for all $w\in W$. Now, such a $c$ exists if and only if
	\[\sup_{w\in W}(\ell(w)-p(w-v'))\stackrel?\le\inf_{w\in W}(p(w+v')-\ell(w)).\]
	For this, we should check that $\ell(w)-p(w-v')\le p(w'+v')-\ell(w')$ for any $w,w'\in W$, which is equivalent to $\ell(w+w')\le p(w-v')+p(w'+v')$. This last inequality follows because $\ell(w)\le p(w)$ and the subadditivity of $p$.
\end{proof}
\begin{proposition} \label{prop:hahn-banach-real}
	Fix a vector space $V$ over $\RR$, a half-seminorm $p$ on $V$, and a linear functional $\ell$ on a subspace $W\subseteq V$ such that $-p(-v)\le\ell(v)\le p(v)$ for all $v\in W$. Then $\ell$ extends to a linear functional $\ell'$ on $V$ such that $-p(-v)\le \ell'(v)\le p(v)$ for all $v\in V$.
\end{proposition}
\begin{proof}
	After \Cref{lem:hahn-banach-one-step}, the rest of this proof is largely formal nonsense. We use Zorn's lemma on the partially ordered set $\mathcal P$ of pairs $(V',\ell')$, where $V'$ is an intermediate subspace, and $\ell'$ is a functional on $V'$ bounded above by $p$; the ordering is given by $(V',\ell')\le(V'',\ell'')$ if and only if $V'\subseteq V''$ and $\ell''|_{V'}=\ell'$. Our application of Zorn's lemma is in two steps.
	\begin{itemize}
		\item We claim that $\mathcal P$ has a maximal element, for which we use Zorn's lemma. First, note $\mathcal P$ is nonempty because it has $(W,\ell)$. Secondly, any ascending chain $\{(W_i,\ell_i)\}_i$ in $\mathcal P$ has upper bound given by setting $V'\coloneqq\bigcup_iW_i$ and defining $\ell'$ as the union of the $\ell_i$s. We can see that $V'$ is still a vector space, and the nature of the partial ordering verifies that $\ell'$ is a well-defined functional extending $\ell$. Thus, so $(V',\ell')$ is indeed an upper bound for our chain.
		\item Let $(V',\ell')$ be a maximal element of $\mathcal P$. We claim that $V'=V$, which will complete the proof. We already have $V'\subseteq V$, so it remains to show the other inclusion. Well, for any $v\in V$, we see that $(V',\ell')$ can be extended up to $V'+\RR v$ by \Cref{lem:hahn-banach-one-step}, so the maximality of $(V',\ell')$ requires $V'+\RR v=V'$. Thus, $v\in V'$, so $V\subseteq V'$ follows.
		\qedhere
	\end{itemize}
\end{proof}
In order to work with the base field $\CC$, we use a trick.
\begin{notation}
	Fix a vector space $V$ over $\CC$. Given a real linear functional $\ell\colon V\to\RR$, we define $C\ell\colon V\to\CC$ by
	\[C\ell(x)\coloneqq\ell(x)-i\ell(ix).\]
	On the other hand, for any complex linear functional $\ell$ on $V$, we define its real part as $\op{Re}\ell\coloneqq\frac12(\ell+\overline\ell)$.
\end{notation}
Here are a few basic facts about this construction.
\begin{lemma} \label{lem:use-complexification}
	Fix a vector space $V$ over $\CC$.
	\begin{listalph}
		\item If $\ell$ is a real linear functional on $V$, then $C\ell\colon V\to\CC$ is a complex linear functional. Furthermore, $\Re C\ell=\ell$.
		\item If $\ell$ is a complex linear functional on $V$, then $C\Re\ell=\ell$.
	\end{listalph}
\end{lemma}
\begin{proof}
	We show these separately.
	\begin{listalph}
		\item First, we check that $C\ell\colon V\to\CC$ is a (complex) linear functional. Indeed, $C\ell$ is already real linear (it is a sum of real linear functionals), so it is enough to check that $C\ell(ix)=iC\ell(x)$ for any $x\in V$, which is true because quantities equal $\ell(ix)+\ell(x)$.

		Secondly, for any $x\in V$, we see $\Re C\ell(x)=\Re(C\ell(x))=\ell(x)$.
		
		\item This is a direct calculation. For any $x\in V$, we see $C\Re\ell(x)=\Re\ell(x)-i\Re\ell(ix)$, which further expands to
		\[\frac{\ell(x)+\overline{\ell(x)}-i\ell(ix)-i\overline{\ell(ix)}}2,\]
		which is $\ell(x)$ after the dust settles.
		\qedhere
	\end{listalph}
\end{proof}
\begin{theorem}[Hahn--Banach existence] \label{thm:hahn-banach-exist}
	Fix a vector space $V$ over $\FF$, a half-seminorm $p$ on $V$, and a linear functional $\ell$ on a subspace $W\subseteq V$ such that $\ell\le p$. Then $\ell$ extends to a linear functional $\ell'$ on $V$ such that $\left|\ell'\right|\le p$.
\end{theorem}
\begin{proof}
	The case of $\FF=\RR$ follows from \Cref{prop:hahn-banach-real}, so we may take $\FF=\CC$. For this, we freely use \Cref{lem:use-complexification}. Even in this case, \Cref{prop:hahn-banach-real} provides some real functional $\ell'$ on $V$ extending $\Re\ell$ such that $\left|\ell'\right|\le p$. Then $C\ell'$ extends $C\Re\ell=\ell$, so it remains to check that $\left|C\ell'\right|\le p$. Well, for any $x\in V$, there is a unit complex number $\mu$ such that $C\ell'(\mu x)=\left|C\ell'(x)\right|$. But this means $C\ell'(\mu x)$ equals $\Re C\ell'(\mu x)$, which is just $\ell'(\mu x)$, which we now know to be upper-bounded by $p(\mu x)$ by construction. However, $p(\mu x)=p(x)$ because $p$ is a half-seminorm!
\end{proof}
We are now ready to give some applications. Let's start with showing that linear functionals separate convex sets.
\begin{theorem}[Hahn--Banach separation] \label{thm:hahn-banach-separate}
	Fix a Fr\'echet space $V$ over $\FF$. Given disjoint convex subsets $X$ and $Y$ of $V$ where $X\subseteq V$ is open, there is a continuous linear functional $\ell$ on $V$ and a constant $c$ so that
	\[\Re\ell(x)<c\le\Re\ell(y)\]
	for all $x\in X$ and $y\in Y$.
\end{theorem}
\begin{proof}
	We proceed in steps.
	\begin{enumerate}
		\item We quickly reduce to the case of $V=\RR$: given the statement over $\RR$, whatever real functional $\ell$ the statement would produce even with $\FF=\CC$ can be upgraded to a complex linear functional by working with $C\ell$, which then still satisfies the required inequality because $\Re C\ell=\ell$ by \Cref{lem:use-complexification}.

		\item Thus, we may assume that $\FF=\RR$. As another quick reduction, we remark that it is enough to find a continuous linear functional which merely achieves
		\[\ell(x)\stackrel?<\ell(y)\]
		for all $x\in X$ and $y\in Y$. Then we note that $\ell$ is surjective, so \Cref{thm:omt} shows that $\ell(X)$ is open! Thus, $\ell(x)$ has a supremum $c$ which it does not achieve, and this constant $c$ suffices for the statement.

		\item It remains to construct such a continuous linear functional $\ell$; equivalently, we want to have $\ell(x-y)<0$ for all $x\in X$ and $y\in Y$. We will construct $\ell$ via \Cref{prop:hahn-banach-real}. By translation, we may assume that $0\in X$, which means that we want $Y$ to take on large values. To this end, we define our starting functional on $\RR y_0$ for some (nonzero!) basepoint $y_0\in Y$ to send $y_0\mapsto1$.

		\item To apply \Cref{prop:hahn-banach-real}, we must still build a half-seminorm, which we do with \Cref{prop:convex-to-seminorm}. Because we want to upper-bound $\ell(X-Y)$, it would make sense to $X-Y$ as our convex set, but we need to have an open neighborhood of $0$. Instead, we take
		\[U\coloneqq X-(-y_0+Y).\]
		This is a sum of convex sets and hence convex; similarly, this is a union of open sets and hence an open neighborhood of $0$. Thus, \Cref{prop:convex-to-seminorm} provides us with a half-seminorm $p$ on $V$ such that $p\colon V\to\RR$ has $U=\{v\in V:p(v)<1\}$, and we remark that $p$ is continuous by \Cref{lem:seminorm-continuity-grab-bag}. (Technically, we do not know if $U$ is balanced, but convexity of $U$ along with $0\in U$ shows that $tU\subseteq U$ for all $t\in[0,1)$, which is good enough to show that $p$ is a half-seminorm.)

		\item We are now ready to apply \Cref{prop:hahn-banach-real}. Quickly, we check that $-p(-y_0)\le1\le p(y_0)$, which holds because $y_0\notin U$ (because this would imply $X\cap Y\ne\emp$). Thus, \cref{prop:hahn-banach-real} provides us with a linear functional $\ell$ on $V$ such that $\ell(x_0-y_0)=1$ and $-p(-v)\le\ell(v)\le p(v)$. In particular, $\ell$ is continuous by \Cref{lem:linear-func-cont-grab-bag}.

		\item We complete the proof. It remains to check that $\ell(x)<\ell(y)$ for all $x\in X$ and $y\in Y$. Well, for any $x\in X$ and $y\in Y$, we see that $x-(-y_0+y)\in U$, so $p(y_0+x-y)<1$, but $\ell(y_0)=1$, so $\ell(y_0+x-y)<\ell(y_0)$, so
		\[\ell(x)<\ell(y),\]
		as required.
		\qedhere
	\end{enumerate}
\end{proof}
Here are some applications. If one has a compact convex set, then one can improve the separation result.
\begin{corollary} \label{cor:separate-compact}
	Fix a Fr\'echet space $V$ over $\FF$. Given disjoint convex subsets $K$ and $A$ where $K\subseteq V$ is compact and $A\subseteq V$ is closed, there is a continuous linear functional $\ell$ on $V$ and constants $c$ and $d$ so that
	\[\Re\ell(k)\le c<d\le\Re\ell(a)\]
	for $k\in K$ and $a\in A$.
\end{corollary}
\begin{proof}
	We apply \Cref{thm:hahn-banach-separate}. For this, we need to produce an open convex set. Well, because $K$ is compact, the main claim that there is an open neighborhood $U$ of $0$ such that $(K+U)\cap A=\emp$. Let's explain how this completes the proof. To start, note that we may shrink $U$ to be convex so that $K+U$ is open and convex. Then \Cref{thm:hahn-banach-separate} provides us with a continuous linear functional $\ell$ and constant $d$ for which
	\[\Re\ell(k+u)<d\le\Re\ell(a)\]
	for all $k+u\in K+U$ and $a\in A$. However, $K$ is compact, so the continuous function $k\mapsto\Re\ell(k)$ must achieve its supremum, which we take to be $c$, and we are done after noting that we must have $c<d$.

	It remains to show the main claim, which is purely topological in nature. Suppose that every open neighborhood $U$ of $0$ has $(K+U)\cap A\ne\emp$; we will show that $K\cap A\ne\emp$. Consider the collection $\{K\cap (A-U)\}$ of nonempty open subsets of $K$, which we note is closed under intersections, so we can extend it to a filter and then an ultrafilter $\mc F$. The compactness of $K$ then shows that $\mc F\to k$ for some $k\in K$; by translation, we may assume $k=0$. Unwinding, this means that $\mc F$ contains every open neighborhood $U$ of $k$, so $K\cap(A-U)$ has nontrivial intersection with $U$, so $A$ has nontrivial intersection with $U+U$, so we see $0\in\overline A$ (by shrinking $U$). But because $A$ is closed, we conclude $0\in A\cap K$!
	% In other words, each $U$ provides us with points $k_U\in K$ and $u_U\in U$ such that $k_U+u_U\in A$. Because $K$ is compact, the net(!) $\{k_U\}$ admits a limit point $k\in K$; passing to a subnet, we may assume that we have convergence to $k$. But $\{u_U\}\to0$ by definition, so we conclude that the net $\{k_U+u_U\}$ of elements in $A$ still converge to $k$, so $k\in A$!
\end{proof}
In the special case where $K$ is a point, we are able to say something about the existence of continuous functionals.
\begin{corollary} \label{cor:subspace-closure-by-func}
	Fix a Fr\'echet space $V$ over $\FF$. Given a vector $v\in V$ and subspace $W\subseteq V$, there is a continuous linear functional $\ell$ on $V$ with $\ell|_W=0$ and $\ell(v)=1$ if and only if $v\notin\overline W$.
\end{corollary}
\begin{proof}
	Of course, if such an $\ell$ exists, then $\overline W\subseteq\ker\ell$, so $v\notin\overline W$.

	In the other direction, we use \Cref{cor:separate-compact} with $K=\{v\}$ and $A=\overline W$ to get a linear functional $\ell$ such that $\Re\ell(v)<\Re\ell(w)$ for all $w\in\overline W$. However, $\Re\ell$ is a (real) linear functional, so it must vanish on $W$ if it is not surjective, so $\Re\ell|_W=0$, so $\ell|_W=0$ by \Cref{lem:use-complexification}. Thus, we see $\Re\ell(v)<0$, so we can complete the proof by dividing $\ell$ by $\ell(v)$.
\end{proof}
At long last, we are ready to extend some functionals.
\begin{corollary} \label{cor:h-b-extend-cont}
	Fix a Fr\'echet space $V$ over $\FF$. Any continuous linear functional $\ell$ on a subspace $W\subseteq V$ extends to a continuous linear functional on $V$.
\end{corollary}
\begin{proof}
	If $\ell=0$, then there is nothing to do. Otherwise, we use \Cref{cor:subspace-closure-by-func}. Choose a vector $w_0\in W$ such that $\ell(w_0)=1$. Thus, we see that $w_0\notin\overline{\ker\ell}$, so \Cref{cor:subspace-closure-by-func} gives us a continuous linear functional $\ell'$ on the full space $V$ for which $\ell'|_{\ker\ell}=0$ and $\ell'(w_0)=\ell(w_0)=1$. Because $W=\RR w_0+\ker\ell$, we conclude that $\ell'$ does in fact extend $\ell$, so we are done.
\end{proof}
\begin{remark}
	It may appear that we can use \Cref{lem:linear-func-cont-grab-bag} to immediately extend continuity from $W$ up to $V$. However, there is something tricky going on: a priori, $\ell$ on $W$ is only bounded by a continuous seminorm from $W$, and it is not obvious that such a seminorm should also extend to a suitable continuous seminorm on $V$!
\end{remark}

\section{Examples of Fr\'echet Spaces}
To gain some experience with our new definition, we provide some examples of Fr\'echet spaces.
\begin{definition}[Banach]
	A topological vector space $V$ is \textit{Banach} if and only if its topology is given by a norm, and $V$ is complete.
\end{definition}
\begin{example}
	Any Banach space is Fr\'echet. For example, one can take the countable family of seminorms to just be the single (actual) norm and conclude by \Cref{cor:locally-convex-by-seminorm}.
\end{example}
Of course, we are going to be interested in examples of Fr\'echet spaces which are not necessarily Banach spaces.
\begin{example} \label{ex:c-x-frechet}
	Fix a Hausdorff topological space $X$.
	\begin{listalph}
		\item Then the space $C(X)$ of continuous functions $X\to\CC$ has a topology where $\{f_i\}\to f$ if and only if $f_i\to f$ uniformly on compact sets. Furthermore, $C(X)$ is locally convex.
		\item If $X$ is compact, then $C(X)$ is a Banach space.
		\item If $X$ is locally compact and second countable, then $C(X)$ is Fr\'echet.
	\end{listalph}
\end{example}
\begin{proof}
	Let's begin by explaining how to give $C(X)$ a topology so that it is always locally convex. Consider the collection $\mc K$ of compact subsets of $X$, which we give a partial ordering by inclusion. For each compact set $K$, we define a seminorm $\nu_K\colon C(X)\to\RR$ by
	\[\nu_K(f)\coloneqq\max_{x\in K}\left|f(x)\right|.\]
	To see that $\nu_K$ is a seminorm, we note that $\nu_K$ is surely homogeneous, and $\nu_K$ is subadditive because $\left|f(x)+g(x)\right|\le\left|f(x)+g(x)\right|$.
	
	We may thus give $C(X)$ the topology given by these seminorms. Quickly, note that $C(X)$ is Hausdorff because $f\ne g$ has some $x\in X$ for which $f(x)\ne g(x)$, and then we have $\nu_{\{x\}}(f-g)>0$. Thus, we see that this topology is locally convex for free! Lastly, observe that $\{f_i\}\to f$ if and only if $\{\nu_K(f_i-f)\}\to0$ for all compact subsets $K\subseteq X$; by translation, we may assume that $f=0$. But of course, having $\{\nu_K(f_i-f)\}\to0$ is equivalent to having uniform convergence on compact sets.

	Next up, let's suppose that $X$ is compact, and we will show that $C(X)$ i Banach. Well, we claim that the topology is actually induced by the single seminorm $\nu_X$, which is actually a norm. To see that the topology is induced by $\nu_X$, note that $\nu_K\le\nu_X$ for all $K$, so having $\{\nu_K(f_i-f)\}\to0$ for all $K\subseteq X$ is equivalent to having $\{\nu_X(f_i-f)\}\to0$. To see that $\nu_X$ is a norm, it remains to check positivity, for which we note that any nonzero continuous function $f$ is nonzero at some $x\in X$, so $\nu_X(f)\ge\left|f(x)\right|$ is positive.

	Lastly, we assume that $X$ is locally compact and second countable. This means that $X$ will admit a countable cover by pre-compact open subsets, so taking the closure shows that $X$ receives a countable cover $\{K_j\}_{j\in\NN}$ by compact sets. By replacing each $K_j$ with $\bigcup_{j'\le j}K_{j'}$, we may assume that
	\[K_0\subseteq K_1\subseteq\cdots.\]
	Accordingly, we claim that the topology is generated by the seminorms $\{\nu_{K_j}\}_{j\in\NN}$. Well, for any compact set $K$, we see $\{\nu_K(f_i-f)\}\to0$ is implied by having this convergence for any larger compact set, so it will be implied by $\{\nu_{K_n}(f_i-f)\}\to0$ for some $n$ sufficiently large.
\end{proof}
\begin{remark}
	Suppose $X$ is compact. Then it turns out that $C(X)$ is separable if and only if $X$ is a metric space. Let's sketch the backwards direction: being metrizable implies that $X$ is second countable, so there is a countable dense subset $\{x_i\}_{i\in\NN}$ of $X$. Consider the algebra $\mc A$ over $\CC$ of functions generated by $1$ and $x\mapsto d(x_i,x)$. Note $x\mapsto d(x_i,x)$ is continuous, so $\mc A\subseteq C(X)$; additionally, $\mc A$ separates points by construction. It follows that $\mc A$ is dense in $C(X)$; considering corresponding algebra over $\QQ$ provides us with a countable dense subset of $C(X)$.
\end{remark}
Here is an example of the sort of check we can do.
\begin{example} \label{ex:pushforward-cont}
	Let $f\colon X\to Y$ be a continuous map. Then the map $(-\circ f)\colon C(Y)\to C(X)$ is well-defined and continuous.
	% If every compact subset $K\subseteq Y$ has a compact subset $K\subseteq Y$ for which $f(K)=L$, then $(-\circ f)$ is continuous.
	We may write this operation as $f^*\colon C(Y)\to C(X)$.
\end{example}
\begin{proof}
	The map is well-defined because the composite of continuous maps is continuous. It remains to show continuity. Well, suppose that we have a converging net $\{\ell_i\}\to\ell$ in $C(Y)$. We would like to show that $\{\ell_i\circ f\}\to(\ell\circ f)$ in $C(X)$. Convergence is equivalent to converging uniformly on compacts, so we choose a compact set $K\subseteq X$, and we would like to check that
	\[\norm{(\ell_i\circ f-\ell\circ f)|_K}_\infty\stackrel?\to0.\]
	Well, this norm is bounded by $\norm{(\ell_i-\ell)|_{f(K)}}_\infty$, which goes to $0$ because $f(K)$ is compact and $\{\ell_i\}\to\ell$ uniformly on compact sets.
\end{proof}

\section{Categorical Properties}
For my own personal use, we review some categorical properties of certain full subcategories of the category of topological vector spaces, which is equipped with continuous linear maps for its morphisms. We recall that all topological vector spaces are assumed to be Hausdorff.

Because they are easier, let's begin with some limits.
\begin{definition}[kernel]
	Fix a continuous map $f\colon V\to W$ of topological vector spaces. Then we define the \textit{kernel} $\ker f$ as the vector space kernel equipped with the subspace topology.
\end{definition}
\begin{remark} \label{rem:kernel-exists}
	Because $\ker f$ is a subspace, we see that it continues to be a Hausdorff topological space. In fact, it is the categorical kernel: it is already the kernel in the category of vector space, so any map $V'\to V$ vanishing under $W$ factors via a linear map into $\ker f$. But $\ker f$ is just a subspace, so the induced map $W\to V'$ is continuous.
\end{remark}
\begin{remark} \label{rem:kernel-complete}
	Note that $\ker f\subseteq V$ is closed because $f$ is continuous, so if $V$ is complete, we see that $\ker f$ is also complete.
\end{remark}
\begin{remark} \label{rem:kernel-frechet}
	If $V$ is locally convex, then the subspace topology inherits the exact set of seminorms, so $\ker f$ is also locally convex. Once we combine this argument with \Cref{rem:kernel-complete}, we see that $V$ being Fr\'echet implies that $\ker f$ is Fr\'echet.
\end{remark}
\begin{definition}[product]
	Fix a family $\{V_i\}_{i\in I}$ of topological vector spaces. Then we define the \textit{product} $\prod_{i\in I}V_i$ as the vector space product equipped with the product topology.
\end{definition}
\begin{remark} \label{rem:product-exists}
	When checking that $\prod_{i\in I}V_i$ is a categorical product, we note that the maps produced by being the vector space product and being the topological product are the same, so the result follows.
\end{remark}
\begin{remark} \label{rem:product-complete}
	If each $V_i$ is complete, then we claim that the product $\prod_{i\in I}V_i$ is still complete. Indeed, given any Cauchy net $\{(v_{ij})_i\}_{j\in I}$, we see that each projection $\{v_{ij}\}_{j\in I}$ continues to be a Cauchy net and therefore has a limit $\{v_{ij}\}\to v_j$. But now having each projection converge means that $\{(v_{ij})\}_{j\in I}\to(v_j)_j$.
\end{remark}
\begin{remark} \label{rem:product-frechet}
	If each $V_i$ is locally convex with its topology determined by some family of seminorms $\{p_{i\alpha}\}_\alpha$, then the definition of the product means that the topology on the product are determined by the seminorms
	\[\prod_{i\in I}V_i\stackrel{\op{pr}_j}\onto V_j\stackrel{p_{j\alpha}}\to\RR.\]
	This argument, when combined with \Cref{rem:product-complete}, shows that a countable product of Fr\'echet spaces continues to be Fr\'echet.
\end{remark}
\begin{proposition}
	Fix a diagram $\{V_i\}_{i\in I}$ of topological vector spaces. Then the limit
	\[\lim_{i\in I}V_i\]
	exists in the category of topological vector spaces. If each $V_i$ is locally convex, so is the limit. If $I$ is countable and each $V_i$ is Fr\'echet, then so is the limit.
\end{proposition}
\begin{proof}
	Existence follows from \Cref{rem:kernel-exists,rem:product-exists}. The remaining claims follow from \Cref{rem:kernel-frechet,rem:product-frechet}.
\end{proof}
\begin{example} \label{ex:c-x-frechet-cat-theory}
	We recover \Cref{ex:c-x-frechet}. Fix a locally compact topological space $X$. Then we claim that
	\[C(X)\stackrel?=\lim_{K\subseteq X}C(K),\]
	where the limit is taken over all compact subsets $K\subseteq X$. Indeed, there is a canonical map $f\mapsto(f|_K)_K$ from the left to the right; conversely, any tuple $(f_K)_K$ on the right extends to a unique function $f\in C(X)$ by construction of the limit. Lastly, we note that the map that we just described is actually a homeomorphism because a convergent net on the left $\{f_i\}\to f$ converges if and only if $\{f_i|_K\}\to f|_K$ uniformly for all compacts $K\subseteq X$, which is exactly the topology on the right-hand side.
	
	We conclude that $C(X)$ is locally convex; if $X$ is second countable, then we can replace the limit over all compact subsets with a countable ascending chain, thereby showing that the limit is also Fr\'echet.
\end{example}
We now turn to some colimits, which are harder.
\begin{definition}[quotient]
	Fix a closed subspace $W$ of a topological vector space $V$. Then we form the \textit{quotient} $V/W$ to be the vector space $V/W$ to be equipped with the quotient topology induced by the projection $V\onto V/W$.
\end{definition}
\begin{remark}
	We claim that this quotient is the categorical quotient. Indeed, for any topological vector space $V'$ and some map $f\colon V\to V'$ for which $W\subseteq\ker f$, we get an induced map $\overline f\colon V/W\to V'$, which we need to be continuous. Well, a map $\overline f\colon V/W\to V'$ is continuous if and only if the composite $V\onto V/W\to V'$ is continuous, which is true.
\end{remark}
\begin{lemma}
	Fix a closed subspace $W$ of a topological vector space $V$.
	\begin{listalph}
		\item If $V$ is locally convex, then $V/W$ is locally convex.
		\item If $V$ is Fr\'echet, then $V/W$ is Fr\'echet.
	\end{listalph}
\end{lemma}
\begin{proof}
	We show the parts separately.
	\begin{listalph}
		\item For each seminorm $p$ on $V$, we produce a continuous seminorm $\overline p\colon V/W\to\RR$ by
		\[\overline p(v+W)\coloneqq\inf_{w\in W}p(v+w).\]
		Because $p$ is a seminorm, we see that $\overline p$ is also a seminorm. Additionally, we note $\overline p$ is continuous: using \Cref{lem:seminorm-continuity-grab-bag}, it is enough to see that $\{v+W:\overline p(v+W)<1\}$ is open, which is true because this subset equals
		\[\{v+W:\overline p(v+W)<1\}\stackrel?=\{v:p(v)<1\}+W.\]
		Certainly any vector $v$ for which $p(v)<1$ has $\overline p(v)<1$. Conversely, if $\overline p(v+W)<1$, then there is $w\in W$ for which $p(v+w)<1$.

		Lastly, we have to check that these seminorms actually produce the correct topology on $V/W$. Well, the open sets of $V/W$ are exactly projections of the open sets of $V$, so we are done because the previous paragraph explains that the open sets produced by our seminorms $\overline p$ are exactly the projections of the open sets produced by the seminorms $p$ of $V$.

		\item The argument in (a) shows that $V/W$ has its topology determined by countable many seminorms, so it remains to show that $V/W$ is complete. Because $V/W$ is metrizable (\Cref{prop:locally-convex-to-metric}), it is enough to check that Cauchy sequences in $V/W$ converge. As such, choose a Cauchy sequence $\{v_i+W\}_{i\in\NN}$, which means that $\overline p(v_i-v_j+W)\to0$ as $i,j\to\infty$. It is enough to show that any subsequence of this Cauchy sequence converges, so we may as well assume that $\overline p(v_{i+1}-v_i+W)<2^{-i}$ for each $i$. By definition of $\overline p$, we may thus, find vectors $\{w_i\}$ for which
		\[\overline p(v_{i+1}+w_{i+1}-(v_i+w_i))<2^{-i+1}.\]
		Thus, the sequence $\{v_i+w_i\}_{i\in\NN}$ is rapidly Cauchy in $V$, so it has a limit to some vector $v\in V$. Projecting back down, we find that $\{v_i+w_i\}\to v$ implies that $\{v_i+W\}\to v+W$, so we are done.
		\qedhere
	\end{listalph}
\end{proof}
For our coproduct, we will retreat to locally convex vector spaces.
\begin{definition}[coproduct]
	Fix a family $\{V_i\}_{i\in I}$ of locally convex vector spaces. Then we define the \textit{coproduct} as given by the coproduct $\bigoplus_{i\in I}V_i$ of vector spaces, and we equip it with a topology whereby an open neighborhood basis of the origin takes the form
	\[\op{conv}\Bigg(\sum_{i\in I}j_i(U_i)\Bigg),\]
	where $j_i\colon V_i\to\bigoplus_{i\in I}V_i$ is the inclusion, $U_i\subseteq V_i$ is an open neighborhood of $0$ for each $i$. (This sum denotes the collection of any finite sum of vectors.)
\end{definition}
\begin{remark}
	Let's verify that we have produced a categorical coproduct: given continuous inclusions $V_i\to W$ for each $i\in I$ into some locally convex vector space $W$, we need to check that the algebraically induced linear map $\bigoplus_{i\in I}V_i\to W$ is continuous. Well, it's enough to check continuity on an open neighborhood basis of the identity, so we note that the pre-image of some convex open neighborhood $U$ of $W$ consists of the finitely supported sequences $(v_i)_{i\in I}$ where $v_i\in j_i^{-1}(U)$ for each $i$. The given open neighborhood basis does in fact cover such an open subset!
\end{remark}
For our application, we want to know something about the dual.
\begin{definition}[dual]
	Fix a topological vector space $V$ over $\FF$. Then we define the \textit{topological dual} $V^*$ as the vector space of all continuous linear functionals $V\to\FF$. We make $V^*$ into a topological vector space by equipping it with the weak topology such that the evaluation maps $\op{ev}_v\colon V^*\to\RR$ are continuous for every $v\in V$.
\end{definition}
\begin{remark}
	Given a continuous linear map $f\colon V\to W$, we see that composition provides a map $(-\circ f)\colon W^*\to V^*$. We may also call this map $f^*$, and it makes $(-)^*$ into a functor.
\end{remark}
\begin{lemma} \label{lem:dual-exact}
	Given an exact sequence
	\[0\to A\to B\to C\to0\]
	of topological vector spaces, the sequence
	\[0\to C^*\to B^*\to A^*\to0\]
	is exact if $B$ is Fr\'echet.
\end{lemma}
\begin{proof}
	We show exactness by hand.
	\begin{itemize}
		\item Exact at $C^*$: by construction of the quotient $C=B/A$, we see that the map $B\to C$ is surjective. Thus, a functional $\ell\in C^*$ vanishes if and only if the composite $B\to C\to\FF$ vanishes.
		\item Exact at $A^*$: we are being asked to show that any continuous linear functional on a closed subspace $A\subseteq B$ extends continuously to $B$, which is exactly \Cref{cor:h-b-extend-cont}.
		\item Exact at $B^*$: we are given a continuous linear functional $\ell$ on $B$ which vanishes on $A$. Thus, $\ell$ factors through as a continuous linear functional on $C=B/A$ because we have a categorical quotient, so we are done.
		\qedhere
	\end{itemize}
\end{proof}
\begin{lemma} \label{lem:dual-coprod-to-prod}
	Fix a family $\{V_i\}_{i\in I}$ of locally convex topological vector spaces. Then the canonical map
	\[\bigoplus_{i\in I}V_i^*\to\Bigg(\prod_{i\in I}V_i\Bigg)^*\]
	is a continuous injection. If the family is countable, then it is an isomorphism.
\end{lemma}
\begin{proof}
	The canonical map is induced by dualizing the canonical map $\prod_{i\in I}V_i\to\bigoplus_{i\in I}V_i$, which in turn is induced by the projections $\op{pr}_j\colon\prod_{i\in I}V_i\to V_j$. Thus, we see that the map sends a finitely supported family $(\ell_i)_{i\in I}$ of linear functionals to the linear functional on $\prod_{i\in I}V_i$ defined by
	\[(v_i)_i\mapsto\sum_{i\in I}\ell_i(v_i).\]
	Here are the remaining checks on this map, which we call $\alpha$.
	\begin{itemize}
		\item Injective: if this target linear functional vanishes, then the inclusions $V_j\to\prod_{i\in I}V_i$ allows us to check that each individual linear functional in $(\ell_i)_{i\in I}$ vanishes.
		\item Surjective if $I$ is countable: fix a continuous linear functional $\ell$ on $\prod_{i\in I}V_i$, and we define $\ell_i$ by composing $\ell$ with the inclusion $V_i\to\prod_{i\in I}V_i$. It remains to show that $\ell=\alpha((\ell_i)_{i\in I})$. For this, we identify $I$ with $\NN$. Then for any vector $(v_i)_{i\in\NN}\in\prod_{i\in\NN}V_i$, we note that the sequence of vectors
		\[w_n\coloneqq\sum_{i=0}^nv_i\]
		approaches $v$ as $n\to\infty$, which we can see by definition of the product topology. It follows that $\ell(v)$ is the limit of the $\ell(w_n)$, which equals
		\[\ell(w_n)=\sum_{i=0}^n\ell_i(v_i).\]
		Thus, it remains to show that $\ell_i$ vanishes for all but finitely many $i\in\NN$. Well, otherwise, we can select $v_i$ so that $\ell_i(v_i)=1$ for infinitely many $i\in\NN$, from which we find that the sequence $\{\ell(w_n)\}_{n\in\NN}$ fails to converge.
		% \item Isomorphism if $I$ is countable: it remains to show that the inverse map is continuous if $I$ is countable. By the previous check, the inverse map is just given by sending $\ell\in\left(\prod_{i\in I}V_i\right)^*$ to the tuple $(\ell\circ j_i)_{i\in I}$, which we showed is in $\bigoplus_{i\in I}V_i^*$. To check that this is continuous, it is enough to check that the individual maps $\left(\prod_{i\in I}V_i\right)^*\to V_i^*$ are continuous.
		\item Open if $I$ is countable: it is enough to check that $\alpha$ is open at $0$. As such, we choose a family of open neighborhoods $U_i\subseteq V_i^*$ of $0$ for each $i\in I$, and we need to show that
		\[\op{conv}\Bigg(\sum_{i\in I}j_i(U_i)\Bigg)\]
		contains an open neighborhood of $0\in\left(\prod_{i\in I}V_i\right)^*$. Fix our finite subset $S\subseteq I$ so that $U_i$ is nonempty exactly when $i\in S$. Now, $U_i\subseteq V_i^*$ is open in the weak topology, so there is a finite family of vectors $\{v_{ij}\}_{j\in S_i}$ so that $U_i$ contains an open neighborhood of the form
		\[\bigcap_{j\in S_i}\{\ell:\left|\ell(v_{ij})\right|<\varepsilon_i\}\]
		for some fixed very small $\varepsilon_i>0$. This means that the image of our open set in $\left(\prod_{i\in I}V_i\right)^*$ is the convex hull of some finite sum of such vectors. But if we select any finite number of these conditions, we see that we cut out an open set in the weak topology of $\left(\prod_{i\in I}V_i\right)^*$, so we are done!
		\qedhere
	\end{itemize}
\end{proof}
\begin{proposition}
	Fix a countable diagram $\{V_i\}_{i\in I}$ of Fr\'echet spaces. Then the canonical map
	\[\colim_{i\in I}V_i^*\to\left(\lim_{i\in I}V_i\right)^*\]
	is an isomorphism.
\end{proposition}
\begin{proof}
	This follows from combining \Cref{lem:dual-exact,lem:dual-coprod-to-prod}.
\end{proof}
\begin{example} \label{ex:c-x-as-colim}
	Continuing \Cref{ex:c-x-frechet-cat-theory}, we see that
	\[C(X)^*=\colim_{K\subseteq X}C(K)^*,\]
	where the induced maps $C(K)^*\to C(X)^*$ are given by sending $\mu\in C(K)^*$ to the map $f\mapsto\mu(f|_K)$.
\end{example}

\end{document}