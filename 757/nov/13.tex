% !TEX root = ../notes.tex

\documentclass[../notes.tex]{subfiles}

\begin{document}

\section{November 13}
Today, we continue discussing characters. We began class by proving an estimate from last class, which I have added there.

\subsection{Regularity of Characters}
We are interested in proving the following regularity result.
\begin{proposition}
	Fix a Hilbert space representation $V$ of $G$. Then $\chi_V$ is analytic when restricted to the regular semisimple locus $G^{\mathrm{rs}}$ of $G$.
\end{proposition}
In fact, we will show the following stronger result.
\begin{proposition} \label{prop:eigendistrib-is-analytic}
	Fix a conjugation-invariant $Z\mf g$-eigendistribution $\delta$. Then $\delta|_{G^{\mathrm{rs}}}$ is analytic.
\end{proposition}
Here, being an eigendistribution refers to the $Z\mf g$-action given by extending the action $\mf g\cong\op{Vect}(G)^G$ to an action $Z\mf g\subseteq U\mf g\to\op{Diff}G$, and $\op{Diff}G$ is able to act on $\op{Dist}G\subseteq C_c^\infty(G)^*$.

This proposition will in turn follow from the following analytic result.
\begin{proposition}
	A distribution on a complex manifold $X$ satisfying a holonomic system of differential equations is given by an analytic function.
\end{proposition}
\begin{proof}
	Omitted. It is perhaps easier than our elliptic result from earlier in the course. Roughly speaking, one can do some favorable coordinate changes in order to make the holonomic system simply asserting that the distribution has certain derivatives vanishing, which amounts to asking for the function to be constant in those directions.
\end{proof}
\begin{proof}[Proof of \Cref{prop:eigendistrib-is-analytic}]
	We will check that $\delta$ is cut out by a holonomic system of differential equations. Fix some regular semisimple $g$, and identify $T_g^*G=\mf g$ as usual, and we see that $\CC[T_gG]=\CC[\mf g]$. Now, for each $z\in Z\mf g$, we get some $z\in\CC[\mf g]$, and our distribution is cut out by $z-\lambda$ for some $\lambda$, and we would like to show that these cut out a holonomic system when combined with the differential operators given by $[x,\mf g]$ (corresponding to being conjugation-invariant). Choosing $x$ to be regular semisimple reveals that our roots must be orthogonal to $\mf t^\perp$ and therefore live in $\mf t$.

	Thus, the main part is to compute $\overline z|_{\mf t}$. Note we may as well take $\overline z$ to be given by one of the various $G$-invariant polynomials $P$. By the Harish-Chandra isomorphism, we may as well compare this to the differential of $\mf t\onto\mf t/W$, so we may consider the action of $Z$ on $\CC[G]^G$ spanned by the known characters $\chi_{V_\lambda}$. More precisely, we note that we have a composite
	\[\CC[G]^G\cong\CC[T]^W\subseteq\CC[T].\]
	Note $\CC[G]^G$ has an action by $Z\mf g$, so on $T$ we have a differential-invariant differential operator $d_p\in\CC[\mf t^*]\subseteq\op{Diff}T$. On the other hand, we have a Weyl character formula
	\[\chi_{V_\lambda}=\frac1\delta\sum_wt^{w(\lambda+\rho)},\]
	and the operator $d_p$ acts by $\delta^{-1}d_p\delta$. Looping over all $\overline z_{p,x}|_{\mf t}$ gives all possible $\ov d_{p,x}$, so we get all $W$-invariant polynomials, which have no common zero anyway.
\end{proof}
\begin{example}
	Consider $G=\op{SL}(2,\RR)$, and we compute $\chi_V(g)$ where $g$ is an element in $\op{SO}(2,\RR)$; we may as well consider $g$ computed against some $z\in S^1$. Now, the principal series representations $P_+(s)$ have
	\[\sum_{n\in2\ZZ}z^n,\]
	which vanishes automatically for $z\notin\{\pm1\}$, so we get a linear combination of $\delta_1$ and $\delta_{-1}$. For $L_m$, we can compute that we get $z^m/\left(1-z^2\right)$ by summing a geometric series.
\end{example}
% Here, holonomic means that we are cut out as the vanishing set of some differential operators $\{D_i\}$ such that the symbols $\overline D_i$ of each $D_i$ has its zero set in $T^*_xX_\CC$ given by $\{0\}$ for all $x\in X$.
% \begin{lemma}
% 	Fix a Lie group $G$, and consider the map $Z\mf g\to\op{Diff}G$ sending $z$ to $dz$. Then the zero set of the symbol $dz$ on $T^*G$ is $\mf t^\perp$ for a regular semisimple $g$.
% \end{lemma}
% This will then show that $\chi_V|_{G^{\mathrm{rs}}}$ satisfies a holonomic system.

\subsection{The Cartan Decomposition}
We now aim to explain what is meant by ``discrete series.'' Fix a real group $G$, and we choose a maximal split torus $A\subseteq G$. It turns out that the Lie algebra $\mf a$ is maximal abelian in $\mf k^\perp$, where $K\subseteq G$ is a maximal compact subgroup. Its centralizer $\mf z(\mf a)$ is some Levi, so we may choose a real parabolic $P$ with Levi subgroup $L$ with the correct Lie algebra. It turns out that the semisimple part of $L$ is compact: non-compact reductive groups have a nontrivial split torus, which would violate the maximality of $A$.

Now, writing $\mf p=\mf l\oplus\mf u$, we define $\mf a_+\subseteq\mf a$ to be the subspace with positive eigenvalues from $\mf u$.
\begin{proposition}[Cartan decomposition]
	One has
	\[G=KA^-K,\]
	where $A^-=\{\exp(-x):x\in\mf a_+\}$.
\end{proposition}
\begin{example}
	Take $G=\op{SL}(2,\RR)$ so that $K=\op{SO}(2,\RR)$. Then $G/K$ is just the upper-half plane $\mc H$, where the natural action by $G$ preserves the negative curvature metric. Then we would like to show that $KA^-\onto G/K$, which amounts to showing that we can get any point in $\mc H$ by first scaling it appropriately by $A^-=\{\op{diag}(a,1/a):a>0\}$ and then taking an orbit along $K$. Roughly speaking, the action by $A$ translates the necessary distance, and then $K$ rotates along points equidistant from $i$.
\end{example}
\begin{example}
	Take $G=\op{SL}(n,\RR)$ so that $K=\op{SO}(n,\RR)$. Then $G/K$ parameterizes positive-definite quadratic forms: for every pair of quadratic forms, one can find a basis which is orthogonal for both. More explicitly, if we let $Q_1$ be the standard quadratic form, then any other form $Q_2$ can be diagonalized by an orthonormal basis for $Q_1$. (This identification is known as the principal axis reduction.) This manipulation has basically shown that $KA^-$ surjects onto $G/K$.
\end{example}
\begin{remark}
	Because $K$ is compact, asymptotics of a function $f$ on $G$ is controlled by $f|_{A^-}$. For matrix coefficients, this reduces to an algebraic calculation of some eigenvalues of $\mf a$ on $M/\mf u^nM$.
\end{remark}

\end{document}