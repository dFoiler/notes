% !TEX root = ../notes.tex

\documentclass[../notes.tex]{subfiles}

\begin{document}

\section{November 25}
Today we continue discussing the discrete series.

\subsection{The Discrete Series}
Throughout, $G$ is a real reductive group, and we choose letters $P=MAN$ as usual.
\begin{theorem} \label{thm:discrete-series}
	Let $G$ be a real reductive group, and let $V$ be an irreducible admissible representation of $G$. Then the following are equivalent.
	\begin{listroman}
		\item All matrix coefficients of $V$ are in $L^2(G)$.
		\item All exponents $\lambda$ of $V$ have $\lambda<-\rho$, where $\rho$ is the half-sum of the roots of $\mf a$ in $\mf n$. Here, $\lambda<-\rho$ means that $-\rho-\lambda$ is a nonzero sum of roots of $\mf a$ in $\mf n$ with nonnegative coefficients.
	\end{listroman}
\end{theorem}
\begin{remark}
	By definition of $\rho$, one has that $e^{2\rho}(a)=\det(e^a;\mf n)$ for $a\in\mf a$.
\end{remark}
\begin{remark}
	The notation $\lambda<-\rho$ is justified as the usual partial order on weights. Indeed, there is some non-reduced root system given by the action of $\mf a$.
\end{remark}
\begin{example}
	Take $G=\op{SL}(2,\RR)$, and we will recover the discrete series. Here, $K=\op{SO}(2)$. Let $\mf b=\mf a\oplus\mf n$ be a Borel subalgebra of $\mf g$ in (distinct from the one giving rise to $\mf k$), and one finds that $L_m^+$ is the induction $U\mf g\otimes_{\mf b}\CC_m$. It then turns out that $L_m^+/\mf n^dL_m^+$ has $\mf a$-weights $\{m,m+2,\ldots,m(d-1)\}$, so it satisfies the criteria of \Cref{thm:discrete-series} if and only if $m>1$.
\end{example}
\begin{example}
	Consider the group $\op{SL}(2,\RR)$. Then 
\end{example}
\begin{proof}[Proof of \Cref{thm:discrete-series}]
	For simplicity, we will assume that the group is split, but it is not so important. The proof proceed in steps.
	\begin{enumerate}
		\item We set up some calculations. Note that $\dim A=\op{rank}G$ and $\dim K=\#\Phi^+$; the second equality follows by hunting for $\theta$-invariant subspaces in the decomposition $\mf g=\mf u_-\oplus\mf a\oplus\mf u_+$, where we know that $\theta$ swaps $\mf u_-$ and $\mf u_+$ and acts by $-1$ on $\mf a$.

		The moral is that the map $c\colon K\times A\times K\to G$ should be generically a diffeomorphism. To see this, we note that the differential at some $a\in A$ is the map $\mf k\oplus\mf a\oplus\mf k\to\mf g$ defined by
		\[dc_a\colon(x,y,z)\mapsto\op{Ad}_a(x)+y+z.\]
		Indeed, we can apply right translations to move the tangent space for $z$ to the identity, everything commutes at the $y$ coordinate, and the $x$ coordinate needs to use left translation to move to the identity, but this adds an adjoint action.
		
		Now, recall that $\theta$ acts on $\mf a$ by $-1$, so $\theta$ will send $\mf u^+\oplus\mf a$ to $\mf u^-\oplus\mf a$. It follows that
		\[\mf k=\{n+\theta(n):n\in\mf u\}\]
		by hunting for $\theta$-invariants. As such, we note that $\op{Ad}_a(n)\to0$ as $a\to\infty$, so we may ignore this term. It follows that
		\[\det dc|_a\sim\det(\mathrm{Ad}_a;\mf u^-)=e^{2\rho}(a).\]

		\item We bound the asymptotics of the matrix coefficients (on $A^+$) in terms of the exponents. To start, choose smooth $K$-finite vectors $v\in V$ and $v^*\in V^\lor$; they vanish on all but finitely many $K$-isotypic components. Now, a given $e^a\in A$ will act by a bounded operator, so
		\[m_{v,v^*}(e^a)=\langle v^*,gv\rangle\lesssim_{v,v^*}\norm{e^a}.\]
		But now $a\in\mf a^+$ can be expanded in local coordinates as a sum of simple roots, so
		\[m_{v,v^*}(e^a)\lesssim_{v,v^*}e^{R\langle a,\rho\rangle}\]
		for some $R>0$.
		
		Now, for any $n\in\mf n$, we may write $v'=nv$, and then there is a better bound. To see this, we may as well assume that $n\in\mf g_\alpha$ for some root $\alpha$, and then we write
		\begin{align*}
			m_{v',v^*}(e^a) &= \langle v^*,e^anv\rangle \\
			&= \langle v^*,\op{Ad}_{e^a}(n)(av)\rangle \\
			&= \langle v^*,e^{\alpha(a)}nav\rangle \\
			&= e^{\alpha(a)}m_{(v^*)',v}(e^a),
		\end{align*}
		where $(v^*)'=-n(v^*)$. Because $\alpha(a)<0$, we have a smaller bound: our right-hand side is bounded above by $e^{\langle a,R\rho+\alpha\rangle}$ up to a constant.

		Iterating the above process, we see that we can adjust our vectors so that the matrix coefficient is smaller than an arbitrary exponential.

		\item For technical reasons, we complete the argument only in the case of $G=\op{SL}(2,\RR)$. (For higher-rank situations, one has to handle interactions between roots, which is not difficult but is technical.) In particular, for any $r$, we can choose $v$ and $v^*$ so that $m_{v,v^*}(e^a)\lesssim e^{r\langle a,\rho\rangle}$.
		
		Now, the first step tells us that $V$ is in $L^2$ if and only if some matrix coefficient satisfies the bound
		\[m_{v,v^*}(e^a)\lesssim e^{ta}\]
		for some $t<-1$. To see this, note that being in $L^2$ is equivalent to having some nontrivial map to $L^2(G)$, which may as well be chosen as a matrix coefficient. Thus, we are left to checking the convergence of
		\[\int_G\left|m_{v,v^*}(g)\right|^2\,dg.\]
		Using the Cartan decomposition and the first step, this becomes
		\[\int_{K\times A^+\times K}\left|m_{v^*,v}(k_1ak_2)\right|^2e^{2\rho}(a)\,dk_1\,da\,dk_2.\]
		The integrals over $K$ do not change convergence because they are compact, so we are only interested in the convergence of the integral over $A^+$, whence the claim upon computing $e^{2\rho}$.

		We now turn to understanding the exponents of $V_{\mathrm{fin}}/\mf u^nV_{\mathrm{fin}}$. This space is finite-dimensional and admits $\mf a$ action, which we left to an $A$-action on $V_{\mathrm{sm}}/\mf u^nV_{\mathrm{sm}}$. Now, the image of
		\[V_{\mathrm{fin}}/\mf u^nV_{\mathrm{fin}}\to V_{\mathrm{sm}}/\mf u^nV_{\mathrm{sm}}\]
		is forced to be $A$-invariant, and the $A$-action of the image must integrate the $\mf a$-action. Now, by definition, these exponents $\lambda$ are the eigenvalues of the $\mf a$-action, so they become the eigenvalues of the $A$-action, so we are left to ask for the exponents to be bounded above by $-1$ by the previous paragraph. This completes the proof.
		\qedhere
		% Note that bounding on $A^+$ is enough by using the Cartan decomposition: for any $f\in L^2(G)$, we see that
		% \[\int_G\left|f\right|^2\,dg=\int_{A^+}\widetilde f(g)\,da,\]
		% where $\widetilde f$ is some $K\times K$ average of $f$ (up to some adjustment of measures). Taking an average over a compact group is okay, but keeping track of the measures will be mildly technical. Thus, one finds that we need to estimate some quotient $dg/dk_1\,da\,dk_2$, and it will be about $e^{-\rho(a)}$.
	\end{enumerate}
\end{proof}
\begin{definition}[discrete series]
	Fix a real reductive group $G$. Then an irreducible admissible representation $V$ of $G$ is in the \textit{discrete series} if and only if it satisfies one of the equivalent conditions of \Cref{thm:discrete-series}.
\end{definition}
\begin{remark}
	For \Cref{thm:discrete-series}, it is enough to merely check the exponents acting on $V_{\mathrm{fin}}/\mf uV_{\mathrm{fin}}$. Indeed, the point is that the eigenvalues are decreasing as we pass to higher powers of $\mf u$. To see this, we argue inductively: additional eigenvalues on $V_{\mathrm{fin}}/\mf u^{i+1}V_{\mathrm{fin}}$ will come from $\mf u^iV_{\mathrm{fin}}/\mf u^{i+1}V_{\mathrm{fin}}$. But the surjection
	\[\mf u\otimes\op{gr}_iV_{\mathrm{fin}}\onto\op{gr}_{i+1}V_{\mathrm{fin}}\]
	tells us that the eigenvalues $\lambda$ for $\mf u^iV_{\mathrm{fin}}/\mf u^{i+1}V_{\mathrm{fin}}$ can be written as $\alpha+\lambda'$ where $\alpha$ is a root of $A$ acting on $\mf u$, and $\lambda'$ is an eigenvalue for $\mf u^{i-1}V_{\mathrm{fin}}/\mf u^{i}V_{\mathrm{fin}}$.
\end{remark}
\begin{definition}[tempered]
	Fix a real reductive group $G$. Then an irreducible admissible representation $V$ of $G$ is \textit{tempered} if and only if all exponents $\lambda$ of $V$ have $\lambda\le-\rho$, where notation is as in \Cref{thm:discrete-series}.
\end{definition}
\begin{remark}
	By the same argument as in \Cref{thm:discrete-series}, it turns out that being tempered is equivalent to $V$ being found in $L^{2+\varepsilon}(G)$ for all $\varepsilon>0$.
\end{remark}
\begin{example}
	It turns out that unitary induction sends tempered representations to tempered representations. However, it does not send discrete series representations to discrete representations.
\end{example}
\begin{remark}
	It turns out that a representation is tempered if and only if it is a summand of some unitary induction of a discrete series representation. Note that these inductions are unitary and hence semisimple.
\end{remark}

\subsection{The Langlands Classification}
There is a classification of all admissible representations (up to infinitesimal equivalence) in terms of tempered representations. This is known as the Langlands classification. Let's say more about this.
\begin{theorem}[Langlands classification] \label{thm:langlands-classification}
	Fix a real reductive group $G$. For every admissible irreducible representation $V$, there is a unique (up to conjugacy) triple $(P,\rho,)$, where $P\subseteq G$ is a parabolic, $\rho$ is a tempered distribution of the Levi $L\subseteq P$, and a dominant real-valued character $\lambda\colon L\to\RR^+$. Here, dominant means that $\langle\lambda,\alpha^\lor\rangle>0$ for each positive coroot $\alpha^\lor$.
\end{theorem}
The Langlands classification takes the triple $(P,\rho,\lambda)$ to an irreducible quotient of the parabolic induction of $\rho\otimes e^\lambda$. We also remark that there is a left adjoint $r^G_P$ to parabolic induction $i^G_P$ which simply takes $V_{\mathrm{fin}}$ to $V_{\mathrm{fin}}/\mf uV_{\mathrm{fin}}$.
\begin{example}
	It turns out that $V_{\mathrm{fin}}/\mf uV_{\mathrm{fin}}$ is always nonzero when $V_{\mathrm{fin}}$ is. Indeed, $V_{\mathrm{fin}}$ being nonzero yields an embedding $M\into i_P^GN$. But the vanishing of $V_{\mathrm{fin}}/\mf uV_{\mathrm{fin}}$ will force $N$ to be finite-dimensional, which causes problems by passing to finite-dimensional theory. This is related to the Casselman submodule theorem.
\end{example}
The proof of the Langlands classification rests on the following combinatorial lemma.
\begin{lemma}[Langlands] \label{lem:langlands}
	Fix a real reductive group $G$, and let $\Lambda$ be the weight lattice.
	% Set $\mf t_\RR^*\coloneqq\Lambda\otimes_\ZZ\RR$, and let $\Lambda^+$ denote the dominant weights. Further, let $R^-$ denote the cone
	Further, set
	\[(\mf t_\RR^*)^+\coloneqq\{\lambda:(\lambda,\alpha)\ge0\text{ for all positive }\alpha\},\]
	and
	\[(\mf t_\RR^*)^-\coloneqq\left\{\sum_{i}a_i\alpha_i:a_i\le0\text{ for all }i\right\}.\]
	% \[\left\{\sum_{i}n_i\alpha_i:n_i\le0\text{ for all }i\right\}.\]
	Then every $\lambda\in\mf t_\RR^*$ can be written uniquely as a sum $\lambda_1+\lambda_2$ such that $\lambda_1\in(\mf t_\RR^*)^+$ and $\lambda_2\in(\mf t_\RR^*)^-$ and $(\lambda_1,\lambda_2)=0$.
\end{lemma}
\begin{proof}[Sketch]
	Simply let $\lambda_1$ be the closest point to $\lambda$ in $(\mf t_\RR^*)^+$.
\end{proof}
\begin{proof}[Sketch of \Cref{thm:langlands-classification}]
	We argue for split groups, for simplicity. Choose a minimal exponent $\lambda$ of $V_{\mathrm{fin}}$ in the standard partial order.
	\begin{enumerate}
		\item To determine the desired parabolic, we write $\op{Re}\lambda=\lambda_1-\lambda_2$ by \Cref{lem:langlands}, and we choose $P$ as large as possible so that $\lambda_1$ is a character of $L$, meaning that $\op{Lie}L$ is generated by $\mf a$ and the roots $\alpha$ for which $(\alpha,\lambda_1)=0$.
		\item Next, to recover the tempered representation of $L$, consider the submodule of $r^G_P(V_{\mathrm{fin}})$ where the center of $L$ acts by $\lambda_1$, and this turns out to be tempered up to twisting by $e^{\lambda_1}$.
		\qedhere
	\end{enumerate}
\end{proof}

\end{document}