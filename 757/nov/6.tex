% !TEX root = ../notes.tex

\documentclass[../notes.tex]{subfiles}

\begin{document}

\section{October 30}
Today, we move on to character theory. We started class with some facts about the nilpotent cone, which I have moved to the appropriate section of the notes.

\subsection{Constructing Characters}
Our characters will be a certain kind of trace, so we need to explain why we can take some traces of infinite-dimensional representations.
\begin{definition}[Hilbert--Schmidt]
	Fix a bounded operator $A$ on a Hilbert space $V$, and choose a topological orthonormal basis $\{v_i\}$ of $V$. Then we say that $A$ is \textit{Hilbert--Schmidt} if and only if
	\[\sum_i\left|Av_i\right|^2<\infty.\]
\end{definition}
\begin{remark}
	If $A$ is defined on a dense subset of $V$ but still satisfies the inequality, then it will extend to a Hilbert--Schmidt operator on $V$.
\end{remark}
\begin{definition}[trace class]
	Fix a bounded operator $A$ on a Hilbert space $V$, and choose a topological orthonormal basis $\{v_i\}$ of $V$. Then $A$ is \textit{trace class} if and only if
	\[\sum_i(Ev_i,v_i)<\infty.\]
\end{definition}
\begin{remark} \label{rem:hs-to-trace-class}
	If $A$ and $B$ are both Hilbert--Schmidt, then $AB$ is trace class.
\end{remark}
\begin{proposition}
	Fix an irreducible Hilbert space representation $V$ of a Lie group $G$. Then any $f\in C_c^\infty(G)$ acts by a trace class operator on $V$.
\end{proposition}
\begin{proof}
	The idea is to use the Casimir operators $C_k$ for $K$. In particular, we may write
	\[V_{\mathrm{fin}}=\bigoplus_\nu W_\nu\otimes V_\nu,\]
	where $\nu$ varies over irreducible representations $V_\nu$ of $K$, and $W_\nu$ is the multiplicity space.

	It follows from the proof of Harish-Chandra admissibility that
	\[\dim W_\nu\le\dim V_\nu\]
	in our situation. By the representation theory of Lie algebras, one can check that $\dim V_\nu$ is $O\left(\left|\nu\right|^d\right)$, where $d$ is the number of positive roots of $\mf k$. Now, the Casimir element $C_{\mf k}$ acts by a scalar $c_\nu$, which is positive and of the order $\left|\nu\right|^2$.

	We now claim that the operator $\left(1+C_{\mf k}\right)^{-N}$ is Hilbert--Schmidt for large $N$, where we are only defining this operator a priori on $V_{\mathrm{fin}}$. Indeed, by combining the above bounds, it is enough to see that
	\[\sum_{\nu\in\Lambda_{\mf k}^+}\frac{(\dim V_\nu)^2}{\left|\nu\right|^{2N}}\le\sum_{\nu\in\Lambda_{\mf k}^+}\frac{\left|\nu\right|^{2d}}{\left|\nu\right|^{2N}}\]
	converges, which occurs as soon as (say) $N>d+1$.

	Now, write our $f\in C_c^\infty(G)$ as
	\[f=\left(1+C_{\mf k}\right)^{-N}\cdot\left((1+C_{\mf k})^{-N}\cdot(1+C_{\mf k})^{2N}f\right).\]
	Notably, $(1+C_{\mf k})^{2N}f$ is just some function in $C_c^\infty(G)$. This is then passed through a product of Hilbert--Schmidt operators, so it is trace class by \Cref{rem:hs-to-trace-class}.
\end{proof}
\begin{definition}[character]
	Fix a Hilbert space representation $V$ of $G$. Then we define the \textit{character} $\chi_V$ to be the distribution on $G$ for which
	\[\langle\chi_V,f\,dg\rangle\coloneqq\tr\left(f\,dg;V\right).\]
\end{definition}
\begin{remark}
	If $V$ and $V'$ are infinitesimally equivalent, then their characters agree. Indeed, everything in the definition can be passed down to $V_{\mathrm{fin}}\cong V'_{\mathrm{fin}}$, so the result follows from continuity.
\end{remark}

\subsection{Regularity of Characters}
We are interested in proving the following regularity result.
\begin{proposition}
	Fix a Hilbert space representation $V$ of $G$. Then $\chi_V$ is regular when restricted to the regular semisimple locus $G^{\mathrm{rs}}$ of $G$.
\end{proposition}
We need the following lemma.
\begin{lemma}
	Fix a Lie group $G$, and consider the map $Z\mf g\to\op{Diff}G$ sending $z$ to $dz$. Then the zero set of the symbol $dz$ on $T^*G$ is $\mf t^\perp$ for a regular semisimple $g$.
\end{lemma}
This will then show that $\chi_V|_{G^{\mathrm{rs}}}$ satisfies a holonomic system.

\end{document}