% !TEX root = ../notes.tex

\documentclass[../notes.tex]{subfiles}

\begin{document}

\section{November 6}
Today, we move on to character theory. We started class with some facts about the nilpotent cone, which I have moved to the appropriate section of the notes.

\subsection{Constructing Characters}
Our characters will be a certain kind of trace, so we need to explain why we can take some traces of infinite-dimensional representations.
\begin{definition}[Hilbert--Schmidt]
	Fix a bounded operator $A$ on a Hilbert space $V$, and choose a topological orthonormal basis $\{v_i\}$ of $V$. Then we say that $A$ is \textit{Hilbert--Schmidt} if and only if
	\[\sum_i\left|Av_i\right|^2<\infty.\]
\end{definition}
\begin{remark}
	If $A$ is defined on a dense subset of $V$ but still satisfies the inequality, then it will extend to a Hilbert--Schmidt operator on $V$.
\end{remark}
\begin{definition}[trace class]
	Fix a bounded operator $A$ on a Hilbert space $V$, and choose a topological orthonormal basis $\{v_i\}$ of $V$. Then $A$ is \textit{trace class} if and only if
	\[\sum_i(Ev_i,v_i)<\infty.\]
\end{definition}
\begin{remark} \label{rem:hs-to-trace-class}
	If $A$ and $B$ are both Hilbert--Schmidt, then $AB$ is trace class.
\end{remark}
\begin{lemma} \label{lem:upper-bound-multiplicity}
	Fix $M$ an irreducible $(\mf g,K)$-module, and expand $M=\bigoplus_\nu(W_\nu\otimes V_\nu)$ into irreducibles $V_\nu$ of $K$. Then there is a constant $C_M$ for which
	\[\dim W_\nu\le C_M\dim V_\nu.\]
\end{lemma}
\begin{proof}
	Start with some $V_\nu\subseteq V$, so we receive a surjection $U\mf g\otimes_{U\mf k} V_\nu\onto V$ because $V$ is irreducible. Writing $\mf g=\mf k\oplus\mf p$, the PBW theorem grants that $\op{Sym}\mf p\otimes_\CC V_\nu\onto V$, and taking the associated graded produces a finitely generated module $\op{gr}M$ over $Sg$, which we note has $K$-equivariant support, so it is contained in $\mf k^\perp\cap\mc N$. (It is contained in $\mc N$ by considerations with the infinitesimal character.)

	We now note that $G$ acts on $\mc N$ with finitely many orbits; the same argument shows that $K_\CC$ acts on $\mc N\cap\mf p$ with finitely many orbits. Indeed, start with some faithful representation $\mf g\subseteq\mf{sl}(n)$, and we may write $\mf{sl}(n)=\mf g\oplus\mf g'$, where $\mf g'$ is some $\mf g$-invariant complement. Now, $\mc N=\mc N_{\mf{sl}(n)}\cap\mf g$, and we have the result for $\mf{sl}(n)$ by hand, so it is enough to show that any orbit $\OO$ for $\op{SL}(n)$ has $\OO\cap\mf g$ divide up into finitely many $G$-orbits (by the compactness of $G$). For this, it is enough to show that each $G$-orbit in $\OO\cap\mf g$ is open, which amounts to showing that
	\[T_x\OO\cap\mf g\stackrel?=T_x(G(x)).\]
	But $T_x\OO=[x,\mf{sl}(n)]$, which we may expand as $[x,\mf g]\oplus[x,\mf g']$, so both sides above are $[x,\mf g]$. We similarly find that $K_\CC$ acts on $\mc N\cap\mf p$ by finitely many orbits be redoing the above argument: it is enough to check that a $G$-orbit $\OO\subseteq\mf g$ has $\OO\cap\mf p$ split into finitely many $K$-orbits, which follows from the same tangent space calculation because $[x,\mf g]=[x,\mf k]\oplus[x,\mf p]$.

	We now return to the proof. It is enough to check that any $K$-equivariant $S\mf p$-module $N$ has
	\[[N:V_\nu]\le C_N\dim V_\nu.\]
	By dividing up the support of $N$ by taking subquotients, we may assume that $N$ is a torsion-free module supported on the closure of a $K$-orbit $\OO\subseteq\mc N\cap\mf p$. But then $N$ can be embedded as sections of an equivariant bundle on the orbit, which embeds into $\CC[K]^r$, and the result now follows from Peter--Weyl.
\end{proof}
\begin{remark}
	The $G$-orbits $\OO$ in $\mf g^*$ are symplectic, so they are in particular even-dimensional. It even turns out that $\mf p\cap\OO$ is a Lagrangian subvariety.
\end{remark}
\begin{proposition}
	Fix an irreducible Hilbert space representation $V$ of a Lie group $G$. Then any $f\in C_c^\infty(G)$ acts by a trace class operator on $V$.
\end{proposition}
\begin{proof}
	The idea is to use the Casimir operators $C_k$ for $K$. In particular, we may write
	\[V_{\mathrm{fin}}=\bigoplus_\nu W_\nu\otimes V_\nu,\]
	where $\nu$ varies over irreducible representations $V_\nu$ of $K$, and $W_\nu$ is the multiplicity space.

	By \Cref{lem:upper-bound-multiplicity}, we have a constant $C$ for which
	\[\dim W_\nu\le C\dim V_\nu\]
	in our situation. By the representation theory of Lie algebras, one can check that $\dim V_\nu$ is $O\left(\left|\nu\right|^d\right)$, where $d$ is the number of positive roots of $\mf k$. Now, the Casimir element $C_{\mf k}$ acts by a scalar $c_\nu$, which is positive and of the order $\left|\nu\right|^2$.

	We now claim that the operator $\left(1+C_{\mf k}\right)^{-N}$ is Hilbert--Schmidt for large $N$, where we are only defining this operator a priori on $V_{\mathrm{fin}}$. Indeed, by combining the above bounds, it is enough to see that
	\[\sum_{\nu\in\Lambda_{\mf k}^+}\frac{(\dim V_\nu)^2}{\left|\nu\right|^{2N}}\le C\sum_{\nu\in\Lambda_{\mf k}^+}\frac{\left|\nu\right|^{2d}}{\left|\nu\right|^{2N}}\]
	converges, which occurs as soon as (say) $N>d+1$.

	Now, write our $f\in C_c^\infty(G)$ as
	\[f=\left(1+C_{\mf k}\right)^{-N}\cdot\left((1+C_{\mf k})^{-N}\cdot(1+C_{\mf k})^{2N}f\right).\]
	Notably, $(1+C_{\mf k})^{2N}f$ is just some function in $C_c^\infty(G)$. This is then passed through a product of Hilbert--Schmidt operators, so it is trace class by \Cref{rem:hs-to-trace-class}.
\end{proof}
\begin{definition}[character]
	Fix a Hilbert space representation $V$ of $G$. Then we define the \textit{character} $\chi_V$ to be the distribution on $G$ for which
	\[\langle\chi_V,f\,dg\rangle\coloneqq\tr\left(f\,dg;V\right).\]
\end{definition}
\begin{remark}
	If $V$ and $V'$ are infinitesimally equivalent, then their characters agree. Indeed, everything in the definition can be passed down to $V_{\mathrm{fin}}\cong V'_{\mathrm{fin}}$, so the result follows from continuity.
\end{remark}

\end{document}