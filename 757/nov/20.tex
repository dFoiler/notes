% !TEX root = ../notes.tex

\documentclass[../notes.tex]{subfiles}

\begin{document}

\section{November 20}
Today, we begin discussing the discrete series.

\subsection{Exponents}
We now aim to explain what is meant by ``discrete series.'' Fix a real group $G$, and we choose a maximal split torus $A\subseteq G$. It turns out that the Lie algebra $\mf a$ is maximal abelian in $\mf k^\perp$, where $K\subseteq G$ is a maximal compact subgroup. The centralizer $\mf z(\mf a)$ is some Levi, so we may choose a real parabolic $P$ with Levi subgroup $L$ with the correct Lie algebra. It turns out that the semisimple part $M$ of $L$ is compact: non-compact reductive groups have a nontrivial split torus, which would violate the maximality of $A$.

Now, the Levi decomposition allows us to write $P=MAN$ where $N$ is the unipotent radical of $P$. Namely, note $L$ is reductive, so $\mf l$ is the sum of its semisimple part $\mf m$ and its center $\mf a$, so $L=MA$ follows.
\begin{remark}
	Note $G$ is split if and only if $\dim\mf a=\op{rank}\mf g$; this then implies that $\mf m=0$ and that $P$ is a Borel subgroup. On the other hand, $P$ being Borel is equivalent to $\mf a$ containing a regular element, which is equivalent to $G$ being quasi-split. Recall that any group is an inner twist of a quasi-split group, but not every group can e an inner twist of a split group; for example, $G=\op{SU}(n,n)$ has $G_\CC=\op{SL}(2n,\CC)$ is quasi-split but not split.
\end{remark}
\begin{definition}
	Let $V$ be an admissible representation. Then an \textit{exponent} of $V$ is an eigenvalue of $\mf a$ acting on $M/\mf n^dM$ for some $d>0$, where $M=V_{\mathrm{fin}}$.
\end{definition}
The following lemma makes sense of this definition in the irreducible case.
\begin{lemma}
	Let $M$ be an irreducible admissible $(\mf g,K)$-module. Then $M$ is finitely generated over $\mf n$, so $M/\mf n^dM$ is finite-dimensional for all $d>0$.
\end{lemma}
\begin{proof}
	We may complexify everything and pass to $\op{gr}M$, induced by some filtration on $M$ which we may assume to be finite-dimensional at all steps, compatible with the PBW filtration, and $K$-invariant. Then $\op{gr}M$ is a coherent sheaf on $\mf g$ supported on $\mf k^\perp\cap\mc N$. (Support being in $\mc N$ follows because $M$ is irreducible and therefore has an infinitesimal character.) Certainly we know that $\op{gr}M$ is finitely generated over $\op{gr}U\mf g=S\mf g$ by the irreducibility, so it is enough to check that the projection
	\[\mf k^\perp\cap\mc N\to\mf g/\mf p\]
	is finite. Because this map is invariant under scaling, it is enough to merely check that the kernel is zero, so we need to compute the intersection $\mc N\cap\mf k^\perp\cap\mf p$.

	To this end, choose a maximal abelian subspace $\mf a\subseteq\mf k^\perp$, and the Cartan involution $\theta$ from $\mf k$ admits $\theta|_{\mf a}=-1$. Now, $\theta$ sends $\mf p$ to the opposite parabolic $\mf p^-$, so $\mf p\cap\mf k^\perp$ is contained in $\mf p\cap\mf p^-$, which is $\mf a\oplus\mf m$. But $\theta$ acts on these two factors by $(-1)$ and $(+1)$, respectively, so we conclude that $|mf p\cap\mf k^\perp=\mf a$. However, $\mf a$ is toral, so $\mc N\cap\mf a=0$.
\end{proof}
% \begin{remark}
% 	Continue with $M\coloneqq V_{\mathrm{fin}}$, and we assume $M$ is irreducible. It turns out that $M$ is finitely generated over $\mf n$, so $M/\mf n^dM$ is finite-dimensional for all $d>0$. To see this finite generation, 
% \end{remark}
% Now, writing $\mf p=\mf l\oplus\mf u$, we define $\mf a_+\subseteq\mf a$ to be the subspace with positive eigenvalues from $\mf u$.

\end{document}