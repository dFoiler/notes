% !TEX root = ../notes.tex

\documentclass[../notes.tex]{subfiles}

\begin{document}

\section{November 20}
Today, we begin discussing the discrete series.

\subsection{Asymptotics and Exponents}
We now aim to explain what is meant by ``discrete series.'' Fix a real group $G$, and we choose a maximal split torus $A\subseteq G$. It turns out that the Lie algebra $\mf a$ is maximal abelian in $\mf k^\perp$, where $K\subseteq G$ is a maximal compact subgroup. The centralizer $\mf z(\mf a)$ is some Levi, so we may choose a real parabolic $P$ with Levi subgroup $L$ with the correct Lie algebra. It turns out that the semisimple part $M$ of $L$ is compact: non-compact reductive groups have a nontrivial split torus, which would violate the maximality of $A$.

Now, the Levi decomposition allows us to write $P=MAN$ where $N$ is the unipotent radical of $P$. Namely, note $L$ is reductive, so $\mf l$ is the sum of its semisimple part $\mf m$ and its center $\mf a$, so $L=MA$ follows.
\begin{remark}
	Note $G$ is split if and only if $\dim\mf a=\op{rank}\mf g$; this then implies that $\mf m=0$ and that $P$ is a Borel subgroup. On the other hand, $P$ being Borel is equivalent to $\mf a$ containing a regular element, which is equivalent to $G$ being quasi-split. Recall that any group is an inner twist of a quasi-split group, but not every group can e an inner twist of a split group; for example, $G=\op{SU}(n,n)$ has $G_\CC=\op{SL}(2n,\CC)$ is quasi-split but not split.
\end{remark}
\begin{definition}
	Let $V$ be an admissible representation. Then an \textit{exponent} of $V$ is an eigenvalue of $\mf a$ acting on $M/\mf n^dM$ for some $d>0$, where $M=V_{\mathrm{fin}}$.
\end{definition}
The following lemma makes sense of this definition in the irreducible case.
\begin{lemma}
	Let $M$ be an irreducible admissible $(\mf g,K)$-module. Then $M$ is finitely generated over $\mf n$, so $M/\mf n^dM$ is finite-dimensional for all $d>0$.
\end{lemma}
\begin{proof}
	We may complexify everything and pass to $\op{gr}M$, induced by some filtration on $M$ which we may assume to be finite-dimensional at all steps, compatible with the PBW filtration, and $K$-invariant. Then $\op{gr}M$ is a coherent sheaf on $\mf g$ supported on $\mf k^\perp\cap\mc N$. (Support being in $\mc N$ follows because $M$ is irreducible and therefore has an infinitesimal character.) Certainly we know that $\op{gr}M$ is finitely generated over $\op{gr}U\mf g=S\mf g$ by the irreducibility, so it is enough to check that the projection
	\[\mf k^\perp\cap\mc N\to\mf g/\mf p\]
	is finite. Because this map is invariant under scaling, it is enough to merely check that the kernel is zero, so we need to compute the intersection $\mc N\cap\mf k^\perp\cap\mf p$.

	To this end, choose a maximal abelian subspace $\mf a\subseteq\mf k^\perp$, and the Cartan involution $\theta$ from $\mf k$ admits $\theta|_{\mf a}=-1$. Now, $\theta$ sends $\mf p$ to the opposite parabolic $\mf p^-$, so $\mf p\cap\mf k^\perp$ is contained in $\mf p\cap\mf p^-$, which is $\mf a\oplus\mf m$. But $\theta$ acts on these two factors by $(-1)$ and $(+1)$, respectively, so we conclude that $|mf p\cap\mf k^\perp=\mf a$. However, $\mf a$ is toral, so $\mc N\cap\mf a=0$.
\end{proof}
% \begin{remark}
% 	Continue with $M\coloneqq V_{\mathrm{fin}}$, and we assume $M$ is irreducible. It turns out that $M$ is finitely generated over $\mf n$, so $M/\mf n^dM$ is finite-dimensional for all $d>0$. To see this finite generation, 
% \end{remark}
\begin{theorem} \label{thm:discrete-series}
	Let $G$ be a real reductive group, and let $V$ be an irreducible admissible representation of $G$. Then the following are equivalent.
	\begin{listroman}
		\item All matrix coefficients of $V$ are in $L^2(G)$.
		\item All exponents $\lambda$ of $V$ have $\lambda<-\rho$, where $\rho$ is the half-sum of the roots of $\mf a$ in $\mf n$. Here, $\lambda<-\rho$ means that $-\rho-\lambda$ is a nonzero sum of roots of $\mf a$ in $\mf n$ with nonnegative coefficients.
	\end{listroman}
\end{theorem}
\begin{remark}
	By definition of $\rho$, one has that $e^{2\rho}(a)=\det(\mathrm{Ad}_a;\mf n)$.
\end{remark}
\begin{remark}
	The notation $\lambda<-\rho$ is justified as the usual partial order on weights. Indeed, there is some non-reduced root system given by the action of $\mf a$.
\end{remark}
\begin{example}
	Take $G=\op{SL}(2,\RR)$, and we will recover the discrete series. Here, $K=\op{SO}(2)$. Let $\mf b=\mf a\oplus\mf n$ be a Borel subalgebra of $\mf g$ in (distinct from the one giving rise to $\mf k$), and one finds that $L_m^+$ is the induction $U\mf g\otimes_{\mf b}\CC_m$. It then turns out that $L_m^+/\mf n^dL_m^+$ has $\mf a$-weights $\{m,m+2,\ldots,m(d-1)\}$, so it satisfies the criteria of \Cref{thm:discrete-series} if and only if $m>1$.
\end{example}
\begin{proof}[Proof of \Cref{thm:discrete-series}]
	For simplicity, we will assume that the group is split, but it is not so important. The proof proceed in steps.
	\begin{enumerate}
		\item We bound the asymptotics of the matrix coefficients (on $A^+$) in terms of the exponents. It follows that $\dim A=\op{rank}G$ and $\dim K=\#\Phi^+$; the second equality follows by hunting for $\theta$-invariant subspaces in the decomposition $\mf g=\mf u_-\oplus\mf a\oplus\mf u_+$, where we know that $\theta$ swaps $\mf u_-$ and $\mf u_+$ and acts by $-1$ on $\mf a$.

		The moral is that the mp $K\times A\times K\to G$ should be generically a diffeomorphism. To see this, we note that the differential at some $a\in A$ is the map $\mf k\oplus\mf a\oplus\mf k\to\mf g$ defined by
		\[(x,y,z)\mapsto\op{Ad}_a(x)+y+z.\]
		This allows to estimate some asymptotics.

		\item We see that bounding on $A^+$ is enough by using the Cartan decomposition: for any $f\in L^2(G)$, we see that
		\[\int_G\left|f\right|^2\,dg=\int_{A^+}\widetilde f(g)\,da,\]
		where $\widetilde f$ is some $K\times K$ average of $f$ (up to some adjustment of measures). Taking an average over a compact group is okay, but keeping track of the measures will be mildly technical. Thus, one finds that we need to estimate some quotient $dg/dk_1\,da\,dk_2$, and it will be about $e^{-\rho(a)}$.
	\end{enumerate}
\end{proof}
% Now, writing $\mf p=\mf l\oplus\mf u$, we define $\mf a_+\subseteq\mf a$ to be the subspace with positive eigenvalues from $\mf u$.

\end{document}