% !TEX root = ../notes.tex

\documentclass[../notes.tex]{subfiles}

\begin{document}

\section{November 18}
We continue our discussion of characters.

\subsection{Characters of the Principal Series}
Let's work out some characters.
\begin{example}
	We work out the characters $\chi_{L_{\pm m}}(g)$, where $g$ is a compact toral elements of $\op{SL}(2,\RR)$.
\end{example}
\begin{proof}
	We run through the classification of \Cref{thm:g-k-for-sl2}. Let's start with the case that $g$ is compact, and we are working with some $L_{\pm m}$. Then we give each $g\in K$ a diagonalization $\op{diag}(\lambda,\ov\lambda)$ where $\lambda\in S^1$. Then the character is given by
	\[\chi(g)=\frac{\lambda^{\pm m}}{1-\lambda^{-2}}.\]
	Technically speaking, the series used for this calculation does not converge because $\left|\lambda\right|=1$, but this expression does make sense when integrating against a function $\varphi\in C_c^\infty(S^1)$ satisfying $\varphi(\pm1)=0$.
\end{proof}
On the other hand, if we have some regular semisimple $g$ which is not compact, then $\chi_{L_m}(g)=\chi_{L_{-m}}(g)$: note that $\op{SL}(2,\RR)$ admits an outer automorphism given by $g\mapsto g^{-\intercal}$ (equivalently, this is conjugation by an element in $\op{GL}(2,\RR)$ with determinant $-1$), and this outer automorphism exhibits an isomorphism $L_m\to L_{-m}$. Furthermore, for $g$ which is not compact, we see that $g^{-\intercal}$ is conjugate to $g$, which is not the case for compact $g$. (At a high level, this is true because we can find a matrix in $\op{GL}(2,\RR)$ with determinant $-1$ which commutes with $g$.) Thus, we see that
\[\chi_{L_m}+\chi_{L_{-m}}+\chi_{V_{m-2}}=\chi_{P_\pm(s)}\]
for suitable principal series, so it remains to compute the character for the principal series.

Thus, we now turn to the principal series $V_s^\pm$. As usual, identify $V_s^\pm$ with functions on $\RR^2\setminus\{0\}$ with some homogeneity properties, and we want to compute the action of some $f\in C_c^\infty(G)$ on $V_s^\pm$. Explicitly, $V_s$ consists of functions $f_0\colon\left(\RR^2\setminus\{0\}\right)\to\CC$ for which $f_0(tz)=\pm\left|t\right|^{-is-1}f_0(z)$. We now chase around the following diagram.
% https://q.uiver.app/#q=WzAsMyxbMCwxLCJHIl0sWzEsMCwiR1xcdGltZXMgRy9CIl0sWzIsMSwiXFxkaXNwbGF5c3R5bGVcXGZyYWN7Ry9VXFx0aW1lcyBHL1V9VCJdLFsxLDBdLFsxLDJdXQ==&macro_url=https%3A%2F%2Fraw.githubusercontent.com%2FdFoiler%2Fnotes%2Fmaster%2Fnir.tex
\[\begin{tikzcd}[cramped]
	& {G\times G/B} \\
	G && {\displaystyle\frac{G/U\times G/U}T}
	\arrow[from=1-2, to=2-1]
	\arrow[from=1-2, to=2-3]
\end{tikzcd}\]
Here, the left map is the projection, and the right map is given by the action map $G\times G/U\to G/U\times G/U$. One then pulls $f$ back and pushes forward to produce an integral operator on the quotient $(G/U)^2/T$. To compute the trace of this integral operator, one merely compute the integral along the diagonal $G/B\into(G/U)^2/T$.

To do this, we now observe that the pre-image of the diagonal up in $G\times G/B$ is given by
\[\widetilde G=\{(g,B'):g\in G,g\in B'\},\]
where we view $G/B$ as the space of Borel subgroups. Thus, we are now chasing around the following diagram.
% https://q.uiver.app/#q=WzAsMyxbMCwxLCJUIl0sWzEsMCwiXFx3aWRldGlsZGUgRyJdLFsyLDEsIkciXSxbMSwyLCJcXHBpIl0sWzEsMF1d&macro_url=https%3A%2F%2Fraw.githubusercontent.com%2FdFoiler%2Fnotes%2Fmaster%2Fnir.tex
\[\begin{tikzcd}[cramped]
	& {\widetilde G} \\
	T && G
	\arrow[from=1-2, to=2-1]
	\arrow["\pi", from=1-2, to=2-3]
\end{tikzcd}\]
We are interested in integrating over the fibers of $\pi$ to compute the integral over the diagonal up in $\widetilde G$, so we need to say something about $\pi$. Over $\CC$, we note that $\pi_\CC$ is generically $\left|W\right|$-to-$1$. Indeed, for $g$ regular and semisimple, we are asking for Borel subgroups containing $g$, which more or less corresponds to choosing a flag from $g$, but if $g$ is regular semisimple, then we are basically just reordering the eigenvalues of $g$.

We are now able to consider the differential $d\pi_{(g,B')}$. Well, the bundle sequence $B\to \widetilde G\to G/B$ induces a morphism
% https://q.uiver.app/#q=WzAsMTAsWzAsMCwiMCJdLFsxLDAsIlxcbWYgYiJdLFsyLDAsIlRfeyhnLEIpfVxcd2lkZXRpbGRlIEciXSxbMywwLCJcXG1mIGcvXFxtZiBiIl0sWzQsMCwiMCJdLFswLDEsIjAiXSxbMSwxLCJcXG1mIGIiXSxbMiwxLCJUX2dHIl0sWzMsMSwiXFxtZiBnL1xcbWYgYiJdLFs0LDEsIjAiXSxbMCwxXSxbMSwyXSxbMiwzXSxbMyw0XSxbNSw2XSxbNiw3XSxbNyw4XSxbMiw3LCJkXFxwaSJdLFs4LDldLFszLDhdLFsxLDYsIiIsMSx7ImxldmVsIjoyLCJzdHlsZSI6eyJoZWFkIjp7Im5hbWUiOiJub25lIn19fV1d&macro_url=https%3A%2F%2Fraw.githubusercontent.com%2FdFoiler%2Fnotes%2Fmaster%2Fnir.tex
\[\begin{tikzcd}[cramped]
	0 & {\mf b} & {T_{(g,B)}\widetilde G} & {\mf g/\mf b} & 0 \\
	0 & {\mf b} & {T_gG} & {\mf g/\mf b} & 0
	\arrow[from=1-1, to=1-2]
	\arrow[from=1-2, to=1-3]
	\arrow[equals, from=1-2, to=2-2]
	\arrow[from=1-3, to=1-4]
	\arrow["{d\pi}", from=1-3, to=2-3]
	\arrow[from=1-4, to=1-5]
	\arrow[from=1-4, to=2-4]
	\arrow[from=2-1, to=2-2]
	\arrow[from=2-2, to=2-3]
	\arrow[from=2-3, to=2-4]
	\arrow[from=2-4, to=2-5]
\end{tikzcd}\]
of short exact sequences, and the right-hand map is given by $\mathrm{Ad}_g-1$. Notably, $\left|d\pi\right|$ can be computed on the trivial line bundle $\Omega_{\widetilde G}^{\mathrm{top}}$ as
\[\det\left(\mathrm{Ad}_g-1;\mf g/\mf b\right)=\prod_{\alpha\in\Phi^+}(1-\alpha(g)).\]
The same sort of determinant argument works over $\RR$, but we now have to be careful about which regular semisimple $g\in G$ are actually in the image of $\pi$: one finds that $g$ is in the image of $\pi$ if and only if $g$ is diagonalizable over $\RR$ and with $\left|W\right|$ preimages. In total, by ``integrating'' over the fiber of $g$ in $\widetilde G$, one finds that $\chi_V(g)$ vanishes if $g$ is not diagonalizable over $\RR$; otherwise,
\[\chi_V(g)=\frac1{\prod_{\alpha\in\Phi^+}\left|1-\alpha(g)\right|}\sum_w\nu(wg),\]
where $\nu$ is $-is-1$ from earlier.
\begin{example}
	For $\op{SL}(2,\RR)$ and $g=\op{diag}\left(a,a^{-1}\right)$, we get
	\[\pm\frac{\left|a\right|^\nu+\left|a\right|^{-\nu}}{\left|a-a^{-1}\right|}.\]
\end{example}

\subsection{The Cartan Decomposition}
Here is our statement.
\begin{proposition}[Cartan decomposition]
	Let $G$ be a real group with maximal compact $K$, and let $A$ be a maximal split torus.
	\[G=KA^+K,\]
	where $A^+=\{\exp(x):x\in\mf a^+\}$, where $\mf a^+$ is the positive part of $\mf a$ (for the Cartan involution).
\end{proposition}
\begin{proof}[Sketch]
	One starts with the polar decomposition $G=PK$, which arises because $\mf g=\mf k\oplus\mf p$ by definition of $\mf p$. To sketch this polar decomposition, one chooses an embedding $G\into\op{SL}(n,\CC)$ with compatible Cartan involution, approximately allowing us to pass to $\op{SL}(n,\CC)$. Note then that $K$ is $\op{SU}(n)\cap G$ and $P$ is the intersection $P_{\op{SL}(n)}\cap G$. Then given $g\in G$, one can let $p$ be a square root of $g\theta\left(g^{-1}\right)$, which exists as soon as one checks some positivity property of the eigenvalues. Then one checks that $p\in P$ and $k\coloneqq p^{-1}g$ is in $K$.

	We now pass to the Cartan decomposition. Well, as above, we may let $A$ be a maximal split torus in $G$, and we may assume that $\mf a\subseteq\mf p$. In light of the polar decomposition, we basically have to show that
	\[\mf a^+=\mf a/W(\mf a,K)\to\mf p/K\]
	is surjective; in fact, it is an isomorphism. This is something that one can check more or less directly. Once we have this claim, one writes any $g\in G$ as $xu$ where $x\in P$ and $u\in K$, and then we know that $x$ can be written as $u'a(u')^{-1}$ for $u'\in K$ and $a\in A^+$, so the Cartan decomposition follows.
\end{proof}
\begin{example}
	Take $G=\op{SL}(2,\RR)$ so that $K=\op{SO}(2,\RR)$. Then $G/K$ is just the upper-half plane $\mc H$, where the natural action by $G$ preserves the negative curvature metric. Then we would like to show that $KA^-\onto G/K$, which amounts to showing that we can get any point in $\mc H$ by first scaling it appropriately by $A^-=\{\op{diag}(a,1/a):a>0\}$ and then taking an orbit along $K$. Roughly speaking, the action by $A$ translates the necessary distance, and then $K$ rotates along points equidistant from $i$.
\end{example}
\begin{example}
	Take $G=\op{SL}(n,\RR)$ so that $K=\op{SO}(n,\RR)$. Then $G/K$ parameterizes positive-definite quadratic forms: for every pair of quadratic forms, one can find a basis which is orthogonal for both. More explicitly, if we let $Q_1$ be the standard quadratic form, then any other form $Q_2$ can be diagonalized by an orthonormal basis for $Q_1$. (This identification is known as the principal axis reduction.) This manipulation has basically shown that $KA^-$ surjects onto $G/K$.
\end{example}
\begin{example}
	The polar decomposition $G=PK$ of $G=\op{SL}(n,\RR)$ has $K=\op{SO}(n,\RR)$. Then $\mf p$ consists of self-adjoint operators, and $P$ consists of the positive self-adjoint operators.
\end{example}
\begin{remark}
	Because $K$ is compact, asymptotics of a function $f$ on $G$ is controlled by $f|_{A^-}$. For matrix coefficients, this reduces to an algebraic calculation of some eigenvalues of $\mf a$ on $M/\mf u^nM$.
\end{remark}

\subsection{Asymptotics and Exponents}
We now aim to explain what is meant by ``discrete series.'' Fix a real group $G$, and we choose a maximal split torus $A\subseteq G$. It turns out that the Lie algebra $\mf a$ is maximal abelian in $\mf k^\perp$, where $K\subseteq G$ is a maximal compact subgroup. The centralizer $\mf z(\mf a)$ is some Levi, so we may choose a real parabolic $P$ with Levi subgroup $L$ with the correct Lie algebra. It turns out that the semisimple part $M$ of $L$ is compact: non-compact reductive groups have a nontrivial split torus, which would violate the maximality of $A$.

Now, the Levi decomposition allows us to write $P=MAN$ where $N$ is the unipotent radical of $P$. Namely, note $L$ is reductive, so $\mf l$ is the sum of its semisimple part $\mf m$ and its center $\mf a$, so $L=MA$ follows.
\begin{remark}
	Note $G$ is split if and only if $\dim\mf a=\op{rank}\mf g$; this then implies that $\mf m=0$ and that $P$ is a Borel subgroup. On the other hand, $P$ being Borel is equivalent to $\mf a$ containing a regular element, which is equivalent to $G$ being quasi-split. Recall that any group is an inner twist of a quasi-split group, but not every group can e an inner twist of a split group; for example, $G=\op{SU}(n,n)$ has $G_\CC=\op{SL}(2n,\CC)$ is quasi-split but not split.
\end{remark}
\begin{definition}
	The exponents of $V$ are the eigenvalues of $A$ acting on $M/\mf u^iM$ for some $i$.
\end{definition}
Next class, we will give a criterion for $V$ to be square integrable in terms of the exponents.
% Now, writing $\mf p=\mf l\oplus\mf u$, we define $\mf a_+\subseteq\mf a$ to be the subspace with positive eigenvalues from $\mf u$.

\end{document}