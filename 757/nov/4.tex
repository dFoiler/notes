% !TEX root = ../notes.tex

\documentclass[../notes.tex]{subfiles}

\begin{document}

\section{October 30}
Today, we continue discussing the representation theory of $\op{SL}(2,\RR)$.

\subsection{Unitary Irreducible Representations}
Last class, we classified the irreducible $(\mf g,K)$-modules for $G=\op{SL}(2,\RR)$, where $\mf g=\mf{sl}(2,\CC)$ and $K=\op{SO}(2,\RR)$. Let's calculate which are unitary.
\begin{proposition}
	The unitary irreducible $(\mf g,K)$-modules for $(\mf g,K)=(\mf{sl}(2,\CC),\op{SO}(2,\RR))$ are as follows.
	\begin{itemize}
		\item The trivial representation.
		\item Discrete series.
		\item Unitary principal series: $P_s^\pm$ for $s\in i\RR$.
		\item Complementary principal series: $P_s^+$ for $\left|s\right|<1$.
	\end{itemize}
\end{proposition}
\begin{proof}
	Our explicit choice of basis on $\mf{sl}(2,\CC)$ satisfies that $\overline h=-h$ and $\overline e=f$ and $\overline f=e$. We now examine \Cref{thm:g-k-for-sl2} to find which $M$ admit an invariant Hermitian form $(-,-)$ for which
	\[(xv,w)+(v,\overline xw)=0\]
	for all $v,w\in M$. In other words, we need to satisfy the equations
	\[\begin{cases}
		(hv,w)=(v,hw), \\
		(ev,w)=-(v,fw), \\
		(fv,w)=-(v,ew).
	\end{cases}\]
	Give $M$ a weight basis $\{w_n\}_{n\in\ZZ}$, where $w_n\in M[n]$ is either nonzero or a basis vector. Then the first equation implies $(w_n,w_m)=0$ whenever $n\ne m$, so it remains to examine $(w_n,w_n)$. Well, observe
	\begin{align*}
		(ew_n,ew_n) &= -(few_n,w_n) \\
		&= \left(\frac{(n+1)^2}4-c\right)(w_n,w_n)
	\end{align*}
	because $-fe=\frac{(h+1)^2}2-c$, where $c$ is a Casimir eigenvalue. We are now ready to do our classification.
	\begin{itemize}
		\item If $M$ is finite-dimensional, then we see that $M$ is unitary if and only if $M$ is trivial. From the formulae, we can see this because $e$ is nilpotent, so it cannot have the above properties. From first principles, one can note that it is not possible to embed $\op{PSL}(2,\RR)$ into a compact group, so there cannot be a finite-dimensional unitary irreducible representation.

		\item If $M$ has discrete series $L^{\pm}_{\pm m}$, then $c=(m-1)^2/4$ by a direct calculation with the Verma module, so the quantity
		\[\frac{(n+1)^2}4-c\]
		is always positive (whenever $w_n$ and $ew_n$ are both positive), so our Hermitian invariant form can be chosen to be positive-definite. Namely, one defines $(w_n,w_n)$ inductively.

		\item As in the previous point, we can see that $P^\pm_s$ is unitary if and only if
		\[\frac{(n+1)^2}4-\frac{s^2}4\]
		is real and positive for all $n$ for which $w_n\ne0$ and $ew_n\ne0$. (In this case, $w_n\ne0$ implies $ew_n\ne0$, so we may ignore this extra condition.) Notably, we have $s\notin 11+2\ZZ$ when $P_s^+$ admits even weights, and $s\notin2\ZZ$ when $P_s^-$ admits odd weights.
		
		The condition certainly holds when $s\in i\RR$; otherwise, $s$ is real. With odd weights, we see that there are no real $s$ because $(-1+1)^2/4=0$. With even weights, we see that we merely need to achieve $s^2<1$, which is equivalent to $\left|s\right|<1$.
		\qedhere
	\end{itemize}
\end{proof}

\subsection{Realizations}
We now give realizations for our irreducible $(\mf g,K)$-modules. There is nothing to say for the finite-dimen\-sional representations. For $P_s^\pm$, we found these earlier in the semester inside even or odd functions on $\RR^2\setminus\{0\}$, transforming in a specified way under dilations via
\[f(tx)=t^{s-1}f(x),\]
where $t$ is a positive real number. We computed the $K$-finite vectors in a problem set, and it does indeed agree with the described principal series.

Let's discuss the unitary structure on these principal series.
\begin{itemize}
	\item For $s\in i\RR$, these representations should be unitary, and one can see it on these functions as the usual inner product on $L^2\left(\RP^1\right)$ (which should technically be $\frac{1-s}2$-densities): we have
	\[\langle f,g\rangle\coloneqq\int_{\RP^1}f\overline g,\]
	where the point is that $f\overline g(tx)=t^{-2}(x)$, so $f\overline g$ succeeds at being a volume form on $\RP^2$.
	\item For the complimentary series $P_s^+$ for $0<s<1$, we realize $P_s^+$ in functions $\RR$ as $f(x)\,dx^{(1-s)/2}$ (where the action is given by fractional linear transformations), where the inner product is
	\[\langle f,g\rangle\coloneqq\int_{\RR^2}f(x)\overline g(y)\,\frac{dx\,dy}{\left|x-y\right|^{1+s}}.\]
	One can formally check that this pairing is invariant: simply direct check dilations, translations, and $x\mapsto1/x$. To explain where this formula comes from, it can be rewritten as $\langle f*\overline g,\left|x\right|^{-1-s}\rangle$. The Fourier transform of this is $\langle\mc Ff\cdot\overline{\mc Fg},\mc F\left|x\right|^{-1-s}\rangle$. We can think of $\left|x\right|^s$ as a tempered distribution satisfying the differential equation $sxy=y'$, and taking the Fourier transform of this differential equation reveals that the Fourier transform of $\left|x\right|^{-1-s}$ is $\left|x\right|^{-s}$ times a constant.
\end{itemize}
Lastly, we discuss the discrete series. Intuitively, we want our functions on the ``circle'' $\RP^1$ to extend to holomorphic functions on the full disk. For example, $L_m^+$ is realized as functions on the unit disk
\[D\coloneqq\{z\in\CC:\left|z\right|<1\}.\]
(Formally, we should really work with the densities $f\,(du)^{m/2}$.) Note $\op{SL}(2,\RR)$ acts on $D$ (one can transform $D$ to the upper-half plane, and then the upper half-plane admits a natural action by fractional linear transformations), preserving the metric $\frac{du\,d\overline u}{\left(1-\left|u\right|\right)^2}$. The invariant pairing is thus given by
\[\left(f\,(du)^{m/2},g\,(du)^{m/2}\right)\coloneqq\int_Dfg\,\mu.\]
The same story works for $L_m^-$. It turns out that taking $K$-finite vectors actually recovers the discrete series.

\subsection{The Rest of this Course}
Let's outline what will be done in the remainder of the course.
\begin{definition}[discrete series]
	Fix a Lie group $G$. An irreducible representation $V$ of $G$ is \textit{discrete series} if and only if the matrix coefficients
	\[\ell_{v,w}(g)\coloneqq(gv,w)\]
	lie in $L^2(G)$ for all $v,w\in V$.
\end{definition}
\begin{remark}
	When $v$ and $w$ are $K$-finite, we see that $\ell_{v,w}$ is real analytic, so this is a ``global'' condition about the growth of its coordinates. Note that it is enough to check the condition for $K$-finite vectors because they form a dense subset.
\end{remark}
\begin{remark}
	Suppose $V$ is discrete series. Fixing some nonzero vector $w$, we find that there is a non\-zero map $V\to L^2(G)$ given by $v\mapsto\ell_{v,w}$. Thus, because $V$ is irreducible, $V$ embeds into $L^2(G)$, so $V$ is unitary.
\end{remark}
For an admissible representation $V$ of $G$, we have a character which is best understood as a distribution on $G$. We will show later that the restriction to regular semisimple elements is an analytic function, and the distribution is given locally be some $L^1$-function.

From here, we will turn our attention to two questions: classification and character formulae.
\begin{itemize}
	\item For classification, there is the Langlands classification, which explains how to construct representations from the discrete series by induction. It then remains to classify the discrete series, for which there are two approaches: one can use $D$-modules and Beilinson--Bernstein localization, which classifies them from $K_\CC$-orbits in $G_\CC/B_\CC$. Alternatively, one can view these as a special case of the local Langlands conjecture for a local field $F$, which provide a correspondence between irreducible representations of $G(F)$ and homomorphisms from the Galois group of $F$ to the Langlands dual group of $G$ (and some extra data).
	\item For character formulae, we note that principal series are basically inductions of finite-dimensional representations, so the characters are given by the Frobenius formula. (One may call these ``standard representations.'') One would then like to express irreducible characters as linear combinations of these standard ones. For complex groups, this is equivalent to the Kazhdan--Lusztig conjectures; in general, these amount to the Lusztig--Vogan generalizations.
\end{itemize}
Let's begin discussing characters.
\begin{remark}
	It turns out that any irreducible $(\mf g,K)$-module can be realized inside a Hilbert space, but then the $G$-action may not be unitary. For example, the principal series $P_s^\pm$ was realized as densities $f(x)\,dx^{(1-s)/2}$, where $dx$ can be chosen to be some $K$-invariant differential. Upon identifying the various $s$ as providing the same vector space, we recover the given realizations.
\end{remark}
% \begin{proposition}
% 	Fix an admissible representation $V$ of a Lie group $G$. For $f\in C_c^\infty(G)$, the operator $f\,dg$ on $V$ is of trace class.
% \end{proposition}
\begin{definition}[character]
	Fix an admissible representation $V$ of a Lie group $G$. Then we define the distribution $\chi_V$ on $G$ by
	\[\langle\chi_V,f\,dg\rangle=\tr\left(f\,dg;V\right)\]
	for each $f\in C_c^\infty(G)$.
\end{definition}
\begin{remark}
	For this to make sense, we need to show that the operator $f\,dg$ on $V$ is of trace class.
\end{remark}
\begin{theorem}
	Fix an admissible representation $V$ of a Lie group $G$. Then $\chi_V$ is an analytic function when restricted to the regular semisimple locus of $G$.
\end{theorem}
We will prove this next class. There are two ingredients. First, $\chi_V$ is conjugation-invariant. Second, the map $U\mf g\to\op{Diff}(G)$ (by having $G$ act on itself by left translations) allows $Z\mf g$ to act on smooth functions, so there is $\lambda\in\mf h^*/W$ for which the maximal ideal $\mf m_\lambda\subseteq Z\mf g$ has
\[\langle f\,dg,zf\rangle=0\]
for all $z\in\mf m_\lambda$. The idea is that the map $T\times G/T\to G$ given by $(t,x)\mapsto\left(xtx^{-1}\right)$ is a local isomorphism at a given $g\in T$. Then the second ingredient turns out to provide a holonomic system of differential equations on our distribution, from which being analytic follows. We will provide more details next class.

\end{document}