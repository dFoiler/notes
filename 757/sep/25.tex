% !TEX root = ../notes.tex

\documentclass[../notes.tex]{subfiles}

\begin{document}

\section{September 25}
Today we finish our proof of the Admissibility theorem.

\subsection{End of the Proof of Admissibility}
Let's explain how we can finish the proof.
\begin{proof}[Proof of \Cref{thm:hc-admissibility}]
	Fix a unitary admissible representation $V$ of $G$, and we would like to check that $\dim\op{Hom}(\rho,V)\le\dim\rho$ for any irreducible representation $\rho$ of $K$. Of course, this is equivalent to checking that $\dim V_\rho\le(\dim\rho)^2$.

	For this, we let $e_\rho\coloneqq\chi_\rho\,dg\in\op{Meas}_c(K)$ be the projector onto the $\rho$-isotypic component, which we silently pushforward to $\op{Meas}_c(G)$. Now, set
	\[\op{Meas}_c(G)^\rho\coloneqq e_\rho*\op{Meas}_c(G)*e_\rho\]
	to be a subalgebra (without the same identity element). Now, for any nonzero $\mu\in\op{Meas}_c(G)^\rho$, we see that $\mu$ only has nonzero action on $\rho$-isotypic components, so \Cref{lem:approx-to-nonzero-action} forces $\mu$ to act in a nontrivial way on $W_\rho$ for some irreducible representation $W$. The moral is that $\op{Meas}_c(G)^\rho$ will act faithfully on
	\[\bigoplus_{W\in\op{IrRep}(G)}W_\rho,\]
	where $W$ varies over finite-dimensional irreducible representations of $G$. Thus, \Cref{prop:fd-dim-bound} tells us that $\dim W_\rho\le(\dim\rho)^2$. We now apply \Cref{thm:alk-id}, which tells us that any collection of measures $\mu_1,\ldots,\mu_{2(\dim\rho)^2}$ in $\op{Meas}_c(G)^\rho$ has
	\[\sum_{\sigma\in S_{2n}}(-1)^{\left|\sigma\right|}A_{\sigma(1)}\cdots A_{\sigma(2n)}\]
	act by $0$ on $V$ and hence on $V_\rho$. (Here, $n=(\dim\rho)^2$.)

	Now, suppose for the sake of contradiction that $\dim\op{Hom}_K(\rho,V)>(\dim\rho)$ so that we may find a set $S\subseteq V_\rho$ of $2(\dim\rho)^2+1$ linearly independent vectors, and then \Cref{cor:density-of-unitary-ops} allows us to find $\mu_i$s approximating matrices $M_i$ of $\op{span}S$ such that
	\[\sum_{\sigma\in S_{2n}}(-1)^{\left|\sigma\right|}\mu_{\sigma(1)}\cdots \mu_{\sigma(2n)}\ne0\]
	because now the space is larger. (For example, one can take the $M_i$s to be distinct elementary matrices.) On the other hand, projecting down to $\op{Meas}_c(G)^\rho$ produces a contradiction of the action on $V_\rho$.
\end{proof}
It remains to prove \Cref{prop:fd-dim-bound}, which we will do later.

\subsection{Infinitesimal Equivalence}
It will be worthwhile to explain how much a $(\mf g,K)$-module remembers a representation of $G$.
\begin{definition}[infinitesimmaly equivalent]
	Fix a real connected semisimple Lie group $G$ with maximal compact subgroup $K\subseteq G$. Two admissible representations $V$ and $W$ of $G$ are \textit{infinitesimally equivalent} if and only if $V_{\mathrm{fin}}$ and $W_{\mathrm{fin}}$ are isomorphic as $(\mf g,K)$-modules.
\end{definition}
\begin{remark}
	Certainly if $V$ and $W$ are isomorphic representations, then this descends to an isomorphism of the $(\mf g,K)$-modules $V_{\mathrm{fin}}$ and $W_{\mathrm{fin}}$.
\end{remark}
\begin{remark}
	If $V$ is unitary and irreducible (and thus admissible by \Cref{thm:hc-admissibility}), we may grant $V$ a $G$-invariant Hermitian form (which is unique up to scalar by Schur's lemma because a Hermitian form amounts to the data of an isomorphism $\ov V\to V^*$). The same sort of argument shows that $V_{\mathrm{fin}}$ admits a unique Hermitian pairing by a Schur's lemma argument, but then $V$ is the completion of $V_{\mathrm{fin}}$, so we can recover $V$ from $V_{\mathrm{fin}}$. Thus, infinitesimal equivalence implies isomorphism in this case.
\end{remark}
\begin{remark}
	There are principal series representations for $\op{SL}(2,\RR)$ which are infinitesimally equivalent but not isomorphic: one can use $L^p(\RP^1)$ for $p>1$ or $C^k(\RP^1)$ or $C^\infty(\RP^1)$ and so on, which all have the same underlying $(\mf g,K)$-module but complete it to a Fr\'echet space in different ways.
\end{remark}
\begin{remark}
	In general, it is a difficult problem to take a $(\mf g,K)$-module $M$ and ask if there actually is a representation $V$ with $M\cong V_{\mathrm{fin}}$. This is answered positively (with some growth conditions) by the Casselman--Wallach globalization theorem.
\end{remark}

\subsection{Proof of Algebraic Finiteness}
We now turn to the proof of \Cref{prop:fd-dim-bound}. To begin, we take complexifications: a finite-dimensional representation of $G$ is algebraic, so we get an algebraic representation of $G_\CC$. Our proof will require two big inputs. The first input is a structure result on our representations.
\begin{theorem}[Borel--Weil]
	Fix a complex semisimple Lie algebra $G$. For each algebraic representation $V$ of $G$ of highest weight $\lambda$, then $V_\lambda$ is the global sections of a line bundle $\mc L_\lambda$ on $G/B$, where $B$ is the Borel subgroup. Equivalently, $V_\lambda$ is contained in $\op{Ind}_B^G\chi$ for some character $\chi$.
\end{theorem}
\begin{proof}
	Because $B=T\rtimes U$, it is enough to find ourselves in $\op{Ind}_U^G\chi$ for some character $\chi$ on $T$. Now, write
	\[\CC[G]=\bigoplus_\lambda V_\lambda\otimes V_\lambda^\lor\]
	by a variant of the Peter--Weyl theorem. But $V_\lambda^*=V_{\lambda^\lor}$, and the $U$-invariants is $1$-dimensional by the highest-weight theory, so we take $\chi$ to be $\chi_\lambda$.
\end{proof}
\begin{example}
	With $G=\op{SL}(2,\CC)$, the Borel subgroup $B$ is the upper-triangular matrices, so $G/B$ is $\CP^1$. It turns out that the usual finite-dimensional representations $\op{Sym}^nV_{\mathrm{std}}$.
\end{example}
\begin{remark}
	In general $G/B$ will be some kind of flag variety. Sometimes, $\mc L_\lambda$ will come from a further quotient $G/P$ where $P\supseteq B$ is a parabolic subgroup, and it turns out that taking these sorts of pullbacks do not change the global sections.
\end{remark}
Our second input explains that $K$ is large.
\begin{proposition}
	Fix a real semisimple Lie algebra $G$ with maximal compact subgroup $K$. Then $K_\CC$ has an open dense orbit in $G_\CC/B_\CC$. In fact, there are only finitely many orbits.
\end{proposition}
\begin{proof}
	We use the Bruhat decomposition. Let $W$ be the Weyl group, which is $N_G(T)/T$ for some torus $T\subseteq G$. Then the natural map
	\[W\to B_\CC\backslash G_\CC/B_\CC\]
	turns out to be a bijection. Approximately speaking, this arises from an action of $W$ on the flag variety $G_\CC/B_\CC$.
\end{proof}
\begin{example}
	Consider $G=\op{SL}(3,\RR)$ and $K=\op{SO}(3,\RR)$. Then $G_\CC/B_\CC$ is the moduli space of maximal flags of $\CC^3$, which amounts to the data of a point and a line in $\CP^2$ (where the point is on the line). Letting $q$ be our quadratic form, it turns out that our orbits are given by how the point and line intersect the conic; for example, the line could be tangent to the conic with the point elsewhere, the line could be tangent with the point on the conic, the line could intersect the conic, and so on.
\end{example}
\begin{proof}[Proof of \Cref{prop:fd-dim-bound}]
	We have realized $V$ as some $V_\lambda$ living inside some $\op{Ind}_{B_\CC}^{G_\CC}\chi$ for some character $\chi$. We now pull back along the open dense orbit
	\[K_\CC\to G_\CC/B_\CC,\]
	which is injective by the density. The point is that we have identifications
	\[\op{Ind}_{B_\CC}^{G_\CC}\chi=\op{Ind}^{K_\CC}_{K_\CC\cap B_\CC}\chi\subseteq\CC[K_\CC],\]
	of $K$-representations, but we are now done because the Peter--Weyl theorem bounds the multiplicity of an irreducible representation of $K$ in $\CC[K_\CC]$.
\end{proof}

\end{document}