% !TEX root = ../notes.tex

\documentclass[../notes.tex]{subfiles}

\begin{document}

\section{September 30}
There is no class on Thursday.

\subsection{A Little More Structure Theory}
It remains to prove \Cref{prop:almost-iwasawa-decomp}.
\begin{proof}[Proof of \Cref{prop:almost-iwasawa-decomp}]
	The orbits are locally closed by Chevalley's theorem, so it is enough to merely show there are finitely many orbits. We are going to deduce our result from the Bruhat decomposition, which asserts that $B\backslash G/B$ is in bijection with the Weyl group $W$, which is $N_G(T)/T$ for some maximal torus $T\subseteq G$.

	Note $\theta$ acts on $G$ and hence on $X\coloneqq G/B$. (For example, one can realize $X$ as the collection of Borel subgroups of $G$.) Now, let $Z\subseteq X^2$ be the graph of $\theta$, and note that $Z$ is the collection of fixed points of the automorphism
	\[\iota\colon(x,y)\mapsto(\theta(y),\theta(x))\]
	of $X^2$.

	Now, the Bruhat decomposition provides a decomposition of $B\backslash X$ as the Weyl group, so we let $X_w$ be the orbit corresponding to a given $w\in W$. We now claim that $Z\cap X_w^2$ is a finite union of orbits of $K$; by looping over all $w$, this shows that $Z$ has finitely orbits of $K$, but $Z\cong X$ as spaces with $K$-action, so the proposition follows. For this, it is enough to show that the induced map
	\[\op{Lie}K\to T_{(x,y)}Z\cap T_{(x,y)}X_w^2\]
	for any given $(x,y)$: this will imply that all orbits are open, which means that there are finitely many, by quasicompactness. To show this claim, we note that $T_xX=\mf g/\mf b_x$, where $\mf b_x$ is the Lie algebra of the corresponding Borel algebra, so $T_{(x,y)}X^2=(\mf g/\mf b_x)\oplus(\mf g/\mf b_y)$. Thus, $T_{(x,y)}Z=(\mf g/\mf b_x\oplus\mf g/\mf b_y)^\iota$, and $T_{(x,y)}X_w^2$ is a $G$-orbit, so it is the (diagonal) image of $\mf g$. We thus see that the intersection is
	\[\{(X,X)\in(\mf g/\mf b_x\oplus\mf g/\mf b_y):X\in\mf g\}^{\iota},\]
	which means that $X$ is fixed by $\iota$, which is equivalent to $X\in\op{Lie}K$.
\end{proof}
\begin{remark}
	This argument does nothing to allow us to compute the number of orbits of $K\backslash G/B$. Indeed, the main claim achieves finiteness via compactness! However, one can calculate the number of orbits in terms of subsets of some quotient of the Weyl group.
\end{remark}

\subsection{Cartan Involutions}
To close up our discussion of the admissibility theorem, we explain a little about Cartan involutions and the classification of real semisimple groups.
\begin{theorem} \label{thm:cartan-exist}
	Every semisimple real Lie group $G$ has a Cartan involution $\theta$ which is unique up to conjugacy. Furthermore, the pair $(G,\theta)$ is in bijection with its complexification $(G_\CC,\theta_\CC)$, which in turn is in bijection with complex involutions of $G_\CC$.
\end{theorem}
\begin{theorem} \label{thm:classify-real-forms}
	Let $G_\CC$ be a semisimple connected complex Lie group. Then real forms of $G_\CC$ are in bijection with complex involutions; the map sends a real form $G$ of $G_\CC$ to the complexification of the Cartan involution.
\end{theorem}
This is a case of a Galois descent problem, in the special case of the extension $\RR\subseteq\CC$. Here is the way Galois descent is typically done: given a Galois extension $F\subseteq E$ and an object $Y_E$, we would like to understand objects (e.g., schemes) $X$ over $F$ with $X_E\cong Y_E$. Supposing that there is one such object $X$ over $F$, one typically finds that the other objects are classified by $\mathrm H^1(\op{Gal}(E/F),\op{Aut}X)$.
\begin{example}
	In our situation, we suppose that we are given a real Lie group $G$, and we would like to classify real Lie groups $G'$ with $G_\CC\cong G'_\CC$. Well, $G$ has a Cartan involution $\theta$, so a choice of isomorphism $G_\CC\cong G'_\CC$ grants $G'$ the involution $\theta'\coloneqq a\circ\theta$, where $a$ denotes the conjugation (which is an automorphism of $G$). We thus see that $(\theta')^2=aa^\theta$, so $a^\theta=a^{-1}$ is the required $1$-cocycle. Notably, the cocycles $a_1$ and $a_2$ produce the same real form if and only if they are the same up to some twisted conjugacy
	\[a_2=ba_1b^{-\theta}.\]
\end{example}
\begin{example}
	If $G$ is a real semisimple Lie group, then we can try to classify conjugacy classes of maximal tori in $G$. Well, fix maximal torus $T$ in $G$. Then any other torus $T'$ is conjugate to $T$ up in $G_\CC$, so we can write $T'_\CC=gT_\CC g^{-1}$ for some $g\in G_\CC$. Now, we define $x\coloneqq g^{-1}g^\theta$ to normalize $T$; we remark also that $\theta(x)=x^{-1}$. Changing $g$ only changes $x$ up to twisted conjugacy, sending $x$ to (say) $y^{-1}xy^\theta$. It turns out that such $x$ classify our maximal tori.
\end{example}
\begin{example}
	We can run the same argument on conjugacy classes: given a single complex conjugacy class in $G_\CC$, one can ask how the complex conjugacy class splits into real conjugacy classes. Well, given $g_1,g_2\in G$ equipped with $x\in G_\CC$ for which $g_2=xg_1x^{-1}$, we see that $x^{-1}\theta(x)$ commutes with $g_1$, and we expect to be classified by such elements up to twisted conjugacy. Such a splitting is seen as early as the regular semisimple elements of $\op{SL}(2,\RR)$.
\end{example}
\begin{proof}[Proof of \Cref{thm:cartan-exist}]
	We first show the existence of our Cartan involutions in the complex case. Let $K$ be a maximal compact subgroup of $G$ so that $G=K_\CC$. Now, $\CC[G]$ is $C(K)_{\mathrm{fin}}$, which is algebraically just
	\[\bigoplus_\lambda V_\lambda\otimes V_\lambda^*.\]
	Now, each $V_\lambda\otimes V_\lambda^*$ has a $\CC$-antilinear involution given by the conjugate transpose. Composing this with the swapping involution $V_\lambda\otimes V_\lambda^*\cong V_\lambda^*\otimes V_\lambda$, which yields a complex involution $\theta$ for which
	\[\theta f(g)=\overline{f(g)}\]
	for all $g\in K$. We now receive an involution $\theta$ of $G$ for which $G^\theta=K$, which we claim is the required Cartan involution: writing $\mf g=\mf k\oplus i\mf k$, we see that the Killing form $(-,-)_{\mf g}$ when restricted to $\mf k$ is twice the Killing form $(-,-)_{\mf k}$ (because the Killing form is the trace of the composite of the adjoint maps) and when restricted to $i\mf k$ is negative twice the Killing form $(-,-)_{\mf k}$ (for the same reason). Positive-definiteness of $(-,\theta-)$ follows.

	It remains to construct Cartan involutions in the real case. Given a real Lie group $G$, we know that $G_\CC$ has a Cartan involution $\theta$, and $G_\CC$ also has a maximal compact subgroup $K_c$. We would like to choose $\theta_0$ so that $K_c=G_\CC^{\theta_0}$ and so that $\theta_0$ and $\theta$ commute. Then $\sigma\coloneqq\theta\theta_0$ is an involution and provides a Cartan involution on $G$.

	To do so, we consider $X\coloneqq G_\CC/K_c$. This quotient has an invariant metric (because we can take any metric on $G_\CC$ and then average it over $K_c$). We now claim that this metric has nonpositive sectional curvature.
	\begin{example}
		The quotient $\op{SL}(2,\CC)/\op{SU}(2)$ is hyperbolic $3$-space.
	\end{example}
	We will also be able to show that $X$ is simply connected. We now need two geometric facts.
	\begin{theorem}[Toponogov]
		If $X$ is a simply connected Riemannian manifold with constant nonpositive sectional curvature, then every two points are connected with a unique geodesic, and the sum of the angles of any triangle is less than $180^\circ$.
	\end{theorem}
	\begin{proposition}
		Let $X$ be a space where the sum of the angles of any triangle is less than $180^\circ$. Then any compact group acting on $X$ has a fixed point.
	\end{proposition}
	The second fact follows by taking the image of an orbit and then taking the circumcenter to be the fixed point. View $X$ as parameterizing maximal compact subgroups in $G_\CC$. Then for each $x\in X$, we note that there is a geodesic segment connecting $x$ to $\theta(x)$, we let $y$ be the midpoint, and it follows that $\theta(y)=y$ while $\op{Stab}y$ is conjugate to $K_c$. Then we can construct $\theta_0$ using $y$, and this construction works.
\end{proof}
\begin{remark}
	Consider the Riemann curvature tensor for the $G$-invariant metric on $M\coloneqq G/K$. Then there is $R\in\Omega^2(\op{End}T_M)$ given by $Z\mapsto([X,Y],Z)$ at the identity $T_1M=\mf g/\mf k$. One can use the Cartan involution $\theta$ on $\mf g$ to decompose $\mf g$ into $\{\pm1\}$ eigenspaces, which allows one to decompose $R$.
\end{remark}

\end{document}