% !TEX root = ../notes.tex

\documentclass[../notes.tex]{subfiles}

\begin{document}

\section{September 16}
Today, we walk the roads laid by Harish-Chandra.

\subsection{Admissible Representations}
For today, $G$ is a connected Lie group, $K\subseteq G$ is a compact Lie subgroup, and $\mf g$ is the Lie algebra of $G$. We will soon take $G$ to be reductive and $K$ to be a maximal compact subgroup, but we will not have to do this for quite a while.
\begin{notation}
	Fix a Fr\'echet representation $V$ of a compact Lie subgroup $K$. Given an irreducible representation $\rho$ of $K$, we may write
	\[V_\rho\coloneqq\op{Hom}_K(\rho,V).\]
\end{notation}
\begin{lemma}
	Fix a Fr\'echet representation $V$ of a compact Lie subgroup $K$. Then the image of the canonical map
	\[\bigoplus_{\rho\in\op{IrRep}(K)}(V_\rho\otimes\rho)\to V\]
	is exactly $V_{\mathrm{fin}}$.
\end{lemma}
\begin{proof}
	The canonical map is given by evaluation: we send $\varphi\otimes v\in V_\rho\otimes\rho$ to $\varphi(v)\in V$. Note that this map is $K$-invariant: for any $g\in K$, we have $g(\varphi\otimes v)=\varphi(gv)=g\varphi(v)$.

	We have two inclusions to show. Quickly, to show that the image of the canonical map is contained in $V_{\mathrm{fin}}$, we note that it is enough to show that $\varphi(v)\in V_{\mathrm{fin}}$ for any $\varphi\in V_\rho$ and $v\in V$. Indeed, for any $g\in K$, we see that $g\varphi(v)=\varphi(gv)$, so
	\[\op{span}\{g\varphi(v):g\in K\}\subseteq\im\varphi,\]
	which is finite-dimensional because the domain of $\varphi$ is finite-dimensional.

	Thus, it remains to show that the canonical map surjects onto $V_{\mathrm{fin}}$. Fix some $v\in V$ and consider the subspace $W$ spanned by the vectors $gv$ as $g$ varies over $G$. It is enough to show that $W$ is in the image of our canonical map.
	
	The main point is to reduce to the case that $W$ is irreducible. We know that $W$ is finite-dimensional, and because $K$ is compact, we know that finite-dimensional representations of $K$ are semisimple, so we may write $W$ as a direct sum of irreducible representations. For example, we may write
	\[W=\bigoplus_{i=1}^nW_i,\]
	where the $W_i$s are irreducible representations of $K$, and so we see that it is enough to show that each $W_i\subseteq W\subseteq V$ is in the image of the canonical map. Thus, we may assume that $W$ is irreducible.

	Now, choose $\rho\in\op{IrRep}(K)$ so that $W\cong\rho$. Then we see that there is a map $\varphi\colon\rho\to V$ with image exactly $W$ (given by the isomorphism $\rho\to W$), so $W$ is given by the image of the canonical map on the vectors $\{\varphi\otimes v:v\in\rho\}$.
\end{proof}
% \begin{remark}
% 	There is a canonical evaluation map $V_\rho\otimes\rho\to V$ sending $\varphi\otimes v\mapsto\varphi(v)$. This map is $K$-invariant
% \end{remark}
% \begin{remark}
% 	One finds that
% 	\[V_{\mathrm{fin}}=\bigoplus_{\rho\in\op{IrRep}(K)}V_\rho\otimes\rho\]
% 	as representations of $K$.
% \end{remark}
Now that we have basically realized the $V_\rho$s as summands of $V$, one may hope to find a projector back down to $V$.
\begin{lemma}
	
\end{lemma}
\begin{remark}
	Fixing Haar measures, one finds that $V_\rho\otimes\rho$ can be realized as the image of the idempotent $e_\rho$ given by sending $1_K$ through the composite
	\[C^\infty(K)\into\op{Meas}_c(K)\subseteq\op{Meas}_c(G).\]
	By choosing an approximate identity in the usual way, one can show that $V_\rho\otimes\rho\subseteq V_{\mathrm{sm}}$.
\end{remark}
Admissibility is simply a size constraint on the size of the representation.
\begin{definition}[admissible]
	Fix a connected Lie group $G$ with compact Lie subgroup $K$. A Fr\'echet representation $V$ is \textit{$K$-admissible} if and only if
	\[\dim V_\rho<\infty\]
	for each irreducible representation $\rho$ of $K$. We may just say that $V$ is \textit{admissible} if $K$ is understood from context.
\end{definition}
\begin{example}
	Take $G=\op{SL}(2,\RR)$ and $K=\op O(2)$. Let $L_s$ be the $s$-densities of $\op{SL}(2,\RR)$ acting on $\RP^1$. It turns out that the action of $K$ on $\RP^1$ is transitive, and one can find that $L_s$ is admissible because $\dim\op{Hom}_K(\rho,L_s)\le1$ for all irreducible representations $\rho$ of $K$.
\end{example}
It turns out that the admissible representations are the ones that admit passage to algebra.
\begin{lemma}
	Fix a connected Lie group $G$ with compact Lie subgroup $K$. For any Fr\'echet representation $V$, we have $V_{\mathrm{fin}}\subseteq V_{\mathrm{sm}}$.
\end{lemma}
\begin{proof}
	Matrix coefficients are smooth.
\end{proof}
Of course, if we want to understand $V$, it is of course not just enough to restrict to $K$ and look at $V_{\mathrm{fin}}$ on its own. To introduce more data, we note that there is an action of the Lie algebra $\mf g$.
\begin{notation}
	Fix a Fr\'echet representation $V$ of a Lie group $G$ with Lie algebra $\mf g$. Then for each $X\in\mf g$, we define an operator on $V_{\mathrm{sm}}$ by
	\[Xv\coloneqq\lim_{t\to0}\frac{\gamma(y)v-v}t,\]
	where $\gamma\colon(-1,1)\to G$ is some path with $\gamma(0)=1$ and $\gamma'(0)=X$.
\end{notation}
\begin{remark}
	It turns out that this defines a Lie algebra representation.
\end{remark}
\begin{remark}
	One can check that $V_{\mathrm{fin}}$ is preserved by this $\mf g$-action.
\end{remark}
We are now ready to make the following definition.
\begin{definition}[Harish-Chandra pair]
	Fix a Lie algebra $\mf g$ and compact Lie group $K$. Suppose that there is a map $\varphi\colon\op{Lie}K\to\mf g$ and an action of $K$ on $\mf g$ for which $\varphi$ preserves the two actions. In this situation, we call $(\mf g,K)$ a \textit{Harish-Chandra pair}.
\end{definition}
\begin{defihelper}[$(\mf g,K)$-module] \nirindex{gK module@$(\mf g,K)$-module}
	Fix a Lie algebra $\mf g$ and compact Lie group $K$ for which $(\mf g,K)$ is a Harish-Chandra pair.
	\begin{listalph}
		\item A \textit{$(\mf g,K)$-module} is a vector space $M$ with actions by $K$ and $\mf g$ for which $M$ is a sum of finite-dimensional continuous representations of $K$ and the two induced actions by $\op{Lie}K$ coincide.
		\item A $(\mf g,K)$-module $M$ is \textit{admissible} if and only if $\dim\op{Hom}_K(\rho,M)$ is finite for all irreducible representations $\rho$ of $K$.
		\item An admissible $(\mf g,K)$-module $M$
	\end{listalph}
\end{defihelper}
\begin{remark}
	Here are some motivational notes. Complex representations of $\mf g$ more or less reduce to representations of $\mf g_\CC$. Furthermore, finite-dimensional representations of $K$ are basically the algebraic ones, which also pass to $K_\CC$. Thus, in our definition of $(\mf g,K)$-module, we may as well make everything complex and require the $K$-action on $M$ to be algebraic.
\end{remark}
\begin{example}
	Fix a compact connected real Lie group $K$, and we set $G\coloneqq K_\CC$, which we think of as a real Lie group via restriction of scalars. Then we set $\mf g$ to be $\op{Lie}G=\mf k\otimes\CC$, where $\mf k\coloneqq\op{Lie}K$; note $\mf g_\CC$ is isomorphic to $\mf g\oplus\mf g$. Using the diagonal action $\mf g\to\mf g\oplus\mf g$, we receive a Harish-Chandra pair. In this way, we see that a Harish-Chandra module is the same as a $(\mf g,\mf g)$-module which is admissible for the diagonal copy of $\mf g$.
\end{example}
\begin{example}
	Consider the $\mf g$-bimodule $U\mf g$, where the first copy of $\mf g$ acts by $X\colon Y\mapsto XY$, and the second copy of $\mf g$ acts by $X\colon Y\mapsto-YX$. This turns out to not be admissible.
\end{example}

\subsection{Weakly Analytic Vectors}
We are going to want access to some weaker smoothness conditions.
\begin{definition}[matrix coefficient]
	Fix a Fr\'echet representation $V$ of a Lie group $G$. For any $v\in V$ and $\ell\in V^*$, we define the \textit{matrix coefficient} $G\to\CC$ by
	\[g\mapsto\ell(gv).\]
\end{definition}
\begin{definition}[weakly analytic]
	Fix a Fr\'echet representation $V$ of a Lie group $G$. A vector $v\in V$ is \textit{weakly analytic} if and only if the matrix coefficient map $g\mapsto\ell(gv)$ is real analytic for all $\ell\in V^*$.
\end{definition}
Here is the main result we will need.
\begin{theorem}[Harish-Chandra analyticity] \label{thm:h-c-analytic}
	Fix a semisimple Lie group $G$ with compact subgroup $K$. For each admissible Fr\'echet representation $V$ of a Lie group $G$, every $v\in V_{\mathrm{fin}}$ is weakly analytic.
\end{theorem}
To prove this, one uses elliptic regularity, so we spend a moment establishing some theory.
\begin{definition}[differential operator]
	Fix a smooth manifold $X$ of dimension $n$. Then a \textit{differential operator} $d$ is a section of the vector bundle which is locally on a chart $(x_i)$ is given by finite sums
	\[\sum_{i\colon\{1,\ldots,n\}\to\NN}\varphi_i\del x_1^{i_1}\cdots\del x_n^{i_n}.\]
	The \textit{order} of the operator is the maximum of $\sum_ji_j$ as $i\colon\{1,\ldots,n\}\to\NN$ varies such that $\varphi_i\ne0$.
\end{definition}
\begin{remark}
	The differential operators form a graded vector bundle. We define the principal symbol $\sigma(d)$ to be the entry in the associated graded ring.
\end{remark}
\begin{remark}
	We find that $\sigma(d)$ is a function on $T^*X$, and it is a polynomial of degree $N$.
\end{remark}
\begin{definition}[elliptic]
	The operator $d$ is \textit{elliptic} if and only if $\sigma(d)(x,\xi)\ne0$ for all $x$ and $\xi$ nonzero.
\end{definition}

\end{document}