% !TEX root = ../notes.tex

\documentclass[../notes.tex]{subfiles}

\begin{document}

\section{September 16}
Today, we walk the roads laid by Harish-Chandra.

\subsection{Admissible Representations}
For today, $G$ is a connected Lie group, $K\subseteq G$ is a compact Lie subgroup, and $\mf g$ is the Lie algebra of $G$. We will soon take $G$ to be reductive and $K$ to be a maximal compact subgroup, but we will not have to do this for quite a while.
\begin{notation}
	Fix a Fr\'echet representation $V$ of a compact Lie subgroup $K$. Given an irreducible representation $\rho$ of $K$, we may write $V_\rho$ for the image of the canonical map $\op{Hom}_K(\rho,V)\otimes\rho\to V$.
\end{notation}
\begin{remark}
	The canonical map is given by evaluation: we send $\varphi\otimes v\in V_\rho\otimes\rho$ to $\varphi(v)\in V$. Note that this map is $K$-invariant: for any $g\in K$, we have $g(\varphi\otimes v)=\varphi(gv)=g\varphi(v)$.
\end{remark}
\begin{lemma} \label{lem:decompose-v-fin-isotypic}
	Fix a Fr\'echet representation $V$ of a compact Lie group $K$. Then the image of the canonical map
	\[\bigoplus_{\rho\in\op{IrRep}(K)}V_\rho\to V\]
	is exactly $V_{\mathrm{fin}}$.
\end{lemma}
\begin{proof}
	We have two inclusions to show. Quickly, to show that the image of the canonical map is contained in $V_{\mathrm{fin}}$, we note that it is enough to show that $\varphi(v)\in V_{\mathrm{fin}}$ for any $\varphi\colon\rho\to V$ and $v\in V$. Indeed, for any $g\in K$, we see that $g\varphi(v)=\varphi(gv)$, so
	\[\op{span}\{g\varphi(v):g\in K\}\subseteq\im\varphi,\]
	which is finite-dimensional because the domain of $\varphi$ is finite-dimensional.

	Thus, it remains to show that the canonical map surjects onto $V_{\mathrm{fin}}$. Fix some $v\in V$ and consider the subspace $W$ spanned by the vectors $gv$ as $g$ varies over $G$. It is enough to show that $W$ is in the image of our canonical map.
	
	The main point is to reduce to the case that $W$ is irreducible. We know that $W$ is finite-dimensional, and because $K$ is compact, we know that finite-dimensional representations of $K$ are semisimple, so we may write $W$ as a direct sum of irreducible representations. For example, we may write
	\[W=\bigoplus_{i=1}^nW_i,\]
	where the $W_i$s are irreducible representations of $K$, and so we see that it is enough to show that each $W_i\subseteq W\subseteq V$ is in the image of the canonical map. Thus, we may assume that $W$ is irreducible.

	Now, choose $\rho\in\op{IrRep}(K)$ so that $W\cong\rho$. Then we see that there is a map $\varphi\colon\rho\to V$ with image exactly $W$ (given by the isomorphism $\rho\to W$), so $W$ is given by the image of the canonical map on the vectors $\{\varphi\otimes v:v\in\rho\}$.
\end{proof}
% \begin{remark}
% 	There is a canonical evaluation map $V_\rho\otimes\rho\to V$ sending $\varphi\otimes v\mapsto\varphi(v)$. This map is $K$-invariant
% \end{remark}
% \begin{remark}
% 	One finds that
% 	\[V_{\mathrm{fin}}=\bigoplus_{\rho\in\op{IrRep}(K)}V_\rho\otimes\rho\]
% 	as representations of $K$.
% \end{remark}
Now that we have basically realized the $V_\rho$s as summands of $V$, one may hope to find a projector back down to $V$.
\begin{lemma}
	Fix a Fr\'echet representation $V$ of a compact Lie group $K$. For an irreducible representation $\rho$ of $K$, let $\chi_\rho\colon K\to\CC$ denote its character. Then the operator $\overline\chi_\rho\,dg\colon V\to V$ fixes $V_\rho\subseteq V$ and annihilates $V_{\rho'}\subseteq V$ for any other irreducible representation $\rho'$.
\end{lemma}
\begin{proof}
	The main point is to show that $\overline\chi_\rho\,dg$ fixes $V_\rho\subseteq V$. Well, select some $\varphi\in V_\rho$ and $v\in\rho$, and we want to show that $\chi_\rho\,dg$ fixes $\varphi(v)$. Because $\varphi$ is $K$-invariant, we see that $\varphi$ commutes with the action of $\op{Meas}_c^0(K)$ and therefore commutes with $\op{Meas}_c(K)$ by sequential density. Thus,
	\[\chi_\rho\,dg\cdot\varphi(v)=\varphi(\overline\chi_\rho\,dg\cdot v).\]
	It remains to show that $\overline\chi_\rho\,dg$ fixes $v\in\rho$. Well, by \Cref{rem:meas-action-on-compact}, we know that $\ov\chi_\rho\,dg$ acts by the operator $\int_G\ov\chi_\rho(g)\rho(g)\,dg$, which is the identity by \Cref{ex:char-avg-proj}.

	Thus, we have shown that $\overline\chi_\rho\,dg$ fixes $V_\rho$. Then for any other distinct irreducible representation $\rho'$ of $K$, we analogously know that $\overline\chi_{\rho'}\,dg$ fixes $V_{\rho'}$, so
	\[\overline\chi_\rho\,dg|_{V_{\rho'}}=(\overline\chi_\rho\,dg*\overline\chi_{\rho'}\,dg)|_{V_{\rho'}},\]
	where we have used the associativity of \Cref{rem:meas-action-assoc}. However, a calculation with the definition of $*$ (via \Cref{ex:compute-meas-as-convolution}) shows that
	\[\overline\chi_\rho\,dg*\overline\chi_{\rho'}\,dg=\overline{\chi_\rho*\chi_{\rho'}}\,dg,\]
	which vanishes when $\rho\not\cong\rho'$ by \Cref{cor:character-convolution}.
\end{proof}
As an application, we refine \Cref{prop:smooth-dense}.
\begin{proposition} \label{prop:isotypic-smooth-dense}
	Suppose $K$ is a compact Lie subgroup of a Lie group $G$. For any Fr\'echet representation $V$ of $G$, we claim that $V_\rho\cap V_{\mathrm{sm}}$ is dense in $V_{\rho}$ for each irreducible representation $\rho$ of $V$.
\end{proposition}
\begin{proof}
	Choose some $v\in V_\rho$ to approximate by smooth vectors. Following \Cref{prop:smooth-dense}, we choose an approximate identity $\{\varphi_i\}_i$ so that $\{\varphi_i\,dg\}\to\delta_1$. Thus, because $\ov\chi_\rho\,dg$ fixes $v$ automatically, we see that
	\[\lim_{i\to\infty}(\varphi_i\,dg*\ov\chi_\rho\,dg)\cdot v=v,\]
	where we have silently used the associativity of \Cref{rem:meas-action-assoc}. Using the ambient compactness of everything (we may as well pass to the compact subset of $G$ given by the product of the support of $\varphi_i$ and $K$), it now follows that
	\[\varphi_i\,dg*\ov\chi_\rho\,dg=(\varphi_i*\ov\chi_\rho)\,dg\]
	by \Cref{ex:compute-meas-as-convolution}. (Note $\ov\chi_\rho$ is technically only in $L^1(G)$, but this convolution result still holds via a continuity argument.) Thus, \Cref{lem:compute-deriv-on-meas-action} allows us to complete the proof as soon as we know that $\varphi_i*\ov\chi_\rho$ is smooth, which follows because $\varphi_i$ is smooth (and differentiation under the integral sign in local coordinates).
\end{proof}
\begin{remark}
	By summing along \Cref{lem:decompose-v-fin-isotypic}, we see that \Cref{prop:isotypic-smooth-dense} implies that $V_{\mathrm{fin}}\cap V_{\mathrm{sm}}$ is dense in $V_{\mathrm{fin}}$ and hence dense in $V$ by \Cref{prop:finite-is-dense}.
\end{remark}
% \begin{remark}
% 	: if $K$ is a compact Lie subgroup of a Lie group $G$, we claim that $V_\rho\cap V_{\mathrm{sm}}$ is dense in $V_{\rho}$ for each irreducible representation $\rho$ of $V$. 
% \end{remark}
% \begin{remark}
% 	Fixing Haar measures, one finds that $V_\rho\otimes\rho$ can be realized as the image of the idempotent $e_\rho$ given by sending $1_K$ through the composite
% 	\[C^\infty(K)\into\op{Meas}_c(K)\subseteq\op{Meas}_c(G).\]
% 	By choosing an approximate identity in the usual way, one can show that $V_\rho\otimes\rho\subseteq V_{\mathrm{sm}}$.
% \end{remark}
Finite-dimensionality arguments are important above, so we introduce a suitable size constraint.
\begin{definition}[admissible]
	Fix a Lie group $G$ with compact Lie subgroup $K$. A Fr\'echet representation $V$ is \textit{$K$-admissible} if and only if
	\[\dim V_\rho<\infty\]
	for each irreducible representation $\rho$ of $K$. We may just say that $V$ is \textit{admissible} if $K$ is understood from context.
\end{definition}
\begin{example}
	Take $G=\op{SL}(2,\RR)$ and $K=\op O(2)$. Let $L_s$ be the $s$-densities of $\op{SL}(2,\RR)$ acting on $\RP^1$. It turns out that the action of $K$ on $\RP^1$ is transitive, and one can find that $L_s$ is admissible because $\dim\op{Hom}_K(\rho,L_s)\le1$ for all irreducible representations $\rho$ of $K$.
\end{example}
We will shortly see that the admissible representations are the ones that admit passage to algebra by using the $K$-finite vectors. In fact, admissibility is essentially the hypothesis that allows us to just use $K$-finite vectors without worrying too much about smoothness.
\begin{lemma} \label{lem:get-g-k-module}
	Fix a Lie group $G$ with compact Lie subgroup $K$, and fix a Fr\'echet representation $V$ of $G$.
	\begin{listalph}
		\item If $V$ is admissible, then $V_{\mathrm{fin}}\subseteq V_{\mathrm{sm}}$.
		\item The subspace $V_{\mathrm{fin}}\cap V_{\mathrm{sm}}$ is invariant under the action of $\mf g$.
	\end{listalph}
\end{lemma}
\begin{proof}
	We show (a) and (b) separately.
	\begin{listalph}
		\item By \Cref{lem:decompose-v-fin-isotypic}, it is enough to check that $V_\rho\subseteq V_{\mathrm{sm}}$ for each irreducible representation $\rho$ of $V$. Well, \Cref{prop:isotypic-smooth-dense} assures us that $V_\rho\cap V_{\mathrm{sm}}$ is dense in $V_\rho$. But $V$ is admissible, so equality follows because $V_\rho$ is finite-dimensional!
		\item Fix some $v\in V_{\mathrm{fin}}\cap V_{\mathrm{sm}}$, and we want to show that $Xv\in V_{\mathrm{fin}}\cap V_{\mathrm{sm}}$ as well for any $X\in\mf g$. Certainly $Xv\in V_{\mathrm{sm}}$ by definition of smoothness, so the main point is to show that $Xv\in V_{\mathrm{fin}}$. Well, for any $g\in K$, \Cref{lem:group-vs-lie-action} tells us that $g(Xv)=(\op{Ad}_gX)(gv)$, so it follows that
		\[\op{span}\{g(Xv):g\in K\}\subseteq\op{span}\{Y(gv):Y\in\mf g\text{ and }g\in K\}.\]
		However, the latter space is finite-dimensional because it is the image of the finite-dimensional space $\op{span}\{gv:g\in G\}$ under a finite-dimensional space $\mf g$ of operators. Thus, $Xv\in V_{\mathrm{fin}}$.
		\qedhere
	\end{listalph}
\end{proof}
\begin{remark}
	One does not expect $V_{\mathrm{fin}}$ to admit an action by $G$ in general, so it is perhaps remarkable that we managed to retain a $\mf g$-action.
\end{remark}

\subsection{\texorpdfstring{$(\mf g,K)$}{(g, K)}-Modules}
Given a Lie group $G$ with compact Lie subgroup $K\subseteq G$, we now see that any Fr\'echet representation $V$ yields an interesting subspace $V_{\mathrm{fin}}\subseteq V$ with both a $K$-action and a $\mf g$-action. We now codify this sort of information into an algebraic object.
% Of course, if we want to understand $V$, it is of course not just enough to restrict to $K$ and look at $V_{\mathrm{fin}}$ on its own. To introduce more data, we note that there is an action of the Lie algebra $\mf g$.
% \begin{notation}
% 	Fix a Fr\'echet representation $V$ of a Lie group $G$ with Lie algebra $\mf g$. Then for each $X\in\mf g$, we define an operator on $V_{\mathrm{sm}}$ by
% 	\[Xv\coloneqq\lim_{t\to0}\frac{\gamma(y)v-v}t,\]
% 	where $\gamma\colon(-1,1)\to G$ is some path with $\gamma(0)=1$ and $\gamma'(0)=X$.
% \end{notation}
% \begin{remark}
% 	It turns out that this defines a Lie algebra representation.
% \end{remark}
% \begin{remark}
% 	One can check that $V_{\mathrm{fin}}$ is preserved by this $\mf g$-action.
% \end{remark}
% We are now ready to make the following definition.
\begin{definition}[Harish-Chandra pair]
	A \textit{Harish-Chandra pair} is a pair $(\mf g,K)$ of a Lie algebra $\mf g$ and a compact Lie group $K$ equipped with a map $\varphi\colon\op{Lie}K\to\mf g$ and an action map $\op{Ad}_\bullet\colon K\to\op{GL}(\mf g)$ such that
	\[d\op{Ad}_e(X)(Y)=[X,Y]\]
	for any $X\in\op{Lie}K$ and $Y\in\mf g$.
	% Fix a Lie algebra $\mf g$ and compact Lie group $K$. Suppose that there is a map $\varphi\colon\op{Lie}K\to\mf g$ and an action of $K$ on $\mf g$ for which $\varphi$ preserves the two actions. In this situation, we call $(\mf g,K)$ a \textit{Harish-Chandra pair}.
\end{definition}
\begin{example} \label{ex:prototypical-g-k-pair}
	If $G$ is a Lie group with compact Lie subgroup $K$, then the embedding $\op{Lie}K\subseteq\op{Lie}G$ makes $(\op{Lie}G,K)$ into a Harish-Chandra pair. Indeed, we can define the action map $\op{Ad}_\bullet K\to\op{GL}(\op{Lie}G)$ as the restriction of the usual adjoint map $\op{Ad}_\bullet\colon G\to\op{GL}(\op{Lie}G)$.
\end{example}
\begin{defihelper}[$(\mf g,K)$-module] \nirindex{gK module@$(\mf g,K)$-module}
	Fix a Lie algebra $\mf g$ and compact Lie group $K$ for which $(\mf g,K)$ is a Harish-Chandra pair.
	\begin{listalph}
		\item A \textit{$(\mf g,K)$-module} is a vector space $M$ with actions by $K$ and $\mf g$ such that $M$ is a direct sum of finite-dimensional continuous representations of $K$, and the two induced actions of $\op{Lie}K$ coincide: for any $X\in\op{Lie}K$, we have
		\[\varphi(X)v=\frac d{dt}\exp(tX)v\bigg|_{t=0}.\]
		Furthermore, we require that $g(Xv)=(\op{Ad}_gX)(gv)$ for all $g\in K$ and $X\in\mf g$ and $v\in M$.
		\item A $(\mf g,K)$-module $M$ is \textit{admissible} if and only if $\dim\op{Hom}_K(\rho,M)$ is finite for all irreducible representations $\rho$ of $K$.
		\item An admissible $(\mf g,K)$-module $M$ is \textit{Harish-Chandra} if and only if it is finitely generated over $U\mf g$.
	\end{listalph}
\end{defihelper}
\begin{example} \label{ex:prototypical-g-k-mod}
	Fix a Lie group $G$ with Lie algebra $\mf g$ and compact Lie subgroup $K$. Then \Cref{ex:prototypical-g-k-pair} tells us that the embedding $\op{Lie}K\subseteq\mf g$ makes $(\mf g,K)$ into a Harish-Chandra pair. Now, for any Fr\'echet representation $V$ of $G$, we claim
	\[V_{\mathrm{fin}}\cap V_{\mathrm{sm}}\]
	is a $(\mf g,K)$-module. Indeed, it has a $\mf g$-action by \Cref{lem:get-g-k-module} and a $K$-action by \Cref{rem:sm-is-g-inv}. The identity $Xv=\frac d{dt}\exp(tX)v\big|_{t=0}$ for $X\in\op{Lie}K$ follows by definition of the $\mf g$-action, and $g(Xv)=(\op{Ad}_gX)(gv)$ follows by \Cref{lem:group-vs-lie-action}.
\end{example}
\begin{example}
	Continuing from the set-up of \Cref{ex:prototypical-g-k-mod}, we note that if $V$ is admissible, then $V_{\mathrm{fin}}\cap V_{\mathrm{sm}}=V_{\mathrm{fin}}$ by \Cref{lem:get-g-k-module}. Furthermore, $V_{\mathrm{fin}}$ has $\dim\op{Hom}_K(\rho,V_{\mathrm{fin}})<\infty$ for all irreducible representations $\rho$ of $K$ because $V$ is admissible, so we conclude that $V_{\mathrm{fin}}$ is an admissible $(\mf g,K)$-module.
\end{example}
\begin{remark}
	One should really think of a $(\mf g,K)$-module as something algebraic. Complex representations of $\mf g$ more or less reduce to representations of $\mf g_\CC$. Furthermore, finite-dimensional representations of $K$ are basically the algebraic ones, which also pass to $K_\CC$. Thus, in our definition of $(\mf g,K)$-module, we may as well make everything complex and require the $K$-action on $M$ to be algebraic.
\end{remark}
As an aside, we note that the inclusion of the hypothesis $g(Xv)=(\op{Ad}_gX)(gv)$ is just a byproduct of us not currently requiring $K$ to be connected.
\begin{lemma}
	Fix a Lie algebra $\mf g$ and compact Lie group $K$ for which $(\mf g,K)$ is a Harish-Chandra pair. Suppose that we have a vector space $M$ with actions by $\mf g$ and $K$ such that $M$ is a direct sum of finite-dimensional continuous representations of $K$, and
	\[\varphi(X)v=\frac d{dt}\exp(tX)v\bigg|_{t=0}\]
	for any $X\in\mf g$ and $v\in M$. If $K$ is connected, then we have
	\[g(Xv)=(\op{Ad}_gX)gv\]
	for any $g\in K$ and $X\in\mf g$ and $v\in M$.
\end{lemma}
\begin{proof}
	Let $H\subseteq K$ be the subset of $g\in K$ for which the identity holds for any $X\in\mf g$ and $v\in M$. We would like to show that $H=K$. Because $K$ is connected, it is enough to show that $H$ is closed under multiplication and contains an open neighborhood of the identity; we will run these checks separately.
	\begin{enumerate}
		\item We show that $H$ is closed under multiplication. Well, if $g,h\in H$, then any $X\in\mf g$ and $v\in M$ has
		\begin{align*}
			(gh)(Xv) &= g(h(Xv)) \\
			&= g(\op{Ad}_hX)(hv) \\
			&= (\op{Ad}_g\op{Ad}_hX)(ghv) \\
			&= (\op{Ad}_{gh}X)(gh)v,
		\end{align*}
		so $gh\in H$.

		\item We show that $H$ contains an open neighborhood of the identity. Because $\exp\colon\op{Lie}K\to K$ is a local diffeomorphism, it is enough to check that $\exp(Y)\in H$ for any $Y\in\op{Lie}K$. For this, we fix some $X\in\mf g$ and $v\in V$ (for now) and define the two paths $\gamma_1,\gamma_2\colon\RR\to V$ by
		\[\begin{cases}
			\gamma_1(t) \coloneqq \exp(tY)X\exp(-tY)v, \\
			\gamma_2(t) \coloneqq (\op{Ad}_{\exp(tY)}X)v.
		\end{cases}\]
		We will complete the proof by showing that $\gamma_1=\gamma_2$. For this, we may a few preparatory remarks. By replacing $M$ with $\op{span}K\mf gKv$, we may assume that $M$ is finite-dimensional for this check. Then every finite-dimensional representation of a compact Lie group is real analytic (indeed, this follows by tracking through the representation on the exponential map, which itself is real analytic), so $\gamma_1$ and $\gamma_2$ are real analytic, so we may compare them only using their Taylor expansions.

		It remains to compute some derivatives.
		\begin{itemize}
			\item By passing through $\varphi$, we know that the operator $\exp(tY)$ is the same as $\exp(t\varphi(Y))$, so
			\[\gamma_1(t)=\exp(t\varphi(Y))X\exp(-t\varphi(Y))v,\]
			which expands to the Taylor series
			\[\gamma_1(t)=\sum_{i,j=0}^\infty\frac{(-1)^j}{i!j!}t^{i+j}\varphi(Y)^iX\varphi(Y)^jv.\]
			\item By construction of $\op{Ad}_\bullet$, we know that $\gamma_2'(0)=[\varphi(Y),X]v$. Then the fact that $\op{Ad}_\bullet$ is a group homomorphism implies that $\gamma_2'(t)=\op{Ad}_{\exp(tY)}[\varphi(Y),X]v$, so an induction gives
			\[\gamma_2^{(n)}(t)=[\cdots[\varphi(Y),[\varphi(Y),X]]\cdots]v,\]
			where there are $n$ opening braces $[$ and $n$ closing braces $]$.
		\end{itemize}
		A term-by-term comparison of the preceding Taylor series completes the proofs. Indeed, in the expansion of the $2^n$ terms of $[\cdots[\varphi(Y),[\varphi(Y),X]]\cdots]$, we see that are basically choosing some number $i$ times to use the $\varphi(Y)X$ term or the $-X\varphi(Y)$ term in the expansion of $[\varphi(Y),X]$, so we receive the term $\varphi(Y)^iX\varphi(Y)^{n-j}$ an absolute total of $\binom ni$ times; this count must then be corrected by the sign $(-1)^{n-i}$.
		\qedhere
		% Because $\gamma_1$ and $\gamma_2$ are smooth (because they are the composite of smooth maps), it is enough to note that $\gamma_1(0)=\gamma_2(0)=Xv$ and then verify that $\gamma_1'(t_0)=\gamma_2'(t_0)$ for all $t_0$.
		% \begin{itemize}
		% 	\item On one hand, we see that $\gamma_1'(t_0)$ is
		% 	\[\frac d{dt}\exp(tY)(Xv)\bigg|_{t=t_0}=\exp(t_0Y)\frac d{dt}\exp(tY)(Xv)\bigg|_{t=0},\]
		% 	which is $\exp(t_0Y)\varphi(Y)(Xv)$ by the compatibility of the $(\mf g,K)$-module. Continuing this calculation shows that the $n$th derivative is $\exp(t_0Y)\varphi(Y)^n(Xv)$.
		% 	\item On the other hand, we can use the product rule (viewing the outputs as matrices in $\op{End}(V)$!) to compute that $\gamma_2'(t_0)$ is
		% 	\[\frac d{dt}(\op{Ad}_{\exp(tY)}X)(\exp(tY)v)\bigg|_{t=t_0}=\frac d{dt}\op{Ad}_{\exp(tY)}X\bigg|_{t=t_0}\exp(t_0Y)v+\op{Ad}_{\exp(t_0Y)}X\frac d{dt}\exp(tY)v\bigg|_{t=t_0}.\]
		% 	As in the previous point, this evaluates to
		% 	\[[\varphi(Y),\op{Ad}_{\exp(t_0Y)}X]v+(\op{Ad}_{\exp(t_0)Y}X)\varphi(Y)v.\]
		% 	Notably, because the action of $\mf g$ on $V$ is a Lie algebra representation, this collapses down to $\varphi(Y)(\op{Ad}_{\exp(t_0Y)}X)v$. Continuing this calculation shows that the next derivative is $\varphi(Y)[\varphi(Y),\op{Ad}_{\exp(t_0Y)}X]v$
		% \end{itemize}
		% We now compare the results. Because the action of $\mf g$ on $V$ is a Lie algebra homomorphism
	\end{enumerate}
\end{proof}
Here are the typical sources of $(\mf g,K)$-modules.
\begin{example}
	Fix a compact connected real Lie group $K$, and we set $G\coloneqq K_\CC$, which we think of as a real Lie group via restriction of scalars. Then we set $\mf g$ to be $\op{Lie}G=\mf k\otimes\CC$, where $\mf k\coloneqq\op{Lie}K$; note $\mf g_\CC$ is isomorphic to $\mf g\oplus\mf g$. Using the diagonal action $\mf g\to\mf g\oplus\mf g$, we receive a Harish-Chandra pair. In this way, we see that a Harish-Chandra module is the same as a $(\mf g,\mf g)$-module which is admissible for the diagonal copy of $\mf g$.
\end{example}
\begin{example}
	Consider the $\mf g$-bimodule $U\mf g$, where the first copy of $\mf g$ acts by $X\colon Y\mapsto XY$, and the second copy of $\mf g$ acts by $X\colon Y\mapsto-YX$. This turns out to not be admissible.
\end{example}

\subsection{Weakly Analytic Vectors}
We are going to want access to some weaker smoothness conditions.
\begin{definition}[matrix coefficient]
	Fix a Fr\'echet representation $V$ of a Lie group $G$. For any $v\in V$ and $\ell\in V^*$, we define the \textit{matrix coefficient} $G\to\CC$ by
	\[g\mapsto\ell(gv).\]
\end{definition}
\begin{definition}[weakly analytic]
	Fix a Fr\'echet representation $V$ of a Lie group $G$. A vector $v\in V$ is \textit{weakly analytic} if and only if the matrix coefficient map $g\mapsto\ell(gv)$ is real analytic for all $\ell\in V^*$.
\end{definition}
Here is the main result we will need.
\begin{theorem}[Harish-Chandra analyticity] \label{thm:h-c-analytic}
	Fix a semisimple Lie group $G$ with compact subgroup $K$. For each admissible Fr\'echet representation $V$ of a Lie group $G$, every $v\in V_{\mathrm{fin}}$ is weakly analytic.
\end{theorem}
To prove this, one uses elliptic regularity, so we spend a moment establishing some theory.
\begin{definition}[differential operator]
	Fix a smooth manifold $X$ of dimension $n$. Then a \textit{differential operator} $d$ is a section of the vector bundle which is locally on a chart $(x_i)$ is given by finite sums
	\[\sum_{i\colon\{1,\ldots,n\}\to\NN}\varphi_i\del x_1^{i_1}\cdots\del x_n^{i_n}.\]
	The \textit{order} of the operator is the maximum of $\sum_ji_j$ as $i\colon\{1,\ldots,n\}\to\NN$ varies such that $\varphi_i\ne0$.
\end{definition}
\begin{remark}
	The differential operators form a graded vector bundle. We define the principal symbol $\sigma(d)$ to be the entry in the associated graded ring.
\end{remark}
\begin{remark}
	We find that $\sigma(d)$ is a function on $T^*X$, and it is a polynomial of degree $N$.
\end{remark}
\begin{definition}[elliptic]
	The operator $d$ is \textit{elliptic} if and only if $\sigma(d)(x,\xi)\ne0$ for all $x$ and $\xi$ nonzero.
\end{definition}

\end{document}