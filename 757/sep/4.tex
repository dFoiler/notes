% !TEX root = ../notes.tex

\documentclass[../notes.tex]{subfiles}

\begin{document}

\section{September 4}
Welcome to the class.

\subsection{Administrative Notes}
Here are some administrative notes.
\begin{itemize}
	\item There will be problem sets every two weeks, due on Fridays. They are not expected to be too time-consuming.
	\item Technically, this course is a sequel to 18.745--18.755, but one can get away with a bit less. In particular, we will assume familiarity with some basic notions in Lie theory, things about simple complex Lie algebras (as related to compact Lie groups), the theory of roots and weights, and this theory of finite-dimensional representations. For example, things like the Peter--Weyl theorem may come up.
	\item We will largely follow \href{https://amathr.org/wp-content/uploads/2024/10/Etingofbook2.pdf}{Etingof's lecture notes}.
\end{itemize}
This course is about the representations of Lie groups, especially those which are not necessarily compact. For example, we may focus on real reductive Lie groups such as $\op{SL}_n(\RR)$, and there is a new feature here that we must care about infinite-dimensional representations.

One of our motivations comes from quantum physics, where one finds groups acting on infinite-dimen\-sional Hilbert spaces. Another motivation is number-theoretic: one uses this theory to set up the arch\-imedean theory of automorphic forms.

\subsection{Finite Groups}
Let's recall some background. As one does, let's begin with the representation theory of finite groups. We split this into a few theorems.
\begin{theorem}[Maschke]
	Let $G$ be a finite group. Then every finite-dimensional complex representation is unitary (by averaging any given Hermitian form) and semisimple.
\end{theorem}
\begin{theorem}[Peter--Weyl]
	Let $G$ be a finite group. Then there is a decomposition
	\[\CC[G]=\bigoplus_{V\in\op{IrRep}(G)}\op{End}_\CC(V)\]
	of $\CC$-algebras. It follows that the characters $\{{\tr_V}\}_{V\in\op{IrRep}(G)}$ form an orthonormal basis of the class functions $G\to\CC$.
\end{theorem}
\begin{remark}
	Let $G$ be a finite group. For $f\in\CC[G]$, there is a dimension formula
	\[f(1)=\frac1{\left|G\right|}\sum_{V\in\op{IrRep}(G)}\dim V_i\cdot\tr_{V_i}f.\]
	Indeed, this follows from writing $\langle\varphi,\psi\rangle=\sum_V\tr_V(\varphi\psi')$ (where $\psi'\colon g\mapsto\psi(g^{-1})$) and then noting that $\langle\varphi,\psi'\rangle=\frac1{\left|G\right|}(\varphi*\psi')(1)$.
\end{remark}

\subsection{Compact Groups}
We now move up to the representation theory of compact connected Lie groups. Here is a generalization of Maschke's theorem.
\begin{theorem}
	Let $G$ be a compact Lie group with Lie algebra $\mf g$. Then the irreducible representations of $G$ are semisimple and unitary. If $G$ is simply connected, they are in bijection with the finite-dimensional irreducible representations of $\mf g$, which are in bijection with the dominant weights of $\mf g$.\todo{Write out lie theorems}
\end{theorem}
\begin{example}
	Take $G=\op U(1)$. Because $G$ is abelian, all irreducible representations are one-dimensional. These representations $\op U(1)\to\CC^\times$ are indexed by $n\in\ZZ$, given by $z\mapsto z^n$. Note that these are not in bijection with the representations of $\op{Lie}G$ because $G$ is not simply connected!
\end{example}
\begin{example}
	Take $G=\op{SU}(n)$ so that $\mf g=\mf{sl}_n(\CC)$. A weight is a character of the maximal torus $T$, for which we can take to be the subgroup of diagonal matrices. Explicitly,
	\[T=\{\op{diag}(z_1,\ldots,z_n):z_1\cdots z_n=1\},\]
	so the weight lattice is $\ZZ^n/\ZZ(1,\ldots,1)$, and a weight $\lambda=(\lambda_1,\ldots,\lambda_n)$ is dominant when the entries are increasing.
\end{example}
\begin{example}
	For example, with $\op{SU}(2)$, we have an isomorphism $\ZZ^2/\ZZ(1,1)\to\ZZ$ given by $(\lambda_1,\lambda_2)\mapsto\lambda_2-\lambda_1$, and the dominant weights are the nonnegative integers. One finds that weight $n$ corresponds to the $n$th symmetric power of the standard representation of $\op{SU}(2)$.
\end{example}
Here is a generalization of the Peter--Weyl theorem.
\begin{theorem}[Peter--Weyl]
	Let $G$ be a compact Lie group. The canonical map
	\[\bigoplus_{\text{dominant }\lambda}V_\lambda\otimes V_\lambda^*\to C^\infty(G)\]
	is an embedding with dense image; here $V_\lambda$ refers to the irreducible representation corresponding to the dominant weight $\lambda$.
\end{theorem}
\begin{example}[Fourier analysis]
	With $G=\op U(1)$, then one can calculate that $V_n\otimes V_n^*\to C^\infty(G)$ has image given by the $n$th power map $\op U(1)\to\op U(1)$. Thus, we are asserting that the collection of such polynomials are dense in the collection of all smooth functions $\op U(1)\to\CC$. If we identify $\op U(1)$ with $\RR/\ZZ$ via the exponential map, then this is asserting that the exponentials $z\mapsto e^{2\pi inz}$ have dense span in the collection of all smooth functions $\RR/\ZZ\to\CC$.
\end{example}
One can characterize the image of the Peter--Weyl map.
\begin{definition}[finite]
	A function $f\in C^\infty(G)$ is \textit{$G$-finite} if and only if the span of $\{gf:g\in G\}$ is finite-dimensional. We let $C_{\mathrm{fin}}(G)$ be the space of $G$-finite functions.
\end{definition}
\begin{remark}
	It turns out that the map
	\[\bigoplus_{\text{dominant }\lambda}V_\lambda\otimes V_\lambda^*\to C^\infty(G)\]
	has image given by the space of $G$-finite functions.
\end{remark}
\begin{remark}
	The vector space $C_{\mathrm{fin}}(G)$ has two ring structures: there is pointwise multiplication (in $\CC$) and also convolution given by
	\[(\varphi*\psi)\colon g\mapsto\int_G\varphi(x)\psi\left(x^{-1}g\right)\,dx,\]
	where $dx$ is a Haar measure for $G$ normalized so that $\int_Gdx=1$. (Because $G$ is compact, $dx$ is bi-invariant.)
\end{remark}
\begin{remark}
	The convolution algebra is non-unital, so one sometimes upgrades to the algebra of distributions, where we have the unit $1_1$. Similar remarks hold for $C^\infty(G)$ and even $C(G)$.
\end{remark}
\begin{remark}[algebraic groups]
	In fact, $C_{\mathrm{fin}}(G)$ also has a comultiplication given by pulling back along the multiplication map $m\colon G\times G\to G$. Namely, the comultiplication is the composite
	\[C_{\mathrm{fin}}(G)\stackrel{m^*}\to C_{\mathrm{fin}}(G\times G)=C_{\mathrm{fin}}(G)\otimes C_{\mathrm{fin}}(G).\]
	Thus, we have a Hopf algebra, which allows us to associate a complex algebraic group $G_{\mathrm{alg}}$ to $G$, and it turns out that unitary representations of $G$ all arise from algebraic representations of $G_{\mathrm{alg}}$. Conversely, one can take a complex algebraic group $G_{\mathrm{alg}}$ and then find a maximal compact subgroup $G\subseteq G_{\mathrm{alg}}(\CC)$ which is unique up to conjugacy.
\end{remark}

\subsection{Unitary Representations}
We will be interested in the unitary representations of Lie groups $G$, which we no longer assume to be compact.
\begin{remark}
	If $G$ is simple and not compact, then all unitary representations are infinite-dimensional. Proceeding by contraposition, suppose that $G$ is simple and admits a finite-dimensional unitary representation $\rho\colon G\to\op{SU}(n)$. Because $G$ is simple, this is an embedding, and because $\op{SU}(n)$ is compact, we conclude that $G$ must then also be compact.
\end{remark}
Thus, we see that we will be interested in infinite-dimensional representations. Of course, one still must add in topologies everywhere, though this point is more technical now that our vector spaces are not finite-dimensional. For example, for unitary representations, we are looking for actions of $G$ on Hilbert spaces, though we will find occasion to look at more general topological vector spaces.

Our main source of examples of representations arise from more general group actions.
\begin{example}
	If $G$ acts on a ``geometric space'' $X$, then we receive an induced action of $G$ on classes of functions on $X$. For example, when $X$ is a reasonably nice topological space, then we can think about $G$ acting on $L^2(X)$; when $X$ is a manifold, we can think about $G$ acting on $C^\infty(X)$.
\end{example}
\begin{example}
	The action of $G$ on $G$ itself by left multiplication gives rise to some ``regular'' representations.
\end{example}
\begin{example}
	Many matrix groups such as $\op{SL}(n,\RR)$ admit standard group action on a vector space. Note that this standard action may not be unitary!
\end{example}
Let's begin building our language.
\begin{defihelper}[subrepresentation, irreducible] \nirindex{subrepresentation} \nirindex{irreducible}
	Fix a group $G$ and topological vector space $V$. If $\rho\colon G\to\op{GL}(V)$ is a continuous representation, then a \textit{subrepresentation} is a closed subspace $W\subseteq V$ which is $G$-invariant. We say that $\rho$ is \textit{irreducible} if and only if there are no proper nontrivial subspaces.
\end{defihelper}
Note the hypothesis that $W\subseteq V$ is closed for our subrepresentations!
\begin{example}
	The action of $\op{SL}_n(\RR)$ on $\RR^n$ has no nontrivial proper subrepresentations and hence is irreducible.
\end{example}
\begin{remark}
	The action of $G$ on itself makes $L^2(G)$ a representation of $G$. However, this representation frequently fails to be irreducible. For example, $L^2(G)$ has many automorphisms, so it cannot be irreducible by a suitable version of Schur's lemma. In some cases, we can see this more concretely: taking $G=\RR$, then we know $\RR$ is isomorphic to its dual, so $L^2(\RR)\cong L^2\left(\RR^{2\lor}\right)$, and this right-hand side has more obvious subrepresentations given by the subspace of functions which vanish on a given subset of positive measure.
\end{remark}
Nonetheless, it is possible to write down some irreducible representations.
\begin{example}[Heisenberg group]
	Fix a positive integer $n\ge1$. Then we define the Heisenberg group $H_n$ as the matrix group
	\[H_n\coloneqq\left\{\begin{bmatrix}
		1 & a & c \\
		0 & 1_n & b \\
		0 & 0 & 1
	\end{bmatrix}:a\in\RR^{1\times n},b\in\RR^{n\times1},c\in\RR\right\}.\]
	It turns out that $H_n$ admits a natural action on $L^2(\RR^n)$, which is an irreducible representation. Quickly, the $a$-coordinate will act by translation on the $\RR^n$, and the $b$-coordinate will act by a character $b\mapsto e^{2\pi i\langle b,-\rangle}$. One finds a similar action on $C^\infty(\RR^n)$ and $\mc S(\RR^n)$.\todo{Check it}
\end{example}

\end{document}