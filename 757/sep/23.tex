% !TEX root = ../notes.tex

\documentclass[../notes.tex]{subfiles}

\begin{document}

\section{September 23}
Today we continue discussing the admissibility theorem.

\subsection{Real Structure Theory}
Last class, we defined the notion of a real semisimple group, but it will be convenient to work in the larger class of real reductive groups. We will take the following convention.
\begin{definition}[reductive]
	A \textit{real reductive group} is one of the form $G(\RR)$ where $G$ is some connected algebraic group defined over $\RR$ for which $G(\CC)$ is complex and reductive. Then a complex reductive group is a direct product of a semisimple group with a torus.
\end{definition}
\begin{remark}
	By descent, it is enough to only provide the data of the connected complex reductive algebraic group $G_\CC$ along with an involution $\theta$ of $G$. The real reductive group is then attained as $G_\CC^\theta$.
\end{remark}
\begin{remark}
	For a complex reductive group $G$, the commutator $G'$ is semisimple, and the map
	\[G'\times Z(G)^\circ\to G\]
	is surjective with finite kernel.
\end{remark}
\begin{example}
	Technically, we required our real semisimple groups to be connected, which is not preserved under base change. This means that there are real semisimple groups which are not reductive! For example, consider $\op{PGL}(n,\RR)$ is not a real semisimple Lie group according to our definitions because it fails to be connected---we should technically work with $\op{PGL}(n,\RR)^+$ instead, but that is not an algebraic subgroup and therefore not reductive. Approximately speaking, this is a failure of the group $\op{PGL}$ to be simply connected.
\end{example}
\begin{remark}
	In general, given a complex algebraic group $G_\CC$ with an involution $\theta$, the number of connected components of $G_\CC^\theta$ is finite and typically small. This can be studied with Galois cohomology: passing to the simply connected cover $\widetilde G$, it inherits an extended involution (which we also call $\theta$), and then the map $\widetilde g\mapsto\widetilde g^{-1}\theta(\widetilde g)$ is a $2$-cocycle in $\mathrm H^1(\op{Gal}(\CC/\RR),Z(\widetilde G))$.
\end{remark}
\begin{example}
	Here is a more serious problem: the group $\op{SL}(n,\RR)$ is not simply connected---it has fundamental group $\ZZ/2\ZZ$. Thus, it admits a covering by some $G\to\op{SL}(n,\RR)$, but this group $G$ fails to be algebraic and thus fails to be reductive. Indeed, $\op{Lie}G=\op{Lie}\op{SL}(n,\RR)=\mf{sl}(n,\RR)$, so any finite-dimensional representation $G\to\op{GL}(N,\RR)$, which induces $\mf{sl}(n,\RR)\to\mf{gl}(N,\RR)$, but this now extends to $\op{SL}(n,\CC)\to\op{GL}(N,\CC)$. Thus, our representation of $G$ factors through $\op{SL}(n,\RR)$ and thus cannot be faithful.
\end{example}
We are going to assume the following bit of structure theory.
\begin{theorem}[Cartan]
	Fix a real semisimple complex Lie group $G$. Then $G$ admits a Cartan involution $\theta\colon G\to G$, and $\theta$ is unique up to conjugation.
\end{theorem}
As discussed in \Cref{rem:cartan-to-compact}, the fixed points $K\coloneqq G^\theta$ form a compact subgroup of $G$; it turns out that $K^\theta$ is maximal compact.
\begin{remark}
	Once we are given $\theta$, we may extend it to a complex involution $\theta_\CC$ of $G_\CC$, and then we see $K_\CC=G_\CC^{\theta_\CC}$. There is also a complex anti-linear involution $\theta_0$ so that $G_\CC^{\theta_0}=G$, which then provides a bijection between real forms and such anti-linear involutions.
\end{remark}

\subsection{Proof of the Admissibility Theorem}
We now turn our attention towards the proof of \Cref{thm:hc-admissibility}. The first step is to establish a density result and a Schur's lemma.
\begin{lemma}
	Fix an associative algebra $A$, and let $V$ be a representation with $\op{End}_A(V)=\CC$. Then given any linearly independent vectors $v_1,\ldots,v_n$ and $w_1,\ldots,w_n\in V$, there is an $a\in A$ for which $a(v_i)=w_i$ for each $i$.
\end{lemma}
\begin{proof}
	Fix the vector $v\coloneqq(v_1,\ldots,v_n)$ in $V^{\oplus n}$, and we need to show that $Av=V^{\oplus n}$. Well, writing $V^{\oplus n}=\CC^{\oplus n}\otimes V$, it follows from $\op{End}_A(V)=\CC$ that every submodule looks like $U\otimes V$ where $U\subseteq\CC^n$ is a subspace: the tensor subcategory $\langle V\rangle^\otimes\subseteq\mathrm{Mod}_\CC(A)$ is seen to be equivalent to $\mathrm{Vec}_\CC$. However, $U\subsetneq\CC^n$ allows us to find a nonzero vector $u'\in U^\perp$, which in turn will have $u'\cdot v=0$, which contradicts the linear independence of $\{v_1,\ldots,v_n\}$.
\end{proof}
\begin{proposition}[Schur's lemma]
	Fix an irreducible Hilbert space representation $V$ of $\CC$ of a Lie group $G$. Then any bounded self adjoint operator is a scalar.
\end{proposition}
\begin{proof}[Sketch]
	The spectral theorem presents $V$ as $L^2(X,\mu)$ where $T$ corresponds to multiplication by a measurable function $\varphi$.

	Now, if $\varphi$ is not a scalar, then $X$ is not a point, so we can choose some $f\in C_c^\infty(G)$ for which $X_0\coloneqq(f\circ\varphi)^{-1}(\{0\})$ and $X_1\coloneqq(f\circ\varphi)^{-1}(\{1\})$ both have positive measure, so $X_0$ is a closed subspace invariant under $G$, which is our contradiction.
\end{proof}
Here is our density result.
\begin{corollary}[Density]
	Fix an irreducible Hilbert space representation $V$ of $\CC$ of a Lie group $G$. Then for any linearly independent vectors $v_1,\ldots,v_n\in V$ and some $w_1,\ldots,w_n\in V$ and some error bound $\varepsilon>0$, there is $\varepsilon>0$ and $\mu\in\op{Meas}_c(G)$ such that
	\[\norm{\varphi(v_i)-w_i}<\varepsilon.\]
\end{corollary}
\begin{proof}
	Once again, fix $v\coloneqq(v_1,\ldots,v_n)$, and we would like to check that the closure of $\op{Meas}_c(G)$ acting on $v$ achieves all $V^{\oplus n}$. Additionally, we still have that all closed subrepresentations of $V^{\oplus n}$ take the form $U\otimes V$ for some subset $U$. Indeed, if $W\subseteq V$ is a closed subrepresentation, then there is a bounded self-adjoint projection operator $\op{pr}\colon V\to W$. But Schur's lemma shows that endomorphisms of $V^{\oplus n}$ look like matrices, so the result follows after a little work.
\end{proof}
Next time, we will show the following.
\begin{proposition}
	Fix a finite-dimensional irreducible representation $V$ of $G$. Then $\dim V_\rho\le\dim\rho$ for irreducible representation $\rho$ of $K$.
\end{proposition}
We now move towards the completion of the proof.
\begin{lemma}
	For any $\mu\in\op{Meas}_c(G)$, there is a finite-dimensional irreducible representation of $G$ for which the action of $\mu$ on $V$ is nonzero.
\end{lemma}
\begin{proof}
	By smooth approximation to $\delta_1$, we may assume that $\mu=\varphi\,dg$ for some $\varphi\in C_c^\infty(G)$. Indeed, if $\mu$ acted by $0$ on all finite-dimensional irreducible representations, then $\mu*f\,dg$ would also act by $0$ always.\todo{}

	Now, consider the ring $\CC[G_\CC]$ of polynomial functions on $G_\CC$, which because $G$ is reductive, we see that $\CC[G_\CC]$ must be a sum of finite-dimensional irreducible representations. Now, by Stone--Weierstrass, there is a polynomial $P\in\CC[G_\CC]$ such that $\left|P(g)-f(g)\right|<\varepsilon$ for every $g\in\op{supp}f$. We will thus take the representation generated by $P$.

	It remains to check that the action of $\mu$ on $\ov P$ is nonzero, but this is given by
	\[\int_G f(g)\ov P(g)\,dg,\]
	which is approximately $\int_Gf(g)\ov f(g)\,dg$, which is of course positive.
\end{proof}
To conclude, we need the following identity.
\begin{theorem}[Amitsur--Levitzki--Kolchin]
	Choose some $n\times n$ matrices $A_1,\ldots,A_{2n}$. Then
	\[\sum_{\sigma\in S_{2n}}\left(-1\right|^{\left|\sigma\right|}A_{\sigma(1)}\cdots A_{\sigma(2n)}=0.\]
\end{theorem}
To explain how this helps us, if we choose an irreducible representation $\rho$ of $K$, then any measures $\mu_1,\ldots,\mu_{2n}$ satisfy the above identity on the isotypic components. Then we are able to conclude by the density theorem: if the dimension of the $\rho$-isotypic component was too large, then the density result provides a measure with large image, allowing us to conclude.

\end{document}