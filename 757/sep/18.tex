% !TEX root = ../notes.tex

\documentclass[../notes.tex]{subfiles}

\begin{document}

\section{September 18}
Today, we talk about elliptic operators.

\subsection{Applications of Weakly Analytic Vectors}
Here is the local version of our statement.
\begin{theorem}[Elliptic regularity]
	Fix an open subset $U\subseteq\RR^n$ and a smooth function $f$ on $U$. Suppose that $d$ is an elliptic differential operator with real analytic coefficients and for which $df=0$. Then $f$ is real analytic.
\end{theorem}
\begin{example}
	Take $n=1$ so that $d$ looks like $\sum_{i=0}^nf_i\del_x^i$. Being elliptic amounts to requiring that $\varphi_n(x)\ne0$ for all $x$, so after cancelling out, we are solving the differential equation
	\[f^{(n)}+\varphi_{n-1}f^{(n-1)}+\cdots+\varphi_0=0,\]
	and we are asking for $f$ to be real analytic when the $\varphi_\bullet$s are. This follows from some theory of first-order differential equations.
\end{example}
\begin{example}
	With $U=\RR^n$, we see $d$ looks like $\sum_If_I\del_x^I$. Suppose that all the $f_I$s are constant except for $f_0$, which we ask to be Schwartz. Then by shifting $f$, we are trying to solve $df=f_0$, which one can solve by taking a Fourier transform.
\end{example}
By a partition of unity argument, one can prove a global statement.
\begin{corollary} \label{cor:global-elliptic-reg}
	Fix a real analytic manifold $X$ and a smooth function $f$ on $X$. Suppose that $d$ is an elliptic differential operator with real analytic coefficients and for which $df=0$. Then $f$ is real analytic.
\end{corollary}
And here is our application.
\begin{proof}[Proof of \Cref{thm:h-c-analytic}]
	We proceed in steps.
	\begin{enumerate}
		\item We construct a differential operator $C$ on $G$. Fix a left-invariant Riemannian metric on $G$; this amounts to the choice of some $B\in\op{Sym}^2(\mf g^*)$. By suitably averaging over $K$, we can assume that it is also right-invariant by $K$. Then $B^{-1}$ is a $K$-invariant element of $\op{Sym}^2(\mf g)$, which then maps to some $K$-invariant quadratic element $C$ of $U\mf g$. Now, $\mf g$ acts by first-order differential operators on $G$, so $U\mf g$ acts by higher-order differential operators; for example, $C$ is a differential operator.

		\item We claim that $C$ is elliptic. Upon identifying $T^*G\cong G\times\mf g^*$ via right translations, the principal symbol of $C$ is exactly the quadratic element $B^{-1}$ that we started with: we find $\sigma(d_C)(g,\xi)=B^{-1}(\xi,\xi)$, which of course is positive for every $g$ and $\xi$.

		\item We complete the proof. Fix $v$ and $\xi$, and we note that $C(f_{v,\xi})=f_{C(v),\xi}$ by a direct calculation. Further, the action of $C$ commutes with the $K$-action, so it preserves the finite-dimensional $K$-isotypic components. Accordingly, we fix some irreducible representation $\rho$ of $K$, let $P$ be the minimal polynomial of $C$ acting on $V_\rho$, and we see that $P(C)f_{v,\xi}=0$. But $P(C)$ is still elliptic (its symbol is just the $(\deg P)$th power of the symbol of $C$), so we are done by \Cref{cor:global-elliptic-reg}.
		\qedhere
	\end{enumerate}
\end{proof}
This has many nice applications.
\begin{corollary} \label{cor:pass-to-g-k}
	Fix a Lie group $G$ with compact subgroup $K\subseteq G$, and let $V$ be an admissible Fr\'echet representation.
	\begin{listalph}
		\item The action of $G$ on $V$ is uniquely determined by its action of $\mf g$ on $V_{\mathrm{fin}}$.
		\item Closed subrepresentations of $V$ are in bijection with $(\mf g,K)$-submodules of $V_{\mathrm{fin}}$.
		\item The functor $(-)_{\mathrm{fin}}\colon\mathrm{Rep}(G)\to\mathrm{Mod}(\mf g,K)$ is faithful.
	\end{listalph}
\end{corollary}
\begin{proof}
	We show these independently.
	\begin{listalph}
		\item Because $V_{\mathrm{fin}}\subseteq V$ is dense, the action is determined by the action on the subspace $V_{\mathrm{fin}}$. But the action of $g$ on some $v\in V_{\mathrm{fin}}$ is then determined by the Taylor series of $\ell(gv)$ as $\ell$ varies over all $\ell\in V^*$ (because $g\mapsto\ell(gv)$ is real analytic!). These Taylor series coefficients can now be read off of the derivatives.

		\item In one direction, we can take a closed subrepresentation $W\subseteq V$ and produce the $(\mf g,K)$-submodule $W_{\mathrm{fin}}\subseteq V_{\mathrm{fin}}$.
		
		In the other direction, for any $W\subseteq V_{\mathrm{fin}}$ which is a $(\mf g,K)$-module, we claim that the closure $\overline W\subseteq V$ is a $G$-submodule. Well, for $g\in G$ and $w\in W$, we would like to check that $gw\in\overline W$, for which it is enough to check that $\ell(gw)=0$ for every linear functional $\ell$ vanishing on $W$. But the function $g\mapsto\ell(gw)$ is real analytic, so we may determine it by determining its Taylor expansion at $g=1$, which passes to the Lie algebra action on $W_{\mathrm{fin}}$ and can be seen to vanish.

		Lastly, we need to check that these maps are mutually inverse, which we omit.

		\item This follows from density of $V_{\mathrm{fin}}$ in $V$.
		\qedhere
	\end{listalph}
\end{proof}
\begin{remark}
	One cannot expect $(-)_{\mathrm{fin}}$ to be fully faithful because the same pair $(\mf g,K)$ can arise from different groups $G$.
\end{remark}
\begin{remark}
	The proofs of (a) and (b) do not use the $K$-action anywhere, so we can actually go down to just $\mf g$-modules. This produces a stronger statement when $K$ fails to be connected.
\end{remark}
We will also be able to prove a variant of Schur's lemma.
\begin{lemma}[Dixmier] \label{lem:dixmier}
	Fix an associative $\CC$-algebra $A$, and suppose that $M$ is a simple $A$-module with at most countable dimension. Then $\op{End}_A(M)=\CC$.
\end{lemma}
\begin{proof}
	For brevity, set $D\coloneqq\op{End}_A(M)$. Because $M$ is simple, we know that $D$ is a division algebra over $\CC$: any nonzero element $\varphi\colon M\to M$ must have trivial kernel and full image and therefore be invertible. Further, for any nonzero $m\in M$, we see that $m$ generates $M$, so the map $D\to M$ given by $\varphi\mapsto\varphi(m)$ is injective, so $\dim D\le\dim M$. It follows that $D$ is a division algebra over $\CC$ with at most countable dimension over $\CC$.

	Now, suppose for the sake of contradiction that $D\ne\CC$. Then because $\CC$ is algebraically closed, $D$ contains an element $t\in D$ which is transcendental over $\CC$. However, $\CC(t)$ has uncountable dimension over $\CC$; for example, the elements $\left\{\frac1{t-a}\right\}_{a\in\CC}$ are linearly independent.
\end{proof}
\begin{corollary}[Schur's lemma]
	Fix a Lie group $G$ with compact subgroup $K\subseteq G$, and let $V$ be an admissible Fr\'echet representation. If $V$ is irreducible, then $\op{End}_G(V)=\CC$.
\end{corollary}
\begin{proof}
	By \Cref{cor:pass-to-g-k}, we see that $V_{\mathrm{fin}}$ is an irreducible $(\mf g,K)$-module, from which we see that it is enough to show that $\op{End}_{\mf g}(V_{\mathrm{fin}})=\CC$. The result now follows from \Cref{lem:dixmier} with the algebra $U\mf g$.
\end{proof}
\begin{remark}
	If $M$ is finite-dimensional, then there is another proof: any endomorphism $\varphi\colon M\to M$ still has an eigenvalue $\lambda$, and the eigenspace $M[\lambda]$ is a nonzero submodule of $M$, so $M[\lambda]=M$, so $\varphi=\lambda\id_M$.
\end{remark}

\subsection{The Admissibility Theorem}
Here is our next goal.
\begin{theorem}[Harish-Chandra]
	Fix a real semisimple group $G$ with maximal compact subgroup $K$. If $V$ is an irreducible unitary representation, then
	\[\dim V_\rho\le\dim\rho\]
	for each irreducible representation $\rho$ of $K$. In particular, $V$ is admissible.
\end{theorem}
This is the first time we actually want $K$ to be a maximal compact subgroup, which is why we are asking for $G$ to be semisimple.

Let's begin by recalling some definitions.
\begin{definition}[semisimple]
	Fix a connected real Lie group $G$. Then $G$ is \textit{semisimple} if and only if $G$ embeds into some $\op{GL}(n,\RR)$ (i.e., $G$ is linear) and has semisimple Lie algebra.
\end{definition}
\begin{definition}[split]
	Fix a connected real Lie group $G$. Then $G$ is \textit{split} if and only if $G$ contains a maximal torus $T$ isomorphic to $\left(\RR^\times\right)^n$.
\end{definition}
\begin{example}
	The group $\op{SL}(n,\RR)$ is split and semisimple with maximal torus given by the diagonal matrices.
\end{example}
\begin{example}
	The group $\op{SO}(p,q)$ is semisimple, but it only succeeds at being split when $p-q\in\{-1,0,+1\}$.
\end{example}
Real groups come with a Cartan involution.
\begin{definition}[Cartan involution]
	Fix a semisimple real Lie group $G$ with Lie algebra $\mf g$. Consider the Killing form $(-,-,)$ of a real semisimple Lie algebra $\mf g$. A \textit{Cartan involution} is an involution $\theta$ of $G$ for which the form $C(-,-)\coloneqq(-,\theta-)$ is positive-definite.
\end{definition}
\begin{remark}
	It turns out that such involutions exist and are unique up to conjugation.
\end{remark}
\begin{remark}
	If $(\theta(x),\theta(y)=(x,y)$, then we find that $C(x,y)=C(y,x)$.
\end{remark}
\begin{remark}
	The subgroup $K\coloneqq G^\theta$ acts on $\mf g$ while preserving $C$, so it follows that $K$ embeds into some $\op{SO}(n)$ and is therefore a compact subgroup. It turns out that $K$ is a maximal compact subgroup.
\end{remark}
\begin{example}
	With $G=\op{SL}(n)$, one finds that $\theta(g)\coloneqq g^{-\intercal}$ is a Cartan involution, so one finds that $K=\op{SO}(n)$.
\end{example}
\begin{example}
	If $K$ is a compact group, then we can try out $G=K_\CC$, and one can check that we get a Cartan involution $\theta$ which acts by complex conjugation on $G$ and therefore on $\op{Lie}G=\op{Lie}K\otimes\CC$ so that $K=G^\theta$ again.
\end{example}

\end{document}