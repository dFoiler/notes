% !TEX root = ../notes.tex

\documentclass[../notes.tex]{subfiles}

\begin{document}

\section{October 14}
Today we discuss the Harish-Chandra isomorphism.

\subsection{The Harish-Chandra Isomorphism}
We now try to describe the center of $U\mf g$.
\begin{notation}
	Fix a Lie algebra $\mf g$. Then we let $Z\mf g$ denote the center $Z(U\mf g)$.
\end{notation}
Well, for $z\in Z\mf g$, we see that $z$ must act on $M_\lambda$ by a scalar (namely, the scalar given by the image of $v_\lambda$).
\begin{definition}
	Fix a complex reductive Lie algebra $\mf g$, and fix a Cartan subalgebra $\mf h\subseteq\mf g$. Then we define the \textit{shifted Weyl action} of $W$ on $\mf h^*$ by
	\[w\cdot\lambda\coloneqq w(\lambda+\rho)-\rho,\]
	where $\rho$ is half the sum of the positive roots.
\end{definition}
\begin{theorem}[Harish-Chandra isomorphism] \label{thm:hc-iso}
	Fix a complex reductive Lie algebra $\mf g$ with Cartan subalgebra $\mf h\subseteq\mf g$. For $z\in Z\mf g$, let $P_z(\lambda)$ be the scalar by which $z$ acts on $M_\lambda$.
	\begin{listalph}
		\item For $z\in Z\mf g$, the function $P_z(\lambda)$ is a polynomial in $\lambda\in\mf h^*$, invariant under the shifted Weyl action.
		\item The map $z\mapsto P_z$ produces an isomorphism $Z\mf g\cong\CC[\mf h^*]^{W,\cdot}$.
	\end{listalph}
\end{theorem}
\begin{remark}
	Here, $\CC[\mf h^*]$ means the polynomials on $\mf h^*$, note polynomials with coefficients in $\mf h^*$.
\end{remark}
Here is a related tool.
\begin{theorem}[Chevalley restriction] \label{thm:chev-res}
	Fix a complex reductive Lie group $G$ with Lie algebra $\mf g$ with Cartan subalgebra $\mf h\subseteq\mf g$. Then the restriction map $\CC[\mf g]\to\CC[\mf h]$ restricts to an isomorphism
	\[\CC[\mf g]^G\to\CC[\mf h]^W.\]
\end{theorem}
\begin{example}
	With $\mf g=\mf{gl}(n)$, we can show that $\CC[\mf g]^G$ is generated by the polynomials $X\mapsto\tr\land^iX$ as $i$ varies. Accordingly, these are precisely the symmetric polynomials when restricted to the diagonal matrices $\mf h\subseteq\mf g$.
\end{example}
\begin{proof}[Proof of \Cref{thm:chev-res}]
	Let $T\subseteq G$ be the Lie subgroup corresponding to $\mf h\subseteq\mf g$. Identifying $W$ with $N(T)/T$ in the usual way, we see that $\CC[\mf g]^G$ of course maps to $\CC[\mf h]^W$. We now check injectivity and surjectivity separately.
	\begin{itemize}
		\item Injective: if $P\in\CC[\mf g]^G$ vanishes on $\mf h$, then $P$ vanishes on every semisimple element in $\mf g$ because any semisimple element can be placed in a Cartan subalgebra and then conjugated to $\mf h$. Thus, $P=0$ because semisimple elements are dense in $\mf g$.
		\item Surjective: note $\CC[\mf h]$ is spanned by the polynomials $\lambda^n$ where $\lambda$ is a dominant integral weight (indeed, we may even take $\lambda$ to consist of fundamental weights, which provide a basis of $\mf h^*$), so $\CC[\mf h]^W$ is spanned by the polynomials
		\[Q_{\lambda,n}\coloneqq\sum_{w\in W}(w\lambda)^n.\]
		We now extend this polynomial to an element of $\CC[\mf g]^G$. Well, define $P_{\lambda,n}$ by
		\[P_{\lambda,n}(x)\coloneqq\tr\left(x^n;V_\lambda\right),\]
		where $V_\lambda$ is the finite-dimensional irreducible representation of highest weight $\lambda$. The representations $V_\lambda$ are finite-dimensional, so $P_{\lambda,n}$ is in fact a polynomial, and we can see that it is invariant under the $G$-action because we took a trace. When restricted to $\mf h$, we see that $P_{\lambda,n}$ is a sum of $Q_{\lambda,n}$ plus some $Q_{\mu,n}$s for $\mu<\lambda$ (which are the other weights of $\lambda$). Thus, one can induct to show that $Q_\lambda$ is in the image.
		\qedhere
	\end{itemize}
\end{proof}
\begin{remark}
	Here is one way to view the construction at the end of the argument: from Peter--Weyl, we expect $\CC[G]$ to decompose into sums of $V_\lambda\otimes V_\lambda^*$, but $V_\lambda\otimes V_\lambda^*$ corresponds to the function given by the character.
\end{remark}
\begin{corollary} \label{cor:apply-chev-res}
	Fix a complex reductive Lie group $G$ with Lie algebra $\mf g$ with Cartan subalgebra $\mf h\subseteq\mf g$. For any dominant integral weight $\lambda$, the group $\op{Hom}_G(V_\lambda,\CC[\mf g])$ is finitely generated over $\CC[\mf g]^G$.
\end{corollary}
\begin{proof}
	Note $\op{Hom}_G(V_\lambda,\CC[\mf g])=(V_\lambda^*\otimes\CC[\mf g])^G$ which then injects to $V_\lambda^*\otimes\CC[\mf h]$ by the same injectivity argument given in \Cref{thm:chev-res}. By some geometric invariant theory, we see that $\CC[\mf h]$ is finitely generated over $\CC[\mf h]^W$.\footnote{More generally, if a finite group $\Gamma$ acts on a complex vector space $V$, then any $f\in\CC[V]$ is the root of the polynomial $\prod_{\gamma\in\Gamma}(t-\gamma(f))=0$, so $f$ is the root of a polynomial in $\CC[V]^\Gamma[t]$.} The result now follows because $V_\lambda^*$ is also finitely generated.
\end{proof}
\begin{proof}[Proof of \Cref{thm:hc-iso}]
	For (a), fix some $z\in Z\mf g$, and let's begin by showing that $P_z(\lambda)$ is a polynomial in $\lambda\in\mf h^*$. Well, recall that $M_\lambda$ has the single generator $v_\lambda$, which is a singular vector of weight $\lambda$. Now, the PBW theorem tells us that elements of $U\mf g$ can be written uniquely as a polynomial in the basis $\{f_\alpha\}_\alpha\sqcup\{h_i\}_i\sqcup\{e_\alpha\}_\alpha$. In particular, $z$ takes the form $P$ for some polynomial $P$ in the $h_i$s plus some monomials using at least one $f_\alpha$ or $e_\alpha$. However, $z$ must be in weight $0$ because it is central: indeed, commuting with various $h\in\mf h$ is the same as saying that $z$ has weight $0$. Thus, all other monomials outside $P(h)$ contain at least one of $e_\alpha$, which automatically kills $v_\lambda$. We conclude that
	\[P_z(\lambda)=P(\lambda),\]
	and now the right-hand side is definitionally a polynomial in $\lambda$.

	We next check that $P_z(\lambda)$ is invariant under the shifted Weyl action. On the homework, we showed that there is a canonical map $M_{s_i\cdot\lambda}\to M_\lambda$ whenever $\langle\lambda,\alpha_i^\lor\rangle\ge0$. We conclude that $P_z(\lambda)=P_z(s_i\cdot\lambda)$ in this case. Letting $\lambda$ vary over all dominant integral weights (which are Zariski dense), the invariance follows.

	It remains to check that we have actually provided an isomorphism $Z\mf g\to\CC[\mf h^*]^{W,\cdot}$. For this, we will use \Cref{thm:chev-res}. By the PBW theorem, there is an isomorphism $\CC[\mf g^*]\to U\mf g$ of $G$-modules (where we are using the adjoint action). Thus, we get an isomorphism
	\[\CC[\mf g^*]^G\to(U\mf g)^G=Z\mf g\]
	of vector spaces. Applying \Cref{thm:chev-res} (and noting that there is an isomorphism $\mf g\cong\mf g^*$ by using an invariant form), we receive an isomorphism $\CC[\mf g^*]^G\to\CC[\mf h^*]^W$. We now claim that the composite
	\[\CC[\mf h^*]^W\from\CC[\mf g^*]^G\to(U\mf g)^G=Z\mf g\to\CC[\mf h^*]^W\]
	is the identity ``up to lower-order terms.'' Namely, we claim that a homogeneous polynomial $P\in\CC[\mf h^*]^W$ of degree $d$ goes to the same polynomial $P$ only with lower-order terms, which then completes the argument because it forces the full composite to be an isomorphism. To prove the claim, we see that it is enough to check it on polynomials of $\CC[\mf g^*]$ which have been averaged over $S_n$. However, this averaging process does not change the highest-order term, and the shifted action by $W$ also does not change the highest-order term, so we are done!
\end{proof}
% There is more that can be said about the center.

% \begin{remark}
% 	It turns out that many of these arguments about the adjoint action of $G$ on $\mf g$ can be translated into similar results about the action of $K_\CC$ on $\mf p=\mf k^\perp$. To explain the notation, we are fixing some real form $G_\theta$ where $\theta$ is some involution, and then we have a decomposition $\mf g=\mf k\oplus\mf p$, where $\theta$ acts by $+1$ on $\mf k$ and by $-1$ on $\mf p$. Lastly, $K\subseteq G$ is the subgroup corresponding to $\mf k\subseteq\mf g$. In this situation, it turns out that a generic $X\in\mf p$ with centralizer $\mf g_X$ makes $\mf g_X\cap\mf p$ an abelian semisimple subalgebra, which is the torus one should now use (instead of the Cartan). For example, one can argue as in \Cref{cor:apply-chev-res} to show that
% 	\[\op{Hom}_K(V_\nu,\CC[\mf p])\]
% 	is finitely generated over $\CC[\mf g]^G$, which in turn implies the admissibility of $(\mf g,K)$-modules.
% \end{remark}

\end{document}