% !TEX root = ../notes.tex

\documentclass[../notes.tex]{subfiles}

\begin{document}

\section{October 23}
We continue.

\subsection{The Kostant Slice}
Our story continues with the Kostant slice.
\begin{definition}[principal triple]
	Fix a semisimple Lie algebra $\mf g$ with Cartan subalgebra $\mf h\subseteq\mf g$. Then we define the $\mf{sl}(2)$-triple $(e,f,h)$ so that $e\coloneqq\sum_ie_i$ and $h\coloneqq2\rho^\lor$. This is the \textit{principal $\mf{sl}(2)$} triple.
\end{definition}
\begin{example}
	If $\mf g=\mf{sl}(n)$, then $e$ is the nilpotent matrix supported only just above the diagonal, where it has $1$. In other words, $e$ is a single Jordan block with eigenvalue $0$. Note that the centralizer $\mf g_e$ consists of powers of $e$, which can be checked directly.
\end{example}
\begin{definition}[Kostant slice]
	Fix a semisimple Lie algebra $\mf g$ with Cartan subalgebra $\mf h\subseteq\mf g$, and let $(e,f,h)$ be the principal $\mf{sl}(2)$-triple. Then we define the \textit{Kostant slice} to be $S\coloneqq f+\mf g_e$.
\end{definition}
\begin{theorem} \label{thm:kostant-slice}
	Fix a semisimple Lie algebra $\mf g$ with Cartan subalgebra $\mf h\subseteq\mf g$, and let $(e,f,h)$ be the principal $\mf{sl}(2)$-triple with Kostant slice $S$.
	\begin{listalph}
		\item There is an isomorphism $S\to\mf t/W$ given by restriction, meaning that the map $\CC[\mf g]^G\to\CC[S]$ is an isomorphism.
		\item There is an isomorphism $S\to(f+\mf b)/U$ given by inclusion, and the action map $(U\times S)\to(f+\mf b)$ induces an isomorphism, where $\mf b$ is the positive Borel subalgebra.
	\end{listalph}
\end{theorem}
\begin{remark}
	This is related to Whittaker models.
\end{remark}
\begin{remark}
	The Borel $\mf b\subseteq\mf g$ is unique once we require that it contains $e$. (Such a Borel certainly exists by using the simple roots in $e$.) This $\mf b$ turns out to be the sum of eigenspaces where $h$ acts with nonnegative eigenvalues; this subspace actually only depends on $e$ because it is the subspace on which $e^m$ acts by zero, where $m$ is chosen to give us half the total dimension.
\end{remark}
\begin{example}
	If $\mf g=\mf{sl}(n)$, then $S$ consists of $f$ plus the upper-triangular matrices which are constant on the diagonals. One can use this description to prove (a) by hand.
\end{example}
\begin{lemma} \label{lem:get-kostant-slice}
	Fix free homogeneous generators $P_1,\ldots,P_r$ of $\CC[\mf g]^G$, and consider the maps $dP_i\colon\mf g\to\mf g$ given by the differential. Then $dP_i|_x\in T_x\mf g_x$ for all regular $x$, and these elements form a basis.
\end{lemma}
\begin{proof}
	Here, $dP_i$ is technically a map $\mf g\to\mf g^*$. The fact that $dP_i(x)\in\mf g_x$ follows because the polynomial $P_i$ is $G$-invariant. For example, $P_i(\op{Ad}_{\exp(y)}x)=P_i(x)$ for all $y$, so we conclude that $dP_i$ vanishes on $[\mf g,x]$. However, under the identification $\mf g=\mf g^*$, we see that $[\mf g,x]$ is the complement of $\mf g_x$: an element $z$ commutes with $x$ if and only if $([x,z],y)=0$ for all $y$, which in turn is equivalent to $(z,[x,y])=0$ for all $y$, which means that $z$ is in the complement of $[\mf g,x]$. (The implications also reverse.)

	Consider the map $\mf g\to\mf h/W$ induced by Chevalley restriction. For a fixed $x$, we may write $\mf g=V\oplus[\mf g,x]$ for some complement $V$ (say, under the Killing form); the previous paragraph reveals that $V=\mf g_x$. Then we claim that the $dP_i|_x$s span if and only if the composite
	\[V\stackrel x\to\mf g\to\mf h/W\]
	is an isomorphism on tangent spaces, where the left map is given by $v\mapsto(x+v)$. Indeed, this is for size reasons: $\dim V=\dim\mf g_x=r$ because $x$ is regular, and the tangent spaces of $\mf h/W$ are also $r$-dimensional for regular $x$ (because $\dim\mf h=r$). By Chevalley restriction, the $dP_i$s span the target tangent space, so the above isomorphism will then guarantee that they span (equivalently, form a basis of) the source tangent space. % Indeed, under the identification of $\mf g$ with $\mf g^*$, we see that $[\mf g,x]$ is the complement of $\mf g_x$: an element $z$ commutes with $x$ if and only if $([x,z],y)=0$ for all $y$, which in turn is equivalent to $(z,[x,y])=0$ for all $y$, which means that $z$ is in the complement of $[\mf g,x]$. (The implications also reverse.) The point is that $V=[\mf g,x]^\perp=\mf g_x$

	% Let $\mf g^{\mathrm{reg}}$ consist of the regular elements, which we know to be Zariski open. For fixed regular $x$, we note that $\mf g$ has a natural diagonal action on the trivial bundle $\underline{\mf g}$ on $\mf g$ (thought of as $\mf g\times\mf g$), so we may define a subbundle $\underline{\mf g_x}\subseteq\underline{\mf g}$ as the centralizer of $x$. We are given $r$ sections $\{dP_i\}$ of this vector bundle, and we would like to show that they trivialize the bundle. For rank reasons, it is enough to show that they span.

	We are only going to show the spanning away from a codimension-$2$ subset of the regular locus $\mf g^{\mathrm{reg}}$ in $\mf g$. To see that this is enough, we consider the vector bundle $\mc V$ over $\mf g^{\mathrm{reg}}$ with fibers given by $\mf g_x$.\footnote{In short, this vector bundle arises by consider the kernel of the natural action of $\mf g$ on the trivial bundle $\mf g$ over $\mf g$ given by the adjoint. This is a vector bundle over the regular locus.} Then the spanning of the sections $dP_i$ amount to saying that the induced map $\pi\colon\CC^r\to\mc V$ is an isomorphism away from a codimension-$2$ subset of $\mf g^{\mathrm{reg}}$. However, this means that the polynomial $\land^r\pi\colon\mf g^{\mathrm{reg}}\to\CC$ is nonzero away from a codimension-$2$ subset, so it must be nonzero everywhere!
	
	Thus, we are allowed to ignore a codimension-$2$ subset of regular elements in $\mf g$. We now handle our $x\in\mf g$ in cases.
	\begin{itemize}
		\item We work with regular semisimple elements $x$. Here, we may choose $\mf h$ to contain $x$. Because $x$ is regular, we see $\mf h=\mf g_x$. We thus would like to check that
		\[\mf h\stackrel x\to\mf g\to\mf h/W\]
		is an isomorphism on tangent spaces. In other words, we would like to show that the projection $\mf h\to\mf h/W$ is an isomorphism on tangent spaces at $x$.
		
		Indeed, this quotient is \'etale at $x$ by regularity, so this is not so bad. Alternatively, we can show the spanning directly. Note that $W$ acts freely on the orbit of $x$ by the regularity, so regularity provides us with an isomorphism $T^*_x(\mf t/W)\to T^*_x(\mf h)$, which means that any $\xi\in\mf h^*$ can be lifted to an invariant polynomial in $\CC[\mf t]^W$. (Indeed, choose $P_0$ with $dP_0|_x=\xi$ and vanishing at the rest of the orbit of $Wx$, and then $P=\left|W\right|^{-1}\sum_wwP_0$ will work.)
		%Set $\mf t\coloneqq\mf g_x$ to be a maximal torus. Consider the projection $\pi\colon\mf g\to\mf g/G$, which restricts to $\mf t\to\mf t/W$ (where the quotients mean invariants on the polynomial algebras). Note that $W$ acts freely on the orbit of $x$ by the regularity, so regularity provides us with an isomorphism $d\pi\colon T^*_x(\mf t/W)\to T^*_x(\mf t)$, which means that any $\xi\in\mf t^*$ can be lifted to an invariant polynomial in $\CC[\mf t]^W$. (Indeed, choose $P_0$ with $dP_0|_x=\xi$ and vanishing at the rest of the orbit of $Wx$, and then $P=\left|W\right|^{-1}\sum_wwP_0$ will work.) This polynomial is then lifted to $\CC[\mf g]^G$, completing the proof.

		\item We now work with generic elements $x$ which are regular but not semisimple. The Jordan decomposition allows us to write $x=s+n$ for semisimple $s$ and nilpotent $n$; we go ahead and extend $s$ to a Cartan subalgebra $\mf h$. Now, we cannot have that $s$ is regular, so a generic such $x$ will admit a unique root $\alpha$ for which $\langle s,\alpha\rangle=0$. Because $s$ and $n$ are forced to commute, we can see from the root decomposition of $\mf g$ that $n$ equals $e_\alpha$ up to a scalar, so we may as well assume $n=e_\alpha$. To compute $\mf g_x$, we note that $\mf g_x\subseteq\mf g_s$ by the uniqueness of the Jordan decomposition, and
		\[\mf g_s=\alpha^\perp\oplus\mf{sl}(2)_\alpha.\]
		It follows that $\mf g_x=\alpha^\perp\oplus\CC f_\alpha$ by further asking which elements of $\mf g_s$ (given above) commute with $n=e_\alpha$.

		% Fix a Cartan subalgebra $\mf h$, and consider elements of the form $x=s+e$, where $s\in\mf h$ has $\langle s,\alpha\rangle=0$ for some simple root $\alpha$ but nonzero elsewhere, and $e\coloneqq e_\alpha$. One can check that $x$ is regular by computing its centralizer, which is most of $\mf t$ but has some issue at $e$. For example, there is a natural projection $\mf b\to\mf t$ which induces a map $\mf z(x)\to\mf t$ because $\mf z(x)$ lives in $\mf b$.

		By a similar argument as in the previous point, we find that $\mf h/\langle s_\alpha\rangle\to\mf h/W$ has non-degenerate differential at the image of $x$ (namely, at $s$). On the other hand, the map $\mf g_x\to\mf h/\langle s_\alpha\rangle$ is the identity on $\alpha^\perp$, but we have an additional piece from $f_\alpha$ which makes up for our last lost dimension after reducing the story to $\mf{sl}(2)_\alpha$.
		\qedhere
	\end{itemize}
\end{proof}
\begin{example}
	With $\mf g=\mf{sl}(n)$, then $P_i(X)=\tr X^{i+1}$ (for $i\in\{1,2,\ldots,n-1\}$). We claim that $dP_i(X)=(i+1)X^i$. Because diagonalizable matrices are Zariski open, we may restrict our attention to diagonal matrices, where
	\[P_i(\op{diag}(a_1,\ldots,a_n))=\sum_ja_j^{i+1}.\]
	Thus, $dP_i=\sum_j(i+1)a_j^i\,da_j$, which under the identification $\mf g\to\mf g^*$ goes to the claimed matrix.
\end{example}
\begin{proof}[Proof of \Cref{thm:kostant-slice}(a)]
	We show the parts separately. For the injectivity, it is enough to check that $G(S)$ contains a dense open subset in $\mf g$, for which it is enough to check that $T(G\times S)\to T\mf g=\mf g$ is surjective at any given point. For example, at $(1,f)\mapsto f$, this map is simply $\mf g\oplus\mf g_e\to T_f\mf g=\mf g$ given by $(x,s)\mapsto([f,x]+s)$. The surjectivity now follows from the representation theory of $\mf{sl}(2)$: every vector in a representation is either in the image of $\op{ad}_f$ or in the kernel of $\op{ad}_e$ (which is $\mf g_e$).
		
	We now turn to surjectivity, for which we use \Cref{lem:get-kostant-slice}. We may as well consider the composite $\CC[\mf t]^W\cong\CC[\mf g]^G\to\CC[S]$, and we will use graded Nakayama. Note that a grading is equivalent data to an action by $\mathbb G_{m,\CC}$, meaning that the $n$th graded piece has an action by $(-)^n$.

	As such, we would like to give $\CC[S]$ an action by $\mathbb G_m$ as well. The obvious scaling won't work because it moves $f$. TO fix this, lift $\mf{sl}(2)\to\mf g$ to some representation $\op{SL}(2)\to G$, and we will have $t$ act by the image of $\op{Ad}(\op{diag}(\sqrt t,1/\sqrt t))$. Explicitly,
	\[t\cdot s\coloneqq\op{Ad}_{\op{diag}(\sqrt t,1/\sqrt t)}(ts).\]
	Technically, this formula does not look algebraic because of the square roots, but it is: the principal representation $\mf g$ of $\mf{sl}(2)$ has only even weights, so the action of $\op{SL}(2)$ will factor through $\op{SL}(2)/\{\pm1\}=\op{PGL}(2)$. As such, the action on $\CC[S]$ is given by $t$ acting by $\op{Ad}(\op{diag}(t,1))(ts)$.
	
	Now, $\CC[S]$ is a polynomial algebra with generators in degrees $m_i+1$, where $\{2m_1,\ldots,2m_r\}$ are the highest weights of $\mf g$ as the principal representation. Indeed, just $m_i$ would come from acting on $S$ by $ts$, but the extra adjoint shifts by $1$.

	We now are equipped with an inclusion $\CC[\mf t]^W\into\CC[S]$ of positively graded polynomial algebras, and \Cref{lem:get-kostant-slice} tells us that the differential map $S\to\mf t/W|_f$ is an isomorphism. Unwinding the definition of the tangent space, we have an isomorphism
	\[\frac{\CC[\mf t]^W_+}{\left(\CC[\mf t]^W\right)_+^2}\onto\frac{\CC[S]_+}{\left(\CC[S]\right)_+^2},\]
	which now upgrades to an isomorphism of the algebras by graded Nakayama (i.e., doing an induction on the degree).
\end{proof}
\begin{remark}
	The proof of (a) shows that $\deg P_i=m_i+1$. Indeed, we know that the generators of each of the polynomial algebras $\CC[S]$ and $\CC[\mf t]^W$ must have the same degree!
\end{remark}
\begin{remark}
	Let's contextualize part (b). Under the identification $\mf g\cong\mf g^*$, the space $\mf b$ goes to $\mf u^\perp$ because $\mf b=\mf t\oplus\mf u$. As such, $f+\mf b$ is identified with
	\[\{\xi:(\xi,x)=(f,x)\text{ for }x\in\mf u\}.\]
	Now, note that $(f,x)=0$ for $x\in[\mf u,\mf u]$: one can write $x$ into the eigenspaces of roots $e_\alpha$, but this decomposition can feature no simple roots, so $(f,x)=0$ follows. Accordingly, the functional $\psi\coloneqq(f,-)$ is a Lie algebra homomorphism $\mf u\to\CC$, and the pre-image of $\psi$ along the projection $\mf g\onto\mf u^*$ is $f+\mf b$. As such,
	\[\CC[f+\mf b]=\CC[\mf g^*]\otimes_{\CC[\mf u^*]}\CC_\psi.\]
	One can think of this latter algebra is a degeneration of $U\mf g\otimes_{U\mf u}\otimes\CC_\psi$, whose $U$-invariants are given by $\op{End}_{\mf g}(U\mf g\otimes_{U\mf u}\CC_\psi)$.
\end{remark}

\end{document}