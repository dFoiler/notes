% !TEX root = ../notes.tex

\documentclass[../notes.tex]{subfiles}

\begin{document}

\section{October 23}
We continue.

\subsection{The Kostant Slice}
Our story continues with the Kostant slice.
\begin{definition}[principal triple]
	Fix a semisimple Lie algebra $\mf g$ with Cartan subalgebra $\mf h\subseteq\mf g$. Then we define the $\mf{sl}(2)$-triple $(e,f,h)$ so that $e\coloneqq\sum_ie_i$ and $h\coloneqq2\rho^\lor$. This is the \textit{principal $\mf{sl}(2)$} triple.
\end{definition}
\begin{example}
	If $\mf g=\mf{sl}(n)$, then $e$ is the nilpotent matrix supported only just above the diagonal, where it has $1$. In other words, $e$ is a single Jordan block with eigenvalue $0$. Note that the centralizer $\mf g_e$ consists of powers of $e$, which can be checked directly.
\end{example}
\begin{definition}[Kostant slice]
	Fix a semisimple Lie algebra $\mf g$ with Cartan subalgebra $\mf h\subseteq\mf g$, and let $(e,f,h)$ be the principal $\mf{sl}(2)$-triple. Then we define the \textit{Kostant slice} to be $S\coloneqq f+\mf g_e$.
\end{definition}
\begin{theorem} \label{thm:kostant-slice}
	Fix a semisimple Lie algebra $\mf g$ with Cartan subalgebra $\mf h\subseteq\mf g$, and let $(e,f,h)$ be the principal $\mf{sl}(2)$-triple with Kostant slice $S$.
	\begin{listalph}
		\item There is an isomorphism $S\to\mf t/W$ given by restriction, meaning that the map $\CC[\mf g]^G\to\CC[S]$ is an isomorphism.
		\item There is an isomorphism $S\to(f+\mf b)/U$ given by inclusion, and the action map $(U\times S)\to(f+\mf b)$ induces an isomorphism.
	\end{listalph}
\end{theorem}
\begin{remark}
	This is related to Whittaker models.
\end{remark}
\begin{remark}
	The Borel $\mf b\subseteq\mf g$ is unique once we require that it contains $e$. (Such a Borel certainly exists by using the simple roots in $e$.) This $\mf b$ turns out to be the sum of eigenspaces where $h$ acts with nonnegative eigenvalues; this subspace actually only depends on $e$ because it is the subspace on which $e^m$ acts by zero, where $m$ is chosen to give us half the total dimension.
\end{remark}
\begin{example}
	If $\mf g=\mf{sl}(n)$, then $S$ consists of $f$ plus the upper-triangular matrices which are constant on the diagonals. One can use this description to prove (a) by hand.
\end{example}
\begin{lemma} \label{lem:get-kostant-slice}
	Fix free homogeneous generators $P_1,\ldots,P_r$ of $\CC[\mf g]^G$, and consider the maps $dP_i\colon\mf g\to\mf g$ given by the differential. Then $dP_i(x)\in\mf g_x$ for all $x$. If $x$ is regular, then these elements form a basis.
\end{lemma}
\begin{proof}
	Here, $dP_i$ is technically a map $\mf g\to\mf g^*$. The fact that $dP_i(x)\in\mf g_x$ follows because the polynomial $P_i$ is $G$-invariant. For example, $P_i(\exp(x)y)=P_i(y)$ for all $y$, so we conclude that $dP_i(x)$ commutes with $x$.\todo{}

	Let $\mf g^{\mathrm{reg}}$ consist of the regular elements, which we know to be Zariski open. For fixed regular $x$, we note that $\mf g$ has a natural diagonal action on the trivial bundle $\underline{\mf g}$ on $\mf g$ (thought of as $\mf g\times\mf g$), so we may define a subbundle $\underline{\mf g_x}\subseteq\underline{\mf g}$ as the centralizer of $x$. We are given $r$ sections $\{dP_i\}$ of this vector bundle, and we would like to show that they trivialize the bundle. For rank reasons, it is enough to show that they span.
	
	Observe that, if not, then the points in $\mf g$ where the $dP_i$ fail to trivialize the bundle is given by where the determinant vanishes, which is cut out by a single equation, so this bad locus would have codimension at most $1$. Thus, it is enough to check the trivialization holds outside a subset of codimension $2$. We will check it for regular semisimple elements (which is where the discriminant does not vanish) and then for a Zariski open subset of regular elements even where the discriminant does vanish.
	\begin{itemize}
		\item We work with regular semisimple elements $x$. Set $\mf t\coloneqq\mf g_x$ to be a maximal torus. Consider the projection $\pi\colon\mf g\to\mf g/G$, which restricts to $\mf t\to\mf t/W$ (where the quotients mean invariants on the polynomial algebras). Note that $W$ acts freely on the orbit of $x$ by the regularity, so regularity provides us with an isomorphism $d\pi\colon T^*_x(\mf t/W)\to T^*_x(\mf t)$, which means that any $\xi\in\mf t^*$ can be lifted to an invariant polynomial in $\CC[\mf t]^W$. (Indeed, choose $P_0$ with $dP_0|_x=\xi$ and vanishing at the rest of the orbit of $Wx$, and then $P=\left|W\right|^{-1}\sum_wwP_0$ will work.) This polynomial is then lifted to $\CC[\mf g]^G$, completing the proof.

		\item Fix a Cartan subalgebra $\mf t$, and consider elements of the form $x=s+e$, where $s\in\mf t$ has $\langle s,\alpha\rangle=0$ for some simple root $\alpha$ but nonzero elsewhere, and $e\coloneqq e_\alpha$. One can check that $x$ is regular by computing its centralizer, which is most of $\mf t$ but has some issue at $e$. For example, there is a natural projection $\mf b\to\mf t$ which induces a map $\mf z(x)\to\mf t$ because $\mf z(x)$ lives in $\mf b$.

		By a similar argument as in the previous point, we find that $\mf t/\langle1,s_\alpha\rangle\to\mf t/W$ has non-degenerate differential at the image of $x$, but the map $\mf t\to\mf t/\langle1,s\rangle$ has a simple root at $x$. On the other hand, $\mf z(x)\to\mf t$ has a simple root at the image of $x$, so we conclude that $\mf z(x)\to\mf t/W$ should have no zero.
		\qedhere
	\end{itemize}
\end{proof}
\begin{example}
	With $\mf g=\mf{sl}(n)$, then $P_i(X)=\tr X^{i+1}$ (for $i\in\{1,2,\ldots,n-1\}$). We claim that $dP_i(X)=(i+1)X^i$. Because diagonalizable matrices are Zariski open, we may restrict our attention to diagonal matrices, where
	\[P_i(\op{diag}(a_1,\ldots,a_n))=\sum_ja_j^{i+1}.\]
	Thus, $dP_i=\sum_j(i+1)a_j^i\,da_j$, which under the identification $\mf g\to\mf g^*$ goes to the claimed matrix.
\end{example}
\begin{remark}
	It should be the case that $dP_i(x)\in\mf z(\mf g_x)$ as well.
\end{remark}
\begin{proof}[Proof of \Cref{thm:kostant-slice}(a)]
	We show the parts separately. For the injectivity, it is enough to check that $G(S)$ contains a dense open subset in $\mf g$, for which it is enough to check that $T(G\times S)\to T\mf g=\mf g$ is surjective at any given point. For example, at $(1,f)\mapsto f$, this map is simply $\mf g\oplus\mf g_e\to T_f\mf g=\mf g$ given by $(x,s)\mapsto([f,x]+s)$. The surjectivity now follows from the representation theory of $\mf{sl}(2)$: every vector in a representation is either in the image of $\op{ad}_f$ or in the kernel of $\op{ad}_e$ (which is $\mf g_e$).
		
	We now turn to surjectivity, for which we use \Cref{lem:get-kostant-slice}. We may as well consider the composite $\CC[\mf t]^W\cong\CC[\mf g]^G\to\CC[S]$, and we will use graded Nakayama. Note that a grading is equivalent data to an action by $\mathbb G_{m,\CC}$, meaning that the $n$th graded piece has an action by $(-)^n$.

	As such, we would like to give $\CC[S]$ an action by $\mathbb G_m$ as well. The obvious scaling won't work because it moves $f$. TO fix this, lift $\mf{sl}(2)\to\mf g$ to some representation $\op{SL}(2)\to G$, and we will have $t$ act by the image of $\op{Ad}(\op{diag}(\sqrt t,1/\sqrt t))$. Explicitly,
	\[t\cdot s\coloneqq\op{Ad}_{\op{diag}(\sqrt t,1/\sqrt t)}(ts).\]
	Technically, this formula does not look algebraic because of the square roots, but it is: the principal representation $\mf g$ of $\mf{sl}(2)$ has only even weights, so the action of $\op{SL}(2)$ will factor through $\op{SL}(2)/\{\pm1\}=\op{PGL}(2)$. As such, the action on $\CC[S]$ is given by $t$ acting by $\op{Ad}(\op{diag}(t,1))(ts)$.
	
	Now, $\CC[S]$ is a polynomial algebra with generators in degrees $m_i+1$, where $\{2m_1,\ldots,2m_r\}$ are the highest weights of $\mf g$ as the principal representation. Indeed, just $m_i$ would come from acting on $S$ by $ts$, but the extra adjoint shifts by $1$.

	We now are equipped with an inclusion $\CC[\mf t]^W\into\CC[S]$ of positively graded polynomial algebras, and \Cref{lem:get-kostant-slice} tells us that the differential map $S\to\mf t/W|_f$ is an isomorphism. Unwinding the definition of the tangent space, we have an isomorphism
	\[\frac{\CC[\mf t]^W_+}{\left(\CC[\mf t]^W\right)_+^2}\onto\frac{\CC[S]_+}{\left(\CC[S]\right)_+^2},\]
	which now upgrades to an isomorphism of the algebras by graded Nakayama (i.e., doing an induction on the degree).
\end{proof}
\begin{remark}
	The proof of (a) shows that $\deg P_i=m_i+1$. Indeed, we know that the generators of each of the polynomial algebras $\CC[S]$ and $\CC[\mf t]^W$ must have the same degree!
\end{remark}
\begin{remark}
	Let's contextualize part (b). Under the identification $\mf g\cong\mf g^*$, the space $\mf b$ goes to $\mf u^\perp$ because $\mf b=\mf t\oplus\mf u$. As such, $f+\mf b$ is identified with
	\[\{\xi:(\xi,x)=(f,x)\text{ for }x\in\mf u\}.\]
	Now, note that $(f,x)=0$ for $x\in[\mf u,\mf u]$: one can write $x$ into the eigenspaces of roots $e_\alpha$, but this decomposition can feature no simple roots, so $(f,x)=0$ follows. Accordingly, the functional $\psi\coloneqq(f,-)$ is a Lie algebra homomorphism $\mf u\to\CC$, and the pre-image of $\psi$ along the projection $\mf g\onto\mf u^*$ is $f+\mf b$. As such,
	\[\CC[f+\mf b]=\CC[\mf g^*]\otimes_{\CC[\mf u^*]}\CC_\psi.\]
	One can think of this latter algebra is a degeneration of $U\mf g\otimes_{U\mf u}\otimes\CC_\psi$, whose $U$-invariants are given by $\op{End}_{\mf g}(U\mf g\otimes_{U\mf u}\CC_\psi)$.
\end{remark}

\end{document}