% !TEX root = ../notes.tex

\documentclass[../notes.tex]{subfiles}

\begin{document}

\section{October 7}
We began with an alternative proof of \Cref{thm:cartan-exist}, which has been added to the last notes.

\subsection{Another Proof of Real Cartan Involutions}
Let's give an alternative proof of \Cref{thm:cartan-exist}. We will once again assume that they exist in the complex case.
\begin{lemma} \label{lem:make-involutions-commute}
	Let $G$ be a semisimple real Lie group with a Cartan involution $\theta$. For any other involution $\sigma$, there is a conjugate involution $\sigma'=g\sigma g^{-1}$ commuting with $\theta$.
\end{lemma}
\begin{proof}
	We do some clever linear algebra. As usual, let $(-,-)$ be the Killing form, and let $B_\theta$ be the bilinear form $B_\theta(X,Y)\coloneqq(X,\theta Y)$ to be negative-definite (by definition of the Cartan involution). Now, set $s\coloneqq\theta\sigma$. We note the following properties of $s$.
	\begin{itemize}
		\item Note $s$ is self-adjoint for $B_\theta$ by a direct calculation:
		\[B_\theta(sX,Y)=(\theta\sigma X,\theta Y)\stackrel*=(\sigma X,Y)\stackrel*=(X,\sigma Y)=B_\theta(X,sY).\]
		Here, the key equalities $\stackrel*=$ follow because $\theta$ and $\sigma$ are involutions preserving the group structure, so they must preserve the Killing form.
		\item Note $\theta s\theta^{-1}=s^{-1}=\sigma s\sigma^{-1}$ by a direct calculation.
	\end{itemize}
	It follows that the action of $s$ may be diagonalized with real eigenvalues, say as
	\[\mf g=\bigoplus_i\mf g_{r_i},\]
	where the eigenvalues of $s$ are $\{r_i\}_i$. Notably, $\left|s\right|^t$ (acting on $\mf g_{r_i}$ by $\left|r_i\right|^t$) can be thought of as the product of various $\exp(\left|t\delta\right|)$s where $\delta$ is the log of $r_i$, so the linear operator $\left|s\right|^t$ even belongs to $G$. We thus achieve $\theta\left|s\right|^t\theta^{-1}=\left|s\right|^{-t}$ and similar for $\sigma$.

	Now, define $\sigma'$ by $s\coloneqq\left|s\right|\sigma'$ where $\sigma'$ acts by sign of $r_i$ on the various $\mf g_{r_i}$s. One can calculate that $\sigma'$ commutes with both $\theta$ and $\sigma$, so the involution
	\[\left|s\right|^{1/2}\sigma\left|s\right|^{-1/2}=\left|s\right|\sigma=\sigma's\sigma=\sigma'\theta\sigma\sigma=\sigma'\theta\]
	is conjugate to $\sigma$ and will commute with $\theta$.
\end{proof}
\begin{proof}[Proof of \Cref{thm:cartan-exist}]
	We already know that $G_\CC$ admits a Cartan involution $\sigma$ with $G=G_\CC^\sigma$. We also know that $G_\CC$ admits a compact form $G_\CC^{\theta_c}=K_c$, where $\theta_c$ is some other compact form. \Cref{lem:make-involutions-commute} allows us to assume that $\sigma$ and $\theta_c$ commute (by merely conjugating $\sigma$), so $\theta_c$ descends to an action on $G$. Now, the Killing form of $G$ is half the killing form of $G_\CC$, so $\theta_c|_G$ is easily checked to be the desired Cartan involution.

	Next, let's show the uniqueness: if we have two Cartan involutions $\theta$ and $\theta'$ on $G$, then by \Cref{lem:make-involutions-commute}, we are allowed to assume that $\theta\theta'=\theta'\theta$. Thus, either $\theta=\theta'$ or (by considering diagonalization) there is $X\in\mf g$ for which $\theta(X)=-\theta'(X)$. But then $B_\theta(X,X)=-B_{\theta'}(X,X)$, so both cannot be negative definite.

	Lastly, we need to show the result about twists. We have already described a map sending $G$ to its Cartan involution of $G_\CC$, unique up to conjugation. The reverse map sends the Cartan involution $\sigma$, which we may as well assume to commute with $\theta$, to the real group $G_\CC^{\sigma\theta}$. Perhaps we should check that this reverse map is well-defined. The main difficulty is that replacing $\sigma$ with a conjugate commuting with $\theta$ is not known to be unique because we had a choice of $\theta$. Well, given two Cartan involutions $\theta$ and $\theta'$ commuting with $\sigma$, up to conjugation above, we may assume that
	\[\theta|_{G^\sigma}=\theta'|_{G^\sigma}.\]
	We now claim that $\theta$ is uniquely determined by the restriction to $G^\sigma$. Well, by taking products, we may assume that $G_\CC$ is either almost simple (namely, $\mf g$ is simple) or is the sum of two factors permuted by $\sigma$. In both cases, we note that $\mf g^{\sigma=-1}$ is a simple $\mf k_\CC$-module: in the second case, we find that $\mf g^{\sigma=-1}$ is simple and so isomorphic to $\mf g^{\sigma}$ as a $\mf g^\sigma$-module, so the result follows; in the first case, any submodule $V\subsetneq\mf g^{\sigma=-1}$ makes $V\oplus\mf g^\sigma$ a Lie ideal of $\mf g$, which is a contradiction. To complete the proof of the claim, we decompose
	\[\mf g=\mf g^\sigma\oplus\mf g^{\sigma=-1}.\]
	Now, $B_\theta|_{\mf g^{\sigma=-1}}$ is a $G^\sigma$-invariant pairing (here we use the fact that they already agree on $\mf g^\sigma$), so it is unique up to scalar by Schur's lemma, and by definiteness, it is unique up to a positive scalar. It follows that
	\[\theta|_{\mf g^{\sigma=-1}}=r\theta'|_{\mf g^{\sigma=-1}}\]
	for some $r>0$, but then $r^2=1$ because we have an involution, so we are done.
\end{proof}
\begin{remark}
	These proofs can be upgraded to the reductive case.
\end{remark}

\subsection{More on Twisting}
Here are some motivating examples.
\begin{example}
	Consider $G_\CC=\op{SL}(n)$. Then we have involutions $\sigma$ given by conjugating by
	\[(\underbrace{1,\ldots,1}_p,\underbrace{-1,\ldots,-1}_q)\]
	where $p+q=n$ (and $q$ is even). One finds that $G^\sigma=\op{SL}(p)\times\op{SL}(q)$ by a direct calculation, which we note is the complexification of $\op{SU}(p)\times\op{SU}(q)$, which is a maximal compact subgroup of the real form $\op{SU}(p,q)$.
\end{example}
\begin{example}
	For $n\ge3$, there is an automorphism on $G_\CC=\op{SL}(n)$ given by $g\mapsto g^{-\intercal}$. This produces the split forms.
\end{example}
\begin{remark}
	In general, there is an exact sequence
	\[1\to G_{\mathrm{ad}}\to\op{Aut}G\to\op{Out}G\to1.\]
	Considerations around fixing a maximal torus shows that $\op{Out}G$ are given by automorphisms of the Dynkin diagram; for example, this group is only nontrivial in the simply laced case.
\end{remark}
\begin{definition}[inner twist]
	Two twists $G$ and $G'$ over $\RR$ are \textit{inner twists} if and only if the corresponding involution has trivial image in $\op{Out}G$.
\end{definition}
\begin{definition}[quasisplit]
	A form $G$ is \textit{quasisplit} if and only if the corresponding involution has trivial image in $G_{\mathrm{ad}}$.
\end{definition}
\begin{remark}
	It turns out that a form is quasisplit if and only if there is a Borel subgroup defined over $\RR$.
\end{remark}
\begin{example}
	One can see that the split form is quasisplit.
\end{example}
\begin{example}
	A complex group $G$ viewed as a real group is quasisplit but not split. Indeed, $G$ is the fixed points of $G_\CC= G\times G$ given by swapping the two factors, which is an outer automorphism.
\end{example}
\begin{remark}
	Given two (regular semisimple) elements $g,g'\in G$ conjugate in $G_\CC$ (i.e., they are stably conjugate), then we can consider the set
	\[\left\{x\in G_\CC:xgx^{-1}\right\},\]
	which is a torsor for a maximal torus $T\coloneqq Z_{G_\CC}(g)$. Note that $g$ and $g'$ are actually conjugate if and only if this torsor admits a real point. It then turns out that the $G_\CC$-conjugacy class of $g$ in $G$ becomes some subset of $\mathrm H^1(\op{Gal}(\CC/\RR);T)$. The answer turns out to be the kernel of the map
	\[\mathrm H^1(\op{Gal}(\CC/\RR);T)\to\mathrm H^1(\op{Gal}(\CC/\RR);G)\]
	The left group can be seen to consist of the fixed points of the involution $t\mapsto\overline t^{-1}$, which amounts to the connected components of $T_\CC^\sigma$, where $\sigma$ is $t\mapsto\theta_c\big(\overline t^{-1}\big)$. For example, if $G$ is compact, then $T$ is compact, and it turns out that this map is injective.
\end{remark}

\subsection{Globalization}
Next class we will prove the following.
\begin{theorem}[Harish-Chandra globalization]
	Fix a real semisimple Lie group $G$, and set $\mf g\coloneqq\op{Lie}G_\CC$ and $K$ to be a maximal compact subgroup as usual. For any irreducible unitary $(\mf g,K)$-module $M$, then there is a unitary representation $V$ of $G$ for which $M=V_{\mathrm{fin}}$ and $V=\overline M$.
\end{theorem}
\begin{remark}
	In particular, it follows that $M$ is admissible by \Cref{thm:hc-admissibility}.
\end{remark}
\begin{remark}
	One can purely algebraically prove that any irreducible $(\mf g,K)$-module is admissible. The key tool is the action of the Casimir elements (in $Z\mf g\subseteq U\mf g$) on $M$. The point is that the Schur lemma forces every element in $Z\mf g$ to act by a scalar, which produces an ``infinitesimal character'' $\chi\colon Z\mf g\to\CC$. This will be useful once we understand the structure of $Z\mf g$.
\end{remark}

\end{document}