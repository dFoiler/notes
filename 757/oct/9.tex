% !TEX root = ../notes.tex

\documentclass[../notes.tex]{subfiles}

\begin{document}

\section{October 7}
Today we discuss globalization.

\subsection{Globalization}
Next class we will prove the following.
\begin{theorem}[Harish-Chandra globalization] \label{thm:hc-global}
	Fix a real semisimple Lie group $G$, and set $\mf g\coloneqq\op{Lie}G_\CC$ and $K$ to be a maximal compact subgroup as usual, and we assume that $K$ is connected. For any irreducible unitary $(\mf g,K)$-module $M$, then there is a unitary representation $V$ of $G$ for which $M=V_{\mathrm{fin}}$ and $V=\overline M$.
\end{theorem}
\begin{remark}
	In particular, it follows that $M$ is admissible by \Cref{thm:hc-admissibility}.
\end{remark}
\begin{remark}
	One can purely algebraically prove that any irreducible $(\mf g,K)$-module is admissible. The key tool is the action of the Casimir elements (in $Z\mf g\subseteq U\mf g$) on $M$. The point is that the Schur lemma forces every element in $Z\mf g$ to act by a scalar, which produces an ``infinitesimal character'' $\chi\colon Z\mf g\to\CC$. This will be useful once we understand the structure of $Z\mf g$.
\end{remark}
\begin{proof}[Proof of \Cref{thm:hc-global}]
	Let $(-,-)$ be the invariant unitary inner product on $M$. Viewing $M$ as a representation of $K$, we may decompose
	\[M=\bigoplus_\nu M_\nu,\]
	where $\nu$ varies over the various dominant integral weights of $K$ (and hence of $K_\CC^\circ$), and $M_\nu$ is the corresponding isotypic component of the highest weight representation $V_\nu$. As such, we may write $M_\nu=V_\nu\otimes W_\nu$.

	The main claim is that there is a quadratic form $q$ on $\mf h^*$ such that
	\[\left|Xv\right|^2\stackrel?\le\left|v\right|^2\cdot q(\nu)\left|X\right|^2,\]
	for any $X\in\mf g$ and $v\in M$; here, $\left|X\right|^2$ is computed with respect to the bilinear form $-1\cdot(-,\theta-)$ of a fixed Cartan involution $\theta$. Let's show this. Because $(-,-)$ is invariant under $\mf g$, we know $(X'v',w')+(v',X'w')=0$ for any $v',w'\in M$ and $X'\in\mf g$, so we see that $\left|Xv\right|^2$ is
	\[(Xv,Xv)=-\left(X^2v,v\right),\]
	which we will do by fixing an orthonormal basis $\{X_i\}_i$ of $\mf g$ and estimating $\sum_i\left(X_i^2v,v\right)$ instead. Here, the orthonormal basis of $\mf g$ is taken with respect to the bilinear form $-1\cdot(-,\theta-)$. In fact, decomposing $\mf g=\mf k\oplus\mf p$, where $\theta$ acts by $+1$ on $\mf k$ and acts by $-1$ on $\mf p$, we see that
	\[\sum_iX_i^2=C_\mf g-2C_\mf k,\]
	where $C_\mf g$ is the Casimir element of $\mf g$, and $C_\mf k$ is the Casimir element of $\mf k$. (Here, $C_\mf k$ means the Casimir element of the adjoint action of $\mf k$ on $\mf g$. This equality follows because the Killing form on $\mf k$ is negative-definite by definition of the Cartan!) Now, $C_\mf g$ commutes with the $\mf g$-action on $M$, so by Schur's lemma, we see that $C_\mf g$ must act by a scalar.
	
	Let's discuss the Casimir action. Continuing, one can check that a Casimir element $C_\mf k$ acts on a highest weight representation $V_\nu$ by the scalar $(\nu+\rho)^2-(\rho)^2$, where $(-,-)$ is some choice of invariant bilinear form. Indeed, write $C_\mf k$ as, up to some constant,
	\[\sum_\alpha e_\alpha f_\alpha+\sum_\alpha f_\alpha e_\alpha+\sum_ih_i^2,\]
	but this can be rearranged to $2\sum_\alpha f_\alpha e_\alpha+\sum_\alpha h_\alpha^2+\sum_ih_i^2$, and then it is a matter of calculating the action on a singular vector $v$ of $V_\nu$, which is now easier because $e_\alpha v=0$. Summing completes the calculation. The main claim now follows after an expansion of this action.

	We now complete the argument. Set $V$ to be the completion of $M$ for the norm topology given by $(-,-)$. To upgrade the $\mf g$-action to a $G$-action, it basically remains to check that
	\[\exp(X)v=\sum_{n\ge0}\frac{X^nv}{n!}\]
	converges to $V$ for any $v\in V$. By continuity, it is enough to check this for $v\in M$, and by decomposing into locally finite pieces, it is enough to check this for $v\in M_\nu$, which roughly speaking follows from the lemma as well as the fact that the $X^nv$ is supported in weights of the form $\nu+\alpha$ where $\alpha$ is a sum of the weights of $K$ acting on $\mf g$. (Indeed, this latter statement follows by decomposing $\mf g\otimes M_\nu$ into weight spaces of $K$.) In particular, the inequality
	\[\left|Xv\right|\le\left|X\right|\cdot\left|v\right|\cdot\sqrt{q(\nu)},\]
	so we find that
	\[\left|X^nv\right|\le\left|X\right|^n\left|v\right|\cdot c(c+R)(c+2R)\cdots(c+(n-1)R),\]
	where $c=\sqrt{q(\nu)}$ and $R$ is the maximum among the $\sqrt{q(\alpha)}$ where $\alpha$ is one of weights appearing in the action of $K$ on $\mf g$. It follows that $\exp(X)v$ converges for $\left|X\right|<R^{-1}$.
	% One can show this with the Borel--Weil theorem; the argument is slightly involved: write $V_\nu$ as $\Gamma(H/B,L_\nu)$ for some line bundle $\mc L$. Then $\mf g\otimes V_\nu=\Gamma(H/B,\mf g\otimes L_\nu)$. Then one can calculate with the associated graded rings.

	Once we know that $\exp(X)v$ converges for small $X$, the Campbell--Hausdorff formula extends this to an action of the universal cover $\widetilde G$. To see that our action actually factors through $G$, we let $\widetilde K\subseteq\widetilde G$ be the preimage of $K$. So because the action of $\widetilde K$ factors through $K$, it follows that the action of $\widetilde G$ factors through $G$ because $\pi_1(G)=\pi_1(K^\circ)$.\footnote{In fact, one can show that $G/K^\circ$ is contractible, which follows from the polar decomposition. Indeed, writing $\mf g=\mf k\oplus\mf p$ as above, then one can show that $\exp(\mf p)$ is a Euclidean space. Thus, one finds that $G/K^\circ$ is some Euclidean space.} (Indeed, $\pi_1(G)$ is the kernel of the map $\widetilde G\onto G$ and similarly for $K$.)
\end{proof}

\subsection{Verma Modules}
We should say something about Verma modules.
\begin{definition}
	Fix a complex semisimple Lie algebra $\mf g$, and fix a Cartan subalgebra $\mf h$ and decomposition $\mf g=\mf n_+\oplus\mf h\oplus\mf n_-$, and we set $\mf b\coloneqq\mf n_+\oplus\mf h$. For a weight $\lambda\in\mf h^*$, we define the \textit{Verma module}
	\[M_\lambda\coloneqq\op{ind}_{\mf b}^{\mf g}\CC_\lambda.\]
\end{definition}
\begin{remark}
	Equivalently, one can think of $M_\lambda$ as $U\mf g\otimes_{U\mf b}\CC_\lambda$. As such, $M_\lambda$ is generated by the singular vector $v_\lambda$, which lives in weight $\lambda$.
\end{remark}
\begin{remark}
	It follows from the PBW theorem that
	\[U\mf n_-\cong M_\lambda,\]
	where the isomorphism is given by $X\mapsto Xv_\lambda$.
\end{remark}
\begin{remark}
	It follows that $\op{Hom}_{\mf g}(M_\lambda,M)$ is given by the collection of singular vectors $v\in M$ of weight $\lambda$. For example, if $\alpha^\lor$ is a simple coroot with $\langle\lambda^\lor,\lambda\rangle$ equal to a nonnegative integer $m$, then $f_\alpha^{m+1}v_\lambda$ is a vector in $M_\lambda$ of weight $\lambda-(m+1)\alpha$. Furthermore, one can check that $f_\alpha^{m+1}v_\lambda$ is singular!
\end{remark}

\end{document}