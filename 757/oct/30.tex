% !TEX root = ../notes.tex

\documentclass[../notes.tex]{subfiles}

\begin{document}

\section{October 30}
I have regained access to my computer. Thank you, Hui!

\subsection{Extensions of the Kostant Slice}
Let $e$ be any nilpotent. By the Jacobson--Morosov theorem, we can fit $e$ into an $\mathfrak{sl}(2)$-triple in $\mf g$. Then
\[S_f\coloneqq f+\mf g_e\]
is another slice to the orbit of $Gf$, meaning that $T_f\mf g=T_fGf\oplus T_fS_f$. Of course, one can compute $T_fGf=[f,\mf g]$ and $T_fS_f=\mf g_e$.

It is interesting to consider the kernel $S_f^\circ$ of the composite $S_f\subseteq\mf g\to\mf h/W$. The kernel of the map $\mf g\to\mf h/W$ is just $\mc N$, so we see $S_f^\circ=S_f\cap\mc N$. If $\mf g$ is simple, then $G$ acts on $\mc N$ with finitely many orbits. For symplectic reasons, all orbits are even-dimensional. It turns out that there is a unique open orbit, consisting of the regular nilpotents; and then it turns out that there is a unique orbit of codimension $2$, called the ``subregular'' nilpotents.
\begin{example}
	Let $J_m$ be the $m\times m$ nilpotent Jordan block of size $m$. With $\mf g=\mf{sl}(n)$, then the regular orbit is the orbit of $J_n$, and the subregular orbit is the orbit of $J_{n-1}\oplus J_1$.
\end{example}
Now, in the subregular case, it turns out that $\dim S_f^\circ=2$, and it can be seen to be a rational surface singularity of the type $\AA^2/\Gamma=\Spec\CC[x,y]^\Gamma$, where $\Gamma\subseteq\op{SL}(2)$ is a finite subgroup.
\begin{theorem}
	There is a bijection between all the following items.
	\begin{listroman}
		\item Finite subgroups of $\op{SL}(2)$.
		\item Rational surface singularities.
		\item Simply laced Dynkin graphs.
		\item Simply laced simple Lie algebras.
	\end{listroman}
\end{theorem}
\begin{proof}[Sketch]
	We have already explained how to go from (iv) to (i); it turns out that adding in non-simply-laced simple Lie algebras does not do much.
	\begin{itemize}
		\item Starting with a finite subgroup $\Gamma\subseteq\op{SL}(2)$, we build a graph whose vertices are the nontrivial irreducible representations of $\Gamma$, and we draw an edge between vertices $\rho$ and $\rho'$ if and only if $\op{Hom}(\rho\otimes\mathrm{std},\rho')\ne0$. (Note that this condition is symmetric because $\mathrm{std}$ is self-dual, and $\dim\op{Hom}(-,-)$ is symmetric.) It turns out that this is a simply laced Dynkin graph, and it provides a bijection between (i) and (iii).
		
		For example, with the subgroup
		\[\Gamma=\left\{\begin{bmatrix}
			\zeta^a \\ & \zeta^{-a}
		\end{bmatrix}:a\in\ZZ\right\},\]
		where $\zeta$ is some root of unity, we recover type $A$ graphs. Dihedral groups recover type $D$ graphs. It turns out that $A_5\subseteq\op{SL}(2)$ recovers $E_8$; other platonic solids yield $E_6$ and $E_7$.

		\item A rational surface singularity $S$ (with singularity at $p$) admits a minimal resolution $\pi\colon\widetilde S\onto S$. It turns out that $\pi^{-1}(\{p\})$ is some union of projective lines. One builds a graph by declaring the lines to be vertices, and two vertices are connected if and only if the lines intersect.
		\qedhere
	\end{itemize}
\end{proof}
\begin{remark}
	One can recover the resolution $\widetilde S$ from the $S\coloneqq S_f^\circ\cap\mc N$ arising from a simply laced Lie algebra $\mf g$ by resolving $\widetilde N$ as $T^*(G/B)$. (This is the ``Springer resolution.'') Here, $T_x(G/B)$ can be thought of as $\mf g/\mf b_x$, so $T_x^*(G/B)$ can be identified with $\mf b_x^\perp$, which is simply the unipotent radical $\mf u_x$.
\end{remark}
\begin{remark}
	It turns out that the composite
	\[T^*(G/B)\to\mc N\subseteq\mf g^*\]
	is the moment map. It is interesting to compute the pre-image of $J_1\oplus J_{n-1}$ in the case $\mf{sl}(n)$ where we expect some $\PP^1$s intersected as described by the type-$A$ graph.
\end{remark}
\begin{remark}
	A non-simply laced diagram $D$ can be ``unfolded'' (for example, $G_2$ is simply $D_4$ modulo an $S_3$-action), which allows us to extend some of the results to include non-simply-laced cases.
\end{remark}
\begin{remark}
	One can generalize the isomorphism $S\to(f+\mf b)/U=\mf g^*/(U,\psi)$. In particular, if $e$ is even (i.e., $h$ acts only by even eigenvalues in $\mf g$), then the linear functional $\psi_f\colon x\mapsto(f,x)$ as before has weight $2$ and thus vanishes on $[\mf u,\mf u]$ for weight reasons. Thus, there continues to be an isomorphism $S\to\mf g^*/(U,\psi_f)$. For example, for $\mf{sl}(n)$, the condition on $e$ is equivalent to all blocks having sizes of the same parity.
\end{remark}
\begin{remark}
	One can generalize the isomorphism $S\to(f+\mf b)/U=\mf g^*/(U,\psi)$ without hypotheses on $e$. The idea is that
	\[\omega(x,y)\coloneqq (f,[x,y])\]
	turns out to be non-degenerate (and alternating) on $\mf g_1$. Then we can choose a Lagrangian $\mc L\subseteq\mf g_1$ and define $\mf u\coloneqq\mc L\oplus\mf g_{\ge2}$. We still have $\psi_f|_{[\mf u,\mf u]}=0$, and the story goes through. (As a sanity check, one can verify that $\dim\mf g-\dim S=2\dim\mf u$.) % Indeed, $\dim\mf g-\dim S$ is the dimension of $\dim Gf$
\end{remark}
\begin{remark}[Kostant--Rallis slice]
	Suppose $\mf g$ admits an involution $\theta$, so we can decompose $\mf g$ into $\mf k\oplus\mf p$ as the $(+1)$- and $(-1)$-eigenspaces for $\theta$. For a generic nilpotent $e$ in $\mf p$, one can extend it to an $\mathfrak{sl}(2)$-triple with $f\in\mf p$ and $h\in\mf k$. It turns out that there is a bijection
	\[f+(\mf g_e\cap\mf p)\to\mf p/K.\]
	The left-hand side $S_f\coloneqq f+(\mf g_e\cap\mf p$ is known as the Kostant--Rallis slice. It turns out that $\mf p/K$ is further isomorphic (as in Chevalley restriction) to $\mf a/W^\theta$, where $\mf a\subseteq\mf p$ is a maximal abelian subalgebra, and $W^\theta$ is the subgroup of the Weyl group preserving $\mf a$.
\end{remark}

\subsection{Representations of \texorpdfstring{$\op{SL}(2,\mathbb R)$}{SL(2,R)}}
Set $G\coloneqq\op{SL}(2,\RR)$ and $K\coloneqq\op{SO}(2,\RR)$. Then we may let $\mf g\coloneqq\mf{sl}(2,\CC)$ with $K_\CC=\op{SO}(2,\CC)\cong\CC^\times$. We are interested in classifying irreducible $(\mf g,K)$-modules.

Well, let $V$ be some irreducible $(\mf g,K)$-module. It is convenient to give $\mf g$ a basis with $h=i\begin{bsmallmatrix}
	0 & 1 \\ -1 & 0
\end{bsmallmatrix}$; this generates $\mf k_\CC$, and it is multiplied by $i$ to make it conjugate to the usual value of $h$. For completeness, we note
\[e=\frac12\begin{bmatrix}
	1 & i \\
	i & -1
\end{bmatrix}\qquad\text{and}\qquad f=\frac12\begin{bmatrix}
	1 & -i \\ 1 & -1
\end{bmatrix}\]
as well. Notably, complex conjugation acts by $(e,f,h)\mapsto(-h,f,e)$, as one would expect from the compact involution. Now, the $\mf k$-action on $V$ must integrate to $K$, which means that it admits a grading $V=\bigoplus_{n\in\ZZ}V_n$ so that $t\in K$ acts by $t^n$, which then means that $h$ acts with eigenvalue $n\in\ZZ$.

Now, by standard $\mf{sl}(2)$-theory, we see that $e\colon V\to V$ acts by weight $2$ and $f\colon V\to V$ acts by weight $-2$, so if $V$ is irreducible, then it must either have all even or all odd weights. To gain further information, we recall the Casimir element $C\coloneqq ef+fe+\frac12 h^2$ must act by a scalar $c\in\CC$ on $V$. Thus, by the PBW theorem, one can conclude that $\dim V_n\le1$ for each $n$: any element of $U\mf g$ of weight $0$ can be written as a polynomial in $\CC[h,C]$, so
\[(U\mf g\cdot x_n)_n=(U\mf g)_0\cdot x_n=\CC x_n\]
for any given $x_n\in V_n$, so $V_n=\CC x_n$ follows by irreducibility.

Additionally, we note that if $V_n=0$ for any $n$, we conclude that $V=V_{<n}\oplus V_{>n}$ because both spaces are $\mf{sl}(2)$-invariant, so we must have $V=V_{<n}$ or $V=V_{>n}$. (In fact, the same claim holds whenever $e$ or $f$ have a kernel.) It follows that
\[\{n\in\ZZ:V_n\ne0\}\]
is either a segment (i.e., evens or odds in $[a,b]$) or a ray (i.e., even or odds in $[a,\infty)$ or $(-\infty,a]$) or everything (i.e., even or odds in $\ZZ$). The case of a segment corresponds to the finite-dimensional representations of $\mf{sl}(2,\RR)$, so we allow ourselves to ignore them. The remaining possibilities can all be realized. Let's start with the rays.
\begin{itemize}
	\item The ray $(-\infty,m]$ is realized by a Verma module of highest weight $m$ with Borel given by $\op{span}\{h,e\}$. To be irreducible, one needs $m\le0$ because $m\ge0$ admit finite-dimensional quotients. Then one can check by hand that these are actually irreducible because any vector generates: any vector can be pushed along $e$ until it is supported in $V_m$, and then it can be pushed back along $f$ to generate the rest of the representation.
	\item The ray $[m,\infty)$ is realized by a Verma module of highest weight $m$ with Borel given by $\op{span}\{h,f\}$. To be irreducible, one needs $m\ge0$ because $m\le0$ admit finite-dimensional representations. Once again, one checks by hand that these representations are actually irreducible.
\end{itemize}
These are called the discrete series representations, and they are labeled by their highest weight as $L_m$.

Lastly, we consider the cases $2\ZZ$ and $1+2\ZZ$. Here, we may choose $v_n\in V_n$ for which $fv_n=v_{n-2}$ for all $n$, from which we can use the Casimir element to compute
\[ev_n=efv_{n+2}=\left(c-\frac{(n+1)^2}4\right)v_{n+2}.\]
Thus, one sees that this provides an irreducible module if and only if $c\ne\frac14(n+1)^2$ for all $n$. For convenience, we can write $c=s^2/4$, and we now have the formula
\[efv_{n+2}=\frac{(s-n-1)(s+n+1)}4v_n\]
where now we are required to have $s$ not among the weights of $V$. One rescale the basis to $w_n\coloneqq c_nv_n$ so that the formulae look like
\[\begin{cases}
	ew_n=\frac12(s-1-n)w_{n+2}, \\
	fw_n=\frac12(s+1+n)w_{n-2}.
\end{cases}\]
We call this module $P_s^+$ in the even-weight case and $P_s^-$ in the odd-weight case. In total, we have collected the following result.
\begin{theorem} \label{thm:g-k-for-sl2}
	The irreducible $(\mf g,K)$-modules for $(\mf g,K)=(\mf{sl}(2,\CC),\op{SO}(2,\RR))$ are as follows.
	\begin{itemize}
		\item Finite-dimensional modules $V_m$.
		\item Discrete series: Verma modules $L_m$ for $m\ne0$.
		\item Principal series: $P_s^+$ (for $s\notin1+2\ZZ$) and $P_s^-$ (for $s\notin2\ZZ$).
	\end{itemize}
	There are isomorphisms $P^{\pm}s=P^{\pm}_{-s}$ and no other isomorphisms.
\end{theorem}

\end{document}