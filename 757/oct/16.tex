% !TEX root = ../notes.tex

\documentclass[../notes.tex]{subfiles}

\begin{document}

\section{October 16}
Today we say more about the structure theory.

\subsection{Admissibility of Irreducible \texorpdfstring{$(\mf g,K)$}{ (g, K)}-Modules}
Here is our main result.
\begin{proposition} \label{prop:generic-split-element}
	Fix a semisimple complex connected Lie group $G$, and let $\theta\colon G\to G$ be an involution. Set $K\coloneqq G^\theta$, which we let have Lie algebra $\mf k=\mf g^\theta$, and then we decompose $\mf g=\mf k\oplus\mf p$, where $\mf p\coloneqq\mf k^\perp$.
	\begin{listalph}
		\item For generic $x\in\mf p$, we may arrange for $x$ to be semisimple and for the dimension of $\mf g_x$ (which is the centralizer of $x$) is minimal. Then $\mf a\coloneqq\mf g_x\cap\mf p$ is abelian and semisimple.
		\item For generic $x\in\mf p$, the orbit $K(\mf a)$ contains a dense Zariski open subset in $\mf p$.
	\end{listalph}
\end{proposition}
\begin{example}
	If $G$ is a real group, then $G_\CC\cong G\times G$ has an automorphism $\theta$ given by switching the two factors. Then $K=G$, so $\mf k=\mf g$, and $\mf p=\mf g$ as well.
\end{example}
\begin{proof}
	If $\theta$ is the identity, then $\mf p=0$, and there is nothing to do. Thus, we may assume $\mf p\ne0$, and then we go ahead and choose $x\in\mf p$ generic. For example, $x\ne0$, so there is a unique Jordan decomposition $x=x_s+x_u$, thereby implying $x_s,x_u\in\mf p$. For example, if $\mf p$ has no semisimple elements, then we are required to have $\mf p$ consist solely of nilpotent elements, but this is false because the Killing form vanishes on nilpotent elements while the Killing form is non-degenerate on $\mf p$.

	We thus see that we can choose generic $x\in\mf p$ to be semisimple with $\mf h\coloneqq\mf g_x$ of minimal dimension. Certainly $\mf h$ is reductive, so we can write $\mf h=\mf h'\oplus\mf h_0$ where $\mf h'$ is semisimple and $\mf h_0$ and is abelian. Decomposing $\mf h$ into $\theta$-eigenspaces, we claim that $\theta$ is trivial on $\mf h'$. This implies (a) because it shows that $\mf g_x\cap\mf p=\mf h_0$ is abelian.
	
	It remains to show the claim. Otherwise, we can find nonzero semisimple $y\in(\mf h')^-$ (as before), meaning $y\in(\mf h'\cap\mf p)$. But then $x$ and $y$ are commuting semisimple elements, so a generic (semisimple!) linear combination $x-ty\in\mf p$ has centralizer $\mf g_x\cap\mf g_y$ (simply choose $t$ so that the eigenvalues of $x$ and $ty$ do not overlap). But then $\mf g_{x+ty}$ is strictly smaller than $\mf h$, contradicting the minimality.

	This follows by a tangent space calculation. It is enough to show that the action map $\rho\colon K\times\mf a\to\mf p$ is surjective on the differentials at some point (because this provides us with an open neighborhood). For this, we compute the differential at $(1,x)\in K\times\mf a$. This is some map $d\rho\colon\mf k\oplus\mf a\to\mf p$, and it sends $(y,b)$ to $[y,x]+b$ (because conjugation action goes to the Lie bracket). To show that this map surjects onto $\mf p$, we note that the adjoint action of $x$ on $\mf g$ has an eigenspace decomposition
	\[\mf g=\mf g_x\oplus\op{ad}_x\mf g\]
	because $\mf g_x$ is the eigenspace of $\op{ad}_x$ with eigenvalue $0$. Intersecting such a decomposition with $\mf p$, we see that $\op{ad}_x$ swaps $\mf k$ and $\mf p$ (because $x\in\mf p$), so we get
	\[\mf p=\mf a\oplus\op{ad}_x\mf k,\]
	so the surjectivity follows.
\end{proof}
\begin{remark}
	Here, we can take ``generic'' to mean ``Zariski open.'' In fact, the argument shows that generic elements of $\mf p$ are semisimple.
\end{remark}
\begin{corollary}
	Fix a semisimple real connected group $G$.
	\begin{listalph}
		\item For any irreducible representation $V_\nu$ of $K$, the space $\op{Hom}_K(V_\nu,\CC[\mf p])$ is finitely generated over $\CC[\mf g]^G$.
		\item Any irreducible $(\mf g,K)$-module is admissible.
	\end{listalph}
\end{corollary}
\begin{proof}
	Let's start with (a). It is worthwhile to explain how $\CC[\mf g]^G$ acts on $\op{Hom}_K(V_\nu,\CC[\mf p])$. Well, a function $\CC[\mf g]^G$ can be restricted to $\mf p$, and it remains $K$-invariant, so it may act by ``scaling.''

	Note that $\op{Hom}_K(V_\nu,\CC[\mf p])$ can be identified with $K$-invariant polynomial maps $\mf p\to V_\nu^*$ by definition. However, by \Cref{prop:generic-split-element}, the Zariski density of $K(\mf a)$ implies that such a map is uniquely determined by its restriction to $\mf a$. Thus, we have an embedding
	\[\op{Hom}_K(V_\nu,\CC[\mf p])\into\CC[\mf a]\otimes V_\nu^*.\]
	Now, \Cref{thm:chev-res}, we may as well choose a Cartan $\mf h$ containing $\mf a$, and then $\CC[\mf g]^G$ is identified with $\CC[\mf h]^W$. Because $\mf a\subseteq\mf h$, we know that $\CC[\mf a]$ is finitely generated over $\CC[\mf h]$ and so $\CC[\mf h]^W$ as well, so the result follows.

	We now turn to (b). Fix an irreducible $(\mf g,K)$-module $M$. For a given weight $\nu$ of $K$, we would like to show that the isotypic component $M_\nu$ is finite-dimensional. Fix some $v\in M_\nu$, which we may as well assume is nonzero. Because $M$ is irreducible, the induced map $U\mf g\to M$ given by $X\mapsto Xv$ is surjective, but this map factor through $U\mf g\otimes_{U\mf k}V_\nu$ because $v\in M_\nu$. Now, we have an isomorphism
	\[\CC[\mf p^*]\otimes V_\nu\to U\mf g\otimes_{U\mf k}V_\nu\]
	of $K$-modules by the PBW theorem (with a suitably ordered basis of $\mf g$). Additionally, Schur's lemma tells us that $Z\mf g$ acts by scalars according to some character $\chi$, so we also have a surjection
	\[\CC_\chi\otimes_{Z\mf g}(U\mf g\otimes_{U\mf k}V_\nu)\onto M.\]
	Now, $(U\mf g\otimes_{U\mf k}V_\nu)$ has a filtration given by $\CC[\mf p^*]\otimes V_\nu$, and one can find that the action of $Z\mf g$ is the same one as the action of $\CC[\mf g]^G$ on $\CC[\mf p^*]\otimes V_\nu$ from (a). Thus, we have a surjection
	\[\CC\otimes_{\CC[G]}(\CC[\mf p]\otimes V_\nu)\onto\op{gr}\left(\CC_\chi\otimes_{Z\mf g}(U\mf g\otimes_{U\mf k}V_\nu)\right).\]
	It remains to show that the left-hand object is admissible, which follows from (a): for any $\mu$, we see that
	\[\op{Hom}_K\left(V_\mu,\CC\otimes_{\CC[G]}(\CC[\mf p]\otimes V_\nu)\right)=\op{Hom}_K(V_\mu\otimes V_\nu^*,\CC\otimes_{\CC[\mf g]^G}\CC[\mf p])\]
	but this last space is finite-dimensional by (a).
\end{proof}
\begin{remark}
	Philosophically, the point is that $K$-equiinvariant $\CC[\mf p^*]$-modules can be used to study $(\mf g,K)$-modules. Roughly speaking, $(\mf g,K)$-modules can be ``de-quantized'' (i.e., take a suitable associated graded module) to $K$-equivariant coherent sheaves on $\mf p^*$. Accordingly, $(\mf g\oplus\mf g,G)$-modules can be studied via $G$-equivariant 
\end{remark}
\begin{example}
	A finitely generated $(\mf g,K)$-module $M$ has a support $\op{supp}M\subseteq\mf p^*$. For an appropriately defined filtration of $M$, the module $\op{gr}M$ is a finitely generated $K$-invariant module over $\CC[\mf p^*]$. It turns out that $\op{supp}\op{gr}M$ is a closed subset of $\mf p^*$ not depending on the filtration, so it follows that the dimension of the support is independent of the filtration.
\end{example}

\subsection{The Chevalley--Sheppard--Todd Theorem}
Let's return to studying the center.
\begin{definition}
	Fix a complex vector space $V$. Then a \textit{complex reflection} is an automorphism $g\in\op{GL}(V)$ of finite order such that $V^{\langle g\rangle}\subseteq V$ has codimension $1$.
\end{definition}
\begin{theorem} \label{thm:easy-cst}
	Fix a complex semisimple Lie algebra $\mf g$, and let $W$ act on $\mf h^*$ as usual. % Fix a complex vector space $V$ and a finite group $W\subseteq\op{GL}(V)$, and suppose that $W$ is generated by reflections.
	\begin{listalph}
		\item Then $\CC[\mf h]$ is free as a module over $\CC[\mf h]^W$.
		\item Then $\CC[\mf h]^W$ is a polynomial algebra.
	\end{listalph}
\end{theorem}
\begin{remark}[Chevalley--Sheppard--Todd]
	In fact, one can show that a subgroup $W\subseteq\op{GL}(V)$ is generated by complex reflections if and only if $\CC[V]^W$ is isomorphic to a polynomial algebra.
\end{remark}
\begin{example}
	Letting the Weyl group $W$ act on $\mf h^*$, \Cref{thm:easy-cst} implies that $\CC[\mf h^*]^{W,\cdot}$ is a free commutative algebra.
\end{example}
\begin{example}
	By \Cref{thm:chev-res}, we see that $\CC[\mf g]^G$ is also a free commutative algebra.
\end{example}
\begin{proof}[Proof of \Cref{thm:easy-cst}]
	The main content lies in (a).
	\begin{enumerate}
		\item We will use the Demazure operators. Given a complex reflection $s_\alpha\in W$, we see that $\mf h^{s_\alpha}$ is the kernel of $\alpha$. Thus, for a polynomial $P\in\CC[\mf h]$, we note that $s(P)-P$ automatically vanishes on the kernel of the linear functional $\alpha$, so we may define
		\[D_sP\coloneqq\frac{sP-P}\alpha,\]
		which we know must continue to be a polynomial in $\CC[\mf h]$. The point is that $P$ is invariant under $W$ is equivalent to saying that $sP=P$ for all $s$, which is equivalent to $Ds_P=0$ for all $s$. As such, $D_s(QP)=QD_s(P)$ as long as $Q\in\CC[\mf h]^W$.

		\item Now, because $W$ is a finite group, $\CC[\mf h]$ is finitely generated over $\CC[\mf h]^W$ (for example, this ring extension is integral because $f\in\CC[\mf h]$ is a root of the monic polynomial $\prod_{w\in W}(x-wf)$ in $\CC[\mf h]^W$), so the quotient
		\[A\coloneqq\frac{\CC[\mf h]}{\left(\CC[\mf h]_+^W\right)}\]
		is a finite-dimensional graded vector space; here, the $_+$ means that we are considering polynomials with no constant term, and we are taking the quotient by the generated ideal. As such, we may choose homogeneous polynomials $P_1,\ldots,P_d$ whose images produce a basis in this quotient. A graded version of Nakayama's lemma then implies that they generate $\CC[\mf h]$ over $\CC[\mf h]^W$.\footnote{More precisely, this follows by some induction on degrees: write a given polynomial in $\CC[\mf h]$ as a linear combination in the $P_\bullet$s with coefficients in $\CC[\mf h]_+^W$ and $\CC[\mf h]$. Then we can inductively write the coefficients in $\CC[\mf h]_+^W$ as linear combinations, and the process terminates by a degree argument.}
		
		\item It remains to check that there are no relations among these generators. As such, suppose that we have polynomials $Q_1,\ldots,Q_d\in\CC[\mf h]^W$ for which
		\[\sum_iQ_iP_i=0.\]
		We may as well assume that this relation is minimal. The point is to use the Demazure operators to produce a smaller relation. In particular, for each $i$, we claim that we can find some $D$ in the polynomial algebra of Demazure operators for which %operators $D_{s_1},\ldots,D_{s_n}$ so that
		\[D(P_j)=\begin{cases}
			1 & \text{if }i=j, \\
			0 & \text{if }\deg P_j\le\deg P_i\text{ or }i\ne j.
		\end{cases}\]
		Then we may apply the Demazure operators to our relation, choosing $i$ to be the $P_i$ with maximal degree and $Q_iP_i$ nonzero; this then produces a smaller relation!

		\item It remains to show the main claim of the previous paragraph, constructing the operator $D$. The key step is to show that a homogeneous polynomial $P\in\CC[\mf h]$ admits some $D=D_{s_1}\cdots D_{s_n}$ for which $DP=1$. To see this, note that $A^W$ only has the constants (because higher-degree $W$-invariant polynomials are killed in the quotient of $A$). Thus, there is nothing to do when $P$ is $W$-invariant. On the other hand, if $\ov P\in A$ is not $W$-invariant, then there is a reflection $s$ with $s\ov P\ne\ov P$, so $D_sP\ne0$, so $s\ov P-\ov P=\ov\alpha\cdot\ov{D_sP}$. We also see that $\deg D_sP<\deg P$ for degree reasons, so we may induct down.

		To complete the proof of the main claim, let $\mathrm{Dem}$ be the algebra of Demazure operators, and we note that the canonical pairing
		\[\mathrm{Dem}_d\times A_d\to\CC\]
		is perfect by the previous paragraph. Namely, $A_d\into\mathrm{Dem}_d^*$ by the previous paragraph, so by duality, $\mathrm{Dem}_d\onto A_d^*$, which provides the required operator $D$ by unwinding this statement.
	\end{enumerate}
	We now explain how (a) implies (b). This follows from the following criterion of Serre.
	\begin{proposition}[Serre] \label{prop:serre-criterion}
		Fix $A$ to be a $\NN_{\ge0}$-graded commutative algebra with $A_0=\CC$. Then $A$ is free (as an algebra) if and only if there is a finite free resolution for $\CC$ as an $A$-module.
	\end{proposition}
	More generally, if $\Spec A$ is smooth if and only if it has finite homological dimension, meaning that each maximal ideal $\mf m$ grants $A/\mf m$ a finite projective resolution. To see why this criterion is helpful, note that we may simply choose the finite free resolution of $\CC$ for $\CC[\mf h]$ and then simply restrict our ring to $\CC[\mf h]^W$, thereby producing a free resolution for $\CC$ over $\CC[\mf h]^W$.
\end{proof}
\begin{remark}
	Once we know that $\CC[\mf h]^W$ is free, we automatically know that it is free of rank $\left|W\right|$. For example, one can show this by passing to the fraction fields and using Galois theory. Alternatively, the map $\Spec\CC[\mf h]\to\Spec\CC[\mf h]^W$ is an integral map, so the Krull dimension must be preserved.
\end{remark}
\begin{remark}
	This ring $A$ in the proof turns out to be the cohomology ring $\mathrm H^*(G/B)$. One can calculate that $\dim A_n$ is the number of Weyl elements of given length $n$. This is the sort of thing which can be checked by hand.
\end{remark}

\end{document}