% !TEX root = ../notes.tex

\documentclass[../notes.tex]{subfiles}

\begin{document}

\section{October 28}
We continue.

\subsection{The Jacobson--Morosov Theorem}
For no reason in particular, we begin with a discussion of the Jacobson--Morosov theorem.
\begin{theorem}[Jacobson--Morosov] \label{thm:jacobson-morosov}
	Fix a complex semisimple Lie algebra $\mf g$, and choose a nilpotent element $e\in\mf g$. Then there is an $\mathfrak{sl}(2)$-triple $(e,f,h)$ in $\mf g$.
\end{theorem}
\begin{proof}
	We follow \cite[Section~3.7]{chriss-ginzburg-rep-theory}. Throughout, $(-,-)$ denotes the Killing form on $\mf g$.
	\begin{enumerate}
		\item We show that there is a semisimple element $h\in\mf g$ for which $[e,h]=2e$. The semisimplicity is not much of a constraint: if we can find any $h\in\mf g$ for which $\op{ad}_h(e)=-2e$, then we may consider the Jordan decomposition $h=h_s+h_n$ of $h$, which is induced by the Jordan decomposition of ${\op{ad}_h}\in\mf{gl}(\mf g)$. In particular, because $e$ is an eigenvector of $\op{ad}_h$ with eigenvalue $-2$, the same will be true of the semisimple element $h_s$.

		Thus, it remains to explain why $2e$ is in the image of $\op{ad}_e\colon\mf g\to\mf g$. Well, $\op{ad}_e$ has kernel $\mf g_e$, and we claim that its image is contained in $\mf g_e^\perp$; this implies that $\op{ad}_e$ restricts to an injection $\mf g/\mf g_e\to\mf g_e^\perp$, which is then an isomorphism for dimension reasons. To prove the claim, we choose some $x\in\mf g$ and $y\in\mf g_e$ and compute
		\[(\op{ad}_ex,y)=-(x,[e,y])=0.\]

		\item We explain how to construct $f$ in the case that $\mf g_e$ is nilpotent. We have to show that there is some $f$ for which $[e,f]=h$, and then we may adjust $f$ to satisfy $[h,f]=-2f$ formally: note that
		\[{\op{ad}_h\circ\op{ad}_e}={\op{ad}_e\circ({\op{ad}_h}+2)},\]
		so $\op{ad}_e$ sends eigenvectors of $h$ with eigenvalue $\lambda$ to eigenvectors of $h$ with eigenvalue $\lambda+2$, so one merely needs to replace the given $f$ with $\op{ad}_ef=h$ with the component of $f$ for which the eigenvalue of $\op{ad}_h$ acting on $f$ is $-2$.
		
		We thus want to show that $h$ is in the image of $\op{ad}_e\colon\mf g\to\mf g$. As discussed in the preceding step, it is enough to show that $h\in\mf g_e^\perp$. For this, consider the subspace $\CC h\oplus\mf g_e$ of $\mf g$. This is a Lie subalgebra because $[h,\mf g_e]\subseteq\mf g_e$: for any $x\in\mf g_e$, we see that $[e,[h,x]]$ equals
		\[[[e,h],x]-[[e,x],h]=[2e,x]-[0,h],\]
		which vanishes. In fact, because $\mf g_e$ is nilpotent, the fact that $[h,\mf g_e]\subseteq\mf g_e$ shows that $\CC h\oplus\mf g_e$ is solvable. As such, there is a basis of $\mf g$ so that the adjoint representation restricted to $\CC h\oplus\mf g_e$ is upper-triangular. Because $h$ is semisimple, it must go to a semisimple element, so it goes to a diagonal element under the new basis of $\mf g$. On the other hand, $\mf g_e$ is nilpotent, so it must go to strictly upper triangular elements. It follows that ${\op{ad}_h}\circ{\op{ad}_x}$ is still strictly upper triangular for any $x\in\mf g_e$, so $(h,x)=0$ follows.

		\item We explain how to construct $f$ in the general case. We will do this by induction on $\mf g$. If $\mf g_e$ is already nilpotent (for example, if $\mf g=0$), then we are done by the previous step. Otherwise, the Jordan decomposition implies that $\mf g_e$ contains a nonzero semisimple element $x$. By standard weight theory, the centralizer $\mf g_x$ is reductive (indeed, the Killing form restricted to $\mf g_x$ can be shown to be non-degenerate).

		We now claim that $e$ is contained in the semisimple part of $\mf g_x$, which will complete the proof by passing (inductively) to $\mf g_x$ because $\mf g_x$ is strictly smaller than $\mf g$. (Indeed, $\op{ad}_x$ is semisimple, so $\mf g_x=\mf g$ implies that $x=0$.) Well, $\mf g_x$ splits into its center (which is a torus) and its semisimple part, so because $e$ is nilpotent, it cannot have any component from the torus (by the Jordan decomposition).
		\qedhere
	\end{enumerate}
\end{proof}
\begin{remark}
	In the $\mf{sl}(2)$-triple $(e,f,h)$, we see that $f$ must be nilpotent, and $h$ must be semisimple. Indeed, $(e,f,h)$ arise from a morphism $\varphi\colon\mf{sl}(2)\to\mf g$. Thus, ${\op{ad}}\circ\varphi$ is just some representation of $\mf{sl}(2)$, and the representation theory of $\mf{sl}(2)$ shows that $\begin{bsmallmatrix}
		0 & 0 \\ 1 & 0
	\end{bsmallmatrix}$ must go to a nilpotent element, and $\begin{bsmallmatrix}
		1 & 0 \\ 0 & -1
	\end{bsmallmatrix}$ must go to a semisimple element.
\end{remark}
\begin{remark}
	Given $h$, we can see that $f$ is unique: if $f$ and $f'$ both complete the $\mf{sl}(2)$-triple, then we have two algebra morphisms $\varphi,\varphi'\colon\mf{sl}(2)\to\mf g$ with $\varphi(e)=\varphi'(e)$ and $\varphi(h)=\varphi'(h)$. It follows that $\ker\varphi\subseteq\mf{sl}(2)$ is some Lie ideal containing both $e$ and $h$, so it must totally vanish, thereby implying that $f=\varphi(f)=\varphi'(f)=f'$.
\end{remark}
\begin{remark} \label{rem:h-uniq-jacboson-morosov}
	Let's explain how unique $h$ is. Given two semisimple elements $h$ and $h'$ which extend to $\mf{sl}(2)$-triples $(e,f,h)$ and $(e,f',h')$, we see that $h-h'\in\mf g_e\cap\im\op{ad}_e$. Now, decompose $\mf g$ according to the eigenspaces $\mf g[i]$ of $\op{ad}_h$, which we see must have only integral eigenvalues $i$. It follows that $h-h'$ is contained in
	\[\mf u\coloneqq\bigoplus_{i>0}\mf g_e[i]=\mf g_e\cap\im\op{ad}_e.\]
	It turns out that the corresponding Lie subgroup $U\subseteq G_{\mathrm{ad}}$ has $h+\mf u=U\cdot h$ (roughly speaking, $[\mf u,h]=\mf u$ implies that the orbit is open, and the orbit is closed because $U$ is unipotent), so there is some $u\in U$ for which $u\cdot h=h'$. Thus, we see that $h$ is unique up to multiplication by an element of $\exp(\mf g_e)$.
\end{remark}
\begin{remark}
	Here is an amusing application: for any representation $V$ of $\mf g$, we see that a nilpotent element $e\in\mf g$ must act by a nilpotent operator on $V$. Indeed, \Cref{thm:jacobson-morosov} allows us to pass the statement to $\mf{sl}(2)$, where it follows from the representation theory of $\mf{sl}(2)$.
\end{remark}
We will use \Cref{thm:jacobson-morosov} to construct a canonical filtration on $\mf g$.
\begin{definition}
	Fix a representation $V$ of a complex semisimple Lie algebra $\mf g$, and choose a nilpotent element $e\in\mf g$. Then extend $e$ to an $\mathfrak{sl}(2)$-triple $(e,f,h)$. By the representation theory of $\mf{sl}(2)$, the eigenvalues of $h$ acting on $V$ must all be integers, so we define $V[i]$ to be the eigenspace with eigenvalue $i\in\ZZ$. We then define the filtration
	\[V_{\ge i}\coloneqq\bigoplus_{j\ge i}V[j].\]
\end{definition}
\begin{remark}
	The given filtration does not depend on the choice of $h$. (Of course, its definition shows that it does not depend on the choice of $f$.) Indeed, any other $\mf{sl}(2)$-triple $(e,f',h')$ admits some $u\in G_{\mathrm{ad}}$ as in \Cref{rem:h-uniq-jacboson-morosov} for which $h'=uhu^{-1}$. However, $\op{Ad}_u$ preserves the filtration on $\mf g$ because the Lie subgroup $U$ of \Cref{rem:h-uniq-jacboson-morosov} has its Lie algebra $\mf u$ contained in $\mf g_{\ge1}$, so for any $x\in\mf u$, we see that $\op{ad}_x$ and therefore $\op{Ad}_{\exp(x)}$ preserves the filtration on $\mf g$. The same claim then follows for the representation $V$ by passing to $\mf{gl}(V)$.
\end{remark}
\begin{remark}
	If $V$ is finite-dimensional, then the filtration is exhaustive. Indeed, $h$ has only finitely many eigenvalues.
\end{remark}
% \begin{example}
% 	Let $e\in\mf g$ be a regular nilpotent element. Then we claim that there is a unique Borel subgroup $\mf b$ containing $e$. To see that there is some Borel subgroup, we extend $e$ to an $\mathfrak{sl}(2)$-triple $(e,f,h)$, and then we extend $h$ to a Cartan subalgebra $\mf h\subseteq\mf g$. If $x\in\mf g$ commutes with $h$
% \end{example}

\subsection{Completing the Kostant Slice}
We now prove \Cref{thm:kostant-slice}(b).
\begin{proof}[Proof of \Cref{thm:kostant-slice}]
	It suffices to construct an algebraic bijection $U\times S\to f+\mf b$. There is certainly a natural map $U\times S\to\mf g$ given by $(u,f+x)\mapsto u\cdot(f+x)$. Observe that the action of $h$ on $\mf g$ induces the filtration $\{\mf g_{\ge i}\}_{i\in\ZZ}$. Here are our checks on this action map.
	\begin{itemize}
		\item Well-defined: we claim that the natural map outputs to $f+\mf b$. To start, we note that $\mf g_e\subseteq\mf b$, and $U$ preserves the positive filtration on $\mf g$, so $[\mf u,\mf g_e]\subseteq\mf b$, so $U\cdot\mf g_e\subseteq\mf b$. Thus, it remains to show that $U\cdot f\subseteq f+\mf b$, for which it suffices to show that $[\mf u,f]\subseteq\mf b$. This is true for weight reasons: $f\in\mf g_{-2}$ while $\mf u\subseteq\mf g_{\ge2}$ because $h$ acts with only even weights.

		\item Surjective: given $x\in\mf b$, we need to find $u\in U$ with $u\cdot(f+x)\in f+\mf g_e$. (The surjectivity then follows by inverting $u$.) As above, we will use the natural filtration on $\mf g$; for example, we know that $\mf b=[f,\mf u]+\mf g_e$ because $\mf b=\mf g_{\ge0}$ and $\mf u=\mf g_{\ge2}$. Additionally, we can see that this filtration has $[f,\mf u]\cap\mf g_e=0$ by comparing the filtration with the representation theory of $\mf{sl}(2)$.

		We now show by induction on $d$ that there is some $u\in U$ with $u\cdot(f+x)$ in $f+\mf g_e+\mf g_{\ge d}$. There is nothing to do for $d=0$ because then $\mf g_{\ge0}=\mf b$. Thus, if we suppose that we already have $u\cdot(f+x)\in f+\mf g_e+\mf g_{\ge d}$, we need to find a way to kill the piece $x_d$ in degree $d$. As such, we write
		\[x_d\coloneqq x^\circ+[f,y]\]
		where $x^\circ\in\mf g_e$ and $y\in\mf g_{d-2}$. Then one can check that replacing $u$ with $u\exp(-y)$ will work.

		\item We omit the proof of injectivity. In short, if $u\cdot(f+x)=(f+x')$, then one can find $y\in\mf u$ with $[f+x,y]=0$, which can be seen to produce a problem after going down to $\mf{sl}(2)$.
		\qedhere
	\end{itemize}
\end{proof}

\subsection{Hamiltonian Reduction}
One can view \Cref{thm:kostant-slice} as a variant of Hamiltonian reduction. To explain this, we must define a Poisson variety.
\begin{definition}[Poisson variety]
	A \textit{Poisson variety} is variety $X$ for which the ring of global functions $\OO(X)$ is equipped with a Lie bracket $\{-,-\}$ such that $\{f,-\}$ is a derivation for all $f$. In other words,
	\[\{f,gh\}=h\{f,g\}+g\{f,h\}\]
\end{definition}
\begin{remark}
	An alternating bracket $\{-,-\}$ on $\OO(X)$ can be described by $\pi\in\Gamma\left(X,\land^2TX\right)$, from which the bracket is defined by
	\[\{f,g\}\coloneqq\langle\pi,df\land dg\rangle.\]
	This construction also makes it clear that an alternating bracket $\{-,-\}$ gives rise to a unique $\pi$. Of course, there are some conditions (namely, $[\pi,\pi]=0$) in order to have a Poisson structure on $X$.
\end{remark}
\begin{example}
	If $\pi\in\Gamma\left(X,\land^2TX\right)$ is non-degenerate (i.e., if $\dim X$ is some even integer $2m$ for which $\land^m\pi$ vanishes nowhere), then we may define $\omega\coloneqq\pi^{-1}$ to be a class in $\land^2T^*X=\Omega^2X$. In this case, $[\pi,\pi]=0$ is equivalent to $d\omega=0$, meaning that a non-degenerate Poisson structure on $X$ is equivalent to having a symplectic manifold $(X,\omega)$.
\end{example}
\begin{example}
	As an instance of the above example, we see that the cotangent space $T^M$ of a smooth manifold $M$ admits a symplectic structure $\omega\coloneqq d\eta$, where
	\[\eta\coloneqq\sum_{ij}p_i\,dq_j,\]
	where the $p_\bullet$s and $q_\bullet$s are suitably defined local coordinates. Thus, $T^*M$ admits the structure of a Poisson variety.
\end{example}
\begin{example}
	The affine variety $X\coloneqq\mf g^*$ admits a Poisson structure by taking $\{-,-\}$ to be the usual bracket on linear functionals. For example, any $G$-orbit in $\mf g^*$ gains a symplectic structure as in the previous example, meaning that all $G$-orbits are even-dimensional!
\end{example}
\begin{remark}
	One can unwind the proof that $G$-orbits in $\mf g$ have even dimension as follows: the tangent space of an orbit at some $x\in\mf g$ is $\mf g/[\mf g,x]$, which admits a well-defined alternating non-degenerate pairing
	\[\omega(y,z)\coloneqq([y,z],x),\]
	where $(-,-)$ is the Killing form on $\mf g$. It then follows that the tangent space is even-dimensional, from which the result follows.
\end{remark}
Group actions on Poisson varieties frequently give rise to moment maps.
\begin{definition}[moment map]
	Fix a Lie group $G$ acting on a Poisson variety $X$. Then a \textit{moment map} is a map $\mu\colon\mf g\to\OO(X)$ making the diagram
	% https://q.uiver.app/#q=WzAsMyxbMCwwLCJcXG1mIGciXSxbMSwwLCJcXE9PKFgpIl0sWzEsMSwiXFxvcHtWZWN0fShYKSJdLFswLDEsIlxcbXUiLDAseyJzdHlsZSI6eyJib2R5Ijp7Im5hbWUiOiJkYXNoZWQifX19XSxbMCwyXSxbMSwyXV0=&macro_url=https%3A%2F%2Fraw.githubusercontent.com%2FdFoiler%2Fnotes%2Fmaster%2Fnir.tex
	\[\begin{tikzcd}[cramped]
		{\mf g} & {\OO(X)} \\
		& {\op{Vect}(X)}
		\arrow["\mu", dashed, from=1-1, to=1-2]
		\arrow[from=1-1, to=2-2]
		\arrow[from=1-2, to=2-2]
	\end{tikzcd}\]
	commute; here, the map $\mf g\to\op{Vect}(X)$ is induced by the $G$-action, and the map $\OO(X)\to\op{Vect}(X)$ is given by $f\mapsto df\perp\pi$.
\end{definition}
\begin{remark}
	One can equivalently describe $\mu$ as the data of a $G$-equivariant map $X\to\mf g^*$ satisfying some commutativity constraints.
\end{remark}
\begin{example}
	If $G$ acts on a manifold $M$, then the action extends to $X\coloneqq T^*M$, and there is a moment map $\mu\colon X\to\mf g^*$ given by
	\[\mu(x,\xi)\colon a\mapsto\langle a\cdot x,\xi\rangle,\]
	where $a\cdot x$ refers to the induced $\mf g$-action.
\end{example}
\begin{theorem}[Hamiltonian reduction] \label{thm:hamiltonian-reduction}
	Let $G$ act freely on a manifold $M$, and let $N$ be the quotient. Then it turns out that
	\[T^*N=\mu^{-1}(\{0\})/G,\]
	which is referred to as Hamiltonian reduction.
\end{theorem}
\begin{remark}
	The condition that $G$ acts freely on $M$ means that the induced projection $M\to N$ is a principal $G$-bundle.
\end{remark}
\begin{remark}
	Let's relate \Cref{thm:hamiltonian-reduction} to \Cref{thm:kostant-slice}. Indeed, \Cref{thm:kostant-slice}(b) amounts to the statement that
	\[\mf g^*/(U,\psi_f)=S,\]
	where $\psi_f\colon x\mapsto[f,x]$. This is analogous to a calculation of a tangent space appearing in \Cref{thm:hamiltonian-reduction}.
\end{remark}

\subsection{Whittaker Things}
We close our discussion today by defining some objects related to Whittaker modules.
\begin{definition}[Whittaker generator]
	Fix a complex semisimple Lie algebra $\mf g$, and let $(e,f,h)$ be the principal $\mf{sl}(2)$-triple. Then the \textit{Whittaker generator} is $W\coloneqq U\mf g\otimes_{U\mf u}\CC\psi$, where $\psi\colon\mf u\to\CC$ is defined by $\psi(x)\coloneqq[x,f]$.
\end{definition}
\begin{remark}
	To check that $\psi$ is a character, we note that $\psi$ vanishes on $[\mf u,\mf u]$ for weight reasons.
\end{remark}
\begin{definition}[Whittaker module]
	Fix a complex semisimple Lie algebra. Then a $\mf g$-module $M$ is \textit{Whittaker} if and only if $M\otimes_{U\mf u}\CC_{-\psi}$ has the action of $\mf u$ to be locally nilpotent. In other words, each vector in $M\otimes_{U\mf u}\CC_{-\psi}$ is killed by some power of the $\mf u$-action.
\end{definition}
\begin{theorem}[Kostant]
	Fix a complex semisimple Lie algebra $\mf g$, and let $(e,f,h)$ be the principal $\mf{sl}(2)$-triple.
	\begin{listalph}
		\item There is an isomorphism $Z\mf g\to\op{End}(W)$, and $\op{Ext}^i_{\mf g}(W,W)=0$ for all $i>0$.
		\item The category of Whittaker modules is equivalent to the category of $Z\mf g$-modules. The functor takes a Whittaker module $M$ to $\op{Hom}(W,M)$.
	\end{listalph}
\end{theorem}

\end{document}