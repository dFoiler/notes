% !TEX root = ../notes.tex

\documentclass[../notes.tex]{subfiles}

\begin{document}

\section{October 21}
We began class by completing the proof of \Cref{thm:easy-cst}.

\subsection{Serre's Criterion}
Let's sketch the proof of \Cref{prop:serre-criterion}. If $A$ is a polynomial algebra $\CC[x_1,\ldots,x_n]$, then we can use the Koszul complex. More precisely, if $A=\CC[x]$, then we have the resolution
\[0\to\CC[x]\stackrel x\to\CC[x]\to\CC\to0,\]
so if $A=\CC[x_1,\ldots,x_n]=\CC[x_1]\otimes\cdots\otimes\CC[x_n]$, then we can just take the tensor product of the free resolutions. In a more basis-free way, we can write the Koszul resolution as
\[0\to\land^nV\otimes A\to\cdots\land^2V\otimes A\to V\otimes A\to A,\]
where $V$ is some $n$-dimensional vector space, and $A=k[V^*]$.

Of course, we are more interested in the reverse direction. Quickly, if $\dim A<\infty$ and $A\ne\CC$, then there cannot be a finite free resolution of $\CC$ by considering Euler characteristics. Namely, for a graded vector space $V$, write $P_V(t)\coloneqq\sum_it^i\dim V_i$. Then any free resolution
\[0\to V_n\otimes A\to\cdots\to V_1\otimes A\to V_0\otimes A\to\CC\to0\]
must have
\[P_A(t)\Bigg(\sum_{i=0}^n(-1)^nP_V(t)\Bigg)=1\]
by counting dimensions. We conclude that $P_A(t)=1$, so $A=\CC$.

Thus, we may suppose that $A$ is infinite-dimensional. Then we select some $a\in A$ which is not a zero divisor and which is not in the ideal $I$ of positive-degree elements, and define $B\coloneqq A/(a)$. The point is to show that $k$ admits a finite free resolution over $B$ implies that it admits a finite free resolution over $A$; from here, an induction with some minimal generators shows that $B$ being free implies that $A$ is free. Indeed, take
\[0\to F_n\to\cdots\to F_1\to F_0\to\CC\to0\]
and apply $-\otimes B$. Of course, this doesn't have to be exact, but we can show that it has a direct summand which is a resolution of $k$; the key input is that $\op{Ext}^2(k,k)=0$.

\subsection{The Kostant Slice}
We recall the following definition.
\begin{definition}[semisimple]
	An element $x\in\mf g$ is \textit{regular} if and only if its centralizer $\mf g_x$ has
	\[\dim\mf g_x=\op{rank}r.\]
\end{definition}
\begin{remark}
	Suppose $x$ is semisimple. Observe that $\mf g_x$ is the kernel of the map $[X,-]$, so by placing $X$ in some Cartan subalgebra $\mf h$, we find that $\dim\mf g_x\ge\dim\mf h$, so $\dim\mf g_x\ge\op{rank}\mf g$. Thus, we are saying that $\dim\mf g_x$ is as small as possible.
\end{remark}
\begin{example}
	In the case where $\mf g=\mf{sl}(n)$, one can put a given $x\in\mf g$ in Jordan normal form, and a calculation of the centralizer by hand shows that $x$ is regular if and only if it has a single Jordan block for each eigenvalue, which is equivalent to the minimal polynomial being equal to the characteristic polynomial.
\end{example}
\begin{remark}
	One can check that being regular is a Zariski open condition. For example, the only irregular element in $\mf{sl}(2)$ is zero.
\end{remark}
\begin{proposition}
	Fix a semisimple Lie algebra $\mf g$. Let $\mf h\subseteq\mf g$ be a Cartan subalgebra, with roots $\Phi$ and simple roots $\Delta$. Then the element
	\[e\coloneqq\sum_{\alpha\in\Delta}e_\alpha\]
	is regular.
\end{proposition}
\begin{proof}
	Setting $h\coloneqq2\rho^\lor$, we find that $\langle h,\alpha_i\rangle=2$ for each simple root $\alpha_i$. As such, $[h,e]=2e$. Now, recall that the elements $h_i=[e_i,f_i]$ form basis of $\mf h$, so we may find $f$ of the form $\sum_if_i$ for which $[e,f]=h$. Now, $[h,f_i]=-2f_i$ for each $i$, so $[h,f]=-2f$.

	Thus, $(e,f,h)$ is some $\mf{sl}(2)$-triple, so we may calculate the centralizer of $e$ using the representation theory of $\mf{sl}(2)$. Note that only even-weight representations can appear because $h$ acts on $\mf g_\alpha$ with nonzero even eigenvalues (namely, $2$ for the simple roots) anyway. Then $\mf g_e$ is the number of summands in the decomposition into $\mf{sl}(2)$-representations, which now equals the dimension of the weight-$0$ subspace. But the weight-$0$ subspace is simply the centralizer of $h$, which is exactly $\mf h$ because $h$ acts on each $\mf g_\alpha$ by nonzero eigenvalues anyway.
\end{proof}
\begin{remark}
	It turns out that all regular nilpotent elements are conjugate to $e$.
\end{remark}
\begin{definition}[principal triple]
	Fix a semisimple Lie algebra $\mf g$ with Cartan subalgebra $\mf h\subseteq\mf g$. Then we define the $\mf{sl}(2)$-triple $(e,f,h)$ so that $e\coloneqq\sum_ie_i$ and $h\coloneqq2\rho^\lor$. This is the \textit{principal $\mf{sl}(2)$} triple.
\end{definition}
\begin{definition}[Kostant slice]
	Fix a semisimple Lie algebra $\mf g$ with Cartan subalgebra $\mf h\subseteq\mf g$, and let $(e,f,h)$ be the principal $\mf{sl}(2)$-triple. Then we define the \textit{Kostant slice} to be $S\coloneqq f+\mf g_e$.
\end{definition}
\begin{theorem}
	Fix a semisimple Lie algebra $\mf g$ with Cartan subalgebra $\mf h\subseteq\mf g$, and let $(e,f,h)$ be the principal $\mf{sl}(2)$-triple with Kostant slice $S$.
	\begin{listalph}
		\item There is an isomorphism $S\to\mf t/W$ given by restriction, meaning that the map $\CC[\mf g]^G\to\CC[S]$ is an isomorphism.
		\item There is an isomorphism $S\to(f+\mf b)/U$ given by inclusion, and the action map $(U\times S)\to(f+\mf b)$ induces an isomorphism.
	\end{listalph}
\end{theorem}
\begin{remark}
	This is related to Whittaker models.
\end{remark}
\begin{example}
	If $\mf g=\mf{sl}(n)$, then the centralizer of $f$ consists of the upper-triangular matrices which are constant on the diagonals. One can use this description to prove (a) by hand.
\end{example}

\end{document}