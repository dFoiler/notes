% !TEX root = ../notes.tex

\documentclass[../notes.tex]{subfiles}

\begin{document}

\section{May 3}

Let's discuss the descent obstruction.

\subsection{The Descent Obstruction}
As motivation, recall that the Brauer--Manin obstruction essentially took local points in $X(\AA_K)$, sent them to $\prod_{v\in V_K}\op{Br}K_v$, and asked if they were in the image of $\op{Br}K$. The points that did in fact come from $\op{Br}K$ are now points which might come from global points. Translating everything into cohomology, we are basically computing the fiber product of the following square in set.
% https://q.uiver.app/?q=WzAsNCxbMCwxLCJIXjIoXFxvcHtHYWx9KEtee1xcbWF0aHJte3NlcH19L0spLEtee1xcbWF0aHJte3NlcH1cXHRpbWVzfSkiXSxbMSwxLCJcXGRpc3BsYXlzdHlsZVxccHJvZF97dlxcaW4gVl9LfUheMihcXG9we0dhbH0oS157XFxtYXRocm17c2VwfX1fdi9LX3YpLEtfdl57XFxtYXRocm17c2VwfVxcdGltZXN9KSJdLFsxLDAsIlgoXFxBQV9LKSJdLFswLDAsIlgoXFxBQV9LKV57XFxhbHBoYT0wfSJdLFszLDJdLFsyLDFdLFswLDFdLFszLDBdXQ==&macro_url=https%3A%2F%2Fraw.githubusercontent.com%2FdFoiler%2Fnotes%2Fmaster%2Fnir.tex
\[\begin{tikzcd}
	{X(\AA_K)^{\alpha=0}} & {X(\AA_K)} \\
	{H^2(\op{Gal}(K^{\mathrm{sep}}/K),K^{\mathrm{sep}\times})} & {\displaystyle\prod_{v\in V_K}H^2(\op{Gal}(K^{\mathrm{sep}}_v/K_v),K_v^{\mathrm{sep}\times})}
	\arrow[from=1-1, to=1-2]
	\arrow[from=1-2, to=2-2]
	\arrow[from=2-1, to=2-2]
	\arrow[from=1-1, to=2-1]
\end{tikzcd}\]
Most of this discussion works for general algebraic groups. Before continuing, let's talk a little about algebraic groups.
\begin{example}
	The group $G=\op{GL}_n$ is an algebraic group over a field $k$. Namely, it is an \'etale sheaf because it is a scheme
	\[{\op{GL}_n}=\Spec k[\{x_{ij}\}_{1\le i,j\le n}]_\delta,\]
	where $\delta\in k[\{x_{ij}\}_{1\le i,j\le n}]$ is the determinant.
\end{example}
\begin{example}
	The group $G=\op{PGL}_n$ should be thought of as the quotient ${\op{GL}_n}/\mathbb G_m$, where the quotient happens on the \'etale set. This turns out to be an affine scheme, given by
	\[{\op{PGL}_n}=(\Spec k[\{x_{ij}\}_{1\le i,j\le n}]_\delta)^{\mathbb G_m},\]
	where $\mathbb G_m$ acts by scalar multiplication.
\end{example}
\begin{remark}
	It turns out that defining  $\op{PGL}_nR$ by $\op{GL}_nR/\mathbb G_mR$ is not even an \'etale sheaf! Well, our definition of $\op{PGL}_n$ is the sheafification of this presheaf, so looking at the exact sequence
	\[1\to\mathbb G_m\to{\op{GL}_n}\to{\op{PGL}_n}\to1,\]
	it is enough to check if $\op{GL}_nR\to\op{PGL}_nR$ is always surjective. Well, it fails to be surjective at, for example, $R=\OO_{\op{PGL}_n}$, which one can check directly.
\end{remark}
Quickly, we recall that $\op{Br}X=H^2(X_{\mathrm{et}},\mathbb G_m)$, which we discussed earlier. In particular, we can recover $H^2(X_{\mathrm{et}},\mathbb G_m)$ as the union of the images from $H^1(X_{\mathrm{et}},\mathrm{PGL}_n)$, so we might as well just look at $\op{PGL}_n$ and examine the following pullback square
% https://q.uiver.app/?q=WzAsNCxbMCwwLCJYKFxcQUFfSyleXFxhbHBoYSJdLFsxLDAsIlgoXFxBQV9LKSJdLFsxLDEsIlxcZGlzcGxheXN0eWxlXFxwcm9kX3t2XFxpbiBWX0t9SF4xKEtfdix7XFxvcHtQR0x9X259KSJdLFswLDEsIkheMShLLHtcXG9we1BHTH1fbn0pIl0sWzAsMV0sWzEsMl0sWzMsMl0sWzAsM11d&macro_url=https%3A%2F%2Fraw.githubusercontent.com%2FdFoiler%2Fnotes%2Fmaster%2Fnir.tex
\[\begin{tikzcd}
	{X(\AA_K)^\alpha} & {X(\AA_K)} \\
	{H^1(K,{\op{PGL}_n})} & {\displaystyle\prod_{v\in V_K}H^1(K_v,{\op{PGL}_n})}
	\arrow[from=1-1, to=1-2]
	\arrow[from=1-2, to=2-2]
	\arrow[from=2-1, to=2-2]
	\arrow[from=1-1, to=2-1]
\end{tikzcd}\]
where $\alpha$ is now some class in $H^1(X,{\op{PGL}_n})$. Taking intersections, we can now see an obstruction
\[X(\AA_K)^{\op{PGL}_n}=\bigcap_{n\ge0}\bigcap_{\alpha\in H^1(X,{\op{PGL}_n})}X(\AA_K)^\alpha.\]
It is a theorem that actually $X(\AA_K)^{\op{PGL}_n}=X(\AA_K)^{\mathrm{Br}}$, but the point of this discussion is that we could actually swap $\op{PGL}_n$ with any reductive group $G$ one pleases. So we define
\[X(\AA_K)^{\mathrm{descent}}=\bigcap_G\bigcap_{\alpha\in H^1(X,G)}X(\AA_K)^\alpha,\]
where the intersection of $G$ is working over affine finite type group $k$-schemes. Notably, because we remarked $X(\AA_K)^{\op{PGL}_n}=X(\AA_K)^{\mathrm{Br}}$, we certainly have $X(\AA_K)^{\mathrm{descent}}\subseteq X(\AA_K)^{\mathrm{Br}}$, so we hope to have a better invariant. Sadly, we are not out of the woods.
\begin{theorem}[Poonen]
	There exist $X$ such that
	\[X(K)\subsetneq X(\AA_K)^{\mathrm{descent}}\subsetneq X(\AA_K)^{\mathrm{Br}}.\]
\end{theorem}
\begin{remark}
	There is a conjectural program which provides an infinite sequence of obstructions to characterize $X(K)$.
\end{remark}

\subsection{Torsors}
Let's quickly recall our discussion of $G$-torsors. Here, $X$ is a $k$-scheme, and $G$ is an algebraic group over $k$. Now, $H^1(X,G)$ was defined using \'etale cohomology, but one can also view this as isomorphism classes of $G$-torsors, where we mod out by principal homogeneous spaces. Here is the definition of a torsor.
\begin{definition}[torsor]
	Fix a $k$-scheme $X$ and algebraic group $G$ over $k$. Then a \textit{$G$-torsor} is an \'etale sheaf $P$ on $X$ with a $G$-action satisfying the following conditions.
	\begin{itemize}
		\item For any \'etale open set $U$ of $X$ and covering $\{U_i\}$ of $U$, one has $P(U_i)\ne\emp$ for some $U_i$.
		\item If $P(U)\ne\emp$, then the action of $G(U)$ on $P(U)$ is simply transitive.
	\end{itemize}
\end{definition}
\begin{example}
	Fix an exact sequence
	\[1\to G\to\widetilde G\stackrel\pi\to H\to1\]
	of algebraic groups. Now, for fixed $\alpha\in\Gamma(X,H)$, the \'etale sheaf given by $U\mapsto\{g\in\widetilde G(U):\pi(g)=\alpha|_U\}$ is a $G$-torsor. Indeed, one can adjust any element in here by some element in $G(U)$ to satisfy the condition.
\end{example}
\begin{remark}
	Faithfully flat descent shows that any $G$-torsor is representable by a scheme.
\end{remark}
And here is the definition of a principal homogeneous space.
\begin{definition}[principal homogeneous space]
	A  principal homogeneous space is an $X$-scheme $S$ with a $G$-action $G_X\times_XS\to S$ which is isomorphic to $G_X$ locally on the \'etale site.
\end{definition}
Now, for $\alpha\in H^1(X,G)$, we observe that we can decompose
\[X(K)=\bigsqcup_{\gamma\in H^1(K,G)}\{x\in X(K):x^*\alpha=\gamma\}.\]
In some sense, we have ``stratified'' $H^1(X,G)$.
\begin{example}
	Suppose that $\alpha\in H^1(X,G)$ arises from the $X$-scheme $S$ with structure map $q\colon S\to X$. Then we see
	\[\{x\in X(K):x^*\alpha=*\}=q(S(K)),\]
	where $*$ is the trivial class in $H^1(X,G)$. Indeed, for $x\in X(K)$, we draw the following square.
	% https://q.uiver.app/?q=WzAsNCxbMCwwLCJTX3giXSxbMSwwLCJTIl0sWzEsMSwiWCJdLFswLDEsIngiXSxbMCwxXSxbMSwyXSxbMywyXSxbMCwzXV0=&macro_url=https%3A%2F%2Fraw.githubusercontent.com%2FdFoiler%2Fnotes%2Fmaster%2Fnir.tex
	\[\begin{tikzcd}
		{S_x} & S \\
		x & X
		\arrow[from=1-1, to=1-2]
		\arrow[from=1-2, to=2-2]
		\arrow[from=2-1, to=2-2]
		\arrow[from=1-1, to=2-1]
	\end{tikzcd}\]
	Now, $\alpha=*$ is trivial if and only if we can trivialize $S_x$, which amounts to finding a $K$-point in $S_x$, so these are equivalent data to finding an element in $q(S(K))$.
\end{example}

\end{document}