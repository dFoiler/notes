% !TEX root = ../notes.tex

\documentclass[../notes.tex]{subfiles}

\begin{document}

\section{April 3}

There will be no class or office hours on April 5 or April 12. Professor Olsson will post details about the term paper later today.

\subsection{Finiteness of Heights}
Last class we constructed a candidate height function $H_K$ on $\PP^n_K$ where $K$ is a number field. To define our height function $h$ on $(E,e)$ over the field $K$, we use the composite
\[E(K)\to\PP^1_K(K)\stackrel{\log H}\to\RR,\]
where $E\to\PP^1_K$ is the hyperelliptic projection. Our next goal is to show that this function satisfies the conditions of being a height function given in \Cref{def:height}. Today we will focus on showing (c).
\begin{lemma}
	Define $H\colon\PP^n(\overline\QQ)\to\RR$ by $H(P)\coloneqq H_K(P)^{1/[K:\QQ]}$ where $P$ is defined over $K$. Then $H(\sigma P)=H(P)$ for any $\sigma\in\op{Gal}(\overline\QQ/\QQ)$.
\end{lemma}
\begin{proof}
	Note that $H$ does not depend on $K$ by \Cref{lem:h-field-invaraint}, so we might as well show that $H_K(P)=H_{\sigma K}(\sigma P)$. By expanding $K$ to be Galois, we might as well assume that $K=\sigma K$. Then for any place $p\in M_\QQ$, we claim
	\[\prod_{v\mid p}\max\{|x_0|_v,\ldots,|x_n|_v\}\stackrel?=\prod_{v\mid p}\max\{|\sigma x_0|_v,\ldots,|\sigma x_n|_v\},\]
	where $P=[x_0:\ldots:x_n]$. But $\sigma$ will only permute these places $v$ dividing $p$ by uniqueness of extending a place from $\QQ$ to $K$, so we finish.
\end{proof}
\begin{lemma} \label{lem:height-on-q}
	Fix a constant $C$. Then the set $\{P\in\PP^n_\QQ(\QQ):H(P)\le C\}$ is finite.
\end{lemma}
\begin{proof}
	By scaling $P$ appropriately, we may assume that $P=[x_0:\ldots:x_n]$ where the $x_i$ are integers with $\gcd(x_0,\ldots,x_n)=1$. Essentially, one just needs to clear denominators.\footnote{Alternatively, one can use the valuative criterion: the map $P\colon\Spec\QQ\to\PP^n$ extends (uniquely) to a map $\Spec\ZZ\to\to\PP^n$. To see that $\gcd(x_0,\ldots,x_n)=1$ with this map, it is enough to note that the relevant map $\ZZ^n\to\ZZ$ is surjective, which comes from the injectivity of the map $\Spec\ZZ\to\PP^n$.} We can now just compute directly. Indeed, for each prime $p$, we note that $|x_i|_p\le1$ for each $i$, and $\gcd(x_0,\ldots,x_n)=1$ enforces that $p$ does not divide at least one $x_i$, so
	\[\max\{|x_0|_p,\ldots,|x_n|_p\}=1.\]
	Thus, all finite places disappear from our computation, and we are left with
	\[H(P)=\max\{|x_0|_\infty,\ldots,|x_n|_\infty\},\]
	and there are indeed only finitely many points $(x_0,\ldots,x_n)$ with $|x_i|_\infty\le C$ for each $i$; in fact, there are at most $(2C+1)^{n+1}$ of them.
\end{proof}
To continue, we use the following notation.
\begin{lemma} \label{lem:roots-to-poly-height}
	Fix a polynomial $f(t)\in\overline\QQ[t]$ of degree $d$, and write
	\[f(t)=a_0t^0+a_1t^1+\cdots+a_dt^d=a_d(t-\alpha_1)\cdots(t-\alpha_d).\]
	Then
	\[H([a_0:\ldots:a_d])\le 2^{d-1}\prod_{i=1}^dH(\alpha_i),\]
	where $H(\alpha_i)\coloneqq H([\alpha_i:1])$.
\end{lemma}
\begin{proof}
	Scaling the coefficients $a_i$ will not affect either side of the inequality, so we may assume that $a_d=1$. Now, by Vieta's formulae, we make work everywhere in $K\coloneqq\QQ(\alpha_1,\ldots,\alpha_d)$. For notional reasons, we set $\varepsilon(v)\coloneqq1+1_{v\mid\infty}$ so that
	\[|x+y|_v\le\varepsilon(v)\max\{|x|_v,|y|_v\}\]
	for each place $v$. For any fixed place $v$, our goal is to show
	\[\max\{|\alpha_0|_v,\ldots,|\alpha_d|_v\}\le\varepsilon(v)^{d-1}\prod_{i=1}^nH(\alpha_i),\]
	which will finish after taking an infinite product over all factors. (Notably, there are $[K:\QQ]$ infinite places counted with multiplicity, so $H=H_K^{1/[K:\QQ]}$ will perfectly cancel these places.) Well, for $d=1$, we have $a_1=-\alpha_1$, so there is nothing to say here. Then for our induction, we reorder our roots so that $|\alpha_1|_v$ is the largest absolute value. We now set
	\[g(t)\coloneqq\frac{f(t)}{t-\alpha_1}=(t-\alpha_2)\cdots(t-\alpha_d)=b_{d-1}t^{d-1}+b_{d-2}t^{d-2}+\cdots+b_1t+b_0,\]
	where (for example) $b_{d-1}=1$. By expanding, we see that $f(t)=(t-\alpha_1)g(t)$ implies $a_i=b_{i-1}-\alpha_1b_i$ for each $i$. In total, we see
	\begin{align*}
		\max_{0\le i\le d}\{|a_i|_v\} &= \max_{0\le i\le d}\{|b_{i-1}-\alpha_1b_i|_v\} \\
		&\le \max_{0\le i\le d}\varepsilon(v)\max\{|b_{i-1}|_v,|\alpha_1b_{i}|_v\} \\
		&\le\varepsilon(v)\max_{i}\{|b_i|_v\}\max\{|\alpha_1|_v,1\} \\
		&\le\varepsilon(v)\Bigg(\varepsilon(v)^{d-2}\prod_{j=2}^n\max\{|\alpha_j|_v,1\}\Bigg)\max\{|\alpha_1|_v,1\} \\
		&\le\varepsilon(v)^{d-1}\prod_{j=1}^n\max\{|\alpha_j|_v,1\},
	\end{align*}
	which is what we wanted.
\end{proof}
\begin{remark}
	There is also a lower bound to the above result, but we don't need it.
\end{remark}
And now here is our main bounding result.
\begin{theorem}
	Fix a constants $C$ and $d>0$. Then
	\[\#\{P\in\PP^n(\overline\QQ):H(P)\le C,[\QQ(P):\QQ]\le d\}<\infty.\]
\end{theorem}
\begin{proof}
	For any $P=[x_0:\ldots:x_n]$, we see
	\[H_{\QQ(P)}(P)=\prod_v\max_i\{|x_i|_v\}\ge\max_i\prod_v\max\{|x_i|_v,1\}=\max_iH_{\QQ(P)}(x_i),\]
	where the second inequality holds because the left factor takes the largest factor for each place $v$, and the second factor includes fewer large factors. Notably, $H(P)\le C$ and $[\QQ(P):\QQ]\le d$ implies that $H(x_i)\le C$ and $[\QQ(x_i):\QQ]\le d$ for each $i$, so by taking the union over all possible coordinates appropriately, it is enough to show that
	\[\#\{x\in\overline\QQ:H(x)\le C,[\QQ(x):\QQ]\le d\}<\infty\]
	for any $C$ and $d$. Well, for any $x$ in the above set, we let its set of Galois conjugates be $\{x_1,\ldots,x_e\}$ so that the previous lemma yields
	\[f_x(t)\coloneqq(t-x_1)\cdots(t-x_e)=a_et^e+a_{e-1}t^{e-1}+\cdots+a_1t+a_0\]
	has
	\[H([a_0:\ldots:a_e])\le 2^{e-1}\prod_iH(x_i)=2^{e-1}H(x)^e\le(2C)^d.\]
	However, each element $x\in\overline\QQ$ yields a unique point $[a_0:\ldots:a_e]$ on the other side, but there are only finitely many of these such points by \Cref{lem:height-on-q}.
\end{proof}
It follows that our height function $h$ has the desired finiteness because the hyperelliptic projection $E\to\PP^1_k$ is a $2$-to-$1$ map, and we have the finiteness of $\PP^1_k$.

\end{document}