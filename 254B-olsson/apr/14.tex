% !TEX root = ../notes.tex

\documentclass[../notes.tex]{subfiles}

\begin{document}

\section{April 14}

Let's quickly take stock on what we've done.
\begin{itemize}
	\item We studied degree-$2$ polynomials, which is essentially studying genus-$0$ algebraic curves.
	\item We studied elliptic curves over number fields, which is essentially studying genus-$1$ algebraic curves.
	\item It is a result of Faltings that smooth projective curves with genus larger than $2$ have finitely many points.
\end{itemize}
\begin{remark}
	Faltings's theorem is quite hard, though it would be approachable after this class. The standard reference is Silverman's \textit{Arithmetic Geometry}.
\end{remark}
Away from curves, not much is known. For the rest of the class, we are going to discuss local-to-global principles for varieties of dimension at least $2$. Curves just turn out to be much simpler than everything else.

\subsection{Local-to-Global via Ad\'eles}
Fix a smooth projective variety $X$ over a number field $K$. Roughly speaking, local-to-global principles are interested in the following question.
\begin{ques}
	Suppose that $X(K_v)$ is nonempty for every place $v\in V_K$. Then must $X(K)$ be nonempty?
\end{ques}
This was true for quadratics, but the answer is no in general.

To ask better questions, we will use the ad\'eles $\AA_K$ as the restricted product. Because there is a natural embedding $K\into\AA_K$, we know that a $K$-point of $X$ will imply an $\mathbb A_K$-point of $X$. So we can ask our question again.
\begin{ques}
	Suppose that $X(\AA_K)\ne\emp$. Then must $X(K)$ be nonempty?
\end{ques}
The answer here is still no, but it will turn out to be interesting why.

To see what's going on, suppose for a moment that we are looking with $X=\Spec A$ affine. Note $A$ is a finitely generated $K$-algebra (in fact finitely presented), so we see
\[A=\frac{K[x_1,\ldots,x_r]}{(f_1,\ldots,f_s)}.\]
Letting $N$ be the product of our denominators, we see that we can ``thicken'' $A$ into a finite scheme over $\OO_K$ as
\[A'\coloneqq\frac{\OO_K[1/N][x_1,\ldots,x_r]}{(f_1,\ldots,f_s)}.\]
Working over each of our affine patches, we see that we have produced $X'$ living in $\PP^r_{\OO_K}$. Equivalently, we could consider the scheme-theoretic image of the map $X\subseteq\PP^r_K\subseteq\PP^r_{\OO_K}$.

Now, in one direction, if there is a $K$-point $\Spec K\to\Spec A$, then we get a $K_v$-point via
\[\Spec K_v\to\Spec K\to\Spec A.\]
Taking the product, we actually get a map
\[\Spec\prod_{v\in V_K}K_v\to\Spec K\to\Spec A.\]
Anyway, the point is that we get a diagram as follows.
% https://q.uiver.app/?q=WzAsMyxbMCwxLCJYKEspIl0sWzEsMSwiXFxkaXNwbGF5c3R5bGVcXHByb2Rfe3ZcXGluIFZfS31YKEtfdikiXSxbMSwwLCJ7XFxkaXNwbGF5c3R5bGVcXHByb2Rfe3ZcXGluIFZfS319J1goS192KSJdLFswLDJdLFsyLDFdLFswLDFdXQ==&macro_url=https%3A%2F%2Fraw.githubusercontent.com%2FdFoiler%2Fnotes%2Fmaster%2Fnir.tex
\[\begin{tikzcd}
	& {{\displaystyle\prod_{v\in V_K}}'X(K_v)} \\
	{X(K)} & {\displaystyle\prod_{v\in V_K}X(K_v)}
	\arrow[from=2-1, to=1-2]
	\arrow[from=1-2, to=2-2]
	\arrow[from=2-1, to=2-2]
\end{tikzcd}\]
Indeed, given any particular $K$-point of $X(K)$, there are only finitely many coordinates, and each of these finitely many coordinates has only finitely many primes dividing the denominator, so we are actually producing a point in the restricted product. We now undo some of our products. For example, we see
\[X\left(\prod_{v\in V_K}K_v\right)=\prod_{v\in V_K}X(K_v)\]
by checking affine-locally, where we are asserting that a map $\prod_{v\in V_K}K_v\to A$ can be built be a product of maps $K_v\to A$ over each $v\in V_K$, which is simply true. Furthermore, we claim
\[{\displaystyle\prod_{v\in V_K}}'X(K_v)=X(\AA_K).\]
Again working affine-locally, we compute
\[X(K)=\op{Hom}_{\OO_K}(A',K)\to\op{Hom}_{\OO_K}\left(A',{\prod_{v\in V_K}}K_v\right)=\prod_{v\in V_K}\op{Hom}_{\OO_K}(A',K_v).\]
But we cannot introduce too many denominators as discussed before, so we see that everything does factor through the restricted product. The point is that if we are hunting for rational points, we should do more than look for $K_v$-points but to make sure that they cohere to an $\mathbb A_K$-point.

\subsection{The Brauer Invariant}
Let's be a little loose for a little in order to give a feeling for what we're going to do. Earlier in the course, we took a field $L$ and defined the Brauer group $\op{Br}L$ as $H^2\left(\op{Gal}(L^{\mathrm{sep}}/L),L^{\mathrm{sep}\times}\right)$. More generally, we can take a scheme $X$ and define
\[\op{Br}X\coloneqq H^2_{\text{\'et}}(X,\mathbb G_m).\]
Now, given a point $y\in X(\AA_K)$, we produce a map
\[\op{Br}X\stackrel{y^*}\to\op{Br}\AA_K=\bigoplus_{v\in V_K}H^2\left(\op{Gal}(K^{\mathrm{sep}}/K),K^{\mathrm{sep}\times}\right)\stackrel\sum\to\QQ/\ZZ,\]
where the last map is by summing coordinate-wise. We now recall from class field theory that
\[0\to\op{Br}K\to\bigoplus_{v\in V_K}\op{Br}K_v\stackrel\sum\to\QQ/\ZZ\to0\]
is a short exact sequence---certainly $\op{Br}K$ lives in the kernel of $\sum$ by the product formula. Thus, given $\mc A\in\op{Br}X$, we can compute
\[\op{Br}_{\mc A}(y)\coloneqq\sum_{v\in V_K}y^*\mc A\]
as an element of $\QQ/\ZZ$, and we know this must be nonzero if we are producing a $K$-point. Indeed, if we start with a point $x\in X(K)$, we put everything together to produce the following diagram.
% https://q.uiver.app/?q=WzAsNixbMSwxLCJcXG9we0JyfUsiXSxbMiwxLCJcXG9we0JyfVxcQUFfSyJdLFszLDEsIlxcUVEvXFxaWiJdLFs0LDEsIjAiXSxbMCwxLCIwIl0sWzEsMCwiXFxvcHtCcn1YIl0sWzUsMCwieF4qIiwyXSxbNSwxLCJ5XioiXSxbNCwwXSxbMCwxXSxbMSwyXSxbMiwzXV0=&macro_url=https%3A%2F%2Fraw.githubusercontent.com%2FdFoiler%2Fnotes%2Fmaster%2Fnir.tex
\[\begin{tikzcd}
	& {\op{Br}X} \\
	0 & {\op{Br}K} & {\op{Br}\AA_K} & {\QQ/\ZZ} & 0
	\arrow["{x^*}"', from=1-2, to=2-2]
	\arrow["{y^*}", from=1-2, to=2-3]
	\arrow[from=2-1, to=2-2]
	\arrow[from=2-2, to=2-3]
	\arrow[from=2-3, to=2-4]
	\arrow[from=2-4, to=2-5]
\end{tikzcd}\]
The point now is that we can check if $y\in X(\AA_K)$ might come from a rational point by checking $\op{Br}_\mc A(y)$ to be zero for every $\mc A\in\op{Br}K$. This turns out to be not be a perfect definition, but it at least gives us some check.

\subsection{Brauer Groups}
To make sense of the previous subsection, we need an actual definition of $\op{Br}X$. Quickly, let's review how we currently think of $\op{Br}K$ for two fields $K$.
\begin{itemize}
	\item We have $H^2\left(\op{Gal}(K^{\mathrm{sep}}/K),K^{\mathrm{sep}\times}\right)$.
	\item We can define the set of finite-dimensional central simple $K$-algebras. (It turns out to be equivalent to ask for $A\otimes_KK^{\mathrm{sep}}=M_n(K^{\mathrm{sep}})$.) Then this has a group operation by $\otimes_K$, but then we mod out by the equivalence $A\sim A'$ if and only if $M_n(A)\cong M_{n'}(A')$ for some $n$ and $n'$.
\end{itemize}
We also discussed a little how to between these definitions. The key point was the Skolem--Noether theorem that
\[\op{Aut}_{K^{\mathrm{sep}}}(M_n(K^{\mathrm{sep}}))=\op{PGL}_n(K^{\mathrm{sep}}).\]
Thus, for example, we can take a central simple algebra $A$, fix an isomorphism $\varphi\colon A\otimes_KK^{\mathrm{sep}}\cong M_n(K^{\mathrm{sep}})$, and then we note that an automorphism $\sigma\in\op{Gal}(K^{\mathrm{sep}}/K)$ yields an element of $\op{PGL}_n(K^{\mathrm{sep}})$ by following around the following square.
% https://q.uiver.app/?q=WzAsNCxbMCwwLCJBXFxvdGltZXNfS0tee1xcbWF0aHJte3NlcH19Il0sWzAsMSwiQVxcb3RpbWVzX0tLXntcXG1hdGhybXtzZXB9fSJdLFsxLDAsIk1fbihLXntcXG1hdGhybXtzZXB9fSkiXSxbMSwxLCJNX24oS157XFxtYXRocm17c2VwfX0pIl0sWzAsMSwiMVxcb3RpbWVzXFxzaWdtYSIsMl0sWzIsMywiTV9uKFxcc2lnbWEpIl0sWzAsMiwiXFx2YXJwaGkiXSxbMSwzLCJcXHZhcnBoaSJdXQ==&macro_url=https%3A%2F%2Fraw.githubusercontent.com%2FdFoiler%2Fnotes%2Fmaster%2Fnir.tex
\[\begin{tikzcd}
	{A\otimes_KK^{\mathrm{sep}}} & {M_n(K^{\mathrm{sep}})} \\
	{A\otimes_KK^{\mathrm{sep}}} & {M_n(K^{\mathrm{sep}})}
	\arrow["1\otimes\sigma"', from=1-1, to=2-1]
	\arrow["{M_n(\sigma)}", from=1-2, to=2-2]
	\arrow["\varphi", from=1-1, to=1-2]
	\arrow["\varphi", from=2-1, to=2-2]
\end{tikzcd}\]
This produces a $1$-cocycle, and then we use a map
\[H^1(\op{Gal}(K^{\mathrm{sep}}/K),\op{PGL}_n(K^{\mathrm{sep}}))\to H^2(\op{Gal}(K^{\mathrm{sep}}/K),K^{\mathrm{sep}\times}),\]
which is what we wanted.

\end{document}