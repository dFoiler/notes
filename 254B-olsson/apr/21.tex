% !TEX root = ../notes.tex

\documentclass[../notes.tex]{subfiles}

\begin{document}

\section{April 21}

We continue discussing the \'etale topology.
\begin{remark}
	Fix a scheme $X$. Then $\mathrm{\acute Et}(X)$ has a final object given by $\id_X\colon X\to X$. However, one should not worry about this too much: the site is not so important but rather the category of sheaves on the site. For example, one could replace $\mathrm{\acute Et}(X)$ with the site $\mathrm{\acute Et}^{\mathrm{aff}}(X)$ of affine \'etale $X$-schemes, but the corresponding categories of sheaves are the same.
\end{remark}
\begin{remark}
	A category equivalent to sheaves on a site is called a ``topos.'' But then there is a natural way to turn into any such topos into a site.
\end{remark}

\subsection{Some Sheaves on the \'Etale Site}
Let's give some examples of sheaves on the \'etale site.
\begin{example}[structure sheaf]
	There is a sheaf $\OO_{X_{\mathrm{et}}}$ sending the object $U\to X$ to $\Gamma(U,\OO_U)$. Given a morphism $f\colon V\to U$ of \'etale open sets, we produce the pullback map $f^*\colon\Gamma(U,\OO_U)\to\Gamma(V,\OO_V)$. This forms a presheaf, which one can check directly, and a descent argument shows that this a sheaf.
\end{example}
Because we have a notion of structure sheaf $\OO_{X_{\mathrm{et}}}$, we have $\OO_{X_{\mathrm{et}}}$-modules and quasicoherent sheaves and so on.
\begin{example} \label{ex:qcoh-et}
	Fix a quasicoherent sheaf $\mc M$ on $X_{\mathrm{zar}}$. Then we can produce an ``upgraded'' quasicoherent sheaf $\mc M_{\mathrm{et}}$ by sending the \'etale open set $g\colon U\to X$ to $\Gamma\left(U,g^*\mc M\right)$. Again, a descent argument shows that this is a sheaf.
\end{example}
\begin{remark}
	One can show that, for a suitable definition of quasicoherent over the \'etale site (namely, being a module affine-locally for the \'etale notion of ``locally''), \Cref{ex:qcoh-et} produces an equivalence of categories
	\[\mathrm{QCoh}(X_{\mathrm{zar}})\to\mathrm{QCoh}(X_{\mathrm{et}}).\]
\end{remark}
So we have not lost any of our good theory of quasicoherent sheaves. However, we have gained some good theory of sheaves like $\ZZ/m\ZZ$.
\begin{example}
	Let's think about $H^2(\op{Gal}(\CC/\RR),\mu_2)$. Let $i\in\op{Gal}(\CC/\RR)$ be complex conjugation, so we note that the quaternions $\mathbb H$ are supposed to produce a class in there; this is essentially an \'etale sheaf of algebras over $\Spec\RR$. As such, fixing an isomorphism $\sigma\colon\mathbb H\otimes_\RR\CC\to M_2(\CC)$, one can go around the square
	% https://q.uiver.app/?q=WzAsNCxbMCwwLCJcXG1hdGhiYiBIXFxvdGltZXNfXFxSUlxcQ0MiXSxbMSwwLCJNXzIoXFxDQykiXSxbMSwxLCJNXzIoXFxDQykiXSxbMCwxLCJcXG1hdGhiYiBIXFxvdGltZXNfXFxSUlxcQ0MiXSxbMCwzLCIxXFxvdGltZXMgaSIsMl0sWzEsMiwiTV8yKGkpIl0sWzAsMSwiXFxzaWdtYSJdLFszLDIsIlxcc2lnbWEiXV0=&macro_url=https%3A%2F%2Fraw.githubusercontent.com%2FdFoiler%2Fnotes%2Fmaster%2Fnir.tex
	\[\begin{tikzcd}
		{\mathbb H\otimes_\RR\CC} & {M_2(\CC)} \\
		{\mathbb H\otimes_\RR\CC} & {M_2(\CC)}
		\arrow["{1\otimes i}"', from=1-1, to=2-1]
		\arrow["{M_2(i)}", from=1-2, to=2-2]
		\arrow["\sigma", from=1-1, to=1-2]
		\arrow["\sigma", from=2-1, to=2-2]
	\end{tikzcd}\]
	to produce a cocycle in $H^1(\op{Gal}(\CC/\RR),\op{PGL}_2(\CC))$, from which a boundary map along
	\[0\to\mu_2\to\op{SL}_2(\CC)\to\op{PGL}_2(\CC)\to0\]
	produces a class in $H^2(\op{Gal}(\CC/\RR),\mu_2)$. However, realize that we are basically doing \'etale cohomology here over $\Spec\RR$.
\end{example}
We take a moment to acknowledge that one can now upgrade Hilbert's theorem 90 to the following.
\begin{theorem}
	Fix a scheme $X$. For any quasicoherent sheaf $\mc M$, there is a natural isomorphism
	\[H^\bullet(X_{\mathrm{zar}},\mc M_{\mathrm{zar}})\to H^\bullet(X_{\mathrm{et}},\mc M_{\mathrm{et}}).\]
\end{theorem}
Roughly speaking, $H^1(\op{Gal}(K^{\mathrm{sep}}/K),\op{GL}_n(K^{\mathrm{sep}}))$ is trying to classify vector bundles over $\Spec K$, which are at a point and should be trivial. We upgraded this to the previous result to talk about quasicoherent sheaves in more generality.
\begin{example}
	There is an \'etale sheaf $\op{GL}_n$ on $X$ by sending the affine open $\Spec A\to X$ to $\op{GL}_n(A)$. Similarly, one has an \'etale sheaf $\op{PGL}_n$ by taking the sheafification of the presheaf given by sending $\Spec A\to X$ to $\op{GL}_n(A)/\op{GL}_1(A)$.
\end{example}
The point is that the \'etale topology allows us to make sense of short exact sequences like
\[0\to\mu_n\to{\op{GL}_1}\to{\op{GL}_1}\to0\]
which is the scheme-theoretic version of the short exact sequence
\[0\to\mu_n(K^{\mathrm{sep}})\to K^{\mathrm{sep}\times}\to K^{\mathrm{sep}\times}\to0\]
in Galois cohomology.

\subsection{Azumaya Algebras}
To continue our story, we would like to generalize our notion of the Brauer group.
\begin{definition}[Azumaya algebra]
	Fix a scheme $X$. An \textit{Azumaya algebra} is a sheaf of $\OO_X$-algebras which is \'etale locally isomorphic to $M_n(\OO_X)$ for some $n$.
\end{definition}
Quickly, recall the following result.
\begin{theorem}[Skolem--Noether]
	Fix a ring $R$. Then $\op{Aut}_R(M_n(R))=\op{PGL}_n(R)$.
\end{theorem}
\begin{corollary}
	Fix an Azumaya algebra $\mc A$ on $X$. Then the \'etale sheaf sending $U$ to the set of isomorphisms $\mc A|_U\to M_n(\OO_U)$ has a simply transitive action by $\op{PGL}_n$.
\end{corollary}
This is referred to as a ``torsor,'' and we will be able to classify them by calculating $H^1(X_{\mathrm{et}},\op{PGL}_n)$.
\begin{definition}[torsor]
	Fix a sheaf $\mc G$ of groups on $X$. Then a $\mc G$-torsor is a sheaf of sets $\mc S$ with $\mc G$-action such that each \'etale open set $U$ has a covering $\{U_i\}$ such that $\mc S(U_i)\ne\emp$ for each $i$, and further, our action is simply transitive. In other words, being simply transitive means that we have an isomorphism $\mc G\times\mc S\to\mc S\times\mc S$ by $(g,s)\mapsto(gs,s)$ is an isomorphism of sheaves of sets.
\end{definition}
Here is how this relates to our corollary.
\begin{proposition}
	Fix an \'etale sheaf $\mc A$ of abelian groups. Then $H^1(X_{\mathrm{et}},\mc A)$ is naturally isomorphic to the set of $\mc A$-torsors up to isomorphism.
\end{proposition}
\begin{proof}
	Working from the definition, we recall that computing cohomology of $\mc A$ amounts to constructing an injective resolution
	\[0\to\mc A\to\mc I^0\to\mc I^1\to\mc I^2\to\cdots\]
	and then computing
	\[H^1(X_{\mathrm{et}},\mc A)=\frac{\ker(\Gamma(\mc I^1)\to\Gamma(\mc I^2))}{\im(\Gamma(\mc I^0)\to\Gamma(\mc I^1))}.\]
	Now, given an element of the kernel $s\in\ker(\Gamma(\mc I^1)\to\Gamma(\mc I^2))$, we produce the $\mc A$-torsor
	\[\mc S\colon U\mapsto\left\{\alpha\in\mc I^0(U):d^0\alpha=s|_U\right\}.\]
	Then one can check that $s$ being adjusted by an element of $\im(\Gamma(\mc I^0)\to\Gamma(\mc I^1))$ corresponds to our $\mc A(U)$-action, which turns $\mc S$ into an $\mathcal A$-torsor. Lastly, one should show that every torsor arises in this way, which is hard.
\end{proof}

\end{document}