% !TEX root = ../notes.tex

\documentclass[../notes.tex]{subfiles}

\begin{document}

\section{April 10}

Today we finish up with the Mordell--Weil theorem.

\subsection{Finishing Mordell--Weil}
It remains to show \Cref{prop:get-magic-g}.
\begin{proof}[Proof of \Cref{prop:get-magic-g}]
	For brevity, set $\mc M\coloneqq\OO_E(e)$. We have the following steps, though we will skip some justifications.
	\begin{enumerate}
		\item By our computation of $\rho$ previously, the map $E\times E\to\PP^2$ is given by the line bundle $\mc M^{\otimes2}\boxtimes\mc M^{\otimes2}$. Explicitly, the map $\rho\colon\PP^1\times\PP^1\to\PP^2$ is given by the line bundle $\OO(1)\boxtimes\OO(1)$ on $\PP^1\times\PP^1$, and noting the canonical isomorphism
		\[\Gamma(\PP^1\times\PP^1,\OO(1)\boxtimes\OO(1))\cong\Gamma(\PP^1,\OO(1))\otimes\Gamma(\PP^1,\OO(1)),\]
		the map $\rho$ is given by the sections $v_1\otimes v_2,u_1\otimes v_2+v_1\otimes u_2,u_1\otimes u_2$.
		\item There is an automorphism $\gamma\colon\PP^1\times\PP^1\to\PP^1\times\PP^1$ by switching the factors, which has order $2$ and thus induces an action of $\ZZ/2\ZZ$ on $\PP^1\times\PP^2$. Notably, $\gamma$ also fixes the line bundle $\OO(1)\boxtimes\OO(1)$ and fixes the sections used to define $\rho$, so we can track that
		% https://q.uiver.app/?q=WzAsMyxbMCwwLCJcXFBQXjFcXHRpbWVzXFxQUF4xIl0sWzIsMCwiXFxQUF4xXFx0aW1lc1xcUFBeMSJdLFsxLDEsIlxcUFBeMiJdLFswLDEsIlxcZ2FtbWEiXSxbMCwyLCJcXHJobyIsMl0sWzEsMiwiXFxyaG8iXV0=&macro_url=https%3A%2F%2Fraw.githubusercontent.com%2FdFoiler%2Fnotes%2Fmaster%2Fnir.tex
		\[\begin{tikzcd}
			{\PP^1\times\PP^1} && {\PP^1\times\PP^1} \\
			& {\PP^2}
			\arrow["\gamma", from=1-1, to=1-3]
			\arrow["\rho"', from=1-1, to=2-2]
			\arrow["\rho", from=1-3, to=2-2]
		\end{tikzcd}\]
		commutes. Roughly speaking, we expect $\gamma$ as an automorphism to detect invariants under switching the factors.
		\item In fact, for any line bundle $\mc L$ on $\PP^2$, we claim
		\[\Gamma(\PP^2,\mc L)\to\Gamma(\PP^1\times\PP^1,\rho^*\mc L)^{\gamma=1}\]
		is an isomorphism, where the superscript denotes $\gamma$-invariants.

		Well, there is a classification of line bundles on $\PP^2$, so we can write $\mc L=\OO(d)$ for some $d$. If $d<0$, then neither side will have any global sections (one can compute this directly or note that the inverse of an ample line bundle has no global sections); if $d=0$, then we are just looking at structure sheaves everywhere, and both will be $k$. It remains to check $d>0$.

		Note that $\rho$ being surjective implies that our map on sections is injective because it is essentially given by some restriction of $\rho^\sharp$. As such, it remains to compute some dimensions. Note $\Gamma(\PP^2,\OO(d))$ has dimension $\binom{d+2}2=\frac{(d+2)(d+1)}2$. On the other side, let $\alpha_0,\ldots,\alpha_d$ denote a basis for $\Gamma(\PP^1,\OO(1))$, and we note that we can give a basis of $\Gamma(\PP^1\times\PP^1,\OO(1)\boxtimes\OO(1))$ by the elements
		\[\begin{cases}
			\alpha_i\otimes\alpha_i & 0\le i\le d, \\
			\alpha_i\otimes\alpha_j+\alpha_j\otimes\alpha_i & 0\le i<j\le d, \\
			\alpha_i\otimes\alpha_j-\alpha_j\otimes\alpha_i & 0\le i<j\le d.
		\end{cases}\]
		The first two cases provide basis elements with $\gamma=1$, and the last case has $\gamma=-1$, so our subspace is spanned by the first two cases. Thus, our dimension is
		\[d+1+\binom{d+1}2=d+1+\frac{(d+1)d}2=\frac{2d+2+d^2+2}2=\frac{d^2+3d+2}2=\frac{(d+1)(d+2)}2,\]
		which indeed matches.

		\item Recall that we have the hyperelliptic projection $x\colon E\to\PP^1$ given by the line bundle $\mc M^{\otimes2}$ and the sections $1,x$. We also have the hyperelliptic involution $\iota\colon E\to E$, and we expect $x$ to project down invariants of $\iota$. Explicitly, for any line bundle $\mc L$ on $\PP^1$, we claim that
		\[\Gamma(\PP^1,\mc L)\to\Gamma(E,x^*\mc L)^{\iota=1}\]
		is an isomorphism. This proof is similar to the above.
		\item Now, the composite $\sigma=\rho\circ(x,x)$ is going to accumulate a number of invariants. Explicitly, we have a subgroup $A\subseteq\op{Aut}(E\times E)$ generated by (the lift of) $\tau$ given by switching the factors and $1\times\iota$ and $\iota\times1$. Then we claim that for any line bundle $\mc L$ on $\PP^2$, we have that
		\[\Gamma(\PP^2,\mc L)\to\Gamma(E\times E,\sigma^*\mc L)^A\]
		is an isomorphism. This comes from combining the previous two steps.
		\item It remains to discuss $G$ intelligently. Then we claim there exists an automorphism $h\colon A\to A$ such that the following commutes for any $g\in A$.
		% https://q.uiver.app/?q=WzAsNCxbMCwwLCJFXFx0aW1lcyBFIl0sWzEsMCwiRVxcdGltZXMgRSJdLFsxLDEsIkVcXHRpbWVzIEUiXSxbMCwxLCJFXFx0aW1lcyBFIl0sWzAsMSwiRyJdLFszLDIsIkciXSxbMCwzLCJnIiwyXSxbMSwyLCJoKGcpIl1d&macro_url=https%3A%2F%2Fraw.githubusercontent.com%2FdFoiler%2Fnotes%2Fmaster%2Fnir.tex
		\[\begin{tikzcd}
			{E\times E} & {E\times E} \\
			{E\times E} & {E\times E}
			\arrow["G", from=1-1, to=1-2]
			\arrow["G", from=2-1, to=2-2]
			\arrow["g"', from=1-1, to=2-1]
			\arrow["{h(g)}", from=1-2, to=2-2]
		\end{tikzcd}\]
		Well, we can compute this explicitly. For $\iota\times1$, we track around the following diagram.
		% https://q.uiver.app/?q=WzAsOCxbMCwwLCJFXFx0aW1lcyBFIl0sWzEsMCwiRVxcdGltZXMgRSJdLFsxLDEsIkVcXHRpbWVzIEUiXSxbMCwxLCJFXFx0aW1lcyBFIl0sWzIsMCwiKFAsUSkiXSxbMiwxLCIoLVAsUSkiXSxbMywwLCIoUCtRLFAtUSkiXSxbMywxLCIoLVArUSwtUC1RKSJdLFswLDEsIkciXSxbMywyLCJHIl0sWzAsMywiZyIsMl0sWzEsMiwiaChnKSJdLFs1LDcsIiIsMCx7InN0eWxlIjp7InRhaWwiOnsibmFtZSI6Im1hcHMgdG8ifX19XSxbNCw1LCIiLDAseyJzdHlsZSI6eyJ0YWlsIjp7Im5hbWUiOiJtYXBzIHRvIn19fV0sWzQsNiwiIiwyLHsic3R5bGUiOnsidGFpbCI6eyJuYW1lIjoibWFwcyB0byJ9fX1dLFs2LDcsIigxLFxcaW90YSlcXGNpcmNcXHRhdVxcY2lyYygxLFxcaW90YSkiLDAseyJzdHlsZSI6eyJ0YWlsIjp7Im5hbWUiOiJtYXBzIHRvIn0sImJvZHkiOnsibmFtZSI6ImRhc2hlZCJ9fX1dXQ==&macro_url=https%3A%2F%2Fraw.githubusercontent.com%2FdFoiler%2Fnotes%2Fmaster%2Fnir.tex
		\[\begin{tikzcd}
			{E\times E} & {E\times E} & {(P,Q)} & {(P+Q,P-Q)} \\
			{E\times E} & {E\times E} & {(-P,Q)} & {(-P+Q,-P-Q)}
			\arrow["G", from=1-1, to=1-2]
			\arrow["G", from=2-1, to=2-2]
			\arrow["g"', from=1-1, to=2-1]
			\arrow["{h(g)}", from=1-2, to=2-2]
			\arrow[maps to, from=2-3, to=2-4]
			\arrow[maps to, from=1-3, to=2-3]
			\arrow[maps to, from=1-3, to=1-4]
			\arrow["{(1,\iota)\circ\tau\circ(1,\iota)}", dashed, maps to, from=1-4, to=2-4]
		\end{tikzcd}\]
		Similar works for $1\times\iota$, and for $\tau$ we have the following.
		% https://q.uiver.app/?q=WzAsOCxbMCwwLCJFXFx0aW1lcyBFIl0sWzEsMCwiRVxcdGltZXMgRSJdLFsxLDEsIkVcXHRpbWVzIEUiXSxbMCwxLCJFXFx0aW1lcyBFIl0sWzIsMCwiKFAsUSkiXSxbMiwxLCIoUSxQKSJdLFszLDAsIihQK1EsUC1RKSJdLFszLDEsIihQK1EsUS1QKSJdLFswLDEsIkciXSxbMywyLCJHIl0sWzAsMywiZyIsMl0sWzEsMiwiaChnKSJdLFs1LDcsIiIsMCx7InN0eWxlIjp7InRhaWwiOnsibmFtZSI6Im1hcHMgdG8ifX19XSxbNCw1LCIiLDAseyJzdHlsZSI6eyJ0YWlsIjp7Im5hbWUiOiJtYXBzIHRvIn19fV0sWzQsNiwiIiwyLHsic3R5bGUiOnsidGFpbCI6eyJuYW1lIjoibWFwcyB0byJ9fX1dLFs2LDcsIigxLFxcaW90YSkiLDAseyJzdHlsZSI6eyJ0YWlsIjp7Im5hbWUiOiJtYXBzIHRvIn0sImJvZHkiOnsibmFtZSI6ImRhc2hlZCJ9fX1dXQ==&macro_url=https%3A%2F%2Fraw.githubusercontent.com%2FdFoiler%2Fnotes%2Fmaster%2Fnir.tex
		\[\begin{tikzcd}
			{E\times E} & {E\times E} & {(P,Q)} & {(P+Q,P-Q)} \\
			{E\times E} & {E\times E} & {(Q,P)} & {(P+Q,Q-P)}
			\arrow["G", from=1-1, to=1-2]
			\arrow["G", from=2-1, to=2-2]
			\arrow["g"', from=1-1, to=2-1]
			\arrow["{h(g)}", from=1-2, to=2-2]
			\arrow[maps to, from=2-3, to=2-4]
			\arrow[maps to, from=1-3, to=2-3]
			\arrow[maps to, from=1-3, to=1-4]
			\arrow["{(1,\iota)}", dashed, maps to, from=1-4, to=2-4]
		\end{tikzcd}\]
		The point is that we can construct this automorphism by hand.

		\item Next up, we claim that $G^*\left(\mc M\boxtimes\mc M\right)\cong\mc M^{\otimes2}\boxtimes\mc M^{\otimes2}$. Define $\Delta_E^{\pm}$ to be the kernel of $1\times\pm1$. Explicitly, $\Delta_E^-=\{(P,-P):P\in E\}$, and $\Delta_E^+=\{(P,P):P\in E\}$. As such, by computing on divisors, we see that
		\[G^*(\mc M\boxtimes\mc M)=\OO_{E\times E}(\Delta_E^++\Delta_E^-).\]
		Let's examine what happens on points. For $x\in E$, we see
		\[G^*(\mc M\boxtimes\mc M)|_{\{x\}\times E}=\OO_E([-x]+[x])=\OO_E(2e)=\mc M^{\otimes2}.\]
		The point is that $\mc G^*(\mc M\boxtimes\mc M)\otimes p_2^*\mc M^{-2}$ is locally trivial, which implies that it is the pullback of a line bundle along $p_1^*$. To see what this line bundle is, we can switch factors, and the claim follows.

		\item Combining the above steps, we calculate
		\begin{align*}
			\Gamma(\PP^2,\OO(1)) &\cong \Gamma(E\times E,\mc M^{\otimes2}\boxtimes\mc M^{\otimes2})^A \\
			&\stackrel{G^*}\to \Gamma(E\times E,\mc M\boxtimes\mc M)^A \\
			&\cong \Gamma(\PP^2,\OO(2)).
		\end{align*}
		Notably, the $G^*$ step is noting that we take invariants to invariants, provided we acknowledge that the invariants are preserved because $G^*$ commutes with $A$ through $h$. Thus, we can pull back our sections, which does indeed give us enough sections of $\OO_{\PP^2}(2)$, so we get our map $g\colon\PP^2\to\PP^2$ of degree $2$ making everything commute. It remains to check that these sections are base-point-free, but they were base-point-free over $E\times E$, and the map $E\times E\to\PP^2$ is surjective, so they will continue to be base-point-free on $\PP^2$.
		\qedhere
	\end{enumerate}
\end{proof}

\end{document}