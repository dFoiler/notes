% !TEX root = ../notes.tex

\documentclass[../notes.tex]{subfiles}

\begin{document}

\section{January 20}

Last time we were in the middle of showing \Cref{lem:reps-0-to-norms}, so we continue where we left off.

\subsection{Hilbert's Theorem 90}
Here is the desired lemma.
\begin{lemma} \label{lem:begins-cohom}
	Fix a field $K$ not of characteristic not $2$. Find $a,b\in K$ such that $[K(\sqrt a,\sqrt b):K]=4$. Then $c\in K^\times$ is in the image of the norm map ${\op N}\colon K(\sqrt a,\sqrt b)\to K(\sqrt{ab})$ if and only if there exist $x\in K(\sqrt a)$ and $y\in K(\sqrt b)$ such that
	\[c=\op N^{K(\sqrt a)}_K(x)\cdot\op N^{K(\sqrt b)}_K(y).\]
\end{lemma}
\begin{proof}[Proof of backward direction]
	Observe that we are still dealing with the tower of fields in \eqref{eq:quad-tower}. Now, note
	\[\op{Gal}(K(\sqrt a,\sqrt b)/K)=\{1,\sigma,\tau,\sigma\tau\},\]
	where $\sigma\colon\sqrt a\mapsto\sqrt a$ and $\sigma\colon\sqrt b\mapsto-\sqrt b$ and $\tau\colon\sqrt a\mapsto-\sqrt a$ and $\tau\colon\sqrt b\mapsto\sqrt b$. (Notably, $\op{Gal}(K(\sqrt a)/K)=\langle\tau\rangle$ and $\op{Gal}(K(\sqrt b)/K)=\langle\sigma\rangle$.) We now want the following to be equivalent.
	\begin{listalph}
		\item There are $x,y\in K(\sqrt a,\sqrt b)$ such that $(\sigma-1)x=(\tau-1)y=0$ and $xy\cdot\sigma\tau(xy)=c$.

		Indeed, $(\sigma-1)x=0$ means $x\in K(\sqrt a)$, and similarly for $y\in K(\sqrt b)$, so this statement is equivalent to $c=\op N^{K(\sqrt a)}_K(x)\cdot\op N^{K(\sqrt b)}_K(y)$ for $x\in K(\sqrt a)$ and $y\in K(\sqrt b)$.

		\item There is $z\in K(\sqrt a,\sqrt b)$ such that $z\cdot\sigma\tau(z)=c$.

		Indeed, note $\sigma\tau(\sqrt{ab})=\sqrt{ab}$, so $\op{Gal}(K(\sqrt{ab})/K)=\{1,\sigma\tau\}$. Thus, this is equivalent to $c$ being in the image of the norm map ${\op N}\colon K(\sqrt a,\sqrt b)\to K(\sqrt{ab})$.
	\end{listalph}
	By setting $z\coloneqq xy$, we thus see that (a) implies (b), so the hard part is showing the reverse direction.

	Showing (b) implies (a) is somewhat harder. Assume (b), and observe that $z\cdot\sigma(z)=\op N^{K(\sqrt a,\sqrt b)}_{K(\sqrt a)}(z)$ is fixed by $\sigma$ and hence in $K(\sqrt a)$. Further, we may compute
	\[\op N^{K(\sqrt a)}_{K}(z\cdot\sigma(z))=\op N^{K(\sqrt a,\sqrt b)}_{K}(z)=z\cdot\sigma(z)\cdot\tau(z)\cdot\sigma\tau(z)\]
	is an element of $K$. Now, we see $z\cdot\sigma\tau(z)=c$ is an element of $K$, so $\sigma(z)\cdot\tau(z)\in K$ as well. Thus, hitting this with $\sigma$, we see
	\[\sigma(z)\cdot\tau(z)=\sigma(\sigma(z)\cdot\tau(z))=z\cdot\sigma\tau(z)=c\]
	also, so we conclude $\sigma(z)\cdot\tau(z)=c$, so in fact $z\cdot\sigma(z)/c\in K(\sqrt a)$ is an element of norm $1$. We now appeal to Hilbert's theorem 90.
	\begin{theorem}[Hilbert 90] \label{thm:h-90}
		Fix a cyclic extension of fields $L/K$ with Galois group $\op{Gal}(L/K)=\langle\sigma\rangle$. If $t\in L$ has $\op N^L_K(t)=1$, then there exists $\alpha\in L$ such that $t=\sigma(\alpha)/\alpha$.
	\end{theorem}
	\begin{remark}
		Of course, any element of the form $\sigma(\alpha)/\alpha$ will have norm $1$ by some telescoping.
	\end{remark}
	We will show \Cref{thm:h-90} via group cohomology later, but we will use it freely for now. Pick up the promised $x\in K(\sqrt a)$ such that
	\[\frac{z\cdot\sigma(z)}c=\frac{\tau(x)}x.\]
	Further, set $y\coloneqq\sigma\tau(z)/x$, and we compute
	\[\tau(y)=\frac{\sigma(z)}{\tau(x)}\stackrel*=\frac{c}{z\cdot x}=\frac{\sigma\tau(z)}{x}=y.\]
	Note we have used the definition of $x$ at $\stackrel*=$. Thus, $y\in K(\sqrt b)$, so to finish the proof, we check
	\[xy\cdot\sigma\tau(xy)=\sigma\tau(z)\cdot(\sigma\tau)^2(z)=z\cdot\sigma\tau(z)=c,\]
	so we are done.
\end{proof}
Roughly speaking, the hard direction of the above proof uses \Cref{thm:h-90} to construct our $\alpha$ and $\beta$, and then everything else is more or less a computation.

\subsection{Hasse--Minkowski}
We are now ready to prove \Cref{thm:hasse-mink}, modulo some more appeals to group cohomology. Here is the statement.
\hmthm*
\begin{proof}
	By adjusting the basis of $V$ as in \Cref{rem:get-diagonal-qform}, we may assume that $Q=a_1x_1^2+\cdots+a_nx_n^2$. Additionally, if any of the variables are $0$, say $a_1=0$, then $(1,0,0,\ldots,0)$ is a nontrivial zero for both $V$ and each $V\otimes_KK_v$, so there is nothing to say. As such, we normalize $Q$ so that $a_1=1$.

	We now induct on $n$. Here are our small cases. If $n=1$, then there are never any zeroes at all by \Cref{lem:reps-0-to-norms}. For $n=2$, we are studying $Q=x_1^2+a_2x_2^2$, so we are done by \Cref{lem:reps-0-to-norms} by appealing to the following result, which we will prove later.
	\begin{theorem}
		Fix a number field $K$. Then $\alpha\in K^\times$ is a square if and only if $\alpha$ is a square in each $K_v$ for all places $v$.
	\end{theorem}
	For $n=3$ and $n=4$, we are again done by \Cref{lem:reps-0-to-norms} upon appealing to the following result.
	\begin{theorem}[Hasse norm] \label{thm:hasse-norm}
		Fix a cyclic extension $L/K$ of number fields. Given $a\in K^\times$, then $a$ is in the image of the norm $L\to K$ if and only if $a$ is in a norm in $K_v$ for all places $v$.
	\end{theorem}
	Roughly speaking, \Cref{lem:reps-0-to-norms} turns statements about quadratic forms into statements about norms, so we get a local-to-global principle via \Cref{thm:hasse-norm}'s local-to-global principle.

	We are now almost ready for the inductive step. We make a few starting comments.
	\begin{itemize}
		\item A quadratic form of the form $Q_1(x_1,\ldots,x_m)-Q_2(y_1,\ldots,y_n)$ will represent $0$ if and only if there exists some $c$ represented by both $Q_1$ and $Q_2$. There isn't really anything to say here.
		\item If $Q$ represents some $c\in K^\times$, then $Q$ represents the entire equivalence class of $c$ in $K^\times/K^{\times2}$. Indeed, this is because $Q$ is a quadratic form and thus homogeneous of degree $2$.
		\item For each place $v$, we have $K_v^{\times2}$ is an open subgroup of $K_v^\times$. Indeed, for archimedean $v$, this reduces to saying $\RR_{>0}\subseteq\RR^\times$ is open, and $\CC^\times=\CC^\times$ is open.

		We can argue for nonarchimedean places $v$ explicitly, but we can give a more abstract argument via Hensel's lemma. Indeed, it suffices to provide a neighborhood of $1$ in $K_v^\times$ (because $K_v^\times$ is a topological group), so we choose
		\[U\coloneqq\left\{a:\left|1^2-a\right|_v<|2\cdot1|_v^2\right\}.\]
		Notably, for each $a\in U$, we see $1$ witnesses the ability to solving $x^2-a=0$ in $K_v$ by Hensel's lemma.
	\end{itemize}
	We now proceed with our induction. Assume $n\ge 5$. We may write
	\[Q(x_1,\ldots,x_n)=ax_1^2+bx_2^2-R(x_3,\ldots,x_n),\]
	for some quadratic form $R$ in $n-2$ variables. To continue, we give another statement which comes from the Hasse norm theorem.
	\begin{theorem}[Hasse norm] \label{thm:focus-on-finite-places}
		Fix a cyclic extension $L/K$ of number fields, and let $Q$ be a quadratic form in $n\ge3$ variables. For each $a\in K^\times$, then there is a finite set of places $S$ such that $Q$ represents $0$ in $K_v$ for each $v\notin S$.
	\end{theorem}
	\begin{proof}
		We give a proof from algebraic geometry. Take $K=\QQ$ for simplicity. For simplicity, take $Q=ax^2+by^2+cz^2$, and note $V(Q)\subseteq\PP^2_\QQ$ is a genus-$0$ curve. For all but finitely many primes $p$, we see $\nu_p(a)=\nu_p(b)=\nu_p(c)=0$, so we can base-change $V(Q)$ to $\ZZ_p$ and then $\FF_p$, where $V(Q)$ remains a genus-$0$ curve. However, a genus-$0$ curve always has a point over a finite field, and then smoothness of $V(Q)$ allows us to lift the $\FF_p$-point back to a $\ZZ_p$-point by Hensel's lemma.
	\end{proof}
	So by \Cref{thm:focus-on-finite-places}, there are finitely many places $S$ for which $R$ does not represent $0$.

	Now, suppose that $Q$ has a nontrivial $0$ in each $V\otimes_KK_v$, and we must show that $Q$ has a nontrivial $0$ in $V$. We can deal with each $v\notin S$ because $R$ represents everything by \Cref{lem:rep-zero-gives-all}. Thus, focusing on some $v\notin S$, we see $Q$ having a nontrivial zero in $V\otimes_KK_v$ implies that there is some $c_v\in K_v$ represented by both $ax_1^2+bx_2^2$, so write
	\[a\alpha_{1,v}^2+b\alpha_{2,v}^2=c_v=R(\alpha_{3,v},\ldots,\alpha_{n,v}).\]
	By approximating,\todo{} we choose $\alpha_i\in K$ arbitrarily close to each $\alpha_{i,v}$ in $K_v$ so that $c=a\alpha_1^2+b\alpha_2^2$ differs from $c_v$ only be a square in $v\in S$. This is possible because $K_v^{\times2}$ is open in $K_v^\times$. Note that $R$ still represents $c$ in each $K_v$ for $v\in S$ because $c$ is only a square away from $c_v$.

	Thus, we see that the form
	\[cY^2-R(x_3,\ldots,x_n)\]
	will represent $0$ in each $K_v$ for all $v$. But this form has $n-1$ variables, so our induction kicks in and tells usu that $cY^2-R$ represents $0$ in $K$, so $R$ represents $c$ in $K$, so $Q$ represents $0$ in $K$. This completes the proof.
\end{proof}
\begin{remark}
	Professor Olsson thinks that the last part of this argument is a little too clever.
\end{remark}

\end{document}