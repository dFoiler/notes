% !TEX root = ../notes.tex

\documentclass[../notes.tex]{subfiles}

\begin{document}

\section{March 3}

Let's get started.

\subsection{The Riemann--Roch Theorem}
We are now ready to state the Riemann--Roch theorem.
\begin{theorem}[Riemann--Roch]
	Fix a $k$-curve $X$. There exists an integer $g\in\ZZ$ such that, for each line bundle $\mc L/X$, we have
	\[h^0(X,\mc L)-h^0(X,\mc L^\lor\otimes\Omega^1_X)=\deg\mc L+1-g.\]
	Here, $h^i(X,\mc L)=\dim_kH^i(X,\mc L)$.
\end{theorem}
\begin{remark}
	One definition of our ``genus'' $g$ is $g\coloneqq h^0\left(X,\Omega^1_X\right)$.
\end{remark}
\begin{example}
	Taking $\mc L=\Omega^1_X$ gives $h^0(X,\Omega^1_X)-h^0(X,\OO_X)=\deg\Omega^1_X+1-g$, so we see $\deg\Omega^1_X=2g-2$.
\end{example}
\begin{example} \label{ex:use-rr}
	In all applications, we are going to ensure that $h^1(X,\mc L)$ vanishes. By Serre duality, we see that $h^1(X,\mc L)=h^0\left(X,L^\lor\otimes\Omega^1_X\right)=\dim_k(\mc L^\lor\otimes\Omega^1_X)(X)$, which will vanish if
	\[\deg\left(\mc L^{\lor}\otimes\Omega^1_X\right)=\deg\Omega^1_X-\deg\mc L\stackrel?<0.\]
	In other words, if $\deg\mc L>\deg\Omega^1_X=2g-2$, then $h^0(X,\mc L)=\deg\mc L+1-g$.
\end{example}
To continue our discussion, we will want to talk about complete linear systems.
\begin{proposition}
	Fix a line bundle $\mc L$ on a $k$-curve $X$. Then $(\Gamma(X,\mc L)\setminus\{0\})/k^\times$ is in natural bijection with effective divisors $D$ such that $\OO_X(D)\cong\mc L$.
\end{proposition}
\begin{proof}
	Given some $s\in\Gamma(X,\mc L)\setminus\{0\}$, we note that $s$ produces a map $s\colon\OO_X\to\mc L$, and because $s$ is nonzero, this map is injective by checking at stalks: for each $x\in X$, then we have the commutative diagram as follows.
	% https://q.uiver.app/?q=WzAsNCxbMCwwLCJcXE9PX3tYLHh9Il0sWzEsMCwiXFxtYyBMX3giXSxbMCwxLCJLKFgpIl0sWzEsMSwiXFxtYyBMX1xcZXRhIl0sWzAsMiwiIiwwLHsic3R5bGUiOnsidGFpbCI6eyJuYW1lIjoiaG9vayIsInNpZGUiOiJ0b3AifX19XSxbMiwzLCJzX1xcZXRhIl0sWzAsMSwic194Il0sWzEsMywiIiwyLHsic3R5bGUiOnsidGFpbCI6eyJuYW1lIjoiaG9vayIsInNpZGUiOiJ0b3AifX19XV0=&macro_url=https%3A%2F%2Fraw.githubusercontent.com%2FdFoiler%2Fnotes%2Fmaster%2Fnir.tex
	\[\begin{tikzcd}
		{\OO_{X,x}} & {\mc L_x} \\
		{K(X)} & {\mc L_\eta}
		\arrow[hook, from=1-1, to=2-1]
		\arrow["{s_\eta}", from=2-1, to=2-2]
		\arrow["{s_x}", from=1-1, to=1-2]
		\arrow[hook, from=1-2, to=2-2]
	\end{tikzcd}\]
	Here, $\eta$ is the generic point of $X$. Now, $s$ being nonzero implies that $s_\eta$ is nonzero in $K(X)$, so the bottom map is injective, so the top map should also be injective.

	Now, for any closed point $x\in X$, our map at stalks $s_x\colon\OO_{X,x}\to\mc L_x$ has $\mc L_x$ equal to some free module of rank $1$, so by pulling back a uniformizer makes this map multiplication by some uniformizer $\pi_x^{n_x}$. We now set
	\[\op{div}_\mc L(s)\coloneqq\sum_{x\in X}n_xx,\]
	and we note that $s|_x$ is trivial at only finitely many points, so this divisor's coefficients vanish for all but finitely many points. We now see that $s\colon\OO_X\to\mc L$ tells us that this map is identified with the inclusion
	\[\mc L\otimes\bigoplus_{x\in X}\mc I_x^{n_x}\to\mc L\]
	by just checking at stalks everywhere (indeed, on the left, we are trivial at every point), so we conclude that $\mc L\cong\OO_X(\op{div}_\mc L(s))$.

	To finish the bijection, we note that adjusting our $s$ by an element of $k^\times$ will not change $\op{div}_\mc L(s)$, and we can check that our map is both injective and surjective as such. We omit the rest of these checks.
\end{proof}
\begin{example}
	If $\deg\mc L<0$, then there are no effective divisors $D$ with $\mc L\cong\OO_X(D)$ because $\deg D\ge0$ for all effective divisors $D$. Thus, we must have $\Gamma(X,\mc L)=0$.
\end{example}
\begin{example}
	If $\deg\mc L=0$, then we see that the only effective divisor of degree $0$ is $D=0$, so we either have $\mc L\cong\OO_X$ and so $\Gamma(X,\mc L)=1$, or we have $\Gamma(X,\mc L)=0$.
\end{example}
\begin{example} \label{ex:rr-for-genus-1-deg-1}
	In the case of $g=1$, one sees that $\deg\Omega^1_X=0$ and so $\deg\mc L=1$ implies $\dim_k\mc L(X)=1$. Roughly speaking, it follows that $\mc L\cong\OO_X(p)$ for some closed point $p\in X$ with residue field $k$. Thus, there is a unique effective divisor $D$ (of degree $1$) such that $\mc L\cong\OO_X(D)$, so we see that $D=p$ for some point $p$ of residue field $k$. As such, $\mc L\cong\OO_X(p)$.
\end{example}
The following special case will be important for us.
\begin{theorem} \label{thm:deg-1-bundle-is-point}
	Fix a $k$-curve $X$ of genus $1$. Then the map $X(k)\to\op{Pic}^1(X)$ sending a point $x\in X$ to $\OO_X(x)$ is a bijection. Here, $\op{Pic}^1$ refers to the degree-$1$ line bundles.
\end{theorem}
\begin{proof}
	We show this in pieces.
	\begin{itemize}
		\item We show surjectivity. Fix a line bundle $\mc L\in\op{Pic}^1(X)$ of degree $1$. Then \Cref{ex:rr-for-genus-1-deg-1} tells us that $\mc L\cong\OO_X(x)$ for some $x\in X$ with residue field $k$. We note that having residue field $k$ is equivalent to being a $k$-point: on one hand, a $k$-point is a morphism $x\colon\Spec k\to X$ must induce the identity on $\Spec k$ with the structure morphism $\Spec k\to X\to\Spec k$, so we see that the residue field at $x$ must be $k$. And conversely, if $x$ has residue field $k$, then we immediately induce our morphism $x\colon\Spec k(x)\to X$.
		\item We show injectivity: suppose that $x,y\in X(k)$ grant $\OO_X(x)=\OO_X(y)$. Well, suppose $x\ne y$. This implies that we have an isomorphism $\OO_X\cong\OO_X(x-y)$. In particular, $1\in\Gamma(X,\OO_X)$ is mapped to some $f\in K(X)$ such that $f$ has a pole at $x$ and a zero at $y$. In particular, this gives a nonconstant map of degree $1$ given by $f\colon X\to\PP^1_k$ by taking the corresponding map $X\setminus\{y\}\to\AA^1_k$ and extending it to $\PP^1_k$. However, this requires that $f$ is an isomorphism of curves, which is a contradiction because $g(X)\ne g(\PP^1_k)$.
		\qedhere
	\end{itemize}
\end{proof}
\begin{remark}
	In fact, we see that the injectivity argument holds for any $k$-curve $X$ of nonzero genus.
\end{remark}

\subsection{Elliptic Curves}
We are now ready to define elliptic curves.
\begin{definition}[elliptic curve]
	Fix a field $k$. An \textit{elliptic curve} is a pair $(E,e)$ where $E$ is a $k$-curve of genus $1$ and $e\in E(k)$.
\end{definition}
\begin{remark} \label{rem:group-law-ec}
	Fix an elliptic curve $(E,e)$ over a field $k$. The idea here is that we have a bijection $\op{Pic}^1(E)\to\op{Pic}^0(E)$ given by $\mc L\mapsto\mc L\otimes\OO_E(-e)$, so combining with \Cref{thm:deg-1-bundle-is-point} tells us that $E(k)$ is in bijection with the abelian group $\op{Pic}^0(E)$. In particular, $E(k)$ has the structure of an abelian group with identity element given by $e$!
\end{remark}
\begin{remark}
	Even when $g(X)>1$, we note that the previous remark grants an inclusion $X(k)\into\op{Pic}^0(X)$. Now, $X(k)$ does not inherit a group law, so we are perhaps motivated to simply work with the group $\op{Pic}^0(X)$. Indeed, it turns out that there is a notion of the ``Jacobian'' which is an abelian variety with $k$-points given by $\op{Pic}^0(X)$.
\end{remark}
\begin{remark}
	When $g(X)=1$, even with no $k$-point in $X(k)$, then there is some scheme-theoretic isomorphism $X\cong\op{Pic}^1(X)$, where now we see $\op{Pic}^1(X)$ has some action by $\op{Pic}^0(X)$. We will return to this later in the course.
\end{remark}
The group law on \Cref{rem:group-law-ec} can be made explicit via the ``chord and tangent'' method. For concreteness, write our elliptic curve as
\[E\colon y^2=x^3+Ax+B,\]
where $E$ really refers to the projective variety in $\PP^2_k$ of the corresponding homogenized polynomial. One ought to check smoothness and genus and so on, but we won't bother for the time being. Notably, our marked point $e\in E(k)$ is given by $[0:1:0]\in E$.

Now, fixing some $p,q\in E(k)$, we let $L$ denote the line connecting them. One can explicitly do the algebra to see that $X\cap L$ will have three intersection points---writing $L$ as $y=mx+b$, we see $m,b\in k$, so plugging in for
\[x^3+Ax+B-(mx+b)^2=0\]
with roots given by $p_x$ and $q_x$ will have a third root $r_x\in k$. One can check that the corresponding point $(r_x,r_y)\in X(k)$ has $p+q=(r_x,-r_y)$, which describes our group law rather explicitly. The point is that adding together the three points coming from $X\cap L$ ought to vanish in the group law because all divisors of the form $X\cap L$ are linearly equivalent.

\end{document}