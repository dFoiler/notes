% !TEX root = ../notes.tex

\documentclass[../notes.tex]{subfiles}

\begin{document}

\section{March 20}

Today we discuss the weak Mordell--Weil theorem.

\subsection{The Weak Mordell--Weil Theorem}
In order to avoid algebraic geometry for a little, we give intuition via the following argument.
\begin{proposition}
	Fix a number field $K$ containing the $r$th roots of unity. Then $\OO_K[1/N]^\times/\OO_K[1/n]^{\times r}$ has finite image in $H^1(\op{Gal}(\overline K/K),\mu_r)$ for any $N$ divisible by $r$.
\end{proposition}
\begin{proof}
	Note that this is roughly automatic by Dirichlet's unit theorem, which tells us that $\OO_K[1/N]$ is a finitely generated abelian group already. However, let's give a more geometric argument.

	The group scheme $\mathbb G_m\coloneqq\Spec\ZZ[x,1/x]$ represents the functor $\mathrm{Sch}\opp\to\mathrm{Ab}$ given by $S\mapsto\Gamma(S,\OO_S)^\times$. One can see this by gluing together the story on affine pieces; alternatively, we directly note that a morphism $\Spec\ZZ[x,1/x]$ amounts to choosing a global section $x\in\Gamma(S,\OO_S)$ which is a unit, which gives what we wanted.

	We now claim that $\OO_K[1/N]^\times/\OO_K[1/N]^{r\times}$ has finite image in $H^1(\op{Gal}(\overline K/K),\mu_r)$. To understand this, consider the short exact sequence
	\[0\to\mu_r\to\overline K^\times\stackrel{(-)^r}\to\overline K^\times\to0\]
	of modules over $G\coloneqq\op{Gal}(\overline K/K)$. Then the long exact sequence yields
	\[K^\times\stackrel{(-)^r}\to K^\times\stackrel\del\to H^1(G,\mu_r)\to0,\]
	where the rightmost zero is by Hilbert's theorem 90. As such, we see that there is a map
	\[\frac{\OO_K[1/N]^\times}{\OO_K[1/N]^{\times r}}\to\frac{K^\times}{K^{\times r}}\stackrel\del\cong H^1(G,\mu_r).\]
	Now, for the statement, we note there is a morphism $[r]\colon\mathbb G_m\to\mathbb G_m$ given by the ring map $\ZZ[x,1/x]\to\ZZ[x,1/x]$ by $x\mapsto x^r$. We now bound ramification. The point is that some $a\in\OO_K[1/N]^\times$ induces a morphism $\Spec\OO_K[1/N]\to\mathbb G_{m,\OO_K[1/N]}$, where we have implicitly base-changed $\mathbb G_m$ here. As such, providing an $r$th root of $a$ is equivalent to looking at the following fiber product.
	% https://q.uiver.app/?q=WzAsNCxbMCwwLCJcXGRpc3BsYXlzdHlsZVxcU3BlY1xcZnJhY3tcXE9PX0tbMS9OXVt4XX17XFxsZWZ0KHheci1hXFxyaWdodCl9Il0sWzEsMCwiXFxTcGVjXFxPT19LWzEvTl0iXSxbMSwxLCJcXG1hdGhiYiBHX3ttLFxcT09fS1sxL05dfSJdLFswLDEsIlxcbWF0aGJiIEdfe20sXFxPT19LWzEvTl19Il0sWzMsMiwiW3JdIl0sWzAsMV0sWzEsMiwiYSJdLFswLDNdXQ==&macro_url=https%3A%2F%2Fraw.githubusercontent.com%2FdFoiler%2Fnotes%2Fmaster%2Fnir.tex
	\[\begin{tikzcd}
		{\displaystyle\Spec\frac{\OO_K[1/N][x]}{\left(x^r-a\right)}} & {\Spec\OO_K[1/N]} \\
		{\mathbb G_{m,\OO_K[1/N]}} & {\mathbb G_{m,\OO_K[1/N]}}
		\arrow["{[r]}", from=2-1, to=2-2]
		\arrow[from=1-1, to=1-2]
		\arrow["a", from=1-2, to=2-2]
		\arrow[from=1-1, to=2-1]
	\end{tikzcd}\]
	Namely, an $r$th root asserts asking for an $\mathcal OO_K[1/N]$-point of this fiber product; for brevity, set $A_a\coloneqq\OO_K[1/N][x]/\left(x^r-a\right)$. Now, we can factor $x^r-a$ into a product of irreducibles $F_i$ and set $L_i\coloneqq K[x]/(F_i(x))$, and we see that
	\[A_a\otimes K=\prod_iL_i.\]
	Note that there is a $\mu_r$-action on $\Spec A_a$ by permuting the $r$th roots of $A$ (here we use $\mu_r\subseteq K$), and this action permutes the factors $L_i$ in the above product decomposition because $x^r-a$ is simply going to decompose itself into linear factors over the algebraic closure, and the $\mu_r$-action permutes the factors of this decomposition.

	Now, to see our finite image, we claim that $\del(a)\in H^1(G,\mu_r)$ actually comes from $H^1(\op{Gal}(L_i/K),\mu_r)$ for any fixed $i$. Indeed, to track our boundary map, we begin by choosing some $b\in\overline K^\times$ with $b^r=a$. However, by adjusting our $b$ (by our transitive $\mu_r$-action!), we may assume that the chosen $b$ comes from $L_i$. Now, the corresponding cocycle in $H^1(G,\mu_r)$ when passed through boundary is
	\[\del a\colon\sigma\mapsto\frac{\sigma b}b,\]
	which we see will actually be defined in $L_i$. Passing this ``restricted'' cocycle through inflation provides what we want.

	We now claim that
	\[A=\prod_i\OO_{L_i}[1/N],\]
	and each $L_i$ is unramified (over $K$) outside $(N)$. We leave this claim as an exercise.
	
	This will provide the desired finiteness: $H^1(\op{Gal}(L_i/K),\mu_r)$ is finite (it's a finite group and a finite module), and there are only finitely many extensions over $K$ of bounded degree and unramified outside some constant set. (To see this second claim, we note that being unramified outside some fix set of primes enforces some boundedness of the discriminant, and we can use the Minkowski bound to finish.) Now to finish the argument, we note that even as we vary $a\in\OO_K[1/N]^\times$, there are only finitely many possibilities of the $L_i$, so we only have to check the image of finitely many maps
	\[\op{Inf}\colon H^1(\op{Gal}(L/K),\mu_r)\to H^1(\op{Gal}(\overline K/K),\mu_r),\]
	so the image remains finite.
\end{proof}
Let's now do the same thing but for elliptic curves.
\weakmwthm*
\begin{proof}
	To be concrete, we write $E$ as $y^2=x^3+ax+b$. (Namely, our number field $K$ has characteristic zero, so this factoring is safe.) Smoothness, then, amounts to requiring $4a^3-27b^2\ne0$. We now choose $N$ both divisible by $6$ and by $r$, where $a,b\in\OO_K[1/N]$ (after clearing denominators!), and $4a^3-27b^2$ is actually a unit in $\OO_K[1/N]$.

	We now define $\mc E$ to be defined by $y^2=x^3+ax+b$ (lying in projective space) to be a scheme over $S\coloneqq\Spec\OO_K[1/N]$. In fact, $\mc E$ remains a group scheme, isomorphic to $\Pic^0_{\mc E/S}$ by running the exact same argument through. (In particular, the key ingredient to defining our rigidified line bundles is having a field-valued point for any field, but this is clear because we have the point $[0:1:0]\in E(\Spec A)$ for any ring $A$.) Notably, $\mc E$ is proper over $S$ (because it's projective), and $\mc E$ is also smooth by computing the Jacobian (namely, $4a^3-27b^2$ is still a unit!).

	We now apply the valuative criterion for properness. This tells us that a map $\Spec K\to E\to\mc E$ will extend to a map $\Spec\OO_K[1/N]\to\mc E$ factoring in the following diagram.
	% https://q.uiver.app/?q=WzAsNCxbMCwwLCJcXFNwZWMgSyJdLFsxLDAsIlxcbWMgRSJdLFswLDEsIlxcU3BlY1xcT09fS1sxL05dIl0sWzEsMSwiXFxTcGVjXFxPT19LWzEvTl0iXSxbMiwzLCIiLDAseyJsZXZlbCI6Miwic3R5bGUiOnsiaGVhZCI6eyJuYW1lIjoibm9uZSJ9fX1dLFswLDJdLFsyLDEsIiEiLDEseyJzdHlsZSI6eyJib2R5Ijp7Im5hbWUiOiJkYXNoZWQifX19XSxbMCwxXSxbMSwzXV0=&macro_url=https%3A%2F%2Fraw.githubusercontent.com%2FdFoiler%2Fnotes%2Fmaster%2Fnir.tex
	\[\begin{tikzcd}
		{\Spec K} & {\mc E} \\
		{\Spec\OO_K[1/N]} & {\Spec\OO_K[1/N]}
		\arrow[Rightarrow, no head, from=2-1, to=2-2]
		\arrow[from=1-1, to=2-1]
		\arrow["{!}"{description}, dashed, from=2-1, to=1-2]
		\arrow[from=1-1, to=1-2]
		\arrow[from=1-2, to=2-2]
	\end{tikzcd}\]
	Now, for our result, we fix some $a\in E(K)$ and study the same fiber product $\mc P_a\coloneqq S\times_{\mc E}\mc E$ arising in the following pullback square.
	% https://q.uiver.app/?q=WzAsNCxbMCwwLCJcXG1jIFBfYSJdLFsxLDAsIlMiXSxbMCwxLCJcXG1jIEUiXSxbMSwxLCJcXG1jIEUiXSxbMiwzLCJbcl0iXSxbMSwzLCJhIiwyXSxbMCwyXSxbMCwxXV0=&macro_url=https%3A%2F%2Fraw.githubusercontent.com%2FdFoiler%2Fnotes%2Fmaster%2Fnir.tex
	\[\begin{tikzcd}
		{\mc P_a} & S \\
		{\mc E} & {\mc E}
		\arrow["{[r]}", from=2-1, to=2-2]
		\arrow["a"', from=1-2, to=2-2]
		\arrow[from=1-1, to=2-1]
		\arrow[from=1-1, to=1-2]
	\end{tikzcd}\]
	Let's now start the proof. We have a short exact sequence as follows.
	\[0\to E[r](\overline K)\to E(\overline K)\stackrel r\to E(\overline K)\to0.\]
	As before, the long exact sequence here induces an inclusion
	\[\del\colon\frac{E(K)}{rE(K)}\into H^1(G,E[r](K)),\]
	where $G\coloneqq\op{Gal}(\overline K/K)$. We will show that the image of this inclusion is finite, which will finish the proof. For psychological reasons, we would like to assume that $E[r](\overline K)=E[r](K)$. Well, find some field $L$ such that $E[r](\overline K)=E[r](L)$, and suppose we have the claim for $L$. Then we can draw the following diagram.
	% https://q.uiver.app/?q=WzAsNSxbMCwyLCJFKEwpL3JFKEwpIl0sWzAsMSwiRShLKS9yRShLKSJdLFsxLDIsIkheMShcXG9we0dhbH0oXFxvdmVybGluZSBML0wpLEVbcl0oXFxvdmVybGluZSBLKSkiXSxbMSwxLCJIXjEoXFxvcHtHYWx9KFxcb3ZlcmxpbmUgTC9MKSxFW3JdKFxcb3ZlcmxpbmUgSykpIl0sWzEsMCwiSF4xKFxcb3B7R2FsfShML0spLEVbcl0oXFxvdmVybGluZSBLKSkiXSxbMSwzLCJcXGRlbF9LIl0sWzAsMiwiXFxkZWxfTCJdLFsxLDBdLFszLDJdLFs0LDNdXQ==&macro_url=https%3A%2F%2Fraw.githubusercontent.com%2FdFoiler%2Fnotes%2Fmaster%2Fnir.tex
	\[\begin{tikzcd}
		& {H^1(\op{Gal}(L/K),E[r](\overline K))} \\
		{E(K)/rE(K)} & {H^1(\op{Gal}(\overline L/L),E[r](\overline K))} \\
		{E(L)/rE(L)} & {H^1(\op{Gal}(\overline L/L),E[r](\overline K))}
		\arrow["{\del_K}", from=2-1, to=2-2]
		\arrow["{\del_L}", from=3-1, to=3-2]
		\arrow[from=2-1, to=3-1]
		\arrow[from=2-2, to=3-2]
		\arrow[from=1-2, to=2-2]
	\end{tikzcd}\]
	One can check by hand that the vertical right sequence is exact (this is on the homework), so if the bottom image is finite, then exactness says that the image of the middle map has size bounded by the product of the size of $H^1(\op{Gal}(L/K),E[r](\overline K))$ and the image of $\del_L$.

	Now, similar to before, this will come down to bounding ramification and degree. Namely, for all $a\in E(K)$, we claim that there is an extension $L_a/K$ unramified outside $(N)$ and of degree bounded by $r^2$ such that $\del(a)$ lies in the image of the map
	\[\op{Inf}\colon H^1(\op{Gal}(L/K),E[r](K))\to H^1(G,E[r](K)).\]
	This will complete the proof of our finiteness because there are only finitely many extensions $L$ of bounded degree and unramified outside $(N)$, so we are really only checking the image of finitely many inflation maps from the finite groups $H^1(\op{Gal}(L/K),E[r](K))$, so the total image of $\del$ will be contained in this finite union of finite sets and hence be finite.

	Roughly speaking,
	\[\del(a)\in\im\left(H^1(\op{Gal}(L_a/K),E[r](K))\to H^1(\op{Gal}(\overline K/K),E[r](K))\right),\]
	where $L_a/K$ is an extension where $\mc P_a(L_a)\ne\emp$. Indeed, this is essentially how $\del$ is defined: if we have $P_a(L_a)\ne\emp$, then we find an $m$th root in $\mc E$ for our point $a$ (over $L_a$!), and then we can find our $\del(a)$ as arising from over $\op{Gal}(L_a/K)$ by tracking through the boundary morphism, where the point is that our choice of lift along $[r]\colon E\to E$ of $a$ may live over this specified $L_a$.

	We now attempt to find such an $L_a$. In words, assuming $m\nmid N$, the point is that $[m]\colon\mc E\to\mc E$ is finite \'etale of degree $r^2$, so
	\[\mc P_a=\Spec\prod_i\OO_{L_i}[1/N],\]
	where the $L_i/K$ is finite and unramified outside $N$. In fact, because the degree of our map is at most $r^2$, each of the $L_i$ has degree at most $r^2$, so we may choose the $L_a$ appropriately.\todo{Maybe?}

	Let's see this directly. We proceed in steps.
	\begin{enumerate}
		\item Note that $\mc P_a\to S$ is a finite map of schemes because it is the base-change of the finite map $[r]\colon\mc E\to\mc E$; namely, we can check that $[r]\colon\mc E\to\mc E$ is proper because $\mc E$ is proper and separated over $S$, and it is quasi-finite because the number of points in a fiber of a point in $\mc E$ can be checked after the base-change to a field, and then we are looking at an elliptic curve and may appeal to \Cref{cor:deg-mult}. In particular, finite maps are affine, so we conclude that $\mc P_a$ can be written as $\Spec A_a$ where $A_a$ is a finite $\OO_K[1/N]$-algebra.

		\item Now, we note that the multiplication map $[r]\colon E\to E$ has degree $r^2$, so we can see that $[r]_*\OO_E$ is locally free of rank $r^2$ (checking on stalks, it's enough to see that $[r]_*\OO_E$ is torsion-free on stalks---over the local ring---but there is no torsion because $E$ is an integral scheme). Continuing, as a closed subscheme, we see
		\[E[r]=\Spec([r]_*\OO_E)(e)\]
		by tracking through what this means: on divisors, we are asking for the points which go to $e$ when multiplied by $r$. Here, we see that $([m]_*\OO_E)(e)$ is a finite-dimensional $K$-algebra $A$, so it is Artinian and therefore a product of Artinian local rings
		\[A=\prod_iA_i,\]
		These $A_i$ have a residue field $\kappa_i\coloneqq A_i/\mf m_i$, and each $\kappa_i$ is separable over $K$ because $[r]$ is separable. Now, when $r$ is coprime with the characteristic (as it is in our case), we have $m^2$ points in this fiber (because our map is separable), so $\sum_i\dim_k\kappa_i=r^2$ by this point-counting. But $A$ needs to also have dimension $r^2$ over $K$, so we conclude that $A_i=\kappa_i$ is forced, so we can write
		\[E[m]=\Spec\prod_iL_i',\]
		where the $L_i'$ are finite and separable over $K$ with degrees summing to $r^2$.

		\item We now claim that the map $\mc P_a\to S$ is flat. Indeed, we may check this stalk-locally: fix some prime $\mf p\in S$, and let $\mf P$ be the pre-image in $A_a$. As such, we are asking for the extension $\OO_{S,\mf p}\to A_{a,\mf P}$ is flat.

		Well, looking at residue fields, we see that the residue fields $A_a(\mf P)$ has dimension $m^2$ as a vector space over $\OO_S(\mf p)$, so we can choose elements $x_1,\ldots,x_{m^2}\in A_{a,\mf P}$ which grant a basis over $A_a(\mf P)$. But then Nakayama's lemma tells us that we have a surjection
		\[\OO_{S,\mf p}^{\oplus r^2}\onto A_{a,\mf P}\]
		given by this basis. In fact, this is injective: we can check injectivity after localizing at the generic point, but then we are looking at an algebra of dimension $r^2$ over $K$ (because we have lifted from a map of degree $r^2$), so the surjectivity of our map of vector spaces over dimension $r^2$.

		\item Next up, we claim that $A_a$ is the integral closure of $\OO_K[1/N]$ sitting inside $A_a\otimes K=\prod_iL_i$ where the $L_i$ are finite separable extensions over $K$, and each $L_i$ is unramified over $K$ outside $(N)$.

		Well, let $B$ be the integral closure of $\OO_K[1/N]$ inside $A_a\otimes K$. Certainly each element of $A_a$ is integral over $\OO_K[1/N]$ because $A_a$ is actually finite over $\OO_K[1/N]$. Thus, we have an embedding $A_a\into B$, giving us the following diagram.
		% https://q.uiver.app/?q=WzAsMyxbMCwwLCJcXFNwZWMgQiJdLFsxLDEsIlMiXSxbMiwwLCJcXFNwZWMgQV9hIl0sWzAsMl0sWzIsMV0sWzAsMV1d&macro_url=https%3A%2F%2Fraw.githubusercontent.com%2FdFoiler%2Fnotes%2Fmaster%2Fnir.tex
		\[\begin{tikzcd}
			{\Spec B} && {\Spec A_a} \\
			& S
			\arrow[from=1-1, to=1-3]
			\arrow[from=1-3, to=2-2]
			\arrow[from=1-1, to=2-2]
		\end{tikzcd}\]
		Here, the map $\Spec B\to\Spec A_a$ is dominant. Now, letting $\Omega$ be an algebraic closure of $K$, we base-change, yielding the following diagram.
		% https://q.uiver.app/?q=WzAsNixbMCwxLCJcXFNwZWMgQiJdLFsxLDIsIlMiXSxbMiwxLCJcXFNwZWNcXE9tZWdhIl0sWzEsMSwiXFxTcGVjIEFfYSJdLFsyLDAsIlxcU3BlYyBBX2FcXG90aW1lc1xcT21lZ2EiXSxbMSwwLCJcXFNwZWMgQlxcb3RpbWVzXFxPbWVnYSJdLFswLDFdLFswLDNdLFszLDFdLFs1LDBdLFs0LDNdLFsyLDFdLFs1LDRdLFs0LDJdXQ==&macro_url=https%3A%2F%2Fraw.githubusercontent.com%2FdFoiler%2Fnotes%2Fmaster%2Fnir.tex
		\[\begin{tikzcd}
			& {\Spec B\otimes\Omega} & {\Spec A_a\otimes\Omega} \\
			{\Spec B} & {\Spec A_a} & \Spec\Omega \\
			& S
			\arrow[from=2-1, to=3-2]
			\arrow[from=2-1, to=2-2]
			\arrow[from=2-2, to=3-2]
			\arrow[from=1-2, to=2-1]
			\arrow[from=1-3, to=2-2]
			\arrow[from=2-3, to=3-2]
			\arrow[from=1-2, to=1-3]
			\arrow[from=1-3, to=2-3]
		\end{tikzcd}\]
		Now, $\Spec A_a\otimes\Omega$ has $r^2$ distinct points, so $B$ must have at least $r^2$ distinct primes as well. It then follows that the map $A_a\to B$ is surjective by making an argument similar to above, arguing that $B$ should be some product of fields and counting points/dimensions.\todo{What?}

		In total, we may write
		\[A_a=\prod_i\OO_{L_i}[1/N]\]
		by taking our integral closures appropriately. Notably, looking over each point, we see that the group $E[r]$ acts on the product of the fields $\prod_iL_i$ (which is the fiber over our point $a$). Thus, each extension $L_i/K$ is Galois, so choosing a particular prime $\mf p\in\Spec\OO_K[1/N]$, we may factor it up in $\OO_{L_i}[1/N]$ and find that all the ramification and inertial data must be the same due to our Galois action. As such, we compute these degrees locally to see
		\[r^2=\sum_if_ie_ig_i,\]
		where $f_i$ is the inertial data, $e_i$ is the ramification data, and $g_i$ is the number of primes, where this $i$th index is the data at $L_i/K$.

		On the other hand, we may count points to see that the number of points lying over $\mf p$ in $A_a$ amounts to the number of separable extensions $\kappa(\mf p)$ lying in the various residue fields, so
		\[r^2=\#\Spec A_a(\overline{\kappa(\mf p)})=\sum_if_ig_i.\]
		In total, we see that each $e_i$ must be $1$, being unramified follows.
		\qedhere
	\end{enumerate}
\end{proof}

\end{document}