% !TEX root = ../notes.tex

\documentclass[../notes.tex]{subfiles}

\begin{document}

\section{February 27}

Today we hope to finish global class field theory but very fast.

\subsection{Tate's Theorem}
We are going to want the following result.
\begin{theorem}[Tate] \label{thm:tate}
	Fix a finite group $G$ and a $G$-module $M$. Suppose that each subgroup $H\subseteq G$ satisfies the following conditions.
	\begin{itemize}
		\item $H^1(H,M)=0$.
		\item $H^2(H,M)$ is cyclic of order $\#H$.
	\end{itemize}
	Then, for each $r$, there is an isomorphism $\widehat H^r(G,\ZZ)\cong\widehat H^{r+2}(G,M)$.
\end{theorem}
\begin{proof}
	We provide a sketch. Roughly speaking, we are going to want to combine two different boundary maps. In particular, the short exact sequence
	\[0\to I_G\to\ZZ[G]\to\ZZ\to0\]
	tells us that $\widehat H^r(G,\ZZ)\cong\widehat H^{r+1}(G,I_G)$. We are now a single index away. Thus, we want to construct a short exact sequence
	\[0\to M\to\mc E\to I_G\to0,\]
	where $\mc E$ the boundary maps $\widehat H^{r+1}(G,I_G)\cong\widehat H^{r+2}(G,M)$ are isomorphisms.

	Fix a $2$-cocycle $\varphi\colon G^2\to M$ representing a generator of $H^2(G,M)$.\todo{Yoneda ext?} Now, as an abelian group, we set
	\[\mc E\coloneqq M\oplus I_G=M\oplus\bigoplus_{g\in G\setminus\{e\}}\ZZ x_g.\]
	We need to give $\mc E$ a $G$ action. For this, we define
	\[h\cdot x_g\coloneqq x_{hg}-x_h+\varphi(h,g),\]
	where $x_1\coloneqq\varphi(1,1)$. One can check that this does in fact define a $G$-action. In particular, one can compute that the map $G\to I_G$ given by $g\mapsto x_g$ goes to the generator of $H^2(G,M)$ under the correct boundary map. One can finish by checking that our boundary maps are isomorphisms, which is good enough.
\end{proof}
\begin{example} \label{ex:artin-recip}
	Given a Galois extension of local fields $L/K$, then we see that the $G$-module $L^\times$ satisfies the above conditions by our discussion of local class field theory. In particular, $H^1(H,M)$ vanishes by Hilbert's theorem 90, and being cyclic followed from our rather lengthy and difficult computation. Then \Cref{thm:tate} promises us an isomorphism
	\[G^{\mathrm{ab}}\cong\frac{I_G}{I_G^2}=H_1(G,I_G)=\widehat H^{-2}(G,\ZZ)\cong\widehat H^0(G,L^\times)=\frac{K^\times}{\op N^L_K(L^\times)}.\]
	In particular, if $G\cong\ZZ/p\ZZ$, then we see that taking $p$th powers kills our equivalence classes, so they must be norms.\todo{Gaussian rationals?}
\end{example}

\subsection{Finishing the Second Inequality}
We are now in the case where $L/K$ has Galois group $\langle\gamma\rangle\cong\ZZ/p\ZZ$, and $K$ contains $\mu_p$. We thus claim that $L=K(\alpha^{1/p})$ for some $\alpha\in K^\times$. This is Kummer theory. Well, for some homomorphism $\chi\colon\langle\gamma\rangle\to\mu_p$, and set
\[W_\chi\coloneqq\{\alpha\in L:g\alpha=\chi(g)\alpha\}.\]
We claim that each of these $W_\chi$ is one-dimensional and have direct sum equal to $L$. For this, it's enough to check over an $\Omega\coloneqq\overline K$. Namely, we are looking for an isomorphism
\[\prod_{\sigma\colon L\into\Omega}\Omega\cong\bigoplus_\chi W_\chi\otimes_K\Omega,\]
and we can check this directly. In particular, we see $\op{Hom}(\langle\gamma\rangle,\mu_p)\cong\ZZ/p\ZZ$, so we can decompose everything appropriately. Namely, pulling back elements of $\ZZ/p\ZZ$ allows us to recover elements of $W_\chi$ to make these one-dimensional and so on.

We are now interested in showing
\[\left[\AA_K^\times:K^\times\op N^L_K(\AA_L^\times)\right]\mid\#G=p.\]
Thus, we want to show that we have ``lots'' of norms in $\AA_K^\times$. As usual, choose a (large) finite subset $S\subseteq V_K$ satisfying the following constraints.
\begin{itemize}
	\item $S$ contains the infinite places.
	\item $S$ contains the places lying over $(p)\in V_\QQ$.
	\item $S$ contains the places where $\alpha$ is not a unit.
\end{itemize}
Now, we consider the (large) field $M\coloneqq K(\sqrt[p]{\OO_{K,S}^\times})$, which is finite over $K$ because $\OO_{K,S}^\times$ is finitely generated by Dirichlet's unit theorem. In fact, carefully tracking the unit theorem allows us to see $[M:K]=p^{\#S}$. Additionally, $M/K$ is unramified outside $S$ by checking at each place.

We are going to want the following result, quickly.
\begin{lemma}
	Fix an abelian extension of number fields $L/K$. Suppose we have a subgroup $D\subseteq\AA_K^\times$ contained in $\op N^L_K(\AA_L^\times)$ such that $K^\times D$ is dense in $\AA_K^\times$. Then $L=K$.
\end{lemma}
\begin{proof}
	We sketch. Roughly speaking, $D\subseteq\op N^L_K(\AA_K^\times)$ and our density result forces the groups
	\[\frac{K_v}{\op N^{L_w}_{K_v}(L_w^\times)}\]
	to be small, for any place $v$ lying under a place $w$. However, we do have a lower bound on this size from the first inequality (or alternatively, from local class field theory), so we will force $L=K$.
\end{proof}
As an application, one can use \Cref{ex:artin-recip} and the above lemma to show that $\op{Gal}(M/L)$ is generated by Frobenius elements $\op{Frob}_v$ for various $v\notin S$. Notably, these Frobenius elements exist because $M/K$ is unramified.

As such, we may find $T\subseteq V_K$ disjoint from $S$ such that the Frobenius elements $\op{Frob}_v$ for $v\in T$ generate $\op{Gal}(M/L)$. We are now equipped to write down
\[E\coloneqq\prod_{v\in S}K_v^{\times p}\times\prod_{v\in T}K_v^\times\times\prod_{v\notin S\cup T}\OO_v^\times.\]
The main claim, now, is that $E\subseteq\op N^L_K(\AA_L^\times)$. We go factor-by-factor.
\begin{itemize}
	\item Given $v\in S$, we know that $p$th powers are norms by \Cref{ex:artin-recip}.
	\item For $v\in T$, our choice of $T$ enforces $L_w=K_v$. In particular, the local Frobenius element of $M/L$ is going to be the same as the local Frobenius element of $M/K$, so the extensions at $L$ and $K$ must coincide.
	\item For $v\notin S\cup T$, our extension is unramified, so we see that all units are norms.
\end{itemize}
In particular, we see that $\left[\AA_K^\times:K^\times\op N^L_K(\AA_L^\times)\right]$ is divisible by $\left[\AA_K^\times:K^\times E\right]$, so we might as well work with $E$. We can now compute
\[\left[\AA_K^\times:K^\times E\right]=\frac{\left[\AA_{K,S\cup T}^\times:E\right]}{\left[\OO_{K,S\cup T}^\times:K^\times\cap E\right]},\]
roughly speaking by examining how $E$ interacts with the id\'eles. One can now compute that $\left[\AA_{K,S\cup T}^\times:E\right]=p^{2\#S}$ and $\left[\OO_{K,S\cup T}^\times:K^\times\cap E\right]=p^{\#S+(\#S-1)}$, so the quotient is in fact size $p$. This completes the proof.
\begin{remark}
	Combining with the first inequality, we must actually have $K^\times E=K^\times\op N^L_K(\AA_L^\times)$, which roughly tells us what our norms are.
\end{remark}

\end{document}