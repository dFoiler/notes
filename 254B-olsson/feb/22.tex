% !TEX root = ../notes.tex

\documentclass[../notes.tex]{subfiles}

\begin{document}

\section{February 22}

Let's try to show global class field theory in a week.

\subsection{The First Inequality}
Fix a finite cyclic extension of global fields $L/K$ with Galois group $G$. Last time we showed $\AA_L^\times/L^\times=\AA_{L,T}^\times/\OO_{L,T}^\times$, where $T$ is some sufficiently large $G$-equivariant subset of $V_L$. In particular, we want $T$ to contain the archimedean places and to generate the class group. Thus, we have an exact sequence
\[1\to\OO_{L,T}^\times\to\AA_{L,T}^\times\to\AA_L^\times/L^\times\to1,\]
so
\begin{equation}
	h(G,\AA_L^\times/L^\times)=\frac{h(G,\AA_{L,T}^\times)}{h(G,\OO_{L,T}^\times)}, \label{eq:reduce-first-ineq}
\end{equation}
provided that these Herbrand quotients are finite. In fact, we can see that these are finite from the following proof: indeed, we have a long exact sequence as follows.
% https://q.uiver.app/?q=WzAsNixbMCwwLCJIXjEoRyxcXE9PX3tMLFR9XlxcdGltZXMpIl0sWzEsMCwiSF4xKEcsXFxBQV97TCxUfV5cXHRpbWVzKSJdLFsyLDAsIkheMShHLFxcQUFfTF5cXHRpbWVzL0xeXFx0aW1lcykiXSxbMCwxLCJIXjIoRyxcXEFBX0xeXFx0aW1lcy9MXlxcdGltZXMpIl0sWzEsMSwiSF4yKEcsXFxBQV97TCxUfV5cXHRpbWVzKSJdLFsyLDEsIkheMihHLFxcT09fe0wsVH1eXFx0aW1lcykiXSxbMCwxXSxbMSwyXSxbMyw0XSxbNCw1XSxbMiw1XSxbMywwXV0=&macro_url=https%3A%2F%2Fraw.githubusercontent.com%2FdFoiler%2Fnotes%2Fmaster%2Fnir.tex
\[\begin{tikzcd}
	{H^1(G,\OO_{L,T}^\times)} & {H^1(G,\AA_{L,T}^\times)} & {H^1(G,\AA_L^\times/L^\times)} \\
	{H^2(G,\AA_L^\times/L^\times)} & {H^2(G,\AA_{L,T}^\times)} & {H^2(G,\OO_{L,T}^\times)}
	\arrow[from=1-1, to=1-2]
	\arrow[from=1-2, to=1-3]
	\arrow[from=2-1, to=2-2]
	\arrow[from=2-2, to=2-3]
	\arrow[from=1-3, to=2-3]
	\arrow[from=2-1, to=1-1]
\end{tikzcd}\]
In particular, we are going to compute $h(G,\AA_{L,T}^\times)$ and $h(G,\OO_{L,T}^\times)$ on the nose, from which finiteness of $h(G,\AA_L^\times/L^\times)$ also will follow. Let's see this.
\begin{theorem}
	Fix a finite cyclic extension of global fields $L/K$ with Galois group $G$, and choose a subset $S\subseteq V_K$ containing the archimedean and ramified places with $T\coloneqq S_L$.
	\begin{listalph}
		\item We have
		\[h\left(G,\AA_{L,T}^\times\right)=\prod_{v\in S}n_v,\]
		where $n_v=[L_w:K_v]$ for a chosen place $w\in V_L$ over $v\in V_K$.
		\item We have
		\[h\left(G,\OO_{L,T}^\times\right)=\prod_{v\in S}n_v.\]
	\end{listalph}
\end{theorem}
\begin{proof}
	Quickly, note that the extension being Galois implies that $n_v=e(w/v)f(w/v)$ does not depend on the choice of $w$, so our products are well-defined. We show these one at a time.
	\begin{listalph}
		\item Observe that
		\[\AA_{L,T}^\times=\prod_{v\in S}\Bigg(\prod_{w\mid v}L_w^\times\Bigg)\times\prod_{v\notin S}\Bigg(\prod_{w\mid v}\OO_w^\times\Bigg).\]
		In particular, this is
		\[\AA_{L,T}^\times=\prod_{v\in S}\op{Ind}^G_{G_w}L_w^\times\times\prod_{v\notin S}\op{Ind}_{G_w}^G\OO_w^\times,\]
		where $w$ is a chosen place over $v$. Taking cohomology and using \Cref{cor:get-to-induced}, this is
		\[H^\bullet(G,\AA_{L,T}^\times)=\prod_{v\in S}H^\bullet(G_w,L_w^\times)\times\prod_{v\notin S}H^\bullet(G_w,\OO_w^\times).\]
		Because $S$ contains all ramified places, we see that $H^\bullet(G_w,\OO_w^\times)=0$ always by \Cref{lem:unramified-units-cohom}, so we have left to compute the left product. Taking Herbrand quotients now, we see
		\[H^\bullet(G,\AA_{L,T}^\times)=\prod_{v\in S}h(G_w,L_w^\times),\]
		so we appropriately claim that $h(G_w,L_w^\times)=n_v$, so we need a little more local class field theory.

		By the usual exact sequence
		\[1\to\OO_w^\times\to L_w^\times\to\ZZ\to0,\]
		we see $h(G_w,L_w^\times)=h(G_w,\ZZ)h(G_w,\OO_w^\times)=n_vh(G_w,\OO_w^\times)$, so we want $h(G_w,\OO_w^\times)=1$. Our argument that $H^1(G_w,\OO_w^\times)=0$ for free, so we want to show $H^2(G_w,\OO_w^\times)=0$, which one can again check by going to residue fields. Roughly speaking, the Brauer--Severi variety argument from \Cref{lem:br-severi-thing} still works, where we are now inputting the fact that the Brauer group of our extension of finite fields is trivial, which is certainly true from our computation of its Herbrand quotient in \Cref{lem:finite-herbrand}.

		\item Roughly speaking, the point is that $\OO_{L,T}^\times$ will embed as a lattice into $V_\RR$, where $V=\op{Mor}(T,\ZZ)$. (The $G$-action on $V$ is given by $(gf)(t)=f\left(g^{-1}t\right)$.) Namely, our embedding $\iota\colon\OO_{L,T}^\times\to V_\RR$ is given by
		\[\iota\colon x\mapsto(\log|x|_w)_{w\in T}.\]
		This is of course a homomorphism, and Dirichlet's unit theorem tells us that the $\ker\iota$ is finite and that the image $\Lambda$ is a lattice of the hyperplane
		\[\sum_{w\in T}x_w=0,\]
		which comes from the product formula and recognizing that $\iota(x)$ is trivial on places outside $T$. Notably, we have a decomposition
		\[V_\RR=\Lambda_\RR\oplus\RR(1,\ldots,1)\]
		in fact of $G$-modules. (Indeed, both modules on the right are $G$-submodules of $V_\RR$.) Setting $\Lambda'\coloneqq\Lambda\oplus\ZZ(1,\ldots,1)$, we see $\Lambda'_\RR=V_\RR$.

		We now compute. Note $h(G,\OO_{L,T}^\times)=h(G,\Lambda)$ because their quotient is finite and contributes nothing by \Cref{lem:finite-herbrand}. On the other hand, we see $h(G,\Lambda')=h(G,\Lambda)h(G,\ZZ)=nh(G,\Lambda)$, so
		\[nh(G,\OO_{L,T}^\times)=h(G,\Lambda')\]
		by rearranging. On the other hand, we note
		\[V=\prod_{v\in S}\Bigg(\prod_{w\mid v}\ZZ\Bigg),\]
		so
		\[H^\bullet(G,V)=\prod_{v\in S}H^\bullet(G_w,\ZZ)\]
		by \Cref{cor:get-to-induced} as usual. Thus,
		\[h(G,V)=\prod_{v\in S}h(G_w,\ZZ)=\prod_{v\in S}n_v.\]
		Combining, we see we want to show $h(G,\Lambda')=h(G,V)$. However, these are both lattices of this real vector space, so with this in mind, it will be enough to give a $G$-module isomorphism $\Lambda'\otimes_\ZZ\QQ\cong V\otimes_\ZZ\QQ$ by \Cref{cor:any-lattice-okay}. Well, using the fact our vector spaces are finite-dimensional, we compute
		\[\op{Hom}_{\QQ[G]}(\Lambda_\QQ',V_\QQ)\cong((\Lambda'_\QQ)^\lor\otimes_\QQ V_\QQ)^G=\ker\left((1-\sigma)\colon(\Lambda'_\QQ)^\lor\otimes_\QQ V_\QQ\to(\Lambda'_\QQ)^\lor\otimes_\QQ V_\QQ\right),\]
		where $\sigma\in G$ is the generator. However, taking the kernel commutes with taking the tensor product with a field because these kernel computations can just look at bases, so we might as well be computing the kernel of
		\[(1-\sigma)\colon(\Lambda'_\RR)^\lor\otimes_\RR V_\RR\to(\Lambda'_\RR)^\lor\otimes_\RR V_\RR,\]
		which we do know is isomorphic to $\op{Hom}_{\RR[G]}(\Lambda'_\RR,V_\RR)$. However, we do now know that there is an isomorphism $\alpha$ in this last group, so we produce an element
		\[\sum_{i=1}^n\beta_i\otimes v_i\in(\Lambda_\QQ')^\lor\otimes_\QQ V_\RR\]
		which corresponds to an $\mathbb R$-isomorphism. Selecting the $v_i$ to be rational vectors sufficiently close to $v_i$, we may assume that the determinant of the corresponding linear map remains nonzero (as it is in the above case), so we get to pull back to the desired $\QQ$-isomorphism $\Lambda'_\QQ\cong V_\QQ$.
		\qedhere
	\end{listalph}
\end{proof}
\begin{corollary}
	Fix a finite cyclic extension of global fields $L/K$ with Galois group $G$, and choose a subset $S\subseteq V_K$ containing the archimedean places with $T\coloneqq S_L$ and generating the ideal class group of $L$. Then $h\left(G,\AA_L^\times/L^\times\right)=n$. In particular, it is finite.
\end{corollary}
\begin{proof}
	This follows from the above theorem, combined with \eqref{eq:reduce-first-ineq}.
\end{proof}

\end{document}