% !TEX root = ../notes.tex

\documentclass[../notes.tex]{subfiles}

\begin{document}

\section{February 17}
The homework is killing me.

\subsection{Applications of Herbrand Quotients}
Fix a finite cyclic group $G$. We continue discussing Herbrand quotients.
\begin{corollary} \label{cor:brauer-finite}
	Fix a finite extension of finite fields $\ell/\kappa$ with Galois group $G$. Then $H^2(G,\ell^\times)=0$.
\end{corollary}
\begin{proof}
	Note $G$ is cyclic because these are finite fields, and $\ell^\times$ is finite, so $h(G,\ell^\times)=1$. However, $H^1(G,\ell^\times)=0$ by Hilbert's theorem 90, so it follows that $H^2(G,\ell^\times)=0$ for free.
\end{proof}
\begin{remark}
	This implies that the Brauer group over $k$ vanishes, by taking the colimit over all $\ell/k$.
\end{remark}
\begin{corollary} \label{cor:any-lattice-okay}
	Fix a cyclic group $G$. Then let $V$ be finite-dimensional $G$-representation over a field $\QQ$. Given $G$-stable lattices $M_1,M_2\subseteq V$, we have $h(G,M_1)=h(G,M_2)$.
\end{corollary}
Here, a lattice is a free $\ZZ$-submodule with $\ZZ$-rank equal to the dimension of $V$.
\begin{proof}
	Let $M_i$ have basis $\{v_{ij}\}_{j=1}^n$, where $n=\dim V$. Notably, we can write
	\[v_{2i}=\sum_{j=1}^nc_{ij}v_{1j}\]
	for some $c_{ij}$. Letting $N$ be the product of the denominators of the $c_{ij}$, we see that multiplication by $N$ grants an inclusion $N\colon M_1\to M_2$. Thus, we get an exact sequence
	\[0\to M_1\stackrel N\to M_2\to M_2/NM_1\to0.\]
	Notably, tensoring this with $\QQ$ makes the left an isomorphism, so $M_2/NM_1$ must be a torsion abelian group which is finitely generated, implying that it must be finite. Thus, $h(G,M_2/NM_1)=1$, so $h(G,M_1)=h(G,M_2)$ follows.
\end{proof}
\begin{remark}
	One can actually describe $G$-representations for cyclic groups $G$ somewhat concretely; let $G=\langle\sigma\rangle$ have order $n$. Namely, given a ring $R$, we would like to discuss $R[G]$-modules, where we see
	\[R[G]=\frac{R[\sigma]}{\left(\sigma^n-1\right)}.\]
	For example, if $R$ contains a primitive $n$th root of unity $\zeta$ (and $R$ has characteristic not dividing $p$), then
	\[R[G]\cong\prod_{i=1}^n\frac{R[\sigma]}{(\sigma-\zeta)}\cong R^n.\]
	Thus, an $R[G]$-module is essentially just a direct sum of $n$ different $R$-modules $M_1,\ldots,M_n$, and then the $G$-action on $M_i$ is given by $\sigma\mapsto\zeta^i$.
\end{remark}
\begin{remark}
	One can use the previous remark to show that $M_1\otimes_K\Omega\cong M_2\otimes_K\Omega$ implies that $M_1\cong M_2$, when $K\subseteq\Omega$ is an inclusion of fields. Roughly speaking, the point is that we can decompose $M_1$ into a direct sum as described above, if we have enough roots of unity, then we are basically prescribing dimension at each graded component.
\end{remark}

\subsection{Herbrand Quotient Computation}
We are going to show that $h(G,\AA_L^\times/L^\times)=n$, when $L/K$ is a cyclic extension of global fields of degree $n$.
\begin{lemma}
	Fix an extension of global fields $L/K$. Then there exists a finite set of places $T$ such that $\AA_L^\times=L^\times\cdot\AA_{L,T}^\times$ so that $\AA_L^\times/L^\times=\AA_{L,T}^\times/L^\times$.
\end{lemma}
\begin{proof}
	The point is to hit all the ideal classes. Fix a set of places $S\subseteq V_K$ satisfying the following conditions.
	\begin{itemize}
		\item $S$ contains the archimedean places.
		\item The finite part of $S_L$, made of primes $\{\mf P_1,\ldots,\mf P_n\}$, generates the ideal class group of $L$.
	\end{itemize}
	In particular, the class group of $L$ is finite, so we can find a finite set $S\subseteq V_K$ satisfying the above conditions.

	We now set $T\coloneqq S_L$. Let's show this works. Fix an id\'ele $(\alpha_w)_{w\in V_L}\in\AA_L^\times$. Then
	\[\prod_{w<\infty}\mf P_w^{\op{val}_w(\alpha_w)}\]
	is an ideal equivalent to some product
	\[I\coloneqq\prod_{i=1}^r\mf P_r^{v_r}.\]
	In other words, there is $\beta\in L^\times$ such that $\op{val}_w(\beta)=\op{val}_w(\alpha_w)-\op{val}_w(I)$. Choosing uniformizers $\pi_w\in\mf p_w$, we note
	\[\beta\cdot\left(\pi_w^{\op{val}_w(I)}\right)_w\]
	has the same valuation as $\alpha$ at every place $w$. In particular, the quotient lives in $\AA_{L,\emp}^\times$, so we are now safe.
\end{proof}
\begin{remark}
	Note that making $T$ larger does not hurt us, so we may assume that $T$ is $G$-stable.
\end{remark}
At the end of the day, we have a diagram which looks like the following.
% https://q.uiver.app/?q=WzAsMTAsWzAsMCwiMCJdLFsxLDAsIlxcT09fe0wsVH1eXFx0aW1lcyJdLFsyLDAsIlxcQUFfe0wsVH1eXFx0aW1lcyJdLFszLDAsIlxcQUFfe0wsVH1eXFx0aW1lcy9cXE9PX3tMLFR9XlxcdGltZXMiXSxbNCwwLCIwIl0sWzQsMSwiMCJdLFswLDEsIjAiXSxbMSwxLCJMXlxcdGltZXMiXSxbMiwxLCJcXEFBX0xeXFx0aW1lcyJdLFszLDEsIlxcQUFfTF5cXHRpbWVzL0xeXFx0aW1lcyJdLFswLDFdLFsxLDJdLFsyLDNdLFszLDRdLFs2LDddLFs3LDhdLFs4LDldLFs5LDVdLFsxLDcsIiIsMSx7InN0eWxlIjp7InRhaWwiOnsibmFtZSI6Imhvb2siLCJzaWRlIjoidG9wIn19fV0sWzIsOCwiIiwxLHsic3R5bGUiOnsidGFpbCI6eyJuYW1lIjoiaG9vayIsInNpZGUiOiJ0b3AifX19XSxbMyw5XV0=&macro_url=https%3A%2F%2Fraw.githubusercontent.com%2FdFoiler%2Fnotes%2Fmaster%2Fnir.tex
\[\begin{tikzcd}
	0 & {\OO_{L,T}^\times} & {\AA_{L,T}^\times} & {\AA_{L,T}^\times/\OO_{L,T}^\times} & 0 \\
	0 & {L^\times} & {\AA_L^\times} & {\AA_L^\times/L^\times} & 0
	\arrow[from=1-1, to=1-2]
	\arrow[from=1-2, to=1-3]
	\arrow[from=1-3, to=1-4]
	\arrow[from=1-4, to=1-5]
	\arrow[from=2-1, to=2-2]
	\arrow[from=2-2, to=2-3]
	\arrow[from=2-3, to=2-4]
	\arrow[from=2-4, to=2-5]
	\arrow[hook, from=1-2, to=2-2]
	\arrow[hook, from=1-3, to=2-3]
	\arrow[from=1-4, to=2-4]
\end{tikzcd}\]
The induced morphism on the right is injective by the Snake lemma, and we note that the map $\AA_{L,T}^\times\to\AA_L^\times/L^\times$ is surjective by the lemma, so in fact the induced morphism on the right is also surjective. Thus, $\AA_L^\times/L^\times=\AA_{L,T}^\times/\OO_{L,T}^\times$. (Here, $\OO_{L,T}^\times$ are the $T$-units, which are the elements of $L^\times$ with vanishing valuation outside $T$.)
\begin{remark}
	The arguments of \Cref{prop:global-to-local-idele} also tell us that
	\[H^\bullet(G,\AA_{L,S_L}^\times)=\prod_{v\in S}H^\bullet(G_w,L_w^\times)\times\prod_{w\notin S}H^\bullet(G_w,\OO_w^\times),\]
	and the right product vanishes if, for example, $S$ contains the places of $K$ which ramify over $L$.
\end{remark}

\end{document}