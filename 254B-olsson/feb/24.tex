% !TEX root = ../notes.tex

\documentclass[../notes.tex]{subfiles}

\begin{document}

\section{February 24}

We finish showing global class field theory. Fix a cyclic extension of global fields $L/K$ with Galois group $G$. We want to show that $\widehat H^1(G,\AA_L^\times/L^\times)$ vanishes, so because $h(G,\AA_L^\times/L^\times)=n$, it suffices to show that $\widehat H^0(G,\AA_L^\times/L^\times)\le n$. This is the second inequality.
\begin{remark}
	The remainder of the proof will be quite technical. Roughly speaking Herbrand quotients play well in short exact sequences (like Euler characteristics), but getting an individual cohomology group is harder.
\end{remark}

\subsection{Remark on Restriction}
We are going to want a little more group cohomology to continue. Fix a finite group $G$ and a subgroup $H\subseteq G$. We have the following result.
\begin{proposition}
	Fix a finite group $G$ and a subgroup $H\subseteq G$. Then $\op{Ind}^G_H$ is both a left and right adjoint for $\op{Res}^G_H$.
\end{proposition}
\begin{proof}
	This proof is somewhat technical, but it's fairly direct. We have to provide the following natural transformations.
	\begin{itemize}
		\item There is a map $N\to\op{Res}^G_H\op{Ind}^G_HN$ for any $H$-module $N$. Well, this map is just given by $f\mapsto f(1)$.
		\item There is a map $\op{Res}^G_H\op{Ind}^G_HN\to N$ for any $G$-module $N$. Well, this map is just given by $f\mapsto f(1)$.
		\item There is a map $M\to\op{Ind}^G_H\op{Res}^G_HM$ for any $G$-module $M$. Well, this map is just given by $m\mapsto(g\mapsto gm)$.
		\item There is a map $\op{Ind}^G_H\op{Res}^G_HM\to M$ for any $G$-module $M$. Well, this map is just given by
		\[\sum_{gH\in G/H}gfg^{-1}.\]
		In particular, the $H$-invariance of $f$ implies that the choice of coset representative $gH$ does not matter.
	\end{itemize}
	We omit the adjunction checks.
\end{proof}
\begin{remark}
	Note that the composition
	\[M\to\op{Ind}^G_H\op{Res}^G_HM\to M\]
	is simply multiplication by $[G:H]$. In particular, if $H$ is the trivial subgroup, then the middle term vanishes, so we see that $H^\bullet(G,M)$ should be $n$-torsion.
\end{remark}
\begin{corollary} \label{cor:res-is-inj}
	Fix a finite group $G$ and a Sylow $p$-subgroup $H$. Then the map
	\[H^\bullet(G,M)[p^\infty]\xrightarrow{\op{Res}}H^\bullet(H,M)[p^\infty]\]
	is injective.
\end{corollary}
\begin{proof}
	Note the composite
	\[H^\bullet(G,M)[p^\infty]\xrightarrow{\op{Res}}H^\bullet(H,M)[p^\infty]\to H^\bullet(G,M)[p^\infty]\]
	is multiplication by $[G:H]$, which is coprime to $p$, so this composite is an isomorphism. Thus, the left map is injective.
\end{proof}
\begin{remark}
	Roughly speaking, it will be beneficial to go down to Sylow $p$-subgroups because these are solvable, so one can imagine we can then reduce to cyclic subgroups with some effort.
\end{remark}

\subsection{The Second Inequality}
We are now ready for our main theorem.
\begin{theorem}
	Fix a Galois extension of number fields $L/K$ with Galois group $G$.
	\begin{listalph}
		\item $\left[\AA_K^\times:K^\times\op N^L_K(\AA_L^\times)\right]$ is finite and divides $[L:K]$.
		\item $H^1(G,\AA_L^\times/L^\times)=0$.
		\item $H^2(G,\AA_L^\times/L^\times)$ is finite with order dividing $[L:K]$.
	\end{listalph}
\end{theorem}
\begin{remark}
	Note that $\AA_K^\times/K^\times N^L_K(\AA_L^\times)=\widehat H^0(G,\AA_L^\times/L^\times)$. To see this, we stare at the usual short exact sequence
	\[1\to L^\times\to\AA_L^\times\to\AA_L^\times/L^\times\to1.\]
	Because $H^1(G,L^\times)=0$, this gives rise to the exact sequence
	\[\frac{K^\times}{\op N^L_KL^\times}\to\frac{\AA_K^\times}{\op N^L_K\AA_L^\times}\to\widehat H^0(G,\AA_L^\times/L^\times)\to0,\]
	so the claim follows.
\end{remark}
\begin{proof}[Reductions]
	Let's provide some reductions.
	\begin{itemize}
		\item If $G$ is cyclic then the above are all equivalent. Indeed, (a) and (c) are equivalent by periodicity of cohomology. Further, we see (c) is equivalent to $\#\widehat H^0(G,\AA_L^\times/L^\times)\le n$. But this is equivalent to $\#\widehat H^1(G,\AA_L^\times/L^\times)\le1$ because $h(G,\AA_L^\times/L^\times)=n$ here. However, this last inequality is equivalent to $\widehat H^1(G,\AA_L^\times/L^\times)=0$, which is (b).
		\item We reduce to the case where $G$ is a $p$-group. Indeed, let $H\subseteq G$ be a Sylow $p$-subgroup. If we are given the theorem in the case $L/L^H$ (where here the Galois group is a $p$-group), then we conclude by restricting via \Cref{cor:res-is-inj} that
		\[\#\widehat H^i(G,\AA_L^\times/L^\times)\left[p^\infty\right]\le\#\widehat H^i(H,\AA_L^\times/L^\times)\le\left[L:L^H\right]=\#H,\]
		so the order of $p$ dividing $\widehat H^i(G,\AA_L^\times/L^\times)$ is less than or equal to the order of $p$ dividing $[L:K]$. Because the cohomology is $[L:K]$-torsion, we conclude that these are the only primes we have to worry about, so the theorem for $p$-groups (each of (a), (b), and (c)) implies the theorem in general by taking $i\in\{0,1,2\}$ by these injections.
		\item We reduce to the case where $G\cong\ZZ/p\ZZ$. Indeed, if not, by the proof of the Sylow theorems, we may assume $G$ is a $p$-group, and there is a nontrivial proper normal subgroup $H\subseteq G$ such that we have the theorem for the extensions $L/L^H$ and $L^H/K$.

		Let's start with (b). By Restriction--Inflation, we know that $H^1(G,\AA_L^\times/L^\times)=0$ will imply that
		\[H^1(G,\AA_L^\times/L^\times)\cong H^1\left(G/H,(\AA_L^\times/L^\times)^H\right)=H^1\left(G/H,\AA_{L^H}^\times/(L^H)^\times\right),\]
		which vanishes because we know the theorem on the extension $L^H/K$. To see the last equality above, we can take $H$-invariants of the exact sequence
		\[1\to L^\times\to\AA_L^\times\to\AA_L^\times/L^\times\to1\]
		to see $\AA_{L^H}^\times/(L^H)^\times\cong(\AA_L^\times/L^\times)^H$ because $H^1(H,L^\times)=0$.

		For (a), we see
		\[\AA_L^\times\supseteq K^\times\op N_K^{L^H}(\AA_{L^H}^\times)\supseteq K^\times\op N^L_K(\AA_L^\times),\]
		and the left index is appropriately bounded by $[L^H:K]$, so it suffices to show that the right index is bounded by $[L:L^H]$. Well, for our bound, we know that the index
		\[\AA_{L^H}^\times\supseteq L^{H\times}\op N^L_{L^H}(\AA_L^\times)\]
		divides $[L:L^H]$. Well, taking $\op N^{L^H}_K$ of this inclusion, we see that the index of
		\[\op N^{L^H}_K(\AA_{L^H}^\times)\supseteq\op N^{L^H}_K(L^{H\times})\op N^{L^H}_K(\AA_L^\times)\]
		will still divide $[L:L^H]$ because there is a surjection from the previous quotient to this quotient. Thus, the index of
		\[K^\times\op N^{L^H}_K(\AA_{L^H}^\times)\supseteq K^\times\op N^{L^H}_K(\AA_L^\times)\]
		still divides $[L^H:K]$ because again there is a surjection from this above quotient to this one. This finishes.
		
		Lastly, for (c), one looks at the long exact sequence and does some tricky thing.

		\item We can even reduce to the case where $\mu_p\in K$. We omit the details of this reduction. Roughly speaking, adjoining $\mu_p$ replaces $K$ with an extension coprime to $p$, so because we are interested in showing that $\left[\AA_K^\times:K^\times\op N^L_K(\AA_L^\times)\right]$ divides some smallish power of $p$, so adding in these factors coprime to $p$ do not affect the argument.
		\qedhere
	\end{itemize}
\end{proof}

\end{document}