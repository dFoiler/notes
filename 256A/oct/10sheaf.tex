% !TEX root = ../notes.tex

\documentclass[../notes.tex]{subfiles}

\begin{document}

\section{October 10}

We now shift gears to talk about quasicoherent sheaves.

\subsection{Quasicoherent Sheaves}
Fix a ringed space $(X,\OO_X)$. We quickly recall the following definition.
\oxmoduledef*
\noindent It is a fact that $\OO_X$-modules form an abelian category, though we will not bother proving this. The usual constructions for direct sum, products, the tensor product, inverse limits, and direct limits. The direct sum, the tensor product, and the direct limit require a sheafification to construct, though we will not on average be too worried about such things.

There are also direct image modules in the obvious way.
\begin{definition}[Direct image module]
	Fix a morphism $f\colon(X,\OO_X)\to(Y,\OO_Y)$ of ringed spaces. If $\mc F$ is an $\mathbb O_X$-module, then $f_*\OO_X$ remains an $\mathbb O_X$-module, so the \textit{direct image module} $f_*\mc F$ will become an $\mathbb O_Y$-module through $f^\sharp\colon\OO_Y\to f_*\OO_X$.
\end{definition}
However, pullback sheaves do change from being ``just'' $f^{-1}\mc F$.
\begin{definition}[Pullback module]
	Given an $\mathbb O_Y$-module $\mc G$, we see $f^{-1}\mc G$ is an $f^{-1}\OO_Y$-module. To make this an $\mathbb O_X$-module, we would like to use the morphism $f^\flat\colon f^{-1}\OO_Y\to\OO_X$, but this only makes $\OO_X$ into an $f^{-1}\OO_Y$-module, so we define
	\[f^*\mc G\coloneqq f^{-1}\mc G\otimes_{f^{-1}\OO_Y}\OO_X\]
	to be the \textit{pullback module} $\OO_X$-module.
\end{definition}
\begin{remark}
	The main point of the pullback module is to have the adjunction
	\[\op{Hom}_{\OO_X}(f^*\mc G,\mc F)\simeq\op{Hom}_{\OO_Y}(\mc G,f_*\mc F).\]
\end{remark}
\begin{example}
	Given a ring map $\varphi\colon B\to A$, we induce a scheme map $f\colon\Spec A\to\Spec B$. Now, there is a functor from $A$-modules $M$ to $\OO_X$-modules $\widetilde M$. Then we see that we can send such a module $\widetilde M$ back to a $B$-module as $f_*(\widetilde M)=\widetilde M$ by taking global sections and using $\varphi$, and we can send a $B$-module $N$ to $f^*(\widetilde N)=\widetilde{N\otimes_BA}$ given the obvious $A$-module structure.
\end{example}
To make the above example precise, we need to define $\widetilde M$.
\begin{definition}
	Fix an affine scheme $X\coloneqq\Spec A$ and an $A$-module $M$. Then we define the $\OO_X$-module $\widetilde M$ as a sheaf on the distinguished base
	\[\widetilde M(D(f))\coloneqq M_f,\]
	with the usual restrictions.
\end{definition}
It is somewhat clear to check that this is in fact a sheaf on the base, which gives the needed sheaf. One can also check that the stalk $\widetilde M_\mf p$ at a given prime $\mf p\in\Spec A$ is just $M_\mf p$.
\begin{example}
	From the above example, we can pass $A$ through to see that $\OO_X$ is an $\mathbb O_X$-module.
\end{example}
We now have the following.
\begin{proposition}
	Fix a ring $A$. The functor from $A$-modules to $\OO_X$-modules taking $M$ to $\widetilde M$ is exact and fully faithful. This equivalence also respects $\oplus$ and $\otimes$.
\end{proposition}
\begin{proof}
	Exactness is checked at stalks. Being fully faithful is approximately as hard as showing \autoref{thm:geoisoppalg}, where the same machinery approximately works.
\end{proof}

We are now ready to define quasicoherent sheaves.
\begin{definition}[Quasicoherent sheaf]
	Fix a scheme $(X,\OO_X)$. An $\mathbb O_X$-module $\mc F$ is \textit{quasicoherent} if and only if $X$ has an affine open cover $\{U_\alpha\}_{\alpha\in\lambda}$ such that $\mc F|_{U_\alpha}\simeq\widetilde{M_\alpha}$ for some $\OO_X(U_\alpha)$-module $M_\alpha$.
\end{definition}
Using \autoref{lem:affine-comm}, it is equivalent to saying that $\mc F|_U\simeq\widetilde M$ for some $\OO_X(U)$-module $M$, for any affine open subscheme $U\subseteq X$.

Namely, for (i), has to show that $\mc F|_U\simeq\widetilde M$ implies that $\mc F|_{U_f}\simeq\widetilde{M_f}$ for some $f\in\OO_X(U)$, which is by construction of $\widetilde M$. For (ii), one has to show that if $A\coloneqq\OO_X(U)$ is generated by some $f_i$ with $\mc F|_{U_{f_i}}\simeq\widetilde{M_i}$ where $M_i$ is an $A_{f_i}$-module gluing together to $\widetilde M$. Well, one just has to check that $\mc F$ is isomorphic to the kernel of the natural map
\[\prod_i\widetilde{M_i}\to\prod_{i,j}M_{ij}.\]

\end{document}