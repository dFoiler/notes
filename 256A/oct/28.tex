% !TEX root = ../notes.tex

\documentclass[../notes.tex]{subfiles}

\begin{document}

\section{October 26}

Today we'll focus on computing divisors for curves over a field.

\subsection{Degrees on Curves}
To discuss curves, we should probably define curves.
\begin{definition}[Curve]
	Fix a field $k$. A \textit{curve} is a one-dimensional variety over $k$.
\end{definition}
Let's begin by discussing Cartier divisors.
\begin{itemize}
	\item Let $X$ be a normal curve. For a Cartier divisor $D$, we define
	\[\deg D\coloneqq\sum_{x\in X}\nu_x(D)[k(x):k].\]
	Here, $\nu_x(D)=\nu_x(f_i)$, where $D=\{(U_i,f_i)\}$ has $x\in U_i$ for a given $i$, which makes sense because $\OO_{X,x}$ is a discrete valuation ring.
	\item If $X$ is not normal, then we should pass to its normalization $\widetilde X$ with normalization map $\iota\colon\widetilde X\to X$. Then we can pull back Cartier divisors $D$ on $X$ to divisors on $\widetilde X$ and compute the degree there.
\end{itemize}
\begin{example}
	Take $X=\PP^1_\QQ$ as a $\QQ$-curve. Then the closed point $x\coloneqq\left(x_0^2-2x_1^2\right)$ has $\deg_\QQ[x]=2$, where the $2$ contribution is coming from $[\QQ(x):\QQ]=2$.
\end{example}
And here is the degree of a Weil divisor.
\begin{itemize}
	\item For a normal curve $X$, and a Weil divisor $D=\sum_xn_x[x]$ on $X$, we again just define
	\[\deg D\coloneqq\sum_{x\in X}n_x[k(x):x].\]
	\item For general curves, apply the same normalization technique to get the definition in general.
\end{itemize}
Here is the usual check.
\begin{prop} \label{prop:deg-of-princ-vanishes}
	Fix a proper $k$-curve $X$. Then $\deg\op{div}f=0$ for any $f\in K(X)$. Thus, $\deg$ descends to a homomorphism $\op{CaCl}X\to\ZZ$.
\end{prop}
We will prove this shortly.
\begin{nex}
	With $X=\AA^1_k=\Spec k[t]$, we note that $\deg\op{div}(t)=\deg[0]=1$. The point here is that the proper condition on $X$ is necessary.
\end{nex}

\subsection{Pulling Back Divisors}
To prove \autoref{prop:deg-of-princ-vanishes}, we need to discuss what we meant when we said we're pulling back Cartier divisors.
% GW S14, V S25
\begin{remark}
	Before continuing, we say out loud some properties of flat morphisms.
	\begin{itemize}
		\item The class of flat morphisms is preserved by composition, is preserved by base change, and is affine-local on the target.
		\item If $f\colon X\to Y$ is flat, and $y=f(x)$ for some $x\in X$, then $\dim\OO_{X,x}=\dim\OO_{Y,y}+\dim\OO_{X_y,x}$.
	\end{itemize}
\end{remark}
Fix some flat morphism $f\colon X\to Y$. If $Z\subseteq Y$ is a closed integral subscheme of codimension $1$, then we note that $X\times_YZ$ is a closed subscheme of $X$. Further, by the flatness, we note that
\[\codim_X(X\times_YZ)=\codim_Y(Z)=1\]
provided that $X\times_YZ$ is nonempty. The point is that this process defines a group homomorphism $\op{Weil}(Y)\to\op{Weil}(X)$ sending some $[Z]$ to the reduced subscheme with topological space $X\times_YZ\subseteq X$, with some suitably defined multiplicities.

Namely, for an irreducible closed subscheme $W$ of $X\times_YZ$, we let the multiplicity of $[W]$ be given by choosing an affine open subscheme $U\subseteq W$. Then we take the stalk $\OO_W(U)_\eta$, where $\eta$ is the generic point, is a zero-dimensional ring and hence Artinian and hence has finite length, so we set the multiplicity to the length of $\OO_W(U)_\eta$.
\begin{example}
	If we have $X\times_YZ\simeq\Spec k[x]/\left(x^2\right)$, then our multiplicity we can see should be $2$, and the corresponding reduction is $k$.
\end{example}
\begin{example}
	If we have $X\times_YZ\simeq\Spec k[x,y]/\left(x^2\right)$, then our reduction is $k[y]$, and our multiplicity is $2$ coming from the length of $k[x,y]/\left(x^2\right)$ at the generic point $(y)$ of $k[y]$.
\end{example}
We can see from the examples that the point of our weird reduction and so on is to keep track of the differential information.

Next we want to pull back Cartier divisors, which we are promised will be less confusing.
\begin{example}
	Fix a dominant morphism $f\colon X\to Y$ of integral $k$-curves. Then a Cartier divisor made of the data $\{(U_i,f_i)\}$ on $Y$ has $f_i\in\mc K(Y)=K(Y)^\times$. But we see that $f\colon X\to Y$ gives us a map $f^\sharp\colon K(Y)\to K(X)$ because dominance forces $f$ to send the generic point of $X$ to the generic point of $Y$. As such, we will do the obvious thing and send
	\[(U_i,f_i)\mapsto\left(f^{-1}U_i,f^\sharp(f_i)\right)\]
	to define our map $f^*\colon\op{CaDiv}Y\to\op{CaDiv}X$.
\end{example}
Generalizing the example, we note that $f^\sharp\colon\OO_Y\to f_*\OO_X$ is going to induce a morphism $f^\sharp\colon\mc K_Y\to f_*\mc K_X$. To make a map on divisors, we want a condition like flatness to preserve dimension and to perhaps send irreducible components to irreducible components. Namely, we are getting a map $f^*\colon\mc K_Y/\OO_Y\to f_*(\mc K_X^*/\OO_X^*)$, which is precisely a map of the Cartier divisors.
\begin{example}
	Set $E=V_+\left(ZY^2=X^3+aXZ^2+bZ^3\right)\subseteq\PP^2_k$ to be an elliptic curve. Then the map $f\colon E\to\PP^1_k$ by $[X:Y:Z]\mapsto[Y:Z]$ (note that $E$ has no points with $Y=Z=0$) will let us pull back. For example, we can pull back the divisor $[0:1]$ of $\PP^1_k$ back to the three roots of $x^3+ax+b$ as points on our elliptic curve. Similarly, we can pull back $[1:0]$ on $\PP^1_k$ to $3[0:1:0]$.
\end{example}

\end{document}