% !TEX root = ../notes.tex

\documentclass[../notes.tex]{subfiles}

\begin{document}

\section{October 31}

Happy Halloween.

\subsection{Pulling Back on Curves}
Let's try to prove that divisors over proper curves have degree zero.
\begin{proposition}
	Fix normal integral proper $k$-curves $X$ and $Y$ and a finite morphism $f\colon X\to Y$. Then any Cartier divisor $D\in\op{CaDiv}(Y)$ have
	\[\deg f^*D=[K(X):K(Y)]\deg D.\]
\end{proposition}
\begin{proof}
	Recall that we have a group homomorphism $f^*\colon\op{CaDiv}(Y)\to\op{CaDiv}(X)$, so by linearly extending and passing to Weil divisors, we may consider $D=[y]$, where $y\in Y$ is a closed point. (Namely, codimension-$1$ integral closed subschemes of $Y$ are just closed points.)

	As such, we want to compute $f^*[y]$. As some motivation, we discuss the pull-back. Well, place $y$ in some affine open subscheme $\Spec B\subseteq Y$, and because $f$ is finite, we see that $f^{-1}(\Spec B)\subseteq X$ is affine, so set $f^{-1}(\Spec B)=\Spec A$, where $f^\sharp\colon B\to A$ makes $A$ a finitely generated $B$-module. Anyway, we approximately have
	\[f^*[y]\approx\left[f^{-1}(y)\right],\]
	where we now have to count our multiplicities. Namely, our scheme is $\Spec A/\mf m_yA$, and localizing for psychological reasons means that we're looking at
	\[\Spec\frac{(B\setminus\mf m_y)^{-1}A}{\mf m_y(B\setminus\mf m_y)^{-1}A}.\]
	Thinking more explicitly, let's say that $\mf m_1,\ldots,\mf m_n\in X$ are the points in $f^{-1}(\{y\})$, so we see
	\[f^*[y]=\sum_{i=1}^n\op{length}_{A_{\mf m_i}}(A_{\mf m_i}/\mf m_yA_{\mf m_i})[\mf m_i],\]
	where we have used the decomposition of Artin rings into a product of its localizations. Taking degrees, we want to compute
	\[\deg f^*[y]=\sum_{i=1}^n\op{length}_{A_{\mf m_i}}(A_{\mf m_i}/\mf m_yA_{\mf m_i}).\]

	We now proceed with the proof more directly. Adding together our lengths into a product, we can write
	\[\deg f^*[y]=\op{length}\frac{(B\setminus\mf m_y)^{-1}A}{\mf m_y(B\setminus\mf m_y)^{-1}A},\]
	which we claim is $[K(X):K(Y)]$ by Nakayama's lemma. Indeed, set $A'\coloneqq(B\setminus\mf m_y)^{-1}A$, which we see is an integral domain. Now, $K(X)=\op{Frac}(A')$, but we see that $A'$ is a torsion-free and the principal ideal domain $B_{\mf m_y}$, so $A'$ must be a free $B_{\mf m_y}$-module. Passing to the fraction fields recovers what the rank should be by Nakayama's lemma.
\end{proof}
It is technically okay in the above proof to work with the non-normal case. We give a few remarks to convince ourselves of this.
\begin{remark}
	In general, when $X$ is not normal, given a closed point $x\in X$ and a Cartier divisor $D\in\op{CaDiv}(X)$ represented by the data $\{(U_i,f_i)\}$, we can define the multiplicity of $x$ at $D$ by
	\[\op{mult}_xD\coloneqq\op{length}_{\OO_{X,x}}(\OO_{X,x}/f_x),\]
	where we are implicitly assuming that $f_x\in\OO_{X,x}$ and extending multiplicatively to work with the entire fraction field.
\end{remark}
\begin{example}
	Fix $X=\Spec k[x,y]/(xy)$, and we compute the multiplicity of $(x,y)$ for the divisor coming from $f=x-y$. Then we see
	\[\op{mult}_{(x,y)}(f)=\op{length}\left(\frac{k[x,y]}{(xy,x-y)}\right)_{(x,y)}=\op{length}\left(\frac{k[x]}{x^2}\right)_{(x)}=2.\]
	On the other hand, our normalization is $\widetilde X=\Spec k[x]\sqcup\Spec k[y]$, so we can just compute
	\[\op{mult}_{(x)}(f)+\op{mult}_{(y)}(f)=1+1=2,\]
	which matches.
\end{example}
\begin{example}
	One could similarly look at the cuspidal cubic $X=\Spec k[x,y]/\left(y^2=x^3\right)$ and recover a multiplicity of $1$ at $(x,y)$.
\end{example}
% L S7.1, S7.3
And here is our result.
\begin{corollary}
	Fix a normal integral proper $k$-curve $X$. For some rational section $f\in K(X)^\times$, we have
	\[\deg\op{div}(f)=0.\]
\end{corollary}
\begin{proof}
	Note we have an induced map $f\colon X\to\PP^1_k$: a global section $f$ of $X$ corresponds to a function $X\to\AA^1_k$, which can then be extended uniquely to $\PP^1_k$. Anyway, we see that $\op{div}(f)=f^*([0]-[\infty])$ by computing the pull-back, which we see has degree $0$.
\end{proof}
\begin{remark}
	We will often set $\deg f\coloneqq[K(X):K(Y)]$. The reason for this is that we can more or less think about the degree of a polynomial as the degree of the covering it introduces, which is approximately the number of roots (counted with multiplicity), which is the degree of the pull-back introduced in the proposition.
\end{remark}
Let's give a few more remarks.
\begin{remark}
	Moret--Bailly has a theorem which lets us turn Weil divisors into Cartier divisors (up to a multiple of $\QQ$). Indeed, if $X$ is normal equipped with a map $X\to\Spec\ZZ_p$ of relative dimension $1$, then Weil divisors are Cartier divisors up to a multiple of $\QQ$.
\end{remark}
\begin{remark}
	If $X$ is normal, and $D\subseteq X$ is a Cartier divisor (up to a multiple of $\QQ$) where we embed $D$ into $X$ as a Weil divisor, as well as some other subscheme $f\colon C\subseteq X$, we can think about our multiplicity of intersections of $D$ and $X$ as $f^*D$ (as a Cartier divisor).
\end{remark}
When $X$ is a projective, regular $k$-scheme, it is still possible to define the degree, but one needs to first choose a line bundle $\mc L$. For example, taking $X=\PP^n_k$ and $Y=V_+(f)$ as some closed subscheme, we define $\deg Y$ as the degree of the divisor on $\PP^1_k$ given by $Y\cap\PP^1_k$ where $\PP^1_k\not\subseteq Y$. Namely, we are defining the degree of $Y$ with its intersections with some line in ``general position.'' But lines are just the intersection of $n-1$ different hyperplanes, and hyperplanes are just divisors of $\OO_X(1)$!

Now, thinking more generally, we can think of $\deg_\mc L(Z)$, where $Z\subseteq X$ is some closed subscheme of dimension $r$, as the degree of the intersections of $Z$ with $r$ total divisors in general position. For example, if $r=\dim X$, then are defining a map $\deg_\mc L\colon\op{Weil}X\to\ZZ$. We would like this map to be nontrivial, but this requires $\mc L$ to be an ample line bundle.

\end{document}