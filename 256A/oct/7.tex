% !TEX root = ../notes.tex

\documentclass[../notes.tex]{subfiles}

\begin{document}

\section{October 7}

There may be algebraic groups on the homework. Cool.

\subsection{Some Proper Facts}
We begin by showing that projective schemes are proper.
\begin{lemma} \label{lem:elim-theory}
	Fix a ring $A$. Then the canonical projection $\pi\colon\PP^n_A\to\Spec A$ is a closed map.
\end{lemma}
\begin{proof}
	We think of $\PP^n_A$ as $\Proj A[x_0,\ldots,x_n]$; for brevity, set $R\coloneqq A[x_0,\ldots,x_n]$.
	
	Fix a closed subset of $\PP^n_A$, which we can write down as $V_+(I)$ for some homogeneous ideal $I\subseteq R$. We would like to show that $\pi(V_+(I))\subseteq\Spec A$ is closed. Suppose that $I$ is generated by the homogeneous elements of positive degree $\{f_\alpha\}_{\alpha\in\lambda}$.

	We need to describe the map $\PP^n_A\to\Spec A$. Given a prime $\mf P\in\Proj R$, we know that $\mf P$ does not contain $(x_0,\ldots,x_n)$, so $\mf P\in D_+(x_i)$ for some $i$. From the $\Proj$ construction, we have isomorphisms
	\[D_+(x_i)\simeq\Spec(R_{x_i})_0\simeq\Spec\frac{A[x_{0/i},\ldots,x_{n/i}]}{(x_{i/i}-1)},\]
	which we now see has an evident canonical projection down to $\Spec A$. Thus, we are taking our prime $\mf P\in D_+(x_i)$ over to $(\mf PR_{x_i})_0$ down to $\pi(\mf P)=A\cap(\mf PR_{x_i})_0$ in $\Spec A$.

	We thus claim that $\mf p\in\pi(V_+(I))$ if and only if the polynomials in $I$ have a common root in $\PP^n_{k(\mf p)}$.\todo{} This essentially holds by unwinding the map above and looking at constants.

	Thus, $\mf p\in\pi(V_+(I))$ is equivalent to saying $V_+(\overline I)\subseteq\Proj k(\mf p)[x_0,\ldots,x_n]$ is nonempty; for brevity, set $R^\mf p\coloneqq k(\mf p)[x_0,\ldots,x_n]$. In other words, we are asking for
	\[V_+(\overline I)\not\subseteq V((x_0,\ldots,x_n)).\]
	Well, by \autoref{prop:proj-nullstellensatz}, this is equivalent to asking for
	\[\rad\overline I\not\supseteq\rad(x_0,\ldots,x_n)=(x_0,\ldots,x_n).\]
	This is equivalent to asking for $x_i\notin\rad\overline I$ for each $i$, which is equivalent to asking for $x_i^N\notin\overline I$ for all $N$ for some $i$. In particular, this implies that
	\[(x_0,\ldots,x_n)^N\not\subseteq\overline I\]
	for each $N$ by looking at the $x_i^N$ term; conversely, if $x_i^N\in\overline I$ for some $N$ for each $i$, then we can find $N$ large enough so that $x_i^N\in\overline I$ for each $i$, so $(x_0,\ldots,x_n)^{Nn}\subseteq\overline I$.

	Now, the condition we're looking at is equivalent to requiring
	\[R^\mf p_N\not\subseteq\overline I\]
	for each $N$. However, now that we see we're focusing on a degree-$N$ part, we see is equivalent to requiring
	\[R^\mf p_N\not\subseteq\sum_{\alpha\in\lambda}\overline{f_\alpha} R^\mf p_{N-\deg \overline{f_\alpha}},\]
	which means we're asking for the map
	\[\bigoplus_{\alpha\in\lambda}R^\mf p_{N-\deg \overline{f_\alpha}}\to R^\mf p_N\]
	by $(r_\alpha)_\alpha\mapsto\sum_\alpha \overline{f_\alpha} r_\alpha$ to not be surjective. Because this is a linear transformation of $k(\mf p)$-vector spaces, this is equivalent to asking for each of the $\dim R^\mf p_N\times\dim R^\mf p_N$-minors of the corresponding matrix to have vanishing determinant. Combining everything over all the primes has carved out a closed subset of $\Spec A$, finishing.\todo{}
\end{proof}
% We may give $S$ an affine open cover. By taking base-changes, we only have to show that $\pi\colon\PP^n_B\to\Spec B$ is a closed map for each ring $B$, but by taking enough pre-images.

% Well, set $R\coloneqq B[X_0,\ldots,X_n]$ and some homogeneous ideal $I\subseteq R$ so that our closed set is $Z\coloneqq V(I)$. We would like to show that $\Spec B\setminus(\pi(V(I)))$ is open. Well, for $\mf q\in\Spec B$, we see that
% \[Z\cap\pi^{-1}(\{\mf q\})=V(I\otimes_Vk(\mf q)).\]
% So $\mf q\in\Spec B\setminus\pi(V(I))$ is equivalent to $V(I\otimes_Bk(\mf q))$ if and only if the radical $\rad I\otimes_Bk(\mf q)$ contains $B_+\otimes_Bk(\mf q)$.

% At this point we want to use Nakayama's lemma. In particular, this is equivalent to $I\otimes_Bk(\mf q)\supseteq R_m\otimes k(\mf q)$ for some positive $m$, which is equivalent to requiring $(R/I)_m\otimes_Bk(\mf q)$ vanishing for some $m>0$, which by Nakayama's lemma is equivalent to requiring $(R/I)_m\otimes\OO_{\Spec B,\mf q}$ to vanish, which is equivalent to having some $f\in B$ with $\mf q\in D(f)$ such that $(R/I)_m\otimes B_f=0$ locally. In total, this is equivalent to having $f\in B$ with $\mf q\in D(f)$ being a subset of $(\Spec B)\setminus\pi(V(I))$, so we have been given the required open sett around $\mf q$.
\begin{proposition} \label{prop:proj-is-proper}
	Fix a scheme $S$. Then the canonical projection $\pi\colon\PP^n_S\to S$ is proper.
\end{proposition}
\begin{proof}
	We proceed in steps.
	\begin{enumerate}
		\item \label{item:reduce-to-affine-proj-proper} We reduce to the affine case. Because being proper is affine-local on the target by \autoref{lem:proper-is-reasonable}, it suffices to show that the restriction $\pi\colon\pi^{-1}U\to U$ is proper for any affine open subscheme $U\subseteq S$, where $U\cong\Spec A$. This allows us to draw the diagram
		% https://q.uiver.app/?q=WzAsOCxbMywwLCJcXFBQXm5fXFxaWiJdLFszLDEsIlxcU3BlY1xcWloiXSxbMiwwLCJcXFBQXm5fUyJdLFsyLDEsIlMiXSxbMSwwLCJcXHBpXnstMX1VIl0sWzEsMSwiVSJdLFswLDEsIlxcU3BlYyBBIl0sWzAsMCwiXFxQUF5uX0EiXSxbMiwzLCJcXHBpIl0sWzcsNCwiIiwwLHsic3R5bGUiOnsiYm9keSI6eyJuYW1lIjoiZGFzaGVkIn19fV0sWzQsMiwiIiwwLHsic3R5bGUiOnsidGFpbCI6eyJuYW1lIjoiaG9vayIsInNpZGUiOiJ0b3AifX19XSxbMiwwXSxbMywxXSxbMCwxXSxbNCw1XSxbNSwzLCIiLDEseyJzdHlsZSI6eyJ0YWlsIjp7Im5hbWUiOiJob29rIiwic2lkZSI6InRvcCJ9fX1dLFs2LDVdLFs3LDZdXQ==&macro_url=https%3A%2F%2Fraw.githubusercontent.com%2FdFoiler%2Fnotes%2Fmaster%2Fnir.tex
		\[\begin{tikzcd}
			{\PP^n_A} & {\pi^{-1}U} & {\PP^n_S} & {\PP^n_\ZZ} \\
			{\Spec A} & U & S & \Spec\ZZ
			\arrow["\pi", from=1-3, to=2-3]
			\arrow[dashed, from=1-1, to=1-2]
			\arrow[hook, from=1-2, to=1-3]
			\arrow[from=1-3, to=1-4]
			\arrow[from=2-3, to=2-4]
			\arrow[from=1-4, to=2-4]
			\arrow[from=1-2, to=2-2]
			\arrow[hook, from=2-2, to=2-3]
			\arrow[from=2-1, to=2-2]
			\arrow[from=1-1, to=2-1]
		\end{tikzcd}\]
		where the rightmost square is a pullback by construction of $\PP^n_S$, and the middle square is a pullback by \autoref{lem:open-fp}, so the rightmost two squares make a pullback square by \autoref{lem:smalls-to-big-pullback}.
		
		Thus, because $\Spec\ZZ$ is final, the fact that we have morphisms $\PP^n_A\to\PP^n_\ZZ$ and $\PP^n_A\to\Spec A\to U$ at all forces a dashed arrow above making the diagram commute, and we note that the total rectangle is a pullback by \autoref{exe:proj-by-base-change}, so the leftmost square is a pullback by \autoref{lem:big-to-small-square}.
		
		In total, because isomorphisms are preserved by base-change, we see our morphism $\PP^n_A\to\pi^{-1}U$ is an isomorphism, so showing that the morphism $\pi^{-1}U\to U$ is proper is equivalent to showing that $\PP^n_A\to\Spec A$ is proper. Namely, isomorphisms are proper by \autoref{ex:finite-is-proper}, and proper morphisms are preserved by composition by \autoref{lem:proper-is-reasonable}.
		
		As such, we relabel our variables so that $S\coloneqq\Spec A$, and we are looking at the canonical projection $\pi\colon\PP^n_A\to\Spec A$.
		
		\item We get rid of our easy adjectives. Notably, this morphism is already separated by \autoref{prop:proj-is-sep}, locally of finite type by \autoref{ex:proj-locally-ft}, and quasicompact by \autoref{ex:proj-is-qc}. It remains to show that $\pi$ is universally closed.
	
		\item We describe what being universally closed requires. By taking a base-change morphism $\varphi\colon S\to\Spec A$, we note that we induce the commutative diagram
		% https://q.uiver.app/?q=WzAsNixbMiwxLCJcXFNwZWNcXFpaIl0sWzEsMSwiXFxTcGVjIEEiXSxbMSwwLCJcXFBQXm5fQSJdLFswLDEsIlMiXSxbMCwwLCJcXFBQXm5fUyJdLFsyLDAsIlxcUFBebl9cXFpaIl0sWzEsMF0sWzIsMV0sWzQsMiwiIiwwLHsic3R5bGUiOnsiYm9keSI6eyJuYW1lIjoiZGFzaGVkIn19fV0sWzQsM10sWzMsMV0sWzUsMF0sWzIsNV1d&macro_url=https%3A%2F%2Fraw.githubusercontent.com%2FdFoiler%2Fnotes%2Fmaster%2Fnir.tex
		\[\begin{tikzcd}
			{\PP^n_S} & {\PP^n_A} & {\PP^n_\ZZ} \\
			S & {\Spec A} & \Spec\ZZ
			\arrow[from=2-2, to=2-3]
			\arrow[from=1-2, to=2-2]
			\arrow[dashed, from=1-1, to=1-2]
			\arrow[from=1-1, to=2-1]
			\arrow[from=2-1, to=2-2]
			\arrow[from=1-3, to=2-3]
			\arrow[from=1-2, to=1-3]
		\end{tikzcd}\]
		where the right square is a pullback by \autoref{exe:proj-by-base-change}, the dashed arrow is induced by $\varphi\colon S\to\Spec A$ and the canonical projection $\PP^n_S\to\PP^n_\ZZ$, the total square is a pullback by construction of $\PP^n_\ZZ$, and so the left square is a pullback by \autoref{lem:big-to-small-square}.

		Namely, we want to show that the canonical projection $\PP^n_S\to S$ is closed, for any scheme $S$.

		\item We reduce to the affine case again. Being closed is affine-local on the target by \autoref{lem:closed-is-local-target}, is preserved by composition by \autoref{lem:closed-morphism-comp}, and contains isomorphisms by \autoref{ex:iso-is-closed-map}. Thus, we can replace the word ``proper'' with ``closed'' everywhere in \autoref{item:reduce-to-affine-proj-proper} so that it suffices to show that the canonical projection $\pi\colon\PP^n_A\to\Spec A$ is closed. However, this is true by \autoref{lem:elim-theory}.
		\qedhere
	\end{enumerate}
\end{proof}
Approximately speaking, we expect proper to mean compact. Here's an example of this.
\begin{proposition}
	Fix an algebraically closed field $k$ and a proper integral $k$-scheme $X$. Then, for any affine $k$-scheme $Y$, every $k$-morphism $f\colon X\to Y$ is constant: there is $y\in Y(k)$ such that $f$ factors through $y$.
\end{proposition}
\begin{remark}
	It follows from \autoref{thm:chev} that, given a field $k$ and an irreducible scheme $X$, then the any map $X_k\to\AA^1_k$ either has image of a point or we have a full open embedding. Technically, we only need $X$ to be (geometrically) connected for this to be true.
\end{remark}
\begin{proof}
	Because $Y$ is an affine $k$-scheme, we can embed $Y\into\AA^I_k$ for some index set $I$, but then we can project down to $\AA^1_k$ as well; by looking at each individual projection, we will be able to get $X\to Y$ to factor through a point we saw from each projection. As such, we get to reduce to the case where $Y=\AA^1_k$ by later pulling everything back through these maps. Now, we have the following diagram.
	% https://q.uiver.app/?q=WzAsMyxbMCwwLCJYIl0sWzIsMCwiXFxBQV4xX2siXSxbMSwxLCJcXFNwZWMgayJdLFswLDFdLFsxLDIsIlxcdGV4dHtzZXBhcmF0ZWR9Il0sWzAsMiwiXFx0ZXh0e3Byb3Blcn0iLDJdXQ==&macro_url=https%3A%2F%2Fraw.githubusercontent.com%2FdFoiler%2Fnotes%2Fmaster%2Fnir.tex
	\[\begin{tikzcd}
		X && {\AA^1_k} \\
		& {\Spec k}
		\arrow[from=1-1, to=1-3]
		\arrow["{\text{separated}}", from=1-3, to=2-2]
		\arrow["{\text{proper}}"', from=1-1, to=2-2]
	\end{tikzcd}\]
	It follows from that the map $X\to\AA^1_k$ is proper by chasing our adjectives around. However, drawing the larger diagram
	% https://q.uiver.app/?q=WzAsNCxbMCwwLCJYIl0sWzIsMCwiXFxQUF4xX2siXSxbMSwxLCJcXFNwZWMgayJdLFsxLDAsIlxcQUFeMV9rIl0sWzEsMl0sWzAsMl0sWzAsM10sWzMsMV1d&macro_url=https%3A%2F%2Fraw.githubusercontent.com%2FdFoiler%2Fnotes%2Fmaster%2Fnir.tex
	\[\begin{tikzcd}
		X & {\AA^1_k} & {\PP^1_k} \\
		& {\Spec k}
		\arrow[from=1-3, to=2-2]
		\arrow[from=1-1, to=2-2]
		\arrow[from=1-1, to=1-2]
		\arrow[from=1-2, to=1-3]
	\end{tikzcd}\]
	tells us that the composite $X\to\PP^1_k$ is proper, so the image $\pi(X)$ is closed in $\PP^1_k$. However, $\AA^1_k$ is not closed in $\PP^1_k$, so because $X$ is connected (it's integral), we must have $\pi(X)$ to be a point in $\PP^1_k$ and thus factor through a point $y\in\AA^1_k$. Because $X$ is reduced (it's integral), we see that our morphism $f$ will factor through $X\to\Spec k(y)$.

	To see this last claim, we note that we already have a map $X\to\AA^1_k$ factoring through $y$ topologically, so because $\Spec k(y)$ is the reduced closed subscheme associated to $\{y\}$, our morphism on the level of sheaves must also be okay. Namely, on the level of sheaves we are looking at a map
	\[\OO_{\AA^1_k,y}\to\OO_X(X)\]
	where $\mf m_y$ is going to $\OO_X(X)$.
\end{proof}

\subsection{The Valuative Criterion}
Here is our result.
\begin{restatable}{theorem}{valuativecrit}
	Fix a scheme morphism $f\colon X\to Y$ of finite type, where $Y$ is locally Noetherian. Then the following are equivalent.
	\begin{listroman}
		\item $f$ is separated/universally closed/proper.
		\item For any discrete valuation ring $A$ with fraction field $K$ with diagram
		% https://q.uiver.app/?q=WzAsNCxbMCwwLCJcXFNwZWMgSyJdLFswLDEsIlxcU3BlYyBBIl0sWzEsMCwiWCJdLFsxLDEsIlkiXSxbMiwzLCJmIl0sWzAsMiwidSciXSxbMSwzLCJ2Il0sWzAsMSwiaiJdLFsxLDIsIiIsMSx7InN0eWxlIjp7ImJvZHkiOnsibmFtZSI6ImRhc2hlZCJ9fX1dXQ==&macro_url=https%3A%2F%2Fraw.githubusercontent.com%2FdFoiler%2Fnotes%2Fmaster%2Fnir.tex
		\[\begin{tikzcd}[ampersand replacement=\&]
			{\Spec K} \& X \\
			{\Spec A} \& Y
			\arrow["f", from=1-2, to=2-2]
			\arrow["{u'}", from=1-1, to=1-2]
			\arrow["v", from=2-1, to=2-2]
			\arrow["j", from=1-1, to=2-1]
			\arrow[dashed, from=2-1, to=1-2]
		\end{tikzcd}\]
		will induce at most one dashed arrow/at least one dashed arrow/exactly one dashed arrow.
	\end{listroman}
\end{restatable}
\begin{example}
	Checking the condition for $A\coloneqq\ZZ_p$ and $Y\coloneqq\Spec\ZZ_p$, we are essentially saying that $X(\ZZ_p)=\QQ_p$.
\end{example}
\begin{proof}[Proof of the easier direction]
	We begin with the easier direction. To show that (i) implies (ii), we allow $A$ to be any valuation ring. We have two claims; namely, proper follows by combining separated with universally closed.
	\begin{itemize}
		\item Separated: recall that if we have two scheme morphisms $\alpha,\beta\colon X'\to Y'$ where $X'$ is reduced and $Y'$ separated, then agreeing on a dense open subset requires them to be identified. In our case, we look at
		% https://q.uiver.app/?q=WzAsMyxbMCwwLCJcXFNwZWMgQSJdLFsyLDAsIlgiXSxbMSwxLCJZIl0sWzEsMiwiZiJdLFswLDEsIlxcd2lkZXRpbGRlIGZfMSIsMCx7Im9mZnNldCI6LTF9XSxbMCwyXSxbMCwxLCJcXHdpZGV0aWxkZSBmXzIiLDIseyJvZmZzZXQiOjF9XV0=&macro_url=https%3A%2F%2Fraw.githubusercontent.com%2FdFoiler%2Fnotes%2Fmaster%2Fnir.tex
		\[\begin{tikzcd}
			{\Spec A} && X \\
			& Y
			\arrow["f", from=1-3, to=2-2]
			\arrow["{\widetilde f_1}", shift left=1, from=1-1, to=1-3]
			\arrow[from=1-1, to=2-2]
			\arrow["{\widetilde f_2}"', shift right=1, from=1-1, to=1-3]
		\end{tikzcd}\]
		where we see that $\Spec A$ is reduced because it's a valuation ring, and $X$ is already separated. Thus, agreeing at the generic point (which is $\Spec K$) requires $\widetilde f_1=\widetilde f_2$. So there is at most one lift.
		\item Universally closed: to begin, we note that we have a fiber product diagram
		% https://q.uiver.app/?q=WzAsNSxbMCwwLCJcXFNwZWMgSyJdLFsxLDEsIlxcU3BlYyBBXFx0aW1lc19ZWCJdLFsyLDEsIlgiXSxbMSwyLCJcXFNwZWMgQSJdLFsyLDIsIlkiXSxbMCwyLCIiLDAseyJjdXJ2ZSI6LTJ9XSxbMCwzLCIiLDIseyJjdXJ2ZSI6Mn1dLFsxLDNdLFsxLDJdLFsyLDRdLFszLDRdLFswLDEsIiIsMSx7InN0eWxlIjp7ImJvZHkiOnsibmFtZSI6ImRhc2hlZCJ9fX1dXQ==&macro_url=https%3A%2F%2Fraw.githubusercontent.com%2FdFoiler%2Fnotes%2Fmaster%2Fnir.tex
		\[\begin{tikzcd}
			{\Spec K} \\
			& {\Spec A\times_YX} & X \\
			& {\Spec A} & Y
			\arrow[curve={height=-12pt}, from=1-1, to=2-3]
			\arrow[curve={height=12pt}, from=1-1, to=3-2]
			\arrow[from=2-2, to=3-2]
			\arrow[from=2-2, to=2-3]
			\arrow[from=2-3, to=3-3]
			\arrow[from=3-2, to=3-3]
			\arrow[dashed, from=1-1, to=2-2]
		\end{tikzcd}\]
		which almost gets us what we want. Note that the left projection is closed because closed embeddings are preserved by base change.
		
		Notably, a lift $\widetilde v\colon\Spec A\to X$ is the same as asking for a map $u'\colon\Spec K\to(\Spec A)\times_YX$ agreeing with the projections everywhere. Well, choose the correct point $x\in\Spec A\times_YX$ to be the image of $u'(\Spec K)$, let $X'$ be its Zariski closure, and give it the unique reduced subscheme structure.

		Now, we have a composite
		\[X'\to\Spec A\times_YX\to\Spec A\]
		which is a closed morphism in total and so has image closed in $\Spec A$. However, the image must contain $\Spec K$ by construction, so it follows that the composite map is surjective. As such, we find $x'\in X'$ which we will have to the closed point in $\Spec A$, and on the level of sheaves we define
		\[A\into\OO_{X',x'}\to\OO_{X',x}\to K,\]
		and this last map is an isomorphism: the ring $\OO_{X',x}$ is a field because $x$ is the generic point, and it contains $A$ and lives inside the fraction field of $A$, so we must have actually $\OO_{X',x}\simeq K$.

		In total, we have a map
		\[A\into\OO_{X',x'}\into K,\]
		but because $\OO_{X',x'}$ is a local ring with two maps, we conclude that we will also have $A=\OO_{X',x'}$. Thus, we have managed to define a scheme morphism from $\Spec A$ to $X$.
		\qedhere
	\end{itemize}
\end{proof}
The case of curves has a nice application.
\begin{cor}
	In the case of curves over a field $k$, let $Y$ be a proper $k$-scheme and $X$ is a normal integral $k$-variety with dimension $1$ (i.e., $X$ is a curve). Then the above result tells us that any rational map $X\dashrightarrow Y$ will extend to a unique morphism of schemes.
\end{cor}
\begin{proof}
	We already understand the uniqueness, so we focus on existence. Notably, for an open dense subset $U\subseteq X$, we know that $X\setminus U$ has dimension $0$ and is therefore finite. Now, for any $x\in X\setminus U$, it happens that $\OO_{X,x}$ is one-dimensional and normal and therefore a discrete valuation ring, so the above result lets us extend $X\dashrightarrow Y$ up to a morphism from $\OO_{X,x}$, but because $Y$ is of finite type, we can actually extend to an open neighborhood around $x$ by bounding our denominators, so inductively repeating this process finishes.
\end{proof}
\begin{corollary}
	There is an equivalence of categories between normal geometric integral projective curves over a field $k$ (equipped with dominant maps) and finitely generated field extensions $K/k$ of transcendence degree $1$.
\end{corollary}
\begin{proof}
	Rational dominant maps correspond to field extensions, and above we have seen that proper normal maps will automatically extend to full scheme morphisms. In the other direction, a finitely generated $K$-algebra gives a quasiprojective curve over $k$, so taking the Zariski closure in a sufficiently large $\PP^n_k$ gets an actually projective curve, and lastly taking normalization grants us the actually normal curve.
\end{proof}
Next class we will finish the proof of the valuative criterion.
\begin{remark}
	The reason we care about the harder direction of the valuative criterion is that, in life, a scheme morphism $\pi\colon X\to Y$ will be thinking about $X$ as a moduli space over a base $Y$, where the valuative criterion has some actually geometric meaning. (Moduli spaces would be a good topic for the term paper.)
\end{remark}
% GW B.13--B.14

\end{document}