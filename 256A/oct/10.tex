% !TEX root = ../notes.tex

\documentclass[../notes.tex]{subfiles}

\begin{document}

\section{October 10}

Today we finish the proof of the valuative criterion.

\subsection{The Valuative Criterion}
Here is our statement, from last class.
\valuativecrit*
\noindent We showed that (i) implies (ii) last class. Today we will sketch that (ii) implies (i), but we are unlikely to actually use the result crucially during the class.
\begin{remark}
	This does not mean that the valuative criterion is useless. If we have a variety which we don't know is immediately projective, then the valuative criterion may be quite useful; e.g., we might want to look at moduli spaces.
\end{remark}
\begin{proof}[Proof of the harder direction]
	We show (ii) implies (i). As before, we have two claims.
	\begin{itemize}
		\item Separated: being separated is affine-local on the target, so we may assume that $Y$ is affine and in particular Noetherian. Because $f\colon X\to Y$ is of finite type, this tells us that $X$ is a Noetherian scheme.
		
		We want to show that the image of $\Delta_f\colon X\to X\times_YX$ is closed, but because we are working with Noetherian schemes, we may use \autoref{thm:chev} to say that $\Delta_f(X)$ is at least constructible, so it suffices to show that $\Delta_f(X)$ is stable under specialization.

		Now, for some $z\in\Delta_f(X)$, suppose $z'\in\overline{\{z\}}$, and we want to show $z'\in\Delta_f(X)$. We now claim that there is a discrete valuation ring $A$ with $\Spec A=\{\eta,s\}$ (here, $\eta$ is generic, and $s$ is closed) and a scheme morphism $u'\colon\Spec A\to X\times_YX$ with $u'(\eta)=z$ and $u'(s)=z'$. To see this finishes the proof, we draw the diagram
		% https://q.uiver.app/?q=WzAsNSxbMCwwLCJcXFNwZWMgQSJdLFsxLDEsIlhcXHRpbWVzX1lYIl0sWzIsMSwiWCJdLFsxLDIsIlgiXSxbMiwyLCJZIl0sWzEsMiwiXFxwaV8yIiwyXSxbMSwzLCJcXHBpXzEiXSxbMiw0LCJmIl0sWzMsNCwiZiIsMl0sWzEsNCwiIiwxLHsic3R5bGUiOnsibmFtZSI6ImNvcm5lciJ9fV0sWzAsMSwidSciLDFdLFswLDMsIiIsMCx7ImN1cnZlIjoyLCJzdHlsZSI6eyJib2R5Ijp7Im5hbWUiOiJkYXNoZWQifX19XSxbMCwyLCIiLDAseyJjdXJ2ZSI6LTIsInN0eWxlIjp7ImJvZHkiOnsibmFtZSI6ImRhc2hlZCJ9fX1dXQ==&macro_url=https%3A%2F%2Fraw.githubusercontent.com%2FdFoiler%2Fnotes%2Fmaster%2Fnir.tex
		\[\begin{tikzcd}
			{\Spec A} \\
			& {X\times_YX} & X \\
			& X & Y
			\arrow["{\pi_2}"', from=2-2, to=2-3]
			\arrow["{\pi_1}", from=2-2, to=3-2]
			\arrow["f", from=2-3, to=3-3]
			\arrow["f"', from=3-2, to=3-3]
			\arrow["\lrcorner"{anchor=center, pos=0.125}, draw=none, from=2-2, to=3-3]
			\arrow["{u'}"{description}, from=1-1, to=2-2]
			\arrow[curve={height=12pt}, dashed, from=1-1, to=3-2]
			\arrow[curve={height=-12pt}, dashed, from=1-1, to=2-3]
		\end{tikzcd}\]
		and note that $\pi_1\circ u'(\eta)=\pi_2 \circ u'(\eta)$ because this is $z$, so we conclude that $\pi_1\circ u'=\pi_2\circ u'$ by checking the uniqueness on (ii). So we see that the morphism $u'\colon\Spec A\to\Delta_f(X)$ must give $z'\in\Delta_f(X)$ by lifting the corresponding specialization from $\eta$.

		Now, to prove the claim, we appeal to commutative algebra. Well, set $Z\coloneqq X\times_YX$, and we consider $\OO_{Z,z'}/\mf m_z$, which comes with it attached an embedding into $Z$. This is a local integral domain with generic point $z$ and closed point $z'$. As an aside, if $\OO_{Z,z'}/\mf m_z$ is one-dimensional, then $B$ we can take a normalization to make $B$ normal, so $B$ is regular (it's a one-dimensional, Noetherian, normal ring), so $B$ is a discrete valuation ring (it's a regular local ring of dimension $1$).

		So we are mostly interested in trying to force ourselves into a one-dimensional case. One way to finish is by specializing one point at a time to find a one-dimensional subscheme of $\Spec\OO_{Z,z'}/\mf m_z$, which can be found in \cite{rising-sea}. Another way to finish is by blowing up: one can blow up some local question like $k[x,y]_{(x,y)}$ by setting $u=x/y$ and $v=y$ to embed into $k[u,v]_{(v)}$, which turns our closed points into codimension-$1$ subschemes (``divisors'').
		\item Universally closed: read Hartshorne.
		\qedhere
	\end{itemize}
\end{proof}
\begin{remark}
	Without the Noetherian assumption, the claim from the proof of the separated case will hold if we have a general valuation ring instead of just a discrete valuation ring. Indeed, starting from $B\coloneqq\OO_{Z,z'}/\mf m_z$, we find a maximal local ring $A$ fitting into
	\[B\subseteq A\subseteq\op{Frac}B\]
	to get a valuation ring by Zorn's lemma, and this will do the trick.
\end{remark}

\end{document}