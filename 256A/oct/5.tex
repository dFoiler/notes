% !TEX root = ../notes.tex

\documentclass[../notes.tex]{subfiles}

\begin{document}

\section{October 5}

We continue.

\subsection{Scheme Equalizers}
Last time we were in the middle of the following proposition.
\gluerats*
\noindent Before going into the proof, we have the following lemma.
\begin{definition}[Equalizer]
	Given objects $X$ and $Y$ with morphisms $\alpha,\beta\colon X\to Y$, we can define the \textit{equalizer} as the limit of the following diagram.
	% https://q.uiver.app/?q=WzAsMixbMCwwLCJYIl0sWzEsMCwiWSJdLFswLDEsIlxcYWxwaGEiLDAseyJvZmZzZXQiOi0xfV0sWzAsMSwiXFxiZXRhIiwyLHsib2Zmc2V0IjoxfV1d&macro_url=https%3A%2F%2Fraw.githubusercontent.com%2FdFoiler%2Fnotes%2Fmaster%2Fnir.tex
	\[\begin{tikzcd}
		X & Y
		\arrow["\alpha", shift left=1, from=1-1, to=1-2]
		\arrow["\beta"', shift right=1, from=1-1, to=1-2]
	\end{tikzcd}\]
\end{definition}
\begin{example}
	In the category $\mathrm{Set}$, we have
	\[\op{eq}(\alpha,\beta)=\{x\in X:\alpha(x)=\beta(x)\},\]
	hence justifying the name ``equalizer.''
\end{example}
\begin{remark} \label{rem:eq-by-pullback}
	It turns out that being an equalizer is equivalent to fitting into the following pullback square.
	% https://q.uiver.app/?q=WzAsNCxbMCwwLCJcXG9we2VxfShcXGFscGhhLFxcYmV0YSkiXSxbMSwwLCJZIl0sWzAsMSwiWCJdLFsxLDEsIllcXHRpbWVzX1hZIl0sWzAsMV0sWzEsM10sWzAsMl0sWzIsMywiKFxcYWxwaGEsXFxiZXRhKSJdLFswLDMsIiIsMSx7InN0eWxlIjp7Im5hbWUiOiJjb3JuZXIifX1dXQ==&macro_url=https%3A%2F%2Fraw.githubusercontent.com%2FdFoiler%2Fnotes%2Fmaster%2Fnir.tex
	\[\begin{tikzcd}
		{\op{eq}(\alpha,\beta)} & Y \\
		X & {Y\times Y}
		\arrow[from=1-1, to=1-2]
		\arrow[from=1-2, to=2-2]
		\arrow[from=1-1, to=2-1]
		\arrow["{(\alpha,\beta)}", from=2-1, to=2-2]
		\arrow["\lrcorner"{anchor=center, pos=0.125}, draw=none, from=1-1, to=2-2]
	\end{tikzcd}\]
\end{remark}
With \autoref{rem:eq-by-pullback}, we can define equalizers of schemes.
\begin{lemma}
	Fix scheme morphisms $\alpha,\beta\colon X'\to Y$ where $Y$ is separated. Then the canonical map $\op{eq}(\alpha,\beta)\to X'$ is a closed embedding.
\end{lemma}
\begin{proof}
	Use the fact that closed embeddings are preserved by base change, combined with the pullback square \autoref{rem:eq-by-pullback}.
\end{proof}
\begin{remark}
	One can use the functor of points interpretation to show that the $T$-points of $\op{eq}(\alpha,\beta)$ are the $T$-points of $X'$ such that $\alpha_T(x)=\beta_T(x)$.
\end{remark}
We are now ready to prove \autoref{prop:glue-rats}.
\begin{proof}[Proof of \autoref{prop:glue-rats}]
	The fact that $\alpha$ and $\beta$ are representatives of the same map promises an open subscheme $V$ of $\op{eq}(\alpha,\beta)\subseteq U\cap U'$ such that $\alpha|_V=\beta|_V$. However, it follows that $\ker(\alpha,\beta)=U\cap U'$ because $U\cap U'$ is reduced (in particular, this gives the uniqueness of the closed subscheme structure, and we have shown that this closed subscheme topologically contains an open dense subset of $U\cap U'$ and therefore must be equal).
\end{proof}

\subsection{Graphs}
Given a morphism $f\colon X\to Y$, we can try to define its graph. Morally, it should be the morphism induced by the following diagram.
% https://q.uiver.app/?q=WzAsNCxbMSwxLCJYXFx0aW1lcyBZIl0sWzIsMSwiWSJdLFsxLDIsIlgiXSxbMCwwLCJYIl0sWzMsMCwiXFxnYW1tYV9mIiwxLHsic3R5bGUiOnsiYm9keSI6eyJuYW1lIjoiZGFzaGVkIn19fV0sWzMsMiwiIiwxLHsiY3VydmUiOjIsImxldmVsIjoyLCJzdHlsZSI6eyJoZWFkIjp7Im5hbWUiOiJub25lIn19fV0sWzMsMSwiZiIsMCx7ImN1cnZlIjotMn1dLFswLDJdLFswLDFdXQ==&macro_url=https%3A%2F%2Fraw.githubusercontent.com%2FdFoiler%2Fnotes%2Fmaster%2Fnir.tex
\[\begin{tikzcd}
	X \\
	& {X\times Y} & Y \\
	& X
	\arrow["{\gamma_f}"{description}, dashed, from=1-1, to=2-2]
	\arrow[curve={height=12pt}, Rightarrow, no head, from=1-1, to=3-2]
	\arrow["f", curve={height=-12pt}, from=1-1, to=2-3]
	\arrow[from=2-2, to=3-2]
	\arrow[from=2-2, to=2-3]
\end{tikzcd}\]
\begin{remark}
	If $Y$ is separated, one can use the pullback square
	% https://q.uiver.app/?q=WzAsNCxbMCwwLCJYIl0sWzAsMSwiWFxcdGltZXMgWSJdLFsxLDAsIlkiXSxbMSwxLCJZXFx0aW1lcyBZIl0sWzIsM10sWzAsMV0sWzEsM10sWzAsMl0sWzAsMywiIiwxLHsic3R5bGUiOnsibmFtZSI6ImNvcm5lciJ9fV1d&macro_url=https%3A%2F%2Fraw.githubusercontent.com%2FdFoiler%2Fnotes%2Fmaster%2Fnir.tex
	\[\begin{tikzcd}
		X & Y \\
		{X\times Y} & {Y\times Y}
		\arrow[from=1-2, to=2-2]
		\arrow[from=1-1, to=2-1]
		\arrow[from=2-1, to=2-2]
		\arrow[from=1-1, to=1-2]
		\arrow["\lrcorner"{anchor=center, pos=0.125}, draw=none, from=1-1, to=2-2]
	\end{tikzcd}\]
	to show that the graph morphism is a closed embedding.
\end{remark}
There are such properties about the graph morphism that we could prove using the same pullback square.

\subsection{A Non-reduced Example}
Let's quickly talk about why we're adding in the reduced condition. Take $X'\coloneqq\Spec k[x,y]/\left(x^2,xy\right)$, which is intuitively the $y$-axis of $\AA^2_k$ with some small differential information in the $x$ direction. Also, we set $Y\coloneqq k[t]$, and can build two scheme morphisms by
\[\arraycolsep=1.4pt\begin{array}{cccc}
	& k[t] &\to& k[x,y]/\left(x^2,xy\right) \\
	f_1\colon & t &\mapsto& y  \\
	f_2\colon & t &\mapsto& x+y
\end{array}\]
has $f_1|_{V((x))}=f_2|_{V((x))}$. Now, we can see that $V((x))$ is open and dense in $X'$ (it contains $D(y)$) even though $f_1\ne f_2$ as scheme morphisms on $X'$. Actually seeing that $f_1\ne f_2$ is a little tricky: note
\[\left(x^2,xy\right)=(x)\cap\left(x^2,xy,y^2\right)\]
is a primary decomposition, so we have associated primes $(x)$ and $(x,y)$. Now, even though $D(y)$ is an open dense subset of $X'$, we see that $V((x))\supseteq D(y)$ has $V((x))\ne X'$. The problem here is that our open dense subset $D(y)$ of $X'$ has missed a closed embedding from $(x,y)$, which our $f_2$ could not see.

\subsection{Birational Maps}
Let's continue studying our rational maps.
\begin{proposition}
	Fix integral schemes $X$ and $Y$. Then a rational map $f\colon X\dashrightarrow Y$ is birational if and only if there are open dense subsets $U\subseteq X$ and $V\subseteq Y$ such that $f$ induces an isomorphism $U\cong V$.\todo{Proof?}
\end{proposition}
One can say more for varieties.
\begin{proposition}
	Fix a field $k$ and fix two $k$-schemes $X$ and $Y$ of finite type. Then there is a (natural) bijection
	\[\{\text{dominant }f\colon X\dashrightarrow Y\}\simeq\op{Hom}_k(\Spec K(X),\Spec K(Y))\simeq\op{Hom}_k(K(Y),K(X)).\]
\end{proposition}
\begin{proof}
	In one direction, start with a dominant map $f\colon X\to Y$. Note that the generic point $\eta_X\in X$ must go to the generic point $\eta_Y\in Y$. Namely, for any nonempty open subset $V\subseteq Y$, we have
	\[f^{-1}(V)\cap U\ne\emp\]
	for any open $U\subseteq X$, so $\eta_X\in f^{-1}(V)$ is forced. Looping through all open subsets $V$ forces $\eta_X\in f^{-1}(\{\eta_Y\})$. Now, our dominant map is going to give a map $f^\sharp\colon\OO_{Y,\eta_Y}\to\OO_{X,\eta_X}$, which is exactly the map $K(Y)\to K(X)$ we were looking for.

	In the other direction, we need to use the finite type condition. To begin, note that we can replace $Y$ with an open affine subset $\Spec B$, which is still dense; one can do the same for $X$ (to $\Spec A$) with no headaches. Then we see $K(Y)=\op{Frac}B$ and $K(X)=\op{Frac}X$. Now, we are given a map $f^\sharp\colon K(Y)\to K(X)$. Because our scheme is of finite type, we see $B=k[x_1,\ldots,x_n]$ for some finite $n$, and then we choose some $s\in A$ with enough dominators so that we have an induced map
	\[f^\sharp\colon B\to A_s,\]
	which induces the required rational map. This map is dominant because it sends the generic point to the generic point.
\end{proof}
\begin{remark}
	In some sense, we are saying that field extensions correspond to dominant morphisms of some corresponding scheme.
\end{remark}
We can thus see that $X$ and $Y$ are birationally equivalent if and only if $K(X)=K(Y)$.
\begin{example}
	We work the elliptic curve $V\left(Y^2Z-X^3-Z^3\right)\subseteq\PP^2_k$. Then there are two birational maps $X\dashrightarrow\PP^1_k$ by $[X:Y:Z]\to X/Z$ and $[X:Y:Z]\to Y/Z$. One can see here that these rational maps in fact extend all the way to give a full scheme morphism to $\PP^1_k$.
\end{example}
\begin{example}
	We work with $X=\Spec k[x,y]/\left(y^2-x^3\right)$. Then the normalization is $\widetilde X=\Spec k[t]$, where our normalization map is by $\widetilde X\to X$ given by $x\mapsto t^2$ and $x\mapsto t^3$. One can check that this map is birational: the fraction fields are $K(X)=k(x)[y]/\left(y^2-x^3\right)$ and $K(Y)=k(t)$, and the isomorphism we can see fairly directly is just given by $t\mapsto y/x$.
\end{example}
\begin{remark}
	Here is how to see that $\widetilde X$ is the normalization of $X$: note $k[t]$ is the integral closure of $k[x,y]/\left(y^2-x^3\right)$ in $K(X)$, where we see that $t$ is the root of some monic polynomial belonging to $y/x$. Thus, $k[t]$ is certainly contained in the integral closure. Then one just needs to show that $k[t]$ is integrally closed in $k(t)$, which is true.
\end{remark}
\begin{remark}
	More generally, the normalization map is birational when our scheme $X$ has finite type over $k$.
\end{remark}

\subsection{Proper Morphisms}
Before defining proper, we should get some intuition from universally closed maps.
\begin{definition}
	A scheme morphism $f\colon X\to Y$ is \textit{closed} if and only if it's a closed map of topological spaces.
\end{definition}
\begin{definition}[Universally closed]
	A scheme morphism $f\colon X\to Y$ is \textit{universally closed} if and only if $f$ remains a closed under any base change.
\end{definition}
\begin{example}
	Integral morphisms are universally closed. Namely, integral morphisms are closed embeddings and stable under base change and so will stay integral and hence closed embeddings.
\end{example}
\begin{example}
	Closed embeddings are also preserved by base change and hence universally closed.
\end{example}
And here is our definition.
\begin{definition}[Proper]
	A scheme morphism $f\colon X\to Y$ is \textit{proper} if and only if $f$ is separated, of finite type, and universally closed.
\end{definition}
\begin{remark}
	One can check that being integral implies being separated and universally closed.
\end{remark}
\begin{example}
	Finite morphisms are proper. Namely, affine implies separated, finite type is clear, and integral implies universally closed.
\end{example}
\begin{remark}
	As usual, proper morphisms are preserved by composition, base change, and is affine local on the target.
\end{remark}


\end{document}