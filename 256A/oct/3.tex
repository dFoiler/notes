% !TEX root = ../notes.tex

\documentclass[../notes.tex]{subfiles}

\begin{document}

\section{October 3}

It's spooky season.

\subsection{The Cancellation Theorem}
We have studied lots of adjectives of morphisms, so it will be useful to have some language for them.
\begin{definition}[Reasonable]
	Fix a category $\mc C$. A class of morphisms $P\subseteq\op{Mor}\mc C$ is \textit{reasonable} if and only if $P$ is preserved by composition and base change.
\end{definition}
Here is our statement.
\begin{theorem}
	Fix objects $X,Y,B$ with morphisms $\alpha\colon X\to B$ and $\beta\colon Y\to B$ and $\varphi\colon X\to Y$ making the diagram
	% https://q.uiver.app/?q=WzAsMyxbMCwwLCJYIl0sWzIsMCwiWSJdLFsxLDEsIkIiXSxbMCwxLCJcXHZhcnBoaSJdLFswLDIsIlxcYWxwaGEiLDJdLFsxLDIsIlxcYmV0YSJdXQ==&macro_url=https%3A%2F%2Fraw.githubusercontent.com%2FdFoiler%2Fnotes%2Fmaster%2Fnir.tex
	\[\begin{tikzcd}
		X && Y \\
		& B
		\arrow["\varphi", from=1-1, to=1-3]
		\arrow["\alpha"', from=1-1, to=2-2]
		\arrow["\beta", from=1-3, to=2-2]
	\end{tikzcd}\]
	commute. Given that $\alpha\in P$ and that the diagonal map $\delta_\beta\colon Y\to Y\times_BY$ has $\delta_\beta\in P$, we have that $\varphi\in P$.
\end{theorem}
\begin{proof}
	This is on the homework.
\end{proof}
\begin{example}
	If $\varphi$ is quasicompact, and $\beta$ is quasiseparated, then $\alpha$ is quasicompact.
\end{example}
\begin{lemma} \label{lem:cancel-sep}
	If we have scheme morphisms $f\colon X\to Y$ and $g\colon Y\to Z$. Then if $g\circ f$ is separated, we see $f$ is as well.
\end{lemma}
\begin{proof}
	Here is the relevant diagram.
	% https://q.uiver.app/?q=WzAsNSxbMCwwLCJYIl0sWzEsMCwiWFxcdGltZXNfWVgiXSxbMiwwLCJZIl0sWzIsMSwiWVxcdGltZXNfWlkiXSxbMSwxLCJYXFx0aW1lc19aWCJdLFsyLDMsIlxcRGVsdGFfZyJdLFswLDEsIlxcRGVsdGEgZiJdLFswLDQsIlxcRGVsdGFfe2dcXGNpcmMgZn0iLDJdLFs0LDNdLFsxLDRdLFsxLDJdXQ==&macro_url=https%3A%2F%2Fraw.githubusercontent.com%2FdFoiler%2Fnotes%2Fmaster%2Fnir.tex
	\[\begin{tikzcd}
		X & {X\times_YX} & Y \\
		& {X\times_ZX} & {Y\times_ZY}
		\arrow["{\Delta_g}", from=1-3, to=2-3]
		\arrow["{\Delta f}", from=1-1, to=1-2]
		\arrow["{\Delta_{g\circ f}}"', from=1-1, to=2-2]
		\arrow[from=2-2, to=2-3]
		\arrow[from=1-2, to=2-2]
		\arrow[from=1-2, to=1-3]
	\end{tikzcd}\]
	Because $\Delta_g$ is locally closed, we see that the map $X\times_YX\to X\times_ZX$ is locally closed by taking base-change. However, $\Delta_{g\circ f}(X)$ is a closed subset, so the commutativity of the diagram forces $\Delta_f(X)$ to be a closed subset, which finishes.
\end{proof}
\begin{lemma}
	Monomorphisms in the category of schemes are separated.
\end{lemma}
\begin{proof}
	The point is that the diagonal map $X\simeq X\times_YX$ given a monic map $f\colon X\to Y$.
\end{proof}

\subsection{Varieties}
We are finally ready to define varieties.
\begin{proposition} \label{prop:proj-is-sep}
	The canonical map $\PP^n_A\to\Spec A$ is separated.
\end{proposition}
We will prove this momentarily, but let's explain why we should care first.
\begin{example}
	Because $\Spec A$ is separated, we see that $\PP^n_A$ is also separated by post-composing to see $\PP^n_A\to\Spec\ZZ$ is separated.
\end{example}
\begin{example} \label{ex:quasiproj-is-sep}
	Quasiprojective schemes $X$ are those with a closed embedding into $\PP^n_A$. Thus, these are also separated $A$-schemes.
\end{example}
These notions let us define a variety.
\begin{definition}[Variety]
	Fix a field $k$. A \textit{variety} over $k$ is a reduced, separated $k$-scheme of finite type.
\end{definition}
\begin{example}
	\autoref{ex:quasiproj-is-sep} tells us that (reduced) quasiprojective schemes are varieties. In particular, all our usual affine and projective varieties are indeed varieties.
\end{example}
Before continuing, we pick up the following result.
\begin{proposition}
	Fix a scheme morphism $f\colon X\to Y$ where $Y=\Spec B$ is affine. Then the following are equivalent.
	\begin{listalph}
		\item $f$ is separated.
		\item Any affine open subset $U_1,U_2\subseteq X$ has $U_1\cap U_2$ affine, and the canonical map
		\[\OO_X(U_1)\otimes_B\OO_X(U_2)\to\OO_X(U_1\cap U_2)\]
		is surjective.
		\item There is an open affine cover $\{U_\alpha\}_{\alpha\in\lambda}$ such that all the intersections $U_\alpha\cap U_\beta$ are affine, and the canonical map
		\[\OO_X(U_1)\otimes_B\OO_X(U_2)\to\OO_X(U_1\cap U_2)\]
		is surjective.
	\end{listalph}
\end{proposition}
\begin{proof}
	Set $\Delta\colon X\to X\times_YX$ to be the diagonal map. Now, recall $U_\alpha\cap U_\beta=\Delta^{-1}(U_\alpha\times_Y\alpha_\beta)$.

	Of course (b) implies (c). We show the difficult part, which is showing that (a) and (c) are equivalent. Notably, $f$ is separated if and only if the canonical map
	\[U_\alpha\cap U_\beta\to U_\alpha\times_YU_\beta\]
	are closed embeddings for each $\alpha$ and $\beta$. Now, $U_\alpha\cap U_\beta$ is a closed subscheme of $\Spec(\OO_X(U_\alpha)\otimes_Y\OO_X(U_\beta))$ and therefore affine. Continuing, we are surjective on the level of stalks, so this is equivalent to asking for the full morphism to be surjective here.\todo{What?}
\end{proof}
\begin{remark}
	This is intended to be similar to the way we think about quasiseparated morphisms. However, having affine intersections is not good enough to be separated, as we saw with the affine line with doubled origin.
\end{remark}
We are now ready to prove \autoref{prop:proj-is-sep}.
\begin{proof}[Proof of \autoref{prop:proj-is-sep}]
	We choose the usual affine open cover of $\PP^n_A$. Namely, set
	\[U_i\coloneqq\Spec\frac{A[x_{0/i},\ldots,x_{n/i}]}{(x_{i/i}-1)}.\]
	Here, we see that these intersections are $U_i\times_AU_j$, which are affine schemes, and we can compute that $U_i\cap U_j$ as a subset of $U_i$ is
	\[U_i\cap U_j=\Spec\frac{A[x_{0/i},\ldots,x_{n/i},x_{j/i}^{-1}]}{(x_{i/i}-1)}.\]
	Now, the map $U_i\cap U_j\to U_i\times_AU_j$ corresponds to the ring map sending $x_{k/i}\mapsto x_{k/i}$ and $x_{k/j}\mapsto x_{k/i}\cdot x_{j/i}^{-1}$, which we see is surjective because look at it.
\end{proof}
\begin{remark}
	Given a $k$-morphism $f\colon X\to Y$ of $k$-varieties, we immediately get that $f$ is separated and of finite type. It is separated by \autoref{lem:cancel-sep}, and it is of finite type because the diagonal map $\Delta_Y\colon Y\to Y\times_kY$ is given to be a closed embedding lifting a morphism of finite type, which makes us finite type immediately. The point here is that focusing on just varieties is good.
\end{remark}

\subsection{Rational Maps}
For this subsection, we let $X$ be a reduced scheme and $Y$ a separated scheme. One can remove this assumption, but it creates problems.

We will use the notation $X\dashrightarrow Y$ to talk about our rational maps before we define it; the reason will be clear from the examples.
\begin{example}
	Consider the map $\PP^3_\ZZ\dashrightarrow\PP^2_\ZZ$ by $[x_0:x_1:x_2:x_3]\mapsto[x_0:x_1:x_2]$. Notably, this is not defined when $x_0=x_1=x_2=0$, which is just a single point in $\PP^3_\ZZ$, so our morphism is technically not well-defined, but that's life.
\end{example}
\begin{example}
	Let $X=V(y^2-x^3-x)$ be an elliptic curve over $k$. Then there is a map $X\dashrightarrow\AA^1_k$ by $(x,y)\mapsto 1/x$ so that this isn't well-defined at $x=0$. Notably, there is a way to extend this map to $\PP^1_k$, but that is for later.
\end{example}
\begin{example}
	Fix an integral scheme $X$ with generic point $\eta\in X$, where $K(X)\coloneqq\OO_{X,\eta}$ is the function field. Now, recall
	\[\op{Mor}(X,\AA^1_\ZZ)\simeq\op{Hom}(\ZZ[x],\OO_X(X))\simeq\OO_X(X),\]
	which embeds into $\OO_{X,\eta}$. The point is that $K(X)$ is supposed to be some extra ``rational'' maps on $X$ to $\AA^1_k$.
\end{example}
With enough examples, here is our definition.
\begin{definition}[Rational]
	Fix a reduced scheme $X$ and a scheme $Y$. A \textit{rational map} $f\colon X\dashrightarrow Y$ is an equivalence class of $(U,\alpha)$, where $U\subseteq X$ is a dense open subset, and $\alpha\colon U\to Y$ is a scheme morphism. The equivalence relation is given by $(U,\alpha)\sim(V,\beta)$ if and only if there is an open dense subset $W\subseteq U\cap V$ with $\alpha|_W=\beta|_W$.
\end{definition}
Here are some adjectives on our rational maps.
\begin{defi}[Dominant]
	Fix a reduced scheme $X$ and a scheme $Y$. A rational map $f\colon X\dashrightarrow Y$ is \textit{dominant} if and only if each $(U,\alpha)$ representing $f$ has $\alpha$ dominant.
\end{defi}
\begin{definition}[Birational]
	Fix a reduced scheme $X$ and a scheme $Y$. A rational map $f\colon X\dashrightarrow Y$ is \textit{birational} if and only if there is another rational map $g\colon Y\dashrightarrow X$ such that $f\circ g$ and $g\circ f$ are equivalent to identities.
\end{definition}
\begin{ex}
	The affine line with doubled origin, $\AA^1\setminus\{0\}$, and $\AA^1$ are all birationally equivalent.
\end{ex}
\begin{example}
	All of $\PP^1\times\PP^1$ and $\AA^1\times\AA^1$ and $\PP^2$ are all birationally equivalent.
\end{example}
Here is a quick sanity check, which explains why we want our codomain to be separated.
\begin{restatable}{proposition}{gluerats} \label{prop:glue-rats}
	Fix open dense subsets $U,U'$ of a reduced scheme $X$ and a separated scheme $Y$. Given $\alpha\colon U\to Y$ and $\beta\colon U'\to Y$ which are both representatives of a rational map $f\colon X\dashrightarrow Y$, we have $\alpha|_{U\cap U'}=\beta|_{U\cap U'}$.
\end{restatable}
\noindent In particular, we can glue together $\alpha$ and $\beta$ and apply transfinite induction induct in order to say that $f$ has a unique representative with largest domain.
\begin{nex}
	Consider the rational map from the affine line with doubled origin to $\AA^1$. Then on each $\Spec k[x]$ for the affine line with doubled origin, there is a map to the affine line, but we can't glue these two maps together to get a full map from the affine line with doubled origin to the affine line.
\end{nex}

\end{document}