% !TEX root = ../notes.tex

\documentclass[../notes.tex]{subfiles}

\begin{document}

\section{October 19}

Today we discuss divisors somewhat.

\subsection{Line Bundles}
Fix a scheme $X$.
\begin{notation}
	Given a scheme $X$, we define $\op{Pic}X$ to be the abelian group of isomorphism classes of line bundles, where the group law is given by $\otimes_{\OO_X}$.
\end{notation}
\begin{remark}
	In some cases, $\op{Pic}X$ can further be given a scheme structure.
\end{remark}
This abelian group is difficult to study in general, but we will see some examples.
\begin{remark}
	Given a scheme morphism $\varphi\colon X\to Y$, a line bundle $\mc L$ on $Y$ can be pulled back to $\varphi^*\mc L$, so we induce a group homomorphism $\op{Pic}Y\to\op{Pic}X$.
\end{remark}

\subsection{Divisors}
Let's now discuss divisors. Fix a Noetherian integral scheme $X$.
\begin{definition}[Weil divisor]
	A \textit{Weil divisor} of a Noetherian integral scheme $X$ is a (formal) finite $\ZZ$-linear combination of codimension-$1$ closed integral subschemes of $X$. We will often write divisors $D$ as
	\[D=\sum_{\substack{Y\subseteq X\\Y\text{ closed}\\\codim Y=1}}n_{Y}[Y],\]
	where the $n_Y$ vanish for all but finitely many $Y$.
\end{definition}
\begin{remark}
	The set of Weil divisors has the obvious group structure by addition pointwise.
\end{remark}
One can allow non-integral (namely, non-reduced) things to sneak into our divisors by adding in this information to the multiplicity, but we will not do so.
\begin{definition}[Effective]
	Fix a Noetherian integral scheme $X$. A Weil divisor $D=\sum_Yn_Y[Y]$ is \textit{effective} if and only if $n_Y\ge0$ for each $Y$.
\end{definition}
\begin{definition}[Support]
	Fix a Noetherian integral scheme $X$. The \textit{support} of a Weil divisor $D=\sum_Yn_Y[Y]$ is
	\[\op{Supp}D\coloneqq\bigcup_{\substack{Y\subseteq X\\n_Y\ne0}}Y.\]
\end{definition}
\begin{example}
	Fix a number field $K/\QQ$ and $X\coloneqq\Spec\OO_K$, which we can see is a Noetherian integral scheme. Then Weil divisors are just $\ZZ$-linear formal combinations of the maximal ideals $[\mf p]$. Namely, we are only looking at maximal ideals because we want to have codimension $1$.
\end{example}
\begin{example}
	Fix a normal quasiprojective curve $C$ over an algebraically closed field $k$. Then a Weil divisor is again just $\ZZ$-linear formal combinations of the closed points on $C$. We are implicitly using the fact that closed points are our codimension-$1$ closed embeddings.
\end{example}
The fact that a point $x\in X$ might have codimension $1$ is saying that $\OO_{X,x}$ is a local ring of dimension $1$. Notably, we are already regular, namely $\dim_{k(x)}\mf m_x/\mf m_x^2=\dim\OO_{X,x}=1$, so the corresponding local rings are all discrete valuation rings. One can even give this valuation explicitly: pick $\pi\in\mf m_x\setminus\mf m_x^2$, and then for any $f\in\op{Frac}\OO_{X,x}$, our valuation is the integer $n$ for which $f/\pi^n\in\OO_{X,x}^\times$.

These notions give us some divisors.
\begin{definition}[Principal divisor] \label{def:div-for-ox}
	Fix a Noetherian integral scheme $X$ whose local rings are regular in codimension-$1$. Given some $f\in K(X)\setminus\{0\}$, we define the \textit{principal divisor}
	\[\op{div}(f)\coloneqq\sum_{Y\subseteq X}v_Y(f)[Y],\]
	where $v_Y$ is the valuation of $f$ above at the generic point of $Y$.
\end{definition}
\begin{remark}
	If $X$ is normal, then the local rings will be regular.
\end{remark}
\begin{remark}
	Thus, we have a function $\op{div}$ which actually defines a group homomorphism from $K(X)^\times$ to the group of Weil divisors.
\end{remark}

\subsection{Divisors for Line Bundles}
We would like to generalize \autoref{def:div-for-ox} to line bundles. More generally, given a line bundle $\mc L$ over $X$, some $s\in\Gamma(U,\mc L)\setminus\{0\}$ for $U\subseteq X$ open and dense is a rational section of $\mc L$. This exactly generalizes $f\in K(X)^\times$.

This gives the following definition.
\begin{definition} \label{def:div-for-line-bundle}
	Fix a line bundle $\mc L$ on a normal Noetherian integral scheme $X$. Give $X$ a finite affine open cover $\{U_i\}_{i=1}^n$ such that $\mc L|_{U_i}\simeq\OO_{U_i}$ for each $i$. Then a rational function $s\in\Gamma(U,\mc L)$ will glue together into a divisor $\op{div}s$.\todo{What?}
\end{definition}
Let's see an example.
\begin{exe}
	Fix $X\coloneqq\PP^1_k$, where $k$ is a field. We run computations on $\OO_X$ and $\OO_X(1)$.
\end{exe}
\begin{proof}
	Here are some examples. We think of $\PP^1_k$ as the one-point compactification of $\AA^1_k=\Spec k[x]$. Namely, with $\PP^1_k=\Proj k[X_0,X_1]$, we have $x=X_1/X_0$. Set $U_0\coloneqq\Spec k[X_1/X_0]$ and $U_1\coloneqq\Spec k[X_0/X_1]$.
	\begin{itemize}
		\item The rational function $1\in\Gamma(X,\OO_X)$ gives rise to $\op{div}1=0$ because it vanishes nowhere.
		\item The rational function $X_0/X_1\in\Gamma(X,\OO_X)$ we only need to compute on the open dense subset $\Spec k[x]$, so we get a zero on $[\infty]$ and a pole at $[0]$, so the divisor here is $[\infty]-[0]$.
		\item The rational function $X_0\in\Gamma(X,\OO_X(1))$ gets twisted over to the generated divisor $[\infty]$. Notably, the number of poles does not match the number of zeroes! Similarly, the rational function $X_1\in\Gamma(X,\OO_X(1))$ goes over to $[0]$.
		\qedhere
	\end{itemize}
\end{proof}
\begin{definition}[Weil divisor class group]
	Fix a Noetherian integral scheme $X$ whose local rings are regular in codimension $1$. Then we define the \textit{Weil divisor class group} $\op{Cl}X$ as the Weil divisors modded out by the principal divisors.
\end{definition}
\begin{remark} \label{rem:pic-to-cl-map}
	The divisor map will send pairs $(\mc L,s)$ of a line bundle and a rational section $s$ on $\mc L$ to the Weil divisors, using \autoref{def:div-for-line-bundle}. Modding out by isomorphism on both sides builds a map $\op{Pic}X\to\op{Cl}X$.
\end{remark}
We would like for the morphism of \autoref{rem:pic-to-cl-map} to be an isomorphism, but it need not be in general. For this, we need more scheme adjectives.

\subsection{Normal, Regular, Smooth Schemes}
Normal will make the map \autoref{rem:pic-to-cl-map} injective, regular will make it an isomorphism, but smoothness is easier to check than regularity.
% V 12.3.13, L 4.1-4.3

We start from commutative algebra. Here is regular.
\begin{definition}[Regular]
	Fix a Noetherian local ring $A$ with maximal ideal $\mf m$. Then $A$ is \textit{regular} if and only if
	\[\dim A=\dim_{A/\mf m}\mf m/\mf m^2.\]
\end{definition}
\begin{definition}[Regular]
	A locally Noetherian scheme $X$ is \textit{regular at $x\in X$} if and only if the stalk $\OO_{X,x}$ is regular. The scheme $X$ is \textit{regular} if and only if it is regular everywhere.
\end{definition}
\begin{remark}
	A scheme $X$ will be regular if and only if it is regular at the closed points.
\end{remark}
\begin{remark}
	The main point here is that $\dim_{A/\mf m}\mf m/\mf m^2$ is supposed to be the dimension of the (co)tangent space at a closed point, so we are kind of asking for the tangent space and the actual scheme to have the same dimension. This is analogous to asking for smoothness, as we will see momentarily.
\end{remark}
\begin{example}
	Set $R\coloneqq\ZZ_p[x,y]/(xy-p)$. Then $R$ is regular: its closed point is $\mf m\coloneqq(p,x,y)$, and then we can directly compute
	\[\frac{\mf m}{\mf m^2}=\frac{(p,x,y)}{(xy-p,p^2,x^2,y^2,px,py,xy)}=\frac{(p,x,y)}{(xy,p,x^2,y^2)}\]
	is generated by $x$ and $y$ as an $\mathbb F_p$-module.
	% L Ex 4.2.3
\end{example}

\end{document}