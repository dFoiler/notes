% !TEX root = ../notes.tex

\documentclass[../notes.tex]{subfiles}

\begin{document}

\section{October 26}

We continue our discussion of divisors.

\subsection{More Back to Divisors}
We continue with the proof of the following result from last class.
\cacltopic*
\begin{proof}
	Last class we constructed this map. We now show that the map is injective: we can check that the map $D\mapsto\OO_X(D)$ is a group homomorphism by its construction, so it suffices to show that it has trivial kernel.

	Well, if we have some $D=\{(U_\alpha,f_\alpha)\}_{\alpha\in\lambda}$ going to $0$ in $\op{Pic}X$, then we see that we are promised an isomorphism
	\[\OO_X\simeq\OO_X(D)\subseteq\mc K_X,\]
	so tracking where the global section $1$ goes through grants us some $f\in\mc K_X$ and get that $f$ ``generates'' $\OO_X(D)$. In particular, we see that the global section $f$ has $\op{div}(f^{-1})=D$ by construction, so $\OO_X(D)$ is in fact a principal Cartier divisor.

	We now show that we have an isomorphism. Fix a line bundle $\mc L$. If we have $\mc L\subseteq\mc K_X$ already, then we can go pick up its affine open cover $\{U_\alpha\}_{\alpha\in\lambda}$ upon which it's locally trivial, and then we can do a spreading argument around each point in $X$ to go find the corresponding Cartier divisor $D$ with $\mc L=\OO_X(D)$.

	So we have to provide an embedding $\mc L\into\mc K_X$. Well, given an open subscheme $j\colon U\subseteq X$ such that $\mc L|_U\simeq\OO_X|_U$, we get to write
	\[\mc L\subseteq j_*(\mc L|_U)\simeq j_*(\OO_X|_U)\subseteq j_*(\mc K_U)\simeq\mc K_X,\]
	which is what we wanted. Notably, the last isomorphism holds because $X$ is integral.
\end{proof}
\begin{remark}
	Let's explain this result philosophically. Note that the short exact sequence
	\[1\to\OO_X^\times\to\mc K_X^\times\to\mc K_X^\times/\OO_X^\times\to1,\]
	so a good enough cohomology theory tells us that we should have an exact sequence
	\[\check H^0(X,\mc K_X^\times)\to\check H^0(X,K_X^\times/\OO_X^\times)\to\check H^1(X,\OO_X^\times)\to\check H^1(X,\mc K_X^\times).\]
	By definition, we get $\check H^0(X,\mc F)=\Gamma(X,\mc F)$, and $\check H^1(X,\OO_X^\times)\simeq\op{Pic}X$, so we have induced a map
	\[\frac{\Gamma(X,\mc K_X^\times/\OO_X^\times)}{\Gamma(X,\mc K_X^\times)}\into\check H^1(X,\OO_X^\times)\simeq\op{Pic}X,\]
	so our map is indeed injective. One can show that $\check H^1(X,\mc K_X^\times)$ vanishes when $X$ is integral by some computation, which is ``why'' we get an isomorphism.
\end{remark}
\begin{remark}
	One can show that effective Cartier divisors form a group isomorphic to pairs $(\mc L,s)$, where $\mc L$ is a line bundle and $s\in\Gamma(X,\mc L)$ where the map $\OO_X\to\mc L$ given by multiplication by $s$ is injective. When $X$ is reduced, we're essentially requiring that $s\ne0$.\todo{Do we want s=0?}
\end{remark}

\subsection{Adding in Regularity}
We now return to the proof of our main result.
\begin{theorem}
	Fix a regular Noetherian scheme $X$. Then we have a natural isomorphism $\op{Pic}X\to\op{Cl}X$.
\end{theorem}
\begin{proof}
	All connected components of $X$ we can show to be integral, so we may assume that $X$ is integral. Note that we certainly have a map
	\[\op{CaCl}X\to\op{Weil}X\]
	which we can see to be injective from the proof of injectivity above. Namely, this is the composite
	\[\op{CaCl}X\to\{(\mc L,s):s\in\Gamma(X,\mc L)\}\to\op{Weil}X\]
	which we can see to be injective.

	It remains to go in the other direction. Fix a Weil divisor $D=\sum_{[Y]}n_Y[Y]$; it suffices to just get a Cartier divisor for just $[Y]$. Now, for each $x\in X$, we note that $\OO_{X,x}$ is a regular local domain and hence factorial. Now, each closed subscheme $Y$ has associated to it an ideal sheaf $\mc I_Y\subseteq\OO_X$, and so we get an ideal
	\[\mc I_{Y,x}\subseteq\OO_{X,x}.\]
	Because $\OO_{X,x}$ is factorial, all its height-$1$ primes are principal (!), and $\mc I_{Y,x}$ has codimension $1$, so we conclude that $\mc I_{Y,x}=(f_x)$ for some $f_x$ by running through all the primes containing it. Because we're looking at germs, we actually have $f_x\in\Gamma(U_x,\OO_X)$ for some open neighborhood $U_x$ of $x$, and it follows from this construction that
	\[f_x/f_y\in\Gamma(U_x\cap U_y,\OO_X)^\times.\]
	In total, we have created a Cartier divisor $\{(U_x,f_x)\}_{x\in X}$, which is what we wanted.
\end{proof}
Let's see some examples.
\begin{example}
	Recall that $\AA^n_k$ has all the adjectives, so
	\[\op{Pic}(\AA^n_k)=\op{Cl}(\AA^n_k)=0.\]
	Namely, $\op{Cl}(\AA^n_k)=0$ because $k[x_1,\ldots,x_n]$ is factorial. Indeed, any closed subscheme $Y\subseteq\AA^n_k$ of codimension $1$ must have $Y=V(f)$ for some $f\in k[x_1,\ldots,x_n]$.
\end{example}
\begin{example}
	Given a number field $K$, we see that $\op{Pic}(\Spec\OO_K)=\op{Cl}\OO_K$, where $\op{Cl}\OO_K$ is the ideal class group.
\end{example}
\begin{exe}
	We show that $\op{Pic}\PP^n_k\simeq\ZZ$ by sending $n\in\ZZ$ to $\OO(n)$.
\end{exe}
\begin{proof}
	To begin, note that this is injective: if $n\ne m$, then by tensoring to ensure that everything is positive, it suffices to do this in the case of $n,m>0$, but then $\dim\Gamma(X,\OO(n))\ne\dim\Gamma(X,\OO(m))$, so $\OO(n)\ne\OO(m)$.

	To continue, we pick up the following result.
	\begin{proposition}
		Fix a Noetherian integral separated scheme $X$ which is regular in codimension $1$. Then any irreducible closed subscheme $Z\subseteq X$ produces the following.
		\begin{listalph}
			\item If $\codim Z\ge2$, then $\op{Cl}X\simeq\op{Cl}(X\setminus Z)$ by taking $Y\mapsto[Y\setminus Z]$.
			\item If $\codim Z=1$, then
			\[\ZZ\stackrel{1\mapsto[Z]}\to\op{Cl}X\stackrel{[Y]\mapsto[Y\setminus Z]}\to\op{Cl}(X\setminus Z)\to0\]
			is exact.
		\end{listalph}
	\end{proposition}
	\begin{proof}
		The map
		\[\op{Weil}X\stackrel{[Y]\mapsto[Y\setminus Z]}\to\op{Weil}(X\setminus Z)\]
		is certainly surjective by taking the Zariski closure to go backwards. If $\codim Z\ge2$, then we can check that there is nothing in the kernel as well. If $\codim Z=1$, then we can show that the only way for us to have a Weil divisor which vanishes is for it to come from $Z$, so there is nothing else to say.
	\end{proof}
	Now, to finish the proof, we take $X=\PP^n_k$ with some affine open subscheme $Z=\AA^{n-1}_k$ so that $X\setminus Z=\AA^n_k$. Then the exact sequence above will read as $\ZZ\onto\op{Pic}\PP^n_k\to0$ by $n\mapsto\OO(n)$, but we already know that this map is injective, so we are done.
\end{proof}
\begin{remark}
	Projective $k$-varieties do have a reasonable notion of the degree of a divisor. For example, in $\PP^n_k$, all closed subschemes of codimension $1$ look like $V_+(f)$ for some homogeneous polynomial $f$, so we define $\deg V_+(f)\coloneqq\deg f$. Notably, we can check that the isomorphism $\op{Pic}\PP^n_k\to\ZZ$ is given by $\deg\colon[Y]\mapsto\deg Y$.
\end{remark}
\begin{remark}
	In contrast, degree is not well-behaved for $\AA^n_k$ because principal divisors do not necessarily have degree $0$, and in particular it would not be helpful to have a map from $\op{Pic}\AA^n_k=0$ anyway.
\end{remark}

\end{document}