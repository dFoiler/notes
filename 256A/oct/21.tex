% !TEX root = ../notes.tex

\documentclass[../notes.tex]{subfiles}

\begin{document}

\section{October 21}

Sleep is less important than problem set, I suppose.

\subsection{Regular Schemes}
We pick up a few more commutative algebra facts.
\begin{theorem}
	A Noetherian regular local ring $A$ is a unique factorization domain. Every unique factorization domain is normal.
\end{theorem}
% GW B.73--75, V , E19.19
\begin{corollary}
	A regular Noetherian scheme is normal.
\end{corollary}
\begin{proof}
	Check affine-locally.
\end{proof}
Let's see some examples.
\begin{example}
	Fix a regular Noetherian local ring $A$ so that we get a unique factorization domain. If $\dim A=0$, then $\mf m/\mf m^2=0$, so $\mf m$ is the unique maximal ideal, so $A$ is a field. If $\dim A=1$, then $A$ is a discrete valuation ring.
\end{example}
\begin{example}
	Fix an algebraically closed field $k$. Then we see $\AA^n_k$ is regular. Recall that we only need to check regularity on closed points, which are maximal ideals $\mf m\coloneqq(x_1-a_1,\ldots,x_n-a_n)\subseteq k[x_1,\ldots,x_n]$. Then we can directly compute our basis as
	\[\dim_k\mf m/\mf m^2=k\langle x_1-a_1,\ldots,x_n-a_n\rangle.\]
\end{example}
\begin{example}
	We check that $\AA^n_\ZZ$ is regular. Fix a closed point $\mf m_x\subseteq\ZZ[x_1,\ldots,x_n]$. Note $\mf m_x\cap\ZZ=(p)$ for some prime $p$, so we can think of $\mf m_x/(p)$ has living in $\FF_p[x_1,\ldots,x_n]$. Attempting to induct downwards, we next look at
	\[\frac{\mf m_x}{(p)}\cap\FF_p[x_1]=(p_1(x_1))\]
	for some irreducible $p_1(x_1)$. Continuing, we can look in
	\[\frac{\mf m_x}{(p,p_1)}\cap\frac{\FF_p[x_1]}{(p_1(x))}=(p_2(x_1,x_2)).\]
	After long enough, we get $\mf m_x=(p,p_1,\ldots,p_n)$, where $p\in\ZZ$ and $p_i\in\FF_p[x_1,\ldots,x_i]$ are all irreducible. At the end of the day, we can compute $\dim_{k(x)}\mf m_x/\mf m_x^2\le n+1$ because of our spanning set, but we must have $\dim\OO_{X,x}=n+1\le\dim_{k(x)}\mf m_x/\mf m_x^2$, so the equality follows.
\end{example}
\begin{nex}
	Fix $A\coloneqq k[x,y,z]/(x^2-yz)$ is not regular. Namely, localizing at $(x,y,z)$ gives a normal but not regular ring.
	% L4.1.9
\end{nex}
\begin{remark}
	In good enough curves, being normal, regular, and smooth are all the same. This explains why we see a singularity in the previous example despite being normal.
\end{remark}

\subsection{Smooth Schemes}
We can kind of feel that being regular is pretty annoying. Here is our definition of smoothness.
\begin{definition}[Smooth]
	Fix a $k$-scheme $X$. Then $X$ is \textit{smooth of dimension $d$} if and only if $X$ has an affine open cover $\{U_\alpha\}_{\alpha\in\lambda}$ such that $U_\alpha\cong\Spec k[x_1,\ldots,x_n]/(f_1,\ldots,f_r)$ has the Jacobian matrix
	\[\left[\frac{\del f_j}{\del x_i}\bigg|_x\right]_{ij}\]
	has rank $n-d$ at each $x$.
\end{definition}
\begin{remark}
	Smooth implies locally of finite type, which we can see from inside the definition.
\end{remark}
\begin{remark}
	Geometrically, we are saying that we locally look like some $\RR^d$. In some sense, this is just using the Implicit function theorem to look locally.
\end{remark}
\begin{example}
	Fix an elliptic curve $\Spec k[x,y]/\left(y^2-f(x)\right)$, where $\deg f=3$ has nonzero discriminant, with $\op{char}k\ne2$. Now, we can compute our Jacobian is
	\[\begin{bmatrix}
		-f'(x) & 2y
	\end{bmatrix}\]
	so that the rank of the Jacobian vanishes if and only if both $y=0$ and $f'(x)=0$. However, this requires $f(x)=0$ and $f'(x)=0$, meaning that $f$ has a double root, which is not allowed when $f$ has vanishing discriminant.
\end{example}
\begin{nex}
	As before, take $\Spec k[x,y,z]/(x^2-yz)$. Then our Jacobian is
	\[\begin{bmatrix}
		2x & -z & -y
	\end{bmatrix}\]
	which vanishes at $(x,y,z)=(0,0,0)$.
\end{nex}
\begin{example}
	Given a field $k$, we see $\AA^n_k$ is smooth by doing the check directly. Similarly, $\PP^n_k$ is smooth by doing the usual affine open cover.
\end{example}
Here's a check for varieties.
\begin{proposition}
	Fix a $k$-variety $X$. If $X$ is smooth at a point $x$, then $X$ is regular at $x$. If $k$ is perfect (and, say, algebraically closed), the converse also holds.
\end{proposition}
\begin{proof}
	As usual, for a regularity check, we only need to check at closed points. Note that our ``affine case'' $\AA^n_k$ is already smooth, so we hope that the Jacobian having the correct rank should save us.

	Locally, because $X$ is a $k$-variety, we may build a map $X\to\AA^n_k$ such that $x\mapsto y$ for some closed point $y\in\AA^n_k$. (We have implicitly replaced $X$ with an affine open subscheme containing $x$.) Now, we see that
	\[\OO_{X,x}\simeq\frac{\OO_{\AA^n_k,y}}{(f_1,\ldots,f_r)}.\]
	The Jacobian having full rank of $n-d$ implies that the elements $\overline f_\bullet$ are linearly independent in $\mf m_y/\mf m_y^2$. (Note that we can filter out from the $f_\bullet$ until we have $r=n-d$ to get the needed linear independence.) Now, we have the following fact.
	\begin{lemma}
		Fix a regular local ring $(A,\mf m)$ and some $f\in\mf m\setminus\{0\}$. Then $A/fA$ is regular if and only if $f\notin\mf m^2$.
	\end{lemma}
	Applying the lemma inductively over all our generators implies that $\OO_{X,x}$ is regular, using the regularity of $\OO_{\AA^n_k}$ which we already have.

	Conversely, take $k$ algebraically closed. From earlier, we still have the map $\OO_{\AA^n_k,y}\to\OO_{X,x}$. However, applying the lemma again, we see that $\dim_k\mf m_x/\mf m_x^2=\dim\OO_{X,x}=d$ if and only if the Jacobian has rank $d$. Indeed, we claim that
	\[\dim\mf m_x/\mf m_x^2=n-\op{rank}J_x,\]
	where $J_x$ is our Jacobian. To see this, simply unwind the above end of the previous direction to go extract the needed $f_\bullet$, using the fact that our field is algebraically closed.
\end{proof}
\begin{nex}
	Take $k\coloneqq\FF_p(t)$ for $p$ odd with $X\coloneqq\Spec k[x,y]/\left(y^2-x^p+t\right)$. Then $\dim X=1$, and we can check by hand that $X$ is smooth at all points with $y\ne0$: our Jacobian is
	\[\begin{bmatrix}
		0 & 2y
	\end{bmatrix}\]
	which we can see will not be smooth at, say, the point $(y)=\left(y,x^p-t\right)$. However, $X$ is actually regular at $y$. Indeed, we can check by hand that $\mf m_{(y)}/\mf m_{(y)}^2\simeq ky$ as $k$-vector spaces, which is dimension $1$.
\end{nex}
Let's generalize our definition of smoothness a little.
\begin{definition}[Smooth]
	A scheme morphism $f\colon X\to Y$ is \textit{smooth of relative dimension $d$} if and only if there is an affine open cover $\{U_\alpha\}_{\alpha\in\lambda}$ of $X$ and $\{V_\alpha\}_{\alpha\in\lambda}$ of $Y$ (with $f(U_\alpha)\subseteq V_\alpha$ for each $\alpha$) such that the induced maps $f|_{U_i}\colon U_i\into V_i$ can be factored through an open embedding
	\[U_i\into\Spec\frac{R[x_1,\ldots,x_{d+r}]}{(f_1,\ldots,f_r)}\to\Spec R_i\]
	(where $R_i\coloneqq\OO_Y(V_i)$) such that the Jacobian $J$ has rank $r$ at all points in $U$.
\end{definition}
Equivalently, we can ask for
\[\det\left[\frac{\del f_j}{\del x_i}\bigg|_x\right]_{1\le i,j\le r}\]
where the $x_i$s have possibly been rearranged.
\begin{remark}
	Note that we are implicitly just saying that open embeddings should be smooth.
\end{remark}
\begin{example}
	The morphism $\AA^n_R\to\Spec R$ and $\PP^n_R\to\Spec R$ are smooth. One can just check this directly from the definition using the obvious open covers.
\end{example}
\begin{remark}
	Smoothness is preserved by composition, base-change, and is affine-local on the target. So the morphisms $\AA^n_S\to S$ and $\PP^n_S\to S$ are both smooth for general schemes $S$.
\end{remark}

\end{document}