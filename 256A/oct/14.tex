% !TEX root = ../notes.tex

\documentclass[../notes.tex]{subfiles}

\begin{document}

\section{October 14}

Term paper topics will be posted later today. They're going to be a bit vague. Other topics are allowed, but email first. You may talk to other people about the topics, but the papers should be different.

\subsection{Coherent Sheaves}
Things are going to get a little abstract. Here we go.

Last time we discussed coherent sheaves for locally Noetherian schemes. Throughout, $\mc F$ is an $\mathcal O_X$-module. Here are some finiteness conditions.
\begin{definition}[Finite type]
	An $\mathcal O_X$-module $\mc F$ is of \textit{finite type} if and only if each $x\in X$ has some open $U\subseteq X$ such that there is some $n$ giving a surjection
	\[(\OO_X|_U)^n\to\mc F|_U\to0.\]
\end{definition}
\begin{example}
	In the affine case, take $X\simeq\Spec A$ and $\mc F=\widetilde M$, where $M$ is an $A$-module. Now, $\widetilde M$ is of finite type if and only if $M$ is a finitely generated $A$-module.
\end{example}
\begin{definition}[Finite presentation]
	An $\mathcal O_X$-module $\mc F$ is of \textit{finite presentation} if and only if each $x\in X$ has some open $U\subseteq X$ such that there are some $n$ and $m$ giving a surjection
	\[(\OO_X|_U)^m\to(\OO_X|_U)^n\to\mc F|_U\to0.\]
\end{definition}
\begin{example}
	In the affine case, take $X\simeq\Spec A$ and $\mc F=\widetilde M$, where $M$ is an $A$-module. Now, $\widetilde M$ is of finite presentation if and only if $M$ is a finitely presented $A$-module.
\end{example}
And now, the third has a different name.
\begin{definition}[Coherent]
	An $\mathcal O_X$-module $\mc F$ is \textit{coherent} if and only if $\mc F$ is of finite type and any $U\subseteq X$ and any map $(\OO_X|_U)^n\to\mc F$ has finite-type kernel.
\end{definition}
Of course, being coherent implies finite presentation implies finite type.

Here are some quick remarks.
\begin{example}
	In the affine case, take $X\simeq\Spec A$ and $\mc F=\widetilde M$, where $M$ is an $A$-module. Now, $\widetilde M$ is coherent if and only if $M$ and any map $A^n\to M$ has finitely generated kernel.
\end{example}
\begin{remark}
	The above conditions can be checked on any given affine open cover, using \autoref{lem:affine-comm} as usual.
\end{remark}
\begin{remark}
	We can see that having finite presentation implies that $\mc F$ is quasicoherent, by pulling back to force $\mc F|_U$ to be a module in the sequence.
\end{remark}
\begin{remark}
	In the locally Noetherian case, everything is equivalent because finitely generated modules are also Noetherian.
\end{remark}
\begin{remark}
	If $\OO_X$ is coherent as an $\mathcal O_X$-module, then finite presentation is equivalent to being coherent. Essentially all our time will be spent assuming $\OO_X$ is coherent, though there are counterexamples.
\end{remark}
\begin{remark}
	Fix a morphism $f\colon X\to Y$ of locally Noetherian schemes $X,Y$.
	\begin{itemize}
		\item If $\mc G$ is quasicoherent over $Y$, then $f^*\mc G$ is also quasicoherent, essentially by checking directly on the affine open subschemes.
		\item If $f$ is finite, then $\mc F$ being quasicoherent over $X$ implies that $f_*\mc F$ is also quasicoherent over $Y$. Approximately speaking, we need $f$ to be finite because if $M$ is finitely generated over an $A$-algebra, this does not imply that $M$ is finitely generated over $A$. One can weaken this in various ways.
	\end{itemize}
\end{remark}

\subsection{Vector Bundles}
In analogy to geometry, we will want to talk about modules which are locally free.
\begin{definition}[Locally free]
	Fix a scheme $(X,\OO_X)$. An $\mathcal O_X$-module $\mc F$ is \textit{free} if and only if $\mc F\cong\OO_X^\lambda$ for some set $\lambda$. Further, $\mc F$ is \textit{locally free} if and only if there is an open cover $\mc U$ of $X$ such that $\mc F|_U$ is a free $\OO_U$-module for each $U\in\mc U$.
\end{definition}
\begin{definition}[Rank]
	Fix a scheme $(X,\OO_X)$ and locally free $\OO_X$-module $\mc F$. Then the \textit{rank} of $\mc F$ at an open subset $U$ is the size of the index set $\lambda$ where $\mc F|_U\cong(\OO_U)^\lambda$.
\end{definition}
\begin{remark}
	If $X$ is connected, then the rank is a constant function. Indeed, we can see directly that it is locally constant.
\end{remark}
\begin{remark}
	Note that locally free implies $\mc F$ is quasicoherent, and being locally free of finite type implies that $\mc F$ is coherent, in the case that $\OO_X$ is coherent.
\end{remark}
And here is our definition.
\begin{definition}[Vector bundle]
	A \textit{vector bundle} over a scheme is a locally free sheaf of finite rank. A \textit{line bundle} is a vector bundle of rank $1$. Some authors call line bundles ``invertible sheaves.''
\end{definition}
\begin{remark}
	Here is a fact from commutative algebra we will want: a module $M$ over a ring $A$ has $\widetilde M$ locally free of finite type as an $\mathcal O_{\Spec A}$-module if and only if $M$ is a finitely generated projective $A$-module.
\end{remark}
And now, here are some affine examples.
% GW 7.13, 11.14
\begin{example}
	Given a field $k$, set $X\coloneqq\AA^1_k$. Then we note finitely generated projective modules over the principal ideal domain $k[x]$ must all actually be free, so the only vector bundles over $X$ are just $\OO_{\AA^1_k}^{r}$, where $r$ is the rank.
\end{example}
\begin{ex}
	Given a field $k$, set $X\coloneqq\AA^n_k$. In this case, all line bundles are isomorphic to $\OO_{\AA^n_k}$! We will prove later when we relate line bundles to divisors, so this result will follow from the fact that $k[x_1,\ldots,x_n]$ is factorial.
\end{ex}
Here's an arithmetic example.
\begin{ex}
	Given a number field $K$, set $X\coloneqq\Spec\OO_K$. Then fractional ideals of $\OO_K$ are finitely generated $\OO_K$-submodules of $K$. The fact that $\OO_K$ is a Dedekind domain implies that nonzero fractional ideals $\mf a$ are invertible; i.e., there is some $\mf b$ such that $\mf a\mf b=\OO_K$, or equivalently, $\mf a_\mf p\simeq(\OO_K)_\mf p$ for each maximal $\mf p\in\Spec\OO_K$. The point is that we can spread out from $\mf p$ to some small open subset around each $\mf p$, so $\widetilde{\mf a}$ becomes a locally free sheaf of rank $1$, which is an ``invertible'' sheaf.
\end{ex}
% Eisenbud 3.11.3--4, V14.2.J
\begin{remark}
	In the above example, we have $\widetilde{\mf a}\otimes\widetilde{\mf b}\cong\widetilde{\mf a\otimes\mf b}\cong\widetilde{\mf a\mf b}$ for fractional ideals $\mf a,\mf b\subseteq K$. In general, the tensor product of two line bundles is also a line bundle, which will turn the set of line bundles into an abelian group. The identity is the trivial sheaf, and the inverse will be constructed later; this inverse is what motivates the term ``invertible sheaf.''
\end{remark}
\begin{remark}
	In $\OO_K$, all invertible modules are isomorphic to some fractional ideal. The point is that there is an isomorphism between $\op{Cl}K$ and the isomorphism classes of line bundles under the group operation $\otimes$. (This isomorphism comes from the fact that principal ideals are the ones giving the trivial line bundle.)
\end{remark}
Let's explain why we're calling these vector bundles.
% V15.1
\begin{example}
	Take $X\coloneqq\PP^1_k=\Proj k[X_0,X_1]$ so that $U_0=\Spec k[x_{1/0}]$ and $U_1=\Spec k[x_{0/1}]$ are our affine open pieces. To build a line bundle $\mc F$ on $X$, we note that $\mc F|_{U_0}$ must be $k[x_{1/0}]$ as discussed above, and we note that $\mc F|_{U_1}$ must be $k[x_{0/1}]$ for the same reason. We need a way to glue these together to a sheaf on $X$, so let's glue by
	\[\arraycolsep=1.4pt\begin{array}{cccccccc}
		k[x_{1/0},x_{1/0}^{-1}] &\simeq& k[x_{0/1},x_{0/1}^{-1}] \\
		f &\mapsto& x_{0/1}f \\
		x_{1/0}f &\mapsfrom& f
	\end{array}\]
	which we can check to satisfy the cocycle condition. This gives the sheaf $\OO_X(1)$, where the $1$ here refers to a twist which we will explain later.
\end{example}
From the above example, we can more or less feel this is very similar to constructing a line bundle: we built something on $X$ which locally looked like the line $\AA^1_k$.
\begin{example}
	More generally, we can take $X\coloneqq\PP^n_k$ covered by the affine open subschemes
	\[U_i\coloneqq\Spec\frac{k[x_{0/i},\ldots,x_{n/i}]}{(x_{i/i}-1)}.\]
	A line bundle $\mc F$ on $X$ must be locally isomorphic to $U_i$ on each $U_i$, so to glue these together, we pick up some integer $m$ and glue by
	\[\arraycolsep=1.4pt\begin{array}{cccccccc}
		k[x_{0/i},\ldots,x_{n/i},x_{j/i}^{-1}]/(x_{i/i}-1) &\simeq& k[x_{0/j},\ldots,x_{n/j},x_{i/j}^{-1}]/(x_{j/j}-1) \\
		f &\mapsto& x_{i/j}^mf \\
		x_{j/i}^mf &\mapsfrom& f
	\end{array}\]
	which we can check to satisfy the cocycle condition. This gives the sheaf $\OO_X(m)$.
\end{example}
Next time we will give a $\Proj$ construction for quasicoherent sheaves, which will recover the above.

\end{document}