% !TEX root = ../notes.tex

\documentclass[../notes.tex]{subfiles}

\begin{document}

\section{October 24}

We continue.

\subsection{More on Smooth Schemes}
Here is a better way to think about smoothness.
\begin{proposition}
	Fix Noetherian schemes $X$ and $Y$ of finite type over a field $k$. A morphism $f\colon X\to Y$ is smooth if and only if
	\begin{listroman}
		\item $f$ is flat, and
		\item the fiber $X_{\overline y}$ is smooth over $\overline k$ for any geometric point $\overline y\in Y(\overline k)$.
	\end{listroman}
\end{proposition}
\begin{proof}
	Omitted.
	% V 26.2
\end{proof}
\begin{remark}
	The point is that we can check smoothness on geometric points, with some small coherence condition on points.
\end{remark}
Wait, what does flat mean?
\begin{definition}[Flat]
	A scheme morphism $f\colon X\to Y$ is \textit{flat} if and only if each $x\in X$ has $\OO_{X,x}$ a flat $\OO_{Y,f(x)}$-module for any point $x\in X$.
	% E Ch6, V S25, GW S14
\end{definition}
\begin{nex} \label{nex:fiber-change-nex}
	Note that $X\coloneqq\Spec k[x,y]/(xy)$ is not flat over $\Spec k[x]$ because the fibers have different dimensions: $(x)$ has one-dimensional fiber while $(x-a)$ has zero-dimensional fiber for any $a\in k^\times$. Alternatively, we can see this algebraically from the fact that $x\in k[x]$ is not a zero-divisor even though $x\in k[x,y]/(xy)$ is.
\end{nex}
From \autoref{nex:fiber-change-nex}, we can kind of feel that dimension-changing is a bad property, and it seems related to breaking flatness. Here is this check.
\begin{theorem}
	Fix a morphism $f\colon X\to Y$ of regular schemes. If all fibers have dimension $n$, then $f$ is flat.
\end{theorem}
\begin{proof}
	We omit this proof, but it is very useful.
	% V 29.2.11, E T18.16
\end{proof}
\begin{remark}
	One can weaken the hypotheses to make $X$ Cohen--Macaulay, but we will not define this here.
\end{remark}
\begin{nex}
	One can build a morphism between smooth schemes which is not smooth. For example, define $f\colon\AA^2_k\to\AA^2_k$ to make the fibers squish in some sense; i.e., send $(x,y)\mapsto(xy,y)$.
\end{nex}
Nonetheless, we have the following result.
\begin{proposition}
	Fix a smooth morphism $f\colon X\to Y$ such that $Y$ is regular and locally Noetherian. Then $X$ is regular.
\end{proposition}
\begin{proof}
	The point is to use Nakayama's lemma. As usual, it suffices to check regularity on closed points, so fix a closed point $x\in X$. By continuity, we note that $y\coloneqq f(x)$ is also a closed point in $Y$. By induction on $\dim Y$ and using flatness, we get
	\[\dim\OO_{X_y,x}=\dim\OO_{X,x}-\dim\OO_{Y,y}.\]
	To see this, note that there is nothing to see in the case where $Y$ is zero-dimensional. Then to upgrade to the one-dimensional case, we note that regularity means that $\OO_{Y,y}$ is cutting out by a single equation, so flatness means the cutting-out polynomial is not a zero-divisor and will appropriately lift to $\OO_{X_y,x}$.

	However, we note $X_y$ is smooth, so we may generate $\mf m_{X_y,x}$ by $\dim\OO_{X_y,x}$ elements, but the map $\OO_{X,x}\to\OO_{X_y,x}$ allows us to actually pick these elements from $\OO_{X,x}$. On the other hand,
	\[\OO_{X_y,x}\simeq\OO_{X,x}/\mf m_y\OO_{X,x},\]
	but the regularity of $Y$ now forces $\mf m_{Y,y}$ to be generated by $\dim\OO_{Y,y}$ elements, which then generates $\OO_{X,x}$ by the requested number of elements as we see from the previous dimension computation.
\end{proof}
\begin{example}
	If $X$ is smooth over a field $k$, then $X$ is regular over $k$.
\end{example}
\begin{example}
	Fix a polynomial $f(x)$, and set $X\coloneqq\Spec\ZZ[1/N][x,y]/\left(y^2-f(x)\right)$ for any $N$ such that $p\mid2\disc f$ implies $p\mid N$. We computed earlier that $X$ is smooth over $\Spec\ZZ[1/N]$, so $X$ is also regular.
\end{example}
\begin{remark}
	To review, smooth schemes over a regular scheme are regular. Regular schemes have factorial local rings, which implies the scheme is normal. Lastly, a scheme being normal implies being regular in codimension $1$.
\end{remark}

\subsection{Back to Divisors}
Here is our result.
\begin{theorem} \label{thm:div-is-iso}
	Fix a Noetherian, integral scheme $X$ which is regular in codimension $1$. Now, consider the induced map
	\[\op{div}\colon\op{Pic}X\to\op{Cl}X,\]
	where $\op{Pic}X$ is of isomorphism classes of line bundles, and $\op{Cl}X$ is the class group of Weil divisors.
	\begin{listalph}
		\item If $X$ is normal, then $\op{div}$ is injective.
		\item If $X$ is locally factorial, then $\op{div}$ is an isomorphism.
	\end{listalph}
\end{theorem}
\begin{proof}[Proof of (a)]
	Unsurprisingly, we will proceed one at a time.
	\begin{enumerate}
		\item We show that the map sending pairs $(\mc L,s)$ (up to isomorphism) to Weil divisors is injective. Suppose that a line bundle $\mc L$ and a rational section $s$ gives $\op{div}(\mc L,s)=0$. This means that $s$ has no poles or zeroes on any closed integral subscheme $Y\subseteq X$ of codimension $1$. In particular, reducing to the affine case, picking an affine open subscheme $U\subseteq X$ and some point $x\in U$ of codimension $1$, we have
		\[s\in U_x.\]
		However, for a normal Noetherian ring $R$, we will have
		\[R=\bigcap_{\codim\mf p=1}R_\mf p,\]
		so it follows that we can lift $s$ to $\mc L(U)$. Gluing over all $U$, we see that $s\in\mc L(X)$.

		However, the map $\OO_X\to\mc L$ given by $1\mapsto s$ is actually an isomorphism because has no zeroes or poles on $X$. Indeed, $s$ having no zeroes implies that the isomorphism $\mc L|_U\simeq\OO_U$ for each $U$ for which $\mc L$ is locally trivial, but $\OO_X|_U\simeq\OO_U$ is an isomorphism and factors through the $1\mapsto s$ map, so we conclude that $\mc L$ is in fact trivial.

		\item We show that the map $\op{Pic}X\to\op{Cl}X$ is injective. Indeed, we can compute that $\op{div}(\mc L,s)=\op{div}f$ for some $f\in K(X)$ implies that $\op{div}(\mc L,f^{-1}s)$, so the above argument forces us to have $\mc L$ trivialized by $f^{-1}s$. So the divisor here will vanish.
		\qedhere
	\end{enumerate}
\end{proof}
To finish the proof of \autoref{thm:div-is-iso}, we need to be able to go backward from Weil divisors to line bundles. We do this by introducing Cartier divisors.
\begin{definition}[Sheaf of total fractions]
	Fix a scheme $X$. Then we define the \textit{sheaf of total fractions} $\mc K_X$ as the sheafification of the presheaf sending each open $U\subseteq X$ to $S_U^{-1}\OO_X(U)$, where $S_U\subseteq\OO_X(U)$ are the non-zero-divisors.
\end{definition}
\begin{definition}[Cartier divisor]
	Fix a scheme $X$. Then a \textit{Cartier divisor} is a global section of $\mc K_X^\times/\mc O_X^\times$. A \textit{principal Cartier divisor} is a Cartier divisor coming from a global section of $\mc K_X^\times$,
	\[\op{CaCl}X\coloneqq\Gamma(X,\mc K_X^\times/\OO_X^\times)/\Gamma(X,\mc K_X^\times).\]
\end{definition}
\begin{remark}
	Concretely, given an affine open cover $\{U_\alpha\}_{\alpha\in\lambda}$, a Cartier divisor of $X$ consists of the data $(U_\alpha,f_\alpha)$ where $f_\alpha\in S_{U_\alpha}^{-1}\mc O_X(U_\alpha)$ such that $f_\alpha/f_\beta\in\OO_{U_\alpha\cap U_\beta}^\times$. This package of data might be called an ``effective'' Cartier divisor when each $f_\alpha$ lives in $\OO_X(U_\alpha)$.
\end{remark}
\begin{restatable}{prop}{cacltopic}
	Fix a scheme $X$. We define an injective group homomorphism $\op{CaCl}X\to\op{Pic}X$ which is an isomorphism when $X$ is integral.
\end{restatable}
\begin{proof}[Construction of the map]
	Fix a Cartier divisor $D$ consisting of the data $(U_\alpha,f_\alpha)$, where $\{U_\alpha\}_{\alpha\in\lambda}$ is some affine open cover. We now define the sheaf $\OO_X(D)$ on $X$ such that
	\[\OO_X(D)|_{U_\alpha}=f_\alpha^{-1}\OO_{U_\alpha}\subseteq\mc K_{U_\alpha}.\]
	Note that this is independent of our precise choice of $U_\alpha$ and $f_\alpha$ to represent $D$ by the coherence condition. We now get a line bundle by setting $f_{ij}=f_i|_{U_i\cap U_j}\cdot f_j^{-1}|_{U_i\cap U_j}$, which makes an element of $\check H(X,\OO_X^\times)$.
\end{proof}

\end{document}