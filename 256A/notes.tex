% LTeX: enabled=false

\documentclass[openany]{book}
\usepackage[utf8]{inputenc}

\newcommand{\nirpdftitle}{256A Notes}
\usepackage{import}
\inputfrom{..}{nir}

\pagestyle{contentpage}

\title{256A: Algebraic Geometry}
\author{Nir Elber}
\date{Fall 2022}
\rhead{\textit{256A: ALGEBRAIC GEOMETRY}}

\begin{document}

\maketitle

\toctrue
\tableofcontents
\tocfalse

\newpage

\chapter{Sheaf Theory}

\epigraph{Hold tight to your geometric motivation as you learn the formal structures which have proved to be so effective in studying fundamental questions.}
{---Ravi Vakil, \cite{rising-sea}}

\foreach \n in {24,26,29,31}
{
	\subfile{aug/\n}
}

\foreach \n in {2}
{
	\subfile{sep/\n}
}

\chapter{Building Schemes}

\epigraph{when it is right, the things you reach for in life, the things you deeply hope for, they will reach back.}
{---Bianca Sparacino, \cite{strength-in-stars}}
% \epigraph{What is the shape of such a surface in the higher dimensional space? Since there are already so varied shapes among the curves on the plane, major complications might occur here. However here things are easy, because we are dealing with algebraic structures.}
% {---Felix Klein, \cite{klein-elem-math-iii}}

\foreach \n in {7,9,12,14,16,19,21}
{
	\subfile{sep/\n}
}

\chapter{Morphisms of Schemes}

\epigraph{I can assure you, at any rate, that my intentions are honourable and my results invariant, probably canonical, perhaps even functorial.}
{---Andre Weil, \cite{weil-functorial}}

\foreach \n in {23,26,28,30}
{
	\subfile{sep/\n}
}

\foreach \n in {3,5,7,10}
{
	\subfile{oct/\n}
}

\chapter{Quasicoherent Sheaves}

\epigraph{There was nothing clever to say, so I said something foolish}
{---Madeline Miller}

\subfile{oct/10sheaf}

\foreach \n in {12,14,17}
{
	\subfile{oct/\n}
}

\chapter{Divisors}

\epigraph{Every person believes that he knows what a curve is until he has learned so much mathematics that the countless possible abnormalities confuse him.}
{---Felix Klein, \cite{klein-elem-math-ii}}

\foreach \n in {19,21,24,26,28,31}
{
	\subfile{oct/\n}
}

\foreach \n in {2}
{
	\subfile{nov/\n}
}

\chapter{Ample Line Bundles}

\foreach \n in {4}
{
	\subfile{nov/\n}
}

\nirprintbib
\nirprintindex

\end{document}

% TODO
% Example 1.135 (surjectivity not on opens but stalks)
% P1.160 exactness at stalks
% S1.5.10 (constant sheaf, skyscraper sheaf, adjuction)
% S2.2.3 (gluing for projective space)
% S2.2.7 (proj structure sheaf)
% example: Spec C -> Spec C as R-varieties is not determined by what it does on closed points
% P3.85 (integral preserves dimension of closed subsets)
% rational maps
% varieties have dense points closed
% variety morphisms (on alg closure) determined by closed points
% abelian variety things

% where is the cohomology in L1.71?
% sheaf theory things without elements
% compatible germs from category theory
% sheaf on a base but from stalks

% stalks of the direct image sheaf

% flasque sheaves play the role of injective modules in cohomology
% noetherian stalks means locally noetherian scheme?

% normalization as completion
% intuition for finiteness adjectives
% why "quasiseparated"?

% is it ever illuminating to work with locally ringed spaces instead of schemes?
% do locally ringed spaces have fiber products?
% how scared should I be of gluing?
% closed map of topological spaces => integral morphism?
% functors from ring to Scheme such as A and P?
% is being separated a purely topological condition?

% why do we expect chevalley to be true?
% why should qs satisfy cancellation (geometrically)?
% should I have expected abelian varieties to be abelian?

% curve has only finitely many singularities?
% curve embedded into P^2