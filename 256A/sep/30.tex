% !TEX root = ../notes.tex

\documentclass[../notes.tex]{subfiles}

\begin{document}

\section{September 30}

We continue the proof of \autoref{thm:chev} from last class.

\subsection{Finishing Chevalley's Theorem}
Recall that we had the diagram
% https://q.uiver.app/?q=WzAsNSxbMCwwLCJcXFNwZWMgQV9zIl0sWzEsMCwiXFxTcGVjIEJfc1t4XzEsXFxsZG90cyx4X25dIl0sWzIsMCwiXFxTcGVjIEJfcyJdLFsyLDEsIlxcU3BlYyBCIl0sWzAsMSwiXFxTcGVjIEEiXSxbNCwzLCJcXHBpIl0sWzAsNCwiIiwwLHsic3R5bGUiOnsidGFpbCI6eyJuYW1lIjoiaG9vayIsInNpZGUiOiJ0b3AifX19XSxbMiwzLCIiLDIseyJzdHlsZSI6eyJ0YWlsIjp7Im5hbWUiOiJob29rIiwic2lkZSI6InRvcCJ9fX1dLFswLDEsIlxccGlfcyJdLFsxLDJdXQ==&macro_url=https%3A%2F%2Fraw.githubusercontent.com%2FdFoiler%2Fnotes%2Fmaster%2Fnir.tex
\[\begin{tikzcd}
	{\Spec A_s} & {\Spec B_s[x_1,\ldots,x_n]} & {\Spec B_s} \\
	{\Spec A} && {\Spec B}
	\arrow["\pi", from=2-1, to=2-3]
	\arrow[hook, from=1-1, to=2-1]
	\arrow[hook, from=1-3, to=2-3]
	\arrow["{\pi_s}", from=1-1, to=1-2]
	\arrow[from=1-2, to=1-3]
\end{tikzcd}\]
from spreading out Noether normalization. Note that $\pi_s$ is a dominant morphism because $B_s[x_1,\ldots,x_n]\into A_s$ is a dominant morphism. In fact, $\pi_s$ is a finite morphism by its construction from Noether normalization, so we conclude that $\pi_s(\Spec A_s)=\Spec B_s[x_1,\ldots,x_n]$ by dominance. The map $\Spec B_s[x_1,\ldots,x_n]\to\Spec B_s$ is now certainly surjective, so in total, we see that $\im\pi$ contains the open subset $\Spec B_s\subseteq\Spec B$.

This is actually good enough to finish the proof.
\begin{example}
	Set $B=\Spec k[x]$. Then the open subsets of $\Spec k[x]$ are only missing finitely many points and so will stay open!
\end{example}
Motivated by how nice Noetherian schemes are, we have the following lemma.
\begin{lemma} \label{lem:better-constructible}
	Fix a Noetherian space $Y$. A subset $E\subseteq Y$ is constructible if and only if all irreducible closed subsets $Z\subseteq Y$ has either $E\cap Z$ or $Z\setminus E$ containing a nonempty open set.
\end{lemma}
\begin{proof}
	This is by Noetherian induction. Namely, if the statement were false, we could make $E$ minimal satisfying the above condition and derive contradiction by looking at the two cases.
\end{proof}
Thus, to finish the proof, we use \autoref{lem:better-constructible}. We want to show that $\pi(X)$ is irreducible, so for some irreducible closed subset $Z\subseteq Y$ we use the same reduction steps as above to show that the restricted map
\[\pi\colon\pi^{-1}Z\to Z\]
has, for any open irreducible subset $U\subseteq\overline{\pi(Z)}$ is strictly contained in $Z$ (so that the restricted morphism is not dominant) or $\pi(U)$ contains an open nonempty subset inside $Z$ (because the argument goes through as soon as the restriction is dominant). In the former case, we will get that $Z\setminus\pi(U)$ contains a nonempty open subset, so \autoref{lem:better-constructible} kicks in to tell us that $\pi(X)$ is constructible.
\begin{remark}
	Even though the statement of \autoref{thm:chev} is non-constructive, one can concretely work through the above proof using a somewhat explicit construction from $\pi_s$. In particular, after peeling off the desired open subset $B_s$, what's left over is approximately speaking some lower-dimensional object, so we can induct downwards.
	% 8.4.4
\end{remark}

\subsection{Closed Embeddings Are Reasonable}
It will be helpful for us to collect some facts about closed embeddings before we talk about separated morphisms in a moment.
To start off, we're well overdue to show that closed morphisms are monic.
\begin{lemma} \label{lem:closed-monic}
	Closed morphisms are monic.
\end{lemma}
\begin{proof}
	We show this by hand. Fix a closed morphism $\iota\colon X\to Y$ and two morphisms $\alpha,\beta\colon S\to X$ such that $\iota\circ\alpha=\iota\circ\beta$. We need to show that $\alpha=\beta$, which we show by hand.
	\begin{itemize}
		\item On the level of topological spaces, we note that any $s\in S$ will have $\iota(\alpha(s))=\iota(\beta(s))$, so $\alpha(s)=\beta(s)$ because $\iota$ is a homeomorphism onto its image.
		\item On the level of sheaves, we are given that $\iota_*\alpha^\sharp\circ\iota^\sharp=\iota_*\beta^\sharp\circ\iota^\sharp$ as sheaf morphisms $\OO_Y\to\alpha_*\iota_*\OO_S$. Now, fix any $x\in X$, and we will show that $\alpha^\sharp_x=\beta^\sharp_x$ at $x$, which will finish by \autoref{prop:stalkmorphism}.

		As an intermediate claim, we show that the composite $\OO_{Y,\iota(x)}\to(\iota_*\OO_X)_{\iota(x)}\to\OO_{X,x}$ is surjective. Well, we know that the map $\iota^\sharp_{\iota(x)}\colon\OO_{Y,\iota(x)}\to(\iota_*\OO_X)_{\iota(x)}$ is already surjective, so we just need to show that the map $(\iota_*\OO_X)_{\iota(x)}\to\OO_{X,x}$ is surjective. For this, we pick up some germ $[(U,s)]\in\OO_{X,x}$ with $U$ containing $x$.

		Now, we note that $\iota$ is a homeomorphism onto its image, so $\iota(U)\subseteq\im\iota$ is an open subset, so there is some open $V\subseteq Y$ such that $\iota(U)=V\cap\im\iota$, which in particular means $\iota^{-1}V=U$. So we see $[(V,s)]\in(\iota_*\OO_X)_{\iota(x)}$ will go to $[(U,s)]$ under the canonical map.

		We now show $\alpha^\sharp_x=\beta^\sharp_x$. For this, we track through the large diagram
		% https://q.uiver.app/?q=WzAsOSxbMCwwLCIoXFxPT19ZKV97XFxpb3RhKHgpfSJdLFsxLDAsIihcXGlvdGFfKlxcT09fWClfe1xcaW90YSh4KX0iXSxbMSwxLCJcXE9PX3tYLHh9Il0sWzIsMSwiKFxcYWxwaGFfKlxcT09fUylfeCJdLFsyLDAsIihcXGlvdGFfKlxcYWxwaGFfKlxcT09fUylfe1xcaW90YSh4KX0iXSxbMywwLCJbKFYscyldIl0sWzQsMCwiWyhWLFxcYWxwaGFeXFxzaGFycF97XFxpb3RhXnstMX1WfXMpXSJdLFszLDEsIlsoXFxpb3RhXnstMX1WLHMpXSJdLFs0LDEsIlsoXFxpb3RhXnstMX1WLFxcYWxwaGFeXFxzaGFycF97XFxpb3RhXnstMX1WfXMpXSJdLFswLDIsIiIsMCx7InN0eWxlIjp7ImhlYWQiOnsibmFtZSI6ImVwaSJ9fX1dLFswLDEsIiIsMix7InN0eWxlIjp7ImhlYWQiOnsibmFtZSI6ImVwaSJ9fX1dLFsxLDIsIiIsMix7InN0eWxlIjp7ImhlYWQiOnsibmFtZSI6ImVwaSJ9fX1dLFsxLDQsIihcXGlvdGFfKlxcYWxwaGFeXFxzaGFycClfeCJdLFs0LDNdLFsyLDMsIlxcYWxwaGFeXFxzaGFycF94Il0sWzUsNiwiIiwwLHsic3R5bGUiOnsidGFpbCI6eyJuYW1lIjoibWFwcyB0byJ9fX1dLFs2LDgsIiIsMCx7InN0eWxlIjp7InRhaWwiOnsibmFtZSI6Im1hcHMgdG8ifX19XSxbNSw3LCIiLDIseyJzdHlsZSI6eyJ0YWlsIjp7Im5hbWUiOiJtYXBzIHRvIn19fV0sWzcsOCwiIiwyLHsic3R5bGUiOnsidGFpbCI6eyJuYW1lIjoibWFwcyB0byJ9fX1dXQ==&macro_url=https%3A%2F%2Fraw.githubusercontent.com%2FdFoiler%2Fnotes%2Fmaster%2Fnir.tex
		\[\begin{tikzcd}
			{(\OO_Y)_{\iota(x)}} & {(\iota_*\OO_X)_{\iota(x)}} & {(\iota_*\alpha_*\OO_S)_{\iota(x)}} & {[(V,s)]} & {[(V,\alpha^\sharp_{\iota^{-1}V}s)]} \\
			& {\OO_{X,x}} & {(\alpha_*\OO_S)_x} & {[(\iota^{-1}V,s)]} & {[(\iota^{-1}V,\alpha^\sharp_{\iota^{-1}V}s)]}
			\arrow[two heads, from=1-1, to=2-2]
			\arrow[two heads, from=1-1, to=1-2]
			\arrow[two heads, from=1-2, to=2-2]
			\arrow["{(\iota_*\alpha^\sharp)_x}", from=1-2, to=1-3]
			\arrow[from=1-3, to=2-3]
			\arrow["{\alpha^\sharp_x}", from=2-2, to=2-3]
			\arrow[maps to, from=1-4, to=1-5]
			\arrow[maps to, from=1-5, to=2-5]
			\arrow[maps to, from=1-4, to=2-4]
			\arrow[maps to, from=2-4, to=2-5]
		\end{tikzcd}\]
		where the left triangle is surjective as shown above, and the right square commutes as shown. Now, for any germ $s\in\OO_{X,x}$, we can compute what $\alpha^\sharp_x$ will do to $s$ by pulling $s$ back to a germ in $\OO_{Y,\iota(y)}$, then going forward across to $(\iota_*\alpha_*\OO_S)_{\iota(x)}$, and lastly going down to $(\alpha_*\OO_S)_x$.
		
		However, all of these steps are independent of $\alpha^\sharp$ in that the composite along the top is the same if we replace $\alpha^\sharp$ with $\beta^\sharp$, and the last downward step only depends on the topological data, which we know from the above aligns. This finishes.
		\qedhere
	\end{itemize}
\end{proof}
\begin{corollary}
	Closed morphisms are quasiseparated.
\end{corollary}
\begin{proof}
	Monomorphisms are quasiseparated by \autoref{ex:mono-is-qs}, so \autoref{lem:closed-monic} finishes.
\end{proof}
Here is composition.
\begin{lemma} \label{lem:pushforward-preserves-surjective}
	Fix a continuous map $f\colon X\to Y$ which is a homeomorphism onto a closed set of $Y$. If $\varphi\colon\mc F\to\mc G$ is a sheaf morphism which is surjective on stalks, then $f_*\varphi\colon f_*\mc F\to f_*\mc G$ is still surjective on stalks.
\end{lemma}
\begin{proof}
	Fix some $y\in Y$ so that we want to show $(f_*\varphi)_y$ is surjective. If $y\notin\im f=\overline{\im f}$, then note $(f_*\mc G)_y=0$ by \autoref{rem:zero-out-of-image}, so there is nothing to say.

	Otherwise, we have $y\in\im f$, so say $y=f(x)$ for $x\in X$. Starting with a germ $[(U,s)]\in (f_*\mc G)_{f(x)}$, we take this to $[(f^{-1}U,s)]\in\mc G_x$, where we can use the surjectivity of $\varphi_x$ to find $[(V,t)]\in\mc F_x$ with $\varphi_V(t)|_W=s|_W$ for some small $W$ containing $x$; restricting $V$ enough, we may assume $\varphi_V(t)=s|_W$ and that $V\subseteq f^{-1}U$.
	
	But now we see that the restriction $f|_{f^{-1}U}\colon f^{-1}U\to U$ is a homeomorphism, so $f(V)\subseteq U$ is an open subset of $U$ and hence of $Y$. At this point we recognize $f^{-1}(f(V))=V$, so the germ $[(f(V),t)]\in(f_*\mc F)_{f(x)}$ will do the trick. This finishes.
\end{proof}
\begin{lemma} \label{cor:closed-embed-compose}
	The class of closed embeddings is preserved by composition.
\end{lemma}
\begin{proof}
	Fix closed embeddings $\varphi\colon X\to Y$ and $\psi\colon Y\to Z$. We need to show that $(\psi\circ\varphi)\colon X\to Z$ is a closed embedding. There are two checks.
	\begin{itemize}
		\item Note that $\varphi$ is a homeomorphism onto a closed subset $\varphi(X)\subseteq Y$. Further, $\psi$ is a homeomorphism onto a closed subset $\psi(Y)\subseteq Z$, so $\psi$ restricted to $\varphi(X)$ will still be a homeomorphism onto $\psi(\varphi(X))$. Notably, $\varphi(X)$ implies that $\psi(\varphi(X))$ is closed.
		\item We check that $(\psi\circ\varphi)^\sharp\colon\OO_Z\to\psi_*\varphi_*\OO_X$ is surjective on stalks. Well, fixing some $z$, we want to show that the composite
		\[\OO_{Z,z}\stackrel{\psi^\sharp_z}\to(\psi_*\OO_Y)_z\stackrel{(\psi_*\varphi^\sharp)_z}\to(\psi_*\varphi_*\OO_X)_z\]
		is surjective. Note that $\psi^\sharp_z$ is already surjective, so it suffices to show that $(\psi_*\varphi^\sharp)_z$ is surjective, which follows from \autoref{lem:pushforward-preserves-surjective}.
		\qedhere
	\end{itemize}
\end{proof}
As we should expect, we now move towards being local on the target.
\begin{lemma} \label{lem:closed-emb-affine-local-1}
	Suppose that $\pi\colon X\to Y$ is a closed embedding. Then, for any open subset $U\subseteq Y$, the restriction $\pi|_{\pi^{-1}U}\colon\pi^{-1}U\to U$ is still a closed embedding.
\end{lemma}
\begin{proof}
	Unsurprisingly, there are two checks.
	\begin{itemize}
		\item On the level of topological spaces, we note that the inverse continuous map $\varphi\colon\pi(Y)\to X$ will restrict to an inverse continuous map $\varphi|_U\colon (U\cap\pi(Y))\to\pi^{-1}(U)$, so we see that $\pi|_{\pi^{-1}U}$ is at least still a homeomorphism onto its image. (Notably, $U\cap\pi(Y)=\pi(\pi^{-1}(U))$.) Additionally, we see that $\im\pi|_{\pi^{-1}U}=U$, so the image is a closed subset of $U$ (in fact, all of $U$).
		\item On stalks, we fix some $y\in U$, and we want to show that the map
		\[(\pi|_{\pi^{-1}U})^\sharp_y\colon(\OO_Y|_U)_y\to\pi_*(\OO_X|_{\pi^{-1}U})_y\]
		is surjective. Well, by definition of our restriction in \autoref{lem:restrictmorphism}, $(\pi|_{\pi^{-1}U})^\sharp_V$ just behaves as $\pi^\sharp_V$ for any open $V\subseteq U$, so the definition of the map on stalks will match as $\pi^\sharp_y$, which we already know is surjective.
		\qedhere
	\end{itemize}
\end{proof}
And here is the converse.
\begin{lemma} \label{lem:closed-emb-affine-local-2}
	Fix a scheme morphism $\pi\colon X\to Y$. Suppose that we have an open cover $\mc U$ on $Y$ such that $\pi|_{\pi^{-1}U}\colon\pi^{-1}U\to U$ is a closed embedding for each $U\in\mc U$. Then $\pi$ is a closed embedding.
\end{lemma}
\begin{proof}
	As usual, we have two checks. For brevity, let the open cover by $\{V_\alpha\}_{\alpha\in\lambda}$ and $\pi_\alpha\colon U_\alpha\to V_\alpha$ by the restrictions, where $U_\alpha\coloneqq\pi^{-1}V_\alpha$.
	\begin{itemize}
		\item On the level of topological spaces, we know that $\pi_\alpha$ onto its image $\im\pi_\alpha$, and that $\im\pi_\alpha$ is always a closed set. For one, we note that
		\[\im\pi_\alpha=\pi\left(\pi^{-1}V_\alpha\right)=\im\pi\cap V_\alpha\]
		is a closed subset in $V_\alpha$ for each $\alpha$. Thus, we conclude that $(Y\setminus\im\pi)\cap V_\alpha=V_\alpha\setminus\im\pi$ is an open subset in $V_\alpha$ and therefore in $Y$ for each $\alpha$, so
		\[Y\setminus\im\pi=\bigcup_{\alpha\in\lambda}(Y\setminus\im\pi)\cap V_\alpha\]
		must be an open subset of $Y$, so $\im\pi$ is closed.

		It remains to show that $\pi$ is a homeomorphism onto $\im\pi$. Well, for each $\alpha$, we are promised a local inverse continuous function $\varphi_\alpha\colon(\im\pi\cap V_\alpha)\to U_\alpha$. Now, for any $y\in\im\pi$, we find some $\alpha$ such that $y\in V_\alpha$ and define
		\[\varphi(y)\coloneqq\varphi_\alpha(y).\]
		Note that these functions glue appropriately: if $\pi(x)\in V_\alpha\cap V_\beta$, then $\varphi_\alpha(\pi(x))=x=\varphi_\beta(\pi(x))$. Thus, these glue to a continuous function $\varphi$ by \autoref{exe:sheafex}. Additionally, we see any $x\in X$ has $f(x)\in V_\alpha$ for some $\alpha$ and therefore
		\[\varphi(\pi(x))=\varphi_\alpha(\pi_\alpha(x))=x,\]
		any $y\in\im\pi$ has $y\in V_\alpha$ for some $\alpha$ and therefore
		\[\pi(\varphi(y))=\pi_\alpha(\varphi_\alpha(y))=y.\]
		Thus, $\varphi$ is indeed a continuous inverse for $\pi$.

		\item We show that $\pi^\sharp\colon\OO_Y\to\pi_*\OO_X$ is surjective on stalks. Fix any $y\in Y$, and pick up some germ $[(V,f)]\in(\pi_*\OO_X)_y$. Now, $y\in V_\alpha$ for some $\alpha$, so we may restrict the germ to $V_\alpha$ to assume that $V\subseteq V_\alpha$.

		In particular, we thus see that actually $[(V,f)]\in((\pi_\alpha)_*\OO_X|_{U_\alpha})_y$, so the surjectivity of $\pi^\sharp_\alpha$ on stalks tells us that there is some germ $[(V,g)]\in(\OO_Y|_{V_\alpha})_y$ (namely, we may assume $g\in\OO_Y(V)$ with $U\subseteq V_\alpha$) with
		\[(\pi_\alpha^\sharp)_y([(V,g)])=[(U,\pi^\sharp_V(g))]=[(V,f)].\]
		This finishes our surjectivity check.
		\qedhere
	\end{itemize}
\end{proof}
\begin{corollary} \label{cor:closed-emb-affine-local}
	The class of closed embeddings is local on the target.
\end{corollary}
\begin{proof}
	Combine \autoref{lem:closed-emb-affine-local-1} and \autoref{lem:closed-emb-affine-local-2}.
\end{proof}
We now show that we are preserved by base change. We will appeal to \autoref{prop:affineclosedsubschemes} even though we will shortly give a different proof of this result. There will be no circular logic.
\begin{lemma} \label{lem:closed-emb-is-finite}
	Closed embeddings are finite and in particular affine.
\end{lemma}
\begin{proof}
	Fix a closed embedding $\pi\colon X\to Y$; because being finite is affine-local on the target by \autoref{cor:finite-is-reasonable}, it suffices to show that $\pi|_{\pi^{-1}U}\colon\pi^{-1}U\to U$ is finite for each affine open $U\subseteq Y$. (Namely, fix any affine open cover of $Y$.)
	
	Well, note $\pi|_{\pi^{-1}U}$ remains a closed embedding by \autoref{lem:closed-emb-affine-local-1}, so we might as well rename $Y$ to $U$ and $X$ to $\pi^{-1}U$ and $\pi$ to $\pi|_{\pi^{-1}U}$. Then $\pi\colon X\to Y$ is a closed embedding, where $Y$ is affine, and we want to show that $\pi$ is finite. Further, because isomorphisms are closed embeddings by \autoref{ex:iso-is-closed} and closed embeddings are preserved by composition by \autoref{cor:closed-embed-compose}, we see that we actually have a closed embedding
	\[X\to Y\to\Spec A,\]
	so we can just assume that $Y$ takes the form $\Spec A$. Thus, by \autoref{prop:affineclosedsubschemes}, we must have $\pi$ factor as
	\[X\cong\Spec A/I\to\Spec A\]
	for some ideal $I\subseteq A$. In particular, we see that $\pi^\sharp$ is surjective on global sections by composing the above morphisms, which means that $\OO_X(X)$ is in fact finitely generated as an $A$-algebra by $\pi^\sharp$ (in fact, with one generator).
\end{proof}
\begin{lemma} \label{lem:closed-embed-base-change}
	The class of closed embeddings is preserved by base change.
\end{lemma}
\begin{proof}
	Suppose we have a pullback square
	% https://q.uiver.app/?q=WzAsNCxbMCwwLCJYXFx0aW1lc19TWSJdLFsxLDAsIlgiXSxbMCwxLCJZIl0sWzEsMSwiUyJdLFsyLDMsIlxccHNpX1kiXSxbMCwxLCJcXHBpX1giXSxbMCwyLCJcXHBpX1kiXSxbMSwzLCJcXHBzaV9YIl0sWzAsMywiIiwyLHsic3R5bGUiOnsibmFtZSI6ImNvcm5lciJ9fV1d&macro_url=https%3A%2F%2Fraw.githubusercontent.com%2FdFoiler%2Fnotes%2Fmaster%2Fnir.tex
	\[\begin{tikzcd}
		{X\times_SY} & X \\
		Y & S
		\arrow["{\psi_Y}", from=2-1, to=2-2]
		\arrow["{\pi_X}", from=1-1, to=1-2]
		\arrow["{\pi_Y}", from=1-1, to=2-1]
		\arrow["{\psi_X}", from=1-2, to=2-2]
		\arrow["\lrcorner"{anchor=center, pos=0.125}, draw=none, from=1-1, to=2-2]
	\end{tikzcd}\]
	of schemes such that $\psi_Y$ is a closed embedding. We would like to show that $\pi_X$ is a closed embedding. Notably, because being a closed embedding is affine-local on the target by \autoref{cor:closed-emb-affine-local} (in fact, local on the target), we may use \autoref{lem:base-change-reduce-to-affine} to assume that $X$ and $S$ are affine.

	Thus, $\psi_Y\colon Y\to S$ being a closed embedding forces $Y$ to be affine by \autoref{prop:affineclosedsubschemes}, where the induced map $\psi_Y^\sharp\colon\OO_S(S)\to\OO_Y(Y)$ on global sections is surjective. So for brevity we set $R\coloneqq\OO_S(S)$ and $A\coloneqq\OO_X(X)$ and $B\coloneqq\OO_Y(Y)$. Thus, by \autoref{lem:affine-fp}, we see that
	\[X\times_SY\simeq\Spec A\times_{\Spec R}\Spec B\simeq\Spec A\otimes_RB,\]
	so we might as well set $X\times_SY$ to be $\Spec A\otimes_RB$, where the canonical projection $X\times_SY\to X$ is given by chaining the canonical maps $\Spec A\otimes_RB\to\Spec A$ (from $A\to A\otimes_RB$) with the canonical isomorphism $\Spec A\otimes X$.

	Notably, because isomorphisms are closed embeddings by \autoref{ex:iso-is-closed}, and closed embeddings are preserved by composition by \autoref{cor:closed-embed-compose}, it suffices to show that the map $\Spec A\otimes_RB\to\Spec A$ is a closed embedding, for which it suffices by \autoref{prop:affineclosedsubschemes} to show that the canonical map $A\to A\otimes_RB$ is surjective.

	Well, we are given that the canonical map $R\to B$ is surjective. Thus, for any tensor $a\otimes b\in A\otimes_RB$, we can find $r\in R$ which goes to $b$, so $a\otimes b=ra\otimes1$, which comes from $ra\in A$ through the inclusion $A\to A\otimes_RB$. Because $A\otimes_RB$ is generated by these elements $a\otimes b$, we are done.
	\qedhere
\end{proof}

\subsection{Locally Closed Embeddings Are Reasonable}
While we're talking about embeddings, we'll go ahead and show that locally closed embeddings have the usual adjectives.
\begin{remark} \label{rem:loc-closed-is-monic}
	Locally closed embeddings are a closed embedding followed by an open embedding, both of which are monic by \autoref{lem:closed-monic} and \autoref{cor:open-embed-monic}, respectively. Because the composite of monomorphisms remains monic, we see that locally closed embeddings are monic; for example, they are quasiseparated by \autoref{ex:mono-is-qs}.
\end{remark}
\begin{lemma} \label{lem:loc-closed-becomes-closed-embed}
	A locally closed embedding with closed image (topologically!) is a closed embedding.
\end{lemma}
\begin{proof}
	Fix a locally closed embedding $\pi\colon X\to Y$ with closed image. By definition, we can factor $\pi$ as $\pi=\iota\circ\pi'$, where $\pi'\colon X\to U$ is a closed embedding, and $\iota\colon U\to Y$ is an open embedding. We have two checks.
	\begin{itemize}
		\item Topologically, both $\pi'$ and $\iota$ are homeomorphisms onto their image, so their composite is as well: let $\iota_0\colon U\to U$ be the continuous inverse of $\iota$, and let $\pi'_0\colon\iota(X)\to X$ be the continuous inverse of $\pi'$.
		
		Then we note $\iota_0|_{\pi'(X)}\colon\pi'(X)\to\pi'(X)$ is still a continuous function, and we can check that $\pi'_0\circ\iota_0|_{\pi'(X)}$ is the inverse we're looking for, which is a matter of writing down the equations and remembering $\iota\circ\iota_0$ and $\iota_0\circ\iota$ and $\pi\circ\pi_0$ and $\pi_0\circ\pi$ are all identities.

		\item On the structure sheaf, we need to check that $\pi^\sharp\colon\OO_Y\to\pi_*\OO_X$ is surjective on stalks. By \autoref{rem:zero-out-of-image}, we only have to check this on $\overline{\im\pi}=\im\pi$. Well, for some $x\in X$, we pick up some germ $[(V,f)]\in(\pi_*\OO_X)_{\pi(x)}$, which means $f\in\OO_X(\pi^{-1}V)$.

		Because $\pi$ factors through $U$, we see that $\pi(x)\in U$, so we may restrict $V$ to $V\cap U$, meaning we may assume $V\subseteq U$, so actually $[(V,f)]\in(\pi'_*\OO_X)_{\pi'(x)}$. However, $\pi'$ is a closed embedding, so $(\pi')^\sharp\colon\OO_Y|_U\to(\pi'_*\OO_X)$ is surjective on stalks, so we may find $[(W,g)]\in\OO_Y|_U$ such that
		\[\pi^\sharp_{f(x)}([(W,g)])=[(W,\pi^\sharp_Wg)]=[(W,(\pi')^\sharp_Wg)]=(\pi')^\sharp_{f(x)}([W,g])=[(V,f)],\]
		which is what we wanted.
		\qedhere
	\end{itemize}
\end{proof}
It turns out that being preserved by composition is the hardest check, so we will do it last. Let's begin with local on the target.
\begin{lemma}
	The class of locally closed embeddings is affine-local on the target.
\end{lemma}
\begin{proof}
	Fix a scheme morphism $\pi\colon X\to Y$ and an affine open cover $\{Y_\alpha\}_{\alpha\in\lambda}$ of $Y$. In one direction, if $\pi$ is a locally closed embedding, then we have an open subscheme $U\subseteq Y$ such that $\pi$ factors through $\pi'\colon X\to U$ as a closed embedding. Now, for any open $V\subseteq Y$, we see that $\pi|_{\pi^{-1}V}$ becomes
	\[\pi^{-1}V\stackrel{\pi'|_{\pi^{-1}V}}\to(U\cap V)\into V.\]
	Notably, $\pi'|_{\pi^{-1}V}=\pi'|_{(\pi')^{-1}(U\cap V)}$ is a closed embedding by \autoref{lem:closed-emb-affine-local-1}, and $U\cap V\into V$ remains an open embedding (it's still an embedding of open subsets, and the structure sheaves continue to match), so we see that $\pi|_{\pi^{-1}V}$ is still a locally closed embedding.
	
	In the other direction, give $Y$ an affine open cover $\{Y_\alpha\}_{\alpha\in\lambda}$ such that the restrictions $\pi_\alpha\colon X_\alpha\to Y_\alpha$ are all locally closed embeddings, where $X_\alpha\coloneqq\pi^{-1}Y_\alpha$. This means that we are promised open subschemes $U_\alpha\subseteq Y_\alpha$ such that each $\pi_\alpha$ factors as
	\[X_\alpha\stackrel{\pi'_\alpha}\to U_\alpha\into Y_\alpha,\]
	where $\pi'_\alpha$ is a closed embedding. Thus, we set
	\[U\coloneqq\bigcup_{\alpha\in\lambda}U_\alpha.\]
	Notably, $U$ is covered by the $U_\alpha$, and $X_\alpha\subseteq(\pi'_\alpha)^{-1}(U_\alpha)\subseteq\pi^{-1}Y_\alpha=X_\alpha$, so in fact $\pi$ factors through $U$ as $\pi'$ as
	\[X\stackrel{\pi'}\to U\into Y.\]
	Further, we see that the restrictions $\pi'|_{\pi^{-1}U_\alpha}\colon X_\alpha\to U_\alpha$ are all closed embeddings, so \autoref{lem:closed-emb-affine-local-2} says that $\pi'$ is a closed embedding, which finishes.
\end{proof}
Now, here is base-change.
\begin{lemma}
	The class of locally closed embeddings is preserved by base change.
\end{lemma}
\begin{proof}
	Fix a closed embedding $\iota\colon X\to U$ and an open embedding $\jmath\colon Z\to U$. Additionally, picking up some auxiliary scheme $Z$ with a map $\varphi\colon Z\to Y$, we would like to show that the induced map $X\times_YZ\to Z$ in
	% https://q.uiver.app/?q=WzAsNCxbMSwwLCJaIl0sWzEsMSwiWSJdLFswLDEsIlgiXSxbMCwwLCJYXFx0aW1lc19ZWiJdLFswLDEsIlxccGkiXSxbMiwxLCJcXGptYXRoXFxjaXJjXFxpb3RhIl0sWzMsMF0sWzMsMl1d&macro_url=https%3A%2F%2Fraw.githubusercontent.com%2FdFoiler%2Fnotes%2Fmaster%2Fnir.tex
	\begin{equation}
		\begin{tikzcd}
			{X\times_YZ} & Z \\
			X & Y
			\arrow["\varphi", from=1-2, to=2-2]
			\arrow["\jmath\circ\iota", from=2-1, to=2-2]
			\arrow[from=1-1, to=1-2]
			\arrow[from=1-1, to=2-1]
		\end{tikzcd} \label{eq:start-loc-closed-fp}
	\end{equation}
	is also a locally closed embedding. For this, we expand the diagram into
	% https://q.uiver.app/?q=WzAsNixbMiwwLCJaIl0sWzIsMSwiWSJdLFsxLDAsIlxcdmFycGhpXnstMX1VIl0sWzAsMSwiWCJdLFsxLDEsIlUiXSxbMCwwLCJYXFx0aW1lc19ZWiJdLFswLDEsIlxcdmFycGhpIl0sWzIsMCwiXFxqbWF0aCciXSxbNCwxLCJcXGptYXRoIl0sWzMsNCwiXFxpb3RhIl0sWzIsNCwiXFx2YXJwaGknIl0sWzUsMywiXFxwaSIsMl0sWzUsMiwiXFxpb3RhJyIsMCx7InN0eWxlIjp7ImJvZHkiOnsibmFtZSI6ImRhc2hlZCJ9fX1dXQ==&macro_url=https%3A%2F%2Fraw.githubusercontent.com%2FdFoiler%2Fnotes%2Fmaster%2Fnir.tex
	\[\begin{tikzcd}
		{X\times_YZ} & {\varphi^{-1}U} & Z \\
		X & U & Y
		\arrow["\varphi", from=1-3, to=2-3]
		\arrow["{\jmath'}", from=1-2, to=1-3]
		\arrow["\jmath", from=2-2, to=2-3]
		\arrow["\iota", from=2-1, to=2-2]
		\arrow["{\varphi'}", from=1-2, to=2-2]
		\arrow["\pi"', from=1-1, to=2-1]
		\arrow["{\iota'}", dashed, from=1-1, to=1-2]
	\end{tikzcd}\]
	where we note that the map $\iota'$ is induced to make the diagram commute because the right square is a pullback square, so we can plug the commutative diagram \autoref{eq:start-loc-closed-fp} into it. In particular, we have that the projection $X\times_YZ\to Z$ is $\jmath'\circ\iota'$.

	Now, we note that the left square is a pullback square by \autoref{lem:smalls-to-big-pullback}. As such, because $\iota$ is a closed embedding, and closed embeddings are preserved by base change by \autoref{lem:closed-embed-base-change}, we know that $\iota'$ is also a closed embedding. So we have written the map $X\times_YZ\to Z$ as the composite $\jmath'\circ\iota'$ where $\iota'$ is a closed embedding and $\jmath'$ an open embedding, which shows that $\jmath'\circ\iota'$ is a locally closed embedding.
\end{proof}
We begin with composition.
\begin{lemma}
	Suppose that $\jmath\colon X\to V$ is an open embedding, and $\iota\colon V\to Y$ is a closed embedding. Then the composite $(\iota\circ\jmath)\colon X\to Y$ is a locally closed embedding.
\end{lemma}
\begin{proof}
	For psychological ease, we identify $V$ topologically with its image in $Y$, and we identify $X$ topologically with its image in $V$ (and thus in $Y$) as well. In particular, on the level of topological spaces, $X$ is an open subset of the closed subspace $V\subseteq Y$, so we can {find an open subset $U\subseteq Y$ such that $X=U\cap V$}. Let $\jmath'\colon U\into Y$ be the corresponding open embedding, and we claim that we can induce a closed embedding $\iota\colon X\to U$ making the diagram
	% https://q.uiver.app/?q=WzAsNCxbMCwwLCJYIl0sWzEsMCwiViJdLFsxLDEsIlkiXSxbMCwxLCJVIl0sWzAsMSwiXFxqbWF0aCJdLFsxLDIsIlxcaW90YSJdLFszLDIsIlxcam1hdGgnIl0sWzAsMywiXFxpb3RhJyIsMl1d&macro_url=https%3A%2F%2Fraw.githubusercontent.com%2FdFoiler%2Fnotes%2Fmaster%2Fnir.tex
	\begin{equation}
		\begin{tikzcd}
			X & V \\
			U & Y
			\arrow["\jmath", from=1-1, to=1-2]
			\arrow["\iota", from=1-2, to=2-2]
			\arrow["{\jmath'}", from=2-1, to=2-2]
			\arrow["{\iota'}"', from=1-1, to=2-1]
		\end{tikzcd} \label{eq:loc-emb-square}
	\end{equation}
	commute; this will finish because it implies that $\iota\circ\jmath=\jmath'\circ\iota'$ is the composite of a closed embedding followed by an open embedding and therefore a locally closed embedding.

	At the very least, note that $X=V\cap U$ is a closed subset of $U$, so we will simply define $\iota'\colon X\to U$ as the identity on topological spaces. In particular, we have that $\iota'$ is a homeomorphism from $X$ onto a closed subset of $U$. We also see that \autoref{eq:loc-emb-square} now commutes on the level of topological spaces because all the maps are the identity on the level of topological spaces.

	It remains to talk about sheaves. In particular, for each open subset $U'\subseteq U$, we must exhibit a map $(\iota')^\sharp_{U'}\colon\OO_U(U')\to\iota'_*\OO_X(U')$. Well, we note that $\OO_U(U')=\OO_Y(U)$, and $\iota'_*\OO_X(U')=\OO_X(X\cap U')=\OO_V(X\cap U')$ (note $X\cap U'$ is open in $V$), so we define our map $(\iota')^\sharp_{U'}$ as either of the diagonal morphisms in
	% https://q.uiver.app/?q=WzAsNCxbMCwwLCJcXE9PX1koVScpIl0sWzEsMCwiXFxPT19WKFUnXFxjYXAgVikiXSxbMSwxLCJcXE9PX1YoVSdcXGNhcCBYKSJdLFswLDEsIlxcT09fWShVJ1xcY2FwIFUpIl0sWzAsMSwiXFxpb3RhXlxcc2hhcnBfe1UnfSJdLFsxLDIsIlxcb3B7cmVzfSJdLFszLDIsIlxcaW90YV5cXHNoYXJwX3tVJ1xcY2FwIFV9Il0sWzAsMywiXFxvcHtyZXN9IiwyXV0=&macro_url=https%3A%2F%2Fraw.githubusercontent.com%2FdFoiler%2Fnotes%2Fmaster%2Fnir.tex
	\[\begin{tikzcd}
		{\OO_Y(U')} & {\OO_V(U'\cap V)} \\
		{\OO_Y(U'\cap U)} & {\OO_V(U'\cap X)}
		\arrow["{\iota^\sharp_{U'}}", from=1-1, to=1-2]
		\arrow["{\op{res}}", from=1-2, to=2-2]
		\arrow["{\iota^\sharp_{U'\cap U}}", from=2-1, to=2-2]
		\arrow["{\op{res}}"', from=1-1, to=2-1]
	\end{tikzcd}\]
	where the diagram commutes because $\iota^\sharp$ is a sheaf morphism. To see that $(\iota')^\sharp$ assembles into a sheaf map, we pick up open sets $U''\subseteq U'$ and note that the diagram
	% https://q.uiver.app/?q=WzAsOCxbMCwwLCJcXE9PX1koVScpIl0sWzEsMCwiXFxPT19WKFUnXFxjYXAgVikiXSxbMSwxLCJcXE9PX1YoVScnXFxjYXAgVikiXSxbMCwxLCJcXE9PX1koVScnKSJdLFsyLDAsIlxcT09fVihVJ1xcY2FwIFgpIl0sWzIsMSwiXFxPT19WKFUnJ1xcY2FwIFgpIl0sWzMsMCwiXFxpb3RhJ18qXFxPT19YKFUnKSJdLFszLDEsIlxcaW90YSdfKihVJycpIl0sWzAsMSwiXFxpb3RhXlxcc2hhcnBfe1UnfSJdLFsxLDIsIlxcb3B7cmVzfSJdLFszLDIsIlxcaW90YV5cXHNoYXJwX3tVJyd9Il0sWzAsMywiXFxvcHtyZXN9IiwyXSxbMSw0LCJcXG9we3Jlc30iXSxbMiw1LCJcXG9we3Jlc30iXSxbNCw1LCJcXG9we3Jlc30iXSxbNCw2LCIiLDAseyJsZXZlbCI6Miwic3R5bGUiOnsiaGVhZCI6eyJuYW1lIjoibm9uZSJ9fX1dLFs1LDcsIiIsMCx7ImxldmVsIjoyLCJzdHlsZSI6eyJoZWFkIjp7Im5hbWUiOiJub25lIn19fV0sWzYsNywiXFxvcHtyZXN9Il1d&macro_url=https%3A%2F%2Fraw.githubusercontent.com%2FdFoiler%2Fnotes%2Fmaster%2Fnir.tex
	\[\begin{tikzcd}
		{\OO_Y(U')} & {\OO_V(U'\cap V)} & {\OO_V(U'\cap X)} & {\iota'_*\OO_X(U')} \\
		{\OO_Y(U'')} & {\OO_V(U''\cap V)} & {\OO_V(U''\cap X)} & {\iota'_*(U'')}
		\arrow["{\iota^\sharp_{U'}}", from=1-1, to=1-2]
		\arrow["{\op{res}}", from=1-2, to=2-2]
		\arrow["{\iota^\sharp_{U''}}", from=2-1, to=2-2]
		\arrow["{\op{res}}"', from=1-1, to=2-1]
		\arrow["{\op{res}}", from=1-2, to=1-3]
		\arrow["{\op{res}}", from=2-2, to=2-3]
		\arrow["{\op{res}}", from=1-3, to=2-3]
		\arrow[Rightarrow, no head, from=1-3, to=1-4]
		\arrow[Rightarrow, no head, from=2-3, to=2-4]
		\arrow["{\op{res}}", from=1-4, to=2-4]
	\end{tikzcd}\]
	commutes: the left square commutes by naturality of $\iota^\sharp$, and the middle square commutes because everything is just restrictions of the same sheaf $\OO_V$.

	Further, we see that we assemble into a morphism of locally ringed spaces: for any point $p\in X\subseteq U$, we need to know that the composite
	\[\arraycolsep=1.4pt\begin{array}{ccccc}
		\OO_{U,p} &\stackrel{(\iota')^\sharp_p}\to& (\iota'_*\OO_{X})_p &\to& \OO_{X,p} \\
		{[(U',s)]} &\mapsto& [(U',(\iota')^\sharp_{U'}(s))] &\mapsto& [(U'\cap X,(\iota')^\sharp_{U'}(s))]
	\end{array}\]
	is a map of local rings. To begin, we note that $(\iota')^\sharp_{U'}(s)=\iota^\sharp_{U'}(s)|_{U'\cap X}$, where the restriction takes place in $\OO_V$, so the entire germ is just $\iota^\sharp_{U'}(s)|_p$. Thus, the above map is
	\[[U',s]\mapsto\iota^\sharp_{U'}(s)|_p,\]
	which we can see is a map of local rings directly because $\iota^\sharp$ is a map of local rings, so any germ on the right side living in $\mf m_{U,p}$ is actually living in $\mf m_{Y,p}$ and will therefore go to $\mf m_{V,p}$ under $\iota^\sharp$. (We are implicitly using the fact that stalks of a restriction sheaf, which is how open embeddings are defined, are identified with stalks of the original sheaf.)

	Lastly, we need to show that our diagram commutes. We already know that it commutes on the level of topological spaces. On the level of sheaves, we pick up some open subset $U'\subseteq Y$, and we need the diagram
	% https://q.uiver.app/?q=WzAsNCxbMCwwLCJcXE9PX1koVScpIl0sWzEsMCwiXFxPT19WKFZcXGNhcCBVJykiXSxbMSwxLCJcXE9PX1goVlxcY2FwIFUnKSJdLFswLDEsIlxcT09fe1UnfShVXFxjYXAgVScpIl0sWzAsMSwiXFxpb3RhXlxcc2hhcnBfe1UnfSJdLFswLDMsIlxcb3B7cmVzfSIsMl0sWzEsMiwiXFxvcHtyZXN9Il0sWzMsMiwiKFxcaW90YScpXlxcc2hhcnBfe1VcXGNhcCBVJ30iXV0=&macro_url=https%3A%2F%2Fraw.githubusercontent.com%2FdFoiler%2Fnotes%2Fmaster%2Fnir.tex
	\[\begin{tikzcd}
		{\OO_Y(U')} & {\OO_V(V\cap U')} \\
		{\OO_{U'}(U\cap U')} & {\OO_X(X\cap U')}
		\arrow["{\iota^\sharp_{U'}}", from=1-1, to=1-2]
		\arrow["{\op{res}}"', from=1-1, to=2-1]
		\arrow["{\op{res}}", from=1-2, to=2-2]
		\arrow["{(\iota')^\sharp_{U\cap U'}}", from=2-1, to=2-2]
	\end{tikzcd}\]
	commutes because it is the same as the larger diagram
	% https://q.uiver.app/?q=WzAsNSxbMCwwLCJcXE9PX1koVScpIl0sWzEsMSwiXFxPT19WKFZcXGNhcCBVXFxjYXAgVScpIl0sWzAsMSwiXFxPT19ZKFVcXGNhcCBVJykiXSxbMiwxLCJcXE9PX1YoWFxcY2FwIFUnKSJdLFsxLDAsIlxcT09fVihWXFxjYXAgVScpIl0sWzAsMiwiXFxvcHtyZXN9IiwyXSxbMiwxLCJcXGlvdGFeXFxzaGFycF97VVxcY2FwIFUnfSJdLFsxLDMsIlxcb3B7cmVzfSJdLFswLDQsIlxcaW90YV5cXHNoYXJwX3tVJ30iXSxbNCwzLCJcXG9we3Jlc30iXSxbNCwxLCJcXG9we3Jlc30iLDJdXQ==&macro_url=https%3A%2F%2Fraw.githubusercontent.com%2FdFoiler%2Fnotes%2Fmaster%2Fnir.tex
	\[\begin{tikzcd}
		{\OO_Y(U')} & {\OO_V(V\cap U')} \\
		{\OO_Y(U\cap U')} & {\OO_V(V\cap U\cap U')} & {\OO_V(X\cap U')}
		\arrow["{\op{res}}"', from=1-1, to=2-1]
		\arrow["{\iota^\sharp_{U\cap U'}}", from=2-1, to=2-2]
		\arrow["{\op{res}}", from=2-2, to=2-3]
		\arrow["{\iota^\sharp_{U'}}", from=1-1, to=1-2]
		\arrow["{\op{res}}", from=1-2, to=2-3]
		\arrow["{\op{res}}"', from=1-2, to=2-2]
	\end{tikzcd}\]
	where now the square commutes by naturality of $\iota^\sharp$, and the triangle commutes because everything is a restriction in $\OO_V$.
\end{proof}
\begin{corollary} \label{cor:loc-closed-embed-composite}
	The class of locally closed embeddings is preserved by composition.
\end{corollary}
\begin{proof}
	Let our closed embeddings be $X_1\to X_2$ and $X_2\to X_3$, where we decompose $X_1\to X_2$ into a closed embedding $\iota_1\colon X_1\to U_1$ and an open embedding $\jmath_1\colon U_1\to X_2$, and we do similar for the map $X_2\to X_3$. Then the composite $X_1\to X_3$ is just
	\[X_1\stackrel{\iota_1}\to U_1\stackrel{\jmath_1}\to X_2\stackrel{\iota_2}\to U_2\stackrel{\jmath_2}\to X_2.\]
	Now, {the composite $\iota_2\circ\jmath_1$ of an open embedding followed by a closed embedding is actually a locally closed embedding as shown in the previous lemma}, so we can write it as $\jmath_2'\circ\iota_1'$ where $\jmath_2'$ is a closed embedding and $\iota_1'$ is an open embedding.
	
	It follows that the full composite $X_1\to X_3$ is the open embedding $\iota_1'\circ\iota_1$ (this is still an open embedding because we're still a homeomorphism onto an open set, and the structure sheaves still match) followed by the closed embedding $\jmath_2\circ\jmath_2'$ (recall closed embeddings are preserved by composition by \autoref{cor:closed-embed-compose}), which in total is a locally closed embedding.
\end{proof}


\subsection{Separated Morphisms}
We are moving towards defining varieties, for which we need to define separated morphisms. Hartshorne proves some difficult criterion for morphisms to be separated and proper, but many of the corollaries of the criterion can be proven without it, so we will avoid proving the criterion for now.

Anyway, in some sense, a scheme or morphism being separated is the correct analogue of being Hausdorff. However, the statement of the Hausdorff condition requires talking about points and open sets, local to the space, so we will want a more morphism-oriented point of view.
\begin{lemma} \label{lem:topo-sep}
	A topological space $X$ is Hausdorff if and only if the diagonal map $\Delta\colon X\to X\times X$ given by $x\mapsto(x,x)$ has closed image.
\end{lemma}
\begin{proof}
	We have two implications.
	\begin{itemize}
		\item Suppose $X$ is Hausdorff. We need to show $(X\times X)\setminus\Delta(X)$ is open. Well, for any $(x,x')\in (X\times X)\setminus\Delta(X)$, we see $x\ne x'$, so we are promised disjoint open subsets $U,U'\subseteq X$ containing $x$ and $x'$ respectively. It follows that $(U\times U')\cap\Delta(X)=\emp$ (any intersection would witness some $y\in U\cap U'$) and contains $(x,x')$. Thus, any $(x,x')$ has been given an open neighborhood $U\times U'\subseteq(X\times X)\setminus\Delta(X)$, finishing.
		\item Suppose $\Delta(X)\subseteq X\times X$ is closed. Then, for any distinct $x,x'\in X$, we see $(x,x')\notin\Delta(X)$. But $(X\times X)\setminus\Delta(X)$ is open, so we can write it out using the distinguished base as
		\[(X\times X)\setminus\Delta(X)=\bigcup_{\alpha\in\lambda}(U_\alpha\times U_\alpha')\]
		for open subsets $U_\alpha,U_\alpha'\subseteq X$. So say $(x,x')\in U_\alpha\times U_\alpha'$, so $x\in U_\alpha$ and $x'\in U_\alpha'$. Further, $(U_\alpha\times U_\alpha')\cap\Delta(X)\ne\emp$, so we again see that $U_\alpha\cap U_\alpha'=\emp$, so we are done.
		\qedhere
	\end{itemize}
\end{proof}
Thus, we are motivated into the following definition.
\begin{definition}[Separated]
	A scheme morphism $\pi\colon X\to Y$ is \textit{separated} if and only if $\Delta\pi\colon X\to X\times_YX$ is a closed embedding. An $S$-scheme $X$ is \textit{separated} if and only if the map $X\to S$ is separated; for example, a scheme $X$ is separated if and only if the map $X\to\Spec\ZZ$ is separated.
\end{definition}
\begin{remark} \label{rem:sep-is-reasonable}
	As usual, the class of separated morphisms is preserved by composition, is preserved by base change, and is affine-local on the target. Here are the checks.
	\begin{itemize}
		\item The class of closed embeddings is preserved by composition by \autoref{cor:closed-embed-compose} and by base change by \autoref{lem:closed-embed-base-change}, so the class of separated morphisms is as well by \autoref{lem:diagonal-composition}.
		\item The class of closed embeddings is preserved by base change by \autoref{lem:closed-embed-base-change}, so the class of separated morphisms is as well by \autoref{lem:diagonal-base-change}.
		\item Lastly, the class of closed embeddings is affine-local on the target by \autoref{cor:closed-emb-affine-local} and by base change by \autoref{lem:closed-embed-base-change}, so the class of separated morphisms is as well by \autoref{lem:diagonal-affine-local}.
	\end{itemize}
\end{remark}
\begin{remark}
	As a sanity check, separated morphisms $\pi\colon X\to Y$ are quasiseparated. Indeed, the diagonal morphism $\Delta\pi\colon X\to X\times_YX$ is a closed embedding and hence quasicompact by \autoref{ex:closed-is-qc}, so we conclude $\pi$ is quasiseparated by \autoref{lem:mainqslemma}.
\end{remark}
\begin{example} \label{ex:mono-is-sep}
	Monomorphisms $\pi\colon X\to Y$ are separated: indeed, the diagonal morphism $\Delta\pi$ is an isomorphism by \autoref{lem:monic-means-diag-iso}, which is a closed embedding by \autoref{ex:iso-is-closed}. For example, isomorphisms are separated, open embeddings are separated by \autoref{cor:open-embed-monic}, closed embeddings are separated by \autoref{lem:closed-monic}, and more generally locally closed embeddings are separated by \autoref{rem:loc-closed-is-monic}.
\end{example}
We're defining separated as having $\Delta\pi$ being a closed embedding, but we might want $\Delta\pi$ to just have closed image from our topological motivation. These turn out to be the same; to see this, we note that the diagonal morphism isn't just any morphism.
\begin{lemma} \label{lem:morphism-of-affine-is-sep}
	Fix a morphism $\pi\colon X\to Y$ of affine schemes. Then the induced diagonal map $\Delta\pi\colon X\to X\times_YX$ is a closed embedding.
\end{lemma}
\begin{proof}
	For brevity, set $\Delta\coloneqq\Delta\pi$. Because isomorphisms are closed embeddings by \autoref{ex:iso-is-closed}, and closed embeddings are preserved by composition by \autoref{cor:closed-embed-compose}, it suffices to show that a morphism of affine schemes $\pi\colon\Spec B\to\Spec A$ makes $\Delta\colon\Spec B\to(\Spec B\times_{\Spec A}\Spec B)$ a closed embedding.
	
	Well, let $f\colon A\to B$ be the ring map associated to $\pi$ by \autoref{thm:geoisoppalg}, and we see we may set
	\[\Spec B\times_{\Spec A}\Spec B=\Spec B\otimes_AB\]
	by \autoref{lem:affine-fp}, where the canonical projections are given by the ring maps $i_1,i_2\colon B\to B\otimes_AB$ by $\iota_1\colon b\mapsto b\otimes1$ and $\iota_2\colon b\mapsto1\otimes b$. Now, the diagonal map $\Delta$ making the diagram
	% https://q.uiver.app/?q=WzAsNSxbMCwwLCJcXFNwZWMgQiJdLFsxLDEsIlxcU3BlYyBCXFxvdGltZXNfQUIiXSxbMiwxLCJcXFNwZWMgQiJdLFsxLDIsIlxcU3BlYyBCIl0sWzIsMiwiXFxTcGVjIEEiXSxbMCwxLCJcXERlbHRhIiwxXSxbMCwyLCIiLDEseyJjdXJ2ZSI6LTIsImxldmVsIjoyLCJzdHlsZSI6eyJoZWFkIjp7Im5hbWUiOiJub25lIn19fV0sWzAsMywiIiwxLHsiY3VydmUiOjIsImxldmVsIjoyLCJzdHlsZSI6eyJoZWFkIjp7Im5hbWUiOiJub25lIn19fV0sWzEsMl0sWzIsNCwiXFxwaSIsMl0sWzMsNCwiXFxwaSJdLFsxLDNdXQ==&macro_url=https%3A%2F%2Fraw.githubusercontent.com%2FdFoiler%2Fnotes%2Fmaster%2Fnir.tex
	\[\begin{tikzcd}
		{\Spec B} \\
		& {\Spec B\otimes_AB} & {\Spec B} \\
		& {\Spec B} & {\Spec A}
		\arrow["\Delta"{description}, from=1-1, to=2-2]
		\arrow[curve={height=-12pt}, Rightarrow, no head, from=1-1, to=2-3]
		\arrow[curve={height=12pt}, Rightarrow, no head, from=1-1, to=3-2]
		\arrow[from=2-2, to=2-3]
		\arrow["\pi"', from=2-3, to=3-3]
		\arrow["\pi", from=3-2, to=3-3]
		\arrow[from=2-2, to=3-2]
	\end{tikzcd}\]
	commute becomes the diagonal morphism $d\colon B\otimes_AB\to B$ making the diagram
	% https://q.uiver.app/?q=WzAsNSxbMCwwLCJCIl0sWzEsMSwiQlxcb3RpbWVzX0FCIl0sWzIsMSwiQiJdLFsxLDIsIkIiXSxbMiwyLCJBIl0sWzEsMCwiZCIsMV0sWzAsMiwiIiwxLHsiY3VydmUiOi0yLCJsZXZlbCI6Miwic3R5bGUiOnsiaGVhZCI6eyJuYW1lIjoibm9uZSJ9fX1dLFswLDMsIiIsMSx7ImN1cnZlIjoyLCJsZXZlbCI6Miwic3R5bGUiOnsiaGVhZCI6eyJuYW1lIjoibm9uZSJ9fX1dLFsyLDEsImlfMiJdLFs0LDIsImYiXSxbNCwzLCJmIiwyXSxbMywxLCJpXzEiLDJdXQ==&macro_url=https%3A%2F%2Fraw.githubusercontent.com%2FdFoiler%2Fnotes%2Fmaster%2Fnir.tex
	\[\begin{tikzcd}
		B \\
		& {B\otimes_AB} & B \\
		& B & A
		\arrow["d"{description}, from=2-2, to=1-1]
		\arrow[curve={height=-12pt}, Rightarrow, no head, from=1-1, to=2-3]
		\arrow[curve={height=12pt}, Rightarrow, no head, from=1-1, to=3-2]
		\arrow["{i_2}", from=2-3, to=2-2]
		\arrow["f", from=3-3, to=2-3]
		\arrow["f"', from=3-3, to=3-2]
		\arrow["{i_1}"', from=3-2, to=2-2]
	\end{tikzcd}\]
	commute, which we see must be surjective because $d\circ i_2=\id_B$. Thus, $\Delta\colon\Spec B\to\Spec B\otimes_AB$ is associated to the surjection $d\colon B\otimes_AB\to B$ and is therefore a closed embedding by \autoref{prop:affineclosedsubschemes}.
\end{proof}
\begin{example} \label{ex:affine-is-sep}
	\autoref{lem:morphism-of-affine-is-sep} tells us that morphisms of affine schemes are automatically separated. Because the class of separated morphisms is affine-local on the target by \autoref{rem:sep-is-reasonable}, we see that all affine morphisms are separated by \autoref{rem:affine-from-local-on-target}.
\end{example}
To get to the affine case, we pick up the following lemma, analogous to \autoref{lem:topo-sep}.
\begin{lemma} \label{lem:pre-image-diagonal}
	Fix a scheme morphism $\varphi\colon X\to Y$, and let $\pi_1,\pi_2\colon X\to X\times_YX$. Given open subschemes $U_1,U_2\subseteq X$, we have $U_1\times_YU_2\simeq\pi_1^{-1}U_1\times_Y\pi_2^{-1}U_2$ canonically, and
	\[U_1\cap U_2=(\Delta\varphi)^{-1}(U_1\times_YU_2),\]
	where $U_1\times_YU_2$ is realized as canonically embedded into $X\times_YX$.
\end{lemma}
\begin{proof}
	For brevity, set $\Delta\coloneqq\Delta\varphi$, and let $\pi_1,\pi_2\colon X\times_YX\to X$ be the canonical projections. Note the diagram
	% https://q.uiver.app/?q=WzAsOSxbMiwyLCJZIl0sWzIsMSwiWCJdLFsxLDIsIlgiXSxbMiwwLCJVXzEiXSxbMCwyLCJVXzIiXSxbMSwxLCJYXFx0aW1lc19ZWCJdLFsxLDAsIlxccGlfMV57LTF9KFVfMSkiXSxbMCwxLCJcXHBpXzJeey0xfShVXzIpIl0sWzAsMCwiXFxwaV8xXnstMX0oVV8xKVxcY2FwXFxwaV8yXnstMX0oVV8yKSJdLFs1LDEsIlxccGlfMSIsMl0sWzUsMiwiXFxwaV8yIl0sWzYsMywiXFxwaV8xIl0sWzEsMCwiXFx2YXJwaGkiXSxbMywxLCIiLDAseyJzdHlsZSI6eyJ0YWlsIjp7Im5hbWUiOiJob29rIiwic2lkZSI6InRvcCJ9fX1dLFs2LDUsIiIsMix7InN0eWxlIjp7InRhaWwiOnsibmFtZSI6Imhvb2siLCJzaWRlIjoidG9wIn19fV0sWzQsMiwiIiwwLHsic3R5bGUiOnsidGFpbCI6eyJuYW1lIjoiaG9vayIsInNpZGUiOiJ0b3AifX19XSxbNyw1LCIiLDAseyJzdHlsZSI6eyJ0YWlsIjp7Im5hbWUiOiJob29rIiwic2lkZSI6InRvcCJ9fX1dLFs3LDQsIlxccGlfMiJdLFs4LDYsIiIsMCx7InN0eWxlIjp7InRhaWwiOnsibmFtZSI6Imhvb2siLCJzaWRlIjoidG9wIn19fV0sWzgsNywiIiwwLHsic3R5bGUiOnsidGFpbCI6eyJuYW1lIjoiaG9vayIsInNpZGUiOiJ0b3AifX19XSxbMiwwLCJcXHZhcnBoaSJdXQ==&macro_url=https%3A%2F%2Fraw.githubusercontent.com%2FdFoiler%2Fnotes%2Fmaster%2Fnir.tex
	\[\begin{tikzcd}
		{\pi_1^{-1}(U_1)\cap\pi_2^{-1}(U_2)} & {\pi_1^{-1}(U_1)} & {U_1} \\
		{\pi_2^{-1}(U_2)} & {X\times_YX} & X \\
		{U_2} & X & Y
		\arrow["{\pi_1}"', from=2-2, to=2-3]
		\arrow["{\pi_2}", from=2-2, to=3-2]
		\arrow["{\pi_1}", from=1-2, to=1-3]
		\arrow["\varphi", from=2-3, to=3-3]
		\arrow[hook, from=1-3, to=2-3]
		\arrow[hook, from=1-2, to=2-2]
		\arrow[hook, from=3-1, to=3-2]
		\arrow[hook, from=2-1, to=2-2]
		\arrow["{\pi_2}", from=2-1, to=3-1]
		\arrow[hook, from=1-1, to=1-2]
		\arrow[hook, from=1-1, to=2-1]
		\arrow["\varphi", from=3-2, to=3-3]
	\end{tikzcd}\]
	has all squares pullback squares, where all but the bottom-right corner follow from \autoref{lem:open-fp}. Notably, Thus, we see that
	\[U_1\times_YU_2\simeq\underbrace{\pi_1^{-1}(U_1)\cap\pi_2^{-1}(U_2)}_{U\coloneqq},\]
	where the canonical projections are given by $\pi_1,\pi_2$. It remains to compute $\Delta^{-1}(U)$. We show our inclusions separately.
	\begin{itemize}
		\item If $x\in U_1\cap U_2$, then we see $\pi_1\Delta(x)=x\in U_1$ and $\pi_2\Delta(x)\in U_2$, so $\Delta(x)\in\pi_1^{-1}U_1\cap\pi_2^{-1}U_2$, so $x\in\Delta^{-1}(U)$.
		\item If $x\in\Delta^{-1}(U)$, then we see $\Delta(x)\in U$, so $x=\pi_1\Delta(x)\in\pi_1(U)\subseteq U_1$, and $x=\pi_2\Delta(x)\in\pi_2(U)\subseteq U_2$, so $x\in U_1\cap U_2$.
	\end{itemize}
	The above inclusions finish the proof.
\end{proof}
\begin{lemma} \label{lem:diag-is-local-closed}
	Fix a scheme morphism $\varphi\colon X\to Y$. Then the diagonal morphism $\Delta\varphi\colon X\to X\times_YX$ is a locally closed embedding.
\end{lemma}
\begin{proof}
	To show that we have a locally closed embedding, we need to find the requested intermediate open embedding $U\subseteq X\times_YX$. The idea is to reduce to the affine case, \autoref{lem:morphism-of-affine-is-sep}. For brevity, let $\Delta\coloneqq\Delta\varphi$, let $\pi_1,\pi_2\colon X\times_YX\to X$ be the canonical projections, and let $\pi\colon X\times_YX\to Y$ be $\pi\coloneqq\varphi\circ\pi_1=\varphi\circ\pi_2$.

	Now, give $Y$ an affine open cover $\{Y_\alpha\}$ and set $X_\alpha\coloneqq\varphi^{-1}Y_\alpha$ so that \autoref{lem:fp-open-cover-base} and \autoref{lem:sub-affine-base-fp} says that the pre-images
	\[\pi^{-1}(Y_\alpha)\simeq X_\alpha\times_{Y_\alpha}X_\alpha\simeq X_\alpha\times_YX_\alpha\]
	cover $X\times_YX$. Continuing, give each $X_\alpha\coloneqq\pi^{-1}Y_\alpha$ an affine cover $\{X_{\alpha\beta}\}_{\beta\in\lambda_\alpha}$ so that \autoref{lem:pre-image-diagonal} tells us
	\[\pi_1^{-1}(X_{\alpha\beta_1})\cap\pi_2^{-1}(X_{\alpha\beta_2})\simeq X_{\alpha\beta_1}\times_{Y}X_{\alpha\beta_2}\simeq X_{\alpha\beta_1}\times_{Y_\alpha}X_{\alpha\beta_2},\]
	where the canonical projections are induced by $\pi_1,\pi_2$. Notably, $\varphi(X_{\alpha\beta})\subseteq Y_\alpha$, so \autoref{lem:sub-affine-base-fp} is still applying here.

	At this point, we set
	\[U\coloneqq\bigcup_{\alpha\in\lambda}\bigcup_{\beta\in\lambda_\alpha}\big(\underbrace{\pi_1^{-1}(X_{\alpha\beta})\cap\pi_2^{-1}(X_{\alpha\beta})}_{U_{\alpha\beta}\coloneqq}\big).\]
	In particular, we see that $U\subseteq X\times_YX$ is open. Notably, $\Delta^{-1}(U_{\alpha\beta})=U_{\alpha\beta}$ by \autoref{lem:pre-image-diagonal}, so taking the union over all $\alpha$ and $\beta$ tells us that $\Delta^{-1}(U)$ is open.

	Thus, \autoref{rem:factor-through-open} tells us that $\Delta$ factors as
	\[X\stackrel{\Delta'}\to U\into X\times_YX,\]
	where $U\into X\times_YX$ is a closed embedding. It remains to show that $\Delta'\colon X\to U$ is a closed embedding. Because closed embeddings are affine-local on the target by \autoref{cor:closed-emb-affine-local}, it suffices to show that the restrictions $\Delta'_{\alpha\beta}\colon\Delta^{-1}U_{\alpha\beta}\to U_{\alpha\beta}$ are all closed embeddings.

	To see this, we recall $\Delta^{-1}U_{\alpha\beta}=X_{\alpha\beta}$ from \autoref{lem:pre-image-diagonal}. Thus, we are left trying to show that
	\[\Delta|_{X_{\alpha\beta}}\colon X_{\alpha\beta}\to U_{\alpha\beta}\]
	are closed embeddings. However, $U_{\alpha\beta}\simeq X_{\alpha\beta}\times_{Y_\alpha}X_{\alpha\beta}$, where the canonical projections are given by $\pi_1|_{U_{\alpha\beta}},\pi_2|_{U_{\alpha\beta}}\colon U_{\alpha\beta}\to X_{\alpha\beta}$, as we said when we constructed $U$. So the fact that
	\[\pi_\bullet|_{U_{\alpha\beta}}\circ\Delta|_{X_{\alpha\beta}}=(\pi_\bullet\circ\Delta)|_{X_{\alpha\beta}}={\id_X}|_{X_{\alpha\beta}}=\id_{X_{\alpha\beta}}\]
	for each $\pi_\bullet$, using the functoriality of restriction (from \autoref{rem:scheme-restriction-functorial}), tells us that $\Delta|_{X_{\alpha\beta}}$ is the diagonal morphism of the morphism of affine schemes $\varphi|_{X_{\alpha\beta}}\colon X_{\alpha\beta}\to Y_\alpha$. So $\Delta|_{X_{\alpha\beta}}$ is a closed embedding by \autoref{lem:morphism-of-affine-is-sep}. This finishes.
\end{proof}
\begin{corollary} \label{cor:separated-by-diag}
	A scheme morphism $\pi\colon X\to Y$ is separated if and only if $\Delta\pi(X)\subseteq X\times_YX$ is closed topologically.
\end{corollary}
\begin{proof}
	By \autoref{lem:diag-is-local-closed}, $\Delta\pi$ is a locally closed embedding, so $\Delta\pi(X)$ being closed makes $\Delta\pi$ is a closed embedding by \autoref{lem:loc-closed-becomes-closed-embed}.
\end{proof}
As an application, we show that separated morphisms satisfy cancellation, like quasiseparated morphisms (\autoref{lem:qs-cancellation}).
\begin{lemma} \label{lem:cancel-sep}
	Fix scheme morphisms $\varphi\colon X\to Y$ and $\psi\colon Y\to Z$. If the composite $\psi\circ\varphi$ is separated, then $\varphi$ is also separated.
\end{lemma}
\begin{proof}
	The key to the proof is to draw the diagram
	% https://q.uiver.app/?q=WzAsNSxbMCwwLCJYIl0sWzEsMCwiWFxcdGltZXNfWVgiXSxbMiwwLCJZIl0sWzIsMSwiWVxcdGltZXNfWlkiXSxbMSwxLCJYXFx0aW1lc19aWCJdLFsyLDMsIlxcRGVsdGFcXHBzaSJdLFswLDEsIlxcRGVsdGFcXHZhcnBoaSJdLFswLDQsIlxcRGVsdGEoXFxwc2lcXHZhcnBoaSkiLDJdLFs0LDNdLFsxLDQsIlxcZGVsdGEiXSxbMSwyXV0=&macro_url=https%3A%2F%2Fraw.githubusercontent.com%2FdFoiler%2Fnotes%2Fmaster%2Fnir.tex
	\[\begin{tikzcd}
		X & {X\times_YX} & Y \\
		& {X\times_ZX} & {Y\times_ZY}
		\arrow["\Delta\psi", from=1-3, to=2-3]
		\arrow["\Delta\varphi", from=1-1, to=1-2]
		\arrow["{\Delta(\psi\varphi)}"', from=1-1, to=2-2]
		\arrow[from=2-2, to=2-3]
		\arrow["\delta", from=1-2, to=2-2]
		\arrow[from=1-2, to=1-3]
	\end{tikzcd}\]
	where the right square is the pullback square of \autoref{lem:magic-diagram}. In particular, \autoref{lem:diag-is-local-closed} tells us that $\delta$ is a locally closed embedding. Now, $\Delta(\psi\circ\varphi)$ is a closed embedding because $\psi\circ\varphi$ is separated, so $\Delta(\psi\circ\varphi)(X)$ is a closed subset of $X\times_ZX$, so
	\[\delta^{-1}(\Delta(\psi\circ\varphi)(X))=\delta^{-1}(\delta(\Delta\varphi(X)))\subseteq X\times_YX\]
	is also a closed subset by the continuity of $\delta$. However, $\delta$ is the composite of morphisms which are homeomorphisms onto their images, so $\delta$ is a homeomorphism onto its image (to be formal, use the arguments of \autoref{lem:loc-closed-becomes-closed-embed}). Thus,
	\[\Delta(\psi\circ\varphi)(X)=\delta(\Delta\varphi(X))\subseteq\im\delta\]
	tells us that $\delta^{-1}(\delta(\Delta\varphi(X)))=\Delta\varphi(X)$ is a closed subset of $X\times_YX$, so $\varphi$ is in fact separated by \autoref{cor:separated-by-diag}.
\end{proof}
Cancellation allows us to check that being separated is a reasonable notion for $S$-schemes.
\begin{corollary} \label{cor:sep-over-sep-is-sep}
	Let $S$ be a separated scheme. Then an $S$-scheme $X$ is separated over $S$ if and only if $X$ is separated (over $\Spec\ZZ$).
\end{corollary}
\begin{proof}
	Let $\varphi\colon X\to S$ be the promised map. Recall that $\Spec\ZZ$ is final by \autoref{cor:spec-z-final}, so we have unique maps $\tau_S\colon S\to\Spec\ZZ$ and $\tau_X\colon X\to\Spec\ZZ$ which notably satisfy
	\[\tau_X=\tau_S\circ\varphi\]
	by the uniqueness of $\tau_X$. Because $S$ is separated, we see $\tau_S$ is separated.

	Now, if $X$ is separated over $S$, then $\varphi$ is separated, so $\tau_X$ is separated because separated morphisms are preserved by composition by \autoref{rem:sep-is-reasonable}; thus, $X$ is separated. Conversely, if $X$ is separated, then $\tau_X$ is separated, so $\varphi$ is separated by \autoref{lem:cancel-sep}; thus, $X$ is separated over $S$.
\end{proof}
Here is a more extended example.
\begin{exe}
	The affine $k$-line with doubled origin is not separated over $\Spec k$.
\end{exe}
\begin{proof}
	We will be a little sketchy here because we'll shortly give a more sophisticated proof of this. Glue the affine schemes $\Spec k[x]$ and $\Spec k[y]$ by the isomorphism $\Spec k[x,x^{-1}]\simeq\Spec k[y,y^{-1}]$, which makes the affine $k$-line with doubled origin $X$.

	We really only have to pay attention to the topological spaces. Note that $X\times_{\Spec k}X$ as a topological space looks like $\AA^2_k$ with a doubled $y$-axis and a doubled $x$-axis and a quadrupled origin; one can check this through by gluing everything. As such, the diagonal map has image consisting of the line $y=x$, with two of the origins. However, the two origins have closure of all four of the origins, so the total image is not closed.
\end{proof}
\begin{remark}
	However, it is true that our scheme is quasiseparated. Namely, we can see somewhat directly that our pre-images are quasicompact.
\end{remark}

\end{document}