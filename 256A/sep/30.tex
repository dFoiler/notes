% !TEX root = ../notes.tex

\documentclass[../notes.tex]{subfiles}

\begin{document}

\section{September 30}

We continue the proof of \autoref{thm:chev} from last class.

\subsection{Finishing Chevalley's Theorem}
Recall that we had the diagram
% https://q.uiver.app/?q=WzAsNSxbMCwwLCJcXFNwZWMgQV9zIl0sWzEsMCwiXFxTcGVjIEJfc1t4XzEsXFxsZG90cyx4X25dIl0sWzIsMCwiXFxTcGVjIEJfcyJdLFsyLDEsIlxcU3BlYyBCIl0sWzAsMSwiXFxTcGVjIEEiXSxbNCwzLCJcXHBpIl0sWzAsNCwiIiwwLHsic3R5bGUiOnsidGFpbCI6eyJuYW1lIjoiaG9vayIsInNpZGUiOiJ0b3AifX19XSxbMiwzLCIiLDIseyJzdHlsZSI6eyJ0YWlsIjp7Im5hbWUiOiJob29rIiwic2lkZSI6InRvcCJ9fX1dLFswLDEsIlxccGlfcyJdLFsxLDJdXQ==&macro_url=https%3A%2F%2Fraw.githubusercontent.com%2FdFoiler%2Fnotes%2Fmaster%2Fnir.tex
\[\begin{tikzcd}
	{\Spec A_s} & {\Spec B_s[x_1,\ldots,x_n]} & {\Spec B_s} \\
	{\Spec A} && {\Spec B}
	\arrow["\pi", from=2-1, to=2-3]
	\arrow[hook, from=1-1, to=2-1]
	\arrow[hook, from=1-3, to=2-3]
	\arrow["{\pi_s}", from=1-1, to=1-2]
	\arrow[from=1-2, to=1-3]
\end{tikzcd}\]
from spreading out Noether normalization. Note that $\pi_s$ is a dominant morphism because $B_s[x_1,\ldots,x_n]\into A_s$ is a dominant morphism. In fact, $\pi_s$ is a finite morphism by its construction from Noether normalization, so we conclude that $\pi_s(\Spec A_s)=\Spec B_s[x_1,\ldots,x_n]$ by dominance. The map $\Spec B_s[x_1,\ldots,x_n]\to\Spec B_s$ is now certainly surjective, so in total, we see that $\im\pi$ contains the open subset $\Spec B_s\subseteq\Spec B$.

This is actually good enough to finish the proof.
\begin{example}
	Set $B=\Spec k[x]$. Then the open subsets of $\Spec k[x]$ are only missing finitely many points and so will stay open!
\end{example}
Motivated by how nice Noetherian schemes are, we have the following lemma.
\begin{lemma} \label{lem:better-constructible}
	Fix a Noetherian space $Y$. A subset $E\subseteq Y$ is constructible if and only if all irreducible closed subsets $Z\subseteq Y$ has either $E\cap Z$ or $Z\setminus E$ containing a nonempty open set.
\end{lemma}
\begin{proof}
	This is by Noetherian induction. Namely, if the statement were false, we could make $E$ minimal satisfying the above condition and derive contradiction by looking at the two cases.
\end{proof}
Thus, to finish the proof, we use \autoref{lem:better-constructible}. We want to show that $\pi(X)$ is irreducible, so for some irreducible closed subset $Z\subseteq Y$ we use the same reduction steps as above to show that the restricted map
\[\pi\colon\pi^{-1}Z\to Z\]
has, for any open irreducible subset $U\subseteq\overline{\pi(Z)}$ is strictly contained in $Z$ (so that the restricted morphism is not dominant) or $\pi(U)$ contains an open nonempty subset inside $Z$ (because the argument goes through as soon as the restriction is dominant). In the former case, we will get that $Z\setminus\pi(U)$ contains a nonempty open subset, so \autoref{lem:better-constructible} kicks in to tell us that $\pi(X)$ is constructible.
\begin{remark}
	Even though the statement of \autoref{thm:chev} is non-constructive, one can concretely work through the above proof using a somewhat explicit construction from $\pi_s$. In particular, after peeling off the desired open subset $B_s$, what's left over is approximately speaking some lower-dimensional object, so we can induct downwards.
	% 8.4.4
\end{remark}

\subsection{Closed Embeddings Are Reasonable}
It will be helpful for us to collect some facts about closed embeddings before we talk about separated morphisms in a moment.
To start off, we're well overdue to show that closed morphisms are monic.
\begin{lemma} \label{lem:closed-monic}
	Closed morphisms are monic.
\end{lemma}
\begin{proof}
	We show this by hand. Fix a closed morphism $\iota\colon X\to Y$ and two morphisms $\alpha,\beta\colon S\to X$ such that $\iota\circ\alpha=\iota\circ\beta$. We need to show that $\alpha=\beta$, which we show by hand.
	\begin{itemize}
		\item On the level of topological spaces, we note that any $s\in S$ will have $\iota(\alpha(s))=\iota(\beta(s))$, so $\alpha(s)=\beta(s)$ because $\iota$ is a homeomorphism onto its image.
		\item On the level of sheaves, we are given that $\iota_*\alpha^\sharp\circ\iota^\sharp=\iota_*\beta^\sharp\circ\iota^\sharp$ as sheaf morphisms $\OO_Y\to\alpha_*\iota_*\OO_S$. Now, fix any $x\in X$, and we will show that $\alpha^\sharp_x=\beta^\sharp_x$ at $x$, which will finish by \autoref{prop:stalkmorphism}.

		As an intermediate claim, we show that the composite $\OO_{Y,\iota(x)}\to(\iota_*\OO_X)_{\iota(x)}\to\OO_{X,x}$ is surjective. Well, we know that the map $\iota^\sharp_{\iota(x)}\colon\OO_{Y,\iota(x)}\to(\iota_*\OO_X)_{\iota(x)}$ is already surjective, so we just need to show that the map $(\iota_*\OO_X)_{\iota(x)}\to\OO_{X,x}$ is surjective. For this, we pick up some germ $[(U,s)]\in\OO_{X,x}$ with $U$ containing $x$.

		Now, we note that $\iota$ is a homeomorphism onto its image, so $\iota(U)\subseteq\im\iota$ is an open subset, so there is some open $V\subseteq Y$ such that $\iota(U)=V\cap\im\iota$, which in particular means $\iota^{-1}V=U$. So we see $[(V,s)]\in(\iota_*\OO_X)_{\iota(x)}$ will go to $[(U,s)]$ under the canonical map.

		We now show $\alpha^\sharp_x=\beta^\sharp_x$. For this, we track through the large diagram
		% https://q.uiver.app/?q=WzAsOSxbMCwwLCIoXFxPT19ZKV97XFxpb3RhKHgpfSJdLFsxLDAsIihcXGlvdGFfKlxcT09fWClfe1xcaW90YSh4KX0iXSxbMSwxLCJcXE9PX3tYLHh9Il0sWzIsMSwiKFxcYWxwaGFfKlxcT09fUylfeCJdLFsyLDAsIihcXGlvdGFfKlxcYWxwaGFfKlxcT09fUylfe1xcaW90YSh4KX0iXSxbMywwLCJbKFYscyldIl0sWzQsMCwiWyhWLFxcYWxwaGFeXFxzaGFycF97XFxpb3RhXnstMX1WfXMpXSJdLFszLDEsIlsoXFxpb3RhXnstMX1WLHMpXSJdLFs0LDEsIlsoXFxpb3RhXnstMX1WLFxcYWxwaGFeXFxzaGFycF97XFxpb3RhXnstMX1WfXMpXSJdLFswLDIsIiIsMCx7InN0eWxlIjp7ImhlYWQiOnsibmFtZSI6ImVwaSJ9fX1dLFswLDEsIiIsMix7InN0eWxlIjp7ImhlYWQiOnsibmFtZSI6ImVwaSJ9fX1dLFsxLDIsIiIsMix7InN0eWxlIjp7ImhlYWQiOnsibmFtZSI6ImVwaSJ9fX1dLFsxLDQsIihcXGlvdGFfKlxcYWxwaGFeXFxzaGFycClfeCJdLFs0LDNdLFsyLDMsIlxcYWxwaGFeXFxzaGFycF94Il0sWzUsNiwiIiwwLHsic3R5bGUiOnsidGFpbCI6eyJuYW1lIjoibWFwcyB0byJ9fX1dLFs2LDgsIiIsMCx7InN0eWxlIjp7InRhaWwiOnsibmFtZSI6Im1hcHMgdG8ifX19XSxbNSw3LCIiLDIseyJzdHlsZSI6eyJ0YWlsIjp7Im5hbWUiOiJtYXBzIHRvIn19fV0sWzcsOCwiIiwyLHsic3R5bGUiOnsidGFpbCI6eyJuYW1lIjoibWFwcyB0byJ9fX1dXQ==&macro_url=https%3A%2F%2Fraw.githubusercontent.com%2FdFoiler%2Fnotes%2Fmaster%2Fnir.tex
		\[\begin{tikzcd}
			{(\OO_Y)_{\iota(x)}} & {(\iota_*\OO_X)_{\iota(x)}} & {(\iota_*\alpha_*\OO_S)_{\iota(x)}} & {[(V,s)]} & {[(V,\alpha^\sharp_{\iota^{-1}V}s)]} \\
			& {\OO_{X,x}} & {(\alpha_*\OO_S)_x} & {[(\iota^{-1}V,s)]} & {[(\iota^{-1}V,\alpha^\sharp_{\iota^{-1}V}s)]}
			\arrow[two heads, from=1-1, to=2-2]
			\arrow[two heads, from=1-1, to=1-2]
			\arrow[two heads, from=1-2, to=2-2]
			\arrow["{(\iota_*\alpha^\sharp)_x}", from=1-2, to=1-3]
			\arrow[from=1-3, to=2-3]
			\arrow["{\alpha^\sharp_x}", from=2-2, to=2-3]
			\arrow[maps to, from=1-4, to=1-5]
			\arrow[maps to, from=1-5, to=2-5]
			\arrow[maps to, from=1-4, to=2-4]
			\arrow[maps to, from=2-4, to=2-5]
		\end{tikzcd}\]
		where the left triangle is surjective as shown above, and the right square commutes as shown. Now, for any germ $s\in\OO_{X,x}$, we can compute what $\alpha^\sharp_x$ will do to $s$ by pulling $s$ back to a germ in $\OO_{Y,\iota(y)}$, then going forward across to $(\iota_*\alpha_*\OO_S)_{\iota(x)}$, and lastly going down to $(\alpha_*\OO_S)_x$.
		
		However, all of these steps are independent of $\alpha^\sharp$ in that the composite along the top is the same if we replace $\alpha^\sharp$ with $\beta^\sharp$, and the last downward step only depends on the topological data, which we know from the above aligns. This finishes.
		\qedhere
	\end{itemize}
\end{proof}
\begin{corollary}
	Closed morphisms are quasiseparated.
\end{corollary}
\begin{proof}
	Monomorphisms are quasiseparated by \autoref{ex:mono-is-qs}, so \autoref{lem:closed-monic} finishes.
\end{proof}
Here is composition.
\begin{lemma} \label{lem:pushforward-preserves-surjective}
	Fix a continuous map $f\colon X\to Y$ which is a homeomorphism onto a closed set of $Y$. If $\varphi\colon\mc F\to\mc G$ is a sheaf morphism which is surjective on stalks, then $f_*\varphi\colon f_*\mc F\to f_*\mc G$ is still surjective on stalks.
\end{lemma}
\begin{proof}
	Fix some $y\in Y$ so that we want to show $(f_*\varphi)_y$ is surjective. If $y\notin\im f=\overline{\im f}$, then note $(f_*\mc G)_y=0$ by \autoref{rem:zero-out-of-image}, so there is nothing to say.

	Otherwise, we have $y\in\im f$, so say $y=f(x)$ for $x\in X$. Starting with a germ $[(U,s)]\in (f_*\mc G)_{f(x)}$, we take this to $[(f^{-1}U,s)]\in\mc G_x$, where we can use the surjectivity of $\varphi_x$ to find $[(V,t)]\in\mc F_x$ with $\varphi_V(t)|_W=s|_W$ for some small $W$ containing $x$; restricting $V$ enough, we may assume $\varphi_V(t)=s|_W$ and that $V\subseteq f^{-1}U$.
	
	But now we see that the restriction $f|_{f^{-1}U}\colon f^{-1}U\to U$ is a homeomorphism, so $f(V)\subseteq U$ is an open subset of $U$ and hence of $Y$. At this point we recognize $f^{-1}(f(V))=V$, so the germ $[(f(V),t)]\in(f_*\mc F)_{f(x)}$ will do the trick. This finishes.
\end{proof}
\begin{lemma} \label{cor:closed-embed-compose}
	The class of closed embeddings is preserved by composition.
\end{lemma}
\begin{proof}
	Fix closed embeddings $\varphi\colon X\to Y$ and $\psi\colon Y\to Z$. We need to show that $(\psi\circ\varphi)\colon X\to Z$ is a closed embedding. There are two checks.
	\begin{itemize}
		\item Note that $\varphi$ is a homeomorphism onto a closed subset $\varphi(X)\subseteq Y$. Further, $\psi$ is a homeomorphism onto a closed subset $\psi(Y)\subseteq Z$, so $\psi$ restricted to $\varphi(X)$ will still be a homeomorphism onto $\psi(\varphi(X))$. Notably, $\varphi(X)$ implies that $\psi(\varphi(X))$ is closed.
		\item We check that $(\psi\circ\varphi)^\sharp\colon\OO_Z\to\psi_*\varphi_*\OO_X$ is surjective on stalks. Well, fixing some $z$, we want to show that the composite
		\[\OO_{Z,z}\stackrel{\psi^\sharp_z}\to(\psi_*\OO_Y)_z\stackrel{(\psi_*\varphi^\sharp)_z}\to(\psi_*\varphi_*\OO_X)_z\]
		is surjective. Note that $\psi^\sharp_z$ is already surjective, so it suffices to show that $(\psi_*\varphi^\sharp)_z$ is surjective, which follows from \autoref{lem:pushforward-preserves-surjective}.
		\qedhere
	\end{itemize}
\end{proof}
As we should expect, we now move towards being local on the target.
\begin{lemma} \label{lem:closed-emb-affine-local-1}
	Suppose that $\pi\colon X\to Y$ is a closed embedding. Then, for any open subset $U\subseteq Y$, the restriction $\pi|_{\pi^{-1}U}\colon\pi^{-1}U\to U$ is still a closed embedding.
\end{lemma}
\begin{proof}
	Unsurprisingly, there are two checks.
	\begin{itemize}
		\item On the level of topological spaces, we note that the inverse continuous map $\varphi\colon\pi(Y)\to X$ will restrict to an inverse continuous map $\varphi|_U\colon (U\cap\pi(Y))\to\pi^{-1}(U)$, so we see that $\pi|_{\pi^{-1}U}$ is at least still a homeomorphism onto its image. (Notably, $U\cap\pi(Y)=\pi(\pi^{-1}(U))$.) Additionally, we see that $\im\pi|_{\pi^{-1}U}=U$, so the image is a closed subset of $U$ (in fact, all of $U$).
		\item On stalks, we fix some $y\in U$, and we want to show that the map
		\[(\pi|_{\pi^{-1}U})^\sharp_y\colon(\OO_Y|_U)_y\to\pi_*(\OO_X|_{\pi^{-1}U})_y\]
		is surjective. Well, by definition of our restriction in \autoref{lem:restrictmorphism}, $(\pi|_{\pi^{-1}U})^\sharp_V$ just behaves as $\pi^\sharp_V$ for any open $V\subseteq U$, so the definition of the map on stalks will match as $\pi^\sharp_y$, which we already know is surjective.
		\qedhere
	\end{itemize}
\end{proof}
And here is the converse.
\begin{lemma} \label{lem:closed-emb-affine-local-2}
	Fix a scheme morphism $\pi\colon X\to Y$. Suppose that we have an open cover $\mc U$ on $Y$ such that $\pi|_{\pi^{-1}U}\colon\pi^{-1}U\to U$ is a closed embedding for each $U\in\mc U$. Then $\pi$ is a closed embedding.
\end{lemma}
\begin{proof}
	As usual, we have two checks. For brevity, let the open cover by $\{V_\alpha\}_{\alpha\in\lambda}$ and $\pi_\alpha\colon U_\alpha\to V_\alpha$ by the restrictions, where $U_\alpha\coloneqq\pi^{-1}V_\alpha$.
	\begin{itemize}
		\item On the level of topological spaces, we know that $\pi_\alpha$ onto its image $\im\pi_\alpha$, and that $\im\pi_\alpha$ is always a closed set. For one, we note that
		\[\im\pi_\alpha=\pi\left(\pi^{-1}V_\alpha\right)=\im\pi\cap V_\alpha\]
		is a closed subset in $V_\alpha$ for each $\alpha$. Thus, we conclude that $(Y\setminus\im\pi)\cap V_\alpha=V_\alpha\setminus\im\pi$ is an open subset in $V_\alpha$ and therefore in $Y$ for each $\alpha$, so
		\[Y\setminus\im\pi=\bigcup_{\alpha\in\lambda}(Y\setminus\im\pi)\cap V_\alpha\]
		must be an open subset of $Y$, so $\im\pi$ is closed.

		It remains to show that $\pi$ is a homeomorphism onto $\im\pi$. Well, for each $\alpha$, we are promised a local inverse continuous function $\varphi_\alpha\colon(\im\pi\cap V_\alpha)\to U_\alpha$. Now, for any $y\in\im\pi$, we find some $\alpha$ such that $y\in V_\alpha$ and define
		\[\varphi(y)\coloneqq\varphi_\alpha(y).\]
		Note that these functions glue appropriately: if $\pi(x)\in V_\alpha\cap V_\beta$, then $\varphi_\alpha(\pi(x))=x=\varphi_\beta(\pi(x))$. Thus, these glue to a continuous function $\varphi$ by \autoref{exe:sheafex}. Additionally, we see any $x\in X$ has $f(x)\in V_\alpha$ for some $\alpha$ and therefore
		\[\varphi(\pi(x))=\varphi_\alpha(\pi_\alpha(x))=x,\]
		any $y\in\im\pi$ has $y\in V_\alpha$ for some $\alpha$ and therefore
		\[\pi(\varphi(y))=\pi_\alpha(\varphi_\alpha(y))=y.\]
		Thus, $\varphi$ is indeed a continuous inverse for $\pi$.

		\item We show that $\pi^\sharp\colon\OO_Y\to\pi_*\OO_X$ is surjective on stalks. Fix any $y\in Y$, and pick up some germ $[(V,f)]\in(\pi_*\OO_X)_y$. Now, $y\in V_\alpha$ for some $\alpha$, so we may restrict the germ to $V_\alpha$ to assume that $V\subseteq V_\alpha$.

		In particular, we thus see that actually $[(V,f)]\in((\pi_\alpha)_*\OO_X|_{U_\alpha})_y$, so the surjectivity of $\pi^\sharp_\alpha$ on stalks tells us that there is some germ $[(V,g)]\in(\OO_Y|_{V_\alpha})_y$ (namely, we may assume $g\in\OO_Y(V)$ with $U\subseteq V_\alpha$) with
		\[(\pi_\alpha^\sharp)_y([(V,g)])=[(U,\pi^\sharp_V(g))]=[(V,f)].\]
		This finishes our surjectivity check.
		\qedhere
	\end{itemize}
\end{proof}
\begin{corollary} \label{cor:closed-emb-affine-local}
	The class of closed embeddings is local on the target.
\end{corollary}
\begin{proof}
	Combine \autoref{lem:closed-emb-affine-local-1} and \autoref{lem:closed-emb-affine-local-2}.
\end{proof}
We now show that we are affine-local on the target. We will appeal to \autoref{prop:affineclosedsubschemes} even though we will shortly give a different proof of this result. There will be no circular logic.
\begin{lemma}
	Closed embeddings are finite and in particular affine.
\end{lemma}
\begin{proof}
	Fix a closed embedding $\pi\colon X\to Y$; because being finite is affine-local on the target by \autoref{cor:finite-is-reasonable}, it suffices to show that $\pi|_{\pi^{-1}U}\colon\pi^{-1}U\to U$ is finite for each affine open $U\subseteq Y$. (Namely, fix any affine open cover of $Y$.)
	
	Well, note $\pi|_{\pi^{-1}U}$ remains a closed embedding by \autoref{lem:closed-emb-affine-local-1}, so we might as well rename $Y$ to $U$ and $X$ to $\pi^{-1}U$ and $\pi$ to $\pi|_{\pi^{-1}U}$. Then $\pi\colon X\to Y$ is a closed embedding, where $Y$ is affine, and we want to show that $\pi$ is finite. Further, because isomorphisms are closed embeddings by \autoref{ex:iso-is-closed} and closed embeddings are preserved by composition by \autoref{cor:closed-embed-compose}, we see that we actually have a closed embedding
	\[X\to Y\to\Spec A,\]
	so we can just assume that $Y$ takes the form $\Spec A$. Thus, by \autoref{prop:affineclosedsubschemes}, we must have $\pi$ factor as
	\[X\cong\Spec A/I\to\Spec A\]
	for some ideal $I\subseteq A$. In particular, we see that $\pi^\sharp$ is surjective on global sections by composing the above morphisms, which means that $\OO_X(X)$ is in fact finitely generated as an $A$-algebra by $\pi^\sharp$ (in fact, with one generator).
\end{proof}
\begin{lemma}
	The class of closed embeddings is preserved by base change.
\end{lemma}
\begin{proof}
	Suppose we have a pullback square
	% https://q.uiver.app/?q=WzAsNCxbMCwwLCJYXFx0aW1lc19TWSJdLFsxLDAsIlgiXSxbMCwxLCJZIl0sWzEsMSwiUyJdLFsyLDMsIlxccHNpX1kiXSxbMCwxLCJcXHBpX1giXSxbMCwyLCJcXHBpX1kiXSxbMSwzLCJcXHBzaV9YIl0sWzAsMywiIiwyLHsic3R5bGUiOnsibmFtZSI6ImNvcm5lciJ9fV1d&macro_url=https%3A%2F%2Fraw.githubusercontent.com%2FdFoiler%2Fnotes%2Fmaster%2Fnir.tex
	\[\begin{tikzcd}
		{X\times_SY} & X \\
		Y & S
		\arrow["{\psi_Y}", from=2-1, to=2-2]
		\arrow["{\pi_X}", from=1-1, to=1-2]
		\arrow["{\pi_Y}", from=1-1, to=2-1]
		\arrow["{\psi_X}", from=1-2, to=2-2]
		\arrow["\lrcorner"{anchor=center, pos=0.125}, draw=none, from=1-1, to=2-2]
	\end{tikzcd}\]
	of schemes such that $\psi_Y$ is a closed embedding. We would like to show that $\pi_X$ is a closed embedding. Notably, because being a closed embedding is affine-local on the target by \autoref{cor:closed-emb-affine-local} (in fact, local on the target), we may use \autoref{lem:base-change-reduce-to-affine} to assume that $X$ and $S$ are affine.

	Thus, $\psi_Y\colon Y\to S$ being a closed embedding forces $Y$ to be affine by \autoref{prop:affineclosedsubschemes}, where the induced map $\psi_Y^\sharp\colon\OO_S(S)\to\OO_Y(Y)$ on global sections is surjective. So for brevity we set $R\coloneqq\OO_S(S)$ and $A\coloneqq\OO_X(X)$ and $B\coloneqq\OO_Y(Y)$. Thus, by \autoref{lem:affine-fp}, we see that
	\[X\times_SY\simeq\Spec A\times_{\Spec R}\Spec B\simeq\Spec A\otimes_RB,\]
	so we might as well set $X\times_SY$ to be $\Spec A\otimes_RB$, where the canonical projection $X\times_SY\to X$ is given by chaining the canonical maps $\Spec A\otimes_RB\to\Spec A$ (from $A\to A\otimes_RB$) with the canonical isomorphism $\Spec A\otimes X$.

	Notably, because isomorphisms are closed embeddings by \autoref{ex:iso-is-closed}, and closed embeddings are preserved by composition by \autoref{cor:closed-embed-compose}, it suffices to show that the map $\Spec A\otimes_RB\to\Spec A$ is a closed embedding, for which it suffices by \autoref{prop:affineclosedsubschemes} to show that the canonical map $A\to A\otimes_RB$ is surjective.

	Well, we are given that the canonical map $R\to B$ is surjective. Thus, for any tensor $a\otimes b\in A\otimes_RB$, we can find $r\in R$ which goes to $b$, so $a\otimes b=ra\otimes1$, which comes from $ra\in A$ through the inclusion $A\to A\otimes_RB$. Because $A\otimes_RB$ is generated by these elements $a\otimes b$, we are done.
	\qedhere
\end{proof}

\subsection{Separated Morphisms}
We are moving towards defining varieties, for which we need to define separated morphisms. Hartshorne proves some difficult criterion for morphisms to be separated and proper, but many of the corollaries of the criterion can be proven without it, so we will avoid proving the criterion for now.
\begin{remark}
	Fix a topological space $X$. Then $X$ being Hausdorff if and only if the diagonal map $\Delta X\to X\times X$ (by $x\mapsto(x,x)$) makes $\Delta(X)\subseteq X\times X$ a closed subset.
\end{remark}
One shows the above remark by winding the definitions around. Separated is going to be the correct analogue for Hausdorff for schemes (recall that the Zariski topology is somewhat terrible), using the above ideas.

These comments are intended to motivate us to look at diagonal morphisms.
\begin{proposition} \label{prop:diag-is-local-closed}
	Fix a scheme morphism $\pi\colon X\to Y$.
	\begin{listalph}
		\item Suppose $X=\Spec A$ and $Y=\Spec B$ are affine. Then the induced diagonal map $\Delta\colon X\to X\times_YX$ (induced by $\id_X\colon X\to X$) is a closed embedding.
		\item In general, $\pi$ is a locally closed embedding.
	\end{listalph}
\end{proposition}
\begin{proof}
	We go in sequence.
	\begin{listalph}
		\item On the level of rings, we are looking at the diagram
		% https://q.uiver.app/?q=WzAsNSxbMSwxLCJBXFxvdGltZXNfQkEiXSxbMiwxLCJBIl0sWzEsMiwiQSJdLFsyLDIsIkIiXSxbMCwwLCJBIl0sWzEsMF0sWzMsMV0sWzMsMl0sWzIsMF0sWzQsMSwiIiwyLHsiY3VydmUiOi0xLCJsZXZlbCI6Miwic3R5bGUiOnsiaGVhZCI6eyJuYW1lIjoibm9uZSJ9fX1dLFs0LDIsIiIsMix7ImN1cnZlIjoxLCJsZXZlbCI6Miwic3R5bGUiOnsiaGVhZCI6eyJuYW1lIjoibm9uZSJ9fX1dLFswLDQsIiIsMCx7InN0eWxlIjp7ImJvZHkiOnsibmFtZSI6ImRhc2hlZCJ9fX1dXQ==&macro_url=https%3A%2F%2Fraw.githubusercontent.com%2FdFoiler%2Fnotes%2Fmaster%2Fnir.tex
		\[\begin{tikzcd}
			A \\
			& {A\otimes_BA} & A \\
			& A & B
			\arrow[from=2-3, to=2-2]
			\arrow[from=3-3, to=2-3]
			\arrow[from=3-3, to=3-2]
			\arrow[from=3-2, to=2-2]
			\arrow[curve={height=-6pt}, Rightarrow, no head, from=1-1, to=2-3]
			\arrow[curve={height=6pt}, Rightarrow, no head, from=1-1, to=3-2]
			\arrow[dashed, from=2-2, to=1-1]
		\end{tikzcd}\]
		so that we want to show that $A\otimes_BA\to A$ induces a closed embedding, which means we want to show that this induced map is surjective. However, this is pretty direct because the map is by $a_1\otimes a_2\mapsto a_1a_2$.
		\item To show that we have a locally closed embedding, we need to find the requested intermediate open embedding $U\subseteq X\times_YX$. It suffices to glue these things together on small open affine subsets.

		As such, we begin by giving $Y$ an affine open cover $\{V_\alpha\}$, and then we give $X$ an affine cover $\{U_{\alpha\beta}\}$ where each open affine is a subset of $\pi^{-1}V_\alpha$ where $V_\alpha\subseteq Y$ is one of the subsets in the affine open cover of $Y$. Thus, $X\times_YX$ is covered by
		\[U_{\alpha\beta_1}\times_{V_\alpha}U_{\alpha\beta_2}=U_{\alpha\beta_1}\times_YU_{\alpha,\beta_2}.\]
		One can show that $\Delta(X)$ will now be contained in the union of the ``diagonal'' elements $U_{\alpha\beta}\times_YU_{\alpha\beta}$.

		Now, for any given point $p\in\Delta(X)$, find some $U_{\alpha\beta}$ containing $p$, and we can show that $\Delta$ now restricts to
		\[\Delta\colon U_{\alpha\beta}\to U_{\alpha\beta}\times_YU_{\alpha\beta},\]
		which is now a closed embedding. Thus, we have a locally closed embedding on $\Delta(X)$.
		\qedhere
	\end{listalph}
\end{proof}
And now here is our definition.
\begin{definition}[Separated]
	A scheme morphism $\pi\colon X\to Y$ is \textit{separated} if and only if $\Delta\colon X\to X\times_YX$ is a closed embedding. A scheme $X$ is \textit{separated} if and only if the map $X\to\Spec\ZZ$ is separated.
\end{definition}
\begin{example}
	Morphisms of affine schemes are separated by \autoref{prop:diag-is-local-closed}.
\end{example}
\begin{example}
	It turns out that affine morphisms are separated.
\end{example}
\begin{example}
	Monomorphisms $\pi\colon X\to Y$ are separated: indeed, the diagonal morphism $\Delta\pi$ is an isomorphism by \autoref{lem:monic-means-diag-iso}, which is a closed embedding by \autoref{ex:iso-is-closed}. For example, open embeddings are separated by \autoref{cor:open-embed-monic}, and closed embeddings are separated by \autoref{lem:closed-monic}.
\end{example}
\begin{corollary} \label{cor:separated-by-diag}
	A scheme morphism $\pi\colon X\to Y$ is separated if and only if $\Delta(X)\subseteq X\times_YX$ is closed topologically.
\end{corollary}
\begin{proof}
	This follows straight from \autoref{prop:diag-is-local-closed}.
\end{proof}
\begin{corollary}
	Separated morphisms $\pi\colon X\to Y$ are quasiseparated.
\end{corollary}
\begin{proof}
	The diagonal map is a closed embedding and hence quasicompact.
\end{proof}
\begin{remark}
	Being separated is local on the target because we check if something is a closed embedding on the target.
\end{remark}
Here is a more extended example.
\begin{exe}
	Glue the affine schemes $\Spec k[x]$ and $\Spec k[y]$ by the isomorphism $\Spec k[x,x^{-1}]\simeq\Spec k[y,y^{-1}]$, which makes a scheme $X$. This scheme is not separated over $\Spec k$.
\end{exe}
\begin{proof}
	We really only have to pay attention to the topological spaces. Note that $X\times_{\Spec k}X$ as a topological space looks like $\AA^2_k$ with a doubled $y$-axis and a doubled $x$-axis and a quadrupled origin; one can check this through by gluing everything. As such, the diagonal map has image consisting of the line $y=x$, with two of the origins. However, the two origins have closure of all four of the origins, so the total image is not closed.
\end{proof}
\begin{remark}
	However, it is true that our scheme is quasiseparated. Namely, we can see somewhat directly that our pre-images are quasicompact.
\end{remark}
We close class with a last remark.
\begin{lemma}
	Separated morphisms are stable under composition.
\end{lemma}
\begin{proof}
	We stare at diagonal maps very hard. Let $f\colon X\to Y$ and $g\colon Y\to Z$ both separated morphisms. Then we draw the large diagram
	% https://q.uiver.app/?q=WzAsNSxbMCwwLCJYIl0sWzIsMCwiWSJdLFsxLDAsIlhcXHRpbWVzX1lYIl0sWzEsMSwiWFxcdGltZXNfWlgiXSxbMiwxLCJZXFx0aW1lc19aWSJdLFsxLDQsIlxcRGVsdGEgX2ciXSxbMCwyLCJcXERlbHRhX2YiXSxbMCwzLCJcXERlbHRhX3tnXFxjaXJjIGZ9IiwyXSxbMiwxXSxbMyw0XSxbMiwzXSxbMiw0LCIiLDEseyJzdHlsZSI6eyJuYW1lIjoiY29ybmVyIn19XV0=&macro_url=https%3A%2F%2Fraw.githubusercontent.com%2FdFoiler%2Fnotes%2Fmaster%2Fnir.tex
	\[\begin{tikzcd}
		X & {X\times_YX} & Y \\
		& {X\times_ZX} & {Y\times_ZY}
		\arrow["{\Delta _g}", from=1-3, to=2-3]
		\arrow["{\Delta_f}", from=1-1, to=1-2]
		\arrow["{\Delta_{g\circ f}}"', from=1-1, to=2-2]
		\arrow[from=1-2, to=1-3]
		\arrow[from=2-2, to=2-3]
		\arrow[from=1-2, to=2-2]
		\arrow["\lrcorner"{anchor=center, pos=0.125}, draw=none, from=1-2, to=2-3]
	\end{tikzcd}\]
	and note that the square is a pullback square. Thus, the map $X\times_YX\to X\times_ZX$ is a closed embedding by pulling it back, so $\Delta_{g\circ f}$ is a closed embedding because it is the composition of two closed embeddings.
\end{proof}
\begin{remark}
	The same diagram shows that if $g\circ f$ is separated implies that $f$ is separated. Indeed, the diagram tells us that the map $X\times_YX\to X\times_ZX$ is locally closed, so the image of $\Delta_f$ is forced to be a closed set in total, by looking purely topologically.
\end{remark}
\begin{lemma}
	Separated morphisms are stable under base change.
\end{lemma}
\begin{proof}
	Let $\pi\colon X\to B$ be our separated morphism, and we have a base extension $\iota\colon B'\to B$. Now we draw our pullback square
	% https://q.uiver.app/?q=WzAsNCxbMCwxLCJCJyJdLFsxLDEsIkIiXSxbMSwwLCJYIl0sWzAsMCwiWCciXSxbMCwxLCJcXGlvdGEiXSxbMiwxLCJcXHBpIl0sWzMsMF0sWzMsMl0sWzMsMSwiIiwxLHsic3R5bGUiOnsibmFtZSI6ImNvcm5lciJ9fV1d&macro_url=https%3A%2F%2Fraw.githubusercontent.com%2FdFoiler%2Fnotes%2Fmaster%2Fnir.tex
	\[\begin{tikzcd}
		{X'} & X \\
		{B'} & B
		\arrow["\iota", from=2-1, to=2-2]
		\arrow["\pi", from=1-2, to=2-2]
		\arrow[from=1-1, to=2-1]
		\arrow[from=1-1, to=1-2]
		\arrow["\lrcorner"{anchor=center, pos=0.125}, draw=none, from=1-1, to=2-2]
	\end{tikzcd}\]
	and track through our diagonal maps.
\end{proof}

\end{document}