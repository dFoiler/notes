% !TEX root = ../notes.tex

\documentclass[../notes.tex]{subfiles}

\begin{document}

\section{September 30}

We continue the proof of \autoref{thm:chev} from last class.

\subsection{Finishing \autoref{thm:chev}}
Recall that we had the diagram
% https://q.uiver.app/?q=WzAsNSxbMCwwLCJcXFNwZWMgQV9zIl0sWzEsMCwiXFxTcGVjIEJfc1t4XzEsXFxsZG90cyx4X25dIl0sWzIsMCwiXFxTcGVjIEJfcyJdLFsyLDEsIlxcU3BlYyBCIl0sWzAsMSwiXFxTcGVjIEEiXSxbNCwzLCJcXHBpIl0sWzAsNCwiIiwwLHsic3R5bGUiOnsidGFpbCI6eyJuYW1lIjoiaG9vayIsInNpZGUiOiJ0b3AifX19XSxbMiwzLCIiLDIseyJzdHlsZSI6eyJ0YWlsIjp7Im5hbWUiOiJob29rIiwic2lkZSI6InRvcCJ9fX1dLFswLDEsIlxccGlfcyJdLFsxLDJdXQ==&macro_url=https%3A%2F%2Fraw.githubusercontent.com%2FdFoiler%2Fnotes%2Fmaster%2Fnir.tex
\[\begin{tikzcd}
	{\Spec A_s} & {\Spec B_s[x_1,\ldots,x_n]} & {\Spec B_s} \\
	{\Spec A} && {\Spec B}
	\arrow["\pi", from=2-1, to=2-3]
	\arrow[hook, from=1-1, to=2-1]
	\arrow[hook, from=1-3, to=2-3]
	\arrow["{\pi_s}", from=1-1, to=1-2]
	\arrow[from=1-2, to=1-3]
\end{tikzcd}\]
from spreading out Noether normalization. Note that $\pi_s$ is a dominant morphism because $B_s[x_1,\ldots,x_n]\into A_s$ is a dominant morphism. In fact, $\pi_s$ is a finite morphism by its construction from Noether normalization, so we conclude that $\pi_s(\Spec A_s)=\Spec B_s[x_1,\ldots,x_n]$ by dominance. The map $\Spec B_s[x_1,\ldots,x_n]\to\Spec B_s$ is now certainly surjective, so in total, we see that $\im\pi$ contains the open subset $\Spec B_s\subseteq\Spec B$.

This is actually good enough to finish the proof.
\begin{example}
	Set $B=\Spec k[x]$. Then the open subsets of $\Spec k[x]$ are only missing finitely many points and so will stay open!
\end{example}
Motivated by how nice Noetherian schemes are, we have the following lemma.
\begin{lemma} \label{lem:better-constructible}
	Fix a Noetherian space $Y$. A subset $E\subseteq Y$ is constructible if and only if all irreducible closed subsets $Z\subseteq Y$ has either $E\cap Z$ or $Z\setminus E$ containing a nonempty open set.
\end{lemma}
\begin{proof}
	This is by Noetherian induction. Namely, if the statement were false, we could make $E$ minimal satisfying the above condition and derive contradiction by looking at the two cases.
\end{proof}
Thus, to finish the proof, we use \autoref{lem:better-constructible}. We want to show that $\pi(X)$ is irreducible, so for some irreducible closed subset $Z\subseteq Y$ we use the same reduction steps as above to show that the restricted map
\[\pi\colon\pi^{-1}Z\to Z\]
has, for any open irreducible subset $U\subseteq\overline{\pi(Z)}$ is strictly contained in $Z$ (so that the restricted morphism is not dominant) or $\pi(U)$ contains an open nonempty subset inside $Z$ (because the argument goes through as soon as the restriction is dominant). In the former case, we will get that $Z\setminus\pi(U)$ contains a nonempty open subset, so \autoref{lem:better-constructible} kicks in to tell us that $\pi(X)$ is constructible.
\begin{remark}
	Even though the statement of \autoref{thm:chev} is non-constructive, one can concretely work through the above proof using a somewhat explicit construction from $\pi_s$. In particular, after peeling off the desired open subset $B_s$, what's left over is approximately speaking some lower-dimensional object, so we can induct downwards.
	% 8.4.4
\end{remark}

\subsection{Separated Morphisms}
We are moving towards defining varieties, for which we need to define separated morphisms. Hartshorne proves some difficult criterion for morphisms to be separated and proper, but many of the corollaries of the criterion can be proven without it, so we will avoid proving the criterion for now.
\begin{remark}
	Fix a topological space $X$. Then $X$ being Hausdorff if and only if the diagonal map $\Delta X\to X\times X$ (by $x\mapsto(x,x)$) makes $\Delta(X)\subseteq X\times X$ a closed subset.
\end{remark}
One shows the above remark by winding the definitions around. Separated is going to be the correct analogue for Hausdorff for schemes (recall that the Zariski topology is somewhat terrible), using the above ideas.

These comments are intended to motivate us to look at diagonal morphisms.
\begin{proposition} \label{prop:diag-is-local-closed}
	Fix a scheme morphism $\pi\colon X\to Y$.
	\begin{listalph}
		\item Suppose $X=\Spec A$ and $Y=\Spec B$ are affine. Then the induced diagonal map $\Delta\colon X\to X\times_YX$ (induced by $\id_X\colon X\to X$) is a closed embedding.
		\item In general, $\pi$ is a locally closed embedding.
	\end{listalph}
\end{proposition}
\begin{proof}
	We go in sequence.
	\begin{listalph}
		\item On the level of rings, we are looking at the diagram
		% https://q.uiver.app/?q=WzAsNSxbMSwxLCJBXFxvdGltZXNfQkEiXSxbMiwxLCJBIl0sWzEsMiwiQSJdLFsyLDIsIkIiXSxbMCwwLCJBIl0sWzEsMF0sWzMsMV0sWzMsMl0sWzIsMF0sWzQsMSwiIiwyLHsiY3VydmUiOi0xLCJsZXZlbCI6Miwic3R5bGUiOnsiaGVhZCI6eyJuYW1lIjoibm9uZSJ9fX1dLFs0LDIsIiIsMix7ImN1cnZlIjoxLCJsZXZlbCI6Miwic3R5bGUiOnsiaGVhZCI6eyJuYW1lIjoibm9uZSJ9fX1dLFswLDQsIiIsMCx7InN0eWxlIjp7ImJvZHkiOnsibmFtZSI6ImRhc2hlZCJ9fX1dXQ==&macro_url=https%3A%2F%2Fraw.githubusercontent.com%2FdFoiler%2Fnotes%2Fmaster%2Fnir.tex
		\[\begin{tikzcd}
			A \\
			& {A\otimes_BA} & A \\
			& A & B
			\arrow[from=2-3, to=2-2]
			\arrow[from=3-3, to=2-3]
			\arrow[from=3-3, to=3-2]
			\arrow[from=3-2, to=2-2]
			\arrow[curve={height=-6pt}, Rightarrow, no head, from=1-1, to=2-3]
			\arrow[curve={height=6pt}, Rightarrow, no head, from=1-1, to=3-2]
			\arrow[dashed, from=2-2, to=1-1]
		\end{tikzcd}\]
		so that we want to show that $A\otimes_BA\to A$ induces a closed embedding, which means we want to show that this induced map is surjective. However, this is pretty direct because the map is by $a_1\otimes a_2\mapsto a_1a_2$.
		\item To show that we have a locally closed embedding, we need to find the requested intermediate open embedding $U\subseteq X\times_YX$. It suffices to glue these things together on small open affine subsets.

		As such, we begin by giving $Y$ an affine open cover $\{V_\alpha\}$, and then we give $X$ an affine cover $\{U_{\alpha\beta}\}$ where each open affine is a subset of $\pi^{-1}V_\alpha$ where $V_\alpha\subseteq Y$ is one of the subsets in the affine open cover of $Y$. Thus, $X\times_YX$ is covered by
		\[U_{\alpha\beta_1}\times_{V_\alpha}U_{\alpha\beta_2}=U_{\alpha\beta_1}\times_YU_{\alpha,\beta_2}.\]
		One can show that $\Delta(X)$ will now be contained in the union of the ``diagonal'' elements $U_{\alpha\beta}\times_YU_{\alpha\beta}$.

		Now, for any given point $p\in\Delta(X)$, find some $U_{\alpha\beta}$ containing $p$, and we can show that $\Delta$ now restricts to
		\[\Delta\colon U_{\alpha\beta}\to U_{\alpha\beta}\times_YU_{\alpha\beta},\]
		which is now a closed embedding. Thus, we have a locally closed embedding on $\Delta(X)$.
		\qedhere
	\end{listalph}
\end{proof}
And now here is our definition.
\begin{definition}[Separated]
	A scheme morphism $\pi\colon X\to Y$ is \textit{separated} if and only if $\Delta\colon X\to X\times_YX$ is a closed embedding. A scheme $X$ is \textit{separated} if and only if the map $X\to\Spec\ZZ$ is separated.
\end{definition}
\begin{example}
	Morphisms of affine schemes are separated by \autoref{prop:diag-is-local-closed}.
\end{example}
\begin{example}
	It turns out that affine morphisms are separated.
\end{example}
\begin{corollary} \label{cor:separated-by-diag}
	A scheme morphism $\pi\colon X\to Y$ is separated if and only if $\Delta(X)\subseteq X\times_YX$ is closed topologically.
\end{corollary}
\begin{proof}
	This follows straight from \autoref{prop:diag-is-local-closed}.
\end{proof}
\begin{corollary}
	Separated morphisms $\pi\colon X\to Y$ are quasiseparated.
\end{corollary}
\begin{proof}
	The diagonal map is a closed embedding and hence quasicompact.
\end{proof}
\begin{remark}
	Being separated is local on the target because we check if something is a closed embedding on the target.
\end{remark}
Here is a more extended example.
\begin{exe}
	Glue the affine schemes $\Spec k[x]$ and $\Spec k[y]$ by the isomorphism $\Spec k[x,x^{-1}]\simeq\Spec k[y,y^{-1}]$, which makes a scheme $X$. This scheme is not separated over $\Spec k$.
\end{exe}
\begin{proof}
	We really only have to pay attention to the topological spaces. Note that $X\times_{\Spec k}X$ as a topological space looks like $\AA^2_k$ with a doubled $y$-axis and a doubled $x$-axis and a quadrupled origin; one can check this through by gluing everything. As such, the diagonal map has image consisting of the line $y=x$, with two of the origins. However, the two origins have closure of all four of the origins, so the total image is not closed.
\end{proof}
\begin{remark}
	However, it is true that our scheme is quasiseparated. Namely, we can see somewhat directly that our pre-images are quasicompact.
\end{remark}
We close class with a last remark.
\begin{lemma}
	Separated morphisms are stable under composition.
\end{lemma}
\begin{proof}
	We stare at diagonal maps very hard. Let $f\colon X\to Y$ and $g\colon Y\to Z$ both separated morphisms. Then we draw the large diagram
	% https://q.uiver.app/?q=WzAsNSxbMCwwLCJYIl0sWzIsMCwiWSJdLFsxLDAsIlhcXHRpbWVzX1lYIl0sWzEsMSwiWFxcdGltZXNfWlgiXSxbMiwxLCJZXFx0aW1lc19aWSJdLFsxLDQsIlxcRGVsdGEgX2ciXSxbMCwyLCJcXERlbHRhX2YiXSxbMCwzLCJcXERlbHRhX3tnXFxjaXJjIGZ9IiwyXSxbMiwxXSxbMyw0XSxbMiwzXSxbMiw0LCIiLDEseyJzdHlsZSI6eyJuYW1lIjoiY29ybmVyIn19XV0=&macro_url=https%3A%2F%2Fraw.githubusercontent.com%2FdFoiler%2Fnotes%2Fmaster%2Fnir.tex
	\[\begin{tikzcd}
		X & {X\times_YX} & Y \\
		& {X\times_ZX} & {Y\times_ZY}
		\arrow["{\Delta _g}", from=1-3, to=2-3]
		\arrow["{\Delta_f}", from=1-1, to=1-2]
		\arrow["{\Delta_{g\circ f}}"', from=1-1, to=2-2]
		\arrow[from=1-2, to=1-3]
		\arrow[from=2-2, to=2-3]
		\arrow[from=1-2, to=2-2]
		\arrow["\lrcorner"{anchor=center, pos=0.125}, draw=none, from=1-2, to=2-3]
	\end{tikzcd}\]
	and note that the square is a pullback square. Thus, the map $X\times_YX\to X\times_ZX$ is a closed embedding by pulling it back, so $\Delta_{g\circ f}$ is a closed embedding because it is the composition of two closed embeddings.
\end{proof}
\begin{lemma}
	Separated morphisms are stable under base change.
\end{lemma}
\begin{proof}
	Let $\pi\colon X\to B$ be our separated morphism, and we have a base extension $\iota\colon B'\to B$. Now we draw our pullback square
	% https://q.uiver.app/?q=WzAsNCxbMCwxLCJCJyJdLFsxLDEsIkIiXSxbMSwwLCJYIl0sWzAsMCwiWCciXSxbMCwxLCJcXGlvdGEiXSxbMiwxLCJcXHBpIl0sWzMsMF0sWzMsMl0sWzMsMSwiIiwxLHsic3R5bGUiOnsibmFtZSI6ImNvcm5lciJ9fV1d&macro_url=https%3A%2F%2Fraw.githubusercontent.com%2FdFoiler%2Fnotes%2Fmaster%2Fnir.tex
	\[\begin{tikzcd}
		{X'} & X \\
		{B'} & B
		\arrow["\iota", from=2-1, to=2-2]
		\arrow["\pi", from=1-2, to=2-2]
		\arrow[from=1-1, to=2-1]
		\arrow[from=1-1, to=1-2]
		\arrow["\lrcorner"{anchor=center, pos=0.125}, draw=none, from=1-1, to=2-2]
	\end{tikzcd}\]
	and track through our diagonal maps.
\end{proof}

\end{document}