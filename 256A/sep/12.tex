% !TEX root = ../notes.tex

\documentclass[../notes.tex]{subfiles}

\begin{document}

\section{September 12}

The classroom is emptier than usual.

\subsection{Projective Schemes from \texorpdfstring{$\mathrm{Proj}$}{ Proj}}
We quickly finish our definition of a projective scheme.
\begin{definition}[Projective scheme]
	Fix a ring $R$. A scheme $(X,\mc O_X)$ is a \textit{projective scheme over $R$} if and only if $(X,\mc O_X)$ is isomorphic (as schemes) to some
	\[\op{Proj}R[x_0,\ldots,x_n]/I\]
	for a homogeneous ideal $I\subseteq R[x_0,\ldots,x_n]$. Equivalently, $(X,\OO_X)$ is isomorphic to some $\op{Proj}S$, where $S$ is a finitely generated graded $R$-algebra.
\end{definition}
Intuitively, the ring map
\[R[x_0,\ldots,x_n]\onto R[x_0,\ldots,x_n]/I\]
will induce an embedding from $(X,\OO_X)$ into $\PP^n_R$. So a projective scheme is really just one which has an embedding into projective space.
\begin{remark}
	It is not totally trivial that we may allow $S$ to be finitely generated from elements outside $S_1$. See \cite[Section~7.4.4]{rising-sea}.
\end{remark}
Here is another equivalent definition.
\begin{definition}[Projective scheme]
	Fix a ring $R$. A scheme $(X,\mc O_X)$ is a \textit{projective scheme over $R$} if and only if there is a ``closed embedding'' $X\into\PP^n_R$ of schemes.
\end{definition}
We haven't defined a closed embedding yet, but we will do this soon.

\subsection{Topological Adjectives}
We start by describing a scheme by focusing on its topological space.
\begin{definition}[Connected]
	A scheme $(X,\mc O_X)$ is \textit{connected} if and only if $X$ is connected as a topological space. In other words, if $X=V_1\sqcup V_2$ for closed subsets $V_1,V_2\subseteq X$, then one of the $V_1=X$ or $V_2=X$.
\end{definition}
\begin{definition}[Irreducible]
	A scheme $(X,\mc O_X)$ is \textit{irreducible} if and only if $X$ is irreducible as a topological space. In other words, we require $X$ to be nonempty, and if $X=V_1\cup V_2$ for closed subsets $V_1,V_2\subseteq X$, then one of the $V_1=X$ or $V_2=X$.
\end{definition}
\begin{example} \label{ex:closurepoint}
	Let $X$ be a topological space. Then, for any $x\in X$, the subset $\overline{\{x\}}\subseteq X$ is irreducible: if closed sets $V_1,V_2\subseteq X$ have $\overline{\{x\}}\in V_1\cup V_2$, then $x\in V_1$ or $x\in V_2$, so $\overline{\{x\}}\subseteq V_1$ or $\overline{\{x\}}\subseteq V_2$.
\end{example}
\begin{example}
	Projective space $\PP^n_R$ is irreducible.\todo{What.}
\end{example}
\begin{nex} \label{nex:speckxymodxy}
	The scheme $\Spec k[x,y]/(xy)$ is connected but not irreducible. The picture is as follows.
	\begin{center}
		\begin{asy}
			unitsize(0.5cm);
			draw((-2,0)--(2,0));
			draw((0,2)--(0,-2));
		\end{asy}
	\end{center}
	We will explain this example in more detail shortly in \autoref{rem:explainspeckxymodxy}.
\end{nex}
We like compact topological spaces, so here is the scheme analogue.
\begin{definition}[Quasicompact]
	A scheme $(X,\OO_X)$ is \textit{quasicompact} if and only if any open cover of the topological space $X$ has a finite subcover.
\end{definition}
\begin{example}
	The scheme $\Spec A$ is quasicompact.
\end{example}
\begin{nex} \label{nex:notquasicompactscheme}
	The infinite disjoint union $\bigsqcup_{i=1}^\infty\Spec\ZZ$ is not quasicompact.\todo{Explain with gluing.}
\end{nex}
\begin{nex}
	The scheme $\op{Proj}k[x_1,x_2,\ldots]$ is not quasicompact.
\end{nex}

\subsection{Components}
Having discussed the entire topological space, we might be interested in studying some interesting subspaces.
\begin{definition}[Connected component]
	Fix a topological space $X$. A \textit{connected component} is a maximal connected subset of $X$.
\end{definition}
\begin{definition}[Irreducible component]
	Fix a topological space $X$. An \textit{irreducible component} is a maximal irreducible subset of $X$.
\end{definition}
Here are some quick facts.
\begin{lemma} \label{lem:componentsexist}
	Fix a topological space $X$.
	\begin{listalph}
		\item If a subset $V\subseteq X$ is irreducible, then $V$ is connected.
		\item If a subset $V\subseteq X$ is irreducible (respectively, connected), then so is $\overline V$.
		\item All points $x\in X$ are contained in an irreducible component. Also, all points of $x\in X$ are contained in a connected component.
	\end{listalph}
\end{lemma}
\begin{proof}
	We go one at a time.
	\begin{listalph}
		\item Suppose that $V\subseteq V_1\sqcup V_2$ where $V_1,V_2\subseteq X$ are closed subsets. Being irreducible forces $V\subseteq V_1$ or $V\subseteq V_2$, so connectivity of $V$ follows.
		\item We have two claims to show.
		\begin{itemize}
			\item Take $V$ irreducible so that we want to show $\overline V$ is irreducible. Suppose $\overline V\subseteq V_1\cup V_2$ where $V_1,V_2\subseteq X$ are closed. Then $V\subseteq V_1\cup V_2$, so $V\subseteq V_1$ or $V\subseteq V_2$, so properties of the closure promise $\overline V\subseteq V_1$ or $\overline V\subseteq V_2$.
			\item Take $V$ connected so that we want to show $\overline V$ is connected. Well, replace the $\cup$ in the previous proof with a $\sqcup$, and the proof goes through verbatim.
		\end{itemize}
		\item Observe that $\{x\}$ is irreducible: if $\{x\}\subseteq V_1\cup V_2$ where $V_1,V_2\subseteq X$ are closed, then $x\in V_1$ or $x\in V_2$, so $\{x\}\subseteq V_1$ or $\{x\}\subseteq V_2$.

		We now apply Zorn's lemma twice.
		\begin{itemize}
			\item Let $\mc I_x$ denote the set of irreducible subsets of $X$ containing $x$. We need to show that $\mc I_x$ has a maximal element, which will finish because any maximal element of $\mc I_x$ will be maximal among all irreducible subsets.
			
			Note $\{x\}\in\mc I_x$ means that $\mc I_x$ is nonempty. We now show that $\mc I_x$ satisfies the ascending chain condition: given a totally ordered set $\lambda$ and nonempty ascending chain $\{V_\alpha\}_{\alpha\in\lambda}\subseteq\mc I_x$, we claim that
			\[V\coloneqq\bigcup_{\alpha\in\lambda}V_\alpha\]
			and contains $x$. That $x\in V$ is clear because $x$ lives in any of the $V_\alpha$. To see irreducibility, suppose that $V\subseteq V_1\cup V_2$.
			
			If $V\subseteq V_1$, then we are done, so suppose that we can find $p\in V\setminus V_1$. This means that $p\in V_\beta\setminus V_1$ for some $\beta\in\lambda$, so $p\in V_\beta\setminus V_1$ for all $\alpha\ge\beta$. However, $V_\alpha\subseteq V_1\cup V_2$ still even though $V_1\not\subseteq V_1$, so we must instead have
			\[V_\alpha\subseteq V_2\]
			for all $\alpha\ge\beta$. It follows $V=\bigcup_{\alpha\ge\beta}V_\alpha\subseteq V_2$.

			\item Let $\mc C_x$ denote the set of connected subsets of $X$ containing $x$. We actually claim that
			\[V\coloneqq\bigcup_{C\in\mc C_x}C\]
			is connected. This will finish because $V$ is a connected component containing $x$: if $V\subseteq V'$ with $V'$ connected, then $x\in V'$, so $V'\in\mc C_x$, so $V'\subseteq V$.
			
			We now check $V$ is connected. Suppose $V\subseteq V_1\sqcup V_2$ for closed subsets $V_1,V_2\subseteq X$.

			The main point is that $x\in V_1$ or $x\in V_2$. Without loss of generality, take $x\in V_1$ so that $x\notin V_2$. Now, any $C\in\mc C_x$ has $C\subseteq V_1\sqcup V_2$, so $C\subseteq V_1$ or $C\subseteq V_2$. However, $x\in C\setminus V_2$, so we must have $C\subseteq V_1$ instead, meaning that actually $V\subseteq C_1$.
			\qedhere
		\end{itemize}
	\end{listalph}
\end{proof}
\begin{remark}
	It follows from the above proof that any connected subset $C$ of $x$ is contained in the connected component of $x$.
\end{remark}
Here is another nice result.
\begin{proposition} \label{prop:opensinirredspace}
	If $X$ is an irreducible topological space, then all nonempty open subsets $U\subseteq X$ have $U$ irreducible and $\overline U$ dense in $X$.
\end{proposition}
\begin{proof}
	We have two claims to show.
	\begin{itemize}
		\item We show $U$ is irreducible. Suppose $U\subseteq V_1\cup V_2$ for closed subsets $V_1,V_2\subseteq X$. It follows that
		\[X\subseteq\big((X\setminus U)\cup V_1\big)\cup V_2\]
		has covered $X$ by closed subsets. It follows that either $V_2=X$ (and hence covers $U$) or $(X\setminus U)\cup V_1=X$ (and so $V_1\supseteq U$).
		\item We show $\overline U=X$. Indeed, we can cover $X$ by closed sets as
		\[X=(X\setminus U)\cup\overline U,\]
		so either $X\setminus U=X$, which is impossible because $U$ is nonempty, or $\overline U=X$, which finishes.
		\qedhere
	\end{itemize}
\end{proof}
Even though irreducible components are a little weird in typical point-set topology, they are of interest in scheme theory.
\begin{lemma} \label{lem:affineirredcomps}
	Fix a ring $A$.
	\begin{listalph}
		\item Given an ideal $I\subseteq A$, the subset $V(I)\subseteq\Spec A$ is irreducible if and only if $\rad I$ is prime.
		\item The irreducible components of $X$ are
		\[\{V(\mf p):\mf p\in\Spec A\text{ is a minimal prime}\}.\]
	\end{listalph}
\end{lemma}
\begin{proof}
	As usual, we go in sequence.
	\begin{listalph}
		\item We have two claims to show.
		\begin{itemize}
			\item Suppose that $\mf p\coloneqq\rad I$ is prime. Then \autoref{prop:easynullstellensatz} tells us
			\[V(I)=V(\rad I)=V(\mf p)=V(I(\{\mf p\}))=\overline{\{\mf p\}},\]
			which is irreducible by \autoref{ex:closurepoint}.
			\item Suppose that $V(I)$ is irreducible; by replacing $I$ with $\rad I$, we may assume that $I$ is radical. We want to show that $I$ is prime. Well, if $a,b\in A$ have $ab\in I$, we want to show $a\notin I$ and $b\notin I$. Well, $ab\in I$ means any $\mf p$ containing $V(I)$ contains $(ab)$, so
			\[V(I)\subseteq V(ab)=V((a))\cup V((b))\]
			using \autoref{lem:zariskitopcheck}. Because $V(I)$ is irreducible, we conclude that $V(I)\subseteq V((a))$ without loss of generality. Thus, $\rad((a))\subseteq\rad I=I$ by \autoref{prop:easynullstellensatz}, so $a\in I$.
		\end{itemize}
		\item The inclusion reversing-bijection of \autoref{prop:easynullstellensatz} takes prime ideals $\mf p\subseteq A$ to $V(\mf p)$, which we have seen is an irreducible closed subset; and it takes closed subsets $X\subseteq\Spec A$, which we know must take the form $V(\mf p)$ for a prime $\mf p$, to a prime ideal $I(X)=I(V(\mf p))=\rad\mf p=\mf p$.

		Thus, the inclusion-reversing bijection restricts to an inclusion-reversing bijection between prime ideals of $A$ and irreducible closed subsets of $\Spec A$. Thus, the maximal irreducible closed subsets of $\Spec A$ correspond (under this bijection) to minimal prime ideals of $A$.
		
		The claim follows, upon remarking that irreducible components are equal to their closure and hence closed (by \autoref{lem:componentsexist} (b)), so maximal irreducible closed subsets are just maximal irreducible subsets.
		\qedhere
	\end{listalph}
\end{proof}
\begin{remark} \label{rem:explainspeckxymodxy}
	We are now ready to explain \autoref{nex:speckxymodxy}.
	\begin{itemize}
		\item Not irreducible: note that any prime $\mf p\in\Spec k[x,y]/(xy)$ contains $xy=0\in\mf p$, so $(x)\subseteq\mf p$ or $(y)\subseteq\mf p$. On the other hand, $(x)$ and $(y)$ are primes (with quotients isomorphic to $k[t]$), so $(x)$ and $(y)$ are our minimal primes. Thus, \autoref{lem:affineirredcomps} tells us that $V((x))$ and $V((y))$ are our irreducible components; in particular, the entire space is not irreducible.
		\item Connected: note $(x,y)\in V((x))$ and $(x,y)\in V((y))$, so the connected component of $(x,y)$ contains $V((x))\cup V((y))$, which is the entire space because we have taken the union of our irreducible components (and all points live in some irreducible component by \autoref{lem:componentsexist}). So the full space is connected.
	\end{itemize}
\end{remark}

\subsection{Closed and Generic Points}
We like our topological spaces to be Hausdorff, but we have seen that this need not happen in our schemes. So let's keep track of the good points we try to be Hausdorff
\begin{definition}[Closed point]
	Fix a topological space $X$. Then a point $x\in X$ is a \textit{closed point} if and only if $\overline{\{x\}}=\{x\}$.
\end{definition}
In the variety setting, we are more interested in counting closed points, which correspond to the ``actual'' points on our variety. As such, we might hope that we have ``lots'' of closed points in our schemes, and under suitable smallness conditions, we do.
\begin{lemma} \label{lem:enoughclosedpoints}
	Let $(X,\OO_X)$ be a quasicompact scheme. Then any nonempty closed subset $V\subseteq X$ contains a closed point.
\end{lemma}
\begin{proof}
	Note that this is essentially ring theory for affine schemes: for an affine scheme $(\Spec A,\OO_{\Spec A})$, we see that a closed subset $V(I)\subseteq\Spec A$ being nonempty forces $I$ to be proper, so $I$ is contained in some maximal ideal $\mf m$. So $\mf m\in V(I)$ while \autoref{prop:easynullstellensatz} says
	\[\overline{\{\mf m\}}=V(I(\{\mf m\}))=V(\mf m)=\{\mf m\}\]
	because $\mf m$ is maximal.

	We now attack the general case. Because $X$ has an affine open cover, the quasicompactness condition gives $V$ (which is closed in a quasicompact space and hence quasicompact) a finite affine open cover $\{U_i\}_{i=1}^n$ so that we have rings $A_i$ such that
	\[(U_i,\OO_X|_{U_i})\cong(\Spec A_i,\OO_{\Spec A_i})\]
	for each $i$. We may assume that none of the $\{V\cap U_i\}_{i=1}^n$ is contained in the union of the other, for otherwise we could remove the offending $U_i$.\footnote{Here we implicitly use the fact that there are only finitely many $U_i$.} Now,
	\[U_1\cap\Bigg(V\cap\bigcap_{i=1}^n(X\setminus U_i)\Bigg)\]
	is a closed subset of $U_1$, so because $(U_1,\OO_X|_{U_1})$ is an affine scheme, it will have a closed point $p\in U_1$.

	Notably, $p\in V$ by construction, so it remains to show that $\{p\}\subseteq X$ is closed. Well, $\{p\}\subseteq U_1$ is closed, so $U_1\setminus\{p\}\subseteq U_1$ is open, so there is some open set $U'\subseteq X$ such that $U'\cap U_1=U_1\setminus\{p\}$. It follows that
	\[X\setminus\{p\}=(U_1\setminus\{p\})\cup\bigcup_{i=2}^nU_i\]
	because $p\notin U_i$ for any $i\ne1$; in particular, $X\setminus\{p\}\subseteq X$ is open, finishing.
\end{proof}
\begin{remark}
	Sadly, there are examples of schemes with no closed points.
\end{remark}
Having kept track of our closed points, we don't want to shame our ``unclosed points,'' so we give them a name as well.
\begin{definition}[Generic point]
	Fix a topological space $X$. Then a point $x\in X$ is a \textit{generic point} of an irreducible subset $V\subseteq X$ if and only if $V=\overline{\{x\}}$.
\end{definition}
\begin{example} \label{ex:uniquegenericforaffine}
	Given a ring $A$, the point $\mf p\in\Spec A$ is the (unique!) generic point of $V(\mf p)$. Indeed, certainly $V(\mf p)=V(I(\{\mf p\}))=\overline{\{\mf p\}}$. On the other hand, if some $\mf q$ has
	\[V(\mf p)=\overline{\{\mf q\}}=V(\mf q),\]
	then $\mf p\supseteq\mf q$ and $\mf q\supseteq\mf p$, so $\mf p=\mf q$.
\end{example}
\begin{example}
	Fix an integral domain $A$. Then $(0)$ is the generic point for $\Spec A$. Notably, $\overline{\{(0)\}}=V((0))=\Spec A$.
\end{example}
The relationship between generic points will be important to keep track of.
\begin{defihelper}[Specialization, generalization] \nirindex{Specialization} \nirindex{Generalization}
	Fix a topological space $X$ and two points $x,y\in X$. We say that $x$ is a \textit{specialization} of $y$ (or equivalently, $y$ is a \textit{generalization} of $x$) if and only if $x\in\overline{\{y\}}$.
\end{defihelper}
\begin{example}
	Given a ring $A$, we see that $\mf q\in\overline{\{\mf p\}}=V(\mf p)$ if and only if $\mf q\supseteq\mf p$.
\end{example}
This provides a sort of ordering on our space. Closed points are the most ``specific,'' so let's keep track of the most generic.
\begin{definition}[Generic point]
	Fix a topological space $X$. A point $p\in X$ is a \textit{generic point} if and only if the only point specializing to $\overline{\{p\}}$ is $p$.
\end{definition}
We saw in \autoref{ex:uniquegenericforaffine} that in fact every irreducible closed subset had a unique generic point. This can be extended so schemes.
\begin{lemma}
	Fix a scheme $(X,\OO_X)$. Then any nonempty irreducible closed subset $Z\subseteq X$ has a unique generic point $p\in X$. In other words, we can write any nonempty irreducible closed subset $Z\subseteq X$ as $Z=\overline{\{p\}}$ for some $p\in X$.
\end{lemma}
\begin{proof}
	Give $X$ an affine open cover $\mc U$ so that each $U\in\mc U$ has $(U,\OO_X|_U)\cong(\Spec A_U,\OO_{\Spec A_U})$ for some ring $A_U$. Now, let $\mc V$ be the open sets $U\in\mc U$ such that $Z\cap U\ne\emp$. As such,
	\[Z=\bigcup_{V\in\mc V}(Z\cap V).\]
	Now, $(V,\OO_X|_V)$ is an affine scheme for each $V\in\mc V$. Further, $Z\cap V\subseteq V$ is a closed subset by the induced topology, and it is irreducible because $Z\cap V\subseteq(V_1\cap V)\cup(V_2\cap V)$ for closed subsets $V_1,V_2\subseteq V$ tells us that
	\[Z\subseteq(X\setminus V)\cup V_1\cup V_2,\]
	so irreducibility of $Z$ and $Z\cap V\ne\emp$ forces $Z\subseteq V_1$ or $Z\subseteq V_2$.

	Thus, \autoref{ex:uniquegenericforaffine} promises a unique point $p_V\in Z\cap V$ such that $Z\cap V=\overline{\{p_V\}}\cap V$ for each $V\in\mc V$. Fixing some $p_V$, we claim that $Z\cap W=\overline{\{p_V\}}\cap W$ for each $W\in\mc V$. Indeed, note
	\[Z\cap W\subseteq(Z\cap V\cap W)\cup(W\setminus V)\subseteq(\overline{\{p_V\}}\cap W)\cup(W\setminus V).\]
	The right-hand side exhibits $Z\cap W$ as the union of two closed subsets of $W$, so the irreducibility of $Z\cap W$ tells us $Z\cap W\subseteq(\overline{\{p_V\}}\cap W)$ or $Z\cap W\subseteq(W\setminus V)$. The second case would imply $Z\subseteq(X\setminus V)\cup(X\setminus W)$, which by irreducibility of $Z$ forces $Z\cap V=\emp$ or $Z\cap W=\emp$, which is assumed false. So instead, we have
	\[Z\cap W\subseteq\overline{\{p_V\}}\cap W,\]
	but of course $p_V\in Z$ forces the other inclusion.

	It follows that
	\[Z=\bigcup_{W\in\mc V}(Z\cap W)=\bigcup_{W\in\mc V}(\overline{\{p_V\}}\cap W)\subseteq\overline{\{p_V\}}.\]
	But $p_V\in Z$ and the fact that $Z$ is closed gives the other inclusion, so we conclude $Z=\overline{\{p_V\}}$. So we have indeed given $Z$ some generic point.

	It remains to show that this generic point is unique. Well, suppose $p,q\in X$ have $\overline{\{p\}}=\overline{\{q\}}$. The affine open cover $\mc U$ from earlier grants us some open set $U$ containing $q$. Note that $p\notin U$ would imply that $\{p\}\subseteq X\setminus U$ and so $\overline{\{p\}}\subseteq X\setminus U$, meaning that $q\notin\overline{\{p\}}$, which is assumed false. So we must have $p\in U$ as well. But then
	\[\overline{\{p\}}\cap U=\overline{\{q\}}\cap U\]
	tells us that $p=q$ by the uniqueness of generic points of irreducible closed subsets in affine schemes, from \autoref{ex:uniquegenericforaffine}.
\end{proof}
\begin{remark}
	It is not in general case that nonempty irreducible closed subsets of a topological space $X$ can be uniquely written as $\overline{\{x\}}$ for some $x\in X$. Here are two examples.
	\begin{itemize}
		\item If $X=\RR$ has the indiscrete topology, the closure of any point is the full space $X$.
		\item If $X=\RR$ has the cofinite topology, the full space $X$ is irreducible (because all proper closed subsets are finite, so the finite union of proper closed subsets cannot cover $X$), but $X$ is not the closure of any point (because all points are closed).
	\end{itemize}
\end{remark}

\subsection{Noetherian Conditions}
Noetherian rings are good, so we will want to push this to our schemes as well.
\begin{definition}[Locally Noetherian]
	A scheme $(X,\OO_X)$ is \textit{locally Noetherian} if and only if $X$ has an open cover $\mc U$ where each $U\in\mc U$ has $(U,\OO_X|_U)$ isomorphic to the affine scheme of a Noetherian ring.
\end{definition}
Noetherian is about making infinite things finite, so we want to add a quasicompact condition this.
\begin{definition}[Noetherian]
	A scheme $(X,\OO_X)$ is \textit{Noetherian} if and only if $X$ is quasicompact and locally Noetherian.
\end{definition}
\begin{example}
	The scheme from \autoref{nex:notquasicompactscheme} is locally Noetherian (it's the infinite disjoint union of $\Spec\ZZ$s, and $\ZZ$ is Noetherian), but it is not quasicompact and hence not Noetherian.
\end{example}
There is also a separate notion of a Noetherian topological space.
\begin{definition}[Noetherian]
	A topological space $X$ is \textit{Noetherian} if and only if the closed subsets of $X$ satisfy the descending chain condition.
\end{definition}
\begin{example} \label{ex:noetherianringisnoetherian}
	If $A$ is a Noetherian ring, then $\Spec A$ is a Noetherian topological space: given a totally ordered set $\lambda$, any descending chain of closed subsets $\{V_\alpha\}_{\alpha\in\lambda}$ gives an ascending chain of $A$-ideals $\{I(V_\alpha)\}_{\alpha\in\lambda}$. Because $A$ is Noetherian, $\{I(V_\alpha)\}_{\alpha\in\lambda}$ must stabilize past some $\beta$, so for $\alpha\ge\beta$, we have $I(V_\alpha)=I(V_\beta)$, so
	\[V_\alpha=\overline{V_\alpha}=V(I(V_\alpha))=V(I(V_\beta))=\overline{V_\beta}=V_\beta,\]
	so the descending chain of closed subsets stabilizes past $\beta$.
\end{example}
\begin{nex}
	Fix a field $k$ and ring $A\coloneqq k[x_1,x_2,x_3,\ldots]/\left(x_1,x_2^2,x_3^3,\ldots\right)$. Then $\Spec A$ a Noetherian topological space even though $A$ is not a Noetherian ring.
	\begin{itemize}
		\item Observe that any prime ideal $\mf p\in\Spec A$ must contain $x_k^k=0$ for each $k\ge1$, so $x_k\in\mf p$. Thus, $\mf m\coloneqq(x_1,x_2,x_3,\ldots)$ is contained in $\mf p$, but $\mf m$ is maximal in $A$ because $A/\mf m\cong k$. So we conclude that $\mf p=\mf m$, meaning that $\Spec A=\{\mf m\}$ has a single point and so is Noetherian as a topological space.
		\item The ascending chain
		\[(x_1)\subsetneq(x_1,x_2)\subsetneq(x_1,x_2,x_3)\subsetneq\cdots\]
		shows that $A$ is not a Noetherian ring.
	\end{itemize}
\end{nex}
Having seen that Noetherian rings give Noetherian spaces, we might hope that a similar result holds for schemes. As in \autoref{lem:enoughclosedpoints}, we must add a quasicompactness hypothesis.
\begin{lemma} \label{lem:noetherianspace}
	If $(X,\OO_X)$ is a Noetherian scheme, then $X$ is a Noetherian topological space.
\end{lemma}
\begin{proof}
	As usual, give $X$ an affine open cover $\mc U$ where the affine schemes are from Noetherian rings; we may make $\mc U$ finite because $X$ is quasicompact. Now, let $\lambda$ be a totally ordered set, and pick up some descending chain $\{V_\alpha\}_{\alpha\in\lambda}$ of closed subsets of $X$.

	Now, for each $U\in\mc U$, so we see that $\{U\cap V_\alpha\}_{\alpha\in\lambda}$ is a descending chain of closed subsets of the affine open set $U$, so because $(U,\OO_X|_U)$ is the affine scheme of a Noetherian ring, \autoref{ex:noetherianringisnoetherian} tells us that $\{U\cap V_\alpha\}$ will stabilize after some $\beta_U$. Thus, we set
	\[\beta\coloneqq\max\{\beta_U:U\in\mc U\},\]
	which exists because $\mc U$ is finite. Then any $\alpha\ge\beta$ and $U\in\mc U$ will have
	\[V_\alpha\cap U=V_{\beta_U}\cap U=V_\beta\cap U,\]
	so $V_\alpha=V_\beta$ after taking the union over $\mc U$. So indeed, our descending chain stabilized after $\beta$.
\end{proof}
\begin{remark}
	More generally, we have shown that a finite union of Noetherian topological spaces is a Noetherian topological space.
\end{remark}
\begin{remark}
	There are some nice philosophical remarks in \cite[Section~3.6.21]{rising-sea} about when we might care about non-Noetherian things.
\end{remark}
As a benefit to keeping things finite, we have the following.
\begin{lemma} \label{lem:opensarecompact}
	Fix a Noetherian topological space $X$. Then any open subset $U\subseteq X$ is quasicompact.
\end{lemma}
\begin{proof}
	We proceed by contraposition. Suppose that we can find an open cover $\mc V$ of $U$ with no finite subcover. In other words, any finite subset of $\mc V$ cannot cover $U$, so we can build some strictly ascending chain of open subsets
	\[V_1\subsetneq V_1\cup V_2\subsetneq V_1\cup V_2\cup V_3\subsetneq\cdots\]
	by choosing each $V_n\in\mc V$ inductively to not be contained in $\bigcup_{k<n}V_k$. (If no such $V_n$ existed, then we would have $U=\bigcup_{V\in\mc V}V\subseteq\bigcup_{k<n}V_k$, granting a finite open subcover.) For brevity, define
	\[V'_n\coloneqq\bigcup_{k\le n}V_k\]
	so that $\{V'_n\}_{n\ge1}$ is a strictly ascending chain of open subsets. Taking complements, we see that $\{X\setminus V'_n\}_{n\ge1}$ is a strictly descending chain of closed subsets, which means that our space $X$ is not Noetherian.
\end{proof}
Here are some applications to affine open subschemes.
\begin{lemma}
	Fix a scheme $(X,\OO_X)$.
	\begin{listalph}
		\item If $(X,\OO_X)$ is locally Noetherian, then any open subset $U\subseteq X$ makes a locally Noetherian subscheme $(U,\OO_X|_U)$.
		\item If $(X,\OO_X)$ is Noetherian, then any open subset $U\subseteq X$ makes a Noetherian subscheme $(U,\OO_X|_U)$.
		\item All stalks $\mc O_{X,p}$ of a locally Noetherian scheme $(X,\OO_X)$ are Noetherian rings.
	\end{listalph}
\end{lemma}
\begin{proof}
	We go in sequence.
	\begin{listalph}
		\item As usual, begin by giving $X$ an affine open cover $\mc U$, and use the locally Noetherian condition to promise that each $V\in\mc U$ has $\varphi^V\colon(V,\OO_X|_V)\cong(\Spec A_V,\OO_{\Spec A_V})$ for a Noetherian ring $A_U$.

		Now, pick up some $V\in\mc U$ and focus on $U\cap V$; it suffices to cover $U\cap V$ with affine open subschemes of Noetherian rings. Notably, $\varphi^V$ makes $U\cap V$ an open subset of $(\Spec A_V,\OO_{\Spec A_V})$. In particular, using the distinguished open base of $\Spec A_V$, we can write
		\[\varphi^V(U\cap V)=\bigcup_{f\in F_V}D(f),\]
		where $F_V\subseteq A_V$ is some subset. Now let $U_f\subseteq U\cap V$ be the pre-image of $D(f)$ under the homeomorphism $\varphi^V$, and we can use our isomorphism to give the scheme isomorphisms
		\[(U\cap V,\OO_X|_U|_{U\cap V})\cong(D(f),\OO_{\Spec A_U}|_{D(f)})\cong(\Spec A_{U,f},\OO_{\Spec_{A_{U,f}}}),\]
		where we have used \autoref{lem:restrictmorphism} for the first isomorphism and \autoref{lem:openbasescheme} for the second isomorphism. Now, $A_{U,f}$ is the localization of the Noetherian ring and hence Noetherian. This is what we wanted.

		\item This follows directly from (a) and \autoref{lem:opensarecompact}: by (a), we see that $(U,\OO_X|_U)$ is locally Noetherian, and because $X$ is a Noetherian topological space by \autoref{lem:noetherianspace}, \autoref{lem:opensarecompact} tells us that $U$ is quasicompact as a topological space. It follows $(U,\OO_X|_U)$ is a Noetherian scheme.

		\item For our point $p\in X$, the locally Noetherian condition promises an open set $U\subseteq X$ containing $p$ and a Noetherian ring $A$ such that
		\[(U,\OO_X|_U)\cong(\Spec A,\OO_{\Spec A}).\]
		Now, \autoref{lem:invimagestalk} tells us that
		\[\OO_{X,p}\simeq(\OO_X|_U)_p=\OO_{\Spec A,p},\]
		which is $A_\mf p$ by \autoref{lem:affinestalk}. This ring $A_\mf p$ is Noetherian because $A$ is Noetherian.
		\qedhere
	\end{listalph}
\end{proof}

\end{document}