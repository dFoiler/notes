% !TEX root = ../notes.tex

\documentclass[../notes.tex]{subfiles}

\begin{document}

\section{September 12}

The classroom is emptier than usual.

\subsection{Projective Schemes from \texorpdfstring{$\mathrm{Proj}$}{Proj}, Continued}
We quickly finish our definition of a projective scheme.
\begin{definition}[Projective scheme]
	Fix a ring $R$. A scheme $(X,\mc O_X)$ is a \textit{projective scheme over $R$} if and only if $(X,\mc O_X)$ is isomorphic (as schemes) to some
	\[\op{Proj}R[x_0,\ldots,x_n]/I\]
	for a homogeneous ideal $I\subseteq R[x_0,\ldots,x_n]$. Equivalently, $(X,\OO_X)$ is isomorphic to some $\op{Proj}S$, where $S$ is a finitely generated graded $R$-algebra.
\end{definition}
Intuitively, the ring map
\[R[x_0,\ldots,x_n]\onto R[x_0,\ldots,x_n]/I\]
will induce an embedding from $(X,\OO_X)$ into $\PP^n_R$. So a projective scheme is really just one which has an embedding into projective space.
\begin{remark}
	It is not totally trivial that we may allow $S$ to be finitely generated from elements outside $S_1$. See \cite[Section~7.4.4]{rising-sea}.
\end{remark}
Here is another equivalent definition.
\begin{definition}[Projective scheme]
	Fix a ring $R$. A scheme $(X,\mc O_X)$ is a \textit{projective scheme over $R$} if and only if there is a ``closed embedding'' $X\into\PP^n_R$ of schemes.
\end{definition}
We haven't defined a closed embedding yet, but we will do this soon.

\subsection{Topological Adjectives}
We start by describing a scheme by focusing on its topological space.
\begin{definition}[Connected]
	A scheme $(X,\mc O_X)$ is \textit{connected} if and only if $X$ is connected as a topological space. In other words, if $X=V_1\sqcup V_2$ for closed subsets $V_1,V_2\subseteq X$, then one of the $V_1=X$ or $V_2=X$.
\end{definition}
\begin{definition}[Irreducible]
	A scheme $(X,\mc O_X)$ is \textit{irreducible} if and only if $X$ is irreducible as a topological space. In other words, we require $X$ to be nonempty, and if $X=V_1\cup V_2$ for closed subsets $V_1,V_2\subseteq X$, then one of the $V_1=X$ or $V_2=X$.
\end{definition}
\begin{example}
	The scheme $\Spec k[x,y]/(xy)$ is connected but not irreducible. The picture is
	\begin{center}
		\begin{asy}
			unitsize(0.5cm);
			draw((-2,0)--(2,0));
			draw((0,2)--(0,-2));
		\end{asy}
	\end{center}
	Notably, $\Spec k[x,y]/(xy)$ is connected, but it is the union of $V(x)$ and $V(y)$.
\end{example}
We like compact topological spaces, so here is the scheme analogue.
\begin{definition}[Quasicompact]
	A scheme $(X,\OO_X)$ is \textit{quasicompact} if and only if any open cover of the topological space $X$ has a finite subcover.
\end{definition}
\begin{example}
	The scheme $\Spec A$ is quasicompact.
\end{example}
\begin{nex}
	The infinite disjoint union $\bigsqcup_{i=1}^\infty\Spec A_i$ is not quasicompact.
\end{nex}
\begin{nex}
	The scheme $\op{Proj}k[x_1,x_2,\ldots]$ is not quasicompact.
\end{nex}

\subsection{Components}
Having discussed the entire topological space, we might be interested in studying some interesting subspaces.
\begin{definition}[Connected component]
	Fix a topological space $X$. A \textit{connected component} is a maximal connected subset of $X$.
\end{definition}
\begin{definition}[Irreducible component]
	Fix a topological space $X$. An \textit{irreducible component} is a maximal irreducible subset of $X$.
\end{definition}
Here are some quick facts.
\begin{lemma}
	Fix a topological space $X$.
	\begin{listalph}
		\item If a subset $V\subseteq X$ is irreducible, then $V$ is connected.
		\item If a subset $V\subseteq X$ is irreducible (respectively, connected), then so is $\overline V$.
		\item All points $x\in X$ are contained in an irreducible component. Also, all points of $x\in X$ are contained in a connected component.
	\end{listalph}
\end{lemma}
\begin{proof}
	We go one at a time.
	\begin{listalph}
		\item Suppose that $V\subseteq V_1\sqcup V_2$ where $V_1,V_2\subseteq X$ are closed subsets. Being irreducible forces $V\subseteq V_1$ or $V\subseteq V_2$, so connectivity of $V$ follows.
		\item We have two claims to show.
		\begin{itemize}
			\item Take $V$ irreducible so that we want to show $\overline V$ is irreducible. Suppose $\overline V\subseteq V_1\cup V_2$ where $V_1,V_2\subseteq X$ are closed. Then $V\subseteq V_1\cup V_2$, so $V\subseteq V_1$ or $V\subseteq V_2$, so properties of the closure promise $\overline V\subseteq V_1$ or $\overline V\subseteq V_2$.
			\item Take $V$ connected so that we want to show $\overline V$ is connected. Well, replace the $\cup$ in the previous proof with a $\sqcup$, and the proof goes through verbatim.
		\end{itemize}
		\item Observe that $\{x\}$ is irreducible: if $\{x\}\subseteq V_1\cup V_2$ where $V_1,V_2\subseteq X$ are closed, then $x\in V_1$ or $x\in V_2$, so $\{x\}\subseteq V_1$ or $\{x\}\subseteq V_2$.

		We now apply Zorn's lemma twice.
		\begin{itemize}
			\item Let $\mc I_x$ denote the set of irreducible subsets of $X$ containing $x$. We need to show that $\mc I_x$ has a maximal element, which will finish because any maximal element of $\mc I_x$ will be maximal among all irreducible subsets.
			
			Note $\{x\}\in\mc I_x$ means that $\mc I_x$ is nonempty. We now show that $\mc I_x$ satisfies the ascending chain condition: given a totally ordered set $\lambda$ and nonempty ascending chain $\{V_\alpha\}_{\alpha\in\lambda}\subseteq\mc I_x$, we claim that
			\[V\coloneqq\bigcup_{\alpha\in\lambda}V_\alpha\]
			and contains $x$. That $x\in V$ is clear because $x$ lives in any of the $V_\alpha$. To see irreducibility, suppose that $V\subseteq V_1\cup V_2$.
			
			If $V\subseteq V_1$, then we are done, so suppose that we can find $p\in V\setminus V_1$. This means that $p\in V_\beta\setminus V_1$ for some $\beta\in\lambda$, so $p\in V_\beta\setminus V_1$ for all $\alpha\ge\beta$. However, $V_\alpha\subseteq V_1\cup V_2$ still even though $V_1\not\subseteq V_1$, so we must instead have
			\[V_\alpha\subseteq V_2\]
			for all $\alpha\ge\beta$. It follows $V=\bigcup_{\alpha\ge\beta}V_\alpha\subseteq V_2$.

			\item Let $\mc C_x$ denote the set of connected subsets of $X$ containing $x$. We actually claim that
			\[V\coloneqq\bigcup_{C\in\mc C_x}C\]
			is connected. This will finish because $V$ is a connected component containing $x$: if $V\subseteq V'$ with $V'$ connected, then $x\in V'$, so $V'\in\mc C_x$, so $V'\subseteq V$.
			
			We now check $V$ is connected. Suppose $V\subseteq V_1\sqcup V_2$ for closed subsets $V_1,V_2\subseteq X$.

			The main point is that $x\in V_1$ or $x\in V_2$. Without loss of generality, take $x\in V_1$ so that $x\notin V_2$. Now, any $C\in\mc C_x$ has $C\subseteq V_1\sqcup V_2$, so $C\subseteq V_1$ or $C\subseteq V_2$. However, $x\in C\setminus V_2$, so we must have $C\subseteq V_1$ instead, meaning that actually $V\subseteq C_1$.
			\qedhere
		\end{itemize}
	\end{listalph}
\end{proof}
\begin{remark}
	It follows from the above proof that any connected subset $C$ of $x$ is contained in the connected component of $x$.
\end{remark}
Here is another nice result.
\begin{proposition}
	If $X$ is an irreducible topological space, then all nonempty open subsets $U\subseteq X$ have $U$ irreducible and $\overline U$ dense in $X$.
\end{proposition}
\begin{proof}
	We have two claims to show.
	\begin{itemize}
		\item We show $U$ is irreducible. Suppose $U\subseteq V_1\cup V_2$ for closed subsets $V_1,V_2\subseteq X$. It follows that
		\[X\subseteq\big((X\setminus U)\cup V_1\big)\cup V_2\]
		has covered $X$ by closed subsets. It follows that either $V_2=X$ (and hence covers $U$) or $(X\setminus U)\cup V_1=X$ (and so $V_1\supseteq U$).
		\item We show $\overline U=X$. Indeed, we can cover $X$ by closed sets as
		\[X=(X\setminus U)\cup\overline U,\]
		so either $X\setminus U=X$, which is impossible because $U$ is nonempty, or $\overline U=X$, which finishes.
		\qedhere
	\end{itemize}
\end{proof}
Even though irreducible components are a little weird in typical point-set topology, they are of interest in scheme theory.
\begin{lemma}
	Fix a ring $A$ and an ideal $I\subseteq A$.
	\begin{listalph}
		\item The subset $V(I)\subseteq\Spec A$ is irreducible if and only if $\rad I$ is prime.
		\item The irreducible components of $X$ are
		\[\{V(\mf p):\mf p\in\Spec A\text{ is a minimal prime}\}.\]
	\end{listalph}
\end{lemma}
\begin{proof}
	As usual, we go in sequence.
	\begin{listalph}
		\item The point is that $I\subseteq JK$ for ideals $J$ and $K$ should imply $I\subseteq J$ or $I\subseteq K$.
		\item Take the maximal ideals among those whose radical is prime and apply $V$.
		\qedhere
	\end{listalph}
\end{proof}
\begin{example}
	Projective space $\PP^n_R$ is irreducible.
\end{example}

\subsection{Generalizing Points}
We like our topological spaces to be Hausdorff, but we have seen that this need not happen in our schemes. So let's keep track of the good points we try to be Hausdorff
\begin{definition}[Closed point]
	Fix a topological space $X$. Then a point $x\in X$ is a \textit{closed point} if and only if $\overline{\{x\}}=\{x\}$.
\end{definition}
\begin{remark}
	In the variety setting, we are more interested in counting closed points, which correspond to the ``actual'' points on our variety.
\end{remark}
\begin{remark}
	In a quasicompact scheme, all closed subsets contain a closed point, so we have ``lots'' of closed points to work with. There are examples of schemes with no closed points.
\end{remark}
Having kept track of our closed points, we don't want to shame our ``unclosed points,'' so we give them a name as well.
\begin{definition}[Generic point]
	Fix a topological space $X$. Then a point $x\in X$ is a \textit{generic point} for an irreducible subset $V\subseteq X$ if and only if $V=\overline{\{x\}}$.
\end{definition}
\begin{example}
	Given a ring $A$, the point $\mf p\in\Spec A$ is the (unique!) generic point of $V(\mf p)$. For example, $(0)$ is the generic point for $\Spec R$ whenever $R$ is an integral domain.
\end{example}
The relationship between generic points will be important to keep track of.
\begin{defihelper}[Specialization, generalization] \nirindex{Specialization} \nirindex{Generalization}
	Fix a topological space $X$ and two points $x,y\in X$. We say that $x$ is a \textit{specialization} of $y$ (or equivalently, $y$ is a \textit{generalization} of $x$) if and only if $x\in\overline{\{y\}}$.
\end{defihelper}
\begin{example}
	Given a ring $A$, we see that $\mf q\in V(\mf p)=\overline{\{\mf p\}}$ if and only if $\mf q\supseteq\mf p$.
\end{example}
This provides a sort of ordering on our space. Closed points are the most ``specific,'' so let's keep track of the most generic.
\begin{definition}[Generic point]
	Fix a topological space $X$. A point $\xi\in X$ is a \textit{generic point} if and only if the only point specializing to $\xi$ is $\xi$.
\end{definition}
\begin{remark}
	Now let $(X,\OO_X)$ be a scheme. If $Z\subseteq X$ is an irreducible closed subset, then there is a unique generic point $\xi\in X$ such that $\overline{\{\xi\}}=Z$.
\end{remark}
By looking at our partially ordered set, we see As such, there is a bijection between generic points of $X$ and irreducible components of $X$.

\subsection{Noetherian Conditions}
Noetherian rings are good, so we will want to push this to our schemes as well.
\begin{definition}[Locally Noetherian]
	A scheme $(X,\OO_X)$ is \textit{locally Noetherian} if and only if $X$ has an open cover $\{U_\alpha\}_{\alpha\in\lambda}$ where each $(U_\alpha,\OO_X|_{U_\alpha})$ is isomorphic to $\Spec A_\alpha$ where $A_\alpha$ is a Noetherian ring.
\end{definition}
Noetherian is about making infinite things finite, so we want to add a quasicompact condition this.
\begin{definition}[Noetherian]
	A scheme $(X,\OO_X)$ is \textit{Noetherian} if and only if $X$ is quasicompact and locally Noetherian.
\end{definition}
\begin{remark}
	If $(X,\OO_X)$ is Noetherian as a scheme, then $X$ is Noetherian as a topological space. Namely, being Noetherian as a topological space means that the closed subsets have the descending chain condition.
\end{remark}
\begin{remark}
	There are some nice philosophical remarks in \cite[Section~3.6.21]{rising-sea} about when we might care about non-Noetherian things.
\end{remark}
As usual, here are some quick facts.
\begin{lemma}
	Fix a scheme $(X,\OO_X)$.
	\begin{listalph}
		\item If $(X,\OO_X)$ is locally Noetherian (respectively, Noetherian), then any open subset $U\subseteq X$ makes a locally Noetherian (respectively, Noetherian) scheme $(U,\OO_X|_U)$.
		\item If $(X,\OO_X)$ is locally Noetherian, then $\OO_{X,x}$ is a Noetherian ring for any $x\in X$.
	\end{listalph}
\end{lemma}
\begin{proof}
	We go in sequence.
	\begin{listalph}
		\item Reduce to the affine case, where we are roughly saying that localizations of Noetherian rings are Noetherian. The Noetherian part comes from tracking through quasicompactness.
		\item Again, check at an affine cover, and we reduce to saying that the localization of a Noetherian ring is Noetherian.
		\qedhere
	\end{listalph}
\end{proof}
One annoying thing about our locally Noetherian definition is that we are being forced to choose a very special affine open cover. However, we can remove this stress because it turns out everything will be Noetherian.
\begin{proposition}
	Fix a locally Noetherian scheme $(X,\OO_X)$. Then any affine open subset $U\subseteq X$ with $(U,\OO_X|_U)\cong\Spec A$ for a ring $A$ has $A$ a Noetherian ring.
\end{proposition}
\begin{proof}
	Fix an affine open cover $\mc U$ for $U$ making everything Noetherian. By intersecting $\Spec A$ with this open cover, we can pick up a finite subcover contained in particular distinguished open sets $D(f_i)$ for various $f_i\in A$. As such, we see that the $A_{f_i}$ are Noetherian because they are contained form our open cover.

	We now upgrade to $A$. Suppose $I\subseteq A$ is an ideal. Then each $IA_{f_i}$ is finitely generated, so find finitely many elements in $I$ which generate $IA_{f_i}$ as an $A_{f_i}$-ideal (by clearing denominators as necessary). All these elements combine to give a finitely generated ideal $J$, which we want to show equals $I$. Namely, we have $J\subseteq I$ of course, and
	\[JA_{f_i}=IA_{f_i}\]
	for each $i$. We will finish the proof next class. Intuitively, what's going on here is that $J$ and $I$ are sheaves with a morphism $J\subseteq I$ which is an isomorphism on our open cover, so they should be the same ideal.
\end{proof}
\begin{cor}
	Fix a ring $A$. Then $\Spec A$ is locally Noetherian if and only if $A$ is Noetherian.
\end{cor}

\end{document}