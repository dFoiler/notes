% !TEX root = ../notes.tex

\documentclass[../notes.tex]{subfiles}

\begin{document}

\section{September 9}

The fun continues.

\subsection{Modules}
We are not going to need modules for quite some time, but we will go ahead and define them now.
\begin{definition}[Module]
	Fix a scheme $(X,\OO_X)$. Then an \textit{$\OO_X$-module} is a sheaf $\mc F$ on $X$ with sheaf morphisms for addition $+\colon\mc F\times\mc F\to\mc F$ and a scalar multiplication $\cdot\colon\OO_X\times\mc F\to\mc F$.
\end{definition}
\begin{remark}
	In particular, for each $U\subseteq X$, we see $\mc F(U)$ is an $\OO_X(U)$-module.
\end{remark}
We will remark that one can define kernels, cokernels, and images as we did for sheaves values in general abelian categories, so the category of $\OO_X$-modules is again an abelian category. We refer to \cite[\S2.6.3]{rising-sea}.

\subsection{Gluing Schemes}
Here is the main idea.
\begin{definition}[Open subscheme]
	Fix a scheme $(X,\OO_X)$ and an open subset $U\subseteq X$. Then we define the scheme $(U,\OO_X|_U)$ to be an \textit{open subscheme}.
\end{definition}
Checking that we in fact have a scheme is annoying but not particularly hard; one merely has to restrict the affine open cover to $U$.
\begin{example}
	Given an affine scheme $(\Spec A,\OO_{\Spec A})$, we note that taking $U\coloneqq D(f)$ has
	\[(U,\OO_{\Spec A}|_U)=(\Spec A_f,\OO_{\Spec A_f}),\]
	where we are using \autoref{exe:distinguishedisaffine}.
\end{example}
Before we define gluing, we should pick up the following.
\begin{proposition} \label{prop:morphismtoaffine}
	Fix a scheme $(X,\OO_X)$. Given an affine scheme $(\Spec A,\OO_A)$, we can define
	\[\op{Mor}_{\mathrm{Sch}_X}((X,\OO_X),(\Spec A,\OO_A))\simeq\op{Hom}_{\mathrm{Ring}}(A,\OO_X(X)).\]
\end{proposition}
This is intended to generalize \autoref{prop:geoisoppalg}.
\begin{remark}
	Reversing the arguments in \autoref{prop:morphismtoaffine} is no longer true.
\end{remark}
\autoref{prop:morphismtoaffine} will follow from the following, where we let the open cover $\mc U$ below comes from the affine open cover.
\begin{proposition}
	Fix schemes $(X,\OO_X)$ and $(Y,\OO_Y)$. Let $\mc U$ be an open cover for $X$; then, given scheme morphisms $\varphi_U\colon U\to Y$ of such that
	\[\varphi_U|_{U\cap U'}=\varphi_{U'}|_{U\cap U'},\]
	there is a unique scheme morphism $\varphi\colon X\to Y$ such that $\varphi|_U=\varphi_U$.
\end{proposition}
\begin{proof}
	Apply force.
\end{proof}
So far we've glued together morphisms. It remains to glue together schemes. This will be similar to the way that we glue together sheaves.
\begin{proposition}
	Fix schemes $(X_i,\OO_i)$ for each $i\in I$, with an open subset $U_{ij}\subseteq X_i$ for each $i$, where $X_{ii}=X_i$; let $X_{ij}$ be the induced open subscheme. Further, pick up some isomorphisms $f_{ji}\colon X_{ij}\to X_{ji}$ satisfying the cocycle condition
	\[f_{ki}=f_{kj}\circ f_{ji},\]
	on $X_{ij}\cap X_{ik}$, where we implicitly assume that $f_{ji}(X_{ij}\cap X_{ik})\subseteq X_{ji}$. Then there is a unique scheme $X$ covered by open subschemes $U_i\subseteq X$, equipped with isomorphisms $\varphi_i\colon X_i\to U_i$ such that $\varphi_i|_{X_{ij}}=\varphi_j\circ f_{ji}$ and $\varphi_i(X_i)\cap\varphi_j(X_j)=\varphi_i(X_{ij})=\varphi_J(X_{ji})$.
\end{proposition}
\begin{proof}
	Glue the topological space first using the cocycle condition. Second, glue the sheaves together as described earlier. Lastly, $X$ is a scheme by using the affine open covers of the various $X_i$. The uniqueness up to unique isomorphism of $X$ follows by keeping track of all the data.
\end{proof}

\subsection{Projective Space by Gluing}
Fix a ring $R$. Let's define $\PP_R^n$ by gluing $n+1$ different affine sets $\AA^n_R$. Intuitively, we want to define projective space to have the topological space of homogeneous coordinates
\[[X_0:X_1:\ldots:X_n],\]
and we would like the $i$th affine piece of this space to be given by
\[\left(\frac{X_0}{X_i},\frac{X_1}{X_i},\ldots,\frac{X_n}{X_i}\right).\]
Notably, this has killed a coordinate with $X_i/X_i=1$.

As such, to glue properly, we define the $i$th affine piece to be
\[X_i\coloneqq\Spec R\left[x_{0/i},x_{1/i},\ldots,x_{(i-1)/i},x_{(i+1)/i},\ldots,x_{n/i}\right].\]
To glue this $X_i$ piece to the $X_j$ piece, we need to force $x_{j/i}$ to be nonzero (namely, to invert it), so we look at the open subscheme
\[X_{ij}\coloneqq\Spec R\left[x_{0/i},x_{1/i},\ldots,x_{(i-1)/i},x_{(i+1)/i},\ldots,x_{n/i},x_{j/i}^{-1}\right].\]
To glue these open subschemes directly, we remember that $x_{i/j}$ is supposed to mean $X_i/X_j$ as a quotient not always defined, so we define our isomorphism as
\[\arraycolsep=1.4pt\begin{array}{cccc}
	f_{ji}\colon& X_{ij} &\to& X_{ji} \\
	& x_{k/i} &\mapsto& x_{k/j}/x_{i/j}
\end{array}\]
from which we can pretty directly check the cocycle condition. (The $f_{ji}$ is an isomorphism because we can see its inverse is $f_{ij}$.) This gives us our definition.
\begin{definition}[Projective space]
	Fix a ring $R$. Then we define \textit{projective $n$-space over $R$}, denoted $\PP^n_R$ to be the scheme obtained from the above gluing data.
\end{definition}
\begin{remark}
	One can see that
	\[\OO_{\PP^n_R}(\PP^n_R)=R.\]
	Indeed, any global section $s\in\OO_{\PP^n_R}(\PP^n_R)$ must restrict to each affine open set $X_i$; however, looking at our gluing data $X_i$ and $X_j$ tells us that we cannot use a non-constant polynomial because having any positive degree (in, say $x_{i/j}$), would induce a denominator when pushing to $X_i$. Thus, $\PP^n_R$ is not an affine scheme unless $n=0$, for we would be asserting that $\PP^n_R$ is the affine scheme $\Spec R$.
\end{remark}

\subsection{Graded Rings}
Another way to look at projective schemes is to approach them from graded rings.
\begin{definition}[Graded rings]
	Fix a commutative monoid $(M,+)$. An \textit{$M$-graded ring} $S$ is a ring $S$ equipped with a decomposition of abelian groups
	\[S=\bigoplus_{d\in M}S_d\]
	such that $S_k\cdot S_k\subseteq S_{k+\ell}$ for any $k,\ell\in M$. By convention, a \textit{graded ring} will be an $\NN$-graded ring.
\end{definition}
\begin{remark}
	If $S$ is an $M$-graded ring, then $S_0\subseteq S$ is a subring. Here are our checks.
	\begin{itemize}
		\item Note $0\in S_0$ and that $S_0$ is closed under addition and subtraction because $S_0$ is an abelian group.
		\item We check $1\in S_0$. Well, suppose $1=\sum_{d\in M}^Ns_d$. Observe that $s_0s_d\in S_0S_d\subseteq S_d$ for each $d\in M$, so by comparing degrees, we are forced to have $s_0s_d=s_d$. But then
		\[s_0=s_0\cdot1=s_0\sum_{d\in M}s_d=\sum_{d\in M}s_d=1,\]
		so $1=s_0\in S_0$ follows.
		\item For $s,s'\in S_0$, we see $ss'\in S_0S_0\subseteq S_0$.
	\end{itemize}
\end{remark}
\begin{remark}
	Certainly, if $S$ is an $\NN$-graded ring, then $S$ is a $\ZZ$-graded ring by just setting $S_d=0$ for $d<0$.
\end{remark}
\begin{example}
	Take $S\coloneqq R[x_0,\ldots,x_n]$ graded by degree; namely, $S_k$ is the set of homogeneous polynomials of degree $k$ with $0$. Because $\deg(fg)=\deg f+\deg g$, we do indeed have $S_kS_\ell\subseteq S_{k+\ell}$.
\end{example}
\begin{example}
	If $S$ is a graded ring, and $f\in S_n$, then $S_f$ is a $\ZZ$-graded ring, where we are allowing negative degrees coming from $1/f$.
\end{example}
We will want our ideals to keep track of the grading, so we have the following definition.
\begin{definition}[Homogeneous element]
	Fix an $M$-graded ring $S$. Then an element $f\in S$ is \textit{homogeneous} if and only if $f\in S_d$ for some $d\in M$. If $s\in S_d\setminus\{0\}$ is nonzero and homogeneous, we set $\deg s\coloneqq S_d$.
\end{definition}
\begin{definition}[Homogeneous ideal]
	Fix an $M$-graded ring $S$. An ideal $I\subseteq S$ is \textit{homogeneous} if and only if $I$ is generated by homogeneous elements.
\end{definition}
\begin{remark} \label{rem:sumprodhomoideal}
	Directly from the definition, we can see that the (arbitrary) sum of homogeneous ideals is homogeneous by just taking the union of the homogeneous generators. Also, if $I=(r_\alpha)_{\alpha\in\lambda}$ and $J=(s_\beta)_{\beta\in\kappa}$ are homogeneous ideals, we see
	\[IJ=(r_\alpha s_\beta)_{(\alpha,\beta)\in\lambda\times\kappa},\]
	so $IJ$ is homogeneous as well; namely, $r_\alpha s_\beta\in S_{\deg r_\alpha}S_{\deg s_\beta}=S_{\deg r_\alpha+\deg s_\beta}$.
\end{remark}
This definition of a homogeneous ideal is easy to think about, but it is not yet clear why it ``respects the grading.''
\begin{lemma} \label{lem:betterhomogeneousideal}
	Fix an $M$-graded ring $S$ and ideal $I\subseteq S$. The following are equivalent.
	\begin{listalph}
		\item $I$ is generated by homogeneous elements.
		\item If $s=\sum_{d\in M}s_d$ lives in $I$, then $s_d\in I$ for each $d\in M$.
	\end{listalph}
\end{lemma}
\begin{proof}
	To see that (b) implies (a), note that $I$ is generated by
	\[I=\Bigg(\sum_{d\in M}s_d:\sum_{d\in M}s_d\in I\Bigg)\subseteq\Bigg(s_d:\sum_{d\in M}s_d\in I\Bigg).\]
	However, $\sum_{d\in M}s_d\in I$ implies $s_d\in I$ for each $d\in M$, so in fact
	\[\Bigg(s_d:\sum_{d\in M}s_d\in I\Bigg)\subseteq I,\]
	giving the needed equality. Thus, we have shown $I$ to be generated by homogeneous elements.

	We now show that (a) implies (b). Suppose $I$ is generated by the homogeneous elements $\{s_\alpha\}_{\alpha\in\lambda}$, where the degree of $s_\alpha$ is $d_\alpha$. Now, for any $s\in I$, write $s=\sum_{d\in M}s_d$ for $s_d\in S_d$. Of course, we can also write
	\[\sum_{d\in M}s_d=s=\sum_{\alpha\in\lambda}r_\alpha s_\alpha\]
	for some $r_\alpha\in S$. Writing $r_\alpha=\sum_{d\in M}r_{\alpha,d}$, we have
	\[\sum_{d\in M}s_d=\sum_{\alpha\in\lambda}\sum_{d\in M}r_{\alpha,d}s_\alpha.\]
	Comparing the $d$th degree on both sides, we see that
	\[s_d=\sum_{\alpha\in\lambda}r_{\alpha,d_\alpha-d}s_d,\]
	which is indeed an element of $I$. This finishes.
\end{proof}
\begin{corollary}
	Fix an $M$-graded ring $S$ and homogeneous ideal $I\subseteq S$. Then, setting $I_d\coloneqq I\cap S_d$, we see $S/I$ is an $M$-graded ring by $(S/I)_d\simeq S_d/I_d$ for each $d\in M$.
\end{corollary}
\begin{proof}
	Note we have the surjection
	\[\arraycolsep=1.4pt\begin{array}{ccccc}
		S &\simeq& \displaystyle\bigoplus_{d\in M}S_d &\onto& \displaystyle\bigoplus_{d\in M}S_d/I_d \\
		\displaystyle\sum_{d\in M}s_d &\mapsto& (s_d)_{d\in M} &\mapsto& (s_d+I_d)_{d\in M}
	\end{array}\]
	which is indeed a surjection because some $(s_d+I_d)_{d\in M}\in\bigoplus_{d\in M}S_d/I_d$ will just lift right back to $(s_d)_{d\in M}\in\bigoplus_{d\in M}S_d$, where $s_d=0$ if $s_d+I_d=I_d$ (which occurs all but finitely often). Additionally, an element $\sum_{d\in M}s_d\in S$ lives in the kernel of this map if and only if $s_d\in I_d$ for each $d\in M$, which by \autoref{lem:betterhomogeneousideal} is equivalent to $\sum_{d\in M}s_d\in I$. So we actually have the isomorphism
	\[\arraycolsep=1.4pt\begin{array}{ccccc}
		S/I &\simeq& \displaystyle\bigoplus_{d\in M}S_d/I_d \\
		\displaystyle\sum_{d\in M}s_d+I &\mapsto& (s_d+I_d)_{d\in M}
	\end{array}\]
	which becomes a grading upon noting that $k,\ell\in M$ with $s_k+I\in(S/I)_k\simeq S_k/I_k$ and $s_\ell+I\in(S/I)_\ell\simeq S_\ell/I_\ell$ will have $s_ks_\ell+I\in(S/I)_{k+\ell}\simeq S_{k+\ell}/I_{k+\ell}$.
\end{proof}
Here are some other quick facts about homogeneous ideals.
\begin{corollary} \label{cor:intersecthomoideals}
	Fix an $M$-graded ring $S$ and homogeneous ideals $\{I_\alpha\}_{\alpha\in\lambda}$. Then $\bigcap_{\alpha\in\lambda}I_\alpha$ is also a homogeneous ideal.
\end{corollary}
\begin{proof}
	Set $I\coloneqq\bigcap_{\alpha\in\lambda}I_\alpha$. We use \autoref{lem:betterhomogeneousideal}. Indeed, if $s=\sum_{d\in M}s_d$ lives in $I$, then $s\in I_\alpha$ for each $\alpha\in\lambda$, so each $d\in M$ has $s_d\in I_\alpha$ for each $\alpha\in\lambda$. Thus, $s_d\in I$ for each $d\in M$.
\end{proof}
\begin{lemma} \label{lem:fasthomoprime}
	Fix an $M$-graded ring $S$ and homogeneous ideal $I$. Then $I$ is prime if and only if, for any homogeneous elements $ab\in I$, we have $a,b\in I$.
\end{lemma}
\begin{proof}
	Certainly if $I$ is prime, then the conclusion holds. Conversely, we need to show that $I$ is prime. Well, suppose $a=\sum_{d\in M}a_d$ and $b=\sum_{d\in M}b_d$ have $ab\notin I$. Expanding,
	\[ab=\sum_{d\in M}\Bigg(\sum_{k+\ell=d}a_kb_\ell\Bigg)\notin I,\]
	so there is some term $a_kb_\ell\notin I$. Using the hypothesis, we see $a_k\notin I$ and $b_\ell\notin I$, so because $I$ is homogeneous, we conclude $a\notin I$ and $b\notin I$ by \autoref{lem:betterhomogeneousideal}.
\end{proof}
\begin{lemma} \label{lem:radhomo}
	Fix an $M$-graded ring $S$ and homogeneous ideal $I$. Then
	\[\rad I=\bigcap_{\substack{\mf p\supseteq I\\\mf p\text{ homogeneous}}}\mf p.\]
	In particular, $\rad I$ is homogeneous.
\end{lemma}
\begin{proof}
	We follow \cite{ms-238203}. The main claim is the first one; that $\rad I$ is homogeneous will follow by \autoref{cor:intersecthomoideals}. Now, for any prime ideal $\mf p$ containing $I$, let $\mf p'$ denote the ideal generated by the homogeneous elements of $\mf p$. We collect the following facts.
	\begin{itemize}
		\item By definition, $\mf p'$ is homogeneous, and $\mf p'\subseteq\mf p$.
		\item Note $\mf p'$ is prime by \autoref{lem:fasthomoprime}: given homogeneous elements $a,b$ with $ab\in\mf p'$, we see $ab\in\mf p$, so $a\in\mf p$ or $b\in\mf p$, so $a\in\mf p'$ or $b\in\mf p'$ by definition of $\mf p'$.
		\item If $s=\sum_{d\in M}s_d$ lives in $I$, then $s_d\in I\subseteq\mf p$ for each $d\in M$, so $s_d\in\mf p'$ for each $d\in M$, so $s\in\mf p'$. Thus, $I\subseteq\mf p'$.
	\end{itemize}
	From the above, we see
	\[\rad I=\bigcap_{\mf p\supseteq I}\mf p\supseteq\bigcap_{\mf p\supseteq I}\mf p'\supseteq\bigcap_{\substack{\mf p\supseteq I\\\mf p\text{ homogeneous}}}\mf p\supseteq\bigcap_{\mf p\supseteq I}\mf p,\]
	which is what we wanted.
\end{proof}
It turns out that some ideals do not carry geometric information.
\begin{definition}[Irrelevant ideal]
	Fix a graded ring $S$. Then the \textit{irrelevant ideal} $S_+$ is the ideal of $S$ generated by the homogeneous elements of positive degree.
\end{definition}
We will see why this ideal is called the irrelevant ideal shortly. For now, note that $S_+$ is a homogeneous ideal, and because
\[(S_+)_d\coloneqq S_+\cap S_d=\begin{cases}
	0 & d=0, \\
	S_d & d>0,
\end{cases}\]
we see that
\[S/S_+\simeq\bigoplus_{d\in\NN}\left(S_d/(S_+)_d\right)=S_0\oplus\bigoplus_{d\in\NN}0\simeq S_0.\]

\subsection{The Topological Space \texorpdfstring{$\mathrm{Proj}$}{\textrm{Proj}}}
Fix a graded ring $S$. We now construct $\op{Proj}S$. Intuitively, we want to have $\op{Proj}R[x_0,\ldots,x_n]=\PP^n_R$ and $\op{Proj}S[x_0,\ldots,x_n]/I=V(I)$ when $I$ is a homogeneous ideal. Rigorously, we are going to retell the affine story but add the word homogeneous everywhere.

Let's speak a little non-rigorously for a moment. In some sense, the point $p=[\lambda_0:\lambda_1:\ldots:\lambda_n]\in\PP^n_R$ should correspond to the ideal of $R[x_0,\ldots,x_n]$ which cuts out this line. Supposing $\lambda_0\ne0$ without loss of generality, we can see that the correct ideal is
\[\mf m_p=(\lambda_0x_1-\lambda_1x_0,\lambda_0x_2-\lambda_2x_0,\ldots,\lambda_0x_n-\lambda_nx_0).\]
In particular, $x_i\in\mf m_p$ if and only if $\lambda_i=0$, so we can encode the condition that $\lambda_i\ne0$ for some $i$ by requiring $\mf p\not\supseteq R[x_0,\ldots,x_n]_+$---namely, our irrelevant ideal $R[x_0,\ldots,x_n]_+$ carves out no points.\footnote{This is why $S_+$ is called the irrelevant ideal.} This gives our definition.
\begin{defihelper}[\textrm{Proj}] \nirindex{Proj@\textrm{Proj}}
	Given a graded ring $S$, we define
	\[\op{Proj}S\coloneqq\left\{\mf p\in\Spec S:\mf p\text{ homogeneous},\mf p\not\supseteq S_+\right\}.\]
\end{defihelper}
Having defined a version of our spectrum, we should give it a Zariski topology.
\begin{defihelper}[Zariski topology] \nirindex{Zariski topology!for \textrm{Proj}}
	Fix a graded ring $S$. Given a homogeneous ideal $\mf a\subseteq S$, define
	\[V_+(\mf a)\coloneqq\{\mf p\in\op{Proj}S:\mf p\supseteq\mf a\}.\]
	In other words, $V_+(\mf a)=V(\mf a)\cap\Proj S$.
\end{defihelper}
\begin{remark} \label{rem:vreversescontain}
	As before, we see homogeneous ideals $\mf a\subseteq\mf b$ give
	\[V_+(\mf b)=\{\mf p\in\Proj S:\mf p\supseteq\mf b\}\subseteq\{\mf p\in\Proj S:\mf p\supseteq\mf a\}=V_+(\mf a).\]
\end{remark}
\begin{remark}
	In light of \autoref{lem:radhomo}, we may say
	\[V_+(\mf a)=V(\mf a)\cap\Proj S=V(\rad\mf a)\cap\Proj S=V_+(\mf a).\]
\end{remark}
Here is the check that we have defined a topology.
\begin{lemma} \label{lem:zariskitopcheckproj}
	Fix a graded ring $S$. Then the subsets $\{V_+(\mf a)\}$ define a topology of closed sets on $\Proj S$. In particular, we have the following.
	\begin{listalph}
		\item $V_+(S_+)=\emp$ and $V_+((0))=\Proj S$.
		\item Arbitrary intersection: homogeneous ideals $\{\mf a_\alpha\}_{\alpha\in\lambda}$ give $\bigcap_{\alpha\in\lambda}V_+(\mf a_\alpha)=V_+\left(\sum_{\alpha\in\lambda}\mf a_\alpha\right)$.
		\item Finite union: homogeneous ideals $\mf a$ and $\mf b$ give $V_+(\mf a\mf b)=V_+(\mf a)\cup V_+(\mf b)$.
	\end{listalph}
\end{lemma}
\begin{proof}
	This largely follows straight from \autoref{lem:zariskitopcheck}.
	\begin{listalph}
		\item Note there is no $\mf p\in\Proj S$ with $\mf p\supseteq S_+$ by construction, so $V_+(S_+)=\emp$. Also, all ideals contain $(0)$, so $V_+((0))=\Proj S$. We also note that $S_+$ and $(0)$ are both homogeneous ideals.

		\item Using \autoref{lem:zariskitopcheck}, we see
		\[V_+\Bigg(\sum_{\alpha\in\lambda}\mf a_\alpha\Bigg)=V\Bigg(\sum_{\alpha\in\lambda}\mf a_\alpha\Bigg)\cap\Proj S=\Bigg(\bigcap_{\alpha\in\lambda}V(\mf a_\alpha)\Bigg)\cap\Proj S=\bigcap_{\alpha\in\lambda}\underbrace{(V(\mf a_\alpha)\cap\Proj S)}_{V_+(\mf a_\alpha)}.\]
		We close by noting that $\sum_{\alpha\in\lambda}\mf a_\alpha$ is a homogeneous ideal by \autoref{rem:sumprodhomoideal}.

		\item Again using \autoref{lem:zariskitopcheck}, we see
		\[V_+(\mf a\mf b)=V(\mf a\mf b)\cap\Proj S=(V(\mf a)\cup V(\mf b))\cap\Proj S=\underbrace{(V(\mf a)\cap\Proj S)}_{V_+(\mf a)}\cup\underbrace{(V(\mf b)\cap\Proj S)}_{V_+(\mf b)}.\]
		We close by noting that $\mf a\mf b$ is a homogeneous ideal by \autoref{rem:sumprodhomoideal}.
		\qedhere
	\end{listalph}
\end{proof}
As before, we will have a distinguished base, but we will be a little more careful.
\begin{defihelper}[Distinguished open sets] \nirindex{Distinguished open sets!for \textrm{Proj}}
	Fix a graded ring $S$. For a homogeneous element $f\in S_+$, we define
	\[D_+(f)\coloneqq\{\mf p\in\Proj S:f\notin\mf p\}.\]
	As before, we see $D_+(f)=D(f)\cap\Proj S$.
\end{defihelper}
Here is the analogue of \autoref{rem:distinguishedbase}.
\begin{lemma}
	Fix a graded ring $S$. The open sets $\{D_+(f)\}_{f\in S_+}$ form a base of the Zariski topology on $\Proj S$.
\end{lemma}
\begin{proof}
	Given any open subset $(\Proj S)\setminus V_+(\mf a)$ and point $\mf p\in(\Proj S)\setminus V_+(\mf a)$, we need to find $f\in S_+$ such that $D_+(f)$ contains $\mf p$ and $D_+(f)\subseteq(\Proj S)\setminus V_+(\mf a)$. In other words, we need $f\notin\mf p$ while $V_+(\mf a)\subseteq V_+((f))$. As such, it will suffice to find $f\notin\mf p$ with $f\in\mf a$ by \autoref{rem:vreversescontain}.

	Note that $\mf a$ is generated by homogeneous elements, so there certainly must exist some homogeneous element in $\mf a$ which is not in $\mf p$. If this element has positive degree, we are done immediately. Otherwise, suppose for contradiction the only homogeneous elements $f\in\mf a\setminus\mf p$ have degree zero. Then any homogeneous $s\in S_+$ of positive degree will have
	\[fs\in\mf a\]
	while $fs$ has positive degree, but then we forced ourselves into having $s\in\mf p$. Thus, $\mf p$ contains all homogeneous elements of $S_+$, so $\mf p\supseteq S_+$ because $S_+$ is homogeneous (!), which contradicts $\mf p\in\Proj S$.
\end{proof}

\subsection{Easy Nullstellensatz for \texorpdfstring{$\mathrm{Proj}$}{\textrm{Proj}}}
For fun, we take a moment to establish the analogue for \autoref{prop:easynullstellensatz}.
\begin{definition}
	Fix a graded ring $S$. Then, given a subset $Y\subseteq\Proj S$, we define
	\[I(Y)\coloneqq\bigcap_{\mf p\in Y}\mf p.\]
\end{definition}
\begin{remark} \label{rem:ireversecontain}
	Identically as in \autoref{lem:basiciprops}, we have $X\subseteq Y\subseteq\Proj S$ implies $I(Y)=\bigcap_{\mf p\in Y}\mf p\subseteq\bigcap_{\mf p\in X}\mf p=I(X)$.
\end{remark}
\begin{remark} \label{rem:iisradhomo}
	Because the intersection of homogeneous radical ideals is homogeneous (\autoref{cor:intersecthomoideals}) and radical, we see that $I(Y)$ is a homogeneous radical ideal for any $Y\subseteq\Proj S$.
\end{remark}
And here is our analogue.
\begin{proposition}
	Fix a graded ring $S$.
	\begin{listalph}
		\item Given a homogeneous ideal $\mf a\subseteq A$, we have $I(V_+(\mf a))=\rad\mf a$.
		\item Given a subset $X\subseteq\Proj S$, we have $V_+(I(X))=\overline X$.
		\item The functions $V_+$ and $I$ provide an inclusion-reversing bijection between radical ideals of $A$ and closed subsets of $\op{Spec}A$.
	\end{listalph}
\end{proposition}
\begin{proof}
	The proof are essentially analogous to \autoref{prop:easynullstellensatz}; we record them for completeness.
	\begin{listalph}
		\item Note
		\[I(V_+(\mf a))=\bigcap_{\mf p\in V_+(\mf a)}\mf a=\bigcap_{\substack{\mf p\supseteq\mf a\\\mf p\text{ homogeneous}}}\mf p=\rad\mf a,\]
		where the last equality follows from \autoref{lem:radhomo}.
		\item Using \autoref{lem:zariskitopcheckproj}, we see
		\[\overline X=\bigcap_{V_+(\mf a)\supseteq X}V_+(\mf a)=V_+\Bigg(\sum_{V_+(\mf a)\supseteq X}\mf a\Bigg).\]
		In particular, $X\subseteq V_+(\mf a)$ if and only if $\mf a\subseteq\mf p$ for all $\mf p\in X$, which means $\mf a\subseteq I(X)$. Thus, $\overline X=V\left(\sum_{\mf a\subseteq I(X)}\mf a\right)=V(I(X))$.
		\item As before, $V_+$ sends radical homogeneous ideals to closed subsets of $\Proj S$ (by definition of the topology), and $I$ sends closed subsets of $\Proj S$ to radical homogeneous ideals by \autoref{rem:iisradhomo}. These mappings are inclusion-reversing by \autoref{rem:vreversescontain} and \autoref{rem:ireversecontain}. Lastly, (a) and (b) show that these mappings compose to the identity.
		\qedhere
	\end{listalph}
\end{proof}

\subsection{The Structure Sheaf for \texorpdfstring{$\mathrm{Proj}$}{\textrm{Proj}}}
One can again check that this makes a topology. In fact, given $f\in S$, we can define
\[D_+(f)\coloneqq(\op{Proj}S)\setminus V((f))\]
and then check that this makes a basis for our topology, essentially for the same reason.
\begin{remark}
	One can check that the map
	\[\arraycolsep=1.4pt\begin{array}{ccc}
		D_+(f) &\simeq& \Spec (S_f)_0 \\
		\mf p &\mapsto& (\mf pS_f)\cap(S_f)_0
	\end{array}\]
	is a homeomorphism.
\end{remark}
As such, we give the open set $D_+(f)$ the structure sheaf $\OO_{\Spec((S_f)_0)}$. To glue these together, we choose the affine open subset
\[\Spec((S_f)_0)_{g^{\deg f}/f^{\deg g}}\subseteq\Spec(S_f)_0\]
and identify them with $\Spec(S_{fg})_0$.

\end{document}