% !TEX root = ../notes.tex

\documentclass[../notes.tex]{subfiles}

\begin{document}

\section{September 21}

Today we return to fiber products and discuss some applications.

\subsection{Representability}
We start with a few definitions.
\begin{definition}[Zariski sheaf]
	A functor $F\colon\mathrm{Sch}\opp\to\mathrm{Set}$ is a \textit{Zariski sheaf} if and only if $F$ is a sheaf when the scheme is viewed as merely a topological space. Namely, each scheme $T$ has
	\[0\to F(T)\to\prod_iF(T_i)\to\prod_{i,j}F(T_i\cap T_j)\]
	exact for any open cover $\{T_i\}$ of $T$.
\end{definition}
\begin{definition}[Open subfunctor]
	Fix functors $F,F'\colon\mathrm{Sch}\opp\to\mathrm{Set}$. Then $F'\subseteq F$ is an \textit{open subfunctor} if and only if each scheme $T$, every natural transformation $\psi\colon h_T\Rightarrow F$ yielding a pullback square
	% https://q.uiver.app/?q=WzAsNCxbMCwwLCJGX3tpLFxccHNpfSJdLFsxLDAsImhfVCJdLFswLDEsIkZfaSJdLFsxLDEsIkYiXSxbMCwxXSxbMiwzXSxbMCwyXSxbMSwzXSxbMCwzLCIiLDEseyJzdHlsZSI6eyJuYW1lIjoiY29ybmVyIn19XV0=&macro_url=https%3A%2F%2Fraw.githubusercontent.com%2FdFoiler%2Fnotes%2Fmaster%2Fnir.tex
	\[\begin{tikzcd}
		{F_{i,\psi}} & {h_T} \\
		{F_i} & F
		\arrow[from=1-1, to=1-2]
		\arrow[from=2-1, to=2-2]
		\arrow[from=1-1, to=2-1]
		\arrow[from=1-2, to=2-2]
		\arrow["\lrcorner"{anchor=center, pos=0.125}, draw=none, from=1-1, to=2-2]
	\end{tikzcd}\]
	already has each $F_{i,\psi}$ represented by a scheme $T_i$ with the natural transformation $F_{i,\psi}\into h_T$ given by an open embedding $T_i\into T$.
\end{definition}
Here is an abstract lemma.
\begin{lemma} \label{lem:abstract-repr}
	A functor $F\colon\mathrm{Sch}\opp\to\mathrm{Set}$ is representable if and only if the following conditions are satisfied.
	\begin{itemize}
		\item $F$ is a Zariski sheaf.
		\item Locally representable: there are representable subfunctors $F_i\subseteq F$ for each $i$, and $F_i\subseteq F$ is an open subfunctor such that $\{F_i\}$ covers $F$. Here, covering means that each field $K$ has $F(\Spec K)=\bigcup_iF_i(\Spec K)$.
	\end{itemize}
\end{lemma}
The point is that each $F_i$ is represented by some $X_i$, and we just want to glue these $X_i$ together. This is the idea of the proof.
\begin{remark}
	One can replace $\mathrm{Sch}\opp$ with $\mathrm{Ring}$ or $\mathrm{Sch}_S\opp$.
\end{remark}
\begin{remark}
	One can show that \autoref{lem:abstract-repr} implies that the fiber product exists. Namely, the fiber product forms a Zariski sheaf, which we can see from the part where we glued to make $W$ in the key case. Then the $F_i$ come from the purely affine case, which was comparatively easier. Lastly, the $F_i$ cover $F$ roughly speaking comes from the rest of the proof.
\end{remark}
We will not need \autoref{lem:abstract-repr} for the time being.

\subsection{Fibers}
As a first application, we discuss fibers. Given a scheme morphism $\varphi\colon Y\to S$, we might be interested in the fibers here to pull-back. Namely, pulling back to a ``subscheme'' $X$ of $S$, we can imagine the fibers of $Y$ over $X$ as the fiber product, as in the following diagram.
% https://q.uiver.app/?q=WzAsNCxbMCwwLCJYXFx0aW1lc19TWSJdLFsxLDAsIlkiXSxbMSwxLCJTIl0sWzAsMSwiWCJdLFswLDFdLFswLDNdLFszLDJdLFsxLDJdLFswLDIsIiIsMSx7InN0eWxlIjp7Im5hbWUiOiJjb3JuZXIifX1dXQ==&macro_url=https%3A%2F%2Fraw.githubusercontent.com%2FdFoiler%2Fnotes%2Fmaster%2Fnir.tex
\[\begin{tikzcd}
	{X\times_SY} & Y \\
	X & S
	\arrow[from=1-1, to=1-2]
	\arrow[from=1-1, to=2-1]
	\arrow[from=2-1, to=2-2]
	\arrow[from=1-2, to=2-2]
	\arrow["\lrcorner"{anchor=center, pos=0.125}, draw=none, from=1-1, to=2-2]
\end{tikzcd}\]
For example, in the case of $Y=\Spec k[x,y,z]/\left(y^2-x(x-1)(x-s)\right)$ and $S=\Spec k[s]$, we see that we have the obvious map $Y\to S$ sending $x,y\mapsto0$.

Now, if we want to understand the fiber at a given point $s_0\in S$ with $s_0\in k$ for concreteness, the corresponding scheme is a $\Spec k$ over $\Spec S$ induced by the map $k[s]\to k$ by $s\mapsto s_0$. Then we can track our fibers in this affine case as given by
% https://q.uiver.app/?q=WzAsNCxbMCwxLCJcXFNwZWMgayJdLFsxLDEsIlxcU3BlYyBrW3NdIl0sWzEsMCwiWSJdLFswLDAsIllcXG90aW1lc197a1tzXX1cXFNwZWMgayJdLFszLDBdLFswLDFdLFszLDJdLFsyLDFdXQ==&macro_url=https%3A%2F%2Fraw.githubusercontent.com%2FdFoiler%2Fnotes%2Fmaster%2Fnir.tex
\[\begin{tikzcd}
	{Y\times_{\Spec k[s]}\Spec k} & Y \\
	{\Spec k} & {\Spec k[s]}
	\arrow[from=1-1, to=2-1]
	\arrow[from=2-1, to=2-2]
	\arrow[from=1-1, to=1-2]
	\arrow[from=1-2, to=2-2]
\end{tikzcd}\]
where we can compute directly from commutative algebra that
\[Y\times_{\Spec k[s]}\Spec k\simeq\Spec\frac{k[x,y]}{\left(y^2-x(x-1)(x-s_0)\right)}\]
here.

As such, we are convinced that the following is a good definition of a fiber.
\begin{definition}[Fiber]
	Fix a scheme $Y$ over a scheme $S$. Given a point $s_0\in S$, the fiber product $Y\times_S\{s_0\}$ is the \textit{fiber} of $Y\to S$ over $s_0$.
\end{definition}
\begin{remark}
	If $X\into S$ is a closed embedding, then $Y\times_SX\into Y$ is a closed embedding as well. In particular, if $s_0\in S$ is a closed point, then our fiber is in fact a closed embedding. More generally, if $S$ is irreducible with generic point $\eta$, we call $Y\times_S\{\eta\}$ the generic fiber.
\end{remark}
Notably, we can check purely topologically that the fiber defined by the fiber product is the correct fiber at a point. Here is the affine case.
\begin{lemma} \label{lem:affine-fibers}
	Fix a ring homomorphism $\pi\colon A\to B$ and a prime $\mf p\in\Spec A$, and let $S\coloneqq\pi^{-1}(A\setminus\mf p)$. Then
	\[\Spec k(\mf p)\times_{\Spec A}\Spec B\simeq\Spec S^{-1}B/\mf p(S^{-1}B)\]
	and is homeomorphic to $(\Spec \pi)^{-1}(\{\mf p\})$.
\end{lemma}
\begin{proof}
	Note $k(\mf p)=(A/\mf p)_\mf p$. Now, we see from the construction of the fiber product that
	\[\Spec k(\mf p)\times_{\Spec A}\Spec B\simeq\Spec B\otimes_A(A/\mf p)_\mf p.\]
	We now note the isomorphisms
	\begin{align*}
		B \otimes_A (A/\mf p)_\mf p &\simeq B\otimes_A A_\mf p \otimes_A A/\mf p \\
		&\simeq S^{-1}B \otimes_A A/\mf p \\
		&\simeq S^{-1}B/\mf p(S^{-1}B),
	\end{align*}
	which finishes the first claim. For the homeomorphism, we know that the localization map $B\to S^{-1}B$ induces a homeomorphism
	\[\Spec S^{-1}B\simeq\{\mf q\in\Spec B:\mf q\cap\pi(S)=\emp\}=\{\mf q\in\Spec B:\mf q\cap\pi(S)=\emp\}=\{\mf q\in\Spec A:\pi^{-1}(\mf q)\subseteq\mf p\}.\]
	Next up, the projection map $S^{-1}B\onto\Spec S^{-1}B/\mf p(S^{-1}B)$ induces a homeomorphism
	\begin{align*}
		\Spec S^{-1}B/\mf p(S^{-1}B) &\cong \{\mf q\in\Spec S^{-1}B:\mf q\supseteq\mf p(S^{-1}B)\} \\
		&\cong \{\mf q\in\Spec S^{-1}B:\mf q\supseteq\pi\mf p\} \\
		&\cong \{\mf q\in\Spec B:\pi^{-1}(\mf q)\subseteq\mf p\text{and}\mf q\supseteq\mf p\} \\
		&= \{\mf q\in\Spec B:\pi^{-1}\mf q=\mf p\},
	\end{align*}
	which is the needed $(\Spec\pi)^{-1}(\{\mf p\})$. Note that we have used our previous homeomorphism in the second $\cong$.
\end{proof}
\begin{remark}
	Tracking through the above proof shows that the canonical map $B\to S^{-1}B/\mf p(S^{-1}B)$ is providing the needed homeomorphism.
\end{remark}
We now proceed with the proof.
\begin{lemma} \label{lem:topological-fiber}
	Fix a scheme morphism $\varphi\colon X\to Y$ and a point $y\in Y$. Fixing $X_y\coloneqq X\times_Y\{y\}$, the canonical projection $X_y\to X$ is a homeomorphism onto $\varphi^{-1}(\{y\})$.
\end{lemma}
\begin{proof}
	We optimize the proof that the fiber product exists, essentially skipping over all the checks because they are covered in any of the books (as well as in class). To begin, place $y\in Y$ inside an affine open subset $U\subseteq Y$, which induces the commutative diagram
	% https://q.uiver.app/?q=WzAsNixbMiwxLCJZIl0sWzIsMCwiWCJdLFsxLDEsIlUiXSxbMSwwLCJcXHBpXnstMX1VIl0sWzAsMSwiXFx7eVxcfSJdLFswLDAsIlhcXHRpbWVzX1lcXHt5XFx9Il0sWzEsMCwiXFxwaSJdLFsyLDAsIlxcam1hdGgiLDAseyJzdHlsZSI6eyJ0YWlsIjp7Im5hbWUiOiJob29rIiwic2lkZSI6InRvcCJ9fX1dLFszLDEsIlxcam1hdGgnIl0sWzQsMiwiXFxpb3RhIiwwLHsic3R5bGUiOnsidGFpbCI6eyJuYW1lIjoiaG9vayIsInNpZGUiOiJ0b3AifX19XSxbNSw0LCJcXHBpJyIsMl0sWzMsMl0sWzUsMywiXFxpb3RhJyIsMCx7InN0eWxlIjp7ImJvZHkiOnsibmFtZSI6ImRhc2hlZCJ9fX1dXQ==&macro_url=https%3A%2F%2Fraw.githubusercontent.com%2FdFoiler%2Fnotes%2Fmaster%2Fnir.tex
	\[\begin{tikzcd}
		{X_y} & {\varphi^{-1}U} & X \\
		{\{y\}} & U & Y
		\arrow["\varphi", from=1-3, to=2-3]
		\arrow["\jmath", hook, from=2-2, to=2-3]
		\arrow["{\jmath'}", from=1-2, to=1-3]
		\arrow["\iota", hook, from=2-1, to=2-2]
		\arrow["{\pi}"', from=1-1, to=2-1]
		\arrow[from=1-2, to=2-2]
		\arrow["{\iota'}", dashed, from=1-1, to=1-2]
	\end{tikzcd}\]
	where in particular $\iota'\colon X\to\varphi^{-1}U$ is induced because the right square is a pullback square (it's made by open embeddings), so we can use the maps $X_y\to\{y\}\to U$ and $X_y\to X$ to induce $\iota$ by pullback.

	Quickly, we note that the left square is a pullback square by \autoref{lem:big-to-small-square}. As such, we may {replace $Y$ with $U=\Spec A$ and $X$ with $\varphi^{-1}U$ and $X_y$ with $U_y=\varphi^{-1}U\times_Y\{y\}$} without actually changing the fiber product. Applying the suitable isomorphisms everywhere, we may directly assume that $Y=\Spec A$ for a ring $A$.

	Now, give $X$ an affine open cover $\widetilde X$. Letting $\iota'\colon X_y\to X$ and $\iota\colon\{y\}\to X$ be our embeddings, we note that the diagram
	% https://q.uiver.app/?q=WzAsNixbMSwyLCJZIl0sWzAsMiwiXFx7eVxcfSJdLFsxLDEsIlgiXSxbMCwxLCJYX3kiXSxbMCwwLCJcXHBpXnstMX1VX1xcYWxwaGEiXSxbMSwwLCJVX1xcYWxwaGEiXSxbMiwwLCJcXHZhcnBoaSJdLFs1LDIsIiIsMCx7InN0eWxlIjp7InRhaWwiOnsibmFtZSI6Imhvb2siLCJzaWRlIjoidG9wIn19fV0sWzMsMSwiXFxwaSIsMl0sWzEsMCwiXFxpb3RhIl0sWzMsMiwiXFxpb3RhJyJdLFs0LDUsIlxcaW90YSciXSxbNCwzLCIiLDAseyJzdHlsZSI6eyJ0YWlsIjp7Im5hbWUiOiJob29rIiwic2lkZSI6InRvcCJ9fX1dXQ==&macro_url=https%3A%2F%2Fraw.githubusercontent.com%2FdFoiler%2Fnotes%2Fmaster%2Fnir.tex
	\[\begin{tikzcd}
		{(\iota')^{-1}U_\alpha} & {U_\alpha} \\
		{X_y} & X \\
		{\{y\}} & Y
		\arrow["\varphi", from=2-2, to=3-2]
		\arrow[hook, from=1-2, to=2-2]
		\arrow["\pi"', from=2-1, to=3-1]
		\arrow["\iota", from=3-1, to=3-2]
		\arrow["{\iota'}", from=2-1, to=2-2]
		\arrow["{\iota'}", from=1-1, to=1-2]
		\arrow[hook, from=1-1, to=2-1]
	\end{tikzcd}\]
	has both squares as pullback squares (the top is an open embedding square, and the bottom is by definition of $X_y$), so we see that $(\iota')^{-1}(U_\alpha)=U_{\alpha,y}$. However, from the affine case, we know that the map $\iota'\colon(\iota')^{-1}U_\alpha\to U_\alpha$ provides a homeomorphism of $(\iota')^{-1}(U_\alpha)$ onto $U_\alpha\cap\varphi^{-1}(\{y\})$. Thus, we see that the full map $\iota'\colon X_y\to X$ provides a homeomorphism of $X_y$ onto
	\[\bigcup_{\alpha\in\lambda}\iota'\left((\iota')^{-1}U_\alpha\right)=\bigcup_{\alpha\in\lambda}(\varphi^{-1}(\{y\})\cap U_\alpha)=\varphi^{-1}(\{y\}).\]
	This finishes the proof.
\end{proof}

\subsection{Base Extension}
We again begin with a special case. Take $S=\Spec K$ to be our base and $X=\Spec K'$ where $K'/K$ is a field extension; the embedding $K\into K'$ induces a map $X\to S$.

Now, if we have a scheme $Y$ over $S$, we might want to pull $Y$ back to a scheme over $X$, where we are applying some base-change operation. To do this, we unsurprisingly want the fiber product, as in the following diagram.
% https://q.uiver.app/?q=WzAsNCxbMCwwLCJYXFx0aW1lc19TWSJdLFsxLDAsIlkiXSxbMSwxLCJTIl0sWzAsMSwiWCJdLFswLDFdLFswLDNdLFszLDJdLFsxLDJdLFswLDIsIiIsMSx7InN0eWxlIjp7Im5hbWUiOiJjb3JuZXIifX1dXQ==&macro_url=https%3A%2F%2Fraw.githubusercontent.com%2FdFoiler%2Fnotes%2Fmaster%2Fnir.tex
\[\begin{tikzcd}
	{X\times_SY} & Y \\
	X & S
	\arrow[from=1-1, to=1-2]
	\arrow[from=1-1, to=2-1]
	\arrow[from=2-1, to=2-2]
	\arrow[from=1-2, to=2-2]
	\arrow["\lrcorner"{anchor=center, pos=0.125}, draw=none, from=1-1, to=2-2]
\end{tikzcd}\]
As an example, if we have $Y=\Spec K[x,y]/\left(y^2-x^3+x\right)$, we can compute that
\[X\times_SY=\Spec\frac{K'[x,y]}{\left(y^2-x^3+x\right)},\]
which agrees with our intuition of what base-change should do.
\begin{example}
	As another quick example, we can compute $\PP^n_K\times_{\Spec K}\Spec K'=\PP^n_{K'}$. For example, we could cleanly define projective $n$-space over a scheme $S$
	\[\PP^n_S=\PP^n_{\Spec\ZZ}\times_{\Spec\ZZ}S\]
\end{example}
This notation is a bit cumbersome, so we will abbreviate it.
\begin{notation}
	Fix a morphism $S'\to S$. If $X$ is a scheme over $S$, we might denote the base-change of $X$ to a scheme over $S'$ as $X_{S'}=X\times_SS'$.
\end{notation}
\begin{remark}
	Given a field extension $K'/K$ and a scheme $X$ over $\Spec K$, we can check that $X_{K'}(K')=X(K')$. This is purely formal: a morphism $\Spec K'\to X$ induces a unique morphism $\Spec K'\to X_{K'}$ making the diagram
	% https://q.uiver.app/?q=WzAsNSxbMSwyLCJcXFNwZWMgSyciXSxbMiwyLCJcXFNwZWMgSyJdLFsyLDEsIlgiXSxbMSwxLCJYX3tLJ30iXSxbMCwwLCJcXFNwZWMgSyciXSxbMywyXSxbMiwxXSxbMywwXSxbMCwxXSxbNCwyLCIiLDIseyJjdXJ2ZSI6LTJ9XSxbNCwwLCIiLDIseyJjdXJ2ZSI6MiwibGV2ZWwiOjIsInN0eWxlIjp7ImhlYWQiOnsibmFtZSI6Im5vbmUifX19XSxbNCwzLCIiLDIseyJzdHlsZSI6eyJib2R5Ijp7Im5hbWUiOiJkYXNoZWQifX19XV0=&macro_url=https%3A%2F%2Fraw.githubusercontent.com%2FdFoiler%2Fnotes%2Fmaster%2Fnir.tex
	\[\begin{tikzcd}
		{\Spec K'} \\
		& {X_{K'}} & X \\
		& {\Spec K'} & {\Spec K}
		\arrow[from=2-2, to=2-3]
		\arrow[from=2-3, to=3-3]
		\arrow[from=2-2, to=3-2]
		\arrow[from=3-2, to=3-3]
		\arrow[curve={height=-12pt}, from=1-1, to=2-3]
		\arrow[curve={height=12pt}, Rightarrow, no head, from=1-1, to=3-2]
		\arrow[dashed, from=1-1, to=2-2]
	\end{tikzcd}\]
	commute. Notably, we are using the identity map $\Spec K'\to\Spec K'$ because $X(K')$ consists of the $K$-morphisms $\Spec K'\to X$.
\end{remark}
In applications, the base-change to the algebraic closure will be especially important because certain aspects become clear only once passing to an algebraic closure. This gives the following definition.
\begin{defihelper}[Goemetrically irreducible, reduced, connected] \nirindex{Geometrically irreducible} \nirindex{Geometrically reduced} \nirindex{Geometrically connected}
	A scheme $X$ over a field $K$ is \textit{geometrically irreducible/reduced/connected} if and only if $X_{\overline K}$ is irreducible/reduced/connected.
\end{defihelper}
\begin{example}
	Fix the scheme $X=\Spec\QQ(\sqrt2)$ over the scheme $\Spec\QQ$. Even though $X$ is irreducible, it is not geometrically irreducible because $X_{\overline\QQ}$ becomes two copies of $\Spec\overline\QQ$. Indeed,
	\[X_{\overline\QQ}=\Spec(\QQ(\sqrt2)\otimes_\QQ\overline\QQ)=\Spec(\overline\QQ\times\overline\QQ)=\Spec\overline\QQ\sqcup\Spec\overline\QQ.\]
	Namely, $\QQ(\sqrt2)\otimes_\QQ\overline\QQ\cong\overline\QQ\times\overline\QQ$ by decomposing some element $(a+b\sqrt2)\otimes\alpha$ as $1\otimes a\alpha+\sqrt2\otimes b\alpha$.
\end{example}
Intuitively, being irreducible but not geometrically irreducible means that passing to the algebraic closure gives rise to Galois conjugate pieces. This allows us to separate geometric information from Galois information.

\subsection{The Relative Frobenius}
For concreteness, fix a scheme $S$ over $\FF_p$. Notably, a scheme $X$ over $S$ has a map $X\to S$, so because the sheaf $p\OO_S$ vanishes, we have that $p\OO_X$ will also vanish. The point is that the $p$th-power map $f\mapsto f^p$ is going to induce a scheme morphism $F_X\colon X\to X$.

To see morphism topologically, let's see an example. When $X=\Spec A$ is affine, we see that $A$ is an $\mathbb F_p$-algebra, and the Frobenius mapping $\varphi\colon a\mapsto a^p$ we can see directly sends $\mf p$ by $\varphi^{-1}$ to itself. Thus, the Frobenius is indeed nothing at all topologically.
\begin{remark}
	The above morphism is called the absolute Frobenius.
\end{remark}

Now, we have another Frobenius morphism $F_S\colon S\to S$, and we can see that the diagram
% https://q.uiver.app/?q=WzAsNCxbMCwwLCJYIl0sWzEsMCwiWCJdLFswLDEsIlMiXSxbMSwxLCJTIl0sWzAsMSwiRl9YIl0sWzIsMywiRl9TIl0sWzAsMl0sWzEsM11d&macro_url=https%3A%2F%2Fraw.githubusercontent.com%2FdFoiler%2Fnotes%2Fmaster%2Fnir.tex
\[\begin{tikzcd}
	X & X \\
	S & S
	\arrow["{F_X}", from=1-1, to=1-2]
	\arrow["{F_S}", from=2-1, to=2-2]
	\arrow[from=1-1, to=2-1]
	\arrow[from=1-2, to=2-2]
\end{tikzcd}\]
commutes. However, the fiber product now promises us a morphism $X\to X\times_SS$, where the two $S$s are different copies with a Frobenius morphism going between them.
\begin{remark}
	The above morphism is called the relative Frobenius.
\end{remark}
\begin{example}
	With $S=\Spec\FF_p[s]$ and $X=\Spec\FF_p[s,x]$, the relative Frobenius keeps $s$ fixed and sends $x\mapsto x^p$.
\end{example}
% L3.2.4

\end{document}