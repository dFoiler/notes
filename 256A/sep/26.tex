% !TEX root = ../notes.tex

\documentclass[../notes.tex]{subfiles}

\begin{document}

\section{September 26}

The second problem set has been graded. I had a few typos.

\subsection{Finiteness Conditions}
We continue our discussion of finiteness properties. Here is our strongest.
\begin{definition}[Finite]
	A scheme morphism $\pi\colon X\to Y$ is \textit{finite} if and only if an affine open subset $\Spec B\subseteq Y$ makes $\pi^{-1}(U)=\Spec A$ also affine in such a way that the induced ring morphism $B\to A$ on global sections makes $A$ a finitely generated $B$-module.
\end{definition}
Here is the weakest.
\begin{definition}[Locally of finite type]
	A scheme morphism $\pi\colon X\to Y$ is \textit{locally of finite type} if and only if an affine open subset $\Spec B\subseteq Y$ with affine open subset $\Spec A\subseteq\pi^{-1}(U)$ inducing the ring morphism $B\to A$ on global sections makes $A$ a finitely generated $B$-algebra.
\end{definition}
Note that there is a key difference between being finitely generated as a module and an algebra.
\begin{example}
	The inclusion of fields $K\into L$ has $L$ as a finitely generated $K$-module, so the induced map $\Spec L\to\Spec K$ is a finite morphism.
\end{example}
\begin{example}
	The ring $k[x,y]$ is finitely generated as $k$-algebra but not as a $k$-module. Thus, the induced map $\Spec k[x,y]\to\Spec k$ is (locally) of finite type but is not finite.
\end{example}
Lastly, here is our medium-strength morphism.
\begin{definition}[Finite type]
	A scheme morphism $\pi\colon X\to Y$ is of \textit{finite type} if and only if $\pi$ is locally of finite type and quasicompact.
\end{definition}
\begin{remark}
	Of course, being finite implies being of finite type, and being of finite type implies being locally of finite type.
\end{remark}
There are the usual equivalent conditions.
\begin{lemma}
	Being finite is affine local on the target: a scheme morphism $\pi\colon X\to Y$ is finite if and only if there is an affine open cover $\{U_\alpha\}_{\alpha\in\lambda}$ with $U_\alpha=\Spec B_\alpha$ and $\pi^{-1}(U_\alpha)=\Spec A_i$ in such a way that $A_\alpha$ is finitely generated as a $B_\alpha$-module.
\end{lemma}
\begin{proof}
	Omitted.
\end{proof}
\begin{lemma}
	Being locally of finite type is affine local on the target: a scheme morphism $\pi\colon X\to Y$ is locally of finite type if and only if there is an affine open cover $\{U_\alpha\}_{\alpha\in\lambda}$ with $U_\alpha=\Spec B_i$ such that any affine open subset $\Spec B_i\subseteq\pi^{-1}(U_\alpha)$ in such a way that $A_\alpha$ is finitely generated as a $B_\alpha$-module.
\end{lemma}
\begin{proof}
	Omitted.
\end{proof}
There is a similar result for being locally of finite type by just requiring that the affine open cover of $\pi^{-1}(U_\alpha)$ be finite.
\begin{remark}
	Closed embeddings are of finite type. Namely, on affine schemes, our closed embeddings all look like $\Spec A/I\to\Spec A$, and $A/I$ is finitely generated as an $A$-module.
\end{remark}
\begin{remark}
	Open embeddings are locally of finite type. In particular, the sufficiently local case looks like $\Spec A_f\to\Spec A$, and $A_f$ is finitely generated as an $A$-algebra.
\end{remark}
Here's an example of our finiteness conditions doing their job.
\begin{lemma}
	Fix a finite scheme morphism $(\pi,\pi^\sharp)\colon (X,\OO_X)\to (Y,\OO_Y)$. For any $y\in Y$, the set $\pi^{-1}(\{y\})$ is finite.
\end{lemma}
\begin{proof}
	There are arguments avoiding the fiber product, but let's just go ahead and use it. Let $X_y\coloneqq X\times_Y\{y\}$ so that the topological space of $X_y$ is $\pi^{-1}(\{y\})$. Here is our diagram.
	% https://q.uiver.app/?q=WzAsNCxbMSwwLCJYIl0sWzEsMSwiWSJdLFswLDEsIlxce3lcXH0iXSxbMCwwLCJYX3kiXSxbMywwXSxbMCwxLCJcXHRleHR7ZmluaXRlfSJdLFsyLDEsIiIsMix7InN0eWxlIjp7InRhaWwiOnsibmFtZSI6Imhvb2siLCJzaWRlIjoidG9wIn19fV0sWzMsMl1d&macro_url=https%3A%2F%2Fraw.githubusercontent.com%2FdFoiler%2Fnotes%2Fmaster%2Fnir.tex
	\[\begin{tikzcd}
		{X_y} & X \\
		{\{y\}} & Y
		\arrow[from=1-1, to=1-2]
		\arrow["{\text{finite}}", from=1-2, to=2-2]
		\arrow[hook, from=2-1, to=2-2]
		\arrow[from=1-1, to=2-1]
	\end{tikzcd}\]
	Namely, the canonical projection $X_y\to\{y\}$ is finite. Now, we have reduced to the case where $Y=\Spec k(y)$ is a field. It follows because the map $X_y\to\{y\}$ is finite that $X_y=\Spec A$ where $A$ is finitely generated as a $k(y)$-module. Thus, $\dim A=\dim k(y)$, so $A$ is Artinian, so $\Spec A$ is finite.
\end{proof}

\subsection{Integral Morphisms}
The following proof will motivate a definition.
\begin{lemma}
	Fix a finite scheme morphism $(\pi,\pi^\sharp)\colon (X,\OO_X)\to (Y,\OO_Y)$. Then $\pi$ is a closed map of topological spaces.
\end{lemma}
\begin{proof}
	One may work in neighborhoods to reduce to the affine case. TO finish, we take our scheme morphism $\pi\colon\Spec A\to\Spec B$, and we note that the induced map on global sections $\pi^\sharp\colon B\to A$ is an integral ring map (namely, $A$ is integral over $\pi^\sharp(B)$). From here, we claim that
	\[\pi(Z)\stackrel?=V\left((\pi^\sharp)^{-1}(I)\right).\]
	For this, we pick up the following result from commutative algebra.
	\begin{lemma}[Lying over]
		Fix an integral extension of rings $\iota\colon R\to S$. Then each prime $\mf p\in\Spec R$ has some prime $\mf q\in\Spec S$ such that $\mf q\cap R=\mf p$.
	\end{lemma}
	\begin{proof}
		See \cite[\S4.4]{eisenbud-comm-alg}.
	\end{proof}
	The above lemma essentially does the computation for us, upon realizing that $B/(\pi^\sharp)^{-1}(I)\to A/I$ is an integral extension of rings.
\end{proof}
\begin{remark}
	Later in life we will continue to care about the ring of global sections being an integral ring map because this is a fairly common thing to see. For example, $\QQ\into\overline\QQ$ is an integral ring map.
\end{remark}
Observe that the above proof only needed $\pi^\sharp$ to be an integral ring homomorphism, so we make this our definition.
\begin{definition}
	A scheme morphism $\pi\colon X\to Y$ is \textit{integral} if and only if $\pi$ is affine and all affine open $\Spec B\subseteq Y$ has $\Spec A=\pi^{-1}(U)$ such that the induced map $\pi^\sharp\colon B\to A$ an integral ring map.
\end{definition}
We continue working with this definition.
\begin{prop}
	Fix an integral scheme morphism $\pi\colon X\to Y$. Then each closed $Z\subseteq X$ has $\pi(Z)$ closed and $\dim Z=\dim\pi(Z)$.
\end{prop}
\begin{proof}
	That $\pi(Z)$ is closed follows from our previous proof. As usual, we reduce to the affine case, where we have an integral extension of rings $\pi^\sharp\colon R\to S$ (where $\pi(Z)=\Spec R$ and $S=\Spec Z$), and we would like to show that $\dim R=\dim S$.

	For this, we combine two commutative algebra results.
	\begin{lemma}
		Fix an integral extension of rings $R\subseteq S$. If $\mf q_1\subsetneq\mf q_2\subseteq S$ are prime ideals, then their intersections of $R$ are still distinct.
	\end{lemma}
	The above result tells us that $\dim Z\le\dim\pi(Z)$. For the other inequality, we need to be able to go up.
	\begin{lemma}[Going up]
		Fix an integral extension of rings $R\subseteq S$. If there are primes
		\[\mf p_1\subsetneq\mf p_2\subsetneq\cdots\subsetneq\mf p_n\subseteq R,\]
		with a partial lift of $\mf q_i$ such that $\mf q_i\cap R=\mf p_i$ for $1\le i\le m$, then we can extend the chain all the way up to $n$.
	\end{lemma}
	The above result gives us $\pi(Z)\le\dim Z$.
\end{proof}
The point is that integral morphisms are similar to finite ones.

\subsection{Quasifinite Morphisms}
Here is yet another finiteness condition.
\begin{definition}[Quasifinite]
	A scheme morphism $\pi\colon X\to Y$ is \textit{quasifinite} if and only if $\pi$ is of finite type and each $y\in Y$ has $\pi^{-1}(\{y\})$ a finite set.
\end{definition}
\begin{remark}
	We have seen that being finite implies being quasifinite.
\end{remark}
Let's get some practice with the definition.
\begin{lemma}
	If $\pi\colon X\to Y$ is a finite morphism of schemes and $U\subseteq X$ is quasicompact, then the restriction $\pi|_U\colon U\to Y$ is quasifinite.
\end{lemma}
\begin{remark}
	Being quasifinite is stable under composition, base change, and is affine local on the target.
\end{remark}
\begin{remark}
	A morphism is finite if and only if it is quasifinite and integral. We have seen the forward direction.
\end{remark}

\subsection{Chevalley's Theorem}
We close lecture by stating Chevalley's theorem.
\begin{theorem}[Chevalley]
	Fix a scheme morphism $\pi\colon X\to Y$ of finite type. If $C\subseteq X$ is constructible, then $\pi(C)$ is also constructible.
\end{theorem}
Here is the appropriate definition.
\begin{definition}[Constructible]
	Fix a Noetherian topological space $X$. Then a subset $C\subseteq X$ is constructible if and only if it is the union of subsets of the form $U\subseteq V$ where $U\subseteq X$ is open and $V\subseteq X$ is closed.
\end{definition}
There is a different definition on the homework; in particular, the collection of constructible sets
\begin{remark}
	It is somewhat important for constructible sets to be living in a Noetherian topological space. The definition must change otherwise.
\end{remark}

\end{document}