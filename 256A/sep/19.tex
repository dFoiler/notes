% !TEX root = ../notes.tex

\documentclass[../notes.tex]{subfiles}

\begin{document}

\section{September 16}

Bump, bump, bump.

\subsection{Fiber Products}
Here is today's main result.
\begin{theorem} \label{thm:fibexist}
	Fix two $S$-schemes $X$ and $Y$. Then the fiber product $X\times_SY$ exists.
\end{theorem}
\begin{remark}
	Even if $X$ and $Y$ are Noetherian, it does not necessarily follow that $X\times_SY$ is Noetherian. For example, taking $\Spec\overline\QQ$ and $\Spec\overline\QQ$ are both Noetherian, but the fiber product turns out to be
	\[\Spec\overline\QQ\otimes_\QQ\overline\QQ.\]
	In particular, this is zero-dimensional but has infinitely many points and is therefore not Noetherian.
\end{remark}
The fiber product is a purely categorical construct, which is the limit of the diagram as follows.
% https://q.uiver.app/?q=WzAsMyxbMCwxLCJYIl0sWzEsMCwiWSJdLFsxLDEsIlMiXSxbMCwyXSxbMSwyXV0=&macro_url=https%3A%2F%2Fraw.githubusercontent.com%2FdFoiler%2Fnotes%2Fmaster%2Fnir.tex
\[\begin{tikzcd}
	& Y \\
	X & S
	\arrow[from=2-1, to=2-2]
	\arrow[from=1-2, to=2-2]
\end{tikzcd}\]
In other words, there are canonical projection maps $X\times_SY\to X$ and $X\times_SY\to Y$ with a suitable universal property. As usual, the universal property means that this is unique up to unique isomorphism.
\begin{remark}
	Even without knowing that $X\times_SY$ exists, we may note that we have a natural isomorphism
	\[\op{Mor}_S(Z,X\times_SY)\cong\op{Mor}_S(Z,X)\times_{\op{Mor}_S(Z,S)}\op{Mor}_S(Z,Y)\]
	coming straight from the universal property. In other words, the $Z$-points of an $S$-fiber product is just going to be the product of the two $S$-points. In this way, we can view the below proof as asking for a particular functor (on the right here) to be representable, which will be enlightening after some thought.
	% EH ch 6, V 10.1
\end{remark}
Schemes are made of affine schemes, so we will begin with affine schemes, with the hope of patching these together later.
\begin{lemma}
	Fix affine schemes $X$, $Y$, and $S$, with $X=\Spec A$ and $Y=\Spec B$ and $S=\Spec R$. Then
	\[X\times_SY=\Spec A\otimes_RB.\]
\end{lemma}
\begin{proof}
	Notably, the canonical maps $X\to S$ and $Y\to S$ makes $A$ and $B$ into $R$-algebras. To see this, we simply compute, for some scheme $S$,
	\begin{align*}
		\op{Mor}_S(Z,\Spec A\otimes_RS) &\simeq \op{Hom}_R(A\otimes_RS,\OO_Z(Z)) \\
		&\simeq \op{Hom}_R(A,\OO_Z(Z)) \times_{\op{Hom}_R(R,\OO_Z(Z))} \op{Hom}_R(B,\OO_Z(Z)),
	\end{align*}
	which unravels into the correct thing. This finishes by the Yoneda lemma. One can write out all of this as some diagram-chase, but it is equivalent to the above computation.
\end{proof}
\begin{remark}
	One can view the above proof as basically preserving the fact that $A\otimes_RB$ is the fiber product of $A$ and $B$ as $R$-algebras.
\end{remark}
We now begin our gluing. Here is the key case to glue.
\begin{lemma} \label{lem:keyfibercase}
	Fix affine schemes $X$ and $S$ and a scheme $Y$, where $X$ and $Y$ are $S$-schemes. Then the fiber product exists.
\end{lemma}
We hope that a gluing will be able to give $X\times_SY$ for $S$ affine and $X$ and $Y$ arbitrary, from which the general case will follow by gluing $S$.

To prove \autoref{lem:keyfibercase}, we will want to glue to get up to $Y$, so we have the following smaller case.
\begin{lemma} \label{lem:helpkeycase}
	Fix affine schemes $X$ and $S$ and a scheme $Y$, where $X$ and $Y$ are $S$-schemes. Given an open embedding $Y\into Y'$, if $X\times_SY'$ exists, then $X\times_SY$ also exists as
	\[X\times_SY=\pi^{-1}_{Y'}(Y),\]
	where $\pi_{Y'}\colon X\times_SY'\to Y'$ is the canonical map.
\end{lemma}
\begin{proof}[Sketch]
	We use the universal property a bunch of times. Then
	% https://q.uiver.app/?q=WzAsNixbMCwyLCJYIl0sWzEsMSwiWSciXSxbMSwyLCJTIl0sWzAsMSwiWFxcdGltZXNfU1knIl0sWzEsMCwiWSJdLFswLDAsIlxccGleey0xfV97WSd9KFkpIl0sWzAsMl0sWzEsMl0sWzMsMCwiXFxwaV9YIl0sWzMsMSwiXFxwaV9ZIiwyXSxbNSw0LCJcXHBpX1kiXSxbNSwzXSxbNCwxXV0=&macro_url=https%3A%2F%2Fraw.githubusercontent.com%2FdFoiler%2Fnotes%2Fmaster%2Fnir.tex
	\[\begin{tikzcd}
		{\pi^{-1}_{Y'}(Y)} & Y \\
		{X\times_SY'} & {Y'} \\
		X & S
		\arrow[from=3-1, to=3-2]
		\arrow[from=2-2, to=3-2]
		\arrow["{\pi_X}", from=2-1, to=3-1]
		\arrow["{\pi_Y}"', from=2-1, to=2-2]
		\arrow["{\pi_Y}", from=1-1, to=1-2]
		\arrow[from=1-1, to=2-1]
		\arrow[from=1-2, to=2-2]
	\end{tikzcd}\]
	we can show has the top and bottom squares are both pullback squares, so the full rectangle is a pullback square, which finishes.
\end{proof}
We can now prove \autoref{lem:keyfibercase}.
\begin{proof}[Proof of \autoref{lem:keyfibercase}]
	Give $Y$ the affine open cover $\{Y_\alpha\}_{\alpha\in\lambda}$, writing $Y_{\alpha\beta}\coloneqq Y_\alpha\cap Y_\beta$ as a subset of $Y_\alpha$ to prepare for our gluing. Now, we see that \autoref{lem:helpkeycase} grants us schemes
	\[X\times_SY_{\alpha\beta}\subseteq X\times_SY_\alpha\qquad\text{and}\qquad X\times_SY_{\beta\alpha}\subseteq X\times_SY_\beta,\]
	which must be identified because $Y_{\alpha\beta}=Y_{\beta\alpha}$. These identified schemes are going to satisfy the cocycle condition because these fiber products are unique up to unique isomorphism.

	To see the cocycle condition more concretely, we can just say out loud that we have the scheme
	\[X\times_SY_{\alpha\beta\gamma}=(X\times_SY_{\alpha\beta})\cap(X\times_SY_{\alpha\gamma})\subseteq X\times_SY_\alpha.\]
	On the other side, we have a similar computation for $Y_{jik}$ and $Y_{kij}$, so chaining these diagrams together will identify the needed isomorphisms by the uniqueness of our various isomorphisms.

	At this point, we have been glued together a (unique) scheme $W$ with ``compatible'' open embeddings $X\times_SY_\alpha\into W$. It remains to verify that $W$ is our fiber product $X\times_SY$. Our map $\pi_X\colon W\to X$ is induced by projecting down to $X$ in all the glued components; our map $\pi_Y\colon W\to Y$ is made by gluing the various maps $\pi_\alpha\colon W\to Y_\alpha$ (which appear from the projections $X\times_SY_\alpha\to Y_\alpha$).
	
	We now verify the universal property. Fix morphisms $\varphi_X\colon Z\to X$ and $\varphi_Y\colon Y\to Z$ making the diagram
	% https://q.uiver.app/?q=WzAsNSxbMSwyLCJYIl0sWzIsMSwiWSJdLFsyLDIsIlMiXSxbMCwwLCJaIl0sWzEsMSwiVyJdLFswLDJdLFsxLDJdLFszLDAsIlxcdmFycGhpX1giLDIseyJjdXJ2ZSI6Mn1dLFszLDEsIlxcdmFycGhpX1kiLDAseyJjdXJ2ZSI6LTJ9XSxbNCwxLCJcXHBpX1kiLDJdLFs0LDAsIlxccGlfWCJdLFszLDQsIiIsMSx7InN0eWxlIjp7ImJvZHkiOnsibmFtZSI6ImRhc2hlZCJ9fX1dXQ==&macro_url=https%3A%2F%2Fraw.githubusercontent.com%2FdFoiler%2Fnotes%2Fmaster%2Fnir.tex
	\[\begin{tikzcd}
		Z \\
		& W & Y \\
		& X & S
		\arrow[from=3-2, to=3-3]
		\arrow[from=2-3, to=3-3]
		\arrow["{\varphi_X}"', curve={height=12pt}, from=1-1, to=3-2]
		\arrow["{\varphi_Y}", curve={height=-12pt}, from=1-1, to=2-3]
		\arrow["{\pi_Y}"', from=2-2, to=2-3]
		\arrow["{\pi_X}", from=2-2, to=3-2]
		\arrow[dashed, from=1-1, to=2-2]
	\end{tikzcd}\]
	commute, and we want to induce a unique arrow where the dashed arrow is. The big diagram commuting gives an internal commuting diagram
	% https://q.uiver.app/?q=WzAsNSxbMSwyLCJYIl0sWzIsMSwiWV9cXGFscGhhIl0sWzIsMiwiUyJdLFswLDAsIlpfXFxhbHBoYSJdLFsxLDEsIlhcXHRpbWVzX1NZX1xcYWxwaGEiXSxbMCwyXSxbMSwyXSxbMywwLCJcXHZhcnBoaV9YIiwyLHsiY3VydmUiOjJ9XSxbMywxLCJcXHZhcnBoaV9ZIiwwLHsiY3VydmUiOi0yfV0sWzQsMSwiXFxwaV9ZIiwyXSxbNCwwLCJcXHBpX1giXSxbMyw0LCJcXHZhcnBoaV9cXGFscGhhIiwxLHsic3R5bGUiOnsiYm9keSI6eyJuYW1lIjoiZGFzaGVkIn19fV1d&macro_url=https%3A%2F%2Fraw.githubusercontent.com%2FdFoiler%2Fnotes%2Fmaster%2Fnir.tex
	\[\begin{tikzcd}
		{Z_\alpha} \\
		& {X\times_SY_\alpha} & {Y_\alpha} \\
		& X & S
		\arrow[from=3-2, to=3-3]
		\arrow[from=2-3, to=3-3]
		\arrow["{\varphi_X}"', curve={height=12pt}, from=1-1, to=3-2]
		\arrow["{\varphi_Y}", curve={height=-12pt}, from=1-1, to=2-3]
		\arrow["{\pi_Y}"', from=2-2, to=2-3]
		\arrow["{\pi_X}", from=2-2, to=3-2]
		\arrow["{\varphi_\alpha}"{description}, dashed, from=1-1, to=2-2]
	\end{tikzcd}\]
	with $Z_\alpha\coloneqq\varphi_Y^{-1}(Y_\alpha)$, where we have already used the universal property to construct the unique morphism $\varphi_\alpha$ making the diagram commute. This gives maps
	\[Z_\alpha\to X\times_SY_\alpha\subseteq W,\]
	and we just show that these various maps will be compatible to give us a unique map $Z\to W$. One can track through all the various universal properties to show that we have in fact induced a unique map making the original diagram commute.

	We now show that we can in fact glue the morphisms $Z_\alpha\to W$. Well, set $Z_{\alpha\beta}\coloneqq\varphi_Y^{-1}(Y_{\alpha\beta})$ to be $Z_\alpha\cap Z_\beta$, and we can see that the diagram
	% https://q.uiver.app/?q=WzAsNCxbMCwwLCJaX3tcXGFscGhhXFxiZXRhfSJdLFsxLDEsIlhcXHRpbWVzX1NZX3tcXGFscGhhXFxiZXRhfSJdLFsyLDEsIlhcXHRpbWVzX1NZX1xcYWxwaGEiXSxbMSwyLCJYXFx0aW1lc19TWV9cXGJldGEiXSxbMSwzXSxbMSwyXSxbMCwxXSxbMCwzLCJcXHZhcnBoaV9cXGJldGEiLDIseyJjdXJ2ZSI6Mn1dLFswLDIsIlxcdmFycGhpX1xcYWxwaGEiLDAseyJjdXJ2ZSI6LTJ9XV0=&macro_url=https%3A%2F%2Fraw.githubusercontent.com%2FdFoiler%2Fnotes%2Fmaster%2Fnir.tex
	\[\begin{tikzcd}
		{Z_{\alpha\beta}} \\
		& {X\times_SY_{\alpha\beta}} & {X\times_SY_\alpha} \\
		& {X\times_SY_\beta}
		\arrow[from=2-2, to=3-2]
		\arrow[from=2-2, to=2-3]
		\arrow[from=1-1, to=2-2]
		\arrow["{\varphi_\beta}"', curve={height=12pt}, from=1-1, to=3-2]
		\arrow["{\varphi_\alpha}", curve={height=-12pt}, from=1-1, to=2-3]
	\end{tikzcd}\]
	commutes, which finishes.
\end{proof}
Note that the same proof will show that we can build the fiber product for $X\times_SY$ when $S$ is affine and $X$ and $Y$ are arbitrary; namely, nowhere did we use in the proof that $X$ was affine (only that the fiber products $X\times_SY_\alpha$ exist), so we can just apply the same gluing process.

Lastly, here is the general case.
\begin{proof}[Proof of \autoref{thm:fibexist}]
	Give $S$ an affine open cover $\{S_\alpha\}_{\alpha\in\lambda}$, with our natural maps $f_X\colon X\to S$ and $f_Y\colon Y\to S$. Then we set $X_\alpha\coloneqq f_X^{-1}(S_\alpha)$ and similar for $Y_\alpha$. By previous work, we have the fiber products $X_\alpha\times_{S_\alpha}Y_\alpha$, so we can just glue these together as usual.
\end{proof}
A little more rigorously, the following result aids the above gluing.
\begin{lemma}
	Fix schemes $X$ and $Y$ over a scheme $S$. If $S\subseteq S'$ is an affine open embedding of schemes such that $X\times_{S'}Y$ exists, we have
	\[f_X^{-1}(S)\times_Sf_Y^{-1}(S)=(f_X\circ\pi_X)^{-1}(S')=(f_Y\circ\pi_Y)^{-1}(S').\]
\end{lemma}
The intuition is that the above lemma should be pulling back the required fiber product along the various legs of the following diagram.
% https://q.uiver.app/?q=WzAsNCxbMCwwLCJYXFx0aW1lc197Uyd9WSJdLFsxLDAsIlkiXSxbMCwxLCJYIl0sWzEsMSwiUyciXSxbMCwxXSxbMSwzXSxbMCwyXSxbMiwzXV0=&macro_url=https%3A%2F%2Fraw.githubusercontent.com%2FdFoiler%2Fnotes%2Fmaster%2Fnir.tex
\[\begin{tikzcd}
	{X\times_{S'}Y} & Y \\
	X & {S'}
	\arrow[from=1-1, to=1-2]
	\arrow[from=1-2, to=2-2]
	\arrow[from=1-1, to=2-1]
	\arrow[from=2-1, to=2-2]
\end{tikzcd}\]
This will give the intersection data in the gluing for \autoref{thm:fibexist}, which will finish the gluing.
\begin{remark}
	We will provide a more categorical viewpoint of this construction next class. This categorical viewpoint will be helpful for when we want to define the grassmanian.
	% V 10.1.8
\end{remark}

\end{document}