% !TEX root = ../notes.tex

\documentclass[../notes.tex]{subfiles}

\begin{document}

\section{September 19}

Bump, bump, bump.
\begin{warn}
	Today we will begin more aggressively notating a scheme $(X,\OO_X)$ by its topological space $X$, where the structure sheaf will always be $\OO_X$. Similarly, a morphism $\varphi\colon X\to Y$ refers to its continuous map, and the map of structure sheaves is $\varphi^\sharp\colon\OO_Y\to\varphi_*\OO_X$.
\end{warn}

\subsection{Fiber Products}
Here is our definition.
\begin{definition}[Fiber product]
	Fix a category $\mc C$. Then, given morphisms $\psi_X\colon X\to S$ and $\psi_Y\colon Y\to S$, the \textit{fiber product} $X\times_SY$ is the limit of the following diagram.
	% https://q.uiver.app/?q=WzAsMyxbMSwxLCJTIl0sWzEsMCwiWCJdLFswLDEsIlkiXSxbMSwwLCJcXHBzaV9YIl0sWzIsMCwiXFxwc2lfWSIsMl1d&macro_url=https%3A%2F%2Fraw.githubusercontent.com%2FdFoiler%2Fnotes%2Fmaster%2Fnir.tex
	\[\begin{tikzcd}
		& X \\
		Y & S
		\arrow["{\psi_X}", from=1-2, to=2-2]
		\arrow["{\psi_Y}"', from=2-1, to=2-2]
	\end{tikzcd}\]
\end{definition}
\begin{example} \label{ex:set-fp}
	In the category $\mathrm{Set}$, one can show that
	\[X\times_SY=\{(x,y)\in X\times Y:\psi_X(x)=\psi_Y(y)\},\]
	where the projections are the canonical ones.
\end{example}
\begin{notation}
	Given a fiber product $X\times_SY$ in a category, we call the resulting square
	% https://q.uiver.app/?q=WzAsNCxbMSwwLCJYIl0sWzAsMSwiWSJdLFsxLDEsIlMiXSxbMCwwLCJYXFx0aW1lc19TWSJdLFszLDFdLFsxLDJdLFszLDBdLFswLDJdLFszLDIsIiIsMSx7InN0eWxlIjp7Im5hbWUiOiJjb3JuZXIifX1dXQ==&macro_url=https%3A%2F%2Fraw.githubusercontent.com%2FdFoiler%2Fnotes%2Fmaster%2Fnir.tex
	\[\begin{tikzcd}
		{X\times_SY} & X \\
		Y & S
		\arrow[from=1-1, to=2-1]
		\arrow[from=2-1, to=2-2]
		\arrow[from=1-1, to=1-2]
		\arrow[from=1-2, to=2-2]
		\arrow["\lrcorner"{anchor=center, pos=0.125}, draw=none, from=1-1, to=2-2]
	\end{tikzcd}\]
	a \textit{pullback square}.
\end{notation}
And here is our main result.
\begin{theorem} \label{thm:fibexist}
	Fix two $S$-schemes $X$ and $Y$. Then the fiber product $X\times_SY$ exists.
\end{theorem}
\begin{remark}
	Even if $X$ and $Y$ are Noetherian, it does not follow that $X\times_SY$ is Noetherian. For example, taking $\Spec\overline\QQ$ and $\Spec\overline\QQ$ are both Noetherian, but the fiber product turns out to be
	\[\Spec\overline\QQ\otimes_\QQ\overline\QQ.\]
	Namely, $\Spec\overline\QQ\otimes_\QQ\overline\QQ$ is zero-dimensional but has infinitely many points and is therefore not Noetherian.
\end{remark}
% The fiber product is a purely categorical construct, which is the limit of the diagram as follows.
% % https://q.uiver.app/?q=WzAsMyxbMCwxLCJYIl0sWzEsMCwiWSJdLFsxLDEsIlMiXSxbMCwyXSxbMSwyXV0=&macro_url=https%3A%2F%2Fraw.githubusercontent.com%2FdFoiler%2Fnotes%2Fmaster%2Fnir.tex
% \[\begin{tikzcd}
% 	& Y \\
% 	X & S
% 	\arrow[from=2-1, to=2-2]
% 	\arrow[from=1-2, to=2-2]
% \end{tikzcd}\]
% In other words, there are canonical projection maps $X\times_SY\to X$ and $X\times_SY\to Y$ with a suitable universal property. As usual, the universal property means that this is unique up to unique isomorphism.
We will provide one proof of \autoref{thm:fibexist} today and another next class involving the representability of functors. To get a taste for this functor business, we note that we can concretely describe the functor of points for the fiber product.
\begin{lemma} \label{lem:repr-fp}
	Fix a category $\mc C$ and two maps $\psi_X\colon X\to S$ and $\psi_Y\colon Y\to S$ such that the fiber product $X\times_SY$ exists. Then, for any $Z\in\mc C$ with a map $\psi_Z\colon Z\to S$,
	\[\op{Mor}_S(Z,X\times_SY)\cong\op{Mor}_S(Z,X)\times_{\op{Mor}_S(Z,S)}\op{Mor}_S(Z,Y).\]
	Here, $\op{Mor}_S$ refers to morphisms in the category over $S$.
\end{lemma}
\begin{proof}
	This follows straight from the universal property. Namely, let $\pi_X\colon X\times_SY\to X$ and $\pi_Y\colon X\times_SY\to Y$ be the canonical projections. Using \autoref{ex:set-fp}, we are really just asking for an isomorphism
	\[\op{Mor}_S(Z,X\times_SY)\cong\{(\varphi_X,\varphi_Y)\in\op{Mor}_S(Z,X)\times\op{Mor}_S(Z,Y):\psi_X\circ\varphi_X=\psi_Y\circ\varphi_Y\}\]
	matching up with the projections. Namely, observe that we have a map from the left to the right simply taking a map $\varphi\mapsto(\pi_X\circ\varphi,\pi_Y\circ\varphi)$, which works because
	\[\psi_X\circ\pi_X\circ\varphi=\psi_Y\circ\pi_Y\circ\varphi\]
	should both be $\psi_Z$. In the reverse direction, we can take a pair of maps $\varphi_X\colon Z\to X$ and $\varphi_Y\colon Z\to Y$ such that $\psi_X\circ\varphi_X=\psi_Y\circ\varphi_Y$ and recover a unique map $\varphi\colon Z\to X\times_SY$ (by the universal property) such that $\pi_X\circ\varphi=\varphi_X$ and $\pi_Y\circ\varphi_Y$.
\end{proof}
The difficulty will be in actually finding a scheme which can represent the functor
\[\op{Mor}_S(-,X)\times_{\op{Mor}_S(-,S)}\op{Mor}_S(-,Y).\]
Namely, even though we are sure that this object is unique up to unique isomorphism (by \autoref{thm:yo-lem}), it is not actually clear that it exists at all!
% \begin{remark}
% 	Even without knowing that $X\times_SY$ exists, we may note that we have a natural isomorphism
% 	\[\op{Mor}_S(Z,X\times_SY)\cong\op{Mor}_S(Z,X)\times_{\op{Mor}_S(Z,S)}\op{Mor}_S(Z,Y)\]
% 	coming straight from the universal property. In other words, the $Z$-points of an $S$-fiber product is just going to be the product of the two $S$-points. In this way, we can view the below proof as asking for a particular functor (on the right here) to be representable, which will be enlightening after some thought.
% 	% EH ch 6, V 10.1
% \end{remark}

\subsection{They Exist}
We will now start marching toward a proof of \autoref{thm:fibexist}. Schemes are made of affine schemes, so we will begin with affine schemes, with the hope of patching these together later.
\begin{lemma} \label{lem:affine-fp}
	Fix affine schemes $X$, $Y$, and $S$, with $X=\Spec A$ and $Y=\Spec B$ and $S=\Spec R$. Then we may set
	\[X\times_SY=\Spec A\otimes_RB,\]
	where the canonical projections $X\times_SY\to X$ and $X\times_SY\to Z$ are induced by the canonical inclusions $\iota_A\colon A\to A\otimes_RB$ and $\iota_B\colon B\to A\otimes_RB$.
\end{lemma}
\begin{proof}
	Let $f_A\colon R\to A$ and $f_B\colon R\to B$ be the maps associated to the maps $\psi_A\colon X\to S$ and $\psi_B\colon Y\to S$. Now, for some scheme $Z$ over $S$ by $\psi_Z$, and we see that
	\[\op{Mor}_S(Z,\Spec A\otimes_RB)\simeq\op{Hom}_R(A\otimes_RB,\OO_Z(Z))\]
	by \autoref{thm:biggeoisalgopp}. Now, it is a fact of commutative algebra\todo{} that $R$-algebras maps $\varphi\colon A\otimes_RS\to\OO_Z(Z)$ are in bijection with pairs of $R$-algebra maps $\varphi_A\colon A\to\OO_Z(Z)$ and $\varphi_B\colon A\to\OO_Z(Z)$ such that $\varphi_A=\varphi\circ\iota_A$ and $\varphi_B=\varphi\circ\iota_B$. In other words, we have a natural isomorphism
	\[\op{Mor}_S(Z,\Spec A\otimes_RB)\simeq\op{Hom}_R(A,\OO_Z(Z))\times_{\op{Hom}_R(R,\OO_Z(Z))}\op{Hom}_R(B,\OO_Z(Z)).\]
	Applying the adjunction again, we see
	\[\op{Mor}_S(Z,\Spec A\otimes_RB)\simeq\op{Mor}_S(Z,X)\times_{\op{Mor}_S(Z,S)}\op{Mor}_S(Z,Y),\]
	which finishes by \autoref{lem:repr-fp}.
	% Notably, the canonical maps $X\to S$ and $Y\to S$ makes $A$ and $B$ into $R$-algebras. To see this, we simply compute, for some scheme $S$,
	% \begin{align*}
	% 	\op{Mor}_S(Z,\Spec A\otimes_RS) &\simeq \op{Hom}_R(A\otimes_RS,\OO_Z(Z)) \\
	% 	&\simeq \op{Hom}_R(A,\OO_Z(Z)) \times_{\op{Hom}_R(R,\OO_Z(Z))} \op{Hom}_R(B,\OO_Z(Z)),
	% \end{align*}
	% which unravels into the correct thing. This finishes by the Yoneda lemma. One can write out all of this as some diagram-chase, but it is equivalent to the above computation.
\end{proof}
\begin{remark}
	One can view the above proof as basically preserving the fact that $A\otimes_RB$ is the fiber coproduct of $A$ and $B$ as $R$-algebras.
\end{remark}
We now begin our gluing. Here is the key case to glue.
\begin{lemma} \label{lem:keyfibercase}
	Fix schemes $X$ and $Y$ over $S$. Given an open cover $\mc U$ of $Y$, if the fiber products $X\times_SU$ exists for each open subscheme $U\in\mc U$, then the fiber product $X\times_SY$ also exists. (The implicit scheme map $U\to S$ is induced by appending the open embedding $U\into X$ to get $U\into X\to S$.)
\end{lemma}
Here is why we care about \autoref{lem:keyfibercase}.
\begin{corollary}
	Fix schemes $X$ and $Y$ over an affine scheme $S$. Then the fiber product $X\times_SY$ exists.
\end{corollary}
\begin{proof}
	We have two steps.
	\begin{enumerate}
		\item Suppose that $X$ and $S$ are affine. Then we can give $Y$ an affine open cover $\mc U$, and we know that the fiber product $X\times_SU$ exists because now everything is affine, so \autoref{lem:affine-fp} is good enough. It follows that $X\times_SY$ also exists by \autoref{lem:keyfibercase}.
		\item Suppose that $S$ is affine. Then we can give $X$ an affine open cover $\mc U$, and we know that the fiber product $U\times_SY$ exists by the previous point. So again, $X\times_SY$ exists by applying the form \autoref{lem:keyfibercase} achieved by swapping the $X$s and $Y$s.
		\qedhere
	\end{enumerate}
\end{proof}
We will figure out how to glue together over $S$ at the last step.

To prove \autoref{lem:keyfibercase}, we have the following smaller cases.
\begin{lemma}
	Fix a scheme morphism $\varphi\colon X\to Y$. Then, for any open $U\subseteq Y$, the square
	% https://q.uiver.app/?q=WzAsNCxbMSwwLCJYIl0sWzEsMSwiWSJdLFswLDAsIlxcdmFycGhpXnstMX1VIl0sWzAsMSwiVSJdLFswLDEsIlxcdmFycGhpIl0sWzIsMywiXFx2YXJwaGl8X3tcXHZhcnBoaV57LTF9VX0iLDJdLFsyLDAsIiIsMCx7InN0eWxlIjp7InRhaWwiOnsibmFtZSI6Imhvb2siLCJzaWRlIjoidG9wIn19fV0sWzMsMSwiIiwyLHsic3R5bGUiOnsidGFpbCI6eyJuYW1lIjoiaG9vayIsInNpZGUiOiJ0b3AifX19XV0=&macro_url=https%3A%2F%2Fraw.githubusercontent.com%2FdFoiler%2Fnotes%2Fmaster%2Fnir.tex
	\[\begin{tikzcd}
		{\varphi^{-1}U} & X \\
		U & Y
		\arrow["\varphi", from=1-2, to=2-2]
		\arrow["{\varphi|_{\varphi^{-1}U}}"', from=1-1, to=2-1]
		\arrow[hook, from=1-1, to=1-2]
		\arrow[hook, from=2-1, to=2-2]
	\end{tikzcd}\]
	is a pullback square.
\end{lemma}
\begin{proof}
	Label our maps as follows.
	% https://q.uiver.app/?q=WzAsNCxbMSwwLCJYIl0sWzEsMSwiWSJdLFswLDAsIlxcdmFycGhpXnstMX1VIl0sWzAsMSwiVSJdLFswLDEsIlxcdmFycGhpIl0sWzIsMywiXFx2YXJwaGl8X3tcXHZhcnBoaV57LTF9VX0iLDJdLFsyLDAsIlxcaW90YSIsMCx7InN0eWxlIjp7InRhaWwiOnsibmFtZSI6Imhvb2siLCJzaWRlIjoidG9wIn19fV0sWzMsMSwiXFxqbWF0aCIsMCx7InN0eWxlIjp7InRhaWwiOnsibmFtZSI6Imhvb2siLCJzaWRlIjoidG9wIn19fV1d&macro_url=https%3A%2F%2Fraw.githubusercontent.com%2FdFoiler%2Fnotes%2Fmaster%2Fnir.tex
	\[\begin{tikzcd}
		{\varphi^{-1}U} & X \\
		U & Y
		\arrow["\varphi", from=1-2, to=2-2]
		\arrow["{\varphi|_{\varphi^{-1}U}}"', from=1-1, to=2-1]
		\arrow["\iota", hook, from=1-1, to=1-2]
		\arrow["\jmath", hook, from=2-1, to=2-2]
	\end{tikzcd}\]
	Observe that the right arrow of the diagram is induced as $(\varphi,\varphi^\sharp)|_U$ by \autoref{lem:restrictmorphism}. The horizontal arrows are the open embeddings of \autoref{ex:open-embed-morphism}, and the diagram commutes by \autoref{rem:open-embed-commutes}.

	It remains to show the universal property. Suppose that $Z$ is a scheme with morphisms $\psi_X\colon Z\to X$ and $\psi_U\colon Z\to U$ such that $\varphi\circ\psi_X=\jmath\circ\psi_U$. We need a unique scheme morphism $\psi\colon Z\to\varphi^{-1}U$ making the diagram
	% https://q.uiver.app/?q=WzAsNSxbMiwxLCJYIl0sWzIsMiwiWSJdLFsxLDEsIlxcdmFycGhpXnstMX1VIl0sWzEsMiwiVSJdLFswLDAsIloiXSxbMCwxLCJcXHZhcnBoaSJdLFsyLDMsIlxcdmFycGhpfF97XFx2YXJwaGleey0xfVV9Il0sWzIsMCwiXFxpb3RhIiwyLHsic3R5bGUiOnsidGFpbCI6eyJuYW1lIjoiaG9vayIsInNpZGUiOiJ0b3AifX19XSxbMywxLCJcXGptYXRoIiwyLHsic3R5bGUiOnsidGFpbCI6eyJuYW1lIjoiaG9vayIsInNpZGUiOiJ0b3AifX19XSxbNCwwLCJcXHBzaV9YIiwwLHsiY3VydmUiOi0yfV0sWzQsMywiXFxwc2lfWSIsMix7ImN1cnZlIjoyfV0sWzQsMiwiXFxwc2kiLDEseyJzdHlsZSI6eyJib2R5Ijp7Im5hbWUiOiJkYXNoZWQifX19XV0=&macro_url=https%3A%2F%2Fraw.githubusercontent.com%2FdFoiler%2Fnotes%2Fmaster%2Fnir.tex
	\begin{equation}
		\begin{tikzcd}
			Z \\
			& {\varphi^{-1}U} & X \\
			& U & Y
			\arrow["\varphi", from=2-3, to=3-3]
			\arrow["{\varphi|_{\varphi^{-1}U}}", from=2-2, to=3-2]
			\arrow["\iota"', hook, from=2-2, to=2-3]
			\arrow["\jmath"', hook, from=3-2, to=3-3]
			\arrow["{\psi_X}", curve={height=-12pt}, from=1-1, to=2-3]
			\arrow["{\psi_U}"', curve={height=12pt}, from=1-1, to=3-2]
			\arrow["\psi"{description}, dashed, from=1-1, to=2-2]
		\end{tikzcd} \label{eq:open-embed-fp}
	\end{equation}
	commute. We show uniqueness and existence separately.
	\begin{itemize}
		\item Uniqueness: on topological spaces, we require any $z\in Z$ to $\psi(z)=\iota(\psi(z))=\psi_X(z)$, so $\psi$ is uniquely determined topologically. On sheaves, we note that we need the diagram
		% https://q.uiver.app/?q=WzAsMyxbMCwwLCJcXE9PX1giXSxbMSwwLCJcXGlvdGFfKihcXE9PX1h8X3tcXHZhcnBoaV57LTF9VX0pIl0sWzEsMSwiXFxPT19aIl0sWzAsMSwiXFxpb3RhXlxcc2hhcnAiXSxbMCwyLCJcXHBzaV5cXHNoYXJwX1giLDJdLFsxLDIsIlxcaW90YV8qXFxwc2leXFxzaGFycCJdXQ==&macro_url=https%3A%2F%2Fraw.githubusercontent.com%2FdFoiler%2Fnotes%2Fmaster%2Fnir.tex
		\[\begin{tikzcd}
			{\OO_X} & {\iota_*(\OO_X|_{\varphi^{-1}U})} \\
			& {\OO_Z}
			\arrow["{\iota^\sharp}", from=1-1, to=1-2]
			\arrow["{\psi^\sharp_X}"', from=1-1, to=2-2]
			\arrow["{\iota_*\psi^\sharp}", from=1-2, to=2-2]
		\end{tikzcd}\]
		to commute. In particular, for open subset $V\subseteq\varphi^{-1}U$, we see that $\iota^\sharp_V=\op{res}_{V,\varphi^{-1}UV}$ is just the identity, so we are asking for the diagram
		% https://q.uiver.app/?q=WzAsMyxbMCwwLCJcXE9PX1goVikiXSxbMSwwLCJcXE9PX1goVikiXSxbMSwxLCJcXE9PX1oiXSxbMCwxLCJcXGlvdGFeXFxzaGFycF9WIiwwLHsibGV2ZWwiOjIsInN0eWxlIjp7ImhlYWQiOnsibmFtZSI6Im5vbmUifX19XSxbMCwyLCIoXFxwc2leXFxzaGFycF9YKV9WIiwyXSxbMSwyLCJcXHBzaV5cXHNoYXJwX1YiXV0=&macro_url=https%3A%2F%2Fraw.githubusercontent.com%2FdFoiler%2Fnotes%2Fmaster%2Fnir.tex
		\[\begin{tikzcd}
			{\OO_X(V)} & {\OO_X(V)} \\
			& {\OO_Z}
			\arrow["{\iota^\sharp_V}", Rightarrow, no head, from=1-1, to=1-2]
			\arrow["{(\psi^\sharp_X)_V}"', from=1-1, to=2-2]
			\arrow["{\psi^\sharp_V}", from=1-2, to=2-2]
		\end{tikzcd}\]
		to commute, which we can see forces $\psi_V$.
		\item Existence: we follow the above formula. Define $\psi(z)\coloneqq\psi_X(z)\in X$. Notably, this makes sense because $\varphi(\psi_X(z))=\psi_Y(z)\in U$, so $\psi_X(z)\in\varphi^{-1}U$. Now, \autoref{eq:open-embed-fp} commutes on topological spaces by tracking everything through:
		\[\iota(\psi(z))=\psi(z)=\psi_X(z)\qquad\text{and}\qquad\varphi(\psi(z))=\varphi(\psi_Z(z))=\psi_Y(z).\]
		On sheaves, for any open $V\subseteq\varphi^{-1}U$, we need a map $\psi_V\colon\OO_{\varphi^{-1}U}(V)\to\psi_*\OO_Z(V)$, but $\OO_{\varphi^{-1}U}(V)=\OO_X(\varphi^{-1}U\cap V)$ and $\psi_*\OO_Z(V)=\OO_Z(\psi^{-1}V)=\OO_Z(\psi_X^{-1}V)$. However, we note that $\im\psi_X\subseteq\varphi^{-1}U$ as discussed previously, so we define our map as the composite
		\[\OO_{\varphi^{-1}U}(V)=\OO_X(\varphi^{-1}U\cap V)\stackrel{(\psi_X^\sharp)_{\varphi^{-1}U\cap V}}\to(\psi_X)_*\OO_Z(\varphi^{-1}U\cap V)=\OO_Z(\psi_X^{-1}V).\]
		We now run the necessary checks.
		\begin{itemize}
			\item Sheaf morphism: given open subsets $V'\subseteq V$, we see that the diagram
			% https://q.uiver.app/?q=WzAsOCxbMCwwLCJcXE9PX3tcXHZhcnBoaV57LTF9VX0oVikiXSxbMCwxLCJcXE9PX3tcXHZhcnBoaV57LTF9VX0oVicpIl0sWzEsMCwiXFxPT19YKFxcdmFycGhpXnstMX1VXFxjYXAgVikiXSxbMSwxLCJcXE9PX1goXFx2YXJwaGleey0xfVVcXGNhcCBWJykiXSxbMiwwLCIoXFxwc2lfWClfKlxcT09fWihcXHZhcnBoaV57LTF9VVxcY2FwIFYpIl0sWzIsMSwiKFxccHNpX1gpXypcXE9PX1ooXFx2YXJwaGleey0xfVVcXGNhcCBWJykiXSxbMywwLCJcXE9PX1ooXFxwc2leey0xfV9YVikiXSxbMywxLCJcXE9PX1ooXFxwc2leey0xfV9YVicpIl0sWzAsMiwiIiwwLHsibGV2ZWwiOjIsInN0eWxlIjp7ImhlYWQiOnsibmFtZSI6Im5vbmUifX19XSxbMSwzLCIiLDAseyJsZXZlbCI6Miwic3R5bGUiOnsiaGVhZCI6eyJuYW1lIjoibm9uZSJ9fX1dLFswLDEsIlxcb3B7cmVzfSIsMl0sWzIsMywiXFxvcHtyZXN9IiwyXSxbNCw1LCJcXG9we3Jlc30iLDJdLFsyLDQsIihcXHBzaV9YXlxcc2hhcnApX3tcXHZhcnBoaV57LTF9VVxcY2FwIFZ9Il0sWzMsNSwiKFxccHNpX1heXFxzaGFycClfe1xcdmFycGhpXnstMX1VXFxjYXAgVid9Il0sWzYsNywiXFxvcHtyZXN9Il0sWzQsNiwiIiwwLHsibGV2ZWwiOjIsInN0eWxlIjp7ImhlYWQiOnsibmFtZSI6Im5vbmUifX19XSxbNSw3LCIiLDAseyJsZXZlbCI6Miwic3R5bGUiOnsiaGVhZCI6eyJuYW1lIjoibm9uZSJ9fX1dXQ==&macro_url=https%3A%2F%2Fraw.githubusercontent.com%2FdFoiler%2Fnotes%2Fmaster%2Fnir.tex
			\[\begin{tikzcd}
				{\OO_{\varphi^{-1}U}(V)} & {\OO_X(\varphi^{-1}U\cap V)} & {(\psi_X)_*\OO_Z(\varphi^{-1}U\cap V)} & {\OO_Z(\psi^{-1}_XV)} \\
				{\OO_{\varphi^{-1}U}(V')} & {\OO_X(\varphi^{-1}U\cap V')} & {(\psi_X)_*\OO_Z(\varphi^{-1}U\cap V')} & {\OO_Z(\psi^{-1}_XV')}
				\arrow[Rightarrow, no head, from=1-1, to=1-2]
				\arrow[Rightarrow, no head, from=2-1, to=2-2]
				\arrow["{\op{res}}"', from=1-1, to=2-1]
				\arrow["{\op{res}}"', from=1-2, to=2-2]
				\arrow["{\op{res}}"', from=1-3, to=2-3]
				\arrow["{(\psi_X^\sharp)_{\varphi^{-1}U\cap V}}", from=1-2, to=1-3]
				\arrow["{(\psi_X^\sharp)_{\varphi^{-1}U\cap V'}}", from=2-2, to=2-3]
				\arrow["{\op{res}}", from=1-4, to=2-4]
				\arrow[Rightarrow, no head, from=1-3, to=1-4]
				\arrow[Rightarrow, no head, from=2-3, to=2-4]
			\end{tikzcd}\]
			commutes, where the only square which isn't made of horizontal identities is a naturality square for $\psi^\sharp_X$.
			\item Morphism of locally ringed spaces: this is essentially inherited directly from $\psi^\sharp_X$. Given $z\in Z$, we need to know that the composite
			\[\arraycolsep=1.4pt\begin{array}{cccccc}
				\OO_{\varphi^{-1}U,\psi(z)} &\stackrel{\psi^\sharp_{\psi(z)}}\to& (\psi_*\OO_Z)_{\psi(z)} &\to& \OO_{Z,z} \\
				{[(V,s)]} &\mapsto& [(V,(\psi_X^\sharp)_{\varphi^{-1}U\cap V}(s))] &\mapsto& (\psi_X^\sharp)_{\varphi^{-1}U\cap V}(s)|_z
			\end{array}\]
			is a map of local rings. Well, we note that any $[(V,s)]\in\OO_{\varphi^{-1}U,\psi(z)}$ is canonically also a germ in $\OO_{X,\psi(z)}$, and the fact that $\psi_Z$ is a morphism of locally ringed spaces tells us $(\psi_X^\sharp)_{\varphi^{-1}U\cap V}(s)|_z\in\mf m_{Z,z}$, which is what we wanted.
			\item Commutes: we already checked that the needed diagram \autoref{eq:open-embed-fp} commutes on the level of topological spaces, so we just need to check that it commutes on the level of sheaves. This has two checks.
			\begin{itemize}
				\item We verify that
				% https://q.uiver.app/?q=WzAsMyxbMCwwLCJcXE9PX1UiXSxbMSwwLCJcXHZhcnBoaV8qXFxPT197XFx2YXJwaGleey0xfVV9Il0sWzEsMSwiKFxccHNpX1UpXypcXE9PX1oiXSxbMCwyLCJcXHBzaV9VXlxcc2hhcnAiLDJdLFswLDEsIlxcdmFycGhpXlxcc2hhcnAiXSxbMSwyLCJcXHBzaV5cXHNoYXJwIl1d&macro_url=https%3A%2F%2Fraw.githubusercontent.com%2FdFoiler%2Fnotes%2Fmaster%2Fnir.tex
				\[\begin{tikzcd}
					{\OO_U} & {\varphi_*\OO_{\varphi^{-1}U}} \\
					& {(\psi_U)_*\OO_Z}
					\arrow["{\psi_U^\sharp}"', from=1-1, to=2-2]
					\arrow["{\varphi^\sharp}", from=1-1, to=1-2]
					\arrow["{\varphi_*\psi^\sharp}", from=1-2, to=2-2]
				\end{tikzcd}\]
				commutes. Indeed, for any open subset $V\subseteq U$, we could verify that
				% https://q.uiver.app/?q=WzAsNixbMCwwLCJcXE9PX1UoVikiXSxbMSwwLCJcXE9PX3tcXHZhcnBoaV57LTF9VX0oXFx2YXJwaGleey0xfVVcXGNhcCBWKSJdLFsxLDEsIlxcT09fWihcXHBzaV57LTF9VikiXSxbMiwwLCJzIl0sWzMsMCwiXFx2YXJwaGleXFxzaGFycF9WKHMpIl0sWzMsMSwiKFxccHNpXlxcc2hhcnBfWClfe1xcdmFycGhpXnstMX1VXFxjYXAgVn0oXFx2YXJwaGleXFxzaGFycF9WcykiXSxbMCwyLCIoXFxwc2lfVV5cXHNoYXJwKV9WIiwyXSxbMCwxLCJcXHZhcnBoaV5cXHNoYXJwX1YiXSxbMSwyLCJcXHBzaV5cXHNoYXJwX3tcXHZhcnBoaV57LTF9VVxcY2FwIFZ9Il0sWzMsNCwiIiwwLHsic3R5bGUiOnsidGFpbCI6eyJuYW1lIjoibWFwcyB0byJ9fX1dLFs0LDUsIiIsMCx7InN0eWxlIjp7InRhaWwiOnsibmFtZSI6Im1hcHMgdG8ifX19XSxbMyw1LCIiLDIseyJzdHlsZSI6eyJ0YWlsIjp7Im5hbWUiOiJtYXBzIHRvIn19fV1d&macro_url=https%3A%2F%2Fraw.githubusercontent.com%2FdFoiler%2Fnotes%2Fmaster%2Fnir.tex
				\[\begin{tikzcd}
					{\OO_U(V)} & {\OO_{\varphi^{-1}U}(\varphi^{-1}U\cap V)} & s & {\varphi^\sharp_V(s)} \\
					& {\OO_Z(\psi^{-1}V)} && {(\psi^\sharp_X)_{\varphi^{-1}U\cap V}(\varphi^\sharp_Vs)}
					\arrow["{(\psi_U^\sharp)_V}"', from=1-1, to=2-2]
					\arrow["{\varphi^\sharp_V}", from=1-1, to=1-2]
					\arrow["{\psi^\sharp_{\varphi^{-1}U\cap V}}", from=1-2, to=2-2]
					\arrow[maps to, from=1-3, to=1-4]
					\arrow[maps to, from=1-4, to=2-4]
					\arrow[maps to, from=1-3, to=2-4]
				\end{tikzcd}\]
				commutes appropriately using the given commuting square.
				\item We verify that 
				% https://q.uiver.app/?q=WzAsMyxbMCwwLCJcXE9PX1giXSxbMSwwLCJcXGlvdGFfKlxcT09fe1xcdmFycGhpXnstMX1VfSJdLFsxLDEsIihcXHBzaV9YKV8qXFxPT19aIl0sWzAsMiwiKFxccHNpX1heXFxzaGFycCkiLDJdLFswLDEsIlxcaW90YV5cXHNoYXJwIl0sWzEsMiwiXFxpb3RhXypcXHBzaV5cXHNoYXJwIl1d&macro_url=https%3A%2F%2Fraw.githubusercontent.com%2FdFoiler%2Fnotes%2Fmaster%2Fnir.tex
				\[\begin{tikzcd}
					{\OO_X} & {\iota_*\OO_{\varphi^{-1}U}} \\
					& {(\psi_X)_*\OO_Z}
					\arrow["{(\psi_X^\sharp)}"', from=1-1, to=2-2]
					\arrow["{\iota^\sharp}", from=1-1, to=1-2]
					\arrow["{\iota_*\psi^\sharp}", from=1-2, to=2-2]
				\end{tikzcd}\]
				commutes. Indeed, for any open subset $V\subseteq U$, we could verify that
				% https://q.uiver.app/?q=WzAsMyxbMCwwLCJcXE9PX1goVikiXSxbMSwwLCJcXE9PX3tcXHZhcnBoaV57LTF9VX0oXFx2YXJwaGleey0xfVVcXGNhcCBWKSJdLFsxLDEsIlxcT09fWihcXHBzaV9YXnstMX1WKSJdLFswLDIsIihcXHBzaV9YXlxcc2hhcnApX1YiLDJdLFswLDEsIlxcaW90YV5cXHNoYXJwX1YiXSxbMSwyLCJcXHBzaV5cXHNoYXJwX3tcXHZhcnBoaV57LTF9VVxcY2FwIFZ9Il1d&macro_url=https%3A%2F%2Fraw.githubusercontent.com%2FdFoiler%2Fnotes%2Fmaster%2Fnir.tex
				\[\begin{tikzcd}
					{\OO_X(V)} & {\OO_{\varphi^{-1}U}(\varphi^{-1}U\cap V)} \\
					& {\OO_Z(\psi_X^{-1}V)}
					\arrow["{(\psi_X^\sharp)_V}"', from=1-1, to=2-2]
					\arrow["{\iota^\sharp_V}", from=1-1, to=1-2]
					\arrow["{\psi^\sharp_{\varphi^{-1}U\cap V}}", from=1-2, to=2-2]
				\end{tikzcd}\]
				commutes directly from the construction of $\psi^\sharp$.
			\end{itemize}
		\end{itemize}
	\end{itemize}
	The above checks complete the proof.
\end{proof}
\begin{lemma} \label{lem:helpkeycase}
	Fix schemes $X$ and $Y$ over a scheme $S$. Given an open subset $U\subseteq Y$, if $X\times_SY$ exists, then $X\times_SU$ also exists.
\end{lemma}
\begin{proof}[Sketch]
	To begin, label our relevant maps as follows.
	% https://q.uiver.app/?q=WzAsNCxbMCwxLCJZIl0sWzEsMCwiWCJdLFsxLDEsIlMiXSxbMCwwLCJYXFx0aW1lc19TWSJdLFsxLDIsIlxccHNpX1giXSxbMCwyLCJcXHBzaV9ZIl0sWzMsMSwiXFxwaV9YIl0sWzMsMCwiXFxwaV9ZIiwyXSxbMywyLCIiLDAseyJzdHlsZSI6eyJuYW1lIjoiY29ybmVyIn19XV0=&macro_url=https%3A%2F%2Fraw.githubusercontent.com%2FdFoiler%2Fnotes%2Fmaster%2Fnir.tex
	\[\begin{tikzcd}
		{X\times_SY} & X \\
		Y & S
		\arrow["{\psi_X}", from=1-2, to=2-2]
		\arrow["{\psi_Y}", from=2-1, to=2-2]
		\arrow["{\pi_X}", from=1-1, to=1-2]
		\arrow["{\pi_Y}"', from=1-1, to=2-1]
		\arrow["\lrcorner"{anchor=center, pos=0.125}, draw=none, from=1-1, to=2-2]
	\end{tikzcd}\]
	Our fiber product is going to be $\pi_Y^{-1}(U)$, which induces the restricted ring map $\pi_U\coloneqq\pi_Y|_{\pi_Y^{-1}U}$ by \autoref{lem:restrictmorphism}. This gives us the diagram
	% https://q.uiver.app/?q=WzAsNixbMSwxLCJZIl0sWzIsMCwiWCJdLFsyLDEsIlMiXSxbMSwwLCJYXFx0aW1lc19TWSJdLFswLDAsIlxccGlfWV57LTF9KFUpIl0sWzAsMSwiVSJdLFsxLDIsIlxccHNpX1giXSxbMCwyLCJcXHBzaV9ZIl0sWzMsMSwiXFxwaV9YIl0sWzMsMCwiXFxwaV9ZIiwyXSxbNSwwLCJcXGptYXRoIiwwLHsic3R5bGUiOnsidGFpbCI6eyJuYW1lIjoiaG9vayIsInNpZGUiOiJ0b3AifX19XSxbNCwzLCJcXGlvdGEiLDAseyJzdHlsZSI6eyJ0YWlsIjp7Im5hbWUiOiJob29rIiwic2lkZSI6InRvcCJ9fX1dLFs0LDUsIlxccGlfVSIsMl1d&macro_url=https%3A%2F%2Fraw.githubusercontent.com%2FdFoiler%2Fnotes%2Fmaster%2Fnir.tex
	\[\begin{tikzcd}
		{\pi_Y^{-1}(U)} & {X\times_SY} & X \\
		U & Y & S
		\arrow["{\psi_X}", from=1-3, to=2-3]
		\arrow["{\psi_Y}", from=2-2, to=2-3]
		\arrow["{\pi_X}", from=1-2, to=1-3]
		\arrow["{\pi_Y}"', from=1-2, to=2-2]
		\arrow["\jmath", hook, from=2-1, to=2-2]
		\arrow["\iota", hook, from=1-1, to=1-2]
		\arrow["{\pi_U}"', from=1-1, to=2-1]
	\end{tikzcd}\]
	which we can see commutes on the left by \autoref{rem:open-embed-commutes}.

	As an intermediate step, we claim that
	% https://q.uiver.app/?q=WzAsNCxbMSwxLCJZIl0sWzEsMCwiWFxcdGltZXNfU1kiXSxbMCwwLCJcXHBpX1leey0xfShVKSJdLFswLDEsIlUiXSxbMSwwLCJcXHBpX1kiLDJdLFszLDAsIlxcam1hdGgiLDAseyJzdHlsZSI6eyJ0YWlsIjp7Im5hbWUiOiJob29rIiwic2lkZSI6InRvcCJ9fX1dLFsyLDEsIlxcaW90YSIsMCx7InN0eWxlIjp7InRhaWwiOnsibmFtZSI6Imhvb2siLCJzaWRlIjoidG9wIn19fV0sWzIsMywiXFxwaV9VIiwyXV0=&macro_url=https%3A%2F%2Fraw.githubusercontent.com%2FdFoiler%2Fnotes%2Fmaster%2Fnir.tex
	\[\begin{tikzcd}
		{\pi_Y^{-1}(U)} & {X\times_SY} \\
		U & Y
		\arrow["{\pi_Y}"', from=1-2, to=2-2]
		\arrow["\jmath", hook, from=2-1, to=2-2]
		\arrow["\iota", hook, from=1-1, to=1-2]
		\arrow["{\pi_U}"', from=1-1, to=2-1]
	\end{tikzcd}\]
	is a pullback square.

	We use the universal property a bunch of times. Then
	% https://q.uiver.app/?q=WzAsNixbMCwyLCJYIl0sWzEsMSwiWSciXSxbMSwyLCJTIl0sWzAsMSwiWFxcdGltZXNfU1knIl0sWzEsMCwiWSJdLFswLDAsIlxccGleey0xfV97WSd9KFkpIl0sWzAsMl0sWzEsMl0sWzMsMCwiXFxwaV9YIl0sWzMsMSwiXFxwaV9ZIiwyXSxbNSw0LCJcXHBpX1kiXSxbNSwzXSxbNCwxXV0=&macro_url=https%3A%2F%2Fraw.githubusercontent.com%2FdFoiler%2Fnotes%2Fmaster%2Fnir.tex
	\[\begin{tikzcd}
		{\pi^{-1}_{Y'}(Y)} & Y \\
		{X\times_SY'} & {Y'} \\
		X & S
		\arrow[from=3-1, to=3-2]
		\arrow[from=2-2, to=3-2]
		\arrow["{\pi_X}", from=2-1, to=3-1]
		\arrow["{\pi_Y}"', from=2-1, to=2-2]
		\arrow["{\pi_Y}", from=1-1, to=1-2]
		\arrow[from=1-1, to=2-1]
		\arrow[from=1-2, to=2-2]
	\end{tikzcd}\]
	we can show has the top and bottom squares are both pullback squares, so the full rectangle is a pullback square, which finishes.
\end{proof}
We can now prove \autoref{lem:keyfibercase}.
\begin{proof}[Proof of \autoref{lem:keyfibercase}]
	Give $Y$ the affine open cover $\{Y_\alpha\}_{\alpha\in\lambda}$, writing $Y_{\alpha\beta}\coloneqq Y_\alpha\cap Y_\beta$ as a subset of $Y_\alpha$ to prepare for our gluing. Now, we see that \autoref{lem:helpkeycase} grants us schemes
	\[X\times_SY_{\alpha\beta}\subseteq X\times_SY_\alpha\qquad\text{and}\qquad X\times_SY_{\beta\alpha}\subseteq X\times_SY_\beta,\]
	which must be identified because $Y_{\alpha\beta}=Y_{\beta\alpha}$. These identified schemes are going to satisfy the cocycle condition because these fiber products are unique up to unique isomorphism.

	To see the cocycle condition more concretely, we can just say out loud that we have the scheme
	\[X\times_SY_{\alpha\beta\gamma}=(X\times_SY_{\alpha\beta})\cap(X\times_SY_{\alpha\gamma})\subseteq X\times_SY_\alpha.\]
	On the other side, we have a similar computation for $Y_{jik}$ and $Y_{kij}$, so chaining these diagrams together will identify the needed isomorphisms by the uniqueness of our various isomorphisms.

	At this point, we have been glued together a (unique) scheme $W$ with ``compatible'' open embeddings $X\times_SY_\alpha\into W$. It remains to verify that $W$ is our fiber product $X\times_SY$. Our map $\pi_X\colon W\to X$ is induced by projecting down to $X$ in all the glued components; our map $\pi_Y\colon W\to Y$ is made by gluing the various maps $\pi_\alpha\colon W\to Y_\alpha$ (which appear from the projections $X\times_SY_\alpha\to Y_\alpha$).
	
	We now verify the universal property. Fix morphisms $\varphi_X\colon Z\to X$ and $\varphi_Y\colon Y\to Z$ making the diagram
	% https://q.uiver.app/?q=WzAsNSxbMSwyLCJYIl0sWzIsMSwiWSJdLFsyLDIsIlMiXSxbMCwwLCJaIl0sWzEsMSwiVyJdLFswLDJdLFsxLDJdLFszLDAsIlxcdmFycGhpX1giLDIseyJjdXJ2ZSI6Mn1dLFszLDEsIlxcdmFycGhpX1kiLDAseyJjdXJ2ZSI6LTJ9XSxbNCwxLCJcXHBpX1kiLDJdLFs0LDAsIlxccGlfWCJdLFszLDQsIiIsMSx7InN0eWxlIjp7ImJvZHkiOnsibmFtZSI6ImRhc2hlZCJ9fX1dXQ==&macro_url=https%3A%2F%2Fraw.githubusercontent.com%2FdFoiler%2Fnotes%2Fmaster%2Fnir.tex
	\[\begin{tikzcd}
		Z \\
		& W & Y \\
		& X & S
		\arrow[from=3-2, to=3-3]
		\arrow[from=2-3, to=3-3]
		\arrow["{\varphi_X}"', curve={height=12pt}, from=1-1, to=3-2]
		\arrow["{\varphi_Y}", curve={height=-12pt}, from=1-1, to=2-3]
		\arrow["{\pi_Y}"', from=2-2, to=2-3]
		\arrow["{\pi_X}", from=2-2, to=3-2]
		\arrow[dashed, from=1-1, to=2-2]
	\end{tikzcd}\]
	commute, and we want to induce a unique arrow where the dashed arrow is. The big diagram commuting gives an internal commuting diagram
	% https://q.uiver.app/?q=WzAsNSxbMSwyLCJYIl0sWzIsMSwiWV9cXGFscGhhIl0sWzIsMiwiUyJdLFswLDAsIlpfXFxhbHBoYSJdLFsxLDEsIlhcXHRpbWVzX1NZX1xcYWxwaGEiXSxbMCwyXSxbMSwyXSxbMywwLCJcXHZhcnBoaV9YIiwyLHsiY3VydmUiOjJ9XSxbMywxLCJcXHZhcnBoaV9ZIiwwLHsiY3VydmUiOi0yfV0sWzQsMSwiXFxwaV9ZIiwyXSxbNCwwLCJcXHBpX1giXSxbMyw0LCJcXHZhcnBoaV9cXGFscGhhIiwxLHsic3R5bGUiOnsiYm9keSI6eyJuYW1lIjoiZGFzaGVkIn19fV1d&macro_url=https%3A%2F%2Fraw.githubusercontent.com%2FdFoiler%2Fnotes%2Fmaster%2Fnir.tex
	\[\begin{tikzcd}
		{Z_\alpha} \\
		& {X\times_SY_\alpha} & {Y_\alpha} \\
		& X & S
		\arrow[from=3-2, to=3-3]
		\arrow[from=2-3, to=3-3]
		\arrow["{\varphi_X}"', curve={height=12pt}, from=1-1, to=3-2]
		\arrow["{\varphi_Y}", curve={height=-12pt}, from=1-1, to=2-3]
		\arrow["{\pi_Y}"', from=2-2, to=2-3]
		\arrow["{\pi_X}", from=2-2, to=3-2]
		\arrow["{\varphi_\alpha}"{description}, dashed, from=1-1, to=2-2]
	\end{tikzcd}\]
	with $Z_\alpha\coloneqq\varphi_Y^{-1}(Y_\alpha)$, where we have already used the universal property to construct the unique morphism $\varphi_\alpha$ making the diagram commute. This gives maps
	\[Z_\alpha\to X\times_SY_\alpha\subseteq W,\]
	and we just show that these various maps will be compatible to give us a unique map $Z\to W$. One can track through all the various universal properties to show that we have in fact induced a unique map making the original diagram commute.

	We now show that we can in fact glue the morphisms $Z_\alpha\to W$. Well, set $Z_{\alpha\beta}\coloneqq\varphi_Y^{-1}(Y_{\alpha\beta})$ to be $Z_\alpha\cap Z_\beta$, and we can see that the diagram
	% https://q.uiver.app/?q=WzAsNCxbMCwwLCJaX3tcXGFscGhhXFxiZXRhfSJdLFsxLDEsIlhcXHRpbWVzX1NZX3tcXGFscGhhXFxiZXRhfSJdLFsyLDEsIlhcXHRpbWVzX1NZX1xcYWxwaGEiXSxbMSwyLCJYXFx0aW1lc19TWV9cXGJldGEiXSxbMSwzXSxbMSwyXSxbMCwxXSxbMCwzLCJcXHZhcnBoaV9cXGJldGEiLDIseyJjdXJ2ZSI6Mn1dLFswLDIsIlxcdmFycGhpX1xcYWxwaGEiLDAseyJjdXJ2ZSI6LTJ9XV0=&macro_url=https%3A%2F%2Fraw.githubusercontent.com%2FdFoiler%2Fnotes%2Fmaster%2Fnir.tex
	\[\begin{tikzcd}
		{Z_{\alpha\beta}} \\
		& {X\times_SY_{\alpha\beta}} & {X\times_SY_\alpha} \\
		& {X\times_SY_\beta}
		\arrow[from=2-2, to=3-2]
		\arrow[from=2-2, to=2-3]
		\arrow[from=1-1, to=2-2]
		\arrow["{\varphi_\beta}"', curve={height=12pt}, from=1-1, to=3-2]
		\arrow["{\varphi_\alpha}", curve={height=-12pt}, from=1-1, to=2-3]
	\end{tikzcd}\]
	commutes, which finishes.
\end{proof}
Note that the same proof will show that we can build the fiber product for $X\times_SY$ when $S$ is affine and $X$ and $Y$ are arbitrary; namely, nowhere did we use in the proof that $X$ was affine (only that the fiber products $X\times_SY_\alpha$ exist), so we can just apply the same gluing process.

Lastly, here is the general case.
\begin{proof}[Proof of \autoref{thm:fibexist}]
	Give $S$ an affine open cover $\{S_\alpha\}_{\alpha\in\lambda}$, with our natural maps $f_X\colon X\to S$ and $f_Y\colon Y\to S$. Then we set $X_\alpha\coloneqq f_X^{-1}(S_\alpha)$ and similar for $Y_\alpha$. By previous work, we have the fiber products $X_\alpha\times_{S_\alpha}Y_\alpha$, so we can just glue these together as usual.
\end{proof}
A little more rigorously, the following result aids the above gluing.
\begin{lemma}
	Fix schemes $X$ and $Y$ over a scheme $S$. If $S\subseteq S'$ is an affine open embedding of schemes such that $X\times_{S'}Y$ exists, we have
	\[f_X^{-1}(S)\times_Sf_Y^{-1}(S)=(f_X\circ\pi_X)^{-1}(S')=(f_Y\circ\pi_Y)^{-1}(S').\]
\end{lemma}
The intuition is that the above lemma should be pulling back the required fiber product along the various legs of the following diagram.
% https://q.uiver.app/?q=WzAsNCxbMCwwLCJYXFx0aW1lc197Uyd9WSJdLFsxLDAsIlkiXSxbMCwxLCJYIl0sWzEsMSwiUyciXSxbMCwxXSxbMSwzXSxbMCwyXSxbMiwzXV0=&macro_url=https%3A%2F%2Fraw.githubusercontent.com%2FdFoiler%2Fnotes%2Fmaster%2Fnir.tex
\[\begin{tikzcd}
	{X\times_{S'}Y} & Y \\
	X & {S'}
	\arrow[from=1-1, to=1-2]
	\arrow[from=1-2, to=2-2]
	\arrow[from=1-1, to=2-1]
	\arrow[from=2-1, to=2-2]
\end{tikzcd}\]
This will give the intersection data in the gluing for \autoref{thm:fibexist}, which will finish the gluing.
\begin{remark}
	We will provide a more categorical viewpoint of this construction next class. This categorical viewpoint will be helpful for when we want to define the Grassmannian.
	% V 10.1.8
\end{remark}

\end{document}