% !TEX root = ../notes.tex

\documentclass[../notes.tex]{subfiles}

\begin{document}

\section{September 19}

Bump, bump, bump.
\begin{warn}
	Today we will begin more aggressively notating a scheme $(X,\OO_X)$ by its topological space $X$, where the structure sheaf will always be $\OO_X$. Similarly, a morphism $\varphi\colon X\to Y$ refers to its continuous map, and the map of structure sheaves is $\varphi^\sharp\colon\OO_Y\to\varphi_*\OO_X$.
\end{warn}

\subsection{Fiber Products}
Here is our definition.
\begin{definition}[Fiber product]
	Fix a category $\mc C$. Then, given morphisms $\psi_X\colon X\to S$ and $\psi_Y\colon Y\to S$, the \textit{fiber product} $X\times_SY$ is the limit of the following diagram.
	% https://q.uiver.app/?q=WzAsMyxbMSwxLCJTIl0sWzEsMCwiWCJdLFswLDEsIlkiXSxbMSwwLCJcXHBzaV9YIl0sWzIsMCwiXFxwc2lfWSIsMl1d&macro_url=https%3A%2F%2Fraw.githubusercontent.com%2FdFoiler%2Fnotes%2Fmaster%2Fnir.tex
	\[\begin{tikzcd}
		& X \\
		Y & S
		\arrow["{\psi_X}", from=1-2, to=2-2]
		\arrow["{\psi_Y}"', from=2-1, to=2-2]
	\end{tikzcd}\]
\end{definition}
\begin{example} \label{ex:set-fp}
	In the category $\mathrm{Set}$, one can show that
	\[X\times_SY=\{(x,y)\in X\times Y:\psi_X(x)=\psi_Y(y)\},\]
	where the projections are the canonical ones.
\end{example}
\begin{notation}
	Given a fiber product $X\times_SY$ in a category, we call the resulting square
	% https://q.uiver.app/?q=WzAsNCxbMSwwLCJYIl0sWzAsMSwiWSJdLFsxLDEsIlMiXSxbMCwwLCJYXFx0aW1lc19TWSJdLFszLDFdLFsxLDJdLFszLDBdLFswLDJdLFszLDIsIiIsMSx7InN0eWxlIjp7Im5hbWUiOiJjb3JuZXIifX1dXQ==&macro_url=https%3A%2F%2Fraw.githubusercontent.com%2FdFoiler%2Fnotes%2Fmaster%2Fnir.tex
	\[\begin{tikzcd}
		{X\times_SY} & X \\
		Y & S
		\arrow[from=1-1, to=2-1]
		\arrow[from=2-1, to=2-2]
		\arrow[from=1-1, to=1-2]
		\arrow[from=1-2, to=2-2]
		\arrow["\lrcorner"{anchor=center, pos=0.125}, draw=none, from=1-1, to=2-2]
	\end{tikzcd}\]
	a \textit{pullback square}.
\end{notation}
And here is our main result.
\begin{theorem} \label{thm:fibexist}
	Fix two $S$-schemes $X$ and $Y$. Then the fiber product $X\times_SY$ exists.
\end{theorem}
\begin{remark}
	Even if $X$ and $Y$ are Noetherian, it does not follow that $X\times_SY$ is Noetherian. For example, taking $\Spec\overline\QQ$ and $\Spec\overline\QQ$ are both Noetherian, but the fiber product turns out to be
	\[\Spec\overline\QQ\otimes_\QQ\overline\QQ.\]
	Namely, $\Spec\overline\QQ\otimes_\QQ\overline\QQ$ is zero-dimensional but has infinitely many points and is therefore not Noetherian.
\end{remark}
% The fiber product is a purely categorical construct, which is the limit of the diagram as follows.
% % https://q.uiver.app/?q=WzAsMyxbMCwxLCJYIl0sWzEsMCwiWSJdLFsxLDEsIlMiXSxbMCwyXSxbMSwyXV0=&macro_url=https%3A%2F%2Fraw.githubusercontent.com%2FdFoiler%2Fnotes%2Fmaster%2Fnir.tex
% \[\begin{tikzcd}
% 	& Y \\
% 	X & S
% 	\arrow[from=2-1, to=2-2]
% 	\arrow[from=1-2, to=2-2]
% \end{tikzcd}\]
% In other words, there are canonical projection maps $X\times_SY\to X$ and $X\times_SY\to Y$ with a suitable universal property. As usual, the universal property means that this is unique up to unique isomorphism.
We will provide one proof of \autoref{thm:fibexist} today and another next class involving the representability of functors. To get a taste for this functor business, we note that we can concretely describe the functor of points for the fiber product.
\begin{lemma} \label{lem:repr-fp}
	Fix a category $\mc C$ and two maps $\psi_X\colon X\to S$ and $\psi_Y\colon Y\to S$ such that the fiber product $X\times_SY$ exists. Then, for any $Z\in\mc C$ with a map $\psi_Z\colon Z\to S$,
	\[\op{Mor}_S(Z,X\times_SY)\cong\op{Mor}_S(Z,X)\times_{\op{Mor}_S(Z,S)}\op{Mor}_S(Z,Y).\]
	Here, $\op{Mor}_S$ refers to morphisms in the category over $S$.
\end{lemma}
\begin{proof}
	This follows straight from the universal property. Namely, let $\pi_X\colon X\times_SY\to X$ and $\pi_Y\colon X\times_SY\to Y$ be the canonical projections. Using \autoref{ex:set-fp}, we are really just asking for an isomorphism
	\[\op{Mor}_S(Z,X\times_SY)\cong\{(\varphi_X,\varphi_Y)\in\op{Mor}_S(Z,X)\times\op{Mor}_S(Z,Y):\psi_X\circ\varphi_X=\psi_Y\circ\varphi_Y\}\]
	matching up with the projections. Namely, observe that we have a map from the left to the right simply taking a map $\varphi\mapsto(\pi_X\circ\varphi,\pi_Y\circ\varphi)$, which works because
	\[\psi_X\circ\pi_X\circ\varphi=\psi_Y\circ\pi_Y\circ\varphi\]
	should both be $\psi_Z$. In the reverse direction, we can take a pair of maps $\varphi_X\colon Z\to X$ and $\varphi_Y\colon Z\to Y$ such that $\psi_X\circ\varphi_X=\psi_Y\circ\varphi_Y$ and recover a unique map $\varphi\colon Z\to X\times_SY$ (by the universal property) such that $\pi_X\circ\varphi=\varphi_X$ and $\pi_Y\circ\varphi_Y$.
\end{proof}
The difficulty will be in actually finding a scheme which can represent the functor
\[\op{Mor}_S(-,X)\times_{\op{Mor}_S(-,S)}\op{Mor}_S(-,Y).\]
Namely, even though we are sure that this object is unique up to unique isomorphism (by \autoref{thm:yo-lem}), it is not actually clear that it exists at all!
% \begin{remark}
% 	Even without knowing that $X\times_SY$ exists, we may note that we have a natural isomorphism
% 	\[\op{Mor}_S(Z,X\times_SY)\cong\op{Mor}_S(Z,X)\times_{\op{Mor}_S(Z,S)}\op{Mor}_S(Z,Y)\]
% 	coming straight from the universal property. In other words, the $Z$-points of an $S$-fiber product is just going to be the product of the two $S$-points. In this way, we can view the below proof as asking for a particular functor (on the right here) to be representable, which will be enlightening after some thought.
% 	% EH ch 6, V 10.1
% \end{remark}

\subsection{Stacking Squares}
We are going to want a few basic facts about pullback squares in life, so we pick them up now.
\begin{lemma} \label{lem:big-to-small-square}
	Suppose that the commutative diagram
	% https://q.uiver.app/?q=WzAsNixbMCwwLCJBIl0sWzEsMCwiQiJdLFswLDEsIkMiXSxbMSwxLCJEIl0sWzAsMiwiRSJdLFsxLDIsIkYiXSxbMiw0LCJcXHZhcnBoaSIsMl0sWzQsNSwiXFxpb3RhIl0sWzIsMywiXFxpb3RhJyJdLFszLDUsIlxccGkiXSxbMCwyLCJcXHZhcnBoaSciLDJdLFsxLDMsIlxccGknIl0sWzAsMSwiXFxpb3RhJyciXV0=&macro_url=https%3A%2F%2Fraw.githubusercontent.com%2FdFoiler%2Fnotes%2Fmaster%2Fnir.tex
	\[\begin{tikzcd}
		A & B \\
		C & D \\
		E & F
		\arrow["\varphi"', from=2-1, to=3-1]
		\arrow["\iota", from=3-1, to=3-2]
		\arrow["{\iota'}", from=2-1, to=2-2]
		\arrow["\pi", from=2-2, to=3-2]
		\arrow["{\varphi'}"', from=1-1, to=2-1]
		\arrow["{\pi'}", from=1-2, to=2-2]
		\arrow["{\iota''}", from=1-1, to=1-2]
	\end{tikzcd}\]
	has the big rectangle and the bottom square both pullback squares. Then the top square is also a pullback.
\end{lemma}
\begin{proof}
	We check the universal property. Fix some object $Z$ with maps $\psi_B\colon Z\to B$ and $\psi_C\colon Z\to C$ such that $\iota'\circ\pi_C=\pi'\circ\pi_B$, and we want a unique morphism $\psi\colon Z\to A$ making the diagram
	% https://q.uiver.app/?q=WzAsNyxbMSwxLCJBIl0sWzIsMSwiQiJdLFsxLDIsIkMiXSxbMiwyLCJEIl0sWzEsMywiRSJdLFsyLDMsIkYiXSxbMCwwLCJaIl0sWzIsNCwiXFx2YXJwaGkiLDJdLFs0LDUsIlxcaW90YSJdLFsyLDMsIlxcaW90YSciXSxbMyw1LCJcXHBpIl0sWzAsMiwiXFx2YXJwaGknIiwyXSxbMSwzLCJcXHBpJyJdLFswLDEsIlxcaW90YScnIl0sWzYsMCwiXFxwaSIsMSx7InN0eWxlIjp7ImJvZHkiOnsibmFtZSI6ImRhc2hlZCJ9fX1dLFs2LDEsIlxccGlfQiIsMCx7ImN1cnZlIjotMn1dLFs2LDIsIlxccGlfQyIsMix7ImN1cnZlIjoyfV1d&macro_url=https%3A%2F%2Fraw.githubusercontent.com%2FdFoiler%2Fnotes%2Fmaster%2Fnir.tex
	\begin{equation}
		\begin{tikzcd}
			Z \\
			& A & B \\
			& C & D \\
			& E & F
			\arrow["\varphi"', from=3-2, to=4-2]
			\arrow["\iota", from=4-2, to=4-3]
			\arrow["{\iota'}", from=3-2, to=3-3]
			\arrow["\pi", from=3-3, to=4-3]
			\arrow["{\varphi'}"', from=2-2, to=3-2]
			\arrow["{\pi'}", from=2-3, to=3-3]
			\arrow["{\iota''}", from=2-2, to=2-3]
			\arrow["\psi"{description}, dashed, from=1-1, to=2-2]
			\arrow["{\psi_B}", curve={height=-12pt}, from=1-1, to=2-3]
			\arrow["{\psi_C}"', curve={height=12pt}, from=1-1, to=3-2]
		\end{tikzcd} \label{eq:cat-theory-lem-full-diag}
	\end{equation}
	commute. We show uniqueness and existence separately.
	\begin{itemize}
		\item Uniqueness: set $\psi_E\coloneqq\varphi\circ\psi_C$ so that $\pi$ makes the diagram
		% https://q.uiver.app/?q=WzAsNSxbMSwxLCJBIl0sWzIsMSwiQiJdLFsxLDIsIkUiXSxbMiwyLCJGIl0sWzAsMCwiWiJdLFsyLDMsIlxcaW90YSciXSxbMCwyLCJcXHZhcnBoaVxcY2lyY1xcdmFycGhpJyIsMl0sWzEsMywiXFxwaVxcY2lyY1xccGknIl0sWzAsMSwiXFxpb3RhJyciXSxbNCwwLCJcXHBpIiwxLHsic3R5bGUiOnsiYm9keSI6eyJuYW1lIjoiZGFzaGVkIn19fV0sWzQsMSwiXFxwaV9CIiwwLHsiY3VydmUiOi0yfV0sWzQsMiwiXFxwaV9FIiwyLHsiY3VydmUiOjJ9XV0=&macro_url=https%3A%2F%2Fraw.githubusercontent.com%2FdFoiler%2Fnotes%2Fmaster%2Fnir.tex
		\begin{equation}
			\begin{tikzcd}
				Z \\
				& A & B \\
				& E & F
				\arrow["{\iota}", from=3-2, to=3-3]
				\arrow["{\varphi\circ\varphi'}"', from=2-2, to=3-2]
				\arrow["{\pi\circ\pi'}", from=2-3, to=3-3]
				\arrow["{\iota''}", from=2-2, to=2-3]
				\arrow["\psi"{description}, dashed, from=1-1, to=2-2]
				\arrow["{\psi_B}", curve={height=-12pt}, from=1-1, to=2-3]
				\arrow["{\psi_E}"', curve={height=12pt}, from=1-1, to=3-2]
			\end{tikzcd} \label{eq:cat-theory-lem-uniq}
		\end{equation}
		commute, but this morphism $\psi$ is uniquely induced by the diagram because the square here is a pullback square.
		\item Existence: as suggested by the above proof, set $\psi_E\coloneqq\varphi\circ\psi_C$, and use the commutativity of \autoref{eq:cat-theory-lem-uniq} to induce a morphism $\pi$. We need to show that the full diagram \autoref{eq:cat-theory-lem-full-diag} commutes. We get $\iota''\circ\psi=\psi_B$ for free, so the main concern is showing $\varphi'\circ\psi=\psi_C$. Well, note that the commutativity of \autoref{eq:cat-theory-lem-uniq} means that both $\varphi'\circ\psi$ and $\psi_C$ can fit the dashed arrow in the diagram
		% https://q.uiver.app/?q=WzAsNSxbMSwxLCJDIl0sWzIsMSwiRCJdLFsxLDIsIkUiXSxbMiwyLCJGIl0sWzAsMCwiWiJdLFsyLDMsIlxcaW90YSJdLFswLDIsIlxcdmFycGhpJyIsMl0sWzEsMywiXFxwaSciXSxbMCwxLCJcXGlvdGEnIl0sWzQsMCwiIiwxLHsic3R5bGUiOnsiYm9keSI6eyJuYW1lIjoiZGFzaGVkIn19fV0sWzQsMSwiXFxwaSdcXGNpcmMgXFxwaV9CIiwwLHsiY3VydmUiOi0yfV0sWzQsMiwiXFxwaV9FIiwyLHsiY3VydmUiOjJ9XV0=&macro_url=https%3A%2F%2Fraw.githubusercontent.com%2FdFoiler%2Fnotes%2Fmaster%2Fnir.tex
		\[\begin{tikzcd}
			Z \\
			& C & D \\
			& E & F
			\arrow["\iota", from=3-2, to=3-3]
			\arrow["{\varphi}"', from=2-2, to=3-2]
			\arrow["{\pi}", from=2-3, to=3-3]
			\arrow["{\iota'}", from=2-2, to=2-3]
			\arrow[dashed, from=1-1, to=2-2]
			\arrow["{\pi'\circ \pi_B}", curve={height=-12pt}, from=1-1, to=2-3]
			\arrow["{\psi_E}"', curve={height=12pt}, from=1-1, to=3-2]
		\end{tikzcd}\]
		even though the square here is a pullback square. We conclude that $\varphi'\circ\pi=\pi_C$ is forced.
		\qedhere
	\end{itemize}
\end{proof}
\begin{lemma} \label{lem:smalls-to-big-pullback}
	Suppose that the commutative diagram
	% https://q.uiver.app/?q=WzAsNixbMCwwLCJBIl0sWzEsMCwiQiJdLFswLDEsIkMiXSxbMSwxLCJEIl0sWzAsMiwiRSJdLFsxLDIsIkYiXSxbMiw0LCJcXHZhcnBoaSIsMl0sWzQsNSwiXFxpb3RhIl0sWzIsMywiXFxpb3RhJyJdLFszLDUsIlxccGkiXSxbMCwyLCJcXHZhcnBoaSciLDJdLFsxLDMsIlxccGknIl0sWzAsMSwiXFxpb3RhJyciXV0=&macro_url=https%3A%2F%2Fraw.githubusercontent.com%2FdFoiler%2Fnotes%2Fmaster%2Fnir.tex
	\[\begin{tikzcd}
		A & B \\
		C & D \\
		E & F
		\arrow["\varphi"', from=2-1, to=3-1]
		\arrow["\iota", from=3-1, to=3-2]
		\arrow["{\iota'}", from=2-1, to=2-2]
		\arrow["\pi", from=2-2, to=3-2]
		\arrow["{\varphi'}"', from=1-1, to=2-1]
		\arrow["{\pi'}", from=1-2, to=2-2]
		\arrow["{\iota''}", from=1-1, to=1-2]
	\end{tikzcd}\]
	has the two square as both pullback squares. Then the big rectangle is a pullback square.
\end{lemma}
\begin{proof}
	We proceed by force. Fix some object $Z$ with morphisms $\psi_B\colon Z\to B$ and $\psi_E\colon Z\to E$ such that $\pi\circ\pi'\circ\psi_B=\iota\circ\psi_E$. We need a unique morphism $\psi\colon Z\to A$ making the diagram
	% https://q.uiver.app/?q=WzAsNSxbMiwyLCJGIl0sWzIsMSwiQiJdLFsxLDIsIkUiXSxbMSwxLCJBIl0sWzAsMCwiWiJdLFs0LDMsIlxccHNpIiwxLHsic3R5bGUiOnsiYm9keSI6eyJuYW1lIjoiZGFzaGVkIn19fV0sWzQsMSwiXFxwc2lfQiIsMCx7ImN1cnZlIjotMn1dLFs0LDIsIlxccHNpX0UiLDIseyJjdXJ2ZSI6Mn1dLFszLDIsIlxcdmFycGhpXFxjaXJjXFx2YXJwaGknIl0sWzMsMSwiXFxpb3RhJyciLDJdLFsyLDAsIlxcaW90YSIsMl0sWzEsMCwiXFxwaVxcY2lyY1xccGknIl1d
	\begin{equation}
		\begin{tikzcd}
			Z \\
			& A & B \\
			& E & F
			\arrow["\psi"{description}, dashed, from=1-1, to=2-2]
			\arrow["{\psi_B}", curve={height=-12pt}, from=1-1, to=2-3]
			\arrow["{\psi_E}"', curve={height=12pt}, from=1-1, to=3-2]
			\arrow["{\varphi\circ\varphi'}", from=2-2, to=3-2]
			\arrow["{\iota''}"', from=2-2, to=2-3]
			\arrow["\iota"', from=3-2, to=3-3]
			\arrow["{\pi\circ\pi'}", from=2-3, to=3-3]
		\end{tikzcd} \label{eq:stacked-pullbacks}
	\end{equation}
	commute. We show uniqueness and existence separately.
	\begin{itemize}
		\item Uniqueness: set $\psi_D\coloneqq\pi'\circ\psi_B$ so that $\varphi'\circ\psi$ makes the diagram
		% https://q.uiver.app/?q=WzAsNSxbMSwxLCJDIl0sWzIsMSwiRCJdLFsxLDIsIkUiXSxbMiwyLCJGIl0sWzAsMCwiWiJdLFsyLDMsIlxcaW90YSIsMl0sWzEsMywiXFxwaSJdLFswLDEsIlxcaW90YSciLDJdLFswLDIsIlxcdmFycGhpIl0sWzQsMCwiXFx2YXJwaGknXFxjaXJjXFxwaSIsMV0sWzQsMSwiXFxwaV9EIiwwLHsiY3VydmUiOi0yfV0sWzQsMiwiXFxwaV9FIiwyLHsiY3VydmUiOjJ9XV0=&macro_url=https%3A%2F%2Fraw.githubusercontent.com%2FdFoiler%2Fnotes%2Fmaster%2Fnir.tex
		\[\begin{tikzcd}
			Z \\
			& C & D \\
			& E & F
			\arrow["\iota"', from=3-2, to=3-3]
			\arrow["\pi", from=2-3, to=3-3]
			\arrow["{\iota'}"', from=2-2, to=2-3]
			\arrow["\varphi", from=2-2, to=3-2]
			\arrow["{\varphi'\circ\psi}"{description}, from=1-1, to=2-2]
			\arrow["{\psi_D}", curve={height=-12pt}, from=1-1, to=2-3]
			\arrow["{\psi_E}"', curve={height=12pt}, from=1-1, to=3-2]
		\end{tikzcd}\]
		commute, so this pullback square tells us that $\psi_C\coloneqq\varphi'\circ\psi$ is unique. Continuing, we see that
		\[\iota'\circ\psi_C=\iota'\circ\varphi'\circ\psi=\pi'\circ\iota''\circ\psi=\pi'\circ\psi_B,\]
		so we see that the diagram
		% https://q.uiver.app/?q=WzAsNSxbMCwwLCJaIl0sWzEsMSwiQSJdLFsyLDEsIkIiXSxbMSwyLCJDIl0sWzIsMiwiRCJdLFswLDEsIlxccHNpIiwxXSxbMCwyLCJcXHBzaV9CIiwwLHsiY3VydmUiOi0yfV0sWzAsMywiXFxwc2lfQyIsMix7ImN1cnZlIjoyfV0sWzEsMywiXFx2YXJwaGknIl0sWzEsMiwiXFxpb3RhJyciLDJdLFsyLDQsIlxccGknIl0sWzMsNCwiXFxpb3RhJyIsMl1d&macro_url=https%3A%2F%2Fraw.githubusercontent.com%2FdFoiler%2Fnotes%2Fmaster%2Fnir.tex
		\[\begin{tikzcd}
			Z \\
			& A & B \\
			& C & D
			\arrow["\psi"{description}, from=1-1, to=2-2]
			\arrow["{\psi_B}", curve={height=-12pt}, from=1-1, to=2-3]
			\arrow["{\psi_C}"', curve={height=12pt}, from=1-1, to=3-2]
			\arrow["{\varphi'}", from=2-2, to=3-2]
			\arrow["{\iota''}"', from=2-2, to=2-3]
			\arrow["{\pi'}", from=2-3, to=3-3]
			\arrow["{\iota'}"', from=3-2, to=3-3]
		\end{tikzcd}\]
		commutes. However, $\psi_C$ and $\psi_B$ are both uniquely determined, so we see that the morphism $\psi$ is thus also uniquely determined.
		\item Existence: we unwind the above proof. Because \autoref{eq:stacked-pullbacks} commutes, we set $\psi_D\coloneqq\pi'\circ\psi_B$ to make the diagram
		% https://q.uiver.app/?q=WzAsNSxbMSwxLCJDIl0sWzIsMSwiRCJdLFsxLDIsIkUiXSxbMiwyLCJGIl0sWzAsMCwiWiJdLFsyLDMsIlxcaW90YSIsMl0sWzEsMywiXFxwaSJdLFswLDEsIlxcaW90YSciLDJdLFswLDIsIlxcdmFycGhpIl0sWzQsMCwiIiwxLHsic3R5bGUiOnsiYm9keSI6eyJuYW1lIjoiZGFzaGVkIn19fV0sWzQsMSwiXFxwc2lfRCIsMCx7ImN1cnZlIjotMn1dLFs0LDIsIlxccHNpX0UiLDIseyJjdXJ2ZSI6Mn1dXQ==&macro_url=https%3A%2F%2Fraw.githubusercontent.com%2FdFoiler%2Fnotes%2Fmaster%2Fnir.tex
		\[\begin{tikzcd}
			Z \\
			& C & D \\
			& E & F
			\arrow["\iota"', from=3-2, to=3-3]
			\arrow["\pi", from=2-3, to=3-3]
			\arrow["{\iota'}"', from=2-2, to=2-3]
			\arrow["\varphi", from=2-2, to=3-2]
			\arrow[dashed, from=1-1, to=2-2]
			\arrow["{\psi_D}", curve={height=-12pt}, from=1-1, to=2-3]
			\arrow["{\psi_E}"', curve={height=12pt}, from=1-1, to=3-2]
		\end{tikzcd}\]
		commute, so the fact that we have a pullback square induces a unique morphism $\psi_C\colon Z\to C$ making the above diagram commute. In particular, $\iota'\circ\psi_C=\psi_D=\pi'\circ\psi_B$ by construction of $\psi_C$, so the diagram
		% https://q.uiver.app/?q=WzAsNSxbMCwwLCJaIl0sWzEsMSwiQSJdLFsyLDEsIkIiXSxbMSwyLCJDIl0sWzIsMiwiRCJdLFswLDEsIlxccHNpIiwxLHsic3R5bGUiOnsiYm9keSI6eyJuYW1lIjoiZGFzaGVkIn19fV0sWzAsMiwiXFxwc2lfQiIsMCx7ImN1cnZlIjotMn1dLFswLDMsIlxccHNpX0MiLDIseyJjdXJ2ZSI6Mn1dLFsxLDMsIlxcdmFycGhpJyJdLFsxLDIsIlxcaW90YScnIiwyXSxbMiw0LCJcXHBpJyJdLFszLDQsIlxcaW90YSciLDJdXQ==&macro_url=https%3A%2F%2Fraw.githubusercontent.com%2FdFoiler%2Fnotes%2Fmaster%2Fnir.tex
		\[\begin{tikzcd}
			Z \\
			& A & B \\
			& C & D
			\arrow[dashed, from=1-1, to=2-2]
			\arrow["{\psi_B}", curve={height=-12pt}, from=1-1, to=2-3]
			\arrow["{\psi_C}"', curve={height=12pt}, from=1-1, to=3-2]
			\arrow["{\varphi'}", from=2-2, to=3-2]
			\arrow["{\iota''}"', from=2-2, to=2-3]
			\arrow["{\pi'}", from=2-3, to=3-3]
			\arrow["{\iota'}"', from=3-2, to=3-3]
		\end{tikzcd}\]
		commutes, thus inducing a unique morphism $\psi\colon Z\to C$ making the above diagram commute. In particular, we have $\iota''\circ\psi=\psi_B$ and $\varphi\circ\varphi'\circ\psi=\varphi\circ\psi_C=\psi_E$, which is what we wanted.
		\qedhere
	\end{itemize}
\end{proof}

\subsection{Fiber Products: Easy Cases}
We will now start marching toward a proof of \autoref{thm:fibexist}. Schemes are made of affine schemes, so we will begin with affine schemes, with the hope of patching these together later.
\begin{lemma} \label{lem:affine-fp}
	Fix affine schemes $X$, $Y$, and $S$, with $X=\Spec A$ and $Y=\Spec B$ and $S=\Spec R$. Then we may set
	\[X\times_SY=\Spec A\otimes_RB,\]
	where the canonical projections $X\times_SY\to X$ and $X\times_SY\to Z$ are induced by the canonical inclusions $\iota_A\colon A\to A\otimes_RB$ and $\iota_B\colon B\to A\otimes_RB$.
\end{lemma}
\begin{proof}
	Let $f_A\colon R\to A$ and $f_B\colon R\to B$ be the maps associated to the maps $\psi_A\colon X\to S$ and $\psi_B\colon Y\to S$. Now, for some scheme $Z$ over $S$ by $\psi_Z$, and we see that
	\[\op{Mor}_S(Z,\Spec A\otimes_RB)\simeq\op{Hom}_R(A\otimes_RB,\OO_Z(Z))\]
	by \autoref{thm:biggeoisalgopp}. Now, it is a fact of commutative algebra\footnote{Namely, the tensor product is the fiber coproduct.} that $R$-algebras maps $\varphi\colon A\otimes_RS\to\OO_Z(Z)$ are in bijection with pairs of $R$-algebra maps $\varphi_A\colon A\to\OO_Z(Z)$ and $\varphi_B\colon B\to\OO_Z(Z)$ such that $\varphi_A=\varphi\circ\iota_A$ and $\varphi_B=\varphi\circ\iota_B$. In other words, we have a natural isomorphism
	\[\op{Mor}_S(Z,\Spec A\otimes_RB)\simeq\op{Hom}_R(A,\OO_Z(Z))\times_{\op{Hom}_R(R,\OO_Z(Z))}\op{Hom}_R(B,\OO_Z(Z)).\]
	Applying the adjunction again, we see
	\[\op{Mor}_S(Z,\Spec A\otimes_RB)\simeq\op{Mor}_S(Z,X)\times_{\op{Mor}_S(Z,S)}\op{Mor}_S(Z,Y),\]
	which finishes by \autoref{lem:repr-fp}.
	% Notably, the canonical maps $X\to S$ and $Y\to S$ makes $A$ and $B$ into $R$-algebras. To see this, we simply compute, for some scheme $S$,
	% \begin{align*}
	% 	\op{Mor}_S(Z,\Spec A\otimes_RS) &\simeq \op{Hom}_R(A\otimes_RS,\OO_Z(Z)) \\
	% 	&\simeq \op{Hom}_R(A,\OO_Z(Z)) \times_{\op{Hom}_R(R,\OO_Z(Z))} \op{Hom}_R(B,\OO_Z(Z)),
	% \end{align*}
	% which unravels into the correct thing. This finishes by the Yoneda lemma. One can write out all of this as some diagram-chase, but it is equivalent to the above computation.
\end{proof}
\begin{remark}
	One can view the above proof as basically preserving the fact that $A\otimes_RB$ is the fiber coproduct of $A$ and $B$ as $R$-algebras.
\end{remark}
We are also going to want to take bigger fiber products and find small ones inside them to be able to be glue them properly. For this, we note that open subschemes induce pullbacks.
\begin{lemma} \label{lem:open-fp}
	Fix a scheme morphism $\varphi\colon X\to Y$. Then, for any open $U\subseteq Y$, the square
	% https://q.uiver.app/?q=WzAsNCxbMSwwLCJYIl0sWzEsMSwiWSJdLFswLDAsIlxcdmFycGhpXnstMX1VIl0sWzAsMSwiVSJdLFswLDEsIlxcdmFycGhpIl0sWzIsMywiXFx2YXJwaGl8X3tcXHZhcnBoaV57LTF9VX0iLDJdLFsyLDAsIiIsMCx7InN0eWxlIjp7InRhaWwiOnsibmFtZSI6Imhvb2siLCJzaWRlIjoidG9wIn19fV0sWzMsMSwiIiwyLHsic3R5bGUiOnsidGFpbCI6eyJuYW1lIjoiaG9vayIsInNpZGUiOiJ0b3AifX19XV0=&macro_url=https%3A%2F%2Fraw.githubusercontent.com%2FdFoiler%2Fnotes%2Fmaster%2Fnir.tex
	\[\begin{tikzcd}
		{\varphi^{-1}U} & X \\
		U & Y
		\arrow["\varphi", from=1-2, to=2-2]
		\arrow["{\varphi|_{\varphi^{-1}U}}"', from=1-1, to=2-1]
		\arrow[hook, from=1-1, to=1-2]
		\arrow[hook, from=2-1, to=2-2]
	\end{tikzcd}\]
	is a pullback square.
\end{lemma}
\begin{proof}
	Label our maps as follows.
	% https://q.uiver.app/?q=WzAsNCxbMSwwLCJYIl0sWzEsMSwiWSJdLFswLDAsIlxcdmFycGhpXnstMX1VIl0sWzAsMSwiVSJdLFswLDEsIlxcdmFycGhpIl0sWzIsMywiXFx2YXJwaGl8X3tcXHZhcnBoaV57LTF9VX0iLDJdLFsyLDAsIlxcaW90YSIsMCx7InN0eWxlIjp7InRhaWwiOnsibmFtZSI6Imhvb2siLCJzaWRlIjoidG9wIn19fV0sWzMsMSwiXFxqbWF0aCIsMCx7InN0eWxlIjp7InRhaWwiOnsibmFtZSI6Imhvb2siLCJzaWRlIjoidG9wIn19fV1d&macro_url=https%3A%2F%2Fraw.githubusercontent.com%2FdFoiler%2Fnotes%2Fmaster%2Fnir.tex
	\[\begin{tikzcd}
		{\varphi^{-1}U} & X \\
		U & Y
		\arrow["\varphi", from=1-2, to=2-2]
		\arrow["{\varphi|_{\varphi^{-1}U}}"', from=1-1, to=2-1]
		\arrow["\iota", hook, from=1-1, to=1-2]
		\arrow["\jmath", hook, from=2-1, to=2-2]
	\end{tikzcd}\]
	Observe that the right arrow of the diagram is induced as $(\varphi,\varphi^\sharp)|_U$ by \autoref{lem:restrictmorphism}. The horizontal arrows are the open embeddings of \autoref{ex:open-embed-morphism}, and the diagram commutes by \autoref{rem:better-restriction}.

	It remains to show the universal property. Suppose that $Z$ is a scheme with morphisms $\psi_X\colon Z\to X$ and $\psi_U\colon Z\to U$ such that $\varphi\circ\psi_X=\jmath\circ\psi_U$. We need a unique scheme morphism $\psi\colon Z\to\varphi^{-1}U$ making the diagram
	% https://q.uiver.app/?q=WzAsNSxbMiwxLCJYIl0sWzIsMiwiWSJdLFsxLDEsIlxcdmFycGhpXnstMX1VIl0sWzEsMiwiVSJdLFswLDAsIloiXSxbMCwxLCJcXHZhcnBoaSJdLFsyLDMsIlxcdmFycGhpfF97XFx2YXJwaGleey0xfVV9Il0sWzIsMCwiXFxpb3RhIiwyLHsic3R5bGUiOnsidGFpbCI6eyJuYW1lIjoiaG9vayIsInNpZGUiOiJ0b3AifX19XSxbMywxLCJcXGptYXRoIiwyLHsic3R5bGUiOnsidGFpbCI6eyJuYW1lIjoiaG9vayIsInNpZGUiOiJ0b3AifX19XSxbNCwwLCJcXHBzaV9YIiwwLHsiY3VydmUiOi0yfV0sWzQsMywiXFxwc2lfWSIsMix7ImN1cnZlIjoyfV0sWzQsMiwiXFxwc2kiLDEseyJzdHlsZSI6eyJib2R5Ijp7Im5hbWUiOiJkYXNoZWQifX19XV0=&macro_url=https%3A%2F%2Fraw.githubusercontent.com%2FdFoiler%2Fnotes%2Fmaster%2Fnir.tex
	\begin{equation}
		\begin{tikzcd}
			Z \\
			& {\varphi^{-1}U} & X \\
			& U & Y
			\arrow["\varphi", from=2-3, to=3-3]
			\arrow["{\varphi|_{\varphi^{-1}U}}", from=2-2, to=3-2]
			\arrow["\iota"', hook, from=2-2, to=2-3]
			\arrow["\jmath"', hook, from=3-2, to=3-3]
			\arrow["{\psi_X}", curve={height=-12pt}, from=1-1, to=2-3]
			\arrow["{\psi_U}"', curve={height=12pt}, from=1-1, to=3-2]
			\arrow["\psi"{description}, dashed, from=1-1, to=2-2]
		\end{tikzcd} \label{eq:open-embed-fp}
	\end{equation}
	commute. We show uniqueness and existence separately.
	\begin{itemize}
		\item Uniqueness: on topological spaces, we require any $z\in Z$ to $\psi(z)=\iota(\psi(z))=\psi_X(z)$, so $\psi$ is uniquely determined topologically. On sheaves, we note that we need the diagram
		% https://q.uiver.app/?q=WzAsMyxbMCwwLCJcXE9PX1giXSxbMSwwLCJcXGlvdGFfKihcXE9PX1h8X3tcXHZhcnBoaV57LTF9VX0pIl0sWzEsMSwiXFxPT19aIl0sWzAsMSwiXFxpb3RhXlxcc2hhcnAiXSxbMCwyLCJcXHBzaV5cXHNoYXJwX1giLDJdLFsxLDIsIlxcaW90YV8qXFxwc2leXFxzaGFycCJdXQ==&macro_url=https%3A%2F%2Fraw.githubusercontent.com%2FdFoiler%2Fnotes%2Fmaster%2Fnir.tex
		\[\begin{tikzcd}
			{\OO_X} & {\iota_*(\OO_X|_{\varphi^{-1}U})} \\
			& {\OO_Z}
			\arrow["{\iota^\sharp}", from=1-1, to=1-2]
			\arrow["{\psi^\sharp_X}"', from=1-1, to=2-2]
			\arrow["{\iota_*\psi^\sharp}", from=1-2, to=2-2]
		\end{tikzcd}\]
		to commute. In particular, for open subset $V\subseteq\varphi^{-1}U$, we see that $\iota^\sharp_V=\op{res}_{V,\varphi^{-1}UV}$ is just the identity, so we are asking for the diagram
		% https://q.uiver.app/?q=WzAsMyxbMCwwLCJcXE9PX1goVikiXSxbMSwwLCJcXE9PX1goVikiXSxbMSwxLCJcXE9PX1oiXSxbMCwxLCJcXGlvdGFeXFxzaGFycF9WIiwwLHsibGV2ZWwiOjIsInN0eWxlIjp7ImhlYWQiOnsibmFtZSI6Im5vbmUifX19XSxbMCwyLCIoXFxwc2leXFxzaGFycF9YKV9WIiwyXSxbMSwyLCJcXHBzaV5cXHNoYXJwX1YiXV0=&macro_url=https%3A%2F%2Fraw.githubusercontent.com%2FdFoiler%2Fnotes%2Fmaster%2Fnir.tex
		\[\begin{tikzcd}
			{\OO_X(V)} & {\OO_X(V)} \\
			& {\OO_Z}
			\arrow["{\iota^\sharp_V}", Rightarrow, no head, from=1-1, to=1-2]
			\arrow["{(\psi^\sharp_X)_V}"', from=1-1, to=2-2]
			\arrow["{\psi^\sharp_V}", from=1-2, to=2-2]
		\end{tikzcd}\]
		to commute, which we can see forces $\psi_V$.
		\item Existence: we follow the above formula. Define $\psi(z)\coloneqq\psi_X(z)\in X$. Notably, this makes sense because $\varphi(\psi_X(z))=\psi_Y(z)\in U$, so $\psi_X(z)\in\varphi^{-1}U$. Now, \autoref{eq:open-embed-fp} commutes on topological spaces by tracking everything through:
		\[\iota(\psi(z))=\psi(z)=\psi_X(z)\qquad\text{and}\qquad\varphi(\psi(z))=\varphi(\psi_Z(z))=\psi_Y(z).\]
		On sheaves, for any open $V\subseteq\varphi^{-1}U$, we need a map $\psi_V\colon\OO_{\varphi^{-1}U}(V)\to\psi_*\OO_Z(V)$, but $\OO_{\varphi^{-1}U}(V)=\OO_X(\varphi^{-1}U\cap V)$ and $\psi_*\OO_Z(V)=\OO_Z(\psi^{-1}V)=\OO_Z(\psi_X^{-1}V)$. However, we note that $\im\psi_X\subseteq\varphi^{-1}U$ as discussed previously, so we define our map as the composite
		\[\OO_{\varphi^{-1}U}(V)=\OO_X(\varphi^{-1}U\cap V)\stackrel{(\psi_X^\sharp)_{\varphi^{-1}U\cap V}}\to(\psi_X)_*\OO_Z(\varphi^{-1}U\cap V)=\OO_Z(\psi_X^{-1}V).\]
		We now run the necessary checks.
		\begin{itemize}
			\item Sheaf morphism: given open subsets $V'\subseteq V$, we see that the diagram
			% https://q.uiver.app/?q=WzAsOCxbMCwwLCJcXE9PX3tcXHZhcnBoaV57LTF9VX0oVikiXSxbMCwxLCJcXE9PX3tcXHZhcnBoaV57LTF9VX0oVicpIl0sWzEsMCwiXFxPT19YKFxcdmFycGhpXnstMX1VXFxjYXAgVikiXSxbMSwxLCJcXE9PX1goXFx2YXJwaGleey0xfVVcXGNhcCBWJykiXSxbMiwwLCIoXFxwc2lfWClfKlxcT09fWihcXHZhcnBoaV57LTF9VVxcY2FwIFYpIl0sWzIsMSwiKFxccHNpX1gpXypcXE9PX1ooXFx2YXJwaGleey0xfVVcXGNhcCBWJykiXSxbMywwLCJcXE9PX1ooXFxwc2leey0xfV9YVikiXSxbMywxLCJcXE9PX1ooXFxwc2leey0xfV9YVicpIl0sWzAsMiwiIiwwLHsibGV2ZWwiOjIsInN0eWxlIjp7ImhlYWQiOnsibmFtZSI6Im5vbmUifX19XSxbMSwzLCIiLDAseyJsZXZlbCI6Miwic3R5bGUiOnsiaGVhZCI6eyJuYW1lIjoibm9uZSJ9fX1dLFswLDEsIlxcb3B7cmVzfSIsMl0sWzIsMywiXFxvcHtyZXN9IiwyXSxbNCw1LCJcXG9we3Jlc30iLDJdLFsyLDQsIihcXHBzaV9YXlxcc2hhcnApX3tcXHZhcnBoaV57LTF9VVxcY2FwIFZ9Il0sWzMsNSwiKFxccHNpX1heXFxzaGFycClfe1xcdmFycGhpXnstMX1VXFxjYXAgVid9Il0sWzYsNywiXFxvcHtyZXN9Il0sWzQsNiwiIiwwLHsibGV2ZWwiOjIsInN0eWxlIjp7ImhlYWQiOnsibmFtZSI6Im5vbmUifX19XSxbNSw3LCIiLDAseyJsZXZlbCI6Miwic3R5bGUiOnsiaGVhZCI6eyJuYW1lIjoibm9uZSJ9fX1dXQ==&macro_url=https%3A%2F%2Fraw.githubusercontent.com%2FdFoiler%2Fnotes%2Fmaster%2Fnir.tex
			\[\begin{tikzcd}
				{\OO_{\varphi^{-1}U}(V)} & {\OO_X(\varphi^{-1}U\cap V)} & {(\psi_X)_*\OO_Z(\varphi^{-1}U\cap V)} & {\OO_Z(\psi^{-1}_XV)} \\
				{\OO_{\varphi^{-1}U}(V')} & {\OO_X(\varphi^{-1}U\cap V')} & {(\psi_X)_*\OO_Z(\varphi^{-1}U\cap V')} & {\OO_Z(\psi^{-1}_XV')}
				\arrow[Rightarrow, no head, from=1-1, to=1-2]
				\arrow[Rightarrow, no head, from=2-1, to=2-2]
				\arrow["{\op{res}}"', from=1-1, to=2-1]
				\arrow["{\op{res}}"', from=1-2, to=2-2]
				\arrow["{\op{res}}"', from=1-3, to=2-3]
				\arrow["{(\psi_X^\sharp)_{\varphi^{-1}U\cap V}}", from=1-2, to=1-3]
				\arrow["{(\psi_X^\sharp)_{\varphi^{-1}U\cap V'}}", from=2-2, to=2-3]
				\arrow["{\op{res}}", from=1-4, to=2-4]
				\arrow[Rightarrow, no head, from=1-3, to=1-4]
				\arrow[Rightarrow, no head, from=2-3, to=2-4]
			\end{tikzcd}\]
			commutes, where the only square which isn't made of horizontal identities is a naturality square for $\psi^\sharp_X$.
			\item Morphism of locally ringed spaces: this is essentially inherited directly from $\psi^\sharp_X$. Given $z\in Z$, we need to know that the composite
			\[\arraycolsep=1.4pt\begin{array}{cccccc}
				\OO_{\varphi^{-1}U,\psi(z)} &\stackrel{\psi^\sharp_{\psi(z)}}\to& (\psi_*\OO_Z)_{\psi(z)} &\to& \OO_{Z,z} \\
				{[(V,s)]} &\mapsto& [(V,(\psi_X^\sharp)_{\varphi^{-1}U\cap V}(s))] &\mapsto& (\psi_X^\sharp)_{\varphi^{-1}U\cap V}(s)|_z
			\end{array}\]
			is a map of local rings. Well, we note that any $[(V,s)]\in\OO_{\varphi^{-1}U,\psi(z)}$ is canonically also a germ in $\OO_{X,\psi(z)}$, and the fact that $\psi_Z$ is a morphism of locally ringed spaces tells us $(\psi_X^\sharp)_{\varphi^{-1}U\cap V}(s)|_z\in\mf m_{Z,z}$, which is what we wanted.
			\item Commutes: we already checked that the needed diagram \autoref{eq:open-embed-fp} commutes on the level of topological spaces, so we just need to check that it commutes on the level of sheaves. This has two checks.
			\begin{itemize}
				\item We verify that
				% https://q.uiver.app/?q=WzAsMyxbMCwwLCJcXE9PX1UiXSxbMSwwLCJcXHZhcnBoaV8qXFxPT197XFx2YXJwaGleey0xfVV9Il0sWzEsMSwiKFxccHNpX1UpXypcXE9PX1oiXSxbMCwyLCJcXHBzaV9VXlxcc2hhcnAiLDJdLFswLDEsIlxcdmFycGhpXlxcc2hhcnAiXSxbMSwyLCJcXHBzaV5cXHNoYXJwIl1d&macro_url=https%3A%2F%2Fraw.githubusercontent.com%2FdFoiler%2Fnotes%2Fmaster%2Fnir.tex
				\[\begin{tikzcd}
					{\OO_U} & {\varphi_*\OO_{\varphi^{-1}U}} \\
					& {(\psi_U)_*\OO_Z}
					\arrow["{\psi_U^\sharp}"', from=1-1, to=2-2]
					\arrow["{\varphi^\sharp}", from=1-1, to=1-2]
					\arrow["{\varphi_*\psi^\sharp}", from=1-2, to=2-2]
				\end{tikzcd}\]
				commutes. Indeed, for any open subset $V\subseteq U$, we could verify that
				% https://q.uiver.app/?q=WzAsNixbMCwwLCJcXE9PX1UoVikiXSxbMSwwLCJcXE9PX3tcXHZhcnBoaV57LTF9VX0oXFx2YXJwaGleey0xfVVcXGNhcCBWKSJdLFsxLDEsIlxcT09fWihcXHBzaV57LTF9VikiXSxbMiwwLCJzIl0sWzMsMCwiXFx2YXJwaGleXFxzaGFycF9WKHMpIl0sWzMsMSwiKFxccHNpXlxcc2hhcnBfWClfe1xcdmFycGhpXnstMX1VXFxjYXAgVn0oXFx2YXJwaGleXFxzaGFycF9WcykiXSxbMCwyLCIoXFxwc2lfVV5cXHNoYXJwKV9WIiwyXSxbMCwxLCJcXHZhcnBoaV5cXHNoYXJwX1YiXSxbMSwyLCJcXHBzaV5cXHNoYXJwX3tcXHZhcnBoaV57LTF9VVxcY2FwIFZ9Il0sWzMsNCwiIiwwLHsic3R5bGUiOnsidGFpbCI6eyJuYW1lIjoibWFwcyB0byJ9fX1dLFs0LDUsIiIsMCx7InN0eWxlIjp7InRhaWwiOnsibmFtZSI6Im1hcHMgdG8ifX19XSxbMyw1LCIiLDIseyJzdHlsZSI6eyJ0YWlsIjp7Im5hbWUiOiJtYXBzIHRvIn19fV1d&macro_url=https%3A%2F%2Fraw.githubusercontent.com%2FdFoiler%2Fnotes%2Fmaster%2Fnir.tex
				\[\begin{tikzcd}
					{\OO_U(V)} & {\OO_{\varphi^{-1}U}(\varphi^{-1}U\cap V)} & s & {\varphi^\sharp_V(s)} \\
					& {\OO_Z(\psi^{-1}V)} && {(\psi^\sharp_X)_{\varphi^{-1}U\cap V}(\varphi^\sharp_Vs)}
					\arrow["{(\psi_U^\sharp)_V}"', from=1-1, to=2-2]
					\arrow["{\varphi^\sharp_V}", from=1-1, to=1-2]
					\arrow["{\psi^\sharp_{\varphi^{-1}U\cap V}}", from=1-2, to=2-2]
					\arrow[maps to, from=1-3, to=1-4]
					\arrow[maps to, from=1-4, to=2-4]
					\arrow[maps to, from=1-3, to=2-4]
				\end{tikzcd}\]
				commutes appropriately using the given commuting square.
				\item We verify that 
				% https://q.uiver.app/?q=WzAsMyxbMCwwLCJcXE9PX1giXSxbMSwwLCJcXGlvdGFfKlxcT09fe1xcdmFycGhpXnstMX1VfSJdLFsxLDEsIihcXHBzaV9YKV8qXFxPT19aIl0sWzAsMiwiKFxccHNpX1heXFxzaGFycCkiLDJdLFswLDEsIlxcaW90YV5cXHNoYXJwIl0sWzEsMiwiXFxpb3RhXypcXHBzaV5cXHNoYXJwIl1d&macro_url=https%3A%2F%2Fraw.githubusercontent.com%2FdFoiler%2Fnotes%2Fmaster%2Fnir.tex
				\[\begin{tikzcd}
					{\OO_X} & {\iota_*\OO_{\varphi^{-1}U}} \\
					& {(\psi_X)_*\OO_Z}
					\arrow["{(\psi_X^\sharp)}"', from=1-1, to=2-2]
					\arrow["{\iota^\sharp}", from=1-1, to=1-2]
					\arrow["{\iota_*\psi^\sharp}", from=1-2, to=2-2]
				\end{tikzcd}\]
				commutes. Indeed, for any open subset $V\subseteq U$, we could verify that
				% https://q.uiver.app/?q=WzAsMyxbMCwwLCJcXE9PX1goVikiXSxbMSwwLCJcXE9PX3tcXHZhcnBoaV57LTF9VX0oXFx2YXJwaGleey0xfVVcXGNhcCBWKSJdLFsxLDEsIlxcT09fWihcXHBzaV9YXnstMX1WKSJdLFswLDIsIihcXHBzaV9YXlxcc2hhcnApX1YiLDJdLFswLDEsIlxcaW90YV5cXHNoYXJwX1YiXSxbMSwyLCJcXHBzaV5cXHNoYXJwX3tcXHZhcnBoaV57LTF9VVxcY2FwIFZ9Il1d&macro_url=https%3A%2F%2Fraw.githubusercontent.com%2FdFoiler%2Fnotes%2Fmaster%2Fnir.tex
				\[\begin{tikzcd}
					{\OO_X(V)} & {\OO_{\varphi^{-1}U}(\varphi^{-1}U\cap V)} \\
					& {\OO_Z(\psi_X^{-1}V)}
					\arrow["{(\psi_X^\sharp)_V}"', from=1-1, to=2-2]
					\arrow["{\iota^\sharp_V}", from=1-1, to=1-2]
					\arrow["{\psi^\sharp_{\varphi^{-1}U\cap V}}", from=1-2, to=2-2]
				\end{tikzcd}\]
				commutes directly from the construction of $\psi^\sharp$.
			\end{itemize}
		\end{itemize}
	\end{itemize}
	The above checks complete the proof.
\end{proof}
\begin{corollary}
	Let $X$ be a scheme and $\iota\colon U\to X$ be an open subscheme. Then $\iota$ is a monic morphism of schemes.
\end{corollary}
\begin{proof}
	\autoref{lem:open-fp} tells us that the square
	% https://q.uiver.app/?q=WzAsNCxbMCwwLCJVIl0sWzEsMCwiWCJdLFswLDEsIlUiXSxbMSwxLCJYIl0sWzAsMSwiXFxpb3RhIiwwLHsic3R5bGUiOnsidGFpbCI6eyJuYW1lIjoiaG9vayIsInNpZGUiOiJ0b3AifX19XSxbMSwzLCIiLDAseyJsZXZlbCI6Miwic3R5bGUiOnsiaGVhZCI6eyJuYW1lIjoibm9uZSJ9fX1dLFswLDIsIiIsMix7ImxldmVsIjoyLCJzdHlsZSI6eyJoZWFkIjp7Im5hbWUiOiJub25lIn19fV0sWzIsMywiXFxpb3RhIiwwLHsic3R5bGUiOnsidGFpbCI6eyJuYW1lIjoiaG9vayIsInNpZGUiOiJ0b3AifX19XV0=&macro_url=https%3A%2F%2Fraw.githubusercontent.com%2FdFoiler%2Fnotes%2Fmaster%2Fnir.tex
	\[\begin{tikzcd}
		U & X \\
		U & X
		\arrow["\iota", hook, from=1-1, to=1-2]
		\arrow[Rightarrow, no head, from=1-2, to=2-2]
		\arrow[Rightarrow, no head, from=1-1, to=2-1]
		\arrow["\iota", hook, from=2-1, to=2-2]
	\end{tikzcd}\]
	is a pullback square. Technically, one should check that $\iota|_U=\iota$, but our restriction is functorial enough for this to be okay. (Plugging in the identity makes all of our constructions trivialize.)

	As such, suppose we have a scheme $Z$ with two morphisms $\varphi,\varphi'\colon Z\to U$ such that $\iota\circ\varphi=\iota\circ\varphi'$. Then note that both $\varphi$ and $\varphi'$ can fill in the dashed arrow of the diagram
	% https://q.uiver.app/?q=WzAsNSxbMSwxLCJVIl0sWzIsMSwiWCJdLFsxLDIsIlUiXSxbMiwyLCJYIl0sWzAsMCwiWiJdLFswLDEsIlxcaW90YSIsMCx7InN0eWxlIjp7InRhaWwiOnsibmFtZSI6Imhvb2siLCJzaWRlIjoidG9wIn19fV0sWzEsMywiIiwwLHsibGV2ZWwiOjIsInN0eWxlIjp7ImhlYWQiOnsibmFtZSI6Im5vbmUifX19XSxbMCwyLCIiLDIseyJsZXZlbCI6Miwic3R5bGUiOnsiaGVhZCI6eyJuYW1lIjoibm9uZSJ9fX1dLFsyLDMsIlxcaW90YSIsMCx7InN0eWxlIjp7InRhaWwiOnsibmFtZSI6Imhvb2siLCJzaWRlIjoidG9wIn19fV0sWzQsMiwiXFx2YXJwaGkiLDIseyJjdXJ2ZSI6Mn1dLFs0LDEsIlxcaW90YVxcY2lyY1xcdmFycGhpIiwwLHsiY3VydmUiOi0yfV0sWzQsMCwiIiwxLHsic3R5bGUiOnsiYm9keSI6eyJuYW1lIjoiZGFzaGVkIn19fV1d&macro_url=https%3A%2F%2Fraw.githubusercontent.com%2FdFoiler%2Fnotes%2Fmaster%2Fnir.tex
	\[\begin{tikzcd}
		Z \\
		& U & X \\
		& U & X
		\arrow["\iota", hook, from=2-2, to=2-3]
		\arrow[Rightarrow, no head, from=2-3, to=3-3]
		\arrow[Rightarrow, no head, from=2-2, to=3-2]
		\arrow["\iota", hook, from=3-2, to=3-3]
		\arrow["\varphi"', curve={height=12pt}, from=1-1, to=3-2]
		\arrow["\iota\circ\varphi", curve={height=-12pt}, from=1-1, to=2-3]
		\arrow[dashed, from=1-1, to=2-2]
	\end{tikzcd}\]
	from which $\varphi=\varphi'$ follows because we have a pullback square.
\end{proof}
And now we manifest what we mean by ``small fiber products inside large ones.''
\begin{cor} \label{cor:sub-fp}
	Fix schemes $X$ and $Y$ over a scheme $S$. Given an open subscheme $U\into Y$, if $X\times_SY$ exists, then $X\times_SU$ also exists and is (canonically) isomorphic to $\pi_Y^{-1}(U)$, where $\pi_Y\colon X\times_SY\to Y$ is the canonical projection.
\end{cor}
\begin{proof}
	To begin, label our relevant maps as follows.
	% https://q.uiver.app/?q=WzAsNCxbMCwxLCJZIl0sWzEsMCwiWCJdLFsxLDEsIlMiXSxbMCwwLCJYXFx0aW1lc19TWSJdLFsxLDIsIlxccHNpX1giXSxbMCwyLCJcXHBzaV9ZIl0sWzMsMSwiXFxwaV9YIl0sWzMsMCwiXFxwaV9ZIiwyXSxbMywyLCIiLDAseyJzdHlsZSI6eyJuYW1lIjoiY29ybmVyIn19XV0=&macro_url=https%3A%2F%2Fraw.githubusercontent.com%2FdFoiler%2Fnotes%2Fmaster%2Fnir.tex
	\[\begin{tikzcd}
		{X\times_SY} & X \\
		Y & S
		\arrow["{\psi_X}", from=1-2, to=2-2]
		\arrow["{\psi_Y}", from=2-1, to=2-2]
		\arrow["{\pi_X}", from=1-1, to=1-2]
		\arrow["{\pi_Y}"', from=1-1, to=2-1]
		\arrow["\lrcorner"{anchor=center, pos=0.125}, draw=none, from=1-1, to=2-2]
	\end{tikzcd}\]
	Our fiber product is going to be $\pi_Y^{-1}(U)$, which induces the restricted ring map $\pi_U\coloneqq\pi_Y|_{\pi_Y^{-1}U}$ by \autoref{lem:restrictmorphism}. This gives us the diagram
	% https://q.uiver.app/?q=WzAsNixbMSwxLCJZIl0sWzIsMCwiWCJdLFsyLDEsIlMiXSxbMSwwLCJYXFx0aW1lc19TWSJdLFswLDAsIlxccGlfWV57LTF9KFUpIl0sWzAsMSwiVSJdLFsxLDIsIlxccHNpX1giXSxbMCwyLCJcXHBzaV9ZIl0sWzMsMSwiXFxwaV9YIl0sWzMsMCwiXFxwaV9ZIiwyXSxbNSwwLCJcXGptYXRoIiwwLHsic3R5bGUiOnsidGFpbCI6eyJuYW1lIjoiaG9vayIsInNpZGUiOiJ0b3AifX19XSxbNCwzLCJcXGlvdGEiLDAseyJzdHlsZSI6eyJ0YWlsIjp7Im5hbWUiOiJob29rIiwic2lkZSI6InRvcCJ9fX1dLFs0LDUsIlxccGlfVSIsMl1d&macro_url=https%3A%2F%2Fraw.githubusercontent.com%2FdFoiler%2Fnotes%2Fmaster%2Fnir.tex
	\[\begin{tikzcd}
		{\pi_Y^{-1}(U)} & {X\times_SY} & X \\
		U & Y & S
		\arrow["{\psi_X}", from=1-3, to=2-3]
		\arrow["{\psi_Y}", from=2-2, to=2-3]
		\arrow["{\pi_X}", from=1-2, to=1-3]
		\arrow["{\pi_Y}"', from=1-2, to=2-2]
		\arrow["\jmath", hook, from=2-1, to=2-2]
		\arrow["\iota", hook, from=1-1, to=1-2]
		\arrow["{\pi_U}"', from=1-1, to=2-1]
	\end{tikzcd}\]
	which we can see commutes on the left by \autoref{rem:better-restriction}. Now, the right square is a pullback square by construction of $X\times_SY$, and the left square is a pullback square by \autoref{lem:open-fp}, so the larger rectangle is a pullback square by \autoref{lem:smalls-to-big-pullback}. This completes the proof.
\end{proof}

\subsection{Fiber Products: Gluing the Factors}
We now begin our gluing. Here is the key idea to keep track of.
\begin{idea}
	After checking the affine case, the gluing follows by chasing universal properties around.
\end{idea}
Indeed, we will have to do no actual algebra in the argument that follows. For this subsection, we will glue along the factors of the fiber product; here is the statement.
\begin{lemma} \label{lem:keyfibercase}
	Fix schemes $X$ and $Y$ over $S$. Given an open cover $\mc U$ of $Y$, if the fiber products $X\times_SU$ exists for each open subscheme $U\in\mc U$, then the fiber product $X\times_SY$ also exists. (The implicit scheme map $U\to S$ is induced by appending the open embedding $U\into X$ to get $U\into X\to S$.)
\end{lemma}
Here is why we care about \autoref{lem:keyfibercase}.
\begin{corollary}
	Fix schemes $X$ and $Y$ over an affine scheme $S$. Then the fiber product $X\times_SY$ exists.
\end{corollary}
\begin{proof}
	We have two steps.
	\begin{enumerate}
		\item Suppose that $X$ and $S$ are affine. Then we can give $Y$ an affine open cover $\mc U$, and we know that the fiber product $X\times_SU$ exists because now everything is affine, so \autoref{lem:affine-fp} is good enough. It follows that $X\times_SY$ also exists by \autoref{lem:keyfibercase}.
		\item Suppose that $S$ is affine. Then we can give $X$ an affine open cover $\mc U$, and we know that the fiber product $U\times_SY$ exists by the previous point. So again, $X\times_SY$ exists by applying the form \autoref{lem:keyfibercase} achieved by swapping the $X$s and $Y$s.
		\qedhere
	\end{enumerate}
\end{proof}
We now prove \autoref{lem:keyfibercase}.
\begin{proof}[Proof of \autoref{lem:keyfibercase}]
	Unsurprisingly, we begin by giving $Y$ the promised open cover $\{Y_\alpha\}_{\alpha\in\lambda}$. To prepare for gluing, we write $Y_{\alpha\beta}\coloneqq Y_\alpha\cap Y_\beta$ for any $\alpha,\beta\in\lambda$ and think of $Y_{\alpha\beta}$ as a subset of $Y_\alpha$ with embedding $\jmath_{\alpha\beta}\colon Y_{\alpha\beta}\to Y_\alpha$. We proceed in steps.
	\begin{enumerate}
		\item By hypothesis on the open cover $\{Y_\alpha\}_{\alpha\in\lambda}$, there are fiber products $W_\alpha\coloneqq X\times_SY_\alpha$; label our pullback square as
		% https://q.uiver.app/?q=WzAsNCxbMCwxLCJYIl0sWzEsMSwiUyJdLFsxLDAsIllfXFxhbHBoYSJdLFswLDAsIlhcXHRpbWVzX1NZX1xcYWxwaGEiXSxbMywyLCJcXHBpX3tZLFxcYWxwaGF9Il0sWzMsMCwiXFxwaV97WCxcXGFscGhhfSIsMl0sWzIsMSwiXFxwc2lfWXxfe1lfXFxhbHBoYX0iXSxbMCwxLCJcXHBzaV9YIiwyXV0=&macro_url=https%3A%2F%2Fraw.githubusercontent.com%2FdFoiler%2Fnotes%2Fmaster%2Fnir.tex
		\[\begin{tikzcd}
			{W_\alpha} & {Y_\alpha} \\
			X & S
			\arrow["{\pi_{Y,\alpha}}", from=1-1, to=1-2]
			\arrow["{\pi_{X,\alpha}}"', from=1-1, to=2-1]
			\arrow["{\psi_Y|_{Y_\alpha}}", from=1-2, to=2-2]
			\arrow["{\psi_X}"', from=2-1, to=2-2]
		\end{tikzcd}\]
		where we explicitly recognize that the canonical map $Y_\alpha\to S$ is the embedding $\jmath_\alpha\colon Y_\alpha\to Y$ followed by $\psi_\alpha$, which is the restriction $\psi_\alpha|_{Y_\alpha}$ (using \autoref{rem:better-restriction}).
		
		\item The construction in \autoref{cor:sub-fp} grants us fiber products
		\[W_{\alpha\beta}\coloneqq(X\times_SY_{\alpha\beta})\coloneqq\pi_{Y,\alpha}^{-1}(Y_{\alpha\beta}).\]
		For brevity, we label the canonical maps as $\pi_{X,\alpha\beta}\coloneqq\pi_{X,\alpha}|_{\pi^{-1}_{Y,\alpha}(Y_{\alpha\beta})}$ and $\pi_{Y,\alpha\beta}\coloneqq\pi_{Y,\alpha}|_{\pi^{-1}_{Y,\alpha}(Y_{\alpha\beta})}$. Now, $Y_{\alpha\beta}=Y_{\beta\alpha}$ by construction, so we must have a canonical isomorphism $W_{\alpha\beta}\simeq W_{\beta\alpha}$. Namely, the universal property of the fiber product promises a unique isomorphism $\varphi_{\alpha\beta}\colon W_{\alpha\beta}\to W_{\beta\alpha}$ such that
		% https://q.uiver.app/?q=WzAsNSxbMiwyLCJTIl0sWzIsMSwiWV97XFxhbHBoYVxcYmV0YX0iXSxbMSwyLCJYIl0sWzEsMSwiWFxcdGltZXNfU1lfe1xcYmV0YVxcYWxwaGF9Il0sWzAsMCwiWFxcdGltZXNfU1lfe1xcYWxwaGFcXGJldGF9Il0sWzIsMCwiXFxwc2lfWCIsMl0sWzEsMCwiXFxwc2lfWXxfe1lfe1xcYWxwaGFcXGJldGF9fSJdLFszLDEsIlxccGlfe1ksXFxiZXRhXFxhbHBoYX0iLDJdLFszLDIsIlxccGlfe1gsXFxiZXRhXFxhbHBoYX0iXSxbNCwzLCJcXHZhcnBoaV97XFxhbHBoYVxcYmV0YX0iLDFdLFs0LDEsIlxccGlfe1ksXFxhbHBoYVxcYmV0YX0iLDAseyJjdXJ2ZSI6LTJ9XSxbNCwyLCJcXHBpX3tYLFxcYWxwaGFcXGJldGF9IiwyLHsiY3VydmUiOjJ9XV0=&macro_url=https%3A%2F%2Fraw.githubusercontent.com%2FdFoiler%2Fnotes%2Fmaster%2Fnir.tex
		\[\begin{tikzcd}
			{W_{\alpha\beta}} \\
			& {W_{\beta\alpha}} & {Y_{\alpha\beta}} \\
			& X & S
			\arrow["{\psi_X}"', from=3-2, to=3-3]
			\arrow["{\psi_Y|_{Y_{\alpha\beta}}}", from=2-3, to=3-3]
			\arrow["{\pi_{Y,\beta\alpha}}"', from=2-2, to=2-3]
			\arrow["{\pi_{X,\beta\alpha}}", from=2-2, to=3-2]
			\arrow["{\varphi_{\alpha\beta}}"{description}, from=1-1, to=2-2]
			\arrow["{\pi_{Y,\alpha\beta}}", curve={height=-12pt}, from=1-1, to=2-3]
			\arrow["{\pi_{X,\alpha\beta}}"', curve={height=12pt}, from=1-1, to=3-2]
		\end{tikzcd}\]
		commutes. For example, when $\alpha=\beta$, we see that $Y_{\alpha\beta}=Y_\alpha=Y_\beta$, and the identity morphism will work for $\varphi_{\alpha\beta}$, so we must have $\varphi_{\alpha\beta}=\id_{Y_\alpha}$.
	
		\item We are going to glue the schemes $W_{\alpha}$ along the isomorphisms $\varphi_{\alpha\beta}$, but we must check the cocycle condition. This is most clearly seen by manually checking triple intersections: given $\alpha,\beta,\gamma\in\lambda$, set $Y_{\alpha\beta\gamma}\coloneqq Y_\alpha\cap Y_\beta\cap Y_\gamma=Y_{\alpha\beta}\cap Y_{\alpha\gamma}$ so that \autoref{cor:sub-fp} grants us a fiber product
		\[W_{\alpha\beta\gamma}\coloneqq X\times_SY_{\alpha\beta\gamma}=\pi_{Y,\alpha}^{-1}(Y_{\alpha\beta\gamma})=\underbrace{\pi_{Y,\alpha}^{-1}(Y_{\alpha\beta})}_{W_{\alpha\beta}}\cap\underbrace{\pi_{Y,\alpha}^{-1}(Y_{\alpha\gamma})}_{W_{\beta\alpha}}.\]
		Notably, $W_{\alpha\beta\gamma}=W_{\alpha\gamma\beta}$. Again, we will label the canonical maps as $\pi_{X,\alpha\beta\gamma}\coloneqq\pi_{X,\alpha}|_{W_{\alpha\beta\gamma}}$ and $\pi_{Y,\alpha\beta\gamma}\coloneqq\pi_{Y,\alpha}|_{W_{\alpha\beta\gamma}}$.

		\item Because $Y_{\alpha\beta\gamma}=Y_{\beta\gamma\alpha}$, we see that there is a unique morphism $W_{\beta\gamma\alpha}\to W_{\beta\gamma\alpha}$ making the diagram
		% https://q.uiver.app/?q=WzAsNSxbMiwyLCJTIl0sWzEsMiwiWCJdLFsyLDEsIllfe1xcYWxwaGFcXGJldGFcXGdhbW1hfSJdLFsxLDEsIldfe1xcYWxwaGFcXGJldGFcXGdhbW1hfSJdLFswLDAsIldfe1xcYmV0YVxcZ2FtbWFcXGFscGhhfSJdLFsxLDAsIlxccHNpX1giLDJdLFs0LDMsIiIsMix7InN0eWxlIjp7ImJvZHkiOnsibmFtZSI6ImRhc2hlZCJ9fX1dLFs0LDIsIlxccGlfe1ksXFxiZXRhXFxnYW1tYVxcYWxwaGF9IiwwLHsiY3VydmUiOi0yfV0sWzMsMiwiXFxwaV97WSxcXGFscGhhXFxiZXRhXFxnYW1tYX0iLDJdLFsyLDAsIlxccHNpX1l8X3tZX3tcXGFscGhhXFxiZXRhXFxnYW1tYX19Il0sWzMsMSwiXFxwaV97WCxcXGFscGhhXFxiZXRhXFxnYW1tYX0iXSxbNCwxLCJcXHBpX3tYLFxcYmV0YVxcZ2FtbWFcXGFscGhhfSIsMix7ImN1cnZlIjoyfV1d&macro_url=https%3A%2F%2Fraw.githubusercontent.com%2FdFoiler%2Fnotes%2Fmaster%2Fnir.tex
		\begin{equation}
			\begin{tikzcd}
				{W_{\beta\gamma\alpha}} \\
				& {W_{\alpha\beta\gamma}} & {Y_{\alpha\beta\gamma}} \\
				& X & S
				\arrow["{\psi_X}"', from=3-2, to=3-3]
				\arrow[dashed, from=1-1, to=2-2]
				\arrow["{\pi_{Y,\beta\gamma\alpha}}", curve={height=-12pt}, from=1-1, to=2-3]
				\arrow["{\pi_{Y,\alpha\beta\gamma}}"', from=2-2, to=2-3]
				\arrow["{\psi_Y|_{Y_{\alpha\beta\gamma}}}", from=2-3, to=3-3]
				\arrow["{\pi_{X,\alpha\beta\gamma}}", from=2-2, to=3-2]
				\arrow["{\pi_{X,\beta\gamma\alpha}}"', curve={height=12pt}, from=1-1, to=3-2]
			\end{tikzcd} \label{eq:trip-intersect-fp}
		\end{equation}
		commute. However, we claim that we can put $\varphi_{\beta\alpha}|_{W_{\beta\gamma\alpha}}$ into the dashed arrow to make the diagram commute. At the very least, note we can induce a restricted morphism $\varphi_{\beta\alpha}|_{W_{\beta\gamma\alpha}}\colon W_{\beta\gamma\alpha}\to W_{\alpha\beta\gamma}$: note that
		\begin{align*}
			\varphi_{\beta\alpha}^{-1}(W_{\alpha\beta\gamma}) &= \{w\in W_{\beta\alpha}:\varphi_{\beta\alpha}w\in W_{\alpha\beta\gamma}\} \\
			&= \{w\in W_{\beta\alpha}:\pi_{Y,\alpha}\varphi_{\beta\alpha}w\in Y_\alpha\cap Y_\beta\cap Y_\gamma\} \\
			&= \{w\in W_{\beta\alpha}:\pi_{Y,\beta}w\in Y_\alpha\cap Y_\beta\cap Y_\gamma\} \\
			&= W_{\beta\gamma\alpha},
		\end{align*}
		so we get the needed morphism by restricting as in \autoref{lem:restrictmorphism}.
		
		There are now two checks; let $\iota_{\alpha\beta\gamma}\coloneqq W_{\alpha\beta\gamma}\subseteq W_{\alpha\beta}$ be the canonical embeddings.
		\begin{itemize}
			\item We check that the top triangle of \autoref{eq:trip-intersect-fp} commutes with $\varphi_{\beta\alpha}|_{W_{\beta\alpha\gamma}}$ in the dashed arrow. Unraveling everything, we are asking for the diagram
			% https://q.uiver.app/?q=WzAsNixbMCwwLCJcXHBpX3tZLFxcYmV0YX1eey0xfShZX1xcYWxwaGFcXGNhcCBZX1xcYmV0YVxcY2FwIFlfXFxnYW1tYSkiXSxbMSwwLCJcXHBpX3tZLFxcYmV0YX1eey0xfShZX1xcYWxwaGFcXGNhcCBZX1xcYmV0YSkiXSxbMiwwLCJZX1xcYWxwaGFcXGNhcCBZX1xcYmV0YSJdLFsyLDEsIllfXFxhbHBoYVxcY2FwIFlfXFxiZXRhIl0sWzEsMSwiXFxwaV97WSxcXGFscGhhfV57LTF9KFlfXFxhbHBoYVxcY2FwIFlfXFxiZXRhKSJdLFswLDEsIlxccGlfe1ksXFxhbHBoYX1eey0xfShZX1xcYWxwaGFcXGNhcCBZX1xcYmV0YVxcY2FwIFlfXFxnYW1tYSkiXSxbMCw1LCJcXHZhcnBoaV97XFxiZXRhXFxhbHBoYX18X3tXX3tcXGJldGFcXGFscGhhXFxnYW1tYX19Il0sWzEsNCwiXFx2YXJwaGlfe1xcYmV0YVxcYWxwaGF9Il0sWzIsMywiIiwwLHsibGV2ZWwiOjIsInN0eWxlIjp7ImhlYWQiOnsibmFtZSI6Im5vbmUifX19XSxbNCwzLCJcXHBpX3tZLFxcYWxwaGFcXGJldGF9Il0sWzEsMiwiXFxwaV97WSxcXGJldGFcXGFscGhhfSJdLFswLDEsIlxcam1hdGhfe1xcYmV0YVxcYWxwaGFcXGdhbW1hfSJdLFs1LDQsIlxcam1hdGhfe1xcYWxwaGFcXGJldGFcXGdhbW1hfSJdXQ==&macro_url=https%3A%2F%2Fraw.githubusercontent.com%2FdFoiler%2Fnotes%2Fmaster%2Fnir.tex
			\[\begin{tikzcd}
				{\pi_{Y,\beta}^{-1}(Y_\alpha\cap Y_\beta\cap Y_\gamma)} & {\pi_{Y,\beta}^{-1}(Y_\alpha\cap Y_\beta)} & {Y_\alpha\cap Y_\beta} \\
				{\pi_{Y,\alpha}^{-1}(Y_\alpha\cap Y_\beta\cap Y_\gamma)} & {\pi_{Y,\alpha}^{-1}(Y_\alpha\cap Y_\beta)} & {Y_\alpha\cap Y_\beta}
				\arrow["{\varphi_{\beta\alpha}|_{W_{\beta\alpha\gamma}}}", from=1-1, to=2-1]
				\arrow["{\varphi_{\beta\alpha}}", from=1-2, to=2-2]
				\arrow[Rightarrow, no head, from=1-3, to=2-3]
				\arrow["{\pi_{Y,\alpha\beta}}", from=2-2, to=2-3]
				\arrow["{\pi_{Y,\beta\alpha}}", from=1-2, to=1-3]
				\arrow["{\jmath_{\beta\alpha\gamma}}", from=1-1, to=1-2]
				\arrow["{\jmath_{\alpha\beta\gamma}}", from=2-1, to=2-2]
			\end{tikzcd}\]
			to commute. The right square commutes by construction of $\varphi_{\alpha\beta}$, and the left square commutes by \autoref{rem:better-restriction}.
			\item We check that the left triangle of \autoref{eq:trip-intersect-fp} commutes with $\varphi_{\beta\alpha}|_{W_{\beta\alpha\gamma}}$ in the dashed arrow. Unraveling everything, we want
			% https://q.uiver.app/?q=WzAsNixbMCwwLCJcXHBpX3tZLFxcYmV0YX1eey0xfShZX1xcYWxwaGFcXGNhcCBZX1xcYmV0YVxcY2FwIFlfXFxnYW1tYSkiXSxbMSwwLCJcXHBpX3tZLFxcYmV0YX1eey0xfShZX1xcYWxwaGFcXGNhcCBZX1xcYmV0YSkiXSxbMiwwLCJYIl0sWzIsMSwiWCJdLFsxLDEsIlxccGlfe1ksXFxhbHBoYX1eey0xfShZX1xcYWxwaGFcXGNhcCBZX1xcYmV0YSkiXSxbMCwxLCJcXHBpX3tZLFxcYWxwaGF9XnstMX0oWV9cXGFscGhhXFxjYXAgWV9cXGJldGFcXGNhcCBZX1xcZ2FtbWEpIl0sWzAsNSwiXFx2YXJwaGlfe1xcYmV0YVxcYWxwaGF9fF97V197XFxiZXRhXFxhbHBoYVxcZ2FtbWF9fSJdLFsxLDQsIlxcdmFycGhpX3tcXGJldGFcXGFscGhhfSJdLFsyLDMsIiIsMCx7ImxldmVsIjoyLCJzdHlsZSI6eyJoZWFkIjp7Im5hbWUiOiJub25lIn19fV0sWzQsMywiXFxwaV97WCxcXGFscGhhXFxiZXRhfSJdLFsxLDIsIlxccGlfe1gsXFxiZXRhXFxhbHBoYX0iXSxbMCwxLCJcXGptYXRoX3tcXGJldGFcXGFscGhhXFxnYW1tYX0iXSxbNSw0LCJcXGptYXRoX3tcXGFscGhhXFxiZXRhXFxnYW1tYX0iXV0=&macro_url=https%3A%2F%2Fraw.githubusercontent.com%2FdFoiler%2Fnotes%2Fmaster%2Fnir.tex
			\[\begin{tikzcd}
				{\pi_{Y,\beta}^{-1}(Y_\alpha\cap Y_\beta\cap Y_\gamma)} & {\pi_{Y,\beta}^{-1}(Y_\alpha\cap Y_\beta)} & X \\
				{\pi_{Y,\alpha}^{-1}(Y_\alpha\cap Y_\beta\cap Y_\gamma)} & {\pi_{Y,\alpha}^{-1}(Y_\alpha\cap Y_\beta)} & X
				\arrow["{\varphi_{\beta\alpha}|_{W_{\beta\alpha\gamma}}}", from=1-1, to=2-1]
				\arrow["{\varphi_{\beta\alpha}}", from=1-2, to=2-2]
				\arrow[Rightarrow, no head, from=1-3, to=2-3]
				\arrow["{\pi_{X,\alpha\beta}}", from=2-2, to=2-3]
				\arrow["{\pi_{X,\beta\alpha}}", from=1-2, to=1-3]
				\arrow["{\jmath_{\beta\alpha\gamma}}", from=1-1, to=1-2]
				\arrow["{\jmath_{\alpha\beta\gamma}}", from=2-1, to=2-2]
			\end{tikzcd}\]
			to commute. Again, the right square commutes by construction of the $\varphi_{\beta\alpha}$, and the left square commutes by \autoref{rem:better-restriction}.
		\end{itemize}

		\item We are now ready to verify the cocycle condition. Essentially, we are asking for the diagram
		% https://q.uiver.app/?q=WzAsMyxbMCwwLCJXX3tcXGFscGhhXFxiZXRhXFxnYW1tYX0iXSxbMSwwLCJXX3tcXGJldGFcXGdhbW1hXFxhbHBoYX0iXSxbMSwxLCJXX3tcXGdhbW1hXFxhbHBoYVxcYmV0YX0iXSxbMCwxLCJcXHZhcnBoaV97XFxhbHBoYVxcYmV0YX18X3tXX3tcXGFscGhhXFxiZXRhXFxnYW1tYX19Il0sWzEsMiwiXFx2YXJwaGlfe1xcYmV0YVxcZ2FtbWF9fF97V197XFxiZXRhXFxnYW1tYVxcYWxwaGF9fSJdLFswLDIsIlxcdmFycGhpX3tcXGFscGhhXFxnYW1tYX18X3tXX3tcXGFscGhhXFxiZXRhXFxnYW1tYX19IiwyXV0=&macro_url=https%3A%2F%2Fraw.githubusercontent.com%2FdFoiler%2Fnotes%2Fmaster%2Fnir.tex
		\[\begin{tikzcd}
			{W_{\alpha\beta\gamma}} & {W_{\beta\gamma\alpha}} \\
			& {W_{\gamma\alpha\beta}}
			\arrow["{\varphi_{\alpha\beta}|_{W_{\alpha\beta\gamma}}}", from=1-1, to=1-2]
			\arrow["{\varphi_{\beta\gamma}|_{W_{\beta\gamma\alpha}}}", from=1-2, to=2-2]
			\arrow["{\varphi_{\alpha\gamma}|_{W_{\alpha\beta\gamma}}}"', from=1-1, to=2-2]
		\end{tikzcd}\]
		to commute. By construction of $\varphi_{\alpha\gamma}|_{W_{\alpha\beta\gamma}}$, it suffices to show we can place the composite $\varphi_{\beta\gamma}|_{W_{\beta\gamma\alpha}}\circ\varphi_{\alpha\beta}|_{W_{\alpha\beta\gamma}}$ into the dashed arrow of
		% https://q.uiver.app/?q=WzAsNSxbMiwyLCJTIl0sWzEsMiwiWCJdLFsyLDEsIllfe1xcYWxwaGFcXGJldGFcXGdhbW1hfSJdLFsxLDEsIldfe1xcZ2FtbWFcXGFscGhhXFxiZXRhfSJdLFswLDAsIldfe1xcYWxwaGFcXGJldGFcXGdhbW1hfSJdLFsxLDAsIlxccHNpX1giLDJdLFs0LDMsIiIsMix7InN0eWxlIjp7ImJvZHkiOnsibmFtZSI6ImRhc2hlZCJ9fX1dLFs0LDIsIlxccGlfe1ksXFxhbHBoYVxcYmV0YVxcZ2FtbWF9IiwwLHsiY3VydmUiOi0yfV0sWzMsMiwiXFxwaV97WSxcXGdhbW1hXFxhbHBoYVxcYmV0YX0iLDJdLFsyLDAsIlxccHNpX1l8X3tZX3tcXGFscGhhXFxiZXRhXFxnYW1tYX19Il0sWzMsMSwiXFxwaV97WCxcXGdhbW1hXFxhbHBoYVxcYmV0YX0iXSxbNCwxLCJcXHBpX3tYLFxcYWxwaGFcXGJldGFcXGdhbW1hfSIsMix7ImN1cnZlIjoyfV1d&macro_url=https%3A%2F%2Fraw.githubusercontent.com%2FdFoiler%2Fnotes%2Fmaster%2Fnir.tex
		\[\begin{tikzcd}
			{W_{\alpha\beta\gamma}} \\
			& {W_{\gamma\alpha\beta}} & {Y_{\alpha\beta\gamma}} \\
			& X & S
			\arrow["{\psi_X}"', from=3-2, to=3-3]
			\arrow[dashed, from=1-1, to=2-2]
			\arrow["{\pi_{Y,\alpha\beta\gamma}}", curve={height=-12pt}, from=1-1, to=2-3]
			\arrow["{\pi_{Y,\gamma\alpha\beta}}"', from=2-2, to=2-3]
			\arrow["{\psi_Y|_{Y_{\alpha\beta\gamma}}}", from=2-3, to=3-3]
			\arrow["{\pi_{X,\gamma\alpha\beta}}", from=2-2, to=3-2]
			\arrow["{\pi_{X,\alpha\beta\gamma}}"', curve={height=12pt}, from=1-1, to=3-2]
		\end{tikzcd}\]
		to make the diagram commute. For this, we note that checking the commutativity of the two needed triangles comes down to writing the diagrams
		% https://q.uiver.app/?q=WzAsMTIsWzAsMCwiV197XFxhbHBoYVxcYmV0YVxcZ2FtbWF9Il0sWzAsMSwiV197XFxiZXRhXFxnYW1tYVxcYWxwaGF9Il0sWzAsMiwiV197XFxnYW1tYVxcYWxwaGFcXGJldGF9Il0sWzEsMCwiWV97XFxhbHBoYVxcYmV0YVxcZ2FtbWF9Il0sWzEsMSwiWV97XFxiZXRhXFxhbHBoYVxcZ2FtbWF9Il0sWzEsMiwiWV97XFxnYW1tYVxcYWxwaGFcXGJldGF9Il0sWzMsMCwiV197XFxhbHBoYVxcYmV0YVxcZ2FtbWF9Il0sWzMsMSwiV197XFxiZXRhXFxnYW1tYVxcYWxwaGF9Il0sWzMsMiwiV197XFxnYW1tYVxcYWxwaGFcXGJldGF9Il0sWzQsMCwiWCJdLFs0LDEsIlgiXSxbNCwyLCJYIl0sWzAsMSwiXFx2YXJwaGlfe1xcYWxwaGFcXGJldGF9fF97V197XFxhbHBoYVxcYmV0YVxcZ2FtbWF9fSIsMl0sWzAsMywiXFxwaV97WSxcXGFscGhhXFxiZXRhXFxnYW1tYX0iXSxbMyw0LCIiLDIseyJsZXZlbCI6Miwic3R5bGUiOnsiaGVhZCI6eyJuYW1lIjoibm9uZSJ9fX1dLFsxLDQsIlxccGlfe1ksXFxiZXRhXFxnYW1tYVxcYWxwaGF9Il0sWzEsMiwiXFx2YXJwaGlfe1xcYmV0YVxcZ2FtbWF9fF97V197XFxiZXRhXFxnYW1tYVxcYWxwaGF9fSIsMl0sWzIsNSwiXFxwaV97WSxcXGdhbW1hXFxhbHBoYVxcYmV0YX0iXSxbNCw1LCIiLDAseyJsZXZlbCI6Miwic3R5bGUiOnsiaGVhZCI6eyJuYW1lIjoibm9uZSJ9fX1dLFs5LDEwLCIiLDIseyJsZXZlbCI6Miwic3R5bGUiOnsiaGVhZCI6eyJuYW1lIjoibm9uZSJ9fX1dLFsxMCwxMSwiIiwyLHsibGV2ZWwiOjIsInN0eWxlIjp7ImhlYWQiOnsibmFtZSI6Im5vbmUifX19XSxbNiw5LCJcXHBpX3tYLFxcYWxwaGFcXGJldGFcXGdhbW1hfSJdLFs3LDEwLCJcXHBpX3tYLFxcYmV0YVxcZ2FtbWFcXGFscGhhfSJdLFs4LDExLCJcXHBpX3tYLFxcZ2FtbWFcXGFscGhhXFxiZXRhfSJdLFs2LDcsIlxcdmFycGhpX3tcXGFscGhhXFxiZXRhfXxfe1dfe1xcYWxwaGFcXGJldGFcXGdhbW1hfX0iLDJdLFs3LDgsIlxcdmFycGhpX3tcXGJldGFcXGdhbW1hfXxfe1dfe1xcYmV0YVxcZ2FtbWFcXGFscGhhfX0iLDJdXQ==&macro_url=https%3A%2F%2Fraw.githubusercontent.com%2FdFoiler%2Fnotes%2Fmaster%2Fnir.tex
		\[\begin{tikzcd}
			{W_{\alpha\beta\gamma}} & {Y_{\alpha\beta\gamma}} && {W_{\alpha\beta\gamma}} & X \\
			{W_{\beta\gamma\alpha}} & {Y_{\beta\alpha\gamma}} && {W_{\beta\gamma\alpha}} & X \\
			{W_{\gamma\alpha\beta}} & {Y_{\gamma\alpha\beta}} && {W_{\gamma\alpha\beta}} & X
			\arrow["{\varphi_{\alpha\beta}|_{W_{\alpha\beta\gamma}}}"', from=1-1, to=2-1]
			\arrow["{\pi_{Y,\alpha\beta\gamma}}", from=1-1, to=1-2]
			\arrow[Rightarrow, no head, from=1-2, to=2-2]
			\arrow["{\pi_{Y,\beta\gamma\alpha}}", from=2-1, to=2-2]
			\arrow["{\varphi_{\beta\gamma}|_{W_{\beta\gamma\alpha}}}"', from=2-1, to=3-1]
			\arrow["{\pi_{Y,\gamma\alpha\beta}}", from=3-1, to=3-2]
			\arrow[Rightarrow, no head, from=2-2, to=3-2]
			\arrow[Rightarrow, no head, from=1-5, to=2-5]
			\arrow[Rightarrow, no head, from=2-5, to=3-5]
			\arrow["{\pi_{X,\alpha\beta\gamma}}", from=1-4, to=1-5]
			\arrow["{\pi_{X,\beta\gamma\alpha}}", from=2-4, to=2-5]
			\arrow["{\pi_{X,\gamma\alpha\beta}}", from=3-4, to=3-5]
			\arrow["{\varphi_{\alpha\beta}|_{W_{\alpha\beta\gamma}}}"', from=1-4, to=2-4]
			\arrow["{\varphi_{\beta\gamma}|_{W_{\beta\gamma\alpha}}}"', from=2-4, to=3-4]
		\end{tikzcd}\]
		(the left rectangle is the top triangle, and the right rectangle is the left triangle), and we note that everything commutes by construction of the morphisms on the left.
		
		\item The above steps allow us to glue together the $W_\alpha$ along the isomorphisms $\varphi_{\alpha\beta}$ to get a single scheme $W$ by \autoref{prop:glue-schemes}. Explicitly, $W$ has an open cover $\{W_\alpha'\}_{\alpha\in\lambda}$ and embeddings $\iota_\alpha'\colon W_\alpha\cong W_{\alpha'}$ such that $\iota_\alpha'(W_{\alpha\beta})=W_\alpha\cap W_\beta$ and $\iota_\beta'\circ\varphi_{\alpha\beta}=\iota_\alpha'|_{W_{\alpha\beta}}$.
		
		However, the only property we needed for $W_\alpha$ is that it is the fiber product of $X\times_SY_\alpha$, so we will (for psychological reasons) go ahead and identify $W_\alpha$ with its image in $W$ so that $W$ has an open subscheme $W_\alpha\subseteq W$ which is a fiber product of $X\times_SY_\alpha$.

		Notably, the gluing process, along with the above identification, tells us that $W_\alpha\cap W_\beta$ was isomorphic to both $\iota_\alpha'(W_{\alpha\beta})$ and $\iota_\beta'(W_{\beta\alpha})$ and therefore serves as a fiber product $X\times_SY_{\alpha\beta}$. However, earlier we had to use the isomorphism $\varphi_{\alpha\beta}$ to translate between these two, but $\iota_\beta'\circ\varphi_{\alpha\beta}=\iota_\alpha'$ tells us that $\varphi_{\alpha\beta}$ becomes literally the identity in $W$.

		Now, we had maps $\pi_{X,\alpha}\colon W_\alpha\to X$ for each $\alpha\in\lambda$ such that
		\[\pi_{X,\alpha}|_{W_{\alpha\beta}}=\pi_{X,\beta}|_{W_{\beta\alpha}}\circ\varphi_{\alpha\beta},\]
		which again upon identifying everything into $W$ will glue by \autoref{prop:glue-morphisms} to grant us a unique morphism $\pi_X\colon W\to X$ such that $\pi_X|_{W_{\alpha}}=\pi_{X,\alpha}$.

		Continuing, we note that we had maps $\pi_{Y,\alpha}\colon W_\alpha\to Y_\alpha$ for each $\alpha\in\lambda$ such that
		\begin{equation}
			\pi_{Y,\alpha}|_{W_{\alpha\beta}}=\pi_{Y,\beta}|_{W_{\beta\alpha}}\circ\varphi_{\alpha\beta}. \label{eq:pi-y-gluing}
		\end{equation}
		Once we've identified everything into $W$, \autoref{eq:pi-y-gluing} tells us that the morphisms $\pi_{Y,\alpha}$ glue together by \autoref{prop:glue-morphisms} to a unique morphism $\pi_Y\colon W\to Y$ such that $\pi_Y|_{W_\alpha}=\pi_{Y,\alpha}$. (Technically, one must post-compose $\pi_{Y,\alpha}$ with the embedding $Y_\alpha\into Y$ first, and then note that we can also post-compose \autoref{eq:pi-y-gluing} with $Y_{\alpha\beta}\into Y$, but this causes no problems.)

		We take a moment to recognize that $\pi_Y^{-1}(Y_\alpha)=W_\alpha$. Indeed, if $w\in W$ has $\pi_Y(w)\in Y_\alpha$, then we can at least place $w$ in some $W_\beta$ so that $\pi_Y(w)=\pi_{Y,\beta}(w)\in Y_\beta$. But then $\pi_{Y,\beta}(w)\in Y_{\alpha\beta}$, so $w\in W_{\alpha\beta}\subseteq W_\beta$, so $w\in W_\alpha$.
		
		\item We now verify the universal property. Fix morphisms $\varphi_X\colon Z\to X$ and $\varphi_Y\colon Y\to Z$ making the diagram
		% https://q.uiver.app/?q=WzAsNSxbMSwyLCJYIl0sWzIsMSwiWSJdLFsyLDIsIlMiXSxbMCwwLCJaIl0sWzEsMSwiVyJdLFswLDIsIlxccHNpX1giLDJdLFsxLDIsIlxccHNpX1kiXSxbMywwLCJcXHZhcnBoaV9YIiwyLHsiY3VydmUiOjJ9XSxbMywxLCJcXHZhcnBoaV9ZIiwwLHsiY3VydmUiOi0yfV0sWzQsMSwiXFxwaV9ZIiwyXSxbNCwwLCJcXHBpX1giXSxbMyw0LCIiLDEseyJzdHlsZSI6eyJib2R5Ijp7Im5hbWUiOiJkYXNoZWQifX19XV0=&macro_url=https%3A%2F%2Fraw.githubusercontent.com%2FdFoiler%2Fnotes%2Fmaster%2Fnir.tex
		\begin{equation}
			\begin{tikzcd}
				Z \\
				& W & Y \\
				& X & S
				\arrow["{\psi_X}"', from=3-2, to=3-3]
				\arrow["{\psi_Y}", from=2-3, to=3-3]
				\arrow["{\varphi_X}"', curve={height=12pt}, from=1-1, to=3-2]
				\arrow["{\varphi_Y}", curve={height=-12pt}, from=1-1, to=2-3]
				\arrow["{\pi_Y}"', from=2-2, to=2-3]
				\arrow["{\pi_X}", from=2-2, to=3-2]
				\arrow[dashed, from=1-1, to=2-2]
			\end{tikzcd} \label{eq:glue-factors-fp-up}
		\end{equation}
		commute, and we want to induce a unique morphism $\varphi\colon Z\to W$ making the diagram commute. We show uniqueness and existence separately.
		\begin{itemize}
			\item Uniqueness: suppose we have some $\varphi\colon Z\to W$ making \autoref{eq:glue-factors-fp-up} commute. Restricting $Y$ with $Y_\alpha$, we see that $\varphi_Y$ will restrict to $\varphi_{Y,\alpha}\coloneqq\varphi_Y|_{\varphi_Y^{-1}Y_\alpha}$ by \autoref{lem:restrictmorphism}. Similarly, we should replace $W$ with $\pi_Y^{-1}(Y_\alpha)=W_\alpha$ so that $\pi_Y$ becomes $\pi_{Y,\alpha}$ and $\pi_X$ becomes $\pi_{X,\alpha}$, and we should replace $Z$ with $Z_\alpha\coloneqq\varphi_Y^{-1}(Y_\alpha)$. Notably, $\varphi|_{Z_\alpha}\colon Z_\alpha\to W_\alpha$ makes sense as restricted by \autoref{lem:restrictmorphism}. Lastly, we should replace $\varphi_X$ with $\varphi_{X,\alpha}\coloneqq\varphi_X|_{Z_\alpha}$.

			In total functoriality of restricting morphisms tells us that the diagram
			% https://q.uiver.app/?q=WzAsNSxbMCwwLCJaX1xcYWxwaGEiXSxbMSwxLCJXX1xcYWxwaGEiXSxbMiwxLCJZX1xcYWxwaGEiXSxbMiwyLCJTIl0sWzEsMiwiWCJdLFswLDEsIlxcdmFycGhpX1xcYWxwaGEiLDFdLFswLDIsIlxcdmFycGhpX3tZLFxcYWxwaGF9IiwwLHsiY3VydmUiOi0yfV0sWzAsNCwiXFx2YXJwaGlfe1gsXFxhbHBoYX0iLDIseyJjdXJ2ZSI6Mn1dLFsxLDIsIlxccGlfe1ksXFxhbHBoYX0iLDJdLFsxLDQsIlxccGlfe1gsXFxhbHBoYX0iXSxbMiwzLCJcXHBzaV97WSxcXGFscGhhfSJdLFs0LDMsIlxccHNpX1giLDJdXQ==&macro_url=https%3A%2F%2Fraw.githubusercontent.com%2FdFoiler%2Fnotes%2Fmaster%2Fnir.tex
			\[\begin{tikzcd}
				{Z_\alpha} \\
				& {W_\alpha} & {Y_\alpha} \\
				& X & S
				\arrow["{\varphi|_{Z_\alpha}}"{description}, from=1-1, to=2-2]
				\arrow["{\varphi_{Y,\alpha}}", curve={height=-12pt}, from=1-1, to=2-3]
				\arrow["{\varphi_{X,\alpha}}"', curve={height=12pt}, from=1-1, to=3-2]
				\arrow["{\pi_{Y,\alpha}}"', from=2-2, to=2-3]
				\arrow["{\pi_{X,\alpha}}", from=2-2, to=3-2]
				\arrow["{\psi_{Y,\alpha}}", from=2-3, to=3-3]
				\arrow["{\psi_X}"', from=3-2, to=3-3]
			\end{tikzcd}\]
			commutes; namely, the two triangles commute by applying \autoref{rem:scheme-restriction-functorial} everywhere---the top triangle has restricted $Y$ to $Y_\alpha$, and the left triangle has restricted $W$ to $W_\alpha$. However, we note that the above diagram features a pullback square, so the morphisms $\varphi|_{Z_\alpha}\colon Z_\alpha\to W_\alpha$ are uniquely determined. It follows that the morphisms $\varphi|_{Z_{\alpha}}\colon Z_\alpha\to W$ are uniquely determined, so \autoref{prop:glue-morphisms} tells us that $\varphi$ itself is uniquely determined.
			\item Existence: we unwrap the construction above. Setting variables as in the previous step (except $\varphi$ of course), we see that the diagram
			% https://q.uiver.app/?q=WzAsNixbMiwwLCJZX1xcYWxwaGEiXSxbMSwxLCJaIl0sWzIsMSwiWSJdLFsxLDIsIlgiXSxbMiwyLCJTIl0sWzAsMCwiWl9cXGFscGhhIl0sWzEsMiwiXFx2YXJwaGlfWSJdLFswLDIsIiIsMSx7InN0eWxlIjp7InRhaWwiOnsibmFtZSI6Imhvb2siLCJzaWRlIjoidG9wIn19fV0sWzIsNCwiXFxwc2lfWSJdLFsxLDMsIlxcdmFycGhpX1giXSxbMyw0LCJcXHBzaV9YIl0sWzUsMSwiIiwxLHsic3R5bGUiOnsidGFpbCI6eyJuYW1lIjoiaG9vayIsInNpZGUiOiJ0b3AifX19XSxbNSwzLCJcXHZhcnBoaV97WCxcXGFscGhhfSIsMix7ImN1cnZlIjoyfV0sWzUsMCwiXFx2YXJwaGlfe1ksXFxhbHBoYX0iXV0=&macro_url=https%3A%2F%2Fraw.githubusercontent.com%2FdFoiler%2Fnotes%2Fmaster%2Fnir.tex
			\[\begin{tikzcd}
				{Z_\alpha} && {Y_\alpha} \\
				& Z & Y \\
				& X & S
				\arrow["{\varphi_Y}", from=2-2, to=2-3]
				\arrow[hook, from=1-3, to=2-3]
				\arrow["{\psi_Y}", from=2-3, to=3-3]
				\arrow["{\varphi_X}", from=2-2, to=3-2]
				\arrow["{\psi_X}", from=3-2, to=3-3]
				\arrow[hook, from=1-1, to=2-2]
				\arrow["{\varphi_{X,\alpha}}"', curve={height=12pt}, from=1-1, to=3-2]
				\arrow["{\varphi_{Y,\alpha}}", from=1-1, to=1-3]
			\end{tikzcd}\]
			commutes for any $\alpha\in\lambda$, by the functoriality of restriction (namely, apply \autoref{rem:scheme-restriction-functorial} everywhere). However, this will induce a unique morphism $\varphi_\alpha\colon Z\to W_\alpha$ making the diagram
			% https://q.uiver.app/?q=WzAsNSxbMiwyLCJTIl0sWzIsMSwiWV9cXGFscGhhIl0sWzAsMCwiWl9cXGFscGhhIl0sWzEsMSwiV19cXGFscGhhIl0sWzEsMiwiWCJdLFsyLDEsIlxcdmFycGhpX3tZLFxcYWxwaGF9IiwwLHsiY3VydmUiOi0yfV0sWzIsMywiXFx2YXJwaGlfXFxhbHBoYSIsMV0sWzIsNCwiXFx2YXJwaGlfe1gsXFxhbHBoYX0iLDIseyJjdXJ2ZSI6Mn1dLFszLDQsIlxccGlfe1gsXFxhbHBoYX0iXSxbMywxLCJcXHBpX3tZLFxcYWxwaGF9IiwyXSxbMSwwLCJcXHBzaV9ZIl0sWzQsMCwiXFxwc2lfWCIsMl1d&macro_url=https%3A%2F%2Fraw.githubusercontent.com%2FdFoiler%2Fnotes%2Fmaster%2Fnir.tex
			\[\begin{tikzcd}
				{Z_\alpha} \\
				& {W_\alpha} & {Y_\alpha} \\
				& X & S
				\arrow["{\varphi_{Y,\alpha}}", curve={height=-12pt}, from=1-1, to=2-3]
				\arrow["{\varphi_\alpha}"{description}, from=1-1, to=2-2]
				\arrow["{\varphi_{X,\alpha}}"', curve={height=12pt}, from=1-1, to=3-2]
				\arrow["{\pi_{X,\alpha}}", from=2-2, to=3-2]
				\arrow["{\pi_{Y,\alpha}}"', from=2-2, to=2-3]
				\arrow["{\psi_{Y,\alpha}}", from=2-3, to=3-3]
				\arrow["{\psi_X}"', from=3-2, to=3-3]
			\end{tikzcd}\]
			commute. To glue the morphisms $\varphi_\alpha$ together, we restrict our diagram to $Y_\alpha\cap Y_\beta$. Then we replace $W_\alpha$ with $W_\alpha\cap W_\beta$ and $Z_\alpha$ with $Z_\alpha\cap Z_\beta$, so functoriality of restriction (namely, \autoref{rem:scheme-restriction-functorial}) tells us that
			% https://q.uiver.app/?q=WzAsNSxbMiwyLCJTIl0sWzIsMSwiWV9cXGFscGhhXFxjYXAgWV9cXGJldGEiXSxbMCwwLCJaX1xcYWxwaGFcXGNhcCBaX1xcYmV0YSJdLFsxLDEsIldfXFxhbHBoYVxcY2FwIFdfXFxiZXRhIl0sWzEsMiwiWCJdLFsyLDEsIlxcdmFycGhpX3tZfXxfe1pfXFxhbHBoYVxcY2FwIFpfXFxiZXRhfSIsMCx7ImN1cnZlIjotMn1dLFsyLDMsIlxcdmFycGhpX1xcYWxwaGF8X3taX1xcYWxwaGFcXGNhcCBaX1xcYmV0YX0iLDFdLFsyLDQsIlxcdmFycGhpX3tYfXxfe1pfXFxhbHBoYVxcY2FwIFpfXFxiZXRhfSIsMix7ImN1cnZlIjoyfV0sWzMsNCwiXFxwaV97WH18X3tXX1xcYWxwaGFcXGNhcCBXX1xcYmV0YX0iXSxbMywxLCJcXHBpX3tZfXxfe1dfXFxhbHBoYVxcY2FwIFdfXFxiZXRhfSIsMl0sWzEsMCwiXFxwc2lfWXxfe1dfXFxhbHBoYVxcY2FwIFdfXFxiZXRhfSJdLFs0LDAsIlxccHNpX1giLDJdXQ==&macro_url=https%3A%2F%2Fraw.githubusercontent.com%2FdFoiler%2Fnotes%2Fmaster%2Fnir.tex
			\[\begin{tikzcd}
				{Z_\alpha\cap Z_\beta} \\
				& {W_\alpha\cap W_\beta} & {Y_\alpha\cap Y_\beta} \\
				& X & S
				\arrow["{\varphi_{Y}|_{Z_\alpha\cap Z_\beta}}", curve={height=-12pt}, from=1-1, to=2-3]
				\arrow["{\varphi_\alpha|_{Z_\alpha\cap Z_\beta}}"{description}, from=1-1, to=2-2]
				\arrow["{\varphi_{X}|_{Z_\alpha\cap Z_\beta}}"', curve={height=12pt}, from=1-1, to=3-2]
				\arrow["{\pi_{X}|_{W_\alpha\cap W_\beta}}", from=2-2, to=3-2]
				\arrow["{\pi_{Y}|_{W_\alpha\cap W_\beta}}"', from=2-2, to=2-3]
				\arrow["{\psi_Y|_{W_\alpha\cap W_\beta}}", from=2-3, to=3-3]
				\arrow["{\psi_X}"', from=3-2, to=3-3]
			\end{tikzcd}\]
			commutes. However, the morphism $Z_\alpha\cap Z_\beta\to W_\alpha\cap W_\beta$ is unique making this diagram commute because the square is a pullback square, under our identification of $W_\alpha\subseteq W$. (Namely, our projections are $\pi_{Y,\alpha\beta}=\pi_{Y,\alpha}|_{W_{\alpha\beta}}=\pi_Y|_{W_\alpha\cap W_\beta}$, and similar for $X$.) Thus, swapping all $\alpha$s and $\beta$s in the above diagram changes nothing except this morphism, so we conclude that
			\[\varphi_\alpha|_{Z_\alpha\cap Z_\beta}=\varphi_\beta|_{Z_\alpha\cap Z_\beta}\]
			for any $\alpha,\beta\in\lambda$. Because the $Y_\alpha$ cover $Y$, we see that the $Z_\alpha$ cover $Z$, so we do indeed glue into a morphism $\varphi\colon Z\to W$ such that $\varphi|_{Z_\alpha}=\varphi_\alpha$. (Technically, to glue, we have to post-compose with the embeddings into $W$ everywhere, but this causes no problems.)

			Now, for any $Z_\alpha$, we see that
			\[(\pi_Y\circ\varphi)|_{Z_\alpha}=\pi_Y|_{W_\alpha}\circ\varphi|_{Z_\alpha}=\pi_{Y,\alpha}\circ\varphi_\alpha=\varphi_{Y,\alpha}=\varphi_Y|_{Z_\alpha}\]
			by repeatedly using functoriality of restriction (\autoref{rem:scheme-restriction-functorial}). Because the $\{Z_\alpha\}_{\alpha\in\lambda}$ form an open cover of $Z$, \autoref{prop:glue-morphisms} tells us that $\pi_Y\circ\varphi=\varphi_Y$. Replacing all $Y$s with $X$s in the above argument tells us that $\pi_X\circ\varphi=\varphi_X$.
		\end{itemize}
	\end{enumerate}
	The above steps have been able to glue together a fiber product and show it satisfies the universal property. This finishes.
\end{proof}

\subsection{Fiber Products: Gluing the Base}
Lastly, here is the general case.
\begin{proof}[Proof of \autoref{thm:fibexist}]
	Give $S$ an affine open cover $\{S_\alpha\}_{\alpha\in\lambda}$, with our natural maps $f_X\colon X\to S$ and $f_Y\colon Y\to S$. Then we set $X_\alpha\coloneqq f_X^{-1}(S_\alpha)$ and similar for $Y_\alpha$. By previous work, we have the fiber products $X_\alpha\times_{S_\alpha}Y_\alpha$, so we can just glue these together as usual.
\end{proof}
A little more rigorously, the following result aids the above gluing.
\begin{lemma}
	Fix schemes $X$ and $Y$ over a scheme $S$. If $S\subseteq S'$ is an affine open embedding of schemes such that $X\times_{S'}Y$ exists, we have
	\[f_X^{-1}(S)\times_Sf_Y^{-1}(S)=(f_X\circ\pi_X)^{-1}(S')=(f_Y\circ\pi_Y)^{-1}(S').\]
\end{lemma}
The intuition is that the above lemma should be pulling back the required fiber product along the various legs of the following diagram.
% https://q.uiver.app/?q=WzAsNCxbMCwwLCJYXFx0aW1lc197Uyd9WSJdLFsxLDAsIlkiXSxbMCwxLCJYIl0sWzEsMSwiUyciXSxbMCwxXSxbMSwzXSxbMCwyXSxbMiwzXV0=&macro_url=https%3A%2F%2Fraw.githubusercontent.com%2FdFoiler%2Fnotes%2Fmaster%2Fnir.tex
\[\begin{tikzcd}
	{X\times_{S'}Y} & Y \\
	X & {S'}
	\arrow[from=1-1, to=1-2]
	\arrow[from=1-2, to=2-2]
	\arrow[from=1-1, to=2-1]
	\arrow[from=2-1, to=2-2]
\end{tikzcd}\]
This will give the intersection data in the gluing for \autoref{thm:fibexist}, which will finish the gluing.
\begin{remark}
	We will provide a more categorical viewpoint of this construction next class. This categorical viewpoint will be helpful for when we want to define the Grassmannian.
	% V 10.1.8
\end{remark}

\end{document}