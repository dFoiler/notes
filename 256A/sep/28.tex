% !TEX root = ../notes.tex

\documentclass[../notes.tex]{subfiles}

\begin{document}

\section{September 28}

Today we discuss Chevalley's theorem.

\subsection{Chevalley's Theorem: Comments}
Here is the statement.
\begin{theorem} \label{thm:chev}
	Fix Noetherian schemes $X$ and $Y$ and a morphism $\pi\colon X\to Y$ of finite type. Then if $C\subseteq X$ is constructible, $\pi(C)$ is also constructible.
\end{theorem}
We will prove \autoref{thm:chev} today, but there are analogues when $X$ and $Y$ need not be Noetherian.
\begin{remark}
	If we want to allow $X$ to be quasicompact and quasiseparated, we can let constructible subsets be finite unions of sets of the form $U\setminus U'$ where $U$ and $U'$ are quasicompact open sets. Letting $\pi\colon X\to Y$ be of ``finite presentation'' it will be true that $\pi$ sends constructible sets to constructible sets.
	
	Here, ``finitely presented'' means locally finitely presented and quasicompact and quasiseparated. Further, locally finitely presented means that any affine open subset $U=\Spec B\subseteq Y$ with an affine open subset $\Spec A\subseteq\pi^{-1}(U)$ has $A$ a finitely presented $B$-algebra; i.e., there is a finitely generated ideal $I\subseteq B[x_1,\ldots,x_n]$ such that $A\cong B[x_1,\ldots,x_n]/I$.
	% GW P10.44, T10.70, C10.71
\end{remark}
Let's see a few examples of \autoref{thm:chev}.
\begin{example}
	Take $\pi\colon X\to\AA^1_k$ with $X$ connected. Then $\pi(X)$ is connected, so $\pi(X)$ is a single point and so $\pi(X)$ contains an open subset of $\AA^1_k$ and is therefore missing only finitely many points.
\end{example}
There are some really nontrivial examples which explain why we want to work with constructible sets in \autoref{thm:chev}.
\begin{example}
	Let $k$ be algebraically closed, for psychological reasons. We define $\pi\colon\Spec k[x,y]\to\Spec k[u,v]$ by taking $\mf m_{(a,b)}\mapsto\mf m_{(a,ab)}$. Namely, this is the scheme morphism coming from the ring homomorphism $k[u,v]\to k[x,y]$ by $u\mapsto x$ and $v\mapsto xy$.
	% Intuitively, we are taking the left plane to the right one.
	% \begin{center}
	% 	\begin{asy}
	% 		unitsize(1cm);
	% 		dot("$(0,0)$", (0,0), S);
	% 		draw((-2,2) -- (2,2));
	% 		label("$\textrm{Spec}\,k[x,y]$
	% 	\end{asy}
	% \end{center}
	In particular, we can see that we map the generic point to generic points, we send a one-dimensional prime $(f(x,y))$ to $f(u,v/u)\cdot u^\bullet$ for some sufficiently large power $u^\bullet$, and in particular we send $(x)\mapsto(u,v)$. Additionally, we have a continuous bijection $\AA^2_k\setminus V((x))$ going to $\AA^2_k\setminus V((u))$. Thus, $\pi(\AA^2_k)$ is $D(u)\cup\{(0,0)\}$, which is very weird.
\end{example}
\begin{remark}
	The above example is a ``typical'' example of a birational map, where away from a closed set we have a continuous bijection, and the remaining closed set gets squeezed down.
\end{remark}
\begin{remark}
	On a Noetherian scheme, closed points contain all the needed topological data. Namely, continuous maps carry the generic points along for the ride after being told what to do with closed points. Thus, we can reason topologically about schemes like $\AA^2_k$ by only paying attention to closed points.
	% H II P2.6, GW 3.34--3.36
\end{remark}

\subsection{Chevalley's Theorem: Proof}
We begin with a few reduction steps to turn this into an affine problem of $\Spec B[x_1,\ldots,x_n]\to\Spec B$. Here are our reduction steps.
\begin{enumerate}
	\item At any point in the proof, because we are only looking at topological spaces, we can replace $X$ and $Y$ with their reductions: note we have a unique map $\pi_{\mathrm{red}}$ making the diagram
	% https://q.uiver.app/?q=WzAsNCxbMCwwLCJYIl0sWzAsMSwiWF97XFxtYXRocm17cmVkfX0iXSxbMSwwLCJZIl0sWzEsMSwiWV97XFxtYXRocm17cmVkfX0iXSxbMCwyLCJcXHBpIl0sWzEsMywiXFxwaV97XFxtYXRocm17cmVkfX0iLDAseyJzdHlsZSI6eyJib2R5Ijp7Im5hbWUiOiJkYXNoZWQifX19XSxbMSwwXSxbMywyXV0=&macro_url=https%3A%2F%2Fraw.githubusercontent.com%2FdFoiler%2Fnotes%2Fmaster%2Fnir.tex
	\[\begin{tikzcd}
		X & Y \\
		{X_{\mathrm{red}}} & {Y_{\mathrm{red}}}
		\arrow["\pi", from=1-1, to=1-2]
		\arrow["{\pi_{\mathrm{red}}}", dashed, from=2-1, to=2-2]
		\arrow[from=2-1, to=1-1]
		\arrow[from=2-2, to=1-2]
	\end{tikzcd}\]
	commute, where the vertical maps are the identity on topological spaces. In particular, $\pi_{\mathrm{red}}$ will still be of finite type, which is something that we can just check on affine open subsets by hand.
	\item We replace $\pi(C)$ in general with just $\im\pi$. Given a constructible subset $C\subseteq X$, we want to show $\pi(C)$ is constructible. However, we may write $C$ as a finite union
	\[C=\bigcup_{i=1}^n(U_i\cap V_i)\]
	of locally closed subsets $U_i\cap V_i$ with
	\[\pi(C)=\bigcup_{i=1}^n\pi(U_i\cap V_i).\]
	In particular, it suffices to show that any given $\pi(U_i\cap V_i)$, so we just replace $X$ with $U_i\cap V_i$, which has a scheme structure as a closed subscheme of an open subscheme. So it suffices to show $\im\pi=\pi(X)$ itself is constructible. Notably, $\pi$ is still of finite type because we're staying Noetherian (namely, everything is quasicompact).
	\item We make $X$ and $Y$ affine. We can write $Y$ as
	\[Y=\bigcup_{i=1}^nY_i\]
	where $Y_i$ is affine. So we let $\pi_i$ be the restricted map $\pi^{-1}Y_i\to Y_i$, so we can write
	\[\pi(X)=\bigcup_{i=1}^n\pi_i\left(\pi^{-1}(Y_i)\right),\]
	and it again suffices to just show that each of the $\pi_i$ are outputting a constructible subset of $Y$. Notably, everything is Noetherian, so $\pi^{-1}Y_i$ is quasicompact, so we can replace $\pi^{-1}Y_i$ with various restrictions to affine subsets.
	\item We are now in the situation where $\pi\colon\Spec A\to\Spec B$ is a morphism of finite type, so we have $A=B[x_1,\ldots,x_n]/I$ where $\pi^\sharp$ is the canonical induced map. We can even force $\Spec A$ and $\Spec B$ to be irreducible because there are only finitely many irreducible components, so we are forcing $A$ and $B$ to be integral domains. Lastly, we will also force $B$ to embed into $A$, which is equivalent to the map $\pi\colon\Spec A\to\Spec B$ being having dense image; to see this, simply replace $\Spec B$ with $\overline{\pi(\Spec A)}$.
\end{enumerate}
To continue the reduction, we recall Noether normalization.
\begin{theorem} \label{thm:noeth-normal}
	Fix a field $K$ and $R$ is a finitely generated $K$-algebra, then we can find $x_1,\ldots,x_n\in R$ such that $R$ is in fact a finitely generated $K[x_1,\ldots,x_n]$-module; i.e., the induced map $K[x_1,\ldots,x_n]\to R$ is a finite ring homomorphism.
\end{theorem}
To apply \autoref{thm:noeth-normal}, we need to upgrade it to go beyond $K$-algebras.
\begin{lemma}
	Fix an integral embedding $B\subseteq A$ such that $A$ is a finitely generated $B$-algebra. Then there is a nonzero $s\in B$ such that the embedding
	\[B_s[x_1,\ldots,x_n]\into A_s\]
	is a finite ring homomorphism.
\end{lemma}
Intuitively, \autoref{thm:noeth-normal} communicates what happens at $(0)$, which is like testing what happens on the ``generic fiber.'' Then the above result ``spreads out'' the generic fiber to all of $D(s)$.
\begin{proof}
	We apply \autoref{thm:noeth-normal}. Set $K\coloneqq\op{Frac}B$, and we have a canonical embedding $B\into K$. Then we set $S\coloneqq B\setminus\{0\}$ so that $K=\into S^{-1}A$ makes $S^{-1}A$ finitely generated over $K$.

	Thus, applying \autoref{thm:noeth-normal}, we can find $x_1,\ldots,x_n\in S^{-1}A$ such that $S^{-1}A$ is finite over $K[x_1,\ldots,x_n]$. We now choose our single element $s\in B$ to be large enough to get the result. To begin, make $s$ divisible by the denominators of the $x_i$. Further, we know $A$ will be generated by some finitely many $y_i$s over $B$, but as elements of $S^{-1}A$, they satisfy a monic polynomial with coefficients in $K[x_1,\ldots,x_n]$. Adding the denominators of all these polynomials to $s$, we see that the $y_i$s are integral over $B_s[x_1,\ldots,x_n]$ and hence contained. It follows that we have generated $A$, so $A_s$ is indeed finite over $B_s[x_1,\ldots,x_n]$.\todo{What?}
\end{proof}
In total, we currently have a diagram which looks like
% https://q.uiver.app/?q=WzAsNSxbMCwwLCJcXFNwZWMgQV9zIl0sWzEsMCwiXFxTcGVjIEJfc1t4XzEsXFxsZG90cyx4X25dIl0sWzIsMCwiXFxTcGVjIEJfcyJdLFsyLDEsIlxcU3BlYyBCIl0sWzAsMSwiXFxTcGVjIEEiXSxbNCwzLCJcXHBpIl0sWzAsNCwiIiwwLHsic3R5bGUiOnsidGFpbCI6eyJuYW1lIjoiaG9vayIsInNpZGUiOiJ0b3AifX19XSxbMiwzLCIiLDIseyJzdHlsZSI6eyJ0YWlsIjp7Im5hbWUiOiJob29rIiwic2lkZSI6InRvcCJ9fX1dLFswLDEsIlxccGlfcyJdLFsxLDJdXQ==&macro_url=https%3A%2F%2Fraw.githubusercontent.com%2FdFoiler%2Fnotes%2Fmaster%2Fnir.tex
\[\begin{tikzcd}
	{\Spec A_s} & {\Spec B_s[x_1,\ldots,x_n]} & {\Spec B_s} \\
	{\Spec A} && {\Spec B}
	\arrow["\pi", from=2-1, to=2-3]
	\arrow[hook, from=1-1, to=2-1]
	\arrow[hook, from=1-3, to=2-3]
	\arrow["{\pi_s}", from=1-1, to=1-2]
	\arrow[from=1-2, to=1-3]
\end{tikzcd}\]
where the vertical embeddings are open. We will finish the proof. The point is that we are almost done ``locally'' by looking at particular open sets.

\end{document}