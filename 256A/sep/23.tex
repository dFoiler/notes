% !TEX root = ../notes.tex

\documentclass[../notes.tex]{subfiles}

\begin{document}

\section{September 23}

Today we place some finiteness conditions on morphisms.
% V 8.1--8.3
% 8.3 has definitions

\subsection{Quasicompact and Quasiseparated}
For today, we will denote a morphism of schemes $\pi\colon X\to Y$.
\begin{definition}[Quasicompact]
	A scheme morphism $\pi\colon X\to Y$ is \textit{quasicompact} if and only if all affine open subschemes $U\subseteq Y$ make $\pi^{-1}(U)\subseteq X$ a quasicompact topological space.
\end{definition}
\begin{example}
	Fix a morphism $\pi\colon X\to Y$, where $X$ is a Noetherian scheme and hence a Noetherian space. Then $\pi$ is quasicompact because any (affine) open subset $U\subseteq Y$ makes $\pi^{-1}U\subseteq X$ an open and hence quasicompact set, where we are using \autoref{lem:opensarecompact}.
\end{example}
\begin{example}
	Any closed embedding $\pi\colon X\to Y$ is quasicompact. Indeed, given an affine open subset $U\subseteq Y$, we note that $\pi^{-1}(U)$ is homeomorphic through $\pi$ to $\pi(X)\cap U$. However, $U$ is quasicompact (it's affine), so the closed subset $\pi(X)\cap U$ is a closed subset in a quasicompact space and therefore quasicompact.
\end{example}
\begin{nex}
	An open embedding need not be quasicompact. For example, an affine scheme can have open subschemes which are not quasicompact.
\end{nex}
To define quasiseparated, we will need to have the adjective on topological spaces.
\begin{definition}[Quasiseperated]
	A topological space $X$ is \textit{quasiseparated} if and only if the intersection of two quasicompact open subsets is still quasicompact.
\end{definition}
\begin{example}
	Any Noetherian space is quasiseparated because any open subset is quasicompact by \autoref{lem:opensarecompact}.
\end{example}
\begin{example} \label{ex:local-noetherian-qs}
	Locally Noetherian schemes $X$ are quasiseparated: given quasicompact open subsets $U,V\subseteq X$, give $U$ a finite affine open covers $\{U_i\}_{i=1}^n$. Then $U_i$ is an affine scheme of a Noetherian ring, so $U_i$ is a Noetherian space by \autoref{ex:noetherianringisnoetherian}. Thus, $U_i\cap V$ is quasicompact for each $i$ by \autoref{lem:opensarecompact}, so the finite union $U\cap V=\bigcup_{i=1}^n(U_i\cap V)$ is still quasicompact.
\end{example}
\begin{example} \label{ex:qs-open-subset}
	If $X$ is a quasiseparated space, and $U\subseteq X$ is an open subset, then $U$ is still quasiseparated: any quasicompact open subsets $U_1,U_2\subseteq U$ are also quasicompact open subsets of $X$, which means $U_1\cap U_2$ is quasicompact because $X$ is quasiseparated.
\end{example}
And here is our definition.
\begin{defi}[Quasiseparated]
	A scheme morphism $\pi\colon X\to Y$ is \textit{quasiseparated} if and only if all affine open subschemes $U\subseteq Y$ makes $\pi^{-1}(U)$ a quasiseparated topological space.
\end{defi}
\begin{remark} \label{rem:affine-qs-condition}
	Equivalently, $\pi\colon X\to Y$ is quasiseparated if and only if given any affine open subset $U\subseteq Y$ and more affine open subsets $V_1,V_2\subseteq\pi^{-1}(U)$, we can give $V_1\cap V_2$ a finite affine open cover.
	\begin{itemize}
		\item If $\pi$ is quasiseparated, then $\pi^{-1}(U)$ is quasiseparated, so $V_1\cap V_2$ is quasicompact and thus has a finite affine open cover (from any affine open cover).
		\item If $\pi$ satisfies the condition, suppose we have quasicompact subsets $V_1,V_2\subseteq\pi^{-1}(U)$. Then we can give $V_1$ and $V_2$ finite affine open covers $\mc V_1$ and $\mc V_2$, and we see
		\[V_1\cap V_2=\Bigg(\bigcup_{W_1\in\mc V_1}W_1\Bigg)\cap\Bigg(\bigcup_{W_2\in\mc V_2}W_2\Bigg)=\bigcup_{W_1\in\mc V_1,W_2\in\mc V_2}(W_1\cap W_2),\]
		where $W_1\cap W_2$ is quasicompact and thus has a finite affine open cover. Synthesizing our finite cover by finite affine open covers, we see that we in total have given $V_1\cap V_2$ a finite affine open cover.
	\end{itemize}
\end{remark}
\begin{example}
	Fix a scheme morphism $\pi\colon X\to Y$. If $X$ is quasiseparated (e.g., $X$ is locally Noetherian, using \autoref{ex:local-noetherian-qs}), then $\pi$ is quasiseparated. Indeed, for any (affine) open subset $V\subseteq Y$, we see that $\pi^{-1}(V)\subseteq X$ is open and therefore quasiseparated by \autoref{ex:qs-open-subset}.
\end{example}
It turns out that a scheme $(X,\OO_X)$ is quasicompact/quasiseparated if and only if its morphism $(X,\OO_X)\to(\Spec\ZZ,\OO_{\Spec\ZZ})$ is quasicompact/quasiseparated; we will show this later.
\begin{remark}
	A scheme being quasiseparated is a very reasonable smallness condition, weaker than being locally Noetherian. We will later define what it means for a morphism/scheme to be ``separated,'' which will be stronger than this and approximately mean Hausdorff.
\end{remark}

\subsection{Quasicompactness is Reasonable}
Here are some equivalent definitions for being quasicompact.
\begin{lemma} \label{lem:non-affine-qc}
	A morphism $\pi\colon X\to Y$ is quasicompact if and only if every quasicompact subset $U\subseteq Y$ has $\pi^{-1}U$ also quasicompact.
\end{lemma}
\begin{proof}
	If the conclusion is true, then $\pi$ is certainly quasicompact because affine open subsets are necessarily quasicompact.

	On the other hand, suppose $\pi$ is quasicompact, and pick up a quasicompact subset $U\subseteq Y$. Now, $U$ as an open subscheme can be given an affine open cover $\mc V$, but because $U$ is quasicompact, we may assume that $\mc V$ is finite. But then
	\[\pi^{-1}(U)=\bigcup_{V\in\mc V}\pi^{-1}(V)\]
	is the finite union of quasicompact sets, where the $\pi^{-1}(V)$ is quasicompact because the $V$ are affine. Thus, $\pi^{-1}(U)$ is quasicompact.
\end{proof}
\begin{lemma} \label{lem:mainqclemma}
	Fix a morphism $\pi\colon X\to Y$ of schemes. Then $\pi$ is quasicompact if and only if there is an affine open cover $\mc U$ of $Y$ such that each $\pi^{-1}(U)$ is quasicompact for each $U\in\mc U$.
\end{lemma}
\begin{proof}
	In one direction, if $\pi$ is quasicompact, then any affine open cover $\mc U$ has each $U\in\mc U$ affine, so we see $\pi^{-1}(U)\subseteq X$ is quasicompact by hypothesis on $\pi$.

	The other direction is harder. Fix an affine open cover $\{U_\alpha\}_{\alpha\in\lambda}$ of $Y$ with $\varphi_\alpha\colon\Spec A_\alpha\cong U_\alpha$, and we are given that $\pi^{-1}(U_\alpha)$ is quasicompact for each $\alpha$. Now, for any quasicompact open subset $U\subseteq Y$, we need to show that $\pi^{-1}(U)$ is quasicompact.

	Well, using the distinguished base of each $U_\alpha\cong\Spec A_\alpha$, we can write
	\[U\cap U_\alpha=\bigcup_{\beta\in\lambda_\alpha}\varphi(D(f_{\alpha,\beta}))\]
	for some elements $\alpha,\beta\in\lambda$ (\autoref{rem:distinguishedbase}). It follows that
	\[U=\bigcup_{\alpha\in\lambda}(U\cap U_\alpha)=\bigcup_{\alpha\in\lambda}\bigcup_{\beta\in\lambda_\alpha}\varphi(D(f_{\alpha,\beta})).\]
	This provides an open cover of $U$, so the quasicompactness of $U$ forces us to have a finite subcover; let $\lambda'$ denote the finite set of $(\alpha,\beta)$ such that $\varphi(D(f_{\alpha,\beta}))$ cover $U$.

	It follows that
	\[\pi^{-1}(U)=\bigcup_{(\alpha,\beta)\in\lambda'}\pi^{-1}(\varphi(D(f_{\alpha,\beta}))).\]
	However, each $\varphi(D(f_{\alpha,\beta}))$ is affine, so their preimages under $\pi$ are quasicompact, so $\pi^{-1}(U)$ is the finite union of quasicompact sets and hence quasicompact. This finishes.
\end{proof}
So here are some quick results.
\begin{cor} \label{cor:qc-from-qc-morphisms}
	Fix a morphism $\pi\colon X\to Y$ of schemes. If $Y$ is affine, then $\pi$ is quasicompact if and only if $X$ is quasicompact.
\end{cor}
\begin{proof}
	We apply \autoref{lem:mainqclemma}. If $\pi$ is quasicompact, then the affine open subset $Y\subseteq Y$ must have $X=\pi^{-1}Y$ quasicompact by definition. Conversely, if $X$ is quasicompact, then we use the affine open cover $\{Y\}$ on $Y$ to note that $\pi$ is quasicompact because $\pi^{-1}(Y)=X$ is, by \autoref{lem:mainqclemma}.
\end{proof}
\begin{example}
	We see from \autoref{cor:qc-from-qc-morphisms} that a scheme $X$ is quasicompact if and only if its unique morphism $X\to\Spec\ZZ$ is quasicompact. (Recall this morphism is unique by \autoref{cor:spec-z-final}.)
\end{example}
\begin{remark}
	In fact, if a class of morphisms $P$ is affine-local on the target, then it is actually local on the target. Indeed, fix a morphism $\pi\colon X\to Y$ and give $Y$ an open cover $\{Y_\alpha\}_{\alpha\in\lambda}$, and give each $Y_\alpha$ an affine open cover $\{U_{\alpha,\beta}\}_{\beta\in\lambda_\alpha}$.
	\begin{itemize}
		\item Suppose $\pi\in P$; we want to show $\pi|_{\pi^{-1}Y_\alpha'}$ is in $P$ for some fixed $\alpha'$. Well, the $\{U_{\alpha,\beta}\}$ form affine open cover of $Y$, so $\pi|_{\pi^{-1}U_{\alpha',\beta}}\in P$ for each $\beta$, so $\pi|_{\pi^{-1}Y_{\alpha'}}\in P$ because $P$ is affine-local on the target.
		\item Suppose $\pi|_{Y_\alpha}\in P$ for each $\alpha$. Then, using the affine open covers of $Y_\alpha$, we see that $\pi|_{\pi^{-1}U_{\alpha,\beta}}\in P$ for each $\alpha$ and $\beta$, so $\pi\in P$ follows.
	\end{itemize}
\end{remark}
In light of the above remark, we will make little distinction between being local on the target and affine-local on the target.

The above results are important enough that we will want to give it a name.
\begin{definition}[Affine-local on the target]
	Let $P$ be a class of morphisms. We say that $P$ is \textit{affine-local on the target} if and only if a morphism $\pi\colon X\to Y$ is in $P$ if and only if there is an affine open cover $\{Y_\alpha\}_{\alpha\in\lambda}$ such that all the restricted maps $\pi|_{\pi^{-1}Y_\alpha}\colon\pi^{-1}Y_\alpha\to Y_\alpha$ are also in $P$.
\end{definition}
\begin{example}
	Quasicompact morphisms are affine-local on the target, from \autoref{lem:mainqclemma}. Certainly if $\pi$ is quasicompact, then for any affine open subset $U\subseteq Y$, we see $\pi^{-1}U$ is quasicompact, so the restriction $\pi|_{\pi^{-1}U}\colon\pi^{-1}U\to U$ is quasicompact by \autoref{cor:qc-from-qc-morphisms}. Conversely, if all the restrictions to $\pi|_{\pi^{-1}Y_\alpha}$ are quasicompact, then because $Y_\alpha\subseteq Y$ is quasicompact, $\pi^{-1}(Y_\alpha)$ is quasicompact for each $\alpha$, so $\pi$ is quasicompact by \autoref{lem:mainqclemma}.
\end{example}
Here are a few niceness checks.
\begin{corollary} \label{cor:qc-is-comp-preserve}
	Fix quasicompact scheme morphisms $\varphi\colon X\to Y$ and $\psi\colon Y\to Z$. Then $\psi\circ\varphi$ is quasicompact.
\end{corollary}
\begin{proof}
	We use \autoref{lem:non-affine-qc}. Pick up any quasicompact subset $W\subseteq Z$. Then $\psi^{-1}(W)\subseteq Y$ is quasicompact by \autoref{lem:non-affine-qc}, so $(\psi\circ\varphi)^{-1}(W)=\varphi^{-1}(\psi^{-1}(W))$ is quasicompact again by \autoref{lem:non-affine-qc}.
\end{proof}
Once more, it will be useful to have language to describe the above.
\begin{definition}[Preserved by composition]
	Let $P$ be a class of morphisms. We say that $P$ is \textit{preserved by composition} if and only if, for any pair of morphisms $\varphi\colon X\to Y$ and $\psi\colon Y\to Z$ in $P$, we have $\psi\circ\varphi$ also in $P$.
\end{definition}
\begin{example}
	By \autoref{cor:qc-is-comp-preserve}, quasicompact morphisms are preserved by composition.
\end{example}
\begin{lemma} \label{lem:qc-base-change}
	Suppose we have a pullback square
	% https://q.uiver.app/?q=WzAsNCxbMCwwLCJYXFx0aW1lc19TWSJdLFsxLDAsIlgiXSxbMCwxLCJZIl0sWzEsMSwiUyJdLFsyLDMsIlxccHNpX1kiXSxbMCwxLCJcXHBpX1giXSxbMCwyLCJcXHBpX1kiXSxbMSwzLCJcXHBzaV9YIl0sWzAsMywiIiwyLHsic3R5bGUiOnsibmFtZSI6ImNvcm5lciJ9fV1d&macro_url=https%3A%2F%2Fraw.githubusercontent.com%2FdFoiler%2Fnotes%2Fmaster%2Fnir.tex
	\[\begin{tikzcd}
		{X\times_SY} & X \\
		Y & S
		\arrow["{\psi_Y}", from=2-1, to=2-2]
		\arrow["{\pi_X}", from=1-1, to=1-2]
		\arrow["{\pi_Y}", from=1-1, to=2-1]
		\arrow["{\psi_X}", from=1-2, to=2-2]
		\arrow["\lrcorner"{anchor=center, pos=0.125}, draw=none, from=1-1, to=2-2]
	\end{tikzcd}\]
	of schemes. If $\pi$ is quasicompact, then $\pi'$ is quasicompact.
\end{lemma}
\begin{proof}
	The main point is to reduce to the affine case, where everything is clear. Let $\pi_S\coloneqq\psi_Y\circ\pi_Y=\psi_X\circ\pi_X$, for brevity. Give $S$ an affine open cover $\{S_\alpha\}_{\alpha\in\lambda}$. For each $\alpha\in\lambda$, we give $\psi_X^{-1}(S_\alpha)\subseteq X$ an affine open cover $\{X_{\alpha,\beta}\}_{\alpha\in\lambda,\beta\in\kappa_\alpha}$. Then we build the tower
	% https://q.uiver.app/?q=WzAsNixbMCwxLCJcXHBpX1Neey0xfShTX1xcYWxwaGEpIl0sWzEsMSwiXFxwc2lfWF57LTF9KFNfXFxhbHBoYSkiXSxbMCwyLCJcXHBzaV9ZXnstMX0oU19cXGFscGhhKSJdLFsxLDIsIlNfXFxhbHBoYSJdLFsxLDAsIlhfe1xcYWxwaGEsXFxiZXRhfSJdLFswLDAsIlxccGlfWF57LTF9KFhfe1xcYWxwaGEsXFxiZXRhfSkiXSxbMiwzLCJcXHBzaV9ZfF97XFxwc2lfWV57LTF9KFNfXFxhbHBoYSl9IiwyXSxbMCwxLCJcXHBpX1h8X3tcXHBpX1Neey0xfShTX1xcYWxwaGEpfSJdLFswLDIsIlxccGlfWXxfe1xccGlfU157LTF9KFNfXFxhbHBoYSl9Il0sWzEsMywiXFxwc2lfWCJdLFs0LDEsIiIsMCx7InN0eWxlIjp7InRhaWwiOnsibmFtZSI6Imhvb2siLCJzaWRlIjoidG9wIn19fV0sWzUsNCwiXFxwaV9YfF97XFxwaV9YXnstMX0oWF97XFxhbHBoYSxcXGJldGF9KX0iXSxbNSwwLCIiLDAseyJzdHlsZSI6eyJ0YWlsIjp7Im5hbWUiOiJob29rIiwic2lkZSI6InRvcCJ9fX1dXQ==&macro_url=https%3A%2F%2Fraw.githubusercontent.com%2FdFoiler%2Fnotes%2Fmaster%2Fnir.tex
	\[\begin{tikzcd}
		{\pi_X^{-1}(X_{\alpha,\beta})} & {X_{\alpha,\beta}} \\
		{\pi_S^{-1}(S_\alpha)} & {\psi_X^{-1}(S_\alpha)} \\
		{\psi_Y^{-1}(S_\alpha)} & {S_\alpha}
		\arrow["{\psi_Y|_{\psi_Y^{-1}(S_\alpha)}}"', from=3-1, to=3-2]
		\arrow["{\pi_X|_{\pi_S^{-1}(S_\alpha)}}", from=2-1, to=2-2]
		\arrow["{\pi_Y|_{\pi_S^{-1}(S_\alpha)}}", from=2-1, to=3-1]
		\arrow["{\psi_X}", from=2-2, to=3-2]
		\arrow[hook, from=1-2, to=2-2]
		\arrow["{\pi_X|_{\pi_X^{-1}(X_{\alpha,\beta})}}", from=1-1, to=1-2]
		\arrow[hook, from=1-1, to=2-1]
	\end{tikzcd}\]
	and note that the bottom square is a pullback square by \autoref{lem:fp-open-cover-base}, the top square is a pullback square by \autoref{lem:open-fp}, so the total rectangle is a pullback square by \autoref{lem:smalls-to-big-pullback}. Now, because being quasicompact is affine-local on the target by \autoref{lem:mainqclemma}, it suffices to show that each restricted map $\pi_X|_{\pi_X^{-1}(X_{\alpha,\beta})}$ is quasicompact. Notably, by \autoref{lem:non-affine-qc}, the restriction $\psi_Y|_{\psi_Y^{-1}(S_\alpha)}$ is quasicompact.

	Thus, we fix some $\alpha$ and $\beta$. We now rename our variables, replacing $S_\alpha$ with $S$, $X_{\alpha,\beta}$ with $X$, and $\psi_Y^{-1}(S_\alpha)$ with $Y$, and we rename our morphisms to fit the pullback square
	% https://q.uiver.app/?q=WzAsNCxbMSwxLCJTIl0sWzEsMCwiWCJdLFswLDEsIlkiXSxbMCwwLCJYXFx0aW1lc19TWSJdLFszLDEsIlxccGlfWCJdLFsxLDAsIlxccHNpX1giXSxbMiwwLCJcXHBzaV9ZIl0sWzMsMiwiXFxwaV9ZIiwyXSxbMywwLCIiLDEseyJzdHlsZSI6eyJuYW1lIjoiY29ybmVyIn19XV0=&macro_url=https%3A%2F%2Fraw.githubusercontent.com%2FdFoiler%2Fnotes%2Fmaster%2Fnir.tex
	\[\begin{tikzcd}
		{X\times_SY} & X \\
		Y & S
		\arrow["{\pi_X}", from=1-1, to=1-2]
		\arrow["{\psi_X}", from=1-2, to=2-2]
		\arrow["{\psi_Y}", from=2-1, to=2-2]
		\arrow["{\pi_Y}"', from=1-1, to=2-1]
		\arrow["\lrcorner"{anchor=center, pos=0.125}, draw=none, from=1-1, to=2-2]
	\end{tikzcd}\]
	though we now have both $X$ and $S$ affine. We are given that $\psi_Y$ is quasicompact, and we would like to show that $\pi_X$ is also quasicompact. By \autoref{cor:qc-from-qc-morphisms}, it suffices to show that $X\times_SY$ is quasicompact.

	However, we note that $S$ is affine, so because $\psi_Y$ is quasicompact, \autoref{cor:qc-from-qc-morphisms} tells us that $Y$ is quasicompact. As such, we give $Y$ a finite affine open cover $\{Y_i\}_{i=1}^n$. By the construction of the fiber product in \autoref{lem:keyfibercase}, we note that $X\times_SY$ is covered by the schemes $X\times_SY_i$.

	But now each scheme $X\times_SY_i$ can have $X\cong\Spec A$ and $S\cong\Spec R$ and $Y_i\cong\Spec B_i$ so that
	\[X\times_SY_i\cong\Spec A\times_{\Spec R}\Spec B_i,\]
	which is isomorphic to $\Spec A\otimes_RB_i$ by \autoref{lem:affine-fp}. Notably, $X\times_SY_i$ is an affine scheme and hence quasicompact, so $X\times_SY$ is a finite union of quasicompact subschemes and therefore quasicompact.
\end{proof}
And here is the name.
\begin{definition}[Preserved by base change]
	Let $P$ be a class of morphisms. We say that $P$ is \textit{preserved by base change} if and only if $\varphi\colon X\to S$ being in $P$ implies that the canonical morphism $\pi_Y\colon X\times_SY\to Y$ is still in $P$, for any scheme $Y$ over $S$.
\end{definition}
% \begin{corollary}
% 	A morphism being quasicompact is affine local on the target: given an affine open cover $\mc U$ of $Y$, the restrictions $\pi|_{\pi^{-1}U}$ for $U\in\mc U$ are all quasicompact if and only if $\pi$ is quasicompact.
% \end{corollary}
% \begin{proof}
% 	Apply \autoref{lem:mainqclemma}.
% \end{proof}
\begin{remark}
	Being preserved under composition and base-change shows that the product of two quasicompact morphisms is quasicompact. Namely, $\varphi\colon X\to Y$ and $\varphi'\colon X'\to Y'$ makes us build the product morphism $\varphi\times\varphi'\colon X\times_{\Spec\ZZ}X'\to Y\times_{\Spec\ZZ}Y'$ by applying base-change twice.
\end{remark}

\subsection{Diagonal Morphisms}
It will turn out that two important classes of morphisms---quasiseparated and separated morphisms---are best thought of in terms of ``diagonal morphisms.'' Thus, we will spend a little time discussing these.
\begin{notation}
	Fix schemes $X_1$ and $X_2$ over a scheme $Y$. Given morphisms $\alpha_1\colon Z\to X_1$ and $\alpha_2\colon Z\to X_2$ making the outer square of
	% https://q.uiver.app/?q=WzAsNSxbMSwyLCJYXzIiXSxbMiwxLCJYXzEiXSxbMiwyLCJZIl0sWzEsMSwiWF8xXFx0aW1lc19ZWF8yIl0sWzAsMCwiWiJdLFszLDFdLFszLDBdLFswLDJdLFs0LDEsIlxcYWxwaGFfMSIsMCx7ImN1cnZlIjotMn1dLFsxLDJdLFs0LDAsIlxcYWxwaGFfMiIsMix7ImN1cnZlIjoyfV0sWzQsMywiIiwxLHsic3R5bGUiOnsiYm9keSI6eyJuYW1lIjoiZGFzaGVkIn19fV0sWzMsMiwiIiwxLHsic3R5bGUiOnsibmFtZSI6ImNvcm5lciJ9fV1d&macro_url=https%3A%2F%2Fraw.githubusercontent.com%2FdFoiler%2Fnotes%2Fmaster%2Fnir.tex
	\[\begin{tikzcd}
		Z \\
		& {X_1\times_YX_2} & {X_1} \\
		& {X_2} & Y
		\arrow[from=2-2, to=2-3]
		\arrow[from=2-2, to=3-2]
		\arrow[from=3-2, to=3-3]
		\arrow["{\alpha_1}", curve={height=-12pt}, from=1-1, to=2-3]
		\arrow[from=2-3, to=3-3]
		\arrow["{\alpha_2}"', curve={height=12pt}, from=1-1, to=3-2]
		\arrow[dashed, from=1-1, to=2-2]
		\arrow["\lrcorner"{anchor=center, pos=0.125}, draw=none, from=2-2, to=3-3]
	\end{tikzcd}\]
	commute, we let $(\alpha_1,\alpha_2)$ denote the induced arrow.
\end{notation}
\begin{example}
	Given a morphism $\varphi\colon X\to Y$, we note that $\varphi\circ{\id_X}=\varphi\circ{\id_X}$, so we have induced a morphism $({\id_X},{\id_X})\colon X\to X\times_YX$. This morphism is called the ``diagonal morphism'' and is denoted $\Delta\varphi\coloneqq({\id_X},{\id_X})$.
\end{example}
Having defined our diagonal morphisms, we now pick up some facts about them. The following pullback square will prove quite helpful.
\begin{lemma}[Magic diagram] \label{lem:magic-diagram}
	Fix morphisms $\alpha_1\colon X_1\to Y$ and $\alpha_2\colon X_2\to Y$ and $\iota\colon Y\to Z$. Then the diagram
	% https://q.uiver.app/?q=WzAsNCxbMCwwLCJYXzFcXHRpbWVzX1lYXzIiXSxbMSwwLCJYXzFcXHRpbWVzX1pZXzIiXSxbMCwxLCJZIl0sWzEsMSwiWV8xXFx0aW1lcyBZXzIiXSxbMCwxXSxbMSwzXSxbMCwyXSxbMiwzXSxbMCwzLCIiLDEseyJzdHlsZSI6eyJuYW1lIjoiY29ybmVyIn19XV0=&macro_url=https%3A%2F%2Fraw.githubusercontent.com%2FdFoiler%2Fnotes%2Fmaster%2Fnir.tex
	\[\begin{tikzcd}
		{X_1\times_YX_2} & {X_1\times_ZY_2} \\
		Y & {Y\times_Z Y}
		\arrow[from=1-1, to=1-2]
		\arrow[from=1-2, to=2-2]
		\arrow[from=1-1, to=2-1]
		\arrow[from=2-1, to=2-2]
		\arrow["\lrcorner"{anchor=center, pos=0.125}, draw=none, from=1-1, to=2-2]
	\end{tikzcd}\]
	equipped with the natural maps is a pullback square. Here, we assume that the relevant fiber products exist.
\end{lemma}
\begin{proof}
	We proceed by force; we begin by naming our maps. Let $\pi_1\colon X_1\times_ZX_2\to X_1$ and $\pi_2\colon X_1\times_ZX_2\to X_2$ be the canonical projections. Analogously, we let $\varpi_1\colon X_1\times_YX_2\to X_1$ and $\varpi_2\colon X_1\times_YX_2\to X_2$ be the canonical projections so that the diagram
	% https://q.uiver.app/?q=WzAsNSxbMCwwLCJYXzFcXHRpbWVzX1lYXzIiXSxbMSwxLCJYXzFcXHRpbWVzX1pYXzIiXSxbMiwxLCJYXzEiXSxbMSwyLCJYXzIiXSxbMiwyLCJaIl0sWzIsNCwiXFxpb3RhXFxhbHBoYV8xIl0sWzMsNCwiXFxpb3RhXFxhbHBoYV8yIiwyXSxbMSwzLCJcXHBpXzIiXSxbMSwyLCJcXHBpXzIiLDJdLFswLDIsIlxcdmFycGlfMSIsMCx7ImN1cnZlIjotMn1dLFswLDMsIlxcdmFycGlfMiIsMix7ImN1cnZlIjoyfV0sWzEsNCwiIiwxLHsic3R5bGUiOnsibmFtZSI6ImNvcm5lciJ9fV0sWzAsMSwiXFx2YXJwaSIsMSx7InN0eWxlIjp7ImJvZHkiOnsibmFtZSI6ImRhc2hlZCJ9fX1dXQ==&macro_url=https%3A%2F%2Fraw.githubusercontent.com%2FdFoiler%2Fnotes%2Fmaster%2Fnir.tex
	\[\begin{tikzcd}
		{X_1\times_YX_2} \\
		& {X_1\times_ZX_2} & {X_1} \\
		& {X_2} & Z
		\arrow["{\iota\alpha_1}", from=2-3, to=3-3]
		\arrow["{\iota\alpha_2}"', from=3-2, to=3-3]
		\arrow["{\pi_2}", from=2-2, to=3-2]
		\arrow["{\pi_2}"', from=2-2, to=2-3]
		\arrow["{\varpi_1}", curve={height=-12pt}, from=1-1, to=2-3]
		\arrow["{\varpi_2}"', curve={height=12pt}, from=1-1, to=3-2]
		\arrow["\lrcorner"{anchor=center, pos=0.125}, draw=none, from=2-2, to=3-3]
		\arrow["\varpi"{description}, dashed, from=1-1, to=2-2]
	\end{tikzcd}\]
	commutes (indeed, $\alpha_1\circ\varpi_1=\alpha_2\circ\varpi_2$) and thus induces the morphism $\varpi$ making the diagram commute. Namely, $\varpi_1=\pi_1\circ\varpi$ and $\varpi_2=\pi_2\circ\varpi$. We take a moment to recognize that $\varpi$ is $(\varpi_1,\varpi_2)$.

	To induce the map $\Delta\iota\colon Y\to Y\times_ZY$, we let $\psi_1,\psi_2\colon Y\times_ZY\to Y$ denote the canonical projections, and we draw the diagram
	% https://q.uiver.app/?q=WzAsNSxbMSwyLCJZIl0sWzIsMSwiWSJdLFsyLDIsIloiXSxbMSwxLCJZXFx0aW1lc19aWSJdLFswLDAsIlkiXSxbMCwyLCJcXGlvdGEiXSxbMSwyLCJcXGlvdGEiLDJdLFszLDEsIlxccHNpXzEiLDJdLFszLDAsIlxccHNpXzIiXSxbNCwxLCIiLDAseyJjdXJ2ZSI6LTIsImxldmVsIjoyLCJzdHlsZSI6eyJoZWFkIjp7Im5hbWUiOiJub25lIn19fV0sWzQsMCwiIiwwLHsiY3VydmUiOjIsImxldmVsIjoyLCJzdHlsZSI6eyJoZWFkIjp7Im5hbWUiOiJub25lIn19fV0sWzQsMywiXFxEZWx0YVxcaW90YSIsMSx7InN0eWxlIjp7ImJvZHkiOnsibmFtZSI6ImRhc2hlZCJ9fX1dLFszLDIsIiIsMCx7InN0eWxlIjp7Im5hbWUiOiJjb3JuZXIifX1dXQ==&macro_url=https%3A%2F%2Fraw.githubusercontent.com%2FdFoiler%2Fnotes%2Fmaster%2Fnir.tex
	\[\begin{tikzcd}
		Y \\
		& {Y\times_ZY} & Y \\
		& Y & Z
		\arrow["\iota", from=3-2, to=3-3]
		\arrow["\iota"', from=2-3, to=3-3]
		\arrow["{\psi_1}"', from=2-2, to=2-3]
		\arrow["{\psi_2}", from=2-2, to=3-2]
		\arrow[curve={height=-12pt}, Rightarrow, no head, from=1-1, to=2-3]
		\arrow[curve={height=12pt}, Rightarrow, no head, from=1-1, to=3-2]
		\arrow["\Delta\iota"{description}, dashed, from=1-1, to=2-2]
		\arrow["\lrcorner"{anchor=center, pos=0.125}, draw=none, from=2-2, to=3-3]
	\end{tikzcd}\]
	and note that the outer square commutes because we're dealing with identities. This induces our desired diagonal morphism $\Delta\iota$.

	Next, we induce the map $(\alpha_1,\alpha_2)\colon X_1\times_ZX_2\to Y\times_ZY$ by noting that the outer ``square'' of
	% https://q.uiver.app/?q=WzAsNyxbMCwwLCJYXzFcXHRpbWVzX1lYXzIiXSxbMSwxLCJZXFx0aW1lc19aWSJdLFsxLDIsIlkiXSxbMiwxLCJZIl0sWzIsMiwiWiJdLFsxLDAsIlhfMSJdLFswLDEsIlhfMiJdLFsxLDMsIlxccHNpXzEiLDJdLFsxLDIsIlxccHNpXzIiXSxbMiw0LCJcXGlvdGEiLDJdLFszLDQsIlxcaW90YSJdLFswLDUsIlxccGlfMSJdLFswLDYsIlxccGlfMiIsMl0sWzUsMywiXFxhbHBoYV8xIl0sWzYsMiwiXFxhbHBoYV8yIiwyXSxbMCwxLCIiLDEseyJzdHlsZSI6eyJib2R5Ijp7Im5hbWUiOiJkYXNoZWQifX19XV0=&macro_url=https%3A%2F%2Fraw.githubusercontent.com%2FdFoiler%2Fnotes%2Fmaster%2Fnir.tex
	\[\begin{tikzcd}
		{X_1\times_YX_2} & {X_1} \\
		{X_2} & {Y\times_ZY} & Y \\
		& Y & Z
		\arrow["{\psi_1}"', from=2-2, to=2-3]
		\arrow["{\psi_2}", from=2-2, to=3-2]
		\arrow["\iota"', from=3-2, to=3-3]
		\arrow["\iota", from=2-3, to=3-3]
		\arrow["{\pi_1}", from=1-1, to=1-2]
		\arrow["{\pi_2}"', from=1-1, to=2-1]
		\arrow["{\alpha_1}", from=1-2, to=2-3]
		\arrow["{\alpha_2}"', from=2-1, to=3-2]
		\arrow[dashed, from=1-1, to=2-2]
	\end{tikzcd}\]
	commutes because $\alpha_1\circ\pi_1=\alpha_2\circ\pi_2$, so we have induced a dashed arrow we name $(\alpha_1,\alpha_2)$. Importantly, $\psi_1\circ(\alpha_1,\alpha_2)=\alpha_1\circ\pi_1=\alpha_2\circ\pi_2=\psi_2\circ(\alpha_1,\alpha_2)$.

	Lastly, we induce the map $X_1\times_YX_2\to Y$ just by $\alpha_1\circ\pi_1=\alpha_2\circ\pi_2$. To see that the magic diagram commutes, we note that there is at most morphism $X_1\times_YX_2\to Y\times_ZY$ which can fill into the dashed arrow of
	% https://q.uiver.app/?q=WzAsNyxbMCwwLCJYXzFcXHRpbWVzX1lYXzIiXSxbMSwxLCJZXFx0aW1lc19aWSJdLFsyLDIsIloiXSxbMiwxLCJZIl0sWzEsMiwiWSJdLFsxLDAsIlhfMSJdLFswLDEsIlhfMiJdLFswLDYsIlxccGlfMiIsMl0sWzEsMywiXFxwc2lfMSIsMl0sWzEsNCwiXFxwc2lfMiJdLFswLDUsIlxccGlfMSJdLFs1LDMsIlxcYWxwaGFfMSJdLFs2LDQsIlxcYWxwaGFfMiIsMl0sWzQsMiwiXFxpb3RhIiwyXSxbMywyLCJcXGlvdGEiXSxbMSwyLCIiLDEseyJzdHlsZSI6eyJuYW1lIjoiY29ybmVyIn19XSxbMCwxLCIiLDEseyJzdHlsZSI6eyJib2R5Ijp7Im5hbWUiOiJkYXNoZWQifX19XV0=&macro_url=https%3A%2F%2Fraw.githubusercontent.com%2FdFoiler%2Fnotes%2Fmaster%2Fnir.tex
	\[\begin{tikzcd}
		{X_1\times_YX_2} & {X_1} \\
		{X_2} & {Y\times_ZY} & Y \\
		& Y & Z
		\arrow["{\varpi_2}"', from=1-1, to=2-1]
		\arrow["{\psi_1}"', from=2-2, to=2-3]
		\arrow["{\psi_2}", from=2-2, to=3-2]
		\arrow["{\varpi_1}", from=1-1, to=1-2]
		\arrow["{\alpha_1}", from=1-2, to=2-3]
		\arrow["{\alpha_2}"', from=2-1, to=3-2]
		\arrow["\iota"', from=3-2, to=3-3]
		\arrow["\iota", from=2-3, to=3-3]
		\arrow["\lrcorner"{anchor=center, pos=0.125}, draw=none, from=2-2, to=3-3]
		\arrow[dashed, from=1-1, to=2-2]
	\end{tikzcd}\]
	to make the diagram commute. However, the diagram
	% https://q.uiver.app/?q=WzAsNyxbMCwwLCJYXzFcXHRpbWVzX1lYXzIiXSxbMSwxLCJZIl0sWzEsMCwiWF8xIl0sWzAsMSwiWF8yIl0sWzIsMiwiWVxcdGltZXNfWlkiXSxbMywyLCJZIl0sWzIsMywiWSJdLFswLDFdLFswLDMsIlxcdmFycGlfMiIsMl0sWzMsMSwiXFxhbHBoYV8yIl0sWzAsMiwiXFx2YXJwaV8xIl0sWzEsNCwiXFxEZWx0YVxcaW90YSIsMV0sWzIsMSwiXFxhbHBoYV8xIiwyXSxbNCw1LCJcXHBzaV8xIiwyXSxbNCw2LCJcXHBzaV8yIl0sWzIsNSwiXFxhbHBoYV8xIl0sWzEsNSwiIiwwLHsibGV2ZWwiOjIsInN0eWxlIjp7ImhlYWQiOnsibmFtZSI6Im5vbmUifX19XSxbMyw2LCJcXGFscGhhXzIiLDJdLFsxLDYsIiIsMix7ImxldmVsIjoyLCJzdHlsZSI6eyJoZWFkIjp7Im5hbWUiOiJub25lIn19fV1d&macro_url=https%3A%2F%2Fraw.githubusercontent.com%2FdFoiler%2Fnotes%2Fmaster%2Fnir.tex
	\[\begin{tikzcd}
		{X_1\times_YX_2} & {X_1} \\
		{X_2} & Y \\
		&& {Y\times_ZY} & Y \\
		&& Y
		\arrow[from=1-1, to=2-2]
		\arrow["{\varpi_2}"', from=1-1, to=2-1]
		\arrow["{\alpha_2}", from=2-1, to=2-2]
		\arrow["{\varpi_1}", from=1-1, to=1-2]
		\arrow["\Delta\iota"{description}, from=2-2, to=3-3]
		\arrow["{\alpha_1}"', from=1-2, to=2-2]
		\arrow["{\psi_1}"', from=3-3, to=3-4]
		\arrow["{\psi_2}", from=3-3, to=4-3]
		\arrow["{\alpha_1}", from=1-2, to=3-4]
		\arrow[Rightarrow, no head, from=2-2, to=3-4]
		\arrow["{\alpha_2}"', from=2-1, to=4-3]
		\arrow[Rightarrow, no head, from=2-2, to=4-3]
	\end{tikzcd}\]
	commutes mostly by construction of $\Delta\iota$, and the diagram
	% https://q.uiver.app/?q=WzAsNyxbMCwwLCJYXzFcXHRpbWVzX1lYXzIiXSxbMSwxLCJYXzFcXHRpbWVzX1pYXzIiXSxbMiwyLCJZXFx0aW1lc19aWSJdLFsyLDEsIlhfMSJdLFsxLDIsIlhfMiJdLFsyLDMsIlkiXSxbMywyLCJZIl0sWzEsNCwiXFxwaV8xIl0sWzAsMSwiXFx2YXJwaSIsMV0sWzEsMywiXFxwaV8yIiwyXSxbMSwyLCIoXFxhbHBoYV8xLFxcYWxwaGFfMikiLDFdLFszLDYsIlxcYWxwaGFfMSJdLFs0LDUsIlxcYWxwaGFfMiIsMl0sWzAsMywiXFx2YXJwaV8xIiwwLHsiY3VydmUiOi0yfV0sWzAsNCwiXFx2YXJwaV8yIiwyLHsiY3VydmUiOjJ9XSxbMiw1LCJcXHBzaV8yIl0sWzIsNiwiXFxwc2lfMSIsMl1d&macro_url=https%3A%2F%2Fraw.githubusercontent.com%2FdFoiler%2Fnotes%2Fmaster%2Fnir.tex
	\[\begin{tikzcd}
		{X_1\times_YX_2} \\
		& {X_1\times_ZX_2} & {X_1} \\
		& {X_2} & {Y\times_ZY} & Y \\
		&& Y
		\arrow["{\pi_1}", from=2-2, to=3-2]
		\arrow["\varpi"{description}, from=1-1, to=2-2]
		\arrow["{\pi_2}"', from=2-2, to=2-3]
		\arrow["{(\alpha_1,\alpha_2)}"{description}, from=2-2, to=3-3]
		\arrow["{\alpha_1}", from=2-3, to=3-4]
		\arrow["{\alpha_2}"', from=3-2, to=4-3]
		\arrow["{\varpi_1}", curve={height=-12pt}, from=1-1, to=2-3]
		\arrow["{\varpi_2}"', curve={height=12pt}, from=1-1, to=3-2]
		\arrow["{\psi_2}", from=3-3, to=4-3]
		\arrow["{\psi_1}"', from=3-3, to=3-4]
	\end{tikzcd}\]
	commutes by construction of both $\omega$ and $(\alpha_1,\alpha_2)$.

	We are now ready to show the universal property. Fix some object $W$ with maps $\varphi_F\colon W\to X_1\times_ZY_2$ and $\varphi_Y\colon W\to Y$ such that $(\alpha_1,\alpha_2)\circ\varphi_F=\Delta\iota\circ\varphi_Y$. Then we need a unique morphism $\varphi\colon W\to X_1\times_YX_2$ making the diagram
	% https://q.uiver.app/?q=WzAsNSxbMSwxLCJYXzFcXHRpbWVzX1lYXzIiXSxbMiwxLCJYXzFcXHRpbWVzX1pYXzIiXSxbMiwyLCJZXFx0aW1lc19aWSJdLFsxLDIsIlkiXSxbMCwwLCJXIl0sWzMsMiwiXFxEZWx0YVxcaW90YSIsMl0sWzAsMSwiXFx2YXJwaSIsMl0sWzEsMiwiKFxcYWxwaGFfMSxcXGFscGhhXzIpIl0sWzAsMywiXFxhbHBoYV8xXFxwaV8xIl0sWzQsMywiXFx2YXJwaGlfWSIsMix7ImN1cnZlIjoyfV0sWzQsMSwiXFx2YXJwaGlfRiIsMCx7ImN1cnZlIjotMn1dLFs0LDAsIlxcdmFycGhpIiwxLHsic3R5bGUiOnsiYm9keSI6eyJuYW1lIjoiZGFzaGVkIn19fV1d&macro_url=https%3A%2F%2Fraw.githubusercontent.com%2FdFoiler%2Fnotes%2Fmaster%2Fnir.tex
	\[\begin{tikzcd}
		W \\
		& {X_1\times_YX_2} & {X_1\times_ZX_2} \\
		& Y & {Y\times_ZY}
		\arrow["\Delta\iota"', from=3-2, to=3-3]
		\arrow["\varpi"', from=2-2, to=2-3]
		\arrow["{(\alpha_1,\alpha_2)}", from=2-3, to=3-3]
		\arrow["{\alpha_1\varpi_1}", from=2-2, to=3-2]
		\arrow["{\varphi_Y}"', curve={height=12pt}, from=1-1, to=3-2]
		\arrow["{\varphi_F}", curve={height=-12pt}, from=1-1, to=2-3]
		\arrow["\varphi"{description}, dashed, from=1-1, to=2-2]
	\end{tikzcd}\]
	commute. We show uniqueness and existence separately.
	\begin{itemize}
		\item Uniqueness: given $\varphi$, we claim that the diagram
		% https://q.uiver.app/?q=WzAsNSxbMCwwLCJXIl0sWzEsMSwiWF8xXFx0aW1lc19ZWF8yIl0sWzIsMSwiWF8xIl0sWzEsMiwiWF8yIl0sWzIsMiwiWSJdLFswLDEsIlxcdmFycGhpIiwxXSxbMCwyLCJcXHBpXzFcXHZhcnBoaV9GIiwwLHsiY3VydmUiOi0yfV0sWzAsMywiXFxwaV8yXFx2YXJwaGlfRiIsMix7ImN1cnZlIjoyfV0sWzEsMiwiXFx2YXJwaV8xIiwyXSxbMSwzLCJcXHZhcnBpXzIiXSxbMiw0LCJcXGFscGhhXzEiXSxbMyw0LCJcXGFscGhhXzIiLDJdXQ==&macro_url=https%3A%2F%2Fraw.githubusercontent.com%2FdFoiler%2Fnotes%2Fmaster%2Fnir.tex
		\[\begin{tikzcd}
			W \\
			& {X_1\times_YX_2} & {X_1} \\
			& {X_2} & Y
			\arrow["\varphi"{description}, from=1-1, to=2-2]
			\arrow["{\pi_1\varphi_F}", curve={height=-12pt}, from=1-1, to=2-3]
			\arrow["{\pi_2\varphi_F}"', curve={height=12pt}, from=1-1, to=3-2]
			\arrow["{\varpi_1}"', from=2-2, to=2-3]
			\arrow["{\varpi_2}", from=2-2, to=3-2]
			\arrow["{\alpha_1}", from=2-3, to=3-3]
			\arrow["{\alpha_2}"', from=3-2, to=3-3]
		\end{tikzcd}\]
		commutes, which will of course uniquely determine $\varphi$ by our pullback. Well, we see $\varpi_i\circ\varphi=\pi_i\circ\varpi\circ\varphi=\pi_i\circ\varphi_F$ for $i\in\{1,2\}$, which is what we wanted.
		\item Existence: as above, we claim that the diagram
		% https://q.uiver.app/?q=WzAsNSxbMCwwLCJXIl0sWzEsMSwiWF8xXFx0aW1lc19ZWF8yIl0sWzIsMSwiWF8xIl0sWzEsMiwiWF8yIl0sWzIsMiwiWSJdLFswLDIsIlxccGlfMVxcdmFycGhpX0YiLDAseyJjdXJ2ZSI6LTJ9XSxbMCwzLCJcXHBpXzJcXHZhcnBoaV9GIiwyLHsiY3VydmUiOjJ9XSxbMSwyLCJcXHZhcnBpXzEiLDJdLFsxLDMsIlxcdmFycGlfMiJdLFsyLDQsIlxcYWxwaGFfMSJdLFszLDQsIlxcYWxwaGFfMiIsMl1d&macro_url=https%3A%2F%2Fraw.githubusercontent.com%2FdFoiler%2Fnotes%2Fmaster%2Fnir.tex
		\[\begin{tikzcd}
			W \\
			& {X_1\times_YX_2} & {X_1} \\
			& {X_2} & Y
			\arrow["{\pi_1\varphi_F}", curve={height=-12pt}, from=1-1, to=2-3]
			\arrow["{\pi_2\varphi_F}"', curve={height=12pt}, from=1-1, to=3-2]
			\arrow["{\varpi_1}"', from=2-2, to=2-3]
			\arrow["{\varpi_2}", from=2-2, to=3-2]
			\arrow["{\alpha_1}", from=2-3, to=3-3]
			\arrow["{\alpha_2}"', from=3-2, to=3-3]
		\end{tikzcd}\]
		commutes, which will induce the desired morphism $\varphi$. Indeed, $\alpha_i\circ\pi_i\circ\varphi_F=\psi_i\circ(\alpha_1,\alpha_2)\circ\varphi_F=\psi_i\circ\Delta_Y\circ\varphi_Y=\varphi_Y$ for each $i\in\{1,2\}$.

		We now run our checks on $\varphi$. On one side, we see that $\alpha_1\circ\varpi_1\circ\varphi=\alpha_1\circ\pi_1\circ\varpi_F$ equals $\varphi_F$ as computed above. On the other side, we see that
		\[\pi_i\circ\varpi\circ\varphi=\varpi_i\circ\varphi=\pi_i\circ\varphi_F,\]
		so both $\varpi\circ\varphi$ and $\varphi_F$ could fill the dashed arrow in the diagram
		% https://q.uiver.app/?q=WzAsNSxbMSwxLCJYXzFcXHRpbWVzX1pYXzIiXSxbMiwxLCJYXzEiXSxbMSwyLCJYXzIiXSxbMiwyLCJaIl0sWzAsMCwiVyJdLFsxLDMsIlxcaW90YVxcYWxwaGFfMSJdLFsyLDMsIlxcaW90YVxcYWxwaGFfMiIsMl0sWzAsMSwiXFx2YXJwaV8xIiwyXSxbMCwyLCJcXHZhcnBpXzIiXSxbNCwxLCJcXHZhcnBpXzFcXHZhcnBoaV9GIiwwLHsiY3VydmUiOi0yfV0sWzQsMiwiXFx2YXJwaV8yXFx2YXJwaGlfRiIsMix7ImN1cnZlIjoyfV0sWzQsMCwiIiwxLHsic3R5bGUiOnsiYm9keSI6eyJuYW1lIjoiZGFzaGVkIn19fV1d&macro_url=https%3A%2F%2Fraw.githubusercontent.com%2FdFoiler%2Fnotes%2Fmaster%2Fnir.tex
		\[\begin{tikzcd}
			W \\
			& {X_1\times_ZX_2} & {X_1} \\
			& {X_2} & Z
			\arrow["{\iota\alpha_1}", from=2-3, to=3-3]
			\arrow["{\iota\alpha_2}"', from=3-2, to=3-3]
			\arrow["{\varpi_1}"', from=2-2, to=2-3]
			\arrow["{\varpi_2}", from=2-2, to=3-2]
			\arrow["{\varpi_1\varphi_F}", curve={height=-12pt}, from=1-1, to=2-3]
			\arrow["{\varpi_2\varphi_F}"', curve={height=12pt}, from=1-1, to=3-2]
			\arrow[dashed, from=1-1, to=2-2]
		\end{tikzcd}\]
		where we know there is space for at most morphism. Thus, $\varpi\circ\varphi=\varphi_F$ follows.
	\end{itemize}
	The above checks complete the proof.
\end{proof}
We now run a few checks on classes of diagonal morphisms.
\begin{notation}
	Given a class of morphisms $P$, we let $\Delta P$ denote the class of morphisms $\pi$ such that $\Delta\pi\in\Delta P$.
\end{notation}
\begin{lemma} \label{lem:diagonal-composition}
	Fix a class $P$ of morphisms which is preserved by composition and base change. Then $\Delta P$ is preserved by composition.
\end{lemma}
\begin{proof}
	Fix morphisms $\varphi\colon X\to Y$ and $\psi\colon Y\to Z$ in $\Delta P$ so that we want to show $\psi\circ\varphi$ is still in $\Delta P$. Namely, we are given that the diagonal morphisms $\Delta\varphi\colon X\to X\times_YX$ and $\Delta\psi\colon Y\to Y\times_ZY$ are in $P$.

	Now, \autoref{lem:magic-diagram} promises us the pullback square
	% https://q.uiver.app/?q=WzAsNCxbMCwwLCJYXFx0aW1lc19ZWCJdLFsxLDAsIlhcXHRpbWVzX1pZIl0sWzAsMSwiWSJdLFsxLDEsIllcXHRpbWVzX1pZIl0sWzAsMSwiXFxkZWx0YSJdLFsxLDNdLFsyLDMsIlxcRGVsdGFcXHBzaSJdLFswLDJdLFswLDMsIiIsMCx7InN0eWxlIjp7Im5hbWUiOiJjb3JuZXIifX1dXQ==&macro_url=https%3A%2F%2Fraw.githubusercontent.com%2FdFoiler%2Fnotes%2Fmaster%2Fnir.tex
	\[\begin{tikzcd}
		{X\times_YX} & {X\times_ZY} \\
		Y & {Y\times_ZY}
		\arrow["\delta", from=1-1, to=1-2]
		\arrow[from=1-2, to=2-2]
		\arrow["\Delta\psi", from=2-1, to=2-2]
		\arrow[from=1-1, to=2-1]
		\arrow["\lrcorner"{anchor=center, pos=0.125}, draw=none, from=1-1, to=2-2]
	\end{tikzcd}\]
	which because $P$ is preserved by base change tells us that the natural map $\delta\colon X\times_YX\to X\times_ZX$ is in $P$.

	To finish, we note that the diagonal morphism $\Delta(\psi\circ\varphi)\colon X\to X\times_ZX$ is the one induced by $\id_X\colon X\to X$, but then the commutative diagram
	% https://q.uiver.app/?q=WzAsOCxbMCwwLCJYIl0sWzEsMSwiWFxcdGltZXNfWVgiXSxbMiwyLCJYXFx0aW1lc19aWCJdLFszLDIsIlgiXSxbMiwzLCJYIl0sWzMsMywiWiJdLFsyLDEsIlgiXSxbMSwyLCJYIl0sWzAsMSwiXFxEZWx0YVxcdmFycGhpIiwxXSxbMCw3LCIiLDEseyJjdXJ2ZSI6MiwibGV2ZWwiOjIsInN0eWxlIjp7ImhlYWQiOnsibmFtZSI6Im5vbmUifX19XSxbMCw2LCIiLDEseyJjdXJ2ZSI6LTIsImxldmVsIjoyLCJzdHlsZSI6eyJoZWFkIjp7Im5hbWUiOiJub25lIn19fV0sWzEsN10sWzEsNl0sWzIsNF0sWzIsM10sWzEsMiwiXFxkZWx0YSIsMV0sWzYsMywiIiwxLHsibGV2ZWwiOjIsInN0eWxlIjp7ImhlYWQiOnsibmFtZSI6Im5vbmUifX19XSxbNyw0LCIiLDEseyJsZXZlbCI6Miwic3R5bGUiOnsiaGVhZCI6eyJuYW1lIjoibm9uZSJ9fX1dLFszLDUsIlxccHNpXFx2YXJwaGkiLDJdLFs0LDUsIlxccHNpXFx2YXJwaGkiXV0=&macro_url=https%3A%2F%2Fraw.githubusercontent.com%2FdFoiler%2Fnotes%2Fmaster%2Fnir.tex
	\[\begin{tikzcd}
		X \\
		& {X\times_YX} & X \\
		& X & {X\times_ZX} & X \\
		&& X & Z
		\arrow["\Delta\varphi"{description}, from=1-1, to=2-2]
		\arrow[curve={height=12pt}, Rightarrow, no head, from=1-1, to=3-2]
		\arrow[curve={height=-12pt}, Rightarrow, no head, from=1-1, to=2-3]
		\arrow[from=2-2, to=3-2]
		\arrow[from=2-2, to=2-3]
		\arrow[from=3-3, to=4-3]
		\arrow[from=3-3, to=3-4]
		\arrow["\delta"{description}, from=2-2, to=3-3]
		\arrow[Rightarrow, no head, from=2-3, to=3-4]
		\arrow[Rightarrow, no head, from=3-2, to=4-3]
		\arrow["\psi\varphi"', from=3-4, to=4-4]
		\arrow["\psi\varphi", from=4-3, to=4-4]
	\end{tikzcd}\]
	tells us that the natural composite map $X\stackrel{\Delta\varphi}\to X\times_YX\stackrel\delta\to X\times_ZX$ must also be $\Delta(\psi\circ\varphi)$ by the uniqueness of the definition of $\Delta(\psi\circ\varphi)$ from the bottom-right pullback square. So because $\Delta\varphi$ and $\delta$ are both in $P$, we conclude that their composite $\Delta(\psi\circ\varphi)$ is also in $P$. Thus, $\psi\circ\varphi$ is in $P$.
\end{proof}
\begin{lemma} \label{lem:diagonal-base-change}
	Fix a class $P$ of morphisms which is preserved by base change. Then $\Delta P$ is preserved by base change.
\end{lemma}
\begin{proof}
	Fix a pullback square
	% https://q.uiver.app/?q=WzAsNCxbMCwwLCJYJyJdLFsxLDAsIlknIl0sWzAsMSwiWCJdLFsxLDEsIlkiXSxbMiwzLCJcXHBpIl0sWzAsMSwiXFxwaSciXSxbMSwzLCJcXHZhcnBoaSJdLFswLDIsIlxcdmFycGhpJyJdXQ==&macro_url=https%3A%2F%2Fraw.githubusercontent.com%2FdFoiler%2Fnotes%2Fmaster%2Fnir.tex
	\[\begin{tikzcd}
		{X'} & {Y'} \\
		X & Y
		\arrow["\pi", from=2-1, to=2-2]
		\arrow["{\pi'}", from=1-1, to=1-2]
		\arrow["\varphi", from=1-2, to=2-2]
		\arrow["{\varphi'}", from=1-1, to=2-1]
	\end{tikzcd}\]
	such that $\pi\in\Delta P$. We want to show that $\pi'\in\Delta P$.

	Well, we let $\pi_1,\pi_2\colon X\times_YX\to X$ and $\pi_1'\colon X'\times_{Y'}X'\to X'$ be the canonical projections so that $\varphi'\pi_1'=\varphi'\pi_2'$, implying $\pi\varphi'\pi_1'=\pi\varphi'\pi_2$, which induces $(\varphi'\pi_1',\varphi'\pi_2')\colon X'\times_{Y'}X'\to X\times_YX$. The key claim, now, is that the square
	% https://q.uiver.app/?q=WzAsNCxbMCwwLCJYJyJdLFsxLDAsIlgnXFx0aW1lc197WSd9WCciXSxbMCwxLCJYIl0sWzEsMSwiWFxcdGltZXNfWVgiXSxbMiwzLCJcXERlbHRhXFxwaSJdLFswLDEsIlxcRGVsdGFcXHBpJyJdLFsxLDMsIihcXHBpXzEnXFx2YXJwaGknLFxccGlfMidcXHZhcnBoaScpIl0sWzAsMiwiXFx2YXJwaGknIl1d&macro_url=https%3A%2F%2Fraw.githubusercontent.com%2FdFoiler%2Fnotes%2Fmaster%2Fnir.tex
	\[\begin{tikzcd}
		{X'} & {X'\times_{Y'}X'} \\
		X & {X\times_YX}
		\arrow["\Delta\pi", from=2-1, to=2-2]
		\arrow["{\Delta\pi'}", from=1-1, to=1-2]
		\arrow["{(\varphi'\pi_1',\varphi'\pi_2')}", from=1-2, to=2-2]
		\arrow["{\varphi'}", from=1-1, to=2-1]
	\end{tikzcd}\]
	is a pullback square. This will finish because $P$ is preserved by base change: note $\Delta\pi\in P$ implies $\Delta\pi'\in P$, so $\pi'\in\Delta P$.
	
	Now, at the very least this square commutes because
	\[\pi_i\circ(\varphi'\pi_1',\varphi'\pi_2')\circ\Delta\pi'=\varphi'\circ\pi_i'\circ\Delta\pi'=\varphi'=\pi_i\circ\Delta\pi\circ\varphi'\]
	for each $i\in\{1,2\}$, so it follows that $(\varphi'\pi_1',\varphi'\pi_2')\circ\Delta\pi'=\Delta\pi\circ\varphi'$ by the definition of maps into the fiber product.

	Continuing, to show that our square is a pullback square, we use \autoref{lem:big-to-small-square}. Namely, we claim that the outer and right squares of
	% https://q.uiver.app/?q=WzAsNixbMCwwLCJYJyJdLFsxLDAsIlgnXFx0aW1lc197WSd9WCciXSxbMCwxLCJYIl0sWzEsMSwiWFxcdGltZXNfWVgiXSxbMiwwLCJZJyJdLFsyLDEsIlkiXSxbMiwzLCJcXERlbHRhXFxwaSJdLFswLDEsIlxcRGVsdGFcXHBpJyJdLFsxLDMsIihcXHBpXzEnXFx2YXJwaGknLFxccGlfMidcXHZhcnBoaScpIl0sWzAsMiwiXFx2YXJwaGknIl0sWzEsNCwiXFx2YXJwaGknXFxwaV8xJyJdLFszLDUsIlxcdmFycGhpXFxwaV8xIl0sWzQsNSwiXFx2YXJwaGkiXSxbMCw0LCJcXHBpJyIsMCx7Im9mZnNldCI6LTIsImN1cnZlIjotMn1dLFsyLDUsIlxccGkiLDIseyJjdXJ2ZSI6Mn1dXQ==&macro_url=https%3A%2F%2Fraw.githubusercontent.com%2FdFoiler%2Fnotes%2Fmaster%2Fnir.tex
	\begin{equation}
		\begin{tikzcd}
			{X'} & {X'\times_{Y'}X'} & {Y'} \\
			X & {X\times_YX} & Y
			\arrow["\Delta\pi", from=2-1, to=2-2]
			\arrow["{\Delta\pi'}", from=1-1, to=1-2]
			\arrow["{(\varphi'\pi_1',\varphi'\pi_2')}", from=1-2, to=2-2]
			\arrow["{\varphi'}", from=1-1, to=2-1]
			\arrow["{\pi'\pi_1'}", from=1-2, to=1-3]
			\arrow["{\pi\pi_1}", from=2-2, to=2-3]
			\arrow["\varphi", from=1-3, to=2-3]
			\arrow["{\pi'}", shift left=2, curve={height=-12pt}, from=1-1, to=1-3]
			\arrow["\pi"', curve={height=12pt}, from=2-1, to=2-3]
		\end{tikzcd} \label{eq:intermediate-pullback-diagonals}
	\end{equation}
	are pullback square. To begin, we note that at least the diagram commutes: namely, the right square has
	\[\pi\circ\pi_1\circ(\varphi'\pi_1',\varphi'\pi_2')=\pi\circ\varphi'\circ\pi_1'=\varphi\circ\pi'\circ\pi_1'.\]
	Further, we see that $\pi'\circ\pi_1'\circ\Delta\pi'=\pi'$ and $\pi\circ\pi_1\circ\Delta\pi=\pi$ by construction of the diagonal morphisms.

	Thus, we note that the outer rectangle of \autoref{eq:intermediate-pullback-diagonals} is in fact a pullback square by hypothesis, so it really only remains to show that the right square of \autoref{eq:intermediate-pullback-diagonals} is a pullback square. For this, we use \autoref{lem:smalls-to-big-pullback}, writing down
	% https://q.uiver.app/?q=WzAsNixbMCwwLCJYJ1xcdGltZXNfe1knfVgnIl0sWzAsMSwiWSciXSxbMSwwLCJYJ1xcdGltZXNfe1l9WCciXSxbMSwxLCJZJ1xcdGltZXNfWVknIl0sWzIsMCwiWFxcdGltZXNfWVgiXSxbMiwxLCJZIl0sWzAsMSwiXFxwaSdcXHBpXzEnIl0sWzEsMywiXFxEZWx0YVxcdmFycGhpIl0sWzAsNCwiKFxcdmFycGhpJ1xccGlfMScsXFx2YXJwaGknXFxwaV8yJykiLDAseyJvZmZzZXQiOi0yLCJjdXJ2ZSI6LTJ9XSxbNCw1LCJcXHBpXFxwaV8xIl0sWzMsNSwiXFx2YXJwaV8xIl0sWzEsNSwiXFx2YXJwaGkiLDIseyJjdXJ2ZSI6Mn1dLFswLDIsIihcXHBpXzEnLFxccGlfMicpIl0sWzIsMywiKFxccGknXFxwaV8xLFxccGknXFxwaV8yKSJdLFsyLDQsIihcXHZhcnBoaScsXFx2YXJwaGknKSJdXQ==&macro_url=https%3A%2F%2Fraw.githubusercontent.com%2FdFoiler%2Fnotes%2Fmaster%2Fnir.tex
	\[\begin{tikzcd}
		{X'\times_{Y'}X'} & {X'\times_{Y}X'} & {X\times_YX} \\
		{Y'} & {Y'\times_YY'} & Y
		\arrow["{\pi'\pi_1'}", from=1-1, to=2-1]
		\arrow["\Delta\varphi", from=2-1, to=2-2]
		\arrow["{(\varphi'\pi_1',\varphi'\pi_2')}", shift left=2, curve={height=-12pt}, from=1-1, to=1-3]
		\arrow["{\pi\pi_1}", from=1-3, to=2-3]
		\arrow["{\varphi\varpi_1}", from=2-2, to=2-3]
		\arrow["\varphi"', curve={height=12pt}, from=2-1, to=2-3]
		\arrow["{(\pi_1',\pi_2')}", from=1-1, to=1-2]
		\arrow["{(\pi'\pi_1',\pi'\pi_2')}", from=1-2, to=2-2]
		\arrow["{(\varphi',\varphi')}", from=1-2, to=1-3]
	\end{tikzcd}\]
	to claim that both squares are pullback squares; here $\varpi_1,\varpi_2\colon Y\times_XY\to Y$ are the canonical projections. The left square commutes and is a pullback square by \autoref{lem:magic-diagram}. The right square commutes because
	\[\pi\circ\pi_1\circ(\varphi',\varphi')=\pi\circ\varphi'\circ\pi_1'=\varphi\circ\pi'\circ\pi_1'=\varphi\circ\varpi_1\circ(\pi'\pi_1',\pi'\pi_2').\]
	At a high level, the right square is a pullback square because
	\[X'\times_YX'\simeq(Y'\times_YX)\times_Y(Y'\times_YX)\simeq(Y'\times_YY')\times_Y(X\times_YX).\]
	We can also see this more directly, but I think I personally draw the line at explicitly proving associativity laws.
\end{proof}
\begin{lemma}
	Fix a class $P$ of morphisms which is local on the target and preserved by base change. Then $\Delta P$ is local on the target.
\end{lemma}
\begin{proof}
	Fix a morphism $\varphi\colon X\to Y$. In one direction, suppose $\varphi\in\Delta P$ and fix some open subset $V\subseteq Y$. Then we note that \autoref{eq:open-embed-fp} tells us
	% https://q.uiver.app/?q=WzAsNCxbMCwxLCJYIl0sWzEsMSwiWSJdLFswLDAsIlxccGleey0xfVYiXSxbMSwwLCJWIl0sWzIsMCwiXFxwaXxfe1xccGleey0xfVZ9IiwyXSxbMywxLCJcXHBpIl0sWzIsMywiIiwwLHsic3R5bGUiOnsidGFpbCI6eyJuYW1lIjoiaG9vayIsInNpZGUiOiJ0b3AifX19XSxbMCwxLCIiLDIseyJzdHlsZSI6eyJ0YWlsIjp7Im5hbWUiOiJob29rIiwic2lkZSI6InRvcCJ9fX1dXQ==&macro_url=https%3A%2F%2Fraw.githubusercontent.com%2FdFoiler%2Fnotes%2Fmaster%2Fnir.tex
	\[\begin{tikzcd}
		{\varphi^{-1}V} & X \\
		X & Y
		\arrow["{\varphi|_{\varphi^{-1}V}}"', from=1-1, to=2-1]
		\arrow["\varphi", from=1-2, to=2-2]
		\arrow[hook, from=1-1, to=1-2]
		\arrow[hook, from=2-1, to=2-2]
	\end{tikzcd}\]
	is a pullback square, so $\varphi|_{\varphi^{-1}V}\in\Delta P$ by \autoref{lem:diagonal-base-change}.

	In the other direction, fix an open cover $\{Y_\alpha\}_{\alpha\in\lambda}$ of $Y$ and suppose that the restrictions $\varphi_\alpha\colon\pi^{-1}Y_\alpha\to Y_\alpha$ all live in $\Delta P$; set $X_\alpha\coloneqq\varphi^{-1}Y_\alpha$ for brevity. Then the pullback squares
	% https://q.uiver.app/?q=WzAsNCxbMCwxLCJZX1xcYWxwaGEiXSxbMSwxLCJZIl0sWzAsMCwiXFx2YXJwaGleey0xfVlfXFxhbHBoYSJdLFsxLDAsIlgiXSxbMiwwLCJcXHZhcnBoaV9cXGFscGhhIiwyXSxbMywxLCJcXHZhcnBoaSJdLFsyLDMsIlxcam1hdGhfXFxhbHBoYSIsMCx7InN0eWxlIjp7InRhaWwiOnsibmFtZSI6Imhvb2siLCJzaWRlIjoidG9wIn19fV0sWzAsMSwiXFxpb3RhX1xcYWxwaGEiLDAseyJzdHlsZSI6eyJ0YWlsIjp7Im5hbWUiOiJob29rIiwic2lkZSI6InRvcCJ9fX1dXQ==&macro_url=https%3A%2F%2Fraw.githubusercontent.com%2FdFoiler%2Fnotes%2Fmaster%2Fnir.tex
	\[\begin{tikzcd}
		{X_\alpha} & X \\
		{Y_\alpha} & Y
		\arrow["{\varphi_\alpha}"', from=1-1, to=2-1]
		\arrow["\varphi", from=1-2, to=2-2]
		\arrow["{\jmath_\alpha}", hook, from=1-1, to=1-2]
		\arrow["{\iota_\alpha}", hook, from=2-1, to=2-2]
	\end{tikzcd}\]
	coming from \autoref{eq:open-embed-fp} grant us the pullback squares
	% https://q.uiver.app/?q=WzAsNCxbMCwxLCJYX1xcYWxwaGFcXHRpbWVzX3tZX1xcYWxwaGF9WF9cXGFscGhhIl0sWzEsMSwiWVxcdGltZXNfWFkiXSxbMCwwLCJYX1xcYWxwaGEiXSxbMSwwLCJYIl0sWzIsMCwiXFxEZWx0YVxcdmFycGhpX1xcYWxwaGEiLDJdLFszLDEsIlxcRGVsdGFcXHZhcnBoaSJdLFsyLDMsIlxcam1hdGhfXFxhbHBoYSIsMCx7InN0eWxlIjp7InRhaWwiOnsibmFtZSI6Imhvb2siLCJzaWRlIjoidG9wIn19fV0sWzAsMV1d&macro_url=https%3A%2F%2Fraw.githubusercontent.com%2FdFoiler%2Fnotes%2Fmaster%2Fnir.tex
	\[\begin{tikzcd}
		{X_\alpha} & X \\
		{X_\alpha\times_{Y_\alpha}X_\alpha} & {Y\times_XY}
		\arrow["{\Delta\varphi_\alpha}"', from=1-1, to=2-1]
		\arrow["\Delta\varphi", from=1-2, to=2-2]
		\arrow["{\jmath_\alpha}", hook, from=1-1, to=1-2]
		\arrow[from=2-1, to=2-2]
	\end{tikzcd}\]
	which essentially finish the proof. However, $X_\alpha\times_{Y_\alpha}X_\alpha$ is really $\pi_1^{-1}X_\alpha\subseteq X\times_YX$ by \autoref{lem:fp-open-cover-base}, where $\pi_1,\pi_2\colon X\times_YX\to X$ are the canonical projections. Notably, we see that the open cover $Y_\alpha$ of $Y$ becomes an open cover $\pi_1^{-1}\varphi^{-1}Y_\alpha$ of $X\times_YX$, so the fact that all the individual restrictions\footnote{Namely, the $\Delta\varphi_\alpha$ is a restriction because the pullback square is ``actually'' being induced by \autoref{eq:open-embed-fp}.} $\Delta\varphi_\alpha\colon X_\alpha\to\pi_1^{-1}X_\alpha$ are in $P$ tells us that $\Delta\varphi$ is also in $P$ because $P$ is local on the target, so $\varphi\in\Delta P$.
\end{proof}

\subsection{Quasiseparatedness is Reasonable}
It turns out to be very convenient to think about a morphism $\varphi\colon X\to Y$ being quasiseparated in terms of the diagonal morphism $\Delta\colon X\to X\times_YX$ induced by $\id_X\colon X\to X$.
\begin{lemma} \label{lem:mainqslemma}
	Fix a morphism $\varphi\colon X\to Y$ of schemes.
	\begin{listalph}
		\item $\pi$ is quasiseparated.
		% \item All quasicompact open subsets $U\subseteq Y$ have $\pi^{-1}(U)\subseteq X$ quasiseparated.
		\item There is an affine open cover $\mc U$ of $Y$ such that each $\varphi^{-1}(U)$ is quasiseparated for each $U\in\mc U$.
		\item The diagonal morphism $\Delta\colon X\to X\times_YX$ is quasicompact.
	\end{listalph}
\end{lemma}
\begin{proof}
	As usual, (a) implies (b) by choosing any affine open cover $\mc U$ of $Y$, which implies that $\varphi^{-1}(U)$ is quasiseparated for each affine open $U\in\mc U$ because $\varphi$ is quasiseparated.

	The other implications are harder. Before going further, we set our variables. Let $\pi_1,\pi_2\colon X\times_XY\to X$ be the canonical inclusions so that $\pi_i\circ\Delta=\id_X$ for $i\in\{1,2\}$ by definition of $\Delta$.
	\begin{itemize}
		\item We show (b) implies (c). Give $Y$ the affine open cover $\{Y_\alpha\}_{\alpha\in\lambda}$ such that $\pi^{-1}Y_\alpha$ is quasiseparated for each $\alpha$. Further, give each $\varphi^{-1}(Y_\alpha)$ an affine open cover $\{U_{\alpha,\beta}\}_{\beta\in\lambda_\alpha}$ and label our diagram as
		% https://q.uiver.app/?q=WzAsOSxbMiwyLCJZX1xcYWxwaGEiXSxbMSwyLCJcXHZhcnBoaV57LTF9WV9cXGFscGhhIl0sWzIsMSwiXFx2YXJwaGleey0xfVlfXFxhbHBoYSJdLFsxLDEsIihcXHZhcnBoaVxccGlfMSleey0xfVlfXFxhbHBoYSJdLFswLDIsIlVfe1xcYWxwaGEsXFxiZXRhfSJdLFsyLDAsIlVfe1xcYWxwaGEsXFxiZXRhJ30iXSxbMCwxLCJcXHBpXzJeey0xfVVfe1xcYWxwaGEsXFxiZXRhfSJdLFsxLDAsIlxccGlfMV57LTF9VV97XFxhbHBoYSxcXGJldGEnfSJdLFswLDAsIlxccGlfMV57LTF9VV97XFxhbHBoYSxcXGJldGEnfVxcY2FwXFxwaV8yXnstMX1VX3tcXGFscGhhLFxcYmV0YX0iXSxbMSwwLCJcXHZhcnBoaSJdLFsyLDAsIlxcdmFycGhpIl0sWzMsMiwiXFxwaV8xIl0sWzMsMSwiXFxwaV8yIl0sWzYsMywiXFx3aWRldGlsZGVcXGptYXRoX3tcXGFscGhhLFxcYmV0YX0iLDAseyJzdHlsZSI6eyJ0YWlsIjp7Im5hbWUiOiJob29rIiwic2lkZSI6InRvcCJ9fX1dLFs0LDEsIlxcam1hdGhfe1xcYWxwaGEsXFxiZXRhfSIsMCx7InN0eWxlIjp7InRhaWwiOnsibmFtZSI6Imhvb2siLCJzaWRlIjoidG9wIn19fV0sWzYsNCwiXFxwaV8yIl0sWzcsMywiXFx3aWRldGlsZGVcXGptYXRoX3tcXGFscGhhLFxcYmV0YSd9IiwwLHsic3R5bGUiOnsidGFpbCI6eyJuYW1lIjoiaG9vayIsInNpZGUiOiJ0b3AifX19XSxbNSwyLCJcXGptYXRoX3tcXGFscGhhLFxcYmV0YSd9IiwwLHsic3R5bGUiOnsidGFpbCI6eyJuYW1lIjoiaG9vayIsInNpZGUiOiJ0b3AifX19XSxbNyw1LCJcXHBpXzEiXSxbOCw2LCJcXHdpZGV0aWxkZVxcam1hdGhfe1xcYWxwaGEsXFxiZXRhJ30iLDAseyJzdHlsZSI6eyJ0YWlsIjp7Im5hbWUiOiJob29rIiwic2lkZSI6InRvcCJ9fX1dLFs4LDcsIlxcd2lkZXRpbGRlXFxqbWF0aF97XFxhbHBoYSxcXGJldGF9IiwwLHsic3R5bGUiOnsidGFpbCI6eyJuYW1lIjoiaG9vayIsInNpZGUiOiJ0b3AifX19XV0=&macro_url=https%3A%2F%2Fraw.githubusercontent.com%2FdFoiler%2Fnotes%2Fmaster%2Fnir.tex
		\begin{equation}
			\begin{tikzcd}
				{\pi_1^{-1}U_{\alpha,\beta'}\cap\pi_2^{-1}U_{\alpha,\beta}} & {\pi_1^{-1}U_{\alpha,\beta'}} & {U_{\alpha,\beta'}} \\
				{\pi_2^{-1}U_{\alpha,\beta}} & {(\varphi\pi_1)^{-1}Y_\alpha} & {\varphi^{-1}Y_\alpha} \\
				{U_{\alpha,\beta}} & {\varphi^{-1}Y_\alpha} & {Y_\alpha}
				\arrow["\varphi", from=3-2, to=3-3]
				\arrow["\varphi", from=2-3, to=3-3]
				\arrow["{\pi_1}", from=2-2, to=2-3]
				\arrow["{\pi_2}", from=2-2, to=3-2]
				\arrow["{\widetilde\jmath_{\alpha,\beta}}", hook, from=2-1, to=2-2]
				\arrow["{\jmath_{\alpha,\beta}}", hook, from=3-1, to=3-2]
				\arrow["{\pi_2}", from=2-1, to=3-1]
				\arrow["{\widetilde\jmath_{\alpha,\beta'}}", hook, from=1-2, to=2-2]
				\arrow["{\jmath_{\alpha,\beta'}}", hook, from=1-3, to=2-3]
				\arrow["{\pi_1}", from=1-2, to=1-3]
				\arrow["{\widetilde\jmath_{\alpha,\beta'}}", hook, from=1-1, to=2-1]
				\arrow["{\widetilde\jmath_{\alpha,\beta}}", hook, from=1-1, to=1-2]
			\end{tikzcd} \label{eq:big-qs-square}
		\end{equation}
		where the key point is that the top-left corner indeed should be $\pi_1^{-1}U_{\alpha,\beta'}\cap\pi_2^{-1}U_{\alpha,\beta}$ as computed in \autoref{cor:intersect-open-subscheme}.

		Now, the bottom-right square of \autoref{eq:big-qs-square} is a pullback square by \autoref{lem:fp-open-cover-base}, and the remaining square are pullback squares by \autoref{lem:open-fp}, so repeated applications of \autoref{lem:smalls-to-big-pullback} tell us that the outer square is a pullback square. In particular, we conclude that $\pi_1^{-1}U_{\alpha,\beta'}\cap\pi_2^{-1}U_{\alpha,\beta}$ is affine using the big pullback square by \autoref{lem:affine-fp}.

		We now remember that we are trying to show that $\Delta$ is quasicompact. Well, being quasicompact is affine-local on the target by \autoref{lem:mainqclemma}, so it suffices to show that $\Delta|_{\Delta^{-1}V}$ is quasicompact as $V$ varies over the various $\pi_1^{-1}U_{\alpha,\beta'}\cap\pi_2^{-1}U_{\alpha,\beta}$. In fact, by \autoref{cor:qc-from-qc-morphisms}, it suffices to show that $\Delta^{-1}(V)$ itself is compact for our various $V$, for which we compute
		\begin{align*}
			\Delta^{-1}(V) &= \left\{x\in X:\Delta(x)\in\pi_1^{-1}U_{\alpha,\beta'}\cap\pi_2^{-1}U_{\alpha,\beta}\right\} \\
			&= \left\{x\in X:\pi_1\Delta(x)\in U_{\alpha,\beta'}\text{ and }\pi_2\Delta(x)\in U_{\alpha,\beta}\right\} \\
			&= U_{\alpha,\beta'}\cap U_{\alpha,\beta},
		\end{align*}
		which is compact because the $U_{\alpha,\beta}$ and $U_{\alpha,\beta'}$ are compact open subsets of the quasiseparated space $\varphi^{-1}Y_\alpha$: indeed, $\varphi^{-1}Y_\alpha$ is quasiseparated because $\varphi$ is quasiseparated, and $Y_\alpha$ is affine. This finishes.

		\item We show (c) implies (a); we use \autoref{rem:affine-qs-condition}. Fix an affine open subset $V\subseteq Y$ and two affine open subsets $U_1,U_2\subseteq\varphi^{-1}V$; we need to show that $U_1\cap U_2$ is quasicompact. Well, by \autoref{lem:magic-diagram}, we are promised a pullback square
		% https://q.uiver.app/?q=WzAsNCxbMCwwLCJWXzFcXHRpbWVzX1hWXzIiXSxbMSwwLCJWXzFcXHRpbWVzX1lYXzIiXSxbMCwxLCJYIl0sWzEsMSwiWFxcdGltZXNfWVgiXSxbMCwxXSxbMSwzXSxbMCwyXSxbMiwzLCJcXERlbHRhIl0sWzAsMywiIiwwLHsic3R5bGUiOnsibmFtZSI6ImNvcm5lciJ9fV1d&macro_url=https%3A%2F%2Fraw.githubusercontent.com%2FdFoiler%2Fnotes%2Fmaster%2Fnir.tex
		\[\begin{tikzcd}
			{V_1\times_XV_2} & {V_1\times_YV_2} \\
			X & {X\times_YX}
			\arrow[from=1-1, to=1-2]
			\arrow[from=1-2, to=2-2]
			\arrow[from=1-1, to=2-1]
			\arrow["\Delta", from=2-1, to=2-2]
			\arrow["\lrcorner"{anchor=center, pos=0.125}, draw=none, from=1-1, to=2-2]
		\end{tikzcd}\]
		where the bottom morphism is $\Delta$. To compute $V_1\times_XV_2$, we note that the maps $V_1\into X$ and $V_2\into X$ are open embeddings, so $V_1\times_XV_2\simeq V_1\cap V_2$ by \autoref{cor:intersect-open-subscheme}. With this in mind, we see that the top morphism above is quasicompact because quasicompactness is preserved by base change by \autoref{lem:qc-base-change}, so to show $V_1\cap V_2$ is quasicompact, it suffices to show that $V_1\times_YV_2$ is quasicompact.

		Well, noting that $V_1,V_2\subseteq\varphi^{-1}U$, we note that
		% https://q.uiver.app/?q=WzAsNCxbMSwxLCJVIl0sWzEsMCwiVl8xIl0sWzAsMSwiVl8yIl0sWzAsMCwiVl8xXFx0aW1lc19ZVl8yIl0sWzEsMCwiXFx2YXJwaGkiXSxbMiwwLCJcXHZhcnBoaSJdLFszLDFdLFszLDJdXQ==&macro_url=https%3A%2F%2Fraw.githubusercontent.com%2FdFoiler%2Fnotes%2Fmaster%2Fnir.tex
		\[\begin{tikzcd}
			{V_1\times_YV_2} & {V_1} \\
			{V_2} & U
			\arrow["\varphi", from=1-2, to=2-2]
			\arrow["\varphi", from=2-1, to=2-2]
			\arrow[from=1-1, to=1-2]
			\arrow[from=1-1, to=2-1]
		\end{tikzcd}\]
		is a pullback square by \autoref{lem:fp-open-cover-base}. Thus, $V_1\times_YV_2$ is affine by \autoref{lem:affine-fp} because now all of $V_1,V_2,U$ are affine, so $V_1\times_YV_2$ is quasicompact.
	\end{itemize}
	The above implications finish the proof.
\end{proof}
Here are some quick applications to being affine-local on the target.
\begin{corollary} \label{cor:qs-by-morphism}
	Fix an affine scheme $Y$ and a scheme morphism $\pi\colon X\to Y$. Then $X$ is quasiseparated if and only if $\pi$ is quasiseparated.
\end{corollary}
\begin{proof}
	If $\pi$ is quasiseparated, then we note that the affine open subscheme $Y\subseteq Y$ forces $X=\pi^{-1}(Y)$ to be quasiseparated. Conversely, if $X$ is quasiseparated, then we note that the affine open cover $\{Y\}$ of $Y$ has $\pi^{-1}(Y)=X$ quasiseparated, so $\pi$ is quasiseparated by \autoref{lem:mainqslemma}.
\end{proof}
\begin{example}
	Using the fact that $\Spec\ZZ$ is final in the category of schemes (by \autoref{cor:spec-z-final}), we can use \autoref{cor:qs-by-morphism} to say that a scheme $X$ is quasiseparated if and only if the canonical morphism $X\to\Spec\ZZ$ is quasiseparated.
\end{example}
% \begin{lemma} \label{lem:qs-by-diag}
% 	Fix a morphism $(\pi,\pi^\sharp)\colon(X,\OO_X)\to(Y,\OO_Y)$ of schemes. Then $(\pi,\pi^\sharp)$ is quasiseparated if and only if the diagonal map $\Delta\colon X\to X\times_YX$ (induced by using $\id_X\colon X\to X$) is quasicompact.
% \end{lemma}
% \begin{proof}
% 	To show $\Delta$ is quasicompact, one can just reduce to an affine open cover.
% 	In the other direction, suppose that the diagonal map $\Delta$ is quasicompact, and we show that $\pi$ is quasiseparated. Well, fix some affine open subset $U\subseteq Y$ with affine open subsets $V_1,V_2\subseteq\pi^{-1}(U)$, and we want to show that $V_1\cap V_2$ is quasicompact. For this, we draw the diagram
% 	% https://q.uiver.app/?q=WzAsNCxbMCwwLCJWXzFcXHRpbWVzX1hWXzIiXSxbMSwwLCJWXzFcXHRpbWVzX1lWXzIiXSxbMSwxLCJYXFx0aW1lc19ZWCJdLFswLDEsIlgiXSxbMywyXSxbMSwyXSxbMCwxXSxbMCwzXV0=&macro_url=https%3A%2F%2Fraw.githubusercontent.com%2FdFoiler%2Fnotes%2Fmaster%2Fnir.tex
% 	\[\begin{tikzcd}
% 		{V_1\times_XV_2} & {V_1\times_YV_2} \\
% 		X & {X\times_YX}
% 		\arrow[from=2-1, to=2-2]
% 		\arrow[from=1-2, to=2-2]
% 		\arrow[from=1-1, to=1-2]
% 		\arrow[from=1-1, to=2-1]
% 	\end{tikzcd}\]
% 	which commutes purely formally by properties of the fiber product. Notably, $V_1,V_2\subseteq\pi^{-1}(U)$ promises that $V_1\times_YV_2=V_1\times_UV_2$. However, the main point is that $V_1\cap V_2=V_1\times_XV_2$ and $\pi^{-1}(V_1\times_UV_2)=V_1\cap V_2$ again. Thus, we note that $V_1\times_YV_2$ being quasicompact tells us that $V_1\cap V_2$ is quasicompact by taking the pre-image using the compactness of $\Delta$.
% \end{proof}
\autoref{lem:mainqslemma} lets us turn questions about morphisms being quasiseparated into questions about them being quasicompact. This will more or less automatically prove that being quasiseparated is preserved under composition and base change. Let's see this.
\begin{lemma} \label{lem:qs-base-change}
	The class of quasiseparated morphisms is preserved by composition.
\end{lemma}
\begin{proof}
	Note that the class of quasicompact morphisms is preserved by composition by \autoref{cor:qc-is-comp-preserve} and preserved by base change by \autoref{lem:qc-base-change}. The result now follows from \autoref{lem:diagonal-composition} by viewing quasiseparated morphisms as the class of morphisms with quasicompact diagonal, by \autoref{lem:mainqslemma}.
\end{proof}
\begin{lemma}
	The class of quasiseparated morphisms is preserved by base change.
\end{lemma}
\begin{proof}
	Note that the class of quasicompact morphisms is preserved by base change by \autoref{lem:qc-base-change}. The result now follows from \autoref{lem:diagonal-base-change} by viewing quasiseparated morphisms as the class of morphisms with quasicompact diagonal, by \autoref{lem:mainqslemma}.
\end{proof}
Quasiseparated morphisms also turn out to satisfy a cancellation property.
\begin{lemma}
	Fix scheme morphisms $\varphi\colon X\to Y$ and $\psi\colon Y\to Z$. If the composite $\psi\circ\varphi$ is quasiseparated, then $\varphi$ is also quasiseparated.
\end{lemma}
\begin{proof}
	We use the fact that being quasiseparated is affine-local on the target, as showed in \autoref{lem:mainqslemma}. Give $Z$ some affine open cover $\{Z_\alpha\}_{\alpha\in\lambda}$, and then give each $\psi^{-1}Z_\alpha$ an affine open cover $\{Y_{\alpha,\beta}\}_{\beta\in\lambda_\alpha}$. It suffices to show that the restriction $\varphi|_{\varphi^{-1}Y_{\alpha,\beta}}\colon\varphi^{-1}Y_{\alpha,\beta}\to Y_{\alpha,\beta}$ is quasiseparated because being quasiseparated is affine-local on the target.

	Thus, fix some $\alpha\in\lambda$ and $\beta\in\lambda_\alpha$. Because $Y_{\alpha,\beta}$ is affine, we want to know that $\varphi^{-1}Y_{\alpha,\beta}$ is quasiseparated by \autoref{cor:qs-by-morphism}. However, $(\psi\circ\varphi)^{-1}Z_\alpha$ is quasiseparated because $\psi\circ\varphi$ is quasiseparated. In particular, the open subset $\varphi^{-1}Y_{\alpha,\beta}\subseteq(\psi\circ\varphi)^{-1}Z_\alpha$ is also quasiseparated by \autoref{ex:qs-open-subset}.
\end{proof}

\subsection{Affine Morphisms}
Here is our definition.
\begin{definition}[Affine]
	A scheme morphism $\pi\colon X\to Y$ if and only if every affine open subset $U\subseteq Y$ has $\pi^{-1}(U)$ affine.
\end{definition}
\begin{example}
	Let $\varphi\colon\Spec B\to\Spec A$ be a morphism of affine schemes; we show $\varphi$ is affine. Well, for any affine open subscheme $\Spec A'\cong U\subseteq\Spec A$, \autoref{lem:open-fp} tells us that
	% https://q.uiver.app/?q=WzAsNCxbMSwxLCJcXFNwZWMgQSJdLFsxLDAsIlxcU3BlYyBCIl0sWzAsMSwiVSJdLFswLDAsIlxcdmFycGhpXnstMX0oVSkiXSxbMSwwLCJcXHZhcnBoaSJdLFsyLDAsIiIsMix7InN0eWxlIjp7InRhaWwiOnsibmFtZSI6Imhvb2siLCJzaWRlIjoidG9wIn19fV0sWzMsMSwiIiwwLHsic3R5bGUiOnsidGFpbCI6eyJuYW1lIjoiaG9vayIsInNpZGUiOiJ0b3AifX19XSxbMywyLCJcXHZhcnBoaXxfe1xcdmFycGhpXnstMX0oVSl9Il1d&macro_url=https%3A%2F%2Fraw.githubusercontent.com%2FdFoiler%2Fnotes%2Fmaster%2Fnir.tex
	\[\begin{tikzcd}
		{\varphi^{-1}(U)} & {\Spec B} \\
		U & {\Spec A}
		\arrow["\varphi", from=1-2, to=2-2]
		\arrow[hook, from=2-1, to=2-2]
		\arrow[hook, from=1-1, to=1-2]
		\arrow["{\varphi|_{\varphi^{-1}(U)}}", from=1-1, to=2-1]
	\end{tikzcd}\]
	is a pullback square, so $\varphi^{-1}(U)\cong\Spec B\times_{\Spec A}U\cong\Spec B\times_{\Spec A}\Spec A'$, which is just $\Spec B\otimes_AA'$ by \autoref{lem:affine-fp}. In particular, $\varphi^{-1}(U)$ is affine.
\end{example}
Here are the usual sanity checks.
\begin{lemma}
	Affine morphisms are preserved by composition.
\end{lemma}
\begin{proof}
	Let $\varphi\colon X\to Y$ and $\psi\colon Y\to Z$ be affine morphisms, and we want to show $\psi\circ\varphi$ is affine. Well, if $W\subseteq Z$ is an affine open subset, then $\psi^{-1}W\subseteq Y$ is affine, so $(\psi\circ\varphi)^{-1}(W)=\varphi^{-1}\psi^{-1}W\subseteq X$ is also affine. This finishes.
\end{proof}
\begin{remark}
	We can easily see that being affine implies being quasicompact and quasiseparated.
\end{remark}
If we want the analogue of \autoref{lem:mainqclemma} for affine morphisms, we must do some extra work.
\begin{lemma}
	A morphism $\pi\colon X\to Y$ is affine if and only if we can provide $Y$ with an affine open cover $\mc U$ such that $\pi^{-1}(U)$ is affine for each $U\in\mc U$.
\end{lemma}
\begin{proof}
	To begin, note that a morphism $\pi\colon\Spec B\to\Spec A$ of affine schemes will pull back distinguished open sets to distinguished open sets.

	Now, call our affine open cover $\{U_\alpha\}_{\alpha\in\lambda}$ of $Y$ satisfying the conclusion. Then for some affine open subset $U=\Spec A$ of $X$, we can intersect $U$ with each $U_\alpha$ to give an affine open cover of $U$ by various distinguished open subsets. Then the pre-image of each of these under $\pi$ is affine by our previous remark, and we can glue these affine schemes to another affine scheme by Exercise~2.17(b) in \cite{hartshorne}, which was on the homework. Notably, the fact we have an affine open cover is what tells us that the chosen elements from our distinguished open subsets will generate the unit ideal of $A$.
\end{proof}
\begin{remark}
	As usual, affine morphisms are preserved by composition, base-change, and it is local on the target.
\end{remark}
Next time we move on to talk about finite morphisms, integral morphisms, and morphisms of finite type.

\end{document}