% !TEX root = ../notes.tex

\documentclass[../notes.tex]{subfiles}

\begin{document}

\section{September 23}

Today we place some finiteness conditions on morphisms.
% V 8.1--8.3
% 8.3 has definitions

\subsection{Quasicompact and Quasiseparated}
For today, we will denote a morphism of schemes $\pi\colon X\to Y$.
\begin{definition}[Quasicompact]
	A scheme morphism $\pi\colon X\to Y$ is \textit{quasicompact} if and only if all affine open subschemes $U\subseteq Y$ make $\pi^{-1}(U)\subseteq X$ a quasicompact topological space.
\end{definition}
\begin{example}
	Fix a morphism $\pi\colon X\to Y$, where $X$ is a Noetherian scheme and hence a Noetherian space. Then $\pi$ is quasicompact because any (affine) open subset $U\subseteq Y$ makes $\pi^{-1}U\subseteq X$ an open and hence quasicompact set, where we are using \autoref{lem:opensarecompact}.
\end{example}
\begin{example} \label{ex:closed-is-qc}
	Any closed embedding $\pi\colon X\to Y$ is quasicompact. Indeed, given an affine open subset $U\subseteq Y$, we note that $\pi^{-1}(U)$ is homeomorphic through $\pi$ to $\pi(X)\cap U$. However, $U$ is quasicompact (it's affine), so the closed subset $\pi(X)\cap U$ is a closed subset in a quasicompact space and therefore quasicompact.
\end{example}
\begin{example} \label{ex:iso-is-qc}
	Of course, isomorphisms of schemes are homeomorphisms by \autoref{lem:betterlocaliso}, so isomorphisms are quasicompact.
\end{example}
\begin{nex}
	An open embedding need not be quasicompact. For example, an affine scheme can have open subschemes which are not quasicompact.
\end{nex}
To define quasiseparated, we will need to have the adjective on topological spaces.
\begin{definition}[Quasiseperated]
	A topological space $X$ is \textit{quasiseparated} if and only if the intersection of two quasicompact open subsets is still quasicompact.
\end{definition}
\begin{example}
	Any Noetherian space is quasiseparated because any open subset is quasicompact by \autoref{lem:opensarecompact}.
\end{example}
\begin{example} \label{ex:local-noetherian-qs}
	Locally Noetherian schemes $X$ are quasiseparated: given quasicompact open subsets $U,V\subseteq X$, give $U$ a finite affine open covers $\{U_i\}_{i=1}^n$. Then $U_i$ is an affine scheme of a Noetherian ring, so $U_i$ is a Noetherian space by \autoref{ex:noetherianringisnoetherian}. Thus, $U_i\cap V$ is quasicompact for each $i$ by \autoref{lem:opensarecompact}, so the finite union $U\cap V=\bigcup_{i=1}^n(U_i\cap V)$ is still quasicompact.
\end{example}
\begin{example} \label{ex:qs-open-subset}
	If $X$ is a quasiseparated space, and $U\subseteq X$ is an open subset, then $U$ is still quasiseparated: any quasicompact open subsets $U_1,U_2\subseteq U$ are also quasicompact open subsets of $X$, which means $U_1\cap U_2$ is quasicompact because $X$ is quasiseparated.
\end{example}
\begin{example} \label{ex:affine-is-qs}
	Affine schemes $X\cong\Spec A$ are quasiseparated. Indeed, any quasicompact open subschemes $U,V\subseteq X$ can be given finite distinguished open covers $\{D(f_i)\}_{i=1}^m$ for $U$ and $\{D(g_j)\}_{j=1}^n$ for $V$, which then by \autoref{rem:intersect-distinguished-base} has the finite affine open cover given by
	\[U\cap V=\Bigg(\bigcup_{i=1}^mD(f_i)\Bigg)\cap\Bigg(\bigcup_{j=1}^nD(g_j)\Bigg)=\bigcup_{i=1}^m\bigcup_{i=1}^n\underbrace{D(f_i)\cap D(g_j)}_{D(f_ig_j)}.\]
\end{example}
And here is our definition.
\begin{defi}[Quasiseparated]
	A scheme morphism $\pi\colon X\to Y$ is \textit{quasiseparated} if and only if all affine open subschemes $U\subseteq Y$ makes $\pi^{-1}(U)$ a quasiseparated topological space.
\end{defi}
\begin{remark} \label{rem:affine-qs-condition}
	Equivalently, $\pi\colon X\to Y$ is quasiseparated if and only if given any affine open subset $U\subseteq Y$ and more affine open subsets $V_1,V_2\subseteq\pi^{-1}(U)$, we can give $V_1\cap V_2$ a finite affine open cover.
	\begin{itemize}
		\item If $\pi$ is quasiseparated, then $\pi^{-1}(U)$ is quasiseparated, so $V_1\cap V_2$ is quasicompact and thus has a finite affine open cover (from any affine open cover).
		\item If $\pi$ satisfies the condition, suppose we have quasicompact subsets $V_1,V_2\subseteq\pi^{-1}(U)$. Then we can give $V_1$ and $V_2$ finite affine open covers $\mc V_1$ and $\mc V_2$, and we see
		\[V_1\cap V_2=\Bigg(\bigcup_{W_1\in\mc V_1}W_1\Bigg)\cap\Bigg(\bigcup_{W_2\in\mc V_2}W_2\Bigg)=\bigcup_{W_1\in\mc V_1,W_2\in\mc V_2}(W_1\cap W_2),\]
		where $W_1\cap W_2$ is quasicompact and thus has a finite affine open cover. Synthesizing our finite cover by finite affine open covers, we see that we in total have given $V_1\cap V_2$ a finite affine open cover.
	\end{itemize}
\end{remark}
\begin{example}
	Fix a scheme morphism $\pi\colon X\to Y$. If $X$ is quasiseparated (e.g., $X$ is locally Noetherian, using \autoref{ex:local-noetherian-qs}), then $\pi$ is quasiseparated. Indeed, for any (affine) open subset $V\subseteq Y$, we see that $\pi^{-1}(V)\subseteq X$ is open and therefore quasiseparated by \autoref{ex:qs-open-subset}.
\end{example}
It turns out that a scheme $(X,\OO_X)$ is quasicompact/quasiseparated if and only if its morphism $(X,\OO_X)\to(\Spec\ZZ,\OO_{\Spec\ZZ})$ is quasicompact/quasiseparated; we will show this later.
\begin{remark}
	A scheme being quasiseparated is a very reasonable smallness condition, weaker than being locally Noetherian. We will later define what it means for a morphism/scheme to be ``separated,'' which will be stronger than this and approximately mean Hausdorff.
\end{remark}

\subsection{Quasicompactness is Reasonable}
Here are some equivalent definitions for being quasicompact.
\begin{lemma} \label{lem:non-affine-qc}
	A morphism $\pi\colon X\to Y$ is quasicompact if and only if every quasicompact subset $U\subseteq Y$ has $\pi^{-1}U$ also quasicompact.
\end{lemma}
\begin{proof}
	If the conclusion is true, then $\pi$ is certainly quasicompact because affine open subsets are necessarily quasicompact.

	On the other hand, suppose $\pi$ is quasicompact, and pick up a quasicompact subset $U\subseteq Y$. Now, $U$ as an open subscheme can be given an affine open cover $\mc V$, but because $U$ is quasicompact, we may assume that $\mc V$ is finite. But then
	\[\pi^{-1}(U)=\bigcup_{V\in\mc V}\pi^{-1}(V)\]
	is the finite union of quasicompact sets, where the $\pi^{-1}(V)$ is quasicompact because the $V$ are affine. Thus, $\pi^{-1}(U)$ is quasicompact.
\end{proof}
\begin{lemma} \label{lem:mainqclemma}
	Fix a morphism $\pi\colon X\to Y$ of schemes. Then $\pi$ is quasicompact if and only if there is an affine open cover $\mc U$ of $Y$ such that each $\pi^{-1}(U)$ is quasicompact for each $U\in\mc U$.
\end{lemma}
\begin{proof}
	In one direction, if $\pi$ is quasicompact, then any affine open cover $\mc U$ has each $U\in\mc U$ affine, so we see $\pi^{-1}(U)\subseteq X$ is quasicompact by hypothesis on $\pi$.

	The other direction is harder. Fix an affine open cover $\{U_\alpha\}_{\alpha\in\lambda}$ of $Y$ with $\varphi_\alpha\colon\Spec A_\alpha\cong U_\alpha$, and we are given that $\pi^{-1}(U_\alpha)$ is quasicompact for each $\alpha$. Now, for any quasicompact open subset $U\subseteq Y$, we need to show that $\pi^{-1}(U)$ is quasicompact.

	Well, using the distinguished base of each $U_\alpha\cong\Spec A_\alpha$, we can write
	\[U\cap U_\alpha=\bigcup_{\beta\in\lambda_\alpha}\varphi(D(f_{\alpha,\beta}))\]
	for some elements $\alpha,\beta\in\lambda$ (\autoref{rem:distinguishedbase}). It follows that
	\[U=\bigcup_{\alpha\in\lambda}(U\cap U_\alpha)=\bigcup_{\alpha\in\lambda}\bigcup_{\beta\in\lambda_\alpha}\varphi(D(f_{\alpha,\beta})).\]
	This provides an open cover of $U$, so the quasicompactness of $U$ forces us to have a finite subcover; let $\lambda'$ denote the finite set of $(\alpha,\beta)$ such that $\varphi(D(f_{\alpha,\beta}))$ cover $U$.

	It follows that
	\[\pi^{-1}(U)=\bigcup_{(\alpha,\beta)\in\lambda'}\pi^{-1}(\varphi(D(f_{\alpha,\beta}))).\]
	However, each $\varphi(D(f_{\alpha,\beta}))$ is affine, so their preimages under $\pi$ are quasicompact, so $\pi^{-1}(U)$ is the finite union of quasicompact sets and hence quasicompact. This finishes.
\end{proof}
So here are some quick results.
\begin{cor} \label{cor:qc-from-qc-morphisms}
	Fix a morphism $\pi\colon X\to Y$ of schemes. If $Y$ is affine, then $\pi$ is quasicompact if and only if $X$ is quasicompact.
\end{cor}
\begin{proof}
	We apply \autoref{lem:mainqclemma}. If $\pi$ is quasicompact, then the affine open subset $Y\subseteq Y$ must have $X=\pi^{-1}Y$ quasicompact by definition. Conversely, if $X$ is quasicompact, then we use the affine open cover $\{Y\}$ on $Y$ to note that $\pi$ is quasicompact because $\pi^{-1}(Y)=X$ is, by \autoref{lem:mainqclemma}.
\end{proof}
\begin{example}
	We see from \autoref{cor:qc-from-qc-morphisms} that a scheme $X$ is quasicompact if and only if its unique morphism $X\to\Spec\ZZ$ is quasicompact. (Recall this morphism is unique by \autoref{cor:spec-z-final}.)
\end{example}
\begin{remark}
	In fact, if a class of morphisms $P$ is affine-local on the target, then it is actually local on the target. Indeed, fix a morphism $\pi\colon X\to Y$ and give $Y$ an open cover $\{Y_\alpha\}_{\alpha\in\lambda}$, and give each $Y_\alpha$ an affine open cover $\{U_{\alpha,\beta}\}_{\beta\in\lambda_\alpha}$.
	\begin{itemize}
		\item Suppose $\pi\in P$; we want to show $\pi|_{\pi^{-1}Y_\alpha'}$ is in $P$ for some fixed $\alpha'$. Well, the $\{U_{\alpha,\beta}\}$ form affine open cover of $Y$, so $\pi|_{\pi^{-1}U_{\alpha',\beta}}\in P$ for each $\beta$, so $\pi|_{\pi^{-1}Y_{\alpha'}}\in P$ because $P$ is affine-local on the target.
		\item Suppose $\pi|_{Y_\alpha}\in P$ for each $\alpha$. Then, using the affine open covers of $Y_\alpha$, we see that $\pi|_{\pi^{-1}U_{\alpha,\beta}}\in P$ for each $\alpha$ and $\beta$, so $\pi\in P$ follows.
	\end{itemize}
\end{remark}
In light of the above remark, we will make little distinction between being local on the target and affine-local on the target.

The above results are important enough that we will want to give it a name.
\begin{definition}[Affine-local on the target]
	Let $P$ be a class of morphisms. We say that $P$ is \textit{affine-local on the target} if and only if a morphism $\pi\colon X\to Y$ is in $P$ if and only if there is an affine open cover $\{Y_\alpha\}_{\alpha\in\lambda}$ such that all the restricted maps $\pi|_{\pi^{-1}Y_\alpha}\colon\pi^{-1}Y_\alpha\to Y_\alpha$ are also in $P$.
\end{definition}
\begin{example} \label{ex:qc-is-affine-local}
	Quasicompact morphisms are affine-local on the target, from \autoref{lem:mainqclemma}. Certainly if $\pi$ is quasicompact, then for any affine open subset $U\subseteq Y$, we see $\pi^{-1}U$ is quasicompact, so the restriction $\pi|_{\pi^{-1}U}\colon\pi^{-1}U\to U$ is quasicompact by \autoref{cor:qc-from-qc-morphisms}. Conversely, if all the restrictions to $\pi|_{\pi^{-1}Y_\alpha}$ are quasicompact, then because $Y_\alpha\subseteq Y$ is quasicompact, $\pi^{-1}(Y_\alpha)$ is quasicompact for each $\alpha$, so $\pi$ is quasicompact by \autoref{lem:mainqclemma}.
\end{example}
Here are a few niceness checks.
\begin{corollary} \label{cor:qc-is-comp-preserve}
	Fix quasicompact scheme morphisms $\varphi\colon X\to Y$ and $\psi\colon Y\to Z$. Then $\psi\circ\varphi$ is quasicompact.
\end{corollary}
\begin{proof}
	We use \autoref{lem:non-affine-qc}. Pick up any quasicompact subset $W\subseteq Z$. Then $\psi^{-1}(W)\subseteq Y$ is quasicompact by \autoref{lem:non-affine-qc}, so $(\psi\circ\varphi)^{-1}(W)=\varphi^{-1}(\psi^{-1}(W))$ is quasicompact again by \autoref{lem:non-affine-qc}.
\end{proof}
Once more, it will be useful to have language to describe the above.
\begin{definition}[Preserved by composition]
	Let $P$ be a class of morphisms. We say that $P$ is \textit{preserved by composition} if and only if, for any pair of morphisms $\varphi\colon X\to Y$ and $\psi\colon Y\to Z$ in $P$, we have $\psi\circ\varphi$ also in $P$.
\end{definition}
\begin{example}
	By \autoref{cor:qc-is-comp-preserve}, quasicompact morphisms are preserved by composition.
\end{example}
\begin{lemma} \label{lem:qc-base-change}
	Suppose we have a pullback square
	% https://q.uiver.app/?q=WzAsNCxbMCwwLCJYXFx0aW1lc19TWSJdLFsxLDAsIlgiXSxbMCwxLCJZIl0sWzEsMSwiUyJdLFsyLDMsIlxccHNpX1kiXSxbMCwxLCJcXHBpX1giXSxbMCwyLCJcXHBpX1kiXSxbMSwzLCJcXHBzaV9YIl0sWzAsMywiIiwyLHsic3R5bGUiOnsibmFtZSI6ImNvcm5lciJ9fV1d&macro_url=https%3A%2F%2Fraw.githubusercontent.com%2FdFoiler%2Fnotes%2Fmaster%2Fnir.tex
	\[\begin{tikzcd}
		{X\times_SY} & X \\
		Y & S
		\arrow["{\psi_Y}", from=2-1, to=2-2]
		\arrow["{\pi_X}", from=1-1, to=1-2]
		\arrow["{\pi_Y}", from=1-1, to=2-1]
		\arrow["{\psi_X}", from=1-2, to=2-2]
		\arrow["\lrcorner"{anchor=center, pos=0.125}, draw=none, from=1-1, to=2-2]
	\end{tikzcd}\]
	of schemes. If $\psi_Y$ is quasicompact, then $\pi_X$ is quasicompact.
\end{lemma}
\begin{proof}
	The main point is to reduce to the affine case, where everything is clear. Let $\pi_S\coloneqq\psi_Y\circ\pi_Y=\psi_X\circ\pi_X$, for brevity. Give $S$ an affine open cover $\{S_\alpha\}_{\alpha\in\lambda}$. For each $\alpha\in\lambda$, we give $\psi_X^{-1}(S_\alpha)\subseteq X$ an affine open cover $\{X_{\alpha,\beta}\}_{\alpha\in\lambda,\beta\in\kappa_\alpha}$. Then we build the tower
	% https://q.uiver.app/?q=WzAsNixbMCwxLCJcXHBpX1Neey0xfShTX1xcYWxwaGEpIl0sWzEsMSwiXFxwc2lfWF57LTF9KFNfXFxhbHBoYSkiXSxbMCwyLCJcXHBzaV9ZXnstMX0oU19cXGFscGhhKSJdLFsxLDIsIlNfXFxhbHBoYSJdLFsxLDAsIlhfe1xcYWxwaGEsXFxiZXRhfSJdLFswLDAsIlxccGlfWF57LTF9KFhfe1xcYWxwaGEsXFxiZXRhfSkiXSxbMiwzLCJcXHBzaV9ZfF97XFxwc2lfWV57LTF9KFNfXFxhbHBoYSl9IiwyXSxbMCwxLCJcXHBpX1h8X3tcXHBpX1Neey0xfShTX1xcYWxwaGEpfSJdLFswLDIsIlxccGlfWXxfe1xccGlfU157LTF9KFNfXFxhbHBoYSl9Il0sWzEsMywiXFxwc2lfWCJdLFs0LDEsIiIsMCx7InN0eWxlIjp7InRhaWwiOnsibmFtZSI6Imhvb2siLCJzaWRlIjoidG9wIn19fV0sWzUsNCwiXFxwaV9YfF97XFxwaV9YXnstMX0oWF97XFxhbHBoYSxcXGJldGF9KX0iXSxbNSwwLCIiLDAseyJzdHlsZSI6eyJ0YWlsIjp7Im5hbWUiOiJob29rIiwic2lkZSI6InRvcCJ9fX1dXQ==&macro_url=https%3A%2F%2Fraw.githubusercontent.com%2FdFoiler%2Fnotes%2Fmaster%2Fnir.tex
	\[\begin{tikzcd}
		{\pi_X^{-1}(X_{\alpha,\beta})} & {X_{\alpha,\beta}} \\
		{\pi_S^{-1}(S_\alpha)} & {\psi_X^{-1}(S_\alpha)} \\
		{\psi_Y^{-1}(S_\alpha)} & {S_\alpha}
		\arrow["{\psi_Y|_{\psi_Y^{-1}(S_\alpha)}}"', from=3-1, to=3-2]
		\arrow["{\pi_X|_{\pi_S^{-1}(S_\alpha)}}", from=2-1, to=2-2]
		\arrow["{\pi_Y|_{\pi_S^{-1}(S_\alpha)}}", from=2-1, to=3-1]
		\arrow["{\psi_X}", from=2-2, to=3-2]
		\arrow[hook, from=1-2, to=2-2]
		\arrow["{\pi_X|_{\pi_X^{-1}(X_{\alpha,\beta})}}", from=1-1, to=1-2]
		\arrow[hook, from=1-1, to=2-1]
	\end{tikzcd}\]
	and note that the bottom square is a pullback square by \autoref{lem:fp-open-cover-base}, the top square is a pullback square by \autoref{lem:open-fp}, so the total rectangle is a pullback square by \autoref{lem:smalls-to-big-pullback}. Now, because being quasicompact is affine-local on the target by \autoref{lem:mainqclemma}, it suffices to show that each restricted map $\pi_X|_{\pi_X^{-1}(X_{\alpha,\beta})}$ is quasicompact. Notably, by \autoref{lem:non-affine-qc}, the restriction $\psi_Y|_{\psi_Y^{-1}(S_\alpha)}$ is quasicompact.

	Thus, we fix some $\alpha$ and $\beta$. We now rename our variables, replacing $S_\alpha$ with $S$, $X_{\alpha,\beta}$ with $X$, and $\psi_Y^{-1}(S_\alpha)$ with $Y$, and we rename our morphisms to fit the pullback square
	% https://q.uiver.app/?q=WzAsNCxbMSwxLCJTIl0sWzEsMCwiWCJdLFswLDEsIlkiXSxbMCwwLCJYXFx0aW1lc19TWSJdLFszLDEsIlxccGlfWCJdLFsxLDAsIlxccHNpX1giXSxbMiwwLCJcXHBzaV9ZIl0sWzMsMiwiXFxwaV9ZIiwyXSxbMywwLCIiLDEseyJzdHlsZSI6eyJuYW1lIjoiY29ybmVyIn19XV0=&macro_url=https%3A%2F%2Fraw.githubusercontent.com%2FdFoiler%2Fnotes%2Fmaster%2Fnir.tex
	\[\begin{tikzcd}
		{X\times_SY} & X \\
		Y & S
		\arrow["{\pi_X}", from=1-1, to=1-2]
		\arrow["{\psi_X}", from=1-2, to=2-2]
		\arrow["{\psi_Y}", from=2-1, to=2-2]
		\arrow["{\pi_Y}"', from=1-1, to=2-1]
		\arrow["\lrcorner"{anchor=center, pos=0.125}, draw=none, from=1-1, to=2-2]
	\end{tikzcd}\]
	though we now have both $X$ and $S$ affine. We are given that $\psi_Y$ is quasicompact, and we would like to show that $\pi_X$ is also quasicompact. By \autoref{cor:qc-from-qc-morphisms}, it suffices to show that $X\times_SY$ is quasicompact.

	However, we note that $S$ is affine, so because $\psi_Y$ is quasicompact, \autoref{cor:qc-from-qc-morphisms} tells us that $Y$ is quasicompact. As such, we give $Y$ a finite affine open cover $\{Y_i\}_{i=1}^n$. By the construction of the fiber product in \autoref{lem:keyfibercase}, we note that $X\times_SY$ is covered by the schemes $X\times_SY_i$.

	But now each scheme $X\times_SY_i$ can have $X\cong\Spec A$ and $S\cong\Spec R$ and $Y_i\cong\Spec B_i$ so that
	\[X\times_SY_i\cong\Spec A\times_{\Spec R}\Spec B_i,\]
	which is isomorphic to $\Spec A\otimes_RB_i$ by \autoref{lem:affine-fp}. Notably, $X\times_SY_i$ is an affine scheme and hence quasicompact, so $X\times_SY$ is a finite union of quasicompact subschemes and therefore quasicompact.
\end{proof}
And here is the name.
\begin{definition}[Preserved by base change]
	Let $P$ be a class of morphisms. We say that $P$ is \textit{preserved by base change} if and only if $\varphi\colon X\to S$ being in $P$ implies that the canonical morphism $\pi_Y\colon X\times_SY\to Y$ is still in $P$, for any scheme $Y$ over $S$.
\end{definition}
We are going to use the above argument for being preserved by base change many times, so we take a second to write it down.
\begin{lemma} \label{lem:base-change-reduce-to-affine}
	Let $P$ be a class of morphisms which is affine-local on the target. Given affine schemes $S$ and $Y$, suppose that $\varphi\colon X\to S$ being in $P$ implies that the canonical morphism $\pi_Y\colon X\times_SY\to Y$ is still in $P$. Then $P$ is preserved by base change.
\end{lemma}
\begin{proof}
	Well, suppose we have a pullback square
	% https://q.uiver.app/?q=WzAsNCxbMCwwLCJYXFx0aW1lc19TWSJdLFsxLDAsIlgiXSxbMCwxLCJZIl0sWzEsMSwiUyJdLFsyLDMsIlxccHNpX1kiXSxbMCwxLCJcXHBpX1giXSxbMCwyLCJcXHBpX1kiXSxbMSwzLCJcXHBzaV9YIl0sWzAsMywiIiwyLHsic3R5bGUiOnsibmFtZSI6ImNvcm5lciJ9fV1d&macro_url=https%3A%2F%2Fraw.githubusercontent.com%2FdFoiler%2Fnotes%2Fmaster%2Fnir.tex
	\[\begin{tikzcd}
		{X\times_SY} & X \\
		Y & S
		\arrow["{\psi_Y}", from=2-1, to=2-2]
		\arrow["{\pi_X}", from=1-1, to=1-2]
		\arrow["{\pi_Y}", from=1-1, to=2-1]
		\arrow["{\psi_X}", from=1-2, to=2-2]
		\arrow["\lrcorner"{anchor=center, pos=0.125}, draw=none, from=1-1, to=2-2]
	\end{tikzcd}\]
	of schemes such that $\psi_Y\in P$; we need to show $\pi_X\in P$.

	For this, we just reduce to the affine case. Let $\pi_S\coloneqq\psi_Y\circ\pi_Y=\psi_X\circ\pi_X$, for brevity. Give $S$ an affine open cover $\{S_\alpha\}_{\alpha\in\lambda}$. For each $\alpha\in\lambda$, we give $\psi_X^{-1}(S_\alpha)\subseteq X$ an affine open cover $\{X_{\alpha,\beta}\}_{\alpha\in\lambda,\beta\in\kappa_\alpha}$. Then we build the tower
	% https://q.uiver.app/?q=WzAsNixbMCwxLCJcXHBpX1Neey0xfShTX1xcYWxwaGEpIl0sWzEsMSwiXFxwc2lfWF57LTF9KFNfXFxhbHBoYSkiXSxbMCwyLCJcXHBzaV9ZXnstMX0oU19cXGFscGhhKSJdLFsxLDIsIlNfXFxhbHBoYSJdLFsxLDAsIlhfe1xcYWxwaGEsXFxiZXRhfSJdLFswLDAsIlxccGlfWF57LTF9KFhfe1xcYWxwaGEsXFxiZXRhfSkiXSxbMiwzLCJcXHBzaV9ZfF97XFxwc2lfWV57LTF9KFNfXFxhbHBoYSl9IiwyXSxbMCwxLCJcXHBpX1h8X3tcXHBpX1Neey0xfShTX1xcYWxwaGEpfSJdLFswLDIsIlxccGlfWXxfe1xccGlfU157LTF9KFNfXFxhbHBoYSl9Il0sWzEsMywiXFxwc2lfWCJdLFs0LDEsIiIsMCx7InN0eWxlIjp7InRhaWwiOnsibmFtZSI6Imhvb2siLCJzaWRlIjoidG9wIn19fV0sWzUsNCwiXFxwaV9YfF97XFxwaV9YXnstMX0oWF97XFxhbHBoYSxcXGJldGF9KX0iXSxbNSwwLCIiLDAseyJzdHlsZSI6eyJ0YWlsIjp7Im5hbWUiOiJob29rIiwic2lkZSI6InRvcCJ9fX1dXQ==&macro_url=https%3A%2F%2Fraw.githubusercontent.com%2FdFoiler%2Fnotes%2Fmaster%2Fnir.tex
	\[\begin{tikzcd}
		{\pi_X^{-1}(X_{\alpha,\beta})} & {X_{\alpha,\beta}} \\
		{\pi_S^{-1}(S_\alpha)} & {\psi_X^{-1}(S_\alpha)} \\
		{\psi_Y^{-1}(S_\alpha)} & {S_\alpha}
		\arrow["{\psi_Y|_{\psi_Y^{-1}(S_\alpha)}}"', from=3-1, to=3-2]
		\arrow["{\pi_X|_{\pi_S^{-1}(S_\alpha)}}", from=2-1, to=2-2]
		\arrow["{\pi_Y|_{\pi_S^{-1}(S_\alpha)}}", from=2-1, to=3-1]
		\arrow["{\psi_X}", from=2-2, to=3-2]
		\arrow[hook, from=1-2, to=2-2]
		\arrow["{\pi_X|_{\pi_X^{-1}(X_{\alpha,\beta})}}", from=1-1, to=1-2]
		\arrow[hook, from=1-1, to=2-1]
	\end{tikzcd}\]
	and note that the bottom square is a pullback square by \autoref{lem:fp-open-cover-base}, the top square is a pullback square by \autoref{lem:open-fp}, so the total rectangle is a pullback square by \autoref{lem:smalls-to-big-pullback}. Now, because $P$ is affine-local on the target, it suffices to show that each restricted map $\pi_X|_{\pi_X^{-1}(X_{\alpha,\beta})}$ is in $P$. Notably, because $P$ is affine-local on the target, the restriction $\psi_Y|_{\psi_Y^{-1}(S_\alpha)}$ is in $P$.

	Thus, we fix some $\alpha$ and $\beta$. We now rename our variables, replacing $S_\alpha$ with $S$, $X_{\alpha,\beta}$ with $X$, and $\psi_Y^{-1}(S_\alpha)$ with $Y$, and we rename our morphisms to fit the pullback square
	% https://q.uiver.app/?q=WzAsNCxbMSwxLCJTIl0sWzEsMCwiWCJdLFswLDEsIlkiXSxbMCwwLCJYXFx0aW1lc19TWSJdLFszLDEsIlxccGlfWCJdLFsxLDAsIlxccHNpX1giXSxbMiwwLCJcXHBzaV9ZIl0sWzMsMiwiXFxwaV9ZIiwyXSxbMywwLCIiLDEseyJzdHlsZSI6eyJuYW1lIjoiY29ybmVyIn19XV0=&macro_url=https%3A%2F%2Fraw.githubusercontent.com%2FdFoiler%2Fnotes%2Fmaster%2Fnir.tex
	\[\begin{tikzcd}
		{X\times_SY} & X \\
		Y & S
		\arrow["{\pi_X}", from=1-1, to=1-2]
		\arrow["{\psi_X}", from=1-2, to=2-2]
		\arrow["{\psi_Y}", from=2-1, to=2-2]
		\arrow["{\pi_Y}"', from=1-1, to=2-1]
		\arrow["\lrcorner"{anchor=center, pos=0.125}, draw=none, from=1-1, to=2-2]
	\end{tikzcd}\]
	though we now have both $Y$ and $S$ affine. Now the hypothesis on $P$ kicks in because we know that the restricted version of $\psi_Y$ is in $P$, implying that $\pi_X$ is in $P$.
\end{proof}
% \begin{corollary}
% 	A morphism being quasicompact is affine local on the target: given an affine open cover $\mc U$ of $Y$, the restrictions $\pi|_{\pi^{-1}U}$ for $U\in\mc U$ are all quasicompact if and only if $\pi$ is quasicompact.
% \end{corollary}
% \begin{proof}
% 	Apply \autoref{lem:mainqclemma}.
% \end{proof}
\begin{remark}
	Being preserved under composition and base-change shows that the product of two quasicompact morphisms is quasicompact. Namely, $\varphi\colon X\to Y$ and $\varphi'\colon X'\to Y'$ makes us build the product morphism $\varphi\times\varphi'\colon X\times_{\Spec\ZZ}X'\to Y\times_{\Spec\ZZ}Y'$ by applying base-change twice.
\end{remark}

\subsection{Isomorphisms Are Reasonable}
For some more practice with the definitions, we work with the class of isomorphisms.
\begin{lemma}
	The class of isomorphisms is preserved by composition.
\end{lemma}
\begin{proof}
	This is purely category theory. Fix isomorphisms $\varphi\colon X\to Y$ and $\psi\colon Y\to Z$, and let $\varphi^{-1}\colon Y\to X$ and $\psi^{-1}Z\to Y$ be their inverses. Then we see that
	\[\psi^{-1}\circ\varphi^{-1}\circ\varphi\circ\psi={\id_Z}\qquad\text{and}\qquad\varphi\circ\psi\circ\psi^{-1}\circ\varphi^{-1}={\id_X},\]
	so $\varphi\circ\psi$ is an isomorphism with inverse $\psi^{-1}\circ\varphi^{-1}$.
\end{proof}
\begin{lemma}
	The class of isomorphisms is preserved by base change.
\end{lemma}
\begin{proof}
	This is also purely category theory. Suppose that
	% https://q.uiver.app/?q=WzAsNCxbMCwxLCJZIl0sWzEsMSwiUyJdLFswLDAsIlciXSxbMSwwLCJYIl0sWzAsMSwiXFxwc2lfWSJdLFszLDEsIlxccHNpX1giXSxbMiwzLCJcXHBpX1giXSxbMiwwLCJcXHBpX1kiXV0=&macro_url=https%3A%2F%2Fraw.githubusercontent.com%2FdFoiler%2Fnotes%2Fmaster%2Fnir.tex
	\[\begin{tikzcd}
		W & X \\
		Y & S
		\arrow["{\psi_Y}", from=2-1, to=2-2]
		\arrow["{\psi_X}", from=1-2, to=2-2]
		\arrow["{\pi_X}", from=1-1, to=1-2]
		\arrow["{\pi_Y}", from=1-1, to=2-1]
	\end{tikzcd}\]
	is a pullback square with $\psi_Y$ an isomorphism. Then we claim that $\pi_X$ is also isomorphism. For this, the main claim is that
	% https://q.uiver.app/?q=WzAsNCxbMCwxLCJZIl0sWzEsMSwiUyJdLFswLDAsIlgiXSxbMSwwLCJYIl0sWzAsMSwiXFxwc2lfWSIsMl0sWzMsMSwiXFxwc2lfWCJdLFsyLDMsIiIsMix7ImxldmVsIjoyLCJzdHlsZSI6eyJoZWFkIjp7Im5hbWUiOiJub25lIn19fV0sWzIsMCwiXFxwc2lfWV57LTF9XFxwc2lfWCJdXQ==&macro_url=https%3A%2F%2Fraw.githubusercontent.com%2FdFoiler%2Fnotes%2Fmaster%2Fnir.tex
	\[\begin{tikzcd}
		X & X \\
		Y & S
		\arrow["{\psi_Y}"', from=2-1, to=2-2]
		\arrow["{\psi_X}", from=1-2, to=2-2]
		\arrow[Rightarrow, no head, from=1-1, to=1-2]
		\arrow["{\psi_Y^{-1}\psi_X}", from=1-1, to=2-1]
	\end{tikzcd}\]
	is a pullback square. Indeed, fixing an object $Z$ and morphisms $\alpha_X\colon Z\to X$ and $\alpha_Y\colon Z\to Y$ making the diagram
	% https://q.uiver.app/?q=WzAsNSxbMSwyLCJZIl0sWzIsMiwiUyJdLFsxLDEsIlgiXSxbMiwxLCJYIl0sWzAsMCwiWiJdLFswLDEsIlxccHNpX1kiLDJdLFszLDEsIlxccHNpX1giXSxbMiwzLCIiLDIseyJsZXZlbCI6Miwic3R5bGUiOnsiaGVhZCI6eyJuYW1lIjoibm9uZSJ9fX1dLFsyLDAsIlxccHNpX1leey0xfVxccHNpX1giXSxbNCwwLCJcXGFscGhhX1kiLDIseyJjdXJ2ZSI6Mn1dLFs0LDMsIlxcYWxwaGFfWCIsMCx7ImN1cnZlIjotMn1dLFs0LDIsIlxcYWxwaGEiLDEseyJzdHlsZSI6eyJib2R5Ijp7Im5hbWUiOiJkYXNoZWQifX19XV0=&macro_url=https%3A%2F%2Fraw.githubusercontent.com%2FdFoiler%2Fnotes%2Fmaster%2Fnir.tex
	\[\begin{tikzcd}
		Z \\
		& X & X \\
		& Y & S
		\arrow["{\psi_Y}"', from=3-2, to=3-3]
		\arrow["{\psi_X}", from=2-3, to=3-3]
		\arrow[Rightarrow, no head, from=2-2, to=2-3]
		\arrow["{\psi_Y^{-1}\psi_X}", from=2-2, to=3-2]
		\arrow["{\alpha_Y}"', curve={height=12pt}, from=1-1, to=3-2]
		\arrow["{\alpha_X}", curve={height=-12pt}, from=1-1, to=2-3]
		\arrow["\alpha"{description}, dashed, from=1-1, to=2-2]
	\end{tikzcd}\]
	minus the dashed arrow, we need to show that there is a unique morphism $\alpha\colon Z\to X$ making the diagram commute. Well, $\alpha$ is unique because we see that we are requiring $\alpha=\alpha_X$ in the commutativity. And $\alpha$ exists because $\alpha_X$ does give $\psi_X\circ\alpha_X=\psi_Y\circ\alpha_X$ by hypothesis on $\alpha_X$ and $\alpha_Y$.

	Thus, we see that $X$ and $W$ are canonically isomorphic. To see that $\pi_X$ is the needed canonical isomorphism, we note that the canonical isomorphism $W\to X$ is the one induced by the diagram
	% https://q.uiver.app/?q=WzAsNSxbMSwyLCJZIl0sWzIsMiwiUyJdLFsxLDEsIlgiXSxbMiwxLCJYIl0sWzAsMCwiVyJdLFswLDEsIlxccHNpX1kiLDJdLFszLDEsIlxccHNpX1giXSxbMiwzLCIiLDIseyJsZXZlbCI6Miwic3R5bGUiOnsiaGVhZCI6eyJuYW1lIjoibm9uZSJ9fX1dLFsyLDAsIlxccHNpX1leey0xfVxccHNpX1giXSxbNCwwLCJcXHBpX1kiLDIseyJjdXJ2ZSI6Mn1dLFs0LDMsIlxccGlfWCIsMCx7ImN1cnZlIjotMn1dLFs0LDIsIiIsMSx7InN0eWxlIjp7ImJvZHkiOnsibmFtZSI6ImRhc2hlZCJ9fX1dXQ==&macro_url=https%3A%2F%2Fraw.githubusercontent.com%2FdFoiler%2Fnotes%2Fmaster%2Fnir.tex
	\[\begin{tikzcd}
		W \\
		& X & X \\
		& Y & S
		\arrow["{\psi_Y}"', from=3-2, to=3-3]
		\arrow["{\psi_X}", from=2-3, to=3-3]
		\arrow[Rightarrow, no head, from=2-2, to=2-3]
		\arrow["{\psi_Y^{-1}\psi_X}", from=2-2, to=3-2]
		\arrow["{\pi_Y}"', curve={height=12pt}, from=1-1, to=3-2]
		\arrow["{\pi_X}", curve={height=-12pt}, from=1-1, to=2-3]
		\arrow[dashed, from=1-1, to=2-2]
	\end{tikzcd}\]
	which we can see must be $\pi_X$ from the upper-right triangle.
\end{proof}
\begin{lemma} \label{lem:iso-is-local-target}
	The class of isomorphisms is local on the target.
\end{lemma}
\begin{proof}
	In one direction, if $\varphi\colon X\to Y$ is an isomorphism, then any open $V\subseteq Y$ makes $\varphi|_{\varphi^{-1}V}\colon\varphi^{-1}V\to V$ is an isomorphism by \autoref{lem:restrictmorphism}.

	In the other direction, fix a morphism $\varphi\colon X\to Y$ of schemes and an open cover $\{V_\alpha\}_{\alpha\in\lambda}$ of $Y$ such that each restricted morphism $\varphi|_{\varphi^{-1}V_\alpha}$ is an isomorphism for each $\alpha\in\lambda$. Then $\varphi$ is an isomorphism.
	
	We construct the inverse by hand. Recall from our construction of restriction (\autoref{lem:restrictmorphism}) that the topological map $\varphi_\alpha\colon\varphi^{-1}(V_\alpha)\to V_\alpha$ is the restricted map $\varphi_\alpha\coloneqq\varphi|_{\varphi^{-1}(V_\alpha)}$, and the sheaf map
	\[\varphi^\sharp_\alpha\colon\OO_Y|_{V_\alpha}\to(\varphi_\alpha)_*(\OO_X|_{\varphi^{-1}(V_\alpha)})\]
	is made of the morphisms
	\[\OO_Y|_{V_\alpha}(V)=\OO_Y(V)\stackrel{\varphi^\sharp_V}\to\varphi_*\OO_X(V)=\OO_X\left(\varphi^{-1}(V)\right)=(\varphi_\alpha)_*(\OO_X|_{\varphi^{-1}(V_\alpha)})(V)\]
	for each $V\subseteq V_\alpha$.

	We now show that $\varphi$ is an isomorphism; it suffices to show that $\varphi$ is a homeomorphism and that $\varphi^\sharp$ is an isomorphism of sheaves (by \autoref{lem:betterlocaliso}). There are two checks.
	\begin{itemize}
		\item We show $\varphi$ is a homeomorphism. Let $\varphi_\alpha\colon\varphi^{-1}(V_\alpha)\to V_\alpha$ be the restricted map. For each $\alpha\in\lambda$, there is a continuous map $\psi_\alpha\colon V_\alpha\to\varphi^{-1}(V_\alpha)$ which is the inverse for $\varphi_\alpha$.

		Now, {for each $y\in Y$, find $\alpha$ with $y\in V_\alpha$, and define $\psi(y)\coloneqq\psi_\alpha(y)$}. This is well-defined: if $y\in V_\alpha\cap V_\beta$, then we can find $x\in\varphi^{-1}(V_\alpha)$ with $\varphi(x)=y$, so it follows that actually $x\in\varphi^{-1}(V_\alpha)\cap\varphi^{-1}(V_\beta)$ as well, so $\psi_\alpha(y)=x=\psi_\beta(y)$.

		To see that $\psi$ is continuous, observe that any $U\subseteq X$ has
		\[\psi^{-1}(U)=\psi^{-1}\left(U\cap\bigcup_{\alpha\in\lambda}\varphi_\alpha^{-1}(V_\alpha)\right)=\bigcup_{\alpha\in\lambda}\psi_\alpha^{-1}\left(U\cap\varphi^{-1}(V_\alpha)\right),\]
		which is open as the arbitrary union of open subsets of $X$.

		Lastly, for any $y\in Y$, find $\alpha$ with $y\in V_\alpha$, and we see $\varphi(\psi(y))=\varphi_\alpha(\psi_\alpha(y))=y$. On the other side, for any $x\in X$, find $\alpha$ with $x\in\varphi^{-1}(V_\alpha)$, so $\psi(\varphi(x))=\psi_\alpha(\varphi_\alpha(x))=x$.

		\item We show $\varphi^\sharp\colon\OO_Y\to\varphi_*\OO_X$ is an isomorphism of sheaves. Define the collection $\mc B$ as the collection of all open subsets of $Y$ contained in a $V_\alpha$ for some $\alpha\in\lambda$. Because $\{V_\alpha\}_{\alpha\in\lambda}$ covers $Y$, we see that $\mc B$ is a base for the topology on $X$: for any $V\subseteq X$, we can write
		\[V=\bigcup_{\alpha\in\lambda}\left(V\cap V_\alpha\right).\]
		Now, we showed on previous homework that the construction of a morphism of sheaves on a base is unique, so {it will roughly suffice to show $\varphi^\sharp\colon\OO_Y\to\varphi_*\OO_X$ is an isomorphism on the base}.

		Well, for any $B\in\mc B$, we want to show that $\varphi^\sharp_V$ is an isomorphism. Well, we can find $\alpha\in\lambda$ with $B\subseteq V_\alpha$, so we note that by hypothesis
		\[\OO_Y|_{V_\alpha}(B)=\OO_Y(B)\stackrel{\varphi^\sharp_V}\to\varphi_*\OO_X(B)=\OO_X\left(\varphi^{-1}(B)\right)=(\varphi_\alpha)_*(\OO_X|_{\varphi^{-1}(V_\alpha)})(B)\]
		is an isomorphism, so in particular $\varphi^\sharp_V$ is an isomorphism.
		
		We will go ahead and construct the inverse morphism for $\varphi^\sharp$, for completeness. We may define $\psi^\sharp_B$ as the inverse of $\varphi^\sharp_B$ at all basis sets $B$. To see that $\psi^\sharp$ assembles to a morphism of sheaves on the base $\mc B$, note that any basis elements $B'\subseteq B$ have the left square of
		% https://q.uiver.app/?q=WzAsOCxbMCwwLCJcXE9PX1goQikiXSxbMSwwLCJcXHZhcnBoaV8qXFxPT19ZKEIpIl0sWzAsMSwiXFxPT19YKEInKSJdLFsxLDEsIlxcdmFycGhpXypcXE9PX1koQicpIl0sWzMsMCwiXFxPT19YKEIpIl0sWzQsMCwiXFx2YXJwaGlfKlxcT09fWShCKSJdLFszLDEsIlxcT09fWChCJykiXSxbNCwxLCJcXHZhcnBoaV8qXFxPT19ZKEInKSJdLFswLDIsIlxcb3B7cmVzfV97QixCJ30iLDJdLFsxLDMsIlxcb3B7cmVzfV97QixCJ30iXSxbMSwwLCJcXHBzaV5cXHNoYXJwX0IiLDJdLFszLDIsIlxccHNpXlxcc2hhcnBfe0InfSIsMl0sWzQsNiwiXFxvcHtyZXN9X3tCLEInfSIsMl0sWzUsNywiXFxvcHtyZXN9X3tCLEInfSJdLFs0LDUsIlxcdmFycGhpXlxcc2hhcnBfQiJdLFs2LDcsIlxcdmFycGhpXlxcc2hhcnBfe0InfSJdXQ==&macro_url=https%3A%2F%2Fraw.githubusercontent.com%2FdFoiler%2Fnotes%2Fmaster%2Fnir.tex
		\[\begin{tikzcd}
			{\OO_X(B)} & {\varphi_*\OO_Y(B)} && {\OO_X(B)} & {\varphi_*\OO_Y(B)} \\
			{\OO_X(B')} & {\varphi_*\OO_Y(B')} && {\OO_X(B')} & {\varphi_*\OO_Y(B')}
			\arrow["{\op{res}_{B,B'}}"', from=1-1, to=2-1]
			\arrow["{\op{res}_{B,B'}}", from=1-2, to=2-2]
			\arrow["{\psi^\sharp_B}"', from=1-2, to=1-1]
			\arrow["{\psi^\sharp_{B'}}"', from=2-2, to=2-1]
			\arrow["{\op{res}_{B,B'}}"', from=1-4, to=2-4]
			\arrow["{\op{res}_{B,B'}}", from=1-5, to=2-5]
			\arrow["{\varphi^\sharp_B}", from=1-4, to=1-5]
			\arrow["{\varphi^\sharp_{B'}}", from=2-4, to=2-5]
		\end{tikzcd}\]
		commutes because it is the same as the square on the right. Thus, $\psi^\sharp$ extends to a full morphism of sheaves $\varphi_*\OO_X\to\OO_Y$. Additionally, for any basis element $B\in\mc B$, we see that
		\[\varphi^\sharp_B\circ\psi^\sharp_B=\id_{\varphi_*\OO_X(B)}=({\id_{\varphi_*\OO_Y}})_B,\]
		so uniqueness of morphisms of sheaves on a base shows that $\varphi^\sharp\circ\psi^\sharp=\id_{\varphi_*\OO_X}$. Reversing the roles of $\varphi^\sharp$ with $\psi^\sharp$ and $\varphi_*\OO_Y$ with $\OO_X$ shows that $\psi^\sharp\colon\varphi^\sharp=\id_{\OO_Y}$, so we see that $\psi^\sharp$ is in fact an inverse for $\varphi^\sharp$. This finishes.
		\qedhere
	\end{itemize}
\end{proof}

\subsection{Diagonal Morphisms}
It will turn out that two important classes of morphisms---quasiseparated and separated morphisms---are best thought of in terms of ``diagonal morphisms.'' Thus, we will spend a little time discussing these.
\begin{notation}
	Fix schemes $X_1$ and $X_2$ over a scheme $Y$. Given morphisms $\alpha_1\colon Z\to X_1$ and $\alpha_2\colon Z\to X_2$ making the outer square of
	% https://q.uiver.app/?q=WzAsNSxbMSwyLCJYXzIiXSxbMiwxLCJYXzEiXSxbMiwyLCJZIl0sWzEsMSwiWF8xXFx0aW1lc19ZWF8yIl0sWzAsMCwiWiJdLFszLDFdLFszLDBdLFswLDJdLFs0LDEsIlxcYWxwaGFfMSIsMCx7ImN1cnZlIjotMn1dLFsxLDJdLFs0LDAsIlxcYWxwaGFfMiIsMix7ImN1cnZlIjoyfV0sWzQsMywiIiwxLHsic3R5bGUiOnsiYm9keSI6eyJuYW1lIjoiZGFzaGVkIn19fV0sWzMsMiwiIiwxLHsic3R5bGUiOnsibmFtZSI6ImNvcm5lciJ9fV1d&macro_url=https%3A%2F%2Fraw.githubusercontent.com%2FdFoiler%2Fnotes%2Fmaster%2Fnir.tex
	\[\begin{tikzcd}
		Z \\
		& {X_1\times_YX_2} & {X_1} \\
		& {X_2} & Y
		\arrow[from=2-2, to=2-3]
		\arrow[from=2-2, to=3-2]
		\arrow[from=3-2, to=3-3]
		\arrow["{\alpha_1}", curve={height=-12pt}, from=1-1, to=2-3]
		\arrow[from=2-3, to=3-3]
		\arrow["{\alpha_2}"', curve={height=12pt}, from=1-1, to=3-2]
		\arrow[dashed, from=1-1, to=2-2]
		\arrow["\lrcorner"{anchor=center, pos=0.125}, draw=none, from=2-2, to=3-3]
	\end{tikzcd}\]
	commute, we let $(\alpha_1,\alpha_2)$ denote the induced arrow.
\end{notation}
\begin{example}
	Given a morphism $\varphi\colon X\to Y$, we note that $\varphi\circ{\id_X}=\varphi\circ{\id_X}$, so we have induced a morphism $({\id_X},{\id_X})\colon X\to X\times_YX$. This morphism is called the ``diagonal morphism'' and is denoted $\Delta\varphi\coloneqq({\id_X},{\id_X})$.
\end{example}
Here is the quickest possible reason to care.
\begin{lemma} \label{lem:monic-means-diag-iso}
	Fix a category $\mc C$ with fiber products. A morphism $\varphi\colon X\to Y$ is monic if and only if the diagonal morphism $\Delta\varphi\colon X\to X\times_YX$ is an isomorphism.
\end{lemma}
\begin{proof}
	Let $\pi_1,\pi_2\colon X\times_YX\to X$ be the canonical projections, and set $\Delta\coloneqq\Delta\varphi$ for brevity. We have two checks.
	\begin{itemize}
		\item Suppose that $\varphi$ is monic. We claim that
		% https://q.uiver.app/?q=WzAsNCxbMCwwLCJYIl0sWzEsMCwiWCJdLFswLDEsIlgiXSxbMSwxLCJZIl0sWzAsMSwiIiwwLHsibGV2ZWwiOjIsInN0eWxlIjp7ImhlYWQiOnsibmFtZSI6Im5vbmUifX19XSxbMCwyLCIiLDIseyJsZXZlbCI6Miwic3R5bGUiOnsiaGVhZCI6eyJuYW1lIjoibm9uZSJ9fX1dLFsyLDMsIlxcdmFycGhpIl0sWzEsMywiXFx2YXJwaGkiXV0=&macro_url=https%3A%2F%2Fraw.githubusercontent.com%2FdFoiler%2Fnotes%2Fmaster%2Fnir.tex
		\[\begin{tikzcd}
			X & X \\
			X & Y
			\arrow[Rightarrow, no head, from=1-1, to=1-2]
			\arrow[Rightarrow, no head, from=1-1, to=2-1]
			\arrow["\varphi", from=2-1, to=2-2]
			\arrow["\varphi", from=1-2, to=2-2]
		\end{tikzcd}\]
		is a pullback square. Indeed, for any object $Z$ with morphisms $\alpha,\beta\colon Z\to X$ such that $\varphi\circ\alpha=\varphi\circ\beta$, we see that $\alpha=\beta$ because $\varphi$ is monic. Now, any morphism $\psi\colon Z\to X$ makes
		% https://q.uiver.app/?q=WzAsNSxbMSwxLCJYIl0sWzIsMSwiWCJdLFsxLDIsIlgiXSxbMiwyLCJZIl0sWzAsMCwiWiJdLFswLDEsIiIsMCx7ImxldmVsIjoyLCJzdHlsZSI6eyJoZWFkIjp7Im5hbWUiOiJub25lIn19fV0sWzAsMiwiIiwyLHsibGV2ZWwiOjIsInN0eWxlIjp7ImhlYWQiOnsibmFtZSI6Im5vbmUifX19XSxbMiwzLCJcXHZhcnBoaSJdLFsxLDMsIlxcdmFycGhpIl0sWzQsMCwiXFxwc2kiLDEseyJzdHlsZSI6eyJib2R5Ijp7Im5hbWUiOiJkYXNoZWQifX19XSxbNCwxLCJcXGFscGhhIiwwLHsiY3VydmUiOi0yfV0sWzQsMiwiXFxiZXRhIiwyLHsiY3VydmUiOjJ9XV0=&macro_url=https%3A%2F%2Fraw.githubusercontent.com%2FdFoiler%2Fnotes%2Fmaster%2Fnir.tex
		\[\begin{tikzcd}
			Z \\
			& X & X \\
			& X & Y
			\arrow[Rightarrow, no head, from=2-2, to=2-3]
			\arrow[Rightarrow, no head, from=2-2, to=3-2]
			\arrow["\varphi", from=3-2, to=3-3]
			\arrow["\varphi", from=2-3, to=3-3]
			\arrow["\psi"{description}, dashed, from=1-1, to=2-2]
			\arrow["\alpha", curve={height=-12pt}, from=1-1, to=2-3]
			\arrow["\beta"', curve={height=12pt}, from=1-1, to=3-2]
		\end{tikzcd}\]
		commute if and only if $\alpha=\psi=\beta$, so we see that $\psi$ both exists and is unique.

		To finish, it follows by the universal property that the diagonal morphism $\Delta\colon X\to X\times_YX$ making the diagram
		% https://q.uiver.app/?q=WzAsNSxbMCwwLCJYIl0sWzIsMSwiWCJdLFsxLDIsIlgiXSxbMiwyLCJZIl0sWzEsMSwiWFxcdGltZXNfWVgiXSxbMCwxLCIiLDAseyJjdXJ2ZSI6LTIsImxldmVsIjoyLCJzdHlsZSI6eyJoZWFkIjp7Im5hbWUiOiJub25lIn19fV0sWzAsNCwiXFxEZWx0YSIsMV0sWzQsMSwiXFxwaV8xIiwyXSxbNCwyLCJcXHBpXzIiXSxbMSwzLCJcXHZhcnBoaSJdLFsyLDMsIlxcdmFycGhpIiwyXSxbMCwyLCIiLDAseyJjdXJ2ZSI6MiwibGV2ZWwiOjIsInN0eWxlIjp7ImhlYWQiOnsibmFtZSI6Im5vbmUifX19XV0=&macro_url=https%3A%2F%2Fraw.githubusercontent.com%2FdFoiler%2Fnotes%2Fmaster%2Fnir.tex
		\[\begin{tikzcd}
			X \\
			& {X\times_YX} & X \\
			& X & Y
			\arrow[curve={height=-12pt}, Rightarrow, no head, from=1-1, to=2-3]
			\arrow["\Delta"{description}, from=1-1, to=2-2]
			\arrow["{\pi_1}"', from=2-2, to=2-3]
			\arrow["{\pi_2}", from=2-2, to=3-2]
			\arrow["\varphi", from=2-3, to=3-3]
			\arrow["\varphi"', from=3-2, to=3-3]
			\arrow[curve={height=12pt}, Rightarrow, no head, from=1-1, to=3-2]
		\end{tikzcd}\]
		commute must be the canonical isomorphism witnessing $\Delta\colon X\cong X\times_YX$.
		\item Suppose that $\Delta$ is an isomorphism. Then we note that $\pi_1\circ\Delta=\pi_2\circ\Delta={\id_X}$ forces $\pi_1$ and $\pi_2$ to also both be inverses by the uniqueness of the inverse.
		
		Now, pick up some object $Z$ with morphisms $\alpha,\beta\colon Z\to X$ such that $\varphi\circ\alpha=\varphi\circ\beta$. We need to show that $\alpha=\beta$.

		Well, we are promised a morphism $(\alpha,\beta)\colon Z\to X\times_YX$ such that $\pi_1\circ(\alpha,\beta)=\alpha$ and $\pi_2\circ(\alpha,\beta)=\beta$, so
		\[\alpha=\pi_1\circ(\alpha,\beta)=\pi_2\circ\Delta\circ\pi_1\circ(\alpha,\beta)=\pi_2\circ(\alpha,\beta)=\beta,\]
		which is what we wanted.
		\qedhere
	\end{itemize}
\end{proof}
Having defined our diagonal morphisms, we now pick up some facts about them. The following pullback square will prove quite helpful.
\begin{lemma}[Magic diagram] \label{lem:magic-diagram}
	Fix morphisms $\alpha_1\colon X_1\to Y$ and $\alpha_2\colon X_2\to Y$ and $\iota\colon Y\to Z$. Then the diagram
	% https://q.uiver.app/?q=WzAsNCxbMCwwLCJYXzFcXHRpbWVzX1lYXzIiXSxbMSwwLCJYXzFcXHRpbWVzX1pZXzIiXSxbMCwxLCJZIl0sWzEsMSwiWV8xXFx0aW1lcyBZXzIiXSxbMCwxXSxbMSwzXSxbMCwyXSxbMiwzXSxbMCwzLCIiLDEseyJzdHlsZSI6eyJuYW1lIjoiY29ybmVyIn19XV0=&macro_url=https%3A%2F%2Fraw.githubusercontent.com%2FdFoiler%2Fnotes%2Fmaster%2Fnir.tex
	\[\begin{tikzcd}
		{X_1\times_YX_2} & {X_1\times_ZY_2} \\
		Y & {Y\times_Z Y}
		\arrow[from=1-1, to=1-2]
		\arrow[from=1-2, to=2-2]
		\arrow[from=1-1, to=2-1]
		\arrow[from=2-1, to=2-2]
		\arrow["\lrcorner"{anchor=center, pos=0.125}, draw=none, from=1-1, to=2-2]
	\end{tikzcd}\]
	equipped with the natural maps is a pullback square. Here, we assume that the relevant fiber products exist.
\end{lemma}
\begin{proof}
	We proceed by force; we begin by naming our maps. Let $\pi_1\colon X_1\times_ZX_2\to X_1$ and $\pi_2\colon X_1\times_ZX_2\to X_2$ be the canonical projections. Analogously, we let $\varpi_1\colon X_1\times_YX_2\to X_1$ and $\varpi_2\colon X_1\times_YX_2\to X_2$ be the canonical projections so that the diagram
	% https://q.uiver.app/?q=WzAsNSxbMCwwLCJYXzFcXHRpbWVzX1lYXzIiXSxbMSwxLCJYXzFcXHRpbWVzX1pYXzIiXSxbMiwxLCJYXzEiXSxbMSwyLCJYXzIiXSxbMiwyLCJaIl0sWzIsNCwiXFxpb3RhXFxhbHBoYV8xIl0sWzMsNCwiXFxpb3RhXFxhbHBoYV8yIiwyXSxbMSwzLCJcXHBpXzIiXSxbMSwyLCJcXHBpXzIiLDJdLFswLDIsIlxcdmFycGlfMSIsMCx7ImN1cnZlIjotMn1dLFswLDMsIlxcdmFycGlfMiIsMix7ImN1cnZlIjoyfV0sWzEsNCwiIiwxLHsic3R5bGUiOnsibmFtZSI6ImNvcm5lciJ9fV0sWzAsMSwiXFx2YXJwaSIsMSx7InN0eWxlIjp7ImJvZHkiOnsibmFtZSI6ImRhc2hlZCJ9fX1dXQ==&macro_url=https%3A%2F%2Fraw.githubusercontent.com%2FdFoiler%2Fnotes%2Fmaster%2Fnir.tex
	\[\begin{tikzcd}
		{X_1\times_YX_2} \\
		& {X_1\times_ZX_2} & {X_1} \\
		& {X_2} & Z
		\arrow["{\iota\alpha_1}", from=2-3, to=3-3]
		\arrow["{\iota\alpha_2}"', from=3-2, to=3-3]
		\arrow["{\pi_2}", from=2-2, to=3-2]
		\arrow["{\pi_2}"', from=2-2, to=2-3]
		\arrow["{\varpi_1}", curve={height=-12pt}, from=1-1, to=2-3]
		\arrow["{\varpi_2}"', curve={height=12pt}, from=1-1, to=3-2]
		\arrow["\lrcorner"{anchor=center, pos=0.125}, draw=none, from=2-2, to=3-3]
		\arrow["\varpi"{description}, dashed, from=1-1, to=2-2]
	\end{tikzcd}\]
	commutes (indeed, $\alpha_1\circ\varpi_1=\alpha_2\circ\varpi_2$) and thus induces the morphism $\varpi$ making the diagram commute. Namely, $\varpi_1=\pi_1\circ\varpi$ and $\varpi_2=\pi_2\circ\varpi$. We take a moment to recognize that $\varpi$ is $(\varpi_1,\varpi_2)$.

	To induce the map $\Delta\iota\colon Y\to Y\times_ZY$, we let $\psi_1,\psi_2\colon Y\times_ZY\to Y$ denote the canonical projections, and we draw the diagram
	% https://q.uiver.app/?q=WzAsNSxbMSwyLCJZIl0sWzIsMSwiWSJdLFsyLDIsIloiXSxbMSwxLCJZXFx0aW1lc19aWSJdLFswLDAsIlkiXSxbMCwyLCJcXGlvdGEiXSxbMSwyLCJcXGlvdGEiLDJdLFszLDEsIlxccHNpXzEiLDJdLFszLDAsIlxccHNpXzIiXSxbNCwxLCIiLDAseyJjdXJ2ZSI6LTIsImxldmVsIjoyLCJzdHlsZSI6eyJoZWFkIjp7Im5hbWUiOiJub25lIn19fV0sWzQsMCwiIiwwLHsiY3VydmUiOjIsImxldmVsIjoyLCJzdHlsZSI6eyJoZWFkIjp7Im5hbWUiOiJub25lIn19fV0sWzQsMywiXFxEZWx0YVxcaW90YSIsMSx7InN0eWxlIjp7ImJvZHkiOnsibmFtZSI6ImRhc2hlZCJ9fX1dLFszLDIsIiIsMCx7InN0eWxlIjp7Im5hbWUiOiJjb3JuZXIifX1dXQ==&macro_url=https%3A%2F%2Fraw.githubusercontent.com%2FdFoiler%2Fnotes%2Fmaster%2Fnir.tex
	\[\begin{tikzcd}
		Y \\
		& {Y\times_ZY} & Y \\
		& Y & Z
		\arrow["\iota", from=3-2, to=3-3]
		\arrow["\iota"', from=2-3, to=3-3]
		\arrow["{\psi_1}"', from=2-2, to=2-3]
		\arrow["{\psi_2}", from=2-2, to=3-2]
		\arrow[curve={height=-12pt}, Rightarrow, no head, from=1-1, to=2-3]
		\arrow[curve={height=12pt}, Rightarrow, no head, from=1-1, to=3-2]
		\arrow["\Delta\iota"{description}, dashed, from=1-1, to=2-2]
		\arrow["\lrcorner"{anchor=center, pos=0.125}, draw=none, from=2-2, to=3-3]
	\end{tikzcd}\]
	and note that the outer square commutes because we're dealing with identities. This induces our desired diagonal morphism $\Delta\iota$.

	Next, we induce the map $(\alpha_1,\alpha_2)\colon X_1\times_ZX_2\to Y\times_ZY$ by noting that the outer ``square'' of
	% https://q.uiver.app/?q=WzAsNyxbMCwwLCJYXzFcXHRpbWVzX1lYXzIiXSxbMSwxLCJZXFx0aW1lc19aWSJdLFsxLDIsIlkiXSxbMiwxLCJZIl0sWzIsMiwiWiJdLFsxLDAsIlhfMSJdLFswLDEsIlhfMiJdLFsxLDMsIlxccHNpXzEiLDJdLFsxLDIsIlxccHNpXzIiXSxbMiw0LCJcXGlvdGEiLDJdLFszLDQsIlxcaW90YSJdLFswLDUsIlxccGlfMSJdLFswLDYsIlxccGlfMiIsMl0sWzUsMywiXFxhbHBoYV8xIl0sWzYsMiwiXFxhbHBoYV8yIiwyXSxbMCwxLCIiLDEseyJzdHlsZSI6eyJib2R5Ijp7Im5hbWUiOiJkYXNoZWQifX19XV0=&macro_url=https%3A%2F%2Fraw.githubusercontent.com%2FdFoiler%2Fnotes%2Fmaster%2Fnir.tex
	\[\begin{tikzcd}
		{X_1\times_YX_2} & {X_1} \\
		{X_2} & {Y\times_ZY} & Y \\
		& Y & Z
		\arrow["{\psi_1}"', from=2-2, to=2-3]
		\arrow["{\psi_2}", from=2-2, to=3-2]
		\arrow["\iota"', from=3-2, to=3-3]
		\arrow["\iota", from=2-3, to=3-3]
		\arrow["{\pi_1}", from=1-1, to=1-2]
		\arrow["{\pi_2}"', from=1-1, to=2-1]
		\arrow["{\alpha_1}", from=1-2, to=2-3]
		\arrow["{\alpha_2}"', from=2-1, to=3-2]
		\arrow[dashed, from=1-1, to=2-2]
	\end{tikzcd}\]
	commutes because $\alpha_1\circ\pi_1=\alpha_2\circ\pi_2$, so we have induced a dashed arrow we name $(\alpha_1,\alpha_2)$. Importantly, $\psi_1\circ(\alpha_1,\alpha_2)=\alpha_1\circ\pi_1=\alpha_2\circ\pi_2=\psi_2\circ(\alpha_1,\alpha_2)$.

	Lastly, we induce the map $X_1\times_YX_2\to Y$ just by $\alpha_1\circ\pi_1=\alpha_2\circ\pi_2$. To see that the magic diagram commutes, we note that there is at most morphism $X_1\times_YX_2\to Y\times_ZY$ which can fill into the dashed arrow of
	% https://q.uiver.app/?q=WzAsNyxbMCwwLCJYXzFcXHRpbWVzX1lYXzIiXSxbMSwxLCJZXFx0aW1lc19aWSJdLFsyLDIsIloiXSxbMiwxLCJZIl0sWzEsMiwiWSJdLFsxLDAsIlhfMSJdLFswLDEsIlhfMiJdLFswLDYsIlxccGlfMiIsMl0sWzEsMywiXFxwc2lfMSIsMl0sWzEsNCwiXFxwc2lfMiJdLFswLDUsIlxccGlfMSJdLFs1LDMsIlxcYWxwaGFfMSJdLFs2LDQsIlxcYWxwaGFfMiIsMl0sWzQsMiwiXFxpb3RhIiwyXSxbMywyLCJcXGlvdGEiXSxbMSwyLCIiLDEseyJzdHlsZSI6eyJuYW1lIjoiY29ybmVyIn19XSxbMCwxLCIiLDEseyJzdHlsZSI6eyJib2R5Ijp7Im5hbWUiOiJkYXNoZWQifX19XV0=&macro_url=https%3A%2F%2Fraw.githubusercontent.com%2FdFoiler%2Fnotes%2Fmaster%2Fnir.tex
	\[\begin{tikzcd}
		{X_1\times_YX_2} & {X_1} \\
		{X_2} & {Y\times_ZY} & Y \\
		& Y & Z
		\arrow["{\varpi_2}"', from=1-1, to=2-1]
		\arrow["{\psi_1}"', from=2-2, to=2-3]
		\arrow["{\psi_2}", from=2-2, to=3-2]
		\arrow["{\varpi_1}", from=1-1, to=1-2]
		\arrow["{\alpha_1}", from=1-2, to=2-3]
		\arrow["{\alpha_2}"', from=2-1, to=3-2]
		\arrow["\iota"', from=3-2, to=3-3]
		\arrow["\iota", from=2-3, to=3-3]
		\arrow["\lrcorner"{anchor=center, pos=0.125}, draw=none, from=2-2, to=3-3]
		\arrow[dashed, from=1-1, to=2-2]
	\end{tikzcd}\]
	to make the diagram commute. However, the diagram
	% https://q.uiver.app/?q=WzAsNyxbMCwwLCJYXzFcXHRpbWVzX1lYXzIiXSxbMSwxLCJZIl0sWzEsMCwiWF8xIl0sWzAsMSwiWF8yIl0sWzIsMiwiWVxcdGltZXNfWlkiXSxbMywyLCJZIl0sWzIsMywiWSJdLFswLDFdLFswLDMsIlxcdmFycGlfMiIsMl0sWzMsMSwiXFxhbHBoYV8yIl0sWzAsMiwiXFx2YXJwaV8xIl0sWzEsNCwiXFxEZWx0YVxcaW90YSIsMV0sWzIsMSwiXFxhbHBoYV8xIiwyXSxbNCw1LCJcXHBzaV8xIiwyXSxbNCw2LCJcXHBzaV8yIl0sWzIsNSwiXFxhbHBoYV8xIl0sWzEsNSwiIiwwLHsibGV2ZWwiOjIsInN0eWxlIjp7ImhlYWQiOnsibmFtZSI6Im5vbmUifX19XSxbMyw2LCJcXGFscGhhXzIiLDJdLFsxLDYsIiIsMix7ImxldmVsIjoyLCJzdHlsZSI6eyJoZWFkIjp7Im5hbWUiOiJub25lIn19fV1d&macro_url=https%3A%2F%2Fraw.githubusercontent.com%2FdFoiler%2Fnotes%2Fmaster%2Fnir.tex
	\[\begin{tikzcd}
		{X_1\times_YX_2} & {X_1} \\
		{X_2} & Y \\
		&& {Y\times_ZY} & Y \\
		&& Y
		\arrow[from=1-1, to=2-2]
		\arrow["{\varpi_2}"', from=1-1, to=2-1]
		\arrow["{\alpha_2}", from=2-1, to=2-2]
		\arrow["{\varpi_1}", from=1-1, to=1-2]
		\arrow["\Delta\iota"{description}, from=2-2, to=3-3]
		\arrow["{\alpha_1}"', from=1-2, to=2-2]
		\arrow["{\psi_1}"', from=3-3, to=3-4]
		\arrow["{\psi_2}", from=3-3, to=4-3]
		\arrow["{\alpha_1}", from=1-2, to=3-4]
		\arrow[Rightarrow, no head, from=2-2, to=3-4]
		\arrow["{\alpha_2}"', from=2-1, to=4-3]
		\arrow[Rightarrow, no head, from=2-2, to=4-3]
	\end{tikzcd}\]
	commutes mostly by construction of $\Delta\iota$, and the diagram
	% https://q.uiver.app/?q=WzAsNyxbMCwwLCJYXzFcXHRpbWVzX1lYXzIiXSxbMSwxLCJYXzFcXHRpbWVzX1pYXzIiXSxbMiwyLCJZXFx0aW1lc19aWSJdLFsyLDEsIlhfMSJdLFsxLDIsIlhfMiJdLFsyLDMsIlkiXSxbMywyLCJZIl0sWzEsNCwiXFxwaV8xIl0sWzAsMSwiXFx2YXJwaSIsMV0sWzEsMywiXFxwaV8yIiwyXSxbMSwyLCIoXFxhbHBoYV8xLFxcYWxwaGFfMikiLDFdLFszLDYsIlxcYWxwaGFfMSJdLFs0LDUsIlxcYWxwaGFfMiIsMl0sWzAsMywiXFx2YXJwaV8xIiwwLHsiY3VydmUiOi0yfV0sWzAsNCwiXFx2YXJwaV8yIiwyLHsiY3VydmUiOjJ9XSxbMiw1LCJcXHBzaV8yIl0sWzIsNiwiXFxwc2lfMSIsMl1d&macro_url=https%3A%2F%2Fraw.githubusercontent.com%2FdFoiler%2Fnotes%2Fmaster%2Fnir.tex
	\[\begin{tikzcd}
		{X_1\times_YX_2} \\
		& {X_1\times_ZX_2} & {X_1} \\
		& {X_2} & {Y\times_ZY} & Y \\
		&& Y
		\arrow["{\pi_1}", from=2-2, to=3-2]
		\arrow["\varpi"{description}, from=1-1, to=2-2]
		\arrow["{\pi_2}"', from=2-2, to=2-3]
		\arrow["{(\alpha_1,\alpha_2)}"{description}, from=2-2, to=3-3]
		\arrow["{\alpha_1}", from=2-3, to=3-4]
		\arrow["{\alpha_2}"', from=3-2, to=4-3]
		\arrow["{\varpi_1}", curve={height=-12pt}, from=1-1, to=2-3]
		\arrow["{\varpi_2}"', curve={height=12pt}, from=1-1, to=3-2]
		\arrow["{\psi_2}", from=3-3, to=4-3]
		\arrow["{\psi_1}"', from=3-3, to=3-4]
	\end{tikzcd}\]
	commutes by construction of both $\omega$ and $(\alpha_1,\alpha_2)$.

	We are now ready to show the universal property. Fix some object $W$ with maps $\varphi_F\colon W\to X_1\times_ZY_2$ and $\varphi_Y\colon W\to Y$ such that $(\alpha_1,\alpha_2)\circ\varphi_F=\Delta\iota\circ\varphi_Y$. Then we need a unique morphism $\varphi\colon W\to X_1\times_YX_2$ making the diagram
	% https://q.uiver.app/?q=WzAsNSxbMSwxLCJYXzFcXHRpbWVzX1lYXzIiXSxbMiwxLCJYXzFcXHRpbWVzX1pYXzIiXSxbMiwyLCJZXFx0aW1lc19aWSJdLFsxLDIsIlkiXSxbMCwwLCJXIl0sWzMsMiwiXFxEZWx0YVxcaW90YSIsMl0sWzAsMSwiXFx2YXJwaSIsMl0sWzEsMiwiKFxcYWxwaGFfMSxcXGFscGhhXzIpIl0sWzAsMywiXFxhbHBoYV8xXFxwaV8xIl0sWzQsMywiXFx2YXJwaGlfWSIsMix7ImN1cnZlIjoyfV0sWzQsMSwiXFx2YXJwaGlfRiIsMCx7ImN1cnZlIjotMn1dLFs0LDAsIlxcdmFycGhpIiwxLHsic3R5bGUiOnsiYm9keSI6eyJuYW1lIjoiZGFzaGVkIn19fV1d&macro_url=https%3A%2F%2Fraw.githubusercontent.com%2FdFoiler%2Fnotes%2Fmaster%2Fnir.tex
	\[\begin{tikzcd}
		W \\
		& {X_1\times_YX_2} & {X_1\times_ZX_2} \\
		& Y & {Y\times_ZY}
		\arrow["\Delta\iota"', from=3-2, to=3-3]
		\arrow["\varpi"', from=2-2, to=2-3]
		\arrow["{(\alpha_1,\alpha_2)}", from=2-3, to=3-3]
		\arrow["{\alpha_1\varpi_1}", from=2-2, to=3-2]
		\arrow["{\varphi_Y}"', curve={height=12pt}, from=1-1, to=3-2]
		\arrow["{\varphi_F}", curve={height=-12pt}, from=1-1, to=2-3]
		\arrow["\varphi"{description}, dashed, from=1-1, to=2-2]
	\end{tikzcd}\]
	commute. We show uniqueness and existence separately.
	\begin{itemize}
		\item Uniqueness: given $\varphi$, we claim that the diagram
		% https://q.uiver.app/?q=WzAsNSxbMCwwLCJXIl0sWzEsMSwiWF8xXFx0aW1lc19ZWF8yIl0sWzIsMSwiWF8xIl0sWzEsMiwiWF8yIl0sWzIsMiwiWSJdLFswLDEsIlxcdmFycGhpIiwxXSxbMCwyLCJcXHBpXzFcXHZhcnBoaV9GIiwwLHsiY3VydmUiOi0yfV0sWzAsMywiXFxwaV8yXFx2YXJwaGlfRiIsMix7ImN1cnZlIjoyfV0sWzEsMiwiXFx2YXJwaV8xIiwyXSxbMSwzLCJcXHZhcnBpXzIiXSxbMiw0LCJcXGFscGhhXzEiXSxbMyw0LCJcXGFscGhhXzIiLDJdXQ==&macro_url=https%3A%2F%2Fraw.githubusercontent.com%2FdFoiler%2Fnotes%2Fmaster%2Fnir.tex
		\[\begin{tikzcd}
			W \\
			& {X_1\times_YX_2} & {X_1} \\
			& {X_2} & Y
			\arrow["\varphi"{description}, from=1-1, to=2-2]
			\arrow["{\pi_1\varphi_F}", curve={height=-12pt}, from=1-1, to=2-3]
			\arrow["{\pi_2\varphi_F}"', curve={height=12pt}, from=1-1, to=3-2]
			\arrow["{\varpi_1}"', from=2-2, to=2-3]
			\arrow["{\varpi_2}", from=2-2, to=3-2]
			\arrow["{\alpha_1}", from=2-3, to=3-3]
			\arrow["{\alpha_2}"', from=3-2, to=3-3]
		\end{tikzcd}\]
		commutes, which will of course uniquely determine $\varphi$ by our pullback. Well, we see $\varpi_i\circ\varphi=\pi_i\circ\varpi\circ\varphi=\pi_i\circ\varphi_F$ for $i\in\{1,2\}$, which is what we wanted.
		\item Existence: as above, we claim that the diagram
		% https://q.uiver.app/?q=WzAsNSxbMCwwLCJXIl0sWzEsMSwiWF8xXFx0aW1lc19ZWF8yIl0sWzIsMSwiWF8xIl0sWzEsMiwiWF8yIl0sWzIsMiwiWSJdLFswLDIsIlxccGlfMVxcdmFycGhpX0YiLDAseyJjdXJ2ZSI6LTJ9XSxbMCwzLCJcXHBpXzJcXHZhcnBoaV9GIiwyLHsiY3VydmUiOjJ9XSxbMSwyLCJcXHZhcnBpXzEiLDJdLFsxLDMsIlxcdmFycGlfMiJdLFsyLDQsIlxcYWxwaGFfMSJdLFszLDQsIlxcYWxwaGFfMiIsMl1d&macro_url=https%3A%2F%2Fraw.githubusercontent.com%2FdFoiler%2Fnotes%2Fmaster%2Fnir.tex
		\[\begin{tikzcd}
			W \\
			& {X_1\times_YX_2} & {X_1} \\
			& {X_2} & Y
			\arrow["{\pi_1\varphi_F}", curve={height=-12pt}, from=1-1, to=2-3]
			\arrow["{\pi_2\varphi_F}"', curve={height=12pt}, from=1-1, to=3-2]
			\arrow["{\varpi_1}"', from=2-2, to=2-3]
			\arrow["{\varpi_2}", from=2-2, to=3-2]
			\arrow["{\alpha_1}", from=2-3, to=3-3]
			\arrow["{\alpha_2}"', from=3-2, to=3-3]
		\end{tikzcd}\]
		commutes, which will induce the desired morphism $\varphi$. Indeed, $\alpha_i\circ\pi_i\circ\varphi_F=\psi_i\circ(\alpha_1,\alpha_2)\circ\varphi_F=\psi_i\circ\Delta_Y\circ\varphi_Y=\varphi_Y$ for each $i\in\{1,2\}$.

		We now run our checks on $\varphi$. On one side, we see that $\alpha_1\circ\varpi_1\circ\varphi=\alpha_1\circ\pi_1\circ\varpi_F$ equals $\varphi_F$ as computed above. On the other side, we see that
		\[\pi_i\circ\varpi\circ\varphi=\varpi_i\circ\varphi=\pi_i\circ\varphi_F,\]
		so both $\varpi\circ\varphi$ and $\varphi_F$ could fill the dashed arrow in the diagram
		% https://q.uiver.app/?q=WzAsNSxbMSwxLCJYXzFcXHRpbWVzX1pYXzIiXSxbMiwxLCJYXzEiXSxbMSwyLCJYXzIiXSxbMiwyLCJaIl0sWzAsMCwiVyJdLFsxLDMsIlxcaW90YVxcYWxwaGFfMSJdLFsyLDMsIlxcaW90YVxcYWxwaGFfMiIsMl0sWzAsMSwiXFx2YXJwaV8xIiwyXSxbMCwyLCJcXHZhcnBpXzIiXSxbNCwxLCJcXHZhcnBpXzFcXHZhcnBoaV9GIiwwLHsiY3VydmUiOi0yfV0sWzQsMiwiXFx2YXJwaV8yXFx2YXJwaGlfRiIsMix7ImN1cnZlIjoyfV0sWzQsMCwiIiwxLHsic3R5bGUiOnsiYm9keSI6eyJuYW1lIjoiZGFzaGVkIn19fV1d&macro_url=https%3A%2F%2Fraw.githubusercontent.com%2FdFoiler%2Fnotes%2Fmaster%2Fnir.tex
		\[\begin{tikzcd}
			W \\
			& {X_1\times_ZX_2} & {X_1} \\
			& {X_2} & Z
			\arrow["{\iota\alpha_1}", from=2-3, to=3-3]
			\arrow["{\iota\alpha_2}"', from=3-2, to=3-3]
			\arrow["{\varpi_1}"', from=2-2, to=2-3]
			\arrow["{\varpi_2}", from=2-2, to=3-2]
			\arrow["{\varpi_1\varphi_F}", curve={height=-12pt}, from=1-1, to=2-3]
			\arrow["{\varpi_2\varphi_F}"', curve={height=12pt}, from=1-1, to=3-2]
			\arrow[dashed, from=1-1, to=2-2]
		\end{tikzcd}\]
		where we know there is space for at most morphism. Thus, $\varpi\circ\varphi=\varphi_F$ follows.
	\end{itemize}
	The above checks complete the proof.
\end{proof}
We now run a few checks on classes of diagonal morphisms.
\begin{notation}
	Given a class of morphisms $P$, we let $\Delta P$ denote the class of morphisms $\pi$ such that $\Delta\pi\in\Delta P$.
\end{notation}
\begin{lemma} \label{lem:diagonal-composition}
	Fix a class $P$ of morphisms which is preserved by composition and base change. Then $\Delta P$ is preserved by composition.
\end{lemma}
\begin{proof}
	Fix morphisms $\varphi\colon X\to Y$ and $\psi\colon Y\to Z$ in $\Delta P$ so that we want to show $\psi\circ\varphi$ is still in $\Delta P$. Namely, we are given that the diagonal morphisms $\Delta\varphi\colon X\to X\times_YX$ and $\Delta\psi\colon Y\to Y\times_ZY$ are in $P$.

	Now, \autoref{lem:magic-diagram} promises us the pullback square
	% https://q.uiver.app/?q=WzAsNCxbMCwwLCJYXFx0aW1lc19ZWCJdLFsxLDAsIlhcXHRpbWVzX1pZIl0sWzAsMSwiWSJdLFsxLDEsIllcXHRpbWVzX1pZIl0sWzAsMSwiXFxkZWx0YSJdLFsxLDNdLFsyLDMsIlxcRGVsdGFcXHBzaSJdLFswLDJdLFswLDMsIiIsMCx7InN0eWxlIjp7Im5hbWUiOiJjb3JuZXIifX1dXQ==&macro_url=https%3A%2F%2Fraw.githubusercontent.com%2FdFoiler%2Fnotes%2Fmaster%2Fnir.tex
	\[\begin{tikzcd}
		{X\times_YX} & {X\times_ZY} \\
		Y & {Y\times_ZY}
		\arrow["\delta", from=1-1, to=1-2]
		\arrow[from=1-2, to=2-2]
		\arrow["\Delta\psi", from=2-1, to=2-2]
		\arrow[from=1-1, to=2-1]
		\arrow["\lrcorner"{anchor=center, pos=0.125}, draw=none, from=1-1, to=2-2]
	\end{tikzcd}\]
	which because $P$ is preserved by base change tells us that the natural map $\delta\colon X\times_YX\to X\times_ZX$ is in $P$.

	To finish, we note that the diagonal morphism $\Delta(\psi\circ\varphi)\colon X\to X\times_ZX$ is the one induced by $\id_X\colon X\to X$, but then the commutative diagram
	% https://q.uiver.app/?q=WzAsOCxbMCwwLCJYIl0sWzEsMSwiWFxcdGltZXNfWVgiXSxbMiwyLCJYXFx0aW1lc19aWCJdLFszLDIsIlgiXSxbMiwzLCJYIl0sWzMsMywiWiJdLFsyLDEsIlgiXSxbMSwyLCJYIl0sWzAsMSwiXFxEZWx0YVxcdmFycGhpIiwxXSxbMCw3LCIiLDEseyJjdXJ2ZSI6MiwibGV2ZWwiOjIsInN0eWxlIjp7ImhlYWQiOnsibmFtZSI6Im5vbmUifX19XSxbMCw2LCIiLDEseyJjdXJ2ZSI6LTIsImxldmVsIjoyLCJzdHlsZSI6eyJoZWFkIjp7Im5hbWUiOiJub25lIn19fV0sWzEsN10sWzEsNl0sWzIsNF0sWzIsM10sWzEsMiwiXFxkZWx0YSIsMV0sWzYsMywiIiwxLHsibGV2ZWwiOjIsInN0eWxlIjp7ImhlYWQiOnsibmFtZSI6Im5vbmUifX19XSxbNyw0LCIiLDEseyJsZXZlbCI6Miwic3R5bGUiOnsiaGVhZCI6eyJuYW1lIjoibm9uZSJ9fX1dLFszLDUsIlxccHNpXFx2YXJwaGkiLDJdLFs0LDUsIlxccHNpXFx2YXJwaGkiXV0=&macro_url=https%3A%2F%2Fraw.githubusercontent.com%2FdFoiler%2Fnotes%2Fmaster%2Fnir.tex
	\[\begin{tikzcd}
		X \\
		& {X\times_YX} & X \\
		& X & {X\times_ZX} & X \\
		&& X & Z
		\arrow["\Delta\varphi"{description}, from=1-1, to=2-2]
		\arrow[curve={height=12pt}, Rightarrow, no head, from=1-1, to=3-2]
		\arrow[curve={height=-12pt}, Rightarrow, no head, from=1-1, to=2-3]
		\arrow[from=2-2, to=3-2]
		\arrow[from=2-2, to=2-3]
		\arrow[from=3-3, to=4-3]
		\arrow[from=3-3, to=3-4]
		\arrow["\delta"{description}, from=2-2, to=3-3]
		\arrow[Rightarrow, no head, from=2-3, to=3-4]
		\arrow[Rightarrow, no head, from=3-2, to=4-3]
		\arrow["\psi\varphi"', from=3-4, to=4-4]
		\arrow["\psi\varphi", from=4-3, to=4-4]
	\end{tikzcd}\]
	tells us that the natural composite map $X\stackrel{\Delta\varphi}\to X\times_YX\stackrel\delta\to X\times_ZX$ must also be $\Delta(\psi\circ\varphi)$ by the uniqueness of the definition of $\Delta(\psi\circ\varphi)$ from the bottom-right pullback square. So because $\Delta\varphi$ and $\delta$ are both in $P$, we conclude that their composite $\Delta(\psi\circ\varphi)$ is also in $P$. Thus, $\psi\circ\varphi$ is in $P$.
\end{proof}
\begin{lemma} \label{lem:diagonal-base-change}
	Fix a class $P$ of morphisms which is preserved by base change. Then $\Delta P$ is preserved by base change.
\end{lemma}
\begin{proof}
	Fix a pullback square
	% https://q.uiver.app/?q=WzAsNCxbMCwwLCJYJyJdLFsxLDAsIlknIl0sWzAsMSwiWCJdLFsxLDEsIlkiXSxbMiwzLCJcXHBpIl0sWzAsMSwiXFxwaSciXSxbMSwzLCJcXHZhcnBoaSJdLFswLDIsIlxcdmFycGhpJyJdXQ==&macro_url=https%3A%2F%2Fraw.githubusercontent.com%2FdFoiler%2Fnotes%2Fmaster%2Fnir.tex
	\[\begin{tikzcd}
		{X'} & {Y'} \\
		X & Y
		\arrow["\pi", from=2-1, to=2-2]
		\arrow["{\pi'}", from=1-1, to=1-2]
		\arrow["\varphi", from=1-2, to=2-2]
		\arrow["{\varphi'}", from=1-1, to=2-1]
	\end{tikzcd}\]
	such that $\pi\in\Delta P$. We want to show that $\pi'\in\Delta P$.

	Well, we let $\pi_1,\pi_2\colon X\times_YX\to X$ and $\pi_1'\colon X'\times_{Y'}X'\to X'$ be the canonical projections so that $\varphi'\pi_1'=\varphi'\pi_2'$, implying $\pi\varphi'\pi_1'=\pi\varphi'\pi_2$, which induces $(\varphi'\pi_1',\varphi'\pi_2')\colon X'\times_{Y'}X'\to X\times_YX$. The key claim, now, is that the square
	% https://q.uiver.app/?q=WzAsNCxbMCwwLCJYJyJdLFsxLDAsIlgnXFx0aW1lc197WSd9WCciXSxbMCwxLCJYIl0sWzEsMSwiWFxcdGltZXNfWVgiXSxbMiwzLCJcXERlbHRhXFxwaSJdLFswLDEsIlxcRGVsdGFcXHBpJyJdLFsxLDMsIihcXHBpXzEnXFx2YXJwaGknLFxccGlfMidcXHZhcnBoaScpIl0sWzAsMiwiXFx2YXJwaGknIl1d&macro_url=https%3A%2F%2Fraw.githubusercontent.com%2FdFoiler%2Fnotes%2Fmaster%2Fnir.tex
	\[\begin{tikzcd}
		{X'} & {X'\times_{Y'}X'} \\
		X & {X\times_YX}
		\arrow["\Delta\pi", from=2-1, to=2-2]
		\arrow["{\Delta\pi'}", from=1-1, to=1-2]
		\arrow["{(\varphi'\pi_1',\varphi'\pi_2')}", from=1-2, to=2-2]
		\arrow["{\varphi'}", from=1-1, to=2-1]
	\end{tikzcd}\]
	is a pullback square. This will finish because $P$ is preserved by base change: note $\Delta\pi\in P$ implies $\Delta\pi'\in P$, so $\pi'\in\Delta P$.
	
	Now, at the very least this square commutes because
	\[\pi_i\circ(\varphi'\pi_1',\varphi'\pi_2')\circ\Delta\pi'=\varphi'\circ\pi_i'\circ\Delta\pi'=\varphi'=\pi_i\circ\Delta\pi\circ\varphi'\]
	for each $i\in\{1,2\}$, so it follows that $(\varphi'\pi_1',\varphi'\pi_2')\circ\Delta\pi'=\Delta\pi\circ\varphi'$ by the definition of maps into the fiber product.

	Continuing, to show that our square is a pullback square, we use \autoref{lem:big-to-small-square}. Namely, we claim that the outer and right squares of
	% https://q.uiver.app/?q=WzAsNixbMCwwLCJYJyJdLFsxLDAsIlgnXFx0aW1lc197WSd9WCciXSxbMCwxLCJYIl0sWzEsMSwiWFxcdGltZXNfWVgiXSxbMiwwLCJZJyJdLFsyLDEsIlkiXSxbMiwzLCJcXERlbHRhXFxwaSJdLFswLDEsIlxcRGVsdGFcXHBpJyJdLFsxLDMsIihcXHBpXzEnXFx2YXJwaGknLFxccGlfMidcXHZhcnBoaScpIl0sWzAsMiwiXFx2YXJwaGknIl0sWzEsNCwiXFx2YXJwaGknXFxwaV8xJyJdLFszLDUsIlxcdmFycGhpXFxwaV8xIl0sWzQsNSwiXFx2YXJwaGkiXSxbMCw0LCJcXHBpJyIsMCx7Im9mZnNldCI6LTIsImN1cnZlIjotMn1dLFsyLDUsIlxccGkiLDIseyJjdXJ2ZSI6Mn1dXQ==&macro_url=https%3A%2F%2Fraw.githubusercontent.com%2FdFoiler%2Fnotes%2Fmaster%2Fnir.tex
	\begin{equation}
		\begin{tikzcd}
			{X'} & {X'\times_{Y'}X'} & {Y'} \\
			X & {X\times_YX} & Y
			\arrow["\Delta\pi", from=2-1, to=2-2]
			\arrow["{\Delta\pi'}", from=1-1, to=1-2]
			\arrow["{(\varphi'\pi_1',\varphi'\pi_2')}", from=1-2, to=2-2]
			\arrow["{\varphi'}", from=1-1, to=2-1]
			\arrow["{\pi'\pi_1'}", from=1-2, to=1-3]
			\arrow["{\pi\pi_1}", from=2-2, to=2-3]
			\arrow["\varphi", from=1-3, to=2-3]
			\arrow["{\pi'}", shift left=2, curve={height=-12pt}, from=1-1, to=1-3]
			\arrow["\pi"', curve={height=12pt}, from=2-1, to=2-3]
		\end{tikzcd} \label{eq:intermediate-pullback-diagonals}
	\end{equation}
	are pullback square. To begin, we note that at least the diagram commutes: namely, the right square has
	\[\pi\circ\pi_1\circ(\varphi'\pi_1',\varphi'\pi_2')=\pi\circ\varphi'\circ\pi_1'=\varphi\circ\pi'\circ\pi_1'.\]
	Further, we see that $\pi'\circ\pi_1'\circ\Delta\pi'=\pi'$ and $\pi\circ\pi_1\circ\Delta\pi=\pi$ by construction of the diagonal morphisms.

	Thus, we note that the outer rectangle of \autoref{eq:intermediate-pullback-diagonals} is in fact a pullback square by hypothesis, so it really only remains to show that the right square of \autoref{eq:intermediate-pullback-diagonals} is a pullback square. For this, we use \autoref{lem:smalls-to-big-pullback}, writing down
	% https://q.uiver.app/?q=WzAsNixbMCwwLCJYJ1xcdGltZXNfe1knfVgnIl0sWzAsMSwiWSciXSxbMSwwLCJYJ1xcdGltZXNfe1l9WCciXSxbMSwxLCJZJ1xcdGltZXNfWVknIl0sWzIsMCwiWFxcdGltZXNfWVgiXSxbMiwxLCJZIl0sWzAsMSwiXFxwaSdcXHBpXzEnIl0sWzEsMywiXFxEZWx0YVxcdmFycGhpIl0sWzAsNCwiKFxcdmFycGhpJ1xccGlfMScsXFx2YXJwaGknXFxwaV8yJykiLDAseyJvZmZzZXQiOi0yLCJjdXJ2ZSI6LTJ9XSxbNCw1LCJcXHBpXFxwaV8xIl0sWzMsNSwiXFx2YXJwaV8xIl0sWzEsNSwiXFx2YXJwaGkiLDIseyJjdXJ2ZSI6Mn1dLFswLDIsIihcXHBpXzEnLFxccGlfMicpIl0sWzIsMywiKFxccGknXFxwaV8xLFxccGknXFxwaV8yKSJdLFsyLDQsIihcXHZhcnBoaScsXFx2YXJwaGknKSJdXQ==&macro_url=https%3A%2F%2Fraw.githubusercontent.com%2FdFoiler%2Fnotes%2Fmaster%2Fnir.tex
	\[\begin{tikzcd}
		{X'\times_{Y'}X'} & {X'\times_{Y}X'} & {X\times_YX} \\
		{Y'} & {Y'\times_YY'} & Y
		\arrow["{\pi'\pi_1'}", from=1-1, to=2-1]
		\arrow["\Delta\varphi", from=2-1, to=2-2]
		\arrow["{(\varphi'\pi_1',\varphi'\pi_2')}", shift left=2, curve={height=-12pt}, from=1-1, to=1-3]
		\arrow["{\pi\pi_1}", from=1-3, to=2-3]
		\arrow["{\varphi\varpi_1}", from=2-2, to=2-3]
		\arrow["\varphi"', curve={height=12pt}, from=2-1, to=2-3]
		\arrow["{(\pi_1',\pi_2')}", from=1-1, to=1-2]
		\arrow["{(\pi'\pi_1',\pi'\pi_2')}", from=1-2, to=2-2]
		\arrow["{(\varphi',\varphi')}", from=1-2, to=1-3]
	\end{tikzcd}\]
	to claim that both squares are pullback squares; here $\varpi_1,\varpi_2\colon Y\times_XY\to Y$ are the canonical projections. The left square commutes and is a pullback square by \autoref{lem:magic-diagram}. The right square commutes because
	\[\pi\circ\pi_1\circ(\varphi',\varphi')=\pi\circ\varphi'\circ\pi_1'=\varphi\circ\pi'\circ\pi_1'=\varphi\circ\varpi_1\circ(\pi'\pi_1',\pi'\pi_2').\]
	At a high level, the right square is a pullback square because
	\[X'\times_YX'\simeq(Y'\times_YX)\times_Y(Y'\times_YX)\simeq(Y'\times_YY')\times_Y(X\times_YX).\]
	We can also see this more directly, but I think I personally draw the line at explicitly proving associativity laws.
\end{proof}
\begin{lemma} \label{lem:diagonal-affine-local}
	Fix a class $P$ of morphisms which is local on the target and preserved by base change. Then $\Delta P$ is local on the target.
\end{lemma}
\begin{proof}
	Fix a morphism $\varphi\colon X\to Y$. In one direction, suppose $\varphi\in\Delta P$ and fix some open subset $V\subseteq Y$. Then we note that \autoref{eq:open-embed-fp} tells us
	% https://q.uiver.app/?q=WzAsNCxbMCwxLCJYIl0sWzEsMSwiWSJdLFswLDAsIlxccGleey0xfVYiXSxbMSwwLCJWIl0sWzIsMCwiXFxwaXxfe1xccGleey0xfVZ9IiwyXSxbMywxLCJcXHBpIl0sWzIsMywiIiwwLHsic3R5bGUiOnsidGFpbCI6eyJuYW1lIjoiaG9vayIsInNpZGUiOiJ0b3AifX19XSxbMCwxLCIiLDIseyJzdHlsZSI6eyJ0YWlsIjp7Im5hbWUiOiJob29rIiwic2lkZSI6InRvcCJ9fX1dXQ==&macro_url=https%3A%2F%2Fraw.githubusercontent.com%2FdFoiler%2Fnotes%2Fmaster%2Fnir.tex
	\[\begin{tikzcd}
		{\varphi^{-1}V} & X \\
		X & Y
		\arrow["{\varphi|_{\varphi^{-1}V}}"', from=1-1, to=2-1]
		\arrow["\varphi", from=1-2, to=2-2]
		\arrow[hook, from=1-1, to=1-2]
		\arrow[hook, from=2-1, to=2-2]
	\end{tikzcd}\]
	is a pullback square, so $\varphi|_{\varphi^{-1}V}\in\Delta P$ by \autoref{lem:diagonal-base-change}.

	In the other direction, fix an open cover $\{Y_\alpha\}_{\alpha\in\lambda}$ of $Y$ and suppose that the restrictions $\varphi_\alpha\colon\pi^{-1}Y_\alpha\to Y_\alpha$ all live in $\Delta P$; set $X_\alpha\coloneqq\varphi^{-1}Y_\alpha$ for brevity. Then the pullback squares
	% https://q.uiver.app/?q=WzAsNCxbMCwxLCJZX1xcYWxwaGEiXSxbMSwxLCJZIl0sWzAsMCwiXFx2YXJwaGleey0xfVlfXFxhbHBoYSJdLFsxLDAsIlgiXSxbMiwwLCJcXHZhcnBoaV9cXGFscGhhIiwyXSxbMywxLCJcXHZhcnBoaSJdLFsyLDMsIlxcam1hdGhfXFxhbHBoYSIsMCx7InN0eWxlIjp7InRhaWwiOnsibmFtZSI6Imhvb2siLCJzaWRlIjoidG9wIn19fV0sWzAsMSwiXFxpb3RhX1xcYWxwaGEiLDAseyJzdHlsZSI6eyJ0YWlsIjp7Im5hbWUiOiJob29rIiwic2lkZSI6InRvcCJ9fX1dXQ==&macro_url=https%3A%2F%2Fraw.githubusercontent.com%2FdFoiler%2Fnotes%2Fmaster%2Fnir.tex
	\[\begin{tikzcd}
		{X_\alpha} & X \\
		{Y_\alpha} & Y
		\arrow["{\varphi_\alpha}"', from=1-1, to=2-1]
		\arrow["\varphi", from=1-2, to=2-2]
		\arrow["{\jmath_\alpha}", hook, from=1-1, to=1-2]
		\arrow["{\iota_\alpha}", hook, from=2-1, to=2-2]
	\end{tikzcd}\]
	coming from \autoref{eq:open-embed-fp} grant us the pullback squares
	% https://q.uiver.app/?q=WzAsNCxbMCwxLCJYX1xcYWxwaGFcXHRpbWVzX3tZX1xcYWxwaGF9WF9cXGFscGhhIl0sWzEsMSwiWVxcdGltZXNfWFkiXSxbMCwwLCJYX1xcYWxwaGEiXSxbMSwwLCJYIl0sWzIsMCwiXFxEZWx0YVxcdmFycGhpX1xcYWxwaGEiLDJdLFszLDEsIlxcRGVsdGFcXHZhcnBoaSJdLFsyLDMsIlxcam1hdGhfXFxhbHBoYSIsMCx7InN0eWxlIjp7InRhaWwiOnsibmFtZSI6Imhvb2siLCJzaWRlIjoidG9wIn19fV0sWzAsMV1d&macro_url=https%3A%2F%2Fraw.githubusercontent.com%2FdFoiler%2Fnotes%2Fmaster%2Fnir.tex
	\[\begin{tikzcd}
		{X_\alpha} & X \\
		{X_\alpha\times_{Y_\alpha}X_\alpha} & {Y\times_XY}
		\arrow["{\Delta\varphi_\alpha}"', from=1-1, to=2-1]
		\arrow["\Delta\varphi", from=1-2, to=2-2]
		\arrow["{\jmath_\alpha}", hook, from=1-1, to=1-2]
		\arrow[from=2-1, to=2-2]
	\end{tikzcd}\]
	which essentially finish the proof. However, $X_\alpha\times_{Y_\alpha}X_\alpha$ is really $\pi_1^{-1}X_\alpha\subseteq X\times_YX$ by \autoref{lem:fp-open-cover-base}, where $\pi_1,\pi_2\colon X\times_YX\to X$ are the canonical projections. Notably, we see that the open cover $Y_\alpha$ of $Y$ becomes an open cover $\pi_1^{-1}\varphi^{-1}Y_\alpha$ of $X\times_YX$, so the fact that all the individual restrictions\footnote{Namely, the $\Delta\varphi_\alpha$ is a restriction because the pullback square is ``actually'' being induced by \autoref{eq:open-embed-fp}.} $\Delta\varphi_\alpha\colon X_\alpha\to\pi_1^{-1}X_\alpha$ are in $P$ tells us that $\Delta\varphi$ is also in $P$ because $P$ is local on the target, so $\varphi\in\Delta P$.
\end{proof}

\subsection{Quasiseparatedness is Reasonable}
It turns out to be very convenient to think about a morphism $\varphi\colon X\to Y$ being quasiseparated in terms of the diagonal morphism $\Delta\colon X\to X\times_YX$ induced by $\id_X\colon X\to X$.
\begin{lemma} \label{lem:mainqslemma}
	Fix a morphism $\varphi\colon X\to Y$ of schemes.
	\begin{listalph}
		\item $\pi$ is quasiseparated.
		% \item All quasicompact open subsets $U\subseteq Y$ have $\pi^{-1}(U)\subseteq X$ quasiseparated.
		\item There is an affine open cover $\mc U$ of $Y$ such that each $\varphi^{-1}(U)$ is quasiseparated for each $U\in\mc U$.
		\item The diagonal morphism $\Delta\colon X\to X\times_YX$ is quasicompact.
	\end{listalph}
\end{lemma}
\begin{proof}
	As usual, (a) implies (b) by choosing any affine open cover $\mc U$ of $Y$, which implies that $\varphi^{-1}(U)$ is quasiseparated for each affine open $U\in\mc U$ because $\varphi$ is quasiseparated.

	The other implications are harder. Before going further, we set our variables. Let $\pi_1,\pi_2\colon X\times_XY\to X$ be the canonical inclusions so that $\pi_i\circ\Delta=\id_X$ for $i\in\{1,2\}$ by definition of $\Delta$.
	\begin{itemize}
		\item We show (b) implies (c). Give $Y$ the affine open cover $\{Y_\alpha\}_{\alpha\in\lambda}$ such that $\pi^{-1}Y_\alpha$ is quasiseparated for each $\alpha$. Further, give each $\varphi^{-1}(Y_\alpha)$ an affine open cover $\{U_{\alpha,\beta}\}_{\beta\in\lambda_\alpha}$ and label our diagram as
		% https://q.uiver.app/?q=WzAsOSxbMiwyLCJZX1xcYWxwaGEiXSxbMSwyLCJcXHZhcnBoaV57LTF9WV9cXGFscGhhIl0sWzIsMSwiXFx2YXJwaGleey0xfVlfXFxhbHBoYSJdLFsxLDEsIihcXHZhcnBoaVxccGlfMSleey0xfVlfXFxhbHBoYSJdLFswLDIsIlVfe1xcYWxwaGEsXFxiZXRhfSJdLFsyLDAsIlVfe1xcYWxwaGEsXFxiZXRhJ30iXSxbMCwxLCJcXHBpXzJeey0xfVVfe1xcYWxwaGEsXFxiZXRhfSJdLFsxLDAsIlxccGlfMV57LTF9VV97XFxhbHBoYSxcXGJldGEnfSJdLFswLDAsIlxccGlfMV57LTF9VV97XFxhbHBoYSxcXGJldGEnfVxcY2FwXFxwaV8yXnstMX1VX3tcXGFscGhhLFxcYmV0YX0iXSxbMSwwLCJcXHZhcnBoaSJdLFsyLDAsIlxcdmFycGhpIl0sWzMsMiwiXFxwaV8xIl0sWzMsMSwiXFxwaV8yIl0sWzYsMywiXFx3aWRldGlsZGVcXGptYXRoX3tcXGFscGhhLFxcYmV0YX0iLDAseyJzdHlsZSI6eyJ0YWlsIjp7Im5hbWUiOiJob29rIiwic2lkZSI6InRvcCJ9fX1dLFs0LDEsIlxcam1hdGhfe1xcYWxwaGEsXFxiZXRhfSIsMCx7InN0eWxlIjp7InRhaWwiOnsibmFtZSI6Imhvb2siLCJzaWRlIjoidG9wIn19fV0sWzYsNCwiXFxwaV8yIl0sWzcsMywiXFx3aWRldGlsZGVcXGptYXRoX3tcXGFscGhhLFxcYmV0YSd9IiwwLHsic3R5bGUiOnsidGFpbCI6eyJuYW1lIjoiaG9vayIsInNpZGUiOiJ0b3AifX19XSxbNSwyLCJcXGptYXRoX3tcXGFscGhhLFxcYmV0YSd9IiwwLHsic3R5bGUiOnsidGFpbCI6eyJuYW1lIjoiaG9vayIsInNpZGUiOiJ0b3AifX19XSxbNyw1LCJcXHBpXzEiXSxbOCw2LCJcXHdpZGV0aWxkZVxcam1hdGhfe1xcYWxwaGEsXFxiZXRhJ30iLDAseyJzdHlsZSI6eyJ0YWlsIjp7Im5hbWUiOiJob29rIiwic2lkZSI6InRvcCJ9fX1dLFs4LDcsIlxcd2lkZXRpbGRlXFxqbWF0aF97XFxhbHBoYSxcXGJldGF9IiwwLHsic3R5bGUiOnsidGFpbCI6eyJuYW1lIjoiaG9vayIsInNpZGUiOiJ0b3AifX19XV0=&macro_url=https%3A%2F%2Fraw.githubusercontent.com%2FdFoiler%2Fnotes%2Fmaster%2Fnir.tex
		\begin{equation}
			\begin{tikzcd}
				{\pi_1^{-1}U_{\alpha,\beta'}\cap\pi_2^{-1}U_{\alpha,\beta}} & {\pi_1^{-1}U_{\alpha,\beta'}} & {U_{\alpha,\beta'}} \\
				{\pi_2^{-1}U_{\alpha,\beta}} & {(\varphi\pi_1)^{-1}Y_\alpha} & {\varphi^{-1}Y_\alpha} \\
				{U_{\alpha,\beta}} & {\varphi^{-1}Y_\alpha} & {Y_\alpha}
				\arrow["\varphi", from=3-2, to=3-3]
				\arrow["\varphi", from=2-3, to=3-3]
				\arrow["{\pi_1}", from=2-2, to=2-3]
				\arrow["{\pi_2}", from=2-2, to=3-2]
				\arrow["{\widetilde\jmath_{\alpha,\beta}}", hook, from=2-1, to=2-2]
				\arrow["{\jmath_{\alpha,\beta}}", hook, from=3-1, to=3-2]
				\arrow["{\pi_2}", from=2-1, to=3-1]
				\arrow["{\widetilde\jmath_{\alpha,\beta'}}", hook, from=1-2, to=2-2]
				\arrow["{\jmath_{\alpha,\beta'}}", hook, from=1-3, to=2-3]
				\arrow["{\pi_1}", from=1-2, to=1-3]
				\arrow["{\widetilde\jmath_{\alpha,\beta'}}", hook, from=1-1, to=2-1]
				\arrow["{\widetilde\jmath_{\alpha,\beta}}", hook, from=1-1, to=1-2]
			\end{tikzcd} \label{eq:big-qs-square}
		\end{equation}
		where the key point is that the top-left corner indeed should be $\pi_1^{-1}U_{\alpha,\beta'}\cap\pi_2^{-1}U_{\alpha,\beta}$ as computed in \autoref{cor:intersect-open-subscheme}.

		Now, the bottom-right square of \autoref{eq:big-qs-square} is a pullback square by \autoref{lem:fp-open-cover-base}, and the remaining square are pullback squares by \autoref{lem:open-fp}, so repeated applications of \autoref{lem:smalls-to-big-pullback} tell us that the outer square is a pullback square. In particular, we conclude that $\pi_1^{-1}U_{\alpha,\beta'}\cap\pi_2^{-1}U_{\alpha,\beta}$ is affine using the big pullback square by \autoref{lem:affine-fp}.

		We now remember that we are trying to show that $\Delta$ is quasicompact. Well, being quasicompact is affine-local on the target by \autoref{lem:mainqclemma}, so it suffices to show that $\Delta|_{\Delta^{-1}V}$ is quasicompact as $V$ varies over the various $\pi_1^{-1}U_{\alpha,\beta'}\cap\pi_2^{-1}U_{\alpha,\beta}$. In fact, by \autoref{cor:qc-from-qc-morphisms}, it suffices to show that $\Delta^{-1}(V)$ itself is compact for our various $V$, for which we compute
		\begin{align*}
			\Delta^{-1}(V) &= \left\{x\in X:\Delta(x)\in\pi_1^{-1}U_{\alpha,\beta'}\cap\pi_2^{-1}U_{\alpha,\beta}\right\} \\
			&= \left\{x\in X:\pi_1\Delta(x)\in U_{\alpha,\beta'}\text{ and }\pi_2\Delta(x)\in U_{\alpha,\beta}\right\} \\
			&= U_{\alpha,\beta'}\cap U_{\alpha,\beta},
		\end{align*}
		which is compact because the $U_{\alpha,\beta}$ and $U_{\alpha,\beta'}$ are compact open subsets of the quasiseparated space $\varphi^{-1}Y_\alpha$: indeed, $\varphi^{-1}Y_\alpha$ is quasiseparated because $\varphi$ is quasiseparated, and $Y_\alpha$ is affine. This finishes.

		\item We show (c) implies (a); we use \autoref{rem:affine-qs-condition}. Fix an affine open subset $V\subseteq Y$ and two affine open subsets $U_1,U_2\subseteq\varphi^{-1}V$; we need to show that $U_1\cap U_2$ is quasicompact. Well, by \autoref{lem:magic-diagram}, we are promised a pullback square
		% https://q.uiver.app/?q=WzAsNCxbMCwwLCJWXzFcXHRpbWVzX1hWXzIiXSxbMSwwLCJWXzFcXHRpbWVzX1lYXzIiXSxbMCwxLCJYIl0sWzEsMSwiWFxcdGltZXNfWVgiXSxbMCwxXSxbMSwzXSxbMCwyXSxbMiwzLCJcXERlbHRhIl0sWzAsMywiIiwwLHsic3R5bGUiOnsibmFtZSI6ImNvcm5lciJ9fV1d&macro_url=https%3A%2F%2Fraw.githubusercontent.com%2FdFoiler%2Fnotes%2Fmaster%2Fnir.tex
		\[\begin{tikzcd}
			{V_1\times_XV_2} & {V_1\times_YV_2} \\
			X & {X\times_YX}
			\arrow[from=1-1, to=1-2]
			\arrow[from=1-2, to=2-2]
			\arrow[from=1-1, to=2-1]
			\arrow["\Delta", from=2-1, to=2-2]
			\arrow["\lrcorner"{anchor=center, pos=0.125}, draw=none, from=1-1, to=2-2]
		\end{tikzcd}\]
		where the bottom morphism is $\Delta$. To compute $V_1\times_XV_2$, we note that the maps $V_1\into X$ and $V_2\into X$ are open embeddings, so $V_1\times_XV_2\simeq V_1\cap V_2$ by \autoref{cor:intersect-open-subscheme}. With this in mind, we see that the top morphism above is quasicompact because quasicompactness is preserved by base change by \autoref{lem:qc-base-change}, so to show $V_1\cap V_2$ is quasicompact, it suffices to show that $V_1\times_YV_2$ is quasicompact.

		Well, noting that $V_1,V_2\subseteq\varphi^{-1}U$, we note that
		% https://q.uiver.app/?q=WzAsNCxbMSwxLCJVIl0sWzEsMCwiVl8xIl0sWzAsMSwiVl8yIl0sWzAsMCwiVl8xXFx0aW1lc19ZVl8yIl0sWzEsMCwiXFx2YXJwaGkiXSxbMiwwLCJcXHZhcnBoaSJdLFszLDFdLFszLDJdXQ==&macro_url=https%3A%2F%2Fraw.githubusercontent.com%2FdFoiler%2Fnotes%2Fmaster%2Fnir.tex
		\[\begin{tikzcd}
			{V_1\times_YV_2} & {V_1} \\
			{V_2} & U
			\arrow["\varphi", from=1-2, to=2-2]
			\arrow["\varphi", from=2-1, to=2-2]
			\arrow[from=1-1, to=1-2]
			\arrow[from=1-1, to=2-1]
		\end{tikzcd}\]
		is a pullback square by \autoref{lem:fp-open-cover-base}. Thus, $V_1\times_YV_2$ is affine by \autoref{lem:affine-fp} because now all of $V_1,V_2,U$ are affine, so $V_1\times_YV_2$ is quasicompact.
	\end{itemize}
	The above implications finish the proof.
\end{proof}
Here are some quick applications of the above lemma.
\begin{example} \label{ex:mono-is-qs}
	All monomorphisms $\varphi\colon X\to Y$ are quasiseparated: indeed, by \autoref{lem:monic-means-diag-iso}, the diagonal morphism $\Delta\varphi$ is an isomorphism, which is quasicompact by \autoref{ex:iso-is-qc}, which is quasiseparated by \autoref{lem:mainqslemma}. For example, open embeddings are quasiseparated by \autoref{cor:open-embed-monic}.
\end{example}
\begin{corollary} \label{cor:qs-by-morphism}
	Fix an affine scheme $Y$ and a scheme morphism $\pi\colon X\to Y$. Then $X$ is quasiseparated if and only if $\pi$ is quasiseparated.
\end{corollary}
\begin{proof}
	If $\pi$ is quasiseparated, then we note that the affine open subscheme $Y\subseteq Y$ forces $X=\pi^{-1}(Y)$ to be quasiseparated. Conversely, if $X$ is quasiseparated, then we note that the affine open cover $\{Y\}$ of $Y$ has $\pi^{-1}(Y)=X$ quasiseparated, so $\pi$ is quasiseparated by \autoref{lem:mainqslemma}.
\end{proof}
\begin{example}
	Using the fact that $\Spec\ZZ$ is final in the category of schemes (by \autoref{cor:spec-z-final}), we can use \autoref{cor:qs-by-morphism} to say that a scheme $X$ is quasiseparated if and only if the canonical morphism $X\to\Spec\ZZ$ is quasiseparated.
\end{example}
\begin{example}
	Because affine schemes are quasiseparated by \autoref{ex:affine-is-qs}, we conclude by \autoref{cor:qs-by-morphism} that any morphism of affine schemes is quasiseparated.
\end{example}
\begin{corollary}
	The class of quasiseparated morphisms is affine-local on the target.
\end{corollary}
\begin{proof}
	In one direction, if $\pi\colon X\to Y$ is quasiseparated, then any affine open cover $\mc U$ of $Y$ will make $\pi^{-1}U$ quasiseparated for any $U\in\mc U$ because $\pi$ is quasiseparated. Thus, $\pi$ is quasiseparated by \autoref{lem:mainqslemma}.

	In the other direction, fix a scheme morphism $\pi\colon X\to Y$ and an affine open cover $\{V_\alpha\}_{\alpha\in\lambda}$ such that $\pi|_{\pi^{-1}V_\alpha}\colon\pi^{-1}V_\alpha\to V_\alpha$ is quasiseparated for each $V_\alpha$. Fixing some $V_\alpha$, we see $\pi^{-1}V_\alpha$ is affine and makes $\pi^{-1}V_\alpha$ quasiseparated because $\pi|_{\pi^{-1}V_\alpha}$ is integral. Thus, $\pi$ is integral by \autoref{lem:mainqslemma}.
\end{proof}
% \begin{lemma} \label{lem:qs-by-diag}
% 	Fix a morphism $(\pi,\pi^\sharp)\colon(X,\OO_X)\to(Y,\OO_Y)$ of schemes. Then $(\pi,\pi^\sharp)$ is quasiseparated if and only if the diagonal map $\Delta\colon X\to X\times_YX$ (induced by using $\id_X\colon X\to X$) is quasicompact.
% \end{lemma}
% \begin{proof}
% 	To show $\Delta$ is quasicompact, one can just reduce to an affine open cover.
% 	In the other direction, suppose that the diagonal map $\Delta$ is quasicompact, and we show that $\pi$ is quasiseparated. Well, fix some affine open subset $U\subseteq Y$ with affine open subsets $V_1,V_2\subseteq\pi^{-1}(U)$, and we want to show that $V_1\cap V_2$ is quasicompact. For this, we draw the diagram
% 	% https://q.uiver.app/?q=WzAsNCxbMCwwLCJWXzFcXHRpbWVzX1hWXzIiXSxbMSwwLCJWXzFcXHRpbWVzX1lWXzIiXSxbMSwxLCJYXFx0aW1lc19ZWCJdLFswLDEsIlgiXSxbMywyXSxbMSwyXSxbMCwxXSxbMCwzXV0=&macro_url=https%3A%2F%2Fraw.githubusercontent.com%2FdFoiler%2Fnotes%2Fmaster%2Fnir.tex
% 	\[\begin{tikzcd}
% 		{V_1\times_XV_2} & {V_1\times_YV_2} \\
% 		X & {X\times_YX}
% 		\arrow[from=2-1, to=2-2]
% 		\arrow[from=1-2, to=2-2]
% 		\arrow[from=1-1, to=1-2]
% 		\arrow[from=1-1, to=2-1]
% 	\end{tikzcd}\]
% 	which commutes purely formally by properties of the fiber product. Notably, $V_1,V_2\subseteq\pi^{-1}(U)$ promises that $V_1\times_YV_2=V_1\times_UV_2$. However, the main point is that $V_1\cap V_2=V_1\times_XV_2$ and $\pi^{-1}(V_1\times_UV_2)=V_1\cap V_2$ again. Thus, we note that $V_1\times_YV_2$ being quasicompact tells us that $V_1\cap V_2$ is quasicompact by taking the pre-image using the compactness of $\Delta$.
% \end{proof}
\autoref{lem:mainqslemma} lets us turn questions about morphisms being quasiseparated into questions about them being quasicompact. This will more or less automatically prove that being quasiseparated is preserved under composition and base change. Let's see this.
\begin{lemma} \label{lem:qs-base-change}
	The class of quasiseparated morphisms is preserved by composition.
\end{lemma}
\begin{proof}
	Note that the class of quasicompact morphisms is preserved by composition by \autoref{cor:qc-is-comp-preserve} and preserved by base change by \autoref{lem:qc-base-change}. The result now follows from \autoref{lem:diagonal-composition} by viewing quasiseparated morphisms as the class of morphisms with quasicompact diagonal, by \autoref{lem:mainqslemma}.
\end{proof}
\begin{lemma}
	The class of quasiseparated morphisms is preserved by base change.
\end{lemma}
\begin{proof}
	Note that the class of quasicompact morphisms is preserved by base change by \autoref{lem:qc-base-change}. The result now follows from \autoref{lem:diagonal-base-change} by viewing quasiseparated morphisms as the class of morphisms with quasicompact diagonal, by \autoref{lem:mainqslemma}.
\end{proof}
Quasiseparated morphisms also turn out to satisfy a cancellation property.
\begin{lemma} \label{lem:qs-cancellation}
	Fix scheme morphisms $\varphi\colon X\to Y$ and $\psi\colon Y\to Z$. If the composite $\psi\circ\varphi$ is quasiseparated, then $\varphi$ is also quasiseparated.
\end{lemma}
\begin{proof}
	We use the fact that being quasiseparated is affine-local on the target, as showed in \autoref{lem:mainqslemma}. Give $Z$ some affine open cover $\{Z_\alpha\}_{\alpha\in\lambda}$, and then give each $\psi^{-1}Z_\alpha$ an affine open cover $\{Y_{\alpha,\beta}\}_{\beta\in\lambda_\alpha}$. It suffices to show that the restriction $\varphi|_{\varphi^{-1}Y_{\alpha,\beta}}\colon\varphi^{-1}Y_{\alpha,\beta}\to Y_{\alpha,\beta}$ is quasiseparated because being quasiseparated is affine-local on the target.

	Thus, fix some $\alpha\in\lambda$ and $\beta\in\lambda_\alpha$. Because $Y_{\alpha,\beta}$ is affine, we want to know that $\varphi^{-1}Y_{\alpha,\beta}$ is quasiseparated by \autoref{cor:qs-by-morphism}. However, $(\psi\circ\varphi)^{-1}Z_\alpha$ is quasiseparated because $\psi\circ\varphi$ is quasiseparated. In particular, the open subset $\varphi^{-1}Y_{\alpha,\beta}\subseteq(\psi\circ\varphi)^{-1}Z_\alpha$ is also quasiseparated by \autoref{ex:qs-open-subset}.
\end{proof}

\subsection{Affine Morphisms Are Reasonable}
Here is our definition.
\begin{definition}[Affine]
	A scheme morphism $\pi\colon X\to Y$ if and only if every affine open subset $U\subseteq Y$ has $\pi^{-1}(U)$ affine.
\end{definition}
\begin{example} \label{ex:affine-morphisms-are-affine}
	Let $\varphi\colon\Spec B\to\Spec A$ be a morphism of affine schemes; we show $\varphi$ is affine. Well, for any affine open subscheme $\Spec A'\cong U\subseteq\Spec A$, \autoref{lem:open-fp} tells us that
	% https://q.uiver.app/?q=WzAsNCxbMSwxLCJcXFNwZWMgQSJdLFsxLDAsIlxcU3BlYyBCIl0sWzAsMSwiVSJdLFswLDAsIlxcdmFycGhpXnstMX0oVSkiXSxbMSwwLCJcXHZhcnBoaSJdLFsyLDAsIiIsMix7InN0eWxlIjp7InRhaWwiOnsibmFtZSI6Imhvb2siLCJzaWRlIjoidG9wIn19fV0sWzMsMSwiIiwwLHsic3R5bGUiOnsidGFpbCI6eyJuYW1lIjoiaG9vayIsInNpZGUiOiJ0b3AifX19XSxbMywyLCJcXHZhcnBoaXxfe1xcdmFycGhpXnstMX0oVSl9Il1d&macro_url=https%3A%2F%2Fraw.githubusercontent.com%2FdFoiler%2Fnotes%2Fmaster%2Fnir.tex
	\[\begin{tikzcd}
		{\varphi^{-1}(U)} & {\Spec B} \\
		U & {\Spec A}
		\arrow["\varphi", from=1-2, to=2-2]
		\arrow[hook, from=2-1, to=2-2]
		\arrow[hook, from=1-1, to=1-2]
		\arrow["{\varphi|_{\varphi^{-1}(U)}}", from=1-1, to=2-1]
	\end{tikzcd}\]
	is a pullback square, so $\varphi^{-1}(U)\cong\Spec B\times_{\Spec A}U\cong\Spec B\times_{\Spec A}\Spec A'$, which is just $\Spec B\otimes_AA'$ by \autoref{lem:affine-fp}. In particular, $\varphi^{-1}(U)$ is affine.
\end{example}
\begin{remark} \label{rem:affine-by-morphism}
	In fact, a scheme morphism $\pi\colon X\to Y$ with $Y$ affine has $\pi$ is affine if and only if $X$ is affine. Indeed, in one direction, if $X$ is affine, then $\pi$ is affine by \autoref{ex:affine-morphisms-are-affine}. In the other direction, if $\pi$ is affine, then the affine open subset $Y\subseteq Y$ tells us that $X=\pi^{-1}Y$ is affine, which finishes.
\end{remark}
\begin{remark} \label{rem:affine-is-qc}
	As an example of using our hypotheses, we note that affine morphisms $\pi\colon X\to Y$ are quasicompact and quasiseparated. Indeed, for any affine open $U\subseteq Y$, we see $\pi^{-1}U$ is affine and therefore quasicompact and quasiseparated (quasiseparatedness by \autoref{ex:affine-is-qs}).
\end{remark}
\begin{remark} \label{rem:affine-from-local-on-target}
	More generally, suppose that $P$ is a class of morphisms which is affine-local on the target and includes morphisms of affine schemes. Then, for any affine morphism $\pi\colon X\to Y$, we see $\pi$ is in $P$: give $Y$ any affine open cover $\{Y_\alpha\}_{\alpha\in\lambda}$, and then note that the restriction $\pi_\alpha\colon\pi^{-1}Y_\alpha\to Y_\alpha$ is in $P$ because this is a morphism of affine schemes. So because $P$ is affine-local on the target, we conclude $\pi$ is in $P$.
\end{remark}
Here are the usual sanity checks.
\begin{lemma}
	Affine morphisms are preserved by composition.
\end{lemma}
\begin{proof}
	Let $\varphi\colon X\to Y$ and $\psi\colon Y\to Z$ be affine morphisms, and we want to show $\psi\circ\varphi$ is affine. Well, if $W\subseteq Z$ is an affine open subset, then $\psi^{-1}W\subseteq Y$ is affine, so $(\psi\circ\varphi)^{-1}(W)=\varphi^{-1}\psi^{-1}W\subseteq X$ is also affine. This finishes.
\end{proof}
It turns out that showing being affine is affine-local on the target is quite tricky. We will slowly build our way up.
\begin{lemma} \label{lem:restrict-affine-morphism}
	Fix an affine morphism $\pi\colon X\to Y$. Then for any open subscheme $V\subseteq Y$, the restriction $\pi|_{\pi^{-1}V}\colon\pi^{-1}V\to V$ is also affine.
\end{lemma}
\begin{proof}
	We chase the definitions. Fix an affine open subscheme $V'\subseteq V$. Then $V'\subseteq Y$ is also an affine open subscheme, so $\pi^{-1}V'\subseteq X$ is an affine open subscheme, so $\pi|_{\pi^{-1}V}^{-1}(V')\subseteq\pi^{-1}V'$ is an affine open subscheme.
\end{proof}
What is hard about being affine-local on the target is the gluing part of \autoref{lem:affine-comm}. We throw the relevant results into some lemmas.
\begin{lemma} \label{lem:qcqs-makes-local-res-isos}
	Suppose that the scheme $X$ which has a finite an affine open cover $\{U_i\}_{i=1}^n$ such that each intersection $U_i\cap U_j$ is quasicompact. Then, for any $f\in\OO_X(X)$, the natural map $\OO_X(X)_f\to\OO_X(X_f)$ (induced by restriction and \autoref{lem:finvertsinxf}) is an isomorphism.
\end{lemma}
It turns out that the hypothesis on $X$ is equivalent to requiring $X$ to be quasicompact and quasiseparated, but we won't bother showing this.
\begin{proof}
	We already have our maps of rings as a localization of a restriction map, where the localization is legal because $f\in\OO_X(X_f)^\times$ by \autoref{lem:finvertsinxf}. Thus, it suffices to show that our map $\iota_f\colon\OO_X(X)_f\to\OO_X(X_f)$ given by $s/f^n\mapsto(s|_{X_f})(f|_{X_f})^{-n}$ is bijective. We have two checks; set $A\coloneqq\OO_X(X)$ for brevity.
	\begin{itemize}
		\item Injective: we only need quasicompactness. For now, fix an affine open subscheme $\varphi\colon U\cong\Spec\OO_X(U)$ setting $B\coloneqq\OO_X(U)$ (using the canonical isomorphism) of $X$. We are given $a/f^n\in A_f$ such that $a|_{X_f}\cdot(f|_{X_f})^{-n}=0$, so actually $a|_{X_f}=0$, so set $a_U\coloneqq a|_U$ so that $a_U|_{U\cap X_f}=a|_{X_f}|_{X_f\cap U}=0$ as well.
			
		Now, we pass through $\varphi^\sharp\colon(\OO_X|_U)\to\varphi_*(\OO_{\Spec B})$. Because $U\cap X_f=\varphi(D(\varphi^\sharp_U(f|_U)))$, as shown above, we see that the full diagram
		% https://q.uiver.app/?q=WzAsOCxbMSwwLCJcXE9PX3tcXFNwZWMgQn0oXFx2YXJwaGleey0xfShVKSkiXSxbMSwxLCJcXE9PX3tcXFNwZWMgQn0oXFx2YXJwaGleey0xfShVXFxjYXAgWF9mKSkiXSxbMiwwLCJcXE9PX3tcXFNwZWMgQn0oXFxTcGVjIEIpIl0sWzIsMSwiXFxPT197XFxTcGVjIEJ9KEQoXFx2YXJwaGleXFxzaGFycF9VKGZ8X1UpKSkiXSxbMywwLCJCIl0sWzMsMSwiQl97XFx2YXJwaGleXFxzaGFycF9VKGZ8X1UpfSJdLFswLDAsIlxcT09fWHxfVShVKSJdLFswLDEsIlxcT09fWHxfVShVXFxjYXAgWF9mKSJdLFswLDIsIiIsMix7ImxldmVsIjoyLCJzdHlsZSI6eyJoZWFkIjp7Im5hbWUiOiJub25lIn19fV0sWzIsNCwiIiwyLHsibGV2ZWwiOjIsInN0eWxlIjp7ImhlYWQiOnsibmFtZSI6Im5vbmUifX19XSxbMSwzLCIiLDIseyJsZXZlbCI6Miwic3R5bGUiOnsiaGVhZCI6eyJuYW1lIjoibm9uZSJ9fX1dLFszLDUsIiIsMix7ImxldmVsIjoyLCJzdHlsZSI6eyJoZWFkIjp7Im5hbWUiOiJub25lIn19fV0sWzQsNV0sWzAsMSwiXFxvcHtyZXN9X3tcXHZhcnBoaV57LTF9KFUpLFxcdmFycGhpXnstMX0oVVxcY2FwIFhfZil9Il0sWzcsMSwiXFx2YXJwaGleXFxzaGFycF97VVxcY2FwIFhfZn0iLDJdLFs2LDcsIlxcb3B7cmVzfV97VSxVXFxjYXAgWF9mfSJdLFs2LDAsIlxcdmFycGhpXlxcc2hhcnBfVSJdXQ==&macro_url=https%3A%2F%2Fraw.githubusercontent.com%2FdFoiler%2Fnotes%2Fmaster%2Fnir.tex
		\begin{equation}
			\begin{tikzcd}
				{\OO_X|_U(U)} & {\OO_{\Spec B}(\varphi^{-1}(U))} & {\OO_{\Spec B}(\Spec B)} & B \\
				{\OO_X|_U(U\cap X_f)} & {\OO_{\Spec B}(\varphi^{-1}(U\cap X_f))} & {\OO_{\Spec B}(D(\varphi^\sharp_U(f|_U)))} & {B_{\varphi^\sharp_U(f|_U)}}
				\arrow[Rightarrow, no head, from=1-2, to=1-3]
				\arrow[Rightarrow, no head, from=1-3, to=1-4]
				\arrow[Rightarrow, no head, from=2-2, to=2-3]
				\arrow[Rightarrow, no head, from=2-3, to=2-4]
				\arrow[from=1-4, to=2-4]
				\arrow["{\op{res}_{\varphi^{-1}(U),\varphi^{-1}(U\cap X_f)}}", from=1-2, to=2-2]
				\arrow["{\varphi^\sharp_{U\cap X_f}}"', from=2-1, to=2-2]
				\arrow["{\op{res}_{U,U\cap X_f}}", from=1-1, to=2-1]
				\arrow["{\varphi^\sharp_U}", from=1-1, to=1-2]
			\end{tikzcd} \label{eq:bigcd}
		\end{equation}
		commutes. In particular, passing $a_U$ with $a_U|_{U\cap X_f}=0$ through the diagram, we see that $\varphi^\sharp_U(a_U)$ vanishes in $B_{\varphi^\sharp_U(f|_U)}$. Thus, there exists a positive integer $n$ such that
		\[0=\varphi^\sharp_U(f|_U)^n\cdot\varphi^\sharp_U(a_U)=\varphi^\sharp_U\left((f|_U)^na_U\right)=\varphi^\sharp_U\left((f^na)|_U\right).\]
		Because $\varphi^\sharp$ is an isomorphism of sheaves (because it is part of an isomorphism of schemes), we see that $\varphi^\sharp_U$ is an isomorphism and is therefore injective, so $(f^na)|_U=0$.
		
		We are now ready to talk about all of $X$. Because $X$ has an affine open cover, the quasicompactness of $X$ promises a finite affine open cover $\{U_i\}_{i=1}^m$, and the argument above promises positive integers $n_i$ such that
		\[\left(f^{n_i}a\right)|_{U_i}=0\]
		for each $i$. Thus, we set $n\coloneqq\max\{n_i:1\le i\le m\}$ so that
		\[\left(f^na\right)|_{U_i}=0\]
		for all $i$. However, $\{U_i\}_{i=1}^m$ forms a cover of $X$, so the identity axiom $\OO_X$ forces $f^na=0$. This finishes.

		\item Surjectivity: we now use the fact that the $U_i\cap U_j$ is quasicompact. We proceed in steps. Fix $b\in\OO_X(X_f)$.
		\begin{enumerate}
			\item For now, fix an affine open subscheme $(U,\OO_X|_U)\cong(\Spec B,\OO_{\Spec B})$ of $(X,\OO_X)$. For concreteness, we set $(\varphi,\varphi^\sharp)\colon(\Spec B,\OO_{\Spec B})\cong(U,\OO_X|_U)$ to be the isomorphism. Set $b_U\coloneqq b|_{U\cap X_f}$.
	
			Now, $\varphi^\sharp\colon\OO_X|_U\to\varphi_*\OO_{\Spec B}$ is an isomorphism, so we build the commutative diagram \autoref{eq:bigcd} again. In particular, we can write $\varphi^\sharp_{U\cap X_f}(b_U)\in B_{\varphi^\sharp(f|_U)}$ as
			\[\varphi^\sharp_{U\cap X_f}(b_U)=\frac{\varphi^\sharp_U(a)}{\varphi^\sharp_U(f|_U)^n}\]
			for some $\varphi^\sharp_U(a)\in B$ (where $a\in\OO_X(U)$) and positive integer $n$. In particular, it follows that
			\[\varphi^\sharp_{U\cap X_f}(a|_{U\cap X_f})=\varphi^\sharp_{U\cap X_f}(f^n|_{U\cap X_f}\cdot b|_{U\cap X_f}),\]
			so because $\varphi^\sharp$ is an isomorphism, we have $a|_{U\cap X_f}=f^n|_{U\cap X_f}\cdot b|_{U\cap X_f}$. We remark that multiplying $a$ by multiples of $f$ will not change this property.

			\item Returning to $X$, we give $X$ the promised affine open cover $\{U_i\}_{i=1}^m$. The argument above generates sections $a_i\in\OO_X(U_i)$ and positive integers $n_i$ such that
			\[a_i|_{U_i\cap X_f}=f^{n_i}|_{U_i\cap X_f}\cdot b|_{U_i\cap X_f}\]
			for each $i$. Letting $n$ be the maximum of the finitely many $n_i$, we can replace $a_i$ with $a_i\cdot f^{n-n_i}|_{U_i}$ (recall multiplying by $f$s doesn't do anything) and multiply the above equation by $f^{n-n_i}$ so that actually
			\begin{equation}
				a_i|_{U_i\cap X_f}=f^n|_{U_i\cap X_f}\cdot b|_{U_i\cap X_f} \label{eq:defineais}
			\end{equation}
			for each $i$. We would like to glue these $a_i$, but the argument is a little technical.

			\item Fix two indices $i$ and $j$ and affine open subset $V\subseteq U_i\cap U_j$ so that we have a ring $B$ equipped with an isomorphism $(\varphi,\varphi^\sharp)\colon\Spec B\cong V$. Now, \autoref{eq:defineais} restricted to $V$ tells us that
			\[a_i|_{V\cap X_f}=f^n|_{V\cap X_f}\cdot b|_{V\cap X_f}=a_j|_{V\cap X_f}.\]
			As such, passing $a_i|_V$ and $a_j|_V$ through the commutative diagram \autoref{eq:bigcd} (where $U$s are replaced with $V$s), we see that
			\[\varphi^\sharp_V(a_i|_V)|_{V\cap X_f}=\varphi^\sharp_V(a_j|_V)|_{V\cap X_f}\]
			as elements in $B_{\varphi^\sharp_V(f|_V)}$, so undoing the localization, there is a positive integer $d$ such that
			\[\varphi^\sharp_V(f|_V)^d\cdot\varphi^\sharp_V(a_i|_V)=\varphi^\sharp_V(f|_V)^d\cdot\varphi^\sharp_V(a_j|_V).\]
			Lastly, undoing $\varphi^\sharp$, we get {$(f|_V)^d\cdot a_i|_V=(f|_V)^d\cdot a_j|_V$}.

			\item We now return to $X$. For each pair of indices $i$ and $j$, we are given that $U_i\cap U_j$ is quasicompact, so provide $U_i\cap U_j$ with a finite affine open cover $\{V_{ijk}\}_{k=1}^{m_{ij}}$. Notably, the above point grants a positive integer $d_{ijk}$ such that
			\[\left(f|_{U_i}^{d_{ijk}}\cdot a_i\right)|_{V_{ijk}}=\left(f|_{U_j}^{d_{ijk}}\cdot a_j\right)|_{V_{ijk}}\]
			for any triple of indices $i$ and $j$ and $k$. Now, let $d$ be the maximum over all the $d_{ijk}$, and {replace each $a_i$ with $(f|_{U_i})^d\cdot a_i$ and $n$ with $n+d$} (again multiplying by $f$ doesn't do anything) so that we see
			\[a_i|_{V_{ijk}}=a_j|_{V_{ijk}}\]
			for any triple of indices $i$ and $j$ and $k$. In particular, letting $k$ vary and applying the identity axiom gives $a_i|_{U_i\cap U_j}=a_j|_{U_i\cap U_j}$, so the gluability axiom gives an $a\in A$ such that $a|_{U_i}=a$.

			In particular, for each $U_i$, we see
			\[a|_{X_f}|_{U_i\cap X_f}=a_i|_{U_i\cap X_f}=\left(f^n|_{X_f}\cdot b\right)|_{U_i\cap X_f},\]
			so the identity axiom assures us that $a|_{X_f}=f^n|_{X_f}\cdot b$. This finishes.
			\qedhere
		\end{enumerate}
	\end{itemize}
\end{proof}
\begin{lemma} \label{lem:affine-local-gluing}
	Fix a scheme $X$. If there exist global sections $f_1,\ldots,f_n\in\OO_X(X)$ such that $X_{f_i}$ is affine for each $i$, then $X$ is affine.
\end{lemma}
\begin{proof}
	We combine the previous two lemmas. Set $A\coloneqq\OO_X(X)$. We are given elements $\{f_i\}_{i=1}^r\subseteq A$ such that $(f_1,\ldots,f_r)=A$, and we know that the $X_{f_i}$ are all affine open subschemes of $X$. The main approach is to show that {the morphism
	\[\eta_X\colon X\to\Spec A\]
	conjured in \autoref{lem:specgamma} is an isomorphism}; for brevity, set $\varphi\coloneqq\eta_X$. To stay organized, we will proceed in steps.
	\begin{enumerate}
		\item We check some open covers. Note that the $D(f_i)$ cover $\Spec A$ by \autoref{rem:topological-affine-comm-ii}.
		
		Additionally, we claim that the $X_{f_i}$ form an affine open cover $X$. Indeed, as computed in \autoref{lem:specgamma}, we have $\varphi^{-1}(D(f))=X_f$ for any $f\in A$, so the fact that the $\{D(f_i)\}$ cover $\Spec A$ forces the $X_{f_i}$ to cover $X$.
		\item We claim that $X_{f_i}\cap X_{f_j}$ is quasicompact for each $i$ and $j$. Indeed, given $f_i$ and $f_j$ and some point $x\in X$, we note that $f_i\notin\mf m_{X,x}$ and $f_j\notin\mf m_{X,x}$ if and only if $f_i\notin\mf m_{X,x}$ so that $x\in X_{f_i}$ and $f_j|_{X_{f_i}}\notin\mf m_{X,x}$. Thus, $X_{f_i}\cap X_{f_j}=(X_{f_i})_{f_j}$ is a distinguished open subscheme of $X_{f_i}$ and is therefore quasicompact.
		
		Explicitly, under the canonical isomorphism $X_{f_i}\cong\Spec\OO_X(X_{f_i})$ given by \autoref{cor:the-affine-iso}, we see that the pre-image of $D(f_j)$ is $(X_{f_i})_{f_j}$ as computed in \autoref{lem:specgamma}.
		\item We now trigger \autoref{lem:qcqs-makes-local-res-isos} using the affine open cover $\{X_{f_i}\}_{i=1}^n$ so that $A_{f_i}\simeq\OO_X(X_{f_i})$ by localizing the restriction map.

		In particular, \autoref{cor:the-affine-iso} lets us set $\varphi_i\colon X_{f_i}\to\Spec\OO_X(X_{f_i})$ to be the canonical isomorphism. Thus, we see that we have a chain of isomorphisms
		\begin{equation}
			X_{f_i}\cong\Spec\OO_X(X_{f_i})\cong\Spec A_{f_i}\cong D(f_i)\subseteq\Spec A. \label{eq:the-affine-local-chain}
		\end{equation}
		We take a moment to compute this on global sections, which will give a map from $\OO_{\Spec A}(D(f_i))\simeq A_{f_i}$ to $\OO_X(X_{f_i})$. Well, this is
		\[\arraycolsep=1.4pt\begin{array}{ccccccccc}
			\OO_{\Spec A}(D(f_i)) &\cong& \OO_{\Spec A_{f_i}}(\Spec A_{f_i}) &\cong& \OO_{\Spec\OO_X(X_{f_i})}(\Spec\OO_X(X_{f_i})) &\cong& \OO_X(X_{f_i}) \\
			a/f_i^n &\mapsto& a/f_i^n &\mapsto& a|_{X_{f_i}}\cdot(f_i|_{X_{f_i}})^{-n} &\mapsto& a|_{X_{f_i}}\cdot(f_i|_{X_{f_i}})^{-n}
		\end{array}\]
		by tracking everything through.
		
		\item Because being an isomorphism is local on the target by \autoref{lem:iso-is-local-target}, we will be done if we show $\varphi|_{X_{f_i}}$ restricted to is the composite of \autoref{eq:the-affine-local-chain}.

		However, using the adjunction of \autoref{thm:biggeoisalgopp}, it suffices to show that the map of \autoref{eq:the-affine-local-chain} agrees with $\varphi|_{X_{f_i}}$ on global sections. Well, $\varphi^\sharp_{D(f_i)}\colon\OO_{\Spec A}(D(f_i))\to\OO_X(X_{f_i})$ by construction in \autoref{lem:specgamma} sends a generic element $a/f_i^n$ to $a|_{X_{f_i}}\cdot(f_i|_{X_{f_i}})^{-n}$, as desired.
		\qedhere
	\end{enumerate}
\end{proof}
\begin{lemma} \label{lem:almost-affine-is-affine-local}
	A morphism $\pi\colon X\to Y$ is affine if and only if we can provide $Y$ with an affine open cover $\mc U$ such that $\pi^{-1}(U)$ is affine for each $U\in\mc U$.
\end{lemma}
\begin{proof}
	Certainly if $\pi$ is affine, then any affine open cover $\mc U$ of $Y$ will have $\pi^{-1}U\subseteq X$ affine for any $U\in\mc U$.

	For the other direction, we use \autoref{lem:affine-comm}. Call an affine open subset $U\subseteq Y$ ``acceptable'' if and only if $\pi^{-1}U\subseteq X$ is also affine. We now run our checks.
	\begin{listroman}
		\item Suppose that $U$ is acceptable and $f\in\OO_X(U)$. Then we want $U_f$ to also be acceptable. Well, we recall from \autoref{rem:pre-image-xfs} that
		\[\pi^{-1}(U_f)=(\pi^{-1}U)_{\pi^\sharp_Uf},\]
		which we now see is a distinguished open subscheme of $U$ and therefore an affine scheme.
		\item Suppose that we have an open subscheme $U\subseteq X$ and sections $f_1,\ldots,f_n\in\OO_X(U)$ with $(f_1,\ldots,f_n)=\OO_X(U)$ such that $U_{f_i}$ is acceptable for each $i$. We would like to show that $U$ is acceptable, for which we need to show $\pi^{-1}U$ is affine.

		Well, we recall from \autoref{rem:pre-image-xfs} that
		\[\pi^{-1}(U_{f_i})=(\pi^{-1}U)_{\pi^\sharp_Uf_i}\]
		for each $i$, so all the $(\pi^{-1}U)_{\pi^\sharp_Uf_i}$ are affine. Furthermore, we note $(f_1,\ldots,f_n)=(1)$ implies some $\OO_Y(U)$-linear combination of the $f_i$ equals $1$, so some $\OO_X(U)$-linear combination of the $\pi^\sharp_Uf_i$ equals $1$. Thus, \autoref{lem:affine-local-gluing} kicks in and tells us that $\pi^{-1}U$ is affine.
		\item Lastly, we note that $Y$ has a cover of acceptable affine open subschemes by hypothesis on $Y$.
	\end{listroman}
	The above checks complete the proof by \autoref{lem:affine-comm}.
\end{proof}
\begin{corollary} \label{cor:affine-is-affine-local}
	The class of affine morphisms is affine-local on the target.
\end{corollary}
\begin{proof}
	One direction of being affine-local on the target is dealt with by \autoref{lem:restrict-affine-morphism}.

	In the other direction, fix a scheme morphism $\pi\colon X\to Y$ and an affine open cover $\{V_\alpha\}_{\alpha\in\lambda}$ such that $\pi|_{\pi^{-1}V_\alpha}\colon\pi^{-1}V_\alpha\to V_\alpha$ is affine for each $V_\alpha$. Fixing some $V_\alpha$, we see $\pi^{-1}V_\alpha$ is affine and makes $\pi^{-1}V_\alpha$ affine because $\pi|_{\pi^{-1}V_\alpha}$ is affine. Thus, $\pi$ is affine by \autoref{lem:almost-affine-is-affine-local}.
\end{proof}
And lastly, here is base change.
\begin{lemma}
	The class of affine morphisms is preserved by base change.
\end{lemma}
\begin{proof}
	Suppose we have a pullback square
	% https://q.uiver.app/?q=WzAsNCxbMCwwLCJYXFx0aW1lc19TWSJdLFsxLDAsIlgiXSxbMCwxLCJZIl0sWzEsMSwiUyJdLFsyLDMsIlxccHNpX1kiXSxbMCwxLCJcXHBpX1giXSxbMCwyLCJcXHBpX1kiXSxbMSwzLCJcXHBzaV9YIl0sWzAsMywiIiwyLHsic3R5bGUiOnsibmFtZSI6ImNvcm5lciJ9fV1d&macro_url=https%3A%2F%2Fraw.githubusercontent.com%2FdFoiler%2Fnotes%2Fmaster%2Fnir.tex
	\[\begin{tikzcd}
		{X\times_SY} & X \\
		Y & S
		\arrow["{\psi_Y}", from=2-1, to=2-2]
		\arrow["{\pi_X}", from=1-1, to=1-2]
		\arrow["{\pi_Y}", from=1-1, to=2-1]
		\arrow["{\psi_X}", from=1-2, to=2-2]
		\arrow["\lrcorner"{anchor=center, pos=0.125}, draw=none, from=1-1, to=2-2]
	\end{tikzcd}\]
	of schemes such that $\psi_Y$ is affine. We would like to show that $\pi_X$ is affine. Notably, because being affine is affine-local on the target by \autoref{cor:integral-affine-local-target}, we may use \autoref{lem:base-change-reduce-to-affine} to assume that $X$ and $S$ are affine.

	However, $\psi_Y$ is affine, so we conclude that $Y=\psi^{-1}Y$ is affine because $S$ is. So because all of $X$ and $Y$ and $S$ are affine, we set $A\coloneqq\OO_X(X)$ and $B\coloneqq\OO_Y(Y)$ and $R\coloneqq\OO_S(S)$ so that \autoref{lem:affine-fp} tells us
	\[X\times_SY\simeq\Spec A\otimes_RB,\]
	so $X\times_SY$ is in fact affine. It follows that $\pi_X$ is affine by \autoref{rem:affine-by-morphism} because $X$ and $X\times_SY=\pi^{-1}X$ are both affine.
\end{proof}
Next time we move on to talk about finite morphisms, integral morphisms, and morphisms of finite type.

\end{document}