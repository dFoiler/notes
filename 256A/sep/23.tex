% !TEX root = ../notes.tex

\documentclass[../notes.tex]{subfiles}

\begin{document}

\section{September 23}

Today we place some finiteness conditions on morphisms.
% V 8.1--8.3
% 8.3 has definitions

\subsection{Quasicompact and Quasiseparated}
For today, we will denote a morphism of schemes $\pi\colon X\to Y$.
\begin{definition}[Quasicompact]
	A scheme morphism $(\pi,\pi^\sharp)\colon(X,\OO_X)\to(Y,\OO_Y)$ is \textit{quasicompact} if and only if all affine open subschemes $(U,\OO_Y|_U)\subseteq(Y,\OO_Y)$ make $\pi^{-1}(U)\subseteq X$ a quasicompact topological space.
\end{definition}
To define quasiseparated, we will need to have the adjective on topological spaces.
\begin{definition}[Quasiseperated]
	A topological space $X$ is \textit{quasiseparated} if and only if the intersection of two quasicompact open subsets is still quasicompact.
\end{definition}
And here is our definition.
\begin{defi}[Quasiseparated]
	A scheme morphism $(\pi,\pi^\sharp)\colon(X,\OO_X)\to(Y,\OO_Y)$ is \textit{quasiseparated} if and only if all affine open subschemes $(U,\OO_Y|_U)\subseteq(Y,\OO_Y)$ makes $\pi^{-1}(U)$ a quasiseparated topological space. In particular, $\pi^{-1}(U)$ has a finite affine open cover.
\end{defi}
\begin{remark}
	Equivalently, $(\pi,\pi^\sharp)$ is quasiseparated if and only if every affine open subsets $V_1,V_2\subseteq\pi^{-1}(U)$ for an affine open subset $U\subseteq X$ has $V_1\cap V_2\subseteq X$ with a finite affine open cover.
\end{remark}
It turns out that a scheme $(X,\OO_X)$ is quasicompact/quasiseparated if and only if its morphism $(X,\OO_X)\to(\Spec\ZZ,\OO_{\Spec\ZZ})$ is quasicompact/quasiseparated.
\begin{remark}
	A scheme being quasiseparated is basically a very reasonable smallness condition, say weaker than being locally Noetherian. For example, this is similar to being Hausdorff; this will be clearer later when we talk about separated morphisms.
\end{remark}

\subsection{Quasicompact Speedrun}
Here are some equivalent definitions for being quasicompact.
\begin{lemma} \label{lem:mainqclemma}
	Fix a morphism $(\pi,\pi^\sharp)\colon(X,\OO_X)\to(Y,\OO_Y)$ of schemes.
	\begin{listalph}
		\item $\pi$ is quasicompact.
		\item All quasicompact open subsets $U\subseteq Y$ have $\pi^{-1}(U)\subseteq X$ quasicompact.
		\item There is an affine open cover $\mc U$ of $Y$ such that each $\pi^{-1}(U)$ is quasicompact for each $U\in\mc U$.
	\end{listalph}
\end{lemma}
\begin{proof}
	The equivalence of (a) and (b) we will omit. The proof that (b) implies (c) is clear, by taking any affine open cover.

	The proof that (c) implies (b) is harder. Fix our open cover as $\{U_\alpha\}_{\alpha\in\lambda}$. Then fix some open $U\subseteq Y$ so that $U$ is covered by the $U\cap U_\alpha$. Now, $\pi^{-1}(U_\alpha)$ is quasicompact and therefore has a finite affine open cover. Then, for any distinguished open subset $D(f_i)\subseteq U_i$, we get to write
	\[\pi^{-1}(D(f_i))\]
	also as a finite union of affine open subsets (namely, of distinguished open basic sets). As such, we can give $U\cap U_\alpha$ a cover by these distinguished open basic elements $D(f_i)$.
\end{proof}
So here are some quick results.
\begin{cor}
	Fix a morphism $(\pi,\pi^\sharp)\colon(X,\OO_X)\to(Y,\OO_Y)$. If $(Y,\OO_Y)$ is affine, then $(\pi,\pi^\sharp)$ is quasicompact if and only if $X$ is quasicompact.
\end{cor}
\begin{proof}
	Apply \autoref{lem:mainqclemma}.
\end{proof}
\begin{corollary}
	Fix a morphism $(\pi,\pi^\sharp)\colon(X,\OO_X)\to(Y,\OO_Y)$. If $(X,\OO_X)$ is Noetherian, then $(\pi,\pi^\sharp)$ is quasicompact.
\end{corollary}
\begin{proof}
	Open subsets of Noetherian spaces are automatically compact.
\end{proof}
\begin{corollary}
	Being quasicompact is preserved under composition.
\end{corollary}
\begin{proof}
	Pull back a quasicompact subset twice.
\end{proof}
\begin{corollary}
	A morphism being quasicompact is preserved by base change. Namely, given the square
	% https://q.uiver.app/?q=WzAsNCxbMCwwLCJYXFx0aW1lc19ZWSciXSxbMSwwLCJZJyJdLFswLDEsIlgiXSxbMSwxLCJZIl0sWzIsMywiXFxwaSJdLFswLDEsIlxccGknIl0sWzAsMl0sWzEsM11d&macro_url=https%3A%2F%2Fraw.githubusercontent.com%2FdFoiler%2Fnotes%2Fmaster%2Fnir.tex
	\[\begin{tikzcd}
		{X\times_YY'} & {Y'} \\
		X & Y
		\arrow["\pi", from=2-1, to=2-2]
		\arrow["{\pi'}", from=1-1, to=1-2]
		\arrow[from=1-1, to=2-1]
		\arrow[from=1-2, to=2-2]
	\end{tikzcd}\]
	the morphism $\pi$ is quasicompact implies that $\pi'$ is quasicompact.
\end{corollary}
\begin{proof}
	Reduce to the affine case.
\end{proof}
\begin{corollary}
	A morphism being quasicompact is affine local on the target: given an affine open cover $\mc U$ of $Y$, the restrictions $\pi|_{\pi^{-1}U}$ for $U\in\mc U$ are all quasicompact if and only if $\pi$ is quasicompact.
\end{corollary}
\begin{proof}
	Apply \autoref{lem:mainqclemma}.
\end{proof}
\begin{remark}
	Being preserved under composition and base-change shows that the product of two quasicompact morphisms is quasicompact. Namely, $\varphi\colon X\to Y$ and $\varphi'\colon X'\to Y'$ makes us build the product morphism $\varphi\times\varphi'\colon X\times_{\Spec\ZZ}X'\to Y\times_{\Spec\ZZ}Y'$ by applying base-change twice.
\end{remark}
Here are some examples of our morphisms.
\begin{example}
	Closed embeddings are always quasicompact.
\end{example}
\begin{nex}
	An open embedding need not be quasicompact. For example, an affine scheme can have open subschemes which are not quasicompact.
\end{nex}

\subsection{Quasiseparated Speedrun}
We begin with the same central lemma from being quasicompact.
\begin{lemma} \label{lem:mainqslemma}
	Fix a morphism $(\pi,\pi^\sharp)\colon(X,\OO_X)\to(Y,\OO_Y)$ of schemes.
	\begin{listalph}
		\item $\pi$ is quasiseparated.
		\item All quasicompact open subsets $U\subseteq Y$ have $\pi^{-1}(U)\subseteq X$ quasiseparated.
		\item There is an affine open cover $\mc U$ of $Y$ such that each $\pi^{-1}(U)$ is quasiseparated for each $U\in\mc U$.
	\end{listalph}
\end{lemma}
\begin{proof}
	Similar to \autoref{lem:mainqclemma}.
\end{proof}
However, there is a more important test to be quasiseparated.
\begin{lemma} \label{lem:qs-by-diag}
	Fix a morphism $(\pi,\pi^\sharp)\colon(X,\OO_X)\to(Y,\OO_Y)$ of schemes. Then $(\pi,\pi^\sharp)$ is quasiseparated if and only if the diagonal map $\Delta\colon X\to X\times_YX$ (induced by using $\id_X\colon X\to X$) is quasicompact.
\end{lemma}
\begin{proof}
	To show $\Delta$ is quasicompact, one can just reduce to an affine open cover.
	
	In the other direction, suppose that the diagonal map $\Delta$ is quasicompact, and we show that $\pi$ is quasiseparated. Well, fix some affine open subset $U\subseteq Y$ with affine open subsets $V_1,V_2\subseteq\pi^{-1}(U)$, and we want to show that $V_1\cap V_2$ is quasicompact. For this, we draw the diagram
	% https://q.uiver.app/?q=WzAsNCxbMCwwLCJWXzFcXHRpbWVzX1hWXzIiXSxbMSwwLCJWXzFcXHRpbWVzX1lWXzIiXSxbMSwxLCJYXFx0aW1lc19ZWCJdLFswLDEsIlgiXSxbMywyXSxbMSwyXSxbMCwxXSxbMCwzXV0=&macro_url=https%3A%2F%2Fraw.githubusercontent.com%2FdFoiler%2Fnotes%2Fmaster%2Fnir.tex
	\[\begin{tikzcd}
		{V_1\times_XV_2} & {V_1\times_YV_2} \\
		X & {X\times_YX}
		\arrow[from=2-1, to=2-2]
		\arrow[from=1-2, to=2-2]
		\arrow[from=1-1, to=1-2]
		\arrow[from=1-1, to=2-1]
	\end{tikzcd}\]
	which commutes purely formally by properties of the fiber product. Notably, $V_1,V_2\subseteq\pi^{-1}(U)$ promises that $V_1\times_YV_2=V_1\times_UV_2$. However, the main point is that $V_1\cap V_2=V_1\times_XV_2$ and $\pi^{-1}(V_1\times_UV_2)=V_1\cap V_2$ again. Thus, we note that $V_1\times_YV_2$ being quasicompact tells us that $V_1\cap V_2$ is quasicompact by taking the pre-image using the compactness of $\Delta$.
\end{proof}
\begin{remark}
	Being quasiseparated is preserved under composition, base change and is affine local on the target. Indeed, this can be reduced to statements about quasicompactness by \autoref{lem:qs-by-diag}.
\end{remark}
\begin{remark}
	If the codomain of a scheme morphism is locally Noetherian, then the morphism is quasiseparated.
\end{remark}

\subsection{Affine Morphisms}
Here is our definition.
\begin{definition}[Affine]
	A scheme morphism $\pi\colon X\to Y$ if and only if every affine open subset $U\subseteq Y$ has $\pi^{-1}(U)$ affine.
\end{definition}
\begin{remark}
	We can easily see that being affine implies being quasicompact and quasiseparated.
\end{remark}
If we want the analogue of \autoref{lem:mainqclemma} for affine morphisms, we must do some extra work.
\begin{lemma}
	A morphism $\pi\colon X\to Y$ is affine if and only if we can provide $Y$ with an affine open cover $\mc U$ such that $\pi^{-1}(U)$ is affine for each $U\in\mc U$.
\end{lemma}
\begin{proof}
	To begin, note that a morphism $\pi\colon\Spec B\to\Spec A$ of affine schemes will pull back distinguished open sets to distinguished open sets.

	Now, call our affine open cover $\{U_\alpha\}_{\alpha\in\lambda}$ of $Y$ satisfying the conclusion. Then for some affine open subset $U=\Spec A$ of $X$, we can intersect $U$ with each $U_\alpha$ to give an affine open cover of $U$ by various distinguished open subsets. Then the pre-image of each of these under $\pi$ is affine by our previous remark, and we can glue these affine schemes to another affine scheme by Exercise~2.17(b) in \cite{hartshorne}, which was on the homework. Notably, the fact we have an affine open cover is what tells us that the chosen elements from our distinguished open subsets will generate the unit ideal of $A$.
\end{proof}
\begin{remark}
	As usual, affine morphisms are preserved by composition, base-change, and it is local on the target.
\end{remark}
Next time we move on to talk about finite morphisms, integral morphisms, and morphisms of finite type.

\end{document}