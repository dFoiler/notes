% !TEX root = ../notes.tex

\documentclass[../notes.tex]{subfiles}

\begin{document}

It is another day.
\begin{remark}
	Facts used on the homework from Vakil which are in Vakil without proof should be proven on the homework.
\end{remark}

\subsection{Sheafification}
We want to show that the category of abelian sheaves is abelian. We discussed kernels; we still need to discuss cokernels.
\begin{example} \label{ex:protoses}
	Fix $X=\CC$ with the usual topology, and give it the sheaf $\mc O_X$ of holomorphic functions. There is a constant sheaf $\underline\ZZ$ returning $\ZZ$ always. Then there is an exact sequence of sheaves
	\begin{equation}
		0\to\underline\ZZ\stackrel{2\pi i}\to\OO_X\stackrel{\exp}\to\OO_X^\times\to1 \label{eq:protoses}
	\end{equation}
	even though the last map is not always surjective for any $U\subseteq\CC$; for example, take $U=\CC\setminus\{0\}$. (However, if $U$ is simply connected, then the map will be surjective.)
\end{example}
\begin{remark}
	Cohomology applied to \autoref{eq:protoses} (with $X$ some smooth projective curve) shows a special case of the Hodge conjecture.
\end{remark}
To fix our broken surjectivity, we introduce sheafification.
\begin{remark}
	Setting
	\[\mc F(U)\coloneqq\im\exp(U)\]
	makes $\mc F$ a presheaf but does not give a sheaf.
\end{remark}
So that's not going to be our sheafification. Here it is.
\begin{definition}
	Fix a presheaf $\mc F$ on $X$ valued in a (concrete) category $\mc C$. The \textit{sheafification} of $\mc F$ is a pair $(\mc F^{\op{sh}},{\op{sh}})$ where ${\op{sh}}\colon\mc F\to\mc F^{\op{sh}}$ satisfies the following universal property: any sheaf $\mc G$ with a presheaf morphism $\varphi\colon\mc F\to\mc G$ has a unique morphism $\psi\colon\mc F^{\op{sh}}\to\mc G$ making the following diagram commute.
	% https://q.uiver.app/?q=WzAsMyxbMCwwLCJcXG1jIEYiXSxbMSwwLCJcXG1jIEZee1xcb3B7c2h9fSJdLFsxLDEsIlxcbWMgRyJdLFswLDIsIlxcdmFycGhpIiwyXSxbMCwxLCJcXG9we3NofSJdLFsxLDIsIlxccHNpIl1d&macro_url=https%3A%2F%2Fraw.githubusercontent.com%2FdFoiler%2Fnotes%2Fmaster%2Fnir.tex
	\[\begin{tikzcd}
		{\mc F} & {\mc F^{\op{sh}}} \\
		& {\mc G}
		\arrow["\varphi"', from=1-1, to=2-2]
		\arrow["{\op{sh}}", from=1-1, to=1-2]
		\arrow["\psi", from=1-2, to=2-2]
	\end{tikzcd}\]
\end{definition}
Of course, there are some checks we should do before using this object.
\begin{lemma}
	The sheafification of a presheaf $\mc F$ on $X$ exists and is unique up to unique isomorphism.
\end{lemma}
\begin{proof}
	The point is to set $\mc F^{\op{sh}}(U)$ to be systems of compatible germs; precisely,
	\[\mc F^{\op{sh}}(U)\coloneqq\Bigg\{(f_p)_{p\in U}\in\prod_{p\in U}\mc F_p:(f_p)_{p\in U}\text{ are compatible germs}\Bigg\}.\]
	The sheafification map is as follows.
	\[\arraycolsep=1.4pt\begin{array}{ccccccccc}
		{\op{sh}}\colon & \mc F(U) &\to& \mc F^{\op{sh}}(U) \\
		& f &\mapsto& (f|_p)_{p\in U}
	\end{array}\]
	One can check that this forms a sheaf and satisfies the universal property. The universal property gives the uniqueness.
\end{proof}
Here are some basic properties.
\begin{proposition}
	Fix a presheaf $\mc F$ on $X$ with a sheafification ${\op{sh}}\colon\mc F\to\mc F^{\op{sh}}$.
	\begin{listalph}
		\item For given $p\in X$, the induced map ${\op{sh}_p}\colon\mc F_p\to(\mc F^{\op{sh}})_p$ on stalks is an isomorphism.
		\item Sheafification is right adjoint to the forgetful functor from sheaves on $X$ to presheaves on $X$.
	\end{listalph}
\end{proposition}
\begin{proof}
	Omitted.
\end{proof}

\subsection{Passing to Stalks}
We are now ready to fix our surjectivity. We start by fixing the image.
\begin{defi}[Sheaf image]
	Fix a morphism $\varphi\colon\mc F\to\mc G$ of sheaves on $X$. Then the \textit{sheaf image $\im\varphi$} of $\varphi$ is the sheafification of the presheaf image
	\[(\im^{\op{pre}}\varphi)(U)=\im\varphi_U.\]
\end{defi}
And now we get the following analogue to \autoref{prop:injonstalks}.
\begin{proposition}
	Fix a morphism $\varphi\colon\mc F\to\mc G$ of sheaves on $X$. The following are equivalent.
	\begin{listalph}
		\item $\varphi_p$ is surjective for each $p$.
		\item $\im\varphi=\mc G$.
		\item $\varphi\colon\mc F\to\mc G$ is an epimorphism in the category of sheaves.
	\end{listalph}
\end{proposition}
\begin{proof}
	Omitted. The main point is to hit everything with the universal property and the fact that stalks don't change under sheafification.
\end{proof}
\begin{remark}
	One can use the above proposition to show that the last map of \autoref{eq:protoses} is an epimorphism.
\end{remark}
And now we fix the cokernel.
\begin{definition}[Sheaf cokernel]
	Fix a morphism $\varphi\colon\mc F\to\mc G$ of sheaves on $X$. Then the \textit{sheaf cokernel $\coker\varphi$} of $\varphi$ is the sheafification of the presheaf cokernel
	\[(\coker^{\op{pre}}\varphi)(U)=\coker\varphi_U.\]
\end{definition}
And so we get some nice results by passing to stalks.
\begin{proposition}
	The category of sheaves on $X$ valued in an abelian category $\mc C$ is abelian. Notably, we have to use the sheaf kernel and sheaf cokernel.
\end{proposition}
\begin{proof}
	Pass all the checks to the stalks, where we get what we want for free because $\mc C$ is abelian.
\end{proof}
\begin{proposition}
	Fix a morphism $\varphi\colon\mc F\to\mc G$ of sheaves on $X$ has
	\[\im\varphi\simeq\ker\coker\varphi.\]
\end{proposition}
\begin{proof}
	Pass to stalks.
\end{proof}
\begin{proposition}
	A sequence
	\[0\to\mc F\to\mc G\to\mc H\to0\]
	of sheaves on $X$ is exact if and only if it is exact at all stalks.
\end{proposition}
\begin{proof}
	Unsurprisingly, pass to stalks.
\end{proof}

\subsection{Sheaf Functors}
We now discuss how to build some new sheaves from old.
\begin{definition}[Direct image sheaf]
	Fix a continuous map $f\colon X\to Y$ of topological spaces. Given a sheaf $\mc F$ on $X$, we define the \textit{direct image sheaf} on $Y$ to be
	\[f_*\mc F(V)\coloneqq\mc F\left(f^{-1}(V)\right).\]
\end{definition}
\begin{remark} \label{rem:directimagefunctor}
	One can check that, given continuous maps $f\colon X\to Y$ and $g\colon Y\to Z$, we have
	\[(f\circ g)_*=f_*\circ g_*\]
	as maps from sheaves to sheaves. As a hint, one should check this on stalks.
\end{remark}
\begin{definition}[Inverse image sheaf]
	Fix a continuous map $f\colon X\to Y$ of topological spaces. Given a sheaf $\mc G$ on $Y$, we define the \textit{inverse image sheaf} $f^{-1}\mc G$ on $X$ to be the sheafification of the presheaf
	\[U\mapsto\colimit_{V\supseteq f(U)}\mc G(V).\]
\end{definition}
\begin{remark}
	One can check, as in \autoref{rem:directimagefunctor}, that $(f\circ g)^{-1}=g^{-1}\circ f^{-1}$ as maps from sheaves to sheaves.
\end{remark}
Here is another way to verify our last remark.
\begin{proposition}
	There is a natural bijection
	\[\op{Mor}_X\left(f^{-1}\mc G,\mc F\right)\simeq\op{Mor}_Y\left(\mc G,f_*\mc F\right).\]
	In other words, we have a pair of adjoint functors.
\end{proposition}
\begin{proof}
	Omitted.
\end{proof}
\begin{remark}
	The functor $f^{-1}$ also has a right adjoint, which will appear on the homework. Thus, $f^{-1}$ as a functor is exact.
\end{remark}

\subsection{More Sheaves}
Let's see a few more examples, for fun.
\begin{definition}[Constant sheaf]
	Fix a set $S$ and a topological space $X$. Then the \textit{constant sheaf} is
	\[\underline S(U)\coloneqq S^{\oplus\pi_0(U)},\]
	where $\pi_0(U)$ is the number of connected components in $U$. Notably, all the stalks of $\underline S$ are $S$.
\end{definition}
\begin{definition}[Skyscraper sheaf]
	Fix a topological space $Y$ and a set $S$. For $y\in Y$, set $X\coloneqq\{y\}$ so that there is a continuous map $\iota\colon X\into Y$. Then we define the \textit{skyscraper sheaf} as
	\[\iota_*S(U)\coloneqq\begin{cases}
		S & y\in U, \\
		\{*\} & y\notin U.
	\end{cases}\]
\end{definition}
\begin{remark}
	For $z\in Y$, we can compute the stalk of the skyscraper sheaf as
	\[(\iota_*S)_z=\begin{cases}
		S & z\in\overline{\{y\}}, \\
		\{*\} & z\notin\overline{\{y\}}.
	\end{cases}\]
\end{remark}
For another remark, we pick up the following definition.
\begin{definition}[Support]
	Fix a sheaf $\mc F$ on a topological space $x$. Then we define the \textit{support} of $\mc F$ to be
	\[\op{supp}\mc F\coloneqq\{x\in X:\#\mc F_x>1\}.\]
\end{definition}
\begin{remark}
	The support of $\iota_*S$ is $\overline{\{y\}}$.
\end{remark}
Here is another result, which explains why we care about the skyscraper sheaf.
\begin{proposition}
	There is a natural bijection
	\[\op{Mor}_{\{y\}}(\mc F_y,\mc G)\simeq\op{Mor}_Y(\mc F,\iota_*\mc G).\]
	In other words, understanding maps from stalks is roughly the same as understanding maps to the corresponding skyscraper sheaf.
\end{proposition}

\subsection{Schemes}
We close lecture by defining schemes. We begin by defining a locally ringed space.
\begin{definition}[Locally ringed space]
	A \textit{locally ringed space} is a sheaf of rings $\mc O_X$ on a topological space $X$ such that all stalks are local rings.
\end{definition}
\begin{example}
	Affine schemes give locally ringed spaces.
\end{example}
\begin{definition}[Morphism of locally ringed spaces]
	A \textit{morphism $\varphi\colon(X,\mc O_X)\to(Y,\mc O_Y)$ of locally ringed spaces} consists of an ordered pair $(f,f^\sharp)$ in such a way that $f\colon X\to Y$ is continuous, and $f^\sharp\colon\mc O_Y\to f_*\mc O_X$ is a morphism of sheaves in such a way that the induced map
	\[f^\sharp_p\colon\mc O_{Y,p}\to(f_*\mc O_X)_p\]
	of stalks is a map of local rings.
\end{definition}
Notably, a morphism of local rings $\varphi\colon(R,\mf m)\to(S,\mf n)$ requires $\varphi^{-1}\mf n=\mf m$.
\begin{definition}[Scheme]
	A \textit{scheme} is a pair $(X,\mc O_X)$ of a topological space $X$ and a locally ringed space $\mc O_X$ such that each $p\in X$ has a $U_p\subseteq X$ such that $(U_p,\mc O_X|_{U_p})$ is isomorphic (as a locally ringed space) to an affine scheme.
\end{definition}

\end{document}