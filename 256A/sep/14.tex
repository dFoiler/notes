% !TEX root = ../notes.tex

\documentclass[../notes.tex]{subfiles}

\begin{document}

\section{September 14}

We continue to study things which go bump in the night.

\subsection{A Better Noetherian}
One annoying thing about our locally Noetherian definition is that we are being forced to choose a very special affine open cover. However, we can remove this stress because it turns out all affine open subsets will be Noetherian.
\begin{proposition}
	Fix a locally Noetherian scheme $(X,\OO_X)$. Then any affine open subset $U\subseteq X$ with $(U,\OO_X|_U)\cong\Spec A$ for a ring $A$ has $A$ a Noetherian ring.
\end{proposition}
\begin{proof}
	Fix an affine open cover $\mc U$ for $U$ making everything Noetherian. By intersecting $\Spec A$ with this open cover, we can pick up a finite subcover contained in particular distinguished open sets $D(f_i)$ for various $f_i\in A$. As such, we see that the $A_{f_i}$ are Noetherian because they are contained form our open cover.

	We now upgrade to $A$. Suppose $I\subseteq A$ is an ideal. Then each $IA_{f_i}$ is finitely generated, so find finitely many elements in $I$ which generate $IA_{f_i}$ as an $A_{f_i}$-ideal (by clearing denominators as necessary). All these elements combine to give a finitely generated ideal $J$, which we want to show equals $I$. Namely, we have $J\subseteq I$ of course, and
	\[JA_{f_i}=IA_{f_i}\]
	for each $i$. Intuitively, what's going on here is that $J$ and $I$ are ``sheaves'' with a morphism $J\subseteq I$ which is an isomorphism on our open cover, so they should be the same ideal.

	Let's see this rigorously. In particular, by taking the direct limit over our localizations, we see
	\[IA_\mf p=JA_\mf p\]
	by essentially taking stalks. Now, suppose $x\in I\setminus J$, and set
	\[\mf a\coloneqq\{a\in A:ax\in J\}.\]
	We can check $\mf a\subseteq A$ is an ideal, and $\mf a$ is proper because $1\notin\mf a$; as such, put $\mf a$ inside a maximal ideal $\mf m$. However,
	\[IA_\mf m=JA_\mf m,\]
	which is going to give us a contradiction: indeed, we get $s\notin\mf m$ with $sx\in J$ because $x\in IA_\mf m=JA_\mf m$, which is a contradiction to the construction of $\mf m$.
\end{proof}
\begin{cor}
	Fix a ring $A$. Then $(\Spec A,\OO_{\Spec A})$ is locally Noetherian if and only if $A$ is Noetherian.
\end{cor}
\begin{proof}
	Note $\Spec A\subseteq\Spec A$ makes $(\Spec A,\OO_{\Spec A})$ an affine open subset of itself.
\end{proof}
\begin{remark}
	The above result says that being ``locally Noetherian'' can be checked locally. It will turn out that this is a general notion, but we will not discuss this generalization for some time.
\end{remark}

\subsection{Reduced Schemes}
Here is the definition.
\begin{definition}[Reduced]
	A scheme $(X,\OO_X)$ is \textit{reduced} if and only if each $p\in X$ give a reduced local ring $\OO_{X,p}$; i.e., we are asking for $\op{nilrad}\OO_{X,p}=(0)$.
\end{definition}
\begin{remark}
	On the homework, we will show that being reduced is equivalent to showing that $\OO_X(U)$ is reduced for all open $U\subseteq X$.
\end{remark}
\begin{example}
	The affine ring $\PP^n_k$ is reduced because its affine open patches are reduced.
\end{example}
\begin{remark}
	In particular, it turns out that an affine scheme $\Spec A$ is reduced if and only if $A$ is reduced. This is also on the homework.
\end{remark}
\begin{nex}
	The scheme $\Spec k[x,y]/(x^2)$ is not reduced because of the stalk at $(x)$.
\end{nex}
Being reduced is a nice property, so we might want to force schemes to be reduced.
\begin{definition}[Reduced scheme associted]
	Given a scheme $(X,\OO_X)$, the \textit{reduced scheme associated} to $(X,\OO_X)$ is the scheme $(X,\OO_X/\mc N)$, where $\mc N(U)\coloneqq\{s\in\OO_X(U):s|_x\in\OO_{X,x}\text{ is nilpotent}\}$ for each $U\subseteq X$.
\end{definition}
This satisfies the following universal property.
\begin{lemma}
	Fix a scheme $(X,\OO_X)$, and let $(X^{\op{red}},\OO_X^{\op{red}})$ be the reduced scheme. Then, for any reduced scheme $(Y,\OO_Y)$ and map $\varphi\colon(Y,\OO_Y)\to(X,\OO_X)$, there is a unique map $\overline\varphi\colon(Y,\OO_Y)\to(X^{\op{red}},\OO_X^{\op{red}})$ making the following diagram commute.
	% https://q.uiver.app/?q=WzAsMyxbMSwwLCIoWCxcXE9PX1gpIl0sWzAsMCwiKFksXFxPT19ZKSJdLFsxLDEsIihYXntcXG9we3JlZH19LFxcT09fWF57XFxvcHtyZWR9fSkiXSxbMiwwXSxbMSwwLCJcXHZhcnBoaSJdLFsxLDIsIlxcb3ZlcmxpbmVcXHZhcnBoaSIsMix7InN0eWxlIjp7ImJvZHkiOnsibmFtZSI6ImRhc2hlZCJ9fX1dXQ==&macro_url=https%3A%2F%2Fraw.githubusercontent.com%2FdFoiler%2Fnotes%2Fmaster%2Fnir.tex
	\[\begin{tikzcd}
		{(Y,\OO_Y)} & {(X,\OO_X)} \\
		& {(X^{\op{red}},\OO_X^{\op{red}})}
		\arrow[from=2-2, to=1-2]
		\arrow["\varphi", from=1-1, to=1-2]
		\arrow["\overline\varphi"', dashed, from=1-1, to=2-2]
	\end{tikzcd}\]
\end{lemma}
\begin{proof}
	Omitted.
\end{proof}
\begin{example}
	The reduced scheme associated to $\Spec k[x,y]/(x^2)$ just becomes $\Spec k[y]$. Intuitively, we are ``deleting'' all of our differential information.
\end{example}

\subsection{Integral Schemes}
Here is the definition.
\begin{definition}[Integral]
	A scheme $(X,\OO_X)$ is \textit{integral} if and only if all open subsets $U\subseteq X$ give an integral domain $\OO_X(U)$.
\end{definition}
\begin{remark}
	Note that $X$ being integral will imply that each $\OO_{X,x}$ is an integral domain by taking the direct limit, but the converse does not hold.
\end{remark}
\begin{proposition}
	A scheme $(X,\OO_X)$ is integral if and only if $(X,\OO_X)$ is reduced and irreducible.
\end{proposition}
\begin{proof}
	In the forward direction, note that $(X,\OO_X)$ is easily reduced. Further, if $(X,\OO_X)$ is not irreducible, then we have two proper closed subsets $V_1,V_2\subseteq X$ covering $X$. This grants nonempty open subsets $U_1,U_2\subseteq X$, and we can force these to be disjoint. It follows
	\[\OO_X(U_1\cup U_2)=\OO_X(U_1)\times\OO_X(U_2)\]
	is not integral.

	In the other direction, suppose $(X,\OO_X)$ is irreducible and integral. Well, we have an open subset $U\subseteq X$ and nonzero $f,g\in\OO_X(U)$ with $fg=0$. We now define
	\[V(a)\coloneqq\{x\in U:g_x\in\mf m_x\},\]
	and we can check that $V(f)$ and $V(g)$ are going to be closed subsets of $U$. Because $X$ is irreducible, $U$ is as well, so $U= V(g)\cup V(f)$ forces $U$ to be contained in one of these closed subsets, so without loss of generality take $U=V(f)$. Going down to an affine open patch makes $V(f)$ equal to a full spectrum $\Spec A$, so we conclude
	\[f\in\bigcap_{\mf p\in\Spec A}\mf p=\op{nilrad}A.\]
	But being reduced now forces $f=0$, finishing.
\end{proof}
\begin{example}
	A reduced scheme whose stalks are integral domains even though $X$ is not irreducible will not make $X$ in total integral. Somehow being integral is a more global property.
\end{example}
Here is a good property of integral schemes.
\begin{proposition}
	An integral scheme $(X,\OO_X)$ has a unique generic point $\xi$ for $X$. Then
	\[\OO_{X,\xi}=\op{Frac}\OO_X(U)\]
	for any $U\subseteq X$.
\end{proposition}
\begin{proof}
	The existence of $\xi$ comes from the fact that irreducible components (which here is only $X$) have a unique generic point. The second claim follows by taking the direct limit over our $U$, trying to get down to the generic point.
\end{proof}
So we get the following nice definition.
\begin{definition}[Function field]
	An integral scheme $(X,\OO_X)$ with generic point $\xi$ has \textit{function field}
	\[\OO_{X,\xi}\coloneqq\op{Frac}\OO_X(X)\]
\end{definition}
The point here is that we can retrieve the field out from some $\Spec k[x,y]$, say. This also allows us to define regular functions.
\begin{definition}[Regular]
	Fix an integral scheme $(X,\OO_X)$ with generic point $\xi$. Then $f\in\OO_{X,\xi}$ is \textit{regular} at a point $x\in X$ if and only if $f$ lifts to $\OO_{X,x}$.
\end{definition}

\subsection{Closed Subschemes}
Open subschemes had natural subscheme structure by just taking restriction. Closed subschemes are a little harder.
\begin{example}
	Set $X\coloneqq\Spec k[x,y]$. Then the closed subset $V(x)$ will have lots of natural homeomorphisms
	\[V(x)\cong\Spec\frac{k[x,y]}{(x^n)},\]
	for any $n\ge1$, so there is no canonical way to set the structure sheaf.
\end{example}
The idea to define a closed subscheme is to instead keep track of the morphism which does the embedding.
\begin{definition}[Closed immersion]
	A scheme morphism $(f,f^\sharp)\colon(Z,\OO_Z)\to(X,\OO_X)$ is a \textit{closed immersion} if and only if the following two conditions hold.
	\begin{itemize}
		\item The map $f\colon Z\to X$ is a homeomorphism from $Z$ onto a closed subset of $X$.
		\item The map $f^\sharp\colon\OO_X\to f_*\OO_Z$ is epic.\todo{Germs should pull back to germs.}
	\end{itemize}
	If in fact $Z\subseteq X$ is a closed subset, then we will say $Z$ is a closed subscheme.
\end{definition}
The main point here is that we would like our closed immersions in the affine case to be induced by $A\onto A/I$ as in \autoref{exe:closedaffinesubscheme}.
\begin{proposition} \label{prop:affineclosedsubschemes}
	Fix an affine scheme $(X,\OO_X)\coloneqq(\Spec A,\OO_{\Spec A})$.
	\begin{listalph}
		\item Each ideal $I\subseteq A$ induces a closed immersion
		\[\Spec A/I\to\Spec A\]
		from the projection map $A\onto A/I$. In particular, this gives $V(I)\subseteq\Spec A$ the structure of a closed subscheme.
		\item The map of (a) provides a bijection between ideals of $A$ and closed subschemes of $\Spec A$.
	\end{listalph}
\end{proposition}
\begin{proof}
	Here we go.
	\begin{listalph}
		\item From \autoref{exe:closedaffinesubscheme}, we already have the natural homeomorphism $\Spec A/I\cong V(I)$. On the level of sheaves, we only need to check surjectivity at stalks, for which we look at the distinguished open base. Namely, at some $D(f)$, we are studying the map
		\[A_f=\OO_X(D(f))\to\OO_Z(D(f+I))=(A/I)_f\simeq A_f/I_f,\]
		which we can see is surjective here. Taking the direct limit shows that we remain surjective on the level of stalks.
		\item This proof will be able to simplified later in life when we have talked about coherent sheaves. Fix a closed subscheme $\iota\colon Z\to X$ with $\iota^\sharp\colon\OO_X\to\iota_*\OO_Z$. Now, define
		\[I_Z\coloneqq\ker(\iota^\sharp_X).\]
		Then one can show that $I_Z$ is the ideal we want, providing the inverse to (a). In particular, one can show that $Z$ is identified with $\Spec A/I_Z$ as schemes. Notably, there is an embedding $Z\into Y$ by first looking on the level of topological spaces and then carrying over to schemes. It remains to show that this is an isomorphism.
		\begin{itemize}
			\item We show that we have a homeomorphism of our topological spaces. Note that $Z$ is quasicompact, so we can write it as a finite union of affine open subschemes. Now, for $Z\subseteq V(s)$, we will show that $V(s)=Y$. We see $s\in B\coloneqq A/I_Z$, so note that $s\in\OO_X(U_i)$ will be nilpotent because of its definition. Looping over entire affine open cover forces $s^n=0$, so the injectivity of the open cover
			\[A/I_Z\into\OO_Z(Z)\]
			forces $s^N=0$ in $A/I_Y$, meaning $Y\subseteq V(s)$.
			\item We now need to show $\OO_Y\to\iota_*\OO_Z$. Surjectivity follows from the construction looking at the global sections. The injectivity follows by looking at stalks and making an argument similar to the above.
			\qedhere
		\end{itemize}
	\end{listalph}
\end{proof}
% GW, T3.42

\end{document}