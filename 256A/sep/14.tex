% !TEX root = ../notes.tex

\documentclass[../notes.tex]{subfiles}

\begin{document}

\section{September 14}

We continue to study things which go bump in the night.

\subsection{The Affine Communication Lemma}
One annoying thing about our locally Noetherian definition is that we are being forced to choose a very special affine open cover.

In fact, in the discussion that follows, we will want to check many properties as being affine-local, so we will go ahead and state the relevant lemma now.
\begin{lemma}[Affine communication] \label{lem:affine-comm}
	Fix a scheme $(X,\OO_X)$, and let $P$ be a class of its affine open subschemes. Suppose the following conditions are met.
	\begin{listroman}
		\item If $U\in P$, then $U_f\in P$ for any $f\in\OO_X(U)$.
		\item Given an affine open subscheme $U\subseteq X$ with some elements $f_1,\ldots,f_n\in\OO_X(U)$ generating $\OO_X(U)$, if $U_{f_i}\in P$ for each $i$, then $U\in P$.
		\item There is an affine open cover $\{U_\alpha\}_{\alpha\in\lambda}$ such that $U_\alpha\in P$ for each $\alpha$.
	\end{listroman}
	Then $P$ contains all affine open subsets of $X$.
\end{lemma}
\begin{remark} \label{rem:topological-affine-comm-ii}
	To explain (ii), we note that $(f_1,\ldots,f_n)=\OO_X(U)$ if and only if the $U_{f_i}$ cover $U$. Set $A\coloneqq\OO_X(U)$ for brevity, and let $\varphi\colon(U,\OO_X|_U)\cong(\Spec A,\OO_{\Spec A})$ be the canonical isomorphism.
	\begin{itemize}
		\item Suppose the $U_{f_i}$ cover $U$; set $A\coloneqq\OO_X(U)$ for brevity. Passing through $\varphi$, we are told that the sets $D(f_i)$ cover $\Spec A$, and we want to show that $(f_1,\ldots,f_n)=A$. Well, each prime $\mf p$ does not contain at least of the $f_i$ by our cover, so $(f_1,\ldots,f_n)$ is contained in no maximal ideal, so this ideal is not proper, so $(f_1,\ldots,f_n)=A$.
		\item Conversely, suppose $(f_1,\ldots,f_n)=A$. Then no prime $\mf p\in\Spec A$ contains all the $f_i$, so the $D(f_i)$ cover $\Spec A$. Passing back through $\varphi$, it follows that the $U_{f_i}$ cover $U$.
	\end{itemize}
\end{remark}
The proof of this result rests on the following result, interesting in its own right.
\begin{lemma} \label{lem:simul-distinguish}
	Fix a scheme $(X,\OO_X)$ and affine open subschemes $(\Spec A,\OO_{\Spec A})\cong (U,\OO_X|_U)\subseteq (X,\OO_X)$ and $(\Spec B,\OO_{\Spec B})\cong (V,\OO_X|_V)\subseteq (X,\OO_X)$. Then, for any $p\in U\cap V$, we can find an open subset $W\subseteq U\cap V$ containing $p$ such that $W\subseteq U$ and $W\subseteq V$ both make $W$ in to a distinguished open subscheme.
\end{lemma}
\begin{proof}
	We follow \cite[Proposition~5.3.1]{rising-sea}. Without loss of generality, we may use \autoref{cor:the-affine-iso} to make $A=\OO_X|_U(U)=\OO_X(U)$ and $B=\OO_X|_V(V)=\OO_X(V)$, where the corresponding scheme isomorphisms are the canonical ones.
	
	Namely, as we computed in \autoref{lem:specgamma}, the underlying homeomorphism associates $D(f)\subseteq\Spec A$ with $U_f$ where $f\in\OO_X(U)=A$ (and similar for $V$), so we will elect to work with the open subsets which look like as our ``distinguished open sets.'' In particular, because we still have a homeomorphism, the fact that distinguished opens $D(f)$ make a basis of the topology on $\Spec A$ means that the open subsets $U_f$ will make a basis of the topology on $U$.

	Now, $U\cap V\subseteq U$ is an open subset containing $p$, so we begin by giving it a distinguished open $U_{f}\subseteq U\cap V$ containing $p$, where $f\in A$. Continuing, $U_{f}\cap V\subseteq V$ is an open subset containing $p$, so we may next find some $g\in B$ such that $V_g\subseteq U_{f}\cap V$ containing $p$.
	
	It might look like we're in an infinite loop, but we're not: $V_g\subseteq U_{f}$ means that $g\in\OO_X(V)$ will restrict to some element $g'$ in $\OO_X(U_{f})\simeq D(f)=\OO_X(U)_{f}$, so we let $g'\coloneqq g''/f^n$. Now, we compute
	\begin{align*}
		V_g &= U_f\cap\{x\in X:g|_x\notin\mf m_{X,x}\} \\
		&= \left\{x\in U_f:g''/f^n|_x\notin\mf m_{X,x}\right\} \\
		&\stackrel*= \{x\in U_f:g''|_x\notin\mf m_{X,x}\} \\
		&= \{x\in U:(fg'')|_x\notin\mf m_{X,x}\} \\
		&= U_{fg''},
	\end{align*}
	which is what we wanted. Namely, $\stackrel*=$ holds by thinking as in an affine scheme: a prime $\mf p\in\Spec\OO_X(U)_f$ has $g''/f^n\notin\mf p$ if and only if $g''\notin\mf p$,
\end{proof}
We are now ready to prove \autoref{lem:affine-comm}
\begin{proof}[Proof of \autoref{lem:affine-comm}]
	We follow \cite[Lemma~5.3.2]{rising-sea}. Use (iii) to pick up an affine open cover $\{U_\alpha\}_{\alpha\in\lambda}$ of $X$.
	
	Now, fix any affine open subscheme $U\subseteq X$, and we need to show $U\in P$. For each $\alpha$ and $p\in U\cap U_\alpha$, we can use \autoref{lem:simul-distinguish} to find some open subset $V_{\alpha,p}\subseteq U\cap U_\alpha$ containing which is simultaneously a distinguished open set of both $U$ and $U_\alpha$. Notably, each $V_{\alpha,p}$ is a distinguished open subset of $U_\alpha$, so $V_{\alpha,p}\in P$ by (i).

	Now, we recognize that we have an open cover
	\[U=\bigcup_{\alpha\in\lambda}(U\cap U_\alpha)=\bigcup_{\alpha\in\lambda}\bigcup_{p\in U\cap U_\alpha}V_{\alpha,p}.\]
	However, $U$ is an affine open subscheme and therefore quasicompact, so we extract a finite subcover. Because these are all distinguished open sets of $U$, we may write our open subcover as $\{U_{f_i}\}_{i=1}^n\subseteq P$ where $\{f_i\}_{i=1}^n\subseteq\OO_X(U)$.
	
	Continuing, $U$ is covered by the $U_{f_i}$, so we claim that $(f_1,\ldots,f_n)=\OO_X(U)$; set $A\coloneqq\OO_X(U)$ for brevity. Letting $\varphi\colon(U,\OO_X|_U)\cong(\Spec A,\OO_{\Spec A})$, we are told that the sets $D(f_i)$ cover $\Spec A$, and we want to show that $(f_1,\ldots,f_n)=A$. Well, each prime $\mf p$ does not contain at least of the $f_i$ by our cover, so $(f_1,\ldots,f_n)$ is contained in no maximal ideal, so this ideal is not proper, so $(f_1,\ldots,f_n)=A$. It follows that $U\in P$ by (ii).
\end{proof}

\subsection{A Better Noetherian}
As a first application of \autoref{lem:affine-comm}, we fix our definition of a locally Noetherian scheme.
\begin{proposition} \label{prop:better-local-noetherian}
	Fix a locally Noetherian scheme $(X,\OO_X)$. Then any affine open subset $U\subseteq X$ with $(U,\OO_X|_U)\cong(\Spec A,\OO_{\Spec A})$ for a ring $A$ has $A$ a Noetherian ring.
\end{proposition}
\begin{proof}
	We use \autoref{lem:affine-comm}. Call an affine open subscheme $U\subseteq X$ ``good'' if and only if $\OO_X(U)$ is Noetherian. Because $(U,\OO_X|_U)\cong(\Spec A,\OO_{\Spec A})$ induces $\OO_X(U)\cong A$ on global sections, we are interested in showing that all affine open subschemes of $X$ are good. We now run the checks of \autoref{lem:affine-comm}.
	\begin{listroman}
		\item Suppose an affine open subscheme $U\subseteq X$ has $\OO_X(U)$ Noetherian. Then, for any $f\in\OO_X(U)$, the canonical isomorphism of \autoref{cor:the-affine-iso} tells us that $\OO_X(U_f)\simeq\OO_X(U)_f$, so we see that $U_f$ is good because the localization of a Noetherian ring is still Noetherian.
		\item Fix an affine open subscheme $U\subseteq X$ with elements $f_1,\ldots,f_n\in\OO_X(U)$ generating $\OO_X(U)$ such that $U_{f_i}$ is good for each $P$. For brevity, set $A\coloneqq\OO_X(U)$, and we are given that $A_{f_i}=\OO_X(U)_{f_i}\simeq\OO_X(U_{f_i})$ (using the canonical isomorphism of \autoref{cor:the-affine-iso}) is Noetherian for each $i$.

		We need to show that $A$ is Noetherian, so pick up some ideal $I\subseteq A$, and we need to show that $I$ is Noetherian. Well, for each $i$, we see $IA_{f_i}\subseteq A_{f_i}$ is an ideal and must be finitely generated, so find generators $\{x_{i,1},\ldots,x_{i,n_i}\}$. Now, for each $x_{i,j}$, we may write
		\[x_{i,j}=\frac{y_{i,j}}{f_i^{e_{i,j}}}\]
		where $y_{i,j}\in I$, and we see that the $y_{i,j}$ will also generate $IA_{f_i}$ as an $A_{f_i}$-module by just multiplying the necessary linear combinations by $1/f_i^{e_{i,j}}$ as is necessary.

		Now, let $J$ be the ideal generated by the $y_{i,j}$ over all $i$ and $j$, which is finitely generated because there are only finitely many $i$ and only finitely many $j$ for each $i$. Thus, we claim $I=J$, which will finish the proof.

		On one hand, we note $J\subseteq I$ because the generators of $J$ are contained in $I$. In the other direction, we note that any $a\in I$ can be embedded as $a/1\in IA_{f_i}$, for which we use our above generators $y_{i,j}$ to write
		\[\frac a1=\sum_{j=1}^{n_i}\frac{a_{i,j}y_{i,j}}{f_i^{d_{i,j}}}\]
		for some $a_{i,j}\in A$ and $d_{i,j}\in\NN$. Collecting denominators on the right-hand side and using the equality in $A_{f_i}$, there is some $N_i$ such that
		\[f^{N_i}a=\sum_{j=1}^{n_i}b_{i,j}y_{i,j}\in J\]
		for some $b_{i,j}\in A$.

		We are now essentially done. Note that
		\[\emp=V(A)=V((f_1,\ldots,f_n))=V(\rad(f_1,\ldots,f_n))=V\left((f_1^{N_1},\ldots,f_n^{N_n})\right),\]
		so it follows $(f_1^{N_1},\ldots,f_n^{N_n})=A$, so we may find $c_1,\ldots,c_n\in A$ such that $\sum_{i=1}^nc_if_i^{N_i}=1$, so
		\[a=\sum_{i=1}^nc_i\left(f_i^{N_i}a\right)\in J.\]
		This finishes showing that $I\subseteq J$.
		\item Note that we are promised an open cover by good affine open subschemes by hypothesis on $(X,\OO_X)$.
	\end{listroman}
	Thus, \autoref{lem:affine-comm} kicks in and finishes the proof.
\end{proof}
\begin{cor}
	Fix a ring $A$. Then $(\Spec A,\OO_{\Spec A})$ is locally Noetherian if and only if $A$ is Noetherian.
\end{cor}
\begin{proof}
	If $A$ is Noetherian, then we use the affine open cover $\{(\Spec A,\OO_{\Spec A})\}$ of $(\Spec A,\OO_{\Spec A})$ to show that $(\Spec A,\OO_{\Spec A})$ is locally Noetherian by definition. Conversely, if $(\Spec A,\OO_{\Spec A})$ is locally Noetherian, then its affine open subscheme $(\Spec A,\OO_{\Spec A})$ must give $A=\OO_{\Spec A}(\Spec A)$ Noetherian.
\end{proof}

\subsection{Reduced Schemes}
Here is the definition.
\begin{definition}[Reduced]
	A scheme $(X,\OO_X)$ is \textit{reduced} if and only if each $p\in X$ give a reduced local ring $\OO_{X,p}$; i.e., we are asking for $\op{nilrad}\OO_{X,p}=(0)$.
\end{definition}
\begin{example} \label{ex:reduced-affine-scheme}
	Fix a reduced ring $A$; we claim that $(\Spec A,\OO_{\Spec A})$ is a reduced scheme. Indeed, by \autoref{lem:affinestalk}, it suffices to show that $A_\mf p$ is reduced for any prime $\mf p\in\Spec A$. Well, suppose $a/x\in A_\mf p$ (where $x\notin\mf p$) has $(a/x)^n=0$ for some $n\in\NN$; if $n=0$, there is nothing to say. Otherwise, $a^n/x^n=0$, so $ya^n=0$ for some $y\notin\mf p$. Multiplying both sides by $y^{n-1}$, we conclude $(ay)^n=0$, so $ay=0$, so $a/x=0$.
\end{example}
We can also think about reduced schemes on the level of open sets.
\begin{lemma} \label{lem:reduced-on-sets}
	A scheme $(X,\OO_X)$ is reduced if and only if $\OO_X(U)$ is a reduced ring for each open subset $U\subseteq X$.
\end{lemma}
\begin{proof}
	We show our implications separately.
	\begin{itemize}
		\item Suppose that $\OO_X(U)$ is reduced for each open $U\subseteq X$. Given $p\in X$, we need to show that $\OO_{X,p}$ is also reduced. Well, fix some germ $f_p\in\OO_{X,p}$ for $p\in U$ such that $f_p^n=0$ for some $n\in\NN$; we need to show that $f_p=0$.
	
		Well, write $f_p=[(U,f)]$ so that $f\in\mc F(U)$ where $U$ contains $p$. Then we see
		\[[(U,0)]=0=f_p^n=[(U,f)]^n=[(U,f^n)]\]
		implies that {there is some open subset $V\subseteq U$ containing $p$ such that $f^n|_V=0$}. Thus, $(f|_V)^n=0$, so $f|_V=0$ because $\OO_X(V)$ is reduced. It follows
		\[f_p=[(U,f)]=[(V,f|_V)]=0,\]
		which is what we wanted.
	
		\item Suppose that $\OO_{X,p}$ is reduced for all $p\in X$. Then, for each open set $U\subseteq X$, we need to show that $\OO_X(U)$ is also reduced. Well, suppose that $f\in\OO_X(U)$ has $f^n=0$ for some $n\in\NN$. We need to show that $f=0$.
	
		For this, we note that all $p\in U$ will have
		\[(f|_p)^n=\left(f^n\right)|_p=0|_p=0\in\OO_{X,p},\]
		so because $\OO_{X,p}$ is reduced, we conclude that $f|_p=0$. Thus, $f$ lives in the kernel of the natural map
		\[\arraycolsep=1.4pt\begin{array}{ccc}
			\OO_X(U) &\to& \displaystyle\prod_{p\in U}\OO_{X,p} \\
			f &\mapsto& (f|_p)_{p\in U}
		\end{array}\]
		which is actually injective because $\OO_X$ is a sheaf. It follows that $f=0$.
		\qedhere
	\end{itemize}
\end{proof}
\begin{remark}
	For example, if a ring $A$ makes $(\Spec A,\OO_{\Spec A})$ into a reduced scheme, then we see $A=\OO_{\Spec A}(\Spec A)$ is a reduced ring. This provides the converse to \autoref{ex:reduced-affine-scheme}.
\end{remark}
\begin{nex}
	For example, the scheme $\Spec k[x,y]/(x^2)$ is not reduced because its ring of global sections is $k[x,y]/(x^2)$, which is not reduced.
\end{nex}
We can also think of the reduced condition on the level of affine open subschemes.
\begin{lemma}
	Fix a scheme $(X,\OO_X)$. The following are equivalent.
	\begin{listalph}
		\item $(X,\OO_X)$ is reduced.
		\item $\OO_X(U)$ is a reduced ring for every affine open subscheme $U\subseteq X$.
		\item There is an affine open cover $\mc U$ of $X$ such that $\OO_X(U)$ is reduced for each $U\in\mc U$.
	\end{listalph}
\end{lemma}
\begin{proof}
	Of course, (a) implies (b) by \autoref{lem:reduced-on-sets}, and (b) implies (c) by choosing any affine open cover $X$.
	
	Lastly, we show that (c) implies (a). Fix our affine open cover $\mc U$, where $\mc U$ is made of affine rings. Well, for any $p\in X$, we need to show that $\OO_{X,p}$ is reduced. For this, pick up some affine open subscheme $U\subseteq X$ containing $p$, where $\OO_X(U)$ is a reduced ring. Then, using the canonical isomorphism of \autoref{cor:the-affine-iso},
	\[(U,\OO_X|_U)\cong(\Spec\OO_X(U),\OO_X|_U)\]
	is a reduced scheme by \autoref{ex:reduced-affine-scheme}. Thus, the stalk $(\OO_X|_U)_p$ is a reduced, but this is canonically isomorphic to $\OO_{X,p}$ by \autoref{lem:invimagestalk}, so we are done.
\end{proof}
\begin{example} \label{ex:proj-is-reduced}
	Projective space $\PP^n_k$ is reduced because its affine open patches are reduced.
\end{example}
Being reduced is a nice property, so we might want to force schemes to be reduced.
\begin{definition}[Reduced scheme associted]
	Given a scheme $(X,\OO_X)$, the \textit{reduced scheme associated} to $(X,\OO_X)$ is the scheme $(X,\OO_X/\mc N)$, where $\mc N(U)\coloneqq\{s\in\OO_X(U):s|_x\in\OO_{X,x}\text{ is nilpotent}\}$ for each $U\subseteq X$.
\end{definition}
This satisfies the following universal property.
\begin{lemma}
	Fix a scheme $(X,\OO_X)$, and let $(X^{\op{red}},\OO_X^{\op{red}})$ be the reduced scheme. Then, for any reduced scheme $(Y,\OO_Y)$ and map $\varphi\colon(Y,\OO_Y)\to(X,\OO_X)$, there is a unique map $\overline\varphi\colon(Y,\OO_Y)\to(X^{\op{red}},\OO_X^{\op{red}})$ making the following diagram commute.
	% https://q.uiver.app/?q=WzAsMyxbMSwwLCIoWCxcXE9PX1gpIl0sWzAsMCwiKFksXFxPT19ZKSJdLFsxLDEsIihYXntcXG9we3JlZH19LFxcT09fWF57XFxvcHtyZWR9fSkiXSxbMiwwXSxbMSwwLCJcXHZhcnBoaSJdLFsxLDIsIlxcb3ZlcmxpbmVcXHZhcnBoaSIsMix7InN0eWxlIjp7ImJvZHkiOnsibmFtZSI6ImRhc2hlZCJ9fX1dXQ==&macro_url=https%3A%2F%2Fraw.githubusercontent.com%2FdFoiler%2Fnotes%2Fmaster%2Fnir.tex
	\[\begin{tikzcd}
		{(Y,\OO_Y)} & {(X,\OO_X)} \\
		& {(X^{\op{red}},\OO_X^{\op{red}})}
		\arrow[from=2-2, to=1-2]
		\arrow["\varphi", from=1-1, to=1-2]
		\arrow["\overline\varphi"', dashed, from=1-1, to=2-2]
	\end{tikzcd}\]
\end{lemma}
\begin{proof}
	On the homework.
\end{proof}
\begin{example}
	The reduced scheme associated to $\Spec k[x,y]/(x^2)$ just becomes $\Spec k[y]$. Intuitively, we are ``deleting'' all of our differential information.
\end{example}

\subsection{Integral Schemes}
Here is the definition.
\begin{definition}[Integral]
	A scheme $(X,\OO_X)$ is \textit{integral} if and only if all nonempty open subsets $U\subseteq X$ give an integral domain $\OO_X(U)$.
\end{definition}
\begin{remark} \label{rem:integral-at-stalks}
	Note that $X$ being integral will imply that each $\OO_{X,x}$ is an integral domain: if $f,g\in\OO_{X,x}$ are germs with $fg=0$ while $f\ne0$, we need to show $g=0$. Well, find a sufficiently small $W$ containing $x$ so that we can define $f=[(W,\widetilde f)]$ and $g=[(W,\widetilde g)]$. Now, we see $\widetilde f\cdot\widetilde g$ needs to restrict to the zero germ, so restrict $W$ enough so that $\widetilde f\widetilde g=0$. However, $\widetilde f\ne0$ because $f\ne0$, so we must instead have $\widetilde g=0$, which finishes.
\end{remark}
\begin{proposition} \label{prop:integral-is-reduced-irred}
	A scheme $(X,\OO_X)$ is integral if and only if $(X,\OO_X)$ is reduced and irreducible.
\end{proposition}
\begin{proof}
	We show the directions separately.
	\begin{itemize}
		\item In the forward direction, note that $(X,\OO_X)$ is easily reduced: indeed, all stalks are reduced because they are all integral domains by \autoref{rem:integral-at-stalks}. Further, if $(X,\OO_X)$ is not irreducible, then we have two proper closed subsets $V_1,V_2\subseteq X$ covering $X$. Setting $U_\bullet\coloneqq X\setminus V_\bullet$, we see that $U_1\cap U_2=\emp$ while $U_1,U_2\ne\emp$.

		Now, we let $u_1\in\OO_X(U_1)$ and $u_2\in\OO_X(U_2)$ correspond to the units. However, $u_1|_{U_1\cap U_2}=0|_{U_1\cap U_2}$, so we can glue $u_1$ and $0$ together into some $e_1\in\OO_X(U_1\cup U_2)$. Similarly, $0|_{U_1\cap U_2}=u_2|_{U_1\cap U_2}$ let us glue $0$ and $u_2$ into some $e_2\in\OO_X(U_1\cup U_2)$.

		Thus, we see that $\OO_X(U_1\cup U_2)$ is not an integral domain: note $e_1|_{U_1}\ne0$ and $e_2|_{U_2}\ne0$, so $e_1,e_2\ne0$, but
		\[(e_1e_2)|_{U_1}=(u_1\cdot0)=0\qquad\text{and}\qquad(e_1e_2)|_{U_2}=(0\cdot u_2)=0,\]
		so $e_1e_2=0$ by gluing.
		\item In the other direction, suppose $(X,\OO_X)$ is irreducible and reduced. Well, fix any open subset $U\subseteq X$ and $f,g\in\OO_X(U)$ such that $fg=0$. We now define
		\[V(a)\coloneqq\{x\in U:g_x\in\mf m_x\},\]
		and we see that $V(f)$ and $V(g)$ are going to be closed subsets of $U$ because they are the complement of the open ones in \autoref{lem:someopensupp}.
		
		Continuing, because $X$ is irreducible, $U$ is as well by \autoref{lem:someopensupp}, so $U= V(g)\cup V(f)$ forces $U$ to be contained in one of these closed subsets, so without loss of generality take $U=V(f)$. We are now ready to claim $f=0$, which will finish.
		
		Because $U$ is an open subscheme, it suffices to show that $f=0$ on an affine open subcover $\{U_\alpha\}_{\alpha\in\lambda}$ of $U$. Well, for each $\alpha\in\lambda$, let $f_\alpha\coloneqq f|_{U_\alpha}$, and we see that $U_\alpha\subseteq U=V(f)$ implies that $f|_p\in\mf m_p$ for each $p\in U_\alpha$. Thus, using the canonical isomorphism $\varepsilon\colon(U,\OO_X|_U)\cong(\Spec\OO_X(U),\OO_{\Spec\OO_X(U)})$ of \autoref{cor:the-affine-iso}, we see
		\[\OO_X(U)_p\simeq\colimit_{U_h\ni p}\OO_X(U_h)\stackrel\varepsilon\simeq\colimit_{D(h)\ni\varepsilon(p)}\OO_X(U)_h\simeq\OO_X(U)_{\varepsilon(p)}\]
		where we have also used \autoref{lem:affinestalk}. Thus, we see $f/1\in\mf p\OO_X(U)_\mf p$ for each prime $\mf p\in\Spec\OO_X(U)$, so
		\[f\in\bigcap_{\mf p\in\Spec A}\mf p=\op{nilrad}A.\]
		But $\OO_X(U)$ is reduced as a ring because $(X,\OO_X)$ is a reduced scheme by \autoref{lem:reduced-on-sets}, so now forces $f=0$.
		\qedhere
	\end{itemize}
\end{proof}
We are now prepared to give some examples.
\begin{example}
	A reduced scheme whose stalks are integral domains even though $X$ is not irreducible will not make $X$ in total integral. Somehow being integral is a more global property, which the irreducibility tracks.
\end{example}
\begin{example}
	Projective space $\PP^n_k$ is integral: this is reduced by \autoref{ex:proj-is-reduced} and irreducible by \autoref{ex:proj-is-irred}.
\end{example}
\begin{example} \label{ex:affine-integral-scheme}
	Fix an integral domain $A$. Then $(\Spec A,\OO_{\Spec A})$ is reduced because $A$ is reduced (\autoref{ex:reduced-affine-scheme}), and $\Spec A$ is irreducible with generic point $(0)$ by using \autoref{ex:closurepoint}:
	\[\overline{\{(0)\}}=V((0))=\{\mf p\in\Spec A:0\in\mf p\}=\Spec A.\]
\end{example}
The above example inspires us into the following.
\begin{lemma} \label{lem:get-function-field}
	An integral scheme $(X,\OO_X)$ has a unique generic point $\xi$ for $X$. Then the following are true.
	\begin{itemize}
		\item If $(X,\OO_X)$ is affine, then $\OO_{X,\xi}\simeq\op{Frac}\OO_X(X)$.
		\item For any affine open subscheme $U\subseteq X$, we have $\OO_{X,\xi}\simeq\op{Frac}\OO_X(U)$. In particular, the stalk $\OO_{X,\xi}$ is a field.
	\end{itemize}
\end{lemma}
\begin{proof}
	Because $X$ is irreducible by \autoref{prop:integral-is-reduced-irred}, we see that the closed irreducible subset $X$ of $X$ has a unique generic point $\xi$ by \autoref{lem:uniq-generic}. We now show the remaining claims in sequence.
	\begin{itemize}
		\item Note that $(0)$ is the generic point of $\Spec\OO_X(X)$ as shown in \autoref{ex:affine-integral-scheme}, so we see that the canonical isomorphism of \autoref{cor:the-affine-iso} gives
		\[\OO_{X,\xi}\simeq\OO_{\Spec\OO_X(X),(0)}\simeq\OO_X(X)_{(0)}.\]
		Namely, the first isomorphism holds because $\xi$ must be taken to a point in $\Spec\OO_X(X)$ whose closure is the entire space, which is just $(0)$ by uniqueness of the generic point in \autoref{lem:uniq-generic}; the second isomorphism holds by \autoref{lem:affinestalk}. The claim now follows upon noticing $\op{Frac}\OO_X(X)=\op{Frac}\OO_X(X)$.
		\item Given an affine open subscheme $U\subseteq X$, we note that $U\cap\overline{\{\xi\}}\ne\emp$ forces $U\cap\{\xi\}\ne\emp$ by properties of the closure, so $\xi\in U$. Thus, we see
		\[\OO_{X,\xi}\simeq(\OO_X|_U)_\xi\simeq\op{Frac}\OO_X(U),\]
		where the first isomorphism is by \autoref{lem:invimagestalk}, and the second isomorphism is by the above.
		\qedhere
	\end{itemize}
\end{proof}
\begin{remark}
	In general, it is not true that $\OO_{X,\xi}$ is the fraction field of $\OO_X(U)$ for any open subset $U\subseteq X$. For example, take $X\coloneqq\PP^1_k$. To begin, recall $\OO_X(X)\simeq k$ from \autoref{rem:global-proj-sections}, so $\op{Frac}\OO_X(X)=k$. However, recalling we can build $\PP^1_k$ by gluing together $\Spec k[x]$s, we see that the function field should be $\op{Frac}k[x]=k(x)$ by \autoref{lem:get-function-field}.
\end{remark}
So we get the following nice definition.
\begin{definition}[Function field]
	An integral scheme $(X,\OO_X)$ with generic point $\xi$ has \textit{function field} $\OO_{X,\xi}$.
\end{definition}
The point here is that we can retrieve the field out from some $\Spec k[x,y]$, say. This also allows us to define regular functions.
\begin{definition}[Regular]
	Fix an integral scheme $(X,\OO_X)$ with generic point $\xi$. Then $f\in\OO_{X,\xi}$ is \textit{regular} at a point $x\in X$ if and only if $f$ lifts to $\OO_{X,x}$.
\end{definition}

\subsection{Closed Subschemes}
Open subschemes had natural subscheme structure by just taking restriction. Closed subschemes are a little harder.
\begin{example}
	Set $X\coloneqq\Spec k[x,y]$. Then the closed subset $V(x)$ will have lots of natural homeomorphisms
	\[V(x)\cong\Spec\frac{k[x,y]}{(x^n)},\]
	for any $n\ge1$, so there is no canonical way to set the structure sheaf.
\end{example}
The idea to define a closed subscheme is to instead keep track of the morphism which does the embedding.
\begin{definition}[Closed immersion]
	A scheme morphism $(f,f^\sharp)\colon(Z,\OO_Z)\to(X,\OO_X)$ is a \textit{closed immersion} if and only if the following two conditions hold.
	\begin{itemize}
		\item The map $f\colon Z\to X$ is a homeomorphism from $Z$ onto a closed subset of $X$.
		\item The map $f^\sharp\colon\OO_X\to f_*\OO_Z$ is epic.
	\end{itemize}
	If in fact $Z\subseteq X$ is a closed subset, then we will say $Z$ is a closed subscheme.
\end{definition}
\begin{remark}
	The geometric intuition behind requiring $f^\sharp\colon\OO_X\to f_*\OO_Z$ to be epic is that we want holomorphic functions on a closed subset $V\subseteq\CC$ to pull back to meromorphic ones on $\CC$. Locally, this means that a germ of a holomorphic function on $V$ should come from a germ on a holomorphic function on $X$, which is exactly what this sheaf morphism being epic means.
\end{remark}
The main point here is that we would like our closed immersions in the affine case to be induced by $A\onto A/I$ as in \autoref{exe:closedaffinesubscheme}.
\begin{proposition} \label{prop:affineclosedsubschemes}
	Fix an affine scheme $(X,\OO_X)\coloneqq(\Spec A,\OO_{\Spec A})$.
	\begin{listalph}
		\item Each ideal $I\subseteq A$ induces a closed immersion
		\[\Spec A/I\to\Spec A\]
		from the projection map $A\onto A/I$. In particular, this gives $V(I)\subseteq\Spec A$ the structure of a closed subscheme.
		\item The map of (a) provides a bijection between ideals of $A$ and closed subschemes of $\Spec A$.
	\end{listalph}
\end{proposition}
\begin{proof}
	Here we go.
	\begin{listalph}
		\item From \autoref{exe:closedaffinesubscheme}, we already have the natural homeomorphism $\Spec A/I\cong V(I)$. On the level of sheaves, we only need to check surjectivity at stalks, for which we look at the distinguished open base. Namely, at some $D(f)$, we are studying the map
		\[A_f=\OO_X(D(f))\to\OO_Z(D(f+I))=(A/I)_f\simeq A_f/I_f,\]
		which we can see is surjective here. Taking the direct limit shows that we remain surjective on the level of stalks.
		\item This proof will be able to simplified later in life when we have talked about coherent sheaves. Fix a closed subscheme $\iota\colon Z\to X$ with $\iota^\sharp\colon\OO_X\to\iota_*\OO_Z$. Now, define
		\[I_Z\coloneqq\ker(\iota^\sharp_X).\]
		Then one can show that $I_Z$ is the ideal we want, providing the inverse to (a). In particular, one can show that $Z$ is identified with $\Spec A/I_Z$ as schemes. Notably, there is an embedding $Z\into Y$ by first looking on the level of topological spaces and then carrying over to schemes. It remains to show that this is an isomorphism.
		\begin{itemize}
			\item We show that we have a homeomorphism of our topological spaces. Note that $Z$ is quasicompact, so we can write it as a finite union of affine open subschemes. Now, for $Z\subseteq V(s)$, we will show that $V(s)=Y$. We see $s\in B\coloneqq A/I_Z$, so note that $s\in\OO_X(U_i)$ will be nilpotent because of its definition. Looping over entire affine open cover forces $s^n=0$, so the injectivity of the open cover
			\[A/I_Z\into\OO_Z(Z)\]
			forces $s^N=0$ in $A/I_Y$, meaning $Y\subseteq V(s)$.
			\item We now need to show $\OO_Y\to\iota_*\OO_Z$. Surjectivity follows from the construction looking at the global sections. The injectivity follows by looking at stalks and making an argument similar to the above.
			\qedhere
		\end{itemize}
	\end{listalph}
\end{proof}
% GW, T3.42

\end{document}