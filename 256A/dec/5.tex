% !TEX root = ../notes.tex

\documentclass[../notes.tex]{subfiles}

\begin{document}

\section{December 5}

Today we talk about the Jacobian of a curve. All fields today will be algebraically closed.
% Arithmetic Geometry, Milne's articles

\subsection{Jacobian of a Curve}
Throughout, $C$ will be a smooth proper (equivalently, projective) $k$-curve of geometric genus $g>0$. Again, $k$ here is algebraically closed, though technically we only need $k$-points to be dense in $C$.\footnote{For example, all smooth $k$-varieties have that closed points $x\in X$ with $k(x)/k$ separable are dense, so technically we only need $k$ to be separably closed.}
\begin{definition}
	Fix a smooth proper curve $C$ over a field $k$. For a connected $k$-scheme $T$ such that $C(T)\ne\emp$, we define
	\[P^0_C\coloneqq\{\mc L\in\op{Pic}(C\times_kT):\deg\mc L_t=0\text{ for }t\in T\}/\pi^*\op{Pic}T,\]
	where $\pi\colon C\times_kT\to T$ is the canonical projection.
\end{definition}
\begin{remark}
	There is a way to allow $T$ to be disconnected or even have $C(T)=\emp$, but to say this precisely would take us too far afield.
\end{remark}
\begin{remark}
	The condition $\mc L_t=0$ only needs to be checked at a single $t\in T$ because the degree is locally constant. Roughly speaking, the reason is that the Euler characteristic is locally constant, and the degree is the difference between the Euler characteristic of $\mc L$ and the structure sheaf as we discussed in our proof of \autoref{thm:rr}.
\end{remark}
In fact, we can represent $P^0_C$ by an abelian variety.
\begin{theorem} \label{thm:get-jac}
	Fix a smooth proper curve $C$ over a field $k$. Then there is an abelian $k$-variety $J$ and a morphism $\iota\colon P^0_C\to J$ such that $i\colon P^0_C(T)\to J(T)$ is a scheme isomorphism for all connected $k$-schemes $T$ such that $C(T)\ne\emp$.
\end{theorem}
\begin{remark}
	More generally, over a base $S$, we can define $P_{X/S}(T)\coloneqq\op{Pic}(X\times_ST)$ so that there is a relative Picard functor $\op{Pic}_{X/S}$ as the ``fppf sheafification'' (i.e., faithfully flat finite presentation) of $P_{X/S}$. In this case, it is a theorem due to Grothendieck that a flat, finitely presented, projective $S$-scheme $X$ with reduced and geometrically irreducible fibers has $\op{Pic}_{X/S}$ represented by a separated $S$-scheme which is locally of finite presentation over $S$.
\end{remark}
To prove \autoref{thm:get-jac}, we will want the following definition.
\begin{definition}[Symmetric power]
	Fix a $k$-variety $X$. Then the \textit{$r$th symmetric power} is a $k$-variety $X^{(r)}$ equipped with a morphism $\pi\colon X^r\to X^{(r)}$ satisfying the following.
	\begin{itemize}
		\item As a topological space, $X^{(r)}$ is the quotient of $X^r$ under the action of $S_r$.
		\item Given an affine open $U\subseteq X$, the induced affine open $U^{(r)}\subseteq X^{(r)}$ (by taking the quotient of $U^r\subseteq X^r$) has $\OO_{X^{(r)}}(U^{(r)})=\OO_{X^r}(U^r)^{S_r}$.
	\end{itemize}
\end{definition}
\begin{example}
	In the case of $X=\Spec A$, then $X^{(r)}=\Spec\left(A^{\otimes r}\right)^{S_r}$. We can glue these together for the construction in the general case.
\end{example}
\begin{remark}
	One can check that $C^{(r)}$ is a smooth $k$-variety by hand, affine-locally.
\end{remark}
The idea behind defining we want to represent (effective) divisors, but a divisor does not care about the order of the points in its sum, so $C^{(r)}$ is a more suitable candidate than $C^r$. In fact, this works.
\begin{theorem} \label{thm:get-div}
	Fix a smooth proper $k$-curve $C$. Then $C^{(r)}$ represents the functor $\op{Div}^r_C$.
\end{theorem}
Wait, what is $\op{Div}^r_C$?
\begin{definition}[Relative effective Cartier divisor]
	A morphism $\pi\colon X\to T$ of $k$-schemes is a \textit{relative effective Cartier divisor} if and only if $\pi$ defines an effective Cartier divisor on $X$ which is flat over $T$ (viewed as a subscheme of $X$). This set of relative effective Cartier divisors of $C\times_kT$ over $T$ of degree $r$ is denoted $\op{Div}^r_C(T)$.
\end{definition}
We now prove \autoref{thm:get-div}.
\begin{proof}[{Proof of \autoref{thm:get-div}}]
	We begin with an easy case, where we are dealing with the relative effective Cartier divisor
	\[D=\sum_{j=1}^mn_js_j(T),\]
	where $s_j\colon T\to C\times T$ is a section of $\pi_T\colon C\times T\to T$. Well, to get $D$, we build our needed morphism $T\to C^{(r)}$ as through $T\to C^r\to C^{(r)}$, where the map $T\to C^r$ is by
	\[(\pi_Cs_1,\ldots,\pi_Cs_1,\ldots,\pi_Cs_m,\ldots,\pi_Cs_m).\]
	One can track everything through to see that this works, for example by building a canonical divisor $D_{\textrm{can}}$ to do the work of getting the needed map $T\to C^{(r)}$ by pullback.

	We now work out the general case by descent. Suppose we have a relative effective Cartier divisor $D$ given by $\varphi_D\colon C\times_kT\to T$ of degree $r$. Then we may fix a finite flat covering $\psi\colon T'\to T$ such that $\psi^*D$ is split as in the easy case above. Then we are looking at a split divisor for $T'$, which gives us $D$ after some descent argument. Explicitly, one wants to use the exactness of
	\[\op{Hom}(T,Z)\to\op{Hom}(T',Z)\rightrightarrows\op{Hom}(T'\times_TT',Z)\]
	to go backward.
\end{proof}
We are now ready to prove \autoref{thm:get-jac}.
\begin{proof}[{Proof of \autoref{thm:get-jac}}]
	Fixing some point $p\in C(k)$, there is an isomorphism $P^0_C(T)\simeq P^r_C(T)$ by sending $\mc L$ to $\mc L\otimes p^*\OO(r[P])$, so we only need to show this for $r$ sufficiently large. We choose $r\ge 2g+1$, where $g$ is the geometric genus.
	
	Note that sending divisors to line bundles gives a natural transformation $\op{Div}^r_C\Rightarrow P^r_C$. Explicitly, we have the following claim: given some $r-g$ total $k$-points $\mf r=(P_1,\ldots,P_{r-g})$ on $C$, we set $D_r$ to be their sum. Then we have three claims, whose proofs we omit for time.
	\begin{enumerate}
		\item There is an open subscheme $C^{\mf r}\subseteq C$ such that
		\[C^{\mf r}(T)=\{D\in\op{Div}^r_C:h^0(C,\OO_C(D_t-D_\mf r))=1\text{ for }t\in T\}\]
		for all $k$-schemes $T$. In fact, we will see that $C^{(r)}$ is covered by the $C^\mf r$.
		\item There is $P^\mf r\subseteq P^r_C$ such that
		\[P^\mf r=\{\mc L\in P^r_C(T):h^0(C,\mc L_t\otimes\OO(-D_\mf r))=1\text{ for }t\in T\}\]
		for all $k$-schemes $T$, where we have some section $f\colon C^\mf r\to P^\mf r$.
		\item Lastly, $P^\mf r$ is representable by a closed subvariety $J^\mf r$ of $C^\mf r$.
	\end{enumerate}
	This finishes the proof of \autoref{thm:get-jac} by gluing $J^\mf r$ over the various $\mf r$, and one can see by checking overlaps that it is what we want.
\end{proof}

\end{document}