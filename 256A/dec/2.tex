% !TEX root = ../notes.tex

\documentclass[../notes.tex]{subfiles}

\begin{document}

\section{December 2}

We discuss the Riemann--Roch theorem today, eliding details.
% L 7.3.2, V 19.4--5

\subsection{The Riemann--Roch Theorem}
Throughout, $X$ will be an integral smooth proper $k$-curve. Recall that, when $k$ is perfect, $X$ being smooth is equivalent to being regular or being normal.
\begin{remark}
	It turns out that all proper curves are projective; more generally all curves are quasiprojective. This will come up in the proof.
	% GW T15.18
\end{remark}
For our set-up, we let $D$ be a Weil divisor over $X$, or equivalently $\mc L$ is a line bundle over $X$. Concretely, we can let $\mc O(D)$ be the line bundle associated to $D$, which is an $\mathcal O_X$-submodule of $\underline{K(X)}$, and it has
\[H^0(X,\OO(D))=\Gamma(X,\OO(D))=\{f\in\mc\underline{K(X)}^\times:\op{div}f+D\ge0\}\cup\{0\}.\]
Intuitively, this is rational functions $f$ on $X$ such that the poles on $f$ have poles ``at worst'' prescribed by $D$.

We are now ready to state the Riemann--Roch theorem.
\begin{notation}
	Let $X$ will be an integral smooth proper $k$-curve, and let $D$ be a Weil divisor on $X$. Then we set $h^0(X,\OO(D))\coloneqq\dim_kH^0(X,\OO(D))$.
\end{notation}
\begin{theorem} \label{thm:rr}
	Let $X$ will be an integral smooth proper $k$-curve. For any Weil divisor $D$ on $X$, we have
	\[h^0(X,\OO(D))-h^0(X,\omega_X\otimes\OO(-d))=\deg D+1-g,\]
	where $\omega_X$ is the canonical sheaf associated to $X$, and $g$ is the geometric genus. In particular, all quantities are finite.
\end{theorem}
\begin{proof}[Sketch]
	We proceed in steps.
	\begin{enumerate}
		\item We argue that both $H^0(X,\OO(D))$ and $H^1(X,\OO(D))$ are finite-dimensional $k$-vector spaces. (Note they are $k$-modules because everything is over $k$; namely, $k=\Gamma(X,\OO_X)$.) As an example of difficulty, it is a result in Hartshorne \cite[Theorem~5.19]{hartshorne} that $H^0(X,\OO(D))$ is finite-dimensional whenever $X$ is projective.

		More generally, fix some affine open subscheme $U\subseteq X$. Choosing some non-constant regular function on $U$, then we can find some embedding $U\to\PP^1_k$, which because $X$ is proper extends to a morphism $X\to\PP^1_k$. By staring at being projective and some of our fibers, one can see that this map $f\colon X\to\PP^1_k$ is finite. It then turns out that
		\[H^0(X,\OO(D))=H^0(\PP^1_k,f_*\OO(D)),\]
		but now the right-hand side is finite-dimensional because we know how line bundles over $\PP^1_k$ behave. (The main point here is that $f_*\OO(D)$ is coherent!)

		We now turn to $H^1$. Using the same finite morphism $f$, we again have
		\[H^0(X,\OO(D))=H^1(\PP^1_k,f_*\OO(D)),\]
		so because $f_*\OO(D)$ is coherent, we can find a surjective map from $H^1(\PP^1_k,\OO(r_1)\oplus\cdots\OO(r_n))$ for some $n$ by using our constructed global morphisms from being ample. (One can see this explicitly by working affine-locally on $\AA^1_k$.) Lastly, we can check that $H^1(\PP^1_k,\OO(r_\bullet))$ are finite-dimensional by an explicit computation.

		\item Define the Euler characteristic $\chi(X,\OO(D))\coloneqq h^0(X,\OO(D))-h^1(X,\OO(D))$. We claim that
		\[\chi(X,\OO(D))\stackrel?=\deg D+\chi(X,\OO_X).\]
		(This proof will technically not use the fact that $X$ is smooth.) Pick up an effective divisor $E$ of $X$ so that $E+D$ is effective. This provides an exact sequence
		\[0\to\OO(D)\to\OO(D+E)\to\OO(D+E)|_E\to0.\]
		Then the long exact sequence in cohomology (plus a fact that cohomology above $H^1(X,-)$ vanishes) shows that
		\[\chi(\OO(D+E))=\chi(\OO(D))+\deg E\]
		after a computation---note we are sweeping under the rug how $\deg E$ enters the picture, but it is a computation from our restricted sheaf.

		To complete the proof, we repeat the above argument for the short exact sequence
		\[0\to\OO_X\to\OO(D+E)\to\OO(D+E)|_{D+E}\to0,\]
		which gives
		\[\chi(\OO(D+E))=\chi(\OO_X)\to\deg(D+E),\]
		which completes the proof.

		\item To finish the proof, we use Serre duality, which provides a perfect pairing
		\[H^0(X,\omega_X\otimes\OO(-D))\times H^1(X,\OO(D))\to H^1(X,\omega_X)\simeq k.\]
		For example, we can compute that
		\[\chi(X,\OO_X)=h^0(X,\OO_X)-h^1(X,\OO_X)=1-h^0(X,\omega_X)=1-g.\]
		More generally, we see
		\[h^0(X,\OO(D))-h^0(X,\omega_X\otimes\OO(-D))=h^0(X,\OO(D))-h^1(X,\OO(D))=\deg D+1-g,\]
		where the last equality is from the previous step.
		\qedhere
	\end{enumerate}
\end{proof}

\subsection{Applications}
Let's give some applications.
\begin{corollary}
	Let $X$ will be an integral smooth proper $k$-curve. Then $\deg\omega_X=2g-2$.
\end{corollary}
\begin{proof}
	Take $\OO(D)=\omega_X$. Then we see $g-1=\deg\omega_X+1-g$, so rearranging finishes.
\end{proof}
\begin{corollary}
	Let $X$ will be an integral smooth proper $k$-curve, and let $\mc L/X$ be a line bundle. If $\deg\mc L>2g-2$, then $h^0(X,\mc L)=\deg\mc L+1-g$.
\end{corollary}
\begin{proof}
	Note $\omega_X\otimes\mc L^\lor$ has negative degree, which forces $h^0(X,\omega_X\otimes\mc L^\lor)=0$. The result follows.
\end{proof}
\begin{corollary}
	Let $X$ will be an integral smooth proper $k$-curve, and let $\mc L/X$ be a line bundle. If $\deg\mc L>2g+1$, then $\mc L$ is very ample. For example, if $\deg\mc L>0$, then $\mc L$ is ample.
\end{corollary}
\begin{proof}
	We may assume that $k$ is algebraically closed. Note we get a rational map $X\dashrightarrow\PP^n_k$ using a basis of the global sections $\Gamma(X,\mc L)$, where $n\coloneqq\dim\Gamma(X,\mc L)$. We would like to get a closed embedding, which we may use a base-change for.

	Now that $k$ is algebraically closed, we appeal to \cite[Proposition~7.3]{hartshorne}, which implies that a map $\varphi\colon X\dashrightarrow\PP^n_k$ is a closed embedding if and only if the following are true.
	\begin{itemize}
		\item Separating points: for distinct $x,y\in X(k)$, there exists $s\in\Gamma(X,\mc L)$ such that $s\in\mf m_x\mc L_x$ while $s\notin\mc m_y\mc L_y$.
		\item Separating tangents: for $x\in X(k)$, the set $\{s\in\Gamma(X,\mc L):s_x\in\mf m_x\mc L_x\}$ spans the Zariski cotangent space $\mf m_x\mc L_x/\mf m_x^2\mc L_x$.
	\end{itemize}
	As such, we claim that having $x,y\in X(k)$ with
	\begin{equation}
		h^0(X,\mc L)-h^0(X,\mc L(-[x]-[y]))=2 \label{eq:small-almost-euler-charactistic}
	\end{equation}
	forces $\varphi$ to be a closed embedding. To see this, we note the exact sequence
	\[0\to\mc L([-x])\to\mc L\to\mc L|_x\to0\]
	grants an exact sequence
	\[0\to H^0(X,\mc L(-[x]))\to H^0(X,\mc L)\to k,\]
	so $h^0(X,\mc L(-[x]-[y]))-h^0(X,\mc L)=2$ forces $h^0(X,\mc L)-h^0(X,\mc L(-[x]))=1$ and $h^0(X,\mc L(-[x]))-h^0(X,\mc L(-[x]-[y]))=1$. Thus, we have $f$ vanishing at $x$ but not at $y$ and vice versa, so we can plug into the above criteria.

	To finish the proof of our result, one can see that $\deg\mc L>2g+1$ grants \autoref{eq:small-almost-euler-charactistic}, so we are now done.
\end{proof}
Let's now apply things to elliptic curves. Namely, we now enforce that the geometric genus equals $1$, and we are given a $k$-point $e\in X(k)$. Computing, we see
\begin{align*}
	h^0(X,\OO([e])) &= 1+1-1=1 \\
	h^0(X,\OO(2[e])) &= 2+1-1=2 \\
	h^0(X,\OO(3[e])) &= 3+1-1=3.
\end{align*}
In particular, we see that we have a non-constant function with a pole of order $2$ at $e$ and a different non-constant function with a pole of order $3$ at $e$. Now, we see that $\OO([3e])$ has degree large enough to enforce a closed embedding $E\into\PP^2_k$, and we note that it is a cubic in $\PP^3_k$ by some intersection arguments. Alternatively, we can compute $h^0(X,\OO_X(6[e]))$ has dimension $6$, which forces a relation among our letters
\[1,\quad x,\quad y,\quad x^2,\quad xy,\quad x^3,\quad y^2.\]
That's enough for today. Next time we discuss the Jacobian, quickly.

\end{document}