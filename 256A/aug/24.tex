% !TEX root = ../notes.tex

\documentclass[../notes.tex]{subfiles}

\begin{document}

A feeling of impending doom overtakes your soul.

\subsection{Administrative Notes}
Here are housekeeping notes.
\begin{itemize}
	\item Here are some housekeeping notes. There is a syllabus on \texttt{bCourses}.
	\item We hope to cover Chapter~II of \cite{hartshorne}, mostly, supplemented with examples from curves.
	\item There are lots of books.
	\begin{itemize}
		\item We use \cite{hartshorne} because it is short.
		\item There is also \cite{rising-sea}, which has more words.
		\item The book \cite{liu-alg-geo-ari} has notes on curves.
		\item There are more books in the syllabus. Professor Tang has some opinions on these.
	\end{itemize}
	\item Some proofs will be skipped in lecture. Not all of these will appear on homework.
	\item Some examples will say lots of words, some of which we won't have good definitions for until later. Do not be afraid of words.
\end{itemize}
Here are assignment notes.
\begin{itemize}
	\item Homework is 70\% of the class.
	\item Homework is due on noon on Fridays. There will be 6--8 problems, which means it is pretty heavy. The lowest homework score will be dropped.
	\item Office hours exist. Professor Tang also answers emails.
	\item The term paper covers the last 30\% of the grade. They are intended to be extra but interesting topics we don't cover in this class.
\end{itemize}

\subsection{Motivation}
We're going to talk about schemes. Why should we care about schemes? The point is that schemes are ``correct.''
\begin{example}
	In algebraic topology, there is a cup product map in homology, which is intended to algebraically measure intersections. However, intersections are hard to quantify when we aren't dealing with, say manifolds.
\end{example}
Here is an example of algebraic geometry helping us with this rigorization.
\begin{theorem}[B\'ezout]
	Let $C_1$ and $C_2$ be curves in $\PP^2(k)$, for some algebraically closed $k$, where $C_1$ and $C_2$ are defined by homogeneous polynomials $f_1$ and $f_2$. Then the ``intersection number'' between the curves $C_1$ and $C_2$ is $(\deg f_1)(\deg f_2)$.
\end{theorem}
This is a nice result, for example because it automatically accounts for multiplicities, which would be difficult to deal with directly using (say) geometric techniques. Schemes will help us with this.
\begin{example}
	Moduli spaces are intended to be geometric objects which represent a family of geometric objects of interest. For example, we might be interested in the moduli space of some class of elliptic curves.
\end{example}
It turns out that the correct way to define these objects is using schemes as a functor; we will see this in this class.
\begin{remark}
	One might be interested in when a functor is a scheme. We will not cover this question in this class in full, but it is an interesting question, and we will talk about this in special cases. 
\end{remark}

\subsection{Elliptic Curves}
For the last piece of motivation, let's talk about elliptic curves, over a field $k$.
\begin{definition}[Elliptic curve]
	An \textit{elliptic curve} over $k$ is a smooth projective curve of genus $1$, with a marked $k$-rational point.
\end{definition}
Remember that we said that we not to be afraid of words. However, we should have some notion of what these words mean: being a curve means that we are one-dimensional, being smooth is intuitive, and having genus $1$ roughly means that base-changing to a complex manifold has one hole. Lastly, the $k$-rational point requires defining a scheme as a functor.

Here's another (more concrete) definition of an elliptic curve.
\begin{definition}[Elliptic curve]
	An \textit{elliptic curve} over $k$ is an affine variety in $\AA^2(k)$ cut out be a polynomial of the form
	\[y^2+a_1xy+a_3y^2=x^3+a_2x^2+a_4x+a_6\]
	with nonzero discriminant plus a point $\mc O$ at infinity.
\end{definition}
\begin{remark}
	Why are these equivalent? Well, the Riemann--Roch theorem approximately lets us take a smooth projective curve of genus $1$ and then write it as an equation; the marked point goes to $\mc O$. In the reverse direction, one merely needs to embed our affine curve into projective space and verify its smoothness and genus.
\end{remark}
Instead of working with affine varieties, we can also give a concrete description of an elliptic curve using projective varieties.
\begin{definition}[Elliptic curve]
	An \textit{elliptic curve} over $k$ is a projective variety in $\PP^2(k)$ cut out be a polynomial of the form
	\[Y^2Z+a_1XYZ+a_3YZ^2=X^3+a_2X^2Z+a_4XZ^2+a_6Z^3\]
	with nonzero discriminant.
\end{definition}
We get the equivalence of the previous two definitions via the embedding $\AA_2(k)\into\PP^2(k)$ by $(x,y)\mapsto[x:y:1]$; the point at infinity $\mc O$ is $[0:1:0]$.

\subsection{Crackpot Varieties}
In order to motivate schemes, we should probably mention varieties, so we will spend some time in class discussing affine and projective varieties. For convenience, we work over an algebraically closed field $k$.
\begin{definition}[Affine variety]
	Given a field $k$, we define \textit{affine $n$-space over $k$}, denoted $\AA^n(k)$. An \textit{affine variety} is a subset $Y\subseteq\AA^n(k)$ of the form
	\[Y=Z(S)\coloneqq\left\{p\in\AA^n(k):f(p)=0\text{ for all }f\in S\right\},\]
	where $S\subseteq k[x_1,\ldots,x_n]$.
\end{definition}
\begin{remark}
	The set $S\subseteq k[x_1,\ldots,x_n]$ in the above definition need not be finite or countable. In certain cases, we can enforce this condition; for example, if $n=1$, then $k[x]$ is a principal ideal domain, so we may force $\#S=1$.
\end{remark}
Note that we have defined vanishing sets $Z(S)$ from subsets $S\subseteq k[x_1,\ldots,x_n]$. We can also go from vanishing sets to subsets.
\begin{definition}
	Fix a field $k$ and subset $Y\subseteq\AA^n(k)$. Then we define the ideal
	\[I(Y)\coloneqq\left\{f\in\AA^n(k):f(p)=0\text{ for all }p\in Y\right\}.\]
\end{definition}
\begin{remark}
	One should check that this is an ideal, but we won't bother.
\end{remark}
So we've defined some geometry. But we're in an algebraic geometry class; where's the algebra?
\begin{theorem}[Hilbert's Nullstellensatz]
	Fix an algebraically closed field $k$ and ideal $J\subseteq k[x_1,\ldots,x_n]$.  Then
	\[I(Z(J))=\rad I,\]
	where $\rad I$ is the radical of $I$.
\end{theorem}
\begin{remark}
	The Nullstellensatz has no particularly easy proof.
\end{remark}
The point of this result is that it ends up giving us a contravariant equivalence of posets of radical ideals and affine varieties.

Why do we care? In some sense, we prefer to work with ideals because it ``remembers'' more information than merely the points on the variety. To see this, note that elements $f\in k[x_1,\ldots,x_n]$ we are viewing as giving functions on $\AA^n(k)$. However, when we work on a variety $Y\subseteq\AA^n(k)$, then sometimes two functions will end up being identical on $Y$. So the correct ring of functions on $Y$ is
\[k[x_1,\ldots,x_n]/I(Y),\]
so indeed keeping track of the (algebraic) ideal $Z(Y)$ gets us some extra (geometric) information.

We will use this discussion as a jumping-off point to discuss affine schemes and then schemes. Affine schemes will have the following data.
\begin{itemize}
	\item A commutative ring $A$, which we should think of as the ring of functions on a variety.
	\item A topological space $\op{Spec}A$, which has more information than merely points on the variety.
	\item A structure sheaf of functions on $\op{Spec}A$.
\end{itemize}
\begin{remark}
	Our topological space $\op{Spec}A$ will contain more points than just the points on the variety. In some sense, these extra points make the topology more apparent.
\end{remark}
\begin{remark}
	Going forward, one might hope to remove requirements that the field $k$ is algebraically closed (e.g., to work with a general ring) or talk about ideals which are not radical. This is the point of scheme theory.
\end{remark}



\end{document}