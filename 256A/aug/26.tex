% !TEX root = ../notes.tex

\documentclass[../notes.tex]{subfiles}

\begin{document}

Let's finish up talking varieties, and then we'll move on to affine schemes.

\subsection{Projective Varieties}
We're going to briefly talk about projective varieties. Let's start with projective space.
\begin{definition}[Projective space]
	Given a field $k$, we define \textit{projective $n$-space over $k$}, denoted $\PP^n(k)$ as
	\[\frac{k^{n+1}\setminus\{(0,\ldots,0\}}{\sim},\]
	where $\sim$ assigns two points being equivalent if and only if they span the same $1$-dimensional subspace of $k^{n+1}$. We will denote the equivalence class of a point $(a_0,\ldots,a_n)$ by $[a_0:\ldots:a_n]$.
\end{definition}
To work with varieties, we don't quite cut out by general polynomials but rather by homogeneous polynomials.
\begin{definition}[Projective variety]
	Given a field $k$ and a set of some homogeneous polynomials $T\subseteq k[x_1,\ldots,x_n]$, we define the \textit{projective variety} cut out by $T$ as
	\[V(T)\coloneqq\left\{p\in\PP^n(k):f(p)=0\text{ for all }f\in T\right\}.\]
\end{definition}
\begin{example} \label{ex:projec}
	The elliptic curve corresponding to the affine algebraic variety in $\AA^2(k)$ cut out by $y^2-x^3-1$ becomes the projective variety in $\PP^2(k)$ cut out by
	\[Y^2Z-X^3-Z^3=0.\]
\end{example}
\begin{remark}
	One can give projective varieties some Zariski topology as well, which we will define later in the class.
\end{remark}
What to remember about projective varieties is that we can cover $\PP^2(k)$ (say) by affine spaces as
\begin{align*}
	\PP^2(k) ={}& \{[X:Y:Z]:X,Y,Z\in k\text{ not all }0\} \\
	={}& \{[X:Y:Z]:X,Y,Z\in k\text{ and }X\ne0\} \\
	{}&\cup \{[X:Y:Z]:X,Y,Z\in k\text{ and }Y\ne0\} \\
	={}& \{[1:y:z]:y,z\in k\} \\
	{}&\cup \{[x:1:z]:x,z\in k\} \\
	\simeq{}&\AA^2(k)\cup\AA^2(k).
\end{align*}
The point is that we can decompose $\PP^2(k)$ into an affine cover.
\begin{example}
	Continuing from \autoref{ex:projec}, we can decompose $Z\left(Y^2Z-X^3-Z^3\right)$ into having an affine open cover by
	\[\underbrace{\left\{(x,y):y^2-x^3-1=0\right\}}_{z\ne0}\cup\underbrace{\left\{(x,z):z-x^3-z^3=0\right\}}_{y\ne0}\cup\underbrace{\left\{(y,z):y^2z-1-z^3=0\right\}}_{x\ne0}.\]
	Notably, we get almost everything from just one of the affine chunks, and we get the point at infinity by taking one of the other chunks.
\end{example}
\begin{remark}
	It is a general fact that we only need two affine chunks to cover our projective curve.
\end{remark}

\subsection{The Spectrum}
The definition of a(n affine) scheme requires a topological space and its ring of functions. We will postpone talking about the ring of functions until we discuss sheaves, so for now we will focus on the space.
\begin{definition}[Spectrum]
	Given a ring $A$, we define the \textit{spectrum}
	\[\op{Spec}A\coloneqq\{\mf p\subseteq A:\mf p\text{ is a prime ideal}\}.\]
\end{definition}
\begin{example}
	Fix a field $k$. Then $\op{Spec}k=\{(0)\}$. Namely, non-isomorphic rings can have homeomorphic spectra.
\end{example}
\begin{exe}
	Fix a field $k$. We show that
	\[\op{Spec}k[x]=\{(0)\}\cup\{(\pi):\pi\text{ is monic, irred.},\deg\pi>0\}.\]
\end{exe}
\begin{proof}
	To begin, note that $(0)$ is prime, and $(\pi)$ is prime for irreducible non-constant polynomials $\pi$ because irreducible elements are prime in principal ideal domains. Additionally, we note that all the given primes are distinct: of course $(0)$ is distinct from any prime of the form $(\pi)$, but further, given monic non-constant irreducible polynomials $\alpha$ and $\beta$, having
	\[(\alpha)=(\beta)\]
	forces $\alpha=c\beta$ for some $c\in k[x]^\times$. But $k[x]^\times=k^\times$, so $c\in k^\times$, so $c=1$ is forced by comparing the leading coefficients of $\alpha$ and $\beta$.

	It remains to show that all prime ideals $\mf p\subseteq k[x]$ take the desired form. Well, $k[x]$ is a principal ideal domain, so we may write $\mf p=(\pi)$ for some $\pi\in k[x]$. If $\pi=0$, then we are done. Otherwise, $\deg\pi\ge0$, but $\deg\pi>0$ because $\deg\pi=0$ implies $\pi\in k[x]^\times$. By adjusting by a unit, we may also assume that $\pi$ is monic. And lastly, note that $(\pi)$ is prime means that $\pi$ is prime, so $\pi$ is irreducible.
\end{proof}
\begin{example}
	With an algebraically closed field $k$, the spectrum of $k[x]$ consists of our prime ideals; using that $k[x]$ is a principal ideal domain, these all look like $(\pi(x))$ for either $\pi=0$ or $\pi$ an irreducible polynomial, but because $k[x]$ is algebraically closed, we find that
	\[\op{Spec}A=\{(0)\}\cup\{(x-a):a\in k\}.\]
	Notably, that each maximal ideal $\mf m=(x-a)$ has a corresponding modulo map
	\[k[x]\onto\frac{k[x]}{\mf m}\simeq k,\]
	which really means ``evaluation at $a$.'' The ideal $(0)$ doesn't do anything interesting in its ``evaluation'' map.
\end{example}
This notion of having modulo being an evaluation will continue to be important.
\begin{ex}
	Similar to $k[x]$, we can classify $\op{Spec}\ZZ$: all ideals are principal, so our primes look like $(p)$ where $p=0$ or is a rational prime.
\end{ex}
\begin{example}
	With $k$ not algebraically closed, we can still classify $\op{Spec}k[x]$ as
	\begin{equation}
		\op{Spec}k[x]=\{(0)\}\cup\{(\pi):\pi\text{ is irreducible},\deg\pi>0\}. \label{eq:speckx}
	\end{equation}
	As always, one can spend the time to check this; the main point is that prime (for elements) is equivalent to irreducible in a principal ideal domain.
\end{example}
\begin{remark}
	One can make the correspondence of \autoref{eq:speckx} into a bijection by forcing $\pi$ to be monic. Namely, $(\pi)=(\pi')$ if and only if they differ by a constant in $k^\times$ because $k[x]^\times=k^\times$.
\end{remark}
\begin{example}
	Concretely, we can imagine the prime point $(\pi)\in\op{Spec}\QQ[x]$ as a Galois equivalence class of algebraic integers. (We are working with algebraic integers to ensure our polynomials are monic.)
\end{example}
Here is a harder example, which we won't really spend the time to elaborate on.
\begin{example}
	Let $k$ be algebraically closed. Then $\op{Spec}k[x,y]$ consists of $\{(0)\}$ and $\{(x-a,y-b):a,b\in k\}$ as usual, but we also get
	\[\{(\pi):\pi\in k[x,y]\text{ irreducible polynomials},\deg\pi>0\}.\]
	The main point is that $k[x,y]$ should have Krull dimension $2$; now, $(0)$ has dimension $2$ (it's the full plane), $(x-a,y-b)$ has dimension $0$ (they're points), and $(\pi)$ have dimension $1$ (they're curves). Proving that these are all the prime ideals requires some effort, and it is a special feature of our setting.
\end{example}
\begin{remark}
	It is remarkable that the number of equations we need to cut out a variety of dimension $d$ is $2-d$. This is not always true.
\end{remark}
The point is that we see working with our prime ideals allows us to realize some part of \autoref{rem:wantmorealggeo} by working with spectra.
\begin{definition}[Affine space]
	Given a ring $R$, we define \textit{affine $n$-space over $R$} as
	\[\AA^n_R\coloneqq\op{Spec}R[x_1,\ldots,x_n].\]
\end{definition}
\begin{example}
	We classify $\op{Spec}k[x]/\left(x^2\right)$. Notably, all prime ideals here must correspond to prime ideals of $k[x]$ containing $\left(x^2\right)$, which is only $(x)$. So $\op{Spec}k[x]/\left(x^2\right)$ has a single point.
\end{example}
\begin{remark}
	In some sense, $\op{Spec}k[x]/\left(x^2\right)$ will be able to let us talk about differential information algebraically: $x$ here in some sense is a very small object $x$ such that $x^2=0$. So we can study a ``function'' $f\in k[x]$ locally at a point $p$ by studying $f(p+x)$.
\end{remark}

\subsection{The Zariski Topology}
Thus far we've defined our space. Where's our topology.
\begin{definition}[Zariski topology] \label{defi:zariski}
	Fix a ring $A$. Then we define
	\[V(T)\coloneqq\{\mf p\in\op{Spec}A:T\subseteq\mf p\}\]
	These make the closed sets of the Zariski topology of $\op{Spec}A$; one can show directly that they make a topology.
\end{definition}
\begin{remark}
	Note that $V(T)=V((T))$, where $(T)$ is the ideal generated by $T\subseteq A$.
\end{remark}
\begin{example}
	We work with $\AA^n_k$. Then, given $f\in k[x_1,\ldots,x_n]$, we want to be convinced that $V(\{f\})$ matches up with the affine $k$-points $(a_1,\ldots,a_n)$ which vanish on $f$. Well, $(a_1,\ldots,a_n)$ corresponds to the ideal $(x_1-a_1,\ldots,x_n-a_n)$, and
	\[\{f\}\subseteq(x_1-a_1,\ldots,x_n-a_n)\]
	is equivalent to $f$ vanishing in the evaluation map
	\[\frac{k[x_1,\ldots,x_n]}{(x_1-a_1,\ldots,x_n-a_n)}\to k.\]
\end{example}
While we're here, let's also generalize \autoref{defi:vansihingideal}.
\begin{definition}
	Fix a ring $A$. Then, given a subset $Y\subseteq\op{Spec}A$, we define
	\[I(Y)\coloneqq\bigcap_{\mf p\in Y}\mf p.\]
\end{definition}
And here is our nice version of \autoref{thm:nullstellensatz}.
\begin{proposition}
	Fix a ring $A$ and an ideal $\mf a\subseteq A$ and a subset $Y\subseteq\op{Spec}A$.
	\begin{listalph}
		\item The ideal $I(Y)$ is radical.
		\item We have that
		\[I(V(\mf a))=\rad\mf a\qquad\text{and}\qquad V(I(Y))=\overline Y,\]
		where $\overline Y$ means the closure in the Zariski topology (i.e., the ``Zariski closure'').
		\item There is a one-to-one inclusion-reversing bijection between radical ideals of $A$ and closed subsets of $\op{Spec}A$.
	\end{listalph}
\end{proposition}
\begin{proof}
	Omitted.
\end{proof}
Here are some more quick remarks.
\begin{remark}
	A ring homomorphism $\varphi\colon A\to B$ will induce a continuous map
	\[\varphi^{-1}\colon\op{Spec}B\to\op{Spec}A.\]
	We will state this more concretely later.
\end{remark}
\begin{example}
	Fix a ring $A$ and ideal $\mf a\subseteq A$. Then the surjection $A\onto A/\mf a$ induces a natural homeomorphism
	\[\op{Spec}A/\mf a\cong V(\mf a)=\{\mf p:\mf p\supseteq\mf a\}\]
	because primes of $A/\mf a$ are approximately primes of $A$ containing $\mf a$. This gives a description of the closed sets of $\op{Spec}A$ as coming from other spectra.
\end{example}
\begin{example}
	A ring homomorphism $A\to k$, where $k$ is a field, induces a map from the single closed point of $\op{Spec}k$ to $\op{Spec}A$. We call the images of these maps the ``$k$-points.''
\end{example}
\begin{example}
	Given a ring $A$ and $f\in A$, the sets
	\[D(f)\coloneqq(\op{Spec}A)\setminus V(f)=\{\mf p:f\notin\mf p\}\]
	form a base of the open sets in $\op{Spec}A$. To see this, we can see directly that
	\[(\op{Spec}A)\setminus V(T)=\bigcup_{f\in T}D(f).\]
	To turn $D(f)$ into a spectrum, we have $D(f)\cong\op{Spec}A_f$, where $A_f$ refers to the localization of $A$ at $f^\NN$. Notably, the prime ideals of $\op{Spec}A_f$ are the primes of $A$ which are disjoint from $\{f\}$. Notably, the homeomorphism
	\[\op{Spec}A_f\cong D(f)\]
	is given by the localization map $A\to A_f$.
\end{example}
\begin{remark}
	Not every open set is of the form $D(f)$ or even $D(S)$ where $S$ is a multiplicative set. For example,
	\[\AA^2_k\setminus\{(0,0)\}\subseteq\AA^2_k\]
	is an open set.
\end{remark}


\end{document}