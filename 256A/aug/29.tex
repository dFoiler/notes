% !TEX root = ../notes.tex

\documentclass[../notes.tex]{subfiles}

\begin{document}

\section{August 29}

Today we talk about the structure sheaf. To review, so far we have defined the spectrum $\Spec A$ of a ring $A$ and given it a topology. The goal for today is to define its structure sheaf. Here is a motivating example.
\begin{example}
	Set $A\coloneqq\CC[x_1,\ldots,x_n]$ so that $\Spec A=\AA^n_k$. Recall that $\{D(f)\}_{f\in A}$ is a base for the Zariski topology, and we would like the functions on this ring to be $A_f$, the rational polynomials which allow some $f$ in the denominator. In other words, these are rational functions on $\CC^n$ whose poles are allowed on $V(\{f\})$ only.
\end{example}

\subsection{Sheaves}
Sheaves are largely a topological object, so we will forget that we are interested in the Zariski topology for now. Throughout, $X$ will be a topological space.
\begin{notation}
	Given a topological space $X$, we let $\op{Op}X$ denote the poset (category) of its open sets.
\end{notation}
Namely, the objects of $\op{Ob}X$ are open sets, and
\[\op{Mor}(V,U)=\begin{cases}
	\{*\} & V\subseteq U, \\
	\emp & \text{else}.
\end{cases}\]
Here is our definition.
\begin{definition}[Presheaf]
	A \textit{presheaf} $\mathcal F$ on a topological space $X$ valued in a category $\mathcal C$ is a contravariant functor $\mathcal F\colon(\op{Ob}X)\opp\to\mathcal C$. More concretely, $\mathcal F$ has the following data.
	\begin{itemize}
		\item Given an open set $U\subseteq X$, we have $\mathcal F(U)\in\mathcal C$.
		\item Given open sets $V\subseteq U\subseteq X$, we have a restriction map ${\op{res}_{U,V}}\colon\mathcal F(U)\to\mathcal F(V)$ in $\mathcal C$.
	\end{itemize}
	This data satisfies the following coherence conditions.
	\begin{itemize}
		\item Identity: given an open set $U\subseteq X$, ${\op{res}_{U,U}}=\id_{\mathcal F(U)}$.
		\item Functoriality: given open sets $W\subseteq V\subseteq U$, the following diagram commutes.
		% https://q.uiver.app/?q=WzAsMyxbMCwwLCJcXG1jIEYoVSkiXSxbMSwwLCJcXG1jIEYoVikiXSxbMSwxLCJcXG1jIEYoVykiXSxbMCwxLCJcXG9we3Jlc31fe1UsVn0iXSxbMSwyLCJcXG9we3Jlc31fe1YsV30iXSxbMCwyLCJcXG9we3Jlc31fe1UsV30iLDJdXQ==&macro_url=https%3A%2F%2Fraw.githubusercontent.com%2FdFoiler%2Fnotes%2Fmaster%2Fnir.tex
		\[\begin{tikzcd}
			{\mc F(U)} & {\mc F(V)} \\
			& {\mc F(W)}
			\arrow["{\op{res}_{U,V}}", from=1-1, to=1-2]
			\arrow["{\op{res}_{V,W}}", from=1-2, to=2-2]
			\arrow["{\op{res}_{U,W}}"', from=1-1, to=2-2]
		\end{tikzcd}\]
	\end{itemize}
\end{definition}
\begin{notation}
	We might call an element $f\in\mc F(U)$ a \textit{section over $U$}.
\end{notation}
As suggested by our language and notation, we should think about (pre)sheaves as mostly being ``sheaves of functions.'' We will see a few examples shortly.
\begin{notation}
	Given $f\in\mathcal F(U)$, we might write $f|_V\coloneqq\op{res}_{U,V}f.$
\end{notation}
\begin{remark}
	In principle, one can have any target category $\mathcal C$ for our presheaf. However, we will only work $\mathrm{Set}$, $\mathrm{Ab}$, $\mathrm{Ring}$, $\mathrm{Mod}_R$ in this class. In particular, we will readily assume that $\mc C$ is a concrete category.
\end{remark}
Now that we've defined an algebraic object, we should discuss its morphisms.
\begin{definition}[Presheaf morphism]
	Fix a topological space $X$. A \textit{presheaf morphism} between $\mathcal F$ and $\mathcal G$ is a natural transformation $\eta\colon\mathcal F\Rightarrow\mathcal G$. In other words, for each open set $U\subseteq X$, we have a morphism $\eta_U\colon\mathcal F(U)\to\mathcal F(V)$; these morphisms make the following diagram commute.
	% https://q.uiver.app/?q=WzAsNCxbMCwwLCJcXG1jIEYoVSkiXSxbMCwxLCJcXG1hdGhjYWwgRihWKSJdLFsxLDAsIlxcbWF0aGNhbCBHKFUpIl0sWzEsMSwiXFxtYXRoY2FsIEcoVikiXSxbMCwyLCJcXGV0YV9VIl0sWzEsMywiXFxldGFfViIsMl0sWzAsMSwiXFxvcHtyZXN9X3tVLFZ9IiwyXSxbMiwzLCJcXG9we3Jlc31fe1UsVn0iXV0=&macro_url=https%3A%2F%2Fraw.githubusercontent.com%2FdFoiler%2Fnotes%2Fmaster%2Fnir.tex
	\[\begin{tikzcd}
		{\mc F(U)} & {\mathcal G(U)} \\
		{\mathcal F(V)} & {\mathcal G(V)}
		\arrow["{\eta_U}", from=1-1, to=1-2]
		\arrow["{\eta_V}"', from=2-1, to=2-2]
		\arrow["{\op{res}_{U,V}}"', from=1-1, to=2-1]
		\arrow["{\op{res}_{U,V}}", from=1-2, to=2-2]
	\end{tikzcd}\]
\end{definition}
We've talked about presheaves a lot; where are sheaves?
\begin{definition}[Sheaf]
	Fix a topological space $X$. A presheaf $\mathcal F\colon(\op{Ob}X)\opp\to\mathcal C$ is a \textit{sheaf} if and only if it satisfies the following for any open set $U\subseteq X$ with an open cover $\mc U$.
	\begin{itemize}
		\item Identity: if $f_1,f_2\in\mc F(U)$ have $f_1|_V=f_2|_V$ for all $V\in\mc U$, then $f_1=f_2$.
		\item Gluability: if we have $f_V\in\mc F(V)$ for all $V\in\mc U$ such that
		\[f_{V_1}|_{V_1\cap V_2}=f_{V_2}|_{V_1\cap V_2}\]
		for all $V_1,V_2\in\mc U$, then there is $f\in\mc F(U)$ such that $f|_V=f_V$ for all $V\in\mc U$.
	\end{itemize}
\end{definition}
% \begin{remark}
% 	Note that the section $f\in\mc F(U)$ promised by the gluability axiom is unique by the identity axiom.
% \end{remark}
Ok, so we've defined the sheaf as an algebraic object, so here are its morphisms.
\begin{definition}[Sheaf morphism]
	A \textit{sheaf morphism} is a morphism of the (underlying) presheaves.
\end{definition}
Because there is an identity natural transformation and because the composition of natural transformations is a natural transformation, we see that we have the necessary data for a category $\mathrm{PreSh}_X$ of presheaves on $X$ and a category $\mathrm{Sh}_X$ of sheaves on $X$.

As an aside, we note that we can succinctly write the sheaf conditions in an exact sequence.
\begin{lemma}
	Fix a topological space $X$ and presheaf $\mc F\colon(\op{Ob}X)\opp\to\mathcal C$, where $\mathcal C$ is an abelian category or $\mathrm{Grp}$. Then $\mc F$ is a sheaf if and only if the sequence
	\begin{equation}
		\arraycolsep=1.4pt\begin{array}{cccccccccccc}
			0 &\to& \mc F(U) &\to& \displaystyle\prod_{V\in\mc U}\mc F(V) &\to& \displaystyle\prod_{V_1,V_2\in\mc U}\mc F(V_1\cap V_2) \\
			  &   & f        &\mapsto& (f|_V)_{V\in\mc U} \\
			  &   &          &       & (f_V)_{V\in\mc U} &\mapsto& (f_{V_1}|_{V_1\cap V_2}-f_{V_2}|_{V_1\cap V_2})_{V_1,V_2}
		\end{array} \label{eq:sheafles}
	\end{equation}
	is exact.
\end{lemma}
\begin{proof}
	In one direction, suppose that $\mc F$ is a sheaf, and we will show that \autoref{eq:sheafles} is exact for any open cover $\mc U$ of an open set $U$.
	\begin{itemize}
		\item Exact at $\mc F(U)$: suppose $f_1,f_2\in\mc F(U)$ have the same image in $\prod_{V\in\mc U}\mc F(V)$. This means that
		\[f_1|_V=f_2|_V\]
		for all $V\in\mc U$, so the identity axiom tells us that $f_1=f_2$.
		\item Exact at $\prod_{V\in\mc U}\mc F(V)$: of course any $f\in\mc F(U)$ goes to $(f|_V)_{V\in\mc U}$, which goes to
		\[f|_{V_1}|_{V_1\cap V_2}-f|_{V_2}|_{V_1\cap V_2}=f|_{V_1\cap V_2}-f|_{V_1\cap V_2}=0\in\prod_{V_1,V_2\in\mc U}\mc F(V_1\cap V_2)\]
		and therefore lives in the kernel. Conversely, suppose $(f_V)_{V\in\mc U}$ vanishes in $\prod_{V_1,V_2}\mc F(V_1\cap V_2)$. Rearranging, this means that
		\[f_{V_1}|_{V_1\cap V_2}=f_{V_2}|_{V_1\cap V_2},\]
		so the gluability axiom tells us that we can find $f\in\mc F(U)$ such that $f|_V=f_V$. This finishes.
	\end{itemize}
	Conversely, suppose that $\mc F$ makes \autoref{eq:sheafles} always exact, and we will show that $\mc F$ is a sheaf. Fix an open cover $\mc U$ of an open set $U$.
	\begin{itemize}
		\item Identity: suppose that $f_1,f_2\in\mc F(U)$ have $f_1|_V=f_2|_V$ for any $V\in\mc U$. This means that $f_1$ and $f_2$ have the same image in $\prod_{V\in\mc U}\mc F(V)$, so the exactness of \autoref{eq:sheafles} at $\mc F(U)$ enforces $f_1=f_2$.
		\item Gluability: suppose that we have $f_V\in\mc F(V)$ for each $V\in\mc U$ in such a way that $f_{V_1}|_{V_1\cap V_2}=f_{V_2}|_{V_1\cap V_2}$ for all $V_1,V_2\in\mc U$. Then the image of $(f_V)_{V\in\mc U}$ in $\prod_{V_1,V_2\in\mc U}\mc F(V_1\cap V_2)$ is
		\[(f_{V_1}|_{V_1\cap V_2}-f_{V_2}|_{V_1\cap V_2})_{V_1,V_2}=(0)_{V_1,V_2},\]
		so exactness of \autoref{eq:sheafles} forces there to be $f\in\mc F(U)$ such that $f|_V=f_V$ for each $V\in\mc U$. This finishes.
		\qedhere
	\end{itemize}
\end{proof}
\begin{remark}
	One might want to continue this left-exact sequence. To see this, we will have to talk about cohomology, which is a task for later in life.
\end{remark}

\subsection{Examples of Sheaves}
Sheaves of functions will be our key example here. Intuitively, any type of function which can be determined ``locally'' will form a sheaf; for example, here are continuous functions.
\begin{remark}
	For most of our examples, the identity axiom is easily satisfied: intuitively, the identity axiom says that two sections are equal if and only if they agree locally. However, gluability is usually the tricky one: it requires us to build a function from local behavior.
\end{remark}
\begin{exe} \label{exe:sheafex}
	Fix topological spaces $X$ and $Y$. For each $U\subseteq X$, let $\mc F(U)$ denote the set of continuous functions $f\colon U\to Y$, and equip these sets with the natural restriction maps. Then $\mc F$ is a sheaf.
\end{exe}
\begin{proof}
	To begin, here are the functoriality checks.
	\begin{itemize}
		\item Identity: for any $f\in\mc F(U)$, we have $f|_U=f$.
		\item Functoriality: if $W\subseteq V\subseteq U$, any $f\in\mc F(U)$ will have $(f|_V|_W)(w)=f(w)=(f|_W)(w)$ for any $w\in W$, so $f|_V|_W=f|_W$ follows.
	\end{itemize}
	Here are sheaf checks. Fix an open cover $\mc U$ of an open set $U\subseteq X$.
	\begin{itemize}
		\item Identity: suppose $f_1,f_2\in\mc U$ have $f_1|_V=f_2|_V$ for all $V\in\mc U$. Now, for all $x\in U$, we see $x\in U_x$ for some $U_x\in\mc U$, so
		\[f_1(x)=(f_1|_{U_x})(x)=(f_2|_{U_x})(x)=f_2(x),\]
		so $f_1=f_2$ follows.
		\item Gluability: suppose we have $f_V\in\mc F(V)$ for each $V\in\mc U$ such that $f_{V_1}|_{V_1\cap V_2}=f_{V_2}|_{V_1\cap V_2}$ for each $V_1,V_2\in\mc U$. Now, for each $x\in\mc U$, find $U_x\in\mc U$ with $x\in U_x$ and set
		\[f(x)\coloneqq f_{U_x}(x).\]
		Note this is well-defined: if $x\in U_x$ and $x\in U_{x'}$, then $f_{U_x}(x)=f_{U_x}|_{U_x\cap U_x'}(x)=f_{U_x'}|_{U_x\cap U_x'}(x)=f_{U_x'}(x)$. Additionally, we see that, for each $V\in\mc U$ and $x\in V$, we have
		\[f|_V(x)=f(x)=f_V(x)\]
		by construction, so we are done.

		Lastly, we need to check that $f$ is continuous. Well, for any open set $V_0\subseteq Y$, we can compute
		\[f^{-1}(V_0)=\{x\in U:f(x)\in V_0\}=\bigcup_{V\in\mc U}\{x\in V:f(x)\in V_0\}=\bigcup_{V\in\mc U}\{x\in V:f_V(x)\in V_0\}=\bigcup_{V\in\mc U}f_V^{-1}(V_0),\]
		which is open as the arbitrary union of open sets because $f_V\colon V\to Y$ is a continuous function.
		\qedhere
	\end{itemize}
\end{proof}
Another key geometric example going forward will be the following.
\begin{exe} \label{exe:diffgeo}
	Set $X\coloneqq\CC$. For each open $U\subseteq X$, let $\mc O_X(U)$ denote the set of holomorphic functions $U\to\CC$, and equip these sets with the natural restriction maps. Then $\mc O_X$ is a sheaf.
\end{exe}
\begin{proof}
	Again, the point here is that being differentiable can be checked locally. Anyway, we note that our presheaf checks are exactly the same as in \autoref{exe:sheafex}, as is the check of the sheaf identity axiom.

	The gluability axiom is also mostly the same. Given an open cover $\mc U$ of an open set $U\subseteq X$, pick up $f_V\in\mc F(V)$ for each $V\in\mc U$ such that
	\[f_{V_1}|_{V_1\cap V_2}=f_{V_2}|_{V_1\cap V_2}.\]
	As before, we note that each $x\in U$ has some $U_x\in\mc U$ containing $x$, so we may define $f\colon U\to\CC$ by $f(x)\coloneqq f_{U_x}(x)$. The arguments of \autoref{exe:sheafex} tell us that this function $f$ is well-defined and has $f|_V=f_V$ for each $V\in\mc U$.

	It remains to check that $f$ is actually holomorphic. This requires that, for each $x\in X$, the limit
	\[\lim_{x'\to x}\frac{f(x)-f(x')}{x-x'}.\]
	However, this limit can be computed locally for $x'\in U_{x}$ because $U_{x}$ contains an open neighborhood around $x$. As such, it suffices to show that the limit
	\[\lim_{x'\to x}\frac{f|_{U_x}(x)-f|_{U_x}(x')}{x-x'}=\lim_{x'\to x}\frac{f_{U_x}(x)-f_{U_x}(x')}{x-x'}\]
	exists, which is true because $f_{U_x}\in\mc F(U_x)$ is holomorphic.
\end{proof}
In contrast, sheaves have trouble keeping track of ``global'' information.
\begin{example}
	For each $U\subseteq\RR$, let $\mc F(\RR)$ denote the set of bounded continuous functions $f\colon\RR\to\RR$, and equip these sets with the natural restriction maps. Then $\mc F$ is not a sheaf: for each open set $(n-1,n+1)$ for $n\in\NN$, the function $f_{(n-1,n+1)}\coloneqq\id_{(n-1,n=1)}$ is bounded and continuous, but the glued function $f=\id_\RR$ is not bounded on all of $\RR$. (We glued using \autoref{exe:sheafex}, which does force the definition of $f$.)
\end{example}

\subsection{Sheaf on a Base}
In light of our sheaf language, we are trying to define a ``structure'' sheaf $\mc O_{\Spec A}$ on $\Spec A$, and we wanted to have
\[\mc O_{\Spec A}(D(f))=A_f.\]
We aren't going to be able to specify a presheaf with this data, but we can specify a sheaf. In some sense, the presheaf is unable to build up locally in the way that a sheaf can, so having the data on a base like $\{D(f)\}_{f\in A}$ need not be sufficient to define the full presheaf.

But as alluded to, we can do this for sheaves. We begin by defining a sheaf on a base.
\begin{definition}[Sheaf on a base]
	Fix a topological space $X$ and a base $\mc B$ for its topology. Then a \textit{sheaf on a base valued in $\mc C$} is a contravariant functor $F\colon\mc B\opp\to\mathcal C$ satisfying the following identity and gluability axioms: for any $B\in\mc B$ with a basic cover $\{B_i\}_{i\in I}$, we have the following.
	\begin{itemize}
		\item Identity: if we have $f_1,f_2\in F(B)$ such that $f_1|_{B_i}=f_2|_{B_i}$ for all $B_i$, then $f_1=f_2$.
		\item Gluability: if we have $f_i\in F(B_i)$ for each $i$ such that $f_i|_{B}=f_j|_{B}$ for each $B\subseteq B_i\cap B_j$, then there is $f\in F(B)$ such that $f|_{B_i}=f_i$ for each $i$.
	\end{itemize}
\end{definition}
\begin{example} \label{ex:forgetsheaf}
	Given a topological space $X$ and a base $\mc B$, any sheaf $\mc F\colon(\op{Op}X)\opp\to\mathcal C$ ``restricts'' to a sheaf on a base $\mc F_\mc B$ by setting $\mc F_\mc B(B)\coloneqq\mc F(B)$ for all $B\in\mc B$ and reusing the same restriction maps. The identity and gluability axioms follow from their (stronger) sheaf counterparts; checking this amounts writing down the axioms.
\end{example}
Morphisms are constructed in the obvious way.
\begin{definition}[Sheaf on a base morphisms]
	Fix a topological space $X$ and a base $\mc B$ for its topology. Then a \textit{morphism} between two sheaves $F$ and $G$ on the base $\mc B$ is a natural transformation of the (underlying) contravariant functors.
\end{definition}
\begin{ex} \label{ex:forgetsheafmorphism}
	Given a topological space $X$ and a base $\mc B$, any sheaf morphism $\eta\colon\mc F\to\mc G$ restricts in the obvious way to a morphism $\eta_\mc B\colon\mc F_\mc B\to\mc G_\mc B$ (namely, $(\eta_\mc B)_B=\eta_B$) on the corresponding sheaves on a base. Checking this amounts to saying out loud that the diagram on the left commutes for any $B'\subseteq B$ because it is the same as the diagram on the right.
	% https://q.uiver.app/?q=WzAsOCxbMCwwLCJcXG1jIEZfXFxtYyBCKEIpIl0sWzAsMSwiXFxtYyBGX1xcbWMgQihCJykiXSxbMSwwLCJcXG1jIEdfXFxtYyBCKEIpIl0sWzEsMSwiXFxtYyBHX1xcbWMgQihCJykiXSxbMywwLCJcXG1jIEYoQikiXSxbMywxLCJcXG1jIEYoQicpIl0sWzQsMCwiXFxtYyBHKEIpIl0sWzQsMSwiXFxtYyBHKEInKSJdLFswLDEsIlxcb3B7cmVzfV97QixCJ30iLDJdLFsyLDMsIlxcb3B7cmVzfV97QixCJ30iXSxbMCwyLCIoXFxldGFfXFxtYyBCKV9CIl0sWzEsMywiKFxcZXRhX1xcbWMgQilfe0InfSIsMl0sWzQsNiwiXFxldGFfQiJdLFs1LDcsIlxcZXRhX3tCJ30iLDJdLFs0LDUsIlxcb3B7cmVzfV97QixCJ30iLDJdLFs2LDcsIlxcb3B7cmVzfV97QixCJ30iXV0=&macro_url=https%3A%2F%2Fraw.githubusercontent.com%2FdFoiler%2Fnotes%2Fmaster%2Fnir.tex
	\[\begin{tikzcd}
		{\mc F_\mc B(B)} & {\mc G_\mc B(B)} && {\mc F(B)} & {\mc G(B)} \\
		{\mc F_\mc B(B')} & {\mc G_\mc B(B')} && {\mc F(B')} & {\mc G(B')}
		\arrow["{\op{res}_{B,B'}}"', from=1-1, to=2-1]
		\arrow["{\op{res}_{B,B'}}", from=1-2, to=2-2]
		\arrow["{(\eta_\mc B)_B}", from=1-1, to=1-2]
		\arrow["{(\eta_\mc B)_{B'}}"', from=2-1, to=2-2]
		\arrow["{\eta_B}", from=1-4, to=1-5]
		\arrow["{\eta_{B'}}"', from=2-4, to=2-5]
		\arrow["{\op{res}_{B,B'}}"', from=1-4, to=2-4]
		\arrow["{\op{res}_{B,B'}}", from=1-5, to=2-5]
	\end{tikzcd}\]
\end{ex}
\begin{remark}
	\autoref{ex:forgetsheaf} and \autoref{ex:forgetsheafmorphism} combine into the data of a forgetful functor $(-)_\mc B$ from sheaves on $X$ to sheaves on a base $\mc B$. Here are the last two checks.
	\begin{itemize}
		\item Identity: given a sheaf $\mc F$ on $X$, note $({\id_{\mc F}})_\mc B\colon\mc F_{\mc B}\to\mc F_{\mc B}$ sends $s\in\mc F_{\mc B}(B)=\mc F(B)$ to itself, so $({\id_{\mc F}})_\mc B=\id{\mc F_{\mc B}}$.
		\item Functoriality: given sheaf morphisms $\varphi\colon\mc F\to\mc G$ and $\psi\colon\mc G\to\mc H$ and some $B\in\mc B$, we see
		\[(\varphi\circ\psi)_{\mc B}(B)=(\varphi\circ\psi)_B=\varphi_B\circ\psi_B=(\varphi_\mc B\circ\psi_\mc B)(B).\]
	\end{itemize}
\end{remark}
We are interested in showing that we can build a sheaf from a sheaf on a base uniquely, but it will turn out to be fruitful to spend a moment to discuss how this behaves on morphisms first for the uniqueness part of this statement.
\begin{lemma} \label{lem:morphismonbase}
	Fix a topological space $X$ with a base $\mc B$ for its topology. Given sheaves $\mc F$ and $\mc G$ on $X$ with values in $\mc C$ and a morphism of the (underlying) sheaves on a base $\eta_\mc B\colon\mc F_\mc B\to\mc G_\mc B$, there is a unique sheaf morphism $\eta\colon\mc F\to\mc G$ such that $(\eta_\mc B)_B=\eta_B$ for each $B\in\mc B$.
\end{lemma}
\begin{proof}
	We show uniqueness before existence.
	\begin{itemize}
		\item Uniqueness: fix any open $U\subseteq X$, and we will try to solve for $\eta_U\colon\mc F(U)\to\mc G(U)$. Well, fix a basic open cover $\mc U$ of $U$; then, for any $B\in\mc U$, we need the following diagram to commute.
		% https://q.uiver.app/?q=WzAsNCxbMCwwLCJcXG1jIEYoVSkiXSxbMSwwLCJcXG1jIEcoVSkiXSxbMCwxLCJcXG1jIEYoQikiXSxbMSwxLCJcXG1jIEcoQikiXSxbMiwzLCJcXGV0YV9CPShcXGV0YV9cXG1jIEIpX0IiLDJdLFswLDEsIlxcZXRhX1UiXSxbMCwyLCJcXG9we3Jlc31fe1UsQn0iLDJdLFsxLDMsIlxcb3B7cmVzfV97VSxCfSJdXQ==&macro_url=https%3A%2F%2Fraw.githubusercontent.com%2FdFoiler%2Fnotes%2Fmaster%2Fnir.tex
		\[\begin{tikzcd}
			{\mc F(U)} & {\mc G(U)} \\
			{\mc F(B)} & {\mc G(B)}
			\arrow["{\eta_B=(\eta_\mc B)_B}"', from=2-1, to=2-2]
			\arrow["{\eta_U}", from=1-1, to=1-2]
			\arrow["{\op{res}_{U,B}}"', from=1-1, to=2-1]
			\arrow["{\op{res}_{U,B}}", from=1-2, to=2-2]
		\end{tikzcd}\]
		In particular, for any $f\in\mc F(U)$, we need $\eta_U(f)|_B=(\eta_\mc B)_B(f|_B)$. Thus, $\eta_U(f)|_B$ is fully specified by the data provided by $\eta_\mc B$, so the identity axiom for $\mc G$ forces $\eta_U(f)$ to be unique.
		\item Existence: to begin, fix any open $U\subseteq X$, and we will define $\eta_U\colon\mc F(U)\to\mc G(U)$. As alluded to above, we let $\mc U$ be the set of basis elements which are contained in $U$ so that $\mc U$ is a (large) basic cover of $U$.
		
		Then, picking up $f\in\mc F(U)$, we will try to use the gluability axiom by setting $g_B\coloneqq(\eta_\mc B)_B(f|_B)$ for each $B\in\mc U$. In particular, for any $B,B'\in\mc U$, any basic $B_0\subseteq B\cap B'$ has
		\[(g_B|_{B\cap B'})|_{B_0}=g_B|_{B_0}=(\eta_\mc B)_B(f|_B)|_{B_0}=(\eta_{\mc B})_{B_0}(f|_B|_{B_0})=\eta_{B_0}(f|_{B_0})=g_{B_0},\]
		which is also $(g_{B'}|_{B\cap B'})_{B_0}$ by symmetry, so the identity axiom applied to $B\cap B'$ implies $g_B|_{B\cap B'}=g_{B'}|_{B\cap B'}$. Thus, the gluability axiom applied to $U$ gives us a unique $g\in\mc G(U)$ such that
		\[g|_B=(\eta_\mc B)_B(f|_B)\]
		for each basic set $B\subseteq U$. We define $\eta_U(f)\coloneqq g$.

		It remains to show that $\eta$ does in fact assemble into a sheaf morphism. Fix open sets $V\subseteq U$, and we need the following diagram to commute.
		% https://q.uiver.app/?q=WzAsNCxbMCwwLCJcXG1jIEYoVSkiXSxbMSwwLCJcXG1jIEcoVSkiXSxbMCwxLCJcXG1jIEYoVikiXSxbMSwxLCJcXG1jIEcoVikiXSxbMiwzLCJcXGV0YV9WIiwyXSxbMCwxLCJcXGV0YV9VIl0sWzAsMiwiXFxvcHtyZXN9X3tVLFZ9IiwyXSxbMSwzLCJcXG9we3Jlc31fe1UsVn0iXV0=&macro_url=https%3A%2F%2Fraw.githubusercontent.com%2FdFoiler%2Fnotes%2Fmaster%2Fnir.tex
		\[\begin{tikzcd}
			{\mc F(U)} & {\mc G(U)} \\
			{\mc F(V)} & {\mc G(V)}
			\arrow["{\eta_V}"', from=2-1, to=2-2]
			\arrow["{\eta_U}", from=1-1, to=1-2]
			\arrow["{\op{res}_{U,V}}"', from=1-1, to=2-1]
			\arrow["{\op{res}_{U,V}}", from=1-2, to=2-2]
		\end{tikzcd}\]
		Well, pick up any $f\in\mc F(U)$. Then, for any basic $B\subseteq V\subseteq U$, we see that
		\[\eta_U(f)|_B=(\eta_\mc B)_B(f|_B)=(\eta_\mc B)_B(f|_V|_B),\]
		so the uniqueness of $\eta_V(f|_V)$ forces $\eta_V(f|_V)=\eta_U(f)|_U$. This finishes.
		\qedhere
	\end{itemize}
\end{proof}

\subsection{Extending a Sheaf on a Base}
We dedicate this subsection to the following result, describing how to extend a sheaf on a base to a full sheaf.
\begin{proposition} \label{prop:sheaffrombase}
	Fix a topological space $X$ with a base $\mc B$ for its topology. Given a sheaf on a base $F\colon\mc B\opp\to\mc C$, there is a sheaf $\mc F$ and isomorphism (of sheaves on a base) $\iota\colon F\to\mc F_B$ satisfying the following universal property: any sheaf $\mc G$ with a morphism (of sheaves on a base) $\varphi\colon F\to\mc G_\mc B$ has a unique sheaf morphism $\psi\colon\mc F\to\mc G$ making the following diagram commute.
	% https://q.uiver.app/?q=WzAsMyxbMCwwLCJGIl0sWzEsMCwiXFxtYyBGX1xcbWMgQiJdLFsxLDEsIlxcbWMgR19cXG1jIEIiXSxbMCwxLCJcXGlvdGEiXSxbMCwyLCJcXHZhcnBoaSIsMl0sWzEsMiwiXFxwc2kiXV0=&macro_url=https%3A%2F%2Fraw.githubusercontent.com%2FdFoiler%2Fnotes%2Fmaster%2Fnir.tex
	\begin{equation}
		\begin{tikzcd}
			F & {\mc F_\mc B} \\
			& {\mc G_\mc B}
			\arrow["\iota", from=1-1, to=1-2]
			\arrow["\varphi"', from=1-1, to=2-2]
			\arrow["\psi_\mc B", from=1-2, to=2-2]
		\end{tikzcd} \label{eq:sheaffrombaseup}
	\end{equation}
\end{proposition}
\begin{proof}
	We begin by providing a construction of $\mc F$. For each open set $U\subseteq X$, define
	\[\mc F(U)\coloneqq\limit_{B\subseteq U}F(B)=\Bigg\{(f_B)_{B\subseteq U}\in\prod_{B\subseteq U}F(B):f_B|_{B'}=f_{B'}\text{ for each }B'\subseteq B\subseteq U\Bigg\}.\]
	(Namely, we are implicitly assuming that our target category has limits.) Observe that, when $V\subseteq U$, the natural surjection
	\[\prod_{B\subseteq U}\mc F(B)\to\prod_{B\subseteq V}\mc F(B)\]
	induces a map $\mc F(U)\to\mc F(V)$. Indeed, an element $(f_B)_{B\subseteq U}\in\mc F(U)$ gets sent to $(f_B)_{B\subseteq V}$, and it is still the case that $B'\subseteq B\subseteq V$ implies $f_B|_{B'}=f_{B'}$ because actually $B'\subseteq B\subseteq U$. Thus, $(f_B)_{B\subseteq V}\in\mc F(V)$, so we have a well-defined map
	\[\arraycolsep=1.4pt\begin{array}{cccc}
		{\op{res}_{U,V}}\colon & \mc F(U) &\to& \mc F(V) \\
		& (f_B)_{B\subseteq U} &\mapsto& (f_B)_{B\subseteq V}
	\end{array}\]
	which will serve as our restrictions. We start by checking that these data assemble into a presheaf.
	\begin{itemize}
		\item When $U=V$, we are sending $(f_B)_{B\subseteq U}\in\mc F(U)$ to itself, so ${\op{res}_{U,U}}=\id_{\mc F(I)}$.
		\item Given $W\subseteq V\subseteq U$, the diagram
		% https://q.uiver.app/?q=WzAsNixbMCwwLCJcXG1jIEYoVSkiXSxbMSwwLCJcXG1jIEYoVikiXSxbMSwxLCJcXG1jIEYoVykiXSxbMywwLCIoZl9CKV97Qlxcc3Vic2V0ZXEgVX0iXSxbNCwwLCIoZl9CKV97Qlxcc3Vic2V0ZXEgVn0iXSxbNCwxLCIoZl9CKV97Qlxcc3Vic2V0ZXEgV30iXSxbMCwxLCJcXG9we3Jlc31fe1UsVn0iXSxbMSwyLCJcXG9we3Jlc31fe1YsV30iXSxbMCwyLCJcXG9we3Jlc31fe1UsV30iLDJdLFszLDQsIiIsMix7InN0eWxlIjp7InRhaWwiOnsibmFtZSI6Im1hcHMgdG8ifX19XSxbNCw1LCIiLDIseyJzdHlsZSI6eyJ0YWlsIjp7Im5hbWUiOiJtYXBzIHRvIn19fV0sWzMsNSwiIiwwLHsic3R5bGUiOnsidGFpbCI6eyJuYW1lIjoibWFwcyB0byJ9fX1dXQ==&macro_url=https%3A%2F%2Fraw.githubusercontent.com%2FdFoiler%2Fnotes%2Fmaster%2Fnir.tex
		\[\begin{tikzcd}
			{\mc F(U)} & {\mc F(V)} && {(f_B)_{B\subseteq U}} & {(f_B)_{B\subseteq V}} \\
			& {\mc F(W)} &&& {(f_B)_{B\subseteq W}}
			\arrow["{\op{res}_{U,V}}", from=1-1, to=1-2]
			\arrow["{\op{res}_{V,W}}", from=1-2, to=2-2]
			\arrow["{\op{res}_{U,W}}"', from=1-1, to=2-2]
			\arrow[maps to, from=1-4, to=1-5]
			\arrow[maps to, from=1-5, to=2-5]
			\arrow[maps to, from=1-4, to=2-5]
		\end{tikzcd}\]
		commutes, which is our functoriality check.
	\end{itemize}
	We now show that these data make a sheaf. Fix an open set $U\subseteq X$ with an open cover $\mc U$. To help our constructions, given any open subset $V\subseteq X$, let $\mc B_V$ denote the collection of basis elements $B$ contained in $V$; notably $\mc B_V$ is a basic cover for $V$. Then, for any open $U'\subseteq U$, we let
	\[\mc S_{U'}\coloneqq\bigcup_{V\subseteq U}\mc S_{U'\cap V}.\]
	Notably, $\mc S_{U'}$ is a basic cover for $U'$ such that any $B\in\mc S_{U'}$ is contained in some element of $\mc U$.
	\begin{itemize}
		\item Identity: suppose that $(f_B)_{B\subseteq U},(g_B)_{B\subseteq U}\in\mc F(U)$ restrict to the same element on any $V\in\mc U$. Now, fix any $B_0\subseteq U$, and we will show $f_{B_0}=g_{B_0}$.
		
		Now consider $\mc S_{B_0}$: for each $B'\in\mc S$, we can find $V\in\mc U$ so that $B'\subseteq V$, for which we know
		\[(f_B)_{B\subseteq V}=(g_B)_{B\subseteq V}.\]
		In particular $f_{B_0}|_{B'}=f_{B'}=g_{B'}=g_{B_0}|_{B'}$ for any $B\in\mc S$, so the identity axiom for the sheaf on a base $F$ forces $f_{B_0}=g_{B_0}$.
		\item Gluability: suppose we are given some $(f_{V,B})_{B\subseteq V}\in\mc F(V)$ for each $V\in\mc U$ such that
		\[(f_{V,B})_{B\subseteq V\cap V'}=(f_{V,B})_{B\subseteq V}|_{V\cap V'}=(f_{V',B})_{B\subseteq V}|_{V\cap V'}=(f_{V',B})_{B\subseteq V\cap V'}\]
		for any $V,V'\in\mc U$. In other words, for any basic $B\subseteq V\cap V'$, we have $f_{V,B}=f_{V',B}$.

		Now, for any basic $B_0\subseteq U$, we will solve for $f_{B_0}$. Using $\mc S_{B_0}$, note that any $B\in\mc S_{B_0}$ has some $V_B\in\mc U$ such that $B\subseteq V_B$, so we will use $f_{V_B,B}$ at this point. Note that if $B\subseteq V'_B$ as well, then $f_{V_B,B}=f_{V'_B,B}$, so our $f_{V_B,B}$ is independent of $V_B$. Continuing, if we have $B\subseteq B_1\cap B_2$, then
		\[f_{V_{B_1},B_1}|_{B}=f_{V_{B_1},B}=f_{V_{B_2},B}=f_{V_{B_2},B_2}|_{B},\]
		so gluability applied to our sheaf $F$ on a base promises us a unique $f_{B_0}$ such that $f_{B_0}|_B=f_{V_B,B}$ for any $B\in\mc S_{B_0}$.

		We now need to show that the $(f_B)_{B\subseteq U}$ assemble into an element of $\mc F(U)$. Namely, if we have $B'_0\subseteq B_0$, we need to show that $f_{B_0}|_{B'_0}=f_{B'_0}$. Well, for any $B\in\mc S_{B_0'}$, we compute
		\[f_{B_0}|_{B_0'}|_B=f_{B_0}|_B=f_{V_B,B}=f_{B_0}|_B,\]
		so the uniqueness of $f_{B_0}$ gives the equality.
	\end{itemize}
	For our next step, we define $\iota_{B_0}\colon F(B)\to\mc F_\mc B(B_0)$ by
	\[\iota_{B_0}(f)\coloneqq(f|_B)_{B\subseteq B_0}.\]
	Here are the checks on $\iota$.
	\begin{itemize}
		\item Well-defined: note $\iota_{B_0}(f)$ is an element of $\mc F_\mc B(B_0)$ because $B'\subseteq B\subseteq B_0$ will have $f|_B|_{B'}=f|_{B'}$.
		\item Natural: if $B\subseteq B'$, then note that the diagrams
		% https://q.uiver.app/?q=WzAsOCxbMCwwLCJGKEJfMCkiXSxbMSwwLCJcXG1jIEZfXFxtYyBCKEJfMCkiXSxbMCwxLCJGKEInXzApIl0sWzEsMSwiXFxtYyBGX1xcbWMgQihCJ18wKSJdLFszLDAsImYiXSxbMywxLCJmfF97QidfMH0iXSxbNCwwLCIoZnxfQilfe0JcXHN1YnNldGVxIEJfMH0iXSxbNCwxLCIoZnxfQilfe0JcXHN1YnNldGVxIEInXzB9Il0sWzAsMSwiXFxpb3RhX0IiXSxbMiwzLCJcXGlvdGFfe0InfSIsMl0sWzAsMiwiXFxvcHtyZXN9X3tCLEInfSIsMl0sWzEsMywiXFxvcHtyZXN9X3tCLEInfSJdLFs0LDYsIiIsMCx7InN0eWxlIjp7InRhaWwiOnsibmFtZSI6Im1hcHMgdG8ifX19XSxbNiw3LCIiLDAseyJzdHlsZSI6eyJ0YWlsIjp7Im5hbWUiOiJtYXBzIHRvIn19fV0sWzQsNSwiIiwyLHsic3R5bGUiOnsidGFpbCI6eyJuYW1lIjoibWFwcyB0byJ9fX1dLFs1LDcsIiIsMix7InN0eWxlIjp7InRhaWwiOnsibmFtZSI6Im1hcHMgdG8ifX19XV0=&macro_url=https%3A%2F%2Fraw.githubusercontent.com%2FdFoiler%2Fnotes%2Fmaster%2Fnir.tex
		\[\begin{tikzcd}
			{F(B_0)} & {\mc F_\mc B(B_0)} && f & {(f|_B)_{B\subseteq B_0}} \\
			{F(B'_0)} & {\mc F_\mc B(B'_0)} && {f|_{B'_0}} & {(f|_B)_{B\subseteq B'_0}}
			\arrow["{\iota_B}", from=1-1, to=1-2]
			\arrow["{\iota_{B'}}"', from=2-1, to=2-2]
			\arrow["{\op{res}_{B,B'}}"', from=1-1, to=2-1]
			\arrow["{\op{res}_{B,B'}}", from=1-2, to=2-2]
			\arrow[maps to, from=1-4, to=1-5]
			\arrow[maps to, from=1-5, to=2-5]
			\arrow[maps to, from=1-4, to=2-4]
			\arrow[maps to, from=2-4, to=2-5]
		\end{tikzcd}\]
		commute, finishing.
		\item Injective: suppose that $f,g\in F(B_0)$ have the same image in $\mc F_\mc B(B_0)$. This means that $(f|_B)_{B\subseteq B_0}=(g|_B)_{B\subseteq B_0}$, so $f=f|_{B_0}=g|_{B_0}=g$, so we are done.
		\item Surjective: fix some $(f_B)_{B\subseteq B_0}\in\mc F_\mc B(B_0)$. Notably, for any basic $B_1,B_2\subseteq B_0$ with some basic  $B\subseteq B_1\cap B_2$, we have
		\[f_{B_1}|_B=f_B=f_{B_2}|_B,\]
		so gluability applied to $F$ promises $f\in F(B_0)$ such that $f|_B=f_B$ for all basic $B\subseteq B_0$. So $\iota_{B_0}(f)=(f_B)_{B\subseteq B_0}$.
	\end{itemize}
	We now begin showing that $\mc F$ satisfies the universal property. Fix some sheaf $\mc G$ on $X$ with a morphism $\varphi\colon F\to\mc G_\mc B$.
	
	In light of \autoref{lem:morphismonbase}, it suffices to show the existence and uniqueness of a morphism $\psi_\mc B\colon\mc F_\mc B\to\mc G_\mc B$ on the base $\mc B$ making \autoref{eq:sheaffrombaseup} commute. Namely, the existence of $\psi_\mc B$ promises a full sheaf morphism $\psi\colon\mc F\to\mc G$ extending via \autoref{lem:morphismonbase}; for uniqueness, two possible $\psi,\psi'\colon\mc F\to\mc G$ with $\psi_\mc B$ and $\psi'_\mc B$ both commuting will enforce $\psi_\mc B=\psi'_\mc B$ and then $\psi=\psi'$ by the uniqueness of \autoref{lem:morphismonbase}.

	Continuing with the proof, we note that the fact that $\iota$ is an isomorphism means that the commutativity of \autoref{eq:sheaffrombaseup} is equivalent to the diagram
	% https://q.uiver.app/?q=WzAsMyxbMCwwLCJGIl0sWzEsMCwiXFxtYyBGX1xcbWMgQiJdLFsxLDEsIlxcbWMgR19cXG1jIEIiXSxbMSwwLCJcXGlvdGFeey0xfSIsMl0sWzEsMiwiXFxwc2lfXFxtYyBCIl0sWzAsMiwiXFx2YXJwaGkiLDJdXQ==&macro_url=https%3A%2F%2Fraw.githubusercontent.com%2FdFoiler%2Fnotes%2Fmaster%2Fnir.tex
	\[\begin{tikzcd}
		F & {\mc F_\mc B} \\
		& {\mc G_\mc B}
		\arrow["{\iota^{-1}}"', from=1-2, to=1-1]
		\arrow["{\psi_\mc B}", from=1-2, to=2-2]
		\arrow["\varphi"', from=1-1, to=2-2]
	\end{tikzcd}\]
	commuting. However, the commutativity of this diagram is equivalent to setting $\psi_\mc B\coloneqq\varphi\circ\iota^{-1}$. Thus, uniqueness of $\psi_\mc B$ is immediate, and existence of $\psi_\mc B$ amounts to noting the composition of natural transformations remains a natural transformation.
\end{proof}
\begin{remark}
	One can also define $\mc F(U)$ as compatible systems of stalks, but we have not defined stalks yet.
\end{remark}
\begin{remark}
	The universal property implies that the pair $(\mc F,\iota)$ is unique up to unique isomorphism, for a suitable notion of unique isomorphism. Namely, the usual abstract nonsense arguments with universal properties is able to show that if we have another sheaf $\mc F'$ with isomorphism $\iota'\colon F\to\mc F'_\mc B$ satisfying the universal property, then $\mc F$ and $\mc F'$ are isomorphic. (This isomorphism $\eta\colon\mc F\cong\mc F'$ is unique if we ask for the corresponding diagram
	% https://q.uiver.app/?q=WzAsMyxbMCwwLCJGIl0sWzEsMCwiXFxtYyBGX1xcbWMgQiJdLFsxLDEsIlxcbWMgRl9cXG1jIEInIl0sWzAsMSwiXFxpb3RhIl0sWzEsMiwiXFxldGFfXFxtYyBCIl0sWzAsMiwiXFxpb3RhJyIsMl1d&macro_url=https%3A%2F%2Fraw.githubusercontent.com%2FdFoiler%2Fnotes%2Fmaster%2Fnir.tex
	\[\begin{tikzcd}
		F & {\mc F_\mc B} \\
		& {\mc F_\mc B'}
		\arrow["\iota", from=1-1, to=1-2]
		\arrow["{\eta_\mc B}", from=1-2, to=2-2]
		\arrow["{\iota'}"', from=1-1, to=2-2]
	\end{tikzcd}\]
	to commute.)
\end{remark}
\begin{remark}
	Here is another universal property: given a sheaf $\mc G$ with a morphism $\varphi\colon\mc G_\mc B\to F$ of sheaves on the base $\mc B$, there is a unique sheaf morphism $\psi\colon\mc G\to\mc F$ making the diagram
	% https://q.uiver.app/?q=WzAsMyxbMCwxLCJcXG1jIEZfXFxtYyBCIl0sWzEsMSwiRiJdLFswLDAsIlxcbWMgR19cXG1jIEIiXSxbMiwxLCJcXHZhcnBoaSJdLFswLDEsIlxcaW90YV57LTF9IiwyXSxbMiwwLCJcXHBzaV9cXG1jIEIiLDJdXQ==&macro_url=https%3A%2F%2Fraw.githubusercontent.com%2FdFoiler%2Fnotes%2Fmaster%2Fnir.tex
	\[\begin{tikzcd}
		{\mc G_\mc B} \\
		{\mc F_\mc B} & F
		\arrow["\varphi", from=1-1, to=2-2]
		\arrow["{\iota^{-1}}"', from=2-1, to=2-2]
		\arrow["{\psi_\mc B}"', from=1-1, to=2-1]
	\end{tikzcd}\]
	commute. Indeed, reversing the arrow $\iota$ shows that we are asking for a unique sheaf morphism $\psi$ such that $\psi_\mc B=\iota\circ\varphi$, which we get from \autoref{lem:morphismonbase}.
\end{remark}
The universal property actually gives a functor from sheaves $F$ on a base $\mc B$ to sheaves $\mc F_B$ on $X$.
\begin{lemma} \label{lem:sheafonabasefunctor}
	Fix a topological space $X$ with a base $\mc B$ for its topology. Then the map sending a sheaf $F$ on a base to its sheaf $\mc E(F)$ describes the action of a functor on objects.
\end{lemma}
\begin{proof}
	Given a sheaf on a base $F$, let $\iota_F\colon F\to\mc E(F)_\mc B$ be the inclusion. Now, given a morphism $\varphi\colon F\to G$ of sheaves on a base, note that there is a unique morphism $\mc E(\varphi)$ making the diagram
	% https://q.uiver.app/?q=WzAsNCxbMCwwLCJGIl0sWzEsMCwiXFxtYyBFKEYpX1xcbWMgQiJdLFswLDEsIkciXSxbMSwxLCJcXG1jIEUoRylfXFxtYyBCIl0sWzAsMSwiXFxpb3RhX0YiXSxbMCwyLCJcXHZhcnBoaSIsMl0sWzEsMywiXFxtYyBFKFxcdmFycGhpKV9cXG1jIEIiLDAseyJzdHlsZSI6eyJib2R5Ijp7Im5hbWUiOiJkYXNoZWQifX19XSxbMiwzLCJcXGlvdGFfRyJdXQ==&macro_url=https%3A%2F%2Fraw.githubusercontent.com%2FdFoiler%2Fnotes%2Fmaster%2Fnir.tex
	\[\begin{tikzcd}
		F & {\mc E(F)_\mc B} \\
		G & {\mc E(G)_\mc B}
		\arrow["{\iota_F}", from=1-1, to=1-2]
		\arrow["\varphi"', from=1-1, to=2-1]
		\arrow["{\mc E(\varphi)_\mc B}", dashed, from=1-2, to=2-2]
		\arrow["{\iota_G}", from=2-1, to=2-2]
	\end{tikzcd}\]
	commute by \autoref{lem:morphismonbase}. We now need to show that this data assembles into a functor.
	\begin{itemize}
		\item Identity: given a sheaf on a base $F$, note that $\id_F$ induces the commutative diagram
		% https://q.uiver.app/?q=WzAsNCxbMCwwLCJGIl0sWzEsMCwiXFxtYyBFKEYpX1xcbWMgQiJdLFswLDEsIkYiXSxbMSwxLCJcXG1jIEUoRilfXFxtYyBCIl0sWzAsMSwiXFxpb3RhX0YiXSxbMCwyLCJcXGlkX0YiLDJdLFsxLDMsIihcXGlkX3tcXG1jIEUoRil9KV9cXG1jIEIiLDAseyJzdHlsZSI6eyJib2R5Ijp7Im5hbWUiOiJkYXNoZWQifX19XSxbMiwzLCJcXGlvdGFfRiJdXQ==&macro_url=https%3A%2F%2Fraw.githubusercontent.com%2FdFoiler%2Fnotes%2Fmaster%2Fnir.tex
		\[\begin{tikzcd}
			F & {\mc E(F)_\mc B} \\
			F & {\mc E(F)_\mc B}
			\arrow["{\iota_F}", from=1-1, to=1-2]
			\arrow["{\id_F}"', from=1-1, to=2-1]
			\arrow["{(\id_{\mc E(F)})_\mc B}", dashed, from=1-2, to=2-2]
			\arrow["{\iota_F}", from=2-1, to=2-2]
		\end{tikzcd}\]
		which makes us conclude $\mc E({\id_F})=\id_{\mc E(F)}$.
		\item Functoriality: given morphisms $\varphi\colon F\to G$ and $\psi\colon G\to H$ of sheaves on a base, we note that the diagram
		% https://q.uiver.app/?q=WzAsNixbMCwwLCJGIl0sWzEsMCwiXFxtYyBFKEYpX1xcbWMgQiJdLFswLDEsIkciXSxbMSwxLCJcXG1jIEUoRylfXFxtYyBCIl0sWzAsMiwiSCJdLFsxLDIsIlxcbWMgRShIKV9cXG1jIEIiXSxbMCwxLCJcXGlvdGFfRiJdLFswLDIsIlxcdmFycGhpIiwyXSxbMSwzLCJcXG1jIEUoXFx2YXJwaGkpX1xcbWMgQiIsMl0sWzIsMywiXFxpb3RhX0ciXSxbMiw0LCJcXHBzaSIsMl0sWzMsNSwiXFxtYyBFKFxccHNpKV9cXG1jIEIiLDJdLFs0LDUsIlxcaW90YV9IIl0sWzEsNSwiXFxtYyBFKFxccHNpXFxjaXJjXFx2YXJwaGkpX1xcbWMgQiIsMCx7Im9mZnNldCI6LTIsImN1cnZlIjotMiwic3R5bGUiOnsiYm9keSI6eyJuYW1lIjoiZGFzaGVkIn19fV1d&macro_url=https%3A%2F%2Fraw.githubusercontent.com%2FdFoiler%2Fnotes%2Fmaster%2Fnir.tex
		\[\begin{tikzcd}
			F & {\mc E(F)_\mc B} \\
			G & {\mc E(G)_\mc B} \\
			H & {\mc E(H)_\mc B}
			\arrow["{\iota_F}", from=1-1, to=1-2]
			\arrow["\varphi"', from=1-1, to=2-1]
			\arrow["{\mc E(\varphi)_\mc B}"', from=1-2, to=2-2]
			\arrow["{\iota_G}", from=2-1, to=2-2]
			\arrow["\psi"', from=2-1, to=3-1]
			\arrow["{\mc E(\psi)_\mc B}"', from=2-2, to=3-2]
			\arrow["{\iota_H}", from=3-1, to=3-2]
			\arrow["{\mc E(\psi\circ\varphi)_\mc B}", shift left=2, curve={height=-12pt}, dashed, from=1-2, to=3-2]
		\end{tikzcd}\]
		commutes, so the uniqueness of the arrow $\mc E(\psi\circ\varphi)_\mc B$ forces $\mc E(\psi\circ\varphi)=\mc E(\psi)\circ\mc E(\varphi)$.
		\qedhere
	\end{itemize}
\end{proof}
\begin{remark}
	In fact, the functor $\mc E$ is the right adjoint to the forgetful functor $(-)_\mc B$ from sheaves on a base $\mc B$ to sheaves on $X$, which also essentially follows from the universal property. We will not bother showing this.
\end{remark}

\end{document}