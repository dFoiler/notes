% !TEX root = ../notes.tex

\documentclass[../notes.tex]{subfiles}

\begin{document}

We finish defining the structure sheaf $\mc O_{\Spec A}$ today.
\begin{remark}
	Let $X$ be a topological space. One can write the conditions for $F$ being a sheaf on a base $\mc B$ as the exactness of
	\[0\to F(B_0)\to\prod_{B\in\mc U}F(B)\to\prod_{B,B'\in\mc U}F(B\cap B'),\]
	where $\mc U$ is a basic cover for $B_0$. Notably, the map into $F(B\cap B')$ is well-defined by using the gluability axiom.
\end{remark}
\begin{remark}
	One complaint about sheaves on a base is that we have to choose a base. To be more canonical, we will discuss stalks today, which treats all points the same.
\end{remark}

\subsection{Finishing the Structure Sheaf}
Recall that we defined
\[\mc O_{\Spec A}(D(f))=A_f\]
with some restriction maps. We need to show that our data will assemble into a sheaf on a base, and from last class, we still need to check the gluability axiom for our structure sheaf. As usual, pick a basic cover
\[D(f)=\bigcup_{i\in\mc I}D(f_i).\]
We also give sections $s_i\in A_{f_i}$ which restrict properly.

For psychological reasons, we may assume $f=1$ by replacing our ring $A$ with $A_f$ and the open set $D(f)$ with $\Spec A_f$. Notably, $D(f_i)=(A_f)_{f_i}$, so no information is really changing.

Also, we will freely take $\mc I$ to be finite. To begin, there is a finite subset $\mc I'\subseteq\mc I$ such that we still have a cover
\[\Spec A=\bigcup_{i\in\mc I'}D(f_i),\]
which essentially holds by some compactness result. More formally, $\{D(f_i)\}_{i\in\mc I}$ covering $\Spec A$ is equivalent to $(f_i)_{i\in\mc I}=A$, from which we can extract our finite subcover. Thus, we should be able to build up our section $s\in A$ from the finite subcover $\mc I'$, and then for any $i\in\mc I\setminus\mc I'$, we can build $s'$ from the finite subcover from $\mc I\cup\{i\}$, and we get $s=s'$ from the identity axiom, from which
\[s|_{D(f_i)}=s'|_{D(f_i)}=s_i,\]
finishing.

Now, with $\mc I$ finite, we may find some very large $n$ so that we can write
\[s_i=a_i/f_i^n\in A_{f_i}\]
for some very large $n$. The coherence among our sections is requiring that $s_i|_{D(f_i)\cap D(f_j)}=s_j|_{D(f_i)\cap D(f_j)}$, which means
\[(f_if_j)^m\left(f_j^na_i-f_i^na_j\right)=0\]
for all $i$ and $j$, for some very large and uniform $m$. Then setting $N$ to be large enough, we can take very large powers everywhere so that we can write
\[s_i=a_i'/f_i^N\]
for a perhaps different $a_i'$, where now we have $f_j^Na_i-f_i^Na_j=0$ for all $i$ and $j$.

Now, note that $D(f_i)=D\left(f_i^N\right)$, so we still have
\[\Spec A=\bigcup_{i\in\mc I}D\left(f_i^N\right),\]
so we can write
\[1=\sum_{i\in\mc I}b_if_i^N.\]
As an aside, this is, roughly speaking, a partition of unity to give functions on each of our affine open sets $D(f_i)$. As such, we set
\[a\coloneqq\sum_{i\in\mc I}b_ia_i'\]
so that, for any $j\in\mc J$,
\[f_j^Na=\sum_{i\in\mc I}b_if_j^Na_i'=\sum_{i\in\mc I}b_if_i^Na_j'=a_j',\]
so $a=a_j'/f_j^N=s_j$ in $A_{f_j}$, which is what we wanted.

Having finished the last of our checks, we give our definition.
\begin{definition}[Affine scheme]
	Fix a ring $A$. An \textit{affine scheme} is the topological space $\Spec A$ (given the Zariski topology) together with the sheaf of rings $\mc O_{\Spec A}$ such that
	\[\mc O_{\Spec A}(D(f))=A_f\]
	for each $f\in A$.
\end{definition}

\subsection{Stalks}
To define a morphism of schemes, we will want to discuss stalks.
\begin{remark}
	We might expect a morphism of affine schemes to be merely a continuous map together with a natural transformation of the structure sheaves, but this will not be enough data. Namely, we want all of our morphisms of affine schemes to be induced by ring homomorphisms, which will require adding a little data to what our morphisms do.
\end{remark}
The extra data in those morphisms will come from stalks.
\begin{definition}[Stalk]
	Fix a presheaf $\mc F$ on a topological space $X$. For a point $p\in X$, we define the \textit{stalk of $\mc F$ at $p\in X$} to be the direct limit
	\[\mc F_p\coloneqq\colimit_{p\in U}\mc F(U).\]
	Concretely, elements of $\mc F_p$ are ordered pairs $(U,s)$ where $s\in\mc F(U)$ with $p\in U$, modded out by an equivalence relation $\sim$. Namely, $(U,s)\sim(U',s')$ if and only if there is $W\subseteq U\cap U'$ such that $s|_W=s'|_W$.
\end{definition}
\begin{definition}[Germ]
	Fix a presheaf $\mc F$ on a topological space $X$. For a point $p\in X$ and section $f\in\mc F(U)$ with $p\in U$, the \textit{germ of $f$ at $p$} is the element
	\[[(U,f)]\in\mc F_p.\]
\end{definition}
\begin{notation}
	I will write the germ of $f\in\mc F(U)$ at $p\in U$ as $f|_p$. This notation is not standard.
\end{notation}
Here are some examples.
\begin{lemma}
	Fix a ring $A$. Then $\mc O_{\Spec A,\mf p}=A_\mf p$.
\end{lemma}
\begin{proof}
	Check this directly using the concrete description of our germs. In particular, denominators are allowed to be anything in $A\setminus\mf p$, which is precisely what gives $A_\mf p$.
\end{proof}
Notably, $\mc O_{\Spec A,\mf p}$ is always a local ring. This will be important.
\begin{example}
	Let $X$ be the topological space $\CC$ (or any Riemann surface), and define $\mc O_X$ to be the sheaf of holomorphic functions $X\to\CC$. Then, for any $p\in X$, we have
	\[\mc O_{X,p}=\Bigg\{\sum_{n=0}^\infty a_n(z-z_0)^n\text{ with positive radius of convergence}\Bigg\}\]
	essentially by complex analysis. Namely, we are using the fact that holomorphic functions are analytic.
\end{example}
Here is why we care about stalks.
\begin{idea} \label{idea:stalks}
	Stalks remember everything about a sheaf.
\end{idea}
Here is the rigorization of \autoref{idea:stalks}.
\begin{proposition}
	Fix presheaves $\mc F$ and $\mc G$ on a topological space $X$ with a morphism $\varphi\colon\mc F\to\mc G$.
	\begin{listalph}
		\item For any $p\in X$, there is a natural map $\varphi_p\colon\mc F_p\to\mc G_p$.
		\item Suppose $\mc F$ is a sheaf. Then there is a natural embedding
		\[\mc F(U)\into\prod_{p\in U}\mc F_p\]
		taking $f\in\mc F(U)$ to its germs $(f|_p)_{p\in U}$. An element $(f_p)_{p\in U}$ is in the image if and only if, for each $p\in U$, there is an open set $U_p$ containing $p$ such that w can find $\widetilde f_p\in\mc F(U_p)$ such that all $q\in U_p$ have $f_q=(\widetilde f_p)|_q$. Intuitively, we are saying that all stalks in a small neighborhood come from a single section.
		\item Suppose $\mc G$ is a sheaf. Given morphisms $\varphi_1,\varphi_2\colon\mc F\to\mc G$ such that $(\varphi_1)_p=(\varphi_2)_p$ for all $p\in X$, we have $\varphi_1=\varphi_2$.
	\end{listalph}
\end{proposition}
\begin{proof}
	We go in sequence.
	\begin{listalph}
		\item This follows from category theory. Alternatively, we can write this explicitly as being induced by
		\[\varphi_p\colon[(U,s)]\mapsto[U,\varphi_U(s)].\]
		We omit check that this is well-defined.
		\item The map is an embedding by the identity axiom. The classification of the image comes from the gluability axiom, roughly speaking.
		\item Fix an open set $U\subseteq X$ so that we need $(\varphi_1)_U=(\varphi_2)_U$. Now, the point is that any $\varphi\colon\mc F\to\mc G$ will make the following diagram commute.
		% https://q.uiver.app/?q=WzAsNCxbMCwwLCJcXG1jIEYoVSkiXSxbMSwwLCJcXGRpc3BsYXlzdHlsZVxccHJvZF97cFxcaW4gVX1cXG1jIEZfcCJdLFswLDEsIlxcbWMgRyhVKSJdLFsxLDEsIlxcZGlzcGxheXN0eWxlXFxwcm9kX3twXFxpbiBVfVxcbWMgR19wIl0sWzIsMywiIiwwLHsic3R5bGUiOnsidGFpbCI6eyJuYW1lIjoiaG9vayIsInNpZGUiOiJ0b3AifX19XSxbMCwxXSxbMSwzLCJcXHByb2RcXHZhcnBoaV9wIl0sWzAsMiwiXFx2YXJwaGkiLDJdXQ==&macro_url=https%3A%2F%2Fraw.githubusercontent.com%2FdFoiler%2Fnotes%2Fmaster%2Fnir.tex
		\[\begin{tikzcd}
			{\mc F(U)} & {\displaystyle\prod_{p\in U}\mc F_p} \\
			{\mc G(U)} & {\displaystyle\prod_{p\in U}\mc G_p}
			\arrow[hook, from=2-1, to=2-2]
			\arrow[from=1-1, to=1-2]
			\arrow["{\prod\varphi_p}", from=1-2, to=2-2]
			\arrow["\varphi"', from=1-1, to=2-1]
		\end{tikzcd}\]
		Namely, the value of $\varphi_U(f)$ is uniquely determined by the top row of the diagram for all $f\in\mc F(U)$.
		\qedhere
	\end{listalph}
\end{proof}
\begin{remark}
	The sheaf conditions on (b) and (c) are unnecessary.
\end{remark}

\subsection{Morphisms Between Sheaves}
We are going to want to do category theory on sheaves, so let's begin. Throughout our target category for our sheaves will be abelian (and concrete).
\begin{remark}
	We will be able to show that the category of sheaves on an abelian category is an abelian category. However, we will not do this in detail because we are not sadistic.
\end{remark}
\begin{definition}[Presheaf kernel]
	Given a morphism of presheaves $\varphi\colon\mc F\to\mc G$ on a topological space $X$, we define the \textit{presheaf kernel} as
	\[(\ker\varphi)(U)\coloneqq\ker(\varphi_U)\]
	for each $U\subseteq X$, where restriction maps are induced by $\mc F$. Then $\ker\varphi$ is our \textit{presheaf kernel}.
\end{definition}
\begin{lemma}
	Given a morphism of sheaves $\varphi\colon\mc F\to\mc G$ on a topological space $X$, the presheaf kernel is a sheaf.
\end{lemma}
\begin{proof}
	Omitted.
\end{proof}
Having a kernel gives us a definition.
\begin{definition}[Injective]
	A morphism of presheaves $\varphi\colon\mc F\to\mc G$ is \textit{injective} if and only if the kernel presheaf $\ker\varphi$ is identically zero. Equivalently, we are asking for $\varphi_U$ to be injective everywhere.
\end{definition}
In our stalk philosophy, we might hope we can detect injectivity at stalks. Indeed, we can.
\begin{proposition}
	Fix a morphism of presheaves $\varphi\colon\mc F\to\mc G$. Given that $\mc F$ is a sheaf, we have that $\varphi$ is 
\end{proposition}
\begin{proof}
	Use the embedding
	\[\mc F(U)\into\prod_{p\in U}\mc F_p\]
	for each $U\subseteq X$.
\end{proof}
There is also the following result.
\begin{proposition}
	Fix a morphism of sheaves $\varphi\colon\mc F\to\mc G$. Then $\varphi$ is an isomorphism if and only if $\varphi_U$ is an isomorphism for all $U\subseteq X$ if and only if $\varphi_p$ is an isomorphism for all $p\in X$.
\end{proposition}
\begin{proof}[Sketch]
	The hard part is showing that if we have isomorphisms at each stalk, then we have an isomorphism of full sheaves. We already know about injectivity, so let's focus on surjectivity. Well, for any $g\in\mc G(U)$, we get a system of compatible germs $(g|_p)_{p\in U}$, and because $\varphi_p$ is an isomorphism, we may set
	\[f_p\coloneqq\varphi_p^{-1}(f|_p).\]
	Then we claim that $f_p$ is a set of compatible germs, which gives rise to a section $f\in\mc F(U)$. Indeed, for each $p\in U$, we can find $U_p\subseteq U$ small enough so that $f_p\in\mc F(U_p)$ and $\varphi_{U_p}(f_p)=g|_{U_p}$. Then we simply have to use gluability directly to take these $f_p\in\mc F(U_p)$ to give our section; for this, we need to check
	\[f_p|_{U_p\cap U_q}=f_q|_{U_p\cap U_q}.\]
	However, $\varphi_{U_p\cap U_q}$ is injective, so it suffices to throw this through $\varphi$ and check equality there. However, both of these equal
	\[g|_{U_p\cap U_q}\]
	when passed through $\varphi$, so we are safe. Note we used the injectivity of $\varphi$ at the end here!
\end{proof}
\begin{remark}
	We are avoiding surjectivity for the moment because it is a little trickier. In particular, a morphism $\varphi$ will be able to be surjective without being each $\varphi_U$ being surjective. However, surjectivity will still be equivalent to surjectivity on the stalks.
\end{remark}

\end{document}