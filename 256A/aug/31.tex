% !TEX root = ../notes.tex

\documentclass[../notes.tex]{subfiles}

\begin{document}

We finish defining the structure sheaf $\mc O_{\Spec A}$ of an affine scheme today.
% \begin{remark}
% 	Let $X$ be a topological space. One can write the conditions for $F$ being a sheaf on a base $\mc B$ as the exactness of
% 	\[0\to F(B_0)\to\prod_{B\in\mc U}F(B)\to\prod_{B,B'\in\mc U}F(B\cap B'),\]
% 	where $\mc U$ is a basic cover for $B_0$. Notably, the map into $F(B\cap B')$ is well-defined by using the gluability axiom.
% \end{remark}
\begin{remark}
	One complaint about sheaves on a base is that we have to choose a base. To be more canonical, we will discuss stalks today, which treats all points the same.
\end{remark}

\subsection{The Structure Sheaf}
We are now ready to define the structure sheaf $\mc O_{\Spec A}$ of a ring $A$, which we will define a sheaf on a base. Recall from \autoref{rem:distinguishedbase} that $\{D(f)\}_{f\in A}$ forms a base of the Zariski topology of $\Spec A$, so it will suffice to set
\[\mc O_{\Spec A}(D(f))\coloneqq A_{S(D(f))},\]
where
\[S(D(f))\coloneqq\{g\in A:V(\{g\})\subseteq(\Spec A)\setminus D(f)\}.\]
In other words, $S(D(f))$ consists of the set of functions in $A$ which only vanish outside $D(f)$ so that we can invert them on $D(f)$.
\begin{remark}
	In essence, $\mc O_{\Spec A}(D(f))$ is supposed to be the functions on $D(f)$, which is why we want to be able to invert functions which only vanish on $(\Spec A)\setminus D(f)$.
\end{remark}
\begin{remark}
	The subset $S(D(f))$ only depends on $D(f)$, not $f$, so $\mc O_{\Spec A}(D(f))$ is well-defined. With that said, we note that $f\in S(D(f))$ gives a natural localization map $A_f\to A_{S(D(f))}$ induced by $\id_A$. Similarly, any $g\in S(D(f))$ has $V((g))\subseteq V((f))$ and so \autoref{prop:easynullstellensatz} tells us that
	\[\rad(f)=I(V((f)))\subseteq I(V((g)))=\rad(g),\]
	so $f\in\rad(g)$, so $f^n=ag$ for some positive integer $n$ and $a\in A$; this means $g\in A_f^\times$, so actually $S(D(f))\subseteq A_f^\times$, allowing another natural localization map $A_{S(D(f))}\to A_f$ induced by $\id_A$. These natural localization maps are inverse (their compositions are induced by $\id_A$), so $\mc O_{\Spec A}(D(f))\cong A_f$.
\end{remark}
\begin{remark}[Nir]
	In class, Professor Tang defined the structure sheaf on a base by $\mc O_{\Spec A}(D(f))\coloneqq A_f$. I have chosen to follow \cite{rising-sea} here because I don't like $\mc O_{\Spec A}(D(f))$ to depend on $f\in A$ when it should only depend on $D(f)$.
\end{remark}
To define our (pre)sheaf on a base, we also need to provide restriction maps. Well, for $f,f'\in A$ with $D(f')\subseteq D(f)$, we see that
\[S(D(f))=\{g\in A:V(\{g\})\subseteq(\Spec A)\setminus D(f)\}\subseteq\{g\in A:V(\{g\})\subseteq(\Spec A)\setminus D(f')\}=S(D(f')),\]
so there is a natural localization map
\[\op{res}_{D(f),D(f')}\colon A_{S(D(f))}\to A_{S(D(f'))}\]
induced by $\id_A$. These data give all the data we need to define a sheaf on a base. We will throw the remaining checks into the following lemma.
\begin{lemma}
	Fix a ring $A$. The above data define a sheaf $\mc O_{\Spec A}$ on the base $\{D(f)\}_{f\in A}$.
\end{lemma}
\begin{proof}
	We begin by showing that the data gives a presheaf.
	\begin{itemize}
		\item Identity: if $D(f)=D(f')$, then $S(D(f))=S(D(f'))$, so the localization map
		\[\op{res}_{D(f),D(f)}\colon A_{S(D(f))}\to S_{D(f)}\]
		is simply $\id_{A_{S(D(f))}}$.
		\item Functoriality: suppose $D(f'')\subseteq D(f')\subseteq D(f)$. Then we note that the diagram
		% https://q.uiver.app/?q=WzAsNixbMCwwLCJBX3tTKEQoZikpfSJdLFsxLDAsIkFfe1MoRChmJykpfSJdLFsxLDEsIkFfe1MoRChmJycpKX0iXSxbMiwwLCJhL2ciXSxbMywwLCJhL2ciXSxbMywxLCJhL2ciXSxbMCwxLCJcXG9we3Jlc30iXSxbMSwyLCJcXG9we3Jlc30iXSxbMCwyLCJcXG9we3Jlc30iLDJdLFszLDQsIiIsMix7InN0eWxlIjp7InRhaWwiOnsibmFtZSI6Im1hcHMgdG8ifX19XSxbNCw1LCIiLDIseyJzdHlsZSI6eyJ0YWlsIjp7Im5hbWUiOiJtYXBzIHRvIn19fV0sWzMsNSwiIiwwLHsic3R5bGUiOnsidGFpbCI6eyJuYW1lIjoibWFwcyB0byJ9fX1dXQ==&macro_url=https%3A%2F%2Fraw.githubusercontent.com%2FdFoiler%2Fnotes%2Fmaster%2Fnir.tex
		\[\begin{tikzcd}
			{A_{S(D(f))}} & {A_{S(D(f'))}} & {a/g} & {a/g} \\
			& {A_{S(D(f''))}} && {a/g}
			\arrow["{\op{res}}", from=1-1, to=1-2]
			\arrow["{\op{res}}", from=1-2, to=2-2]
			\arrow["{\op{res}}"', from=1-1, to=2-2]
			\arrow[maps to, from=1-3, to=1-4]
			\arrow[maps to, from=1-4, to=2-4]
			\arrow[maps to, from=1-3, to=2-4]
		\end{tikzcd}\]
		commutes because everything is induced by $\id_A$, so we are done.
	\end{itemize}
	It remains to check the identity and gluability axioms. For this, we will need a basis set $D(f)$ and a basic cover $\{D(f_\alpha)\}_{\alpha\in\lambda}$. To access this cover, we have the following lemma.
	\begin{lemma} \label{lem:finitesubcover}
		Fix a ring $A$. Then, given $f\in A$ and $\{f_\alpha\}_{\alpha\in\lambda}\subseteq A$, the following are equivalent.
		\begin{listalph}
			\item $D(f)\subseteq\bigcup_{\alpha\in\lambda}D(f_\alpha)$.
			\item $f\in\rad(f_\alpha)_{\alpha\in\lambda}$.
		\end{listalph}
	\end{lemma}
	\begin{proof}
		Note that
		\[\bigcup_{\alpha\in\lambda}D(f_\alpha)=\Spec A\setminus\bigcap_{\alpha\in\lambda}V((f_\alpha))=\Spec A\setminus V((f_\alpha)_{\alpha\in\lambda}),\]
		so (a) is equivalent to $V((f_\alpha)_{\alpha\in\lambda})\subseteq V((f))$. Now, \autoref{prop:easynullstellensatz} tells us that (a) implies
		\[\rad(f)=I(V((f)))\subseteq I(V((f_\alpha)_{\alpha\in\lambda}))=\rad(f_\alpha)_{\alpha\in\lambda},\]
		from which (b) follows. Conversely, if (b) holds, then $\rad(f)\subseteq\rad(f_\alpha)_{\alpha\in\lambda}$ by taking radicals, so \autoref{prop:easynullstellensatz} again promises
		\[V((f_\alpha)_{\alpha\in\lambda})=V(\rad(f_\alpha)_{\alpha\in\lambda})\subseteq V(\rad(f))=V((f)),\]
		which we showed is equivalent to (a).
	\end{proof}
	\begin{corollary} \label{cor:easycompactness}
		Fix a ring $A$. Then any cover $\{D(f_\alpha)\}_{\alpha\in\lambda}$ of $D(f)$ has a finite subcover.
	\end{corollary}
	\begin{proof}
		Note \autoref{lem:finitesubcover} tells us that $f\in\rad(f_\alpha)_{\alpha\in\lambda}$, so there is a positive integer $n$ and finite subset $\lambda\subseteq\lambda'$ so that
		\[f^n=\sum_{\alpha\in\lambda'}a_\alpha f_\alpha,\]
		but then $f\in\rad(f_\alpha)_{\alpha\in\lambda'}$, so $D(f)$ is covered by the (finite) cover $\{D(f_\alpha)\}_{\alpha\in\lambda'}$. 
	\end{proof}
	We now show the identity and gluability axioms separately.
	\begin{itemize}
		\item Identity: note \autoref{cor:easycompactness} promises us some $\lambda'\subseteq\lambda$ such that the $\{D(f_\alpha)\}_{\alpha\in\lambda'}$ still covers $D(f)$. We will now forget about $\lambda$ entirely and deal with the finite $\lambda'$ instead.

		For identity, we suppose that we have $s\in\OO_{\Spec A}(D(f))$ such that $s|_{D(f_\alpha)}=0$ for all $\alpha\in\lambda'$, and we want to show that $s=0$. Under the (canonical) isomorphism $\OO_{\Spec A}(D(f_\alpha))\simeq A_{f_\alpha}$, we see that we must have
		\[f_\alpha^{d_\alpha}s=0\]
		for some $d_\alpha$, for each $\alpha$. Now, $D(f_\alpha)=D\left(f_\alpha^{d_\alpha}\right)$, so the $D\left(f_\alpha^{d_\alpha}\right)$ still cover $D(f)$; it follows from \autoref{lem:finitesubcover} that there is some $d$ for which
		\[f^d=\sum_{\alpha\in\lambda'}c_\alpha f_\alpha^{d_\alpha}.\]
		Multiplying both sides by $s$ (after embedding in $A_{S(D(f))}$) tells us that $f^ds=0$ in $A_{S(D(f))}$, so $s=0$ because $f\in A_{S(D(f))}^\times$.

		\item Finite gluability: fix sections $s_\alpha\in\OO_{\Spec A}(D(f_\alpha))$ such that
		\[s_\alpha|_{D(f_\alpha)\cap D(f_\beta)}=s_\beta|_{D(f_\alpha)\cap D(f_\beta)}.\]
		For concreteness, use $\OO_{\Spec A}(D(f))\simeq A_f$ to write $s_\alpha\coloneqq a_\alpha/f_\alpha^n$, where $n$ is the maximum of all the possibly needed denominators.
		
		Noting that $D(f_\alpha)\cap D(f_\beta)=D(f_\alpha f_\beta)$, our coherence is equivalent to asking for
		\[(f_\alpha f_\beta)^m\left(f_\beta^na_\alpha-f_\alpha^na_\beta\right)=0,\]
		where again $m$ is chosen to be large enough among the finitely many possibilities for $\alpha$ and $\beta$. We now notice that
		\[s_\alpha=\frac{a_\alpha}{f_\alpha^n}=\frac{f_\alpha^ma_\alpha}{f_\alpha^{n+m}},\]
		so we set $b_\alpha\coloneqq f_\alpha^ma_\alpha$ and $g_\alpha\coloneqq f_\alpha^{n+m}$, which means
		\[g_\beta b_\alpha=g_\alpha b_\beta\]
		for all $\alpha,\beta$. Notably, $\rad(f_\alpha)=\rad(g_\alpha)$, so $D(f_\alpha)=D(g_\alpha)$, so the $\{D(g_\alpha)\}_{\alpha\in\lambda}$ still cover $D(f)$, so \autoref{lem:finitesubcover} tells us that we can write
		\[f^n=\sum_{\alpha\in\lambda}c_\alpha g_\alpha\]
		for some positive integer $n$. In particular, we set $s\in\OO_{\Spec A}(D(f))\simeq A_f$ by
		\[s\coloneqq\frac1{f^n}\sum_{\alpha\in\lambda}c_\alpha b_\alpha.\]
		In particular, for any $\beta\in\lambda$, we see
		\[g_\beta s=\frac1{f^n}\sum_{\alpha\in\lambda}c_\alpha g_\beta b_\alpha=\frac1{f^n}\sum_{\alpha\in\lambda}c_\alpha g_\alpha b_\beta=b_\beta\]
		in $A_f$, so our restriction is $s|_{D(g_\beta)}=b_\beta/g_\beta=s_\beta$, which is what we wanted.

		\item Gluability: we show general gluability from finite gluability. Fix sections $s_\alpha\in\mc O_{\Spec A}(D(f_\alpha))$ such that
		\begin{equation}
			s_\alpha|_{D(f_\alpha)\cap D(f_\beta)}=s_\beta|_{D(f_\alpha)\cap D(f_\beta)} \label{eq:sectioncoherence}
		\end{equation}
		for each $\alpha,\beta\in\lambda$. Using \autoref{cor:easycompactness}, we can find a finite subcover using $\lambda'\subseteq\lambda$, and the sections $\{s_\alpha\}_{\alpha\in\lambda'}$ still satisfy \autoref{eq:sectioncoherence}, so finite gluability (and identity!) gives a unique $s\in\OO_{\Spec A}(D(f))$ with
		\[s|_{D(f_\alpha)}=s_\alpha.\]
		We claim that actually $s|_{D(f_\alpha)}=s_\alpha$ for all $\alpha\in\lambda$. Well, for any $\beta\in\lambda$, apply finite gluability to $\lambda'\cup\{\beta\}$ to find $s'\in\OO_{\Spec A}(D(f))$ such that $s'|_{D(f_\alpha)}=s_\alpha$ for all $\alpha\in\lambda'\cup\{\beta\}$.
		
		It follows from the identity axiom that on the open cover $\{D(f_\alpha)\}_{\alpha\in\lambda'}$ that $s=s'$, so we conclude
		\[s|_{D(f_\beta)}=s'|_{D(f_\beta)}=s_\beta\]
		for any $\beta\in\lambda$.
		\qedhere
	\end{itemize}
\end{proof}
Having finished the last of our checks, we see that our data make a sheaf on a base, so \autoref{prop:sheaffrombase} promises a unique sheaf extending this sheaf on a base. This is the (affine) structure sheaf, and it finishes our definition of an affine scheme.
\begin{definition}[Affine scheme]
	Fix a ring $A$. An \textit{affine scheme} is the topological space $\Spec A$ (given the Zariski topology) together with the sheaf of rings $\mc O_{\Spec A}$ such that
	\[\mc O_{\Spec A}(D(f))=A_{S(D(f))}\]
	for each $f\in A$; here $S(D(f))=\{g\in A:D(f)\subseteq D(g)\}$.
\end{definition}
Note that we are somewhat sloppily identifying the outputs of the structure sheaf with its outputs on the base.

\subsection{Stalks}
To define a morphism of schemes, we will want to discuss stalks.
\begin{remark}
	We might expect a morphism of (affine) schemes to be merely a continuous map together with a natural transformation of the structure sheaves (perhaps with some coherence conditions). However, this will not be enough data. Namely, we want all of our morphisms of affine schemes to be induced by ring homomorphisms, and this will require exploiting a little more data.
\end{remark}
The extra data in those morphisms will come from stalks.
\begin{definition}[Stalk]
	Fix a presheaf $\mc F$ on a topological space $X$. For a point $p\in X$, we define the \textit{stalk of $\mc F$ at $p\in X$} to be the direct limit
	\[\mc F_p\coloneqq\colimit_{U\ni p}\mc F(U).\]
	Concretely, elements of $\mc F_p$ are ordered pairs $(U,s)$ where $s\in\mc F(U)$ with $p\in U$, modded out by an equivalence relation $\sim$; here, $(U,s)\sim(U',s')$ if and only if there is $W\subseteq U\cap U'$ such that $s|_W=s'|_W$.
\end{definition}
\begin{remark}
	We go ahead and check that $\sim$ forms an equivalence relation. Fix $(U_i,s_i)$ with $s_i\in\mc F(U_i)$ for $i\in\{1,2,3\}$.
	\begin{itemize}
		\item Reflexive: note $U_1\subseteq U_1$ and $s_1|_{U_1}=s_1=s_1|_{U_1}$, so $(U_1,s_1)\sim(U_1,s_1)$.
		\item Symmetry: if $(U_1,s_1)\sim(U_2,s_2)$, we can find an open $V\subseteq U_1\cap U_2$ with $s_1|_V=s_2|_V$, which implies $s_2|_V=s_1|_V$, so $(U_2,s_2)\sim(U_1,s_1)$.
		\item Transitive: if $(U_1,s_1)\sim(U_2,s_2)$ and $(U_2,s_2)\sim(U_3,s_3)$, we can find open $V_1\subseteq U_1\cap U_2$ and $V_2\subseteq U_2\cap U_3$ such that $s_1|_{V_1}=s_2|_{V_1}$ and $s_2|_{V_2}=s_3|_{V_2}$. Then $V_1\cap V_2\subseteq U_1\cap U_3$, and we can see
		\[s_1|_{V_1\cap V_2}=s_1|_{V_1}|_{V_1\cap V_2}=s_2|_{V_1}|_{V_1\cap V_2}=s_2|_{V_1\cap V_2}=s_2|_{V_2}|_{V_1\cap V_2}=s_3|_{V_2}|_{V_1\cap V_2}=s_3|_{V_1\cap V_2}.\]
	\end{itemize}
\end{remark}
\begin{definition}[Germ]
	Fix a presheaf $\mc F$ on a topological space $X$. For a point $p\in X$ and section $s\in\mc F(U)$ with $p\in U$, the \textit{germ of $f$ at $p$} is the element
	\[[(U,s)]\in\mc F_p.\]
\end{definition}
\begin{notation}
	I will write the germ of $f\in\mc F(U)$ at $p\in U$ as $f|_p$. This notation is not standard.
\end{notation}
Here are some examples.
\begin{lemma}
	Fix a ring $A$. Then $\mc O_{\Spec A,\mf p}=A_\mf p$.
\end{lemma}
\begin{proof}
	Check this directly using the concrete description of our germs. In particular, denominators are allowed to be anything in $A\setminus\mf p$, which is precisely what gives $A_\mf p$.
\end{proof}
Notably, $\mc O_{\Spec A,\mf p}$ is always a local ring. This will be important.
\begin{example}
	Let $X$ be the topological space $\CC$ (or any Riemann surface), and define $\mc O_X$ to be the sheaf of holomorphic functions $X\to\CC$. Then, for any $p\in X$, we have
	\[\mc O_{X,p}=\Bigg\{\sum_{n=0}^\infty a_n(z-z_0)^n\text{ with positive radius of convergence}\Bigg\}\]
	essentially by complex analysis. Namely, we are using the fact that holomorphic functions are analytic.
\end{example}
Here is why we care about stalks.
\begin{idea} \label{idea:stalks}
	Stalks remember everything about a sheaf.
\end{idea}
Here is the rigorization of \autoref{idea:stalks}.
\begin{proposition}
	Fix presheaves $\mc F$ and $\mc G$ on a topological space $X$ with a morphism $\varphi\colon\mc F\to\mc G$.
	\begin{listalph}
		\item For any $p\in X$, there is a natural map $\varphi_p\colon\mc F_p\to\mc G_p$.
		\item Suppose $\mc F$ is a sheaf. Then there is a natural embedding
		\[\mc F(U)\into\prod_{p\in U}\mc F_p\]
		taking $f\in\mc F(U)$ to its germs $(f|_p)_{p\in U}$. An element $(f_p)_{p\in U}$ is in the image if and only if, for each $p\in U$, there is an open set $U_p$ containing $p$ such that w can find $\widetilde f_p\in\mc F(U_p)$ such that all $q\in U_p$ have $f_q=(\widetilde f_p)|_q$. Intuitively, we are saying that all stalks in a small neighborhood come from a single section.
		\item Suppose $\mc G$ is a sheaf. Given morphisms $\varphi_1,\varphi_2\colon\mc F\to\mc G$ such that $(\varphi_1)_p=(\varphi_2)_p$ for all $p\in X$, we have $\varphi_1=\varphi_2$.
	\end{listalph}
\end{proposition}
\begin{proof}
	We go in sequence.
	\begin{listalph}
		\item This follows from category theory. Alternatively, we can write this explicitly as being induced by
		\[\varphi_p\colon[(U,s)]\mapsto[U,\varphi_U(s)].\]
		We omit check that this is well-defined.
		\item The map is an embedding by the identity axiom. The classification of the image comes from the gluability axiom, roughly speaking.
		\item Fix an open set $U\subseteq X$ so that we need $(\varphi_1)_U=(\varphi_2)_U$. Now, the point is that any $\varphi\colon\mc F\to\mc G$ will make the following diagram commute.
		% https://q.uiver.app/?q=WzAsNCxbMCwwLCJcXG1jIEYoVSkiXSxbMSwwLCJcXGRpc3BsYXlzdHlsZVxccHJvZF97cFxcaW4gVX1cXG1jIEZfcCJdLFswLDEsIlxcbWMgRyhVKSJdLFsxLDEsIlxcZGlzcGxheXN0eWxlXFxwcm9kX3twXFxpbiBVfVxcbWMgR19wIl0sWzIsMywiIiwwLHsic3R5bGUiOnsidGFpbCI6eyJuYW1lIjoiaG9vayIsInNpZGUiOiJ0b3AifX19XSxbMCwxXSxbMSwzLCJcXHByb2RcXHZhcnBoaV9wIl0sWzAsMiwiXFx2YXJwaGkiLDJdXQ==&macro_url=https%3A%2F%2Fraw.githubusercontent.com%2FdFoiler%2Fnotes%2Fmaster%2Fnir.tex
		\[\begin{tikzcd}
			{\mc F(U)} & {\displaystyle\prod_{p\in U}\mc F_p} \\
			{\mc G(U)} & {\displaystyle\prod_{p\in U}\mc G_p}
			\arrow[hook, from=2-1, to=2-2]
			\arrow[from=1-1, to=1-2]
			\arrow["{\prod\varphi_p}", from=1-2, to=2-2]
			\arrow["\varphi"', from=1-1, to=2-1]
		\end{tikzcd}\]
		Namely, the value of $\varphi_U(f)$ is uniquely determined by the top row of the diagram for all $f\in\mc F(U)$.
		\qedhere
	\end{listalph}
\end{proof}
\begin{remark}
	The sheaf conditions on (b) and (c) are unnecessary.
\end{remark}

\subsection{Morphisms Between Sheaves}
We are going to want to do category theory on sheaves, so let's begin. Throughout our target category for our sheaves will be abelian (and concrete).
\begin{remark}
	We will be able to show that the category of sheaves on an abelian category is an abelian category. However, we will not do this in detail because we are not sadistic.
\end{remark}
\begin{definition}[Presheaf kernel]
	Given a morphism of presheaves $\varphi\colon\mc F\to\mc G$ on a topological space $X$, we define the \textit{presheaf kernel} as
	\[(\ker\varphi)(U)\coloneqq\ker(\varphi_U)\]
	for each $U\subseteq X$, where restriction maps are induced by $\mc F$. Then $\ker\varphi$ is our \textit{presheaf kernel}.
\end{definition}
\begin{lemma}
	Given a morphism of sheaves $\varphi\colon\mc F\to\mc G$ on a topological space $X$, the presheaf kernel is a sheaf.
\end{lemma}
\begin{proof}
	Omitted.
\end{proof}
Having a kernel gives us a definition.
\begin{definition}[Injective]
	A morphism of presheaves $\varphi\colon\mc F\to\mc G$ is \textit{injective} if and only if the kernel presheaf $\ker\varphi$ is identically zero. Equivalently, we are asking for $\varphi_U$ to be injective everywhere.
\end{definition}
In our stalk philosophy, we might hope we can detect injectivity at stalks. Indeed, we can.
\begin{proposition} \label{prop:injonstalks}
	Fix a morphism of presheaves $\varphi\colon\mc F\to\mc G$. Given that $\mc F$ is a sheaf, we have that $\varphi$ is injective if and only if it is injective on stalks.
\end{proposition}
\begin{proof}
	Use the embedding
	\[\mc F(U)\into\prod_{p\in U}\mc F_p\]
	for each $U\subseteq X$.
\end{proof}
There is also the following result.
\begin{proposition}
	Fix a morphism of sheaves $\varphi\colon\mc F\to\mc G$. Then $\varphi$ is an isomorphism if and only if $\varphi_U$ is an isomorphism for all $U\subseteq X$ if and only if $\varphi_p$ is an isomorphism for all $p\in X$.
\end{proposition}
\begin{proof}[Sketch]
	The hard part is showing that if we have isomorphisms at each stalk, then we have an isomorphism of full sheaves. We already know about injectivity, so let's focus on surjectivity. Well, for any $g\in\mc G(U)$, we get a system of compatible germs $(g|_p)_{p\in U}$, and because $\varphi_p$ is an isomorphism, we may set
	\[f_p\coloneqq\varphi_p^{-1}(f|_p).\]
	Then we claim that $f_p$ is a set of compatible germs, which gives rise to a section $f\in\mc F(U)$. Indeed, for each $p\in U$, we can find $U_p\subseteq U$ small enough so that $f_p\in\mc F(U_p)$ and $\varphi_{U_p}(f_p)=g|_{U_p}$. Then we simply have to use gluability directly to take these $f_p\in\mc F(U_p)$ to give our section; for this, we need to check
	\[f_p|_{U_p\cap U_q}=f_q|_{U_p\cap U_q}.\]
	However, $\varphi_{U_p\cap U_q}$ is injective, so it suffices to throw this through $\varphi$ and check equality there. However, both of these equal
	\[g|_{U_p\cap U_q}\]
	when passed through $\varphi$, so we are safe. Note we used the injectivity of $\varphi$ at the end here!
\end{proof}
\begin{remark}
	We are avoiding surjectivity for the moment because it is a little trickier. In particular, a morphism $\varphi$ will be able to be surjective without being each $\varphi_U$ being surjective. However, surjectivity will still be equivalent to surjectivity on the stalks.
\end{remark}

\end{document}