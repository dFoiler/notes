% !TEX root = ../notes.tex

\documentclass[../notes.tex]{subfiles}

\begin{document}

\section{November 14}

We continue falling. We began class by finishing the proof of \autoref{prop:fix-powers-ample}, which I've edited directly into last class for continuity.
\begin{remark}
	Here are some corollaries to \autoref{prop:fix-powers-ample}.
	\begin{itemize}
		\item One can show that $\mc L$ and $\mc L'$ being very ample implies that $\mc L\otimes\mc L'$ is very ample.
		\item Similarly, we will show on the homework that $\mc L$ and $\mc L'$ are both ample, then $\mc L\otimes\mc L'$ is still ample. The point is that being ample means that a sufficiently large power is still ample.
		\item In fact, if $\mc L$ is any line bundle on a projective scheme $X$ (so that very ample line bundles exist), then there are very ample line bundles $\mc L_1$ and $\mc L_2$ such that $\mc L=\mc L_1\otimes\mc L_2^\lor$. The point is to use (c) of \autoref{prop:fix-powers-ample}: given an ample line bundle $\mc M$, eventually $\mc M^{\otimes n}$ is very ample, and eventually $\mc L\otimes\mc M^{\otimes n}$ is very ample, so we set $\mc L_1=\mc L\otimes\mc M^{\otimes n}$ and $\mc L_2=\mc M^{\otimes n}$.
	\end{itemize}
\end{remark}
\begin{remark}
	One can use the above remarks to define heights for projective varieties: of course, we would like to measure heights via a projective embedding, but we would like to keep track of this projective embedding, for which we use the corresponding very ample line bundle.
\end{remark}
Anyway, today we're going to focus on blowing up.

\subsection{Blowing Up}
% GW 13.18--13.20
% V ch 23
% L 8.1, 8.3
We will have to define blowing up, but we will not really discuss how it helps resolution of singularities. Here is our main example.
\begin{exe}
	Fix an algebraically closed field $k$, basically so that we can talk about closed points without fear. We blow up $\AA^2_k$ at the origin to make $\op{Bl}_{(0,0)}\AA^2_k$.
\end{exe}
\begin{proof}
	Set-theoretically, we will have
	\[\op{Bl}_{(0,0)}\AA^2_k\coloneqq\{(p,[\ell])\in\AA^2_k\times[\ell]:p\in\ell\},\]
	where we are thinking about $\PP^1_k$ as lines going through the origin. We can see that this defines a closed subset (variety-theoretically) of $\AA^2_k\times\PP^1_k$, so we give it the reduced subscheme structure.

	Another way to think of this construction of $\op{Bl}_{(0,0)}\AA^2_k$ is by taking $\AA^2_k=\Spec k[x,y]$ and $\PP^1_k=\Proj k[X,Y]$ so that $\op{Bl}_{(0,0)}\AA^2_k$ is essentially the closed subscheme carved out by $xY-yX=0$. We can see this on the usual affine open cover of $\PP^1_k$, as follows.
	\begin{itemize}
		\item On $\AA^2_k\times D_+(Y)$, we are looking at $\Spec k[x,y,u]/(xu-y)$, where $u$ denotes $X/Y$.
		\item On $\AA^2_k\times D_+(X)$, we are looking at $\Spec k[x,y,v]/(x-yv)$, where $u$ denotes $Y/X$.
	\end{itemize}
	Let's discuss what's going on here. Note there is a map $\pi\colon\op{Bl}_{(0,0)}\AA^2_k\to\AA^2_k$ by sending $(p,[\ell])\mapsto p$. However, something funny is happening with the fibers.
	\begin{itemize}
		\item At the point we blew up, we see $\pi^{-1}(\{(0,0)\})=\PP^1_k$.
		\item Otherwise, $\pi|_{\op{Bl}_{(0,0)}\AA^2_k\setminus\pi^{-1}(\{(0,0)\})}$ is an isomorphism onto its image, which is $\AA^2_k\setminus\{(0,0)\}$.
	\end{itemize}
	This first pre-image is so nice only because what we blew up was so nice, but it will be true in general that a blow-up map will be an isomorphism outside what we're blowing up. Here are some other nice properties, which can be checked by hand.
	\begin{itemize}
		\item The map $\pi$ is proper.
		\item We see $\op{Bl}_{(0,0)}\AA^2_k$ is smooth. One can check this directly from the Jacobian.
	\end{itemize}
	Again, the first property is only so nice because we're blowing up is so nice, but the second property will roughly hold in general.
\end{proof}
\begin{example}
	Let's see our blowing up aide in resolution of singularities: define the curve $C\subseteq\AA^2_k$ by $\Spec k[x,y]/\left(y^2-x^3-x^2\right)$, which is smooth everywhere except a node at the origin. To fix this, we blow up at $(0,0)$, for which we consider
	\[\overline{\pi^{-1}(C\setminus\{0\})}_{\mathrm{red}}\subseteq\op{Bl}_{(0,0)}\AA^2_k.\]
	(In general, we want to consider a scheme-theoretic closure, but the Zariski closure works for now.) Let's see what happens on charts.
	\begin{itemize}
		\item On $D_+(Y)$, we are looking at $\Spec k[x,u]/\left(u^2-x-1\right)$ because $y^2=x^3+x^2$ implies $x^2u^2=x^3+x^2$, which is $u^2=x+1$.
		\item Similarly, on $D_+(X)$, we are looking at $\Spec k[y,v]/\left(1-v^3y-v^2\right)$.
	\end{itemize}
	Notably, because $k$ is algebraically closed, we can see that the above glue together into just $\Spec k[u]$.
\end{example}
The point of the above example is that we were able to recover a normalization map from blowing up.

Now that we're motivated to study blowing up, let's define what it is we're talking about. Our definition will be by universal property.
\begin{definition}[Blowing up]
	Fix a closed subscheme $Z$ of a scheme $X$. Then the \textit{blow-up} of $X$ along $Z$ is a scheme $\op{Bl}_ZX$ equipped with a morphism $\pi\colon\op{Bl}_ZX\to X$ such that $\pi^{-1}(Z)$ (which is $Z\times_X\op{Bl}_ZX$) is an affective Cartier divisor. We also require that $(\op{Bl}_ZX,\pi)$ is final with respect to this data: given any $f\colon W\to X$ such that $f^{-1}(Z)\subseteq W$ is an affective Cartier divisor in $W$, then there is a unique $g\colon W\to\op{Bl}_ZX$ making the relevant data commute.
\end{definition}
Wait, what do we mean when we say that a closed subscheme is an effective Cartier divisor?
\begin{remark}
	We'll say that a closed subscheme is an effective Cartier divisor if and only if the corresponding ideal sheaf is a line bundle. One can see this is a sane definition by tracking through what it means for the corresponding Weil divisor to induce an effective Cartier divisor: essentially, we're asking to locally be cut out by a single equation.
\end{remark}
We'll show that we blew up $\AA^2_k$ at $(0,0)$ properly later.

Here are some more words.
\begin{defi}[Exceptional divisor]
	Fix a closed subscheme $Z$ of a scheme $X$ inducing a blow-up map $\pi\colon\op{Bl}_ZX\to X$. We call $\pi^{-1}(Z)$ the \textit{exceptional divisor}.
\end{defi}
\begin{definition}[Total transform]
	Fix a closed subscheme $Z$ of a scheme $X$ with blow-up $\pi\colon\op{Bl}Z_X\to X$. For a closed subscheme $Y\subseteq X$, we call $\pi^{-1}(Y)$ the \textit{total transform of $Y$}, and we call $\pi^{-1}(Y\setminus Z)$ the \textit{strict transform of $Y$}.
\end{definition}
Here are some basic properties.
\begin{lemma}
	Fix a closed subscheme $Z$ of a scheme $X$ with blow-up $\pi\colon\op{Bl}Z_X\to X$. If $Z\subseteq X$ is an effective Cartier divisor, then $\op{Bl}_ZX=X$.
\end{lemma}
\begin{proof}
	One can show directly that the identity satisfies the universal property.
\end{proof}
\begin{lemma}
	Blowing up is functorial: given a closed subscheme $Z$ of a scheme $X$ and a scheme morphism $f\colon X'\to X$, there is a map $\widetilde f$ making the following diagram commute.
	% https://q.uiver.app/?q=WzAsNCxbMCwwLCJcXG9we0JsfV97Zl57LTF9Wn1YJyJdLFsxLDAsIlxcb3B7Qmx9X1pYIl0sWzAsMSwiWCciXSxbMSwxLCJYIl0sWzAsMSwiXFx3aWRldGlsZGUgZiJdLFsyLDMsImYiXSxbMCwyXSxbMSwzXV0=&macro_url=https%3A%2F%2Fraw.githubusercontent.com%2FdFoiler%2Fnotes%2Fmaster%2Fnir.tex
	\[\begin{tikzcd}
		{\op{Bl}_{f^{-1}Z}X'} & {\op{Bl}_ZX} \\
		{X'} & X
		\arrow["{\widetilde f}", from=1-1, to=1-2]
		\arrow["f", from=2-1, to=2-2]
		\arrow[from=1-1, to=2-1]
		\arrow[from=1-2, to=2-2]
	\end{tikzcd}\]
\end{lemma}
\begin{proof}
	Apply the universal property to $f\circ\pi'$.
\end{proof}
\begin{lemma}
	Fix a closed subscheme $Z$ of a scheme $X$ with blow-up $\pi\colon\op{Bl}Z_X\to X$. If $U\subseteq X$ is open, then $\op{Bl}_{Z\cap U}U\simeq\pi^{-1}U$.
\end{lemma}
\begin{proof}
	Omitted.
\end{proof}
The point of the above lemma is that we can construct blow-ups (affine-)locally.

\end{document}