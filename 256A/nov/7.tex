% !TEX root = ../notes.tex

\documentclass[../notes.tex]{subfiles}

\begin{document}

\section{November 7}

The fun continues.

\subsection{Closed Embeddings to Projective Space}
Let's see another example of \autoref{thm:mor-to-proj-scheme}.
\begin{example}
	Fix a field $k$, and we study $\op{Aut}\PP^n_k$. Now, \autoref{thm:mor-to-proj-scheme} tells us that a morphism $\varphi\colon\PP^n_k\to\PP^n_k$ corresponds to a line bundle $\varphi^*\OO(1)$. In particular, this should be a generator of $\Pic\PP^n_k\simeq\ZZ$, so we are either $\OO(1)$ or $\OO(-1)$. However, we cannot have $\OO(-1)$ because $\OO(-1)$ has no nontrivial global sections to generate $\varphi^*\OO(1)$ at all!
	
	We now see that $\varphi$ is given by $\OO(1)$ and $n+1$ global sections generating $\OO(1)$, but we know that $\dim_k\Gamma(\PP^n_k,\OO(1))=n+1$ from the homework, so we just go ahead and choose $n+1$ sections to generate $\OO(1)$ which will be the same as choosing an invertible matrix in $k^{(n+1)\times(n+1)}$. In total, we see that
	\[\op{Aut}\PP^n_k\simeq\op{GL}_{n+1}(k)/k^\times.\]
\end{example}
In \autoref{thm:mor-to-proj-scheme} we constructed morphisms to projective space, but we are really interested in closed embeddings to projective space (e.g., to talk about projective varieties). So we upgrade to the following.
\begin{proposition}
	Fix an affine scheme $\Spec A$. Then a morphism $\varphi\colon X\to\PP^n_A$ corresponding to the tuple $(\mc L,s_0,\ldots,s_n)$ is a closed embedding if and only if the following hold.
	\begin{listroman}
		\item $X_{s_i}$ is affine.
		\item For each $i$, the corresponding ring map $A[y_0,\ldots,y_n]\to\OO_X(X_{s_i})$ given by $y_j\mapsto s_j/s_i$ is surjective.
	\end{listroman}
\end{proposition}
\begin{proof}
	Read Hartshorne. The main point is that closed embeddings are affine and surjective affine-locally, which we can translate over to projective space.
\end{proof}
\begin{ex} % V 17.4.11
	One can use \autoref{thm:mor-to-proj-scheme} to define a $k$-variety which is proper but not projective. Approximately speaking, we can glue a hyperplane in one copy of $\PP^3$ with a quadratic surface in a different copy of $\PP^3$ to make our scheme $X$, and then do this twice in the reverse direction. This is proper because it's just a gluing of proper schemes, but it's not projective because any closed embedding $\iota\colon X\into\PP^N_k$ will induce a line bundle $\iota^*\OO(1)$ by going back and forth between our copies of $\PP^3$ via some degree maps to track the Picard groups.
\end{ex}
\begin{remark}
	If you don't understand the above example, that's fine. One can survive without an example.
\end{remark}
\begin{remark}
	Notably, this example requires dimension to be at least $2$ because all proper curves are projective, which we will see when we discuss Riemann--Roch later.
\end{remark}
\begin{example} % H Exercise II.7.13
	One can construct a proper $k$-variety $X$ containing a $k$-curve $C\subseteq X$ so that the induced map $\Pic X\to\Pic C$ has image in $\Pic^0C$. This implies that $X$ is not projective: an embedding $\iota\colon X\to\PP^n_k$ will have $\OO(1)$ pulled back all the way to $C$ having positive degree. Namely, we can intersect some hyperplanes which will retain intersection number, which is our degree.
\end{example}

\subsection{Ample Line Bundles}
Last time we defined very ample line bundles as a pullback of $\OO(1)$ from an embedding to projective space. However, this can be too strong, so we have the following restriction.
\begin{definition}[Ample]
	Fix a quasicompact quasiseparated scheme $X$ (and finite type over a scheme $S$). A line bundle $\mc L$ on $X$ is \textit{ample} if and only if each finitely generated quasicoherent sheaf $\mc F$ on $X$ has $\mc F\otimes\mc L^{\otimes n}$ generated by its global sections for sufficiently large $n$.
\end{definition}
\begin{remark}
	If $X$ is Noetherian, we are just working with coherent sheaves $\mc F$.
\end{remark}
\begin{example}
	If $X$ is affine and Noetherian, all line bundles $\mc L$ are ample. Namely, coherent sheaves all look like $\widetilde M$, so the tensor product $\widetilde M\otimes\mc L^{\otimes n}$ is still a coherent sheaf always and therefore generated by its global sections.
\end{example}
Let's see some more examples.
\begin{theorem}[Serre]
	Fix an affine base $S$ and a scheme $X$ over $S$ of finite type. Given a closed embedding $\iota\colon X\into\PP^n_S$, then $\mc L\coloneqq\iota^*\OO(1)$ is an ample line bundle on $X$. Moreover, $\mc F\otimes\iota_*\OO(1)^{\otimes n}$ is generated by finitely many global sections for sufficiently large $n$, for each quasicoherent sheaf $\mc F$ on $X$.
\end{theorem}
\begin{proof}
	We begin by reducing to $X=\PP^n_S$, for which we use the following lemma.
	\begin{lemma} % V 17.6.G
		Fix everything as above. Fix a closed embedding $\iota\colon X\to Y$ of $S$-schemes, where $Y$ is a quasicompact and quasiseparated scheme. If $\mc L/Y$ is ample (with the ``moreover'' finiteness condition from before), then $i^*\mc L/X$ is still ample (still with that moreover finiteness condition).
	\end{lemma}
	\begin{proof}
		Fix a quasicoherent sheaf $\mc F$ on $X$ of finite type. Because $\iota$ is a finite morphism, we see that $\iota_*\mc F$ is a quasicoherent sheaf on $Y$ and still of finite type. (This was an exercise on the homework.) Now, $\mc L$ is ample on $Y$, so $i_*\mc F\otimes\mc L^{\otimes n}$ is generated by its global sections, and in fact finitely many by assumption. Thus, we have a surjection
		\[\OO_Y^m\onto\iota_*\mc F\otimes\mc L^{\otimes n}\]
		for sufficiently large $n$. Now, applying $\iota^*$, which we know is right-exact its adjunction (alternatively, note $\iota^{-1}$ is exact, and then the tensor product is right-exact), we get a surjection
		\[\OO_X^m\onto\iota^*(\iota_*\mc F)\to(\iota^*\mc L)^{\otimes n}.\]
		However, because $\iota$ is affine (it's a closed embedding), we see that we have a surjection $\iota^*\iota_*\mc F\onto\mc F$, so using the right-exactness of the tensor product, we get a surjection
		\[\OO_X^m\onto\mc F\otimes(i^*\mc L)^{\otimes n}.\]
		Thus, we are still generated by finitely many global sections, which is what we wanted.
	\end{proof}
	\begin{remark}
		Technically, we will only need $\iota$ to be a quasicompact locally closed embedding. We will make this upgrade later.
	\end{remark}
	The above lemma allows us to reduce to the case of $X=\PP^n_S$. So we write $S=\Spec A$ with $X=\Proj A[Y_0,\ldots,Y_n]$, where
	\[X=\bigcup_{i=0}^nD_+(Y_i).\]
	Now, each $D_+(Y_i)$ is affine, so for any quasicoherent $\mc F$ of finite type, we are granted some finitely many $t_{ij}'\in\mc F(D_+(Y_i))$ which generate $\mc F|_{D_+(Y_i)}$. Namely, we are using the fact that when everything is affine, or quasicoherent sheaves are generated by global sections.

	To finish the proof, we pick up the following lemma.
	\begin{lemma}
		Fix a quasicompact and quasiseparated scheme $X$ and a line bundle $\mc L$ on $X$. Then any global section $f\in\mc L(X)$ and quasicoherent sheaf $\mc F$ on $X$, all $t'$in $\mc F(X_f)$ has some $n$ such that $f^nt'$ is a global section of $\mc F\otimes\mc L^{\otimes n}$.
	\end{lemma}
	\begin{proof}
		This proof is identical to our homework problem Exercise~II.2.16, which we did on the homework.
	\end{proof}
	In total, we see that the lemma allows us to extend each $t_{ij}'$ to some global sections $Y_i^nt_{ij}'$ of $\mc F\otimes\mc L^{\otimes n}$, where we are still finitely generated everywhere.
\end{proof}
\begin{remark}
	More generally, $\iota$ does not have to be a closed embedding: we could set $\mc L$ to be a very ample line bundle with respect to the projection map $f\colon X\to S$, where $\iota\colon X\into\PP^n_S$ is a locally closed embedding with $\mc L=\iota^*\OO(1)$.
\end{remark}
\begin{remark}
	Again, let $X$ be a proper Noetherian scheme over a Noetherian affine scheme $S$. Then a line bundle $\mc L$ is ample over $X$ if and only if $\mc L^{\otimes n}$ is very ample for some positive $n$. We will upgrade this statement later. In particular, Riemann--Roch will tell us that line bundles over curves of sufficiently large degree are very ample, so all line bundles over curves are ample.
\end{remark}


\end{document}