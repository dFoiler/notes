% !TEX root = ../notes.tex

\documentclass[../notes.tex]{subfiles}

\begin{document}

\section{November 9}

Today we hopefully finish our discussion of ample line bundles.

\subsection{Projective Spaces}
We begin class with an aside on the $\Proj$ construction.
\begin{lemma}
	Given a graded ring $A$ which is a finitely generated $A_0$-algebra, we can find a graded ring $A'$ with $A_0=A'_0$ such that $A'$ is finitely generated over $A_0$ by elements in $A_1$ and $\Proj A=\Proj A'$.
\end{lemma}
\begin{proof}[Sketch]
	The idea is to do something like the Veronese embedding. Fix generators $\{a_i\}_{i=1}^n$ be generators of $A$ over $A_0$, and set $d_i\coloneqq\deg a_i$. Now, set $d\coloneqq nd_1d_2\cdots d_n$ and note that the graded ring
	\[A^{(d)}\coloneqq\bigoplus_{m\ge0}A_{md}\]
	is generated by $A_1^{(d)}=A_d$ because we have fit all the generators in there. Then we showed on the homework that $\Proj A^{(d)}\simeq\Proj A$.
\end{proof}
The point is that working with $\Proj A'$ has a good closed embedding to $\PP^n_{A_0}$ for some large enough $n$ by setting $A'=A_0[s_0,\ldots,s_n]$ and using the obvious morphism of graded rings
\[A_0[x_0,\ldots,x_n]\onto A_0[s_0,\ldots,s_n].\]
\begin{remark}
	One can check that \autoref{thm:ample-on-pn} works fine for any $\Proj A$ where $A$ is finitely generated over $A_0$ by finitely many elements in $A_1$.
\end{remark}

\subsection{A Better Ample}
% Ch 12, 13 of GW
We now return to our discussion of ample line bundles.
\begin{restatable}{proposition}{betterampleprop}
	Fix a quasicompact and quasiseparated scheme $X$ of finite type over an affine scheme and a line bundle $\mc L$ on $X$. Then the following are equivalent.
	\begin{listalph}
		\item $\mc L$ is ample; in other words, for each finitely generated quasicoherent sheaf $\mc F$ on $X$ has $\mc F\otimes\mc L^{\otimes n}$ generated by its global sections for sufficiently large $n$.
		\item The open sets $X_f\coloneqq\{x\in X:f_x\text{ generates }\mc L_x^{\otimes n}\simeq\OO_{X,x}\}$ as $n$ varies over all positive integers and $f$ varies over the global sections of $\mc L^{\otimes n}$ form an affine base for the topology of $X$.
		\item There exists some positive integer $d$ such that there are finitely many global sections $f_\bullet$ of $\mc L^{\otimes n}$ where the $X_{f_i}$ cover $X$.
	\end{listalph}
\end{restatable}
\begin{proof}
	We first show (a) implies (b). Fix some $x\in X$, and place it in some affine open $U\subseteq X$. We need to find an $f$ for which $x\in X_f\subseteq U$. To use the ample condition, we need to pick up some quasicoherent sheaves of finite type, so here we go.
	\begin{lem}
		Fix a quasicompact and quasiseparated scheme $X$. Then there is a coherent ideal sheaf $\mc I$ such that $\mc I$ is a finite type quasicoherent sheaf such that $V(\mc I)=X\setminus U$.
	\end{lem}
	\begin{proof}
		In the case that $X$ is Noetherian, then any ideal sheaf $\mc I$ will do. In general, one can reduce to the affine case because $X$ is quasicompact and quasiseparated, and then the fact that $U$ is quasicompact will let us construct $\mc I$.
	\end{proof}
	We now apply the lemma. By definition of being ample applied to the constructed ideal sheaf $\mc I$, we know there is some $n>0$ and a global section $f$ of $\mc I\otimes\mc L^{\otimes n}$ such that $f_x\notin\mf m_x$. (Namely, $x\in U=X\setminus V(\mc I)$ tells us that $\mc I_x\not\subseteq\mf m_x$, so we can find a germ in $\mc I_x$ not in $\mf m_x$, and the ample condition lets us lift this to a global section.)
	
	It follows $x\in X_f\subseteq U$. Indeed, $x\in X_f$ is straight from the construction, and $X_f\subseteq U$ is because any $y\in X\setminus U$ has all $g\in\Gamma(X,\mc I\otimes\mc L^{\otimes n})$ with $f_y$ not invertible by the construction of $\mc I$.\todo{What?}

	Next we show (b) implies (c). Fix $x$ in an affine open subscheme $U\subseteq X$ with some isomorphism $\varphi\colon\mc L|_U\simeq\OO_U$. Now, (b) promises a global section $f$ of $\mc L^{\otimes n}$ such that $x\in X_f\subseteq U$. But then
	\[\varphi(f|_U)\in\Gamma(U,\OO_U)\]
	is affine, so $X_f=D(\varphi(f|_U))$ is an affine open subscheme containing $x$. Then the union of these $X_f$ (over all $x\in X$) covers $X$, and quasicompactness reduces this to a finite cover.

	Lastly, we show (c) implies (a). This requires a little more work because we have to construct a bunch of global sections to generate. Fix our finite affine open cover by $X_{f_i}$s. Let $\mc F$ be some quasicoherent sheaf of finite type so that $\mc F|_{X_{f_i}}$ corresponds to a finitely generated $\OO_X(X_{f_i})$-module. This finite generation promises us some sections $t_{ij}'\in\Gamma(X_{f_i},\mc F)$ generating $\mc F|_{X_{f_i}}$, but then a result from class grants us some $n_0$ such that any $n\ge n_0$ has
	\[t'_{ij}\otimes f_i^{n}\]
	will extend to some global section $t_{ij}\in\Gamma(X,\mc F\otimes\mc L^{\otimes nd})$ where $f_i\in\Gamma(X,\mc L^{\otimes d})$ by \autoref{lem:extend-section}. Intuitively, the point is that our line bundle has some prescribed allowed poles in our global section, which disappear upon by multiplying by sufficiently many powers of $f$. It follows that any $n\ge n_0$ has $\mc F\otimes\mc L^{\otimes nd}$ generated by these finitely many sections.

	We now need to get $\mc F\otimes\mc L^{\otimes n}$ for any $n$ large enough. Well, we apply the above argument to the finitely many
	\[\mc F,\mc F\otimes\mc L,\mc F\otimes\mc L^{\otimes2},\ldots,\mc F\otimes\mc L^{d-1}\]
	and take the maximum of the given $n$s to upgrade our result.
\end{proof}
\begin{corollary}
	Fix a quasicompact and quasiseparated scheme $X$. Given a quasicompact locally closed embedding $\iota\colon Z\to X$, if $\mc L$ is ample on $X$ then $i^*\mc L$ on $Z$.
\end{corollary}
\begin{proof}
	Given any global section $f$ of $\mc L^{\otimes n}$, we note that the adjunction of $i^*$ and $i_*$ promises that
	\[i^*f\in\Gamma(Z,(i^*\mc L)^{\otimes n})\simeq\Gamma(X,i_*(i^*\mc L^{\otimes n})),\]
	which has a map from $\Gamma(X,\mc L^{\otimes n}$. The point is that the $X_f$ form a base for the topology on $X$, so we find $Z_{i^*f}=X_f\cap Z$ will form a base for the topology on $X$.
\end{proof}
\begin{remark}
	In particular, taking $\PP^n$ in the above corollary tells us that very ample implies ample.
\end{remark}
\begin{remark}
	It will turn out that, in the case where $X$ is proper, there will be a cohomological criterion for a line bundle to be ample. Such a thing does not exist for being very ample because being very ample is a bit too rigid.
\end{remark}
We also note that we have the following result.
\begin{proposition} \label{prop:fix-powers-ample}
	Fix an affine base scheme $S$ so that $f\colon X\to S$ is a finite-type morphism making $X$ a quasicompact and quasiseparated scheme. Given a line bundle $\mc L$ on $X$, the following are equivalent.
	\begin{listalph}
		\item $\mc L$ is ample.
		\item There is some positive integer $n$ for which $\mc L^{\otimes n}$ is very ample for $f$.
		\item Each line bundle $\mc L'$ on $X$ has some positive integer $n_0$ such that $\mc L'\otimes\mc L^{\otimes n}$ is very ample for $f$ for all $n\ge n_0$.
	\end{listalph}
\end{proposition}
\begin{proof}
	We show (b) implies (a). Because $\mc L^{\otimes n}$ is very ample, we note that \autoref{thm:ample-on-pn} can be upgraded with the above results to tell us that $\mc L^{\otimes n}$ is actually ample. To continue, we have the following check.
	\begin{lemma}
		Fix a quasicompact and quasiseparated scheme $X$. Then a line bundle $\mc L$ is ample if and only if $\mc L^{\otimes n}$ for some positive integer $n$.
	\end{lemma}
	\begin{proof}
		The forward direction is clear. In the reverse direction, we note that a global section $f\in\Gamma(X,\mc L^{\otimes d})$ has $X_f=X_{f^n}$ while $f^n\in\Gamma(X,\mc L^{\otimes nd})$. So if the $X_f$ form a base for our topology, then the $X_{f_n}$ will also form a base for our topology.
	\end{proof}
	We show that (a) implies (b). The difficult part here is to construct a projective embedding, so we must construct a map to projective space and then check that it is an embedding. Well, because $\mc L$ is ample, we are promised finitely many global sections $f_i\in\Gamma(X,\mc L^{\otimes d})$ such that the $X_{f_i}$ form an affine open cover for $X$.

	Now, let the $a_{ij}'$ be a finite set of generators for $\Gamma(X_{f_i},\OO_X)$ over $A$ and extend them up to global sections $a_{ij}$ extending $a'_{ij}\otimes f_i^{\otimes n}$ for large enough $n$, as usual. Notably, all these global sections are living in some fixed $\mc L^{\otimes dn}$.

	Now, these generators of $\mc L^{\otimes nd}$ given by the $f_i^n$ and $a_{ij}$ form a map $X\to\PP^n_A$. On each affine piece $X_{f_i}=X_{f_i^n}$, we see that the map looks like
	\[A[f_i^n,a_{ij}]\to A\]
	is by $a_{ij}/f_i^n\mapsto a'_{ij}$, which is surjective. Thus, we have constructed a closed embedding.

	It remains to show that both (a) and (b) imply (c). Well, we know that $\mc L^{\otimes n}$ is very ample for some $n$, so we get a locally closed embedding $X\into\PP^{N_2}_S$ for some $N$. Further, using the ample condition, there is some $m$ such that $\mc L'\otimes\mc L^{\otimes m}$ is generated by finitely many global sections, which then gives us another locally closed embedding $X\to\PP^{N_2}_S$.

	To finish, we note that we have a locally closed embedding
	\[X\stackrel{(f_1,f_2)}\to\PP^{N_1}\times\PP^{N_2},\]
	which then embeds to $\PP^{N_3}$ using the Serge embedding. In total, $i\colon X\into\PP^{N_3}$ is still a locally closed embedding. The corresponding line bundle to this locally closed embedding is $i^*\OO_{\PP^{N_3}}(1)$, which we showed on the homework is the same as $\pi_1^*\OO_{\PP^{N_1}}(1)\otimes\pi_2^*\OO_{\PP^{N_2}}(2)$, which then pulls back to $\mc L^{\otimes n}\otimes\mc L'\otimes\mc L^{\otimes m}=\mc L'\otimes\mc L{\otimes(m+n)}$. This finishes.
\end{proof}
The point is that sufficiently large powers make very ample line bundles.
\begin{remark}
	There is a notion of being relatively ample, but we will not discuss it in this class.
\end{remark}

\end{document}