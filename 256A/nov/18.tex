% !TEX root = ../notes.tex

\documentclass[../notes.tex]{subfiles}

\begin{document}

\section{November 18}

Today we finish blowing up.

\subsection{That's How to Blow Up}
Here is our main result.
\blowupconstrunction*
\begin{proof}
	As we discussed last class, we reduce to the affine case, where our closed subscheme $Z$ of $X=\Spec A$ can be given by $\Spec A/I\into\Spec A$; namely, $Z=\Spec A/I$ with $\mc I_Z=\widetilde I$.

	Here, our blow-up looks like $\Proj B$ where $B$ is the graded $A$-algebra
	\[B\coloneqq\bigoplus_{n\ge0}I^n.\]
	To begin, we check that $\pi^{-1}(Z)$ is an effective Cartier divisor. We will just check affine-locally. Given $a\in I$, we note $D_+(a)\subseteq\Proj B$ looks like $\Spec(B_a)_0=\Spec A[I/a]$, where $A[I/a]$ means the subring of $A_a$ generated by elements of the form $x/a$ for $x\in A$. Indeed, we can see this by writing down what $(B_a)_0$ really means. (Notably, $a\in I\subseteq B$ has degree $1$.)

	We now want to compute $\pi^{-1}(Z)$ in $D_+(a)$ to show that we have an effective Cartier divisor. Well, taking the pre-image as viewed in $A[I/a]$, this pre-image is $\Spec A[I/a]/IA[I/a]$, but $IA[I/a]=(a)A[I/a]$ because any element $x\in I$ can be written as $a\cdot x/a$ in $A[I/a]$. Because we are being cut out by a single section on $D_+(a)$, we see that $\pi^{-1}(Z)$ is in fact an effective Cartier divisor here. Because $\{D_+(a)\}_{a\in I}$ covers $\Proj B$ (namely, $B$ is generated by its degree-$1$ elements), this tells us that $\pi^{-1}(Z)$ is in fact an effective Cartier divisor.

	We now check the universal property. Suppose we have a scheme morphism $f\colon W\to X$ where $f^{-1}Z$ is an effective Cartier divisor, and we need a unique map $g\colon W\to\op{Bl}_ZX$ commuting with the rest of our data. Well, because morphisms of schemes glue uniquely, we may assume that $W$ is affine, so set $W=\Spec C$.

	As such, $f$ corresponds to a ring map $f^\sharp\colon A\to C$, and $f^{-1}Z$ being an effective Cartier divisor means that $f^\sharp(I)C$ is locally principal; by checking even more locally, we can just assume that $f^\sharp(I)C=(t)$, where $t\in C$ is not a zero-divisor. Now, for our universal property, we can check that the scheme-theoretic language translates into needing a unique $A$-algebra map $\varphi\colon B\to C$ extending $f^\sharp$.
	
	Well, for this we want to pick up some affine subscheme $D_+(a)\subseteq B$ and define our map $\varphi$ to a smaller $C$. For this, given $x\in I$, we define $\varphi\colon B\to C$ by sending $x\in I$ to
	\[\varphi(x)\coloneqq\frac{f^\sharp(x)}t\in C\]
	on $D_+(a)$. Notably, $t$ is not a zero-divisor, so this element is unique. In particular, we see that this locally looks like $(B_a)_0\to C_{f(a)/t}$ by
	\[\varphi\left(\frac xa\right)=\frac{f^\sharp(x)}t\left(\frac{f^\sharp(a)}t\right)^{-1}=\frac{f^\sharp(x)}{f^\sharp(a)},\]
	so we are in fact extending $f^\sharp$, and our choice of definition for $\varphi$ was in fact the only choice we had to extend $\varphi^\sharp$. Now, because $f^\sharp(I)C=(t)$, we can check that the $\Spec C_{f^\sharp(a)/t}$ do in fact cover $C$, so we have in fact defined a map $\Spec C\to\Proj B$. Running the relevant checks finishes for commutativity.

	We give a few remarks on uniqueness. For example, even though $\varphi$ is not unique---there might be many options for $t$---the actual localized maps we gave $\Spec C_{f^\sharp(a)/t}\to\Spec(B_a)_0$ do not actually care about our choice of $t$.

	The main point is to show that $(\Proj B)\setminus\pi^{-1}Z$ is isomorphic to $X\setminus Z$, which will make $g$ unique because on $g^{-1}Z$ we must have $g$ dictated by $f$. Well, note that the image of $\Spec C_t$ along any $g$ will go to $(\Proj B)\setminus\pi^{-1}(Z)$ will be covered by the $D_+(a)\setminus V(a)$, which is $\Spec A[I/a]_a=\Spec A_a$, which is $D(a)$ of $X$.
	
	So indeed, we'll be able to glue together to an isomorphism $(\Proj B)\setminus\pi^{-1}(Z)\cong X\setminus Z$. As such, the map $\Spec C_t\to(\Proj B)\setminus\pi^{-1}Z$ is unique, namely as $f|_{W\setminus f^{-1}Z}$. This will finish from the following result.
	\begin{lemma}
		Fix everything as above. Given an $X$-scheme $W$ with structure morphism $f\colon W\to X$ such that $D\coloneqq f^{-1}Z$ is an effective Cartier divisor. Then a morphism $W\setminus D\to\Proj B$ of $X$-schemes admits at most one extension making the relevant data commute.
	\end{lemma}
	\begin{proof}
		Note that the structure morphism $\Proj B\to X$ is separated, essentially coming from the proof that the map $\Proj B\to\Spec\ZZ$ is separated. Now, given $X$-morphisms $g_1,g_2\colon W\to\Proj B$, we note that $\op{eq}(g_1,g_2)$ is a closed subscheme of $W$. However, $g_1|_{W\setminus D}=g_2|_{W\setminus D}$ implies that $W\setminus D\subseteq\op{eq}(g_1,g_2)$, so it follows $\op{eq}(g_1,g_2)=W$. Indeed, this last part follows because $W\setminus D$ is open and dense; rigorously, we are using the fact that $D$ is locally cut out by a single equation, and we note that the localization $A\to A_t$ is injective because $t$ is not a zero-divisor.
	\end{proof}
	The above lemma completes the proof.
\end{proof}
\begin{remark}
	Philosophically, the point here is that we were able to extend a ring map $A\to C$ to a map $B\to C$ because our $IC$ was locally generated by a single equation. This sort of thing can be helpful in other situations, for example when we're dealing with projective morphisms where the chosen global sections don't generate.
\end{remark}
\begin{cor}
	Fix a closed subscheme $Z$ of a scheme $X$ such that $\mc I_Z$ is of finite type. Then the blow-up $\pi\colon\op{Bl}_ZX\to X$ is projective, and $\mc I_Z\OO_{\op{Bl}_ZX}=\OO_{\op{Bl}_ZX}(1)$ is very ample with respect to $\pi$.
\end{cor}
\begin{proof}
	The first claim is by the construction of the blow-up, after looking affine-locally. The second claim is again by the construction.
\end{proof}
% Stacks 080E
\begin{corollary}
	Fix a closed subscheme $Z$ of a scheme $X$. Fix an ideal sheaf $\mc I_Y$ of a closed subscheme $i\colon Y\subseteq X$. Then the strict transform of $Y$ in $\op{Bl}_ZX$ is the image of a closed embedding $\op{Bl}_{Z\cap Y}Y\to\op{Bl}_ZX$.
\end{corollary}
\begin{proof}
	We claim the ideal sheaf of the closed embedding $Z\cap Y\to Y$ is $i^{-1}\left((\mc I_Z+\mc I_Y)/\mc I_Y\right)$ in $\OO_Y=i^{-1}(\OO_X/\mc I_Y)$. Well, we compute
	\[\op{Bl}_ZX=\Proj\bigoplus_{n\ge0}\mc I_Z^n\qquad\text{and}\qquad\op{Bl}_{Z\cap Y}Y=\Proj i^{-1}\bigoplus_{n\ge0}\left(\frac{\mc I_Z+\mc I_Y}{\mc I_Y}\right)^n.\]
	Now, there is a natural map between the graded $\OO_X$-algebras
	\[\bigoplus_{n\ge0}\mc I_Z^n\to\bigoplus_{n\ge0}\left(\frac{\mc I_Z+\mc I_Y}{\mc I_Y}\right)^n\]
	by checking affine-locally, and then we note that this last graded $\OO_X$-algebra is just $i_*\left(\bigoplus_{n\ge0}i^{-1}\left((\mc I_Z+\mc I_Y)/\mc I_Y\right)^n\right)$. We would like this map to be the one induced by the universal property, and we can check that it satisfies by hand. In total, we have been given a closed embedding $\op{Bl}_{Z\cap Y}Y\to\op{Bl}_ZX$.

	It remains to verify that this closed embedding is the strict transform. We can check this affine-locally, so set $X=\Spec A$ with $Y=\Spec A/I_Y$ and $Z=\Spec A/I_Z$. Then we can check by hand that the strict transform in $D_+(a)$ is given by
	\[\Spec\frac{A[I_Z/a]}{I_YA[I_Z/a]},\]
	where we also need to mod out the ring by $a$-power torsion. Now that we understand the strict transform and the blow-up, we can check affine-locally that they align.
\end{proof}
Next we talk about the sheaf of differentials.

\end{document}