% !TEX root = ../notes.tex

\documentclass[../notes.tex]{subfiles}

\begin{document}

\section{November 30}

Today we would like to finish our discussion of differentials.

\subsection{Some Computations}
We run some computations on our differentials using the machinery we've built.
\begin{example}
	We use \autoref{cor:compute-diffs}. Take $A=k$ and $B=k[x,y]/\left(y^2-x^3-x\right)$. Here, $\Omega^1_{B/k}$ is generated as a $B$-module by $dx$ and $dy$ with the relation
	\[0=d\left(y^2-x^3-x\right)=2ydy-\left(3x^2+1\right)dy.\]
\end{example}
Our focus today will be on the smooth case. We will want to talk about the tangent space, so we pick up the following result.
\begin{corollary} \label{cor:zar-cotangent-space}
	Fix a $k$-scheme $X$ and some $k$-point $x\in X(k)$. Then
	\[\mf m_x/\mf m_x^2\simeq\Omega^1_{X/k}\otimes_{\OO_X}k(x).\]
	Here, $k(x)$ is just $k$, so the right-hand side is $\iota_x^*\Omega^1_{X/k}$, where $i\colon\{x\}\into X$ is the canonical closed embedding.
\end{corollary}
\begin{proof}
	This question is local around $x\in X(k)$, so we may assume that $X$ is affine. Now, using \autoref{prop:diff-ses-2} on the closed embedding $i\colon\{x\}\into X$, we get
	\[\frac{\mf m_x}{\mf m_x^2}\onto i_x^*\Omega_{X/k},\]
	noting that $\Omega^1_{\{x\}/k}$ vanishes because $\{x\}\simeq\Spec k$. One can use the left-exactness of \autoref{prop:diff-ses-2} to finish because $i$ is smooth; alternatively, \cite[Lemma~6.2.1]{liu-alg-geo-ari} reduces this to the case where $X=\AA^n_k$ and the computes everything by hand.
\end{proof}
\begin{example}
	Suppose $k$ has characteristic $p$. Consider $B=k[x]/\left(x^p-1\right)$ so that $\Spec B$ is the group scheme $\mu_p$. Then $\Omega^1_{\mu_p/k}$ is a free $B$-module generated by $dx$ with no relations (because $k$ has characteristic $p$!). Now, at the point $x=1$ in $B$, we compute
	\[\frac{\mf m_x}{\mf m_x^2}\simeq\frac{(x-1)}{(x-1)^2+\left(x^p-1\right)}\simeq\frac t{\left(t^2,t^p\right)}\simeq k,\]
	where we have set $t=x-1$. Thus, $i^*_x\Omega^1_{\Spec B/\{(x-1)\}}\simeq\Spec k$.
\end{example}

\subsection{Smoothness}
Our discussion for smooth varieties will work with $k$-varieties to be explicit, though it is possible to work more generally.
\begin{proposition}
	Fix a $k$-scheme $X$ locally of finite type. Given some $x\in X$, the following are equivalent.
	\begin{listalph}
		\item $X$ is smooth in a neighborhood of $x$.
		\item $X$ is smooth at $x$.
		\item $\Omega^1_{X/\{x\}}$ is free of rank $\dim_xX\coloneqq\dim\OO_{X,x}$.
	\end{listalph}
\end{proposition}
\begin{proof}
	By definition $X$ is smooth in a neighborhood of $x$ if and only if there is an open neighborhood $U$ around $x$ such that $\OO_X(U)\simeq k[x_1,\ldots,x_m]/(f_1,\ldots,f_r)$ (with $m=r+\dim\OO_{X,x}$) such that the Jacobian matrix is invertible. (This is equivalent to our usual definition by choosing all the equations we need to be invertible and then using Nakayama's lemma to show those are all the equations.) As such, we see that (a) and (b) are equivalent because this condition is local, and being smooth is an open condition (we're asking for some equation to be nonzero).

	Now, note (a) implies (c) by \autoref{cor:compute-diffs}. Indeed, the point is that being invertible lets us rearrange the relations to subtract out rank appropriately. As such, $\Omega^1_{X/\{x\}}$ is in fact free of the desired rank.

	Lastly, we show (c) implies (a). By taking a base-change, we may assume that $k$ is algebraically closed (which notably does not change any of our ranks). But now being smooth is equivalent to being regular, which is equivalent to being regular at all closed points. But now \autoref{cor:zar-cotangent-space} tells us that a closed $\overline k$-point $y$ has
	\[\dim\mf m_y/\mf m_y^2\simeq\Omega^1_{X_{\overline k}/\overline k}\otimes k(y)=\dim\OO_{X,y},\]
	so we have shown regularity.
\end{proof}
The above proposition motivates the following definition.
\begin{definition}[Smooth]
	Fix a $k$-scheme $X$ locally of finite type. Then $X$ is \textit{smooth of dimension $n$} if and only if $\Omega^1_X$ is locally free of rank $n$.
	% V 26.2.2
\end{definition}
\begin{corollary}
	Fix a perfect field $k$ and an irreducible $k$-variety $X$ of dimension $n$. Then there is an open dense subset $U\subseteq X$ which is smooth of dimension $n$.
\end{corollary}
\begin{proof}
	By a spreading out argument, it's enough to show the result at the generic point: namely, we need $\Omega^1_{K(X)/k}$ to be free of rank $n$. But this is now just a computation. Namely, \cite[Theorem~A.1.3]{eisenbud-comm-alg} tells us that the transcendence degree of $(x)$ being $n$ promises us $t_1,\ldots,t_n\in K$ such that $K/k(t_1,\ldots,t_n)$ is finite and separable. So we can use \autoref{cor:compute-diffs} to finish.
\end{proof}
\begin{nex}
	The condition that $k$ is perfect is necessary. Take $k=\FF_p(t)$, which is not perfect. Then $X=\Spec k[x,y]/\left(y^p-x^p-t\right)$ is not smooth anywhere.
\end{nex}
We now discuss our short exact sequences. We start with \autoref{prop:diff-ses-2}.
\begin{proposition}
	Fix some $k$-schemes $X,Y,Z$ locally of finite type and a smooth closed embedding $i\colon X\to Z$ of smooth $Y$-schemes. Then
	\[\mc N_{X/Z}^\lor\to i^*\Omega^1_{Z/Y}\to\Omega^1_{X/Y}\to0\]
	is exact.
	% H II.8.17, V 22.3.7
\end{proposition}
\begin{proof}
	Because $X$ and $Z$ are smooth, it follows that $i\colon X\to Z$ is a regular embedding. Indeed, $i^*\Omega^1_{Z/k}$ and $\Omega^1_{X/k}$ are locally free, so we can find $f_1,\ldots,f_r$ where $r=\codim_ZX$ such that the $\delta(f_\bullet)$ freely generate the kernel of $i^*\Omega^1_{Z/k}\to\Omega^1_{X/k}$. We claim that these $f_\bullet$ assemble into our desired regular sequence: to see this, pick some $k$-rational point $x\in X$, and then we see that the $f_\bullet$ should be linearly independent in $\mf m_x/\mf m_x^2$ and thus be part of a regular sequence, making the $f_\bullet$ regular.

	Now, let $X'$ be the closed subscheme associated to the quasicoherent ideal sheaf $\mc I'$ generated by the $f_\bullet$. However, $X\subseteq X'$ is a closed subscheme of the same dimension, but $X\into X'$ is regular, so $X\into Z$ is regular using the same regular sequence.

	Now that we know $X\into Z$ is regular, we note that $\mc I/\mc I^2$ is locally free of rank $r$ by \cite[22.2.16]{rising-sea}, so it follows that we are exact on the left. Indeed, we can see this on the level of stalks.
\end{proof}

\subsection{The Canonical Sheaf}
Throughout, $X$ will be a smooth $k$-variety.
\begin{definition}[Canonical sheaf]
	Fix a smooth $k$-variety $X$. The \textit{canonical sheaf} $\mc K_X$ is the line bundle $\bigwedge^{\dim X}\Omega^1_{X/k}$, where we are taking the $(\dim X)$th exterior product.
\end{definition}
The point of $\mc K_X$ 
\begin{definition}[Geometric genus]
	Fix a smooth $k$-variety $X$. Then we define the \textit{geometric genus} to be
	\[\rho_g(X)\coloneqq\dim\Gamma(X,\mc K_X).\]
\end{definition}
\begin{remark}
	One can check that $\rho_g$ is a birational invariant for smooth projective $k$-varieties.
\end{remark}
We now do some fact-collection.
\begin{lemma}
	Fix a closed embedding $i\colon Y\to X$ of smooth $k$-schemes, and set $r\coloneqq\codim_XY$. Then
	\[\mc K_Y=i^*\mc K_X\otimes\bigwedge^r\mc N_{X/Y},\]
	where $\mc N_{X/Y}=\left(\mc I/\mc I^2\right)^\lor$ such that $\mc I$ is the quasicoherent ideal sheaf associated to $\iota$.
\end{lemma}
\begin{proof}
	Taking the wedge product of
	\[0\to\mc I/\mc I^2\to i^*\Omega^1_{X/k}\to\Omega^1_{Y/k}\to0\]
	gives the result after tracking everything through.
\end{proof}
\begin{example}
	Fix an elliptic curve $E\coloneqq\Proj k[X,Y,Z]/\left(ZY^2=X^3-aXZ^2+bZ^3\right)$. Then we can compute $\mc K_E$.
	\begin{itemize}
		\item It's possible to break $E$ into affine pieces and just work with those individually. The main point is that $dx/(2y)$ generates $\Omega^1_{E/k}$ at every point, so $\Omega^1_{E/k}\simeq\OO_E$. As such, $\rho_g(E)=1$.
		\item Using the above lemma, we can compute $\mc K_E\simeq i^*\mc K_{\PP^2}\otimes i^*\OO(E)$, where $\OO(E)$ is the line bundle coming from the divisor $E\subseteq\PP^2_k$. One can compute that $\mc K_{\PP^2_k}$ is $\OO(-3)$, and $\OO(E)=3$ because $E$ is defined by an equation of degree $3$, so it follows that $\mc K_E\simeq\OO_E$.
	\end{itemize}
\end{example}

\end{document}