% !TEX root = ../notes.tex

\documentclass[../notes.tex]{subfiles}

\begin{document}

\section{November 2}

Today we give some motivational remarks for what is going on.

\subsection{Homework Exists}
Let's talk about the homework a little. On the homework, we may assume that $k$ is algebraically closed.

Exercise~6.9(a) provides a relationship between $\op{Pic}X$ and $\op{Pic}\widetilde X$, where $X$ is a proper $k$-curve and $\pi\colon\widetilde X\to X$ is the normalization. In particular, the homework shows that $\pi^*$ is surjective, and the kernel can be easily described. Let's discuss surjectivity somewhat.

% Fulton Ch 1, L for pushforward of divisors (check index)
It turns out that understanding push-forwards for finite morphisms will make sense. In some sense, we want to take a finite morphism $f\colon Y_1\to Y_2$ and then just push directly the closed subset of $Y_1$ to $Y_2$ and use the scheme-theoretic image to get the correct multiplicity.

But the main point of our discussion here is the following ``projection'' formula.
\begin{theorem}
	Suppose that $f\colon\widetilde X\to X$ is a finite morphism so that the pull-back and push-forward of divisors both make sense. Fix a Weil divisor $D$ on $X$ ``away'' from the singular points of $X$. Then $f_*(f^*D)=(\deg f)D$.
\end{theorem}
In particular, for our normalization map $\pi\colon\widetilde X\to X$, we can check that $\deg\pi=1$, so we get a good section $\pi_*$ for $X$. This gives surjectivity, in some sense, and it also tells us that the singular points are the problems here. Computing the kernel of $\pi^*\colon\op{Pic}X\to\op{Pic}\widetilde X$ requires us to keep track of principal divisors, which requires some care.
\begin{remark}
	Later in life, we will discuss the Picard group of smooth curves $X$. It turns out that $\op{Pic}X$ has a connected component with a scheme structure, which is its Jacobian. Then to recover the ``generalized Jacobian'' for curves, we can use the short exact sequence
	\[0\to\ker\pi^*\to\op{Pic}X\to\op{Pic}\widetilde X\to0\]
	because we now know how $\ker\pi^*$ and $\op{Pic}\widetilde X$ behave. We will be able to see some concrete examples on the homework.
\end{remark}
On the homework, we will see two different kinds of singularities.
\begin{definition}[Cusp]
	A curve $X$ has a \textit{cusp} if it locally looks like $y^2=x^3$ for some $\AA^2_k$.
\end{definition}
The issue here is that $(0,0)$ is our singularity, where the point is that the ring
\[\frac{k[x,y]}{\left(y^2-x^3\right)}\]
has normalization $k[t]$ given by the maps $x\mapsto t^2$ and $y\mapsto t^3$. Because we are dealing with cubic curves in $\PP^2_k$ on the homework, we may just assume that our cusp looks like $y^2=x^3$.
\begin{definition}[Node]
	A curve $X$ has a \textit{node} if it locally looks like $y^2=x^2(x+c)$ where $c\ne0$, for some $\AA^2_k$.
\end{definition}
Again, $(0,0)$ is our singularity, and the normalization now has two points. Essentially, locally at $(0,0)$, the curves looks like $\Spec k[x,y]/(xy)$.
\begin{remark}
	One can show that these are the only possible singularities for a cubic. I think on the homework we will be allowed to just assume that our cubic curves look like this.
\end{remark}
As a last remark, we need to define some notation.
\begin{notation}
	The group scheme $\mathbb G_a$ is isomorphic to $\AA^1_k$ as schemes, where the addition is given by the addition in $k$. The group scheme $\mathbb G_m$ is isomorphic to $\AA^1_k\setminus\{0\}$ as schemes, where the addition is given by the multiplication in $k^\times$.
\end{notation}

\subsection{Differential Geometry for Algebraic Geometers}
% Griffiths--Harris (Principle of AG, Ch 1)
Take $k=\CC$, and throughout we will let $X$ be a smooth projective $k$-variety. Smoothness, by definition, grants us some local Implicit function theorem, so $X$ locally looks like some open discs, so we can give $X$ the structure of a complex analytic manifold, which we call $X^{\mathrm{an}}$; notably, $X^{\mathrm{an}}$ can be turned into a scheme by using the sheaf of holomorphic functions as our structure sheaf.
\begin{theorem}[Serre's GAGA]
	Fix a smooth projective $\CC$ variety $X$. Then $\op{Pic}X\cong\op{Pic}X^{\mathrm{an}}$.
\end{theorem}
For example, on $X^{\mathrm{an}}$, we have the usual short exact sequence
\[0\to\ZZ\stackrel{2\pi i}\to\OO_{X^{\mathrm{an}}}\stackrel{\exp}\to\OO_{X^{\mathrm{an}}}^\times\to0\]
which then gives rise to the exact sequence as follows.
% https://q.uiver.app/?q=WzAsNyxbMCwwLCIwIl0sWzEsMCwiSF4xKFhee1xcbWF0aHJte2FufX0sXFxaWikiXSxbMiwwLCJIXjEoWF57XFxtYXRocm17YW59fSxcXE9PX3tYXntcXG1hdGhybXthbn19fSkiXSxbMywwLCJIXjEoWF57XFxtYXRocm17YW59fSxcXE9PX3tYXntcXG1hdGhybXthbn19fV5cXHRpbWVzKSJdLFsxLDEsIkheMihYXntcXG1hdGhybXthbn19LFxcWlopIl0sWzIsMSwiSF4yKFhee1xcbWF0aHJte2FufX0sXFxPT197WF57XFxtYXRocm17YW59fX0pIl0sWzMsMSwiXFxjZG90cyJdLFswLDFdLFsxLDJdLFsyLDNdLFszLDRdLFs0LDVdLFs1LDZdXQ==&macro_url=https%3A%2F%2Fraw.githubusercontent.com%2FdFoiler%2Fnotes%2Fmaster%2Fnir.tex
\[\begin{tikzcd}
	0 & {H^1(X^{\mathrm{an}},\ZZ)} & {H^1(X^{\mathrm{an}},\OO_{X^{\mathrm{an}}})} & {H^1(X^{\mathrm{an}},\OO_{X^{\mathrm{an}}}^\times)} \\
	& {H^2(X^{\mathrm{an}},\ZZ)} & {H^2(X^{\mathrm{an}},\OO_{X^{\mathrm{an}}})} & \cdots
	\arrow[from=1-1, to=1-2]
	\arrow[from=1-2, to=1-3]
	\arrow[from=1-3, to=1-4]
	\arrow[from=1-4, to=2-2]
	\arrow[from=2-2, to=2-3]
	\arrow[from=2-3, to=2-4]
\end{tikzcd}\]
Namely, we see that we have a $0$ on the leftmost point because the map $\OO_{X^{\mathrm{an}}}\to\OO_{X^{\mathrm{an}}}^\times$ is surjective even after taking global sections, where we are using the fact that $X$ should be proper.

Now, $H^1(X^{\mathrm{an}},\OO^\times_{X^{\mathrm{an}}})=\op{Pic}X^{\mathrm{an}}$ by thinking of this cohomology as a line bundle, and $H^2(X^{\mathrm{an}},\ZZ)$ is the same as the usual singular cohomology. Thus, we have an exact sequence
\[0\to\frac{H^1(X^{\mathrm{an}},\OO_{X^{\mathrm{an}}})}{H^1(X^{\mathrm{an}},\ZZ)}\to\op{Pic}X^{\mathrm{an}}\to\ker\left(H^2(X^{\mathrm{an}},\ZZ)\to H^2(X^{\mathrm{an}},\OO_{X^{\mathrm{an}}})\right)\to0.\]
The right-hand side here is $\op{NS}(X)$, the N\'eron--Severi group, which (conjecturally) behaves pretty well as a direct object. For example, this is a finitely generated abelian group because $H^2(X^{\mathrm{an}},\ZZ)$ is a vector space.

Now, Hodge theory says that $H^1(X^{\mathrm{an}},\OO_{X^{\mathrm{an}}})$ is approximately $\CC^g$ where $g$ is half the total dimension of $X^{\mathrm{an}}$, and the denominator $H^1(X^{\mathrm{an}},\ZZ)$ is a lattice inside it of full rank. Thus, our left-hand object is $\CC^g/\ZZ^{2g}$, which is an abelian variety.

So the point is that $\op{Pic}X^{\mathrm{an}}$ can be put as an extension of an abelian variety by a discrete object, which should roughly be what we expect from the homework.
\begin{remark}
	The Lefschetz hyperplane theorem states that a smooth irreducible projective $\CC$-variety $X\subseteq\PP^n_\CC$ will have some hyperplane $H\subseteq\PP^n_\CC$ where $Y\coloneqq X\cap H$ (which is a smooth irreducible $\CC$-variety) has the induced map
	\[\pi_1(Y)\to\pi_1(X)\]
	has a surjection if $\dim Y\ge2$ and an isomorphism if $\dim Y=1$. The point here is that we can compute $\pi_1(X)$ by some inductive process by cutting out hyperplanes, gradually dropping the dimension.
\end{remark}
\begin{remark}
	At the end of the day, we're saying that algebraically trivial line bundles on $X$ can be pulled back to line bundles on $X^{\mathrm{an}}$, where we can use tools of Hodge theory and things.
\end{remark}

\end{document}