% !TEX root = ../notes.tex

\documentclass[../notes.tex]{subfiles}

\begin{document}

\section{November 20}

Today we discuss the sheaf of differentials.
% H II.8, V S22

\subsection{Affine Differentials}
For simplicity, we begin with the affine case. Throughout, we fix an $A$-algebra $B$ and a $B$-module $M$.
\begin{definition}[Derivation]
	Fix an $A$-algebra $B$ and a $B$-module $M$. Then an \textit{$A$-derivation from $B$ to $M$} is an $A$-linear map $d\colon B\to M$ satisfying the ``Leibniz rule''
	\[d(b_1b_2)=b_1d(b_2)+b_2d(b_1).\]
	We let $\op{Der}_A(B,M)$ denote the set of $A$-derivations from $B$ to $M$.
\end{definition}
\begin{remark}
	Given an $A$-derivation $d$ from $B$ to $M$, we claim that $d(a)=0$ for each $a\in A$. Indeed, setting everything to $1$ in the Leibniz rule, we see
	\[d(1)=d(1)+d(1),\]
	so $d(1)=0$ follows. Thus, $A$-linearity implies that $d(a)=a\cdot d(1)=0$ for each $a\in A$.
\end{remark}
\begin{lemma}
	Fix an $A$-algebra $B$ and a $B$-module $M$. Then the set of derivations $\op{Der}_A(B,M)$ is a $B$-submodule of $\op{Hom}_A(B,M)$.
\end{lemma}
\begin{proof}
	The $B$-action on $\op{Hom}_A(B,M)$ is pointwise on the target $M$. To check we have a $B$-submodule, fix $b_1,b_2\in B$ and $d_1,d_2\in\op{Der}_A(B,M)$, we need to show $b_1d_1+b_2d_2\in\op{Der}_A(B,M)$. We have two checks.
	\begin{itemize}
		\item Linearity: technically, this follows because $\op{Hom}_A(B,M)$ is already a $B$-module, but we will show this anyway. Well, for $a_1',a_2'\in A$ and $b_1',b_2'\in B$, we compute
		\begin{align*}
			(b_1d_1+b_2d_2)(a_1'b_1'+a_2'b_2') &= b_1d_1(a_1'b_1'+a_2'b_2')+b_2d_2(a_1'b_1'+a_2'b_2') \\
			&= b_1\big(a_1'd_1(b_1')+a_2'd_1(b_2')\big)+b_2\big(a_1'd_2(b_1')+a_2'd_2(b_2')\big) \\
			&= b_1\big(a_1'd_1(b_1')+a_2'd_1(b_2')\big)+b_2\big(a_1'd_2(b_1')+a_2'd_2(b_2')\big) \\
			&= a_1'\big(b_1d_1(b_1')\big)+a_2'\big(b_1d_1(b_2')\big)+a_1'\big(b_2d_2(b_1')\big)+a_2'\big(b_2d_2(b_2')\big) \\
			&= a_1'\big((b_1d_1+b_2d_2)(b_1')\big)+a_2'\big((b_1d_1+b_2d_2)(b_2')\big),
		\end{align*}
		which is what we wanted.
		\item Leibniz: pick up any $b_1',b_2'\in B$, and we compute
		\begin{align*}
			(b_1d_1+b_2d_2)(b_1'b_2') &= (b_1d_1+b_2d_2)(b_1'b_2') \\
			&= b_1d_1(b_1'b_2')+b_2d_2(b_1'b_2') \\
			&= b_1\big(b_1'd_1(b_2')+b_2'd_1(b_1')\big)+b_2\big(b_1'd_2(b_2')+b_2'd_2(b_1')\big) \\
			&= b_1b_1'd_1(b_2')+b_1b_2'd_1(b_1')+b_2b_1'd_2(b_2')+b_2b_2'd_2(b_1') \\
			&= b_1'\big(b_1d_1(b_2')+b_2d_2(b_2')\big)+b_2'\big(b_1d_1(b_1')+b_2d_2(b_1')\big) \\
			&= b_1'(b_1d_1+b_2d_2)(b_2')+b_2'(b_1d_1+b_2d_2)(b_1').
		\end{align*}
	\end{itemize}
	The above checks complete the proof.
\end{proof}
\begin{exe}
	Fix a ring $A$ and $B=M\coloneqq A[x]$. Then the usual derivative $d\colon B\to M$ defined by
	\[d\colon\sum_{i=0}^\infty a_ix^i\mapsto\sum_{i=0}^\infty (i+1)a_{i+1}x^i\]
	is an $A$-derivation of $A[x]$ on $A[x]$. In fact, $\op{Der}_A(A[x],A[x])$ is a free $A[x]$-module generated by $d$ above.
\end{exe}
\begin{proof}
	We begin by showing that $d$ is in fact a derivation.
	\begin{itemize}
		\item Linearity: given $a,b\in A$ and polynomials $\sum_{i=0}^\infty a_ix^i$ and $\sum_{i=0}^\infty b_ix^i$, we compute
		\begin{align*}
			d\left(a\sum_{i=0}^\infty a_ix^i+b\sum_{i=0}^\infty b_ix^i\right) &= d\left(\sum_{i=0}^\infty (aa_i+bb_i)x^i\right) \\
			&= \sum_{i=0}^\infty(i+1)(aa_{i+1}+bb_{i+1})x^i \\
			&= a\sum_{i=0}^\infty(i+1)a_{i+1}x^i+b\sum_{i=0}^\infty(i+1)b_{i+1}x^i \\
			&= ad\left(\sum_{i=0}^\infty a_ix^i\right)+bd\left(\sum_{i=0}^\infty b_ix^i\right).
		\end{align*}
		\item Leibniz: we will freely use linearity here. Given $i,j\ge0$ where $i+j\ge1$, the main point is to extend
		\begin{equation}
			d\left(x^i\cdot x^j\right)=d\left(x^{i+j}\right)=(i+j)x^{i+j-1}=x^id\left(x^j\right)+x^jd\left(x^i\right) \label{eq:monomial-leibniz}
		\end{equation}
		linearly. Note this computation also works when $i=0$ or $j=0$; the last case of $i=j=0$ still has the total equation $d\left(x^i\cdot x^j\right)=0=x^id\left(x^j\right)+x^jd\left(x^i\right)$.

		We now attack the statement directly. Fix polynomials $\sum_{i=0}^\infty a_ix^i$ and $\sum_{j=0}^\infty b_jx^j$. Then we compute
		\[d\left(\Bigg(\sum_{i=0}^\infty a_ix^i\Bigg)\left(\sum_{j=0}^\infty b_jx^j\right)\right) = d\left(\sum_{n=0}^\infty\Bigg(\sum_{i+j=n}a_ib_j\Bigg)x^n\right) = \sum_{n=0}^\infty\Bigg(\sum_{i+j=n}a_ib_j\Bigg)d\left(x^n\right),\]
		where we have used linearity at the end. Applying \autoref{eq:monomial-leibniz}, we can rearrange to
		\begin{align*}
			d\left(\Bigg(\sum_{i=0}^\infty a_ix^i\Bigg)\left(\sum_{j=0}^\infty b_jx^j\right)\right) &= \sum_{n=0}^\infty\Bigg(\sum_{i+j=n}a_ib_j\left(x^id\left(x^j\right)+x^jd\left(x^i\right)\right)\Bigg) \\
			&= \sum_{n=0}^\infty\Bigg(\sum_{i+j=n}a_ib_j\left(x^id\left(x^j\right)\right)\Bigg)+\sum_{n=0}^\infty\Bigg(\sum_{i+j=n}a_ib_jx^jd\left(x^i\right)\Bigg) \\
			&= \Bigg(\sum_{i=0}^\infty a_ix^i\Bigg)\Bigg(\sum_{j=0}^\infty b_jd\left(x^j\right)\Bigg)+\Bigg(\sum_{j=0}^\infty b_jx^j\Bigg)\Bigg(\sum_{n=0}^\infty a_id\left(x^i\right)\Bigg) \\
			&= \Bigg(\sum_{i=0}^\infty a_ix^i\Bigg)d\Bigg(\sum_{j=0}^\infty b_jx^j\Bigg)+\Bigg(\sum_{j=0}^\infty b_jx^j\Bigg)d\Bigg(\sum_{n=0}^\infty a_ix^i\Bigg),
		\end{align*}
		where we have again used linearity at the end.
	\end{itemize}
	It remains to show that $\op{Der}_A(A[x],A[x])$ is a free $A[x]$-module generated by $d$. Namely, we need to show that the $A[x]$-module morphism $\varphi\colon A[x]\to\op{Der}_A(A[x],A[x])$ given by $p(x)\mapsto p(x)\cdot d$ is an isomorphism. These are both $A[x]$-modules already, so $\varphi$ is $A[x]$-linear.

	Thus, to show that $\varphi$ is an isomorphism, it suffices to show that $\varphi$ is bijective, for which we will give a set-theoretic inverse. Indeed, define $\psi\colon\op{Der}_A(A[x],A[x])\to A[x]$ by $\psi(d')\coloneqq d'(x)$. Here are our checks.
	\begin{itemize}
		\item We check that $\psi\circ\varphi$ is the identity. Indeed, given $p(x)\in A[x]$, we see
		\[\psi(\varphi(p(x)))=\psi(p(x)\cdot d)=\big(p(x)\cdot d\big)(x)=p(x)\cdot d(x)=p(x)\cdot1=p(x).\]
		\item We check that $\varphi\circ\psi$ is the identity. Well, pick up some derivation $d'\in\op{Der}_A(A[x],A[x])$, and set $d''\coloneqq\varphi(\psi(d))$. We want to show $d=d''$. For this, observe that
		\[d''(x)=\varphi(\psi(d))(x)=\varphi(d'(x))(x)=\big(d'(x)\cdot d\big)(x)=d'(x)\cdot d(x)=d'(x).\]
		Thus, we claim that $d''\left(x^n\right)=d'\left(x^n\right)$ for all $n\in\NN$. This is by induction: for $n=0$, we see
		\[d''\left(x^n\right)=d''(1)=0=d'(1)=d'\left(x^n\right).\]
		Now, given $d''\left(x^n\right)=d'\left(x^n\right)$, we compute
		\begin{align*}
			d''\left(x^{n+1}\right) &= d''\left(x\cdot x^n\right) \\
			&= x\cdot d''\left(x^n\right)+x^n\cdot d''(x) \\
			&\stackrel*= x\cdot d'\left(x^n\right)+x^n\cdot d'(x) \\
			&= d'\left(x^n\right),
		\end{align*}
		where $\stackrel*=$ is by the inductive hypothesis and $d''(x)=d'(x)$ above.

		We now finish the proof. For any polynomial $\sum_{i=0}^\infty a_ix^i$, the $A$-linearity of $d''$ and $d'$ give
		\[d''\Bigg(\sum_{i=0}^\infty a_ix^i\Bigg)=\sum_{i=0}^\infty a_id''\left(x^i\right)=\sum_{i=0}^\infty a_id'\left(x^i\right)=d'\Bigg(\sum_{i=0}^\infty a_ix^i\Bigg).\]
		Thus, $d'=d''$ follows.
		\qedhere
	\end{itemize}
\end{proof}
\begin{remark}
	More generally, given a ring $A$ with $B=M=A[x_1,\ldots,x_n]$, taking the ``partial derivative'' with respect to $x_i$ defines a derivation $d_i$, and $\op{Der}_A(B,M)$ is a free $B$-module generated by these $d_i$. We could show this now with some more effort, but it will fall out of theory we establish later.
\end{remark}
We are now ready to give the first definition of our differentials.
\begin{definition}[Sheaf of differentials]
	Fix an $A$-algebra $B$ and a $B$-module $M$. Define $I$ to be the kernel of the diagonal embedding $B\otimes_AB\to B$. Then the \textit{sheaf of differentials} is the $B$-module $\Omega^1_{B/A}\coloneqq I/I^2$.
\end{definition}
\begin{remark}
	In the general case, we will take a separated morphism $X\to Y$ and define $\mc I$ to be the ideal sheaf of the closed embedding $\Delta\colon X\to X\times_YX$. One can check that $\mc I/\mc I^2$ is in fact an $\mathcal O_X$-module because we've modded out by $\mc I^2$.
\end{remark}
\begin{remark}
	There is a map $d_{B/A}\colon B\to\Omega^1_{B/A}$ by sending $b\in B$ to $(1\otimes b-b\otimes1)$. Notably, $1\otimes b-b\otimes1$ is in fact in the kernel of the diagonal map $B\otimes_AB\to B$. In fact, $d_{B/A}$ is a derivation: this is $A$-linear by our tensor products. To check the Leibniz rule, we note that $\Omega^1_{B/A}$ has two $B$-module structures by $b\mapsto(b\otimes1)$ or $b\mapsto(1\otimes b)$, but these are the same. As such,
	\[d_{B/A}(b_1b_2)=1\otimes(b_1b_2)-(b_1b_2)\otimes1=b_1(1\otimes b_2)-b_2(b_1\otimes1)=b_1d_{B/A}(b_2)+b_2d_{B/A}(b_1).\]
\end{remark}
\begin{remark}
	Let's try to motivate our definition of $\Omega^1_{B/A}$. Fix a morphism $X\to Y$ of complex manifolds. Letting $T_\bullet$ denote the tangent bundle, we have a short exact sequence
	\[0\to T_{\Delta X}\to T_{X\times X}|_{\Delta X}\to N_{\Delta/(X\times X)}\to0.\]
	However, $N_{\Delta/(X\times X)}$ is isomorphic to $T_X$ by sending the vector corresponding to $(v_1,v_2)$ to $v_1-v_2$. As such, our $I/I^2$ is some kind of dual of the normal bundle isomorphic to the tangent bundle, which is roughly we want out of $\Omega^1$.
\end{remark}

\subsection{Scheme Differentials}
Throughout, fix a morphism $f\colon X\to Y$ of schemes. We want to define $\Omega^1_{X/Y}$. The point is to pay attention to directions only in $X$ for our derivations.
\begin{definition}[Sheaf of differentials]
	Fix a scheme morphism $f\colon X\to Y$. Because $\Delta\colon X\to X\times_YX$ is a locally closed embedding, we can decompose it into a closed embedding $i\colon X\into U$ and an open embedding $U\into X\times_YX$. Then we define $\mc I$ as the kernel of the map $\OO_U\onto i_*\OO_X$ and set $\Omega^1_{X/Y}\coloneqq\mc I/\mc I^2$. Notably, $\Omega^1_{X/Y}$ is quasicoherent.
\end{definition}
It's a little difficult to handle this, so we define a few more objects.
\begin{definition}[Infinitesimal neighborhood]
	Fix a scheme morphism $f\colon X\to Y$. Then we define the \textit{first infinitesimal neighborhood $P^1_{X/Y}$ of $X$ in $X\times_YX$} is the closed subscheme of $U$ corresponding to $\mc I^2$ in the above definition.
\end{definition}
\begin{remark}
	One can check that $P^1_{X/Y}$ satisfies the following universal property: for any closed embedding $T_0\into T_1$ making the diagram
	% https://q.uiver.app/?q=WzAsNCxbMCwwLCJUXzAiXSxbMSwwLCJUXzEiXSxbMCwxLCJYIl0sWzEsMSwiWFxcdGltZXNfWVgiXSxbMCwxLCIiLDAseyJzdHlsZSI6eyJ0YWlsIjp7Im5hbWUiOiJob29rIiwic2lkZSI6InRvcCJ9fX1dLFsxLDNdLFsyLDMsIiIsMix7InN0eWxlIjp7InRhaWwiOnsibmFtZSI6Imhvb2siLCJzaWRlIjoidG9wIn19fV0sWzAsMl1d&macro_url=https%3A%2F%2Fraw.githubusercontent.com%2FdFoiler%2Fnotes%2Fmaster%2Fnir.tex
	\[\begin{tikzcd}
		{T_0} & {T_1} \\
		X & {X\times_YX}
		\arrow[hook, from=1-1, to=1-2]
		\arrow[from=1-2, to=2-2]
		\arrow[hook, from=2-1, to=2-2]
		\arrow[from=1-1, to=2-1]
	\end{tikzcd}\]
	commute, where $T_0\into T_1$ has ``square zero.''
\end{remark}
\begin{remark}
	Expanding out our definitions, we get a short exact sequence
	\[0\to\mc I/\mc I^2\to\OO_{P^1_{X/Y}}\to\OO_X\to0\]
	of sheaves on $X$.
\end{remark}
We would now like to give the derivation $d\colon B\to\Omega^1_{B/A}$. To see this, we check affine-locally, where we are looking at
\[0\to\Omega^1_{B/A}\to P^1_{B/A}\to B\to0.\]
Explicitly, $P^1_{B/A}=(B\otimes_AB)/I^2$. Now, there are two maps $B\to P^1_{B/A}$ which split this short exact sequence, given by $p_1\colon b\mapsto b\otimes1$ and $p_2\colon b\mapsto1\otimes b$, and the derivation in the affine case is the difference between these two.

Gluing these affine pieces together, we see that the short exact sequence
\[0\to\mc I/\mc I^2\to P^1_{B/A}\to B\to0\]
will have two splittings $p_i\colon P^1_{B/A}\to X\times_YX\to X$, where the last map is the projection onto the $i$th coordinate. As such, we define $d_{X/Y}\colon\OO_X\to\Omega^1_{X/Y}$ as $p_2^*-p_1^*$, which we can see is an $f^{-1}\OO_Y$-derivation.
\begin{remark}
	One can check by hand that $\Omega^1_{B/A}$ is generated by $B$ over the image $d_{B/A}(B)$. Indeed, we can write
	\[\sum_i(b_i\otimes x_i)=\sum_i(b_i\otimes1)(1\otimes x_i-x_i\otimes1)+\sum_i(b_ix_i)\otimes1,\]
	but the rightmost term vanishes because $\sum_i(b_i\otimes x_i)$ living in $\Omega^1_{B/A}$ requires $\sum_ib_ix_i=0$.
\end{remark}
\begin{example}
	Take $B=A[x_1,\ldots,x_n]$. Then we can show that $\Omega^1_{B/A}$ is a free $B$-module generated by the $d_{B/A}(x_i)$.
\end{example}

\subsection{Differentials by Universal Property}
We now define differentials by universal property.
\begin{definition}[Derivation]
	Fix a scheme morphism $f\colon X\to Y$. Then an $\mathcal O_X$-module $\mc F$, a \textit{$Y$-derivation from $\OO_X$ to $\mc F$} is an $f^{-1}\OO_Y$-linear map $d\colon\OO_X\to\mc F$ satisfying the Leibniz rule on affine open subschemes.
\end{definition}
And here is our universal property.
\begin{restatable}{proposition}{differentialup} \label{prop:diff-up}
	Fix a scheme morphism $f\colon X\to Y$. Then the derivation $d_{X/Y}\colon\OO_X\to\Omega^1_{X/Y}$ is universal in the following sense: for any $\OO_X$-module $\mc F$, we see
	\[\op{Hom}_{\OO_X}(\Omega^1_{X/Y},\mc F)\simeq\op{Der}_Y(\OO_X,\mc F)\]
	by $\alpha\mapsto\alpha\circ d_{X/Y}$.
\end{restatable}
\noindent Philosophically, what's happening is that $\Omega^1_{X/Y}$ can ``see'' all possible derivations.
% Stacks 08S2, 08RM; EGA 4.16

\end{document}