% !TEX root = ../notes.tex

\documentclass[../notes.tex]{subfiles}

\begin{document}

\section{November 28}

We continue discussing differentials.

\subsection{Proving the Universal Property}
Today we show \autoref{prop:diff-up}, in the case where $\mc F$ is quasicoherent. (This is still enough to determine $\Omega^1_{X/Y}$ up to unique isomorphism as a quasicoherent sheaf.)
\begin{proof}[{Proof of \autoref{prop:diff-up}}]
	Everything is local, and we discussed what $\Omega^1_{X/Y}$ looks like affine-locally, so we may reduce to the affine case. As such, we may set $X\coloneqq\Spec B$ and $Y\coloneqq\Spec A$ so that $B$ is an $A$-algebra; note we can write $\mc F=\widetilde M$ for some $B$-module $M$. We divide the proof into a few steps.
	\begin{itemize}
		\item We show the given map is injective. Recall that $\Omega^1_{B/A}=B\cdot d_{B/A}(B)$. Now, given a map $\alpha\colon\Omega^1_{B/A}\to M$ such that $\alpha\circ d_{B/A}$ is the zero map, then
		\[\alpha|_{d_{B/A}}=0.\]
		However, $\Omega^1_{X/Y}$ is generated by these elements $d_{B/A}b$ for $b\in B$, so $\alpha=0$.
		\item We show that the given map is surjective. We define
		\[D_B(M)\coloneqq B\oplus M,\]
		and we give this $B$-module a ring structure by $(b_1,m_1)\cdot(b_2,m_2)=(b_1b_2,b_2m_1+b_1m_2)$. (This is the computation that comes out of a tangent vector computation.) Now, we have a split short exact sequence of $B$-modules
		\[0\to M\to B\oplus M\stackrel\pi\to B\to0,\]
		but now a derivation $d\in\op{Der}_A(B,M)$ is in one-to-one correspondence with $A$-algebra maps $f\colon B\to D_B(M)$ such that $\pi\circ f=\id_B$ by sending $d$ to the map $f_d\colon b\mapsto(b,db)$. At this point we can see that our multiplication law is intended to exactly make the Leibniz rule work out.

		We now want to show that this data is the same as $B$-algebra maps $I/I^2\to M$, where $I$ is the kernel of the projection map $B\otimes_AB\to B$. Well, we note that we can construct a map $\varphi\colon B\otimes_AB\to D_B(M)$ by $b_1\otimes b_2\mapsto b_1\cdot f_d(b_2)$, where $d\colon B\to M$ is some chosen derivation. Roughly speaking, this map is the right map induced by
		% https://q.uiver.app/?q=WzAsNCxbMCwwLCJYIl0sWzAsMSwiWCJdLFsxLDAsIlxcU3BlYyBEX0IoTSkiXSxbMSwxLCJYXFx0aW1lc19ZWCJdLFswLDIsIiIsMCx7InN0eWxlIjp7InRhaWwiOnsibmFtZSI6Imhvb2siLCJzaWRlIjoidG9wIn19fV0sWzAsMSwiIiwyLHsibGV2ZWwiOjIsInN0eWxlIjp7ImhlYWQiOnsibmFtZSI6Im5vbmUifX19XSxbMiwzLCIiLDAseyJzdHlsZSI6eyJib2R5Ijp7Im5hbWUiOiJkYXNoZWQifX19XSxbMSwzLCJcXERlbHRhIiwwLHsic3R5bGUiOnsidGFpbCI6eyJuYW1lIjoiaG9vayIsInNpZGUiOiJ0b3AifX19XV0=&macro_url=https%3A%2F%2Fraw.githubusercontent.com%2FdFoiler%2Fnotes%2Fmaster%2Fnir.tex
		\[\begin{tikzcd}
			X & {\Spec D_B(M)} \\
			X & {X\times_YX}
			\arrow[hook, from=1-1, to=1-2]
			\arrow[Rightarrow, no head, from=1-1, to=2-1]
			\arrow[dashed, from=1-2, to=2-2]
			\arrow["\Delta", hook, from=2-1, to=2-2]
		\end{tikzcd}\]
		on the geometric side. (One can check that the top map is a ``thickening of order $1$'' in that the corresponding quasicoherent ideal sheaf has square zero, so the right diagonal map will factor through $P^1_{X/Y}$.) Thus, $\varphi$ will induce a map $\overline\varphi\colon I\to M$ by noting
		% https://q.uiver.app/?q=WzAsMTAsWzAsMCwiMCJdLFsxLDAsIkkiXSxbMSwxLCJNIl0sWzIsMCwiQlxcb3RpbWVzX0FCIl0sWzMsMCwiQiJdLFszLDEsIkIiXSxbNCwwLCIwIl0sWzQsMSwiMCJdLFsyLDEsIkRfQihNKSJdLFswLDEsIjAiXSxbMCwxXSxbMSwzXSxbMyw0XSxbNCw2XSxbOSwyXSxbMiw4XSxbOCw1XSxbNSw3XSxbNCw1LCIiLDEseyJsZXZlbCI6Miwic3R5bGUiOnsiaGVhZCI6eyJuYW1lIjoibm9uZSJ9fX1dLFszLDgsIlxcdmFycGhpIl0sWzEsMiwiXFxvdmVybGluZVxcdmFycGhpIl1d&macro_url=https%3A%2F%2Fraw.githubusercontent.com%2FdFoiler%2Fnotes%2Fmaster%2Fnir.tex
		\[\begin{tikzcd}
			0 & I & {B\otimes_AB} & B & 0 \\
			0 & M & {D_B(M)} & B & 0
			\arrow[from=1-1, to=1-2]
			\arrow[from=1-2, to=1-3]
			\arrow[from=1-3, to=1-4]
			\arrow[from=1-4, to=1-5]
			\arrow[from=2-1, to=2-2]
			\arrow[from=2-2, to=2-3]
			\arrow[from=2-3, to=2-4]
			\arrow[from=2-4, to=2-5]
			\arrow[Rightarrow, no head, from=1-4, to=2-4]
			\arrow["\varphi", from=1-3, to=2-3]
			\arrow["\overline\varphi", from=1-2, to=2-2]
		\end{tikzcd}\]
		commutes. Because $M^2=0$ in $D_B(M)$, we see that $I^2$ goes to $0$ from the map $I\to M$, so we have defined a map $\overline\varphi\colon I/I^2\to M$.

		We can now check that $\overline\varphi$ is the map we wanted: we can compute the composite
		\[\arraycolsep=1.4pt\begin{array}{ccccc}
			B &\stackrel{d_{X/Y}}\to& \Omega^1_{X/Y} &\to& M \\
			b &\mapsto& 1\otimes b-b\otimes1 &\mapsto& 1f_d(B)-bf_d(1)
		\end{array}\]
		where the last part we can compute out as $(0,db)$.
		\qedhere
	\end{itemize}
\end{proof}
\begin{example}
	For example, we can compute that
	\[\op{Hom}_{\OO_X}(\Omega^1_{X/Y},\OO_X)\simeq\op{Der}_Y(\OO_X,\OO_X).\]
\end{example}
\begin{example}
	Set $Y=\Spec A$ and $X=\AA^n_A$. Now,
	\[\op{Hom}_B(\Omega^1_{B/A},M)=\op{Der}_A(B,M),\]
	which we showed above are the homomorphisms $f\colon B\to D_B(M)$ such that $f(b)=(b,m)$ for some $m\in M$ for each $b\in B$. Then one can check that we can recover such an $f$ by being told where each $x_i$ goes, so this is isomorphic to $M^{\oplus n}$. Thus, the Yoneda lemma tells us that $\Omega^1_{B/A}\simeq k[dx_1,\ldots,dx_n]$.
\end{example}

\subsection{Some Differential Facts}
We now do some fact-collection.
\begin{lemma}
	Fix a scheme morphism $f\colon X\to Y$ of finite type where $Y$ is Noetherian. Then $X\times_YX$ is Noetherian, and $\Omega^1_{X/Y}$ is coherent.
\end{lemma}
\begin{proof}
	Omitted.
\end{proof}
\begin{lemma}
	Given a commuting square
	% https://q.uiver.app/?q=WzAsNCxbMCwwLCJYJyJdLFsxLDAsIlgiXSxbMSwxLCJZIl0sWzAsMSwiWSciXSxbMCwxLCJnIl0sWzAsM10sWzMsMl0sWzEsMl1d&macro_url=https%3A%2F%2Fraw.githubusercontent.com%2FdFoiler%2Fnotes%2Fmaster%2Fnir.tex
	\[\begin{tikzcd}
		{X'} & X \\
		{Y'} & Y
		\arrow["g", from=1-1, to=1-2]
		\arrow[from=1-1, to=2-1]
		\arrow[from=2-1, to=2-2]
		\arrow[from=1-2, to=2-2]
	\end{tikzcd}\]
	then there is a natural map $g^*\Omega^1_{X/Y}\to\Omega_{X'/Y'}$. If the square is Cartesian, then this map is an isomorphism.
\end{lemma}
\begin{proof}
	Some thickening argument grants us a map $P^1_{X/Y}\to P^1_{X'/Y'}$ making the diagram
	% https://q.uiver.app/?q=WzAsNixbMCwwLCJYJyJdLFsxLDAsIlgiXSxbMCwyLCJZJyJdLFsxLDIsIlkiXSxbMCwxLCJQXjFfe1gnL1knfSJdLFsxLDEsIlBeMV97WC9ZfSJdLFswLDEsImciXSxbMCw0XSxbNCwyXSxbMiwzXSxbMSw1XSxbNSwzXSxbNCw1LCIiLDEseyJzdHlsZSI6eyJib2R5Ijp7Im5hbWUiOiJkYXNoZWQifX19XV0=&macro_url=https%3A%2F%2Fraw.githubusercontent.com%2FdFoiler%2Fnotes%2Fmaster%2Fnir.tex
	\[\begin{tikzcd}
		{X'} & X \\
		{P^1_{X'/Y'}} & {P^1_{X/Y}} \\
		{Y'} & Y
		\arrow["g", from=1-1, to=1-2]
		\arrow[from=1-1, to=2-1]
		\arrow[from=2-1, to=3-1]
		\arrow[from=3-1, to=3-2]
		\arrow[from=1-2, to=2-2]
		\arrow[from=2-2, to=3-2]
		\arrow[dashed, from=2-1, to=2-2]
	\end{tikzcd}\]
	commute. Comparing kernels everywhere gives the necessary map. One can check the natural map is an isomorphism here by looking affine-locally. Alternatively, we note the universal property is able to give us an inverse map $\Omega^1_{X'/Y'}\to g^*\Omega^1_{X/Y}$ after some effort.
\end{proof}
We now pick up two short exact sequences. Here is our first.
\begin{proposition} \label{prop:diff-ses-1}
	Fix scheme morphisms $f\colon X\to Y$ and $g\colon Y\to Z$. Then the sequence
	\[f^*\Omega^1_{Y/Z}\to\Omega^1_{X/Z}\to\Omega^1_{X/Y}\to0\]
	is exact. If $f$ is smooth, then this is also exact on the left.
\end{proposition}
\begin{proof}
	We only discuss the case where $f$ is not smooth. As usual, we are able to reduce to the affine case, which means we would like
	\[C\otimes\Omega^1_{B/A}\to\Omega^1_{C/A}\to\Omega^1_{C/B}\to0\]
	to be exact. Dualizing, it's enough for
	\[0\to\op{Hom}_C(\Omega^1_{C/B},M)\to\op{Hom}_C(\Omega^1_{C/A},M)\to\op{Hom}_B(\Omega^1_{B/A},M)\]
	being exact for all $C$-modules $M$. But this is the same as
	\[0\to\op{Der}_B(C,M)\to\op{Der}_A(C,M)\to\op{Der}_A(B,M)\]
	being exact for all $C$-modules $M$. But now this is clear: given an $A$-derivation $d\colon C\to M$, asserting that it goes to zero in $\op{Der}_A(B,M)$ is just saying that $db=0$ for each $b$, which is exactly a derivation in $\op{Der}_B(C,M)$.
\end{proof}
\begin{remark}
	Informally, here is what \autoref{prop:diff-ses-1} says: as usual, imagine $X,Y,Z$ as manifolds, but now imagine $Z$ as a point. Dualizing, we are asserting an exact sequence
	\[0\to\mc T_{X/Y}\to\mc T_{X/Z}\to f^*\mc T_{Y/Z}\to0.\]
	Now, the idea is that $Z$ marks some point in $X$ and $Y$, and we have some projection $f\colon X\to Y$. The idea is that $\mc T_{X/Z}\to f^*\mc T_{Y/Z}$ is some projection of a tangent vector, but what remains is a tangent vector in $X$ orthogonal to $Y$, which is an element of $\mc T_{X/Y}$.
\end{remark}
\begin{corollary}
	Fix a scheme morphism $f\colon X\to Y$. Then $\Omega^1_{X/Y}|_U\simeq\Omega^1_{U/Y}$ for any open subscheme $U\subseteq X$.
\end{corollary}
\begin{proof}
	Note that $\Omega^1_{U/X}=0$, which we can see affine-locally using the distinguished base. But then the above result gives us a surjective map $\Omega^1_{X/Y}|_U\to\Omega^1_{X/U}$. To get the isomorphism, we look affine-locally to show that any derivation $\op{Der}_A(B,M)$ can be extended to some localization $\op{Der}_A(B_f,M)$, but this is possible using the quotient rule.
\end{proof}
\begin{corollary}
	Fix a scheme morphism $f\colon X\to Y$. Given $x\in X$, set $y\coloneqq f(x)$, and we see $\left(\Omega^1_{X/Y}\right)_x\simeq\Omega^1_{\OO_{X,x}/\OO_{Y,y}}$.
\end{corollary}
\begin{proof}
	We apply the exact sequence to the sequence of maps
	\[\Spec\OO_{X,x}\to\Spec\OO_{Y,y}\to Y,\]
	and looking locally lets us use the previous corollary to finish.
\end{proof}
Here is the second short exact sequence.
\begin{definition}
	Given a closed embedding $f\colon X\to Y$, we define the \textit{conormal bundle} to be $\mc N_{X/Z}^\lor\coloneqq\mc I/\mc I^2$, where $\mc I$ is the associated quasicoherent ideal sheaf.
\end{definition}
\begin{proposition}
	Fix a closed embedding $i\colon X\to Z$ of $Y$-schemes. Then there is an exact sequence
	\[\mc N_{X/Z}^\lor\to i^*\Omega^1_{Z/Y}\to\Omega^1_{X/Y}\to0.\]
	If the map $X\to Y$ is smooth, then we are exact on the left.
\end{proposition}
\begin{proof}
	We omit most of the proof. However, we go ahead and describe the map $\delta\colon\mc N_{X/Z}^\lor\to i^*\Omega^1_{Z/Y}$. Namely, we can build this map as the composite
	\[\frac{\mc I}{\mc I^2}\to\frac{\Omega^1_{Z/Y}}{\mc I\Omega^1_{Z/Y}}=i^*\Omega^1_{Z/Y},\]
	which we can check is okay.
\end{proof}
\begin{corollary}
	Suppose that $A$ is a ring and $B$ is an $A$-algebra with $B=A[x_1,\ldots,x_n]/(f_1,\ldots,f_r)$. Then
	\[\Omega^1_{B/A}=\frac{\bigoplus_{i=1}^nBdx_i}{\bigoplus_{j=1}^rBdf_j}.\]
\end{corollary}
\begin{proof}
	Use the above short exact sequence.
\end{proof}

\end{document}