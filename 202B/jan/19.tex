% !TEX root = ../notes.tex

\documentclass[../notes.tex]{subfiles}

\begin{document}

Here we go.

\subsection{The Product Algebra}
We quickly pick up the following lemma.
\begin{lemma} \label{lem:partition-algebra}
	Fix finitely many subsets $A_1,\ldots,A_n\subseteq X$, and suppose that these subsets live in an algebra $\mc A$ on $X$. Then there exists a finite partition $\{C_\alpha\}_{\alpha\in\kappa}$ of $X$ of sets in the algebra such that
	\[A_i=\bigsqcup_{\substack{\alpha\in\kappa\\C_\alpha\subseteq A_i}}A_i.\]
\end{lemma}
\begin{proof}
	We basically build a Venn diagram. Choose index set $I$ to be $\{0,1\}^n$, and define $C_\alpha$ for $\alpha\in I$ to be the set of $x\in X$ such that $x\in A_i$ if and only if $\alpha_i=1$. Note that we can write $C_\alpha$ as
	\[C_\alpha\coloneqq\bigcup_{\alpha_i=1}A_i\mathbin{\Bigg\backslash}\bigcup_{\alpha_i=0}A_i.\]
	Now, these $C_\bullet$s of course provide a partition satisfying the needed condition by its construction.
\end{proof}
Anyway, let's return to showing that we have a product algebra. For example, it turns out that the union of two measure rectangles is again a measurable rectangle. Here's the image.
\begin{center}
	\begin{asy}
		unitsize(1cm);
		draw((0,0)--(0,2)--(4,2)--(4,0)--cycle);
		draw((3,-3)--(3,1)--(5,1)--(5,-3)--cycle);
		draw((0,1)--(3,1)--(3,2), dashed);
		draw((4,-3)--(4,0)--(5,0), dashed);
	\end{asy}
\end{center}
And here is our statement.
\begin{lemma}
	Fix measure spaces $(X,\mathcal M,\mu)$ and $(Y,\mathcal N,\nu)$. Then $\mathcal A(X,Y)$ is actually an algebra.
\end{lemma}
\begin{proof}
	Here are our checks.
	\begin{itemize}
		\item Note $\emp\times\emp=\emp$, so $\emp\in\mc A(X,Y)$.
		\item Finite union of rectangles: suppose that we have measurable rectangles $\{A_i\times B_i\}_{i=1}^n$. Then we show that the union is in $\mc A(X,Y)$. Now, the $A_\bullet$s produce some partition $\{C_\alpha\}_{\alpha\in I}\subseteq\mc M$ of $X$ via \Cref{lem:partition-algebra}, and the $B_\bullet$s produce some partition $\{D_\beta\}_{\beta\in J}\subseteq\mc N$ of $Y$ via \Cref{lem:partition-algebra} again. Now
		\[A_i\times B_i=\bigsqcup_{\substack{C_\alpha\subseteq A_i\\D_\beta\subseteq B_i}}C_\alpha\times D_\beta,\]
		so
		\[\bigcup_{i=1}^nA_i\times B_i=\bigcup_{i=1}^n\bigsqcup_{\substack{C_\alpha\subseteq A_i\\D_\beta\subseteq B_i}}C_\alpha\times D_\beta,\]
		so our union is a union of measurable rectangles of the form $C_\alpha\times D_\beta$. But these measurable rectangles are all pairwise disjoint because the $C_\bullet$s and $D_\bullet$s are all pairwise disjoint, so the above union is in $\mc A$.
		\item Finite union: given $E_1,\ldots,E_n\in\mc A$, we need to show the union is in $\mc A$. Well, write
		\[E_i=\bigsqcup_{j=1}^{n_i}A_{ij}\times B_{ij}\]
		for some $A_\bullet\in\mc M$ and $B_\bullet\in\mc N$. Then
		\[\bigcup_{i=1}^nE_i=\bigcup_{i=1}^n\bigcup_{j=1}^{n_i}A_{ij}\times B_{ij}\]
		is a union of measurable rectangles and hence lives in $\mc A$ by the above check.
		\item Complement: given $E\in\mc A$, write
		\[E=\bigcup_{i=1}^nA_i\times B_i\]
		for measurable rectangles $A_\bullet\times B_\bullet$. As before, the $A_\bullet$s produce some partition $\{C_\alpha\}_{\alpha\in I}\subseteq\mc M$ of $X$ via \Cref{lem:partition-algebra}, and the $B_\bullet$s produce some partition $\{D_\beta\}_{\beta\in J}\subseteq\mc N$ of $Y$ via \Cref{lem:partition-algebra} again. This allows us to write
		\[E=\bigsqcup_{i=1}^n\bigsqcup_{\substack{C_\alpha\subseteq A_i\\D_\beta\subseteq B_i}}C_\alpha\times D_\beta,\]
		and then the complement $(X\times Y)\setminus E$ will be the union of the measurable rectangles $C_\alpha\times D_\beta$ not in the above union. But these are still disjoint measurable rectangles, so the union remains in $\mc A$.
		\qedhere
	\end{itemize}
\end{proof}

\subsection{The Product Measure}
Let's now define our product premeasure. Given the measure spaces $(X,\mc M,\mu)$ and $(Y,\mc N,\nu)$, we would like to define
\[\rho\Bigg(\bigsqcup_{i=1}^nA_i\times B_i\Bigg)=\sum_{i=1}^n\mu(A_i)\nu(B_i),\]
but it is not obvious that this is well-defined. Instead of doing this, we will choose the following definition.
\begin{definition}[product premeasure] \label{def:product-premeas}
	Fix measure spaces $(X,\mathcal M,\mu)$ and $(Y,\mathcal N,\nu)$. Given $E\in\mc A(X,Y)$, we define the \textit{product premeasure} $\rho(E)$ as
	\[\rho(E)\coloneqq\int_X\nu(E_x)\,d\mu(x),\]
	where $E_x\coloneqq\{y\in Y:(x,y)\in E\}$.
\end{definition}
\begin{remark}
	One should perhaps check that $E_x$ is always in $\mc N$ and hence measurable. But for this we simply write $E=\bigsqcup_{i=1}^n(A_i\times B_i)$ for measurable rectangles $A_i\times B_i$ and note that
	\[E_x=\{y\in Y:(x,y)\in A_i\times B_i\text{ for some }i\}=\bigcup_{\substack{i=1\\x\in A_i}}^nB_i,\]
	which is a finite union of measurable sets and hence in $\mc N$. In fact, 
\end{remark}
\begin{remark}
	One should perhaps check that $x\mapsto\nu(E_x)$ is integrable. Continuing from the above, we can see that these $B_i$ must be disjoint if $x\in A_i$ for each of these $i$, so actually
	\[\nu(E_x)=\sum_{\substack{i=1\\x\in A_i}}^n\nu(B_i)=\sum_{i=1}^n1_{A_i}(x)\nu(B_i),\]
	which is a linear combination of indicators of $\mu$-measurable sets, so this is a $\mu$-integrable function. Notably, we see that the measure of a measurable rectangle $A\times B$ is in fact $\mu(A)\nu(B)$.
\end{remark}
\begin{remark} \label{rem:double-int-premeasure}
	It is notable that we can write
	\[\rho(E)=\int_X\nu(E_x)\,d\mu(x)=\int_X\int_Y1_{E}(x,y)\,d\nu(y)\,d\mu(x),\]
	where the equality follows because the measure $\nu(E_x)$ is simply integrating $Y$ over the indicator of $1_E(x,y)$.
\end{remark}
We now check that we have a premeasure.
\begin{proposition}
	Fix measure spaces $(X,\mathcal M,\mu)$ and $(Y,\mathcal N,\nu)$. Then the defined product premeasure $\rho$ on $\mc A(X,Y)$ is in fact a premeasure.
\end{proposition}
\begin{proof}
	Here are our checks.
	\begin{itemize}
		\item Note $\rho(\emp)=0$ because $\emp_x=\emp$ always.
		\item Finitely additive: fix disjoint $E_1,E_2\in\mc A(X,Y)$, and we want to compute $\rho(E_1\sqcup E_2)$. Well, we use \Cref{rem:double-int-premeasure} to note
		\begin{align*}
			\rho(E_1\sqcup E_2) &= \int_X\nu((E_1\sqcup E_2)_x)\,d\mu(x) \\
			&= \int_X\int_Y1_{E_1\sqcup E_2}(x,y)\,d\nu(y)\,d\mu(x) \\
			&=  \int_X\int_Y(1_{E_1}(x,y)+1_{E_2}(x,y))\,d\nu(y)\,d\mu(x) \\
		\end{align*}
		Now, by linearity of integration, this is
		\begin{align*}
			\rho(E_1\sqcup E_2) &= \int_X\int_Y1_{E_1}(x,y)\,d\nu(y)\,d\,\mu(x)+\int_X\int_Y1_{E_2}(x,y)\,d\nu(y)\,d\mu(x) \\
			&= \rho(E_1)+\rho(E_2),
		\end{align*}
		as desired.
		\item Countably additive: we use the Monotone convergence theorem. Fix some disjoint subsets $\{E_i\}_{i=1}^\infty\subseteq\mc A(X,Y)$ such that $E\coloneqq\bigcup_{i=1}^\infty E_i$ is in $\mc A(X,Y)$. Proceeding as in the previous check, we see that
		\begin{align*}
			\rho(E) &= \int_X\int_Y1_E(x,y)\,d\nu(y)\,d\mu(x) \\
			&= \int_X\int_Y1_E(x,y)\,d\nu(y)\,d\mu(x) \\
			&= \int_X\int_Y\Bigg(\sum_{i=1}^\infty 1_{E_i}(x,y)\Bigg)\,d\nu(y)\,d\mu(x).
		\end{align*}
		Now, the functions $1_{E_i}$ and $1_E$ are all integrable (for suitably fixed coordinates), so applying the Monotone convergence theorem \cite[Theorem~9.18]{elber-top} tells us that
		\[\rho(E)=\sum_{i=1}^\infty\int_X\int_Y1_{E_i}(x,y)\,d\nu(y)\,d\mu(x)=\sum_{i=1}^\infty\rho(E_i),\]
		as desired.
		\qedhere
	\end{itemize}
\end{proof}
We can now produce our product measure.
\begin{definition}[product measure]
	Fix measure spaces $(X,\mathcal M,\mu)$ and $(Y,\mathcal N,\nu)$. Define the \textit{product $\sigma$-algebra} $\mc M\otimes\mc N$ to be the $\sigma$-algebra generated by $\mc A(X,Y)\subseteq\mc P(X\times Y)$. Then the product premeasure $\rho$ on $\mc A(X,Y)$ extends by \Cref{thm:extension} to a measure $\mu\times\nu$ on $\mc M\otimes\mc N$.
\end{definition}
\begin{remark}
	By \Cref{thm:extension}, if $\mu$ and $\nu$ are both $\sigma$-finite, then one can see that $\rho$ is still $\sigma$-finite by some covering with measurable rectangles, so $\mu\times\nu$ becomes the unique measure on $\mc M\otimes\mc N$ extending $\rho$.
\end{remark}

\subsection{Tonelli's Theorem}
The construction of our product premeasure in \Cref{def:product-premeas} has a ``handedness'' in that we integrate with respect to $Y$ and then with respect to $X$. This is somewhat upsetting, so we work to remedy this.
\begin{theorem}[Tonelli] \label{thm:tonelli}
	Fix $\sigma$-finite measure spaces $(X,\mc M,\mu)$ and $(Y,\mc N,\nu)$. Fix a measurable function $f\colon X\times Y\to[0,\infty]$.
	\begin{listalph}
		\item The function $y\mapsto f(x,y)$ is $\mc N$-measurable.
		\item The function $x\mapsto\int_Yf(x,y)\,d\nu(y)$ is $\mc M$-measurable.
		\item We have
		\[\int_{X\times Y}f\,d(\mu\times\nu)=\int_X\int_Yf(x,y)\,d\nu(y)\,d\mu(x)=\int_Y\int_Xf(x,y)\,d\mu(x)\,d\nu(y).\]
	\end{listalph}
\end{theorem}
\begin{remark}
	Note that, once measurable, we can integrate a nonnegative function if we allow for infinite values. For example, see something like \cite[Proposition~9.22]{elber-top}
\end{remark}
\begin{proof}[Reductions of \Cref{thm:tonelli}]
	We begin with two reductions.
	\begin{itemize}
		\item We reduce to the case where $f$ is the indicator of a function $1_E$. Indeed, having the result for indicators shows the conclusions for any linear combination of these, so we get the result for simple measurable functions, and then we can get the general case by taking monotone limits via the Monotone convergence theorem \cite[Theorem~9.18]{elber-top}.
		
		(Namely, (a) is direct by taking limits, (b) follows by the Monotone convergence theorem to move out the limit out of the integral and then taking limits to get measurable, and (c) is achieved directly by the Monotone convergence theorem repeatedly.)

		\item We reduce to the case where $X$ and $Y$ are spaces of finite measure. Indeed, by the $\sigma$-finiteness of $X$ and $Y$, we can partition each into countable disjoint union of sets of finite measure, and then by taking rectangles, we see that $X\times Y$ is a countable union of disjoint sets of finite measure. So achieving the result on these disjoint sets of finite measure, we can check the conclusions by summing over all the disjoint spaces, again concluding via the Monotone convergence theorem \cite[Theorem~9.18]{elber-top}. Namely, one can do an identical argument to the parenthetical remark of the previous reduction.
	\end{itemize}
	Before doing anything, we note that the $\sigma$-algebra $\mc M\otimes\mc N$ is not obviously generated at finite steps from $\mc A(X,Y)$; in fact, there is no countable constructive procedure to do this. So we are not going to proceed by trying to build up to $\mc M\otimes\mc N$; instead we will have to do something difficult.
\end{proof}

\end{document}