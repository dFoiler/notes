% !TEX root = ../notes.tex

\documentclass[../notes.tex]{subfiles}

\begin{document}

Ok let's begin.
\begin{remark}
	To produce the Lebesgue measure on $\RR^d$, one can imagine completing $(\RR^d,\mc L^d,m^d)$ or completing $\left(\RR^d,\mc B^d,m^d\right)$. Of course, one may just focus on showing that the $\sigma$-algebras are the same because then everything is the disjoint union of a null set and a measurable set.

	For example, we note that $\mc B^1\subseteq\mc L^1$, so completing makes $\widehat{\mc B^d}\subseteq\widehat{\mc L^d}$ by construction of our completion. For the reverse inclusion, by construction of our completion, it suffices to show that $\mc L^d\subseteq\widehat{\mc B^d}$. Looking at these as $\sigma$-algebras, it suffices to show that $\widehat{\mc B^d}$ contains measurable rectangles of $\mc L^1$-sets. Well, each set in $\mc L^1$ can be written as the union of a Borel set and a null set, so we can write the needed measurable rectangle as
	\[(B_1\cup N_1)\times(B_2\cup N_2)\times\cdots\times(B_d\cup N_d).\]
	Expanding out the product, the ``leading term'' $B_1\times\cdots\times B_d$ is Borel, and then the remaining terms have null sets in them, so they are null sets. So the entire thing lives in $\mc B^d$.
\end{remark}

\subsection{Symmetries of Euclidean Space}
We will be interested in affine automorphisms of $\RR^d$.
\begin{definition}[affine]
	An \textit{affine map} $f\colon\RR^d\to\RR^d$ is one which can be written as
	\[f(v)\coloneqq Tv+a\]
	where $T\in\op{GL}_d(\RR)$ and $v_0\in\RR^d$.
\end{definition}
\begin{remark}
	Note that $f$ is linear in its coordinates, so it is continuous. The inverse map is $v\mapsto T^{-1}(v-a)$, which is also affine, so $f$ is in fact a homeomorphism and so sends open sets to open sets.
\end{remark}
Let's quickly check that affine maps preserve Lebesgue sets.
\begin{lemma} \label{lem:preserve-leb}
	Fix a homeomorphism $h\colon X\to X$, where $(X,\mc L,\widehat\mu)$ is a complete Borel measure space.
	\begin{listalph}
		\item $h$ sends Borel sets to Borel sets.
		\item Suppose $h$ sends Borel null sets to null sets. Then $h$ preserves Lebesgue sets.
	\end{listalph}
\end{lemma}
\begin{proof}
	Here we go.
	\begin{listalph}
		\item Let $\mc D$ denote the collection of $E\subseteq X$ such that $h(E)$ is Borel. Well, $h$ is an open map, so $\mc D$ contains open sets. Further, $\mc D$ is a $\sigma$-algebra because taking images preserves unions and complements because $h$ is a bijection. Thus, $\mc D$ contains all Borel sets.
		\item Fix a Lebesgue set $B\cup N$ where $N$ is a null set. Then $h(B\cup N)=h(B)\cup h(N)$ is the union of a Borel set $h(B)$ (by (a)) and a null set $h(N)$ by hypothesis.
		\qedhere
	\end{listalph}
\end{proof}
\begin{remark}
	Note that an affine map $f\colon\RR^d\to\RR^d$ is a homeomorphism, so (a) above tells us that Borel sets get sent to Borel sets.
\end{remark}
We can actually measure our images pretty well.
\begin{proposition}
	Fix an affine map $f\colon\RR^d\to\RR^d$ of the form $f(v)\coloneqq Tv+a$. For a Lebesgue set $E$, we have
	\[\mu(f(E))=\left|\det T\right|\mu(E).\]
\end{proposition}
\begin{proof}
	By definition, we can decompose $f$ into a translation map $\tau\colon v\mapsto v+a$ and a linear map $T\colon v\mapsto Tv$. It then suffices to check the result on translations and linear maps.
	
	Well, for translations, we need $\mu(E+a)=\mu(E)$ for any Lebesgue set $E$. It suffices to do this for Borel sets $E$. Letting $\mc D$ denote the collection of Borel sets $E$ with $\mu(E+a)=\mu(E)$, we note that $\mc D$ contains all cubes (compute the measure as a product of the side lengths via, say, \Cref{thm:tonelli}), so $\mc D$ contains all open sets by \Cref{lem:divide-open-to-cubes}. Further, we can see that $\mc D$ is a $\sigma$-algebra because it is closed under unions and complements because translation is a bijection, and $\mu$ preserves unions and complements (approximately speaking). (The complement argument needs to know that $\mu$ is $\sigma$-finite.)

	For linear maps, we break down our maps even more. We can write any linear map $T$ as a composition of maps of the following kinds.
	\begin{itemize}
		\item Permutations of coordinates: for $\sigma\in\op{Sym}(\{1,\ldots,d\})$, we have the linear map $P_\sigma\colon(x_1,\ldots,x_d)\mapsto(x_{\sigma(1)},\ldots,x_{\sigma(d)})$.
		\item Dilation: for $t\in\RR^\times$, we have the linear map $D_t\colon(x_1,\ldots,x_d)\mapsto(tx_1,x_2,\ldots,x_d)$.
		\item Skew shifts: for $v\in\RR^{d-1}$, we have the linear map $S_v\colon(x_1,\ldots,x_n)\mapsto(x_1,\ldots,x_{d-1},x_d+(x_1,\ldots,x_{d-1})\cdot v)$.
	\end{itemize}
	Gaussian elimination shows any linear map $T$ is the composite of maps of the above form, so it suffices to take of matrices of the above form. We will do this next class.
\end{proof}

\end{document}