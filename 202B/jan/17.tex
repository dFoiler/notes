% !TEX root = ../notes.tex

\documentclass[../notes.tex]{subfiles}

\begin{document}

Let's just get started.

\subsection{Course Notes}
Here are some course notes.
\begin{itemize}
	\item The professor for this course is Michael Christ.
	\item There is a bCourses, which I don't have access to.
	\item There will be an exam in the evening in February.
	\item Problem sets will be due on Fridays.
	\item We will assume analysis on the level of Math 202A; see something like \cite{elber-top}.
	\item The text for the course is \cite{folland}.
\end{itemize}

\subsection{Measures}
Our first topic is to integrate on product spaces. Roughly speaking, we might have some measure spaces $(X,\mathcal M,\mu)$ and $(Y,\mathcal N,\nu)$ with some way to measure on them, and then we will want to measure $X\times Y$. Let's quickly recall what a measure is; we won't bother to recall the definition of a $\sigma$-algebra, but we will refer to \cite[Definition~5.25]{elber-top}. This requires the definition of a $\sigma$-algebra.
\begin{defihelper}[$\sigma$-algebra] \nirindex{sigma-algebra@$\sigma$-algebra} \label{def:sigma-alg}
	Fix a set $X$. Then a collection $\mathcal M\subseteq\mathcal P(X)$ is a \textit{$\sigma$-algebra} if and only if the following conditions are satisfied.
	\begin{itemize}
		\item $\emp\in\mathcal M$.
		\item $\mathcal M$ is closed under countable unions.
		\item $\mathcal M$ is closed under complements.
	\end{itemize}
\end{defihelper}
In the sequel, we will also want to produce $\sigma$-algebras.
\begin{definition}
	Fix a set $X$. Given a collection $\mathcal S\subseteq\mathcal P(X)$, we will say that the smallest $\sigma$-algebra generated by $\mathcal S$ is the $\sigma$-algebra \textit{generated by $\mathcal S$.}
\end{definition}
It is lemma that a smallest (i.e., contained in all other such $\sigma$-algebras) such $\sigma$-algebra exists and is unique. Let's see this.
\begin{lemma}
	Fix a set $X$ and collection $\mathcal S\subseteq\mathcal P(X)$. Then there is a $\sigma$-algebra $\mathcal M$ containing $\mathcal S$ such that $\mathcal M\subseteq\mathcal M'$ for any $\sigma$-algebra $\mathcal M'$ containing $\mathcal S$. This $\mathcal M$ is also unique.
\end{lemma}
\begin{proof}
	There is certainly some $\sigma$-algebra on $X$ containing $\mathcal S$, namely $\mathcal P(X)$. So there is a nonempty collection $\underline{\mathcal M}$ of all $\sigma$-algebras containing $\mathcal S$, and then we define
	\[\mathcal M\coloneqq\bigcap_{\mathcal M'\in\underline{\mathcal M}}\mathcal M'.\]
	Certainly $\mathcal M$ contains $\mathcal S$, and one can check directly that $\mathcal M$ is a $\sigma$-algebra. (See \cite[Lemma~5.28]{elber-top} for details.) And by construction, we see that $\mathcal M\subseteq\mathcal M'$ for any $\sigma$-algebra $\mathcal M'$ containing $\mathcal S$. Lastly, we note that $\mathcal M$ is unique because any two such $\sigma$-algebras $\mathcal M_1$ and $\mathcal M_2$ will be contained in each other and hence equal.
\end{proof}
Anyway, here is our definition of a measure.
\begin{definition}[measure] \label{def:meas}
	Fix a $\sigma$-algebra $\mathcal M$ on a set $X$. Then a \textit{measure} $\mu$ is a countably additive nonnegative function $\mu\colon\mathcal M\to\RR_{\ge0}\cup\{\infty\}$, and we require that $\mu(\emp)=0$. Here, being countably additive means that
	\[\mu\Bigg(\bigsqcup_{i=1}^\infty E_i\Bigg)=\sum_{i=0}^\infty\mu(E_i),\]
	where the sum is allowed to be in $\infty$ (namely, diverge to infinity). We call the triple $(X,\mathcal M,\mu)$ a \textit{measure space}.
\end{definition}
\begin{remark}
	If we have $\mu(\emp)>0$, then the countably additive condition implies that $\mu(\emp)=\infty$ and then $\mu(E)=\infty$ for all $E\in\mathcal M$. This is in fact countably additive, but we would like to exclude it.
\end{remark}
We will want to make our measures somewhat small.
\begin{defihelper}[$\sigma$-finite] \nirindex{sigma-finite@$\sigma$-finite} \label{def:sigma-finite}
	Fix a measure space $(X,\mathcal M,\mu)$. Then $\mu$ is \textit{$\sigma$-finite} if and only if $X$ is a countable union of sets in $\mathcal M$ of finite measure.
\end{defihelper}
This smallness condition is quite tame, and in practice all measures are $\sigma$-finite.

\subsection{The Extension Theorem}
We would like to discuss how to build measures from objects easier to construct. The following generalization of \Cref{def:sigma-alg} will be useful.
\begin{definition}[algebra]
	Fix a set $X$. Then a collection $\mathcal A\subseteq\mathcal P(X)$ is an \textit{algebra} if and only if the following conditions are satisfied.
	\begin{itemize}
		\item $\emp\in\mathcal A$.
		\item $\mathcal A$ is closed under finite unions.
		\item $\mathcal A$ is closed under complements.
	\end{itemize}
\end{definition}
\begin{example}
	Fix an uncountable set $X$, and let $\mathcal A$ denote the collection of finite and cofinite sets. Then $\mathcal A$ is an algebra (the finite union of finite sets is finite, and the finite union of cofinite sets is cofinite), but it need not be a $\sigma$-algebra because the countable union of finite sets need not be finite nor cofinite.
\end{example}
\begin{example}
	Fix $X\coloneqq\RR$, and let $\mathcal A$ denote the collection of finite unions of open or closed intervals. Then $\mathcal A$ is an algebra but not a $\sigma$-algebra.
\end{example}
Additionally, the following generalization of \Cref{def:meas} will be useful.
\begin{definition}[premeasure]
	Fix an algebra $\mathcal A$ on a set $X$. Then a \textit{premeasure} is a function $\rho\colon\mathcal A\to\RR_{\ge0}\cup\{\infty\}$ which is satisfies the following.
	\begin{itemize}
		\item $\rho(\emp)=0$.
		\item Finitely additive: we have $\rho(A\sqcup B)=\rho(A)+\rho(B)$ for $A,B\in\mathcal A$.
		\item Countably subadditive: suppose $\{A_i\}_{i\in\NN}\subseteq\mathcal A$ is pairwise disjoint, and $\bigsqcup_{i=1}^\infty A_i\in\mathcal A$. Then
		\[\rho\Bigg(\bigsqcup_{i=1}^\infty A_i\Bigg)=\sum_{i=1}^\infty\rho(A_i).\]
	\end{itemize}
\end{definition}
And now here is our theorem.
\begin{theorem}[Extension] \label{thm:extension}
	Fix a set $X$ and a premeasure $\rho$ on an algebra $\mathcal A$ over $X$. Then there exists a measure $\mu$ on the $\sigma$-algebra $\mathcal M$ generated by $\mathcal A$ such that $\mu|_{\mathcal A}=\rho$. Additionally, if $\rho$ is $\sigma$-finite, then $\mu$ is unique on $\mathcal M$.
\end{theorem}
Here, $\sigma$-finiteness for $\rho$ takes the same definition as \Cref{def:sigma-finite}.
\begin{proof}[Proof of \Cref{thm:extension}]
	For existence, combine \cite[Lemma~6.16 and Theorems~6.21, 6.24]{elber-top}. Further, uniqueness is \cite[Theorem~6.35]{elber-top}. It will be helpful to say a few words about the construction. Essentially, one builds an ``outer measure'' $\rho^*$ on $\mathcal P(X)$ by
	\[\rho^*(E)\coloneqq\inf\Bigg\{\sum_{n=0}^\infty\rho(A_n):\{A_n\}_{n\in\NN}\subseteq\mathcal A\text{ and }E\subseteq\bigcup_{n=0}^\infty A_n\Bigg\}.\]
	Then one restricts $\rho^*$ to a smaller $\sigma$-algebra over which it becomes a bona fide measure.
\end{proof}

\subsection{Towards Product Measures}
For our product measures, we take the following outline. Fix measure spaces $(X,\mathcal M,\mu)$ and $(Y,\mathcal N,\nu)$.
\begin{enumerate}
	\item We will construct a special $\sigma$-algebra $\mathcal M\otimes\mathcal N$ on $X\times Y$. Then we will construct a measure $\mu\times\nu$ on $\mathcal M\otimes\mathcal N$.
	\item Once the construction is in place, we will find a way to compare ``double integrals'' with ``single integrals.'' Morally, one wants equalities comparing
	\[\iint_{X\times Y}f\,d(\mu\times\nu)\qquad\text{and}\qquad\int_X\left(\int_Yf(x,y)\,d\nu(y)\right)\,d\mu(x).\]
	The moral of the story is that we will be able to compare our product measure with the measures on $X$ and $Y$ which we already understand.
	\item Lastly, we will specialize to Euclidean space $\RR^d$.
\end{enumerate}
Let's go ahead and begin.
\begin{definition}[measurable rectangle]
	Fix measure spaces $(X,\mathcal M,\mu)$ and $(Y,\mathcal N,\nu)$. A \textit{measurable rectangle} $E\subseteq X\times Y$ is a set of the form $A\times B$ where $A\in\mathcal M$ and $B\in\mathcal N$.
\end{definition}
\begin{example}
	The product of the circles $S^1\subseteq\RR^2$ and $S^1\subseteq\RR^2$ is the torus $S^1\times S^1$ in $\RR^4$ (identified with $\RR^2\times\RR^2$).
\end{example}
\begin{definition}[product algebra]
	Fix measure spaces $(X,\mathcal M,\mu)$ and $(Y,\mathcal N,\nu)$. Then we define the \textit{product algebra} $\mathcal A(X,Y)$ as the collection of all finite disjoint unions of measure rectangles.
\end{definition}
\begin{remark}
	The reason that we have taken finite disjoint unions of rectangles is because we know how to measure measurable rectangles, and we know how to sum their measures as disjoint unions.
\end{remark}
It's not totally clear that we have actually defined an algebra. We'll show this shortly. It turns out that the union of two measure rectangles is again a measurable rectangle. Here's the image.
\begin{center}
	\begin{asy}
		unitsize(1cm);
		draw((0,0)--(0,2)--(4,2)--(4,0)--cycle);
		draw((3,-3)--(3,1)--(5,1)--(5,-3)--cycle);
		draw((0,1)--(3,1)--(3,2), dashed);
		draw((4,-3)--(4,0)--(5,0), dashed);
	\end{asy}
\end{center}
And here is our statement.
\begin{lemma}
	Fix measure spaces $(X,\mathcal M,\mu)$ and $(Y,\mathcal N,\nu)$. Then $\mathcal A(X,Y)$ is actually an algebra.
\end{lemma}
We will prove this next class.

\end{document}