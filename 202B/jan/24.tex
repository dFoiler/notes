% !TEX root = ../notes.tex

\documentclass[../notes.tex]{subfiles}

\begin{document}

Let's begin.

\subsection{Addenda to Tonelli's Theorem}
Last class we completed the proof of \Cref{thm:tonelli}. We take a moment to note that there is a ``mirror'' of Tonelli's theorem as follows.
\begin{theorem}[Tonelli] \label{thm:tonelli-2}
	Fix $\sigma$-finite measure spaces $(X,\mc M,\mu)$ and $(Y,\mc N,\nu)$. Fix a measurable function $f\colon X\times Y\to[0,\infty]$. Then the following hold.
	\begin{listalph}
		\item The function $x\mapsto f(x,y)$ is $\mc M$-measurable.
		\item The function $y\mapsto\int_Yf(x,y)\,d\nu(y)$ is $\mc N$-measurable.
		\item We have
		\[\int_{X\times Y}f\,d(\mu\times\nu)=\int_Y\int_Xf(x,y)\,d\mu(x)\,d\nu(y).\]
	\end{listalph}
\end{theorem}
We will not write this proof because one can simply interchange $X$ and $Y$ in the provided proof of \Cref{thm:tonelli}. Perhaps one will complain that the definition of the product premeasure \Cref{def:product-premeas} appears asymmetric, but in fact it does not. Indeed, \Cref{rem:meas-rect} explains that the measure of a measurable rectangle is symmetric, which then explains how to measure anything in $\mc A(X,Y)=\mc A(Y,X)$ symmetrically, and then the Extension \Cref{thm:extension} tells us that this uniquely measures anything in $\mc M\otimes\mc N$ symmetrically.
\begin{corollary}
	Fix $\sigma$-finite measure spaces $(X,\mc M,\mu)$ and $(Y,\mc N,\nu)$. If $f\colon X\times Y\to[0,\infty]$ is $(\mc M\otimes\mc N)$-measurable, then
	\[\int_X\int_Yf(x,y)\,d\nu(y)\,d\mu(x)=\int_Y\int_Xf(x,y)\,d\mu(x)\,d\nu(y).\]
\end{corollary}
\begin{proof}
	Combine \Cref{thm:tonelli,thm:tonelli-2}.
\end{proof}
Perhaps one might worry about spaces which are not $\sigma$-finite. Here are some examples.
\begin{example}
	Fix an uncountable set $X$. Then one can define a measure $\mu$ on $\mc M\coloneqq\mc P(X)$ by $\mu(E)\coloneqq\#E$. This is not $\sigma$-finite because subsets of $X$ has finite measure if and only if it is finite, and $X$ cannot be covered by countably many finite sets.
\end{example}
\begin{example}
	Fix an uncountable set $X$, and let $\mc M$ be the collection of countable and cocountable subsets. Then the function $\mu\colon\mc M\to[0,\infty]$ defined by
	\[\mu(E)\coloneqq\begin{cases}
		0 & \text{if }E\text{ is countable}, \\
		\infty & \text{if }E\text{ is cocountable}
	\end{cases}\]
	is a measure. Now, $X$ fails to be $\sigma$-finite because the sets of finite measure are exactly the countable ones, and $X$ cannot be covered by countably many countable subsets.
\end{example}

\subsection{Fubini's Theorem}
We are now ready to state Fubini's theorem. This requires the following definition.
\begin{definition}
	Fix a measure space $(X,\mc M,\mu)$. Then we define $L^1(\mu)$ consists of the measurable functions $f\colon X\to\CC$ (defined almost everywhere) such that
	\[\int_X\left|f\right|\,d\mu<\infty.\]
\end{definition}
\begin{remark}
	If $f\in L^1(\mu)$, then one sees that $\int_Xf\,d\mu$ makes sense. Namely, one has
	\[\int_Xf\,d\mu=\int_X\Re f\,d\mu+i\int_X\Im f\,d\mu,\]
	and the integrals $\int_X\Re f\,d\mu$ and $\int_X\Im f\,d\mu$ are both bounded by $f\in L^1(\mu)$. Something like \cite[Proposition~9.22]{elber-top} assures us that this makes sense (upon taking differences).
\end{remark}
\begin{theorem}[Fubini] \label{thm:fubini}
	Fix $\sigma$-finite measure spaces $(X,\mc M,\mu)$ and $(Y,\mc N,\nu)$. Fix a measurable function $f\colon X\times Y\to\CC$ such that $\int_{X\times Y}\left|f\right|\,d(\mu\times\nu)<\infty$. Then the following hold.
	\begin{listalph}
		\item For $\mu$-almost every $x\in X$, the function $f_x\colon y\mapsto f(x,y)$ is defined and $\mc N$-measurable and in $L^1(\nu)$.
		\item The function $x\mapsto\int_Yf(x,y)\,d\nu(y)$ is defined almost everywhere and $\mc M$-measurable and in $L^1(\mu)$.
		\item We have
		\[\int_{X\times Y}f\,d(\mu\times\nu)=\int_X\int_Yf(x,y)\,d\nu(y)\,d\mu(x).\]
	\end{listalph}
\end{theorem}
\begin{proof}
	Note that we have the result for nonnegative functions by \Cref{thm:tonelli}. The idea is to reduce to this case. Here we go.
	\begin{itemize}
		\item By writing $f=u+iv$ for $u\coloneqq\Re f$ and $v\coloneqq\Im f$, we may assume that $f$ is real-valued. Explicitly, note the set of functions $f$ satisfying the conclusions of (a)--(c) is a $\CC$-vector space by some addition and scalar multiplication. Notably, we still have the hypotheses that $\int_{X\times Y}\left|u\right|\,d(\mu\times\nu)<\infty$ and $\int_{X\times Y}\left|v\right|\,d(\mu\times\nu)<\infty$.

		\item By writing $f=f^+-f^-$ for $f^+,f^-\ge0$, we will reduce to the case that $f$ is nonnegative. Namely, achieving the result for the two functions
		\[f^+(x)\coloneqq\begin{cases}
			f(x) & \text{if }f(x)\ge0, \\
			0 & \text{if }f(x)\le0,
		\end{cases}\qquad\text{and}\qquad f^-(x)\coloneqq\begin{cases}
			-f(x) & \text{if }f(x)\le0, \\
			0 & \text{if }f(x)\ge0,
		\end{cases}\]
		will achieve the result for $f$ by summing.
	\end{itemize}
	Note that there is a technicality hidden in the above reasoning with linear combinations: for example, for the second reduction, even though we have the conclusion for $f^+_x$ and $f^-_x$ are $\mc N$-measurable for all $x$, their difference might not be in $L^1(\nu)$ always. Well, we note that we can compute
	\[\int_X\left(\int_Yf^+(x,y)\,d\nu(y)\right)\,d\mu(x)=\int_{X\times Y}f^+\,d(\mu\times\nu)<\infty,\]
	so the inner function $x\mapsto\int_Yf^+_x\,d\nu(y)$ must be finite almost everywhere, or else this integral would be infinite! So we do indeed achieve that $f^+_x$ and $f^-_x$ are in $L^1(\nu)$ almost everywhere, so their difference is in $L^1(\nu)$ almost everywhere. The argument for taking linear combinations in (b) is similar.
\end{proof}
Let's see an example of why we want the hypothesis in \Cref{thm:fubini}.
\begin{example}
	Set $X\coloneqq\NN$, and let $\mu$ and $\nu$ denote the counting measures on $\mc M=\mc N\coloneqq\mc P(X)$. Note that $\mc A(X,X)=\mc P(X^2)$, so the product measure $\mu\times\nu$ is defined on all subsets; furthermore, we can see that the measure of a singleton is $1$, so $\mu\times\nu$ is the counting measure. Then we define the function
	\[f(x,y)\coloneqq\begin{cases}
		+1 & \text{if }x=y, \\
		-1 & \text{if }y=x+1, \\
		0 & \text{otherwise}.
	\end{cases}\]
	For each $x$, we compute $\int_Yf(x,y)\,d\nu(y)=0$ because each value of $x$ has two values of $y$ where $f(x,y)$ is nonzero. On the other hand, for each $y$, we compute $\int_Xf(x,y)\,d\mu(x)=0$ if $y>0$ but is $0$ if $y=0$. The problem here is that $\int_{X\times Y}f(x,y)\,d(\mu\times\nu)=0$.
\end{example}
\begin{remark}
	According to Professor Christ, the above example is a ``catastrophic failure'' of a theorem rather than a ``technical'' one.
\end{remark}
\begin{remark}
	By induction, we are able to take products of any finite product of $\sigma$-finite measure spaces. Alternatively, one can redo the entire theory to do measurable rectangular prisms and so on. There are some extra checks here (e.g., does forming products associate meaningfully?), but it will work out in the end, essentially by the uniqueness of the construction provided by \Cref{thm:extension}. Namely, up to the identification of products, we get the identification of the product $\sigma$-algebras and product measures because they should all agree on measurable rectangles, from which everything is generated.
\end{remark}

\end{document}