% !TEX root = ../notes.tex

\documentclass[../notes.tex]{subfiles}

\begin{document}

We began class by completing the proof of \Cref{thm:lp-banach}.
\begin{remark}
	It is possible to have a sequence of functions $\{f_n\}_{n\in\NN}$ in $L^p(X)$ which go to $0$ pointwise but do not go to $0$ in $L^p(X)$. For example, set $f_n\coloneqq1_{[n,n+1]}$, so $\norm{f_n}_p=1$, so $f_n\in L^p(X)$. But of course $\norm{f_n-0}_p=1$ does not go to $0$ as $n\to\infty$.
\end{remark}
\begin{example}
	We show that the triangle inequality fails in $L^p(X)$ for $0<p<1$. Indeed, simply set $f\coloneqq1_{[0,1]}$ and $g\coloneqq1_{[3,4]}$, so $\norm{f+g}_p=2^{1/p}$ while $\norm f_p=\norm g_p=1$.
\end{example}
\begin{remark}
	For $0<p<1$, one does have
	\[\int_X\left|f+g\right|^p\,d\mu\le\int_X\left|f\right|^p\,d\mu+\int_X\left|g\right|^p\,d\mu,\]
	so $L^p(X)$ does become a metric on a vector space via $d(f,g)\coloneqq\int_X\left|f-g\right|^p\,d\mu$, though it will not be a normed vector space.
\end{remark}
\begin{remark}
	The notion of convergence is a bit tricky. Given a sequence of functions $\{f_n\}_{n\in\NN}$ in $L^p(X)$ on a measure space $(X,\mc M,\mu)$, we have the following notions of convergence to a function $f$.
	\begin{itemize}
		\item Perhaps $f_n\to f$ pointwise for all $x\in X$ or pointwise almost everywhere.
		\item Perhaps $f_n\to f$ in norm for $\norm\cdot_p$.
		\item Perhaps $f_n\to f$ uniformly or uniformly almost everywhere.
		\item Perhaps $f_n\to f$ almost uniformly, meaning that any $\varepsilon>0$ has some $E\subseteq X$ such that $\mu(X\setminus E)<\varepsilon$ and $f_n\to f$ uniformly on $E$.
		\item Perhaps $f_n\to f$ in measure, meaning that any $\varepsilon>0$ has $\mu(\{x\in X:\left|f_n(x)-f(x)\right|>\varepsilon\})\to0$ as $n\to\infty$. See \cite[Example~8.11]{elber-top} for a nontrivial example.
	\end{itemize}
	Recall Egorov's theorem \cite[Theorem~9.13]{elber-top} tells us that $f_n\to f$ in measure in a finite measure space implies that any $\varepsilon>0$ has some $E\subseteq X$ such that $\mu(E\setminus X)<\varepsilon$ and $f_n\to f$ almost uniformly on $E\setminus X$. Note then that almost uniform convergence provides a limiting function for a subsequence.
\end{remark}

\subsection{The Dual Space of \texorpdfstring{$L^p(X)$}{Lp}}
Here is our statement.
\begin{theorem} \label{thm:dual-lp}
	Fix $p\in[1,\infty)$, and let $q\coloneqq\frac p{p-1}$ be the conjugate exponent. Further, fix a $\sigma$-finite measure space $(X,\mc M,\mu)$.
	\begin{listalph}
		\item If $p>1$, then $L^p(X)^*$ is naturally isomorphic to $L^q(X)$ as Banach spaces.
		\item If $p=1$, then $L^1(X)^*$ is naturally isomorphic to the space of equivalence classes of essentially bounded functions $f\colon X\to\CC$.
	\end{listalph}
\end{theorem}
Wait, what does essentially bounded mean?
\begin{definition}[essentially bounded]
	Fix a measure space $(X,\mc M,\mu)$. A measurable function $f\colon X\to\CC$ is \textit{essentially bounded} if and only if there is some $M>0$ such that
	\[\mu(\{x:\left|f(x)\right|>M\})=0.\]
	We denote the vector space of such functions by $\mc L^\infty(X)$ and their vector space of equivalence classes by $L^\infty(X)$.
\end{definition}
\begin{remark}
	We note that $\mc L^\infty(X)$ has a semi-norm given by
	\[\norm f_\infty\coloneqq\inf\{A\ge0:\mu(\{x\in X:\left|f(x)\right|>A\})=0\}.\]
	Quickly, note $\mu(\{x\in X:\left|f(x)\right|>\norm f_\infty\})=0$ by some countable union argument of $A$s approaching $\norm f_\infty$. Now, we omit most of the checks that this is actually a semi-norm. As an example check, we note that $\norm{f+g}_\infty\le\norm f_\infty+\norm g_\infty$ because
	\begin{align*}
		\mu(\{x\in X:\left|f(x)+g(x)\right|>\norm f_\infty+\norm g_\infty\})&\le\mu(\{x\in X:\left|f(x)\right|>\norm f_\infty\}) \\
		&\qquad+\mu(\{x\in X:\left|g(x)\right|>\norm fg\infty\}).
	\end{align*}
\end{remark}
\begin{remark}
	We also remark that $L^\infty(X)$ is complete, thus making $L^\infty(X)$ into a Banach space. It suffices to show that an infinite series $\sum_nf_n$ converges when $\sum_n\norm{f_n}_\infty<\infty$. By adjusting equivalence classes, we may assume that the $f_\bullet$s are bounded by $\norm{f_n}_\infty$. Then the infinite series is absolutely convergent and in fact uniformly so, so we get to define $f$ by $f(x)\coloneqq\sum_nf_n(x)$, and we get our uniform convergence for $\norm\cdot_\infty$ by writing out the convergence.
\end{remark}
We now prove \Cref{thm:dual-lp}.
\begin{proof}[Proof of \Cref{thm:dual-lp}]
	We begin with the proof of (a). Let's start by describing the map $T_\bullet\colon L^q(X)\to L^p(X)^*$. Well, for $g\in L^q(X)$, define $T_g\in L^p(X)^*$ by
	\[T_g(f)\coloneqq\int_Xfg\,d\mu.\]
	Note that $fg\in L^1(X)$ by \Cref{prop:holder-ineq}, so $T_g(f)$ is a well-defined quantity (and note that it does not depend on the choice of equivalence class for either $f$ or $g$). We also take a moment to note that $T_g$ is linear in $f$ because multiplication is linear, and in fact $T_\bullet\colon L^q(X)\to L^p(X)^*$ is again linear because multiplication is linear.
	
	To continue we claim that $T_\bullet$ preserves norms. For one inequality, note \Cref{prop:holder-ineq} tells us $\left|T_g(f)\right|\le\norm f_p\cdot\norm g_q$, so
	\[\norm{T_g}_{L^p(X)^*}\le\norm g_q.\]
	For the equality, we need to choose some functions $f$ making $T_g(f)$ large. Well, for each $n\in\NN$, set $E_n\coloneqq\{x:1/n\le\left|g(x)\right|\le n\}$, and define $f_n\coloneqq 1_{E_n}\cdot\ov g\left|g\right|^{q-2}$. To check that $f_n\in L^p(X)$, we compute
	\[\norm{f_n}_p=\left(\int_{E_n}\left|g\right|^q\,d\mu\right)^{(q-1)/q}<\infty.\]
	Further, we see
	\[T_g(f)=\int_{E_n}\left|g\right|^q\,d\mu,\]
	so
	\begin{equation}
		\limsup_{n\to\infty}\frac{T_g(f_n)}{\norm{f_n}_p}=\limsup_{n\to\infty}\left(\int_{E_n}\left|g\right|^q\right)^{1/q}=\norm g_q, \label{eq:stange-norm-computation}
	\end{equation}
	providing our lower bound on $\norm{T_g}_{L^p(X)^*}$. Note we did not actually need $E_n$ in the above argument, but we will reuse this calculation later when we don't know that $g\in L^q(X)$ a priori.

	Taking stock, currently we know that $T_\bullet\colon L^q(X)\to L^p(X)^*$ is an isometric embedding of Banach spaces. We also remark on the side that this map is natural in $X$, but we will not show it. It remains to check that $T_\bullet$ is surjective.

	We begin by using $\sigma$-finiteness to reduce the surjectivity check to the case where $X$ is a finite measure space. Write $X=\bigsqcup_{n=1}^\infty$ where the $X_n$ are finite measure spaces. Now, if we are granted the finite measure space situation, the function $T|_{L^p(X_n)}$ gives us some $g_n\in L^q(X_n)$ such that
	\[T(f)=\int_Xg_nf\,d\mu.\]
	Now, a direct argument as in \eqref{eq:stange-norm-computation} allows us to check that $\sum_{n=1}^\infty g_n$ is in $L^q(X)$, and then we can check that $T=T_g$ holds on simple functions, so it will hold everywhere because simple functions are dense in $L^q(X)$.

	Thus, for the rest of the proof, we may assume that $X$ is a finite measure space. We proceed in steps.
	\begin{enumerate}
		\item We would like to realize $T$ as an integral, so define a measure $\lambda$ on $X$ by
		\[\lambda(E)\coloneqq T(1_E).\]
		Note that this is well-defined for measurable sets $E$: $X$ is finite, so $E$ is finite, so $1_E$ is in $L^p(X)$, so $T(1_E)$ makes sense. We claim that $\lambda$ is a measure, for which we need to check that $\lambda$ is countably additive. Well, write $E=\bigsqcup_{n\in\NN}E_n$ as a disjoint union of countable measurable sets, and we want
		\[T(1_E)=\lambda(E)\stackrel?=\sum_{n=1}^\infty\lambda(E_n)=\sum_{n=1}^\infty T(1_{E_n})\stackrel*=T\Bigg(\sum_{n=1}^\infty1_{E_n}\Bigg).\]
		Here, $\stackrel*=$ will hold because continuity of $T$ allows it to move outside the summation. In particular, we would like to show that the partial sums $\sum_{n=1}^N1_{E_n}$ approaches $1_E$ as $N\to\infty$, but we need this convergence to hold in $L^p(X)$! In other words, we want to look at
		\[\norm{1_E-\sum_{n=1}^N1_{E_n}}_p^p=\int_X1_{E\setminus\bigsqcup_{n=1}^NE_n}\,d\mu=\mu\Bigg(\bigsqcup_{n>N}E_n\Bigg).\]
		Because we are in a finite measure space, we know that this right-hand side vanishes as $N\to\infty$.

		\item We now apply the Radon--Nikodym theorem to write $\lambda$ as an integral. For this, we want to check that $\lambda\ll\mu$. Well, this means that we want to check that $\mu(E)=0$ implies that $T(1_E)=\lambda(E)=0$. The point is that $\mu(1_E)=0$ implies that $\norm{1_E}_p=0$, but then $T(1_E)=0$ because $1_E=0$ in $L^p(X)$! Thus, the Radon--Nikodym theorem provides $g\in L^1(X)$ such that
		\[\lambda(E)=\int_Eg\,d\mu\]
		for any measurable $E$.

		It remains to check that $T=T_g$. We do this in steps.

		\item For simple functions $f=\sum_{i=1}^nc_i1_{E_i}$ where the $E_\bullet$ are disjoint measurable functions, so
		\[T(f)=\sum_{i=1}^nc_i\lambda(E_i)=\int_X\Bigg(\sum_{i=1}^nc_i1_{E_i}\Bigg)g\,d\mu=\int_Xfg\,d\mu.\]
		So we get $T(f)=T_g(f)$.

		\item For bounded functions $f$, we know that $f$ is the uniform limit of simple functions $\{f_n\}_{n\in\NN}$: indeed by taking linear combinations, we may assume that $f$ is real-valued, and then we partition the image of $f$ (which is a bounded set) into very small intervals of length $1/N$, taking the preimage of these intervals to build our $E_i\coloneqq f^{-1}([i/N,(i+1)/N))$. In particular, we may assume that $\left|f(x)-f_n(x)\right|<\frac1n$ for all $x$. But then we see that $f_n\to f$ even in $L^p(X)$ (note $p\ge1$), so $T(f_n)\to T(f)$, and we even have $T_g(f_n)\to T_g(f)$ by the $1/n$ bound and finiteness of our measure space. So we complete by passing to the limit.

		\item We show that $g\in L^q(X)$ for $p>1$. For each $n$, define
		\[G_n(x)\coloneqq\begin{cases}
			\ov{g(x)}\left|g(x)\right|^{q-2} & \text{if }\left|g(x)\right|\le n, \\
			0 & \text{otherwise}.
		\end{cases}\]
		Now, the function $G_n(x)$ is always bounded, so we get $T(G_n)=T_g(G_n)$ by the previous step. The point is that an argument identical to \eqref{eq:stange-norm-computation} shows that
		\[\norm g_q=\limsup_{n\to\infty}\frac{\left|\int_XG_n\,d\mu\right|}{\norm{G_n}_p}=\limsup_{n\to\infty}\frac{\left|T(G_n)\right|}{\norm{G_n}_p}\le\norm T_{L^p(X)^*}\]
		is finite.

		\item We show that $g\in L^\infty(X)$ for $p=1$. If $f$ is bounded, we note that we will have
		\[\left|\int_Xfg\,d\mu\right|\le\norm T_{L^1(X)^*}\norm f_1\]
		by bounding $g$ by its maximum. Now, for each measurable set $E$, we set
		\[f(x)\coloneqq\begin{cases}
			1_E(x)\cdot\ov g(x)/g(x) & \text{if }g(x)\ne0, \\
			0 & \text{if }g(x)=0,
		\end{cases}\]
		whereupon we see that
		\[\int_E\left|g\right|\,d\mu=\left|\int_E\left|g\right|\,d\mu\right|=\left|\int_Xfg\,d\mu\right|=\left|T(f)\right|\le\norm T_{L^1(X)^*}\norm f_1\le T_{L^1(X)^*}\mu(E).\]
		Now, this implies that $\left|g\right|\le T_{L^1(X)^*}$ almost everywhere: if we violated this for strict equality on a set of positive measure $E$, we could plug in $E$ above for contradiction. So $g\in L^\infty(X)$ follows.

		\item Now, we know that $g\in L^q(X)$, so $T_g$ fully makes sense. We have $T=T_g$ on simple functions by a previous step, and simple functions are dense in $L^p(X)$, so we get the equality everywhere by continuity.
		\qedhere
	\end{enumerate}
\end{proof}
\begin{remark}
	The result of (a) is true even without $\sigma$-finite hypotheses, though (b) does require this.
\end{remark}
\begin{remark}
	For $p\in[1,\infty)$, one finds that $L^p\left(\RR^d\right)$ is a separable space. Indeed, $L^p\left(\RR^d\right)$ contains as a dense subset the $\RR$-linear combinations of indicators of measurable sets. By approximating measurable sets close enough, we may assume that we are looking at $\RR$-linear combinations of boxes. But then we may assume that the coefficients and endpoints of the boxes are all rational by another approximation argument, so we are done.
\end{remark}

\end{document}