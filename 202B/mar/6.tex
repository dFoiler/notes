% !TEX root = ../notes.tex

\documentclass[../notes.tex]{subfiles}

\begin{document}

Here we go.

\subsection{Reflexive Banach Spaces}
We begin with a quick digression, upgrading \Cref{rem:ev-x-is-bounded}.
\begin{remark} \label{rem:ev-preserve-norm}
	We note that there is a canonical linear map $\op{ev}_\bullet\colon X\to X^{**}$ which preserves norms. The map is defined by defining $\op{ev}_x\in X^{**}$ as $\op{ev}_x(f)\coloneqq f(x)$ for $f\in X^*$. Note \Cref{rem:ev-x-is-bounded} shows that ${\op{ev}_x}\in X^{**}$, and we can see that $\op{ev}_\bullet$ is linear. It remains to compute the norm of $\op{ev}_x$. For one inequality, we use the argument of \Cref{rem:ev-x-is-bounded} to note
	\[\norm{\op{ev}_x}_{X^{*}}=\sup_{\substack{f\in X^*\\f\ne0}}\frac{\left|f(x)\right|}{\norm f_{X^*}}.\]
	To show the equality, we need to exhibit a good $f$. Well, by \Cref{thm:hb}, one gets $f\in X^*$ so that $\norm f_{X^*}=1$ and $\left|f(x)\right|=\norm x_X$; explicitly, $f$ is defined by extending $f\colon\FF x\to\FF$ defined by $f(x)\coloneqq\norm x_X$. So the above computation tells us that $\norm{\op{ev}_x}$ is at least $\left|f(x)\right|/\norm f_{X^*}=\norm x_X$, as required.
\end{remark}
\begin{remark}
	If $X$ is finite-dimensional, then $\dim X^{**}=\dim X^*=\dim X$, so the injective linear map $\op{ev}_\bullet\colon X\to X^{**}$ must actually be an isomorphism. For infinite-dimensional $X$, then $\op{ev}_\bullet$ need not be an isomorphism.
\end{remark}
In light of the above remark, we pick up the following definition.
\begin{definition}[reflexive]
	A Banach space $X$ is \textit{reflexive} if and only if the evaluation map $\op{ev}_\bullet\colon X\to X^{**}$ is an isomorphism.
\end{definition}
By \Cref{rem:ev-preserve-norm}, being reflexive merely requires surjectivity.
\begin{remark}
	On the homework, we will show that $\im\op{ev}_\bullet$ is dense in $X^{**}$ in the weak-$*$ topology.
\end{remark}
\begin{remark} \label{rem:reflexive-by-ball}
	One can show that the closed unit ball of a Banach space $X$ is compact in the weak topology if and only if $X$ is reflexive.
\end{remark}
\begin{nex}
	\Cref{subsec:dual-c0} tells us that $C_0$ is dual to $L^1(\NN)$. However, one can show that $L^1(\NN)$ is dual to $L^\infty(\NN)$, which is the space of bounded sequences, but the evaluation map $C_0\to L^\infty(\NN)$ fails to be surjective (meaning $C_0$ is not reflexive!). For example, let $\mc L\subseteq L^1(\NN)$ of sequences with a limit; notably, the constant sequence of $(1,1,\ldots)$ is in $\mc L\setminus C_0$, and one can use the Hahn--Banach theorem to bring this up to a linear functional on $L^1(\NN)$ not coming from $C_0$.
\end{nex}
\begin{nex}
	One can also see that $L^1(\NN)$ fails to be reflexive. One can again use the Hahn--Banach theorem or appeal to the following remark.
\end{nex}
\begin{remark}
	By functoriality, if $X$ is reflexive, then $X^*$ is also reflexive. We claim that the converse also holds! Suppose for the sake of contradiction that $X^*$ is reflexive but $X$ is not. Then ${\im\op{ev}^X_\bullet}\subsetneq X^{**}$, so \Cref{thm:hb} permits us to find some bounded linear functional $\lambda\in X^{***}$ such that $\lambda\ne0$ but $\lambda|_{\im\op{ev}^X_\bullet}=0$. However, we will show that $\lambda|_{\im\op{ev}^X_\bullet}=0$ implies $\lambda=0$. Indeed, $X^*$ is reflexive, so we are granted $\lambda_0\in X^*$ such that $\lambda=\op{ev}_{\lambda_0}^{X^*}$; it is enough now to show that $\lambda_0$ vanishes. Well, for each $x\in X$, we see that
	\[\lambda_0(x)=\op{ev}_{\lambda_0}^{X^*}({\op{ev}_x})=\lambda({\op{ev}_x})=0.\]
\end{remark}

\subsection{Weak Compactness Results}
Here is today's main result.
\begin{theorem}[Helly] \label{thm:helly}
	Fix a separable Banach space $(X,\norm\cdot)$. Then the closed unit ball of $X^*$ is sequentially compact in the weak-$*$ topology. In other words, given a sequence of bounded linear functionals $\{f_n\}_{n\ge1}\subseteq X^*$ such that $\norm{f_n}_{X^*}\le1$ for each $n$, there is a subsequence $\{f_{n_k}\}_{k\ge1}$ and $f\in X^*$ such that each $x\in X$ makes $f_n(x)\to f(x)$ as $n\to\infty$.
\end{theorem}
\begin{proof}
	This is a Cantor diagonal argument. Choose a countable dense subset $\{x_n\}_{n\in\NN}$. We now proceed with an inductive construction.
	\begin{enumerate}
		\item Because $\left|f_n(x_1)\right|\le\norm{x_1}_X$ for each $n$, we see that the sequence $\{f_n(x_1)\}_{n\in\NN}$ is bounded, so we can find a subsequence $\{f_{n(i_1)}\}_{i_1\in\NN}$ with a subsequence where the limit $\lim f_{n(i_1)}(x_1)$ exists.
		\item Next up, we again pass to a subsequence where $f_{n(i_1)(i_2)}$ so that $\lim f_{n(i_1)(i_2)}(x_2)$ exists.
	\end{enumerate}
	We have forced ourselves into the notation as above so that we can work with a subsequence $\{g_i\}_{i\ge1}$ where $g_i\coloneqq f_{n(i)\cdots(i)}$ repeated $i$ times. As such, $g_i(x_n)$ converges as $i\to\infty$ for any $n$.

	It remains to show that $\{g_i\}_{i\ge1}$ actually converges to a function $f$ in the closed unit ball. It suffices to show that this subsequence is Cauchy pointwise; then completeness of $X^*$ tells us that our sequence converges to an actual bounded linear functional, and continuity of $\norm\cdot_{X^*}$ means that we will land in the closed unit ball. Now, to show Cauchy, for each $x\in X$, choose $\varepsilon>0$ and some $x_j$ such that $\norm{x_j-x}<\varepsilon$, and we find
	\[\limsup_{m,n\to\infty}\left|g_m(x)-g_n(x)\right|\le\limsup_{m,n\to\infty}\left|g_m(x_j)-g_n(x_j)\right|+2\sup_{n\in\NN}\left|g_n(x)-g_n(x_j)\right|\]
	by some rearrangement. But $g_n$ is in the closed unit ball, so the right-hand term is bounded above by $2\varepsilon$. Additionally, the left term vanishes by construction of the $g_\bullet$, so are done.
\end{proof}
\begin{remark}
	\Cref{thm:helly} is useful because oftentimes one will construct a sequence of some functions (e.g., approximate solutions to a differential equation), and then we want to actually produce a limiting function. Then we see that it will merely remain to check some separability (which is the case when working in any Euclidean space) and a uniform boundedness (which is often not so bad).
\end{remark}
\Cref{thm:helly} has the following analogous result without separability.
\begin{restatable}[Alaoglu]{theorem}{aballthm} \label{thm:alaoglu-ball-weak-compact}
	Let $X$ be a Banach space. Then the closed unit ball of $X^*$ is compact in the weak-$*$ topology.
\end{restatable}
Note that we are working with the dual space here the result is not true for the closed unit ball in the weak topology on $X$; see \Cref{rem:reflexive-by-ball}.

It will be helpful to have the following extension result.
\begin{lemma} \label{lem:extend-from-ball}
	Fix a normed vector space $X$, and let $B\subseteq X$ be the closed unit ball. Suppose $f\colon B\to\FF$ is linear in the sense that $f(x+y)=f(x)+f(y)$ and $f(tz)=tf(z)$ whenever all inputs are in $B$. Then $f$ has a unique linear extension $F\colon X\to\FF$.
\end{lemma}
\begin{proof}
	Define $F(x)\coloneqq f(x/\norm x)$. In fact, one finds that $F(x)=tf\left(\frac1tx\right)$ whenever $\norm{\frac1tx}\le1$ by the given linearity; one can now check that $F$ is linear. One can see that this construction of $F$ is unique.
\end{proof}

\end{document}