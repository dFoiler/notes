% !TEX root = ../notes.tex

\documentclass[../notes.tex]{subfiles}

\begin{document}

Here we go.

\subsection{Functionals on Hilbert Spaces}
We begin with a remark on inner product spaces.
\begin{remark}
	Fix an inner product space $X$. Then the map $\varphi_y(x)\coloneqq\langle x,y\rangle$ satisfies the following properties.
	\begin{itemize}
		\item Linear in $x$: we see $\varphi_y(ax+a'x')=a\langle x,y\rangle+a'\langle x',y\rangle=a\varphi_y(x)+a'\varphi_y(x')$.
		\item Conjugate-linear in $y$: we see $\varphi_{ay+a'y'}(x)=\langle x,ay+a'y'\rangle=\ov a\langle x,y\rangle+\ov{a'}\langle x,y'\rangle=(\ov a\varphi_y+\ov{a'}\varphi_{y'})(x)$.
		\item Bounded: by \Cref{prop:cs-ineq}, we see
		\[\left|f_y(x)\right|\le\norm x\cdot\norm y,\]
		so $\norm{f_y}_{X^*}\le\norm y$. In fact, $f_y(y)=\norm y$, so actually $\norm{f_y}_{X^*}=\norm y$ exactly.
	\end{itemize}
	So we get a conjugate linear isometry $\varphi_\bullet\colon X\to X^*$.
\end{remark}
Being a Hilbert space lets us upgrade this map.
\begin{proposition} \label{prop:hilb-star}
	Fix a Hilbert space $X$. Then the map $y\mapsto\varphi_y$ (where $\varphi_y\colon X\to\RR$ is the map $\varphi_y(x)\coloneqq\langle x,y\rangle$) is a surjection from $X$ onto $X^*$.
\end{proposition}
\begin{proof}
	Let $f$ be a bounded linear functional on $X$. If $f=0$, then $f=\varphi_0$, so we may assume that $f$ is not identically zero. Because $f$ is continuous, we note that
	\[V\coloneqq\ker f=\{x\in X:f(x)=0\}\]
	is a closed subspace of $X$ not equal to $X$. As such, we can find an element of $X$ not in $V$, and then \Cref{prop:ortho-decompose} allows us to make this into a nonzero element $y\in V^\perp$. We are going to show that $f$ is approximately $\varphi_y$.

	Quickly, we claim that $\dim V^\perp=1$ and actually spanned by $y$. Well, suppose that $x\in V^\perp$ and set $\alpha\coloneqq f(x)/f(y)$ so that we would like to show $x=\alpha y$. Well, by rearranging, we know that
	\[f(\alpha y-x)=0,\]
	so $\alpha y-x\in V$, so write $\alpha y=x+v$ for some $v\in V$, but then the uniqueness of \Cref{prop:ortho-decompose} requires $(\alpha y,0)=(x,v)$, so $x=\alpha y$.

	We are now able to show that $f=\varphi_y$. For example, any $x\in X$ by \Cref{prop:ortho-decompose} can be written uniquely as $\alpha y+v$ for some $\alpha\in\FF$ and $v\in V$ so that
	\[\varphi_y(\alpha y+v)=\langle\alpha y+v,y\rangle=\alpha\norm y^2.\]
	On the other hand, $f(x)=\alpha f(y)$, so
	\[\frac{f_y(x)}{f(x)}=\frac{\norm y^2}{f(y)}\]
	for each (nonzero) $x$, so $f$ and $\varphi_y$ only differ by a fixed nonzero element of $\FF$, so we can multiply $y$ through by (the conjugate of) this element to complete.
\end{proof}
\begin{remark}
	One can use \Cref{prop:hilb-star} to show that Hilbert spaces $X$ are reflexive. In fact, this result tells us that $X^*$ should be a Hilbert space because its norm will satisfy the Parallelogram law. Now checking surjectivity of $\op{ev}_\bullet$ is some purely formal composition of surjections.
\end{remark}

\subsection{Orthonormal Sets}
Here is our definition.
\begin{definition}[orthonormal]
	Fix an inner product space $X$. Then a set $S\subseteq X$ is an \textit{orthonormal set} if and only if
	\[\langle v,w\rangle=1_{v=w}\]
	for all $v,w\in S$. We say that $S$ is merely \textit{orthogonal} if and only if distinct $v,w\in S$ have $\langle v,w\rangle=0$.
\end{definition}
\begin{example}
	Consider the Hilbert space $X=L^2(\NN)$. Then the set $\{e_n\}_{n\in\NN}$ where $(e_n)_i\coloneqq1_{i=n}$ is an orthonormal set.
\end{example}
One can define $L^2$ in more generality than we have done so.
\begin{example}
	Fix a measure space $(X,\mc M,\mu)$. Then the vector space $L^2(X,\mc M,\mu)$ of measurable functions $f\colon X\to\CC$ such that $\int_Xf^2\,d\mu<\infty$ is almost an inner product space when given the inner product
	\[\langle f,g\rangle=\int_Xf\ov g\,d\mu.\]
	The issue is that $\langle f,f\rangle=0$ only implies that $\int_X\left|f\right|^2\,d\mu=0$, meaning that $f$ merely vanishes almost everywhere. One fixes this by defining $\ell^2(X,\mc M,\mu)$ to be the vector space of equivalence classes of measurable functions.
\end{example}
We would like to work with infinite sums of our bases, but this requires a reasonable notion of convergence.
\begin{definition}
	Fix a set $S$ of nonnegative real numbers. Then we define the sum $\sum_{s\in S}S$ as
	\[\sup_{\text{finite }S'\subseteq S}\sum_{s'\in S'}s'\]
\end{definition}
\begin{remark}
	If $S$ is countable, then this is just usual convergence because absolute convergence permits us to sum in any order anyway.
\end{remark}
\begin{remark} \label{rem:converge-only-when-countable}
	If $\sum_{s\in S}s$ is actually finite, then $S$ can only have countably many nonzero real numbers. Indeed, suppose the summation is finite. Then for each positive integer $N$, we see that
	\[\sum_{s\in S}s\ge\sum_{\substack{s\in S\\s\ge1/N}}s\ge\frac1N\#\{s\in S:s\ge1/N\},\]
	so $\#\{s\in S:s\ge1/N\}$ is finite for all $N$. Sending $N\to\infty$ tells us that the nonzero elements of $S$ live in a countable union of finite sets, so the nonzero elements of $S$ form a countable set.
\end{remark}
So far we have discussed convergence for sums of real numbers. We now pass to vectors.
\begin{proposition}[Bessel's inequality] \label{prop:bessel-ineq}
	Fix an orthonormal subset $S$ of an inner product space $X$. Then
	\[\sum_{s\in S}\left|\langle x,s\rangle\right|^2\le\norm x^2\]
	for any $x\in X$.
\end{proposition}
\begin{proof}
	By the definition of our summation, it is enough to show that
	\[\sum_{s'\in S'}\left|\langle x,s\rangle\right|^2\le\norm x^2\]
	for any finite subset $S'\subseteq S$. Thus, we may assume that $S$ is finite.

	The trick now is to take the projection onto the span of $S$. Set
	\[y\coloneqq\sum_{s\in S}\langle x,s\rangle s,\]
	which makes sense as a sum because now $S$ is finite. We claim that $\langle x-y,y\rangle=0$. This will complete the proof because then $\norm x^2=\norm{x-y}^2+\norm y^2$, and $\norm y^2$ is the sum $\sum_{s\in S}\left|\langle x,s\rangle\right|^2$ because the $s\in S$ are mutually orthogonal.

	So it remains to show the claim. This is purely formal. Indeed,
	\begin{align*}
		\langle x-y,y\rangle &= \langle x,y\rangle-\langle y,y\rangle \\
		&= \sum_{s\in S}\langle x,\langle x,s\rangle s\rangle+\sum_{s\in S}\langle x,s\rangle\ov{\langle x,s\rangle}\langle s,s\rangle \\
		&= \sum_{s\in S}\ov{\langle x,s\rangle}\langle x, s\rangle+\sum_{s\in S}\langle x,s\rangle\ov{\langle x,s\rangle} \\
		&= 0,
	\end{align*}
	where we have repeatedly used the orthogonality of $S$.
\end{proof}
\begin{lemma} \label{lem:converge-for-vectors}
	Fix a countable orthonormal subset $S$ of a Hilbert space $X$. Given scalars $\{t_s\}_{s\in S}$ in $\FF$ such that $\sum_{s\in S}\left|t_s\right|^2<\infty$, then
	\[\sum_{s\in S}t_ss\]
	converges in $X$.
\end{lemma}
\begin{proof}
	One might complain that the last summation we wrote down might depend on the order of summation. We will show that it does not. If $S$ is finite, there is nothing to do, so we may assume that $S$ is enumerated as $\{u_n\}_{n\in\NN}$, and we set $t_k\coloneqq t_{u_k}$.

	Now, because $X$ is complete, it suffices to show that our partial sums form a Cauchy sequence, meaning that we would like to show that
	\[\lim_{m,n\to\infty}\norm{\sum_{k=m}^nt_ku_k}=0.\]
	But now this is a finite sum, so we may write
	\[\norm{\sum_{k=m}^nt_ku_k}=\sum_{k=m}^n\left|t_k\right|^2\le\sum_{k=m}^\infty\left|t_k\right|^2,\]
	and the right-hand side vanishes as $m\to\infty$ because $\sum_k\left|t_k\right|^2$ converges, so we are done.
	
	Lastly, we check that the order of the summation does not depend on our enumeration. Well, a different enumeration amounts to choosing a permutation $\sigma\colon\NN\to\NN$, and then we want to show that
	\[\lim_{n\to\infty}\norm{\sum_{k=1}^nt_ku_k-\sum_{k=1}^nt_{\sigma(k)}u_{\sigma(k)}}.\]
	Well, for any $\varepsilon>0$, choose $n$ large enough so that $\sum_{k>n}\left|t_k\right|^2<\varepsilon$ and $\sum_{k>n}\left|t_{\sigma(k)}\right|^2<\varepsilon$. Then taking $N$ large enough so that $\{\sigma(1),\ldots,\sigma(N)\}$ contains $\{1,\ldots,n\}$ tells us that the norm
	\[\norm{\sum_{k=1}^Nt_ku_k-\sum_{k=1}^Nt_{\sigma(k)}u_{\sigma(k)}}\le\norm{\sum_{k=n+1}^Nt_ku_k}+\norm{\sum_{k=n+1}^Nt_{\sigma(k)}u_{\sigma(k)}}<2\varepsilon.\]
	Sending $N\to\infty$ permits $\varepsilon\to0^+$ and hence completes the proof.
\end{proof}
\begin{remark} \label{rem:norm-of-series}
	This proof actually shows that
	\[\norm{\sum_{n=1}^\infty tnu_n}^2=\sum_{n=1}^\infty\left|t_n\right|^2\]
	under our hypotheses, by continuity of $\norm\cdot$.
\end{remark}
\begin{remark}
	One might expect the hypothesis on the scalars to be $\sum_s\left|t_s\right|\le1$ so that $\sum_s\norm{t_nu_n}<\infty$. Our lemma is stronger though. Indeed, note that $\sum_s\left|t_s\right|^2<\infty$ is a weaker condition than $\sum_s\left|t_s\right|<\infty$: if $\sum_s\left|t_s\right|<\infty$, then the sequence $\{t_s\}$ is bounded by some $T$, so we may write
	\[\sum_{s\in S}\left|t_s\right|^2\le T\sum_{s\in S}\left|t_s\right|<\infty.\]
	In fact, the sum of squares converging is strictly weaker: one has $\sum_n1/n^2<\infty$ but $\sum_n1/n=\infty$.
\end{remark}
\begin{remark}
	To further motivate our hypothesis on the scalars, note that if $\sum_st_ss$ has any hope of convergence, then
	\[\norm{\sum_{s\in S}t_ss}=\sum_{s\in S}\left|t_s\right|^2\]
	had better converge.
\end{remark}

\end{document}