% !TEX root = ../notes.tex

\documentclass[../notes.tex]{subfiles}

\begin{document}

We began class with a review of topological spaces. We just refer to \cite{elber-top}.

\subsection{Weak Topologies}
It will be helpful to have other topologies to work with on our normed vector spaces. We recall the following notion.
\begin{definition}[weak topology]
	Fix a nonempty set $X$, and let $\{f_\alpha\}_{\alpha\in\kappa}$ be a nonempty collection of maps $f_\alpha\colon X\to Y_\alpha$, where $\{Y_\alpha\}_{\alpha\in\kappa}$ is a collection of topological spaces. Then we define the \textit{weak topology} on $X$ to be the topology generated by the collection
	\[\bigcup_{\alpha\in\kappa}\left\{f^{-1}(U_\alpha):\text{open }U_\alpha\subseteq Y_\alpha\right\}.\]
\end{definition}
\begin{remark}
	Notably, the given collection above is a subbasis because it contains $\emp=f_\alpha^{-1}(\emp)$ and $X=f_\alpha^{-1}(Y_\alpha)$. As such, we can give ourselves a basis by taking finite intersections of the sets in our given collection.
\end{remark}
\begin{remark}
	Observe that the weak topology on $X$ promises that each map $f_\alpha\colon X\to Y_\alpha$ is continuous. Conversely, if $\mc T$ is a topology on $X$ such that the maps $f_\alpha\colon X\to Y_\alpha$ are continuous, then $\mc T$ must contain the weak topology.
\end{remark}
\begin{remark}
	We have the following universal property: a map $g\colon Z\to X$ is continuous if and only if the composites $(f_\alpha\circ g)\colon X\to Y_\alpha$ are all continuous. The forward direction occurs by taking composition. For the backward direction, it suffices to check that any $g^{-1}\left(f_\alpha^{-1}(U_\alpha)\right)\subseteq Z$ is open for any $\alpha\in\kappa$ and open $U_\alpha\subseteq Y_\alpha$, which holds by the continuity of $f_\alpha\circ g$.
\end{remark}
\begin{example}
	One gives the infinite product $X\coloneqq\prod_{\alpha\in\kappa}Y_\alpha$ of topological spaces $\{Y_\alpha\}_{\alpha\in\kappa}$ the weak topology with respect to the projections $\op{pr}_\alpha\colon X\to Y_\alpha$.
\end{example}
\begin{remark}
	Let's explain closures quickly. Let $E\subseteq X$ be a subset as above. Then $x\in\ov E$ if and only if $B\cap E\ne\emp$ for any basic open neighborhood $B\subseteq X$ of $x$. In other words, for any finite collection $S\subseteq\kappa$ of indices and open subsets $U_\alpha\subseteq Y_\alpha$ for $\alpha\in S$, we have
	\[B\cap\bigcap_{\alpha\in S}f_\alpha^{-1}U_\alpha\ne\emp.\]
\end{remark}
\begin{example}
	Consider $X\coloneqq\{0,1\}^\RR$ endowed with the product topology. Set $x\coloneqq0\in X$ and $E$ to be the collection of $y\in X$ such that $y_t=1$ or all but countably many $t\in\RR$. Then actually $x\in\ov E$: indeed, a basic open subset of $X$ containing $x$ merely requires that some finite subset of indices $t\in\RR$ vanish, which $E$ does intersect.

	However, there is no countable sequence $\{x_n\}_{n\in\ZZ^+}$ in $E$ which converge to $x$. Indeed, by taking a union over all $x_n$, we see that there is a cocountable subset $T\subseteq\RR$ such that $(x_n)_t=1$ for all $n\in\ZZ^+$ and $t\in T$. So our sequence cannot approach $x$ (for example) by using a basic open subset of $X$ requiring $y_t=0$ for some $t\in T$.
\end{example}
Here is our chief example of interest.
\begin{definition}
	Fix a normed vector space $(X,\norm\cdot)$. Then the \textit{weak topology} on $X$ is the topology obtained by requiring that all bounded linear functionals $f\in X^*$ are continuous.
\end{definition}
\begin{remark}
	The weak topology on $X$ is Hausdorff: for any distinct $x,y\in X$, it is enough to show that there is a linear functional $f\in X^*$ such that $f(x)=f(y)$. If $x$ and $y$ are linearly independent, then define $f$ by $f(x)=1$ and extend $f$ from $\op{span}\{x\}$ to all $X$ by \Cref{thm:hb}. If they are not linearly dependent, then define $f$ by $f(x)=1$ and $f(y)=0$ and extend $f$ again via \Cref{thm:hb}.
\end{remark}
Let's provide an example of some convergence.
\begin{example}
	Let $X$ be the collection of sequences $\{(x_n)_{n\in\ZZ^+}\}$ converging to $0$, and we give $X$ the norm $\norm\cdot_\infty$ defined by taking the maximum. Now, define the sequence $x_m\in X$ by $(x_m)_n\coloneqq1_{m=n}$, and it turns out that $x_m\to0$ as $m\to\infty$ in the weak topology, even though this is notably false in $X$! In other words, we have $f(x_m)\to0$ as $m\to\infty$ for any bounded linear functional $f\in X^*$.
\end{example}

\end{document}