% !TEX root = ../notes.tex

\documentclass[../notes.tex]{subfiles}

\begin{document}

We spent class by completing the proof of \Cref{thm:dual-lp}.

\subsection{Chebyshev's Inequality}
We close class by proving an inequality. Here is our result.
\begin{proposition}[Chebyshev's inequality] \label{prop:cheby-ineq}
	Fix a measure space $(X,\mc M,\mu)$ and some real number $p\ge1$. Then for any $f\in L^p(X)$ and $t>0$, we have
	\[\mu(\{x\in X:\left|f(x)\right|>t\})\le t^{-p}\norm f_p^p.\]
\end{proposition}
\begin{proof}
	We would like to show that
	\[\int_X\left|f\right|^p\,d\mu=\norm f_p^p\stackrel?\ge t^p\mu(\{x\in X:\left|f(x)\right|>t\}).\]
	Well, one simply notes that the integral on the left can be restricted to
	\[E_t\coloneqq X\setminus\{x\in X:\left|f(x)\right|\le t\},\]
	and then bound $\left|f\right|\ge t$ on $E_t$ (by definition), which is precisely the right-hand side.
\end{proof}
\begin{remark}
	One might complain that $\mu(\{x\in X:\left|f(x)\right|>t\})$ is a much simpler object that $\norm f_p^p$. We work with $\norm f_p$ instead of objects like this because $\norm\cdot_p$ has a triangle inequality.
\end{remark}

\end{document}