% !TEX root = ../notes.tex

\documentclass[../notes.tex]{subfiles}

\begin{document}

Today we finish up our discussion of Lusin's theorem.

\subsection{Lusin's Theorem}
The following lemmas will aide our proof of Lusin's theorem.
\begin{lemma} \label{lem:lusin-helper}
	Fix a measure space $(X,\mc M,\mu)$ and $p\ge1$. Suppose that $f\in L^p(X)$ and $\{f_n\}_{n\in\NN}\subseteq L^p(X)$ with $\norm{f_n-f}_p\to0$ as $n\to\infty$. Then there is a subsequence $\{f_{n_k}\}_{k\in\NN}$ such that $f_{n_k}\to f$ almost everywhere.
\end{lemma}
\begin{proof}
	We may choose our subsequence $\{n_k\}_{k\in\NN}$ to be increasing and so that
	\[\norm{f_{n_k}-f}_p<2^{-k}\]
	for each $k$. Now, set $h_k\coloneqq f_{n_k}-f$, and then \Cref{prop:cheby-ineq} tells us that
	\[\mu(\underbrace{\{x\in X:\left|h_k(x)-f(x)\right|>2^{-k/2}\}}_{E_k\coloneqq})\le\left(2^{-k/2}\right)^{-p}\norm{h_k-f}_p^p=2^{-pk/2}.\]
	We will use these $E_k$ to control our convergence.

	In particular, suppose that $x\in E_k$ for finitely many $k$. In other words, there is some $K$ such that $x\notin E_k$ for $k>K$, so $\left|h_k(x)-f(x)\right|<2^{-k/2}$ for $k>K$, so $f_{n_k}(x)\to f(x)$ as $k\to\infty$. So because
	\[\sum_{k=1}^\infty\mu(E_k)<\infty,\]
	we are able to conclude that the set of $x\in X$ living in infinitely many $E_\bullet$s is a null set by the following lemma.
	\begin{lemma}[Borel--Cantelli]
		Fix a measure space $(X,\mc M,\mu)$. Given a family $\{E_k\}_{k\in\NN}$ of measurable sets such that
		\[\sum_{k=1}^\infty\mu(E_k)<\infty,\]
		then the set of $x\in X$ living in infinitely many $E_\bullet$s is a null set.
	\end{lemma}
	\begin{proof}
		Consider the function $g\coloneqq\sum_{k\in\NN}1_{E_k}$. \Cref{thm:tonelli} grants
		\[\int_Xg\,d\mu=\sum_{k\in\NN}\int_X1_{E_k}\,d\mu=\sum_{k\in\NN}\mu(E_k)<\infty.\]
		Thus, $g$ must be finite almost everywhere, as required.
	\end{proof}
	The above lemma completes the proof of \Cref{lem:lusin-helper}.
\end{proof}
Now, here is our statement of Lusin's theorem. Intuitively, it says that any measurable function is almost continuous (even if it need not be continuous anywhere).
\begin{theorem}[Lusin]
	Fix a locally compact Hausdorff space $X$, and let $\mu$ be a nonnegative Radon measure on $X$ such that $\mu(X)<\infty$. Further, fix a measurable function $f\colon X\to\CC$.
	\begin{listalph}
		\item Suppose $f$ is ``small'' in the sense that $\mu(\{x:f(x)\ne0\})<\infty$. Then any $\varepsilon>0$ has some $g\in C_c(X)$ such that
		\[\mu(\{x\in X:f(x)\ne g(x)\})<\varepsilon.\]
		\item Suppose $f$ is ``small'' in the sense that $\mu(\{x:\left|f(x)\right|>t\})<\infty$ for all $t$. Then any $\varepsilon>0$ has some $g\in C_0(X)$ such that
		\[\mu(\{x\in X:f(x)\ne g(x)\})<\varepsilon.\]
	\end{listalph}
\end{theorem}
\begin{proof}
	We prove these separately.
	\begin{listalph}
		\item Set $X'\coloneqq\{x\in X:f(x)\ne0\}$ for brevity. We begin by claiming that there is a compact set $K\subseteq X$ sequence of continuous functions $\{g_n\}_{n\in\NN}\in C_c(X)$ supported in $K$ such that $\mu(X'\setminus K)<\varepsilon$ and $g_n(x)\to f(x)$ for almost every $x\in K$. (Note that this self-upgrades to $g_n(x)\to f(x)$ for every $x\in K$ by using \Cref{prop:radon-inner-reg-on-sigma-finite} after subtracting out from $K$ the points where we don't have this convergence.) Well, we can converge in norm using \Cref{lem:cc-dense-lp}, which we then upgrade to almost everywhere convergence by \Cref{lem:lusin-helper}.

		We now complete the proof. Construct $K$ and $\{g_n\}_{n\in\NN}$ as in the previous step. Because $\mu(K)<\infty$, Egorov's theorem \cite[Theorem~9.13]{elber-top} provides a subset $L\subseteq K$ such that $\mu(K\setminus L)<\varepsilon$ and $g_{n_i}\to f$ uniformly on $L$ for some subsequence $g_{n_i}$. By again using our inner regularity, we may assume that $L$ is compact, so $f|_L$ becomes continuous. We now use Tietze's extension theorem \cite[Theorem~3.26]{elber-top} to get some $g\in C_c(X)$ such that $g|_L=f|_L$ with arbitrarily small support outside $X'$, so we are done because $\mu(X'\setminus L)<2\varepsilon$ and noting that $g$ will vanish only a little outside $X'$.

		\item We use (a). We claim that we can write
		\[f\stackrel?=\sum_{n=0}^\infty f_n,\]
		where the $f_\bullet$s are measurable, satisfy $\mu(\{x\in X:f_n(x)\ne0\})<\infty$ and $\left|f_n(x)\right|<2^{-n}$ for all $x\in X$ and $n\ge1$. (Note that we are not asserting a bound on $f_0$!) To see how this completes the proof, use (a) to find $g_n\in C_c(X)$ such that
		\[\mu(\{x\in X:g_n(x)\ne f_n(x)\})<2^{-n}\varepsilon\]
		for each $n\ge0$. In fact, because $\norm{f_n}_u\le2^{-n}$, we can achieve $\norm{g_n}_u\le2^{-n}$ by using the approximation trick at the end of the proof of \Cref{thm:rr-2} involving $\varphi$. Now, the bound on the uniform bounds implies that $g\coloneqq\sum_ng_n$ is the uniform limit of continuous functions on $X$, so $g$ is continuous. In fact, $g$ lives in the closure of $C_c(X)$, which we know is $C_0(X)$ by \Cref{prop:co-banach,prop:closure-of-cc}. Lastly, we bound
		\[\mu(\{x\in X:f_n(x)\ne g_n(x)\})\le\mu\left(\bigcup_{n\in\NN}\{x\in X:g_n(x)\ne f_n(x)\}\right)\le\sum_{n=0}^\infty2^{-n}\varepsilon<2\varepsilon,\]
		which vanishes as $\varepsilon\to0^+$.

		It remains to prove the claim. Write $f(x)=\left|f(x)\right|e^{ih(x)}$, where we know now that $h$ is a real-valued measurable function. We now decompose $F\coloneqq\left|f\right|$ into a sum of small real-valued functions $G_\bullet$, and then we will set $g_\bullet\coloneqq G_\bullet e^{ih}$. Well, simply cut up $F$ by
		\[G_0(x)\coloneqq1_{F(x)>1}F(x),\]
		and
		\[G_n(x)\coloneqq1_{1/2^{n-1}\ge F(x)>1/2^{n}}F(x)\]
		for $n\ge1$. Then $F=\sum_{n=0}^\infty G_n$, so $f=\sum_{n=0}^\infty G_ne^{ih}$, and we are done.
		\qedhere
	\end{listalph}
\end{proof}

\end{document}