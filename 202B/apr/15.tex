% !TEX root = ../notes.tex

\documentclass[../notes.tex]{subfiles}

\begin{document}

Today we continue our discussion of the second Riesz representation theorem.

\subsection{Proving the Second Riesz Representation Theorem}
The main claim in the proof is as follows, which allows us to reduce understanding signed functionals into nonnegative ones.
\begin{proposition} \label{prop:decompose-bounded-linear-functional}
	Fix a locally compact Hausdorff space $X$ and a real-valued bounded linear functional $I\colon C_0(X)\to\RR$. Then there are bounded linear nonnegative functionals $I^\pm\colon C_0(X)\to\RR$ such that
	\[I=I^+-I^-.\]
\end{proposition}
For this, we will want the following lemma.
\begin{lemma}
	Fix a locally compact Hausdorff space $X$ and a real-valued bounded linear functional $I\colon C_0(X)\to\RR$. Then define $J\colon C_0(X)\to\RR$ by
	\[J(f)\coloneqq\sup\{I(g)-I(f):g,h\in C_0(X),g\ge0,h\ge0,g+h=f\}\]
	is additive in the following sense: for nonnegative $f_1,f_2\in C_0(\RR)$, we have $J(f_1+f_2)=J(f_1)+J(f_2)$.
\end{lemma}
\begin{proof}
	Approximately speaking, $J$ amounts to an ``absolute value'' of $I$; a similar construction is used in the Hahn--Jordan decomposition theorem. HI

	Anyway, fix nonnegative $f_1$ and $f_2$. In one direction, then fix nonnegative $g_1,h_1,g_2,h_2\in C_0(X)$ such that $f_\bullet=g_\bullet+h_\bullet$, so we see that
	\[J(f_1+f_2)\ge I(g_1+g_2)-I(h_1+h_2)=I(g_1)-I(h_1)+I(g_2)-I(h_2),\]
	so by taking suprema over $g_1,h_1,g_2,h_2$, we see that $J(f_1+f_2)\ge J(f_1)+J(f_2)$.

	In the other direction, fix nonnegative $g,h\in C_0(X)$ such that $g+h=f_1+f_2$. We now need to construct some $g_1,h_1,g_2,h_2$, which will be somewhat hard. Define
	\[g_1\coloneqq\min\{f_1,g\}.\]
	Note $0\le g_1\le f_1$ and $g_1\le g$, so it lives in $C_0(X)$. This allows us to define $h_1\coloneqq f_1-g_1$ and then $g_2\coloneqq g-g_1$ to also be nonnegative functions in $C_0(X)$, and we also get $f_1=g_1+h_1$ and $f_2=g_2+h_2$ by construction. Thus, we see that
	\[J(f_1)+J(f_2)\ge I(g_1)-I(h_1)+I(g_2)+I(h_2)=I(g)-I(h),\]
	which achieves $J(f_1)+J(f_2)\ge J(f_1+f_2)$ by taking a supremum over all $g$ and $h$.
\end{proof}
\begin{remark}
	Fix a nonnegative $f\in C_0(X)$. Defining $I$ and then $J$ as above, we claim that
	\[J(f)\le2\norm f_u\norm I_{C_0(X)^*}.\]
	Indeed, for any decomposition $f=g+h$ with nonnegative $g,h\in C_0(X)$, we find that
	\[J(f)\le\left|I(g)\right|+\left|I(h)\right|\le\norm g_u\norm I_{C_0(X)^*}+\norm f_u\norm I_{C_0(X)^*},\]
	which provides the claim.
\end{remark}
\begin{remark}
	Fix a nonnegative $f\in C_0(X)$. Defining $I$ and then $J$ as above, we claim that
	\[J(f)\ge\left|I(f)\right|.\]
	Namely, one decomposes as $f=f+0$ and $f=0+f$ to see that $J(f)\ge I(f)$ and $J(f)\ge-I(f)$, as needed.
\end{remark}
We are now ready to prove \Cref{prop:decompose-bounded-linear-functional}.
\begin{proof}[Proof of \Cref{prop:decompose-bounded-linear-functional}]
	Define $J$ as in the above lemma, and define
	\[I^+(f)\coloneqq J(f^+)-J(f^-)\]
	where $f=f^+-f^-$ is the decomposition into positive and negative values. We claim that $I^+$ is nonnegative and bounded and linear. For nonnegativity, we note that nonnegative $f\in C_0(X)$ have $f=f^+$ and then $J(f)\ge0$, so we are okay. Being bounded follows by using the bound on $J(f)$ to find that $I^+(f)\le4\norm f_u\norm I_{C_0(X)^*}$.
	
	For linearity, note that we already satisfy a scalar multiplication requirement for nonnegative reals because $J(cf)=cJ(f)$ for $c\ge0$ and $f$ nonnegative; then we get the linearity for $c=-1$ because this amounts to swapping the terms in the definition of $I^+(f)=J(f^+)-J(f^-)$. As such, we really only need some additivity check on $I^+$, for which we use the additivity we know on $J$. Fix $f_1,f_2\in C_0(X)$, and we find that
	\[I^+(f_1+f_2)=I^+(f_1)+I^+(f_2)\]
	is equivalent to
	\[J(f_1^++f_2^+)+J(f_1^-)+J(f_2^-)=J(f_1^+)+J(f_2^+)+J((f_1+f_2)^-)\]
	by some rearranging. By additivity of $J$ from the lemma, it is enough to check that
	\[f_1^++f_2^++f_1^-+f_2^-=f_1^++f_2^++(f_1+f_2)^-,\]
	which amounts to $f_1+f_2=f_1+f_2$ after rearranging back.

	To complete the proof, we set $I^-\coloneqq I^+-I$. The nonnegativity check amounts to knowing that $I^+(f)\ge I(f)$ for nonnegative $f$, which we already know. The boundedness and linearity of $I^-$ follows because $I^+,I\in C_0(X)^*$ already.
\end{proof}
We are now ready to prove \Cref{thm:rr-2}.
\rrtwothm*
\begin{proof}
	Any complex $I\in C_0(X)^*$ can be decomposed into real and imaginary parts; the conclusion is $\CC$-linear, so we may suppose that $I$ is real-valued. Now, we may use \Cref{prop:decompose-bounded-linear-functional} to write $I=I^+-I^-$ where $I^+$ and $I^-$ are bounded linear nonnegative functionals on $C_0(X)$. So \Cref{thm:rr-1} provides Radon measures $\mu^\pm$ such that
	\[I^\pm(f)=\int_Xf\,d\mu^\pm\]
	for any choice of sign and $f\in C_c(X)$. Note that $\mu^\pm$ is a finite measure because
	\[\mu^\pm(X)=\sup_{f<X}I^\pm(f)\le\norm{I^\pm}\]
	by the definition of $\mu^\pm$. As such, we note that $I^\pm$ and $f\mapsto\int_Xf\,d\mu^\pm$ are both continuous (we are using the finiteness of $\mu^\pm$ here!), so the density of $C_c(X)\subseteq C_0(X)$ implies that
	\[I^\pm(f)=\int_Xf\,d\mu^\pm\]
	holds for any $f\in C_0(X)$.

	Lastly, we must actually show that $\norm\mu=\norm I_{C_0(X)^*}$. Again, by squaring and using linearity, we see that we may assume that $I$ is real-valued. Additionally, we certainly know that
	\[\norm I_{C_0(X)^*}=\sup_{f\in C_0(X)^*}\frac{\left|I(f)\right|}{\norm f_u}=\sup_{f\in C_0(X)^*}\frac1{\norm f_u}\left|\int_Xf\,d\mu\right|\le\sup_{f\in C_0(X)}\left|\mu\right|(X)=\left|\mu\right|(X),\]
	so we get the inequality $\norm\mu_{M(X)}\ge\norm I_{C_0(X)^*}$. It remains to achieve the other inequality. For this, we want the following lemma.
	\begin{lemma} \label{lem:cc-dense-lp}
		Fix a locally compact Hausdorff topological space $X$, and let $\lambda$ be a nonnegative Radon measure on $X$. Then $C_c(X,\CC)$ is dense in $L^p(X,\lambda)$.
	\end{lemma}
	\begin{proof}
		Any element of $L^p(X,\lambda)$ can be approximated by a finite linear combination of simple functions, so because $C_c(X,\CC)$ is a subspace, it is enough to show that we can approximate simple functions.
		
		Thus, we want to approximate $f=1_E$ where $E$ is a Borel set of finite measure. Fix some $\varepsilon>0$ to be sent to $0$ later. Then we use regularity (both inner via \Cref{prop:radon-inner-reg-on-sigma-finite} and outer regularity!) to find a compact set $K\subseteq E$ and an open set $E\subseteq U$ such that $\mu(U\setminus K)<\varepsilon$. So \Cref{lem:lch-urysohn} promises $g\in C_c(X)$ such that $g|_K=1$ and $g<U$. Thus, we see that
		\[\norm{f-g}_p^p=\int_X\left|f-g\right|^p\,dx,\]
		and the function $\left|f-g\right|$ vanishes on $K$ and $X\setminus U$, and it always at most $1$, so we get
		\[\norm{f-g}_p<\varepsilon^{1/p}.\]
		Sending $\varepsilon\to0^+$ completes the proof.
	\end{proof}
	We now show that $\norm\mu_{M(X)}\le\norm I_{C_0(X)^*}$. Namely, for each $\varepsilon>0$, we must produce nonzero $f\in C_0(X)^*$ with $\norm f_u\le1$ such that
	\[\left|\int_Xf\,d\mu\right|\stackrel?\ge\norm\mu_{M(X)}-\varepsilon.\]
	Because $\left|\mu\right|$ is a Radon measure, inner regularity for open subsets tells us that
	\[\norm\mu_{M(X)}=\left|\mu\right|(X)=\sup_{f<X}\int_Xf\,d\left|\mu\right|.\]
	Now, we let $h$ be the Radon--Nikodym derivative $h\coloneqq d\mu/d\left|\mu\right|$, and one can show that $\left|h\right|=1$ almost everywhere. Thus, changing variables, we get
	\[\norm\mu_{M(X)}=\sup_{f<X}\left|\int_Xfh\,d\mu\right|.\]
	The problem, now, is that $h$ is merely a measurable function, but the function we need to construct must be continuous.
	
	Well, note $h\in L^1(X,\left|\mu\right|)$ by its construction because $h$ is bounded and $\left|\mu\right|(X)<\infty$. So choose $\varepsilon>0$, and use \Cref{lem:cc-dense-lp} to construct $g\in C_c(X)$ such that $\norm{h-g}_1<\varepsilon$. In fact, we claim we can choose $g$ so that $\norm g_u\le1$ as well: indeed, the main point is that $\left|h\right|=1$ almost everywhere, so we will simply modify the required $g$ to work. Well, define $\varphi\colon\CC\to\CC$ by
	\[\varphi(z)\coloneqq\begin{cases}
		z & \text{if }\left|z\right|\le1, \\
		z/\left|z\right| & \text{if }\left|z\right|>1,
	\end{cases}\]
	and we  note that $\left|\varphi(z)-\varphi(w)\right|\le\left|z-w\right|$ by a direct computation. We now see that $\norm{\varphi\circ g}_u\le1$, and we want to check that we still have $\norm{h-(\varphi\circ g)}_1<\varepsilon$. Well, because $\left|h\right|\le1$ almost everywhere, we have $h=\varphi\circ h$ almost everywhere, so $\norm{h-g}_1\ge\norm{h-\varphi\circ g}_1$, as required.
	
	Next, choose $f$ with $f<X$ and
	\[\left|\int_Xfh\,d\mu\right|>\norm\mu_{M(X)}-\varepsilon.\]
	Then $fg\in C_c(X)\subseteq C_0(X)$, so we are able to compute
	\[\left|\int_Xfg\,d\mu\right|>\left|\int_Xfh\,d\mu\right|-\left|\int_Xf(g-h)\,d\mu\right|>\left(\norm\mu_{M(X)}-\varepsilon\right)-\norm f_u\norm{g-h}_1>\norm\mu_{M(X)}-2\varepsilon.\]
	Sending $\varepsilon\to0^+$ completes our approximation.
\end{proof}

\end{document}