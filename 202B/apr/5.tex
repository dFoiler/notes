% !TEX root = ../notes.tex

\documentclass[../notes.tex]{subfiles}

\begin{document}

We began class by proving \Cref{thm:partition-unity}.

\subsection{Some More Spaces of Functions}
Here are some more spaces of functions.
\begin{definition}
	Fix a topological space $X$.
	\begin{itemize}
		\item $C(X)$ is the space of continuous functions $f\colon X\to\CC$.
		\item $C_c(X)$ is the space of compactly supported continuous functions $f\colon X\to\CC$.
		\item $C_0(X)$ is approximately the space of continuous functions $f$ such that $f(x)\to0$ as $x\to\infty$. Rigorously, we see that $f\in C_0(X)$ if and only if any $\varepsilon>0$ has some compact set $K_\varepsilon\subseteq X$ containing
		\[\{x\in X:\left|f(x)\right|\ge\varepsilon\}.\]
	\end{itemize}
\end{definition}
\begin{remark}
	Note $C_c(X)\subseteq C_0(X)\subseteq C(X)$. The last inclusion has no content, and the first inclusion holds because $f\in C_c(X)$ has a single compact set $K$ containing its support so that any $\varepsilon>0$ has $\left|f\right|^{-1}((\varepsilon,\infty))$.
\end{remark}
\begin{remark} \label{rem:want-duals}
	One might wonder why we are examining $C_0(X)$ instead of the space of bounded continuous function $X\to\CC$. The reason is that the dual space of $C_0(X)$ is much better-behaved, as we will soon see.
\end{remark}
These definitions allow us to make sense of a uniform norm.
\begin{definition}[uniform norm]
	Fix a topological space $X$. For $f\in C_0(X)$, we define
	\[\norm f_u\coloneqq\sup_{x\in X}\left|f(x)\right|.\]
\end{definition}
\begin{remark}
	Let's check that $\norm\cdot_u$ is finite on $C_0(X)$. Well, for $f\in C_0(X)$, we are granted a compact set $K$ such that $\left|f(x)\right|<1$ outside $K$. But then $\norm f_u$ is bounded above by $1$ outside $K$ and has a maximum on $K$, so $\norm f_u$ in total is bounded.
\end{remark}
\begin{remark} \label{rem:uniform-norm-is-norm}
	It is true that $\norm\cdot_u$ becomes a norm on $X$, which we will not check too closely. Homogeneity has little content, and positive-definiteness holds because $\norm f_u=0$ would require that all values vanish. Lastly, the triangle inequality.
\end{remark}
\begin{example}
	Take $X\coloneqq\NN$. Recall $C_0(\NN)$ is the set of sequences $\{a_n\}_{n\in\NN}$ which converge to $0$.
\end{example}
Here are some fun checks.
\begin{proposition}
	For any topological space $X$, the space $C_0(X)$ is a Banach space.
\end{proposition}
\begin{proof}
	In light of \Cref{rem:uniform-norm-is-norm}, we just have to check that $C_0(X)$ is complete. Well, suppose that $\{f_n\}_{n\in\NN}$ is Cauchy for $\norm\cdot_u$. So we are uniformly Cauchy, so our sequence converges to a continuous function $f\colon X\to\CC$; in fact, this convergence is uniform.
	
	It remains to check that $f\in C_0(X)$. Fix some $\varepsilon>0$, and we need a compact set $K$ such that $\left|f(x)\right|<\varepsilon$ outside $K$. Well, convergence in $\norm\cdot_u$ grants some $n\in\NN$ such that $\norm{f-f_n}_u<\varepsilon/2$, and then $f_n\in C_0(X)$ promises some $K$ such that $\left|f_n(x)\right|<\varepsilon$ outside $K$. Combining, we see that $\left|f(x)\right|<\varepsilon$ outside $K$, as needed.
\end{proof}
\begin{proposition}
	For any locally compact Hausdorff space $X$, the space $C_c(X)$ is dense in $C_0(X)$.
\end{proposition}
\begin{proof}
	We use \Cref{lem:lch-urysohn}. Fix $f\in C_0(X)$ and $\varepsilon>0$, and we need $g\in C_c(X)$ such that $\norm{f-g}_u<\varepsilon$.

	The point is to truncate $f$ where it is small. Because $f\in C_0(X)$, there is a compact set $K\subseteq X$ such that $\left|f(x)\right|<\varepsilon$ outside $K$. Then \Cref{lem:lch-urysohn} provides a continuous function $h\colon X\to[0,1]$ such that $h|_K=1$ and $h$ is compactly supported. So we set $g\coloneqq fh$, and we find that $\left|f(x)-g(x)\right|=0$ on $K$ and $\left|f(x)-g(x)\right|=\left|f(x)\right|\cdot\left|1-h(x)\right|$ is bounded by $\varepsilon$ outside $K$. So $g$ works.
\end{proof}
\begin{remark}
	In particular, if $X$ is locally compact Hausdorff, then we see that $C_c(X)$ fails to be complete: its closure in $C(X)$ contains $C_0(X)$!
\end{remark}

\subsection{Functionals and Measures}
In light of \Cref{rem:want-duals}, we will soon want to understand our dual space. We pick up the following definition.
\begin{definition}[nonnegative]
	Fix a topological space $X$. A linear functional $I\colon C_0(X)\to\CC$ is \textit{nonnegative} if and only if $I(f)\ge0$ whenever $f\colon X\to[0,\infty)$.
\end{definition}
\begin{remark}
	By taking the difference, we see that $f\le g$ implies $0\le g-f$, which implies $0\le I(g-f)$, which implies $I(f)\le I(g)$ by linearity.
\end{remark}
We will learn that nonnegative bounded linear functionals give rise to nonnegative measures; loosening the nonnegative hypothesis gives rise to complex measures.
\begin{example}
	Fix a nonnegative Borel measure $\mu$ on a topological space $X$, and suppose that $\mu$ is finite on compact sets. Then the map $C_c(X)\to\RR$ given by
	\[f\mapsto\int_Xf\,d\mu\]
	is a nonnegative functional on $X$. Notably, nonnegativity follows by the nonnegativity of $\mu$.
\end{example}
Being finite on compact sets is something of an annoying condition, though it is true for all measures that come up in nature. So we pick up the following adjective to promise this.
\begin{defihelper}[outer regular, inner regular, Radon] \nirindex{Radon} \nirindex{outer regular} \nirindex{inner regular}
	Fix a Borel measure $\mu$ on a topological space $X$.
	\begin{itemize}
		\item $\mu$ is \textit{outer regular} on $E\subseteq X$ if and only if $\mu(E)=\inf\{\mu(U):U\supseteq E\}$.
		\item $\mu$ is \textit{inner regular} on $E\subseteq X$ if and only if $\mu(E)=\sup\{\mu(K):K\subseteq E\}$.
		\item $\mu$ is \textit{Radon} if and only if $\mu$ is finite on compact sets, outer regular on all sets, and inner regular on open sets.
	\end{itemize}
\end{defihelper}

\end{document}