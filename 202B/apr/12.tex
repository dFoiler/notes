% !TEX root = ../notes.tex

\documentclass[../notes.tex]{subfiles}

\begin{document}

We now move towards the second Riesz representation theorem.

\subsection{More on Regularity}
Let's give a few more remarks on regularity.
\begin{proposition} \label{prop:radon-inner-reg-on-sigma-finite}
	Fix a Radon Borel measure $\mu$ on a locally compact Hausdorff topological space $X$. Then $\mu$ is inner regular on every $\sigma$-finite Borel set.
\end{proposition}
\begin{proof}
	Because $\mu$ is countably additive already, we may just show that $\mu$ is inner regular on sets of finite measure by then taking countable disjoint unions afterwards. Now, we are trying to show that
	\[\mu(E)=\sup_{K\subseteq E}\mu(K).\]
	Certainly we get $\ge$ because $K\subseteq E$ implies $\mu(K)\le\mu(E)$.
	
	So it remains to get $\le$. Fix some $\varepsilon>0$ to be sent to zero later. By outer regularity, we may approximate $E$ by an open neighborhood $U$ with $\mu(U\setminus E)<\varepsilon$. Further, by the inner regularity, we may approximate $U$ by a compact subset $K\subseteq U$ with $\mu(U)$.
	
	We would like to use $K$ to approximate $E$, but it is not clear how to compare the two because $K$ need not be contained in $E$. Well, again using outer regularity, we get an open neighborhood $U'$ of $U\setminus E$ such that $\mu(U')<2\varepsilon$, and then $K\setminus U'$ is compact, and we see
	\[\mu(K\setminus U')\ge\mu(K)-\mu(U')\ge\mu(E)-\varepsilon-\mu(U')\ge\mu(E)-3\varepsilon.\]
	Sending $\varepsilon\to0^+$ provides the required $\le$ inequality.
\end{proof}
\begin{corollary}
	Fix a locally compact Hausdorff $\sigma$-compact space $X$. Then any Radon Borel measure $\mu$ on $X$ is inner regular.
\end{corollary}
\begin{proof}
	Note that any Borel set $E\subseteq X$ is certainly $\sigma$-countable because $X$ itself can be covered by countably many compact sets. But then $\mu$ is countably additive, so we can decompose $E$ as an ascending union of sets of finite measure (compact implies finite measure by Radon), so $E$ is $\sigma$-finite. So we are done by \Cref{prop:radon-inner-reg-on-sigma-finite}.
\end{proof}
\begin{proposition}
	Fix a locally compact Hausdorff topological space $X$, and suppose every open subset $U\subseteq X$ is $\sigma$-compact. Then any Borel measure $\lambda$ on $X$ which is finite on compact sets is in fact Radon.
\end{proposition}
\begin{proof}[Sketch]
	We use \Cref{thm:rr-1}. The point is that we can define a nonnegative linear functional $I\colon C_c(X)\to\RR$ by
	\[I(f)\coloneqq\int_Xf\,d\lambda,\]
	where the integral on the right-hand side is defined because $\lambda$ is a Borel measure which is finite on compact sets. But then \Cref{thm:rr-1} provides a Radon measure $\mu$ such that
	\[\int_Xf\,d\mu=\int_Xf\,d\lambda\]
	for any $f\in C_c(X)$. It remains to check that $\lambda=\mu$, which can be done using the relevant regularities to reduce everything to checking inequalities on open and compact sets.
\end{proof}

\subsection{The Second Riesz Representation Theorem}
To state the second Riesz representation theorem, we want the following notion.
\begin{definition}[complex measure]
	Fix a measurable space $(X,\mc M)$. Then a \textit{complex measure} $\mu$ is a function $\mu\colon\mc M\to\CC$ which is countably additive. Namely, for any countably collection $\{E_n\}_{n\in\NN}$ of subsets in $\mc M$, set $E\coloneqq\bigsqcup_{n\in\NN}E_n$, and then
	\[\mu(E)=\sum_{n\in\NN}\mu(E_n).\]
\end{definition}
\begin{remark}
	We are assuming that the sum converges in our definition of $\mu$. This implies that the sum converges no matter how we arrange the sum. This actually implies that the sum absolutely converges! Indeed, proceeding by contraposition, one can show that a conditionally convergent series can be rearranged to sum to any value by a careful approaching our needed terms.
\end{remark}
\begin{remark}
	Any complex measure can be written as $\alpha+i\beta$ where $\alpha$ and $\beta$ are signed measures: simply take the real and imaginary parts. Further, any signed measure can be written as the difference of nonnegative measures: simply take the positive and negative parts. Notably, given a signed measure $\lambda$, there are unique measures $\lambda^+$ and $\lambda^-$ such that $\lambda=\lambda^+-\lambda^-$ and $\lambda^+$ and $\lambda^-$ are orthogonal (namely, any set can be decomposed into a disjoint union where at most one of the two measures is nonzero on each component). This last sentence is the Hahn decomposition theorem.
\end{remark}
\begin{remark}
	There is a Radon--Nikodym theorem for complex measures.
\end{remark}
We now define Radon measures.
\begin{definition}[Radon]
	A signed measure $\lambda$ is \textit{Radon} if and only if $\lambda^+$ and $\lambda^-$ are Radon. A complex measure $\mu$ is \textit{Radon} if and only if $\Re\mu$ and $\Im\mu$ are both Radon. We let $M(X)$ denote the collection of complex Radon measures on $X$.
\end{definition}
\begin{remark}
	One can show that $M(X)$ is a Banach space (over $\CC$), with norm given by taking the norm of $\int_Xd\left|\mu\right|$ for any $\mu\in M(X)$. Showing that this is a normed vector space is not too hard, and showing completeness amounts to checking that we can take infinite sums of our measures, which is not too hard.
\end{remark}
At long last, here is our theorem.
\begin{restatable}[Riesz representation]{theorem}{rrtwothm} \label{thm:rr-2}
	Fix a locally compact Hausdorff topological space $X$. For any bounded linear functional $I\in C_0(X)^*$, there is a unique Radon complex measure $\mu\in M(X)$ such that
	\[I(f)=\int_Xf\,d\mu\]
	for all $f\in C_0(X)$. This is a norm-preserving bijection between $C_0(X)^*$ and $M(X)$.
\end{restatable}
\noindent What is remarkable here is that $C_0(X)$ is a largely topological construction, but we see that understanding it requires a thorough handle of integration.

We spend the rest of class to describe the outline.
\begin{itemize}
	\item Taking real and imaginary parts, we reduce to the case where $I$ is real-valued.
	\item Then we need to write $I$ as a difference $I^+-I^-$ where $I^+$ and $I^-$ are nonnegative and linear.
	\item From here, we can complete the proof by \Cref{thm:rr-1}.
\end{itemize}
The second step is the hardest. The main point is to define
\[\left|I\right|(f)\coloneqq\sup\left\{I(g)-I(h):g,h\in C_0(X),0\le g,h,g+h=f\right\}.\]
The idea is that $\left|I\right|(f)$ is able to expose any cancellation used in a (continuous) decomposition of $f$. However, one must show that $\left|I\right|$ is linear, which is far from obvious.

\end{document}