% !TEX root = ../notes.tex

\documentclass[../notes.tex]{subfiles}

\begin{document}

We are now ready to state and begin the proof of the Riesz representation theorem.

\subsection{Riesz Representation}
Here is our statement.
\begin{theorem} \label{thm:rr-1}
	Fix a locally compact Hausdorff topological space $X$. Then for any nonnegative linear functional $I\colon C_c(X)\to\CC$. Then there is a unique Radon Borel measure $\mu$ on $X$ such that
	\[I(f)=\int_Xf\,d\mu\]
	for any $f\in C_c(X)$.
\end{theorem}
\begin{remark}
	If one only knows about Riemann integration, one can define
	\[I(f)\coloneqq\int_{\RR^d}f(x)\,dx,\]
	where the right-hand side is a Riemann integral! Then one gets our a Radon Borel measure $\mu$ from \Cref{thm:rr-1}, which can be used as a definition of the Lebesgue measure.
\end{remark}
Anyway, let's prove \Cref{thm:rr-1}.
\begin{proof}[Proof of \Cref{thm:rr-1}]
	We proceed with many steps. The general strategy is to reduce everything to open and compact subsets, where we have some control due to \Cref{lem:lch-urysohn}.
	\begin{enumerate}
		\item We begin with the proof of uniqueness. Suppose that $\mu$ is a Radon Borel measure satisfying the conclusion. Then we claim that
		\[\mu(U)\stackrel?=\sup_{f<U}I(f)\]
		for all open subsets $U\subseteq X$; the outer regularity of $\mu$ then implies that this pins down $\mu$.
	
		In one direction, suppose that $f<U$, and we want to show that $\mu(U)\ge I(f)$. Indeed, $f<U$ implies that
		\[I(f)=\int_Xf\,d\mu\le\int_X1_U\,d\mu=\mu(U).\]
		In the other direction, we must show that $\sup\{I(f):f<U\}\ge\mu(U)$. Well, by inner regularity of $\mu$, it suffices to show that $\mu(K)\le\sup\{I(f):f<U\}$ for any compact subset $K\subseteq U$. Well, \Cref{lem:lch-urysohn} grants $f_K\in C_c(X,[0,1])$ such that $f_K|_K=1$ and $f_K<U$, so
		\[\mu(K)\le\int_Xf_K\,d\mu=I(f_K)\le\sup_{f<U}\{I(f):f<U\},\]
		as desired.

		\item We now move towards our proof of existence. We are going to use Carath\'eodory's Theorem \cite[Theorem~6.21]{elber-top}. Let's set some notation. Motivated by our proof of uniqueness, we define
		\[\rho(U)\coloneqq\sup_{f<U}I(f)\]
		for open subsets $U\subseteq X$, and we define
		\[\rho^*(E)\coloneqq\inf_{E\subseteq U}\rho(U)\]
		for any subset $E\subseteq X$.

		We take a moment that we are going to constantly be dealing with infimums of supremums, which are complicated by nature, so the rest of the proof is going to be rather technical.
		
		\item We claim that $\rho^*(U)=\rho(U)$ for any open subset $U\subseteq X$. Certainly $\rho^*(U)\le\rho(U)$ by definition because $U\subseteq U$. For the other direction, suppose that $U\subseteq V$ for an open subset $V$. Then we must show that $\rho(U)\le\rho(V)$. Expanding, we are asking for
		\[\sup_{f<U}I(f)\le\sup_{f<V}I(f),\]
		for which we can show that $I(f_U)\le\sup\{I(f):f<V\}$ for any $f_U<U$. But this is direct: if $f_U<U$, then $f_U<V$ too, so the inequality follows.

		\item We claim that $\rho^*$ is an outer measure. Certainly $\mu^*$ is monotone: if $E\subseteq E'$, then $\{\rho(U):E\subseteq U\}$ is a superset of $\{\rho(U):E'\subseteq U\}$, so $\inf\{\rho(U):E\subseteq U\}\le\inf\{\rho(U):E'\subseteq U\}$.

		So now we want to show that $\rho^*$ is countably subadditive. Well, for a countable collection of subsets $\{E_i\}_{i\in\NN}\subseteq X$, then we want to show that
		\[\rho^*\Bigg(\bigcup_{i\in\NN}E_i\Bigg)\stackrel?\le\sum_i\rho^*(E_i).\]
		By definition of $\rho^*$ as an infimum, it will be enough to show that
		\[\rho^*\Bigg(\bigcup_{i\in\NN}E_i\Bigg)\stackrel?\le\sum_{i\in\NN}\rho(U_i),\]
		where $\{U_i\}_{i\in\NN}$ has $E_i\subseteq U_i$ for each $i$. (One has to make some argument of choosing $U_i$ such that $\rho^*(E_i)\le\rho(U_i)+\varepsilon/2^i$ or something, but it will go through.) But then $\rho^*$ being monotone means that it will be enough to show that
		\[\rho\Bigg(\sum_{i\in\NN}U_i\Bigg)\stackrel?\le\sum_{i\in\NN}\rho(U_i),\]
		so we are now showing that $\rho$ is countably subadditive. By expanding out $\rho$ as a supremum, we must show that having $f<\bigcup_{i\in\NN}U_i$ implies that we want
		\[I(f)\stackrel?\le\sum_{i\in\NN}\rho(U_i).\]
		This requires us to build some functions $f_i$ out of $f$ with $f_i<U_i$. Well, we use \Cref{thm:partition-unity}. Let $K\coloneqq\op{supp}f$ be the support of $f$, which we know is compact. Then $\{U_i\}_{i\in\NN}$ is an open cover of $K$, so we choose $N$ large enough so that $\{U_i\}_{1\le i\le N}$ is an open cover of $K$. Then \Cref{thm:partition-unity} provides functions $\{g_i\}_{1\le i\le N}$ such that $\sum_{i=1}^Ng_i|_K=1$ but $g_i<U_i$ for each $i$.

		We are now ready to define $f_i\coloneqq fg_i$ for each $1\le i\le N$. Because $f$ is compactly supported, we see that each $f_i$ is compactly supported. Additionally, we know that
		\[\sum_{i=1}^Nf_i=f.\]
		Indeed, this is true on $K$ because $\sum_{i=1}^Ng_i|_K=1$, and it is true outside $K$ because everything in sight vanishes. Thus, we are able to write
		\[I(f)=\sum_{i=1}^NI(f_i)\stackrel*\le\sum_{i=1}^N\rho(U_i)\le\sum_{i\in\NN}\rho(U_i),\]
		where $\stackrel*\le$ is because $I(f_i)\le\rho(U_i)$ (note $f_i<U_i$ because $g_i<U_i$). This is what we wanted.

		\item We would like to apply Carath\'eodory's Theorem \cite[Theorem~6.21]{elber-top} to take the outer measure $\rho^*$ to produce a Borel measure $\mu$. However, \cite[Theorem~6.21]{elber-top} produces a measure $\mu$ on the $\sigma$-algebra of subsets $A\subseteq X$ such that
		\[\rho^*(E)=\rho^*(E\cap A)+\rho^*(E\setminus A)\]
		for every set $E\subseteq X$. To restrict this measure $\mu$ to a Borel measure, we must show that the Borel algebra on $X$ satisfies the above condition. Because the Borel algebra is generated by open subsets of $X$, it is enough to check that
		\[\rho^*(E)\stackrel?=\rho^*(E\cap U)+\rho^*(E\setminus U)\]
		for any open subset $U\subseteq X$ and subset $E\subseteq X$. Note that the previous step's countably subadditive check proves $\le$, so we just need to check $\ge$.

		Well, let's reduce our check to something we can check on opens. By definition of $\rho^*$ as an infimum, it is enough to show that any open subset $V$ containing $E$ will achieve
		\[\rho^*(V)\stackrel?\ge\rho^*(E\cap U)+\rho^*(E\setminus U),\]
		but by monotonicity, it will be enough for
		\[\rho^*(V)\stackrel?\ge\rho^*(V\cap U)+\rho^*(V\setminus U).\]
		Note we may replace the first two $\rho^*$s with $\rho$s. As such, by definition of $\rho$, it is enough to show that
		\[\rho(V)\stackrel?\ge I(f)+\rho^*(V\setminus U)\]
		for any $f<(V\cap U)$. Define $K\coloneqq\op{supp}f$, which is notably closed, and we actually claim that
		\[\rho(V)\stackrel?\ge I(f)+\rho(V\setminus K),\]
		which will be enough by monotonicity: $K\subseteq U$ implies $V\setminus K\supseteq V\setminus U$, which implies $\rho(V\setminus K)=\rho^*(V\setminus K)\ge\rho^*(V\setminus U)$. Well, for this, by definition of $\rho$, we know that it is enough to check
		\[\rho(V)\stackrel?\ge I(f)+I(g)\]
		for any $g<V\setminus K$. Well, $I(f)+I(g)=I(f+g)$, and $f+g$ is supported on $V$, where the $f$ piece is supported on $f$, and the $g$ piece is supported on $V\setminus K$. So the above claim follows by definition of $\rho$.
	\end{enumerate}
	We will continue the proof next class.
\end{proof}

\end{document}