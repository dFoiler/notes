% !TEX root = ../notes.tex

\documentclass[../notes.tex]{subfiles}

\begin{document}

We began class by finishing a proof from last class, into which I have edited directly.

\subsection{Change of Variables}
Before doing anything, here are some lemmas.
\begin{lemma} \label{lem:compact-convex-to-lip}
	Fix a compact convex subset $K\subseteq\RR^d$. Suppose $U\subseteq\RR^d$ is an open subset containing $K$, and suppose $f\colon U\to\RR^n$ is continuously differentiable. Then $f|_K$ is Lipschitz.
\end{lemma}
\begin{proof}
	Fix $x,y\in K$. We would like to approximate $f(x)-f(y)$ by $Df(x)(y-x)$, for which we use the Mean value theorem.
	
	To bring our target down to one dimension, we note
	\[\left|f(y)-f(x)\right|=w\cdot(f(y)-f(x)),\]
	where $w$ is the unit vector in the direction of $f(y)-f(x)$. In order to bring our source down to one dimension, we define
	\[g(t)\coloneqq w\cdot f(x+t(y-x)),\]
	which is well-defined because $K$ is convex. Note $g$ is continuously differentiable because it is the composition of continuously differentiable functions; in particular, we find $g'(t)=w\cdot Df(x+t(y-x))(y-x)$. Everything is now in one dimension, so the Mean value theorem provides $c\in(0,1)$ such that
	\[\frac{g(1)-g(0)}{1-0}=g'(c)=w\cdot Df(x+c(y-x))(y-x).\]
	Thus, we may have the sequence of bounds
	\begin{align*}
		\left|f(y)-f(x)\right| &= \left|g(1)-g(0)\right| \\
		&= \left|w\cdot Df(x+c(y-x))(y-x)\right| \\
		&\le \left|w\right|\cdot\left|Df(x+c(y-x))(y-x)\right| \\
		&\le \left|y-x\right|\cdot\max_{z\in K}\norm{Df(z)}.
	\end{align*}
	Now, $f$ is continuously differentiable, and $K$ is compact, so $Df$ is bounded on $K$. So we say that $f$ is Lipschitz with Lipschitz constant bounded above by $\max_{z\in K}\norm{Df(z)}$.
\end{proof}
Now, the main content of \Cref{thm:change-vars} is in the following result.
\begin{restatable}{lemma}{boundchangevarslem} \label{lem:bound-change-vars}
	Fix a $C^1$-diffeomorphism $\Phi\colon U\to V$ where $U,V\subseteq\RR^d$ is open. Further, let $U'$ be an open subset with compact closure $\ov{U'}\subseteq U$ such that there is $\delta_0>0$ such that any $x\in U'$ has $d(x,U^c)\ge2\delta_0$. Then we have
	\[\mu(\varphi(U'))\le\int_{U'}\left|J_\Phi\right|\,d\mu.\]
\end{restatable}
\begin{remark}
	Here, $d(x,U^c)$ denotes the (infimum of the) distance from $x$ to the set $U^c$. Because $\ov{U'}$ is compact, we note that we also get $d(x,U^c)\ge2\delta_0$ for each $x\in\ov{U'}$ because $x\mapsto d(x,U^c)$ is a continuous function.
\end{remark}
To prove the above lemma, we will want the following lemma.
\begin{lemma} \label{lem:uniform-c1}
	Fix a $C^1$-diffeomorphism $\Phi\colon U\to V$ where $U,V\subseteq\RR^d$ is open. Further, let $U'\subseteq U$ be an open subset with compact closure such that there is $\delta>0$ such that any $x\in U'$ has $d(x,U^c)\ge2\delta_0$. Then there is a decreasing ``remainder'' function $R$ such that $R(t)\to0$ as $t\to0^+$ while the ``Taylor remainder''
	\[\mc R(x,u)\coloneqq\Phi(x+u)-\Phi(x)-D\Phi(x)(u)\]
	satisfies $\left|\mc R(x,u)\right|\le\left|u\right|R(\left|u\right|)$ for all $x\in U$ and $\left|u\right|\le\delta_0$.
\end{lemma}
The content here is that the remainder function $\mc R$ (which ought to go to zero as $u\to0$) is bounded in a way that does not depend on $x$.
\begin{proof}[Proof of \Cref{lem:uniform-c1}]
	For $u$ such that $\left|u\right|\le\delta_0$, the construction of $\delta_0$ implies that any $x\in U'$ has $x+u\in U$; in fact, we have $d(y,U^c)\ge\delta_0$ for all $y$ of the form $x+u$. Notably,
	\[\{y\in U:d(y,U^c)\ge\delta_0\}\]
	is compact: it is closed because $y\mapsto d(y,U^c)$ is continuous, and it is bounded because it is a fixed distance away from the compact set $\ov{U'}$.

	We now use the Mean value theorem to conclude. The proof is similar to \Cref{lem:compact-convex-to-lip}. Fix $x$ and $u$ as in the statement. Then there is a unit vector $w$ such that
	\[w\cdot\mc R(x,u)=\left|\mc R(x,u)\right|.\]
	This allows us to define
	\[g(t)\coloneqq w\cdot\mc R(x,u),\]
	which is now a function $g\colon[0,1]\to\RR$ which is continuously differentiable because everything in sight is continuously differentiable. As such, the Mean value theorem provides $c\in(0,1)$ such that
	\[g(1)-g(0)=g'(c)=w\cdot(D\Phi(x+cu)(u)-D\Phi(x)(u)).\]
	Thus,
	\begin{align*}
		\left|\mc R(x,u)\right| &= \left|g(1)-g(0)\right| \\
		&\le \left|w\right|\cdot\left|D\Phi(x+cu)(u)-D\Phi(x)(u)\right| \\
		&\le \max_{c\in[0,1]}\left|D\Phi(x+cu)(u)-D\Phi(x)(u)\right| \\
		&\le \left|u\right|\cdot\max_{\substack{x'\in\ov{U'}\\\left|u'\right|\le\left|u\right|}}\norm{D\Phi(x'+u')-D\Phi(x')}.
	\end{align*}
	The right-hand factor is our function $R$ in terms of $\left|u\right|$, which is finite because we are taking the maximum of a continuous function on a compact set. Technically, we have not shown that $R$ is decreasing, but we can make it decreasing by replacing $R$ with $t\mapsto\sup\{R(s):s\ge t\}$.
\end{proof}

\end{document}