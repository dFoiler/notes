% !TEX root = ../notes.tex

\documentclass[../notes.tex]{subfiles}

\begin{document}

\section{February 13}
The next homework can be found on pages 596--599. Please do the exercises 2--4, 11--14, and 17--19.

\subsection{Quadratic Forms}
For today, we will focus on symmetric and skew-symmetric bilinear forms. We will also have occasion to consider Hermitian forms, which we go ahead and define.
\begin{definition}[Hermitian]
	Fix a ring $R$ admitting an automorphism $a\mapsto\ov a$ of order $2$. Given a module $E$, a map $g\colon E\times E\to R$ is \textit{Hermitian} if and only if it satisfies the following.
	\begin{itemize}
		\item The function $a\mapsto g(a,b)$ is $R$-linear for any $b\in E$.
		\item Conjugate symmetry: we have $g(a,b)=\ov{g(b,a)}$ for any $a,b\in E$.
	\end{itemize}
\end{definition}
Note that the map $b\mapsto g(a,b)$ is not expected to be $R$-linear due to the conjugate symmetry.

We would like to find good bases for such quadratic forms. For example, when writing out bilinear forms as matrices, it would be nice for these matrices to be block-diagonal with small blocks.
\begin{lemma}
	Fix a module $E$ over a ring $R$. Suppose that $g\colon E\times E\to R$ is symmetric, alternating, or Hermitian. Fix some $a\in E$. Then the following are equivalent for some $a\in E$.
	\begin{listroman}
		\item The map $b\mapsto g(a,b)$ is the zero map.
		\item The map $b\mapsto g(b,a)$ is the zero map.
	\end{listroman}
\end{lemma}
\begin{proof}
	Use the ambient symmetry of $g$.
\end{proof}
This motivates the following definition.
\begin{definition}[kernel]
	Fix a module $E$ over a ring $R$. Suppose that $g\colon E\times E\to R$ is symmetric, alternating, or Hermitian. Then the \textit{kernel} of $g$ is
	\[\ker g\coloneqq\{a\in E:g(a,b)=g(b,a)=0\text{ for all }b\in E\}.\]
	We say that $g$ is \textit{non-degenerate} if and only if $\ker g=0$.
\end{definition}
We remark that a non-degenerate bilinear form gives rise to an isomorphism $E\to E^\lor$ while a non-degen\-erate Hermitian form gives rise to an isomorphism $E\to\ov E^\lor$, where $\ov E^\lor$ denotes the conjugate dual.

We will decompose our quadratic forms into direct sums. The concrete way to do this is via orthogonal decompositions.
\begin{definition}[orthogonal]
	Fix a module $E$ over a ring $R$. Suppose that $g\colon E\times E\to R$ is symmetric, alternating, or Hermitian. Then two $a,b\in E$ are \textit{orthogonal}, written $a\perp b$ if and only if $g(a,b)=0$. Two subspaces $F,G\subseteq E$ are \textit{orthogonal} if and only if $a\perp b$ for all $a\in F$ and $b\in G$. In this last situation, we say that $E$ is an orthogonal direct sum $F\oplus F'$ if and only if we also have $F+F'=E$.
\end{definition}
\begin{definition}[orthogonal basis]
	Fix a free module $E$ of finite rank over a ring $R$. Suppose that $g\colon E\times E\to R$ is symmetric, alternating, or Hermitian. An \textit{orthogonal basis} is a basis of $E$ in which all elements are pairwise orthogonal.
\end{definition}
\begin{remark}
	For the professor, all bases come with an order.
\end{remark}
\begin{remark}
	If $E$ is given an orthogonal basis $\{v_1,\ldots,v_n\}$, then the spans of disjoint subsets produce orthogonal subspaces.
\end{remark}
We are now ready to define a quadratic function.
\begin{definition}[quadratic] \nirindex{quadratic!homogeneous} \nirindex{quadratic!form}
	Fix modules $E$ and $F$ over a ring $R$. A function $f\colon E\to F$ is \textit{quadratic} if and only if it takes the form
	\[f(x)=g(x,x)+h(x)\]
	where $g\colon E\times E\to F$ is $R$-bilinear and $h\colon E\to F$ is $R$-linear. We say further that $f$ is \textit{homogeneous quadratic} if $h=0$. Lastly, a quadratic form is a homogeneous quadratic function with target $R$.
\end{definition}
\begin{remark}
	If $F$ has no $2$-torsion, then the decomposition of $f$ into $g+h$ is unique. Indeed, one can compute that
	\[2g(x,y)=f(x+y)-f(x)-f(y),\]
	from which we can then read off $h$ as $f-g$.
\end{remark}
\begin{remark}
	Suppose further that the multiplication-by-$2$ map $2\colon F\to F$ is an isomorphism. Then for any function $f\colon E\to F$, we find that $2f(x)-\frac12f(2x)$ is $\ZZ$-bilinear; if further $f(2x)=4f(x)$, then $f$ is homogeneous quadratic. This can be checked by a direct computation.
\end{remark}
Let's begin doing some decompositions.
\begin{proposition}
	Fix a field $k$ of characteristic not $2$. If $g$ is a symmetric bilinear form on a finite-dimensional vector space $E$ over $k$, then $E$ admits an orthogonal basis.
\end{proposition}
\begin{proof}
	If $g$ is zero, there is nothing to do. Then one can construct the basis inductively by choosing some $v\notin\ker g$ to start and then inductively passing to the orthogonal complement. For example, it is possible to do this using the Gram--Schmidt process. A careful proof using the Gram--Schmidt process must deal with $\ker g$. In brief, one can decompose $E$ into a direct sum of $\ker g$ with a subspace $E'\subseteq E$ on which $g$ is non-degenerate.
\end{proof}
Over ordered fields, one can add some positivity.
\begin{definition}[ordered field]
	An \textit{ordered field} $k$ is a field $k$ together with a collection of positive elements $P\subseteq k$ satisfying the following.
	\begin{itemize}
		\item For each $x\in k$, exactly one of $x\in P$, $x=0$, or $-x\in P$ is true.
		\item For each $x,y\in P$, one has $x+y\in P$ and $xy\in P$.
	\end{itemize}
\end{definition}
\begin{theorem}[Sylvester] \label{thm:sylvester}
	Fix an ordered field $k$, and let $g$ be a non-degenerate symmetric bilinear form on a vector space $E$ over $k$. Then there is $r\ge0$ such that any orthogonal basis $S$ of $E$ has
	\[r=\#\{v\in S:g(v,v)>0\}.\]
\end{theorem}
\begin{proof}
	Choose two orthogonal bases $\{v_1,\ldots,v_n\}$ and $\{w_1,\ldots,w_n\}$, and set $a_i\coloneqq g(v_i,v_i)$ and $b_i\coloneqq g(w_i,w_i)$ for each $i$. By reordering, we may assume that
	\[\{i:a_i>0\}=\{1,2,\ldots,r\}\qquad\text{and}\qquad\{i:b_i>0\}=\{1,2,\ldots,s\}\]
	for some $r,s\ge0$. We would like to show that $r=s$. By symmetry, it's enough to check that $r\le s$, which is equivalent to checking that $r+(n-s)\le n$, where $n=\dim E$.
	
	For this, we check that the vectors
	\[\{v_1,\ldots,v_r\}\sqcup\{w_{s+1},\ldots,w_n\}\]
	are linearly independent. Well, suppose that we have some linear relation
	\[\sum_{i=1}^rx_iv_i=\sum_{j=s+1}^ny_jw_j.\]
	Taking the norm on both sides, we see that
	\[\sum_{i=1}^rx_i^2a_i\ge0\ge\sum_{j=s+1}^ny_j^2b_j,\]
	with equality in the middle holding if and only if $x_i=0$ and $y_j=0$ for all $i$ and $j$.
\end{proof}
\begin{corollary}
	Fix an ordered field $k$ such that the positive elements are squares. For any non-degen\-erate symmetric bilinear form $g$ on a vector space $E$ over $k$, there is an orthonormal basis $\{v_1,\ldots,v_n\}$ with some $r\ge0$ such that
	\[\begin{cases}
		g(v_i,v_i)=+1 & \text{if }i\le r, \\
		g(v_i,v_i)=-1 & \text{if }i>r.
	\end{cases}\]
\end{corollary}
\begin{proof}
	Simply rescale the given basis by some squares.
\end{proof}
These sorts of results motivate us to define some nice bases.
\begin{definition}[positive-definite]
	Fix an ordered field $k$ and a non-degenreate symmetric bilinear form $g$ on a vector space $E$ over $k$.
	\begin{itemize}
		\item Then $g$ is \textit{positive-definite} if and only if $g(v,v)>0$ for all $v\in E$.
		\item Then $g$ is \textit{negative-definite} if and only if $g(v,v)<0$ for all $v\in E$.
	\end{itemize}
\end{definition}
\begin{remark}
	\Cref{thm:sylvester} allows us to decompose any $E$ into an orthogonal direct sum $E_+$ and $E_-$ so that $g$ restricted to $E_+$ or $E_-$ is positive-definite or negative-definite, respectively.
\end{remark}

\end{document}