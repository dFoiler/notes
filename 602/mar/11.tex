% !TEX root = ../notes.tex

\documentclass[../notes.tex]{subfiles}

\begin{document}

\section{March 4}
I missed last class because I was lazy. The homework from last week is due on the 21st of March. The following homework is due on the 28th of March: problems 13(iii--iv) on page 638 and problems 1 and 2 on page 661.

\subsection{Basic Properties of Tensor Products}
Given an $R$-module $F$, there is a tensor functor $F\otimes-$ defined in the obvious way on modules by $E\mapsto F\otimes_RE$, and sends a morphism $\varphi\colon E'\to E$ to the morphism $\varphi\colon F\otimes_RE'\to F\otimes_RE$ given on pure tensors by $f\otimes e'\mapsto f\otimes\varphi(e')$. In fact, $F\otimes-$ is an $R$-linear functor because the induced map
\[\op{Hom}_R(E',E)\to\op{Hom}_R(F\otimes_RE',F\otimes_RE)\]
on morphisms is $R$-lienar. We will not bother running any of these required functoriality checks because they are mostly a matter of writing out the definitions.
\begin{lemma} \label{lem:tensor-curry}
	Given $R$-modules $E$, $F$, and $G$, there is a natural isomorphism
	\[\op{Hom}_R(E,\op{Hom}_R(F,G))\simeq L^2(E,F;G)\simeq\op{Hom}_R(E\otimes_RF,G).\]
\end{lemma}
\begin{proof}
	The last isomorphism is given by the universal property of $\otimes$. The first isomorphism is given by currying: simply send $\varphi_\bullet\colon E\to\op{Hom}_R(F,G)$ to the $R$-bilinear map $E\times F\to G$ given by $(e,f)\mapsto\varphi_e(f)$. Conversely, given an $R$-bilinear map $\psi\colon E\times F\to G$, one can define $\varphi_\bullet\colon E\to\op{Hom}_R(F,G)$ by $\varphi_e(f)\coloneqq\psi(e,f)$. It is true (but we will not check) that these maps are well-defined, $R$-linear, define inverse isomorphisms, and are natural.
\end{proof}
Taking tensor products commutes with taking direct limits. Here is an example.
\begin{lemma} \label{lem:tensor-sum}
	Given $R$-modules $\{E_\alpha\}_{\alpha\in\lambda}$ and $F$, there is a natural isomorphism
	\[\Bigg(\bigoplus_{\alpha\in\lambda}E_\alpha\Bigg)\otimes_R F=\bigoplus_{\alpha\in\lambda}(E_\alpha\otimes_R F).\]
\end{lemma}
\begin{proof}
	The forward map is defined on pure tensors by
	\[\Bigg(\sum_{\alpha\in\lambda} e_\alpha\Bigg)\otimes f\mapsto\sum_{\alpha\in\lambda}(e_\alpha\otimes f),\]
	and the inverse map is defined analogously on pure tensors.
\end{proof}
Here is another computation.
\begin{lemma} \label{lem:tensor-r}
	Given an $R$-module $F$, there is a natural isomorphism $R\otimes_RF\simeq F$.
\end{lemma}
\begin{proof}
	The forward map is defined on pure tensors by $r\otimes f\mapsto rf$, and the backward map is defined on pure tensors by $f\mapsto 1\otimes f$.
\end{proof}
\begin{remark}
	Here is another version of this statement: let $E$ be free of rank $1$ with basis $e$. Then any element of $F\otimes_RE$ can be written uniquely as a pure tensor of the form $f\otimes e$. Indeed, this amounts to computing the composite isomorphism $E\cong R\cong F\otimes_RR\cong F\otimes_RE$.
\end{remark}
\begin{example} \label{ex:tensor-free}
	Combining \Cref{lem:tensor-sum,lem:tensor-r}, we see that $R^n\otimes_RF\cong F^n$ for any $n\ge0$. For example,
	\[R^n\otimes_RR^m\cong\left(R^m\right)^n\cong R^{mn}.\]
	In other words, for free $R$-modules $E$ and $F$ of finite rank, $\op{rank}_R(E\otimes_RF)=(\op{rank}_RE)(\op{rank}_RF)$.
\end{example}
\begin{remark}
	Many of these lemmas (in particular, \Cref{lem:tensor-sum,lem:tensor-r}) can be checked directly by using the universal property appropriately.
\end{remark}

\subsection{Duality}
We now see that free modules will have especially nice structure, analogously to what we know when the base ring $R$ is a field. For example, there is a nice duality theory.
\begin{definition}[dual]
	Fix an $R$-module $F$. Then the \textit{dual} is $F^\lor\coloneqq\op{Hom}_R(F,R)$.
\end{definition}
\begin{remark}
	If $F$ is a free $R$-module of finite rank, then $F^\lor$ is as well. Indeed, one simply takes a dual basis.
\end{remark}
\begin{lemma}
	For any $R$-module $F$, there is a natural morphism $F\to F^{\lor\lor}$. If $F$ is free of finite rank, then this is an isomorphism.
\end{lemma}
\begin{proof}
	The first assertion is proved by construction: send $f\in F$ to the element $\op{ev}_f\colon F^\lor\to R$ given by $\op{ev}_f(\ell)\coloneqq\ell(f)$. The second assertion follows by an explicit computation with dual bases.
\end{proof}
\begin{lemma}
	For any $R$-modules $E$ and $F$, there is a natural morphism
	\[E^\lor\otimes F\to\op{Hom}_R(E,F).\]
	If $E$ is free of finite rank, then this is an isomorphism.
\end{lemma}
\begin{proof}
	The morphism is constructed on pure tensors by sending $\ell\otimes f$ to the map $E\to F$ defined by $e\mapsto\ell(e)f$. The last sentence holds because an isomorphism $E\cong R^n$ produces isomorphisms $E^\lor\otimes F\cong F^n$ and $\op{Hom}_R(E,F)\cong F^n$.
\end{proof}
\begin{lemma}
	For any $R$-modules $E$ and $F$, there is a natural morphism
	\[E^\lor\otimes F^\lor\to(E\otimes F)^\lor.\]
	If $E$ and $F$ are free of finite rank, then this is an isomorphism.
\end{lemma}
\begin{proof}
	The morphism is constructed on pure tensors by sending $e^\lor\otimes f^\lor$ to the map $E\otimes F\to R$ defined on pure tensors by $(e\otimes f)\mapsto e^\lor(e)\otimes f^\lor(f)$. As usual, the second assertion follows by a computation on the bases.
\end{proof}
\begin{example}
	If $E$ and $F$ are free of finite rank, then we see that
	\[\op{End}_R(E)\otimes_R\op{End}_R(F)\cong E\otimes E^\lor\otimes F\otimes F^\lor\cong\op{End}_R(E\otimes F)\]
	as $R$-modules. In fact, one can check that the composite also respects the ambient ring structure.
\end{example}

\subsection{Reduction and Base Change}
Here is another incarnation of commuting with direct limits.
\begin{lemma}
	Given a right-exact sequence
	\[E'\to E\to E''\to0\]
	of $R$-modules, the sequence
	\[F\otimes_RE'\to F\otimes_RE\to F\otimes_RE''\to0\]
	is still right-exact for any $R$-module $F$.
\end{lemma}
\begin{proof}
	One should first show that the functor $\op{Hom}_R$ has its own exactness properties, and then one can use \Cref{lem:tensor-curry} to complete the proof. In particular, this result follows by formal properties of adjoints.
\end{proof}
\begin{example}[reduction] \label{ex:tensor-reduce}
	By applying $E\otimes-$ to the right-exact sequence $\mf a\to R\to R/\mf a\to0$, we see that there is a natural isomorphism
	\[R/\mf a\otimes_RE\cong E/\mf aE\]
	given by $(r+\mf a)\otimes e\mapsto (re+\mf aE)$.
\end{example}
It is not always true that the functor $F\otimes_R-$ preserves left-exactness. This motivates the following definition.
\begin{definition}[flat]
	An $R$-module $F$ is flat if and only if any injection $E'\into E$ induces an injection $F\otimes_RE'\into F\otimes_RE$.
\end{definition}
\begin{example}
	Any free $R$-module is flat by \Cref{ex:tensor-free}.
\end{example}
\begin{nex}
	The $\ZZ$-module $\ZZ/2\ZZ$ is not flat. For example, there is an injection $\ZZ\into\ZZ$ given by multiplication by $2$, but this becomes the zero map upon tensoring with $\ZZ/2\ZZ$.
\end{nex}
Thus far we have been tensoring by $R$-modules to produce $R$-modules. However, more should be possible. For example, given a vector space $V$ over $\RR$, we know that we can think of $V\otimes_\RR\CC$ as a vector space over $\CC$! The following definition explains this construction.
\begin{definition}[base change]
	Fix a ring homomorphism $R\to R'$. For any $R$-module $M$, one can view $R'\otimes_RM$ as an $R'$-module, where the action by $R'$ is defined on pure tensors by
	\[r'(s'\otimes m)\coloneqq(r's'\otimes m).\]
\end{definition}
\begin{example}
	Given an ideal $\mf a\subseteq R$, we may use the ring homomorphism $R\onto R/\mf a$ to turn any $R$-module $E$ into the $(R/\mf a)$-module $R/\mf a\otimes_RE$. We recall from \Cref{ex:tensor-reduce} that $R/\mf a\otimes_RE$ is $E/\mf aE$.
\end{example}
\begin{example}
	For any abelian group $M$, we see that $M\otimes_\ZZ\FF_p=M/pM$, which is a vector space over $\FF_p$.
\end{example}

\end{document}