% !TEX root = ../notes.tex

\documentclass[../notes.tex]{subfiles}

\begin{document}

\section{April 17}
I missed last class, so this will be rough.

\subsection{Matching on the Geometric Side}
We have been interested in two functionals $\mathrm{RTF}^T$ and $\mathrm{RTF}^A$ of $S(G(\AA_F))$, which are invariant $T\times T$ and $A\times(A,\eta)$-invariant respectively. We have two expansions for these functionals, one geometric and one automorphic. The automorphic one is already somewhat understood. The geometric side is given by some sum
\[\sum_{\xi\in T(F)\backslash G(F)/T(F)}\op{Vol}([T_\xi])\OO_\xi\]
of orbital integrals.

Last class, we discussed that it is easier to sum over the GIT quotient $T\backslash G//T=\Spec F[G]^{T\times T}$, which is canonically isomorphic to $A\backslash G//A$, which are both birational with $\AA^1$ via the map $\begin{bsmallmatrix}
	a & b \\ c & d
\end{bsmallmatrix}\mapsto ad/(ad-bc)$. In particular, for open $U\subseteq\AA^1$, say avoiding $0$ and $1$, we find that $A\backslash G|_U$ is isomorphic to $A\times U$, which tells us that the contribution to the relative trace formula for $\xi\notin\{0,1\}$ are given by the orbital integrals
\[\OO_\xi^A(\varphi_A)=\int_{A(\AA_F)}\varphi_A(\xi a)\eta(a)\,da\qquad\text{and}\qquad\OO_\xi^T(\varphi_T)=\int_{T(\AA_F)}\varphi_T(\xi t)\,dt,\]
where $\varphi_A\in S(A\backslash G)$ and $\varphi_T\in S(T\backslash G)$. These now split into some local integrals, so we now find that we have local orbital integrals which we call $\pi_{A!}\varphi_A(\xi)\in C^\infty(U)$ and $\pi_{T!}\varphi_T(\xi)\in C^\infty(U)$. We are now ready to state a more precise form of \Cref{thm:jacquet-compare}.
\begin{theorem} \label{thm:better-compare}
	Fix everything as above.
	\begin{listalph}
		\item Matching: for each $f$, there is $f'$ such that $\pi_{T!}(f)=\pi_{A!}(f')$.
		\item Fundamental lemma for Hecke algebra: if $v$ is nonarchimedean and $K_v/F_v$ is unramified, and if $f\in\mc H(G(F_v),G(\OO_v))$, then we may take $f'=f$.
	\end{listalph}
\end{theorem}
\begin{remark}
	If $K_v/F_v$ is split, there is nothing to do because $A$ and $T$ both refer to the split torus.
\end{remark}
This matching of terms on the geometric side of the relative trace formula is something of a miracle. Surely it is expected because Waldspurger's theorem produces a matching of the relative trace formulae on the automorphic side. Our identification away from $\xi\in\{0,1\}$ above tells us that the content of the above theorem is producing the matching for $\xi\in\{0,1\}$. It remains to understand limits approaching the ``singularities.'' By applying an automorphism of $G$, we may as well focus on $\xi=1$.

For $X=A\backslash G$, we find that $\PP^1\times\PP^1\setminus\Delta\PP^1\cong X$ by $(x,y)\mapsto\frac x{x-y}$. We are thus interested in the behavior around the singularity $(\infty,0)$. To visualize, we linearize $X$: there is a birational $A$-equivariant map $\AA^2\to X$ given by $(x,y)\mapsto\left(x^{-1},y\right)$, where $A$ acts on $\AA^2$ by $a\cdot(x,y)=\left(a^{-1}x,ay\right)$, and importantly, this map is an isomorphism in a neighborhood of $(0,0)\mapsto(\infty,0)$. (Note that these orbits are hyperbolas now.) This linearization then tells us how to handle the matching around $\xi=1$.
\begin{remark}
	This linearization is not a miracle: its existence is a consequence of Luna's \'etale slice theorem.
\end{remark}
Let's now do our matching. Choose $f\in S^2\left(\AA^2\right)$. Then we produce a function $\pi_{A!}f(\zeta)$, where $\zeta\in\AA^2$ is our coordinate, and then the Mellin transform
\[(\mc M\pi_{A!}f)(\chi)=\int_{F^\times}\pi_{A!}f(\zeta)\chi^{-1}(\zeta)\,d^\times\zeta,\]
which can be thought of as a double Tate integrals, but now its simple poles live at $\chi\in\{1,\eta\}$. By undoing the Mellin transform, we find that $\pi_{A!}f(\zeta)=c_1(\zeta)+\eta(\zeta) c_2(\zeta)$, where $c_1,c_2\in C^\infty(U)$ are some smooth functions. Going all the way back to our coordinate $\xi\in X$, we see that $\pi_{A!}f(\xi)=c_1(\xi)+\eta(\xi-1)c_2(\xi)$ in a neighborhood of $\xi=1$ (where we technically changed the smooth function silently).

We now turn to the non-split torus $T$. Well, note that $T$ is basically given by $A$ after applying a Galois twist in $\mathrm H^1(\op{Gal}(\ov F/F),\op{Aut}(G,A))$. Note that this Galois group can be seen to be $\mathrm H^1(\op{Gal}(K/F),\ZZ/2\ZZ)$, where the canonical generator of $\op{Aut}(G,A)$ is given by conjugation by the nontrivial Weyl element. Thus, we can apply this automorphism everywhere in the previous discussion. For example, we still admit a linearization $\AA^2\to T\backslash G$, but now the $T$-action on $\AA^2$ must be twisted: it is the natural action of $T$ on $V\coloneqq\op{Res}_{K/F}\AA^1$. (For example, the orbits can be seen to be circles in $\AA^2$ because the quotient map $V\onto V//T=\AA^1$ is simply the norm. Note we already saw this norm in the split case when dealing with $A\backslash G$.)
\begin{remark}
	The quotient $V\onto V//T$ is now no longer surjective on $F$-points: indeed, the norm map $\op N_{K/F}\colon K^\times\to F^\times$ is not surjective! Nonetheless, our fibers are always a trivializable $T$-torsor, which is already trivial when $\zeta\in F^\times$ is in the image of the norm map and trivializing over an extension otherwise. The fiber over $0$ has a single point over $F$, but there are two lines over $\ov F$ which are swapped by the Galois action.
\end{remark}
We now claim that the collection of functions $\pi_{T!}(S(F^2))$ are just given by restrictions of smooth functions $S(F)$ to the image of $\op N_{K/F}(K)$. Surely we are looking at functions on $\op N_{K/F}(K)$, but to see that they should be smooth, we note that the image of $\pi_{T!}(f)$ is the same as $\pi_{T!}(\widetilde f)$ where $\widetilde f$ is the average of $f$ over the $T$-orbit. Because $T$ is compact, we know $\widetilde f$ continues to be smooth! So $\pi_{T!}(\widetilde f)=\widetilde f|_{\op N_{K/F}(K)}$.

Let's now think about what our matching is asking for. After linearization, for each $f\in S(F^2)$, we would like to find $f'\in S(F^2)$ with $\pi_{T!}(f)=\pi_{A!}(f')$. Now, the direction of the matching is important. Note that $\pi_{T!}(f)$ looks like
\[c1_{\op N_{K/F}(K^\times)}=\frac c2+\eta\frac c2\]
for some smooth function $c$. This has the form of the functions $\pi_{A!}(f')$ we computed previously, so matching is confirmed!
\begin{remark}
	To check the fundamental lemma claim, one can directly compute that this formula sends $1_{\OO_F^2}$ to $1_{\OO_F^2}$. This is a basic case of the Hecke algebra check we wanted.
\end{remark}
Let's review our intended application for a moment. This comparison of relative trace formulae is able to show that if $\pi$-component of the relative trace formula for $T$ is nonzero, then the same is true for the $\pi$-component of the relative trace formula for $A$. This implies a weak version of Waldspurger's formula because it provides non-vanishing of our $L$-functions by checking what the $\pi$-component of $\op{RTF}^A$ is.

If we wanted to go the other way, we need to be able to do more matching. We even knew that we would have problems much earlier because we have been working with $\op{PGL}_2$ instead of with division algebras. The moral is that we should look at $G'=PD^\times$ where $D$ is the division algebra over $F$ we described long ago. Then it turns out that there is a canonical identification with $T\backslash G//T$ and $T\backslash G'//T$ with $\AA^1$ so that we have
\[\AA^1(F)=\im(T(F)\backslash G(F))\cup\im(T(F)\backslash G'(F))\]
away from $0$ and $1$. In fact, one can check that the fibers over the $T(F)$-points on the left factor or the right factor, so we get the extra degree of freedom we lost from earlier! Keeping track of everything, one can check that this does in fact let us recover the full matching in the opposite direction. This sort of thing is required for a full proof of Waldspurger's formula instead of our mere non-vanishing corollary.

\end{document}