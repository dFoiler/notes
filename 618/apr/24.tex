% !TEX root = ../notes.tex

\documentclass[../notes.tex]{subfiles}

\begin{document}

\section{April 24}
Today we complete our discussion of Waldspurger's formula.

\subsection{Completing Waldspurger's Formula}
Let's recall where we are right now. Thus far, we noticed a matching of $L$-functions, which motivated us to compare trace formulae. This led us to a matching of orbital integrals, which can be stated as an isomorphism
\[\bigoplus_RS\left(T\backslash G^R(\AA_F)\right)_{T(\AA_F)}\cong S(A\backslash G(\AA_F))_{A(\AA_F),\eta},\]
where the subscripts denote co-invariants, such that the orbital integrals matched on both sides, and there was some alignment with the Hecke algebras. (It turns out that taking co-invariants is the same as having an isomorphism of the push-forwards by $\pi_{\bullet}$.)

We then began our spectral decomposition, which amounts to choosing some good test functions to read off particular automorphic pieces from the relative trace formula. To be precise, we fixed a finite but large set of places $\Sigma$ (e.g., containing the nonarchimedean places), and we considered the relative trace formulae as functionals on the restricted Hecke algebra
\[\mathcal H^\Sigma=\bigotimes_{v\notin\Sigma}'\mc H(G(F_v),G(\OO_v))=\bigotimes_{v\notin\Sigma}'\CC[T_v],\]
where the target is given by taking eigenvalues (also called Satake parameters). In particular, after choosing some matched test functions $f_\Sigma$ and $f'_\Sigma$, we produced equalities
\begin{align*}
	\op{RTF}^A_\Sigma(h) &= \int_{\widehat G} \widehat h(\pi)J_\pi^A(1_{G(\OO^\Sigma)}\otimes f_\Sigma')\,d\pi, \\
	\op{RTF}^T_{\Sigma}(h) &= \sum_R\int_{\widehat{G^R}}\widehat h(\pi^R)J^A_{\pi^R}(1_{G(\OO^\Sigma)}\otimes f_\Sigma)\,d\pi^R.
\end{align*}
After some regularization, these expressions are absolutely convergent. We would like to think of the expressions $J_\pi^A(\cdot)\,d\pi$ as a measure, for which we are integrating some polynomial $\widehat h$ against. We would like to conclude that these measures $J_\pi^A(1_{G(\OO^\Sigma)}\otimes f_\Sigma')\,d\pi$ and $\sum_RJ_{\pi^R}^A(1_{G(\OO^\Sigma)}\otimes f_\Sigma)\,d\pi^R$ are the same once we know that they are the same on polynomials on $\prod_{v\notin\Sigma}\CC_v$. Our tool will be the Stone--Weierstrass theorem.
\begin{theorem}[Stone--Weierstrass] \label{thm:stone-weier}
	Fix a compact Hausdorff space $X$. Then a subalgebra $\mc A\subseteq C(X)$ is dense if it satisfies the following.
	\begin{listalph}
		\item $\mc A$ separates points: for any two $x,y\in X$, there is $f\in\mc A$ with $f(x)\ne f(y)$.
		\item $\mc A$ is closed under conjugation.
	\end{listalph}
\end{theorem}
Of course, our space $\prod_{v\notin\Sigma}\CC_v$ is pretty far from compact, but we will be able to force compactness into our application. In particular, the Satake parameters $\widehat h(\pi^\Sigma)$ lie in the subset of unitary irreducible automorphic representations (because these came from an $L^2$ decomposition), where unitary means that $\pi^\lor\cong\ov\pi$. It turns out that this subset $U\subseteq\prod_{v\notin\Sigma}\CC_v$ of unitary representations is compact, so we are now okay. We can see that $U$ is compact for $\mathrm{PGL}_2$ explicitly: locally over some $F_v$ (for $p$-adic $v$), we see that $\pi_v$ should look like $\pi_v=\op{Ind}_B^G\left|\cdot\right|^s$, and then our space of Satake parameters looks like $q_v^s$ up to inversion (because of the unitary condition) and thus is $q_v^s+q_v^{-s}$. However, the unitary representations are exactly given by $\Re s=0$ (the principal series) and $s\in[-1,1]$ (the complementary series), which then makes our space of Satake parameters $q_v^s+q_v^{-s}$ a compact space!

We may thus apply \Cref{thm:stone-weier} restricted to $U$. Polynomials always separate points, so it only remains to check that our polynomials are closed under conjugation. Well, for $h\in\mc H^\Sigma$, we note that one can define a dual $h^\lor(g)\coloneqq h\left(g^{-1}\right)$ which has
\[\widehat{h^\lor}(\pi^\lor)=\widehat h(\pi)\]
because $h$ and $h^\lor$ are transpose operators. Because $\pi$ is unitary, we thus find that
\[\widehat{h^\lor}(\pi^\lor)=\widehat{h^\lor}(\ov\pi)=\overline{\widehat{\ov{h^\lor}}(\pi)},\]
from which we find that $h^\lor$ and $h$ are complex conjugates. This completes the argument for the spectral decomposition.

\subsection{A Little Gross--Zagier Formula}
Let's quickly say something about the Gross--Zagier formula. We continue with $T\into G$ and $T\into G^R$, and we see that this produces embeddings of Shimura varieties $S_T\into S_G$ and $S_T\into S_{G^R}$. Roughly speaking, given an automorphic representation $\pi$ of $G$ (corresponding to a modular form of weight $2$), we would like to relate the $\pi$-part of the N\'eron--Tate height of $\langle S_T,S_T\rangle_{\mathrm{NT}}$ with the derivative of an $L$-function.

Motivated by the above discussion on relative trace formula, we define a relative trace formula with a parameter $s$. Thus, we are interested in comparing
\[\frac d{ds}\bigg|_{s=0}\mathrm{RTF}^A_s(f')\]
and
\[H(f)\coloneqq\sum_R\langle S_T|f|S_T\rangle_{\mathrm{NT}},\]
where $f$ is some test Hecke operator in $\bigoplus_RS\left(T\backslash G^R(\AA_F^\infty)\right)$.

On the geometric side, our relative trace formula will look like
\[\sum_\xi\frac d{ds}\bigg|_{s=0}\prod_v\OO_{\xi_v}^{A,s}=\sum_\xi\sum_v\prod_{w\ne v}\OO_{\xi_w}(\OO^A_{\xi_v})',\]
where the prime now denotes a derivative at $s=0$. This is now a kind of ``Euler sum,'' so we would like a similar manifestation for our heights. In particular, there is a theory of local heights. Let's describe this: the N\'eron--Tate height is a pairing $J(F)\times J(F)\to\CC$, where $J$ is the Jacobian of our modular curve, and it turns out that this height pairing has a manifestation in Arakelov intersection theory, where we find
\[\langle D_1,D_2\rangle_{\mathrm{NT}}=\langle\widehat D_1,\widehat D_2\rangle_{\mathrm{Arakelov}}=\sum_v\langle\widehat D_{1,v},\widehat D_{2,v}\rangle_v.\]
The moral is that our $H(f)$ can be decomposed into a sum of local heights. Of course, we are still looking at a local height pairing of some global objects, and we still need to express these local heights as Euler products!

The next step is to do a $p$-adic uniformization of our formal neighborhoods of points on the Shimura varieties. In particular, we find that this formal neighborhood looks like
\[G(F)\backslash\mc N_v\times G(\AA_f^v)/U^v,\]
where $U^v$ is a level, and $\mc N_v$ is a Rappaport--Zink space or a local Shimura variety (which is some kind of generalized Lubin--Tate space). Thus, the local height pairing can be found to expand as
\[H_v(f)=\sum_\xi\prod_{w\ne v}\OO_{\xi_w}\langle\mc N^T_v|F_v|\mc N^T_v\rangle.\]
Here, the orbital integrals $\OO_{\xi_w}$ come above from the $G(\AA_f^v)$ term, and the remaining term is some local height pairing of some special cycles on a local Shimura variety. There is now an arithmetic fundamental lemma which explains that the height pairing term matches up with the derivative.

\end{document}