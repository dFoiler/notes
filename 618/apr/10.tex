% !TEX root = ../notes.tex

\documentclass[../notes.tex]{subfiles}

\begin{document}

\section{April 10}
We continue.

\subsection{The Relative Trace Formula}
Recall our standing assumption that $\chi$ is trivial, which means that the central character of $\pi$ is trivial, which means that $\pi$ descends to an automorphic form of $G=\op{PGL}_2$. We would now like to say something about relative trace formulae. For $f\in S(G(\AA_F))$, we recall that $f\cdot\delta_T$ is the distribution
\[(f\cdot\delta_T)(x)=\int_{G(\AA_F)}\delta_T(xg)f(g)\,dg,\]
where $x\in G(F)\backslash G(\AA_F)$. Ultimately, we are interested in computing $\langle f\cdot\delta_T,\delta_T\rangle$ for Waldspurger's formula, so let's say a bit about how one can manipulate this object. For general $\varphi$, we note
\begin{align*}
	\langle f\cdot\delta_T,\varphi\rangle &= \int_{G(F)\backslash G(\AA_F)}\varphi(x)(f\cdot\delta_T)(x)\,dx \\
	&= \int_{G(F)\backslash G(\AA_F)}\int_{G(\AA_F)}\delta_T(xg)f(g)\varphi(x)\,dg\,dx \\
	&= \int_{G(F)\backslash G(\AA_F)}\delta_T(y)\int_{G(\AA_F)}f(g)\varphi\left(yg^{-1}\right)\,dg\,dy,
\end{align*}
where we have applied the variable substitution $y=xg$ at the end. We are now integrating over the distribution $\delta_T(y)$, so we can write this integral as
\[\langle f\cdot\delta_T,\varphi\rangle=\int_{[T]}\int_{G(\AA_F)}f\left(g^{-1}\right)\varphi(yg)\,dg\,dy.\]
This rearranges into
\[\int_{G(\AA_F)}f\left(g^{-1}\right)\int_{[T]}\varphi(yg)\,dy\,dg=\int_{T(\AA_F)\backslash G(\AA_F)}\underbrace{\int_{T(\AA_F)}f\left(g^{-1}t\right)\,dt}_{\Phi^\lor(g)\coloneqq}\int_{[T]}\varphi(yg)\,dy\,dg.\]
Combining the integrals, we are left with
\[\int_{T(F)\backslash G(\AA_F)}\Phi^\lor(g)\varphi(g)\,dg\]
by keeping track of the ambient invariance of everything in sight. (Namely, we see that $\varphi$ is being integrated over $T(F)\backslash T(\AA_F)$, so we may as well combine the integrals suitable.) Now, $\varphi$ is further $G(F)$-invariant (it is a function on $G(F)\backslash G(\AA_F)$), so we can rewrite the integral as
\[\int_{G(F)\backslash G(\AA_F)}\varphi(g)\Bigg(\sum_{\gamma\in T(F)\backslash G(F)}\Phi^\lor(\gamma g)\Bigg)\,dg.\]
This is now basically a pairing of some sum with $\varphi$, so the moral is that $f\cdot\delta_T$ is basically equal to this ``theta'' $\theta(f_T^\lor)$ series of $\Phi^\lor$. Thus, we are able to write
\[\langle f\cdot\delta_T,\delta_T\rangle=\int_{[T]}\theta\left(f_T^\lor\right)(t)\,dt.\]
One can check that there are no convergence issues. (Namely, $\delta_T$ has finite support given by the CM points, so $f\cdot\delta_T$ has compact support and is of rapid decay, so the pairing has rapid decay as well.) Additionally, we remark that a direct computation can check that this expression is left- and right-invariant by $T(\AA_F)$. This is giving us two ways to think about the pairing $\langle f\cdot\delta_T,\delta_T\rangle$.
\begin{itemize}
	\item Writing $\langle f\cdot\delta_T,\delta_T\rangle=\langle\delta_T,f^\lor\cdot\delta_T\rangle$ lets us see this as a functional on $S(G(\AA_F))$ which is left- and right-invariant by $T(\AA_F)$. So this is a distribution on $T\backslash G/T$.
	\item Writing this as $\int_{[T]}\theta(f_T^\lor)(t)\,dt$ lets us see this as a $T(\AA_F)$-invariant functional on $S(T\backslash G)$. So this is a distribution on $X/T$ where $X=T\backslash G$.
	\item If we can write $f=f_1^\lor\cdot f_2$, then we find that
	\[\langle f\cdot\delta_T,\delta_T\rangle=\langle f_2\cdot\delta_T,f_1\cdot\delta_T\rangle=\left\langle\theta(f_{2,T}^\lor),\theta(f_{1,T}^\lor)\right\rangle.\]
	This last expression is invariant under the action of the diagonal embedding by $G$ into $S(X)\otimes S(X)$, where $X=T\backslash G$. So this is a distribution on $(X\times X)/G$, which is again basically the same thing.
\end{itemize}
Viewing $\theta$ as a ``Poincar\'e series,'' we note the second is writing a Fourier expansion of $\theta$, and the third is an inner product of Poincar\'e series.

We are now able to say something about the relative trace formula. Morally, it is an equality between a spectral and geometric expansion of the functional $f\cdot\langle f\cdot\delta_T,\delta_T\rangle$. We can even now say something about what this spectral expansion is: it follows by doing a spectral decomposition of functions in $L^2([G])$. Let's explain how this works for $G=\op{PGL}_2$. Here, $L^2([G])$ decomposes as
\[\bigoplus_{\pi\text{ cuspidal}}\pi\oplus\bigoplus_{\dim\chi=1}\CC_\chi\oplus\int I_B^G(\chi)\,d\chi.\]
Let's explain these terms. The first sum is over the ``discrete series'' cuspidal representations of $G$. The second sum is over the one-dimensional characters. Lastly, the last integral refers to the ``continuous spectrum''; the integral means that we are taking a Hilbert space direct sum. Note that we already see a continuous spectrum on $L^2(\RR)$ because this decomposes into a Hilbert space direct sum of the characters. Note the integral is taken over unitary id\'ele class characters on the central torus $A\cong\mathbb G_m$, which we think about as a quotient of the Borel subgroup by its unipotent radical.
\begin{remark}
	The reason we write the decomposition in the above form is that it explains that any $\varphi_1,\varphi_2\in L^2([G])$ will have inner product $\langle\varphi_1,\varphi_2\rangle$ equal to a sum over the cuspidal and character pieces plus the value of an integral over the continuous spectrum. 
\end{remark}
Let's say something about the characters of $G$. Note that there are no nontrivial algebraic characters: they would have to be powers of $\det$, but only the trivial character is trivial on the diagonal copy of $\mathbb G_m\subseteq G$. On the other hand, there are many characters of $G(\AA_F)$: note that one can produce lots of characters on $[\mathrm{GL}_2]$ by taking determinant to $\AA_F^\times$ and then applying an id\'ele class character $\chi$ to $\CC^\times$. One can check that this factors through $[G]$ as soon as $\chi$ is quadratic, so our characters are basically quadratic id\'ele class characters.

We are now ready to give the spectral side of the relative trace formula. Put simply, we are able to write
\[\langle\theta(f_{2,T}^\lor),\theta(f_{1,T}^\lor)\rangle=\int_{\widehat G}\langle\theta(f_{2,T}^\lor),\theta(f_{1,T}^\lor)\rangle_\pi\,d\pi,\]
where these inner products are
\[\sum_{\varphi\in\op{ON}(\pi)}\langle\theta(f_{2,T}^\lor),\varphi\rangle\langle\ov\varphi,\theta(f_{1,T}^\lor)\rangle,\]
where the sum is taken over an orthonormal basis of $\pi$. Using what we already know about these pairings, we can re-expand these $\theta$s to see this equals
\[\sum_{\varphi\in\op{ON}(\pi)}\langle f_2\cdot\delta_T,\varphi\rangle\langle \ov\varphi,f_1\cdot\delta_T\rangle=\sum_{\varphi\in\op{ON}(\pi)}\int_{[T]}f_2^\lor\cdot\varphi(t)\,dt\int_{[T]}f_1^\lor\cdot\ov\varphi(t)\,dt.\]
We now hope to obtain a geometric expansion for this functional. Then we hope to be able to compare it to a relative trace formula for $\langle f\cdot\delta_A,\delta_{A,\eta}\rangle$, which should produce Waldspurger's formula.

Let's now say something about the geometric expansion. To begin, we note that we may expand
\[\langle\theta(\Phi_1),\theta(\Phi_2)\rangle_{[G]}=\int_{G(F)\backslash G(\AA_F)}\sum_{(\gamma_1,\gamma_2)\in X\times X(F)}\Phi_1(x_1g)\Phi_2(x_2g)\,dg.\]
We now divide this up by $G(F)$ orbits as
\[\int_{G(F)\backslash G(\AA_F)}\sum_{\xi\in(X\times X)(F)/G(F)}\sum_{\gamma\in G_\xi(F)\backslash G(F)}\Phi_1\otimes\Phi_2(\xi\gamma g),\]
which can be recombined into the sum
\[\sum_{\xi\in X\times X(F)/G(F)}\op{vol}([G_\xi])\int_{G_\xi(\AA_F)\backslash G(\AA_F)}\Phi_1\otimes\Phi_2(\xi g)\,dg\]
of orbital integrals. This is great we are now integrating over purely adelic objects, so good choices of $\Phi_1$ and $\Phi_2$ will turn these integrals into purely local problems. We take a moment to remark that one can alternatively work with other formulations of our pairing to give different kinds of orbital integrals.

At long last, here is our statement.
\begin{theorem}[Relative trace formula]
	Fix everything as above. Then
	\[\int_{\widehat G}\langle\theta(f_{2,T}^\lor),\theta(f_{1,T}^\lor)\rangle_\pi\,d\pi\]
	equals
	\[\sum_{\xi\in X\times X(F)/G(F)}\op{vol}([G_\xi])\int_{G_\xi(\AA_F)\backslash G(\AA_F)}\theta(f_{2,T}^\lor)\otimes\theta(f_{1,T}^\lor)(\xi g)\,dg,\]
\end{theorem}
There are similar expressions for the pairing $\langle f\cdot\delta_A,\delta_{A,\eta}\rangle$, but some regularization is needed because our integrals are no longer compactly supported. What saves us in this case is that we are comparing $\delta_A$ with the twist $\delta_{A,\eta}$, so one is able to create cancellation by playing the characters off of each other.

Let's at least explain what the orbital integrals look like. For example, when looking at the expression over $A\backslash G/A$, we choose some $[\xi]\in A\backslash G/A$ and $f\in S(G(\AA_F))$, and then our orbital integral looks like
\[\int_{A_\xi\backslash A\times A(\AA_F)}f\left(a_1^{-1}\xi a_2\right)\eta(a)\,d^\times a.\]
Next time, we will explain how to relate the two relative trace formulae. Notably, we will upgrade \Cref{thm:jacquet-compare} to compare the two geometric sides with a notable equality of the orbital integrals for any chosen $\xi$.


\end{document}