% !TEX root = ../notes.tex

\documentclass[../notes.tex]{subfiles}

\begin{document}

\section{April 8}
Welcome back.

\subsection{Recollections on Waldspurger's Formula}
For context, we are working in the context of smooth admissible representations. Globally over a number field $F$, this means that we are looking at some kind of representation $\pi$ of $G(\AA_F)$, which then has a decomposition into a restricted tensor product $\bigotimes'_v\pi_v$. For our setup, we take $G$ to be a suitable form of $\op{GL}_2$: let $K/F$ be a quadratic extension (where $F$ will be set to $\QQ$ later), we let $T$ be the torus $\op{Res}_{K/F}\mathbb G_{m,F}$, which is naturally embedded in $G\coloneqq\op{GL}_F(K)$, which is non-canonically isomorphic to $\op{GL}_2$.

Now, for $\chi$ and $\pi$ automorphic representations of $T$ and $G$ (respectively), we may set $\widetilde G\coloneqq(T\times G)/\mathbb G_m$. For Waldspurger's formula, we set $\delta_T\colon\chi\boxtimes\pi\to\CC$ to be the distribution given by integration over $\int_{[\mathbb G_m\backslash T]}$. Waldspurger's formula will compute the spectral decomposition of $\delta_T$. Morally, we would like to expand
\[\langle\delta_T,\delta_T\rangle\approx\int_{\widehat{\widetilde G}}\langle\delta_T,\delta_T\rangle_{\chi\boxtimes\pi}\,d(\chi\boxtimes\pi)\]
in terms of the special values $L(1/2,\pi\times\chi)$. We can think about $\langle\delta_T,\delta_T\rangle_{\chi\boxtimes\pi}$ formally as $\sum_{\varphi\in\op{ON}(\pi)}\left|\ell_\chi(\varpi)\right|^2$ (here, $\op{ON}(\pi)$ is an orthonormal basis), where
\[\ell_\chi(\varphi)\coloneqq\int\varphi(t)\chi(t)\,dt\]
for $\varphi\in\pi$. However, this latter sum is infinite, so we must do some regularization. Here are two ways to regularize.
\begin{itemize}
	\item Perhaps we can try to write a formula for $\left|\ell_\chi\right|^2$ directly as some map $\pi\otimes\ov\pi\to\CC$.
	\item Alternatively, we can add a Schwarz function to smoothen the sum. Choose a Schwarz function $f$ on $G(\AA_F)$, and we will instead compute $\langle f\cdot\delta_T,\delta_T\rangle$, where $f\cdot\delta_T$ refers to some convolution
	\[(f\cdot\varphi)(x)\coloneqq\int_{G(\AA_F)}f(g)\varphi(xg)\,dg.\]
	In particular, we may manipulate
	\begin{align*}
		\langle f\cdot\delta_T,\delta_T\rangle_{\chi\boxtimes\pi} &= \sum_\varphi\langle f\cdot\delta_T,\chi\varphi\rangle\overline{\langle\chi\varphi,\delta_T\rangle} \\
		&= \sum_\varphi\int_{[T]}f^\lor\cdot\varphi(t)\chi(t)\,dt\overline{\int_{[T]}\varphi(t)\chi(t)\,dt},
	\end{align*}
	where $f^\lor(g)\coloneqq f\left(g^{-1}\right)$.
	This sum is now finite: for example, on nonarchimedean places, $f$ becomes compactly supported, so $f\cdot\varphi$ must live in $J$-invariants of $\pi$ for some open subgroup $J\subseteq G(F_v)$, which is a finite-dimensional space due to the smoothness of $\pi$.
\end{itemize}
These approaches are in fact equivalent. Namely, having the first computation done implies that we ought to be able to compute $\langle f\cdot\delta_T,\delta_T\rangle_{\chi\boxtimes\pi}$ because our final sum above may as well be plugging $(f\cdot\varphi)\otimes\widetilde\varphi$ into $\ell_\chi\otimes\ell_{\chi^{-1}}$. Conversely, given the second computation, we note that the map $S(G(\AA_F))\to\op{End}(\pi)$ given by $f\mapsto(f\cdot-)$ surjects onto the ``smooth'' endomorphisms of $\pi$ (namely, having finite-dimensional output), so if we want to compute $(\ell_\chi\otimes\ell_{\chi^{-1}})(\varphi_1\otimes\varphi_2)$, we simply choose $f_1$ and $f_2$ so that their endomorphisms project $\pi$ and $\widetilde\pi$ onto the spans of $\varphi_1$ and $\varphi_2$ (respectively). Then $\ell_\chi(\varphi_1)\ell_{\chi^{-1}}(\varphi_2)$ is seen to be proportional to the sum
\[\sum_\varphi\ell_\chi(f_1\cdot\varphi)\ell_{\chi^{-1}}(f_2\cdot\widetilde\varphi),\]
but this last sum is an instance of the computation of $\langle f\cdot\delta_T,\delta_T\rangle_{\chi\boxtimes\pi}$.
\begin{remark}
	Here is a more succinct way to give the previous paragraph. The construction of $\ell_\chi$ implies that $\ell_\chi\otimes\ell_{\chi^{-1}}$ lives in
	\[\op{Hom}_{T(\AA_F)\times T(\AA_F)}(\pi\otimes\widetilde\pi,\CC_{\chi^{-1}\boxtimes\chi}).\]
	On the other hand, the functional $f\mapsto\langle f\cdot\delta_T,\delta_T\rangle$ lives in
	\[\op{Hom}_{T(\AA_F)\times T(\AA_F)}(S(G(\AA_F)),\CC_{\chi\boxtimes\chi^{-1}}),\]
	possibly up to correcting some signs. Then the equivalence of our two computations comes down to the image of $S(G(\AA_F))$ in endomorphisms (thought of as $\widetilde\pi\widehat\otimes\pi$) is large enough so that we can read off the value of any functional.
\end{remark}
\begin{remark}
	For reference, it is true that
	\[\dim\op{Hom}_{T(\AA_F)\times T(\AA_F)}(\pi\otimes\widetilde\pi,\CC_{\chi^{-1}\boxtimes\chi})=1.\]
\end{remark}
Now, Waldspurger's formula provides a local decomposition of $\ell_\chi\otimes\ell_{\chi^{-1}}$ as
\[\ell_\chi\otimes\ell_{\chi^{-1}}\approx\prod_v\alpha_v,\]
where $\alpha_v\in\op{Hom}_{T(F_v)\times T(F_v)}(\pi_v\otimes\widetilde\pi_v,\CC_{\chi^{-1}\boxtimes\chi})$ is defined by
\[\alpha_v(\varphi\otimes\widetilde\varphi)=\int_{\mathbb G_m\backslash T(F_v)}\langle\pi(t)\varphi,\widetilde\varphi\rangle\,dt.\]
Here, the $\approx$ in our statement means that the equality is only true up to some explicit volume factors which notably depend on choices of Haar measures.

Let's explain what this has to do with special values. If $v$ is nonarchimedean with $K_v/F_v$ unramified, and if we further assume $\pi_v^{G(\OO_v)}\ne0$ and $\chi_v|_{T(\OO_v)}=1$ and $\varphi\otimes\widetilde\varphi\in\pi_v^{G(\OO_v)}\otimes\widetilde\pi_v^{G(\OO_v)}$, then we are going to recover something about our local $L$-factors. (Note that these hypotheses hold for all but finitely many $v$.) Namely, one can calculate
\[\alpha_v(\varphi\otimes\widetilde\varphi)\approx\frac{L(1/2,\pi_v\times\chi_v)}{L(1,\mathrm{Ad},\pi_v)L(1,\eta_v)},\]
where $\eta\colon\AA_K^\times\to\CC^\times$ is the quadratic (Dirichlet) character associated to $K/F$.
\begin{remark}
	These conditions are potentially bizarre. Suppose that $v$ splits in two up in $K$, meaning that $K_v=F_v\oplus F_v$. Then we note $\mathbb G_m\backslash T$ is just $\mathrm U_1$, which is the kernel of the norm map $K^\times\to F^\times$. But now $\mathrm U_1(K_v)=F_v^\times$ because the norm map $F_v\times F_v\to F_v$ is just the product.
\end{remark}
We are now ready for a statement.
\begin{theorem}[Waldsburger's formula]
	Fix notation as above. For a set $S$ of places of $F$ which is large enough, we have
	\[\ell_\chi\otimes\ell_{\chi^{-1}}\approx\frac{L^S(1/2,\pi\times\chi)}{L^S(\pi,\mathrm{Ad},1)L^S(\eta,1)}\prod_{v\in S}\alpha_v,\]
	up to some volume factors.
\end{theorem}
\begin{remark}
	In terms of our second question, we may equivalently ask for our functional $S(G(\AA_F))\to\CC$ to suitably factor into local pieces when the input $f\in G(\AA_F)$ does.
\end{remark}
Let's recall something about our $L$-functions. To do this appropriately, we think of $T$ as $\op{GSO}_2$, where the ambient quadratic form is given by the norm; similarly, we can think of $G$ as $\op{GSO}_3$. (Here, $\op{GSO}_\bullet$ refers to $\op{SO}_\bullet$ plus scalars and is in particular connected, unlike $\op{GO}_\bullet$ which may be disconnected.) Now, $\check T$ is still $\op{GSO}_2$ and has $^LT=\op{GO}_2$ due to a nontrivial Galois action. Similarly, $\check G=\op{GSp}_2$. Now, one may compute that
\[\check{\widetilde G}=\left(\check T\times\check G\right)^{\det=1}=\left(\mathrm{GSO}_2\times\mathrm{GSp}_2\right)^{\det=1},\]
where the point is that the modding out by $\mathbb G_m$ is turned into a determinant $1$ condition. Thus, we see that $^L\widetilde G$ will have a standard four-dimensional symplectic representation $r$ given by $\chi$ and $\pi$, so we can build a degree four $L$-function $L(\chi\boxtimes\pi,r,s)$.

This construction of the $L$-function is not very helpful for us. Instead, we will use the theory of base change. Morally, note that there is an embedding of Langlands parameters $W_F\to{^LG}$ to Langlands parameters $W_K\to{^LG}$ simply by embedding the Weil groups. Thus, we expect to have a ``base-change'' functor from automorphic representations of $G_F$ to automorphic representations of $G_K$ which agrees with this. This is conjectural for general extensions $K/F$, but it is known if $K/F$ is quadratic (due to Langlands) or solvable (due to Arthur--Clozel). The moral is that we can think of $\pi$ has a base-change to some automorphic representation $\Pi_{K/F}$ of $G_K$, and we will have
\[L(\chi\times\pi,s)=L(\chi\otimes\Pi_{K/F},\mathrm{Std},s),\]
where now the right-hand side is the standard $L$-function for (a form of) $\mathrm{GL}_{2,K}$. Perhaps one is confused because the right-hand side is an $L$-function of degree $2$, but this is okay because $K/F$ is a quadratic extension which absorbed a factor of $2$.

For simplicity, we will take $\chi=1$ (so that the central character $\omega_\pi$ of $\pi$ is trivial) for the remainder of the course. Then it turns out that
\[L(\chi\times\pi,s)=L(\pi,s)L(\pi\otimes\eta,s).\]
Let's explain this in terms of Langlands parameters. Taking $\chi=1$ amounts to taking the trivial Langlands parameter of $^LT$, but $^LT$ already has some Galois action, so it amounts to the canonical projection $W_F\to\op{Gal}(K/F)$. Thus, tensoring with $\pi$ basically gives us two copies of $\pi$, one for each character of $\op{Gal}(K/F)$, so we get $\pi\oplus\pi\eta$.

Anyway, these are now some standard $L$-functions, which have integral representations, allowing us to do some computations. Namely, we would like to compare $\langle\delta_T,\delta_T\rangle_\pi$ to $\langle\delta_A,\delta_{A,\eta}\rangle_\pi$ where $A$ is the diagonal torus of elements of the form $\begin{bsmallmatrix}
	* \\ & 1
\end{bsmallmatrix}$. In fact, we will show the following.
\begin{theorem}[Jacquet] \label{thm:jacquet-compare}
	Fix everything as above. Then for $f\in S(G(\AA_F))$, one can construct $f'\in S(G(\AA_F))$ such that
	\[\langle f\cdot\delta_T,\delta_T\rangle=\langle f'\cdot\delta_A,\delta_{A,\eta}\rangle.\]
	Further, for $h\in S(G(F_v))^{G(\OO_v)\times G(\OO_v)}$, we have $h*f=h*f'$.
\end{theorem}
Let's explain the application to Waldspurger's theorem. The left-hand side admits a spectral decomposition
\[\langle f\cdot\delta_T,\delta_T\rangle=\int_{\widehat G}\underbrace{\langle f\cdot\delta_T,\delta_T\rangle_\pi}_{J_\pi(f)\coloneqq}\,d\pi,\]
so we give the right-hand side a similar spectral decomposition, writing
\[\langle f'\cdot\delta_A,\delta_{A,\eta}\rangle=\int_{\widehat G}\langle f\cdot\delta_A,\delta_{A,\eta}\rangle_\pi\,d\pi.\]
Now, one can try to use Hecke algebras to try to separate the contribution of each $\pi$ on each side matches up. For example, if $\langle f\cdot\delta_T,\delta_T\rangle_\pi$ is not identically zero, then $\langle f\cdot\delta_A,\delta_{A,\eta}\rangle_\pi$ will also not be identically zero, which by an argument with the functional equation implies that $L(\pi,1/2)L(\eta\otimes\eta,1/2)$ fails to vanish.

\end{document}