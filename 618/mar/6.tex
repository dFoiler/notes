% !TEX root = ../notes.tex

\documentclass[../notes.tex]{subfiles}

\begin{document}

\section{March 6}
Today we continue with the overview of our main theorems.

\subsection{The Gross--Zagier Formula}
For the Gross--Zagier formula, we consider the map $\mathrm{Sh}_T\to\mathrm{Sh}_T\times\overline{\mathrm{Sh}}_G$. Thus, we are producing some divisor on a variety, and the Gross--Zagier formula explains how to compute its height.
\begin{theorem}[Gross--Zagier, rough version]
	For any irreducible automorphic representation $\chi\boxtimes\pi$ of $T\times G$ (where $\pi_\infty$ is a weight-$2$ modular form), the corresponding isotypic component of $\langle\mathrm{Sh}_T,\mathrm{Sh}_T\rangle_{\mathrm{NT}}$ (where $\mathrm{NT}$ denotes the N\'eron--Tate height) is proportional to the central value $L'(\chi\times\pi,1/2)$.
\end{theorem}
Once again, let's be more precise about what this inner product means. Choose a weight-$2$ cusp form $f\in S_2(\Gamma)$ for some congruence subgroup $\Gamma$. Then $\overline{\mathrm{Sh}}_G=X(\Gamma)$, which surjects onto the abelian variety $A_f$ attached to $f$. The composite
\[\mathrm{Sh}_T\subseteq\mathrm{Sh}_G\subseteq\overline{\mathrm{Sh}}_G\onto A_f\]
produces the so-called Heegner points $t\mapsto[t]$ of $A_f$. Then we want to compute the height of the divisor $\sum_{t\in\mathrm{Sh}_T(\Gamma)}\chi(t)[t]$.
\begin{remark}
	As we have written the two theorems, they appear to be quite similar, and indeed, all known proofs of the Gross--Zagier formula are inspired by corresponding proofs of Waldspurger's theorem. However, in the current setting, there is no way to prove both at the same time.
\end{remark}
\begin{remark}
	In the function field setting, it is possible to give a uniform proof. Roughly speaking, Waldspurger's theorem is about shtukas with zero legs, and the Gross--Zagier formula is about shtukas with one leg.
\end{remark}
\begin{remark}
	The classical application of the Gross--Zagier formula is to argue that the Heegner points are producing non-torsion points because their N\'eron--Tate height is nonzero. This has an application to the Birch and Swinnerton-Dyer conjecture. We note that the difference between $L'(\chi\times\pi,1/2)$ and $L'(E,1)$ is merely one of automorphic vs. motivic normalization of $L$-functions.
\end{remark}

\subsection{Automorphic \texorpdfstring{ $L$}{ L}-Functions}
We will need to define the $L$-function $L(\chi\times\pi,s)$. Let's begin with the general formalism of automorphic $L$-functions.

Throughout this section, fix a reductive group $G$ over a global field $F$. Recall that one can realize an automorphic form $\pi$ (which is some kind of representation of $G(\AA_F)$) as a subrepresentation $\pi\subseteq C^\infty([G])$. Further, $\pi$ factors as $\pi=\bigotimes_v'\pi_v$, where $\pi_v$ is unramified for all but finitely many $v$ (taken to be nonarchimedean), meaning that $\pi_v^{K_v}\ne0$ (where $K_v\coloneqq G(\OO_v)$ comes from a choice of integral model) and hence produces a nonzero module of the local Hecke algebra $\mc H(G_v,K_v)$.

Now, by the Satake isomorphism, this Hecke algebra can be realized as $\CC[X_v]$ where $X_v$ is some variety; thus, irreducible $\mc H(G_v,K_v)$-modules can be realized as $\CC$-points in
\[\op{Hom}(\mc H,\CC)=X_v(\CC).\]
It is worth having a more explicit description of $X_v$: recall that there is a dual group $\check G$ and then a description of the $L$-group as $^LG=\check G\rtimes\op{Gal}(\ov F/F)$, and then for a nonarchimedean place $v$, we have
\[X_v\coloneqq\check G\mathrm{Fr}_v/\check G=\Spec\CC[\check G\mathrm{Fr}_v]^{\check G},\]
where the quotient is taken in the sense of geometric invariant theory. (Here, $\check G$ acts on $\check G\mathrm{Fr}_v$ by conjugation.) Morally speaking, $X_v(\CC)$ should be thought of as semisimple Frobenius-twisted conjugacy classes in $\check G$. In particular, $c_v\in X_v(\CC)$ produces a homomorphism $W_{\QQ_p}\to{^LG}$ taking the Frobenius to $c_v\mathrm{Fr}_v$. We may call $c_v\in X_v(\CC)$ a ``Satake parameter'' and the map $W_{\QQ_p}\to{^LG}$ a ``Langlands parameter.''

In total, we see that $\pi_v$ (when unramified) admits a Langlands parameter $c_v(\pi_v)\colon W_{\QQ_p}\to{^LG}$, which is trivial on inertia.
\begin{remark}
	The local Langlands conjecture asks one to attach a Langlands parameter $W_v\to{^LG}$ for any irreducible representation $\pi_v$ of $G(F_v)$, not necessarily unramified. Of course, the previous sentence does not technically have content because we would like the association to satisfy some coherence conditions pinning it down.
\end{remark}
\begin{definition}
	Fix a reductive group $G$ over a number field $F$. Assume the local Langlands conjecture for $G_v$ for all places $v$. Choose some algebraic representation $r\colon{^LG}\to\op{GL}(V)$. For some irreducible representation $\pi$ of $G(\AA_F)$, write $\pi=\bigotimes_v\pi_v$, and we define
	\[L(\pi,r,s)\coloneqq\prod_vL(\pi_v,r,s),\]
	where $L(\pi_v,r,s)$ is defined as the local Artin $L$-function:
	\[L(\pi_v,r,s)\coloneqq L(r\circ\varphi_v,s)\coloneqq\frac1{\det\left(1-q_v^{-s}r\varphi_v(\mathrm{Fr}_v);V^{r\varphi_v(I_v)}\right)},\]
	where $\varphi_v$ is the local Langlands parameter.
\end{definition}
\begin{remark}
	Consider $G=\mathrm{GL}_{2,F}$ so that $\check G=\mathrm{GL}_{2,\CC}$, and we take $r\colon\mathrm{GL}_2\to\mathrm{GL}_{k+1}$ to be the $k$th symmetric power. Then a Satake parameter of $\pi_v$ viewed as a conjugacy class looks like $\op{diag}(\alpha_v,\beta_v)$ for some $\alpha_v$ and $\beta_v$. Thus, we can compute
	\[L(\pi_v,r,s)=\prod_{i=0}^k\frac1{1-q_v^{-s}\alpha_v^i\beta_v^{k-i}}.\]
\end{remark}
It is worth recording the following conjecture, which explains the expected analytic properties.
\begin{conj}[Langlands] \label{conj:langlands}
	Suppose $\pi$ is an automorphic irreducible representation of $G(\AA_F)$. Then $L(\pi,r,s)$ admits a meromorphic continuation to $\CC$ (with prescribed poles) and a functional equation of the form
	\[L(\pi,r,s)=\varepsilon(\pi,r,s)L(\pi^\lor,r,1-s),\]
	where $\varepsilon(\pi,r,s)$ is $\varepsilon(\pi,r,1/2)$ times an exponential in $\frac12-s$.
\end{conj}
\begin{remark}
	Technically speaking, this is a ``second-order'' conjecture because it already depends on the local Langlands conjecture!
\end{remark}
\begin{remark}
	The Langlands functoriality conjecture predicts that the homomorphism $r\colon{^LG}\to\op{GL}(V)$ yields a map $\ell_r$ from automorphic representations of $G$ to automorphic representations of $\op{GL}_{\dim V}$ which is compatible with $L$-functions:
	\[L(\pi,r,s)=L(\ell_r(\pi),\mathrm{Std},s).\]
	Thus, we could realize $L(\pi,r,s)$ as an $L$-function for the group $\op{GL}_n$ and the standard representation, where the conjecture is already known by Godement--Jacquet (extending Riemann--Iwasawa--Tate for $\mathrm{GL}_1$).
\end{remark}
\begin{notation}
	There is some standard simplified notation for our $L$-functions. If $G$ is classical, we may write $L(\pi,s)$ for $L(\pi,\mathrm{Std},s)$. If $G=G_1\times G_2$ is a product of two classical groups, we may write $L(\pi_1\times\pi_2,s)$ for $L(\pi_1\boxtimes\pi_2,\otimes,s)$, where $\otimes$ is the tensor product representation on the dual side.
\end{notation}
\begin{remark}
	\Cref{conj:langlands} is known in the listed cases when the classical group is $\op{GL}_n$.
\end{remark}
For our applications, our map $r\colon{^LG}\to\op{GL}(V)$ will factor through $\op{Sp}(V)$. In this case, it turns out that
\[\varepsilon\left(\pi,r,\frac12\right)=\pm1.\]
Roughly speaking, one knows (in general) that we can expand $\varepsilon(\pi,r,s)=\prod_v\varepsilon(\pi_v,r,s)$, and it turns out that these local root numbers $\varepsilon(\pi_v,r,1/2)$ are also $\pm1$. For example, $\varepsilon(\pi_v,r,1/2)$ in the Iwasawa--Tate case is realized as some normalized Gauss sum.

We are now ready to return to our situation with $T=\op{Res}_{K/\QQ}\mathbb G_{m,K}$ and $G=\op{GL}_2$.
\begin{itemize}
	\item Note $\check T=\mathbb G_{m,\CC}^2$ and so $^LT=\mathbb G_m^2\rtimes\op{Gal}(K/\QQ)$, where the Galois action acts by permuting the two factors (seen by tracking through the Galois action on $\check T$). In particular, $^LT$ is isomorphic to $\op{GO}_2$.
	\item For $G=\op{GL}_2$, we see $^LG=\op{GL}_{n,\CC}\times\op{Gal}(\ov\QQ/\QQ)$.
\end{itemize}
We are interested in automorphic representations $\chi\boxtimes\pi$ of $T\times G$, and we will further be interested in those which admit $T^\Delta$ functionals, where $T^\Delta$ is the diagonal copy of $T$ embedded in $T\times G$. For example, we want the restriction of $\chi\boxtimes\pi$ to $\mathbb G_m\subseteq T\times G$ to be trivial, which amounts to asking for $\chi\omega_\pi=1$, where $\omega_\pi$ is the central character of $\pi$. Thus, we may as well think about $\chi\boxtimes\pi$ as being an automorphic representation of $\widetilde G\coloneqq(T\times G)/\mathbb G_m$, where $\mathbb G_m$ is embedded diagonally. Then one can compute
\[^L\widetilde G=\mathrm{GO}_2\times_{\mathbb G_m}\mathrm{GL}_2\]
In particular, the tensor product representation $r$ embeds $^L\widetilde G$ into $\mathrm{Sp}_4$. Thus, we may define
\[L(\chi\times\pi,s)\coloneqq L(\chi\boxtimes\pi,r,s),\]
which by construction is symplectic, and we know that its $\varepsilon$-factor should be $\pm1$ as the central point.

\end{document}