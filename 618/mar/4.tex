% !TEX root = ../notes.tex

\documentclass[../notes.tex]{subfiles}

\begin{document}

\section{March 4}
Today we continue with our discussion of geometric class field theory.

\subsection{Recovering Class Field Theory}
The following content is extremely sketchy.
We would like to show that the map
\[\alpha\colon\pi_1(C,\ov\eta)^{\mathrm{ab}}\to\widehat{\op{Pic}_C(\FF_q)}\]
is an isomorphism, recovering class field theory. We already know that it is surjective because of our construction: it is not hard to see that we surject onto $\op{Pic}_C(\FF_q)$ by our construction, so we successfully hit a dense subset by the continuous map $\alpha$.

For the injectivity, it is enough to check this on the level of (enough) characters. Fix $\Lambda\coloneqq\ZZ_\ell$ for $\ell\ne p$ (or more generally, any ring of integers in any finite extension of $\QQ_\ell$), and we will show that the induced map
\[\op{Hom}(\widehat{\op{Pic}_C(\FF_q)},\Lambda^\times)\to\op{Hom}(\pi_1(C,\ov\eta),\Lambda^\times)\]
is surjective. This will achieve the required injectivity after looping over all $\ell$. Now, this right-hand side is given by a one-dimensional $\ell$-adic local system, so we would like a geometric avatar for the left-hand side.
\begin{definition}[character local system]
	Fix a smooth commutative algebraic group $G$ over $\ov\FF_q$. A \textit{character local system} on $G$ is a pair $(\mc L,\psi)$ where $\mc L$ is a one-dimensional local system, and
	\[\psi\colon m^*\mc L\to\op{pr}_1^*\mc L\otimes\op{pr}_2^*\mc L\]
	is an isomorphism satisfying a cocycle condition expected by an associativity law. We let $\op{CharLoc}(G)$ denote this group of character local systems.
\end{definition}
\begin{example}
	Given a character $\chi\colon G(\ov\FF_q)\to\Lambda^\times$, one can construct the local system $\mc L_\chi$ on $G$ as being trivialized by the Lang map $L\colon G\to G$ and twisted by $\psi$ upstairs; in particular, the multiplication arises from having a multiplication upon restriction along $L$. Conversely, given a character local system $(\mc L,\psi)$, one can take trace of the Frobenius in order to define a map
	\[\op{CharLoc}(G)\to\widehat{\op{Hom}(G(\ov\FF_q),\Lambda^\times)}.\]
\end{example}
\begin{remark}
	It turns out that local systems, constructed in the above manner, are able to produce all characters of a given finite group which looks like $G(\FF_q)$, if we allow more Frobenius traces and more local systems.
\end{remark}
The moral of the story is that we are looking for an isomorphism
\[\mathrm{AJ}^*\colon\op{CharLoc}(\mathrm{Pic}_C)\to\op{Loc}^1_C(C)\]
given by class field theory. Namely, we would like to find an inverse for this map.
\begin{remark}
	For motivation, let's recall why $\op{Pic}_C$ is a variety. It is enough to check that $\op{Pic}_C^d$ is a scheme for any $d$. For this, we note that $\op{Pic}_C$ has a covering by the sheaf of pairs $(\mc L,s)$ of a line bundle together with a nonzero section $s$; by reading off the zeroes of $s$, we see that the sheaf of such pairs are simply given by effective divisors, which of course is $\bigsqcup_{d\ge0}C^{(d)}$. Now, for large enough degree $d$ (compared to the genus), Riemann--Roch explains how many sections a line bundle has: on degree $d$, we get a fiber bundle $C^{(d)}\to\op{Pic}^d$ with fibers $\PP^{d-g+1}$. Figuring out how to take a quotient completes the construction.
\end{remark}
The remark explains that we may understand local systems on $\mathrm{Pic}_C$ as basically local systems on $C^{(d)}$ (because the fibers of the covering are simply connected for $d\gg0$!), so we can imagine taking a local system $\mc L$ on $C$ and producing a local system $\mc L^{(d)}$ on $C^{(d)}$ by some kind of symmetric power construction.
\begin{remark}
	Let's record one step which checks that the produced sheaf $r(\mc L)$ on $\op{Pic}^d$ is actually a character local system. We will check what happens under the ``Hecke operators'' $T_c\colon\op{Pic}\to\op{Pic}$ given by twisting by a closed point $c\in C$: for $d\gg0$, it turns out that
	\[T_c^*r(\mc L)|_{\op{Pic}_C^{d+1}}=r(\mc L)|_{\op{Pic}_C^d}\otimes\mc L_c,\]
	so we see that $r(\mc L)$ is a ``Hecke eigensheaf.'' One can then use this property to extend $r(\mc L)$ uniquely to all of $\op{Pic}_C$ and then check that it is the inverse of $\mathrm{AJ}^*$.
\end{remark}
\begin{remark}
	More generally, it is true that shtukas produces a map from ``automorphic representations'' (which look like representations of $\op{Pic}_C(\FF_q)$) to ``Galois representations'' (which look like representations of $\pi_1^{\mathrm{\acute et}}(C,\ov\eta)$).
\end{remark}

\end{document}