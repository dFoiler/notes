% !TEX root = ../notes.tex

\documentclass[../notes.tex]{subfiles}

\begin{document}

\section{March 13}
We continue. As usual, $K$ is an imaginary quadratic field.

\subsection{A Little on Automorphic Representations}
Let's recall a working definition of automorphic representation.
\begin{definition}[automorphic representation]
	Fix a reductive group $G$ over a number field $F$. Then an \textit{automorphic representation} of $G$ is an irreducible representation $\pi$ of $G(\AA_F)$ equipped with an embedding $\pi\into C^\infty([G])$.
\end{definition}
\begin{remark}
	This definition is not technically correct. We should allow subquotients of finite-length modules, and we should only work with $(\mf g,K)$-modules at the real infinite places.
\end{remark}
\begin{remark}
	It turns out that $\pi$ admits a restricted tensor product decomposition $\pi\cong\bigoplus'_v\pi_v$. Being a restricted tensor product means that $\pi_v$ is unramified for almost all $v$ (namely, $\pi_v^{G(\OO_v)}\ne0$), and then we require that an element of $\bigoplus'_v\pi_v$ takes the value of the $G(\OO_v)$-fixed vector for all but finitely many $v$.
\end{remark}
\begin{example}
	Work with $G=\mathrm{GL}_{2,\QQ}$. Choose $f\in S_2(\Gamma_0(N))$ to be a Hecke eigenform for $T_p$ for $p$ away from $N$. Some argument involving double quotients explains how $f$ produces some $F\in C^\infty([G])$, which then spans some automorphic representation $\pi$ of $G(\AA_\QQ)$. It turns out that $\pi_\infty$ then lives in the short exact sequence
	\[0\to\pi_\infty\to\mathrm{Meas}^\infty\left(\PP^1(\RR)\right)\to\CC\to0.\]
\end{example}
\begin{remark} \label{rem:better-cuspidal-rep-from-eigenform}
	Let's give a more explicit description of the previous example. Suppose further that $f$ has rational coefficients so that the quotient $J_0(N)\to A_f$ corresponding to $f$ is an elliptic curve. We also suppose that $f$ (and thus $A_f$) does not have complex multiplication. Then it turns out that the finite part $\pi_f\cong\bigoplus_{p<\infty}'\pi_p$ is given by
	\[\pi_f=\colimit\op{Hom}(J_0(N),A_f)_\QQ.\]
	Notably, this becomes a representation of $G(\AA_{\QQ,f})$ by the action locally. One checks this by understanding how the $G(\AA_{\QQ,f})$ action on $A_f$ corresponds to the Hecke action on $f$.
\end{remark}

\subsection{Fixing our Theorems}
We now begin fixing our theorem. We will focus on cases where $\chi\boxtimes\pi$ has trivial central character, which means that it descends to an automorphic representation of $(T\times G)/\mathbb G_m$, where this $\mathbb G_m$ is embedded diagonally. Then we see that we should also mod out our embedding $T\into(T\times G)$ by $\mathbb G_m$ which becomes the embedding
\[T/\mathbb G_m\into(T\times G)/\mathbb G_m.\]
Now, $T/\mathbb G_m$ is simply $\mathrm U_1$, so our pairings should really be taken over $\mathrm U_1$. Thus, in Waldspurger's formula, we will want to integrate over $\mathrm U_1$ instead of $T$.

For the Gross--Zagier formula, we choose a Hecke eigenform $f\in S_2(\Gamma)$ for some congruence subgroup $\Gamma$ and with rational coefficients (as in \Cref{rem:better-cuspidal-rep-from-eigenform}); let $F\in\pi_f$ be the distinguished element of the cuspidal automorphic representation. Further, we choose a CM point $P$ given by an embedding $\mathrm{Sh}_T\to\mathrm{Sh}_T\times\mathrm{Sh}_G$. Now, it turns out that there is a distinguished Hodge class of line bundles in $\mathrm{Sh}_G$, which provides a good choice of Abel--Jacobi map
\[\mathrm{AJ}\colon\mathrm{Sh}_G\to J,\]
where $J$ is technically an inverse limit $\limit J_0(N)$. In particular, the image of $P\in X_0(N)$ is defined over $K^{\mathrm{ab}}$, so its image in the Jacobian is as well.

We now use $\chi$ to define a divisor. Viewing $F$ as an element of $\op{Hom}(J,E_f)_\QQ$, we see that we can write down $F(P)$ as morally being an element of $E_f(K^{\mathrm{ab}})_\QQ$. Then we may integrate
\[P_\chi(F)\coloneqq\int_{\op{Gal}(K^{\mathrm{ab}}/K)}\chi(\tau)F(\tau(P))\,d\tau.\]
Because $\chi$ is an idele class character $\op{Gal}(K^{\mathrm{ab}}/K)\to\ov\QQ\times$ (where we have used class field theory), it descends to a finite quotient of $\op{Gal}(K^{\mathrm{ab}}/K)$. Similarly, $P$ is defined over some finite extension of $K$. Thus, this integral really descends to a finite sum. In total, we are producing an element in $E_f(K^{\mathrm{ab}})\otimes_\ZZ\ov\QQ$.
\begin{remark}
	The theory of complex multiplication explains that the Galois action on $P$ should agree with what we find with the reciprocity map. In particular, viewing $P$ as an element $z_0\in\mathcal H$ of the upper-half plane, then $\tau(P)$ is simply
	\[\left[z_0,\op{Art}_K^{-1}(\tau)\right]\in\mc H\times.\]
\end{remark}
Now, the Gross--Zagier formula is about the inner N\'eron--Tate height of $\langle P_\chi(F),P_{\chi^{-1}}(\widetilde F)\rangle_{\mathrm{NT}}$, where $\widetilde F$ is the corresponding element of the contragredient $\widetilde\pi_f$.
\begin{remark}
	There is something bizarre about the current formulation: we would like the N\'eron--Tate height to be proportional to $L'(1/2,\pi\times\chi)$, but the divisor we constructed depends on a choice of modular form $f\in\pi$!
\end{remark}
Let's try to remedy the above remark. Note that there is a canonical pairing
\[\pi_f\otimes\widetilde\pi_f\to\CC,\]
which is $\left(T(\AA_f),\chi^{-1}\right)\times(T(\AA_f),\chi)$-equivariant. Thus, we improve our formulation to ask for
\[\left\langle P_\chi(F),P_{\chi^{-1}}(\widetilde F)\right\rangle_{\mathrm{NT}}=cL'\left(\frac12,\pi_f\times\chi\right)\prod_p\alpha_p\left(F_p\otimes\widetilde F_p\right),\]
where $c$ is some global explicit constant, and $\alpha_p\colon\pi_p\otimes\widetilde\pi_p\to\CC$ is the canonical pairing; explicitly,
\[\alpha_p\left(F_p\otimes\widetilde F_p\right)\approx c_p\int_{T(\QQ_p)/\QQ_p^\times}\chi_p(t)\left\langle\pi_p(t)F,\widetilde F\right\rangle\,dt.\]
The $c$s and $c_p$s are some explicit constants; for example, the $c_p$s should be chosen so that the product is $1$ for all but finitely many terms. Namely, we should take the $c_p$s to come from some Tamagawa measures.

This motivates us to go back and adjust the statement of the Waldspurger formula as follows: we have
\[\sum_\varphi\int_{[T/\mathbb G_m]}\chi(t)\varphi(t)\,dt\cdot\int_{[T/\mathbb G_m]}\chi^{-1}(t)\widetilde\varphi(t)\,dt=cL\left(\frac12,\pi\times\chi\right)\prod_p\alpha_p(\varphi_p\otimes\widetilde\varphi_p),\]
where the $\alpha_p$s are some similarly scaling factors. For example, one needs to choose constants $c_p$ so that $\alpha_p(\varphi_p\otimes\widetilde\varphi_p)$
\begin{remark}
	We now see that one can view both of our theorems as providing Euler product decompositions of our ``geometrically motivated'' objects.
\end{remark}
There is a representation-theoretic obstruction to the statement of our current theorems. There is a theorem of Saito--Tunnell which states that
\[\op{Hom}_{T(\QQ_v)}(\chi_v\boxtimes\pi_v,\CC)\]
vanishes unless $\chi_v(-1)\eta_p(-1)=\varepsilon_p(-1)$, where $\eta_v=\varepsilon(1/2,\pi_v\times\chi_v)$ is the local root number. (In fact, this result holds for more general quadratic extensions, but we will not need this. This is an instance of a more general phenomenon called the Gan--Gross--Prasad conjectures.) Do note that $\chi_v(-1)\eta_p(-1)=\varepsilon_p(-1)$ reads as $1\cdot1=1$ for all but finitely many $v$.

The problem here is that our left-hand side of both theorems will vanish unless $\chi_v(-1)\eta_p(-1)=\varepsilon_p(-1)$ holds for every prime $p$! To fix this, Gross and Zagier imposed a ``Heegner condition'' that $f$ should be a new form for $\Gamma_0(N)$ such that each $p\mid N$ has $p^2\nmid N$ and $p$ is split or ramified in $K$, which implies that $\eta_p(-1)=\varepsilon_p(-1)$. (If $K=\QQ(\sqrt{-D})$, then this Heegner condition is asking for $D\pmod{4N}$ to be a square.)

Let's now relate this to $\varepsilon$-factors. Recall that we have a global functional equation of the form
\[L(s,\pi\times\chi)=\varepsilon(s,\pi\times\chi) L(1-s,\pi\times\chi),\]
and we see that the sign $\varepsilon\coloneqq\varepsilon(1/2,\pi\times\chi)$ admits a product factorization as $\prod_v\varepsilon_p(1/2,\pi\times\chi)$. So if we are given $\prod_v\chi_v(-1)\eta_v(-1)=1$ by the Heegner condition, we find the following.

Let's now fully correct the Waldspurger formula. If $\varepsilon=-1$, then Waldspurger's formula has no content: $L(1/2,\pi\times\chi)$ vanishes, and the local represen\-tation-theoretic obstruction tells us that $\chi_v(-1)\eta_p(-1)=\varepsilon_p(-1)$ failing to hold at some $p$ implies that the Euler product also vanishes. On the other hand, $\varepsilon=1$ may find some content in $L(1/2,\pi\times\chi)$, but of course it is possible for $\chi_v(-1)\eta_p(-1)=\varepsilon_p(-1)$ to fail at an even number of places. Thus, the theorem as stated is wrong, and to fix it, we should pass from $\mathrm{GL}_2$ to the inner form $D^\times$ (where $D$ is a quaternion algebra over $\QQ$). Then we find that the theorem as stated holds where we choose $D$ of the form $D_\Sigma^\times$, where $\Sigma$ is the even set of places where $\chi_v(-1)\eta_p(-1)=\varepsilon_p(-1)$ fails. Here, $D_\Sigma$ is the element of the Brauer group chosen by the short exact sequence
\[0\to\op{Br}F\to\bigoplus_v\op{Br}F_v\to\QQ/\ZZ\to0.\]
We now turn to the Gross--Zagier formula. If $\varepsilon=1$, then $L'(1/2,\pi\times\chi)=0$ for functional equation reasons. To study its left-hand side, then we find a representation-theoretic failure at $\infty$: because $K_\infty=\CC$, we see that $\eta_\infty(-1)=-1$, but $\chi_\infty=\varepsilon_\infty=1$. Because $\varepsilon=1$, we see that $\chi_v(-1)\eta_p(-1)=\varepsilon_p(-1)$ fails at an odd set of finite primes $p$. In total, $\op{Hom}_{T(\AA_f)}(\pi_f\otimes\chi,\CC)$ vanishes, so the left-hand side must vanish.

On the other hand, having $\varepsilon=-1$ may reveal some content. Namely, $\chi_v(-1)\eta_p(-1)=\varepsilon_p(-1)$ will fail at an even set $\Sigma$ of finite places $p$. This time around, we replace $\mathrm{GL}_{2,\QQ}$ with its inner form $D_\Sigma^\times$, and the Shimura variety $\mathrm{Sh}_{\mathrm{GL}_2}$ is replaced by an inverse limit of Shimura curves. Then one can recover the correct Gross--Zagier theorem.

\end{document}