% !TEX root = ../notes.tex

\documentclass[../notes.tex]{subfiles}

\begin{document}

\section{March 4}
Let's now say a few words about what the second part of the course is going to be about.

\subsection{Waldspurger's Theorem}
We are interested in the theorems of Waldspurger and Gross--Zagier, which are related to the embedding $T\to G$ of the torus $\op{Res}_{K/\QQ}\mathbb G_{m,K}$ (where $K/\QQ$ is imaginary quadratic) and $G=\op{GL}_{2,\QQ}$. Now, for any group $H$ over $\QQ$, we have an adelic quotient $[H]\coloneqq H(\QQ)\backslash H(\AA_\QQ)$. Thus, so the embedding $T\into G$ produces a map $[T]\into[G]$, which roughly corresponds to the choice of an elliptic curve with CM by $K$.

Now, let $\delta_T$ be the pushforward of the Haar measure on $[T]$ to a distribution on $[T\times G]$; after fixing a level structure, this is basically a finite sum over the CM points. Then Waldspurger's theorem explains how to compute $\langle\delta_T,\delta_T\rangle$. Of course, it is a little tricky to make sense of such an inner product, but we will side-step this issue by only looking at some components.
\begin{theorem}[Waldspurger, rough version]
	For any irreducible automorphic representation $\chi\boxtimes\pi$ of $T\times G$, the corresponding isotypic component of $\langle\delta_T,\delta_T\rangle$ is proportional to the central value $L(\chi\times\pi,1/2)$.
\end{theorem}
Namely, to compute this isotypic component, one may instead compute
\[\sum_{\varphi}\langle\varphi,\delta_T\rangle\langle\delta_T,\chi\otimes\varphi\rangle,\]
where $\varphi$ varies over some orthnormal basis of $\pi$. But now we see that this sum is
\[\sum_\varphi\left|\int_{[T]}\varphi(t)\chi(t)\,dt\right|^2,\]
and this is what we will choose to compute.

\end{document}