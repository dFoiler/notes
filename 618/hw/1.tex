% !TEX root = ../notes.tex

\documentclass[../notes.tex]{subfiles}

\begin{document}

\section{Homework 1}
Our exposition follows \cite[Section~3]{serre-cm}.

\subsection{The Composite of Hilbert Class Fields}
Throughout, it will be helpful to remember the ``profinite completion'' notation as
\[\widehat\OO_K=\prod_{v<\infty}\OO_{K,v}.\]
This is an isomorphism of rings.

We begin with some general result on class field theory.
\begin{lemma} \label{lem:artin-ideles-fixing-hilbert}
	Fix a totally imaginary number field $K$, and let $H$ be the Hilbert class field of the order $\OO_K$. Then the restriction of the global Artin map
	\[\OO_K^\times\backslash\widehat\OO_K^\times\to\op{Gal}(K^{\mathrm{ab}}/H)\]
	is a surjection of topological groups.
\end{lemma}
\begin{proof}
	We begin with the usual global Artin map as
	\[K^\times\backslash\AA_{K,f}^\times\to\op{Gal}(K^{\mathrm{ab}}/K).\]
	Note that this map is surjective by \cite[Corollary~8.2.2]{neukirch-cohom}, where we note that we have removed the archimedean complexes (which are all complex here!). Now, an idele class $x\in K^\times\backslash\AA_{K,f}^\times$ fixes $H$ if and only if $x_v\in\OO_{K,v}^\times$ for all finite places $v$, which is equivalent to $x\in\widehat\OO_K^\times$.\footnote{Technically, one only knows that we can adjust $x$ by an element of $K^\times$ to live in $\widehat\OO_K^\times$. Please forgive me for this sort of abuse of notion throughout.} Thus, the global Artin map restricts to a continuous surjection
	\[\OO_K^\times\backslash\widehat\OO_K^\times\to\op{Gal}(K^{\mathrm{ab}}/H),\]
	which is what we wanted.
	% Now, we note that both sides of this map are compact Hausdorff topological groups, so a continuous isomorphism of groups automatically upgrades to an isomorphism of topological groups.
\end{proof}
As a first attempt to constructing $K^{\mathrm{ab}}$, we take the union of all the Hilbert class fields as we vary over orders $\OO\subseteq\OO_K$. This gets pretty close.
\begin{proposition} \label{prop:union-hilbert-class-fields}
	Fix a totally imaginary quadratic number field $K$. Let $K^?$ be the composite of the Hilbert class fields $H_\OO$ as $\OO$ varies over orders in $\OO_K$, and let $K^{??}$ be the composite of $K^?$ and $\QQ^{\mathrm{ab}}$. Then $\op{Gal}(K^{\mathrm{ab}}/K^{??})$ is a product of groups of order $2$.
\end{proposition}
\begin{proof}
	We characterize the subgroup $O\subseteq\OO_K^\times\backslash\widehat\OO_K^\times$ corresponding to $\op{Gal}(K^{\mathrm{ab}}/K^{??})\subseteq\op{Gal}(K^{\mathrm{ab}}/H)$. In particular, \Cref{lem:artin-ideles-fixing-hilbert} tells us that we would like to show that $\op{Art}_K(O)$ is a product of groups of order $2$. Quickly, we note that we merely check that $\op{Art}_K(O)$ is $2$-torsion, for then $O$ becomes a vector space over $\FF_2$ and thus a product of groups of order $2$.
	
	Anyway, we will have $x\in\OO_K^\times\backslash\widehat\OO_K^\times$ being in $O$ if and only if $\op{Art}_K(x)\in\op{Gal}(K^{\mathrm{ab}}/H)$ fixes $K^{??}$. This splits into two checks.
	\begin{enumerate}
		\item We require $\op{Art}_K(x)$ to fix $H_\OO$ for each order $\OO\subseteq\OO_K$. In other words, for each $f\ge1$, we must have $\op{Art}_K(x)$ fix $H_{\OO_f}$, where $\OO_f=\ZZ+f\OO_K$ is the order of conductor $f$. By definition of $H_{\OO_f}$, this is equivalent to having $x\in\OO_K^\times\backslash\widehat\OO_f^\times$ for each $f\ge1$. Now, we recall from the proof of \Cref{prop:better-order-class-group} that
		\[\widehat\OO_f^\times=\left\{y\in\widehat\OO_K^\times:y\pmod f\in(\ZZ/f\ZZ)^\times\right\}.\]
		Thus, fixing a representative $x\in\widehat\OO_K^\times$ in the quotient, we see that for each $f\ge1$, we are granted some unit $u_f\in\OO_K^\times$ such that $u_fx\in\widehat\OO_f^\times$. However, for $f$ sufficiently divisible, we see that $\OO_K^\times$ will embed into $\OO_K/(f)$, meaning that there will be at most one unit $u_f$ (up to sign) satisfying $u_fx\in\widehat\OO_f^\times$. Thus, we see that there will in fact be a single unit $u\in\OO_K^\times$ (not depending on $f$) such that $ux\in\widehat\OO_f^\times$ for all $f\ge1$, which upon taking the limit over $f$ means that $ux\in\widehat\ZZ^\times$. In other words, we are requiring that our $x\in\OO_K^\times\backslash\widehat\OO_K^\times$ actually comes from $\ZZ^\times\backslash\widehat\ZZ^\times$.
		\item We require $\op{Art}_K(x)$ to fix $\QQ^{\mathrm{ab}}$. In other words, we need $\op{Res}_{\QQ^{\mathrm{ab}}}\op{Art}_K(x)$ to be trivial. Thus, we recall that the square
		% https://q.uiver.app/#q=WzAsNCxbMCwxLCJcXFFRXlxcdGltZXNcXGJhY2tzbGFzaFxcQUFfXFxRUV5cXHRpbWVzIl0sWzAsMCwiS15cXHRpbWVzXFxiYWNrc2xhc2hcXEFBX0teXFx0aW1lcyJdLFsxLDAsIlxcb3B7R2FsfShLXntcXG1hdGhybXthYn19L0spIl0sWzEsMSwiXFxvcHtHYWx9KFxcUVFee1xcbWF0aHJte2FifX0vXFxRUSkiXSxbMiwzLCJcXG9we1Jlc30iXSxbMSwwLCJcXG9wIE5fe0svXFxRUX0iLDJdLFsxLDIsIlxcb3B7QXJ0fV9LIl0sWzAsMywiXFxvcHtBcnR9X1xcUVEiXV0=&macro_url=https%3A%2F%2Fraw.githubusercontent.com%2FdFoiler%2Fnotes%2Fmaster%2Fnir.tex
		\[\begin{tikzcd}
			{K^\times\backslash\AA_{K,f}^\times} & {\op{Gal}(K^{\mathrm{ab}}/K)} \\
			{\QQ^\times\backslash\AA_{\QQ,f}^\times} & {\op{Gal}(\QQ^{\mathrm{ab}}/\QQ)}
			\arrow["{\op{Art}_K}", from=1-1, to=1-2]
			\arrow["{\op N_{K/\QQ}}"', from=1-1, to=2-1]
			\arrow["{\op{Res}}", from=1-2, to=2-2]
			\arrow["{\op{Art}_\QQ}", from=2-1, to=2-2]
		\end{tikzcd}\]
		commutes. Now, in light of the previous check, we would like to check when some $ux\in\widehat\ZZ^\times$ (where $\widehat\ZZ^\times$ is implicitly embedded in $\widehat\OO_K^\times\subseteq\AA_{K,f}^\times$) vanishes in $\op{Gal}(\QQ^{\mathrm{ab}}/\QQ)$. Equivalently, we would like to check when
		\[\op N_{K/\QQ}(ux)=x^2\]
		lives in the kernel of the global Artin map $\op{Art}_\QQ$. However, the restriction of the global Artin map $\op{Art}_\QQ$ to $\widehat\ZZ^\times\to\op{Gal}(\QQ^{\mathrm{ab}}/\QQ)$ is injective (one could use \cite[Corollary~8.2.2]{neukirch-cohom} again).
		% simply an isomorphism of topological groups by the Kronecker--Weber theorem (simply take an inverse limit of the Galois groups of the cyclotomic extensions $\QQ(\zeta_n)$),
		Thus, $x^2\in\widehat\ZZ^\times$ only succeeds at being in the kernel of $\op{Art}_\QQ$ only if it is already trivial.
	\end{enumerate}
	The above two checks combine to show that $O$ is a $2$-torsion group, as required.
\end{proof}
\begin{remark}
	Recall from \Cref{cor:hilbert-class-field-by-cm} that $H_\OO$ is the field of definition of some $E\in Y_\OO$ (over $K$). Thus, one can view $K^?$ as the extension of $K$ obtained by adjoining all $j$-invariants $j(\OO)$ as $\OO\subseteq\OO_K$ varies over all orders.
\end{remark}
\begin{remark}
	Recall from the Kronecker--Weber theorem that $\QQ^{\mathrm{ab}}$ is the maximal cyclotomic extension of $\QQ$. Thus, we can also view $K^{??}$ the extension of $K^?$ achieved by adjoining all roots of unity.
\end{remark}

\subsection{The Composite of Torsion}
Again, fix an imaginary quadratic field $K$. In order to actually compute $K^{\mathrm{ab}}$, we add in some torsion. We will work with the maximal order $\OO_K$, for convenience, and we will let $H$ be the Hilbert class field. Fix an elliptic curve $E\in Y_{\OO_K}$. Let's quickly explain why adding torsion could help.
\begin{lemma}
	Fix an imaginary quadratic field $K$ with Hilbert class field $H$. For $E\in Y_{\OO_K}$, let $H_E$ be the extension obtained by adding in the coordinates of the torsion points $E_{\mathrm{tors}}$ of $E$ to $H$. Then $\op{Gal}(H_E/H)$ embeds in $\widehat\OO_K^\times$ via the Galois action on $E_{\mathrm{tors}}$.
\end{lemma}
\begin{proof}
	Quickly, we recall from \Cref{cor:hilbert-class-field-by-cm} that $E$ has a model over $H$. Let $TE$ be the product $\prod_pT_pE$ of the Tate modules. Because $E$ has complex multiplication by $\OO_K$, one may choose compatible level structure trivializations $E[n]\to\OO_K/n\OO_K$, eventually producing a level structure trivialization
	\[\tau\colon TE\to\widehat\OO_K.\]
	Now, each $\sigma\in\op{Gal}(\ov H/H)$ induces a compatible system of $\mathcal O$-linear morphisms $\sigma\colon E[n]\to E[n]$ for each $n$, so we get a canonical map
	\[\op{Gal}(\ov H/H)\to\op{Aut}_\OO TE.\]
	Further, note that $\sigma\in\op{Gal}(\ov H/H)$ fixes $TE$ if and only if it fixes all torsion points of $E$, which is equivalent to fixing $H_E$; thus, the above Galois representation factors as an injective map
	\[\op{Gal}(H_E/H)\into\op{Aut}_\OO TE.\]
	The result follows after using the level structure isomorphism $\tau$.
\end{proof}
\begin{remark}
	In particular, it follows that $H_E/H$ is an abelian extension.
\end{remark}
We would like to descend the extension $H_E/H$ to an abelian extension of $K$, which we will then check produces $K^{\mathrm{ab}}$. Ultimately, this will require us to produce a Galois representation of $\op{Gal}(K^{\mathrm{ab}}/K)$, but constructing such a thing from a fixed $E\in Y_{\OO_K}$ does not make sense because $E$ is not defined over $K$ in general. Instead, we will have to use the fact that $\op{Gal}(H/K)$ acts on $Y_{\OO_K}$ to use the entire torsor of elliptic curves in $Y_{\OO_K}$ to produce a Galois representation of $\op{Gal}(\ov K/K)$.

Eventually, we will want to make arguments akin to \Cref{prop:union-hilbert-class-fields} in order to prove our theorem, so it will help to start using class field theory.
\begin{notation}
	Fix an imaginary quadratic field $K$ with Hilbert class field $H$. For $E\in Y_{\OO_K}$, we define $\theta_E$ as the composite representation
	\[\AA_H^\times\stackrel{\op{Art}}\to\op{Gal}(\ov H/H)\to\op{Aut}_\OO TE\cong\widehat\OO_K^\times,\]
	where the last isomorphism is obtained by choosing a level structure isomorphism for $TE$.
\end{notation}
\begin{lemma} \label{lem:relate-theta-to-units}
	Fix an imaginary quadratic field $K$ with Hilbert class field $H$, and choose some $E\in Y_{\OO_K}$. For any $x\in\widehat\OO_H^\times$, we have
	\[\theta_E(x)\op N_{H/K}(x)\in\OO_K^\times.\]
\end{lemma}
\begin{proof}
	The main idea is to compare the Galois and class group actions on $(E,\tau)$, where $\tau\colon TE\to\widehat\OO_K$ is a choice of level structure trivialization.
	% The norm will arise from the diagram
	% % https://q.uiver.app/#q=WzAsNCxbMCwwLCJIXlxcdGltZXNcXGJhY2tzbGFzaFxcQUFfe0gsZn1eXFx0aW1lcyJdLFsxLDAsIlxcb3B7R2FsfShIXntcXG1hdGhybXthYn19L0gpIl0sWzAsMSwiS15cXHRpbWVzXFxiYWNrc2xhc2hcXEFBX3tLLGZ9XlxcdGltZXMiXSxbMSwxLCJcXG9we0dhbH0oS157XFxtYXRocm17YWJ9fS9LKSJdLFsxLDMsIlxcb3B7UmVzfSJdLFswLDIsIlxcb3AgTl97SC9LfSIsMl0sWzAsMSwiXFxvcHtBcnR9X0giXSxbMiwzLCJcXG9we0FydH1fSyIsMl1d&macro_url=https%3A%2F%2Fraw.githubusercontent.com%2FdFoiler%2Fnotes%2Fmaster%2Fnir.tex
	% \[\begin{tikzcd}
	% 	{H^\times\backslash\AA_{H,f}^\times} & {\op{Gal}(H^{\mathrm{ab}}/H)} \\
	% 	{K^\times\backslash\AA_{K,f}^\times} & {\op{Gal}(K^{\mathrm{ab}}/K)}
	% 	\arrow["{\op{Art}_H}", from=1-1, to=1-2]
	% 	\arrow["{\op N_{H/K}}"', from=1-1, to=2-1]
	% 	\arrow["{\op{Res}}", from=1-2, to=2-2]
	% 	\arrow["{\op{Art}_K}", from=2-1, to=2-2]
	% \end{tikzcd}\]
	% which commutes by properties of the global Artin map.

	Let's be more explicit. The following computation is potentially off by a sign, but it is not so significant. By \Cref{thm:main-cm-level}, we see that
	\[\op{Art}_K(\op N_{H/K}(x))(E,\tau)=(\OO_K,\op N_{H/K}(x))\star(E,\tau).\]
	(In particular, $\op N_{H/K}(x)$ is trivial in $\op{Cl}(\OO_K)$, so the corresponding line bundle is $\OO_K$; we are using $x$ for our choice of trivialization.) We now compute both sides.
	\begin{itemize}
		\item On the left, we find $\op{Art}_K(\op N_{H/K}(x))=\op{Art}_H(x)$ by compatibility of the global Artin map in extensions. In particular, this Galois element fixes $H$, so it fixes our chosen model $E$ over $H$. However, it does adjust the level structure isomorphism to $\left(E,\tau\circ\op{Art}_K(\op N_{H/K}(x))^{-1}\right)$ by definition of this Galois action.
		\item On the right, we note $\OO\star E=E$, so the elliptic curve is still fixed. As for the level structure isomorphism, we have
		\[\arraycolsep=1.4pt\begin{array}{cccccccccc}
			E[N] &\to& \op{Hom}_\OO(\OO/N\OO,E[N]) &\to& \op{Hom}_\OO(\OO/N\OO,\OO/N\OO) &\to& \OO/N\OO \\
			p &\mapsto& (1\mapsto p) &\mapsto& (1\mapsto\tau(\op N_{H/K}(x)p)) &\mapsto& \tau(\op N_{H/K}(x)p)
		\end{array}\]
		by tracking through the construction.
	\end{itemize}
	In total, we see that
	\[\left(E,\tau\circ\op{Art}_K(\op N_{H/K}(x))^{-1}\right)=\left(E,\tau\circ\op N_{H/K}(x)\right),\]
	so there is an automorphism $u\in\op{Aut}(E)$ such that
	\[\tau\circ\op{Aut}_H(x)^{-1}=\tau\circ\op N_{H/K}(x)\circ u.\]
	Undoing $\tau$ and applying the definition of $\theta_E$, we find that $\theta_E(x)\op N_{H/K}(x)=u^{-1}$, so the result follows. (Note $\op{Aut}(E)=\OO_K^\times$!)
\end{proof}
The above lemma allows us to make the following definition.
\begin{notation}
	Fix an imaginary quadratic field $K$ with Hilbert class field $H$, and choose some $E\in Y_{\OO_K}$. Then we define the homomorphism $\rho_E\colon\widehat\OO_H^\times\to\OO_K^\times$ by $\rho_E\coloneqq\theta_E\cdot\op N_{H/K}$.
\end{notation}
We now see that the presence of these units $\OO_K^\times$ (i.e., nontrivial automorphisms of $E$) will present a problem when we are trying to glue together the various Galois actions. Thus, our next step is to quotient them out. This presents the following cases.
\begin{lemma}
	Fix an imaginary quadratic field $K$ with Hilbert class field $H$, and choose some fixed $E\in Y_{\OO_K}$. Then there is a model of $E$ over $H$ equipped with a cyclic projection $\pi\colon E\to\PP^1$ with Galois group $\OO_K^\times=\op{Aut}(E)$.
\end{lemma}
\begin{proof}
	Our proof will be explicit, though it will require some casework. Everything in sight is characteristic $0$, so we may write out a model for $E$ over $H$ as cut out from $\PP^2$ by the homogeneous equation $Y^2Z=X^3+aXZ^2+bZ^3$. It will be helpful to note that there is a map $x\colon E\to\PP^1$ given on affine points by $(x,y)\mapsto x$ and mapping together the points at infinity. We now have the following cases.
	\begin{itemize}
		\item Suppose that $K\notin\{\QQ(i),\QQ(\zeta_3)\}$. Then $\OO_K^\times=\{\pm1\}$, and we define $\pi$ to be $x$. Namely, we see that $\pi([X:Y:Z])=\pi([X':Y':Z'])$ if and only if $[X:Y:Z]=[X:\pm Y:Z]$, which is equivalent to $[X:Y:Z]$ and $[X':Y':Z']$ being in the same orbit by $\OO_K^\times=\{\pm1\}$.

		\item Suppose that $K=\QQ(i)$ so that $\OO_K=\{\pm1,\pm i\}$. We work with the explicit model $E_0\colon Y^2Z=X^3+X$, and we note that $i\in\OO_K$ may act by fixing the point at infinity and sending $i\colon(x,y)\mapsto(-x,iy)$. (In particular, we see that this is an endomorphism of $E$ of order $4$, so it induces an injection $\ZZ[i]\into\op{End}(E_0)$, which becomes an isomorphism because $\op{End}(E_0)$ is either $\ZZ$ or an order in an imaginary quadratic field.) As such, we see that affine points $(x,y)$ and $(x',y')$ are in the same orbit if and only if $x=\pm x'$.
		
		Thus, in this case, we set $\pi$ to be $x^2$ (namely, the value of $x^2$ on affine points, still sending the point to infinity to $[1:0]\in\PP^1$), and we see that $\pi([X:Y:Z])=\pi([X':Y':Z'])$ if and only if these are both the point at infinity or having these be affine points $(x,y)$ and $(x',y')$ with $x=\pm x'$. This last condition is equivalent to having our points be in the same orbit by $\OO_K^\times$.

		\item Suppose $K=\QQ(\zeta_3)$ so that $\OO_K=\left\{\pm1,\pm\zeta_3,\pm\zeta_3^2\right\}$. We work with the explicit model $E_0\colon Y^2Z=X^3+Z^3$, and we note that $\zeta_3\in\OO_K$ may act by fixing the point at infinity and sending $\zeta_3\colon(x,y)\mapsto(\zeta_3x,y)$. (In particular, this is certainly an endomorphism of order $3$, which then induces the needed injection $\ZZ[\zeta_3]\into\op{End}(E_0)$.) As such, we see that affine points $(x,y)$ and $(x',y')$ are in the same orbit if and only if $x^3=(x')^3$.

		Thus, in this case, we set $\pi$ to be $x^3$ as in the previous case, and we conclude in the same way that $\pi$ is a cyclic projection with Galois group $\OO_K^\times$.
		\qedhere
	\end{itemize}
\end{proof}
\begin{remark}
	It is worth noting that the construction of $\pi$ is totally explicit as soon as we have a chosen model for $E$. This will later enable us to rather explicitly write down abelian extensions of $K$ from torsion points of $E$.
\end{remark}
\begin{remark}
	In fact, provided we are working with models of $E$ in short Weierstrass form $Y^2Z=X^3+aXZ^2+bZ^3$, the specific choice of model in the last two cases doesn't matter to the construction of $\pi$.
	\begin{itemize}
		\item When $K=\QQ(i)$, then we see that we are looking for elliptic curves with $j$-invariant $1728$, which all look like $Y^2Z=X^3+aXZ^2$. Of course, we then see that we can still take $i\colon(x,y)\mapsto(ix,-y)$, so the same argument goes through with the same $\pi$.
		\item When $K=\QQ(\zeta_3)$, then now we want elliptic curves with $j$-invariant $0$, which all look like $Y^2Z=X^3+bZ^3$. The action by $\zeta_3$ can still be taken as $(x,y)\mapsto(\zeta_3x,y)$, so the same argument still goes through with the same $\pi$.
	\end{itemize}
\end{remark}
\begin{notation}
	Fix an imaginary quadratic field $K$ with Hilbert class field $H$, and choose some fixed $E\in Y_{\OO_K}$. Choose a model of $E$ over $H$, and choose a cyclic projection $\pi\colon E\to\PP^1$ with Galois group $\OO_K^\times$. Then we set
	\[L(E)\coloneqq H(\pi(E_{\mathrm{tors}})).\]
\end{notation}
More precisely, we are asking to adjoin the affine points, which can be identified with elements of $\ov H$ already. (The point at infinity does not help us because it is defined over $H$.)
% \begin{lemma} \label{lem:l-e-independent-cm}
% 	Fix an imaginary quadratic field $K$ with Hilbert class field $H$. Then the extension $L(E)$ does not depend on the choice of $E\in Y_{\OO_K}$ (or model of $E$).
% \end{lemma}
% \begin{proof}
% 	Given $E\in Y_{\OO_K}$ with a given model over $H$, the assertion that some $\sigma\in\op{Gal}(\ov H/H)$ fixes $L_E$ is equivalent to $\sigma$ fixing $\pi(E[N])$ for each $N$ where $\pi\colon E\to\PP^1$ is the associated quotient by $\OO_K^\times$; tracking the Galois action back up to $E$, we see that we are asking for $\sigma$ to fix the quotient $E[N]/\OO_K^\times$. Equivalently, we are asking for $\sigma$ to fix $TE/\OO_K^\times$.

% 	Thus, to show the lemma, we pick up two $E_1,E_2\in Y_{\OO_K}$, give them models over $H$, and we would like to show that $\sigma$ fixes $TE/\OO_K^\times$ if and only if it fixes $TE'/\OO_K^\times$. By symmetry, it's enough to check one of the inclusions, for which we use an isogeny. Well, over $\CC$, we see directly that $E$ and $E'$ are isogenous, so we can find some isogeny $\varphi\colon E\to E'$ defined over a finite extension of $H$. Because $\varphi\colon TE\to TE'$ is surjective, it will be enough to compare $\varphi\sigma(p)$ and $\sigma\varphi(p)$. Well, choose an isogeny $E'\to E$, and then we see that composition with this isogeny will Well, $\sigma$ is an isomorphism and hence preserves orders, so we see that $\sigma^{-1}\varphi\sigma(p)$ and $\varphi(p)$ have the same order, so there will be a unit in $\OO_K^\times$ taking one to the other.
% \end{proof}
Anyway, we are now ready to prove our theorem.
\begin{theorem}
	Fix an imaginary quadratic field $K$ with Hilbert class field $H$, and choose some fixed $E\in Y_{\OO_K}$. Then $L(E)$ is the maximal abelian extension of $K$.
\end{theorem}
\begin{proof}
	% As discussed in the proof of \Cref{lem:l-e-independent-cm},
	Note that fixing $L(E)$ is the same as fixing torsion up to units, which is equivalent to fixing $TE/\OO_K^\times$. In other words, the extension $L(E)$ of $H$ can be described as given by the kernel of
	\[\ov\rho\colon\op{Gal}(\ov H/H)\to\op{Aut}_{\OO_K}\left(TE/\OO_K^\times\right).\]
	We note that $\ov\rho$ in fact factors through $\op{Aut}_{\OO_K}TE$, which is abelian (it is $\widehat\OO_K^\times$), so we may as well pass to the abelianization, meaning that $L(E)$ can be described as given by the kernel of
	\[\ov\rho\colon\op{Gal}(H^{\mathrm{ab}}/H)\to\op{Aut}_{\OO_K}\left(TE/\OO_K^\times\right).\]
	We now use class field theory. Because $\op{Art}_H\colon H^\times\backslash\AA_{H,f}^\times\to\op{Gal}(H^{\mathrm{ab}}/H)$ is surjective, it is enough to check that $\ov\rho(\op{Art}_H(x))$ is trivial if and only if $\op{Art}_H(x)|_{K^{\mathrm{ab}}}$ is trivial. We now compare these two conditions.
	\begin{itemize}
		\item Note that $\ov\rho(\op{Art}_H(x))$ is trivial if and only if the Galois action by $\op{Art}_H(x)$ (which we recall is succinctly given by $\theta_E(x)$) only ever adjusts elements of $E$ by units. By \Cref{lem:relate-theta-to-units}, it is equivalent to ask for $\op N_{H/K}(x)\in\OO_K^\times$.
		\item By compatibility of the global Artin maps, $\op{Art}_H(x)|_{K^{\mathrm{ab}}}$ is trivial if and only if $\op{Art}_K(\op N_{H/K}(x))$ is trivial. Note $\op{Art}_K(\op N_{H/K}(x))$ already fixes $H$, so we may as well take $\op N_{H/K}(x)\in\OO_K^\times\backslash\widehat\OO_K^\times$ by definition of the Hilbert class field. But now \cite[Corollary~8.2.2]{neukirch-cohom} explains that the Artin map being trivial on such an element $\op N_{H/K}(x)$ requires $\op N_{H/K}(x)\in\OO_K^\times$.
	\end{itemize}
	The above two checks complete the proof.
\end{proof}
\begin{remark}
	We note that the above theorem implies that $L(E)$ is independent of the choice of $E$. One can also imagine showing this more directly by noting that two $E,E'\in Y_{\OO_K}$ will have an isogeny $\varphi\colon E\to E'$ between them (as this can be seen on the level of $\CC$-points). This isogeny allows us to relate the Galois action on the torsion of $E$ to the Galois action on $E'$, provided that we can ensure that $\varphi$ has a relatively small field of definition, which is technically not obvious.
\end{remark}

\end{document}