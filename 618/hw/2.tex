% !TEX root = ../notes.tex

\documentclass[../notes.tex]{subfiles}

\begin{document}

\section{Homework 2}
In this section, we work out some examples related to Dieudonn\'e modules. Before going any further, we set up some notation. Throughout, $k$ denotes a perfect field of positive characteristic $p$, and $W$ denotes the ring scheme of Witt vectors. Whenever appropriate, $F$ and $V$ are the Frobenius and Verschiebung morphisms, respectively. As in our Dieudonn\'e theory, it will be helpful to define the pro-group
\[W_\infty\coloneqq\colimit W_n.\]

\subsection{Some Duals}
Before getting into any heavy computation, we give the Hopf algebra structure of the algebraic groups $\mathbb G_a$ and $\mathbb G_m$.
\begin{lemma}
	For any ring $k$, we give the Hopf algebra structure on $\mathbb G_a$.
\end{lemma}
\begin{proof}
	The group scheme $\mathbb G_a$ is defined on the level of $R$-points by returning the additive group $\mathbb G_a(R)\coloneqq R$. On the level of sets, we thus see that $\mathbb G_a$ is represented by the affine scheme $\Spec k[t]$, where the point is that
	\[\mathbb G_a(R)=\op{Mor}_k(\Spec R,\mathbb G_a)=\op{Hom}_k(k[t],R)=R,\]
	where the last equality is given by $f\mapsto f(t)$.

	It remains to give the comultiplication and coidentity on $k[t]$. The coidentity $\Spec k\to\mathbb G_a$ needs to pick out $0\in\mathbb G_a(R)$ for each $R$, so it will be given by the map $k[t]\to k$ defined as $t\mapsto0$. The comultiplication $m\colon\mathbb G_a\times\mathbb G_a\to\mathbb G_a$ corresponds to the addition map $R\times R\to R$ on $R$-points, which on the level of the schemes comes from the ring map $m^\sharp\colon k[t]\to k[t]\otimes_kk[t]$ by
	\[m^\sharp(t)\coloneqq(t\otimes1)+(1\otimes t).\]
	Indeed, a pair $(r,s)\in R\times R$ corresponds to a ring homomorphism $\varphi^\sharp\colon k[t]\otimes_kk[t]\to R$ given by $\varphi^\sharp(t\otimes1)=r$ and $\varphi^\sharp(1\otimes t)=s$. Then we require that $m\circ\varphi$ to correspond to the $R$-point $r+s\in\mathbb G_a(R)$, which can be computed via $(m\circ\varphi)^\sharp(t)=r+s$.
\end{proof}
\begin{lemma}
	For any ring $k$, we give the Hopf algebra structure on $\mathbb G_m$.
\end{lemma}
\begin{proof}
	We proceed as in the previous lemma, but we will be a bit briefer because many of the arguments are similar.
	\begin{itemize}
		\item On $R$-points, we have $\mathbb G_m(R)=R^\times$, so we find that $\mathbb G_m=\Spec k\left[t,t^{-1}\right]$ as schemes.
		\item For the coidentity, we need to pick out $1\in\mathbb G_m(R)$ for each $R$, so this is given by the scheme map $\Spec k\to\mathbb G_m$ defined by the ring map $k\left[t,t^{-1}\right]\to k$ via $t\mapsto1$.
		\item For the comultiplication, we need to define the ring map which gives the natural transformation $\mathbb G_m\times\mathbb G_m\to\mathbb G_m$ defined by multiplication. Well, $m^\sharp\colon k\left[t,t^{-1}\right]\to k\left[t,t^{-1}\right]\otimes k\left[t,t^{-1}\right]$ defined by
		\[m^\sharp(t)\coloneqq(t\otimes t)\]
		can be checked to work.
		\qedhere
	\end{itemize}
\end{proof}
We now compute some duals of some finite groups.
\begin{exercise}
	For any ring $k$ of positive characteristic $p$, we show that the Cartier dual of the finite group $\alpha_p$ is $\alpha_p$.
\end{exercise}
\begin{proof}
	Because we should do it at least once, we prove this using the definition of Cartier duality by characters, working directly with the algebraic group $\alpha_p^\lor(R)\coloneqq\op{Hom}_R(\alpha_{p,R},\mathbb G_{m,R})$.
	\begin{enumerate}
		\item We begin by giving the Hopf algebra structure. By definition, $\alpha_p$ is given on $R$-points by
		\[\alpha_p(R)\coloneqq\left\{r\in R:r^p=0\right\}.\]
		In particular, we see that the inclusion $\alpha_p(R)\subseteq R$ defines an embedding $\alpha_p\into\mathbb G_a$. For example, the definition of $\alpha_p$ implies that $\alpha_p$ is represented by the scheme $\Spec k[t]/\left(t^p\right)$, where $f\colon\Spec R\to\Spec k[t]/\left(t^p\right)$ corresponds to $f^\sharp(t)\in\alpha_p(R)$. Thus, the natural inclusion $\iota\colon\alpha_p\into\mathbb G_a$ is seen to be given on schemes by $\iota^\sharp(t)=t$ (e.g., by plugging in some test ring $R$). Because we already know the coidentity and comultiplication on $\mathbb G_a$, the same data now passes to the closed subscheme: our coidentity $\Spec k\to\alpha_p$ is given on rings by $t\mapsto0$, and our comultiplication $\alpha_p\times\alpha_p\to\alpha_p$ is given by $t\mapsto(t\otimes1)+(1\otimes t)$.

		\item We now begin our duality computation. For a general ring $k$ of positive characteristic $p$, we would like to compute the group scheme $\op{Hom}_k(\alpha_p,\mathbb G_m)$, for which by base-changing it is enough to just compute the $k$-points for now. This embeds into the set
		\[\op{Mor}_k(\alpha_p,\mathbb G_m)=\mathbb G_m\left(k[t]/\left(t^p\right)\right),\]
		which of course consists of the units in $k[t]/\left(t^p\right)$. Well, because the Frobenius is the zero map on $k[t]/\left(t^p\right)$, we see that any polynomial $f(t)\in k[t]/\left(t^p\right)$ has $f(t)^p=f(0)$ and therefore is a unit if and only if $f(0)\in k^\times$.

		\item Let's unwind the morphism $\alpha_p\to\mathbb G_m$ given by some such polynomial $f$ so that we can check when it is a homomorphism of groups. On $R$-points, we may consider some $r\in\alpha_p(R)$, which corresponds to the ring homomorphism $k[t]/\left(t^p\right)\to R$ defined by $t\mapsto r$. Then composing with $f$ appropriately sends this to the ring homomorphism $k\left[t,t^{-1}\right]\to R$ given by $t\mapsto f(r)$, which of course corresponds to $f(r)\in\mathbb G_m(R)$. Thus, we are interested in when the map
		\[\arraycolsep=1.4pt\begin{array}{cccc}
			\alpha_p &\to& \mathbb G_m \\
			r &\mapsto& f(r)
		\end{array}\]
		is actually a group homomorphism.
		
		\item We now check which $f$ produce a homomorphism. For example, we must have $f(0)=1$. More generally, we must have $f(r+s)=f(r)f(s)$ to hold for any $r,s\in\alpha_p(R)$; this implies having $f(s+t)=f(t)f(s)$ for $s,t\in k[s,t]/\left(s^p,t^p\right)$, and this latter condition can be seen to be sufficient as well. Now, write $f(t)=\sum_{i=0}^{\infty}a_it^i$ (for example, $a_0=1$) so that
		\[f(s+t)=\sum_{i,j\ge0}a_{i+j}\binom{i+j}js^it^j\qquad\text{and}\qquad f(s)f(t)=\sum_{i,j\ge0}a_ia_js^it^j.\]
		In particular, looking at the coefficients of $s^it$, this implies that $a_{i+1}(i+1)=a_ia_1$ for each $i\ge0$, so we are forced to have
		\[f(t)=\sum_{i=0}^{p-1}\frac{a_1^i}{i!}t^i\]
		and so $a_1^p=0$. On the other hand, for any $a\in\alpha_p(R)$, the polynomial $f_a(t)\coloneqq\sum_{i=0}^{p-1}\frac{a^i}{i!}t^i$ can be seen to satisfy $f(s+t)=f(s)f(t)$: the coefficient of $s^it^j$ is
		\[a^{i+j}\cdot\frac1{(i+j)!}\binom{i+j}j=a^{i+j}\cdot\frac{1}{i!j!}.\]

		\item At this point, we have produced a bijection $\alpha_p(k)\to\op{Hom}_k(\alpha_p,\mathbb G_m)$ given by $a\mapsto f_a$. We will be done as soon as we check that this upgrades to a natural isomorphism of groups. To check that this is an isomorphism, we must check that $f_{a+b}(t)=f_a(t)f_b(t)$, which is true because the coefficient of $t^i$ is
		\[\frac{(a+b)^i}{i!}=\sum_{m+n=i}\frac{a^{m}b^n}{m!n!}.\]
		Lastly, for naturality, we suppose that we have a ring homomorphism $\varphi\colon k\to k'$, and we note that the diagram
		% https://q.uiver.app/#q=WzAsOCxbMCwwLCJcXGFscGhhX3AoaykiXSxbMCwxLCJcXGFscGhhX3AoaycpIl0sWzEsMCwiXFxvcHtIb219X2soXFxhbHBoYV9wLFxcbWF0aGJiIEdfbSkiXSxbMiwwLCJhIl0sWzMsMCwiKHJcXG1hcHN0byBmX2EocikpIl0sWzEsMSwiXFxvcHtIb219X3trJ30oXFxhbHBoYV9wLFxcbWF0aGJiIEdfbSkiXSxbMiwxLCJcXHZhcnBoaShhKSJdLFszLDEsIihyXFxtYXBzdG8gZl97XFx2YXJwaGkoYSl9KHIpKSJdLFswLDJdLFswLDFdLFsxLDVdLFsyLDVdLFszLDYsIiIsMCx7InN0eWxlIjp7InRhaWwiOnsibmFtZSI6Im1hcHMgdG8ifX19XSxbNiw3LCIiLDAseyJzdHlsZSI6eyJ0YWlsIjp7Im5hbWUiOiJtYXBzIHRvIn19fV0sWzMsNCwiIiwyLHsic3R5bGUiOnsidGFpbCI6eyJuYW1lIjoibWFwcyB0byJ9fX1dLFs0LDcsIiIsMix7InN0eWxlIjp7InRhaWwiOnsibmFtZSI6Im1hcHMgdG8ifX19XV0=&macro_url=https%3A%2F%2Fraw.githubusercontent.com%2FdFoiler%2Fnotes%2Fmaster%2Fnir.tex
		\[\begin{tikzcd}[cramped]
			{\alpha_p(k)} & {\op{Hom}_k(\alpha_p,\mathbb G_m)} & a & {(r\mapsto f_a(r))} \\
			{\alpha_p(k')} & {\op{Hom}_{k'}(\alpha_p,\mathbb G_m)} & {\varphi(a)} & {(r\mapsto f_{\varphi(a)}(r))}
			\arrow[from=1-1, to=1-2]
			\arrow[from=1-1, to=2-1]
			\arrow[from=1-2, to=2-2]
			\arrow[maps to, from=1-3, to=1-4]
			\arrow[maps to, from=1-3, to=2-3]
			\arrow[maps to, from=1-4, to=2-4]
			\arrow[from=2-1, to=2-2]
			\arrow[maps to, from=2-3, to=2-4]
		\end{tikzcd}\]
		commutes.
		\qedhere
	\end{enumerate}
\end{proof}
\begin{exercise}
	For any $n\ge1$, we show the Cartier dual of the group scheme $\ZZ/n\ZZ$ is $\mu_n$.
\end{exercise}
\begin{proof}
	Once again, because it is good practice, we prove using $(\ZZ/n\ZZ)^\lor(R)\coloneqq\op{Hom}_R(\ZZ/n\ZZ,\mathbb G_m)$.
	\begin{enumerate}
		\item As before, we begin by giving the Hopf algebra structure on $\ZZ/n\ZZ$. We will be more brief. Recall
		\[\ZZ/n\ZZ(R)=\op{Mor}_{\mathrm{cts}}(\Spec R,\ZZ/n\ZZ),\]
		which becomes a group via the addition on $\ZZ/n\ZZ$. Thus, the functor $\ZZ/n\ZZ\colon\mathrm{Alg}_k\to\mathrm{Set}$ is represented by the affine scheme
		\[\bigsqcup_{*\in\ZZ/n\ZZ}\Spec k=\Spec k^{\ZZ/n\ZZ}.\]
		We may alternatively write $k^{\ZZ/n\ZZ}$ as $k_0\times\cdots\times k_{n-1}$, where each $k_\bullet$ equals $k$. The coidentity $\Spec k\to\Spec k^{\ZZ/n\ZZ}$ needs to pick out $0\in\ZZ/n\ZZ(R)$ for each $R$, so it is given by embedding $k\into k_0\into k^{\ZZ/n\ZZ}$. Lastly, the comultiplication $m$ is given on the $k_\bullet$s by gluing together the diagonal embeddings
		\[k_i\into\bigoplus_{i_1+i_2=i}k_{i_1}\otimes k_{i_2}\subseteq k^{\ZZ/n\ZZ}\otimes k^{\ZZ/n\ZZ},\]
		where indices are taken$\pmod n$. To check that this works, we choose $r,s\in\ZZ/n\ZZ(R)$, which correspond to some maps $r^\sharp,s^\sharp\colon k^{\ZZ/n\ZZ}\to R$, and we can see that $m^\sharp\circ(r,s)$ is adding appropriately by checking on the $k_\bullet$s.

		\item We calculate $\op{Mor}_k(\ZZ/n\ZZ,\mathbb G_m)$ as
		\[\mathbb G_m(\ZZ/n\ZZ)=\left(k^{\ZZ/n\ZZ}\right)^\times.\]
		Thus, we produce a morphism $\ZZ/n\ZZ\to\mathbb G_m$ for each element $(a_0,\ldots,a_{n-1})$ where $a_i\in k^\times$ for each $i$. To be explicit, we note that any such unit $(a_0,\ldots,a_{n-1})$ sends $f\in\ZZ/n\ZZ(R)$ to $f(a_0,\ldots,a_{n-1})\in R^\times$.

		\item 
	\end{enumerate}
\end{proof}
% \begin{exercise}
% 	For any $n\ge1$, we show the Cartier dual of the group scheme $\mu_n$ is $\ZZ/n\ZZ$.
% \end{exercise}
% \begin{proof}
% 	Once again, because it is good practice, we prove this using the definition $\mu_n^\lor(R)\coloneqq\op{Hom}_R(\mu_n,\mathbb G_m)$.
% 	\begin{enumerate}
% 		\item As before, we begin by giving the Hopf algebra structure on $\mu_n$. We will be more brief. Recall
% 		\[\mu_n(R)=\left\{r\in R^\times:r^n=1\right\},\]
% 		so $\mu_n$ has a natural inclusion into $\mathbb G_m$. In particular, we see that $\mu_n=\Spec k[t]/\left(t^n-1\right)$, and the inclusion $\mu_n\into\mathbb G_m$ is a closed embedding given on rings by the map $k\left[t,t^{-1}\right]\to k[t]/\left(t^n-1\right)$ defined by $t\mapsto t$. The coidentity and comultiplication data are just inherited from $\mathbb G_m$.

% 		\item We calculate $\op{Hom}_k(\mu_n,\mathbb G_m)$ as
% 		\[\mathbb G_m(\mu_n)=\left(\frac{k[t]}{\left(t^n-1\right)}\right)^\times.\]
% 		Thus, we produce a morphism $\mu_n\to\mathbb G_m$ for each unit polynomial $f(t)\in k[t]/\left(t^n-1\right)$, which amounts to requiring $\left(f(t),t^n-1\right)=k[t]$. As before, we note that such a polynomial $f$ defines the natural transformation $\mu_n\to\mathbb G_m$ given on $R$-points by $r\mapsto f(r)$.

% 		\item 
% 	\end{enumerate}
% \end{proof}

\end{document}