% !TEX root = ../notes.tex

\documentclass[../notes.tex]{subfiles}

\begin{document}

\section{Homework 2}
In this section, we work out some examples related to Dieudonn\'e modules. Before going any further, we set up some notation. Throughout, $k$ denotes a perfect field of positive characteristic $p$, and $W$ denotes the ring scheme of Witt vectors. Whenever appropriate, $F$ and $V$ are the Frobenius and Verschiebung morphisms, respectively. As in our Dieudonn\'e theory, it will be helpful to define the pro-group
\[W_\infty\coloneqq\colimit W_n.\]
We remark that various parts of the following discussion were motivated by the discussion \href{https://people.math.ethz.ch/~pink/ftp/FGS/CompleteNotes.pdf}{here}.

\subsection{Some Cartier Duals}
Before getting into any heavy computation, we give the Hopf algebra structure of the algebraic groups $\mathbb G_a$ and $\mathbb G_m$.
\begin{lemma}
	For any ring $k$, we give the Hopf algebra structure on $\mathbb G_a$.
\end{lemma}
\begin{proof}
	The group scheme $\mathbb G_a$ is defined on the level of $R$-points by returning the additive group $\mathbb G_a(R)\coloneqq R$. On the level of sets, we thus see that $\mathbb G_a$ is represented by the affine scheme $\Spec k[t]$, where the point is that
	\[\mathbb G_a(R)=\op{Mor}_k(\Spec R,\mathbb G_a)=\op{Hom}_k(k[t],R)=R,\]
	where the last equality is given by $f\mapsto f(t)$.

	It remains to give the comultiplication and coidentity on $k[t]$. The coidentity $\Spec k\to\mathbb G_a$ needs to pick out $0\in\mathbb G_a(R)$ for each $R$, so it will be given by the map $k[t]\to k$ defined as $t\mapsto0$. The comultiplication $m\colon\mathbb G_a\times\mathbb G_a\to\mathbb G_a$ corresponds to the addition map $R\times R\to R$ on $R$-points, which on the level of the schemes comes from the ring map $m^\sharp\colon k[t]\to k[t]\otimes_kk[t]$ by
	\[m^\sharp(t)\coloneqq(t\otimes1)+(1\otimes t).\]
	Indeed, a pair $(r,s)\in R\times R$ corresponds to a ring homomorphism $\varphi^\sharp\colon k[t]\otimes_kk[t]\to R$ given by $\varphi^\sharp(t\otimes1)=r$ and $\varphi^\sharp(1\otimes t)=s$. Then we require that $m\circ\varphi$ to correspond to the $R$-point $r+s\in\mathbb G_a(R)$, which can be computed via $(m\circ\varphi)^\sharp(t)=r+s$.
\end{proof}
\begin{lemma}
	For any ring $k$, we give the Hopf algebra structure on $\mathbb G_m$.
\end{lemma}
\begin{proof}
	We proceed as in the previous lemma, but we will be a bit briefer because many of the arguments are similar.
	\begin{itemize}
		\item On $R$-points, we have $\mathbb G_m(R)=R^\times$, so we find that $\mathbb G_m=\Spec k\left[t,t^{-1}\right]$ as schemes.
		\item For the coidentity, we need to pick out $1\in\mathbb G_m(R)$ for each $R$, so this is given by the scheme map $\Spec k\to\mathbb G_m$ defined by the ring map $k\left[t,t^{-1}\right]\to k$ via $t\mapsto1$.
		\item For the comultiplication, we need to define the ring map which gives the natural transformation $\mathbb G_m\times\mathbb G_m\to\mathbb G_m$ defined by multiplication. Well, $m^\sharp\colon k\left[t,t^{-1}\right]\to k\left[t,t^{-1}\right]\otimes k\left[t,t^{-1}\right]$ defined by
		\[m^\sharp(t)\coloneqq(t\otimes t)\]
		can be checked to work.
		\qedhere
	\end{itemize}
\end{proof}
We now compute some duals of some finite groups.
\begin{exercise} \label{exe:cartier-dual-alpha-p}
	For any ring $k$ of positive characteristic $p$, we show that the Cartier dual of the finite group $\alpha_p$ is $\alpha_p$.
\end{exercise}
\begin{proof}
	Because we should do it at least once, we prove this using the definition of Cartier duality by characters, working directly with the algebraic group $\alpha_p^\lor(R)\coloneqq\op{Hom}_R(\alpha_{p,R},\mathbb G_{m,R})$.
	\begin{enumerate}
		\item We begin by giving the Hopf algebra structure. By definition, $\alpha_p$ is given on $R$-points by
		\[\alpha_p(R)\coloneqq\left\{r\in R:r^p=0\right\}.\]
		In particular, we see that the inclusion $\alpha_p(R)\subseteq R$ defines an embedding $\alpha_p\into\mathbb G_a$. For example, the definition of $\alpha_p$ implies that $\alpha_p$ is represented by the scheme $\Spec k[t]/\left(t^p\right)$, where $f\colon\Spec R\to\Spec k[t]/\left(t^p\right)$ corresponds to $f^\sharp(t)\in\alpha_p(R)$. Thus, the natural inclusion $\iota\colon\alpha_p\into\mathbb G_a$ is seen to be given on schemes by $\iota^\sharp(t)=t$ (e.g., by plugging in some test ring $R$). Because we already know the coidentity and comultiplication on $\mathbb G_a$, the same data now passes to the closed subscheme: our coidentity $\Spec k\to\alpha_p$ is given on rings by $t\mapsto0$, and our comultiplication $\alpha_p\times\alpha_p\to\alpha_p$ is given by $t\mapsto(t\otimes1)+(1\otimes t)$.

		\item We now begin our duality computation. For a general ring $k$ of positive characteristic $p$, we would like to compute the group scheme $\op{Hom}_k(\alpha_p,\mathbb G_m)$, for which by base-changing it is enough to just compute the $k$-points for now. This embeds into the set
		\[\op{Mor}_k(\alpha_p,\mathbb G_m)=\mathbb G_m\left(k[t]/\left(t^p\right)\right),\]
		which of course consists of the units in $k[t]/\left(t^p\right)$. Well, because the Frobenius is the zero map on $k[t]/\left(t^p\right)$, we see that any polynomial $f(t)\in k[t]/\left(t^p\right)$ has $f(t)^p=f(0)$ and therefore is a unit if and only if $f(0)\in k^\times$.

		\item Let's unwind the morphism $\alpha_p\to\mathbb G_m$ given by some such polynomial $f$ so that we can check when it is a homomorphism of groups. On $R$-points, we may consider some $r\in\alpha_p(R)$, which corresponds to the ring homomorphism $k[t]/\left(t^p\right)\to R$ defined by $t\mapsto r$. Then composing with $f$ appropriately sends this to the ring homomorphism $k\left[t,t^{-1}\right]\to R$ given by $t\mapsto f(r)$, which of course corresponds to $f(r)\in\mathbb G_m(R)$. Thus, we are interested in when the map
		\[\arraycolsep=1.4pt\begin{array}{cccc}
			\alpha_p &\to& \mathbb G_m \\
			r &\mapsto& f(r)
		\end{array}\]
		is actually a group homomorphism.
		
		\item We now check which $f$ produce a homomorphism. For example, we must have $f(0)=1$. More generally, we must have $f(r+s)=f(r)f(s)$ to hold for any $r,s\in\alpha_p(R)$; this implies having $f(s+t)=f(t)f(s)$ for $s,t\in k[s,t]/\left(s^p,t^p\right)$, and this latter condition can be seen to be sufficient as well. Now, write $f(t)=\sum_{i=0}^{\infty}a_it^i$ (for example, $a_0=1$) so that
		\[f(s+t)=\sum_{i,j\ge0}a_{i+j}\binom{i+j}js^it^j\qquad\text{and}\qquad f(s)f(t)=\sum_{i,j\ge0}a_ia_js^it^j.\]
		In particular, looking at the coefficients of $s^it$, this implies that $a_{i+1}(i+1)=a_ia_1$ for each $i\ge0$, so we are forced to have
		\[f(t)=\sum_{i=0}^{p-1}\frac{a_1^i}{i!}t^i\]
		and so $a_1^p=0$. On the other hand, for any $a\in\alpha_p(R)$, the polynomial $f_a(t)\coloneqq\sum_{i=0}^{p-1}\frac{a^i}{i!}t^i$ can be seen to satisfy $f(s+t)=f(s)f(t)$: the coefficient of $s^it^j$ is
		\[a^{i+j}\cdot\frac1{(i+j)!}\binom{i+j}j=a^{i+j}\cdot\frac{1}{i!j!}.\]

		\item At this point, we have produced a bijection $\alpha_p(k)\to\op{Hom}_k(\alpha_p,\mathbb G_m)$ given by $a\mapsto f_a$. We will be done as soon as we check that this upgrades to a natural isomorphism of groups. To check that this is an isomorphism, we must check that $f_{a+b}(t)=f_a(t)f_b(t)$, which is true because the coefficient of $t^i$ is
		\[\frac{(a+b)^i}{i!}=\sum_{m+n=i}\frac{a^{m}b^n}{m!n!}.\]
		Lastly, for naturality, we suppose that we have a ring homomorphism $\varphi\colon k\to k'$, and we note that the diagram
		% https://q.uiver.app/#q=WzAsOCxbMCwwLCJcXGFscGhhX3AoaykiXSxbMCwxLCJcXGFscGhhX3AoaycpIl0sWzEsMCwiXFxvcHtIb219X2soXFxhbHBoYV9wLFxcbWF0aGJiIEdfbSkiXSxbMiwwLCJhIl0sWzMsMCwiKHJcXG1hcHN0byBmX2EocikpIl0sWzEsMSwiXFxvcHtIb219X3trJ30oXFxhbHBoYV9wLFxcbWF0aGJiIEdfbSkiXSxbMiwxLCJcXHZhcnBoaShhKSJdLFszLDEsIihyXFxtYXBzdG8gZl97XFx2YXJwaGkoYSl9KHIpKSJdLFswLDJdLFswLDFdLFsxLDVdLFsyLDVdLFszLDYsIiIsMCx7InN0eWxlIjp7InRhaWwiOnsibmFtZSI6Im1hcHMgdG8ifX19XSxbNiw3LCIiLDAseyJzdHlsZSI6eyJ0YWlsIjp7Im5hbWUiOiJtYXBzIHRvIn19fV0sWzMsNCwiIiwyLHsic3R5bGUiOnsidGFpbCI6eyJuYW1lIjoibWFwcyB0byJ9fX1dLFs0LDcsIiIsMix7InN0eWxlIjp7InRhaWwiOnsibmFtZSI6Im1hcHMgdG8ifX19XV0=&macro_url=https%3A%2F%2Fraw.githubusercontent.com%2FdFoiler%2Fnotes%2Fmaster%2Fnir.tex
		\[\begin{tikzcd}[cramped]
			{\alpha_p(k)} & {\op{Hom}_k(\alpha_p,\mathbb G_m)} & a & {(r\mapsto f_a(r))} \\
			{\alpha_p(k')} & {\op{Hom}_{k'}(\alpha_p,\mathbb G_m)} & {\varphi(a)} & {(r\mapsto f_{\varphi(a)}(r))}
			\arrow[from=1-1, to=1-2]
			\arrow[from=1-1, to=2-1]
			\arrow[from=1-2, to=2-2]
			\arrow[maps to, from=1-3, to=1-4]
			\arrow[maps to, from=1-3, to=2-3]
			\arrow[maps to, from=1-4, to=2-4]
			\arrow[from=2-1, to=2-2]
			\arrow[maps to, from=2-3, to=2-4]
		\end{tikzcd}\]
		commutes.
		\qedhere
	\end{enumerate}
\end{proof}
\begin{exercise}
	For any $n\ge1$, we show the groups $\ZZ/n\ZZ$ and $\mu_n$ are Cartier dual.
\end{exercise}
\begin{proof}
	Because it is significantly faster, we will show that these two groups are Cartier dual by appealing to their Hopf algebra structure.
	\begin{enumerate}
		\item We give the Hopf algebra structure on $\mu_n$. Recall
		\[\mu_n(R)=\left\{r\in R^\times:r^n=1\right\},\]
		so $\mu_n$ has a natural inclusion into $\mathbb G_m$. In particular, we see that $\mu_n=\Spec k[t]/\left(t^n-1\right)$, and the inclusion $\mu_n\into\mathbb G_m$ is a closed embedding given on rings by the map $k\left[t,t^{-1}\right]\to k[t]/\left(t^n-1\right)$ defined by $t\mapsto t$. The coidentity and comultiplication data are just inherited from $\mathbb G_m$.

		\item We now give the Hopf algebra structure on $\ZZ/n\ZZ$. Recall
		\[\ZZ/n\ZZ(R)=\op{Mor}_{\mathrm{cts}}(\Spec R,\ZZ/n\ZZ),\]
		which becomes a group via the addition on $\ZZ/n\ZZ$. Thus, the functor $\ZZ/n\ZZ\colon\mathrm{Alg}_k\to\mathrm{Set}$ is represented by the affine scheme
		\[\bigsqcup_{*\in\ZZ/n\ZZ}\Spec k=\Spec k^{\ZZ/n\ZZ}.\]
		We may alternatively write $k^{\ZZ/n\ZZ}$ as $k_0\times\cdots\times k_{n-1}$, where each $k_\bullet$ equals $k$. The coidentity $\Spec k\to\Spec k^{\ZZ/n\ZZ}$ needs to pick out $0\in\ZZ/n\ZZ(R)$ for each $R$, so it is given by embedding $k\into k_0\into k^{\ZZ/n\ZZ}$. Lastly, the comultiplication $m$ is given on the $k_\bullet$s by gluing together the diagonal embeddings
		\[k_i\into\bigoplus_{i_1+i_2=i}k_{i_1}\otimes k_{i_2}\subseteq k^{\ZZ/n\ZZ}\otimes k^{\ZZ/n\ZZ},\]
		where indices are taken$\pmod n$. To check that this works, we choose $r,s\in\ZZ/n\ZZ(R)$, which correspond to some maps $r^\sharp,s^\sharp\colon k^{\ZZ/n\ZZ}\to R$, and we can see that $m^\sharp\circ(r,s)$ is adding appropriately by checking on the $k_\bullet$s.

		% \item We calculate $\op{Mor}_k(\ZZ/n\ZZ,\mathbb G_m)$ as
		% \[\mathbb G_m(\ZZ/n\ZZ)=\left(k^{\ZZ/n\ZZ}\right)^\times.\]
		% Thus, we produce a morphism $\ZZ/n\ZZ\to\mathbb G_m$ for each element $(a_0,\ldots,a_{n-1})$ where $a_i\in k^\times$ for each $i$. To be explicit, we note that any such unit $(a_0,\ldots,a_{n-1})$ sends $f\in\ZZ/n\ZZ(R)$ to $f(a_0,\ldots,a_{n-1})\in R^\times$.

		\item The Cartier dual of $\mu_n$ should be given by the dual Hopf algebra of the Hopf algebra $k[\mu_n]$. Thus, we would like to show that $k[\mu_n]^\lor$ is $k[\ZZ/n\ZZ]$. We begin by noting that there is a vector-space isomorphism $\varphi\colon k[\mu_n]^\lor\to k[\ZZ/n\ZZ]$ given by sending the dual basis $\{\ell_1,\ldots,\ell_{n-1}\}\subseteq k[\mu_n]^\lor$ of $\left\{1,t,\ldots,t^{n-1}\right\}\subseteq k[\mu_n]$ to the standard basis $\{e_0,\ldots,e_{n-1}\}\subseteq k^{\ZZ/n\ZZ}$. It remains to argue that $\varphi$ preserves the Hopf algebra structures.
		\begin{itemize}
			\item The identity $1\in k[\mu_n]$ produces the coidentity $\op{ev}_1\colon\ell\mapsto\ell(1)$ of $k[\mu_n]^\lor$, which by definition of $\varphi$ goes to the coidentity $e_0\in k[\ZZ/n\ZZ]$.
			\item The coidentity $k[\mu_n]\to k$ is given by $t\mapsto1$, which produces the identity ${\ell_1}+\cdots+{\ell_{n-1}}\in k[\mu_n]^\lor$, which is sent to the identity $e_0+\cdots+e_{n-1}\in k^{\ZZ/n\ZZ}$.
			\item The multiplication in $k[\mu_n]$ is given by $t^i\cdot t^j=t^{i+j}$, which defines a comultiplication on $k[\mu_n]^\lor$ which sends $\ell_i$ to
			\[\sum_{i_1+i_2=i}{\ell_{i_1}}\otimes{\ell_{i_2}}.\]
			This is the correct comultiplication on $k[\ZZ/n\ZZ]$.
			\item The comultiplication in $k[\mu_n]$ is given by $t\mapsto t\otimes t$, which produces the multiplication given by extending the conditions
			\[\ell_i\otimes\ell_j\mapsto\begin{cases}
				\ell_i & \text{if }i=j, \\
				0 & \text{else},
			\end{cases}\]
			$k$-linearly. This does in fact give the multiplication on $k^{\ZZ/n\ZZ}$.
			\qedhere
		\end{itemize}
	\end{enumerate}
\end{proof}

\subsection{Some Dieudonn\'e Modules}
In this subsection, we will compute some Dieudonn\'e modules.
\begin{notation}
	Fix a ring $R$ of characteristic $p$. Then we define $_mW_n$ as the kernel of the $m$-fold Frobenius map $F^m\colon W_n\to W_n^{(p^m)}$.
\end{notation}
\begin{example}
	Because $W_1=\mathbb G_a$, we see that $_1W_1=\alpha_p$.
\end{example}
\begin{remark}
	It turns out that the Cartier dual of $_mW_n$ is $_nW_m$. For example, we showed in \Cref{exe:cartier-dual-alpha-p} that the dual of $\alpha_p={_1W_1}$ is itself.
\end{remark}
\begin{exercise} \label{exe:m-nwn}
	Fix a perfect field $k$ of positive characteristic $p$. The Dieudonn\'e module $M({_nW_n})$ is isomorphic to the Dieudonn\'e module $D_k/\left(D_kF^n+D_kV^n\right)$.
\end{exercise}
\begin{proof}
	We proceed in steps.
	\begin{enumerate}
		\item We compute the order of $_nW_n$. In fact, we claim that the order of $_mW_n$ is $p^{mn}$, which we will show by induction. We begin with the base cases ${_0W_0}={_1W_0}={_0W_1}=0$, which has order $0$. Then, for any $n$ and $m$, we note that there are short exact sequences
		\[0\to{_1W_n}\to{_{m+1}W_n}\stackrel F\to{_mW_n}\to0\]
		and
		\[0\to{_mW_n}\stackrel V\to{_mW_{n+1}}\onto{_mW_1}\to0\]
		from which we see that $\#(_{m+1}W_n)=\#({_mW_n})\cdot\#({_1W_n})$ and $\#({_mW_{n+1}})=\#({_mW_n})\cdot\#({_mW_1})$. The claim now follows by induction on $m$ and $n$.

		\item We upper-bound the length of the $W(k)$-module $D_k/(D_kF^n+D_kV^n)$. Note that any element of $D_k$ can be written as a finite polynomial in the form
		\[\sum_{m<0}a_mV^{-m}+a_0+\sum_{m>0}a_mF^m,\]
		where $a_m\in W(k)$ for each $m$. Thus, any element in the quotient $D_k/\left(D_kF^n+D_kV^n\right)$ can be written as a polynomial in the form
		\[\sum_{m=1}^{n-1}a_{-m}V^{m}+a_0+\sum_{m=1}^{n-1}a_mF^m.\]
		In fact, for an element of $D_k/\left(D_kF^n+D_kV^n\right)$ in the above form, we see that $a_{-m}V^m$ has $a_m$ only defined up to $p^{n-m}W(k)$ because $p^{n-m}V^m\in D_kV^n$, so we may view $a_m$ as an element of the quotient $W(k)/p^{n-m}W(k)=W_{n-m}(k)$, which is a $W(k)$-module of length $n-m$. A similar remark holds for the terms $a_mF^m$, so upon taking the appropriate filtrations, we see that the length of the total $W(k)$-module is bounded above by
		\[1+2+\cdots+(n-1)+n+(n-1)+\cdots+2+1=n^2.\]

		\item We exhibit an injection $D_k/\left(D_kF^n+D_kV^n\right)\to M({_nW_n})$ of Dieudonn\'e modules. In light of the embedding ${_nW_n}\subseteq W_n\subseteq W_\infty$, it is enough to exhibit a map $D_k\to\op{Hom}_k({_nW_n},W_n)$ and show that its kernel is $D_kF^n+D_kV^n$. Well, there is certainly a map $D_k\to\op{Hom}_k(W,W)$ because $D_k$ is the endomorphism algebra of $W$; further, we can see that all the endomorphisms in $D_k$ descend to endomorphisms $W_n\to W_n$ and hence restrict to morphisms $_nW_n\to W_n$.

		Now, certainly $F^n$ and $V^n$ induce the zero morphism $_nW_n\to W_n$ because $F^n$ and $V^n$ vanish on $_nW_n$. It remains to show that these two elements generate the kernel. Well, the previous step established a ``normal form'' for elements in $D_k/\left(D_kF^n+D_kV^n\right)$ as given by polynomials
		\[\sum_{m=1}^{n-1}a_{-m}V^{m}+a_0+\sum_{m=1}^{n-1}a_mF^m,\]
		where $a_{\pm m}\in W_{n-\left|m\right|}(k)$ for each $m$. It is enough to show that each of the elements produces a nonzero homomorphism $_nW_n\to W_n$ when at least one of the coefficients is nonzero. This can be checked on $k[t]/\left(t^{p^n}\right)$-points, where we have to take a nilpotent thickening to ensure that $_nW_n$ has some points.

		\item In light of the bounds on the lengths on $D_k/\left(D_kF^n+D_kV^n\right)$ and $M({_nW_n})$ as $W(k)$-modules, the injection provided in the previous step completes the proof.
		\qedhere
	\end{enumerate}
\end{proof}
\begin{remark}
	The previous exercise shows that $M(\alpha_p)=M({_1W_1})$ is isomorphic to the $W(k)$-module $W_1(k)=k$ where $F$ and $V$ both act by $0$.
\end{remark}
An understanding of $_nW_n$ allows us to prove the following duality result.
\begin{proposition}
	Fix a perfect field $k$ of positive characteristic $p$. For a connected unipotent group $G$, we have $M(G^\lor)=M(G)^\lor$.
\end{proposition}
\begin{proof}
	The point is to check the group $_nW_n$ and then reduce to this case. For brevity, we set $_nE_n\coloneqq M({_nW_n})$, which we view as the endomorphism algebra of $_nW_n$.
	\begin{enumerate}
		\item We check the result for $G={_nW_n}$. Because $_nW_n^\lor={_nW_n}$, this amounts to checking that $_nE_n^\lor={_nE_n}$. This is more or less a direct computation, for which we will not write out all the details. The proof of \Cref{exe:m-nwn} describes elements in $_nE_n$ as being represented by polynomials
		\[\sum_{m=1}^{n-1}a_{-m}V^m+a_0+\sum_{m=1}^{n-1}a_mF^m,\]
		where $a_{\pm m}\in W_{n-\left|m\right|}(k)$ for each $m$. The proof actually shows that this is a unique decomposition for any element in $_nE_n$, so we conclude that
		\[_nE_n\cong W_1(k)\oplus W_2(k)\oplus\cdots\oplus W_{n-1}(k)\oplus W_n(k)\oplus W_{n-1}(k)\oplus\cdots\oplus W_2(k)\oplus W_1(k)\]
		as $W(k)$-modules. We now see that $_nE_n$ is self-dual as a $W(k)$-module, and the duality flips the action of $F$ and $V$ in such a way to make $_nE_n$ also self-dual as a $D_k$-module.

		\item We prove the general case. Let $p^n$ be the order of $G$, and then we see that any map $G\to W_\infty$ must factor through $_nW_n$ because $G$ is killed by $p^n$. Thus, we may write $M(G)=\op{Hom}_k(G,{_nW_n})$, and the same argument shows $M(G^\lor)=\op{Hom}_k(G^\lor,{_nW_n})$.

		We will slowly move $M(G^\lor)$ to $M(G)^\lor$. Because $(\cdot)^\lor$ is an anti-equivalence, we see that
		\[M(G^\lor)=\op{Hom}_k(G^\lor,{_nW_n})=\op{Hom}_k({_nW_n},G).\]
		Now, we would like $M(G)$ to appear, so we use the fact that $M$ is an anti-equivalence as well to see that
		\[\op{Hom}_k({_nW_n},G)=\op{Hom}_{D_k}(M(G),M({_nW_n})=\op{Hom}_{D_k}(M(G),{_nE_n}).\]
		We would now like to make $M(G)^\lor$ to appear, so we use the fact that $(\cdot)^\lor$ (on Dieudonn\'e modules) is an anti-equivalence on $W(k)$-modules of finite length (which can be checked more or less directly by breaking everything into cyclic modules) to see that
		\[\op{Hom}_{D_k}(M(G),{_nE_n})=\op{Hom}_{D_k}({_nE_n^\lor},M(G)^\lor)=\op{Hom}_{D_k}({_nE_n},M(G)^\lor).\]
		Now, certainly $\op{Hom}_{D_k}(D_k,M(G)^\lor)=M(G)$, so we are really interested in showing that all morphisms $D_k\to M(G)^\lor$ contain $F^nD_k+V^nD_k$ in the kernel. Well, we already argued at the beginning of this step that $F^n$ and $V^n$ vanish on $M(G)$ because $G$ has order $p^n$, so the same is true of $M(G)^\lor$.
		\qedhere
	\end{enumerate}
\end{proof}
\begin{exercise} \label{exe:m-z-pn}
	Fix a perfect field $k$ of positive characteristic $p$. The Dieudonn\'e module $M(\ZZ/p^n\ZZ)$ is isomorphic to $W_n(k)\subseteq W_\infty(k)$.
\end{exercise}
\begin{proof}
	We proceed in steps.
	\begin{enumerate}
		\item For completeness, we check that $W_n(k)$ is a $W(k)$-module of length $n$. The ``filtration'' induced by taking kernels in the sequence
		\[W_n(k)\onto W_{n-1}(k)\onto\cdots\onto W_2(k)\onto W_1(k)\onto0\]
		implies that $W_n(k)$ certainly has length at most $n$. On the other hand, we note that the quotients
		\[\frac{W_{i+1}(k)}{W_i(k)}=\frac{W(k)/p^{i+1}W(k)}{W(k)/p^iW(k)}\cong\frac{W(k)}{pW(k)}\]
		is a simple $W(k)$-module (it is isomorphic to the field $k$), so the above filtration is maximal.

		\item Note that $\ZZ/p^n\ZZ$ is \'etale, so it has order equal to $\#\ZZ/p^n\ZZ(k)=p^n$. Thus, $M(\ZZ/p^n\ZZ)$ should be a $W(k)$-module of length $n$, so we will be able to conclude that $M(\ZZ/p^n\ZZ)=W_n(k)$ as $W(k)$-modules as soon as we produce an injection $W_n(k)\into M(\ZZ/p^n\ZZ)$. We will actually exhibit an injection $W_n(k)\into\op{Hom}_k(\ZZ/p^n\ZZ,W_n)$ of $W(k)$-modules, where $W_n$ has the $W(k)$-action induced by $W_n(k)\into W_\infty(k)$. %Namely, we are not using the action on $W_n$ induced by $W_n\subseteq W_\infty$; this provides no issue because there is an embedding $W_n\into W_\infty$ of groups given by $\sigma^{-n}$.
		
		Well, for $\alpha\in W_n(k)$, we define a natural transformation $\op{ev}_\alpha\colon\ZZ/p^n\ZZ\to W_n$ on $k$-points by $\op{ev}_\alpha\colon z\mapsto z\alpha$, which makes sense because $W_n(k)$ is $p^n$-torsion.\footnote{The same definition works on $R$-points whenever $\Spec R$ is connected, so we can produce a definition for arbitrary $R$ by passing to connected components.} This definition on $k$-points reveals that our map $\op{ev}_\bullet\colon W_n(k)\into\op{Hom}_k(\ZZ/p^n\ZZ,W_n)$ is an injective group homomorphism, and it is $W(k)$-invariant because
		\[\op{ev}_{\lambda\alpha}(z)=z\lambda\alpha=\left(\lambda\op{ev}_\alpha\right)(z).\]

		\item The previous step provides an isomorphism $\op{ev}_\bullet\colon W_n(k)\to\op{Hom}_k(\ZZ/p^n\ZZ,W_n)$ of $W(k)$-modules. Because $F$ and $V$ commute with the action of $\ZZ$ (after all, $F,V\colon W(k)\to W(k)$ are ring endomorphisms), we see that $\op{ev}_\bullet$ is also an isomorphism of Dieudonn\'e modules, so we are done.
		\qedhere
	\end{enumerate}
\end{proof}
\begin{exercise} \label{ex:m-qp-zp}
	Fix a perfect field $k$ of positive characteristic $p$. The Dieudonn\'e module $M(\QQ_p/\ZZ_p)$ is isomorphic to $W(k)$.
\end{exercise}
\begin{proof}
	Note $\QQ_p/\ZZ_p$ is simply $\colimit\ZZ/p^\bullet\ZZ$, where the transition maps $\ZZ/p^n\ZZ\into\ZZ/p^{n+1}\ZZ$ are given by mutlipli\-cation-by-$p$. Thus,
	\[M(\QQ_p/\ZZ_p)=\limit M(\ZZ/p^\bullet\ZZ),\]
	where the transition maps are given by
	% https://q.uiver.app/#q=WzAsOCxbMSwxLCJNKFxcWlovcF5uXFxaWikiXSxbMCwxLCJNXFxsZWZ0KFxcWlovcF57bisxfVxcWlpcXHJpZ2h0KSJdLFswLDAsIldfe24rMX0oaykiXSxbMSwwLCJXX24oaykiXSxbMiwxLCIoMVxcbWFwc3RvXFxhbHBoYSkiXSxbMiwwLCJcXGFscGhhIl0sWzMsMSwiKDFcXG1hcHN0byBwXFxhbHBoYSkiXSxbMywwLCJwXFxhbHBoYSJdLFs1LDQsIiIsMCx7InN0eWxlIjp7InRhaWwiOnsibmFtZSI6Im1hcHMgdG8ifX19XSxbNCw2LCIiLDAseyJzdHlsZSI6eyJ0YWlsIjp7Im5hbWUiOiJtYXBzIHRvIn19fV0sWzcsNiwiIiwyLHsic3R5bGUiOnsidGFpbCI6eyJuYW1lIjoibWFwcyB0byJ9fX1dLFs1LDcsIiIsMix7InN0eWxlIjp7InRhaWwiOnsibmFtZSI6Im1hcHMgdG8ifX19XSxbMiwzXSxbMywwLCJcXG9we2V2fV9cXGJ1bGxldCJdLFsxLDAsInAiXSxbMiwxLCJcXG9we2V2fV9cXGJ1bGxldCIsMl1d&macro_url=https%3A%2F%2Fraw.githubusercontent.com%2FdFoiler%2Fnotes%2Fmaster%2Fnir.tex
	\[\begin{tikzcd}[cramped]
		{W_{n+1}(k)} & {W_n(k)} & \alpha & {\alpha} \\
		{M\left(\ZZ/p^{n+1}\ZZ\right)} & {M(\ZZ/p^n\ZZ)} & {(1\mapsto\sigma^{-n-1}\alpha)} & {(1\mapsto p\sigma^{-n-1}\alpha)}
		\arrow[from=1-1, to=1-2]
		\arrow["{\op{ev}_\bullet}"', from=1-1, to=2-1]
		\arrow["{\op{ev}_\bullet}", from=1-2, to=2-2]
		\arrow[maps to, from=1-3, to=1-4]
		\arrow[maps to, from=1-3, to=2-3]
		\arrow[maps to, from=1-4, to=2-4]
		\arrow["p", from=2-1, to=2-2]
		\arrow[maps to, from=2-3, to=2-4]
	\end{tikzcd}\]
	where the vertical isomorphisms are given by \Cref{exe:m-z-pn}, where we have implicitly adjusted the isomorphisms so that all maps in sight are $W(k)$-invariant. (Note that $\sigma^{-n}\alpha\in W_n(k)$ goes to the same element as $V\sigma^{-n}\alpha\in W_{n+1}(k)$ in $W_\infty(k)$!) Thus, we see that we get the $W(k)$-module $W(k)$ after passing to the inverse limit. The Frobenius element acts by $\sigma$ everywhere, even after adjusting the $W(k)$-action, so we see that this Dieudonn\'e module is in fact $W(k)$. In particular, we are recalling that we do not have to define a Verschiebung because it is now uniquely given by the Frobenius element and the condition $FV=VF=p$.
\end{proof}
\begin{exercise}
	Fix a perfect field $k$ of positive characteristic $p$. The Dieudonn\'e module $M(\mu_{p^n})$ is isomorphic to the $W(k)$-module $W_n(k)\subseteq W_\infty(k)$, where $F=p\sigma$ and $V=\sigma^{-1}$.
\end{exercise}
\begin{proof}
	By construction of $M$, we have
	\[M(\mu_{p^n})=M(\ZZ/p^n\ZZ)^\lor=\op{Hom}_{W(k)}(W_n(k),W_\infty(k)).\]
	Note that $W_n(k)$ is $V^n$-torsion, so the right-hand side is actually $\op{Hom}_{W_n(k)}(W_n(k),W_n(k))$. Now, we note that there is an embedding of $W(k)$-modules $W_n(k)\into\op{Hom}_{W_n(k)}(W_n(k),W_n(k))$ given by sending $\alpha\in W_n(k)$ to the map $\alpha\colon\lambda\mapsto\lambda\alpha$. Because $M(\mu_{p^n})$ needs to be a $W(k)$-module of length $n$, we conclude that this embedding is an isomorphism.

	It remains to describe the action of $F$ and $V$. For $F$, we note that the diagram
	% https://q.uiver.app/#q=WzAsOCxbMCwwLCJXX24oaykiXSxbMSwwLCJcXG9we0hvbX1fayhXX24oayksV19uKGspKSJdLFswLDEsIldfbihrKSJdLFsxLDEsIlxcb3B7SG9tfV9rKFdfbihrKSxXX24oaykpIl0sWzIsMCwiXFxhbHBoYSJdLFszLDAsIigxXFxtYXBzdG9cXGFscGhhKSJdLFszLDEsIigxXFxtYXBzdG9cXHNpZ21hKFxcYWxwaGEgVjEpKSJdLFsyLDEsIlxcc2lnbWEocFxcYWxwaGEpIl0sWzEsMywiRiJdLFswLDIsIkYiLDJdLFswLDFdLFsyLDNdLFs0LDcsIiIsMix7InN0eWxlIjp7InRhaWwiOnsibmFtZSI6Im1hcHMgdG8ifX19XSxbNyw2LCIiLDIseyJzdHlsZSI6eyJ0YWlsIjp7Im5hbWUiOiJtYXBzIHRvIn19fV0sWzQsNSwiIiwwLHsic3R5bGUiOnsidGFpbCI6eyJuYW1lIjoibWFwcyB0byJ9fX1dLFs1LDYsIiIsMCx7InN0eWxlIjp7InRhaWwiOnsibmFtZSI6Im1hcHMgdG8ifX19XV0=&macro_url=https%3A%2F%2Fraw.githubusercontent.com%2FdFoiler%2Fnotes%2Fmaster%2Fnir.tex
	\[\begin{tikzcd}[cramped]
		{W_n(k)} & {\op{Hom}_{W(k)}(W_n(k),W_n(k))} & \alpha & {(1\mapsto\alpha)} \\
		{W_n(k)} & {\op{Hom}_{W(k)}(W_n(k),W_n(k))} & {\sigma(p\alpha)} & {(1\mapsto\sigma(\alpha V1))}
		\arrow[from=1-1, to=1-2]
		\arrow["F"', from=1-1, to=2-1]
		\arrow["F", from=1-2, to=2-2]
		\arrow[maps to, from=1-3, to=1-4]
		\arrow[maps to, from=1-3, to=2-3]
		\arrow[maps to, from=1-4, to=2-4]
		\arrow[from=2-1, to=2-2]
		\arrow[maps to, from=2-3, to=2-4]
	\end{tikzcd}\]
	commutes, and for $V$, we note that the diagram
	% https://q.uiver.app/#q=WzAsOCxbMCwwLCJXX24oaykiXSxbMSwwLCJcXG9we0hvbX1fayhXX24oayksV19uKGspKSJdLFswLDEsIldfbihrKSJdLFsxLDEsIlxcb3B7SG9tfV9rKFdfbihrKSxXX24oaykpIl0sWzIsMCwiXFxhbHBoYSJdLFszLDAsIigxXFxtYXBzdG9cXGFscGhhKSJdLFszLDEsIlxcbGVmdCgxXFxtYXBzdG9cXHNpZ21hXnstMX0oXFxhbHBoYSBGMSlcXHJpZ2h0KSJdLFsyLDEsIlxcc2lnbWFeey0xfShcXGFscGhhKSJdLFsxLDMsIlYiXSxbMCwyLCJWIiwyXSxbMCwxXSxbMiwzXSxbNCw3LCIiLDIseyJzdHlsZSI6eyJ0YWlsIjp7Im5hbWUiOiJtYXBzIHRvIn19fV0sWzcsNiwiIiwyLHsic3R5bGUiOnsidGFpbCI6eyJuYW1lIjoibWFwcyB0byJ9fX1dLFs0LDUsIiIsMCx7InN0eWxlIjp7InRhaWwiOnsibmFtZSI6Im1hcHMgdG8ifX19XSxbNSw2LCIiLDAseyJzdHlsZSI6eyJ0YWlsIjp7Im5hbWUiOiJtYXBzIHRvIn19fV1d&macro_url=https%3A%2F%2Fraw.githubusercontent.com%2FdFoiler%2Fnotes%2Fmaster%2Fnir.tex
	\[\begin{tikzcd}[cramped]
		{W_n(k)} & {\op{Hom}_{W(k)}(W_n(k),W_n(k))} & \alpha & {(1\mapsto\alpha)} \\
		{W_n(k)} & {\op{Hom}_{W(k)}(W_n(k),W_n(k))} & {\sigma^{-1}(\alpha)} & {\left(1\mapsto\sigma^{-1}(\alpha F1)\right)}
		\arrow[from=1-1, to=1-2]
		\arrow["V"', from=1-1, to=2-1]
		\arrow["V", from=1-2, to=2-2]
		\arrow[maps to, from=1-3, to=1-4]
		\arrow[maps to, from=1-3, to=2-3]
		\arrow[maps to, from=1-4, to=2-4]
		\arrow[from=2-1, to=2-2]
		\arrow[maps to, from=2-3, to=2-4]
	\end{tikzcd}\]
	commutes.
\end{proof}
\begin{exercise}
	Fix a perfect field $k$ of positive characteristic $p$. The Dieudonn\'e module $M(\mu_{p^\infty})$ is isomorphic to the $W(k)$-module $W(k)$, where $F=p\sigma$.
\end{exercise}
\begin{proof}
	The same argument as in \Cref{ex:m-qp-zp} shows that $M(\mu_{p^\infty})=\limit M(\mu_{p^\bullet})$ is $W(k)$ as a $W(k)$-invariant and that the isomorphism $M(\mu_{p^\infty})\to W(k)$ is $\sigma$-invariant as well. In particular, the action of $F$ on the finite quotients $W_n(k)$ of $M(\mu_{p^\infty})$ are all given by the same endomorphism $p\sigma$, so we conclude that the action of $F$ on $W(k)$ should be given by $p\sigma$ as well.
\end{proof}

% \begin{exercise}
% 	For any $n\ge1$, we show the Cartier dual of the group scheme $\mu_n$ is $\ZZ/n\ZZ$.
% \end{exercise}
% \begin{proof}
% 	Once again, because it is good practice, we prove this using the definition $\mu_n^\lor(R)\coloneqq\op{Hom}_R(\mu_n,\mathbb G_m)$.
% 	\begin{enumerate}
% 		\item As before, we begin by giving the Hopf algebra structure on $\mu_n$. We will be more brief. Recall
% 		\[\mu_n(R)=\left\{r\in R^\times:r^n=1\right\},\]
% 		so $\mu_n$ has a natural inclusion into $\mathbb G_m$. In particular, we see that $\mu_n=\Spec k[t]/\left(t^n-1\right)$, and the inclusion $\mu_n\into\mathbb G_m$ is a closed embedding given on rings by the map $k\left[t,t^{-1}\right]\to k[t]/\left(t^n-1\right)$ defined by $t\mapsto t$. The coidentity and comultiplication data are just inherited from $\mathbb G_m$.

% 		\item We calculate $\op{Hom}_k(\mu_n,\mathbb G_m)$ as
% 		\[\mathbb G_m(\mu_n)=\left(\frac{k[t]}{\left(t^n-1\right)}\right)^\times.\]
% 		Thus, we produce a morphism $\mu_n\to\mathbb G_m$ for each unit polynomial $f(t)\in k[t]/\left(t^n-1\right)$, which amounts to requiring $\left(f(t),t^n-1\right)=k[t]$. As before, we note that such a polynomial $f$ defines the natural transformation $\mu_n\to\mathbb G_m$ given on $R$-points by $r\mapsto f(r)$.

% 		\item 
% 	\end{enumerate}
% \end{proof}

\end{document}