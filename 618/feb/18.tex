% !TEX root = ../notes.tex

\documentclass[../notes.tex]{subfiles}

\begin{document}

\section{February 18}

Before we do anything, let's find a sign in Lubin--Tate theory that the reciprocity map
\[\op{Art}\colon\QQ_p^\times\to\op{Gal}(\QQ_p^{\mathrm{ab}}/\QQ_p)\]
restricts to $\ZZ_p^\times\to I_p$, where $I_p$ is the inertial subgroup, and the inverse map $I_p\to\ZZ_p^\times$ is given by the cyclotomic character. One way to see this is by Lubin--Tate theory, where we need to compute the character $\eta_p$, which can be done by using the explicit Lubin--Tate formal group.

Another way to see this is by global class field theory (and the Kronecker--Weber theorem), which explains that one has
\[\QQ_p^\times\into\QQ^\times\backslash\AA_\QQ^\times\onto(\ZZ/N\ZZ)^\times\simeq\op{Gal}(\QQ(\zeta_N)/\QQ)\]
at ``finite level'' $\QQ(\zeta_N)\subseteq\QQ^{\mathrm{ab}}$; namely, we see that $u\in\ZZ\setminus p\ZZ$ in $\QQ_p^\times$ goes to a $u$-power Frobenius on the left, up to a sign depending on the normalization of class field theory.

\subsection{Cartier Duality}
Today we continue with our $p$-divisible groups.
\begin{definition}
	Fix a base ring $A$. Then a \textit{finite group $G$ over $A$} is a finite abelian group scheme over $A$ which is locally free over $A$.
\end{definition}
\begin{remark}
	Explicitly, we are asking for the coordinate ring $A[G]$ of $G$ to be a Hopf algebra which is locally free and of finite rank over $A$.
\end{remark}
\begin{example}
	Fix a ring $A$. Then one has constant groups ${\ZZ/N\ZZ}$ and ${\mu_N}$.
\end{example}
\begin{example}
	The kernel of an isogeny of abelian varieties is also seen to be a finite group over the base.
\end{example}
\begin{example}
	Over $\ov\FF_p$, we let $\alpha_p$ be the kernel of Frobenius fitting into the diagram
	\[1\to\alpha_p\to\mathbb G_a\stackrel{(\cdot)^p}\to\mathbb G_a.\]
	In particular, we see that $\alpha_p=\Spec\ov\FF_p[x]/(x^p)$.
\end{example}
Let's give a little structure theory.
\begin{lemma} \label{lem:conn-et-sequence}
	Fix a finite group $G$ over a base ring $A$. Then the connected component $G^\circ\subseteq G$ is a subgroup, and the quotient $G^{\mathrm{\acute et}}\coloneqq G/G^\circ$ is \'etale. In short, there is a short exact sequence
	\[1\to G^\circ G\to G^{\mathrm{\acute et}}\to1.\]
\end{lemma}
\begin{proof}
	The fact that $G^\circ$ is a subgroup is a rather formal fact. We won't check that the quotient is \'etale, but the idea is that the only obstruction to being separable should come down to having a large infinitesimal neighborhood.
\end{proof}
These finite groups have a theory of ``Cartier'' duality, analogous to Pontryagin duality in harmonic analysis.
\begin{definition}[Cartier dual]
	Fix a base finite group $G$ over a base ring $A$. Then we define the \textit{Cartier dual} as the sheaf $\mathrm{Alg}(A)\to\mathrm{AbGrp}$ given by
	\[G^\lor\colon R\mapsto\op{Hom}(G(R),R^\times).\]
\end{definition}
\begin{remark}
	Let's check that $G^\lor$ is represented by an affine scheme over $A$ which is locally free over $A$. Indeed, one finds that the Hopf algebra $A[G^\lor]$ should be
	\[\op{Hom}(A[G],A),\]
	where $A[G]$ is the Hopf algebra of $G$. One sees that the above is locally free of finite rank over $A$ because the same was true of $A[G]$, and one sees that $A[G]$ is a Hopf algebra of an abelian group because implies the same for $A[G^\lor]$ essentially by interchanging the multiplication and comultiplication.
\end{remark}
\begin{remark}
	The above remark explains why $G^{\lor\lor}=G$.
\end{remark}
\begin{example}
	Consider $A=\FF_p$.
	\begin{itemize}
		\item If $\gcd(p,N)=1$, then the dual of ${\ZZ/N\ZZ}$ is ${\mu_N}$. Note that both of these schemes are \'etale over $A$.
		\item The dual of $\ZZ/p\ZZ$ is $\mu_p$. Note that $\ZZ/p\ZZ$ is \'etale while $\mu_p$ is connected.
		\item The dual of $\alpha_p$ is $\alpha_p$. Note that these are both connected.
	\end{itemize}
	These can be seen by explicitly writing out the multiplication and comultiplication everywhere.
\end{example}
\begin{remark}
	Cartier duality provides a contravariant exact functor.
\end{remark}
This whole connected and \'etale business is a bit subtle on the duality, so we pick up some definitions, motivated by the above calculations.
\begin{definition}
	Fix a finite group $G$ over a perfect field $k$.
	\begin{itemize}
		\item We say $G$ is \textit{unipotent} if and only if $G^\lor$ is connected.
		\item We say $G$ is \textit{multiplicative} if and only if $G$ is \'etale.
	\end{itemize}
\end{definition}
\begin{example}
	Over $\FF_p$, we see that $\ZZ/p\ZZ$ and $\alpha_p$ are both unipotent, and $\mu_N$ is multiplicative.
\end{example}
\begin{remark}
	This notion is poorly behaved when working with different characteristics simultaneously. The problem is that the duals in different characteristics are likely to behave differently depending on (for example) when the characteristic divides the order.
\end{remark}
We now see that a group $G$ can be split into pieces which are \'etale multiplicative, \'etale unipotent, connected multiplicative, and connected unipotent. Notably, no two groups of any individual pair of adjectives will have any no nontrivial morphisms between each other: connected and \'etale groups have no nontrivial morphisms between each other, and the same is carries over for the duals.

We now recall some of our story of $p$-divisible groups.
\begin{definition}[formal $p$-divisible group]
	A \textit{formal $p$-divisible group} is a formal group $\mc G$ together with a $p$-power map $[p]\colon\mc G\to\mc G$ which makes the underlying map $A[[X]]\to A[[X]]$ into a free module of finite rank over itself.
\end{definition}
\begin{remark}
	Note that this implies that $\mc G[p]$ is a finite group over $A$: namely, $\mc G[p]$ is $A[[X]]$ modulo the power series given by $[p]$, and then we see that
\end{remark}
\begin{remark}
	Eventually, we will find that if $A$ is a perfect field, then $\mc G[p]$ has prime-power order, and it turns out to have no component which is \'etale multiplicative.
\end{remark}

\subsection{Witt Vectors}
We would like to classify $p$-divisible groups, which will eventually turn into some semilinear algebra in the form of Dieudonn\'e modules.

This requires some discussion of Witt vectors. We will not go into the formal definition involving power series here. Instead, let's give a little motivation. One can view the ring $\OO\coloneqq k[[t]]$ as being a formal neighborhood of $0\in\AA^1_{k}$. For example, an element $\gamma\in k[[t]]$ can be thought of as the germ of some arc going through $\gamma(0)\in\AA^1_{k}$. The moral is that we can think of $k[[t]]$ as the $k$-points of some infinite-dimensional ``arc space'' $L^+\AA^1_k$ of $\AA^1_{k}$. It is not so hard to actually describe $L^+\AA^1_k$: it should be an inverse limit
\[L^+\AA^1=\limit L_n^+\AA^1,\]
where $L_n^+\AA^1=\AA^{n+1}$ contains information of the first $(n+1)$ coefficients for the arc here. One can find a ring structure on these schemes (the multiplication map can be seen on the level of the coordinates), and we see that the $\FF_q$-points are $\OO$.

This perspective provides some new insights. For example, we see that $\OO^\times$ are the $\FF_q$-points of a Zariski open subset which is dense. Indeed, $\OO^\times$ consists of the condition that the constant term is nonzero, which is a Zariski open and dense condition. This provides a new topology for us to do some interesting sheaf theory.

The Witt vectors bring this story to characteristic $0$. Namely, one finds that $W$ is an inverse limit of some ring schemes $W_n$. Given a perfect field $k$ of positive characteristic, one finds that $W(k)$ is some complete local ring with residue field $k$.
\begin{example}
	One finds $W(\FF_p)=\ZZ_p$. The idea is to take some sequence $(a_0,a_1,\ldots)$ to the power series
	\[a_0+a_1p+a_2p^2+\cdots.\]
	One can reverse-engineer this to figure out how to write out addition and multiplication of the tuples.
\end{example}
\begin{example}
	More generally, one has that $W(\FF_q)=\ZZ_q\subseteq\QQ_q$.
\end{example}
The moral of the story is that we managed to think of $\ZZ_q$ as the $\FF_q$-points of a (limit of) scheme(s).

Importantly, $W(k)$ comes with a Frobenius $F$.
\begin{definition}
	Fix a perfect field $k$ of positive characteristic $p$, and let $\sigma$ be the Frobenius on $k$. Then we define the \textit{Frobenius} $F\colon W(k)\to W(k)$ by
	\[F\cdot(a_1,a_2,\ldots)\coloneqq\left(\sigma(a_1),\sigma^2(a_2),\sigma^3(a_3),\ldots\right).\]
\end{definition}
There is also a dual operation called the Verschiebung. Roughly speaking, this morphism should be the Cartier dual of the Frobenius $F\colon W_n\to W_n$, where now we think of $W_n$ a finite group over $k$.
\begin{definition}
	Fix a perfect field $k$ of positive characteristic $p$. On the level of the power series, $V\colon W(k)\to W(k)$ is given by shifting coordinates over to the left by $1$; for example, we see that $V$ is nilpotent on the $W_n$s and hence nilpotent in the limit on $W_\infty$.
\end{definition}

\subsection{Back to Finite Groups}
Let's return to finite groups. It turns out that $W_n$ (now viewed as a functor) is a $p$-power torsion group over $k$, which can be seen from the construction we did not give: one finds that there are embeddings $p\colon W_n\into W_{n+1}$ (which is multiplication by $p$), seen on the level of infinite tuples as shifting the power series coefficients over by $1$. One finds that $W_1(k)=\FF_p$, so one further finds that $W_n$ is unipotent by some induction.
\begin{theorem} \label{thm:witt-for-finite-groups}
	Fix a perfect field $k$ of positive characteristic $p$. Then the rings $W_n$ generate the category of unipotent $p$-power groups over $k$ in the following sense: the functor $G\to M(G)$ given by
	\[M\colon G\mapsto\colimit\op{Hom}(G,W_\bullet)\]
	is a contravariant fully faithful embedding into the category of $\op{End}(W_\infty)$-modules.
\end{theorem}
\begin{remark}
	More succinctly, we can think about the functor $M$ as
	\[\op{Hom}(G,W_\infty),\]
	where $W_\infty$ is some direct limit. For example, one can think of $W_\infty(\FF_p)$ as $\QQ_p/\ZZ_p$ because the direct limit morphisms are approximately compatible with identifications
	\[W_n(\FF_p)=\ZZ_p/p^n\ZZ_p\cong\frac1{p^n}\ZZ_p/\ZZ_p\subseteq\QQ_p/\ZZ_p.\]
\end{remark}
\begin{remark}
	This is a fairly typical statement in representation theory: one has a category we'd like to understand, so we pick up some particularly interesting objects $W_\bullet$ in it, and it turns out that we can read off the entire category from these elements as modules in some sense.
\end{remark}
This is a nice result because it tells us that our $p$-power unipotent groups are found in some linear algebra.

Let's spend a moment describing $\op{End}(W_\infty)$. Of course, $W(k)$ has an action on $W_\infty$, but there is more: there is a Frobenius morphism $F\colon W_\infty\to W_\infty$ (glued together from the Frobenius morphism above), and there is a Verschiebung $V$ morphism. In fact, we claim that these are all the endomorphisms.
\begin{lemma}
	Fix a perfect ring $k$ of positive characteristic. The ring $\op{End}(W_\infty)$ is the algebra generated by $W(k)$, $F$, and $V$ subject to the requirements
	\[\begin{cases}
		F\circ\lambda = (F\lambda)\circ F, \\
		V\circ\lambda = (V\lambda)\circ V.
	\end{cases}\]
\end{lemma}
\begin{remark}
	The essential image of $M$ in \Cref{thm:witt-for-finite-groups} can now be described as requiring that $V$ acts as a nilpotent operator.
\end{remark}

\end{document}