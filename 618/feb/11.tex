% !TEX root = ../notes.tex

\documentclass[../notes.tex]{subfiles}

\begin{document}

\section{February 11}

Let's continue Lubin--Tate theory.

\subsection{Remarks on Models}
The proof of \Cref{thm:supersingular-for-cm} raises a bizarre question: what is the element $\mu\in\OO$ which induces the Frobenius action by $\mathrm{Frob}_{\mf P}$ on $E$? (Here, we are using the notation of the proof of \Cref{thm:supersingular-for-cm}.) However, this question does not make sense: the hunt for an element $\mu$ depends on a choice of model of $E$ over $H$.

Let's write out this out. Letting $E$ be defined over the Hilbert class field $H$ of $\OO=\OO_K$, we note that its model is not unique and indeed depends on the choice of cocycle in the continuous cohomology group
\[\mathrm H^1(\op{Gal}(\ov\QQ/H),\op{Aut}_\OO(E)).\]
Because $\op{Aut}_\OO(E)=\OO^\times$, we see that the Galois action is trivial, so this is $\op{Hom}_{\mathrm{cont}}(\op{Gal}(\ov\QQ/H),\OO^\times)$. By choosing $\mf p$ and then $\mf P$ carefully, we may assume that $E$ actually admits a model over $\OO_{\mf P}$ (namely, we want $\mf p$ to be totally split or similar). Then the model of the reduction$\pmod{\mf P}$ of $E$ is going to be defined up to an element of
\[\op{Hom}_{\mathrm{cont}}(\op{Gal}(\ov\QQ_{\mf P}/\QQ_{\mf P}),\OO^\times)\supseteq\op{Hom}_{\mathrm{cont}}(\op{Gal}(\ov\FF_{\mf P}/\FF_{\mf P}),\OO^\times)=\OO^\times.\]
Upon changing the model, it turns out that $\mathrm{Frob}_{\mf P}\in\op{End}E$ gets twisted by a unit in $\OO^\times$ so the desired $\mu$ is seen to be only defined up to a unit.

To be explicit, let's choose some $c\colon\op{Gal}(\ov\FF_{\mf P}/\FF_{\mf P})\to\OO^\times$, and let's figure out what the new elliptic curve $E'$ is.
\begin{remark}
	For motivation, let's give the general construction. More generally, if we are dealing with a class of geometric objects $X$ (e.g., schemes) which satisfy some version of descent over a field $K$ (namely, objects $X'$ over $K$ which are isomorphic over $K^{\mathrm{sep}}$ can be described by a form over $K^{\mathrm{sep}}$ along with some descent data), then forms of $X$ over $K$ can be identified with $\mathrm H^1(\op{Gal}(K^{\mathrm{sep}}/K),\op{Aut}_{\ov K}(X))$. To give the map in one direction, suppose that we have a form $X'$ equipped with a specified isomorphism $\varphi\colon X_L\to X'_L$ over a finite Galois extension $L/K$. This produces a diagram
	% https://q.uiver.app/#q=WzAsMTAsWzAsMCwiWF9MIl0sWzAsMSwiXFxTcGVjIEwiXSxbMSwxLCJcXFNwZWMgTCJdLFsxLDAsIlhfTCJdLFsyLDAsIlhfTCciXSxbMiwxLCJcXFNwZWMgTCJdLFszLDAsIlgnX0wiXSxbMywxLCJcXFNwZWMgTCJdLFs0LDAsIlhfTCJdLFs0LDEsIlxcU3BlYyBMIl0sWzAsMywiXFxzaWdtYSJdLFsxLDIsIlxcc2lnbWEiXSxbMCwxXSxbMywyXSxbMiw1LCIiLDEseyJsZXZlbCI6Miwic3R5bGUiOnsiaGVhZCI6eyJuYW1lIjoibm9uZSJ9fX1dLFszLDQsIlxcdmFycGhpIl0sWzQsNiwiXFxzaWdtYV57LTF9Il0sWzUsNywiXFxzaWdtYV57LTF9Il0sWzQsNV0sWzYsN10sWzcsOSwiIiwxLHsibGV2ZWwiOjIsInN0eWxlIjp7ImhlYWQiOnsibmFtZSI6Im5vbmUifX19XSxbNiw4LCJcXHZhcnBoaV57LTF9Il0sWzgsOV1d&macro_url=https%3A%2F%2Fraw.githubusercontent.com%2FdFoiler%2Fnotes%2Fmaster%2Fnir.tex
	\[\begin{tikzcd}
		{X_L} & {X_L} & {X_L'} & {X'_L} & {X_L} \\
		{\Spec L} & {\Spec L} & {\Spec L} & {\Spec L} & {\Spec L}
		\arrow["\sigma", from=1-1, to=1-2]
		\arrow[from=1-1, to=2-1]
		\arrow["\varphi", from=1-2, to=1-3]
		\arrow[from=1-2, to=2-2]
		\arrow["{\sigma^{-1}}", from=1-3, to=1-4]
		\arrow[from=1-3, to=2-3]
		\arrow["{\varphi^{-1}}", from=1-4, to=1-5]
		\arrow[from=1-4, to=2-4]
		\arrow[from=1-5, to=2-5]
		\arrow["\sigma", from=2-1, to=2-2]
		\arrow[equals, from=2-2, to=2-3]
		\arrow["{\sigma^{-1}}", from=2-3, to=2-4]
		\arrow[equals, from=2-4, to=2-5]
	\end{tikzcd}\]
	from which the cocycle $c\colon\op{Gal}(L/K)\to\op{Aut}_{\ov K}(X)$ is given by the above composite: $c(\sigma)=\varphi^{-1}\circ\sigma^{-1}\circ\varphi\circ\sigma$, so for $x\in X(\ov K)$, we have $\sigma(\varphi(c(\sigma)(x)))=\varphi(\sigma(x))$.
\end{remark}
We may identify $E'_{\ov K}=E_{\ov K}$, so the concern is the descent down to $K$. Given $x\in E(\ov K)$, the new Galois action (which we denote by $\cdot'$) is given by
\[\sigma\cdot x=\sigma\cdot'c(\sigma)x,\]
essentially by construction of $\sigma$. Note that $E'$ continues to have good reduction at $\mf P$ because we chose our cocycle to factor through $\op{Gal}(\ov\FF_{\mf P}/\FF_{\mf P})$. In total, we see that the Frobenius action has been adjusted by the unit $c(\sigma)$: if $E$ had $[\mu]=\mathrm{Frob}_{\mf P,E}$, then we will have $\left[c(\sigma)^{-1}\mu\right]=\mathrm{Frob}_{\mf P,E'}$.

\subsection{Proof of Lubin--Tate Theory}
Let's relate this back to Lubin--Tate theory. The moral of the story is that the Lubin--Tate group $\mc G$ provided by \Cref{thm:lt-construction} really does not care about the choice of uniformizer $\varpi$: adjusting $\varpi$ really only adjusts $\mc G$ up to essentially a choice of model.
\begin{lemma}
	Fix a $p$-adic field $(K,\OO,\mf p)$. Choose two uniformizers $\varpi$ and $\varpi'$. Then the Lubin--Tate formal groups $\mc G_{\varpi}$ and $\mc G_{\varpi'}$ are isomorphic over the ring of integers $\widehat\OO$ of the completion $\widehat K$ of $K^{\mathrm{unr}}$.
\end{lemma}
\begin{proof}
	For brevity, let our formal groups by $\mc G$ and $\mc G'$, and we let $F$ and $F'$ be the formal group laws.
	
	Write $\varpi'=\varpi u$ for some unit $u\in\OO^\times$, and we would essentially like to twist $\mc G_{\varpi'}$ by a $1$-cocycle corresponding to $u$ to exhibit the isomorphism. However, note that there is some complication because we need to know that sending $\mathrm{Frob}_{\mf p}\mapsto u$ actually extends to a continuous map $\op{Gal}(K^{\mathrm{unr}}/K)\to\op{Aut}_{\widehat\OO}\mc G_{\varpi,\widehat\OO}$.\footnote{If we only wanted to a homomorphism out of the Weil group $W_{K^{\mathrm{unr}}/K}=\ZZ$ instead of $\op{Gal}(K^{\mathrm{unr}}/K)$, then this would not be a concern.} Let $\check\OO$ be the ring of integers in $K^{\mathrm{unr}}$. It turns out that we extend to a continuous map to merely $\op{Gal}(K^{\mathrm{unr}}/K)\to\op{Aut}\mc G_{\check\OO}$ if and only if $u$ has finite order, which does not need to be the case! One could work with $\mc G_{\check O/\mf p^\bullet}$ for some power $\mf p^\bullet$ because the coefficients are now finite; then we can take the limit as $\mf p^\bullet$ increases in power, so we do get our desired $1$-cocycle.
	
	Everything still satisfies descent, so the choice of $1$-cocyle produces a twist $\widetilde{\mc G}$ of $\mc G$, still defined over $\OO$, with Galois action given by
	\[\sigma\cdot f(X)=\sigma\cdot'f([u]X)\]
	for $f\in\widehat\OO[[X]]$.
	We would like to show that $\widetilde{\mc G}$ is isomorphic to $\mc G'$. Thus, we are on the hunt for a power series $\varphi\in\widehat\OO[[X]]$ witnessing the isomorphism. This formal power series can be constructed (and checked to work) using \Cref{prop:lt-optimize}.

	Roughly speaking, one needs to know that one has $\varphi$ with an inverse function $\varphi^{-1}$ (both in $\widehat\OO[[X]]$) such that
	\[\sigma(f\varphi)\left(\varphi^{-1}(X)\right)=(\sigma\cdot'f)(X).\]
	Then one can check everything formally.
\end{proof}
We are now ready to prove \Cref{thm:lt}. Of course, $\mc G_\varpi$ will be the Lubin--Tate group. Recall that
\[[\varpi](X)\equiv\begin{cases}
	X^q\pmod{\mf p}, \\
	\pi X\pmod{X^2},
\end{cases}\]
where $q\coloneqq\#\FF_{\mf P}$. Note that we are free to choose an object $\mc G_\varpi$ from the isomorphism class, and our proof of existence allows us to choose $[\varpi]$ to be anyone in $\mc F_\varpi$, so we take $[\varpi]=\pi X+X^q$. Note that the formal group law for $\mc G_\varpi$ is potentially complicated, but this choice of $[\varpi]$ will be easier for our calculation. Then $\mc G[\mf p^n]$ consists of the roots of $[\varpi^n]=[\varpi]^{\circ n}$. However, one can calculate that $[\varpi]^{\circ n}$ is a separable polynomial of degree $q^n$. This implies that $\mc G_\varpi[\mf p^n]$ and $\OO/\mf p^n$ have the same order $q^n$; thus, choosing $a\in\mc G_\varpi[\mf p^n]\backslash\mc G_\varpi\left[\mf p^{n-1}\right]$, we see that the induced map $\mc O/\mf p^n\to\mc G[\mf p^n]$ given by $u\mapsto ua$ must be injective.

Let's say a bit about why $[\varpi]^{\circ n}$ is separable. Define
\[\Phi_n(X)\coloneqq\frac{[\varpi]^{\circ(n+1)}(X)}{[\varpi]^{\circ n}(X)},\]
which we can compute as $[\varpi]^{\circ(n-1)}(X)^{q-1}+\varpi$. By induction, one finds that this is an Eisenstein polynomial of degree $(q-1)q^{n-1}$, so for example it is irreducible. Thus, $[\varpi]^{\circ n}$ is a product of irreducible polynomials of all different degrees, thereby proving that it is separable.

In fact, the fact that our polynomial is Eisenstein implies that its roots are totally ramified over $K$, so we are in fact constructing a totally ramified extension of $K$. More precisely, we see that the extension $K(\mc G_\varpi[\mf p^n])/K$ is a totally ramified Galois extension of degree $(q-1)q^{n-1}$. Now, set $K_n\coloneqq K(\mc G_\varpi[\mf p^n])$ and $K_\varpi=\bigcup_{n\ge1} K_n$ for brevity, and we see that we have constructed a character
\[\op{Gal}(K_n/K)\to\op{Aut}_\OO(\mc G_\varpi[\mf p^n]).\]
In fact this, character is surjective because the Galois action is transitive on the roots of the irreducible polynomial $\Phi_n(X)$. Now, this character extends in the limit to a character
\[\eta_\varpi\colon\op{Gal}(K_\varpi/K)\to\op{Aut}_\OO(T_\varpi\mc G_\varpi).\]
This right-hand side is isomorphic to $\OO^\times$, and we will treat it as such.

It remains to show that the above map is identified with local the local Artin map. In particular, we want to show that $\eta_\varpi(\op{Art}_K(\varpi^nu))=u$. Recall local class field theory provides local Artin map $K^\times/\op N_{L/K}(L^\times)\to\op{Gal}(L/K)$ for abelian extensions $L/K$. Thus, for example, we see that $\op{Art}_K\varpi$ acts trivially on $K_\infty$ under the local Artin map because it is a norm (as seen from the Eisenstein polynomial constructed before!), so $\eta(\op{Art}_K(\varpi))=1$. As such, it remains to see that $\eta_\varpi(\op{Art}_Ku)=u$ for units $u\in\OO^\times$. Well, it is actually simply enough to check $\eta_\varpi(\op{Art}_K\varpi')=\varpi'$ for other uniformizers $\varpi'$, for which we will show
\[\eta^{-1}(u)\sigma=\op{Art}_K\varpi'.\]
This follows formally from having $\mc G_{\varpi,\widehat\OO}=\mc G_{\varpi',\widehat\OO}$.

\end{document}