% !TEX root = ../notes.tex

\documentclass[../notes.tex]{subfiles}

\begin{document}

\section{February 27}
Today we continue discussing shtukas.

\subsection{Shtukas in Geometric Class Field Theory}
Fix a smooth projective irreducible curve $C$ over a finite field $\FF_q$, and we will work with the function field $K\coloneqq\FF_q(C)$. We would like to construct its field extensions; for today, we will be content with abelian extensions (and thus use geometric class field theory), and actually today we will focus on constructing the maximal abelian unramified extension of $K$. Note that adding an unramified condition is helpful because
\[\op{Gal}(K^{\mathrm{unr}}/K)=\pi_1^{\mathrm{\acute et}}(C,\ov\eta),\]
where $\ov\eta\colon K^{\mathrm{sep}}\into C$ is some geometric point.
\begin{remark}
	In short, $\pi_1^{\mathrm{\acute et}}(C,\ov\eta)$ is the Tannakian group of the category of finite \'etale covers of $C$. In a few more words, we recall that there is an equivalence between finite \'etale covers $C'\to C$ to finite sets with an action by $\pi_1^{\mathrm{\acute et}}(C,\ov\eta)$ given by sending $C'\to C$ to the fiber $C'_{\ov\eta}$. The reason that $\pi_1^{\mathrm{\acute et}}(C,\ov\eta)$ only sees unramified extensions of $K$ is because we are looking for merely \'etale covers of $C$.
\end{remark}
Thus, we are currently interested in computing the subfield of $K^{\mathrm{sep}}$ coming from $\pi_1^{\mathrm{\acute et}}(C,\ov\eta)^{\mathrm{ab}}$. By (geometric) class field theory, we are looking at
\[\pi_1^{\mathrm{\acute et}}(C,\ov\eta)^{\mathrm{ab}}\cong \widehat{K^\times\backslash\AA_K^\times/\widehat\OO_K^\times},\]
where the $\widehat\OO_K^\times$ is present because we are on the hunt for an unramified extension. Do note that there is a decomposition
\[0\to\pi_1^{\mathrm{\acute et}}(C_{\ov\FF_q},\ov\eta)\to\pi_1^{\mathrm{\acute et}}(C,\ov\eta)\to\op{Gal}(\ov\FF_q/\FF_q)\to0,\]
so we may as well focus on trying to produce a cover with Galois group $\pi_1^{\mathrm{\acute et}}(C_{\ov\FF_q},\ov\eta)$. We will the get the remaining $\op{Gal}(\ov\FF_q/\FF_q)$ extensions by extending the field of coefficients.

We now recall that this right-hand side is the completion of $\op{Pic}C$; note that $\op{Pic}C$ upgrades to a scheme $\op{Pic}_{C/\FF_q}$, which is a moduli space of rigidified line bundles. In order for this moduli space to be well-defined, it is helpful to assume that $C$ has an $\mathbb F_q$-rational point $\infty\in C(\FF_q)$; we will identify $\eta$ and $\infty$ in the sequel. Then $\op{Pic}_{C/\FF_q}$ on $S$-points classifies isomorphism classes of pairs $(\mc L,\alpha)$ of line bundles $\mc L$ on $C\times S$ together with a trivialization $\alpha\colon\mc L|_{S\times\infty}\to\OO_S$ at $\infty$.
\begin{remark}
	If we wanted to remove the rigidification, we can work without it as ${\op{Pic}_{C/\FF}}/\mathbb G_m$, in which case the identification with $K^\times\backslash\AA_K^\times/\widehat\OO_K^\times$ becomes an identification of bundles.
\end{remark}
Notably, $\op{Pic}_{C/\FF_q}$ lives in the short exact sequence
\[0\to\mathrm{Jac}_C\to\mathrm{Pic}_{C/\FF_q}\stackrel\deg\to\ZZ\to0,\]
where $\mathrm{Jac}_C=\mathrm{Pic}_{C/\FF_q}^\circ$. In fact, this is an avatar of the short exact sequence of the previous paragraph: after taking abelianizations and completions appropriately, class field theory provides an isomorphism
% https://q.uiver.app/#q=WzAsMTAsWzAsMSwiMCJdLFsxLDEsIlxcb3B7SmFjfV97Qy9cXEZGX3F9KFxcRkZfcSkiXSxbMiwxLCJcXHdpZGVoYXR7XFxvcHtQaWN9X3tDL1xcRkZfcX0oXFxGRl9xKX0iXSxbMywxLCJcXHdpZGVoYXRcXFpaIl0sWzQsMSwiMCJdLFswLDAsIjAiXSxbMSwwLCJcXHBpXzFee1xcbWF0aHJte1xcYWN1dGUgZXR9fShDX3tcXG92XFxGRl9xfSxcXG92XFxldGEpIl0sWzIsMCwiXFxwaV8xXntcXG1hdGhybXtcXGFjdXRlIGV0fX0oQyxcXG92XFxldGEpIl0sWzMsMCwiXFxvcHtHYWx9KFxcb3ZcXEZGX3EvXFxGRl9xKSJdLFs0LDAsIjAiXSxbNSw2XSxbNiw3XSxbNyw4XSxbOCw5XSxbMCwxXSxbMSwyXSxbMiwzXSxbMyw0XSxbNiwxXSxbNywyXSxbOCwzXV0=&macro_url=https%3A%2F%2Fraw.githubusercontent.com%2FdFoiler%2Fnotes%2Fmaster%2Fnir.tex
\[\begin{tikzcd}[cramped]
	0 & {\pi_1^{\mathrm{\acute et}}(C_{\ov\FF_q},\ov\eta)} & {\pi_1^{\mathrm{\acute et}}(C,\ov\eta)} & {\op{Gal}(\ov\FF_q/\FF_q)} & 0 \\
	0 & {\op{Jac}_{C/\FF_q}(\FF_q)} & {\widehat{\op{Pic}_{C/\FF_q}(\FF_q)}} & {\widehat\ZZ} & 0
	\arrow[from=1-1, to=1-2]
	\arrow[from=1-2, to=1-3]
	\arrow[from=1-2, to=2-2]
	\arrow[from=1-3, to=1-4]
	\arrow[from=1-3, to=2-3]
	\arrow[from=1-4, to=1-5]
	\arrow[from=1-4, to=2-4]
	\arrow[from=2-1, to=2-2]
	\arrow[from=2-2, to=2-3]
	\arrow[from=2-3, to=2-4]
	\arrow[from=2-4, to=2-5]
\end{tikzcd}\]
of short exact sequences.

Recall that our end goal is to produce a large abelian \'etale cover $C'\to C$; in particular, we want a cover with Galois group $\op{Jac}_{C/\FF_q}(\FF_q)$. The idea is to produce a large cover of $\op{Jac}_C$ and then pull back along the Abel--Jacobi map $\mathrm{AJ}\colon C\to\op{Jac}_C$ (which is given by $p\mapsto[p]-\deg p[\infty]$ on closed points). In other words, we would like an isogeny $\mathrm{Jac}_{C/\FF_q}\to\mathrm{Jac}_{C/\FF_q}$ with kernel given by the $\FF_q$-rational points. This is the Lang map.
\begin{definition}[Lang map]
	Fix a group scheme $G$ over a finite field $\FF_q$. Let $\sigma\colon G\to G$ denote the absolute Frobenius morphism. (Namely, $\sigma$ is the identity on topological spaces, and $\sigma$ is the $q$-power map on structure sheaves.) Then the \textit{Lang map} is the map $L\colon G\to G$ given by
	\[L(g)\coloneqq\sigma(g)g^{-1}.\]
\end{definition}
\begin{theorem}[Lang] \label{thm:lang}
	Fix a connected group scheme $G$ of finite type over a finite field $\FF_q$. Then the Lang map $L\colon G\to G$ is surjective.
\end{theorem}
\begin{remark}
	We are essentially trying to show that $\mathrm H^1(W_{\ov\FF_q},G(\ov\FF_q))$ is trivial. Indeed, this implies that each $1$-cocycle $c\colon W_{\ov\FF_q}\to G(\ov\FF_q)$ takes the form $\mathrm{Frob}_q\mapsto\sigma(g)g^{-1}$ for some $g$. In particular, we can define a $1$-cocycle as sending $\mathrm{Frob}_q$ to some general point of $G(\ov\FF_q)$ (and extending to be a $1$-cocycle), and then it must be a coboundary. We remark that this also implies that the continuous $\mathrm H^1(\op{Gal}(\ov\FF_q/\FF_q),G(\ov\FF_q))$ is trivial.
\end{remark}
\begin{remark}
	Here is another example application, just for fun: if $X=H\backslash G$ is some homogeneous space, then $X(\FF_q)=H(\FF_q)\backslash G(\FF_q)$. This follows by taking Galois cohomology of the exact sequence
	\[1\to H(\ov\FF_q)\to G(\ov\FF_q)\to X(\ov\FF_q)\to1.\]
\end{remark}
Anyway let's prove \Cref{thm:lang}.
\begin{proof}[Proof of \Cref{thm:lang}]
	This is a statement about $\ov\FF_q$-points, so we may adjust the scheme structure so that $G$ is geometrically reduced. For some $x\in G(\ov\FF_q)$, and we consider the coboundary action of $G$ on $G$ given by $g\cdot x\coloneqq \sigma(g)xg^{-1}$. We claim that each $x$ lies in an open orbit, which will complete the proof because there is only one open orbit because $G$ is connected. (Namely, we find that all points live in the same orbit of the coboundary action, so in particular, they are in the same orbit as the identity).

	It remains to show the claim that any given $x\in G(\ov\FF_q)$ lies in an open orbit, for which it is enough to check that the map $G_{\ov\FF_q}\to G_{\ov\FF_q}$ given by $g\mapsto\sigma(g)xg^{-1}$ is surjective on tangent spaces, where $\sigma$ is now the relative Frobenius (it is the base-change of $\sigma$ to $\ov\FF_q$). We may check this at the identity of the first $G_{\ov\FF_q}$. For this computation, we consider the more general map $G\times G\to G$ given by $(g_1,g_2)\mapsto\sigma(g_1)xg_2^{-1}$; at the identity, this map is zero on the $g_1$ coordinate (it is the Frobenius) and an isomorphism on the $g_2$ coordinate (because inversion is negation). Thus, the composite
	\[G\stackrel\Delta\to G\times G\to G,\]
	which we see is given by $g\mapsto\sigma(g)xg^{-1}$, is surjective on tangent spaces at the identity.
\end{proof}
\begin{corollary}
	Fix an abelian variety $A$ over a finite field $\FF_q$. Then $L\colon A\to A$ is an \'etale isogeny.
\end{corollary}
\begin{proof}
	We showed that $L$ is surjective, so it is an isogeny. We checked that it is smooth in the previous proof, so the result follows.
\end{proof}
\begin{remark}
	For example, this implies that $\ker L$ is smooth over $\FF_q$. Note that $\ker L\subseteq A$ has geometric points $\ker L(\ov\FF_q)=A(\FF_q)$. In fact, $\ker L$ is a constant group scheme.
\end{remark}
We may now define the \'etale cover $C'\to C$ to be the base-change sitting in the following diagram.
% https://q.uiver.app/#q=WzAsNCxbMSwxLCJcXG9we0phY31fQyJdLFsxLDAsIlxcb3B7SmFjfV9DIl0sWzAsMSwiQyJdLFswLDAsIkMnIl0sWzEsMCwiTCJdLFsyLDBdLFszLDJdLFszLDFdXQ==&macro_url=https%3A%2F%2Fraw.githubusercontent.com%2FdFoiler%2Fnotes%2Fmaster%2Fnir.tex
\begin{equation}
	\begin{tikzcd}[cramped]
		{C'} & {\op{Jac}_{C/\FF_q}} \\
		C & {\op{Jac}_{C/\FF_q}}
		\arrow["\mathrm{AJ}", from=1-1, to=1-2]
		\arrow[from=1-1, to=2-1]
		\arrow["L", from=1-2, to=2-2]
		\arrow["\mathrm{AJ}", from=2-1, to=2-2]
	\end{tikzcd} \label{eq:aj-diagram-to-shtuka}
\end{equation}
Let's compute the fiber over $\ov\infty$ on the left: this corresponds to the identity of $\op{Jac}_C$, and so the fiber over this identity is supposed to be $\ker L$, which we know to be $\op{Jac}_{C/\FF_q}(\FF_q)$. Now, deck transformations are able to act on this fiber, so we produce an action map
\[\pi_1^{\mathrm{\acute et}}(C,\ov\eta)\to\op{Aut}(\op{Jac}_{C/\FF_q}(\FF_q)).\]
By fixing the base-point $0\in\op{Jac}_{C/\FF_q}(\FF_q)$, we produce a function $\alpha$ from $\pi_1^{\mathrm{\acute et}}(C,\ov\eta)$ to $\op{Jac}_{C/\FF_q}(\FF_q)$, which we will check to be a homomorphism later. Our main claim is that
\[\alpha(\mathrm{Frob}_c)\stackrel?=\mathrm{AJ}(c)\]
for each closed point $c\in C$. This allows us to explicate some class field theory.
\begin{theorem}
	Fix everything as above. Consider the Weil group $W(C,\ov\eta)$. Then we define the map $\varphi\colon W(C,\ov\eta)\to\op{Pic}_{C/\FF_q}(\FF_q)$ by
	\[\varphi(\gamma)\coloneqq\alpha(\gamma)+\deg(\gamma)[\infty].\]
	Then $\varphi$ is the inverse of the reciprocity map.
\end{theorem}
\begin{proof}
	The map is certainly continuous, so we may check it on a dense subset of $W(C,\ov\eta)$. Thus, we may check it on Frobenius elements $\mathrm{Frob}_c$ for closed points $c\in C$, where it follows from the main claim as soon as we recall that $\mathrm{AJ}(c)=[c]-\deg(c)[\infty]$.
\end{proof}
Let's now prove the main claim. We need to unravel what $\mathrm{Frob}_c$ means: it is the image of the Frobenius through
\[\pi_1^{\mathrm{\acute et}}(c,\ov c)\into\pi_1^{\mathrm{\acute et}}(C,\ov c).\]
In particular,  to live in this subgroup, $\mathrm{Frob}_c$ is required to fix the geometric point $\ov c\into C$, so it will fix the scheme-theoretic points. Furthermore, to be the Frobenius, it needs to act as $q^{\deg(c)}$-Frobenius on the residue field of $c$. We would like to know that $\mathrm{AJ}(c)$ (which acts by multiplication on the fiber of $C'\to C$) agrees with this action of $\mathrm{Frob}_c$.
\begin{remark}
	We've gone pretty far from shtukas, so now let's have them return: it turns out that $C'$ is the moduli space $\mathrm{Sht}^0$ of shtukas for the group $\mathrm{GL}_1$ with the legs in $C\times\{\infty\}$ and coweights $(1,-1)$. This can be seen by tracking around the diagram \eqref{eq:aj-diagram-to-shtuka}. For example, for each closed point $c\in C$ of degree $1$, we produce the line bundle $\OO([c]-[\infty])$, and its fiber through $L$ consists of the rigidified line bundles $\mc L$ such that $\sigma(\mc L)\otimes\mc L^{-1}\cong\OO([c]-[\infty])$. In other words, we are admitting an isomorphism $\sigma(\mc L)\cong\mc L([c]-[\infty])$ (chosen by the rigidifications!) and hence defines a shtuka with the legs $c$ and $\infty$. The coweights are read off of the divisor $[c]-[\infty]$.
\end{remark}
We are now ready to prove the claim. We want to know that the deck transformation $\mathrm{Frob}_c$ is the same as the action of $\op{AJ}(c)$ on $\op{Jac}_{C/\FF_q}(\FF_q)$. Indeed, note $\op{AJ}(c)$ acts by $\mc L\mapsto\mc L([c]-\deg(c)[\infty])$. Thus, on the fiber over $c$, we find
\[\mc L([c]-\deg(c)[\infty])\cong\sigma^{\deg(c)}(\mc L),\]
which of course agrees with the action of $\mathrm{Frob}_c$ on the line bundle! (Namely, it is fixing the fiber, and it is acting by Frobenius on residue fields.) 

\end{document}