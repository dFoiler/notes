% !TEX root = ../notes.tex

\documentclass[../notes.tex]{subfiles}

\begin{document}

\section{February 25}

Today we will say something about shtukas.

\subsection{Drinfeld's Shtukas}
We would like to do explicit class field theory (and eventually the Langlands correspondence) for function fields $\FF_q(C)$, which is the field of functions on a curve $C$ over $\FF_q$, and we may assume that the curve is smooth proper because we are only interested in this function field. 
Shtukas will provide a global analogue for our Drinfeld modules.

Here is our definition of a shtuka.
\begin{definition}[shtuka]
	Fix a test scheme $S$ over $\FF$. An \textit{$S$-Shtuka of rank $h$ on $C$} is a triple $(\mc E,x,\tau)$ as the following.
	\begin{itemize}
		\item $\mc E$ is a vector bundle of rank $h$ on $C\times S$.
		\item $x$ is some finite collection of points $\{x_i\}_{i\in S}\subseteq C(S)$.
		\item $\tau$ is an isomorphism of vector bundles outside the points $x$ between $\mc E$ and its Frobenius twist $\mathrm{Frob}_S^*\mc E$.
	\end{itemize}
	One typically refers to the points $x$ as ``legs.''
\end{definition}
Here, $\mathrm{Frob}_S$ refers to the relative Frobenius.
\begin{example}
	Let's explain how to associate a shtuka to a Dieudonn\'e module. Indeed, a Dieudonn\'e module $M$ is a free module over $W(k)$, which can be thought of as a vector bundle over $\op{Spec}\ZZ_p$ together with a Frobenius-semilinear isomorphism $F$; further, this $F$ becomes an isomorphism after ``removing'' the special point and hence defines a shtuka.

	We note further that the Dieudonn\'e module had more data: the map $F$ extends over the leg (which is the special fiber), which is not required for a general shtuka.
\end{example}
The above example motivates us to understand what happens to $\tau$ at the legs $x$ of a shtuka. Notably, we would like to understand how precisely $\tau$ fails to be an isomorphism at the legs.

For this, fix a shtuka $(\mc E,x,\tau)$. In a neighborhood of some twist $x_i$, we may choose two isomorphisms $\mc E\otimes\OO\cong\OO^h$ and $\mathrm{Frob}_S^*\mc E\otimes\OO\cong\OO^h$, where $\OO\coloneqq\OO_{x_i}$. Then $\tau$ is providing an morphism of $\OO^h$ which upgrades to an isomorphism after inverting a uniformizer. The moral of the story is that we have produced an element of $\op{GL}_h(\mathrm{Frac}\OO)$, but it is only defined in the double coset space
\[\op{GL}_h(\OO)\backslash\op{GL}_h(\mathrm{Frac}\OO)/\op{GL}_h(\OO).\]
One can now use the Cartan decomposition to expand out this double coset space using ``dominant'' co\-weights; in short, one sends a dominant weight $(\lambda_1\ge\cdots\ge\lambda_h)$ of $\op{GL}_h$ to the element $(t^{\lambda_1},\ldots,t^{\lambda_h})$, where $t\in\OO$ is a uniformizer.
\begin{example}
	The trivial cocharacter corresponds to having the trivial double coset element, which corresponds to the isomorphism $\tau$ extending over the legs.
\end{example}
\begin{remark}
	The condition $FM\supseteq pM$ for a Dieudonn\'e module is then asking for the double coset element to look like $(p,\ldots,p,1,\ldots,1)$, which corresponds to the cocharacter having values in $\{-1,0,+1\}$, which corresponds to being a miniscule coweight. In general, miniscule conditions should be thought of as avoiding difficult technicalities (locally, where we must pass to diamonds) or being possible at all (globally).
\end{remark}
\begin{remark}
	Adding in a height $h$ constraint to the Dieudonn\'e module corresponds to controlling
	\[\dim_kM/FM=h,\]
	which controls the number of $1$s we can have in the cocharacter. %Thus, Lubin--Tate groups, which corresponded to Dieudonn\'e modules of dimension $1$
\end{remark}

\end{document}
