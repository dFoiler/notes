% !TEX root = ../notes.tex

\documentclass[../notes.tex]{subfiles}

\begin{document}

\section{February 20}

Here we go.

\subsection{Frobenius and Verschiebung}
Let's write down a Frobenius and Verschiebung for finite groups. For motivation, we recall the construction of the relative Frobenius. Fix a scheme $X$ over a field $k$ of positive characteristic $p$. Then we draw the following pullback diagram.
% https://q.uiver.app/#q=WzAsNSxbMSwxLCJYXnsocCl9Il0sWzIsMSwiWCJdLFsyLDIsIlxcU3BlYyBrIl0sWzEsMiwiXFxTcGVjIGsiXSxbMCwwLCJYIl0sWzMsMiwiXFxzaWdtYSJdLFswLDEsIlxcc2lnbWEiXSxbMCwzXSxbMSwyXSxbNCwxLCJcXHNpZ21hIiwwLHsiY3VydmUiOi0yfV0sWzQsMywiIiwwLHsiY3VydmUiOjJ9XSxbNCwwLCJGX3tYL2t9IiwxLHsic3R5bGUiOnsiYm9keSI6eyJuYW1lIjoiZGFzaGVkIn19fV1d&macro_url=https%3A%2F%2Fraw.githubusercontent.com%2FdFoiler%2Fnotes%2Fmaster%2Fnir.tex
\[\begin{tikzcd}
	X \\
	& {X^{(p)}} & X \\
	& {\Spec k} & {\Spec k}
	\arrow["{F_{X/k}}"{description}, dashed, from=1-1, to=2-2]
	\arrow["\sigma", curve={height=-12pt}, from=1-1, to=2-3]
	\arrow[curve={height=12pt}, from=1-1, to=3-2]
	\arrow["\sigma", from=2-2, to=2-3]
	\arrow[from=2-2, to=3-2]
	\arrow[from=2-3, to=3-3]
	\arrow["\sigma", from=3-2, to=3-3]
\end{tikzcd}\]
Concretely, $X^{(p)}$ is $X\to\Spec k$ pulled back everywhere by Frobenius. On affine open subsets of finite type, one writes $U=\Spec k[x_1,\ldots,x_n]/(f_1,\ldots,f_n)$, and then $X^{(p)}$ is cut out by the same variables but where the coefficients of the polynomials are $\sigma(f_\bullet)$. Then we note that our map $F_{X/k}$ above is the relative Frobenius, which is understood to send the variables $x_\bullet\mapsto x_\bullet^p$ on affine open subsets of finite type.

We are now ready to define the Frobenius and Verschiebung.
\begin{definition}[Frobenius]
	Fix an affine commutative group $G$ over a field $k$ of positive characteristic $p$. This is the map $F_G\colon G\to G^{(p)}$ given in the above construction.
\end{definition}
\begin{remark}
	Let's give a more concrete definition when the group is finite. Write $A=k[G]$ for the underlying Hopf algebra of $G$. Then our map $A^{(p)}\to A$ is given by
	\[a\mapsto\underbrace{(a\otimes\cdots\otimes a)}_p\mapsto a^p,\]
	where the last map is the $k$-linear multiplication.
\end{remark}
For the Verschiebung, one should imagine taking a Cartier dual.
\begin{definition}[Verschiebung]
	Fix an affine commutative group $G$ over a field $k$ of positive characteristic $p$. Let $A$ be the underlying Hopf algebra, and then we define the Verschiebung morphism $G^{(p)}\to G$ to be given by $A\to A^{(p)}$ by the composite
	\[\arraycolsep=1.4pt\begin{array}{ccccccc}
		A &\to& (A\otimes\cdots\otimes A)^{S_p} &\to& A^{(p)} \\
		a &\mapsto& a\otimes\cdots\otimes a &\mapsto& a
	\end{array}\]
	where the last map is uniquely defined by sending those invariant tensors as shown.
\end{definition}
\begin{remark}
	Note that this last map notably sends $(ca\otimes\cdots\otimes ca)\mapsto c^pa$, which explains why the target should be $A^{(p)}$ and $A$: this last map is not $k$-linear but only $\sigma$-semilinear!
\end{remark}
For each of our finite group examples, let's explain what happens to the Frobenius and Verschiebung.
\begin{example}
	Because Frobenius acts by $0$ on $\alpha_p$, we see that the Verschiebung is $0$ as well.
\end{example}
\begin{example}
	Frobenius acts as an isomorphism on the constant groups $\ZZ/N\ZZ$ for any $N$.
\end{example}
\begin{example}
	Frobenius acts by $0$ on $\mu_p$ and as an isomorphism on $\mu_N$ when $\gcd(p,N)=1$.
\end{example}
This suggests the following classification.
\begin{proposition}
	Fix a finite group $G$ over a base field $k$ of positive characteristic $p$.
	\begin{listalph}
		\item Then $G$ is connected if and only if the Frobenius $F$ is nilpotent.
		\item Then $G$ is \'etale if and only if the Frobenius $F$ is an isomorphism.
		\item Dually, $G$ is unipotent if and only if the Verschiebung $V$ is nilpotent.
		\item Dually, $G$ is multiplicative if and only if the Verschiebung $V$ is an isomorphism.
	\end{listalph}
\end{proposition}
These last two conditions have more intuitive descriptions.
\begin{proposition}
	Fix a finite group $G$ over a base field $k$ of positive characteristic $p$.
	\begin{listalph}
		\item The group $G$ is unipotent if and only if $G$ embeds into as a subgroup of upper-triangular unipotent matrices. More abstractly, there is a filtration of normal subgroups with quotients embedding in $\mathbb G_a$.
		\item The group $G$ is multiplicative if and only if $G$ embeds into a split torus.
	\end{listalph}
\end{proposition}

\subsection{Semilinear Algebra for Fun and Profit}
We are now ready to state a classification theorem.
\begin{theorem} \label{thm:witt-for-finite-groups}
	Fix a perfect field $k$ of positive characteristic $p$. Then the rings $W_n$ generate the category of affine unipotent finite type abelian groups over $k$ in the following sense: the functor $G\to M(G)$ given by
	\[M\colon G\mapsto\colimit\op{Hom}(G,W_\bullet)\]
	is a contravariant fully faithful embedding into $\op{End}(W_\infty)$-modules, where $W_\infty=\colimit W_n$.
\end{theorem}
\begin{remark}
	More succinctly, we can think about the functor $M$ as
	\[\op{Hom}(G,W_\infty),\]
	where $W_\infty$ is some direct limit. For example, one can think of $W_\infty(\FF_p)$ as $\QQ_p/\ZZ_p$ because the direct limit morphisms are approximately compatible with identifications
	\[W_n(\FF_p)=\ZZ_p/p^n\ZZ_p\cong\frac1{p^n}\ZZ_p/\ZZ_p\subseteq\QQ_p/\ZZ_p.\]
\end{remark}
\begin{remark}
	This is a fairly typical statement in representation theory: one has a category we'd like to understand, so we pick up some particularly interesting objects $W_\bullet$ in it, and it turns out that we can read off the entire category from these elements as modules in some sense.
\end{remark}
This is a nice result because it tells us that our $p$-power unipotent groups are found in some linear algebra.

Let's spend a moment describing $\op{End}(W_\infty)$. Of course, $W$ has an action on $W_\infty$: admittedly one has to twist the action a bit because the embeddings $W_n\to W_{n+1}$ are not actually $W$-equivariant, but one can fix it by defining
\[w\star x\coloneqq \sigma^{1-n}(w_n)x\]
for $w\in W$ and $x\in W_n$, where the multiplication on the right is happening in $W_n$. But there are more endomorphisms: there is a Frobenius morphism $F\colon W_\infty\to W_\infty$ (glued together from the Frobenius morphism above), and there is a Verschiebung $V$ morphism. In fact, we claim that these are all the endomorphisms.
\begin{definition}[Dieudonn\'e module]
	Fix a perfect field $k$ of positive characteristic. Then we define the \textit{Dieudonn\'e ring} as being generated by the Witt vectors $W(k)$ and two extra variables $F$ and $V$ subject to the requirements
	\[\begin{cases}
		F\circ\lambda = \sigma\lambda\circ F, \\
		V\circ\sigma\lambda = \lambda\circ V,
	\end{cases}\]
	where $\lambda\in W(k)$.
\end{definition}
\begin{lemma}
	Fix a perfect field $k$ of positive characteristic. Then $\op{End}_kW_n=D_k/D_kV^n$ for any $n\ge0$, and $\op{End}_kW_\infty=D_k$.
\end{lemma}
\begin{remark}
	The essential image of $M$ in \Cref{thm:witt-for-finite-groups} can now be described as requiring that $V$ acts as a nilpotent operator.
\end{remark}
\begin{remark}
	One can now restrict \Cref{thm:witt-for-finite-groups} to an equivalence between the category finite unipotent groups over $k$ and the category of $D_k$-modules which have finite length as a $W(k)$-module.
\end{remark}
With the above remark in hand, we are able to classify all $p$-power torsion groups with some Cartier duality.
\begin{theorem}
	Fix a perfect field $k$ of positive characteristic $p$. Then there is an equivalence $M$ of categories between the category of finite $p$-power torsion abelian groups over $k$ and the category of $D_k$-modules of finite length as a $W(k)$-module. It is categorized as follows.
	\begin{listroman}
		\item If $G$ is unipotent, then $M(G)$ is defined as in \Cref{thm:witt-for-finite-groups}.
		\item For any finite $G$, one has $M(G^\lor)=M(G)^\lor$.
	\end{listroman}
\end{theorem}
\begin{remark}
	In particular, it is true that \'etale-\'etale groups are never $p$-power torsion.
\end{remark}
% \subsection{Back to \texorpdfstring{$p$}{p}-Divisible Groups}
We now go back to classify $p$-divisible groups. Here is our result.
\begin{theorem}
	Fix a perfect field $k$ of positive characteristic $p$. For a $p$-divisible group $\mc G=\{G_n\}_{n\ge1}$, we define the functor $M$ out of $p$-divisible groups by
	\[M\mc G\coloneqq\limit M(G_\bullet).\]
	Then $M$ is an equivalence between the following two categories.
	\begin{itemize}
		\item The category of $p$-divisible groups (of height $h$).
		\item The category of $D_k$-modules which are free of rank $h$ as $W(k)$-modules and such that $FV=p$.
	\end{itemize}
\end{theorem}
\begin{remark}
	The condition that we have a $D_k$-action on $M$ with $FV=p$ can be recast as requiring that our modules have a $\sigma$-semilinear endomorphism $F$ (over $W(k)$) such that $pM\subseteq FM$. Roughly speaking, one can construct $V$ uniquely from this condition by attempting to invert the isomorphism
	\[pM[1/p]\cong FM[1/p].\]
\end{remark}
Let's indicate how some of these conditions come about.
\begin{itemize}
	\item The fact that $G_1$ is $p$-torsion turns out to require that $FV=p$. Namely, $G_1$ being $p$-torsion means that $FV$ should vanish on $MG_1$, and similarly, we see that $FV$ on $G_n$ becomes the map $p\colon G_n\to G_{n-1}$ which should be some kind of surjection.
	\item The rank condition comes about because one finds that $\deg G_n=p^{nh}$ for each $n\ge1$.
\end{itemize}
Lastly, we go all the way back to formal $p$-divisible groups. Because we haven't yet, let's give a definition.
\begin{definition}
	A \textit{formal $p$-divisible group} $\mc G$ of dimension $d$ consists of the data of a formal group law on the formal neighborhood $A\coloneqq k[[X_1,\ldots,X_d]]$ such that the multiplication map $[p]\colon\mc G\to\mc G$ is an isogeny (i.e., the induced map $[p]\colon A\to A$ presents $A$ as a locally free module over itself).
\end{definition}
\begin{remark}
	There is a functor sending such formal $p$-diivisible groups $\mc G$ to $p$-divisible groups given by
	\[\mc G\mapsto\left\{\mc G[p^n]\right\}_{n\ge1}.\]
	This functor is fully faithful, and it preserves dimensions in some sense.
\end{remark}
\begin{remark}
	There is a way to read the dimension of $\mc G$ off of only(!) the $p$-divisible group $\{G_n\}_{n\ge1}$ where $G_n\coloneqq\mc G[p^n]$. Roughly speaking, it turns out that one can do some tangent space computations. Notably, it turns out that $h(G)=h(G')$, but it turns out that $\dim\mc G+\dim\mc G'=h$.
\end{remark}
\begin{example}
	Here are two examples.
	\begin{itemize}
		\item The $p$-divisible group $G=\QQ_p/\ZZ_p$ has dimension $0$ because it is \'etale everywhere.
		\item The $p$-divisible group $G=\mu_{p^\infty}$ has dimension $1$ because it comes from the formal group $\widehat{\mathbb G}_m$.
	\end{itemize}
	Do note that both of these examples have height $1$.
\end{example}
\begin{remark}
	Our CM and Lubin--Tate stories can be seen as explaining what happens with our formal groups of dimension $1$ and varying height. Similarly, local Langlands for $\mathrm{GL}_2(\QQ_p)$ works with dimension $1$ and varying height. However, there is a use for higher dimension: abelian varieties of dimension $g$ produce formal $p$-divisible groups of height $2g$, and abelian varieties are relevant to us because they produce the CM theory for CM fields.
\end{remark}

\subsection{A Little Local Langlands}
Let's finish today's lectures by indicating what local Langlands looks like for $\mathrm{GL}_h$, for fixed height $h$.

Let's build a tower of moduli spaces. We begin with a moduli space $\mc M_0$ of (connected) one-dimensional $p$-divisible groups of height $h$ (over $\ZZ_p$ or something like this). More generally, for $n\ge0$, we let $\mc M_n$ be the moduli space of one-dimensional $p$-divisible groups of height $h$ but with $p^n$-level structure; for example, in the \'etale case, we choose to fix an isomorphism $G_n\cong\left(\frac1{p^n}\ZZ/\ZZ\right)^h$. This produces a tower
\[\cdots\to\mc M_{n+1}\to\mc M_n\to\mc M_{n-1}\to\cdots\to\mc M_2\to\mc M_1\to\mc M_0,\]
and we will call the limiting tower by $\mc M_\infty$. We would like to say that something about $\mathrm H^\bullet(\mc M_\infty)$ realizes the local Langlands correspondence. Namely, we would like to have something like
\[\mathrm H^\bullet(\mc M_\infty)\approx\bigoplus_{\rho\colon W_{\QQ_p}\to\mathrm{GL}_h(\QQ_p)}\rho\boxtimes\mathrm{LLC}(\rho),\]
where $\rho$ is a representation of the Weil group $W_{\QQ_p}$, and $\mathrm{LLC}(\rho)$ is a representation of $\mathrm{GL}_h(\QQ_p)$. Here are some reasons why this doesn't work.
\begin{enumerate}[label=(\roman*)]
	\item Morally, the trivializations admit some action by $\mathrm{GL}_h(\ZZ_p)$ instead of $\mathrm{GL}_h(\QQ_p)$, so something must be a bit wrong.
	\item There is an additional action by a group $J=\op{Aut}\mc G_0$, where $\mc G_0$ is the unique connected one-dimensional $p$-divisible group of height $h$ over $\ov\FF_p$. Concretely, $\mc G_0$ can be realized as $W(\ov\FF_p)^h=\ZZ_{p^\infty}^h$, where we twist the Frobenius $\sigma$-action by the automorphism
	\[\sigma\star v=\begin{bmatrix}
		0 & 1 & 0 & \cdots & 0 & 0 \\
		0 & 0 & 1 & \cdots & 0 & 0 \\
		0 & 0 & 0 & \cdots & 0 & 0 \\
		\vdots & \vdots & \vdots & \ddots & \vdots & \vdots \\
		0 & 0 & 0 & \cdots & 0 & 1 \\
		p & 0 & 0 & \cdots & 0 & 0
	\end{bmatrix}(\sigma v).\]
	Then one can realize $J$ as the units of some division algebra, which one can see from some explicit construction of the Dieudonn\'e module of $\mc G_0$. One finds that $J$ looks like the units in some division algebra $D$.
\end{enumerate}
To fix this, maybe we should write something like
\[\mathrm H^\bullet(\mc M_\infty)\approx\bigoplus_{\rho\colon W_{\QQ_p}\to\mathrm{GL}_h(\QQ_p)}\rho\boxtimes\mathrm{LLC}(\rho)\boxtimes\mathrm{JL}(\mathrm{LLC}(\rho)),\]
where the last term is a representation of $D_h(\QQ_p)$. This still doesn't work. Here is another reason.
\begin{enumerate}[label=(\roman*), resume]
	\item The moduli space $\mc M_\infty$ doesn't even make sense. We swept the trivializations under the rug, but they present a genuine problem to make $\mc M_\infty$ live over $\ZZ_p$ because our groups are frequently connected over the special fiber. Instead, $\mc M_0$ will only be able to be pro-representable over Artinian algebras: it sends an Artinian local ring $A$ with residue field $k$ to a pair $(\mc G,\iota)$, where $\mc G$ is a $p$-divisible group over $A$, and $\iota\colon\mc G_0\to\mc G|_k$ is some isomorphism.
	
	We will now only be able to lift $\mc M_0$ to the formal scheme $\op{Spf}\ZZ_p$, and the $\mc M_n$s will only be able to live over $\op{Spf}\ZZ_p$ minus a point (which is merely a rigid analytic space) because the trivializations we are asking for do not live over the special fiber! (Namely, the special fiber has all of its torsion trivial.)

	\item Any kind of rigid cohomology of $\mathrm H^\bullet(\mc M_n)$ we choose will be enormous. The problem is that we end up looking at some sort of meromorphic functions on a non-compact space, so one needs to do some fixing.

	To fix this, one should add some finiteness conditions coming from vanishing cycles next to the special fiber.
\end{enumerate}

\end{document}