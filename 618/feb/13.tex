% !TEX root = ../notes.tex

\documentclass[../notes.tex]{subfiles}

\begin{document}

\section{February 11}

I didn't have time to edit my notes from last class. Maybe I will do it over the weekend.

\subsection{Where We're Going}
For the next few classes, we will provide a guide to class field theory for function fields. Let's compare two stories.
\begin{itemize}
	\item We remark that Lubin--Tate theory provides an understanding of $\op{Gal}(K^{\mathrm{ab}}/K)$, but we remark that the local Langlands correspondence provides a full description of $\op{Gal}(\ov K/K)$ by using the moduli of $p$-divisible groups. We remark that one can understand $p$-divisible groups by semilinear algebra (using Dieudonn\'e modules). Roughly speaking, these can be generalized to local Shimura varieties, and then Scholze explains how to pass from here to local shtukas.
	\item On the other hand, for function fields $\FF_q(\Sigma)$ (where $\Sigma$ is some curve over $\FF_q$) has a description of its abelian closure via geometric class field theory, but in fact one can go to the full algebraic closure (as in the local case) by working with the moduli of shtukas. There is a classical story before shtukas (akin to working with elliptic curves) known as Drinfeld modules.
	\item Returning to our story of complex multiplication, we note that our CM theory is more or less working with moduli of abelian varieties, which in our situation come down to Shimura varieties of PEL type. This story can be generalized to more general Shimura varieties, and one would like to have some notion of shtuka, but such a thing is not available yet.
\end{itemize}

\subsection{\texorpdfstring{$p$}{p}-Divisible Groups}
Looking at the above remarks, our first task is to move to formal $p$-divisible groups (such as our Lubin--Tate groups) to $p$-divisible groups. (Here, $p$-divisible simply means that $[p]\colon\mc G\to\mc G$ is finite.) Namely, we will produce a fully faithful embedding between these categories, where formal $p$-divisible groups get sent to connected $p$-divisible groups. As for our bases, we fix a perfect field $k$ of characteristic $p>0$, and we would like our base $A$ to be an Artinian local ring with residue field $k$. (For example, one can recover local fields by taking some kind of limit over $k$.)
\begin{nex}
	The formal group $\widehat{\mathbb G}_a$ has formal group law given by $F[[X,Y]]=X+Y$. Multiplication by $p$ is given by $[p]=pX$. Thus, this is not a $p$-divisible group (over, say, the ring of integers $\OO$ of a local field).
\end{nex}
\begin{example}
	We work with the Lubin--Tate formal group $\mc G_\varpi$. Then $[\varpi]\colon\mc G_\varpi\to\mc G_\varpi$ is a finite map with finite kernel $\mc G_\varpi[\mf p]$. Taking a power of $\varpi$, we find that $[p]\colon\mc G_\varpi\to\mc G_\varpi$ continues to be a finite map. More precisely, if $(p)=\mf p^e$, then our kernel $\mc G_\varpi[\mf p^e]$ becomes a finite group scheme of order $q^e=p^{[K:\QQ_p]}$. This exponent $[K:\QQ_p]$ is called the ``height.''
\end{example}
Now, given a formal $p$-divisible groups $\mc G$, we see that we have a sequence of finite groups
\[1\into\mc G[p]\into\mc G\left[p^2\right]\into\cdots.\]
This motivates the following definition.
\begin{definition}
	A \textit{$p$-divisible group} over a base local ring $A$ is a sequence of finite flat abelian group schemes
	\[\mc G_0\into\mc G_1\into\mc G_2\into\cdots\]
	such that $\mc G_n$ is the kernel of $[p^n]\colon\mc G_{n+1}\to\mc G_{n+1}$ (which we assume to be finite) and such that there is a surjection $[p]\colon\mc G_{n+1}\to\mc G_{n+1}$. The $p$-divisible group is \textit{connected} if and only if every $\mc G_\bullet$ is connected (at the special fiber). The \textit{height} of this $p$-divisible group is $\log_p\left|G_1\right|$.
\end{definition}
\begin{remark}
	There is a lot of ambient symmetry. For example, one can take duals to receive some Cartier duality.
\end{remark}
\begin{example}
	For $\widehat{\mathbb G}_m$, we see that we get the $p$-divisible group $\mc G_n=\mu_{p^n}$, which is indeed connected. We remark that the Zariski cotangent space of $\mc G_n$ is $\ZZ/p^n\ZZ$.
\end{example}
A priori, it is a bit bizarre that one can recover a full formal $p$-divisible group by just understanding the $p$-torsion. Roughly speaking, the point is to pass to the limit of torsion to recover a formal neighborhood of the identity; for example, in the above example, taking the limit over $\ZZ/p^n\ZZ$ recovers a formal neighborhood $\ZZ_p$ of the identity.

To explain a bit of what we've done, we note that our theory of complex multiplication worked with a moduli space of elliptic curves with complex multiplication, which is a discrete moduli space. Then Lubin--Tate theory worked with some moduli space of Lubin--Tate groups (maybe with level structure to keep track of the Tate module), which was again discrete. So our next step, perhaps, is to discuss moduli of $p$-divisible groups.
\begin{remark}
	With complex multiplication, we had to fix an order $\OO$. With Lubin--Tate theory, we had to fix a uniformizer $\varpi$. In our moduli space of $p$-divisible groups, we will have to fix something to build our moduli space as well, which will be the special fiber. These choices must occur (roughly speaking) because we eventually need to produce some totally ramified extensions, and there is no canonical way to do this.
\end{remark}
Next class, we will set up the local Langlands correspondence for $\QQ_p$ by using the moduli of $p$-divisible groups of height $h$. In particular, such objects produce $h$-dimensional Galois representations.
\begin{remark}
	It is now enough to only work with $\QQ_p$ because understanding the algebraic closure of our local $p$-adic fields is the same as $\overline{\QQ}_p$.
\end{remark}

\end{document}