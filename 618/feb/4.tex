% !TEX root = ../notes.tex

\documentclass[../notes.tex]{subfiles}

\begin{document}

\section{February 4}

We began class by reviewing the proof of the Main theorem of class field theory. Today we start Lubin--Tate theory.

\subsection{Remarks on Hilbert's 12th Problem}
Let's state the Kronecker--Weber theorem in motivating way.
\begin{theorem}[Kronecker--Weber]
	The field $\QQ^{\mathrm{ab}}$ equals the field $\QQ$ adjoining the torsion points of the group scheme $\mathbb G_{m,\QQ}$.
\end{theorem}
Namely, the torsion points of $\mathbb G_{m,\QQ}$ are the roots of unity, so this amounts to saying that $\QQ^{\mathrm{ab}}=\QQ^{\mathrm{cyclo}}$, as usual.

From our theory of complex multiplication, we showed the following, which is roughly Kronecker's Jugendtraum.
\begin{theorem}[Kronecker's Jugendtraum]
	Fix an imaginary quadratic number field $K$. Then $K^{\mathrm{ab}}$ is the field $K$ after adjoining the torsion points of a certain group scheme.
\end{theorem}
This group scheme was typically an elliptic curve with complex multiplication, but sometimes we had to take a quotient for reasons related to units.

The moral of the story is that one may look for a group scheme $G$ over $\ov\QQ$ such that its torsion points generate $K^{\mathrm{ab}}$ for a given number field $K$. Perhaps we would expect to see some similar torsor of an idele class group and Galois group so that we could tell a similar story. Such a thing does not exist for general number fields currently, but there is something for function fields using the theory of shtukas.
\begin{remark}
	Note that the above examples all had $\dim G=1$. This roughly has to do with the fact that we are looking for abelian extensions. For CM fields $K$, one can attempt to look at more extensions by looking at the Galois action on the moduli space of abelian varieties.
\end{remark}
Lubin--Tate theory answers our call for an explicit class field theory, but it does not work for number fields: instead, we will work with local $p$-adic fields. We remark that many of our arguments will work in positive characteristic as well, but we will not pay so much attention to this.

\subsection{Overview of Lubin--Tate Theory}
Let's give a quick ``lay of the land'' for Lubin--Tate theory. Instead of working with an algebraic group $G$, we will work with the local analogue, which is a formal group. Approximately speaking, a formal group expressed a group in a formal neighborhood of the identity.

By way of motivation, let $G$ be a $1$-dimensional commutative algebraic group, and we let $0\in G$ be the identity. Then the formal neighborhood $\widehat G_0$ of the identity amounts to a formal group. For example, if $G$ is a group over a field $K$ of dimension $1$, then $G$ in a neighborhood of the identity looks something like $K[[X]]$, and there is a multiplication law which roughly amounts to a formal power series in $K[[X,Y]]$. Roughly speaking, this includes the tangent space (and hence the Lie algebra) as a quotient, but we will also be interested in keeping track of higher-order data. Here is our definition.
\begin{definition}[formal group]
	Fix a ring $A$. Then a \textit{formal group law} $F$ over $A$ is a formal power series $F(X,Y)\in A[[X,Y]]$ which satisfies the following conditions.
	\begin{listalph}
		\item Associativity: $F(F(X,Y),Z)=F(X,F(Y,Z))$.
		\item Identity: $F(0,Y)=Y$ and $F(X,0)=X$.
		\item Additivity: $F(X,Y)\equiv X+Y\pmod{X^2,XY,Y^2}$.
	\end{listalph}
	If in addition $F(X,Y)=F(Y,X)$, then we say that $F$ is \textit{commutative}. A \textit{formal group} $\mc G$ is the data of the formal group law $F_{\mc G}$ but labeled separately.
\end{definition}
\begin{remark}
	Perhaps one should try to distinguish between the formal group law $F$ and the actual (infinitesimal) formal group that it defines. We will not attempt to do so.
\end{remark}
\begin{remark}
	As a slightly formal comment, we remark that the composite $F(G_1(\underline X),\ldots,G_n(\underline X))$ of formal power series is well-defined provided that none of the $G_\bullet$s have constant terms. The point is that $G_\bullet(\underline X)^n$ will only have terms of degree at least $n$, so if we want to compute the coefficient of some term in $F(G_1(\underline X),\ldots,G_n(\underline X))$, we only have to compute terms of each $G_\bullet$ up to the prescribed degree. Using this construction of composition, it is not hard to check things such as an associative law $(f\circ g)\circ h=f\circ(g\circ h)$ by reducing to a computation of lower-order terms for polynomials.
\end{remark}
Technically speaking, we have defined a $1$-dimensional formal group. Here, the fact that we are working with $A[[X,Y]]$ is meant to signify that we are working in a formal neighborhood. Namely, if we simply ignored all the higher-order data, we could recover a Lie algebra.

With any object, we want to have homomorphisms.
\begin{defihelper}[homomorphism] \nirindex{formal group!homomorphism}
	Fix a ring $A$. Then a \textit{homomorphism} of formal groups $\mc G_1$ and $\mc G_2$ over $A$ is the data of a formal power series $f\in XA[[X]]$ such that
	\[F_{\mc G_2}(f(X),f(Y))=f(F_{\mc G_1}(X,Y)).\]
	Note that these composites make sense because all power series in sight lack a constant term.
\end{defihelper}
With homomorphisms, we may define modules.
\begin{definition}
	Fix rings $\OO$ and $A$ with a homomorphism $h\colon\OO\to A$. Then a \textit{formal $\OO$-module over $A$} is a commutative $1$-dimensional formal group $\mc G$ equipped with a homomorphism $\OO\to\op{End}\mc G$ lifting $h$ at the level of Lie algebras. Explicitly, we require that the endomorphism $[\alpha]\in XA[[X]]$ of $\mc G$ belonging to some $\alpha\in\OO$ satisfies
	\[[\alpha]\equiv h(A)X\pmod{X^2}.\]
\end{definition}
\begin{remark}
	Morally, the Lie algebra condition is just asserting that our $\OO$-action makes sense. The condition in this definition should be compared with \Cref{rem:lie-alg-coherence}.
\end{remark}
Let's see an example.
\begin{example}
	Consider $F$ which is the formal group law coming from $\mathbb G_m$. Namely, for $z\in\mathbb G_m$, we want our coordinate to be $X=z-1$. Then we calculate
	\[F(X,Y)=(X+1)(Y+1)-1=X+Y+XY,\]
	and it can be checked to be a formal group law. For each integer $n\in\ZZ$, there is an endomorphism on $\mathbb G_m$ given by $z\mapsto z^n$, which we can map back to an endomorphism on the level of the formal group as providing the endomorphism $[n]=(X+1)^n-1$. One can check directly that this is an endomorphism:
	\[F([n]X,[n]Y)=(X+1)^n(Y+1)^n-1=[n]F(X,Y).\]
\end{example}
\begin{remark}
	If $A=\ZZ_p$, then it turns out that the action by $\ZZ$ above can be extended to an action by $\ZZ_p$. Namely, for $n\in\ZZ_p$, one has a formal power series
	\[[n]=\sum_{i\ge1}\binom niX^i.\]
	This cannot be trivial: it works for $\ZZ_2$, but it does not work for $\ZZ_4$!
\end{remark}
Next time, we will attempt to explain the preceding remark. Approximately speaking, we will show that with $A=\OO$ (for local field $K$) and uniformizer $\varpi$ admits a unique formal $\OO$-module $F_\varpi$ over $\OO$ such that
\[[\varpi_K]X\equiv X^q\pmod{\mf p_K}.\]
For example, it follows that the torsion $F_\varpi\left[\mf p^n\right]$ are contained in some maximal ideal $\widehat{\mf p}$ belonging to an algebraic closure of $K$. Thus, we can define a Tate module
\[T_\varpi F_\varpi\coloneqq\limit F_\varpi\left[\mf p^\bullet\right].\]
It turns out that $T_\varpi F_\varpi$ is a free $\OO$-module of rank $1$, so the Galois action on torsion here grants a character
\[\op{Gal}(\ov K/K)\to\OO^\times.\]
This turns out to produce the ``totally ramified'' part of local class field theory. Explicitly, if $\op{Art}_K\colon K^\times\to\op{Gal}(K^{\mathrm{ab}}/K)$ is the local Artin reciprocity map, then the fixed field $K_\varpi\coloneqq\left(K^{\mathrm{ab}}\right)^{\op{Art}_K(\varpi)}$ is generated by the torsion elements $\bigcup_{n\ge0}F_\varpi\left[\mf p_K^n\right]$.
\begin{example}
	Let's explain what we are trying to generalize. Take $K=\QQ_p$. Then $\QQ_p^{\mathrm{ab}}=\QQ(\zeta_{p^\infty})\QQ_p^{\mathrm{unr}}$, where the unramified extensions are simply given by prime-to-$p$ cyclotomic extensions. We will find that the uniformizer $p\in\ZZ_p$ produces the formal group law $X^p-X$, whose torsion gives exactly the $p^\infty$th roots of unity $\mu_{p^\infty}$.
\end{example}
One can even recover some part of the theory of complex multiplication here.

\end{document}