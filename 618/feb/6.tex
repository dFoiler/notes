% !TEX root = ../notes.tex

\documentclass[../notes.tex]{subfiles}

\begin{document}

\section{February 6}

Today we actually start Lubin--Tate theory. Throughout, for brevity, we may refer to a $p$-adic local field as a triple $(K,\OO,\mf p)$ to mean that $K$ is a $p$-adic field, $\OO$ is its ring integer, and $\mf p$ is the unique maximal ideal of $\OO$.

\subsection{The Main Theorem}
Now, $K$ is a $p$-adic field with rings of integers $\OO\subseteq K$. For a base ring $A$, we have a notion of a formal $\OO$-module law $F$ over $A$.

We now discuss a construction which we will use throughout Lubin--Tate theory.
\begin{notation}
	Fix a $p$-adic field $K$ with ring of integers $\OO$ and maximal ideal $\mf p\subseteq\OO$. Let $(\widehat K,\widehat\OO,\widehat{\mf p})$ be a complete extension of $(K,\OO,\mf p)$. Given a formal $\OO$-module $\mc G$ over the base ring $\OO$, we note that $\widehat{\mf p}$ becomes a group with addition given by
	\[a+_{\mc G}b\coloneqq\mc G_F(a,b).\]
\end{notation}
\begin{remark}
	Let's explain why this is a group. The fact that $\widehat{\mf p}$ consists of elements of absolute value strictly less than $1$, the power series $F(X,Y)\in\OO[[X,Y]]$ will converge absolutely, so this definition at least makes sense. Then $\widehat{\mf p}$ becomes a group under this addition by directly translating the conditions of the formal group law.
\end{remark}
\begin{example}
	For example, we could take $\widehat K\in\{K,K^{\mathrm{unr}},\widehat{\ov K}\}$. In particular, the extension need not be finite or even algebraic.
\end{example}
\begin{definition}[Tate module]
	Fix a $p$-adic field $K$ with ring of integers $\OO$ and maximal ideal $\mf p\subseteq\OO$. Let $(\widehat K,\widehat\OO,\widehat{\mf p})$ be a completion of the algebraic closure of $(K,\OO,\mf p)$. Given a formal $\OO$-module $\mc G$ over the base ring $\OO$, we define the torsion subgroup
	\[\mc G[\mf p^n]\coloneqq\{x\in\widehat{\mf p}:[a]x=0\}\]
	for any $n\ge0$. We also write $\mc G[\mf p^\infty]\coloneqq\bigcup_{n\ge0}\mc G[\mf p^n]$ and the Tate module $T_\varpi \mc G=\limit\mc G[\mf p^\bullet]$.
\end{definition}
\begin{remark}
	Choose a uniformizer $\varpi\in\mf p$. Then we claim that $\mc G[\mf p^n]=\mc G[\varpi^n]$ for each $n\ge0$. Indeed, certainly $\mc G[\mf p^n]\subseteq\mc G[\varpi^n]$ because $\varpi^n\in\mf p^n$. For the reverse inclusion, we note that any $a\in\mf p^n$ takes the form $a=\varpi^nb$ for some $b\in\OO$, so $\mc G[a]\subseteq\mc G[\varpi^n]$. We conclude that $\mc G[\mf p^n]=\bigcap_{a\in\mf p^n}\mc G[a]$ is contained in $\mc G[\varpi^n]$.
\end{remark}
Now, here is our main theorem.
\begin{theorem}[Lubin--Tate] \label{thm:lt}
	Fix a $p$-adic field $K$ with ring of integers $\OO$ and maximal ideal $\mf p\subseteq\OO$, and let $\varpi\in\mf p$ be a uniformizer. Then there is a formal $\OO$-module $\mc G_\varpi$ (defined up to isomorphism) such that
	\[K_\varpi\coloneqq K(\mc G[\mf p^\infty])\]
	is a maximal totally ramified abelian extension of $K$; it is in fact the such extension fixed by $\op{Art}_K(\varpi)\in\op{Gal}(K^{\mathrm{ab}}/K)$, where $\op{Art}_K\colon K^\times\to\op{Gal}(K^{\mathrm{ab}}/K)$ is the local Artin map.
\end{theorem}
\begin{remark}
	Let's give some indication of what is going on under the hood. We will find that $T_\varpi  G_\varpi$ is a free rank-$1$ module over $\OO$. Then the Galois action of $\op{Gal}(K^{\mathrm{ab}}/K)$ on $T_\varpi G_{\varpi}$ will induce a character $\eta_\varpi\colon\op{Gal}(\ov K/K)\to\OO^\times$ defined by
	\[\eta_\varpi(\sigma)\cdot v=\sigma(v)\]
	for any $v\in T_\varpi G_\varpi$. We will show that $\eta_\varpi$ is characterized by having $\op{Art}_K^{-1}(a)=\varpi^\bullet\eta_\varpi(a)$ for some power $\varpi^\bullet$.
\end{remark}
\begin{remark}
	Roughly speaking, the ``uniqueness up to isomorphism'' corresponds to choosing a different ``local coordinate'' at the identity of the ambient group scheme.
\end{remark}
\begin{remark}
	For some torsion point $x\in\mc G[\mf p^\infty]$, we note that the ring $\OO[x]$ (and hence the field $K(x)$) is well-defined up to isomorphism of $\mc G$. Indeed, a homomorphism $\varphi\colon\mc G\to\mc G'$ translates $x$ to $\varphi_F(x)$, which we note converges absolutely because $x\in\widehat{\mf p}$. Then the completeness of $\OO[x]$ (because $K(x)$ is finite over $K$) implies hat $\varphi_F(x)\in\OO[x]$. Thus, if $\varphi$ is an isomorphism, we find $\OO[x]=\OO[\varphi_F(x)]$.
\end{remark}
We are going to use the $\mf p$-torsion to construct the maximal ramified extensions of $K$.

\subsection{Relation to Complex Multiplication}
The proof of \Cref{thm:lt} is rather elementary: one simply needs to manipulate certain power series with certain constraints. To motivate the following discussion, let's relate Lubin--Tate theory to the theory of complex multiplication we already built.

As before, we let $K$ be a quadratic imaginary extension of $\QQ$. We know that the action of $\op{Gal}(\ov K/K)$ on $Y_{\OO_K}(\infty)$ produces a character
\[\op{Gal}(K^{\mathrm{ab}}/K)\to K^\times\backslash\AA_{K,f}^\times,\]
which provides some version of explicit class field theory. We remark that an explicit computation of this map requires the choice of an elliptic curve $E$ with complex multiplication by an order $\OO\subseteq\OO_K$; in the sequel, we will go ahead and take $\OO=\OO_K$.

For a choice of finite prime $v$ (and a prime of $\ov K$ lying over it), we suppose that we want to understand the abelian extensions of $K_v$. Compatibility between local and global class field theory produces a commutative diagram
% https://q.uiver.app/#q=WzAsNCxbMCwwLCJLX3ZeXFx0aW1lcyJdLFsxLDAsIlxcb3B7R2FsfShLX3Zee1xcbWF0aHJte2FifX0vS192KSJdLFswLDEsIkteXFx0aW1lc1xcYmFja3NsYXNoXFxBQV97SyxmfV5cXHRpbWVzIl0sWzEsMSwiXFxvcHtHYWx9KEtee1xcbWF0aHJte2FifX0vSykiXSxbMCwyLCIiLDAseyJzdHlsZSI6eyJ0YWlsIjp7Im5hbWUiOiJob29rIiwic2lkZSI6InRvcCJ9fX1dLFswLDEsIiIsMix7InN0eWxlIjp7InRhaWwiOnsibmFtZSI6Imhvb2siLCJzaWRlIjoidG9wIn19fV0sWzIsMywiIiwwLHsic3R5bGUiOnsidGFpbCI6eyJuYW1lIjoiaG9vayIsInNpZGUiOiJ0b3AifX19XSxbMSwzLCIiLDIseyJzdHlsZSI6eyJib2R5Ijp7Im5hbWUiOiJkYXNoZWQifX19XV0=&macro_url=https%3A%2F%2Fraw.githubusercontent.com%2FdFoiler%2Fnotes%2Fmaster%2Fnir.tex
\[\begin{tikzcd}
	{K_v^\times} & {\op{Gal}(K_v^{\mathrm{ab}}/K_v)} \\
	{K^\times\backslash\AA_{K,f}^\times} & {\op{Gal}(K^{\mathrm{ab}}/K)}
	\arrow[hook, from=1-1, to=1-2]
	\arrow[hook, from=1-1, to=2-1]
	\arrow[dashed, from=1-2, to=2-2]
	\arrow[hook, from=2-1, to=2-2]
\end{tikzcd}\]
and so $\op{Gal}(K_v^{\mathrm{ab}}/K_v)$ embeds into $\op{Gal}(K^{\mathrm{ab}}/K)$. Thus, abelian extensions of $K_v$ can be found as closed subgroups of $\op{Gal}(K_v^{\mathrm{ab}}/K_v)$, which then produce closed subgroups of $\op{Gal}(K^{\mathrm{ab}}/K)$. The commutativity of the diagram now translates into the statements that abelian extensions of $K_v$ can be found as localizations of abelian extensions of $K$.

For simplicity, we will assume that the Hilbert class field $H$ of $\OO_K$ is contained in $K_v$, so $E$ is defined over $K_v$, and we will assume that $E$ has good reduction at $v$. Namely, $E$ becomes defined over $\OO_v$, and a formal neighborhood of the identity $e\in E$ produces a formal group $\mc G_E$ over $\OO_v$.
\begin{remark}
	Note $\mc G_E$ has an $\mathcal O_K$-action coming from $E$, and $T_\varpi \mc G_E$ also has an action by $\OO_v$, but it is not clear that these actions will agree.
\end{remark}
Now, we are interested in constructing a maximal totally ramified abelian extension of $K_v$. If $\ell$ is a rational prime away from $\mf p$, then the Galois action
\[\op{Gal}(K_v^{\mathrm{ab}}/K_v)\to\op{Aut}T_\ell E\]
has finite image for topological reasons: this is a homomorphism from a $v$-adic group to an $\ell$-adic group. So when we are on the hunt for totally ramified extensions, we should be looking for torsion at $\mf p$.

Now, \Cref{thm:lt} explains that we may as well look at $\mf p$-torsion for $\mc G_E$ already living in the maximal ideal $\widehat{\mf p}$ associated to a completion of the algebraic closure $\ov K$. This means that we expect the $\mf p$-power-torsion to reduce to $0$ in a reduction, and this is something that we can check directly.
\begin{theorem} \label{thm:supersingular-for-cm}
	Fix an elliptic curve $E$ with complex multiplication by an order $\OO$ over some imaginary quadratic field $K$. Suppose that $E$ has good reduction at some prime $\mf P$ of the Hilbert class field $H$ of $\OO$, and let $(p)$ be the rational prime lying under $\mf P$. Then the reduction $\ov E$ at $\mf P$ is supersingular if and only if there is a unique prime $\mf p$ in $K$ over $(p)$.
\end{theorem}
\begin{proof}
	The hypotheses and conclusion are all isogeny-invariant: namely, the good reduction by the N\'eron--Ogg--Shafarevich criterion, the supersingular by the criterion given by $p$-torsion, and the uniqueness of the prime $\mf p$ does not depend on $E$ at all. Thus, we can use an isogeny to transform $E$ into an elliptic curve with complex multiplication by $\OO_K$; explicitly, there is an isogeny over $\CC$ because all the lattices with CM by an order in $K$ are homothetic.

	Having $\OO=\OO_K$ is helpful for the following reason: we claim that $\mathrm{Frob}_{\mf P}$ as an endomorphism of $\ov E$ will lift to an endomorphism of $E$. If $\op{End}\ov E$ is an order in a quadratic field $K$, then there is nothing to do because we are forced to have $\op{End}\ov E=\OO_K=\op{End}E$. Otherwise, $\op{End}\ov E$ is an order in a quaternion algebra, then we note that the Frobenius commutes with the action by $\OO$, so we instead claim that the reduction
	\[\op{End}_\OO E\to\op{End}_{\OO}\ov E\]
	is surjective, which will complete the claim paragraph. Because $\op{End}(\ov E)_\QQ$ is a quaternion algebra, we see that $K$ is a maximal subfield, so its centralizer is itself, so certainly $\op{End}_\OO(E)_\QQ=K=\op{End}_\OO(\ov E)_\QQ$. However, any element in any of these endomorphism rings must be integral over $\ZZ$, so we conclude that $\op{End}_\OO(E)=\OO_K=\op{End}_\OO(\ov E)$.
	\begin{remark}
		Note that we used the trick of noting $\op{End}_\OO\ov E=\op{End}_\OO E$ previously at the end of the proof of the main theorem of complex multiplication.
	\end{remark}
	We now proceed with the proof. There are two cases.
	\begin{itemize}
		\item Suppose that $(p)$ splits as $(p)=\mf p_1\mf p_2$ in $K$. In particular, $\op N_{K/\QQ}\mf p=p$. We would like to check that $\ov E$ fails to be supersingular, which means that we are on the hunt for a nontrivial element in $\ov E[p]$. Note that it is enough to find a separable endomorphism of $\ov E$ of $p$-power degree: the kernel will be nontrivial because the morphism is separable, and it provides $p$-power torsion for degree reasons.
		
		Choose $\mf p\in\{\mf p_1,\mf p_2\}$ lying under $\mf P$, and let $\sigma(\mf p)$ be the other prime, which we note is the complex conjugate. We may choose some positive integer $m$ so that $\mf p^m$ is principal; say $\mf p^m=\mu\OO_K$. Then $\sigma(\mf p)^m=\sigma(\mu)\OO_K$. We claim that $\sigma(\mu)\colon E\to E$ is the required morphism. Here are our checks.
		\begin{itemize}
			\item The degree can is $p$-power because $\deg\mu=\deg\sigma(\mu)$, and $\mu\sigma(\mu)=\op N_{K/\QQ}\mu=p^m$.
			\item To see that $\sigma(\mu)$ is separable after the reduction, we note that $\sigma(\mu)$ acts on the tangent space $\op{Lie}E=\OO_K$ by multiplication-by-$\sigma(\mu)$,\footnote{This equality of morphisms can be checked after base-changing with $\CC$, where the nature of the complex multiplication is clear when everything is presented as $\CC$ modulo some lattice with endomorphisms given by $\OO_K$.} which is nontrivial in $\OO_K\subseteq\OO_L/\mf P$ because $\sigma(\mu)\notin\mf P$ by construction!
		\end{itemize}

		\item Suppose that $(p)$ has only prime upstairs in $K$, and let $\mf p$ be the prime living above it. Then we recall from previously that $\mathrm{Frob}_{\mf P}$ must be an endomorphism in $\op{End}E=\OO_K$, so we may find $\mu\in\OO_K$ with $\mu=\mathrm{Frob}_{\mf P}$. For example, this implies that $\op N_{K/\QQ}\mu=\deg\mathrm{Frob}_{\mf P}=\#\FF_{\mf P}$ is a power of $p$, so $\mu\in\mf p$. Now, the dual of $\mathrm{Frob}_{\mf P}$, labeled $\mathrm{Frob}_{\mf P}^\lor$ must be equal to $\sigma(\mu)$ because
		\[\mu\sigma(\mu)=\op N_{K/\QQ}\mu=\deg\mathrm{Frob}_{\mf P}=\mathrm{Frob}_{\mf P}\circ\mathrm{Frob}_{\mf P}^\lor.\]
		Thus, we also find that $\sigma(\mu)\in\mf p$; it must have the same valuation, so we are granted some unit $u\in\OO_K^\times$ such that $\sigma(\mu)=u\mu$: they have the same norm and thus both generate the same principal ideal (which is a power of $(p)$)! Then
		\[\mathrm{Frob}_{\mf P}^\lor=\sigma(\mu)=u\mathrm{Frob}_{\mf P}\]
		stays purely inseparable, so the morphism $[\deg\mathrm{Frob}_{\mf P}]$ is purely inseparable, so $\ov E[p]=1$. We conclude that $\ov E$ is supersingular.
		\qedhere
	\end{itemize}
\end{proof}
The moral of the story is that when $K_{\mf p}/\QQ_p$ is in fact quadratic (namely, there is more than one prime of $K$ living above $(p)$), all $p$-power torsion of $E$ can be seen on the level of $\widehat{\mf p}$ with group structure given by $\mc G_E$. Thus, we are actually allowed to look at some formal group with $\widehat{\mf p}$.

\subsection{Construction of Some Formal Groups}
We now begin saying something about Lubin--Tate groups. We return to $K$ being a $p$-adic field with ring of integers $\OO$ and maximal ideal $\mf p$.
\begin{theorem}[Lubin--Tate] \label{thm:lt-construction}
	Fix a $p$-adic field $(K,\OO,\mf p)$. For a given uniformizer $\varpi\in\mf p$, there is a unique (up to isomorphism) formal $\OO$-module $\mc G$ over the base $\OO$ such that
	\[[\varpi]\equiv X^{\#\FF_{\mf p}}\pmod{\mf p}.\]
\end{theorem}
\begin{remark}
	We recall that $[\varpi]$ being an endomorphism of $\mc G$ already requires that
	\[[\varpi]\equiv\varpi X\pmod{X^2},\]
	which does reduce to $0\pmod{\mf p,X^2}$.
\end{remark}
\begin{example}
	Let $\mc G$ be the formal group attached to $\mathbb G_m$ with formal group law $\mc G_F[X,Y]=(X+1)(Y+1)-1$. For a rational prime $p$, our endomorphism is given by
	\[[p](X)=(X+1)^p-1,\]
	which we note is $X^p\pmod{p}$.
\end{example}
\begin{remark}
	Let's try to motivate the condition $[\varpi]\equiv X^{\#\FF_{\mf P}}\pmod{\mf p}$. We again return to the setting of complex multiplication with the imaginary quadratic field $K$ and elliptic curve $E$ with complex multiplication by $\OO_K$. Assuming good enough reduction everywhere, we found that $\mathrm{Frob}_{\mf P}$ was found in $\op{End}(E)=\OO_K$, say equal to some $\mu\in\OO_K$. Thus, it is natural to expect that we have an endomorphism which reduces to $X^{\#\FF_{\mf p}}\pmod{\mf p}$.
\end{remark}
Let's now try to move towards a proof of \Cref{thm:lt-construction}. One can optimize a significant amount of the argument by passing to the following general proposition.
\begin{notation}
	Fix a $p$-adic field $(K,\OO,\mf p)$, and set $q\coloneqq\#\FF_{\mf p}$. For a uniformizer $\varpi$ of $K$, we define the collection $\mc F_\varpi$ as the set of $f(X)\in\OO[[X]]$ such that
	\[f(X)\equiv\begin{cases}
		\varpi X & \pmod{X^2}, \\
		X^q & \pmod{\mf p}.
	\end{cases}\]
\end{notation}
\begin{proposition} \label{prop:lt-optimize}
	Fix a $p$-adic field $(K,\OO,\mf p)$ and a uniformizer $\varpi$. For $f,g\in\mc F_\varpi$ and a linear polynomial $\ell_1(X_1,\ldots,X_n)$, there is a unique power series $\ell(X_1,\ldots,X_n)\in\OO[[X_1,\ldots,X_n]]$ satisfying the following.
	\begin{listalph}
		\item $\ell(X_1,\ldots,X_n)\equiv\ell_1(X_1,\ldots,X_n)\pmod{(X_iX_j)_{ij}}$, where $(X_iX_j)_{ij}$ refers to modding out by terms of degree at least $2$.
		\item $f\circ\ell=\ell\circ g^n$.
	\end{listalph}
\end{proposition}
\begin{proof}
	The proof is quite elementary, more or less boiling down to a manipulation of some polynomials. For brevity, we will write $\underline X$ for the full tuple $(X_1,\ldots,X_n)$, and we let $I_d$ denote the ideal of $\OO[[\underline X]]$ consisting of polynomials of degree strictly larger than $d$. Note that each class $\OO[[\underline X]]/I_d$ is uniquely represented by a polynomial of degree $d$.
	
	We will construct $\ell$ inductively. Indeed, note that a power series $\ell$ has equivalent data to a sequence of polynomials $\{\ell_d\}_{d\ge1}$ where
	\[\ell_{d'}\equiv\ell_d\pmod{I_{d+1}}\]
	whenever $d'\ge d$. Namely, one can reconstruct $\ell$ by reading the terms of degree at most $d$ from $\ell_d$. So we are tasked with constructing such a sequence $\{\ell_d\}_{d\ge1}$ of polynomials of degree $d$ with the given $\ell_1$ (in order to satisfy (i)) and so that
	\[f\circ\ell_d\equiv\ell_d\circ g^n\pmod{I_{d+1}}\]
	in order to satisfy (ii). (Namely, once we are done constructing $\ell$, we see that the computation of the terms of degree at most $d$ of both $f\circ\ell$ and $\ell\circ g^n$ is allowed to reduce all the power series to only looking at terms of degree at most $d$.) The construction of the $\{\ell_d\}_{d\ge1}$ will show that $\ell$ exists; in fact, we will show that these polynomials $\ell_d$ are unique, which then shows that the terms of $\ell$ of degree at most $d$ are unique and thus that $\ell$ itself is unique.

	Let's now proceed with the computation, which we do inductively; note that $\ell_1$ have been given. Suppose that we have already constructed $\ell_d$ uniquely, and we would like to show that $\ell_{d+1}$ is unique. If $\ell_{d+1}$ exists, then its terms of degree at most $d$ must agree with $\ell_d$ by the uniqueness, so we may as well look for $\ell_{d+1}$ of the form $\ell_d+q_{d+1}$, where $q_{d+1}$ is some homogeneous polynomial of degree $d+1$. On one hand, we see
	\[(f\circ\ell_{d+1})\equiv(f\circ\ell_d)+\varpi q_{d+1}\pmod{I_{d+2}}.\]
	On the other hand, we see
	\[(\ell_{d+1}\circ g^n)\equiv(\ell_d\circ g)+\varpi^{d+1}q_{d+1}\pmod{I_{d+2}},\]
	where we have quietly used the fact that $q_{d+1}(\varpi\underline X)=\varpi^{d+1}q_{d+1}(\underline X)$. Thus, we see that we want to show that there is a unique homogeneous polynomial $q_{d+1}$ of degree $d+1$ such that
	\[q_{d+1}\equiv\frac1\varpi\cdot\frac{(f\circ\ell_d)-(\ell_d\circ g^n)}{\varpi^{d}-1}\pmod{I_{d+2}}.\]
	The right-hand term on the right-hand side certainly defines a power series in $\OO[[X]]$ (note $\varpi^d-1\in\OO^\times$), so to check that the entire right-hand side defines a power series in $\OO[[X]]$, it is enough to check that
	\[(f\circ\ell_d)-(\ell_d\circ g^n)\stackrel?\equiv0\pmod\varpi.\]
	Well, $f(X)\equiv g(X)\equiv X^{\#\FF_{\mf p}}\pmod{\mf p}$, so this check reduces to using the Frobenius automorphism of $\FF_{\mf p}$.
\end{proof}
Everything that follows is essentially a corollary of the proposition. As one application of the proposition, we define the required ``Lubin--Tate'' formal groups.
\begin{corollary}
	Fix a $p$-adic field $(K,\OO,\mf p)$. For a given uniformizer $\varpi\in\mf p$ and $f\in\mc F_\varpi$, there is a unique commutative $1$-dimensional formal group $\mc G$ over $\OO$ such that $f$ is an endomorphism.
\end{corollary}
\begin{proof}
	\Cref{prop:lt-optimize} explains that there is a unique power series $F\in\OO[[X,Y]]$ such that $F(X,Y)\equiv X+Y\pmod{X^2,XY,Y^2}$ and $f(F(X,Y))=F(f(X),f(Y))$, so it remains to check that this $F$ is actually a commutative formal group law.
	\begin{itemize}
		\item Associativity: note that each $G\in\{F_f(X,F_f(Y,Z)),F_f(F_f(X,Y),Z)\}$ has constant term $0$, linear terms $X+Y+Z$, and satisfies $G\circ f^3=f\circ G$. However, such $G$ is unique by \Cref{prop:lt-optimize}.
		\item Commutativity: note that each $G\in\{F_f(X,Y),F_f(Y,Z)\}$ has constant term $0$, linear terms $X+Y$, and satisfies $G\circ f^2=f\circ G$. However, such $G$ is unique by \Cref{prop:lt-optimize}.
		\qedhere
	\end{itemize}
\end{proof}
\begin{definition}[Lubin--Tate formal group]
	Fix a $p$-adic field $(K,\OO,\mf p)$. For a given uniformizer $\varpi\in\mf p$ and $f\in\mc F_\varpi$, we define the \textit{Lubin--Tate formal group} $\mc G_f$ to be the unique commutative $1$-dimensional formal group $\mc G$ over $\OO$ such that $f$ is an endomorphism. If the uniformizer $\varpi$ is unclear, we may write $\mc G_{\varpi,f}$.
\end{definition}
We expect these formal groups to be isomorphic to each other and to be $\OO$-modules. Thus, we will want an ample supply of homomorphisms and endomorphisms between them. Let's see this.
\begin{corollary}
	Fix a $p$-adic field $(K,\OO,\mf p)$. Further, fix a uniformizer $\varpi\in\mf p$ and elements $f,g\in\mc F_\varpi$.
	\begin{listalph}
		\item For each $a\in\OO$, there is a unique power series $[a]_{g,f}\in X\OO[[X]]$ such that $[a]_{g,f}(X)\equiv aX\pmod{X^2}$ and $[a]_{g,f}\circ f=g\circ[a]_{g,f}$.
		\item The power series $[a]_{g,f}$ is a homomorphism $\mc G_f\to\mc G_g$ of formal groups.
	\end{listalph}
\end{corollary}
\begin{proof}
	Note (a) follows directly from \Cref{prop:lt-optimize}. For (b), we let $F_f$ and $F_g$ be the corresponding formal groups, and we see that we would like to show that
	\[F_g\circ[a]_{g,f}^2=[a]_{g,f}\circ F_f.\]
	For this, we use \Cref{prop:lt-optimize}: both sides have vanishing constant term, linear terms equal to $aX+aY$, and we see that
	\[F_g\circ[a]_{g,f}^2\circ f^2=g\circ F_g\circ[a]_{g,f}^2,\]
	and
	\[[a]_{g,f}\circ F_f\circ f^2=g\circ [a]_{g,f}\circ F_f,\]
	completing the proof by uniqueness.
\end{proof}
\begin{notation}
	Fix a $p$-adic field $(K,\OO,\mf p)$. Further, fix a uniformizer $\varpi\in\mf p$ and elements $f,g\in\mc F_\varpi$. For each $a\in\OO$, we let $[a]_{g,f}\colon\mc G_f\to\mc G_g$ be the induced formal group homomorphism. If $f=g$, we may simply write $[a]_f\coloneqq[a]_{f,f}$.
\end{notation}
\begin{example} \label{ex:lubin-tate-hom-1}
	We see $[1]_f=X$ because $X\equiv X\pmod{X^2}$ and commutes with $f$.
\end{example}
\begin{example} \label{ex:lubin-tate-hom-pi}
	We see $[\varpi]_f=f$ because $f(X)\equiv\varpi X\pmod{X^2}$ and commutes with $f$.
\end{example}
\begin{corollary} \label{cor:ring-struct-lubin-tate-hom}
	Fix a $p$-adic field $(K,\OO,\mf p)$. Further, fix a uniformizer $\varpi\in\mf p$ and elements $f,g,h\in\mc F_\varpi$ and $a,b\in\OO$.
	\begin{listalph}
		\item We have $[a+b]_{g,f}=[a]_{g,f}+[b]_{g,f}$.
		\item We have $[a]_{h,g}\circ [b]_{g,f}=[ab]_{h,f}$.
	\end{listalph}
\end{corollary}
\begin{proof}
	All power series in sight have no constant term. For (a), both sides produce a power series $q$ with linear term $(a+b)X$ and satisfying $q\circ f=g\circ q$. For (b), both sides produce a power series $q$ with linear term $abX$ and satisfying $q\circ f=h\circ q$.
\end{proof}
Now is as good as a time as any to prove \Cref{thm:lt-construction}.
\begin{proof}[Proof of \Cref{thm:lt-construction}]
	We begin with uniqueness up to isomorphism. Pick up two such formal groups $\mc G_1$ and $\mc G_2$. Because $\mc G_\bullet$ is an $\mathcal O$-module, we see the power series $f_\bullet\coloneqq[\varpi]_{\mc G_\bullet}$ satisfies $f_\bullet(X)\equiv\varpi X\pmod{X^2}$. Thus, by hypothesis, we see that $f_\bullet\in\mc F_\varpi$, so we see that $\mc G=\mc G_{f_\bullet}$ by the uniqueness of this formal group. So we have left to show that the formal groups
	\[\{\mc G_f:f\in\mc F_\varpi\}\]
	are all isomorphic. Well, $[1]_{g,f}\colon\mc G_f\to\mc G_g$ and $[1]_{f,g}\colon\mc G_g\to\mc G_f$ are inverse homomorphisms: combining \Cref{ex:lubin-tate-hom-1} with \Cref{cor:ring-struct-lubin-tate-hom}, we see that they compose to $[1]_f=X$ and $[1]_g=X$.

	For existence, we see that we will want to choose our $\mc G$ to be of the form $\mc G_f$ for some $f$, and we see that \Cref{cor:ring-struct-lubin-tate-hom} tells us that there is a ring homomorphism $\OO\to\op{End}\mc G_f$ given by $a\mapsto[a]_f$. (Note that $1\mapsto\id_{\mc G_f}$ by \Cref{ex:lubin-tate-hom-1}.) Lastly, we see that $[\varpi]=f$ has $[\varpi]\equiv X^{\#\FF_{\mf p}}\pmod{\mf p}$ by \Cref{ex:lubin-tate-hom-pi}, completing the construction.
\end{proof}

\end{document}