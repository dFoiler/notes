% !TEX root = ../notes.tex

\documentclass[../notes.tex]{subfiles}

\begin{document}

\section{January 23}

Here we go.

\subsection{Proper Ideals}
We would like to move towards proving \Cref{thm:main-cm-1}. As usual, $K$ will be an imaginary quadratic field, and we go ahead and fix an order $\OO\subseteq\OO_K$. For example, we would like to show that $E$ given by $\CC/\OO$ is in fact defined over an abelian extension of $K$. Let's begin by getting a better understanding of the class group.
\begin{lemma} \label{lem:proper-ideal-order}
	Fix an imaginary quadratic field $K$ and an order $\OO\subseteq\OO_K$. Then the following are equivalent for a fractional ideal $\mf a$ of $\OO$.
	\begin{listalph}
		\item $\mf a$ is proper.
		\item $\mf a$ is locally free of rank $1$ over $\OO$.
		\item There is a fractional ideal $\mf a^*$ such that $\mf a\mf a^*=\OO$.
	\end{listalph}
\end{lemma}
\begin{proof}
	Let $f$ be the conductor of $\OO$. Before doing anything, we introduce some notation: for a lattice $\Lambda\subseteq K$, we define the dual lattice
	\[\Lambda^\lor\coloneqq\{\alpha\in K:\alpha\Lambda\subseteq\OO_K\}.\]
	For example, we claim that $\Lambda^\lor\cong\op{Hom}_{\OO_K}(\Lambda,\OO_K)$. Indeed, given $a\in\Lambda^\lor$, multiplication by $a$ produces a morphism $\Lambda\to\OO_K$; this map is certainly injective and $\OO_K$-invariant. For surjectivity, we choose a morphism $a\colon\Lambda\to\OO_K$; by tensoring with $\QQ$, this produces a morphism $K\to K$, which must be given by multiplication by some $a\in K$, and we see $a\in\Lambda^\lor$ by construction.

	We now show the implications separately.
	\begin{itemize}
		\item We show (c) implies (b). This is some moderately technical commutative algebra. Note $\mf a\mf a^*=\OO$ implies that $\mf a\otimes_\OO\mf a^*=\OO$. Then for each prime $\mf p$ of $\OO$, we see that $\mf a_{\mf p}\otimes_{\OO_{\mf p}}\mf a^*_{\mf p}=\OO_{\mf p}$.
		
		Thus, we reduce to the following commutative algebra problem: given a local ring $R$ and two finite $R$-modules $M$ and $N$ such that there is an isomorphism $\psi\colon M\otimes_RN\to R$, we would like to show that $M$ and $N$ are free of rank $1$. It is enough to check that $M$ and $N$ are projective, which implies free (because we are over a local ring) thereby completing the proof after a rank computation. By symmetry, we may focus on $M$, and we now see that it is enough to realize $M$ inside a free $R$-module of finite rank.
		
		Well, choose $\xi\coloneqq\sum_{i=1}^nx_i\otimes y_i$ in $M\otimes_RN$ such that $\psi(\xi)=1$. Then we consider the composite
		\[M=M\otimes R\stackrel\psi=M\otimes(M\otimes N)=(M\otimes M)\otimes N\cong(M\otimes M)\otimes N\cong M\otimes(M\otimes N)\stackrel\psi=M\otimes R=M,\]
		where the $\cong$ is given by swapping the two coordinates. In total, one can compute that this automorphism of $M$ sends $x\in M$ to $x\otimes1$ to $x\otimes\xi$ to $\sum_i x\otimes x_i\otimes y_i$ to $\sum_ix_i\otimes x\otimes y_i$ to $\sum_i\psi(x\otimes y_i)x_i$. We conclude that the map $M\to R^n$ given by sending $x$ to the $n$-tuple $(\psi(x\otimes y_i))_i$ is a split monomorphism and hence provides the required embedding.

		\item We show (b) implies (c). Here, (b) implies that $\mf a$ is projective locally of rank $1$, so we may think about it as a line bundle, and we know how to invert line bundles: define $\mf a^*\coloneqq\op{Hom}_\OO(\mf a,\OO)$, which we note is a fractional ideal because (arguing as above) it may be realized as the $\OO$-stable sublattice $\{\alpha\in K:\alpha\mf a\subseteq\OO\}$. Now, $\mf a\mf a^*$ is isomorphic to $\mf a\otimes_\OO\mf a^*$, so it remains to check that
		\[\mf a\otimes_\OO\op{Hom}_\OO(\mf a,\OO)\]
		is isomorphic to $\OO$. Well, there certainly is a map to $\OO$ given by evaluation, and this map is locally an isomorphism (this amounts to checking the result at $\mf a=\OO$), so we are done.

		\item We show (c) implies (a). Choose an endomorphism $\alpha\colon\mf a\to\mf a$, and we must show that $\alpha$ is given by multiplication by an element of $\OO$. Tensoring with $\QQ$, we see that $\alpha$ must be a scalar in $K$ which we also call $\alpha$. Then we want to check $\alpha\in\OO$. Well, $\alpha\mf a\subseteq\mf a$ implies $\alpha\mf a\mf a^*\subseteq\mf a\mf a^*$, so $\alpha\in\OO$ follows.

		% \item We show (b) implies (a). Choose an endomorphism $\alpha\colon\mf a\to\mf a$, and we must show that $\alpha$ is given by multiplication by an element of $\OO$. Tensoring with $\QQ$, we see that $\alpha$ must be a scalar in $K$ which we also call $\alpha$. Then we want to check $\alpha\in\OO$. Because $\alpha\in K$ already, it is enough to check that the valuation at all primes $\mf p$ is nonnegative. Well, at each $\mf p$, we may choose an isomorphism $\mf a_{\mf p}\to\OO_{\mf p}$ of modules, and then we see that the commutativity of the diagram
		% % https://q.uiver.app/#q=WzAsNCxbMCwwLCJcXG1mIGFfe1xcbWYgcH0iXSxbMSwwLCJcXG1mIGFfe1xcbWYgcH0iXSxbMCwxLCJcXE9PX3tcXG1mIHB9Il0sWzEsMSwiXFxPT197XFxtZiBwfSJdLFswLDEsIlxcYWxwaGEiXSxbMiwzLCJcXGFscGhhIl0sWzAsMl0sWzEsM11d&macro_url=https%3A%2F%2Fraw.githubusercontent.com%2FdFoiler%2Fnotes%2Fmaster%2Fnir.tex
		% \[\begin{tikzcd}
		% 	{\mf a_{\mf p}} & {\mf a_{\mf p}} \\
		% 	{\OO_{\mf p}} & {\OO_{\mf p}}
		% 	\arrow["\alpha", from=1-1, to=1-2]
		% 	\arrow[from=1-1, to=2-1]
		% 	\arrow[from=1-2, to=2-2]
		% 	\arrow["\alpha", from=2-1, to=2-2]
		% \end{tikzcd}\]
		% implies $\alpha\in\OO_{\mf p}$ for all primes $\mf p$.

		\item We show (a) implies (c). The key difficulty is gaining access to some proper fractional ideals. We begin with a basic case: if $K=\QQ(\alpha)$ with $\alpha$ satisfying the minimal integral polynomial $ax^2+bx+c$ (so that $a\alpha\in\OO_K$), then we claim that $\ZZ+\alpha\ZZ$ is a proper fractional ideal of the order $\ZZ+a\alpha\ZZ$. Indeed, note that $\beta(\ZZ+\alpha\ZZ)\subseteq\ZZ+\alpha\ZZ$ if and only if $\beta,\beta\alpha\in\ZZ+\tau\ZZ$. So we may write out $\beta=m+n\tau$ for $m,n\in\ZZ$, but then $\beta\alpha\in\ZZ+\alpha\ZZ$ if and only if
		\[\beta\alpha-m=n\alpha^2=-\frac{cn}a+\frac{bn}a\alpha,\]
		which in turn is equivalent to $a\mid n$. We conclude that $\beta(\ZZ+\alpha\ZZ)\subseteq\ZZ+\alpha\ZZ$ if and only if $\beta\in\ZZ+a\alpha\ZZ$, as required.
		
		We are now ready to attack the implication directly. Write $\mf a=\alpha\ZZ+\beta\ZZ$ for some $\alpha,\beta\in K$; scaling $\mf a$ by an element of $K$ does not adjust the hypothesis nor the conclusion, so we may assume that $\beta=1$. Because $\mf a$ is proper, we know that $\OO=\ZZ+a\tau\ZZ$, where $ax^2+bx+c$ is the minimal integral polynomial for $\tau$. Now, let $\overline{\mf a}$ be the complex conjugate ideal, and we see that
		\[\mf a\overline{\mf a}=\ZZ+\tau\ZZ+(\tau+\ov\tau)\ZZ+(\tau\ov\tau)\ZZ.\]
		Now, $\tau+\ov\tau=-b/a$ and $\tau\ov\tau=c/a$, so we conclude that $\mf a\cdot a\overline{\mf a}=\OO$, as required.
		\qedhere
	\end{itemize}
\end{proof}
\begin{remark}
	In particular, we see that the set of proper ideals is closed under multiplication and inversion, allowing us to define $\op{Cl}(\OO)$ as we did last class.
\end{remark}
\begin{remark}
	Alternatively, we see that we can describe $\op{Cl}(\OO)$ as isomorphism classes of line bundles $\op{Pic}(\OO)$. Indeed, any fractional ideal produces a line bundle, and principal ideals are trivial line bundles, so we obtain a map $\op{Cl}(\OO)\to\op{Pic}(\OO)$. This map is surjective by the above lemma; to see that it is injective, note that a proper fractional ideal $\mf a\subseteq K$ which is isomorphic to the unit $\OO$ must be principal generated by the image of $1$ under the given $\OO$-module isomorphism $\OO\to\mf a$.
\end{remark}

\subsection{An Adelic Class Group}
To remind ourselves that class field theory should show up somewhere, we note $\op{Cl}(\OO)$ comes from a ray class group.
\begin{proposition} \label{prop:better-order-class-group}
	Fix an imaginary quadratic field $K$ and an order $\OO\subseteq\OO_K$ written as $\OO=\ZZ+f\OO_K$. Then $\op{Cl}(\OO)$ is canonically isomorphic to the following two groups.
	\begin{listalph}
		\item Ideal-theoretic: the ray class group of fractional ideals of $\OO_K$ (prime to $f$) modulo the principal ideals $(\alpha)$ (prime to $f$) such that $\alpha\pmod f$ is in $(\ZZ/f\ZZ)^\times$.
		\item Adele-theoretic: $T(\QQ)\backslash T(\AA_{\QQ,f})/T(\widehat\ZZ)$, where $T$ is the algebraic group $\OO^\times$. Explicitly, $T$ is the subgroup $\op{GL}_\OO(\OO)\subseteq\op{GL}_\ZZ(\OO)$.
	\end{listalph}
\end{proposition}
Let's give a few remarks before proceeding with the proof.
\begin{remark} \label{rem:explicit-torus-order}
	Let's be more explicit about $T$. One has that $T(R)=\op{GL}_{R\otimes\OO}(R\otimes\OO)$; here, there may be some confusion about why an $R$ appears in the subscript, but this follows by reminding ourselves that $\op{GL}_\ZZ(\OO)$ should be thought of as $\op{GL}_2(\ZZ)$ (seen by choosing a basis), so its $R$-points are $\op{GL}_2(R)$. For example $T_\QQ=\op{GL}_K(K)=\op{Res}_{K/\QQ}\mathbb G_{m,K}$.
\end{remark}
\begin{remark}
	We note $T$ is isomorphic to $\op{GL}_\OO(M)\subseteq\op{GL}_\ZZ(M)$ for any $\OO$-module $M$ which is free of rank $1$. Indeed, an isomorphism $M\cong\OO\otimes_\OO M$ produces a morphism $\op{GL}_\OO(\OO)\to\op{GL}_\OO(M)$, which can be checked to be an isomorphism locally everywhere.
\end{remark}
\begin{remark}
	It is worth keeping in a safe place the isomorphism for (a): one sends an ideal $\mf b\subseteq\OO_K$ to $\mf b\cap\OO$.
\end{remark}
\begin{remark} \label{rem:idele-double-quotient-example}
	Let's provide some motivation for the bijection between (a) and (b). Over $\QQ$, the prototypical class group looks something like
	\[\QQ^+\backslash\AA_{\QQ,f}^\times/\prod_p(1+f\ZZ_p)^\times,\]
	which we claim is isomorphic to $(\ZZ/f\ZZ)^\times$. Roughly speaking, this is by the Chinese remainder theorem. By adjusting an idele by a rational scalar, we may identify $\QQ^\times\backslash\AA_{\QQ,f}^\times$ with $\prod_p\ZZ_p^\times$. Then we see that $\ZZ_p^\times/(1+f\ZZ_p)$ is isomorphic to $\left(\ZZ/p^{\nu_p(f)}\ZZ\right)^\times$ by computing the kernel of the surjection $\ZZ_p^\times\onto\left(\ZZ/p^{\nu_p(f)}\ZZ\right)^\times$ as $1+f\ZZ_p$. The result now follows by gluing together our primes $p$ together by the Chinese remainder theorem.
\end{remark}
\begin{remark} \label{rem:class-group-as-double-quotient}
	Let's provide some geometric motivation for the bijection between $\op{Cl}(\OO)$ and the double quotient (b). Geometrically, we would like to work with a (not necessarily smooth) projective curve $C$ over $\FF_q$ with function field $F=\FF_q(C)$. Then $\op{Cl}(\OO)$ becomes $\op{Pic}(C)$, which we claim is in bijection with $F^\times\backslash\AA_F^\times/\OO_F^\times$. Well, the latter is in bijection with divisors modulo principal divisors, which we note is in bijection with $\op{Pic}(C)$ by looking at the trivializations at various points of $\mc L$.
\end{remark}
\begin{corollary} \label{cor:cl-o-finite}
	Fix an imaginary quadratic field $K$ and an order $\OO\subseteq\OO_K$ written as $\OO=\ZZ+f\ZZ$. Then $\op{Cl}(\OO)$ is finite.
\end{corollary}
\begin{proof}
	It is enough to see that $T(\QQ)\backslash T(\AA_{\QQ,f})/T(\widehat\ZZ)$ is finite. Well, by \Cref{rem:explicit-torus-order}, we see $T(\QQ)\backslash T(\AA_{\QQ,f})$ is the idele class group $K^\times\backslash\AA_K^\times$. And because $\widehat\ZZ\subseteq\AA_{\QQ,f}$ is an open subgroup, we see that $T(\widehat\ZZ)\subseteq\AA_{K,f}^\times$ continues to be an open subgroup. Properties of the topology of the idele class group allow us to conclude that our double quotient is finite.
\end{proof}
Let's now begin the proof.
\begin{proof}[Proof of \Cref{prop:better-order-class-group}]
	Before doing anything, we recall from our computation of $T$ in \Cref{rem:explicit-torus-order} that $T(\QQ)=K^\times$ and $T(\AA_{\QQ,f})=\AA_{K,f}^\times$. For this proof, for a $\ZZ$-algebra $R$, we will write $R_p$ for the ring $\OO$ localized at the set of elements coprime $p$, and we write $\widehat R_p$ for its completion; we do similar for the ring $\OO_K$. (However, we still write $\ZZ_p$ for the completion.)
	
	Let's begin with the isomorphism between (a) and (b), which is more or less purely formal. We have the following steps, following \Cref{rem:idele-double-quotient-example}.
	\begin{enumerate}
		\item Before doing anything, we compute
		\[T(\widehat\ZZ)=\prod_p(\OO\otimes_\ZZ\ZZ_p)^\times.\]
		Now, the natural inclusion $\OO\otimes_\ZZ\ZZ_p\into\OO_K\otimes_\ZZ\ZZ_p$ means that we can realize $\widehat\OO_p\coloneqq\OO\otimes_\ZZ\ZZ_p$ inside $\widehat\OO_{K,p}\coloneqq\OO_K\otimes_\ZZ\ZZ_p$. But viewing $\OO$ and $\OO_K$ as sublattices of $K$, we see $\OO=\ZZ\oplus f\tau\ZZ$ (where $\OO_K=\ZZ+\tau\ZZ$), so $\widehat\OO_p=\ZZ_p\oplus f\tau\ZZ_p$. We conclude that
		\[\widehat\OO_p^\times=\left\{\alpha\in\widehat\OO_{K,p}^\times:\alpha\pmod f\in(\ZZ_p/f\ZZ_p)^\times\right\}.\]

		\item We construct a map from (b) to (a). Begin with an idele $x\in\AA_{K,f}^\times$, and we need to produce an ideal. Well, we may adjust by an element of $K^\times$ so that $x_p\pmod f\in(\ZZ_p/f\ZZ_p)^\times$ for each $p\mid f$. (This condition should be understood by identifying $\ZZ_p$ with its image in $\widehat K^\times_p$.) Then this produces a fractional ideal
		\[\prod_{\mf p}\mf p^{\nu_{\mf p}(x_{\mf p})}.\]
		This is coprime to $f$ by construction. Here are checks that this is well-defined.
		\begin{itemize}
			\item We should check that this map does not depend on the choice of scalar in $K^\times$. Thus, if we adjust again by some other $\alpha\in K^\times$, we want to land in the same ideal class. Because we need $(\alpha x)_p\pmod f\in(\ZZ_p/f\ZZ_p)^\times$ for each $p\mid f$, we must have $\alpha\pmod f\in(\ZZ/f\ZZ)^\times$. Then we see that
			\[\prod_{\mf p}\mf p^{\nu_{\mf p}(\alpha x_{\mf p})}=(\alpha)\prod_{\mf p}\mf p^{\nu_{\mf p}(x_{\mf p})}\]
			lives in the same ideal class.
			\item We check that the ideal class is not change if we adjust $x$ by an element $y\in T(\widehat\ZZ)$. Well, for each prime $p$, we see $y_p\in\widehat\OO_p^\times$, so $y_p\pmod f\in(\ZZ_p/f\ZZ_p)^\times$, so the same is true for $x_py_p$. We conclude that we are producing the fractional ideal
			\[\prod_{\mf p}\mf p^{\nu_{\mf p}(x_{\mf p}y_{\mf p})},\]
			but of course $y_{\mf p}\in\widehat\OO_{K,p}^\times$ means that none of the valuations have actually changed.
		\end{itemize}
		While we're here, we also note that our map is surjective because this is fairly easy: for any ideal in $\OO_K$ coprime to $f$, one can read off its valuations at each prime $\mf p$ to recover an idele in $\AA_{K,f}^\times$ mapping to that ideal.

		\item We show that the constructed map is surjective. Suppose an idele $x\in T(\AA_{\QQ,f})$ goes to a principal ideal, and we want to show that $x$ is trivial in the double quotient. As in the construction of the map, we go ahead and assume that $x_p\pmod f\in(\ZZ_p/f\ZZ_p)^\times$ for each $p\mid f$. Now, are given some $\alpha\in K$ such that $\alpha\pmod f\in(\ZZ_p/f\ZZ_p)^\times$ and
		\[(\alpha)\prod_{\mf p}\mf p^{\nu_{\mf p}(x_{\mf p})}=1.\]
		Thus, we see that $\alpha x_{\mf p}\in\widehat\OO_{K,\mf p}^\times$ for each prime $\mf p$, so this lives in $\widehat\OO_p^\times$ for each $p\nmid f$ automatically. For $p\mid f$, it remains to note that $\alpha x_{\mf p}\in(\ZZ_p/f\ZZ_p)^\times$ as well by construction of $\alpha$. Synthesizing, $\alpha x\in T(\widehat\ZZ)$, implying that $x$ is trivial in the double quotient.
	\end{enumerate}
	It remains to show that $\op{Cl}(\OO)$ is the same as (b). We follow the idea of \Cref{rem:class-group-as-double-quotient}; we have the following steps.
	\begin{enumerate}
		\item We define the map from $\op{Cl}(\OO)$ to Cartier divisors. Choose $\mf a\in\op{Cl}(\OO)$. Because $\mf a$ is locally free of rank $1$, we are granted an open cover $\mc U$ of $\Spec\ZZ$ together with some isomorphisms $\varphi_U\colon\mf a_U\to\OO_U$. Now, for any two $U,V\in\mc U$, there is a composite isomorphism
		\[K=(\OO_U)_\QQ\stackrel{\varphi_U}\cong(\mf a_U)_\QQ=\mf a_\QQ=(\mf a_V)_\QQ\stackrel{\varphi_V}\cong(\OO_V)_\QQ=K\]
		of $K$-modules, so we have produced an element $\alpha_{UV}=\varphi_V\circ\varphi_U^{-1}$ in $\OO_{U\cap V}$. For example, by construction, we see that the collection $\{\alpha_{UV}\}_{U,V\in K}$ satisfies a cocycle condition $\alpha_{VW}\alpha_{UV}=\alpha_{UW}$. Thus, we have in fact provided a Cartier divisor.

		\item While we're here, we explain that each Cartier divisor $\{\alpha_{UV}\}_{U,V\in\mc U}$ does in fact produce some $\OO$-module $\mf a$ which is locally free of rank $1$. Indeed, one has ``local'' line bundles $\mf a_U\coloneqq\OO_\OO|_U$ on each $U\in\mc U$, and the elements $\alpha_{UV}\in\OO_{U\cap V}$ provide transition maps $\mf a_U|_{U\cap V}\to\mf a_V|_{U\cap V}$ which satisfy a suitable cocycle condition. The standard argument gluing sheaves is now able to glue these sheaves into a sheaf $\mf a$ on $\OO$ which is locally free of rank $1$.

		\item In the sequel, we will want to be able to check when two Cartier divisors define the same module $\mf a$. This amounts to computing the kernel of the map from Cartier divisors to line bundles on $\OO$ given in the previous paragraph. (Multiplication of Cartier divisors is defined pointwise on a refinement of the relevant open covers.)

		Well, suppose there is an isomorphism $\psi\colon\OO\to\mf a$, where $\mf a$ arises from the Cartier divisor $\{\alpha_{UV}\}_{U,V\in\mc U}$. Then each $U\in\mc U$ produces a composite $(\varphi_U\circ\psi)\colon\OO\to\OO_U$; thus, we see that $\psi$ amounts to the same amount of data as a tuple of elements $\{\beta_U\}_{U\in\mc U}$. However, for the morphisms $\beta_U\colon\OO\to\OO_U$ to glue together to a morphism $\psi\colon\OO\to\mf a$ (which will be locally an isomorphism and hence globally an isomorphism), we need the diagram
		% https://q.uiver.app/#q=WzAsMyxbMCwwLCJcXE9PIl0sWzEsMCwiXFxPT19VIl0sWzEsMSwiXFxPT19WIl0sWzAsMiwiXFxiZXRhX1YiLDJdLFswLDEsIlxcYmV0YV9VIl0sWzEsMiwiXFxhbHBoYV97VVZ9Il1d&macro_url=https%3A%2F%2Fraw.githubusercontent.com%2FdFoiler%2Fnotes%2Fmaster%2Fnir.tex
		\[\begin{tikzcd}
			\OO & {\OO_U} \\
			& {\OO_V}
			\arrow["{\beta_U}", from=1-1, to=1-2]
			\arrow["{\beta_V}"', from=1-1, to=2-2]
			\arrow["{\alpha_{UV}}", from=1-2, to=2-2]
		\end{tikzcd}\]
		to commute, which amounts to the equality $\alpha_{UV}\beta_U=\beta_V$.

		\item We define a map from Cartier divisors to the double quotient. Choose a Cartier divisor $\{\alpha_{UV}\}_{U,V\in\mc U}$. To construct our idele in $\AA_{K,f}^\times=\prod_p(\widehat K_p^\times,\widehat\OO_{K,p}^\times)$, we fix an open subset $U_0\in\mc U$, and we define $S\coloneqq(\Spec\ZZ)\setminus U_0$, which we note is a finite set. Now, for each $p\in S$, we may choose a neighborhood $U_p\in\mc U$, allowing us to construct the tuple
		\[(\alpha_{0p})_{p\in S}\in\AA^\times_{K,S}\subseteq\AA_{K,f}^\times,\]
		where $\alpha_{0p}\coloneqq\alpha_{U_0U_p}$.

		We would like to check that the element in $T(\QQ)\backslash T(\AA_{\QQ,f})/T(\widehat\ZZ)$ depends only on the choice of Cartier divisor. We go through our choices one at a time.
		\begin{itemize}
			\item Choosing a different neighborhood $U_p'$ of $p$ will adjust $\alpha_{0p}$ to some $\alpha_{0p'}$. However, $\alpha_{0p}=\alpha_{p'p}\alpha_{0p'}$ by the cocycle condition, and $\alpha_{p'p}\in\OO_p^\times$ (because $p\in U_p\cap U_p'$), so this only adjusts this coordinate by an element of $\OO_p^\times\subseteq T(\ZZ_p)$, which is legal. Thus, the tuple is well-defined in $T(\AA_{\QQ,f})/T(\widehat\ZZ)$.
			\item Shrinking $U_0$ by (say) removing a prime $q$ adds a new coordinate $\alpha_{0q}$. However, $\alpha_{0q}\in T(\ZZ_q)$ because $\alpha_{0q}$ began as providing an isomorphism $\OO_{U_0}\to\OO_{U_0}$ and hence lives in $\OO_{U_0}^\times\subseteq\OO_q^\times$. Thus, the tuple is well-defined in $T(\AA_{\QQ,f})/T(\widehat\ZZ)$.
			\item We explain that changing $U_0$ to some different $U_0'\in\mc U$ will not change the class. The previous check explains that we may shrink both $U_0$ and $U_0'$ to not adjust the class, so we may assume that $U_0$ and $U_0'$ are equal as sets. Then there is some $\alpha_{00'}\in T(\QQ)$ which allows us to identify $\alpha_{0p}=\alpha_{0'p}\alpha_{00'}$. Thus, the entire tuple is still well-defined in $T(\QQ)\backslash T(\AA_{\QQ,f})/T(\widehat\ZZ)$.
		\end{itemize}
		While we're here, we note that the above checks also explain that our map from Cartier divisors to the double quotient is well-defined up to refining the open cover of the Cartier divisor. Because isomorphisms of Cartier divisors really amount to the existence of a common refinement (by tracking through the comments in the previous step), we see that we have in fact defined a map $\op{Cl}(\OO)\to T(\QQ)\backslash T(\AA_{\QQ,f})/T(\widehat\ZZ)$.

		\item We argue that the map is surjective. Choose some $x\in T(\AA_{\QQ,f})$, which we consider as a double coset in the double quotient. For all $p\nmid f$, we see that $\widehat\OO_p^\times=\widehat\OO_{K,p}^\times$. So because $x_p\in\widehat\OO_{K,p}^\times$ for all but finitely primes $p$, we see that we can adjust $x$ by an element of $T(\widehat\ZZ)$ until $x\in\AA_{K,S}^\times$ for some finite set $S$. Furthermore, for each $p\in S$, we still know $T(\ZZ_p)\subseteq\widehat\OO_{K,p}^\times$ is an open subgroup of finite index, so one can still adjust by an element of $T(\widehat\ZZ)$ until $x_p\in K_p^\times$ for each $p\in S$.

		We are now ready to construct our Cartier divisor. The index set for our open cover will be $I\coloneqq\{0\}\cup S$, where $U_0=(\Spec\ZZ)\setminus S$, and $U_p$ is an open neighborhood of $p$ chosen small enough so that $x_p\in\OO_{U_0}^\times$. Now, we define $x_0\coloneqq1$ and define
		\[\alpha_{ij}\coloneqq x_jx_i^{-1}\]
		for any $i,j\in S$. Then the tuple $\{\alpha_{ij}\}_{i,j\in I}$ satisfies the cocycle condition and maps to $x$ by construction, so we are done.

		\item We argue that the map is injective. Suppose a Cartier divisor $\{\alpha_{UV}\}_{U,V\in\mc U}$ is trivial in the double quotient after producing the element $x\in\AA_{K,f}^\times$; we would like to show that the Cartier divisor corresponds to the trivial line bundle. Working through the construction, we may as well replace $\mc U$ with the open cover $\{U_0\}\cup\{U_p\}_{p\in S}$ where $S=(\Spec\ZZ)\setminus U_0$. (One can check that a line bundle is trivial on any open cover.) Because the Cartier divisor is trivial in the double quotient, we are granted $\beta\in K^\times$ and elements $\beta_p\in\widehat\OO_p^\times$ (for each prime $p$) such that
		\[\alpha_{0p}=\beta\beta_{0p},\]
		where $\alpha_{0p}=1$ for $p\notin S$. In particular, we conclude $\beta_{0p}\in K^\times\cap\widehat\OO_{p}^\times$, so $\beta_{0p}\in \OO_p^\times$. We take a moment to note that we can adjust $\beta$ by uniformizers of primes $\mf p$ not lying over primes $p\in S$ because this will not change $\beta_{0p}\in\OO_p^\times$; thus, we may assume $\beta\in\OO_{U_0}$.
		
		We now set $\beta_0\coloneqq\beta$ and $\beta_p\coloneqq\beta_{0p}$ for each $p\in S$ and note that $\alpha_{ij}\beta_i=\beta_j$ for any $i,j\in\{0\}\cup S$ by construction, which witnesses that the Cartier divisor is equivalent to the trivial one (upon maybe shrinking the open neighborhoods $U_p$ so that $\beta_p\in\OO_{U_p}$ for each $p\in S$).
		\qedhere
	\end{enumerate}
\end{proof}
\begin{remark} \label{rem:galois-invariant-better-cl-group}
	By construction of all the maps, one can see that the group isomorphisms are $\op{Gal}(K/\QQ)$-equivariant.
\end{remark}
\begin{remark}
	The last bijectivity check can be seen as an instance of fpqc descent. Here is an example of the statement we need: for any prime $p$ of $\ZZ$, the category of free modules over $\OO\otimes\ZZ_{(p)}$ of rank $1$ is equivalent to the category of triples $(M_\QQ,M_p,\tau)$, where $M_\QQ$ is a rank $1$ module over $\OO\otimes\QQ$, $M_p$ is a free module of rank $1$ over $\OO\otimes\ZZ_p$, and $\tau$ is an isomorphism between $M_\QQ\otimes_\QQ\QQ_p\to M_p\otimes_{\ZZ_p}\QQ_p$. (The forgetful functor can be seen to be fully faithful, and for essential surjectivity, one notes that one can recover $M$ as a $\ZZ_{(p)}$-module from the data in $\tau$, and the $\OO$-module structure is unique.) This sort of statement would allow us to work in formal neighborhoods of $p$; for example, in the injectivity check, given two $M,M'\in\op{Cl}(\OO)$, one gets to say that having $M_{(p)}\cong M'_{(p)}$ from the isomorphism on the completion, and then we can glue the isomorphisms together.
\end{remark}

\end{document}