% !TEX root = ../notes.tex

\documentclass[../notes.tex]{subfiles}

\begin{document}

\section{January 28}

Here we go.

\subsection{The Class Group Action}
We now relate our order to elliptic curves. For now, this will happen by letting $\op{Cl}(\OO)$ act on the collection of elliptic curves $E$. Over $\CC$, the action we would like to send $\CC/\mf a$ to $\CC/\mf a'$ by the action of the class $\mf a(\mf a')^{-1}$. However, because we are interested in arithmetic information, we will want to make this construction work in algebraic geometry. Here is our construction.
\begin{definition}
	Fix an algebraically closed field $k$ of characteristic $0$, and choose an order $\OO$ of a quad\-ratic imaginary field $K$. Then an elliptic curve $E$ has complex multiplication by $\OO$ if and only if there is an isomorphism $\OO\to\op{End}(E)$. We let $Y_\OO$ denote the set of such elliptic curves up to isomorphism (not necessarily preserving the isomorphism $\OO\to\op{End}(E)$).
\end{definition}
\begin{remark}
	There is no way to fix the isomorphism $\op{End}(E)\to\OO$. However, upon choosing such an isomorphism, it becomes unique up to a ring automorphism of $\OO$, which upon tensoring up to $K$ provides equivalent data to $\op{Gal}(K/\QQ)$.
\end{remark}
\begin{remark} \label{rem:lie-alg-coherence}
	Upon choosing an isomorphism $\OO\to\op{End}(E)$, taking the differential provides a map
	\[\OO\to\op{End}_k(\op{Lie}(E)).\]
	The right-hand side is $k$, so we are given a map $K\into k$. Notably, we have not embedded $K$ into $k$ to start out, so this is genuinely new information arising from a choice: changing the isomorphism $\OO\to\op{End}_k(E)$ up to the Galois element in $\op{Gal}(K/\QQ)$ will similarly adjust the embedding $K\into k$ by the same Galois element.
\end{remark}
\begin{remark} \label{rem:chosen-o-structure-from-field-embedding}
	We could alternatively choose an embedding $K\subseteq k$ and then let $Y_\OO$ be the collection of isomorphism classes of elliptic curves $E$ with complex multiplication by $\OO$, together with the choice of isomorphism $\OO\to\op{End}(E)$ to be compatible with the embedding $K\subseteq k$. The point is that exactly one of the two isomorphisms $\OO\to\op{End}(E)$ will be compatible with the embedding $K\subseteq k$ because both are uniquely determined up to an element of $\op{Gal}(K/\QQ)$.
\end{remark}
\begin{definition}
	Fix an algebraically closed field $k$ of characteristic $0$, and choose an order $\OO$ of a quad\-ratic imaginary field $K$ embedded in $k$. Given $\mf a\in\op{Cl}(\OO)$ and $E\in Y_\OO$, we define the action map
	\[\mf a\star E\coloneqq\op{Hom}_\OO(\mf a,E).\]
	Namely, we have defined an fpqc sheaf $(\mf a\star E)(S)\coloneqq\op{Hom}_\OO(\mf a,E(S))$; here, we are viewing $\mf a$ as a constant $k$-scheme, which then produces an fpqc sheaf.
\end{definition}
\begin{remark}
	Note that $\op{Hom}_\OO(\mf a,E)$ of course has an action by $\OO$ via its action on $\mf a$. Note that the action of $\OO$ on $E$ is well-defined (even though merely $E\in Y_\OO$) because we chose an embedding $K\into k$ to start!
\end{remark}
\begin{remark}
	Let's check that $\mf a\star E$ is in fact represented by an elliptic curve. Because $\mf a$ is finitely presented, we have some exact sequence $\OO^m\to\OO^n\to\mf a\to0$, so there is a short exact sequence
	\[0\to(\mf a\star E)\to E^n\to E^m\]
	of fpqc sheaves, so we realize $\op{Hom}_\OO(\mf a,E)$ as a commutative group scheme over $k$.
	
	To check that $\mf a\star E$ is an elliptic curve, we restrict $(\mf a\star E)$ as a sheaf to the category of fpqc covers $S$ of $k$ equipped with an $\mathcal O$-action, in which case $(\mf a\star E)(S)=\op{Hom}_{\OO(S)}(\mf a(S),E(S))$. Because $\mf a$ is locally free of rank $1$ (over $\OO$), we may find an open cover $\mc U$ of $\op{Spec}\OO$ trivializing $\mf a$ so that
	\[(\mf a\star E)|_U=\op{Hom}_{\OO|_U}(\mf a|_U,E|_U)=E|_U\]
	for each $U\in\mc U$. Thus, we see that $\op{Hom}_\OO(\mf a,E)|_U$ is an elliptic curve, which is a fact that glues together to tell us that $(\mf a\star E)$ is an elliptic curve.
\end{remark}
\begin{remark}
	We take a moment to note that we have produced a well-defined group action. For example, adjusting $\mf a$ or $E$ up to isomorphism of course only adjusts $\mf a\star E$ up to isomorphism, which can be seen by tracking through the construction. Additionally, we see that $\OO\star E=\op{Hom}_\OO(\OO,E)=E$, which can be seen on the level of the sheaves. Lastly, we note that
	\[(\mf a\otimes\mf b)\star E=\op{Hom}_\OO(\mf a\otimes\mf b,E)=\op{Hom}_\OO(\mf a,\op{Hom}_\OO(\mf b,E))=\mf a\star(\mf b\star E)\]
	by the tensor--hom adjunction.
\end{remark}
\begin{example} \label{ex:cl-action-on-cm-over-c}
	Over $k=\CC$, we may write $E(\CC)=\CC/\Lambda$. Then we claim that $(\mf a\star E)(\CC)\cong \CC/\left(\Lambda\mf a^{-1}\right)$. (Here, $\Lambda\mf a^{-1}$ is the product of these fractional ideals, which is a proper fractional ideal because the product of line bundles is a line bundle; see \Cref{lem:proper-ideal-order}.) Well, as $\OO$-modules, we see that
	\[\op{Hom}_\OO(\mf a,E(\CC))=\mf a^{-1}\otimes_\OO(\CC/\Lambda).\]
	Then we note $\mf a^{-1}$ is a line bundle and hence flat, so we apply $\mf a^{-1}\otimes-$ to the exact sequence
	\[0\to\Lambda\to\CC\to\CC/\Lambda\to0\]
	of $\OO$-modules to see that $\mf a^{-1}\otimes_\OO\CC/\Lambda$ is isomorphic to $\CC/\left(\mf a^{-1}\Lambda\right)$, where the embedding $\mf a^{-1}\otimes_\OO\Lambda\into K\subseteq\CC$ is given by multiplication.
\end{example}
Now, here is the punchline of our action.
\begin{proposition} \label{prop:simply-transitive-cl-action}
	Fix an algebraically closed field $k$ of characteristic $0$, and choose an order $\OO$ of a quad\-ratic imaginary field $K$. Then the action of $\op{Cl}(\OO)$ on $Y_\OO$ is simply transitive.
\end{proposition}
\begin{proof}
	Choose two elliptic curves $E$ and $E'$, and we need to show that there is a unique $\mf a\in\op{Cl}(\OO)$ such that $\mf a\star E=E'$. For this, we reduce to $\CC$: the elliptic curves $E$ and $E'$ are defined with finitely many equations, so they will be defined over an algebraically closed field $k$ of finite transcendence degree over $\QQ$, so we define $E$ and $E'$ over $\CC$. Then we may write $E(\CC)=\CC/\Lambda$ and $E'(\CC)=\CC/\Lambda'$, and according to \Cref{ex:cl-action-on-cm-over-c}, we are on the hunt for a unique class $\mf a$ such that $\Lambda\star\mf a^{-1}=\Lambda'$. This follows from the group structure of $\op{Cl}(\OO)$.
\end{proof}
\begin{corollary} \label{cor:finitely-many-cm}
	Fix an algebraically closed field $k$ of characteristic $0$, and choose an order $\OO$ of a quad\-ratic imaginary field $K$. Then the set $Y_\OO$ is finite. 
\end{corollary}
\begin{proof}
	Combine \Cref{prop:simply-transitive-cl-action} with the finiteness of $\op{Cl}(\OO)$ given in \Cref{cor:cl-o-finite}.
\end{proof}

\subsection{The Galois Action}
We now add a Galois action to the mix. Note $\op{Gal}(k/\QQ)$ acts on $Y_\OO$ by applying some $\sigma\in\op{Gal}(k/\QQ)$ directly to the equations cutting out some $E\in Y_\OO$ to produce an elliptic curve $\sigma(E)$. Then we can also apply $\sigma$ to any endomorphism of $\sigma(E)$, so $\sigma(E)$ continues to live in $Y_\OO$.

Let's check that the Galois action and the $\op{Cl}(\QQ)$-action behave.
\begin{lemma} \label{lem:galois-and-cl-action-commute}
	Fix an algebraically closed field $k$ of characteristic $0$, and choose an order $\OO$ of a quad\-ratic imaginary field $K$ embedded in $k$. Choose $\sigma\in\op{Gal}(k/\QQ)$ and $\mf a\in\op{Cl}(\OO)$, and fix an embedding $K\subseteq k$. Then, acting on $Y_\OO$, we have
	\[\sigma\circ\mf a=\sigma(\mf a)\circ\sigma.\]
\end{lemma}
\begin{proof}
	Choose some $E\in Y_\OO$, and we would like to check that $\sigma(\mf a\star E)\cong\sigma(\mf a)\star\sigma(E)$. We may do this on the level of fpqc sheaves: choose some fpqc cover $S$ of $k$, and we see that
	\begin{align*}
		\sigma(\mf a\star E)(S) &= \sigma((\mf a\star E)(S)) \\
		&= \sigma(\op{Hom}_\OO(\mf a,E)(S)) \\
		&= \sigma(\op{Hom}_\OO(\mf a,E(S))).
	\end{align*}
	On the other hand, we find
	\begin{align*}
		(\sigma(\mf a)\star\sigma(E))(S) &= \op{Hom}_\OO(\sigma(\mf a),\sigma(E))(S) \\
		&= \op{Hom}_\OO(\sigma(\mf a),\sigma(E)(S)) \\
		&= \op{Hom}_\OO(\sigma(\mf a),\sigma(E(S))).
	\end{align*}
	These two $\OO$-modules now agree by pulling out the $\sigma$ in the last equation. (We are perhaps using some assertion that $\OO$ is Galois-stable, so the subscript $\OO$ does not need to change.)
\end{proof}
Note that the action of $\op{Gal}(k/\QQ)$ on $\op{Cl}(\OO)$ will factor through $\op{Gal}(K/\QQ)$, so we really only need to understand the action of the complex conjugation element $\sigma\in\op{Gal}(K/\QQ)$ on $\op{Cl}(\OO)$.
\begin{lemma} \label{lem:galois-action-on-cl}
	Fix an order $\OO$ of a quadratic imaginary field $K$, and let $\sigma\in\op{Gal}(K/\QQ)$ be the nontrivial element. For any $\mf a\in\op{Cl}(\OO)$, we have
	\[\sigma(\mf a)=\mf a^{-1}.\]
\end{lemma}
\begin{proof}
	We are interested in showing that $\mf a\cdot\sigma(\mf a)$ is trivial in $\op{Cl}(\OO)$. Using (a) of \Cref{prop:better-order-class-group} (and noting the Galois action is the natural one by \Cref{rem:galois-invariant-better-cl-group}), it is enough to check that any prime $\mf p$ of $\OO_K$ (coprime to $f$) has $\mf p\cdot\sigma(\mf p)=(\alpha)$ for some $\alpha$ such that $\alpha\pmod f\in(\ZZ/f\ZZ)^\times$. Letting $(p)\coloneqq\mf p\cap\ZZ$ be the prime lying under $\mf p$, we find two cases.
	\begin{itemize}
		\item If $(p)$ is split or ramified, then the two primes (counted with multiplicity) above $(p)$ are $\mf p$ and $\sigma(\mf p)$, so $\mf p\cdot\sigma(\mf p)=(p)$ is trivial in $\op{Cl}(\OO)$.
		\item If $(p)$ is inert, then $\mf p=\sigma(\mf p)=(p)$, so $\mf p\cdot\sigma(\mf p)=\left(p^2\right)$ continues to be trivial in $\op{Cl}(\OO)$.
		\qedhere
	\end{itemize}
\end{proof}
The moral of the story is that we can glue our actions together to produce an action by the semidirect product $\op{Gal}(k/\QQ)\rtimes\op{Cl}(\OO)$.

Here is a punchline of having a Galois action.
\begin{proposition} \label{prop:cm-defined-over-number-field}
	Fix an algebraically closed field $k$ of characteristic $0$, and choose an order $\OO$ of a quad\-ratic imaginary field $K$. Then all elliptic curves $E\in Y_\OO$ are defined over a fixed algebraic number field.
\end{proposition}
\begin{proof}
	Define the subfield $L\subseteq k$ so that $\op{Gal}(k/L)$ is the kernel of the action map $\op{Gal}(k/\QQ)\to\op{Sym}(Y_\OO)$. Namely, because $Y_\OO$ is a finite set (by \Cref{cor:finitely-many-cm}), we see that the kernel of the action map is finite-index; additionally, the action commutes with restriction (suitably understood), so the action map is continuous, so the kernel is an open subgroup of finite index. We conclude that $L$ exists and is finite over $\QQ$.

	We now check that $L$ works. Given $E\in Y_\OO$, we would like to know that the equations cutting out $E$ can be descended to $L$. By Galois descent, it is enough to check that $E$ is isomorphic to $\sigma(E)$ for all $\sigma\in\op{Gal}(k/L)$.\footnote{This point is somewhat subtle: just because $E\cong\sigma(E)$ for all $\sigma\in\op{Gal}(k/L)$, how do we know that there is actually a model of $E$ with coefficients in $L$? This sort of question is what the machinery of (Galois) descent is supposed to answer.} However, this last statement is true by construction of $L$.
\end{proof}
\begin{remark} \label{rem:cm-defined-over-bounded-field}
	In fact, the proof shows that the degree of $L$ over $\QQ$ is at most $\#\op{Sym}(Y_\OO)=(\#\op{Cl}(\OO))!$.
\end{remark}
\begin{example} \label{ex:cm-defined-over-q}
	If $\OO$ has class number $1$, then $Y_\OO$ has only one element, so the proof shows that all the elliptic curves in $Y_\OO$ are defined over $\QQ$! For example, the elliptic curve $E\colon y^2=x^3+1$ has complex multiplication by $\ZZ[\zeta_3]\subseteq\QQ(\zeta_3)$. Note that it is important that $Y_\OO$ did not keep track of the isomorphism $\OO\into\op{End}(E)$ because this does not have to be defined over $\QQ$.
\end{example}

\subsection{Stating the Main Theorem}
We are now ready to (re)state the main theorems of complex multiplication. Roughly speaking, this says that our two actions agree under class field theory. Formally, we write down a character $\chi$ to measure how the two actions interact.
\begin{notation}
	Fix an algebraically closed field $k$ of characteristic $0$, and choose an order $\OO$ of a quad\-ratic imaginary field $K$ embedded in $k$. Then we define a character $\chi\colon\op{Gal}(k/K)\to\op{Cl}(\OO)$ by sending $\sigma$ to the element $\chi(\sigma)\in\op{Cl}(\OO)$ such that
	\[\sigma(E)=\chi(\sigma)\star E.\]
\end{notation}
\begin{remark}
	Let's check that this $\chi$ makes sense. Note that $\chi(\sigma)$ is uniquely defined given $E$ (by \Cref{prop:simply-transitive-cl-action}), and one can check that it does not depend on the choice of $E$ by checking that the equation remains true after replacing $E$ with $(\mf a\star E)$ in the equation above (using \Cref{lem:galois-and-cl-action-commute}).
\end{remark}
And here is our theorem.
\begin{theorem}[Main] \label{thm:main-cm}
	Fix an algebraically closed field $k$ of characteristic $0$, and choose an order $\OO$ of a quad\-ratic imaginary field $K$ embedded in $k$. Then $\chi$ is a quotient of the (inverse of the) Artin reciprocity map
	\[K^\times\backslash\AA_{K,f}^\times\into\op{Gal}(\ov\QQ/K)^{\mathrm{ab}},\]
	where we realize $\op{Cl}(\OO)$ as a quotient of the idele class group via \Cref{prop:better-order-class-group}. This Artin reciprocity map sends a uniformizer $\varpi_{\mf p}$ to the arithmetic Frobenius element $\mathrm{Frob}_{\mf p}$.
\end{theorem}
Here is an example corollary, extending \Cref{rem:cm-defined-over-bounded-field}.
\begin{corollary} \label{cor:hilbert-class-field-by-cm}
	Fix an algebraically closed field $k$ of characteristic $0$, and choose an order $\OO$ of a quad\-ratic imaginary field $K$ embedded in $k$. Fix any $E\in Y_\OO$.
	\begin{listalph}
		\item The elliptic curve $E$ is defined over the ring class field of $\OO$ and no smaller extension of $K$.
		\item The field $K(j(E))$ is the ring class field of $\OO$.
		\item We have $[\QQ(j(E)):\QQ]=[K(j(E)):K]=\#\op{Cl}(\OO)$.
	\end{listalph}
\end{corollary}
\begin{proof}
	Here we go. Throughout, let $H$ be the ray class field of $\OO$.
	\begin{listalph}
		\item By Galois descent, it is enough to check that $E$ is isomorphic to $\sigma(E)$ for any $\sigma\op{Gal}(k/H)$. Well, $\sigma(E)=\chi(\sigma)\star E$ by definition of $\chi$, so we would like to know that the kernel of $\chi$ is $\op{Gal}(k/H)$.
		
		We now apply \Cref{thm:main-cm}. By definition of $H$, the Artin reciprocity map provides an isomorphism
		\[\op{Cl}(\OO)\cong K^\times\backslash\AA^\times_{K,f}/\widehat\OO^\times\cong\op{Gal}(H/K),\]
		where the first isomorphism is given by \Cref{prop:better-order-class-group}. The inverse of this composite is $\chi$ by \Cref{thm:main-cm}, so we conclude that $\op{Gal}(k/H)$ is in fact the kernel of $\chi$.

		\item The field of definition of $E$ is $K(j(E))$ by properties of the $j$-invariant, so this follows from (a). Let's quickly review the argument that the field of definition of $E$ is $K(j(E))$. The coefficients of $E$ generate $K(j(E))$, so $K(j(E))$ certainly contains the field of definition of $E$. Conversely, one can write down an elliptic curve cut out by
		\[y^2=x^3-\frac{27j(E)}{j(E)-1728}x-\frac{27j(E)}{j(E)-1728}\]
		with $j$-invariant $j(E)$ and manifestly defined over $K(j(E))$. (Technically, this only works for $j\ne1728$. For $j=1728$, one can provide a separate construction of an elliptic curve with $j$-invariant $1728$.) This elliptic curve which is isomorphic to $E$ (over $k$) because the $j$-invariant determines isomorphism class over an algebraic closure; thus, $E$ is defined over $K(j(E))$.
		
		\item The second equality again follows from (a) and the fact that $K(j(E))$ is the field of definition for $E$; namely, $\#\op{Cl}(\OO)=[H:K]$.
		
		We will have to work a little harder to show $[\QQ(j(E)):\QQ]=\#\op{Cl}(\OO)$. Note $[\QQ(j(E)):\QQ]$ is equal to the degree of the minimal polynomial of $j(E)$ over $\QQ$, which equals the number of Galois conjugates of $j(E)$. However, $\sigma(j(E))=j(\sigma(E))$, so we see that we are counting the number of $j$-invariants in Galois orbit of $E\in Y_\OO$. As discussed in (a), \Cref{thm:main-cm} explains how to exchange the Galois action on $Y_\OO$ with the class group action on $Y_\OO$ via the character $\chi$. In particular, we see that $\chi$ is surjective onto $\op{Cl}(\OO)$, so the Galois orbit of $E\in Y_\OO$ is all of $Y_\OO$ and in particular has size $\#\op{Cl}(\OO)$ (using \Cref{prop:simply-transitive-cl-action}).
		% To begin, we note that $\QQ(j(E))$ is Galois over $\QQ$. Well, $K(j(E))$ is a Hilbert class field, so it is Galois over $\QQ$: we can characterize it as a maximal abelian extension with prescribed ramification (and this ramification information is defined in terms of the conductor, so it is Galois-invariant). But now $\op{Gal}(K(j(E))/\QQ(j(E)))$ is either trivial or generated by complex conjugation, so we note that complex conjugation always commutes with any other automorphism, so $\op{Gal}(K(j(E))/\QQ(j(E)))$ is a normal subgroup of $\op{Gal}(K(j(E))/\QQ)$, so $\QQ(j(E))/\QQ$ is a Galois extension.
		% Now, because $\QQ(j(E))$ is Galois over $\QQ$, the degree $[\QQ(j(E)):\QQ]$ equals the degree of the minimal polynomial of $j(E)$, which is the number of its Galois conjugates.
		\qedhere
	\end{listalph}
\end{proof}
\begin{remark}
	The algebraic numbers $j(E)$ for $E\in Y_\OO$ are called ``singular moduli'' in the literature. There is a relation to supersingular elliptic curves.
\end{remark}
In private communication with Professor Yiannis Sakellaridis, I asserted an incorrect version of the following corollary. Here is what I think is a corrected version.
\begin{corollary}
	Fix an algebraically closed field $k$ of characteristic $0$, and choose an order $\OO$ of a quadratic imaginary field $K$ embedded in $k$. Then the following are equivalent.
	\begin{listroman}
		\item The fields of definition of all $E\in Y_{\OO}$ are all equal.
		\item For any $E\in Y_{\OO}$, the field $\QQ(j(E))$ is Galois over $\QQ$.
		\item For any $E\in Y_\OO$, the extension $K(j(E))/\QQ$ is abelian.
		\item The class group $\op{Pic}(\OO)$ is $2$-torsion.
	\end{listroman}
\end{corollary}
\begin{proof}
	We show the implications separately.
	\begin{itemize}
		\item We show that (i) and (ii) are equivalent. By \Cref{thm:main-cm}, the character $\chi$ is surjective, so the Galois group $\op{Gal}(k/K)$ acts transitively on $Y_\OO$ (see \Cref{prop:simply-transitive-cl-action}). Thus, fixing any $E_0\in Y_\OO$, we find $Y_\OO=\{\sigma(E_0):\sigma\in\op{Gal}(k/\QQ)\}$, so their fields of definition are given by
		\[\{\QQ(j(\sigma(E_0))):\sigma\in\op{Gal}(k/\QQ)\}=\{\sigma(\QQ(j(E_0))):\sigma\in\op{Gal}(k/\QQ)\}.\]
		Thus, all these fields of definition are equal if and only if $\QQ(j(E_0))$ is Galois over $\QQ$.

		\item We show that (ii) and (iii) are equivalent. Of course (iii) implies (ii) because any subextension of an abelian extension succeeds at being Galois. For the converse, note that we already know $K(j(E))/\QQ$ is Galois by class field theory: one can classify $K(j(E))/K$ as the maximal abelian extension with some prescribed ramification information dictated by the conductor of $f$, which is then seen to produce a field Galois over $\QQ$. Now, given (ii), the fact that $\QQ(j(E))/\QQ$ is a Galois extension means that in fact it is an abelian extension because then the natural map
		\[\op{Gal}(K(j(E))/K)\to\op{Gal}(\QQ(j(E))/\QQ)\]
		is an isomorphism. But now $K(j(E))=K\cdot\QQ(j(E))$ is a composite of abelian extensions over $\QQ$ and hence abelian.

		\item We show that (iii) and (iv) are equivalent. The main claim is that the exact sequence
		\[1\to\op{Gal}(K(j(E))/K)\to\op{Gal}(K(j(E))/\QQ)\to\op{Gal}(K/\QQ)\to1\]
		always splits. Indeed, there is a splitting map given by inverting the natural restriction isomorphism
		\[\op{Gal}(K(j(E))/\QQ(j(E)))\to\op{Gal}(K/\QQ).\]
		We now proceed with the argument, starting with (iii). Note $\op{Gal}(K(j(E))/\QQ)$ is now a semidirect product $\op{Gal}(K/\QQ)\ltimes\op{Gal}(K(j(E))/K)$, so $\op{Gal}(K(j(E))/\QQ)$ is abelian if and only if the induced action of $\op{Gal}(K/\QQ)$ on $\op{Gal}(K(j(E))/K)$ is trivial.
		
		We now translate this into a Galois action on the class group. We need to know when the nontrivial element $\sigma\in\op{Gal}(K(j(E))/\QQ(j(E)))$ commutes with $\op{Gal}(K(j(E))/K)$. By the Chebotarev density theorem, it is enough to check this on Frobenius elements $\mathrm{Frob}_{\mf p}$. Then using the Artin reciprocity isomorphism
		\[\op{Cl}(\OO)\to\op{Gal}(K(j(E))/K)\]
		given by $[\mf p]\mapsto\mathrm{Frob}_{\mf p}$, we see $\sigma$ acts on the right by
		\[\sigma\mathrm{Frob}_{\mf p}\sigma^{-1}=\mathrm{Frob}_{\sigma\mf p}\]
		and hence on the left by the usual action of $\op{Gal}(K/\QQ)$ on the class group.

		It remains to understand when the action of $\op{Gal}(K/\QQ)$ on $\op{Cl}(\OO)$ is trivial. Well, the nontrivial element of $\op{Gal}(K/\QQ)$ acts by inversion on $\op{Cl}(\OO)$ by \Cref{lem:galois-action-on-cl}, which is a trivial action if and only if $\op{Cl}(\OO)$ is $2$-torsion.
		\qedhere
	\end{itemize}
\end{proof}
\begin{example}
	Consider the maximal order $\OO=\OO_K$ of $K=\QQ(\sqrt{-5})$. It turns out that $\op{Cl}(\OO)\cong(\ZZ/2\ZZ)$. One can show that the minimal polynomial of one of the $j(E)$ for $E\in Y_\OO$ is
	\[x^2 - 1264000x - 681472000.\]
	One can compute then that $\QQ(j(E))=\QQ(\sqrt5)$.
\end{example}

\end{document}