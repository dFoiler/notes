% !TEX root = ../notes.tex

\documentclass[../notes.tex]{subfiles}

\begin{document}

\section{September 23}
Today, we compute some Selmer groups of the congruent number of elliptic curve.

\subsection{The Weil Pairing}
Even though we are not going to use many of the results in this subsection in the future, it is useful to give some general facts and conjectures in order to build intuition about Selmer groups of elliptic curves, following \cite{poonen-rains-selmer}. For later use, we begin with a discussion of the Weil pairing, following \cite[Section~III.3]{silverman}, though we remark that one can generalize everything to abelian varieties without too much trouble.
\begin{definition}[Weil pairing]
	Fix an elliptic curve $E$ over a field $K$. For each positive integer $m$, we define the \textit{Weil pairing}
	\[e_m\colon E[m]\times E[m]\to\mu_m\]
	as follows. Fix $S,T\in E[m]$. Choose functions $f$ and $g$ in $K(E)$ for which $\op{div}f=m[T]-m[\infty]$ and $f\circ[m]=g^m$. Now, the function $E\to\PP^1$ defined by $X\mapsto g(X+S)/g(X)$ turns out to be constant, so we define $e_m(S,T)$ to be this constant value.
\end{definition}
\begin{remark} \label{rem:produce-functions-on-e}
	Let's explain why the functions $f$ and $g$ exist. The isomorphism $\op{Pic}^0(E)\to E$ of group schemes shows that a divisor $\sum_in_i[P_i]$ in $\op{Pic}^0(E)$ vanishes (i.e., arises from a function unique up to $\ov K^\times$) if and only if the associated sum $\sum_in_iP_i$ is $0$ in $E$. This explains why there is some $f$ for which $\op{div}f=m[T]-m[\infty]$. Furthermore, we can select $g$ for which $\op{div}g=[m]^*[T]-[m]^*[\infty]$, which can be computed as $\sum_{T'\in E[m]}([T+T']-[T'])$. As such, $f\circ[m]$ and $g^m$ have the same divisor, so we can force an equality by multiplying by a suitable scalar.
\end{remark}
\begin{remark}
	Let's explain why $X\mapsto g(X+S)/g(X)$ is constant and outputs to $\mu_m$. For the constancy, we note that this is a function between two connected curves, so it is enough to check that it fails to be surjective. Well, $g(X+S)^m=f(mX+mS)=f(mX)$ is also equal to $g(X)^m=f(mX)$, so $g(X+S)/g(X)$ must output to the finite set of roots of unity (or $\infty$). Thus, this function is indeed not surjective. Lastly, the output is to $\mu_m$ because there must be some $X$ for which $g(X+S)\ne\infty$ and $g(X)\ne0$ (after all, $g$ has only finitely many zeroes and poles).
\end{remark}
\begin{remark}
	The value $e_m(S,T)$ does not depend on the choices $f$ and $g$: we know $f$ is unique up to scalar, so $g$ is also unique up to scalar, so the quotient $g(X+S)/g(X)$ is well-defined.
\end{remark}
\begin{remark}
	When generalizing to abelian varieties, the correct Weil pairing is defined between an abelian variety $A$ and its ``dual'' $\op{Pic}^0A$.
\end{remark}
\begin{example} \label{ex:compute-e2}
	Here is basically the only example that can be done by hand: take $m=2$, and suppose that $E$ is the projective closure of $y^2=(x-a)(x-b)(x-c)=x^3-s_1x^2+s_2x-s_3$. We will compute $f$ and $g$ for the $2$-torsion point $T\coloneqq(a,0)$. Indeed, take $f(x)\coloneqq x-a$; this only can have a root at $T$, and it has a double root there because the tangent line is vertical. Thus, $\op{div}f=2[T]-2[\infty]$. Continuing, with the help of a computer algebra system and the doubling formula for an elliptic curve, one can check that
	\[f\circ[2]=\left(\frac{x^2-2ax-2a^2+2s_1a-s_2}{2y}\right)^2,\]
	so we may take $g$ to be the function $\left(x^2-2ax-2a^2+2s_1a-s_2\right)/2y$. We will not bother to compute $g(X+S)/g(X)$ for various points $X$ and $S$.
\end{example}
Here are our checks on this pairing.
\begin{lemma} \label{lem:weil-pairing}
	Fix an elliptic curve $E$ over a field $K$. For each positive integer $m$, the Weil pairing $e_m$ is bilinear, alternating, non-degenerate, and Galois-invariant. Furthermore, given two positive integers $m$ and $k$, we have that
	\[e_{mk}=e_m\circ([k],{\id}).\]
\end{lemma}
\begin{proof}
	We run our checks one at a time. Whenever torsion points, we will silently produce $f$ and $g$ as in the Weil pairing.
	\begin{itemize}
		\item Linear on the left: given $T\in E[m]$, we produce $f$ and $g$ as usual. Then for $S_1,S_2\in E[m]$, the identity $e_m(S_1+S_2,T)=e_m(S_1,T)e_m(S_2,T)$ can be expanded into the equality
		\[\frac{g(X+S_1+S_2)}{g(X)}=\frac{g(X+S_1+S_2)}{g(X+S_2)}\cdot\frac{g(X+S_2)}{g(X)}.\]

		\item Linear on the right: given $T_1,T_2\in E[m]$, we produce the functions $f_1$, $f_2$, $f_3$, $g_1$, $g_2$, and $g_3$, where the pair $(f_1,g_1)$ is for the torsion point $T\coloneqq T_1+T_2$. We need a way to relate these functions, so we remark that
		\[\op{div}\frac{f_3}{f_1f_2}=m([T_1+T_2]-[T_1]-[T_2]+[\infty]),\]
		so as discussed in \Cref{rem:produce-functions-on-e}, we may produce a function $h$ with $\op{div}h=[T_1+T_2]-[T_1]-[T_2]+[\infty]$. Adjusting $h$ by a scalar, we can achieve the equality $f_3=f_1f_2h^m$, so taking $m$th powers gives $g_3=g_1g_2(h\circ[m])^m$.
		
		Now, for any $S\in E[m]$, we see that $e_m(S,T_1+T_2)$ equals $g_3(X+S)/g_3(X)$, which now expands into
		\[\underbrace{\frac{g_1(X+S)}{g_1(X)}}_{e_m(S,T_1)}\cdot\underbrace{\frac{g_2(X+S)}{g_2(X)}}_{e_m(S,T_2)}\cdot\underbrace{\frac{h(mX+mS)^m}{h(mX)^m}}_{1}.\]

		\item Alternating: we need to check that $e_m(T,T)=1$ for $T\in E[m]$. Producing $f$ and $g$ as usual, we would like to show that $g(X+T)=g(X)$. The trick is to consider the function
		\[\prod_{i=0}^{m-1}f\circ\tau_{iT},\]
		where $iT$ is translation by $T$. A direct expansion with $\op{div}f=m[T]-m[\infty]$ shows that the divisor of the above function vanishes (it is $\sum_im([(i+1)T]-m[iT])$, which telescopes), so it is constant. Composing with $[m]$ and taking $m$th roots, we see that
		\[\prod_{i=0}^{m-1}g\circ\tau_{iT'}\]
		is also constant, where $T'$ has been chosen so that $mT'=T$. For example, we should get the same value plugging in $X$ and $X+T'$. Taking the quotient causes the terms $0<i<m-1$ to vanish from both products, leaving us with $g(X+T)/g(X)=1$.

		\item Non-degenerate: because the pairing is already alternating, it is enough to show that $e_m(-,T)=1$ implies that $T=\infty$. Well, choose $f$ and $g$ as usual, and we are given that $g(X+S)=g(X)$ for any $S\in E[m]$. Thus, $E$ factors through the elliptic curve $E/E[m]$, so we receive a function $h$ for which $g=h\circ[m]$. But now $h^m=f$, so $\op{div}h=[T]-[\infty]$. Because $E\ne\PP^1$, we are forced to have $T=\infty$.

		\item Galois-invariant: fix $S,T\in E[m]$, and choose $\sigma\in\op{Gal}(K^{\mathrm{sep}}/K)$. Picking up functions $f$ and $g$ as usual, we note that $(\sigma f)\circ[m]=(\sigma g)^m$ and $\op{div}\sigma f=m[\sigma T]-m[\infty]$, so $e_m(\sigma S,\sigma T)$ is
		\[\frac{\sigma g(\sigma X+\sigma S)}{\sigma g(\sigma X)}=\sigma\left(\frac{g(X+S)}{g(X)}\right),\]
		which of course is $\sigma e_m(S,T)$.

		\item Lastly, we need to check that $e_{mk}(S,T)=e_m(kS,T)$ for $S\in E[mk]$ and $T\in E[m]$. Well, choose $f$ and $g$ as usual, and then we note that $f^k$ and $g\circ[k]$ work to define $e_{mk}$, so $e_{mk}(S,T)$ equals
		\[\frac{g(kX+kS)}{g(kX)},\]
		which is $e_m(kS,T)$.
		\qedhere
	\end{itemize}
\end{proof}
In the sequel, it will be helpful to have another presentation of the Weil pairing.
\begin{notation}
	Fix a smooth proper curve $C$ over a field $K$. For each function $f\in C(K)$ and divisor $D$ on $K$ such that $\op{supp}D$ and $\op{supp}\op{div}f$ are disjoint, we write $D=\sum_in_i[P_i]$ and define
	\[f(D)=\prod_if(P_i)^{n_i}.\]
\end{notation}
\begin{definition}[Weil pairing]
	Fix an elliptic curve $E$ over a field $K$. For each positive integer $m$, we define the \textit{Weil pairing}
	\[\widetilde e_m\colon E[m]\times E[m]\to\mu_m\]
	as follows. Given $S,T\in E[m]$, choose divisors $D_S$ and $D_T$ of disjoint support such that $D_S\equiv[S]-[\infty]$ and $D_T\equiv[T]-[\infty]$. Then choose functions $f_S$ and $f_T$ for which $\op{div}f_S=mD_S$ and $\op{div}f_T=mD_T$, and we define $\widetilde e_m(S,T)\coloneqq f_S(D_T)/f_T(D_S)$.
\end{definition}
\begin{remark}
	It is possible to find divisors $D_S$ and $D_T$ by (say) taking $D_S\coloneqq[S]-[\infty]$ and $D_T\coloneqq[R+S]-[R]$ for some auxiliary torsion point $R\in E(\ov K)$ of order larger than $m$. Then $f_S$ and $f_T$ are uniquely defined up to scalars (see \Cref{rem:produce-functions-on-e}), and the values $f_S(D_T)$ and $f_T(D_S)$ are well-defined because the supports are disjoint, and $\deg D_T=\deg D_S=0$. Note that we have not yet shown that $\widetilde e$ is independent of the choice of divisors $D_S$ and $D_T$!
\end{remark}
To see that this definition is well-defined (and agrees with the one provided earlier), we need the following tool.
\begin{theorem}[Weil reciprocity] \label{thm:weil-recip}
	Fix a smooth proper curve $C$ over an algebraically closed field $K$. For any $f,g\in K(C)$ with disjoint supports, we have
	\[f(\op{div}g)=g(\op{div}f).\]
\end{theorem}
\begin{proof}
	We have two steps.
	\begin{enumerate}
		\item We handle the case where $C=\PP^1$. Here, $f$ and $g$ are just some rational functions in the coordinate $x$. By changing coordinates (and using the fact that $\ov K$ is infinite while $\op{supp}f$ and $\op{supp}g$ are finite), we may assume that neither $f$ nor $g$ have neither a root nor pole at $\infty$. After adjusting by scalars (which does not change the validity of the conclusion), we may now set
		\[f(x)=\prod_{i=1}^M(x-a_i)^{m_i}\qquad\text{and}\qquad g(x)=\prod_{j=1}^N(x-b_j)^{n_j}\]
		for some $a_1,\ldots,a_M,b_1,\ldots,b_N\in K$ and $m_1,\ldots,m_M,n_1,\ldots,n_N\in\ZZ$. Because $\op{supp}f$ and $\op{supp}g$ avoid $\infty$, we have $\op{div}f=\sum_im_i[a_i]$ and $\op{div}g=\sum_jn_j[b_j]$. Now, on one hand $f(\op{div}g)$ equals
		\[\prod_{j=1}^Nf(b_j)^{n_j}=\prod_{j=1}^N\prod_{i=1}^M(b_j-a_i)^{m_in_j}.\]
		Similarly, we find that $g(\op{div}f)$ equals
		\[\prod_{i=1}^M\prod_{j=1}^N(a_i-b_j)^{m_in_j}.\]
		Now, these two values differ by $(-1)$ to the power of $\sum_{i,j}m_in_j$, which of course is zero because $\sum_im_i=\sum_jn_j=0$.
		\item We handle the general case. View $g$ as a rational map $C\to\AA^1$, which can then be extended to a regular map $g\colon C\to\PP^1$ because $C$ is proper. We now formally manipulate divisors. Note that $\op{div}g=g^*\op{div}\id_{\PP^1}$ by definition, so $f(\op{div}g)$ equals
		\[f(g^*\op{div}{\id_{\PP^1}})=(g_*f)(\op{div}{\id_{\PP^1}}).\]
		Now, by the first step, this is
		\[\id_{\PP^1}(\op{div}g_*f)=\id_{\PP^1}(g_*\op{div}f).\]
		This right-hand side collapses to $g(\op{div}f)$, so we are done.
		\qedhere
	\end{enumerate}
\end{proof}
We now give a few remarks about our definition of $\widetilde e$.
\begin{remark}
	We show that $\widetilde e$ is independent of the choices of $D_S$ and $D_T$. By symmetry, it will be enough to show that it is independent of the choice of $D_S$. Accordingly, suppose we choose another divisor $D_S'$ linearly equivalent to $D_S$, and then we choose another function $f_S'$ with divisor equal to $mD_S'$. Then we want to show that
	\[\frac{f_S(D_T)}{f_T(D_S)}\stackrel?=\frac{f_S'(D_T)}{f_T(D_S')}.\]
	Well, $D_S\equiv D_S'$ implies that there is a function $g$ with $\op{div}g=D_S'-D_S$. Then we see $\op{div}f_S'=\op{div}f_Sg^m$, so after adjusting by a scalar, we may assume that $f_S'=f_Sg^m$. The above equality then rearranges into $f_T(\op{div}g)=g(D_T)^m$, which follows from \Cref{thm:weil-recip}.
\end{remark}
\begin{remark}
	We note that we can directly check that $\widetilde e(S,T)^m=1$. Indeed, this quotient is
	\[\frac{f_S(mD_T)}{f_T(mD_S)}=\frac{f_S(\op{div}f_T)}{f_T(\op{div}f_S)},\]
	which is $1$ by \Cref{thm:weil-recip}.
\end{remark}
\begin{proposition} \label{prop:alternative-weil}
	Fix an elliptic curve $E$ over a field $K$ and a positive integer $m$. For any $S,T\in E[m]$, we have
	\[\widetilde e(S,T)=e(T,S).\]
\end{proposition}
\begin{proof}
	We follow \cite[Exercise~3.16]{silverman}; note that an argument for this exercise is provided in the back matter. For $m=1$ or $S=T$, everything is trivial, so there is nothing to do. Otherwise, we proceed in steps.
	\begin{enumerate}
		\item We set up notation. Choose points $S'$ and $T'$ with $mS'=S$ and $mT'=T$. Further, by adjusting $T'$ by an element of $E[m]$, we may assume that $S'\ne T'$ (which we do for technical reasons), and we let $R$ be an auxiliary point so that $2R=S'-T'$; note then that the divisors $D_S\coloneqq[S]-[\infty]$ and $D_T\coloneqq[T+mR]-[mR]$ have disjoint support. Then $mD_S$ and $mD_T$ are linearly equivalent to $0$, so we may find functions $f_S$ and $f_T$ with $\op{div}f_S=mD_S$ and $\op{div}f_T=mD_T$, and as in \Cref{rem:produce-functions-on-e}, we may find functions $g_S$ and $g_T$ for which $f_S\circ[m]=g_S^m$ and $f_T\circ[m]=g_T^m$. Before going further, we note that, as in \Cref{rem:produce-functions-on-e}, we have
		\[\op{div}g_S=\sum_{P\in E[m]}([P+S']-[P])\qquad\text{and}\qquad\op{div}g_T=\sum_{P\in E[m]}([P+T'+R]-[P+R]).\]

		\item By plugging in our various pairings, we see that we are interested in showing
		\[\frac{f_S(D_T)}{f_T(D_S)}\stackrel?=\frac{g_S(X+S)}{g_S(X)}.\]
		To start, a direct calculation shows that the left-hand side is
		\[\frac{f_S(mT'+mR)/f_S(mR)}{f_T(mS')/f_T(\infty)}=\left(\frac{g_S(T'+R)/g_S(R)}{g_T(S')/g_T(\infty)}\right)^m.\]
		The trick is that the function
		\[\frac{g_S(X+T'+R)/g_S(X+R)}{g_T(X+S')/g_T(X)}\]
		is a constant function. This will use the fact that $2R=S'-T'$. Indeed, it is enough to show that the divisor of this function vanishes, so we calculate its divisor to be
		\begin{align*}
			&\sum_{P\in E[m]}([P+S'-T'-R]-[P-T'-R])-([P+S'-R]-[P-R]) \\
			&\qquad-([P+T'-S'+R]-[P-S'+R])+([P+T'+R]-[P+R]).
		\end{align*}
		Plugging in $S'=T'+2R$, we get
		\begin{align*}
			&\sum_{P\in E[m]}([P+R]-[P-T'-R])-([P+T'+R]-[P-R]) \\
			&\qquad-([P-R]-[P-T'-R])+([P+T'+R]-[P+R]),
		\end{align*}
		which vanishes term-wise.

		\item Continuing, we see that $\widetilde e(S,T)$ is
		\[\prod_{i=0}^{m-1}\frac{g_S((i+1)T'+R)/g_S(iT'+R)}{g_T(iT'+S')/g_T(iT')}\]
		by the constancy of the preceding paragraph, which collapses to
		\[\frac{g_S(T+R)}{g_S(R)}\prod_{i=0}^{m-1}\frac{g_T(iT')}{g_T(iT'+S)}.\]
		The left-hand term is simply $e(T,S)$ by its construction, so it remains to show that the right-hand product is $1$. For this, we see that we have to show that the function
		\[\prod_{i=0}^{m-1}g_T(iT'+X)\]
		is constant, which we achieve with another divisor calculation: the divisor of this function is
		\[\sum_{P\in E[m]}\Bigg(\sum_{i=0}^{m-1}[P+T'+R-iT']-[P+R-iT']\Bigg),\]
		which collapses to
		\[\sum_{P\in E[m]}([P+mT'+R]-[P+R]),\]
		which now vanishes because $mT'\in E[m]$.
		\qedhere
	\end{enumerate}
\end{proof}
This last interpretation of the Weil pairing can also be viewed as a commutator pairing (as in \Cref{prop:commutator-gives-cup}).
\begin{definition}[Heisenberg group]
	Fix an elliptic curve $E$ over a field $K$ and an integer $m$. For an extension $L/K$, we define the \textit{Heisenberg group} $\mc H_m(L)$ to consist of pairs $(x,f)$ where $x\in E(L)$ and $f$ is a rational function in $L(E)$ for which $\op{div}f=m[x]-m[\infty]$. The group operation is given by
	\[(x,f)\cdot(y,g)\coloneqq(x+y,\tau_yf\cdot g),\]
	where $\tau_y$ denotes the translation action $\tau_yf(t)\coloneqq f(t-y)$.
\end{definition}
\begin{remark}
	Note that $\mc H_m(L)$ is a subgroup of $E(L)\ltimes L(E)^\times$, where $E(L)$ acts on $L(E)^\times$ by translation. Indeed, we just have to check closure under the group operation (and inversion), for which we note that $(x,f),(y,g)\in\mc H_m$ have
	\[\op{div}(\tau_yf\cdot g)=m[x+y]-m[y]+m[y]-m[\infty].\]
	Similarly, $(x,f)^{-1}=\left(-x,\tau_{-x}f^{-1}\right)$ has $\op{div}\tau_{-x}f^{-1}=m[-x]-m[\infty]$.
\end{remark}
\begin{remark} \label{rem:heisenberg-ses}
	Note that having some $f$ with $\op{div}f=m[x]-m[\infty]$ implies that $m\cdot x=\infty$ in $E(L)$; conversely, any such $x\in E(L)[m]$ has some $f\in L(E)$ with $\op{div}f=m[x]-m[\infty]$, and this $f$ is unique up to scalar. Thus, there is a short exact sequence
	\[1\to L^\times\to\mc H_m(L)\to E[m](L)\to0,\]
	where $L^\times\to\mc H_m(L)$ is given by sending $\alpha\in L^\times$ to the constant function $\alpha$. Note that $L^\times$ embeds into the center of $\mc H_m(L)$!
\end{remark}
\begin{lemma} \label{lem:weil-by-commutator}
	Fix an elliptic curve $E$ over a field $K$. For each positive integer $m$ and $S,T\in E[m]$, we have
	\[\widetilde e_m(S,T)=\varphi(S,T)^{-1},\]
	where $\varphi\colon E[m]\otimes E[m]\to\ov K^\times$ is the commutator pairing induced by the short exact sequence of \Cref{rem:heisenberg-ses}.
\end{lemma}
\begin{proof}
	This is a direct calculation. Choose functions $f_S$ and $f_T$ for which $\op{div}f_S=m[S]-m[\infty]$ and $\op{div}f_T=m[T]-m[\infty]$. Then the commutator pairing $\varphi(S,T)$ is simply
	\begin{align*}
		(S,f_S)(T,f_T)(S,f_S)^{-1}(T,f_T)^{-1} &= (S+T,\tau_Tf_S\cdot f_T)(S+T,\tau_{S}f_T\cdot f_S)^{-1} \\
		&= (S+T,\tau_Tf_S\cdot f_T)\left(-S-T,\tau_{-T}f_T^{-1}\cdot\tau_{-S-T}f_S^{-1}\right) \\
		&= (0,\tau_{-S}f_S\cdot\tau_{-S-T} f_T\cdot\tau_{-T}f_T^{-1}\cdot\tau_{-S-T}f_S^{-1}).
	\end{align*}
	Thus, the commutator pairing outputs the value of the constant function
	\[X\mapsto\frac{f_S(X+S)f_T(X+S+T)}{f_T(X+T)f_S(X+S+T)}.\]
	In order to make $\widetilde e$ appear, we need to take quotients by rational functions with disjoint support. Choose an auxiliary point $P\in E(\ov K)$ which is not $m$-torsion. We would like to compute the quotient
	\[\frac{f_T([P+S+T]-[P+T])}{f_S([P+S+T]-[P+S])}.\]
	Well, the function $\tau_{P+T}f_S$ has $\op{div}\tau_{P+T}f_S=m[P+S+T]-m[P+T]$, so this quotient is
	\[\frac{f_T([P+S+T]-[P+T]}{\tau_{P+T}f_S([T]-[\infty])}\cdot\frac{f_S([-P]+[-P-T])}{f_S([P+S+T]-[P+S])}.\]
	The left term is $\widetilde e(T,S)$, so we are left to show that
	\[\frac{f_S(-P)f_S(P+S)}{f_S(-P-T)f_S(P+T+S)}\stackrel?=1.\]
	We now claim that the function $X\mapsto f_S(-X)f_S(X+S)$ is constant, which will complete the proof. Indeed, the divisor of this function is $m[-S]-m[\infty]+m[\infty]-m[-S]=0$.
\end{proof}

\subsection{Some Maximal Isotropic Subspaces}
The Weil pairing now interacts with cohomology as follows.
\begin{lemma} \label{lem:kummer-is-isotropic}
	Fix a field $K$ of characteristic $0$ and a positive integer $m$.
	\begin{listalph}
		\item The Weil pairing induces a symmetric cup-product pairing
		\[\mathrm H^1(K;E[m])\times\mathrm H^1(K;E[m])\to\mathrm H^2(K;\mu_m).\]
		In fact, this pairing on $\mathrm H^1(K;E[m])$ is induced by a quadratic function $\mathrm H^1(K;E[m])\to\mathrm H^2(K;\mathbb G_m)$.
		\item The boundary map $\mathrm H^1(K;E[m])\to\mathrm H^2(K;\mathbb G_m)$ vanishes on $E(K)/mE(K)$. Thus, the subspace $E(K)/mE(K)$ of $\mathrm H^1(K;E[m])$ is isotropic.
	\end{listalph}
\end{lemma}
\begin{proof}
	We handle the parts separately.
	\begin{listalph}
		\item This pairing is defined by
		\[(c_1,c_2)\mapsto\mathrm H^2(e_m)(c_1\cup c_2),\]
		which is symmetric because $\cup$ and $e_m$ are both alternating.
		
		It remains to produce the desired quadratic function. We follow \cite[Corollary~4.7]{poonen-rains-selmer}. To begin, note \Cref{lem:weil-by-commutator} tells us that the Weil pairing $e_m$ is induced by a commutator pairing, so \Cref{prop:commutator-gives-cup} implies that the pairing on $\mathrm H^1(K;E[m])$ is induced by the boundary map $q\colon\mathrm H^1(K;E[m])\to\mathrm H^2(K;\mathbb G_m)$. To show that $q$ is a quadratic function, we note that $\langle m,n\rangle_q=q(m+n)q(m)^{-1}q(n)^{-1}$ is already understood to be symmetric and bilinear, so we only have to check that $q(km)=k^2q(m)$ for all $m$ and $k\in\ZZ$. By the bilinearity we already have, it is enough to just check that $q(-m)=q(m)$ for all $m$. To this end, we note that there is a morphism
		% https://q.uiver.app/#q=WzAsMTAsWzAsMCwiMSJdLFsxLDAsIlxcbWF0aGJiIEdfbSJdLFsyLDAsIlxcbWMgSF9tIl0sWzMsMCwiRVttXSJdLFs0LDAsIjAiXSxbMCwxLCIxIl0sWzEsMSwiXFxtYXRoYmIgR19tIl0sWzIsMSwiXFxtYyBIX3stbX0iXSxbMywxLCJFW21dIl0sWzQsMSwiMCJdLFswLDFdLFsxLDJdLFsyLDNdLFszLDRdLFs1LDZdLFs2LDddLFs3LDhdLFs4LDldLFsxLDYsIiIsMSx7ImxldmVsIjoyLCJzdHlsZSI6eyJoZWFkIjp7Im5hbWUiOiJub25lIn19fV0sWzMsOCwiLTEiXSxbMiw3XV0=&macro_url=https%3A%2F%2Fraw.githubusercontent.com%2FdFoiler%2Fnotes%2Fmaster%2Fnir.tex
		\[\begin{tikzcd}[cramped]
			1 & {\mathbb G_m} & {\mc H_m} & {E[m]} & 0 \\
			1 & {\mathbb G_m} & {\mc H_{m}} & {E[m]} & 0
			\arrow[from=1-1, to=1-2]
			\arrow[from=1-2, to=1-3]
			\arrow[equals, from=1-2, to=2-2]
			\arrow[from=1-3, to=1-4]
			\arrow[from=1-3, to=2-3]
			\arrow[from=1-4, to=1-5]
			\arrow["{-1}", from=1-4, to=2-4]
			\arrow[from=2-1, to=2-2]
			\arrow[from=2-2, to=2-3]
			\arrow[from=2-3, to=2-4]
			\arrow[from=2-4, to=2-5]
		\end{tikzcd}\]
		of short exact sequence, where the map $\mc H_m\to\mc H_{m}$ is given by $(x,f)\mapsto(-x,\iota f)$; here, $\iota\colon\ov K(E)\to\ov K(E)$ is defined by $\iota g(p)\coloneqq g(-p)$. Certainly this map is well-defined, and it is a group homomorphism because $(x,f)(y,g)=(x+y,\tau_yf\cdot g)$ goes to $(-x,\iota f)(-y,\iota g)=(-x-y,\tau_{-y}\iota f\cdot\iota g)$. Functoriality of the boundary now gives us a commuting square
		% https://q.uiver.app/#q=WzAsNCxbMCwwLCJcXG1hdGhybSBIXjEoSztFW21dKSJdLFsxLDAsIlxcbWF0aHJtIEheMihLO1xcbWF0aGJiIEdfbSkiXSxbMSwxLCJcXG1hdGhybSBIXjIoSztcXG1hdGhiYiBHX20pIl0sWzAsMSwiXFxtYXRocm0gSF4xKEs7RVttXSkiXSxbMSwyLCIiLDAseyJsZXZlbCI6Miwic3R5bGUiOnsiaGVhZCI6eyJuYW1lIjoibm9uZSJ9fX1dLFswLDEsInEiXSxbMCwzLCItMSIsMl0sWzMsMiwicSJdXQ==&macro_url=https%3A%2F%2Fraw.githubusercontent.com%2FdFoiler%2Fnotes%2Fmaster%2Fnir.tex
		\[\begin{tikzcd}[cramped]
			{\mathrm H^1(K;E[m])} & {\mathrm H^2(K;\mathbb G_m)} \\
			{\mathrm H^1(K;E[m])} & {\mathrm H^2(K;\mathbb G_m)}
			\arrow["q", from=1-1, to=1-2]
			\arrow["{-1}"', from=1-1, to=2-1]
			\arrow[equals, from=1-2, to=2-2]
			\arrow["q", from=2-1, to=2-2]
		\end{tikzcd}\]
		which is exactly what we wanted to show.

		\item We will show this directly on the level of cocycles, following \cite[Proposition~4.9]{poonen-rains-selmer}. Note that the first sentence implies the second one: \Cref{prop:commutator-gives-cup} combined with \Cref{lem:weil-by-commutator} tells us that the pairing on $\mathrm H^1(K;E[m])$ is given by the boundary map.

		We now show the first sentence. It is enough to show the boundary $\delta\colon\mathrm H^0(K;E)\to\mathrm H^1(K;E[m])$ factors through $\mathrm H^1(K;\mc H_m)$ because then the composite
		\[\mathrm H^1(K;\mc H_m)\to\mathrm H^1(K;E[m])\to\mathrm H^2(K;\mathbb G_m)\]
		vanishes by \Cref{lem:h1-to-h2}. We will show this factoring directly on the level of cocycles.
		
		Fix some $x\in E(K)$ and choose some $y\in E(\ov K)$ with $y=mx$ so that $\delta(g)\coloneqq gy-y$. We are tasked with finding some $1$-cocycle $c$ of $\mc H_m$ such that $\delta=\op{pr}_1(c)$, where $\op{pr}_1\colon\mc H_m\to E[m]$ is the canonical projection. Thus, we need to find functions $f_g\in\ov K(E)$ such that $\op{div}f_g=m[\delta(g)]-m[\infty]$ so that $c(g)\coloneqq(\delta(g),f_g)$ is a $1$-cocycle of $\mc H_m$. Note that such functions certainly exist because $\delta(g)\in E[m]$, and they are unique up to scalar.

		Now, checking that $c$ is a $1$-cocycle amounts to verifying the equality $c(gh)=gc(h)\cdot c(g)$. This is equivalent to
		\[(\delta(gh),f_{gh})\stackrel?=\left(g\delta(h)+\delta(g),\tau_{\delta(g)}gf_h\cdot f_g\right).\]
		The left coordinates are equal because $\delta$ is a $1$-cocycle, so it remains to produce the functions $f_\bullet$ achieving $f_{gh}=\tau_{gy-y}gf_h\cdot f_g$; equivalently, we need to achieve $\tau_yf_{gh}=g\tau_yf_h\cdot\tau_yf_g$. Well, note that the divisors of both sides are
		\[m[ghy]-m[y]=m[ghy]-m[gy]+m[gy]-m[y],\]
		so these functions are certainly the same up to a scalar. To fix the scalar, note that we may assume $y\ne\infty$ (otherwise, $x=\infty$, and we can choose all the functions $f_\bullet$ to be constants), so we may normalize all functions so that $\tau_yf_g(\infty)=1$ for all $g$. Then the equality $\tau_yf_{gh}=g\tau_yf_h\cdot\tau_yf_g$ follows from the equality at the point $\infty$.
		\qedhere
	\end{listalph}
\end{proof}
We now produce some local and global isotropy results. Here is the local one.
\begin{proposition} \label{prop:local-is-lagrangian}
	Fix a local field $K$ of characteristic $0$ and a positive integer $m$.
	\begin{listalph}
		\item The Weil pairing makes $\mathrm H^1(K;E[m])$ a non-degenerate quadratic space. Explicitly, this is a quad\-ratic space over $\ZZ/m\ZZ$ if $K$ is nonarchimedean and over $\ZZ/\gcd(m,2)\ZZ$ if $K$ is archimedean.
		\item The submodule $E(K)/mE(K)$ of $\mathrm H^1(K;E[m])$ is maximal isotropic.
	\end{listalph}
\end{proposition}
\begin{proof}
	These claims follow from local duality. For (a), note the Weil pairing on $E[m]$ is perfect and Galois-invariant (by \Cref{lem:weil-pairing}), so it induces an isomorphism $E[m]\cong\op{Hom}(E[m],\mu_m)$ of (finite discrete) Galois modules. Thus, we can think of the pairing on $\mathrm H^1(K;E[m])$ as fitting into the commutative diagram
	% https://q.uiver.app/#q=WzAsMTAsWzAsMCwiXFxtYXRocm0gSF4xKEs7RVttXSkiXSxbMSwwLCJcXG1hdGhybSBIXjEoSztFW21dKSJdLFsxLDEsIlxcbWF0aHJtIEheMShLO0VbbV0pIl0sWzAsMSwiXFxtYXRocm0gSF4xKEs7RVttXV4qKSJdLFsyLDAsIlxcbWF0aHJtIEheMihLO1xcbXVfbSkiXSxbMiwxLCJcXG1hdGhybSBIXjIoSztcXG11X20pIl0sWzMsMCwiKGNfMSxjXzIpIl0sWzQsMCwiZV9tKGNfMVxcY3VwIGNfMikiXSxbMywxLCIoZ1xcbWFwc3RvIGVfbShjXzEoZyksLSksY18yKSJdLFs0LDEsIihnLGgpXFxtYXBzdG8gZV9tKGNfMShnKSxnY18yKGgpKSJdLFswLDEsIlxcdGltZXMiLDMseyJzdHlsZSI6eyJib2R5Ijp7Im5hbWUiOiJub25lIn0sImhlYWQiOnsibmFtZSI6Im5vbmUifX19XSxbMywyLCJcXHRpbWVzIiwzLHsic3R5bGUiOnsiYm9keSI6eyJuYW1lIjoibm9uZSJ9LCJoZWFkIjp7Im5hbWUiOiJub25lIn19fV0sWzEsMiwiIiwwLHsibGV2ZWwiOjIsInN0eWxlIjp7ImhlYWQiOnsibmFtZSI6Im5vbmUifX19XSxbMCwzLCJlX20iXSxbMSw0LCJcXGN1cCJdLFsyLDUsIlxcY3VwIl0sWzQsNSwiIiwxLHsibGV2ZWwiOjIsInN0eWxlIjp7ImhlYWQiOnsibmFtZSI6Im5vbmUifX19XSxbNiw4LCIiLDEseyJzdHlsZSI6eyJ0YWlsIjp7Im5hbWUiOiJtYXBzIHRvIn19fV0sWzYsNywiIiwxLHsic3R5bGUiOnsidGFpbCI6eyJuYW1lIjoibWFwcyB0byJ9fX1dLFs4LDksIiIsMSx7InN0eWxlIjp7InRhaWwiOnsibmFtZSI6Im1hcHMgdG8ifX19XSxbNyw5LCIiLDEseyJzdHlsZSI6eyJ0YWlsIjp7Im5hbWUiOiJtYXBzIHRvIn19fV1d&macro_url=https%3A%2F%2Fraw.githubusercontent.com%2FdFoiler%2Fnotes%2Fmaster%2Fnir.tex
	\[\begin{tikzcd}[cramped, column sep=small]
		{\mathrm H^1(K;E[m])} & {\mathrm H^1(K;E[m])} & {\mathrm H^2(K;\mu_m)} & {(c_1,c_2)} & {e_m(c_1\cup c_2)} \\
		{\mathrm H^1(K;E[m]^*)} & {\mathrm H^1(K;E[m])} & {\mathrm H^2(K;\mu_m)} & {(g\mapsto e_m(c_1(g),-),c_2)} & {(g,h)\mapsto e_m(c_1(g),gc_2(h))}
		\arrow["\times"{marking, allow upside down}, draw=none, from=1-1, to=1-2]
		\arrow["{e_m}", from=1-1, to=2-1]
		\arrow["\cup", from=1-2, to=1-3]
		\arrow[equals, from=1-2, to=2-2]
		\arrow[equals, from=1-3, to=2-3]
		\arrow[maps to, from=1-4, to=1-5]
		\arrow[maps to, from=1-4, to=2-4]
		\arrow[maps to, from=1-5, to=2-5]
		\arrow["\times"{marking, allow upside down}, draw=none, from=2-1, to=2-2]
		\arrow["\cup", from=2-2, to=2-3]
		\arrow[maps to, from=2-4, to=2-5]
	\end{tikzcd}\]
	where the bottom row is perfect by \Cref{thm:tate}. Thus, the induced pairing is perfect, and we see that we actually have a quadratic space over $\ZZ/m\ZZ$ because the target of the pairing is $\mathrm H^2(K;\mu_m)\cong\ZZ/m\ZZ$ by local class field theory.

	% We omit the proof of (b) because it is on the homework.
	We now move on to (b), following \cite[Theorem~8.2]{tate-galois-cohomology}. We begin by recalling the Kummer short exact sequence
	\[0\to\frac{E(K)}{mE(K)}\to\mathrm H^1(K;E[m])\to\mathrm H^1(K;E)[m]\to0\]
	which is induced by the short exact sequence $0\to E[m]\to E\to E\to0$. Roughly speaking, the result will follow by comparing this short exact sequence with its dual. By \Cref{lem:kummer-is-isotropic}, we see that $E(K)/mE(K)$ is an isotropic subspace of $\mathrm H^1(K;E[m])$, so the Weil pairing descends to a pairing
	\[\frac{E(K)}{mE(K)}\times\mathrm H^1(K;E)[m]\to\mathrm H^2(K;\mu_m),\]
	where we are viewing $\mathrm H^1(K;E)[m]$ as the quotient of $\mathrm H^1(K;E[m])$ by $E(K)/mE(K)$.
	
	The main claim is that this last Weil pairing is perfect. For now, let's content ourselves with explaining why this completes the proof. We would like to show that a class $c\in\mathrm H^1(K;E[m])$ has $x\cup c=0$ for all $x\in E(K)/mE(K)$ if and only if $c\in E(K)/mE(K)$. In other words, the functional $-\cup c$ on $E(K)/mE(K)$ vanishes if and only if $c$ vanishes in the quotient $\mathrm H^1(K;E)[m]$, which is exactly the main claim.
	
	We now show the main claim. Let $(-)^\lor\coloneqq\op{Hom}_\ZZ(-,\QQ/\ZZ)$ denote the Pontryagin dual. Then we note that the Weil pairing induces a morphism
	% https://q.uiver.app/#q=WzAsMTAsWzAsMCwiMCJdLFsxLDAsIlxcZGlzcGxheXN0eWxlXFxmcmFje0UoSyl9e21FKEspfSJdLFsyLDAsIlxcbWF0aHJtIEheMShLO0VbbV0pIl0sWzMsMCwiXFxtYXRocm0gSF4xKEs7RSlbbV0iXSxbNCwwLCIwIl0sWzAsMSwiMCJdLFsxLDEsIlxcbWF0aHJtIEheMShLO0UpW21dXlxcbG9yIl0sWzIsMSwiXFxtYXRocm0gSF4xKEs7RVttXSleXFxsb3IiXSxbMywxLCJcXGxlZnQoXFxkaXNwbGF5c3R5bGVcXGZyYWN7RShLKX17bUUoSyl9XFxyaWdodCleXFxsb3IiXSxbNCwxLCIwIl0sWzAsMV0sWzEsMl0sWzIsM10sWzMsNF0sWzUsNl0sWzYsN10sWzcsOF0sWzgsOV0sWzEsNl0sWzIsN10sWzMsOF1d&macro_url=https%3A%2F%2Fraw.githubusercontent.com%2FdFoiler%2Fnotes%2Fmaster%2Fnir.tex
	\[\begin{tikzcd}[cramped]
		0 & {\displaystyle\frac{E(K)}{mE(K)}} & {\mathrm H^1(K;E[m])} & {\mathrm H^1(K;E)[m]} & 0 \\
		0 & {\mathrm H^1(K;E)[m]^\lor} & {\mathrm H^1(K;E[m])^\lor} & {\left(\displaystyle\frac{E(K)}{mE(K)}\right)^\lor} & 0
		\arrow[from=1-1, to=1-2]
		\arrow[from=1-2, to=1-3]
		\arrow[from=1-2, to=2-2]
		\arrow[from=1-3, to=1-4]
		\arrow[from=1-3, to=2-3]
		\arrow[from=1-4, to=1-5]
		\arrow[from=1-4, to=2-4]
		\arrow[from=2-1, to=2-2]
		\arrow[from=2-2, to=2-3]
		\arrow[from=2-3, to=2-4]
		\arrow[from=2-4, to=2-5]
	\end{tikzcd}\]
	of short exact sequences. We showed in (a) that the middle morphism is an isomorphism, so it follows that the left morphism is injective, and the right morphism is surjective. But the size of a finite abelian group and its Pontryagin dual are the same, so it follows that the left and right morphisms are isomorphisms, which is what we wanted to show!
\end{proof}
	% \begin{listalph}
	% 	\item This pairing is defined by
	% 	\[(c_1,c_2)\mapsto\mathrm H^2(e_m)(c_1\cup c_2),\]
	% 	which is symmetric because $\cup$ and $e_m$ are both alternating.
	% 	\item This follows from local duality. Indeed, the Weil pairing on $E[m]$ is perfect and Galois-invariant, so it induces an isomorphism $E[m]\cong\op{Hom}(E[m],\mu_m)$ of (finite discrete) Galois modules. The cup-product pairing in the statement is now exactly the pairing of \Cref{thm:tate}.
	% 	\item 
	%
		% Fix $P,Q\in E(K)/mE(K)$, and we want to show that the $2$-cocycle $\mathrm H^2(e_m)(\delta_P\cup\delta_Q)$ is a $2$-coboundary. Expanding out the definitions, we see that we should choose $P'$ and $Q'$ with $P=mP'$ and $Q=mQ'$, and then we want to show the map
		% \[(\sigma,\tau)\mapsto e_m(\sigma P'-P',\sigma\tau Q'-\sigma Q')\]
		% is a $2$-coboundary. For this, we will use \Cref{prop:alternative-weil}, which allows us to instead show that the $2$-cocycle
		% \[c\colon(\sigma,\tau)\mapsto\widetilde e_m(\sigma P'-P',\sigma\tau Q'-\sigma Q')^{-1}\]
		% is a $2$-coboundary.
	%
		% Accordingly, we now set up some notation. If $P=Q$, then there is nothing to do, so we omit this case. For each $\sigma\in\op{Gal}(K^{\mathrm{sep}}/K)$, we define the divisors $D_{P,\sigma}\coloneqq[\sigma P']-[P']$ and $D_{Q,\sigma}\coloneqq[\sigma Q']-[Q']$. Note that any $D_{P,\sigma}$ and any $D_{Q,\tau}$ have disjoint support: $P\ne Q$ forces $\sigma P'\ne\tau Q'$ for any $\sigma$ or $\tau$ because $m\sigma P'=P$ and $m\tau Q'=Q$. We also note that $D_{P,\sigma\tau}=\sigma D_{P,\tau}+D_{P,\sigma}$ and similarly for $Q$. Continuing, we select functions $f_{P,\sigma}$ and $f_{Q,\sigma}$ so that $\op{div}f_{P,\sigma}=mD_{p,\sigma}$ and $\op{div}f_{Q,\sigma}=$. By choosing another point $S$ not in the Galois orbit of the previously defined points, we may normalize the functions $f_\bullet$ so that they evaluate to $1$ on $S$. Then we see that
		% \[f_{P,\sigma\tau}=\sigma f_{P,\tau}\cdot f_{P,\sigma}\]
		% because they have the same divisor and the same nonzero value at $S$; a similar statement holds for $Q$.
	%
		% We are now ready to calculate. To begin, note $c(\sigma,\tau)$ is
		% \begin{align*}
		% 	\widetilde e_m(\sigma P'-P',\sigma\tau Q'-Q')\widetilde e_m(\sigma P'-P',\sigma Q'-Q')^{-1}
		% 	&= \frac{f_{Q,\sigma\tau}(D_{P,\sigma})}{f_{P,\sigma}(D_{Q,\sigma\tau})}\cdot\frac{f_{P,\sigma}(D_{Q,\sigma})}{f_{Q,\sigma}(D_{P,\sigma})}.
		% \end{align*}
		% We now recall $f_{Q,\sigma\tau}=\sigma f_\tau\cdot f_\sigma$ and $D_{Q,\sigma\tau}=\sigma D_\tau+D_\sigma$, so $c(\sigma,\tau)$ equals
		% \[\frac{(\sigma f_{Q,\tau})(D_{P,\sigma})}{f_{P,\sigma}(\sigma D_{Q,\tau})}=\frac{\sigma f_{Q,\tau}(P')}{\sigma f_{Q,\tau}(\sigma^{-1}P')}\cdot\frac{f_{P,\sigma}(\sigma Q)}{f_{P,\sigma}(\sigma\tau Q)}.\]
		% We will complete the proof by showing that each factor of the right-hand side is a $2$-coboundary. For example, the $1$-cochain $\sigma\mapsto f_{Q,\sigma}(P')$ produces the $1$-coboundary sending $(\sigma,\tau)$ to
		% \[\frac{\sigma f_{Q,\tau}(P')\cdot f_{Q,\sigma}(P')}{f_{Q,\sigma\tau}(P')}=\frac{\sigma f_{Q,\tau}(P')}{\sigma f_{Q,\tau}(\sigma^{-1}P')}.\]
		% Similarly, the $1$-cochain $\sigma\mapsto f_{P,\sigma}(\sigma Q')$ produces the $2$-coboundary sending $(\sigma,\tau)$ to
		% \[\frac{\sigma f_{P,\tau}(\tau Q')\cdot f_{P,\sigma}(\sigma Q')}{f_{P,\sigma\tau}(\sigma\tau Q')}=\frac{\sigma f_{P,\tau}(\tau Q')\cdot f_{P,\sigma}(\sigma Q')}{\sigma f_{P,\tau}(\sigma^{-1}\sigma\tau Q)\cdot f_{P,\sigma}(\sigma\tau Q)}.\]
		% Multiplying these two $2$-coboundaries together completes the proof.
	%
	% 	\item You may show this on the homework!
	% 	\qedhere
	% \end{listalph}
% \begin{notation}
% 	Fix a number field $K$. Then we set
% 	\[\mathrm H^1(\AA_K;E[m])\coloneqq\prod_v\left(\mathrm H^1(K_v;E[m]),\mathrm H^1_{\mathrm{ur}}(K_v;E[m])\right).\]
% \end{notation}
% \begin{remark}
% 	If $p$ is a prime, then it turns out that the map $\mathrm H^1(K;E[p])\to\mathrm H^1(\AA_K;E[p])$ is injective when $p$ is prime. However, this is very special: this fails when replacing $p$ by something which is not a prime or when replacing $E$ by a higher-dimensional abelian variety.
% \end{remark}
Here is the global isotropy result.
\begin{remark} \label{rem:e-on-adele}
	For any local field $K_v$, we note that $E(K_v)=E(\OO_v)$ if $E$ has good reduction at $v$. At a high level, this follows from the valuative criterion of properness or the theory of N\'eron models. More directly, one can see that a point $[X:Y:Z]\in\PP^2(K_v)$ satisfying the equation defining $E$ may have its coordinates adjusted until all coordinates are in $\OO_v$ by homogeneity.
\end{remark}
\begin{proposition} \label{prop:global-is-lagrangian}
	Fix a number field $K$ and a positive integer $m$.
	\begin{listalph}
		\item The Weil pairing makes $\mathrm H^1(\AA_K;E[m])$ a non-degenerate quadratic space over $\ZZ/m\ZZ$.
		\item The image of $E(\AA_K)/mE(\AA_K)$ in $\mathrm H^1(\AA_K;E[m])$ is maximal isotropic.
		\item The image of $\mathrm H^1(K;E[m])$ in $\mathrm H^1(\AA_K;E[m])$ is maximal isotropic.
	\end{listalph}
\end{proposition}
\begin{proof}
	Before we do anything, let's describe the pairing on $\mathrm H^1(\AA_K;E[m])$. It is the one induced from \Cref{rem:weil-pairing-on-adele} and the Weil pairing (see also the diagram in \Cref{prop:local-is-lagrangian}). More explicitly, for two classes $c$ and $c'$, the pairing is
	\[\langle c,c'\rangle\coloneqq\sum_{v}\langle\op{loc}_v c,\op{loc}_vc'\rangle_v,\]
	where $\langle-,-\rangle_v$ is the (local) Weil pairing on $\mathrm H^1(K_v;E[m])$ of \Cref{prop:local-is-lagrangian} (and \Cref{thm:tate}). In particular, the fact that the Weil pairing is antisymmetric implies that the pairing on $\mathrm H^1(\AA_K;E[m])$ is symmetric. Furthermore, because all the local pairings output to $\mathrm H^2(K_v;\mu_m)\into\ZZ/m\ZZ$, we see that we receive a quadratic space over $\ZZ/m\ZZ$.
	\begin{listalph}
		\item This is a local statement; in particular, it is just a ``sum'' of local results. In light of the above remarks, it only remains to show the non-degeneracy.
		% The pairing on $\mathrm H^1(\AA_K;E[m])$ is the one induced from \Cref{rem:weil-pairing-on-adele} and the Weil pairing (see also the diagram in \Cref{prop:local-is-lagrangian}); in particular, the fact that the Weil pairing is antisymmetric implies that the pairing on $\mathrm H^1(\AA_K;E[m])$ is symmetric. The fact that it is perfect is simply the ``sum'' of the local results.
		For this, suppose that $c\in\mathrm H^1(\AA_K;E[m])$ has $\langle c,c'\rangle=0$ for all $c'\in\mathrm H^1(\AA_K;E[m])$, so we want to show that $c=0$. By definition of $\mathrm H^1(\AA_K;E[m])$, it is enough to show that $\op{loc}_vc=0$ for all places $v$. Well, for each $v_0$, and class $c'_{v_0}\in\mathrm H^1(K_v;E[m])$, we can define a class $c'\in\mathrm H^1(\AA_K;E[m])$ by
		\[\op{loc}_{v}c'\coloneqq\begin{cases}
			c'_{v_0} & \text{if }v=v_0, \\
			0 & \text{otherwise}.
		\end{cases}\]
		Then $c'\in\mathrm H^1(\AA_K;E[m])$ and $\langle c,c'\rangle=\langle\op{loc}_{v_0}c,c'_{v_0}\rangle_{v_0}$ must vanish for all $c'_{v_0}$, so it follows that $\op{loc}_{v_0}c=0$ by \Cref{prop:local-is-lagrangian}.

		\item This is also a local statement. Indeed, given $c\in\mathrm H^1(\AA_K;E[m])$, we have to show $c\in E(\AA_K)/mE(\AA_K)$ if and only if $c$ annihilates $E(\AA_K)/mE(\AA_K)$. Certainly if $c$ is in $E(\AA_K)/mE(\AA_K)$, then summing \Cref{prop:local-is-lagrangian} shows that $c$ annihilates $E(\AA_K)/mE(\AA_K)$.
		
		Conversely, suppose $c$ annihilates $E(\AA_K)/mE(\AA_K)$, and we will show $\op{loc}_{v_0}c\in E(K_{v_0})/mE(K_{v_0})$ for all $v_0$. Well, by \Cref{prop:local-is-lagrangian}, it is enough to show that
		\[\langle\op{loc}_{v_0}c,c'_{v_0}\rangle_{v_0}=0\]
		for all $c'_{v_0}\in\mathrm H^1(K_{v_0};E[m])$, but this follows by using $c'_{v_0}$ to construct a class $c'\in\mathrm H^1(\AA_K;E[m])$ exactly as in (a).

		\item This is a global statement. Indeed, we will use Pitou--Tate duality in the form of \Cref{thm:pitou-tate}. Let's explain what we can show without global duality: we can show that $\mathrm H^1(K;E[m])$ is isotropic. In fact, the quadratic form $\mathrm H^1(\AA_K;E[m])\to\ZZ/m\ZZ$ vanishes on $\mathrm H^1(K;E[m])$. To see this, note the commutativity of the diagram
		% https://q.uiver.app/#q=WzAsNCxbMCwwLCJcXG1hdGhybSBIXjEoSztFW21dKSJdLFswLDEsIlxcbWF0aHJtIEheMShcXEFBX0s7RVttXSkiXSxbMSwxLCJcXG1hdGhybSBIXjIoXFxBQV9LO1xcbWF0aGJiIEdfbSkiXSxbMSwwLCJcXG1hdGhybSBIXjIoSztcXG1hdGhiYiBHX20pIl0sWzAsMV0sWzAsM10sWzEsMl0sWzMsMl1d&macro_url=https%3A%2F%2Fraw.githubusercontent.com%2FdFoiler%2Fnotes%2Fmaster%2Fnir.tex
		\[\begin{tikzcd}[cramped]
			{\mathrm H^1(K;E[m])} & {\mathrm H^2(K;\mathbb G_m)} \\
			{\mathrm H^1(\AA_K;E[m])} & {\mathrm H^2(\AA_K;\mathbb G_m)}
			\arrow[from=1-1, to=1-2]
			\arrow[from=1-1, to=2-1]
			\arrow[from=1-2, to=2-2]
			\arrow[from=2-1, to=2-2]
		\end{tikzcd}\]
		causes the composite $\mathrm H^1(K;E[m])\to\ZZ/m\ZZ$ to factor through
		\[\mathrm H^2(K;\mathbb G_m)\to\mathrm H^2(\AA_K;\mathbb G_m)\to\QQ/\ZZ.\]
		This last composite vanishes because it is part of the fundamental exact sequence of global class field theory.

		It remains to show that $\mathrm H^1(K;E[m])$ is maximal isotropic. Well, suppose that we have a class $c\in\mathrm H^1(\AA_K;E[m])$ such that $c\cup c'=0$ for all $c'\in\mathrm H^1(K;E[m])$, and we would like to show that $c\in\mathrm H^1(K;E[m])$. Well, \Cref{thm:pitou-tate} applied with $\Sigma$ equal to the set of all places shows that we have an exact sequence
		\[\mathrm H^1(K;E[m])\to\mathrm H^1(\AA_K;E[m])\to\mathrm H^1(K;E[m]^*)^\lor.\]
		Identifying $\mathrm H^1(K;E[m]^*)$ with $\mathrm H^1(K;E[m])$ via the Weil pairing, we see that $c\in\mathrm H^1(\AA_K;E[m])$ vanishes in $\mathrm H^1(K;E[m])^\lor$ by hypothesis! Thus, $c$ must come from a class in $\mathrm H^1(K;E[m])$.
		\qedhere
	\end{listalph}
\end{proof}
We now summarize our results as follows.
\begin{theorem} \label{thm:selmer-by-lagrangian}
	Fix an elliptic curve $E$ over a number field $K$, and choose a positive integer $m$. Then $\op{Sel}_m(E/K)$ sits in the following pullback square.
	% https://q.uiver.app/#q=WzAsNCxbMCwwLCJcXG9we1NlbH1fbShFL0spIl0sWzEsMCwiXFxtYXRocm0gSF4xKEs7RVttXSkiXSxbMSwxLCJcXG1hdGhybSBIXjEoXFxBQV9LO0VbbV0pIl0sWzAsMSwiRShcXEFBX0spL3BFKFxcQUFfSykiXSxbMCwzXSxbMywyXSxbMCwxXSxbMSwyXSxbMCwyLCIiLDEseyJzdHlsZSI6eyJuYW1lIjoiY29ybmVyIn19XV0=&macro_url=https%3A%2F%2Fraw.githubusercontent.com%2FdFoiler%2Fnotes%2Fmaster%2Fnir.tex
	\[\begin{tikzcd}[cramped]
		{\op{Sel}_m(E/K)} & {\mathrm H^1(K;E[m])} \\
		{E(\AA_K)/mE(\AA_K)} & {\mathrm H^1(\AA_K;E[m])}
		\arrow[from=1-1, to=1-2]
		\arrow[from=1-1, to=2-1]
		\arrow["\lrcorner"{anchor=center, pos=0.125}, draw=none, from=1-1, to=2-2]
		\arrow[from=1-2, to=2-2]
		\arrow[hook, from=2-1, to=2-2]
	\end{tikzcd}\]
	The images of the bottom and right arrows are maximal isotropic subspaces with respect to the pairing induced by the Weil pairing.
	% \begin{listalph}
	% 	\item The images of the bottom and right arrows are maximal isotropic subspaces with respect to the pairing induced by the Weil pairing.
	% 	\item If $m$ is prime, then the right vertical arrow is injective.
	% \end{listalph}
\end{theorem}
\begin{proof}
	The pullback square is the one in the definition of the Selmer group by \Cref{rem:get-adelic-cohom,rem:e-on-adele}. %The rest of the statement is \cite[Theorem~4.14]{poonen-rains-selmer}. In particular, see \cite[Example~4.18]{poonen-rains-selmer}.
	The rest of the result is a combination of \Cref{prop:local-is-lagrangian,prop:global-is-lagrangian}.
\end{proof}
\begin{remark}
	The moral is that we may view $\op{Sel}_p(E/K)$ may be viewed as an intersection of two maximal isotropic subspaces. Such a ``random'' intersection is expected to be rather transverse, which perhaps explains why $\op{Sel}_p(E/K)$ is finite-dimensional.
\end{remark}
\begin{remark}
	If $m$ is prime, then the right vertical arrow in \Cref{thm:selmer-by-lagrangian} is injective; see \cite[Theorem~4.14]{poonen-rains-selmer} for a proof. This is a rather sensitive result because it depends on a certain vanishing of $\Sha$ result \cite[Proposition~3.3(e)]{poonen-rains-selmer}. It is not expected to be true if $E[m]$ is replaced by a different module; see \Cref{rem:local-to-global-not-inj} below for an example.
	% For example, on the homework, you may show that the map
	% \[\mathrm H^1\big(\QQ(\sqrt 7);\mu_8\big)\to\mathrm H^1\big(\AA_{\QQ(\sqrt 7)};\mu_8\big)\]
	% fails to be injective.
\end{remark}
\begin{example}[Counterexamples to Grunwald--Wang] \label{ex:grunwald-wang}
	We have the following.
	\begin{listalph}
		\item The positive integer $16$ is an eighth power in $\QQ_v$ for all places $v$ except $v=2$. However, $16$ is not an eighth power in $\QQ_2$.
		\item The positive integer $16$ is an eighth power in $\QQ(\sqrt7)_v$ for all places $v$. However, $16$ is not an eighth power in $\QQ(\sqrt7)$.
	\end{listalph}
\end{example}
\begin{proof}
	Before showing either of the parts, we claim that $16$ is an eighth power in a field $K$ if and only if $\sqrt2\in K$ or $\sqrt{-2}\in K$ or $\sqrt{-1}\in K$. Certainly the reverse direction holds because
	\[\big(\sqrt2\big)^{8}=\left(\sqrt{-2}\right)^{8}=(1+i)^{8}=16.\]
	In the forward direction, if $16$ is an eighth power, then we see that one of $4$ or $-4$ is a fourth power. If $-4$ is a fourth power, then $-4$ is a square, so either $2=0$ (in which case $\sqrt2\in K$) or $-1$ is a square. Otherwise, $4$ is a fourth power, so one of $2$ or $-2$ is a square, as desired.
	\begin{listalph}
		\item We do this by casework on the place $v$. For example, if $v$ is archimedean, the result follows because $\sqrt2\in\RR$. Additionally, for $v=2$, we see that $v_2(16)=4$ is not divisible by $8$, so $16$ cannot possibly be an eighth power.

		It remains to handle places $v$ given by odd primes $p$. 
		% To this end, we note that
		% \[\big(\sqrt2\big)^{8}=\left(\sqrt{-2}\right)^{8}=(1+i)^{8}=16,\]
		% so for any field $K$, we see that $16$ is an eighth power in any of the extensions $K(\sqrt2)$ or $K(\sqrt{-2})$ or $K(i)$. Thus, 
		It is enough to show that $\{\sqrt2,\sqrt{-2},i\}\cap\QQ_p\ne\emp$ for each odd prime $p$. In other words, we are asking for one of the quadratics to $x^2-2$ or $x^2+2$ or $x^2-1$ to admit a root. By Hensel's lemma, it is equivalent for one of these quadratics to admit a root over $\FF_p$, which in turn is equivalent to having
		\[1\stackrel?\in\left\{\left(\frac2p\right),\left(\frac{-2}p\right),\left(\frac{-1}p\right)\right\}.\]
		To see this, we note that
		\[\left(\frac2p\right)\left(\frac{-2}p\right)\left(\frac{-1}p\right)=1,\]
		so it is not possible for all three Legendre symbols to equal $-1$!

		\item Set $K\coloneqq\QQ(\sqrt7)$ for brevity. By (a), to show that $16$ is an eighth power in all $K_v$, we only have to handle places $v$ above $2$. Note that $7$ is not a square in $\QQ_2$ (indeed, it is not a square$\pmod4$), so there is only one place $\QQ_2(\sqrt7)$ above $2$. It remains to show that $16$ is an eighth power in $\QQ_2(\sqrt7)$, which is true because $\sqrt{-1}\in\QQ_2(\sqrt7)$ because $\sqrt{-1}/\sqrt7\in\QQ_2^{\times2}$.

		Lastly, we should show that $16$ is not an eighth power in $K$. It is enough to show that none of $\sqrt{-1}$ or $\sqrt2$ or $\sqrt{-2}$ are in $K$, which holds because $K=\QQ(\sqrt7)$ is a quadratic extension avoiding all those elements. Explicitly, $K$ is totally real, so $-1$ and $-2$ cannot be squares, and we can see that there is no $a,b\in\ZZ$ for which
		\[\left(a+b\sqrt7\right)^2=a^2+7b^2+14ab\sqrt7\]
		can equal $2$: this would require $a=0$ or $b=0$, but certainly neither $2$ nor $7/2$ is a square in $\ZZ$.
		\qedhere
		% In fact, we will show that any totally real field $L$ in which $16$ is an eighth power has $\sqrt2\in L$, which will complete the proof because $\sqrt2\notin K$. To see this, if $16$ is an eighth power in $L$, then one of $4$ or $-4$ is a fourth power; however, $L$ is totally real, so $-4$ cannot be a square, so we see that $4$ must be a fourth power. Similarly, this means that one of $2$ or $-2$ is a square in $L$, but $-2$ is not a square, so $2$ is a square in $L$. Thus, $\sqrt2\in L$, as required.
	\end{listalph}
\end{proof}
\begin{remark} \label{rem:local-to-global-not-inj}
	\Cref{ex:h1-of-mu-n} identifies $\mathrm H^1(K;\mu_8)$ with $K^\times/K^{\times8}$, so the second part of \Cref{ex:grunwald-wang} shows that the map
	\[\mathrm H^1(\QQ(\sqrt7);\mu_8)\to\mathrm H^1\left(\AA_{\QQ(\sqrt7)};\mu_8\right)\]
	fails to be injective because the nontrivial class $16\in K^\times/K^{\times8}$ vanishes in $\mathrm H^1\big(\AA_{\QQ(\sqrt7)};\mu_8\big)$.
\end{remark}

\subsection{Conjectures on the Selmer Group}
While we're here, we acknowledge that now is as good as time as any to recall/give the definition of the Tate--Shafarevich group.
\begin{definition}[Tate--Shafarevich group]
	Fix a number field $K$ and a discrete Galois module $M$. Then we define the \textit{Tate--Shafarevich group} $\Sha(M/K)$ as
	\[\Sha(M/K)\coloneqq\ker\Bigg(\mathrm H^1(K;M)\to\prod_v\mathrm H^1(K_v;M)\Bigg).\]
\end{definition}
\begin{lemma} \label{lem:selmer-ses}
	Fix an elliptic curve $E$ over a number field $K$. For each positive integer $m$, there is an exact sequence
	\[0\to E(K)/mE(K)\to\op{Sel}_m(E/K)\to\Sha(E/K)[m]\to0.\]
\end{lemma}
\begin{proof}
	Functoriality of evaluating $E$ on a field yields a morphism
	% https://q.uiver.app/#q=WzAsMTAsWzAsMCwiMCJdLFsxLDAsIkVbbV0oXFxvdiBLKSJdLFsyLDAsIkUoXFxvdiBLKSJdLFszLDAsIkUoXFxvdiBLKSJdLFs0LDAsIjAiXSxbMCwxLCIwIl0sWzEsMSwiXFxkaXNwbGF5c3R5bGVcXHByb2RfdkVbbV0oXFxvdntLX3Z9KSJdLFsyLDEsIlxcZGlzcGxheXN0eWxlXFxwcm9kX3ZFKFxcb3Z7S192fSkiXSxbMywxLCJcXGRpc3BsYXlzdHlsZVxccHJvZF92RShcXG92e0tfdn0pIl0sWzQsMSwiMCJdLFswLDFdLFsxLDJdLFsyLDMsIm0iXSxbMyw0XSxbNSw2XSxbNiw3XSxbNyw4LCJtIl0sWzgsOV0sWzEsNiwiIiwxLHsic3R5bGUiOnsidGFpbCI6eyJuYW1lIjoiaG9vayIsInNpZGUiOiJ0b3AifX19XSxbMiw3LCIiLDEseyJzdHlsZSI6eyJ0YWlsIjp7Im5hbWUiOiJob29rIiwic2lkZSI6InRvcCJ9fX1dLFszLDgsIiIsMSx7InN0eWxlIjp7InRhaWwiOnsibmFtZSI6Imhvb2siLCJzaWRlIjoidG9wIn19fV1d&macro_url=https%3A%2F%2Fraw.githubusercontent.com%2FdFoiler%2Fnotes%2Fmaster%2Fnir.tex
	\[\begin{tikzcd}[cramped]
		0 & {E[m](\ov K)} & {E(\ov K)} & {E(\ov K)} & 0 \\
		0 & {\displaystyle\prod_vE[m](\ov{K_v})} & {\displaystyle\prod_vE(\ov{K_v})} & {\displaystyle\prod_vE(\ov{K_v})} & 0
		\arrow[from=1-1, to=1-2]
		\arrow[from=1-2, to=1-3]
		\arrow[hook, from=1-2, to=2-2]
		\arrow["m", from=1-3, to=1-4]
		\arrow[hook, from=1-3, to=2-3]
		\arrow[from=1-4, to=1-5]
		\arrow[hook, from=1-4, to=2-4]
		\arrow[from=2-1, to=2-2]
		\arrow[from=2-2, to=2-3]
		\arrow["m", from=2-3, to=2-4]
		\arrow[from=2-4, to=2-5]
	\end{tikzcd}\]
	of short exact sequences, where everything in sight is a continuous Galois module. Taking Galois cohomology thus produces another morphism
	% https://q.uiver.app/#q=WzAsMTAsWzAsMCwiMCJdLFsxLDAsIkUoSykvbUUoSykiXSxbMCwxLCIwIl0sWzEsMSwiXFxkaXNwbGF5c3R5bGVcXHByb2RfdkUoS192KS9tRShLX3YpIl0sWzIsMCwiXFxtYXRocm0gSF4xKEs7RVttXSkiXSxbMywwLCJcXG1hdGhybSBIXjEoSztFKVttXSJdLFsyLDEsIlxcZGlzcGxheXN0eWxlXFxwcm9kX3ZcXG1hdGhybSBIXjEoS192O0VbbV0pIl0sWzQsMCwiMCJdLFs0LDEsIjAiXSxbMywxLCJcXGRpc3BsYXlzdHlsZVxccHJvZF92XFxtYXRocm0gSF4xKEtfdjtFKVttXSJdLFswLDFdLFsxLDRdLFs0LDVdLFs1LDddLFsyLDNdLFszLDZdLFs2LDldLFs5LDhdLFsxLDNdLFs0LDZdLFs1LDldXQ==&macro_url=https%3A%2F%2Fraw.githubusercontent.com%2FdFoiler%2Fnotes%2Fmaster%2Fnir.tex
	\[\begin{tikzcd}[cramped]
		0 & {E(K)/mE(K)} & {\mathrm H^1(K;E[m])} & {\mathrm H^1(K;E)[m]} & 0 \\
		0 & {\displaystyle\prod_vE(K_v)/mE(K_v)} & {\displaystyle\prod_v\mathrm H^1(K_v;E[m])} & {\displaystyle\prod_v\mathrm H^1(K_v;E)[m]} & 0
		\arrow[from=1-1, to=1-2]
		\arrow[from=1-2, to=1-3]
		\arrow[from=1-2, to=2-2]
		\arrow[from=1-3, to=1-4]
		\arrow[from=1-3, to=2-3]
		\arrow[from=1-4, to=1-5]
		\arrow[from=1-4, to=2-4]
		\arrow[from=2-1, to=2-2]
		\arrow[from=2-2, to=2-3]
		\arrow[from=2-3, to=2-4]
		\arrow[from=2-4, to=2-5]
	\end{tikzcd}\]
	of short exact sequences. Now, the kernel of the rightmost vertical arrow is $\Sha(E/K)[m]$ by definition of $\Sha(E/K)$. Accordingly, we claim that we may take a pullback of the top short exact sequence to produce yet another morphism
	% https://q.uiver.app/#q=WzAsMTAsWzAsMSwiMCJdLFsxLDEsIkUoSykvbUUoSykiXSxbMiwxLCJcXG1hdGhybSBIXjEoSztFW21dKSJdLFszLDEsIlxcbWF0aHJtIEheMShLO0UpW21dIl0sWzQsMSwiMCJdLFszLDAsIlxcU2hhKEUvSylbbV0iXSxbMiwwLCJcXG9we1NlbH1fbShFL0spIl0sWzEsMCwiRShLKS9tRShLKSJdLFswLDAsIjAiXSxbNCwwLCIwIl0sWzAsMV0sWzEsMl0sWzIsM10sWzMsNF0sWzgsN10sWzcsNl0sWzYsNV0sWzUsOV0sWzcsMSwiIiwxLHsibGV2ZWwiOjIsInN0eWxlIjp7ImhlYWQiOnsibmFtZSI6Im5vbmUifX19XSxbNiwyLCIiLDEseyJzdHlsZSI6eyJ0YWlsIjp7Im5hbWUiOiJob29rIiwic2lkZSI6InRvcCJ9fX1dLFs1LDMsIiIsMSx7InN0eWxlIjp7InRhaWwiOnsibmFtZSI6Imhvb2siLCJzaWRlIjoidG9wIn19fV0sWzYsMywiIiwxLHsic3R5bGUiOnsibmFtZSI6ImNvcm5lciJ9fV1d&macro_url=https%3A%2F%2Fraw.githubusercontent.com%2FdFoiler%2Fnotes%2Fmaster%2Fnir.tex
	\[\begin{tikzcd}[cramped]
		0 & {E(K)/mE(K)} & {\op{Sel}_m(E/K)} & {\Sha(E/K)[m]} & 0 \\
		0 & {E(K)/mE(K)} & {\mathrm H^1(K;E[m])} & {\mathrm H^1(K;E)[m]} & 0
		\arrow[from=1-1, to=1-2]
		\arrow[from=1-2, to=1-3]
		\arrow[equals, from=1-2, to=2-2]
		\arrow[from=1-3, to=1-4]
		\arrow[hook, from=1-3, to=2-3]
		\arrow["\lrcorner"{anchor=center, pos=0.125}, draw=none, from=1-3, to=2-4]
		\arrow[from=1-4, to=1-5]
		\arrow[hook, from=1-4, to=2-4]
		\arrow[from=2-1, to=2-2]
		\arrow[from=2-2, to=2-3]
		\arrow[from=2-3, to=2-4]
		\arrow[from=2-4, to=2-5]
	\end{tikzcd}\]
	of short exact sequences. Here, the middle term of the top short exact sequence is in fact $\op{Sel}_m(E/K)$: this fiber product should consist of the elements of $\mathrm H^1(K;E[m])$ which vanish in $\prod_v\mathrm H^1(K_v;E)$, which by exactness is equivalent to their image along $\mathrm H^1(K;E[m])\to\prod_v\mathrm H^1(K_v;E[m])$ coming from $\prod_vE(K_v)/mE(K_v)$.
	
	It is now totally formal that the top row is exact: exactness on the right follows because the pullback of an epimorphism is an epimorphism. Further, exactness elsewhere amounts to saying that $E(K)/mE(K)$ is the kernel of $\op{Sel}_m(E/K)\to\Sha(E/K)[m]$, which follows because pullbacks commute with kernels (recall limits commute with limits).
\end{proof}
% It turns out that $\op{Sel}_m(E/K)$ fits into an exact sequence of the form
% \[0\to E(K)/mE(K)\to\mathrm{Sel}_m(E/K)\to\Sha(E/K)[m]\to0.\]
Thus, we see that $\op{Sel}_m(E/K)$ contains contributions from three interesting invariants of $E$: the $m$-torsion $E[m]$, the algebraic rank $\op{rank}_\ZZ E(K)$, and $\Sha(E/K)$. Of course, the $m$-torsion is the least interesting, so we introduce some notation to get rid of it.
\begin{notation}
	Fix an elliptic curve $E$ over a number field $K$. For each prime $p$, we define
	\[S_p(E/K)\coloneqq\dim_{\FF_p}\op{Sel}_p(E/K)-\dim_{\FF_p}E(K)[p].\]
\end{notation}
\begin{remark} \label{rem:reduced-selmer}
	Let $r$ be the algebraic rank of $E$ over $K$ so that $E(K)\cong E(K)_{\mathrm{tors}}\oplus\ZZ^{\oplus r}$. Thus, for any prime $p$,
	\[\frac{E(K)}{pE(K)}\cong\frac{E(K)_{\mathrm{tors}}}{pE(K)_{\mathrm{tors}}}\oplus\left(\frac\ZZ{p\ZZ}\right)^{\oplus r}\]
	Note $E(K)_{\mathrm{tors}}$ is some finite abelian group, so the kernel and cokernel of $p\colon E(K)_{\mathrm{tors}}\to E(K)_{\mathrm{tors}}$ have the same size by an Euler characteristic argument. Thus,
	\[\dim_{\FF_p}E(K)/pE(K)=\dim_{\FF_p}E(K)[p]+\op{rank}_\ZZ E(K).\]
	\Cref{lem:selmer-ses} now implies that $\op{rank}_\ZZ E(K)+\dim_{\FF_p}\Sha(E/K)[p]=S_p(E/K)$.
\end{remark}
Let's make some ``parity conjectures.'' Fix an elliptic curve $E$ over a number field $K$.
\begin{itemize}
	\item Note that $\Sha(E/K)$ is known to have an alternating ``Cassels--Tate'' pairing and is expected to be finite, so its size is conjectured to be a square.
	\item Similarly, $E[m](K)$ has a Weil pairing, which is a perfect alternating pairing on it, so it similarly follows that the size is a square.
\end{itemize}
For example, for taking $m$ to be a prime $p$, this produces the following conjecture via \Cref{rem:reduced-selmer}.
\begin{conj}[Partity for Mordell--Weil rank]
	Fix an elliptic curve $E$ over a number field $K$. Then for each prime $p$,
	\[S_p(E/K)\stackrel?\equiv\op{rank}E(K)\pmod2.\]
\end{conj}
By comparing with the Birch and Swinnerton-Dyer conjecture, we can make a parity conjecture comparing to modular forms.
\begin{conj}[Parity for global root number] \label{conj:root-number}
	Fix an elliptic curve $E$ over a number field $K$ with an attached modular form $f_E$. Then
	\[(-1)^{S_p(E/K)}=\varepsilon(f_E/K),\]
	where $\varepsilon(f_E/K)$ is the sign of the $L$-function's functional equation.
\end{conj}
\begin{remark}
	There is a purely local definition of $\varepsilon(f_E/K)$ which does not require us to know that there is an attached modular form.
\end{remark}
\begin{remark}
	\Cref{conj:root-number} is known if $K=\QQ$ by Nekov\'a\v{r} and Dokchitser--Dokchitser. If $E[p](K)$ is nontrivial, it is still known by Dokchitser--Dokchitser again. There are other results by \v{C}esnavi\v{c}ius.
\end{remark}

\subsection{\texorpdfstring{$2$}{2}-Descent} \label{subsec:descent}
In this subsection, we explain how to compute $2$-Selmer groups of elliptic curves $E$ over a number field $K$ for which $E[2](K)=E[2](\ov K)$.

To begin, suppose that $K$ is an arbitrary field of characteristic $0$, to be set to be a number field shortly. Writing $E$ into Weierstrass form $y^2=f(x)$ for a cubic $x$, one sees that the roots of $f$ produce the nontrivial $2$-torsion points of $f$. (This follows from the usual group law of $E$.) Thus, $f$ is required to fully factor over $K$, allowing us to write $E$ as the projective closure of the affine curve cut out by
\[y^2=(x-a_1)(x-a_2)(x-a_3)\]
for some $a_1,a_2,a_3\in K$. In this situation, we see that%\todo{}
\[E[2]=\{\infty,(a_1,0),(a_2,0),(a_3,0)\}.\]
Now, $E[2]$ has trivial Galois action, so we may identify it with the isomorphic Galois module $\mu_2^{\oplus2}$. For symmetry reasons, it will in fact be easier to identify it with the ``trace zero'' hyperplane $H$ of $\mu_2^{\oplus3}$: namely, we embed $E[2]$ into $\mu_2^{\oplus3}$ by
\[\begin{cases}
	\infty \mapsto (+1,+1,+1), \\
	(a_1,0) \mapsto (+1,-1,-1), \\
	(a_2,0) \mapsto (-1,+1,-1), \\
	(a_3,0) \mapsto (-1,-1,+1).
\end{cases}\]
Namely, the image of this embedding is $H=\{(\varepsilon_1,\varepsilon_2,\varepsilon_3)\in\mu_2^{\oplus3}:\varepsilon_1\varepsilon_2\varepsilon_3=1\}$, which is the kernel of the product map $H\to\mu_2$.
\begin{remark} \label{rem:projection-via-weil}
	This embedding can be explained by the Weil pairing: it is given by
	\[S\mapsto \big(e_2(S,(a_1,0)),e_2(S,(a_2,0)),e_2(S,(a_3,0))\big).\]
	Indeed, note that $e_2$ is linear and alternating by \Cref{lem:weil-pairing}, so it must have $e_2(\infty,T)=e_2(T,T)=1$ for each $T\in E[2]$. However, because $E[2]\cong(\ZZ/2\ZZ)^2$, if $e_2(S,T)=1$ for any $S\notin\{\infty,T\}$, then $e_2(-,T)$ is trivial, violating the non-degeneracy of \Cref{lem:weil-pairing}.
\end{remark}
Thus, we may identify $\mathrm H^1(K;E[2])=\mathrm H^1(K;H)$, which tracking through the functoriality of \Cref{ex:h1-of-mu-n} gives
\[\mathrm H^1(K;H)\cong\left\{(\alpha,\beta,\gamma):K^\times/K^{\times2}:\alpha\beta\gamma\in K^{\times2}\right\}.\]
In order to compute the $2$-Selmer group, we need to understand the image of the map $E(K)/2E(K)\to\mathrm H^1(K;E[2])=\mathrm H^1(K;H)$.
\begin{proposition} \label{prop:compute-delta-e}
	Fix an elliptic curve $E$ over a field $K$ which is the projective closure of $y^2=(x-a_1)(x-a_2)(x-a_3)$. Identifying $E[2]$ with the trace-zero hyperplane $H\subseteq\mu_2^{\oplus3}$, the boundary map $\delta\colon E(K)/2E(K)\to\mathrm H^1(K;H)$ is the map
	\[\delta\colon\begin{cases}
		(x,y) \mapsto (x-a_1,x-a_2,x-a_3) & \text{if }y\ne0, \\
		\infty \mapsto (1,1,1), \\
		(a_1,0) \mapsto ((a_1-a_2)(a_1-a_3),a_1-a_2,a_1-a_3), \\
		(a_2,0) \mapsto (a_2-a_1,(a_2-a_1)(a_2-a_3),a_2-a_3), \\
		(a_3,0) \mapsto (a_3-a_1,a_3-a_2,(a_3-a_1)(a_3-a_2)).
	\end{cases}\]
\end{proposition}
\begin{proof}
	Our exposition is taken from \cite[Theorem~X.1.1]{silverman} and the discussion after it. To be explicit, let $\delta_K$ be the isomorphism identifying $K^\times/K^{\times2}\to\mathrm H^1(K;\mu_2)$; it sends $\alpha\in K^\times/K^{\times2}$ to the $1$-cocycle $\sigma\mapsto\sigma\sqrt\alpha/\sqrt\alpha$.

	The idea is to compute $\delta$ using the Weil pairing, via \Cref{rem:projection-via-weil}. Because $e_2$ is linear and Galois-invariant, we any $T\in E[2]$ produces a map $e_2(-,T)\colon E[2]\to\mu_2$, so any $P\in E(K)/2E(K)$ functorially produces a $1$-cocycle
	\[\sigma\mapsto e_2(\delta(P)(\sigma),T)\]
	in $\mathrm H^1(K;\mu_2)$, which must be identified with $\delta_K(b(P,T))$ for some uniquely defined $b(P,T)\in K^\times/K^{\times2}$. In fact, by \Cref{rem:projection-via-weil}, we see that $e_2(-,(a_i,0))\colon E[2]\to\mu_2$ is projection onto the $i$th coordinate of $E[2]\into H$. Thus, $b(P,(a_i,0))$ will continue to be the $i$th coordinate in $\mathrm H^1(K;H)\into\left(K^\times/K^{\times2}\right)^3$.

	We thus see that we will be content with computing $b(P,T)$ for $T\in E[2]\setminus\{\infty\}$; say $T\coloneqq(a_i,0)$. To begin, fix some $Q\in E(K^{\mathrm{sep}})$ with $2Q=P$, and fix some $\beta\in\ov K$ with $\beta^2=b(P,T)$. On one hand, we see that $\delta_K(b(P,T))(\sigma)=\sigma\beta/\beta$. On the other hand, choosing $f$ and $g$ as in \Cref{ex:compute-e2}, we see that
	\[e_2(\delta(P)(\sigma),T)=\frac{g(X+\sigma Q-Q)}{g(X)}.\]
	Now, provided that $g(Q)\ne0$, which is equivalent to $g(Q)^2=f(2Q)=f(P)\ne0$, we may plug in $Q$ to see $e_2(\delta(P)(\sigma),T)=g(\sigma Q)/g(Q)$, so
	\[\frac{\sigma g(Q)}{g(Q)}=\frac{\sigma\beta}{\beta}.\]
	Thus, $\delta_K(g(Q))=\delta_K(\beta)$, so $g(Q)$ and $\beta$ represent the same class in $K^\times/K^{\times2}$. Accordingly, up to squares, we can compute $b(P,T)$ as $\beta^2=g(Q)^2$, which is $f(2Q)$ by construction of the Weil pairing, which is $f(P)$ (as usual, provided this makes sense).

	We now recall that $f(x)=x-a_i$, so we find that the $i$th coordinate of $\delta(x,y)$ will be $x-a_i$ whenever $a_i\ne0$. To finish up the calculation, we note that $\delta(\infty)=(1,1,1)$ because identities go to identities, and the remaining $i$th coordinate of $\delta(a_i,0)$ can be computed from the other two because all three coordinates must multiply to be a square.
\end{proof}
\begin{corollary} \label{cor:test-im-delta-descent}
	Fix an elliptic curve $E$ over a field $K$ which is the projective closure of $y^2=(x-a_1)(x-a_2)(x-a_3)$. Identifying $E[2]$ with the trace-zero hyperplane $H\subseteq\mu_2^{\oplus3}$, a triple $(\alpha,\beta,\gamma)\in\mathrm H^1(K;H)$ is in the image of the boundary map from $E(K)/2E(K)$ if and only if the conic $T_{(\alpha,\beta,\gamma)}\subseteq\PP(1,1,1,2,1)$ cut out by the affine equations
	\[\begin{cases}
		\alpha u^2=x-a_1, \\
		\beta v^2=x-a_2, \\
		\gamma w^2=x-a_3
	\end{cases}\]
	admits a solution. (Namely, the coordinates $u$, $v$, and $w$ have weight $1$, and $x$ has weight $2$.)
\end{corollary}
\begin{proof}
	Let's begin by showing that admitting a solution implies being in the image of $\delta$. In projective coordinates $[U:V:W:X:Z]$, the equations are
	\[\begin{cases}
		\alpha U^2=X-a_1Z^2, \\
		\beta V^2=X-a_2Z^2, \\
		\gamma W^2=X-a_3Z^2.
	\end{cases}\]
	The points at infinity occur with $Z=0$, where we see that we have a point if and only if $\alpha U^2=\beta V^2=\gamma W^2$, which amounts to requiring that $(\alpha,\beta,\gamma)=(1,1,1)$ in $\left(K^\times/K^{\times2}\right)^3$.

	Otherwise, we are allowed to work in affine coordinates, setting $Z=1$. The idea is to use a solution to construct an explicit pre-image, using the calculation of \Cref{prop:compute-delta-e}. The presence of a solution means that $\alpha(x-a_1)$, $\beta(x-a_2)$, and $\gamma(x-a_3)$ are all squares, which in turn means that we can find $y$ for which
	\[y^2=(x-a_1)(x-a_2)(x-a_3).\]
	We now see that $(\alpha,\beta,\gamma)$ is the image of $(x,y)$ along $\delta$: this is immediately apparent if $y\ne0$ (i.e., $x\notin\{a_1,a_2,a_3\}$), but even if (say) $(x,y)=(a_1,0)$, then $\beta(x-a_2)$ and $\gamma(x-a_3)$ are nonzero squares and thus uniquely determine $\alpha\in K^\times/K^{\times2}$, so we still find that $(\alpha,\beta,\gamma)=\delta(a_1,0)$. (A similar argument works for $(x,y)=(a_2,0)$ and $(x,y)=(a_3,0)$---one just has to rearrange the indices.)

	This argument also tells us how to show that being in the image of $\delta$ implies that we admit a solution.
	\begin{itemize}
		\item We handled $\delta(\infty)$ in the first paragraph. 
		\item For $(x,y)\in E(K)$ with $y\ne0$, we see that $\delta(x,y)=(x-a_1,x-a_2,x-a_3)$, so $T_{\delta(x,y)}$ admits the solution $(u,v,w,x)=(1,1,1,x)$.
		\item For the remaining points $(x,y)$ with $y=0$, it is by symmetry enough to only handle $(x,y)=(a_1,0)$. Then $\delta(x,y)=(\alpha,\beta,\gamma)$ has $\beta=a_1-a_2$ and $\gamma=a_1-a_3$, so $T_{\delta(x,y)}$ admits the solution $(u,v,w,x)=(0,1,1,a_1)$.
		\qedhere
	\end{itemize}
	% which makes it apparent that there are no points at infinity live at $T=0$, and the remaining points are on the affine chart $T=1$.
\end{proof}
\begin{remark}
	Here is a more geometric argument for \Cref{cor:test-im-delta-descent}. To understand the image of this map $\delta$, it is equivalent to understand the kernel of the next map in the long exact sequence, which is
	\[\mathrm H^1(K;E[2])\to\mathrm H^1(K;E)[2].\]
	Now, $\mathrm H^1(K;E)$ classifies principal homogeneous spaces \cite[Section~X.3]{silverman}, which are trivial if and only if they admit a $K$-rational point (after all, principal homogeneous spaces for $E$ are twists of $E$). Thus, it is enough to check that the principal homogeneous space associated to the triple $(\alpha,\beta,\gamma)$ admits a $K$-rational point, but one can check that this principal homogeneous space is exactly the conic $T_{(\alpha,\beta,\gamma)}$!
\end{remark}
\begin{remark} \label{rem:test-im-delta-descent-hilbert}
	After rearranging, solving the system in \Cref{cor:test-im-delta-descent} is equivalent to solving the (projective closure of the) system
	\[\begin{cases}
		\alpha u^2-\beta v^2=a_2-a_1, \\
		\alpha u^2-\gamma w^2=a_3-a_1.
	\end{cases}\]
\end{remark}
% \Cref{rem:test-im-delta-descent-hilbert} suggests that we can begin to handle solving these equations using the Hilbert symbol.
% \begin{definition}[Hilbert symbol]
%	
% \end{definition}
% In our immediate application
While we're here, we give some general remarks for how big these groups should be. In the case of $2$-descent, one can get away with just doing Kummer theory.
\begin{example} \label{ex:size-im-delta-v}
	Fix an elliptic curve $E$ over a number field $K$ which is the projective closure of $y^2=(x-a_1)(x-a_2)(x-a_3)$. Then
	\[\dim_{\FF_2}E(K_v)/2E(K_v)=\begin{cases}
		0 & \text{if }K_v=\CC, \\
		1 & \text{if }K_v=\RR, \\
		2 & \text{if }v\text{ is odd}, \\
		2+[K_v:\QQ_2] & \text{if }v\text{ is even}.
	\end{cases}\]
\end{example}
\begin{proof}
	By \Cref{thm:selmer-by-lagrangian}, the image of $E(K_v)/2E(K_v)\to\mathrm H^1(K_v;E[2])$ should have dimension equal to
	\[\frac12\dim_{\FF_2}\mathrm H^1(K_v;E[2])=\dim_{\FF_2}\mathrm H^1(K_v;\mu_2).\]
	By \Cref{ex:h1-of-mu-n}, we are left to compute $K_v^\times/K_v^{\times2}$. In the archimedean cases, we directly see that $\CC^\times/\CC^{\times2}=1$ (because $\CC$ is algebraically closed) and $\RR^\times/\RR^{\times2}=\RR^\times/\RR^+=\{\pm1\}$.
	
	Otherwise, we suppose that $K$ is a finite extension of $\QQ_p$, and we claim that
	\[K_v^\times\cong\ZZ\times\FF_v\times\mu_{p^\infty}(K_v)\times\OO_v\]
	as abelian groups. To begin, note $K_v^\times\cong\ZZ\times\OO_v^\times$ by using the valuation; additionally, by modding out by $\mf p_v$, we find that $\OO_v^\times\cong\FF_v\times(1+\mf p_v)$.
	
	Now, recall that the exponential map $\exp\colon\mf p_v\to(1+\mf p_v)$ identifies open neighborhoods of the identity of $K_v$ and $K_v^\times$, so it follows that $\OO_v^\times$ is a finitely generated $\OO_v$-module. Because $\OO_v$ is a principal ideal domain, it follows that $\OO_v^\times$ is isomorphic to its torsion times its free part. The free part of $\OO_v^\times$ has rank $1$ because the exponential map identifies a finite-index open subgroup with $\OO_v$. Lastly, the torsion of $(1+\mf p_v)$ must be $p$-power (because $(1+\varpi)^n\equiv1+n\varpi\pmod{\mf p_v^{2m}}$ for any $\varpi\in\mf p^m$), and conversely, the $p$-power torsion of $K_v^\times$ all lives in $\OO_v^\times$ by looking at the valuation and is in fact $1\pmod{\mf p_v}$ by looking$\pmod{\mf p_v}$. The claim follows.

	To complete the calculation, we write
	\[\frac{K_v^\times}{K_v^{\times2}}\cong\frac{\ZZ}{2\ZZ}\times\frac{\FF_v^\times}{\FF_v^{\times2}}\times\frac{\mu_{p^\infty}(K_v)}{\mu_{p^\infty}(K_v)^2}\times\frac{\OO_v}{2\OO_v}.\]
	Continuing, we note that the kernel and cokernel of the endomorphisms $2\colon\FF_v^\times\to\FF_v^\times$ and $2\colon\mu_{2^\infty}(K_v)\to\mu_{2^\infty}(K_v)$ have the same size, so we see
	\[\dim_{\FF_2}\frac{K_v^\times}{K_v^{\times2}}\cong1+\dim_{\FF_2}\FF_v^\times[2]+\dim_{\FF_2}\mu_{p^\infty}(K_v)[2]+\dim_{\FF_2}\frac{\OO_v}{2\OO_v}.\]
	We now have two cases.
	\begin{itemize}
		\item If $v$ is odd, then $\FF_v^\times[2]\cong\ZZ/2\ZZ$ and $\mu_{p^\infty}(K_v)[2]=0$ and $\OO_v/2\OO_v=0$. In total, we get $2$ dimensions.
		\item If $v$ is even, then $\FF_v^\times[2]=0$ and $\mu_{p^\infty}(K_v)[2]=\mu_2(K_v)=\{\pm1\}$ and $\OO_v/2\OO_v\cong(\ZZ/2\ZZ)^{[K_v:\QQ_2]}$. In total, we get $2+[K_v:\QQ_2]$ dimensions.
		\qedhere
	\end{itemize}
	%  Otherwise, if $v$ is even, then $\FF_v^\times/\FF_v^{\times2}=1$, and $\OO_v/2\OO_v\cong(\ZZ/2\ZZ)^{[K_v:\QQ_2]}$; additionally, the kernel and cokernel of $2\colon\mu_{2^\infty}(K_v)\to\mu_{2^\infty}(K_v)$ have the same size. Totaling these calculations completes.
\end{proof}
\begin{remark}
	For an odd prime $\ell$, we see that $K_v^\times/K_v^{\times\ell}$ is always $\{1\}$ when $K_v\in\{\RR,\CC\}$. When $v$ is finite and $p$-adic, we similarly find that
	\[\dim_{\FF_\ell}\frac{K_v^\times}{K_v^{\times\ell}}=1+\dim_{\FF_\ell}\FF_v[\ell]+\dim_{\FF_\ell}\mu_{p^\infty}(K_v)[\ell]+\dim_{\FF_\ell}\frac{\OO_v}{\ell\OO_v}.\]
	Here are our two cases.
	\begin{itemize}
		\item If $v\nmid\ell$, then the dimension is $1+1_{\ell\mid\#\FF_v-1}+0+0$.
		\item If $v\mid\ell$, then the dimension is $1+0+1_{\zeta_p\in K_v}+[K_v:\QQ_p]$.
	\end{itemize}
	% If $v\nmid\ell$, then the dimension is $1+1+0+0=2$ when $\ell\mid\#\FF_v-1$ and $1+0+0+0=1$ otherwise. Otherwise, if $v\mid\ell$, then the dimension is $1+0+\dim_{\FF_\ell}\mu_{\ell^\infty}(K_v)$
\end{remark}
However, with a little more theory, one can say more.
\begin{lemma} \label{lem:e-has-finite-index-o}
	Fix an elliptic curve $E$ over a nonarchimedean local field $K_v$. Then $E(K_v)$ admits a finite-index subgroup isomorphic to $\OO_v$.
\end{lemma}
\begin{proof}
	The proof of this result is fairly involved, so we will be sketchy; we refer to \cite[Proposition~VII.6.3]{silverman} for more details. Let $E_0(K_v)\subseteq E(K)$ denote the collection of points which reduce to a non-singular point in $E(\FF_v)$, and we let $E_1(K_v)\subseteq E(K)$ denote the collection of points which reduce to the identity of $E(\FF_v)$. 
	
	The main point is to show that $E(K_v)/E_0(K_v)$ is finite, but for now let's explain why it completes the proof. It turns out that the canonical maps
	\[0\to E_1(K_v)\to E_0(K_v)\to E(\FF_v)\to0\]
	assemble into a short exact sequence \cite[Proposition~VII.2.1]{silverman}. Thus, it is enough to show that $E_1(K_v)$ admits a finite-index subgroup isomorphic to $\OO_v$. Well, $E_1(K_v)$ is isomorphic to $G_E(\mf m_v)$, where $G_E$ is the one-dimensional formal group of $E$ \cite[Proposition~VII.2.2]{silverman}. Then the canonical filtration $G_E(\mf m_v^\bullet)$ shows that $G_E(\mf m_v^i)/G_E\left(\mf m_v^{i+1}\right)$ is finite for all $i$, and for $i$ large enough, there is a logarithm map establishing that $G_E(\mf m_v^i)$ is isomorphic to $\OO_v$. This completes the proof modulo the finiteness of $E(K_v)/E_0(K_v)$.

	It remains to show that $E(K_v)/E_0(K_v)$, for which we follow \cite[Exercise~7.6]{silverman}. Because $K_v$ is a topological field, we see that $E(K_v)\subseteq\PP^2(K_v)$ is a topological group. In fact, $E(K_v)\subseteq\PP^2(K_v)$ is a closed subset of the compact space $\PP^2(K_v)$, so $E(K_v)$ is compact. The reduction map $\PP^2(K_v)\to\PP^2(\FF_v)$ is a continuous map to a finite discrete space, so $E_0(K_v)\subseteq E(K_v)$ is an open subgroup. Compactness then forces $E_0(K_v)$ to be finite-index in $E(K_v)$.
\end{proof}
\begin{proposition}
	Fix an elliptic curve $E$ over field $\QQ_p$. For any prime $\ell$,
	\[\dim_{\FF_\ell}E(\QQ_p)/\ell E(\QQ_p)=\dim_{\FF_\ell}E(\QQ_p)[\ell]+\begin{cases}
		1 & \text{if }p=\ell, \\
		0 & \text{if }p\ne\ell.
	\end{cases}\]
	Thus, $\dim_{\FF_\ell}E(\QQ_p)/\ell E(\QQ_p)\in\{0,1,2,3\}$, where $3$ is only possible when $p=\ell$.
\end{proposition}
\begin{proof}
	Set $K_v\coloneqq\QQ_\ell$ for brevity. We will use \Cref{lem:e-has-finite-index-o}, which grants us an exact sequence
	\[0\to\OO_v\to E(K_v)\to C\to0,\]
	where $C$ is some finite abelian group. This exact sequence has an endomorphism given by multiplication by $\ell$. Applying the Snake lemma to this endomorphism yields the exact sequence
	\[0\to\OO_v[\ell]\to E(K_v)[\ell]\to C[\ell]\to\frac{\OO_v}{\ell\OO_v}\to\frac{E(K_v)}{\ell E(K_v)}\to\frac C{\ell C}\to0.\]
	The result will follow by taking dimensions of this exact sequence; for example, just the right-exact part of this sequence immediately shows us that $E(K_v)/\ell E(K_v)$ is finite. We thus have
	\[\dim_{\FF_\ell}\frac{E(K_v)}{\ell E(K_v)}-\dim_{\FF_\ell}E(K_v)[\ell]=\dim_{\FF_\ell}\frac{\OO_v}{\ell\OO_v}-\dim_{\FF_\ell}\OO_v[\ell]+\dim_{\FF_\ell}\frac C{\ell C}-\dim_{\FF_\ell}C[\ell].\]
	Now, we note that $\OO_v[\ell]=0$ because $K_v$ has characteristic $0$. Continuing, and $\#C_\ell=\#(C/\ell C)$ because the kernel and cokernel of the endomorphism $\ell\colon C\to C$ should have the same size. We are left with
	\[\dim_{\FF_\ell}\frac{E(K_v)}{\ell E(K_v)}=\dim_{\FF_\ell}\frac{\OO_v}{\ell\OO_v}+\dim_{\FF_\ell}E(K_v)[\ell].\]
	We now note that $\ell=p$ implies that $\OO_v/\ell\OO_v\cong\ZZ/\ell\ZZ$; otherwise, $\OO_v/\ell\OO_v=0$.
\end{proof}

\subsection{Congruent Number Elliptic Curves}
We now return to the congruent number elliptic curves $E_d\colon y^2=x(x-d)(x+d)$, where $d\in\ZZ$ is some squarefree positive integer. It turns out that $E_d$ is a quadratic twist of $E_1\colon y^2=x^3-x$, and these elliptic curves have complex multiplication by $\ZZ[i]$. Importantly, the $2$-torsion
\[E[2]=\{\infty,(0,0),(+d,0),(-d,0)\}\]
is fully defined over $\QQ$. Here is a bit more about what is known.
\begin{remark}[Birch--Stephens]
	Fix a squarefree positive integer $d$. It is known that
	\[\varepsilon(E_d/\QQ)=\begin{cases}
		+1 & \text{if }d\equiv 1,2,3\pmod8, \\
		-1 & \text{if }d\equiv 5,6,7\pmod8.
	\end{cases}\]
	Furthermore, they computed
	\[S_2(E_d/\QQ)\equiv\begin{cases}
		0\pmod2 & \text{if }d\equiv1,2,3\pmod8, \\
		1\pmod2 & \text{if }d\equiv5,6,7\pmod8.
	\end{cases}\]
\end{remark}
They proved this using calculations of Selmer groups. We will show the following.
\begin{theorem} \label{thm:cong-selmer}
	Fix an odd positive prime integer $d=p$, and let $E_p$ be the projective closure of $y^2=x(x-p)(x+p)$. Then
	\[S_2(E_p/\QQ)=\begin{cases}
		2 & \text{if }p\equiv1\pmod8, \\
		0 & \text{if }p\equiv3\pmod8, \\
		1 & \text{if }p\equiv5,7\pmod8.
	\end{cases}\]
\end{theorem}
\begin{remark}
	In the first case $p\equiv1\pmod8$, it is possible to get both $0$ and $2$ for the Mordell--Weil rank. Indeed, for many small primes $p$, $\op{rank}E_p(\QQ)=0$, but $\op{rank}E_{41}(\QQ)=2$.
\end{remark}
\begin{remark}
	It has been verified by Heegner--Monsky that $p\equiv5,7\pmod8$ implies $\op{rank}E_p(\QQ)=1$. This requires the construction of non-torsion points, which uses Heegner points.
\end{remark}
We are going to use $2$-descent. As in \Cref{subsec:descent}, we identify $\mathrm H^1(\QQ;E_d)$ with $\mathrm H^1(\QQ;H)$. We begin with two technical calculations.
\begin{lemma} \label{lem:cong-local-condition}
	Fix an odd positive squarefree integer $d$, and let $E_d$ be the elliptic curve over $\QQ$ which is the projective closure of $y^2=x(x-d)(x+d)$. We will compute the image of $\delta_v\colon E_d(\QQ_v)/2E_d(\QQ_v)\to\mathrm H^1(\QQ_v;H)$ for each place $v$.
	\begin{listalph}
		\item If $v\nmid2d\infty$, then the image of $\delta_v$ consists of the triples $(\alpha,\beta,\gamma)$ such that $v(\alpha)=v(\beta)=v(\gamma)=0$.
		\item The image of $\delta_v$ contains the triples
		\[S\coloneqq\{(1,1,1),(-1,-d,d),(d,2,2d),(-d,-2d,2)\}.\]
		\item If $v\mid d\infty$, then the image of $\delta_v$ is $S$.
		\item If $v=2$, the image of $\delta_v$ is $\op{span}(S\cup\{(1,5,5)\})$.
	\end{listalph}
\end{lemma}
\begin{proof}
	We show the parts in sequence.
	\begin{listalph}
		\item If $v\nmid2d\infty$, then $E_d$ has good reduction at the finite place $v$, so by \Cref{lem:ur-of-elliptic-curve}, the image of $\delta_v$ is $\mathrm H^1_{\mathrm{ur}}(\QQ_v;H)$. The result now follows by looking coordinate-wise via \Cref{ex:unramified-mu-m}.
		\item The given set $S$ is precisely the image of $E_d[2]$. Indeed,
		\begin{align*}
			\delta_v(\infty) &= (1,1,1), \\
			\delta_v(0,0) &= (-1,-d,d), \\
			\delta_v(d,0) &= (d,2,2d), \\
			\delta_v(-d,0) &= (-d,-2d,2).
		\end{align*}
		\item If $v=\infty$, then we have a linearly independent set $\{(-1,-1,+1)\}$, which spans the image of $\delta_v$ by \Cref{ex:size-im-delta-v}. Similarly, if $v\mid d$, then we have a linearly independent set $\{(-1,-d,d),(d,2,2d)\}$ (because $d$ is squarefree), which spans the image of $\delta_v$ by \Cref{ex:size-im-delta-v}.
		\item %You may do this for the homework!
		%
		Because $-1$, $2$, and $5$ are linearly independent in $\QQ_2^\times/\QQ_2^{\times2}$, we see that the triples $(1,5,5)$, $(-1,-d,d)$, and $(d,2,2d)$ are linearly independent triples. Thus,
		\[\dim_{\FF_2}\op{span}(S\cup\{1,5,5\})=3,\]
		which also equals $\dim_{\FF_2}\im\delta_v$ by \Cref{ex:size-im-delta-v}. Thus, it suffices to show that $\op{span}(S\cup\{1,5,5\})\subseteq\im\delta_v$.
		
		By (b), it is enough to check that $(1,5,5)\in\im\delta_v$. For this, we will use \Cref{rem:test-im-delta-descent-hilbert}, which tells us that we need to produce a nonzero solution to the system
		\[\begin{cases}
			x^2-5y^2=+dw^2, \\
			x^2-5z^2=-dw^2,
		\end{cases}\]
		in $\QQ_2$. The solubility of this system does not change if we change $d$ by an element of $\QQ_2^{\times2}$, so we may assume $d\in\{\pm1,\pm5\}$. Similarly, by symmetry, we may adjust the sign of $d$, so we may assume that $d\in\{1,5\}$. In these cases, we may set $(x,w)=(1,2)$ so that we need $5y^2\in\{-3,-19\}$ and $5z^2\in\{5,21\}$, both of which are possible.
		\qedhere
	\end{listalph}
\end{proof}
We will also need the following technical result.
\begin{lemma} \label{lem:conic-almost-everywhere}
	Fix an odd positive squarefree integer $d$, and let $E$ be the elliptic curve over $\QQ$ which is the projective closure of $y^2=x(x-d)(x+d)$. Further, fix some triple $(\alpha,\beta,\gamma)\in\mathrm H^1(\QQ;H)$. If $(\alpha,\beta,\gamma)$ is in the image of $\delta_v\colon E_d(\QQ_v)/2E_d(\QQ_v)\to\mathrm H^1(\QQ_v;H)$ for each place $v\ne2$, then it is also in the image of $\delta_2$.% Let $T\subseteq\PP^n_\ZZ$ be a smooth projective conic. If $T$ admits solutions in $\QQ_v$ for all but one place $v_0$, then $T$ admits solutions in $\QQ_{v_0}$ as well.
\end{lemma}
\begin{proof}
	% You may show this for the homework!
	We use \Cref{cor:test-im-delta-descent}, freely using the calculations of \Cref{lem:cong-local-condition}. Fix a triple $(\alpha,\beta,\gamma)$ which is in the image of $\delta_v$ for each place $v\ne2$; we may as well represent $(\alpha,\beta,\gamma)$ as a triple of squarefree integers. We will show that $(\alpha,\beta,\gamma)\in\im\delta_2$.
	\begin{enumerate}
		\item Quickly, note that $\alpha$ is odd (modulo squares). Surely this is the case for $(\alpha,\beta,\gamma)\notin\delta_2(E[2])$. Otherwise, by \Cref{prop:compute-delta-e}, we know $\alpha$ equals the $x$-coordinate of some $(x,y)\in E(\QQ_2)$, meaning
		\[y^2=x\left(x^2-d^2\right).\]
		We now compute some valuations to show that $\nu_2(x)$ is even, which means $\alpha$ is odd.
		\begin{itemize}
			\item There is nothing to do if $\nu_2(x)=0$.
			\item If $\nu_2(x)>0$, then the valuations are $2\nu_2(y)=\nu_2(x)$, so $\nu_2(x)$ is even.
			\item If $\nu_2(x)<0$, then the valuations are $2\nu_2(y)=-3\nu_2(x)$, so $\nu_2(x)$ is still even.
		\end{itemize}
	
		\item We make some reductions. By \Cref{lem:cong-local-condition}, we know that $(\alpha,\beta,\gamma)$ is unramified for $v\nmid2d\infty$, so the prime factorizations of $\alpha$, $\beta$, and $\gamma$ are supported in the prime factors of $2d$.
	
		For our next few reductions, we note that multiplying any triple by the image of $\delta(E[2])$ will not change whether it is in the image of $\delta_2$. For example, with $\alpha$ odd, we see that we may multiply by the triple $(d,2,2d)$ to force $\beta$ to be odd, which in turn forces $\gamma$ to be odd too. Multiplying by the triple $(-1,-d,d)$, we may assume that $\alpha\pmod8$ is in $\{1,5\}$. Similarly, multiplying by the triple $(1,5,5)$, we may assume that $\beta\pmod8$ is in $\{1,3\}$.
	
		\item We complete the proof using Hilbert symbols. By \Cref{rem:test-im-delta-descent-hilbert}, we know that the equations
		\[\begin{cases}
			\alpha x^2-\beta y^2=-dw^2, \\
			\alpha x^2-\gamma z^2=+dw^2
		\end{cases}\]
		has solutions in each $\QQ_v$ for $v\ne2$. Thus, $(-d\alpha,d\beta)_v=(d\alpha,-d\gamma)_v=(2d\beta,-2d\gamma)_v=1$ for each $v\ne2$, so Hilbert reciprocity\footnote{In this case, it is possible to unwind the application of Hilbert reciprocity into merely applications of Quadratic reciprocity, but this language is convenient anyway.} implies that
		\[(-d\alpha,d\beta)_2=(d\alpha,-d\gamma)_2=(2d\beta,-2d\gamma)_2=1.\]
		We now show that $(\alpha,\beta,\gamma)\in\im\delta_2$ directly. We may consider this triple up to $\QQ_2^{\times2}$, meaning that we may assume $\alpha\in\{1,5\}$ and $\beta\in\{1,3\}$. For example, because everything is odd and $\alpha\in\{1,5\}$, we see that $(\alpha,d)_2=(\alpha,\beta)_2=(\alpha,\gamma)_2=1$.\footnote{In particular, $(5,-)_2$ vanishes on odds. It is not hard to show $(5,-5)_2=1$ (because $5\cdot1^2-5\cdot1^2=0^2$), and $(5,-1)_2=1$ because $5\cdot1^2-1\cdot1^2=2^2$.} As such, $(-d\alpha,d\beta)_2=(-d,\beta)_2$ and $(d\alpha,-d\gamma)_2=(d,\beta)_2$, so $(-1,\beta)_2=1$. Thus, $\beta\ne3$ is forced,\footnote{We need to show that $(-1,3)_2=-1$, which holds because $3x^2-y^2=z^2$ has no nontrivial solutions by$\pmod8$ considerations.} so $\gamma=1$ follows.
		\qedhere
	\end{enumerate}
\end{proof}
We now proceed with the proof of \Cref{thm:cong-selmer}.
\begin{proof}[Proof of \Cref{thm:cong-selmer}]
	We have identified $\mathrm H^1(\QQ;E_d[2])$ with the trace-zero hyperplane of $\left(\QQ^\times/\QQ^{\times2}\right)^3$. Now, let $\mc L_v\subseteq\left(\QQ^\times/\QQ^{\times2}\right)^3$ be the corresponding local condition of the Selmer group at the place $v$, as computed in \Cref{lem:cong-local-condition}. Thus,
	\[\op{Sel}_2(E_d/\QQ)\cong\left\{(\alpha,\beta,\gamma):(\alpha,\beta,\gamma)\in\mc L_v\text{ for all }v\right\}.\]
	The local conditions $v\nmid2d\infty$ show that $\alpha$, $\beta$, and $\gamma$ should (up to squares) be supported on primes dividing $2d$; adjusting these rationals up to squares, we may assume that they are all integers dividing $2d$.

	Now, we are not actually interested in computing the Selmer group on the nose. Instead, we would like to compute the (dimension of the) quotient by $E[2]$. Well, examining the local condition at $\infty$, we see that taking a quotient by the subgroup generated by $\delta(0,0)=(-1,-d,d)$ corresponds exactly to assuming $(\alpha,\beta,\gamma)$ are all positive---a priori, none are negative or exactly $\alpha$ and $\beta$ are negative. Similarly, examining the local condition at $2$, we see that taking a quotient by the subgroup generated by $\delta(d,0)=(d,2,2d)$ corresponds exactly to assuming that $(\alpha,\beta,\gamma)$ are all odd. Thus,
	\[\frac{\op{Sel}_2(E_d/\QQ)}{E[2]}\subseteq\left\{(\alpha,\beta,\gamma)\in\ZZ_{>0}^3:\alpha,\beta,\gamma\mid d,\alpha\beta\gamma\text{ is square}\right\}.\]
	By \Cref{lem:cong-local-condition}, all triples $(\alpha,\beta,\gamma)$ in the above set are automatically in the local condition at a place $v\nmid2d$, so there are only finitely many more places to check.
	
	Only now do we use the fact that $d=p$ is prime. By \Cref{lem:conic-almost-everywhere} combined with \Cref{cor:test-im-delta-descent}, we are allowed to avoid checking $(\alpha,\beta,\gamma)\in\mc L_{v_0}$ for a single place $v_0$; we choose $v_0=2$, so it only remains to check the place at the prime $p$. In other words, we are interested in which of the triples
	\[\{(1,1,1),(1,p,p),(p,1,p),(p,p,1)\}\]
	live in $\mc L_p=\{(1,1,1),(-1,-p,p),(p,2,2p),(-p,-2p,2)\}$. (Note that the elements of $\mc L_p$ are only defined up to squares!) We handle these one at a time.
	\begin{itemize}
		\item We see that $(1,1,1)\in\mc L_p$ always.
		\item By examining valuations (even$\pmod2$), we see that $(1,p,p)\in\mc L_p$ if and only if it is $(-1,-p,p)$ up to squares, which is equivalent to $-1$ being square, which is equivalent to $p\equiv1\pmod4$.
		\item By examining valuations, we see that $(p,1,p)\in\mc L_p$ if and only if it is $(p,2,2p)$ up to squares, which is equivalent to $2$ being square, which is equivalent to $p\equiv\pm1\pmod8$.
		\item Lastly, we similarly find that $(p,p,1)\in\mc L_p$ if and only if it is $(-p,-2p,p)$ up to squares, which is equivalent to $-1$ and $2$ being squares, which is equivalent to $p\equiv1\pmod8$.
	\end{itemize}
	Totaling the above cases completes the proof.
\end{proof}
% We are going to do $2$-descent for elliptic curve $E$ over $K$ with $E[2](\ov K)=E[2](K)$. In this case, we may write
% \[E\colon y^2=(x-a_1)(x-a_2)(x-a_3).\]
% Because $E[2]$ is defined over $K$, the Galois action is trivial. Thus, we see $\mathrm H^1(K;E[2])$ is simply $\mathrm H^1(K;\mu_2)^2$, and we can compute this cohomology using the ``Kummer'' exact sequence
% \[1\to\mu_2\to\ov K^\times\stackrel2\to\ov K^\times\to1,\]
% which in Galois cohomology produces an isomorphism $\mathrm H^1(K;\mu_2)\cong K^\times/K^{\times2}$ by Hilbert's theorem 90. We may now identify $\mathrm H^1(K;E[2])$ with
% \[\left\{(\alpha,\beta,\gamma)\in(K^\times/K^{\times2})^3:\alpha\beta\gamma=1\right\}.\]
% It turns out that this map (approximately) sends some point $(x,y)$ of $E(K)/2E(K)$ to the triple $(x-a_1,x-a_2,x-a_3)$ when $E$ has the form $y^2=(x-a_1)(x-a_2)(x-a_3)$. Technically speaking, we should note that we send $\infty$ to the identity $(1,1,1)$, and we send any two-torsion point like $(a_1,0)$ to the triple whose last two coordinates are $a_1-a_2$ and $a_3-a_2$.

% Of course, we would like a way to know if an interesting triple $(\alpha,\beta,\gamma)$ is in the image without having to find points in $E(K)$ first. Here is one such test.
% \begin{lemma}
% 	Fix an elliptic curve $E\colon y^2=(x-a_1)(x-a_2)(x-a_3)$ over a finite extension $K$ of $\QQ_p$. Then a triple $(\alpha,\beta,\gamma)$ lies in the image of the above map if and only if the system of equations
% 	\[\begin{cases}
% 		\alpha u^2=x-a_1, \\
% 		\beta v^2=x-a_2, \\
% 		\gamma w^2=x-a_3
% 	\end{cases}\]
% 	admits a solution. We will call this conic $T_{(\alpha,\beta,\gamma)}$.
% \end{lemma}
% \begin{proof}[Sketch]
% 	By cancelling out $x$, we see that we are trying to solve the system of equations
% 	\[\begin{cases}
% 		\alpha u^2-\beta v^2=-a_1+a_2, \\
% 		\alpha u^2-\gamma v^2=-a_1+a_3.
% 	\end{cases}\]
% 	This cuts out a curve in $\PP^3$, and it turns out that this is precisely the $E$-torsor corresponding to the image of $(\alpha,\beta,\gamma)$ in $\mathrm H^1(K;E)[2]$. Having a rational point is equivalent to the torsor being trivial, so we are done.
% \end{proof}
% In our situation, we now want to compute the image of
% \[E(K_v)/2E(K_v)\to\mathrm H^1(K_v;E[2]).\]
% The image is supposed to be $2$-dimensional because it should be a maximal isotropic subspace. Certainly it is at most $2$-dimensional because it is isotropic.
% \begin{example} \label{ex:delta-v-image-supp-d}
% 	In our application to $E_d$ for $d$ odd, we can check this by hand by examining the image of torsion. Namely, the image of $E_d[2]$ maps to the triples given by
% 	\[\{(1,1,1),(-1,-d,d),(d,2,2d),(-d,-2d,2)\},\]
% 	and these are distinct elements at places $v$ dividing $2d$.
% \end{example}
% We now turn to the end of the calculation. We are looking at a triple $(\alpha,\beta,\gamma)\in\mathrm H^1(K;E[2])$, and we want to check if it satisfies some local conditions as follows.
% \begin{itemize}
% 	\item At $v\nmid 2d\infty$, then we are asking for $(\alpha,\beta,\gamma)$ to live in the image of $\mathrm H^1_{\mathrm{ur}}(K_v;E[2])$. This group is being identified with $\OO_v^\times/\OO_v^{\times2}$, so this amounts to requiring that $\alpha$, $\beta$, and $\gamma$ have prime factorizations supported at primes dividing $2d$.
% 	\item At $v\mid d\infty$, then we take the image of $\delta_v$ as computed from \Cref{ex:delta-v-image-supp-d}.
% \end{itemize}
% Now, it turns out that we can check that a triple $(\alpha,\beta,\gamma)$ has adelic solutions if and only if it has adelic solutions over just $\AA_\QQ^{(2)}$, so we don't have to give a local condition at $2$.
\begin{corollary} \label{cor:selmer-trivial-bound}
	Fix an odd positive squarefree integer $d$, and let $E_d$ be the projective closure of $y^2=x(x-d)(x+d)$. Let $\nu(d)$ be the number of positive integers of $d$. Then
	\[S_2(E_d)\le2\nu(d).\]
\end{corollary}
\begin{proof}
	The proof of \Cref{thm:cong-selmer} shows that
	\[\frac{\op{Sel}_2(E_d/\QQ)}{E[2]}\subseteq\left\{(\alpha,\beta,\gamma)\in\ZZ_{>0}^3:\alpha,\beta,\gamma\mid d,\alpha\beta\gamma\text{ is square}\right\}.\]
	The right-hand side has a basis over $\FF_2$ given by $(p,q,pq)$ where $p$ and $q$ are primes dividing $d$, so this space has dimension $2\nu(n)$.
\end{proof}
% For example, if $p$ is prime, then we find that we are looking at the triples generated by
% \[(\alpha,\beta)\in\{(1,1),(1,p),(p,1),(p,p)\}\]
% satisfy all the local conditions at $p$ are not. One can run all the calculations to prove \Cref{thm:cong-selmer}.


\end{document}