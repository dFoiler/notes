% !TEX root = ../notes.tex

\documentclass[../notes.tex]{subfiles}

\begin{document}

\section{September 23}
Today, we compute some Selmer groups of the congruent number of elliptic curve.

\subsection{Selmer Groups of Elliptic Curves}
Even though we are not going to use many of the results in this subsection in the future, it is useful to give some general facts and conjectures in order to build intuition about Selmer groups of elliptic curves, following \cite{poonen-rains-selmer}.

Fix an elliptic curve $E$ over a field $K$. For any positive integer $m$, recall that there is a Weil pairing
\[W\colon E[m]\times E[m]\to\mu_m\]
which is Galois-invariant, alternating, and perfect.
\begin{proposition}
	Fix a local field $K_v$ and a prime $p$. Then the Weil pairing induces a cup-product pairing
	\[\mathrm H^1(K_v;E[p])\times\mathrm H^1(K_v;E[p])\stackrel\cup\to\ZZ/p\ZZ\]
	which is perfect and symmetric.
\end{proposition}
% \begin{notation}
% 	Fix a number field $K$. Then we set
% 	\[\mathrm H^1(\AA_K;E[m])\coloneqq\prod_v\left(\mathrm H^1(K_v;E[m]),\mathrm H^1_{\mathrm{ur}}(K_v;E[m])\right).\]
% \end{notation}
\begin{remark}
	If $p$ is a prime, then it turns out that the map $\mathrm H^1(K;E[p])\to\mathrm H^1(\AA_K;E[p])$ is injective when $p$ is prime. However, this is very special: this fails when replacing $p$ by something which is not a prime or when replacing $E$ by a higher-dimensional abelian variety.
\end{remark}
Let's apply the Weil pairing to our Selmer groups.
\begin{remark} \label{rem:e-on-adele}
	For any local field $K_v$, we note that $E(K_v)=E(\OO_v)$ if $E$ has good reduction at $v$. At a high level, this follows from the valuative criterion of properness or the theory of N\'eron models. More directly, one can see that a point $[X:Y:Z]\in\PP^2(K_v)$ satisfying the equation defining $E$ may have its coordinates adjusted until all coordinates are in $\OO_v$ by homogeneity.
\end{remark}
\begin{theorem}
	Fix an elliptic curve $E$ over a number field $K$, and choose a positive integer $m$. Then $\op{Sel}_m(E/K)$ sits in the following pullback square.
	% https://q.uiver.app/#q=WzAsNCxbMCwwLCJcXG9we1NlbH1fbShFL0spIl0sWzEsMCwiXFxtYXRocm0gSF4xKEs7RVttXSkiXSxbMSwxLCJcXG1hdGhybSBIXjEoXFxBQV9LO0VbbV0pIl0sWzAsMSwiRShcXEFBX0spL3BFKFxcQUFfSykiXSxbMCwzXSxbMywyXSxbMCwxXSxbMSwyXSxbMCwyLCIiLDEseyJzdHlsZSI6eyJuYW1lIjoiY29ybmVyIn19XV0=&macro_url=https%3A%2F%2Fraw.githubusercontent.com%2FdFoiler%2Fnotes%2Fmaster%2Fnir.tex
	\[\begin{tikzcd}[cramped]
		{\op{Sel}_m(E/K)} & {\mathrm H^1(K;E[m])} \\
		{E(\AA_K)/pE(\AA_K)} & {\mathrm H^1(\AA_K;E[m])}
		\arrow[from=1-1, to=1-2]
		\arrow[from=1-1, to=2-1]
		\arrow["\lrcorner"{anchor=center, pos=0.125}, draw=none, from=1-1, to=2-2]
		\arrow[from=1-2, to=2-2]
		\arrow[hook, from=2-1, to=2-2]
	\end{tikzcd}\]
	\begin{listalph}
		\item If $m$ is prime, then the right vertical arrow is injective.
		\item The images of the bottom and right arrows are maximal isotropic subspaces with respect to the pairing induced by the Weil pairing.
	\end{listalph}
\end{theorem}
\begin{proof}
	The pullback square is exactly the one in the definition of the Selmer group by \Cref{rem:get-adelic-cohom,rem:e-on-adele}. The rest of the statement is \cite[Theorem~4.14]{poonen-rains-selmer}. In particular, see \cite[Example~4.18]{poonen-rains-selmer}.
\end{proof}
\begin{remark}
	The moral is that we may view $\op{Sel}_p(E/K)$ may be viewed as an intersection of two maximal isotropic subspaces. Such a ``random'' intersection is expected to be rather transverse, which perhaps explains why $\op{Sel}_p(E/K)$ is finite-dimensional.
\end{remark}
\begin{remark}
	Part (a) is a rather sensitive result because it depends on a certain vanishing of $\Sha$ result \cite[Proposition~3.3(e)]{poonen-rains-selmer}. For example, it is not expected to be true if $E$ is replaced by a different module or if $m$ is no longer prime. For example, on the homework, you may show that the map
	\[\mathrm H^1\big(\QQ(\sqrt 7);\mu_8\big)\to\mathrm H^1\big(\AA_{\QQ(\sqrt 7)};\mu_8\big)\]
	fails to be injective.
\end{remark}
% Now, $\op{Sel}_m(E)$ still sits in the pullback square as follows.
% Here, $E(\AA_K)/pE(\AA_K)=\prod_vE(K_v)/pE(K_v)$ because $E(K_v)/pE(K_v)=E(\OO_v)/pE(\OO_v)$ whenever $E$ has good reduction at $v$. (This follows from the theory of N\'eron models. More directly, it follows from smoothness.)
% E(Kv) = E(Ov) by the Neron model
% \begin{remark}
% 	It also turns out that the image of $E(\AA_K)/pE(\AA_K)$ in $\mathrm H^1(\AA_K;E[p])$ is another maximal isotropic subspace.
% \end{remark}
% \begin{remark}
% 	One expects $\prod_v\mc L_v$ and $\mathrm H^1(K;M)$ to be very large, but they tend to be rather transverse in $\prod_v\mathrm H^1(K_v;M)$. For example, in the elliptic curve case, the Weil pairing makes $\prod_v\mathrm H^1(K_v;E[m])$ into a quadratic space, and it turns out that $\prod_v\mc L_v$ is isotropic (by \Cref{thm:tate}), and the image of $\mathrm H^1(K;E[m])$ is isotropic by some global duality.
% \end{remark}
% \begin{remark}
% 	While the map $\prod_v\mc L_v\to\prod_v\mathrm H^1(K_v;M)$ is certainly injective, the vertical map is not always expected to be. In short, it is injective in the case $E[p]$ when $p$ is prime by combining the Chebotarev density theorem and the fact that $p$-Sylow subgroups of $\op{GL}(E[p])$ are cyclic \cite[Theorem~4.16]{poonen-rains-selmer}. However, it will frequently fail to be injective outside this case due to $\Sha$ problems.
% \end{remark}
% The moral, is that we can view $\op{Sel}_p(E/K)$ as the intersection of two maximal isotropic subspaces.

\subsection{Conjectures on the Selmer Group}
While we're here, we acknowledge that now is as good as time as any to recall/give the definition of the Tate--Shafarevich group.
\begin{definition}[Tate--Shafarevich group]
	Fix a number field $K$ and a discrete Galois module $M$. Then we define the \textit{Tate--Shafarevich group} $\Sha(M/K)$ as
	\[\Sha(M/K)\coloneqq\ker\Bigg(\mathrm H^1(K;M)\to\prod_v\mathrm H^1(K_v;M)\Bigg).\]
\end{definition}
\begin{lemma} \label{lem:selmer-ses}
	Fix an elliptic curve $E$ over a number field $K$. For each positive integer $m$, there is an exact sequence
	\[0\to E(K)/mE(K)\to\op{Sel}_m(E/K)\to\Sha(E/K)[m]\to0.\]
\end{lemma}
\begin{proof}
	Functoriality of evaluating $E$ on a field yields a morphism
	% https://q.uiver.app/#q=WzAsMTAsWzAsMCwiMCJdLFsxLDAsIkVbbV0oXFxvdiBLKSJdLFsyLDAsIkUoXFxvdiBLKSJdLFszLDAsIkUoXFxvdiBLKSJdLFs0LDAsIjAiXSxbMCwxLCIwIl0sWzEsMSwiXFxkaXNwbGF5c3R5bGVcXHByb2RfdkVbbV0oXFxvdntLX3Z9KSJdLFsyLDEsIlxcZGlzcGxheXN0eWxlXFxwcm9kX3ZFKFxcb3Z7S192fSkiXSxbMywxLCJcXGRpc3BsYXlzdHlsZVxccHJvZF92RShcXG92e0tfdn0pIl0sWzQsMSwiMCJdLFswLDFdLFsxLDJdLFsyLDMsIm0iXSxbMyw0XSxbNSw2XSxbNiw3XSxbNyw4LCJtIl0sWzgsOV0sWzEsNiwiIiwxLHsic3R5bGUiOnsidGFpbCI6eyJuYW1lIjoiaG9vayIsInNpZGUiOiJ0b3AifX19XSxbMiw3LCIiLDEseyJzdHlsZSI6eyJ0YWlsIjp7Im5hbWUiOiJob29rIiwic2lkZSI6InRvcCJ9fX1dLFszLDgsIiIsMSx7InN0eWxlIjp7InRhaWwiOnsibmFtZSI6Imhvb2siLCJzaWRlIjoidG9wIn19fV1d&macro_url=https%3A%2F%2Fraw.githubusercontent.com%2FdFoiler%2Fnotes%2Fmaster%2Fnir.tex
	\[\begin{tikzcd}[cramped]
		0 & {E[m](\ov K)} & {E(\ov K)} & {E(\ov K)} & 0 \\
		0 & {\displaystyle\prod_vE[m](\ov{K_v})} & {\displaystyle\prod_vE(\ov{K_v})} & {\displaystyle\prod_vE(\ov{K_v})} & 0
		\arrow[from=1-1, to=1-2]
		\arrow[from=1-2, to=1-3]
		\arrow[hook, from=1-2, to=2-2]
		\arrow["m", from=1-3, to=1-4]
		\arrow[hook, from=1-3, to=2-3]
		\arrow[from=1-4, to=1-5]
		\arrow[hook, from=1-4, to=2-4]
		\arrow[from=2-1, to=2-2]
		\arrow[from=2-2, to=2-3]
		\arrow["m", from=2-3, to=2-4]
		\arrow[from=2-4, to=2-5]
	\end{tikzcd}\]
	of short exact sequences, where everything in sight is a continuous Galois module. Taking Galois cohomology thus produces another morphism
	% https://q.uiver.app/#q=WzAsMTAsWzAsMCwiMCJdLFsxLDAsIkUoSykvbUUoSykiXSxbMCwxLCIwIl0sWzEsMSwiXFxkaXNwbGF5c3R5bGVcXHByb2RfdkUoS192KS9tRShLX3YpIl0sWzIsMCwiXFxtYXRocm0gSF4xKEs7RVttXSkiXSxbMywwLCJcXG1hdGhybSBIXjEoSztFKVttXSJdLFsyLDEsIlxcZGlzcGxheXN0eWxlXFxwcm9kX3ZcXG1hdGhybSBIXjEoS192O0VbbV0pIl0sWzQsMCwiMCJdLFs0LDEsIjAiXSxbMywxLCJcXGRpc3BsYXlzdHlsZVxccHJvZF92XFxtYXRocm0gSF4xKEtfdjtFKVttXSJdLFswLDFdLFsxLDRdLFs0LDVdLFs1LDddLFsyLDNdLFszLDZdLFs2LDldLFs5LDhdLFsxLDNdLFs0LDZdLFs1LDldXQ==&macro_url=https%3A%2F%2Fraw.githubusercontent.com%2FdFoiler%2Fnotes%2Fmaster%2Fnir.tex
	\[\begin{tikzcd}[cramped]
		0 & {E(K)/mE(K)} & {\mathrm H^1(K;E[m])} & {\mathrm H^1(K;E)[m]} & 0 \\
		0 & {\displaystyle\prod_vE(K_v)/mE(K_v)} & {\displaystyle\prod_v\mathrm H^1(K_v;E[m])} & {\displaystyle\prod_v\mathrm H^1(K_v;E)[m]} & 0
		\arrow[from=1-1, to=1-2]
		\arrow[from=1-2, to=1-3]
		\arrow[from=1-2, to=2-2]
		\arrow[from=1-3, to=1-4]
		\arrow[from=1-3, to=2-3]
		\arrow[from=1-4, to=1-5]
		\arrow[from=1-4, to=2-4]
		\arrow[from=2-1, to=2-2]
		\arrow[from=2-2, to=2-3]
		\arrow[from=2-3, to=2-4]
		\arrow[from=2-4, to=2-5]
	\end{tikzcd}\]
	of short exact sequences. Now, the kernel of the rightmost vertical arrow is $\Sha(E/K)[m]$ by definition of $\Sha(E/K)$. Accordingly, we claim that we may take a pullback of the top short exact sequence to produce yet another morphism
	% https://q.uiver.app/#q=WzAsMTAsWzAsMSwiMCJdLFsxLDEsIkUoSykvbUUoSykiXSxbMiwxLCJcXG1hdGhybSBIXjEoSztFW21dKSJdLFszLDEsIlxcbWF0aHJtIEheMShLO0UpW21dIl0sWzQsMSwiMCJdLFszLDAsIlxcU2hhKEUvSylbbV0iXSxbMiwwLCJcXG9we1NlbH1fbShFL0spIl0sWzEsMCwiRShLKS9tRShLKSJdLFswLDAsIjAiXSxbNCwwLCIwIl0sWzAsMV0sWzEsMl0sWzIsM10sWzMsNF0sWzgsN10sWzcsNl0sWzYsNV0sWzUsOV0sWzcsMSwiIiwxLHsibGV2ZWwiOjIsInN0eWxlIjp7ImhlYWQiOnsibmFtZSI6Im5vbmUifX19XSxbNiwyLCIiLDEseyJzdHlsZSI6eyJ0YWlsIjp7Im5hbWUiOiJob29rIiwic2lkZSI6InRvcCJ9fX1dLFs1LDMsIiIsMSx7InN0eWxlIjp7InRhaWwiOnsibmFtZSI6Imhvb2siLCJzaWRlIjoidG9wIn19fV0sWzYsMywiIiwxLHsic3R5bGUiOnsibmFtZSI6ImNvcm5lciJ9fV1d&macro_url=https%3A%2F%2Fraw.githubusercontent.com%2FdFoiler%2Fnotes%2Fmaster%2Fnir.tex
	\[\begin{tikzcd}[cramped]
		0 & {E(K)/mE(K)} & {\op{Sel}_m(E/K)} & {\Sha(E/K)[m]} & 0 \\
		0 & {E(K)/mE(K)} & {\mathrm H^1(K;E[m])} & {\mathrm H^1(K;E)[m]} & 0
		\arrow[from=1-1, to=1-2]
		\arrow[from=1-2, to=1-3]
		\arrow[equals, from=1-2, to=2-2]
		\arrow[from=1-3, to=1-4]
		\arrow[hook, from=1-3, to=2-3]
		\arrow["\lrcorner"{anchor=center, pos=0.125}, draw=none, from=1-3, to=2-4]
		\arrow[from=1-4, to=1-5]
		\arrow[hook, from=1-4, to=2-4]
		\arrow[from=2-1, to=2-2]
		\arrow[from=2-2, to=2-3]
		\arrow[from=2-3, to=2-4]
		\arrow[from=2-4, to=2-5]
	\end{tikzcd}\]
	of short exact sequences. Here, the middle term of the top short exact sequence is in fact $\op{Sel}_m(E/K)$: this fiber product should consist of the elements of $\mathrm H^1(K;E[m])$ which vanish in $\prod_v\mathrm H^1(K_v;E)$, which by exactness is equivalent to their image along $\mathrm H^1(K;E[m])\to\prod_v\mathrm H^1(K_v;E[m])$ coming from $\prod_vE(K_v)/mE(K_v)$.
	
	It is now totally formal that the top row is exact: exactness on the right follows because the pullback of an epimorphism is an epimorphism. Further, exactness elsewhere amounts to saying that $E(K)/mE(K)$ is the kernel of $\op{Sel}_m(E/K)\to\Sha(E/K)[m]$, which follows because pullbacks commute with kernels (recall limits commute with limits).
\end{proof}
% It turns out that $\op{Sel}_m(E/K)$ fits into an exact sequence of the form
% \[0\to E(K)/mE(K)\to\mathrm{Sel}_m(E/K)\to\Sha(E/K)[m]\to0.\]
Thus, we see that $\op{Sel}_m(E/K)$ contains contributions from three interesting invariants of $E$: the $m$-torsion $E[m]$, the algebraic rank $\op{rank}_\ZZ E(K)$, and $\Sha(E/K)$. Of course, the $m$-torsion is the least interesting, so we introduce some notation to get rid of it.
\begin{notation}
	Fix an elliptic curve $E$ over a number field $K$. For each prime $p$, we define
	\[S_p(E/K)\coloneqq\dim_{\FF_p}\op{Sel}_p(E/K)-\dim_{\FF_p}E(K)[p].\]
\end{notation}
\begin{remark} \label{rem:reduced-selmer}
	Let $r$ be the algebraic rank of $E$ over $K$ so that $E(K)\cong E(K)_{\mathrm{tors}}\oplus\ZZ^{\oplus r}$. Thus, for any prime $p$,
	\[\frac{E(K)}{pE(K)}\cong\frac{E(K)_{\mathrm{tors}}}{pE(K)_{\mathrm{tors}}}\oplus\left(\frac\ZZ{p\ZZ}\right)^{\oplus r}\]
	Note $E(K)_{\mathrm{tors}}$ is some finite abelian group, so the kernel and cokernel of $p\colon E(K)_{\mathrm{tors}}\to E(K)_{\mathrm{tors}}$ have the same size by an Euler characteristic argument. Thus,
	\[\dim_{\FF_p}E(K)/pE(K)=\dim_{\FF_p}E(K)[p]+\op{rank}_\ZZ E(K).\]
	\Cref{lem:selmer-ses} now implies that $\op{rank}_\ZZ E(K)+\dim_{\FF_p}\Sha(E/K)[p]=S_p(E/K)$.
\end{remark}
Let's make some ``parity conjectures.'' Fix an elliptic curve $E$ over a number field $K$.
\begin{itemize}
	\item Note that $\Sha(E/K)$ is known to have an alternating ``Cassels--Tate'' pairing and is expected to be finite, so its size is conjectured to be a square.
	\item Similarly, $E[m](K)$ has a Weil pairing, which is a perfect alternating pairing on it, so it similarly follows that the size is a square.
\end{itemize}
For example, for taking $m$ to be a prime $p$, this produces the following conjecture via \Cref{rem:reduced-selmer}.
\begin{conj}[Partity for Mordell--Weil rank]
	Fix an elliptic curve $E$ over a number field $K$. Then for each prime $p$,
	\[S_p(E/K)\stackrel?\equiv\op{rank}E(K)\pmod2.\]
\end{conj}
By comparing with the Birch and Swinnerton-Dyer conjecture, we can make a parity conjecture comparing to modular forms.
\begin{conj}[Parity for global root number] \label{conj:root-number}
	Fix an elliptic curve $E$ over a number field $K$ with an attached modular form $f_E$. Then
	\[(-1)^{S_p(E/K)}=\varepsilon(f_E/K),\]
	where $\varepsilon(f_E/K)$ is the sign of the $L$-function's functional equation.
\end{conj}
\begin{remark}
	There is a purely local definition of $\varepsilon(f_E/K)$ which does not require us to know that there is an attached modular form.
\end{remark}
\begin{remark}
	\Cref{conj:root-number} is known if $K=\QQ$ by Nekov\'a\v{r} in many cases and Dokchitser--Dokchitser in the remaining cases. If $E[p](K)$ is nontrivial, it is still known by Dokchitser--Dokchitser again. There are other results by \v{C}esnavi\v{c}ius.
\end{remark}

% \subsection{\texorpdfstring{$2$}{2}-Descent}
% In this subsection, we explain how to compute $2$-Selmer groups of elliptic curves $E$ over a number field $K$ for which $E[2](K)=E[2](\ov K)$.

% To begin, suppose that $K$ is an arbitrary field of characteristic $0$, to be set to be a number field shortly. Writing $E$ into Weierstrass form $y^2=f(x)$ for a cubic $x$, one sees that the roots of $f$ produce the nontrivial $2$-torsion points of $f$. (This follows from the usual group law of $E$.) Thus, $f$ is required to fully factor over $K$, allowing us to write $E$ as the projective closure of the affine curve cut out by
% \[y^2=(x-a_1)(x-a_2)(x-a_3)\]
% for some $a_1,a_2,a_3\in K$. In this situation, we see that\todo{}
% \[E[2]=\{\infty,(a_1,0),(a_2,0),(a_3,0)\}.\]
% Now, $E[2]$ has trivial Galois action, so we may identify it with the isomorphic Galois module $\mu_2^{\oplus2}$. For symmetry reasons, it will in fact be easier to identify it with the ``trace zero'' hyperplane $H$ of $\mu_2^{\oplus3}$: namely, we embed $E[2]$ into $\mu_2^{\oplus3}$ by
% \[\begin{cases}
% 	\infty \mapsto (+1,+1,+1), \\
% 	(a_1,0) \mapsto (+1,-1,-1), \\
% 	(a_2,0) \mapsto (-1,+1,-1), \\
% 	(a_3,0) \mapsto (-1,-1,+1).
% \end{cases}\]
% Namely, the image of this embedding is $H=\{(\varepsilon_1,\varepsilon_2,\varepsilon_3)\in\mu_2^{\oplus3}:\varepsilon_1\varepsilon_2\varepsilon_3=1\}$, which is the kernel of the product map $H\to\mu_2$. Thus, we may identify $\mathrm H^1(K;E[2])=\mathrm H^1(K;H)$, which tracking through the functoriality of \Cref{ex:h1-of-mu-n} gives
% \[\mathrm H^1(K;H)\cong\left\{(\alpha,\beta,\gamma):K^\times/K^{\times2}:\alpha\beta\gamma\in K^{\times2}\right\}.\]
% In order to compute the $2$-Selmer group, we need to understand the image of the map $E(K)/2E(K)\to\mathrm H^1(K;E[2])=\mathrm H^1(K;H)$.
% \begin{proposition}
% 	Fix an elliptic curve $E$ over a field $K$ which is the projective closure of $y^2=(x-a_1)(x-a_2)(x-a_3)$. Identifying $E[2]$ with the trace-zero hyperplane $H\subseteq\mu_2^{\oplus3}$, the map $\delta\colon E(K)/2E(K)\to\mathrm H^1(K;H)$ is the map
% 	\[\delta\colon\begin{cases}
% 		(x,y) \mapsto (x-a_1,x-a_2,x-a_3) & \text{if }y\ne0, \\
% 		\infty \mapsto (1,1,1), \\
% 		(a_1,0) \mapsto ((a_1-a_2)(a_1-a_3),a_1-a_2,a_1-a_3), \\
% 		(a_2,0) \mapsto (a_2-a_1,(a_2-a_1)(a_2-a_3),a_2-a_3), \\
% 		(a_3,0) \mapsto (a_3-a_1,a_3-a_2,(a_3-a_1)(a_3-a_2)).
% 	\end{cases}\]
% \end{proposition}
% \begin{proof}
% 	This follows from \cite[Theorem~X.1.1]{silverman} and the discussion after it.
% \end{proof}
% \begin{corollary}
% 	Fix an elliptic curve $E$ over a field $K$ which is the projective closure of $y^2=(x-a_1)(x-a_2)(x-a_3)$. Identifying $E[2]$ with the trace-zero hyperplane $H\subseteq\mu_2^{\oplus3}$, a triple $(\alpha,\beta,\gamma)\in\mathrm H^1(K;H)$ is in the image of the boundary map from $E(K)/2E(K)$ if and only if the conic $T_{(\alpha,\beta,\gamma)}\subseteq\PP^4$ cut out by the affine equations
% 	\[\begin{cases}
% 		\alpha u^2=x-a_1, \\
% 		\beta v^2=x-a_2, \\
% 		\gamma w^2=x-a_3
% 	\end{cases}\]
% 	admits a solution.
% \end{corollary}
% \begin{proof}
% 	In projective coordinates $[U:V:W:X:Y]$, the equations are
% 	\[\begin{cases}
% 		\alpha U^2=(X-a_1Y)Y, \\
% 		\beta V^2=(X-a_2Y)Y, \\
% 		\gamma W^2=(X-a_3Y)Y.
% 	\end{cases}\]
% 	The points at infinity occur with $Y=0$
% 	% which makes it apparent that there are no points at infinity live at $T=0$, and the remaining points are on the affine chart $T=1$.
% \end{proof}
% \begin{remark}
% 	For our application to computing Selmer groups, it is enough to merely understand the image of this map $\delta$. The next term 
% \end{remark}

\subsection{Congruent Number Elliptic Curves}
We now return to the congruent number elliptic curves $E_d\colon y^2=x(x-d)(x+d)$, where $d\in\ZZ$ is some squarefree positive integer. It turns out that $E_d$ is a quadratic twist of $E_1\colon y^2=x^3-x$, and these elliptic curves have complex multiplication by $\ZZ[i]$. Importantly, the $2$-torsion
\[E[2]=\{\infty,(0,0),(+d,0),(-d,0)\}\]
is fully defined over $\QQ$. Here is a bit more about what is known.
\begin{remark}[Birch--Stephens]
	Fix a squarefree positive integer $d$. It is known that
	\[\varepsilon(E_d/\QQ)=\begin{cases}
		+1 & \text{if }d\equiv 1,2,3\pmod8, \\
		-1 & \text{if }d\equiv 5,6,7\pmod8.
	\end{cases}\]
	Furthermore, they computed
	\[S_2(E_d/\QQ)\equiv\begin{cases}
		0\pmod2 & \text{if }d\equiv1,2,3\pmod8, \\
		1\pmod2 & \text{if }d\equiv5,6,7\pmod8.
	\end{cases}\]
\end{remark}
They proved this using calculations of Selmer groups. We will show the following.
\begin{theorem} \label{thm:cong-selmer}
	Suppose that $d$ is an odd prime $p$. Then
	\[S_2(E_p/\QQ)=\begin{cases}
		2 & \text{if }p\equiv1\pmod8, \\
		0 & \text{if }p\equiv2,3\pmod8, \\
		1 & \text{if }p\equiv5,7\pmod8.
	\end{cases}\]
\end{theorem}
\begin{remark}
	In the first case $p\equiv1\pmod8$, it is possible to get both $0$ and $2$ for the Mordell--Weil rank. Indeed, for many small primes $p$, $\op{rank}E_p(\QQ)=0$, but $\op{rank}E_{41}(\QQ)=2$.
\end{remark}
\begin{remark}
	It has been verified by Heegner--Monsky that $p\equiv5,7\pmod8$ implies that $\op{rank}E_p(\QQ)=1$. This requires the construction of non-torsion points, which uses Heegner points.
\end{remark}
We are going to do $2$-descent for elliptic curve $E$ over $K$ with $E[2](\ov K)=E[2](K)$. In this case, we may write
\[E\colon y^2=(x-a_1)(x-a_2)(x-a_3).\]
Because $E[2]$ is defined over $K$, the Galois action is trivial. Thus, we see $\mathrm H^1(K;E[2])$ is simply $\mathrm H^1(K;\mu_2)^2$, and we can compute this cohomology using the ``Kummer'' exact sequence
\[1\to\mu_2\to\ov K^\times\stackrel2\to\ov K^\times\to1,\]
which in Galois cohomology produces an isomorphism $\mathrm H^1(K;\mu_2)\cong K^\times/K^{\times2}$ by Hilbert's theorem 90. We may now identify $\mathrm H^1(K;E[2])$ with
\[\left\{(\alpha,\beta,\gamma)\in(K^\times/K^{\times2})^3:\alpha\beta\gamma=1\right\}.\]
It turns out that this map (approximately) sends some point $(x,y)$ of $E(K)/2E(K)$ to the triple $(x-a_1,x-a_2,x-a_3)$ when $E$ has the form $y^2=(x-a_1)(x-a_2)(x-a_3)$. Technically speaking, we should note that we send $\infty$ to the identity $(1,1,1)$, and we send any two-torsion point like $(a_1,0)$ to the triple whose last two coordinates are $a_1-a_2$ and $a_3-a_2$.

Of course, we would like a way to know if an interesting triple $(\alpha,\beta,\gamma)$ is in the image without having to find points in $E(K)$ first. Here is one such test.
\begin{lemma}
	Fix an elliptic curve $E\colon y^2=(x-a_1)(x-a_2)(x-a_3)$ over a finite extension $K$ of $\QQ_p$. Then a triple $(\alpha,\beta,\gamma)$ lies in the image of the above map if and only if the system of equations
	\[\begin{cases}
		\alpha u^2=x-a_1, \\
		\beta v^2=x-a_2, \\
		\gamma w^2=x-a_3
	\end{cases}\]
	admits a solution. We will call this conic $T_{(\alpha,\beta,\gamma)}$.
\end{lemma}
\begin{proof}[Sketch]
	By cancelling out $x$, we see that we are trying to solve the system of equations
	\[\begin{cases}
		\alpha u^2-\beta v^2=-a_1+a_2, \\
		\alpha u^2-\gamma v^2=-a_1+a_3.
	\end{cases}\]
	This cuts out a curve in $\PP^3$, and it turns out that this is precisely the $E$-torsor corresponding to the image of $(\alpha,\beta,\gamma)$ in $\mathrm H^1(K;E)[2]$. Having a rational point is equivalent to the torsor being trivial, so we are done.
\end{proof}
In our situation, we now want to compute the image of
\[E(K_v)/2E(K_v)\to\mathrm H^1(K_v;E[2]).\]
The image is supposed to be $2$-dimensional because it should be a maximal isotropic subspace. Certainly it is at most $2$-dimensional because it is isotropic.
\begin{example} \label{ex:delta-v-image-supp-d}
	In our application to $E_d$ for $d$ odd, we can check this by hand by examining the image of torsion. Namely, the image of $E_d[2]$ maps to the triples given by
	\[\{(1,1,1),(-1,-d,d),(d,2,2d),(-d,-2d,2)\},\]
	and these are distinct elements at places $v$ dividing $2d$.
\end{example}
We now turn to the end of the calculation. We are looking at a triple $(\alpha,\beta,\gamma)\in\mathrm H^1(K;E[2])$, and we want to check if it satisfies some local conditions as follows.
\begin{itemize}
	\item At $v\nmid 2d\infty$, then we are asking for $(\alpha,\beta,\gamma)$ to live in the image of $\mathrm H^1_{\mathrm{ur}}(K_v;E[2])$. This group is being identified with $\OO_v^\times/\OO_v^{\times2}$, so this amounts to requiring that $\alpha$, $\beta$, and $\gamma$ have prime factorizations supported at primes dividing $2d$.
	\item At $v\mid d\infty$, then we take the image of $\delta_v$ as computed from \Cref{ex:delta-v-image-supp-d}.
\end{itemize}
Now, it turns out that we can check that a triple $(\alpha,\beta,\gamma)$ has adelic solutions if and only if it has adelic solutions over just $\AA_\QQ^{(2)}$, so we don't have to give a local condition at $2$.
\begin{corollary} \label{cor:selmer-trivial-bound}
	Fix $d$ odd, and let $n$ be the number of prime factors of $d$. Then
	\[S_2(E_d)\le2\log_2n.\]
\end{corollary}
\begin{proof}
	Up to $E_d[2]$, we may assume that our triples $(\alpha,\beta,\gamma)$ have $\alpha$ and $\beta$ are positive factors of $d$ and $\alpha\beta\gamma=1$. These two degrees of freedom produce the dimension bound.
\end{proof}
For example, if $p$ is prime, then we find that we are looking at the triples generated by
\[(\alpha,\beta)\in\{(1,1),(1,p),(p,1),(p,p)\}\]
satisfy all the local conditions at $p$ are not. One can run all the calculations to prove \Cref{thm:cong-selmer}.


\end{document}