% !TEX root = ../notes.tex

\documentclass[../notes.tex]{subfiles}

\begin{document}

\section{September 18}
Today, we will continue to review Galois cohomology.

\subsection{Local Duality}
Akin to \Cref{prop:procyclic-cohom}, we have the following duality statement for local fields.
\begin{theorem}[Tate] \label{thm:tate}
	Fix a finite extension $K$ of $\QQ_p$, set $G\coloneqq\op{Gal}(\ov K/K)$ for brevity, and let $M$ be a finite discrete $G$-module.
	\begin{listalph}
		\item Finiteness: the modules $\mathrm H^i(K;M)$ are finite for all $i$ and vanishes for $i\ge3$.
		\item Duality: for a $G$-module $M$, we define the $G$-module $M^*\coloneqq\op{Hom}_\ZZ(M,\mu_\infty(\ov K))$. Then there is a perfect pairing
		\[\mathrm H^i(K;M)\times\mathrm H^{2-i}(K;M^*)\to\QQ/\ZZ.\]
		\item Euler characteristic formula: one has
		\[\frac{\#\mathrm H^0(K;M)\cdot\#\mathrm H^2(K;M)}{\#\mathrm H^1(K;M)}=\frac1{\#(\OO_K/(\#M)\OO_K)}.\]
	\end{listalph}
\end{theorem}
\begin{remark}
	One can define the pairing via a cup product
	\[\cup\colon\mathrm H^i(K;M)\times\mathrm H^{2-i}(K;M^*)\to\mathrm H^2(K;\mu_\infty),\]
	and it turns out that the target is isomorphic to $\QQ/\ZZ$ (via the ``local invariant'' map of local class field theory).
\end{remark}
\begin{remark}
	One calls (c) an Euler characteristic formula because the invariant
	\[\chi(M)\coloneqq\frac{\#\mathrm H^0(K;M)\cdot\#\mathrm H^2(K;M)}{\#\mathrm H^1(K;M)}\]
	behaves like an Euler characteristic. Indeed, it is like an alternating sum of cohomology groups.
\end{remark}
\begin{remark}
	It is possible to check \Cref{thm:tate} explicitly for $M\in\{\ZZ/m\ZZ,\mu_m\}$.
\end{remark}
In order to relate local fields with finite fields, we should explain how one can recover an unramified cohomology.
\begin{definition}[inertia group]
	Fix a local field $K$ with finite residue field $k$. Then the Galois action on $K$ preserves the absolute value and therefore descends to $\OO_K/\mf p_K=k$. We define the \textit{inertia subgroup} $I_K$ of $\op{Gal}(\ov K/K)$ to fit in the short exact sequence
	\[1\to I_K\subseteq\op{Gal}(\ov K/K)\to\op{Gal}(\ov k/k)\to1.\]
\end{definition}
\begin{remark}
	Let $K^{\mathrm{ur}}$ be the maximal unramified extension of $K$. Then we see that $\op{Gal}(K^{\mathrm{ur}}/K)$ is simply $\op{Gal}(\ov K/K)/I_K$.
\end{remark}
\begin{definition}[unramified]
	Fix a local field $K$. Then a $\op{Gal}(\ov K/K)$-module $M$ is \textit{unramified} if and only if $I_K$ acts trivially on $M$. In this case, we define the \textit{unramified cohomology} $\mathrm H^i_{\mathrm{ur}}(K;M)$ as the image of
	\[\op{Inf}\colon\mathrm H^i(\op{Gal}(K^{\mathrm{ur}}/K);M)\to\mathrm H^i(\op{Gal}(\ov K/K);M).\]
\end{definition}
\begin{remark}
	By \Cref{prop:procyclic-cohom}, we see that only the unramified cohomology which has a chance of being nonzero is indices $0$ and $1$.
\end{remark}
We are now able to relate our two dualities.
\begin{theorem}
	Fix a finite extension $K$ of $\QQ_p$. Let $M$ be a discrete Galois module, and suppose further that $M$ is unramified and that $\#M$ is coprime to $p$. Then $M^*$ is still unramified, and under the duality pairing
	\[\mathrm H^i(K;M)\times\mathrm H^{2-i}(K;M^*)\to\QQ/\ZZ,\]
	the two subgroups $\mathrm H^1_{\mathrm{ur}}(K;M)$ and $\mathrm H^1_{\mathrm{ur}}(K;M^*)$ are annihilators of each other.
\end{theorem}
\begin{proof}
	One can check directly that $\mathrm H^1_{\mathrm{ur}}(K;M)$ and $\mathrm H^1_{\mathrm{ur}}(K;M^*)$ annihilate each other because the cup product lands in $\mathrm H^2_{\mathrm{ur}}(K;\ZZ/(\#M)\ZZ)$, which automatically vanishes by \Cref{prop:procyclic-cohom}. Because we have a perfect pairing, it now remains to show that these two groups have the same size.
	
	By \Cref{prop:procyclic-cohom}, we see that $\mathrm H^1_{\mathrm{ur}}(K;M)$ is
	\[\mathrm H^1\left(M\stackrel{\sigma-1}\to M\right)=\coker\left(M\stackrel{\sigma-1}\to M\right).\]
	But because $M$ is finite, we see that the size of this cokernel equals the size of this kernel, so we conclude that $\#\mathrm H^1_{\mathrm{ur}}(K;M)=\#\mathrm H^0_{\mathrm{ur}}(K;M)$, but this is just $\#\mathrm H^0(K;M)$ because $M$ is unramified. One similarly deduces that $\#\mathrm H^1_{\mathrm{ur}}(K;M^*)=\#\mathrm H^0(K;M^*)$, which is $\#\mathrm H^2(K;M)$ by \Cref{thm:tate}. We now complete the proof with an Euler characteristic calculation because we know $\chi(M)=1$ by \Cref{thm:tate}.
\end{proof}
% \subsection{}
Here is why unramified cohomology will be relevant to our story.
\begin{lemma}
	Fix a finite extension $K$ of $\QQ_p$, and fix an elliptic curve $E$ of good reduction. Further, choose an integer $m$ coprime to $p$. Then the image of the map
	\[0\to E(K)/mE(K)\to\mathrm H^1(K;E[m])\]
	coincides with $\mathrm H^1_{\mathrm{ur}}(K;E[m])$.
\end{lemma}
\begin{remark}
	The map $E(K)/mE(K)\to\mathrm H^1(K;E[m])$ is induced by the ``Kummer'' exact sequence
	\[0\to E[m]\to E\stackrel m\to E\to0.\]
	Indeed, taking Galois cohomology produces an exact sequence
	\[E(K)\stackrel m\to E(K)\to\mathrm H^1(K;E[m]).\]
\end{remark}
The point of this lemma is that we are interested in $E(K)/mE(K)$, which appears to be some difficult invariant including the rank of $E$. However, $E[m]$ is just some explicitly computable torsion, so we find that we are actually able to handle $E(K)/mE(K)$ over local fields! For example, it turns out that $E[m](K)$ descends to the residue field in $E[m](k)$, which is contained in $E(k)$.

\subsection{Selmer Groups}
We are now allowed to make the following global definition.
\begin{definition}
	Fix a number field $K$, and fix a finite discrete Galois module $M$. Furthermore, for each place $v$ of $K$, choose a subset $\mc L_v\subseteq\mathrm H^1(K_v;M)$, and we require that $\mc L_v=\mathrm H^1_{\mathrm{ur}}(K_v;M)$ for all but finitely many $v$. Then we define the \textit{Selmer group} with respect to $\mc L$ to be the pullback in the following square.
	% https://q.uiver.app/#q=WzAsNCxbMCwwLCJcXG9we1NlbH1fe1xcbWMgTH0oTSkiXSxbMSwwLCJcXG1hdGhybSBIXjEoSztNKSJdLFsxLDEsIlxcZGlzcGxheXN0eWxlXFxwcm9kX3ZcXG1hdGhybSBIXjEoS192O00pIl0sWzAsMSwiXFxkaXNwbGF5c3R5bGVcXHByb2RfdlxcbWMgTF92Il0sWzMsMl0sWzAsM10sWzAsMV0sWzEsMl0sWzAsMiwiIiwxLHsic3R5bGUiOnsibmFtZSI6ImNvcm5lciJ9fV1d&macro_url=https%3A%2F%2Fraw.githubusercontent.com%2FdFoiler%2Fnotes%2Fmaster%2Fnir.tex
	\[\begin{tikzcd}[cramped]
		{\op{Sel}_{\mc L}(M)} & {\mathrm H^1(K;M)} \\
		{\displaystyle\prod_v\mc L_v} & {\displaystyle\prod_v\mathrm H^1(K_v;M)}
		\arrow[from=1-1, to=1-2]
		\arrow[from=1-1, to=2-1]
		\arrow["\lrcorner"{anchor=center, pos=0.125}, draw=none, from=1-1, to=2-2]
		\arrow[from=1-2, to=2-2]
		\arrow[from=2-1, to=2-2]
	\end{tikzcd}\]
	The vertical maps are induced by the maps $\op{Gal}(\ov{K_v}/K_v)\to\op{Gal}(\ov K/K)$ given restricting an automorphism.
\end{definition}
\begin{example}
	If $E$ is an elliptic curve over a global field $K$, we can define $M\coloneqq E[m]$ and choose $\mc L_v$ to be the image of the map
	\[0\to E(K_v)/mE(K_v)\to\mathrm H^1(K;E[m])\]
	for each place $v$. We may write $\op{Sel}_m(E)$ for $\op{Sel}_{\mc L}(E[m])$ in this situation.
\end{example}
\begin{remark}
	It is undesirable to require that $\mc L_v=\mathrm H^1_{\mathrm{ur}}(K_v;M)$ for all places of $v$ because we do not expect $M$ to be unramified at all $v$.
\end{remark}
\begin{remark}
	One expects $\prod_v\mc L_v$ and $\mathrm H^1(K;M)$ to be very large, but they tend to be rather transverse. For example, in the elliptic curve case, the Weil pairing makes $\prod_v\mathrm H^1(K_v;E[m])$ into a quadratic space, and it turns out that both of our input spaces are in some sense Lagrangian with respect to this pairing.
\end{remark}
Here is our finiteness result.
\begin{theorem}
	Fix a number field $K$, and fix a finite discrete Galois module $M$. Furthermore, for each place $v$ of $K$, choose a subset $\mc L_v\subseteq\mathrm H^1(K_v;M)$, and we require that $\mc L_v=\mathrm H^1_{\mathrm{ur}}(K_v;M)$ for all but finitely many $v$. Then $\op{Sel}_{\mc L}(M)$ is finite.
\end{theorem}
\begin{proof}
	We proceed in steps. The point is to reduce the case to where $M$ has a trivial Galois action, from which we will be able to apply class field theory.
	\begin{enumerate}
		\item We note that making $\mc L$ larger cannot help us, so we may assume that either $\mc L_v=\mathrm H^1(K_v;M)$ or $\mc L_v=\mathrm H^1_{\mathrm{ur}}(K_v;M)$ for all places $v$, and we let $S$ to be the finite set in which the former occurs. We also may as well enlarge $S$ so that $M$ is unramified outside $S$. From now one, we will abbreviate $\op{Sel}_{\mc L}(M)$ to $\op{Sel}_S(M)$.

		\item We reduce to the case where $M$ has the trivial Galois action. Indeed, because $M$ is finite and discrete, the continuity of the action provides a finite extension $L$ of $K$ for which $\op{Gal}(\ov K/L)$ acts trivially on $M$. We now sit in the following diagram.
		% https://q.uiver.app/#q=WzAsNixbMiwwLCJcXG9we1NlbH1fe1N9KE0pIl0sWzMsMCwiXFxvcHtTZWx9X3tTJ31cXGxlZnQoXFxvcHtSZXN9XntcXG9we0dhbH0oXFxvdiBLL0spfV97XFxvcHtHYWx9KFxcb3YgSy9MKX1NXFxyaWdodCkiXSxbMywxLCJcXG1hdGhybSBIXjEoTDtNKSJdLFsyLDEsIlxcbWF0aHJtIEheMShLO00pIl0sWzEsMSwiXFxtYXRocm0gSF4xKFxcb3B7R2FsfShML0spO00pIl0sWzAsMSwiMCJdLFszLDIsIlxcb3B7UmVzfSJdLFswLDFdLFsxLDJdLFswLDNdLFs0LDMsIlxcb3B7SW5mfSJdLFs1LDRdXQ==&macro_url=https%3A%2F%2Fraw.githubusercontent.com%2FdFoiler%2Fnotes%2Fmaster%2Fnir.tex
		\[\begin{tikzcd}[cramped]
			&& {\op{Sel}_{S}(M)} & {\op{Sel}_{T}\left(\op{Res}^{\op{Gal}(\ov K/K)}_{\op{Gal}(\ov K/L)}M\right)} \\
			0 & {\mathrm H^1(\op{Gal}(L/K);M)} & {\mathrm H^1(K;M)} & {\mathrm H^1(L;M)}
			\arrow[from=1-3, to=1-4]
			\arrow[from=1-3, to=2-3]
			\arrow[from=1-4, to=2-4]
			\arrow[from=2-1, to=2-2]
			\arrow["{\op{Inf}}", from=2-2, to=2-3]
			\arrow["{\op{Res}}", from=2-3, to=2-4]
		\end{tikzcd}\]
		Here, $T$ is the collection of primes of $L$ sitting above $S$. By definition of the Selmer group, the square is a pullback square, and the horizontal line is exact by the Inflation--Restriction exact sequence. Thus, finiteness for the restricted module implies finiteness for $\op{Sel}_S(M)$ because $\mathrm H^1(\op{Gal}(L/K);M)$ is finite (as the cohomology group of a finite module over a finite group).

		\item It remains to show finiteness when the Galois action is trivial. Because $M$ is a finite abelian group, it is a sum of cyclic groups (with trivial action), so we may assume that $M$ is some cyclic group $\ZZ/m\ZZ$. Thus, we see that $\op{Sel}_S(\ZZ/m\ZZ)$ now embeds into $\mathrm H^1(K;\ZZ/m\ZZ)$, which is the same as
		\[\op{Hom}(\op{Gal}(\ov K/K),\ZZ/m\ZZ).\]
		It turns out that a given character $\chi$ is unramified at some place $v\in S$ if and only if $\chi|_{I_v}=1$. Thus, we want to show that there are only finitely many Galois characters which are unramified outside $S$. For this, we see that $\chi$ factors through an extension $L$ of $K$ which is of degree at most $m$ over $K$ and unramified outside $S$, of which there are only finitely many by the Hermite--Minkowski theorem.\footnote{Technically speaking, Hermite--Minkowski only bounds the number of fields with bounded discriminant, but the discriminant is supported in $S$ and the primes have exponent bounded by the degree.}
		\qedhere
	\end{enumerate}
\end{proof}

\subsection{An Extended Example}
Fix a nonzero integer $n$ and consider the ``congruent number'' elliptic curve
\[E_n\colon y^2=x(x-n)(x+n).\]
One can show by some rearrangement that $E_n$ is a quadratic twist of the elliptic curve $y^2=x^3-x$. One can show that $E(\QQ)_{\mathrm{tors}}=E[2]$, which is precisely the set
\[E_n[2]=\{\infty,(0,0),(n,0),(-n,0)\}.\]
We are going to prove the following.
\begin{theorem}
	We have
	\[\dim_{\FF_2}\op{Sel}_2(E_p)=2+\begin{cases}
		0 & \text{if }p\equiv1,3\pmod 8, \\
		1 & \text{if }p\equiv5,7\pmod 8.
	\end{cases}\]
\end{theorem}
\begin{example}
	For any elliptic curve $E$ over $\QQ$, one has
	\[\dim_{\FF_2}\op{Sel}_2(E)=\dim_{\FF_2}E[2](\ov\QQ)+\op{rank}E(\QQ)+\dim_{\FF_2}\Sha(E)[2].\]
	Thus, one sees that $p\equiv1,3\pmod8$ must have rank $0$ and trivial $\Sha(E)[2]$. If $\Sha$ is finite, then the fact that it should be square further implies that the rank is $1$ and $\Sha(E)[2]$ is trivial.
\end{example}
Let's spend some time setting up the calculation. We work with a general elliptic curve $E$ over a number field $K$. Because $E[2]$ is defined over $K$, the Galois action is trivial. Thus, we see that $\mathrm H^1(K;E[2])$ is simply $\mathrm H^1(K;\mu_2)^2$, and we can compute this cohomology using the ``Kummer'' exact sequence
\[1\to\mu_2\to\ov K^\times\stackrel2\to\ov K^\times\to1,\]
which in Galois cohomology produces an isomorphism $\mathrm H^1(K;\mu_2)\cong K^\times/K^{\times2}$ by Hilbert's theorem 90. We may now identify $\mathrm H^1(K;E[2])$ with
\[\left\{(\alpha,\beta,\gamma)\in(K^\times/K^{\times2})^3:\alpha\beta\gamma=1\right\}.\]
It turns out that this map (approximately) sends some point $(x,y)$ of $E(K)/2E(K)$ to the triple $(x-a_1,x-a_2,x-a_3)$ when $E$ has the form $y^2=(x-a_1)(x-a_2)(x-a_3)$. Technically speaking, we should note that we send $\infty$ to the identity $(1,1,1)$, and we send any two-torsion point like $(a_1,0)$ to the triple whose last two coordinates are $a_1-a_2$ and $a_3-a_2$.

Of course, we would like a way to know if an interesting triple $(\alpha,\beta,\gamma)$ is in the image without having to find points in $E(K)$ first. Here is one such test.
\begin{lemma}
	Fix an elliptic curve $E\colon y^2=(x-a_1)(x-a_2)(x-a_3)$ over a finite extension $K$ of $\QQ_p$. Then a triple $(\alpha,\beta,\gamma)$ lies in the image of the above map if and only if the system of equations
	\[\begin{cases}
		\alpha u^2=x-a_1, \\
		\beta v^2=x-a_2, \\
		\gamma w^2=x-a_3
	\end{cases}\]
	admits a solution.
\end{lemma}

\end{document}