% !TEX root = ../notes.tex

\documentclass[../notes.tex]{subfiles}

\begin{document}

\section{September 30}
This week, we will prove Koymans--Pagano's theorem.

\subsection{The Theorem and Its Application}
Quickly, let's recall the statement that we are going to sketch, which is \cite[Theorem~2.4]{koymans-pagano-h10}.
\begin{theorem}[Koymans--Pagano] \label{thm:kp}
	Fix a quadratic extension $K(i)/K$ of number fields, and further assume that $K$ has at least $32$ real places. Then there is an elliptic curve $E$ over $K$ such that
	\[\op{rank}E(K)=\op{rank}E(K(i))>0.\]
\end{theorem}
Let's say something about the application of \Cref{thm:kp}, which is to Hilbert's $10$th problem.
\begin{definition}[Diophantine]
	Fix a number field $K$. Then a subset $S\subseteq\OO_K$ is \textit{Diophantine} if and only if there is an affine scheme $X$ of finite type over $\OO_K$ and a morphism $\pi\colon X\to\AA^1_{\OO_L}$ such that $S=\pi(X(\OO_K))$.
\end{definition}
\begin{remark}
	In other words, there are polynomials $f_1,\ldots,f_m\in\OO_K[t,x_1,\ldots,x_n]$ such that $S$ consists of the values $t\in\OO_L$ for which
	\[\begin{cases}
		f_1(t,x_1,\ldots,x_n)=0, \\
		\qquad\vdots \\
		f_m(t,x_1,\ldots,x_n)=0
	\end{cases}\]
	admits a solution in $\OO_L$. Indeed, the above is equivalent to asserting that $S$ is the image of the map
	\[\Spec\frac{\OO_K[t,x_1,\ldots,x_n]}{(f_1,\ldots,f_m)}\to\AA^1_{\OO_K}\]
	given by projecting onto the $t$ coordinate.
\end{remark}
\begin{remark}
	One can remove the ``affine'' hypothesis from $X$. Indeed, it is enough to show that the collection of Diophantine subsets is closed under finite unions. Well, one simply has to note that $t$ satisfies one of two systems of equations $\{f_i(t,\ov x)\}_i$ or $\{g_j(t,\ov y)\}_j$ if and only if it satisfies the single system of equations
	\[\{f_i(t,\ov x)g_j(t,\ov y)\}_{i,j}.\]
\end{remark}
\begin{remark}
	One can reduce to a system of equations having a single equation, possibly in many variables. It is enough to explain how to take a system of two equations $f(t,\ov x)$ and $g(t,\ov y)$ and produce a single equation. Well, choose a homogeneous polynomial $h\in\OO_L[x,y]$ with no nonzero solution. Then $t$ satisfies both $f(t,\ov x)$ and $g(t,\ov y)$ if and only if $t$ satisfies the composite polynomial $h(f(t,\ov x),g(t,\ov x))$.
\end{remark}
Then \Cref{thm:kp} is the key input in the following result.
\begin{theorem}[Koymans--Pagano] \label{thm:kp-application}
	Fix a number field $K$. Then $\ZZ\subseteq\OO_K$ is Diophantine.
\end{theorem}
\begin{remark}
	Let's explain the significance of this result. Building on work of Davis, Putnam, and Robinson, Matiyasevich used Pell equations to show that there is no algorithm which can determine if a subset of $\ZZ$ is Diophantine \cite{mati-h10}. \Cref{thm:kp-application} then shows that there is no algorithm which can determine if a subset of $\OO_K$ is Diophantine: indeed, because $\ZZ\subseteq\OO_K$ is Diophantine, every subset Diophantine subset of $\ZZ$ is now a Diophantine subset of $\OO_K$.
\end{remark}
\Cref{thm:kp-application} is proven thanks to the following reduction of \cite{shlap-elliptic-curve-h10}.
\begin{theorem} \label{thm:ab-var-bring-down-h10}
	Fix a finite extension $L/K$ of number fields. If there is an abelian variety $A$ over $K$ such that
	\[\op{rank}A(L)=\op{rank}A(K)>0,\]
	then $\OO_K\subseteq\OO_K$ is a Diophantine subset.
\end{theorem}
In order to explain how this applies, we need the following ``base case'' of \cite{denef-h10}.
\begin{theorem}[Denef] \label{thm:h10-real}
	Fix a totally real number field $K$. Then $\ZZ\subseteq\OO_K$ is Diophantine.
\end{theorem}
We will also need some flexibility to make our reductions.
\begin{proposition} \label{prop:h10-reductions}
	Fix an extension $L/K$ of number fields.
	\begin{listalph}
		\item Diophantine subsets of $\OO_K$ are closed under finite intersection.
		\item If $\ZZ\subseteq\OO_K$ is Diophantine, and $\OO_K\subseteq\OO_L$ is Diophantine, then $\ZZ\subseteq\OO_L$ is Diophantine.
		\item If $\ZZ\subseteq\OO_L$ is Diophantine, then $\ZZ\subseteq\OO_K$ is Diophantine.
	\end{listalph}
\end{proposition}
\begin{proof}
	We show the parts separately, following \cite[Proposition~1]{denef-lipshitz-diophantine}.
	\begin{listalph}
		\item Take the fiber product of the schemes whose image are giving the provided Diophantine subsets.
		\item We may cut out $\ZZ\subseteq\OO_L$ by first cutting out $\OO_K\subseteq\OO_L$ by polynomials and then appending the polynomials which cut out $\ZZ\subseteq\OO_K$.
		\item This follows because $\OO_L$ is a finitely presented $\OO_K$-module. Indeed, one can simply replace all coefficients in $\OO_L$ with formal sums of a generating set from $\OO_K$ subject to certain relations.
		\qedhere
	\end{listalph}
\end{proof}
\begin{proof}[Proof that \Cref{thm:kp} implies \Cref{thm:kp-application}]
	We follow \cite[Theorem~2.6]{koymans-pagano-h10}. To match our application, we change variables somewhat. Let $F$ be a number field so that we want to show $\ZZ\subseteq\OO_F$ is Diophantine. We let $E$ be some totally real multiquadratic field of degree $64$, and we set $L$ to be the normal closure of $EF(i)$. By \Cref{prop:h10-reductions}, it is enough to $\ZZ\subseteq\OO_L$ is Diophantine. We now proceed in steps.
	\begin{enumerate}
		\item Suppose $L=K(i)$ for some field $K$ containing $E$ with at least one real place. Then we claim that $\OO_K\subseteq\OO_L$ is Diophantine. By \Cref{thm:ab-var-bring-down-h10}, it is enough to find an elliptic curve $E$ with $\op{rank}E(L)=\op{rank}E(K)>0$, which is what we will use \Cref{thm:kp} for. For this, it remains to show that $K$ has at least $32$ real places. In fact, we will show that the number of real embeddings is
		\[\frac12\#C_{\op{Gal}(L/\QQ)}(\op{Gal}(L/K)).\]
		To see why this completes the proof, note that this centralizer equals $[L:\QQ]$ divided by the number of elements conjugate to an element in $\op{Gal}(L/K)$, but $E\subseteq K$ means that being conjugate to an element in $\op{Gal}(L/K)$ requires being conjugate to an element in $\op{Gal}(L/E)$ (because $E/\QQ$ is Galois). So the size of the centralizer is lower-bounded by $[E:\QQ]=64$.

		It remains to count these real embeddings. Fix some real embedding $\sigma\colon K\into\RR$, which we extend to some embedding $\widetilde\sigma\colon L\into\CC$. Then the other archimedean embeddings of $K$ look like $\widetilde\sigma\circ\tau$ for $\tau\in\op{Gal}(L/\QQ)$. Now, we note that $\widetilde\sigma\circ\tau$ is a real embedding of $K$ if and only if it is equal to its conjugate, which one can check is equivalent to commuting with $\op{Gal}(L/K)$. Pairing off the (complex!) embeddings $\tau$ of $L$ completes the counting.

		\item We now let $M$ be the intersection of all fields $K$ containing $E$ with at least one real place and for which $L=K(i)$. By the previous step and \Cref{prop:h10-reductions}, we see that $\OO_M\subseteq\OO_L$ is Diophantine, so by \Cref{prop:h10-reductions} again, it is enough to show that $\ZZ\subseteq\OO_M$ is Diophantine. In fact, we claim that $M$ is totally real, from which the claim follows by \Cref{thm:h10-real}. Indeed, to see that $M$ is totally real, we note that $M$ is Galois over $\QQ$ because it is an intersection of a Galois-invariant collection of fields. Further, $M$ has at least one real places, so it follows that $M$ is totally real!
		\qedhere
	\end{enumerate}
\end{proof}
% This basically allows us to propagate a negative solution of Hilbert's $10$th problem from $K$ to $L$. One also knows that negative solutions for intersections and for totally real fields, so the restricted quadratic extensions provided by \Cref{thm:kp} complete the argument. The exact deductions are fairly involved; see \cite[Section~2]{koymans-pagano-h10}.

\subsection{Reduction to Selmer Groups}
We now outline the proof of \Cref{thm:kp}, which will be done by controlling some quadratic twists.
% In fact, they show that there is an elliptic curve $E$ over $K$ and $t\in K^\times/K^{\times2}$ such that we have control of the ranks of the ``twists'' $\op{rank}E^t(K)=0$ and $\op{rank}E^{-t}(K)>0$.
\begin{definition}[quadratic twist]
	For an elliptic curve $E$ over a field $K$ (not of characteristic two) cut out by the equation $y^2=x^3+ax^2+bc+c$, the \textit{quadratic twist} $E^t$ by some $t\in K^\times/K^{\times2}$ is the elliptic curve cut out by the equation.
	\[ty^2=x^3+ax^2+bx+c.\]
\end{definition}
\begin{remark}
	There is an isomorphism from $E$ to $E^t$ over the quadratic extension $\QQ(\sqrt t)$ given by $(x,y)\mapsto(x,y/\sqrt t)$. Note that this isomorphism is not defined over $K$!
\end{remark}
\begin{remark}
	Note that the given equation for $E^t$ is equivalent to
	\[\left(t^2y\right)^2=(tx)^3+at(tx)^2+bt^2(tx)+ct^3,\]
	so $E^t$ could also be cut out by the equation
	\[y^2=x^3+atx^2+bt^2x+c.\]
	For example, in the case where $E$ is cut out by $y^2=(x-a_1)(x-a_2)(x-a_3)$, this latter equation becomes the convenient equation $y^2=(x-a_1t)(x-a_2t)(x-a_3t)$.
\end{remark}
\begin{remark} \label{rem:decompose-quadratic-jump}
	Here is the important point of quadratic twists: the isomorphism $E^t_L\cong E_L$ given by $(x,y)\mapsto(x,y/\sqrt t)$ induces an embedding
	\[\iota\colon E^t(K)\into E(L).\]
	Now, write $\op{Gal}(L/K)=\{1,\sigma\}$, and we claim that the image of $\iota$ is exactly the subgroup of $E(L)$ such that $\sigma$ acts by $-1$: indeed, $-(x,y)=(x,-y)$, so $\sigma(x,y)=-(x,y)$ if and only if $x\in K$ and $y\in K/\sqrt t$. Writing $(x',y')\coloneqq(x,y/\sqrt t)$, we then see that $(x,y)\in E(L)$ if and only if $(x',y')\in E^t(K)$.
\end{remark}
\begin{lemma} \label{lem:rank-jump-twist}
	Fix a number field $K$, and let $L=K(\sqrt t)$ be a quadratic extension, where $t\in K^\times/K^{\times2}$ is nontrivial. For any elliptic curve $E$ over $K$, we have
	\[\op{rank}E(L)=\op{rank}E(K)+\op{rank}E^t(K).\]
\end{lemma}
\begin{proof}
	For concreteness, we present $E$ and $E^t$ as cut out by the equations $y^2=x^3+ax^2+bx+c$ and $ty^2=x^3+ax^2+bx+c$, respectively. Because $E(K)$ is a finitely generated group, we can compute its rank as $\dim_\QQ E(K)_\QQ$. This applies to the other elliptic curves as well, so we are left to show
	\[\dim_\QQ E(L)_\QQ=\dim_\QQ E(K)_\QQ+\dim_\QQ E^t(K)_\QQ.\]
	Now, because $E$ is defined over $K$, the set $E(L)$ admits an action by $\op{Gal}(L/K)$. But $\op{Gal}(L/K)$ has two elements, which we denote $\{1,\sigma\}$. Then $\sigma^2=1$ splits over $\QQ$ and so implies that the representation $E(L)$ of $\op{Gal}(L/K)$ splits into eigenspaces of $\sigma$ as $V_+\oplus V_-$ where $\sigma$ acts on $V_+$ by $+1$ and acts on $V_-$ by $-1$. We now compute these spaces.
	\begin{itemize}
		\item We claim $E(K)_\QQ=V_+$. Indeed, $(x,y)\otimes q\in E(L)\otimes\QQ$ is fixed by $\sigma$ if and only if $(x,y)\in E(K)$.
		\item We claim $E^t(K)_\QQ\cong V_-$. Indeed, $(x,y)\otimes q\in E(L)\otimes\QQ$ has eigenvalue $-1$ if and only if $(x,y)$ comes from $E^t(K)$ via \Cref{rem:decompose-quadratic-jump}.
	\end{itemize}
	Summing the above two calculations completes the proof.
\end{proof}
As such, to show \Cref{thm:kp}, we will instead sketch the following.
\begin{theorem}[Koymans--Pagano] \label{thm:kp-by-twist}
	Fix a quadratic extension $K(i)/K$ of number fields, and further assume that $K$ at least $32$ real places. Then there is an elliptic curve $E$ over $K$ and $t\in K^\times/K^{\times2}$ for which $E[2](\ov K)=E[2](K)$ and
	\[\dim_{\FF_2}\op{Sel}_2(E^t/K)=2\qquad\text{and}\qquad\op{rank}E^{-t}(K)>0.\]
\end{theorem}
\begin{proof}[Proof that \Cref{thm:kp-by-twist} implies \Cref{thm:kp}]
	We claim that $E^{-t}$ is the elliptic curve we are looking for, so we have to show that $\op{rank}E^{-t}(K(i))=\op{rank}E^{-t}(K)$. Well, by \Cref{lem:rank-jump-twist}, it is enough to show that
	\[\op{rank}\left(E^{-t}\right)^{-1}(K)=0.\]
	But of course, $\left(E^{-t}\right)^{-1}=E^t$ (for example, by unwinding our definition of the quadratic twists), so we want to show that $\op{rank}E^t(K)=0$. Now, \Cref{lem:selmer-ses} tells us that
	\[\dim_{\FF_2}\op{Sel}_2(E^t/K)\ge\dim_{\FF_2}\frac{E(K)}{2E(K)}.\]
	Because $E[2](K)=E[2](\ov K)$, this right-hand side is $2+\op{rank}E(K)$, so $\dim_{\FF_2}\op{Sel}_2(E^t/K)=2$ enforces $\op{rank}E(K)=0$.
\end{proof}
\begin{remark}
	An examination of the inequality at the end produced by \Cref{lem:selmer-ses} shows that this proof has also verified that the elliptic curve constructed in \Cref{thm:kp-by-twist} has $\Sha(E/K)[2]=0$. It may be interesting to construct curves with controlled $\Sha$ in addition.
\end{remark}

\subsection{Twisting for Rank}
We will only manage to sketch \Cref{thm:kp-by-twist}. Our argument is based on two ideas.
\begin{itemize}
	\item One can use $2$-Selmer groups to control ranks under quadratic twists. Because this is a bound on Selmer groups, this only gives an upper bound of a rank.
	\item One can sometimes construct explicit non-torsion rational points on elliptic curves. This only gives a lower bound of a rank.
\end{itemize}
We will return to the first point shortly. For now, let's explain the second. It is fairly explicit: suppose we write $E$ as the projective closure of
\[y^2=(x-a_1)(x-a_2)(x-a_3)\]
for $a_1,a_2,a_3\in\OO_K$. If we plug in $x=c/d$, then our right-hand side is
\[\prod_{i=1}^3(x-a_i)=\frac1{d^4}\cdot d\prod_{i=1}^3(c-a_id).\]
Thus, if we set $t\coloneqq d\prod_{i=1}^3(c-a_id)$, then $E^t$ is the projective closure by $ty^2=(x-a_1)(x-a_2)(x-a_3)$, which has the rational point $\left(c/d,1/d^2\right)$ by construction. If $c/d$ is ``generic,'' then this should even provide us with a non-torsion rational point.
\begin{lemma} \label{lem:bound-torsion-degree}
	Fix an elliptic curve $E$ defined over a number field $K$. For any positive integer $d$,
	\[\bigcup_{[L:K]<d}E(L)_{\mathrm{tors}}\]
	is finite.
\end{lemma}
\begin{proof}
	We use the theory of the N\'eron--Tate height $\hat h$, which is summarized in \cite[Theorem~9.3]{silverman}. Indeed, we note that $\hat h$ vanishes on $E(\ov K)_{\mathrm{tors}}$, so the union in question is a set of bounded height and degree, so it is finite by Northcott's property (of heights).
\end{proof}
\begin{proposition} \label{prop:bound-2-torsion-twist}
	Fix an elliptic curve $E$ defined over a number field $K$. For all but finitely many $t\in K^\times/K^{\times2}$, we have $E^t(K)_{\mathrm{tors}}=E^t(K)[2]$.
\end{proposition}
\begin{proof}
	We use \Cref{lem:bound-torsion-degree}, following \cite[Lemma~3.2]{koymans-pagano-h10}. It is enough to show that
	\[E^t(K(\sqrt t))_{\mathrm{tors}}\stackrel?\subseteq E^t(K)[2]\]
	for all but finitely many $t$.
	% Well, note $E^t(K)[2]$ is identified with $E(K)[2]$ using the usual equations, and $E(K)[2]=E(K(\sqrt t))[2]$ for all but finitely many $t$ (because a given $2$-torsion point is either already defined over $K$ or itself defines a nontrivial extension of $K$ which will be in at most one quadratic extension). Thus, it is enough to show that
	% \[E^t(K(\sqrt t))_{\mathrm{tors}}\stackrel?\subseteq E(K(\sqrt t))[2]\]
	% for all but finitely many $t$. Well, the left-hand side embeds into $E(K(\sqrt t))_{\mathrm{tors}}$ by \Cref{rem:decompose-quadratic-jump}, so it is enough to show that
	% \[E(K(\sqrt t))_{\mathrm{tors}}\stackrel?\subseteq E(K(\sqrt t))[2]\]
	% for all but finitely many $t$.
	
	We now apply \Cref{lem:bound-torsion-degree}: after embedding $E^t(K)\into E(K(\sqrt t))$ via \Cref{rem:decompose-quadratic-jump}, the union over all $K(\sqrt t)$ of the left-hand side must be a finite set. Thus, for all but finitely many $t$, we need to have $E(K)_{\mathrm{tors}}\subseteq E(K)_{\mathrm{tors}}$ for a distinct $s\in K^\times/K^{\times2}$. But after the embedding of \Cref{rem:decompose-quadratic-jump}, we see that the intersection of $E^t(K)$ and $E^s(K)$ in $E(K(\sqrt s,\sqrt t))$ must live in $E(K)$ and thus by fixed by the Galois actions. But the image of $E^t(K)$ is the Galois submodule with eigenvalue $-1$, so it follows that the intersection is contained in $E(K)[2]=E^t(K)[2]$.
\end{proof}
Thus, we will be able to focus on calculating Selmer ranks for twists of specific type.
\begin{theorem}[Koymans--Pagano] \label{thm:kp-selmer-rank}
	Fix a quadratic extension $K(i)/K$ of number fields, and further assume that $K$ has at least $32$ real places. Then there is an elliptic curve $E$ over $K$ cut out by $y^2=(x-a_1)(x-a_2)(x-a_3)$ for $a_1,a_2,a_3\in\OO_K$ with the following property: there are infinitely many $t\in K^\times/K^{\times2}$ of the form $d\prod_{i=1}^3(c+a_id)$ (where $m,c,d\in\OO_K$) and
	\[\dim_{\FF_2}\op{Sel}_2(E^t/K)=2.\]
\end{theorem}
\begin{proof}[Proof that \Cref{thm:kp-selmer-rank} implies \Cref{thm:kp-by-twist}]
	We only have to check that $\op{rank}E^{-t}(K)>0$. As argued above \Cref{lem:bound-torsion-degree}, when $t$ takes the form $d\prod_{i=1}^3(c+a_id)$, the twist $E^{-t}$ cut out by the equation
	\[-d(c+a_1d)(c+a_2d)(c+a_3d)y^2=(x-a_1)(x-a_2)(x-a_3)\]
	contains the $K$-rational point $\left(-c/d,1/d^2\right)$. Now, by \Cref{prop:bound-2-torsion-twist}, for all but finitely many $t$, the only torsion points of $E^{-t}(K)$ are $2$-torsion, which are either $\infty$ or are an affine point $(x,y)$ with $y=0$. This means that the constructed point $\left(-c/d,1/d^2\right)$ is in fact non-torsion, so $\op{rank}E^{-t}(K)>0$ follows.
\end{proof}
% \begin{remark}
% 	Later, Zywina used this method to also get $\op{rank}E(K)=1$ at the same time, which amounts to requiring us to do another $2$-descent to bound the Selmer group of $E^D$.
% \end{remark}

\end{document}