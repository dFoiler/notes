% !TEX root = ../notes.tex

\documentclass[../notes.tex]{subfiles}

\begin{document}

\section{September 16}
Welcome to the second class of the semester. The note-taker is furiously eating lunch.

\subsection{Construction of Group Cohomology}
For the next few weeks, we are going to focus on proving \Cref{thm:zywina-koymans-pagano}. This will be done using Selmer groups.

We begin by recalling the definition of group cohomology.
\begin{definition}[module]
	Fix a group $G$. Then a \textit{$G$-module} is an abelian group $M$ equipped with an action by $G$ for which $1m=m$ for all $m\in M$ and $g(m+n)=gm+gn$ for all $g\in G$ and $m,n\in M$.
\end{definition}
\begin{remark}
	Equivalently, a $G$-module is a module for the ring $\ZZ[G]$.
\end{remark}
\begin{definition}[invariants]
	Fix a group $G$. Then there is a functor $(-)^G\colon\mathrm{Mod}_{\ZZ[G]}\to\mathrm{Ab}$ given on objects by sending a $G$-module $M$ to the subset
	\[M^G\coloneqq\{m\in M:gm=m\text{ for all }g\in G\}.\]
	On morphisms, it sends $f\colon M\to N$ to the restriction $f\colon M^G\to N^G$.
\end{definition}
\begin{remark}
	One can show that there is a natural isomorphism
	\[\op{Hom}_{\ZZ[G]}(\ZZ,-)\Rightarrow(-)^G.\]
	It sends a map $f\colon\ZZ\to M$ to $f(1)$; the inverse sends $m\in M^G$ to the map $f\colon\ZZ\to M$ given by $k\mapsto km$.
\end{remark}
\begin{definition}[group cohomology]
	Fix a group $G$. The \textit{group cohomology groups} $\mathrm H^\bullet(G;-)$ are the right-derived functors for the invariants functor $(\cdot)^G\colon\mathrm{Mod}_{\ZZ[G]}\to\mathrm{Ab}$.
\end{definition}
\begin{remark}
	In light of the natural isomorphism $(-)^G=\op{Hom}_{\ZZ[G]}(\ZZ,-)$, we see that
	\[\mathrm H^\bullet(G,-)=\mathrm{Ext}^\bullet_{\ZZ[G]}(\ZZ,-).\]
\end{remark}

\subsection{Some Calculations}
Because we are now dealing with $\mathrm{Ext}$ groups, there are two ways to compute $\mathrm H^\bullet(G,M)$.
\begin{itemize}
	\item We can build an injective resolution of $M$, apply $(-)^G$, and take cohomology.
	\item We can build a projective resolution of $\ZZ$, apply $\op{Hom}_{\ZZ[G]}(-,M)$, and take cohomology.
\end{itemize}
The second is easier for the purposes of calculation.
\begin{example} \label{ex:compute-group-cohom}
	It turns out that there is a free resolution
	\[\cdots\to\ZZ\left[G^3\right]\to\ZZ\left[G^2\right]\to\ZZ[G]\to\ZZ\to0.\]
	Here, the map $\ZZ[G]\to\ZZ$ sends $\sum_ga_gg$ to $\sum_ga_g$. In general, the map $d_{n+1}\colon\ZZ\left[G^{n+1}\right]\to\ZZ\left[G^n\right]$ is given by $\ZZ$-linearly extending
	\[d_{n+1}(g_0,\ldots,g_n)\coloneqq\sum_{i=0}^n(-1)^i(g_0,\ldots,g_{i-1},g_{i+1},\ldots,g_n).\]
	One can check that this is a free resolution of $\ZZ$. We let $\mc P_\bullet$ be the above complex where we have truncated off $\ZZ$, so we see that $\mathrm H^i(G;M)$ is
	\[\mathrm{Ext}^i_{\ZZ[G]}(\ZZ,M)=\mathrm H^i(\mathrm{Hom}_G(\mc P_\bullet,M)).\]
\end{example}
\begin{remark}
	If $G$ is finite and $M$ is finite, then a direct calculation of the cohomology via the resolution in \Cref{ex:compute-group-cohom} implies that $\mathrm H^i(G;M)$ is finite in all degrees.
\end{remark}
While the combinatorics in \Cref{ex:compute-group-cohom} becomes difficult for large $n$, we can be fairly explicit about $n=1$. In this case, one can show that $\mathrm H^1(G;M)$ is isomorphic to the quotient of the crossed homomorphisms by the principal crossed homomorphisms.
\begin{definition}[crossed homomorphism]
	Fix a group $G$ and a $G$-module $M$. Then a \textit{crossed homomorphism} is a function $f\colon G\to M$ for which
	\[f(gh)=gf(h)+f(g)\]
	for all $g,h\in G$.
\end{definition}
\begin{example}[principal crossed homomorphism] \nirindex{crossed homomorphism!principal crossed homomorphism}
	For any $m\in M$, we can define a map $f\colon G\to M$ by
	\[f(g)\coloneqq(g-1)m.\]
	This is a crossed homomorphism, which amounts to checking
	\[(gh-1)m\stackrel?=g(h-1)m+(g-1)m.\]
	We call such a crossed homomorphism ``principal.''
\end{example}
\begin{example}
	If the action of $G$ on $M$ is trivial, then a crossed homomorphism is just a group homomorphism. Additionally, all the principal crossed homomorphisms vanish, so we see that
	\[\mathrm H^1(G;M)=\op{Hom}_\ZZ(G,M).\]
	For example, $\mathrm H^1(1;\ZZ)=\ZZ$ is infinite.
\end{example}
In the case where $G$ is cyclic, there is an easier resolution than the one in \Cref{ex:compute-group-cohom}.
\begin{proposition}
	Fix a finite cyclic group $G$. Then for any $G$-module $M$ and index $i>0$, we have
	\[\mathrm H^i(G;M)=\begin{cases}
		M^G/\im\op N_G & \text{if }i\text{ is even}, \\
		\ker{\op N_G}/\im(\sigma-1) & \text{if }i\text{ is odd}.
	\end{cases}\]
	In particular, $\left\{\mathrm H^i(G;M)\right\}_{i>0}$ is $2$-periodic.
\end{proposition}
\begin{proof}
	Suppose that $G$ is finite cyclic of order $n$ and generated by some $\sigma$. We will build an explicit resolution for $\ZZ$. We start with the degree map $\ZZ[G]\onto\ZZ$ has kernel generated by $(\sigma-1)$, so we can surject onto its kernel via the map $(\sigma-1)\colon\ZZ[G]\to\ZZ[G]$. On the other hand, the kernel of $(\sigma-1)$ is exactly isomorphic to $\ZZ$, given by the elements of the form $k\sum_{i=0}^{n-1}\sigma^i$ where $k$ is some integer. In other words, the kernel of $(\sigma-1)$ is given by the norm map $\op N_G\colon\ZZ[G]\to\ZZ[G]$, where $\op N_G(x)\coloneqq\sum_{g\in G}gx$. Because we are back at $\ZZ$, we see that we can iterate to produce a resolution
	\[\cdots\stackrel{(\sigma-1)}\to\ZZ[G]\stackrel{\op N_G}\to\ZZ[G]\stackrel{(\sigma-1)}\to\ZZ[G]\stackrel{\deg}\to\ZZ\to0.\]
	We now compute cohomology. After truncating and applying $\op{Hom}_{\ZZ[G]}(-,M)$, we receive the complex
	\[0\to M\stackrel{\sigma-1}\to M\stackrel{\op N_G}\to M\stackrel{\sigma-1} M\to\cdots,\]
	where the leftmost $M$ lives in degree $0$. For example, we can see that $\mathrm H^0(G;M)$ is $\ker(\sigma-1)$, which is $\{m\in M:\sigma m=m\}$, which is $M^G$. Continuing, for $i>0$, we see that
	\[\mathrm H^i(G;M)=\begin{cases}
		M^G/\im\op N_G & \text{if }i\text{ is even}, \\
		\ker{\op N_G}/\im(\sigma-1) & \text{if }i\text{ is odd},
	\end{cases}\]
	as desired.
\end{proof}

\subsection{Change of Group}
We will get some utility out of having more functors.
\begin{definition}[induction]
	Fix a subgroup $H\subseteq G$. Then there is an \textit{induction} functor $\op{Ind}_H^G\colon\mathrm{Mod}_H\to\mathrm{Mod}_G$ given on objects by sending any $H$-module $N$ to $\op{Ind}_H^GN$, defined as the module of functions $f\colon G\to N$ for which $f(hx)=hf(x)$ for any $h\in H$. This is a $G$-module with action given by
	\[(gf)(x)\coloneqq f(xg).\]
\end{definition}
With an induction, we also have a restriction.
\begin{definition}[restriction]
	Fix a subgroup $H\subseteq G$. Then there is a \textit{restriction} functor $\op{Res}^G_H\colon\mathrm{Mod}_G\to\mathrm{Mod}_H$ given on objects by sending any $G$-module $M$ to the same abelian group equipped with an $H$-action via the inclusion $H\subseteq G$. This functor is the identity on morphisms.
\end{definition}
\begin{remark}
	We can define a map $M\to\op{Ind}_H^G\op{Res}_H^GM$ given by sending $m\in M$ to the map $f\colon G\to M$ defined by $f(g)\coloneqq gm$. This gives part of the adjunction.
\end{remark}
Here are the results on induction and restriction.
\begin{proposition}[Frobenius reciprocity]
	Fix a subgroup $H$ of a finite group $G$. Then $\op{Ind}_H^G$ and $\op{Res}^G_H$ are adjoints of each other. In particular, $\mathrm{Ind}_H^G\colon\mathrm{Mod}_H\to\mathrm{Mod}_G$ is an exact functor.
\end{proposition}
\begin{proposition}[Shapiro's lemma]
	Fix a subgroup $H$ of a finite group $G$. Then there is a natural isomorphism
	\[\mathrm H^\bullet\left(G;\op{Ind}_H^G(-)\right)\simeq\mathrm H^i(H;-).\]
\end{proposition}
It turns out that restriction has a sort of dual.
\begin{definition}[corestriction]
	Fix a finite-index subgroup $H$ of a group $G$. Then we define the \textit{corestriction} $\op{Cores}\colon\mathrm H^i(H;M)\to\mathrm H^i(G;M)$ map by extending the map $M^H\to M^G$ in degree $0$ defined by
	\[m\mapsto\sum_{gH\in G/H}gm.\]
\end{definition}
\begin{remark}
	It turns out that the composite
	\[\mathrm H^i(G;M)\stackrel{\op{Res}}\to\mathrm H^i(H;M)\stackrel{\op{Cores}}\to\mathrm H^i(G;M)\]
	is multiplication by $[G:H]$. For example, if $G$ is finite, we can set $H$ to be the trivial group so that the middle term vanishes in positive degree; thus, we see that $\mathrm H^i(G;M)$ is $\left|G\right|$-torsion for $i>0$.
\end{remark}
Our last functor allows us to take quotients.
\begin{definition}[inflation]
	Fix a normal subgroup $H$ of a group $G$. Then for any $G$-module $M$, there is an inflation map $\mathrm H^\bullet\left(G/H;M^H\right)\to\mathrm H^\bullet(G;M)$ defined as the composite
	\[\mathrm H^\bullet\left(G/H,M^H\right)\to\mathrm H^\bullet\left(G;M^H\right)\to\mathrm H^\bullet(G;M).\]
	The left map exists via the forgetful functor $\mathrm{Mod}_{G/H}\to\mathrm{Mod}_G$ induced by the quotient $G\onto G/H$. The right map exists by functoriality of $\mathrm H^\bullet(G;-)$.
\end{definition}
Here is the result we need on inflation.
\begin{proposition}[Inflation--restriction]
	Fix a $G$-module $M$. Then there is an exact sequence
	\[0\to\mathrm H^1(G/H;M^H)\stackrel{\op{Inf}}\to\mathrm H^1(G;M)\stackrel{\op{Res}}\to\mathrm H^1(H;M)^{G/H}.\]
\end{proposition}

\subsection{Galois Cohomology}
We quickly explain how to take cohomology for profinite groups.
\begin{example}
	Fix a finite field $k$ with $q$ elements. Then $\op{Gal}(\ov k/k)$ is a profinite group with topological generator given by the Frobenius. Explicitly,
	\[\op{Gal}(\ov k/k)=\lim_n\op{Gal}(\FF_{q^n}/\FF_q)=\lim_n\ZZ/n\ZZ=\widehat\ZZ.\]
\end{example}
\begin{definition}[discrete]
	Fix a profinite group $G$. Then a $G$-module $M$ is \textit{discrete} if and only if the stabilizer $\op{Stab}_G(m)$ is open for all $m\in M$.
\end{definition}
\begin{remark}
	Equivalently, we are asking for the action map $G\times M\to M$ to be continuous, where $M$ has been given the discrete topology: the fiber over the open set $\{m\}$ of $M$ contains the open subset $\op{Stab}_G(m)\times\{m\}$.
\end{remark}
\begin{definition}[continuous group cohomology]
	Fix a profinite group $G$, and write $G=\lim_H G/H$, where the limit varies over open normal subgroups. Then we define
	\[\mathrm H^i_{\mathrm{cts}}(G;M)\coloneqq\colim_{\text{ open normal }H\subseteq G}\mathrm H^i\left(G/H;M^H\right).\]
	We will frequently write $\mathrm H^i(G;M)$ for $\mathrm H^i_{\mathrm{cts}}(G;M)$ whenever $G$ is profinite. In particular, we will never use ordinary group cohomology for profinite groups $G$.
\end{definition}
\begin{remark}
	Equivalently, following \Cref{ex:compute-group-cohom}, we can define $\mathrm H^i_{\mathrm{cts}}(G;M)$ as
	\[\mathrm H^i(\op{Hom}_{\mathrm{cont}}(\mc P_\bullet,M)),\]
	where we are now requiring that the maps from $\mc P_j\to M$ be continuous.
\end{remark}
We can now upgrade our calculation for cyclic groups to procyclic groups.
\begin{proposition} \label{prop:procyclic-cohom}
	Fix a procyclic group $G$ with generator $\sigma$. Fix a finite discrete $G$-module $M$. Then
	\[\mathrm H^i\left(G;M\right) = \begin{cases}
		M^{G} & \text{if }i=0, \\
		M/(\sigma-1) & \text{if }i=1, \\
		0 & \text{if }i\ge2.
	\end{cases}\]
\end{proposition}
\begin{remark}
	Equivalently, the cohomology of $M$ is computed via the two-term complex
	\[0\to M\stackrel{\sigma-1}\to M\to 0.\]
\end{remark}
This allows us to say something about Galois cohomology.
\begin{notation}
	Fix a field $k$ and a commutative group scheme $X$ over $k$. Then we set the notation
	\[\mathrm H^i(k;X)\coloneqq\mathrm H^i\left(\op{Gal}(\ov k/k);X(\ov k)\right).\]
\end{notation}
\begin{example}
	Fix a finite field $k$. From \Cref{prop:procyclic-cohom}, we see that $\mathrm H^i(k;M)=0$ for $i\ge2$ for any finite discrete $\op{Gal}(\ov k/k)$-module $M$.
\end{example}

\end{document}