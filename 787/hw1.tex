% !TEX root = notes.tex

\documentclass[notes.tex]{subfiles}

\begin{document}

\setlength{\headheight}{13.40561pt}
\addtolength{\topmargin}{-0.05557pt}

\chapter{Homework 1}

The homework problems have been recorded as propositions.

\section{The Cohomology of \texorpdfstring{$\widehat {\mathbb Z}$}{ Z hat}}
We begin by recalling the cohomology of cyclic groups.
\begin{lemma} \label{lem:compute-group-cohom-cyclic-hw1}
	Fix a finite cyclic group $G$ generated by $\sigma$. Then for any $G$-module $M$ and index $i>0$, we have
	\[\mathrm H^i(G;M)=\begin{cases}
		M^G/\im\op N_G & \text{if }i\text{ is even}, \\
		\ker{\op N_G}/\im(\sigma-1) & \text{if }i\text{ is odd}.
	\end{cases}\]
	In particular, $\left\{\mathrm H^i(G;M)\right\}_{i>0}$ is $2$-periodic.
\end{lemma}
\begin{proof}
	Suppose that $G$ is finite cyclic of order $n$ and generated by some $\sigma$. We will build an explicit resolution for $\ZZ$. We start with the degree map $\ZZ[G]\onto\ZZ$ has kernel generated by $(\sigma-1)$, so we can surject onto its kernel via the map $(\sigma-1)\colon\ZZ[G]\to\ZZ[G]$. On the other hand, the kernel of $(\sigma-1)$ is exactly isomorphic to $\ZZ$, given by the elements of the form $k\sum_{i=0}^{n-1}\sigma^i$ where $k$ is some integer. In other words, the kernel of $(\sigma-1)$ is given by the norm map $\op N_G\colon\ZZ[G]\to\ZZ[G]$, where $\op N_G(x)\coloneqq\sum_{g\in G}gx$; equivalently, we can view $\op N_G$ as multiplication by the norm element ${\op N_G}\coloneqq\sum_{g\in G}g$. The kernel of $\op N_G$ can be calculated as the image of $(\sigma-1)$ again, so we see that we can iterate to produce a resolution
	\[\cdots\stackrel{(\sigma-1)}\to\ZZ[G]\stackrel{\op N_G}\to\ZZ[G]\stackrel{(\sigma-1)}\to\ZZ[G]\stackrel{\deg}\to\ZZ\to0.\]
	We now compute cohomology. After truncating and applying $\op{Hom}_{\ZZ[G]}(-,M)$, we receive the complex
	\[0\to M\stackrel{\sigma-1}\to M\stackrel{\op N_G}\to M\stackrel{\sigma-1}\to M\to\cdots,\]
	where the leftmost $M$ lives in degree $0$. For example, we can see that $\mathrm H^0(G;M)$ is $\ker(\sigma-1)$, which is $\{m\in M:\sigma m=m\}$, which is $M^G$. Continuing, for $i>0$, we see that
	\[\mathrm H^i(G;M)=\begin{cases}
		M^G/\im\op N_G & \text{if }i\text{ is even}, \\
		\ker{\op N_G}/\im(\sigma-1) & \text{if }i\text{ is odd},
	\end{cases}\]
	as desired.
\end{proof}
It is worthwhile to explain the functoriality properties of \Cref{lem:compute-group-cohom-cyclic-hw1}, which are rather bizarre.
\begin{lemma} \label{lem:compute-group-cohom-cyclic-funct-hw1}
	Fix a surjection $G'\to G$ of cyclic groups. Then given a morphism $f\colon M\to M'$ where $M$ is a $G$-module, and $M'$ has the induced $G'$-module structure, the induced map
	\[f\colon\mathrm H^i(G;M)\to\mathrm H^i(G';M')\]
	is $(\#G'/\#G)^{\floor{i/2}}f$ once these groups have been identified with subquotients of $M$ and $M'$ via \Cref{lem:compute-group-cohom-cyclic-hw1}.
\end{lemma}
\begin{proof}
	Denote the surjection $G'\to G$ by $g$, and we choose a generator $\sigma'$ for $G'$, and then we define $\sigma\coloneqq g(\sigma')$, which we see must generate $G$. We also set $m\coloneqq\#G'/\#G$ for brevity. Now, the identities $g(\sigma'-1)=(\sigma-1)$ and $g({\op N_{G'}})=m\op N_G$ produce the morphism
	% https://q.uiver.app/#q=WzAsMTYsWzYsMCwiXFxaWiJdLFs3LDAsIjAiXSxbNywxLCIwIl0sWzYsMSwiXFxaWiJdLFs1LDAsIlxcWlpbRyddIl0sWzUsMSwiXFxaWltHXSJdLFs0LDAsIlxcWlpbRyddIl0sWzQsMSwiXFxaWltHXSJdLFszLDAsIlxcWlpbRyddIl0sWzMsMSwiXFxaWltHXSJdLFsyLDAsIlxcWlpbRyddIl0sWzEsMCwiXFxaWltHJ10iXSxbMCwwLCJcXGNkb3RzIl0sWzAsMSwiXFxjZG90cyJdLFsxLDEsIlxcWlpbR10iXSxbMiwxLCJcXFpaW0ddIl0sWzQsMCwiXFxkZWciXSxbNSwzLCJcXGRlZyIsMl0sWzMsMl0sWzAsMV0sWzAsMywiIiwxLHsibGV2ZWwiOjIsInN0eWxlIjp7ImhlYWQiOnsibmFtZSI6Im5vbmUifX19XSxbNCw1LCJnIl0sWzYsNCwiKFxcc2lnbWEnLTEpIl0sWzcsNSwiKFxcc2lnbWEtMSkiLDJdLFs2LDcsImciXSxbOCw2LCJcXG9wIE5fe0cnfSJdLFs5LDcsIlxcb3AgTl9HIiwyXSxbOCw5LCJtZyJdLFsxMywxNF0sWzE0LDE1LCJcXG9wIE5fRyIsMl0sWzE1LDksIihcXHNpZ21hLTEpIiwyXSxbMTIsMTFdLFsxMSwxMCwiXFxvcCBOX3tHJ30iXSxbMTAsOCwiKFxcc2lnbWEnLTEpIl0sWzEwLDE1LCJtZyJdLFsxMSwxNCwibV4yZyJdXQ==&macro_url=https%3A%2F%2Fraw.githubusercontent.com%2FdFoiler%2Fnotes%2Fmaster%2Fnir.tex
	\[\begin{tikzcd}[cramped]
		\cdots & {\ZZ[G']} & {\ZZ[G']} & {\ZZ[G']} & {\ZZ[G']} & {\ZZ[G']} & \ZZ & 0 \\
		\cdots & {\ZZ[G]} & {\ZZ[G]} & {\ZZ[G]} & {\ZZ[G]} & {\ZZ[G]} & \ZZ & 0
		\arrow[from=1-1, to=1-2]
		\arrow["{\op N_{G'}}", from=1-2, to=1-3]
		\arrow["{m^2g}", from=1-2, to=2-2]
		\arrow["{(\sigma'-1)}", from=1-3, to=1-4]
		\arrow["mg", from=1-3, to=2-3]
		\arrow["{\op N_{G'}}", from=1-4, to=1-5]
		\arrow["mg", from=1-4, to=2-4]
		\arrow["{(\sigma'-1)}", from=1-5, to=1-6]
		\arrow["g", from=1-5, to=2-5]
		\arrow["\deg", from=1-6, to=1-7]
		\arrow["g", from=1-6, to=2-6]
		\arrow[from=1-7, to=1-8]
		\arrow[equals, from=1-7, to=2-7]
		\arrow[from=2-1, to=2-2]
		\arrow["{\op N_G}"', from=2-2, to=2-3]
		\arrow["{(\sigma-1)}"', from=2-3, to=2-4]
		\arrow["{\op N_G}"', from=2-4, to=2-5]
		\arrow["{(\sigma-1)}"', from=2-5, to=2-6]
		\arrow["\deg"', from=2-6, to=2-7]
		\arrow[from=2-7, to=2-8]
	\end{tikzcd}\]
	of chain complexes. Now, given a morphism $f\colon M\to M'$ where $M$ is a $G$-module, and $M'$ has the induced $G'$-module structure, we may apply $\op{Hom}_G(-,M)$ and $\op{Hom}_{G'}(-,M')$ to get another morphism of chain complexes
	% https://q.uiver.app/#q=WzAsMTQsWzAsMCwiMCJdLFsxLDAsIk0iXSxbMiwwLCJNIl0sWzMsMCwiTSJdLFs0LDAsIk0iXSxbNSwwLCJNIl0sWzAsMSwiMCJdLFsxLDEsIk0nIl0sWzIsMSwiTSciXSxbMywxLCJNJyJdLFs0LDEsIk0nIl0sWzUsMSwiTSciXSxbNiwwLCJcXGNkb3RzIl0sWzYsMSwiXFxjZG90cyJdLFswLDFdLFsxLDIsIihcXHNpZ21hLTEpIl0sWzIsMywiXFxvcCBOX0ciXSxbMyw0LCIoXFxzaWdtYS0xKSJdLFs0LDUsIlxcb3AgTl9HIl0sWzYsN10sWzEsNywiZiJdLFs3LDgsIihcXHNpZ21hJy0xKSIsMl0sWzgsOSwiXFxvcCBOX3tHJ30iLDJdLFs5LDEwLCIoXFxzaWdtYSctMSkiLDJdLFsyLDgsImYiXSxbMyw5LCJtZiJdLFs0LDEwLCJtZiJdLFsxMCwxMSwiXFxvcCBOX3tHJ30iLDJdLFs1LDExLCJtXjJmIl0sWzUsMTJdLFsxMSwxM11d&macro_url=https%3A%2F%2Fraw.githubusercontent.com%2FdFoiler%2Fnotes%2Fmaster%2Fnir.tex
	\[\begin{tikzcd}[cramped]
		0 & M & M & M & M & M & \cdots \\
		0 & {M'} & {M'} & {M'} & {M'} & {M'} & \cdots
		\arrow[from=1-1, to=1-2]
		\arrow["{(\sigma-1)}", from=1-2, to=1-3]
		\arrow["f", from=1-2, to=2-2]
		\arrow["{\op N_G}", from=1-3, to=1-4]
		\arrow["f", from=1-3, to=2-3]
		\arrow["{(\sigma-1)}", from=1-4, to=1-5]
		\arrow["mf", from=1-4, to=2-4]
		\arrow["{\op N_G}", from=1-5, to=1-6]
		\arrow["mf", from=1-5, to=2-5]
		\arrow[from=1-6, to=1-7]
		\arrow["{m^2f}", from=1-6, to=2-6]
		\arrow[from=2-1, to=2-2]
		\arrow["{(\sigma'-1)}"', from=2-2, to=2-3]
		\arrow["{\op N_{G'}}"', from=2-3, to=2-4]
		\arrow["{(\sigma'-1)}"', from=2-4, to=2-5]
		\arrow["{\op N_{G'}}"', from=2-5, to=2-6]
		\arrow[from=2-6, to=2-7]
	\end{tikzcd}\]
	induced by $f$ and the above morphism. Taking cohomology, it follows that the induced map $\mathrm H^i(G;M)\to\mathrm H^i(G';M')$ is given by $m^{\floor{i/2}}f$ by a computation on the corresponding cocycles.
\end{proof}
\begin{proposition} % \label{prop:procyclic-cohom}
	Let $\mf g=\widehat{\ZZ}$, a profinite group (the inverse limit $\lim_n\ZZ/n\ZZ$), which is isomorphic to the absolute Galois group of a finite field. Let $M$ be a discrete $\mf g$-module (i.e., the stabilizer of ever $m\in M$ is an open subgroup of $\mf g$; equivalently, $M=\bigcup_HM^H$ is the union of the invariants of open subgroups of $\mf g$). Recall from Milne's notes \cite[Section~II.4]{milne-cft} that the cohomology of a profinite group $G$ acting continuously on $G$ can be computed as the inductive limit, via the inflations, over all open normal subgroups $H$:
	\[\mathrm H^i(G;M)=\colim_H\mathrm H^i(G/H;M^H).\]
	\begin{listalph}
		\item Denote by $\sigma$ the topological generator $1$ in $\mf g$. Let $M'$ be the set of elements annihilated by $1+\sigma+\sigma^2+\cdots+\sigma^n$ for some $n\in\NN$. Show that $\mathrm H^1(\mf g;M)=M'/(\sigma-1)M$. In particular, if $M$ is torsion, we have $\mathrm H^1(\mf g;M)=M/(\sigma-1)M$.
		\item If $M$ is divisible (i.e., the multiplication-by-$n$ map is surjective for all $n\ge1$) or finite torsion, then $\mathrm H^2(\mf g;M)=0$.
		\item Show that $\mathrm H^i(\mf g;\QQ)=0$ for all $i\ge1$. Here, $\QQ$ has the trivial $\mf g$-action.
		\item Show that the group $\mf g$ has cohomological dimension $1$; i.e., for all finite discrete $\mf g$-modules $M$, we have $\mathrm H^i(\mf g;M)=0$ for $i>1$, and there exists $M$ such that $\mathrm H^1(\mf g;M)\ne0$.
		\item Show that for any discrete $\mf g$-module $M$, we have $\mathrm H^i(\mf g;M)=0$ when $i\ge3$.
	\end{listalph}
	% Fix a procyclic group $G$ isomorphic to $\widehat\ZZ$ with generator $\sigma$. Fix a discrete $G$-module $M$. Then
	% \[\mathrm H^i\left(G;M\right) = \begin{cases}
	% 	M^{G} & \text{if }i=0, \\
	% 	M'/(\sigma-1)M & \text{if }i=1, \\
	% 	0 & \text{if }i\ge3,
	% \end{cases}\]
	% where $M'$ is the subset of elements annihilated by $1+\sigma+\cdots+\sigma^n$ for some $n$. Furthermore, if $M$ is divisible or torsion as a group, then $\mathrm H^2(G;M)=0$.
\end{proposition}
\begin{proof}
	We proceed in steps, following \cite[Section~XIII.1]{serre-local-fields}. Before doing anything, we set up some notation around the cohomology of (pro)cyclic groups. For example, let's describe the open subgroups of $\mf g$. Set $H_m\coloneqq\overline{\langle\sigma^m\rangle}$ for brevity, which we note is the kernel of the map $\mf g\to\ZZ/m\ZZ$ defined by $\sigma\mapsto1$, so $H_m$ is a closed and normal subgroup of finite index, so $H_m$ is also open because $\mf g$ is compact. In fact, every open normal subgroup $H\subseteq \mf g$ takes this form: being open, we see that $H$ is finite index because $\mf g$ is compact, and being normal, we see that $H$ must be then be the kernel of some map $\mf g\to A$ where $A$ is a finite group. Replacing $A$ with the (cyclic!) image of $\mf g$, we may find an $m\ge1$ for which $A=\ZZ/m\ZZ$ where $\sigma\in \mf g$ is sent to $1$, so we see that $H=H_m$.

	Thus, we see that
	\[\mathrm H^i(\mf g;M)=\colim_{m\ge1}\mathrm H^i\left(\mf g/H_m;M^{\sigma^m}\right),\]
	where $M^{\sigma^m}=M^{\overline{\langle\sigma^m\rangle}}$ by continuity. Because $\mf g/H_m$ is cyclic, \Cref{lem:compute-group-cohom-cyclic-hw1} tells us that $i>0$ has
	\[\mathrm H^i(\mf g;M)=\begin{cases}
		\colim_{m\ge1}M^\mf g/\op N_{\mf g/H_m}\left(M^{\sigma^m}\right) & \text{if }i\text{ is even}, \\
		\colim_{m\ge1}\ker{\op N_{\mf g/H_m}}/(1-\sigma)M^{\sigma^m} & \text{if }i\text{ is odd}.
	\end{cases}\]
	We now see that we have to use \Cref{lem:compute-group-cohom-cyclic-funct-hw1} to compute the internal maps: the map between the $m$th term and the $m'$th term (where $m\mid m'$) is given by multiplication by $(m'/m)^{\floor{i/2}}$. The various parts of the problem amount to calculations with this.
	\begin{listalph}
		% \item We set up notation. 

		% \item We apply the cohomology of cyclic groups. By plugging in the definitions, we see that
		% \[\mathrm H^i(G;M)=\colim_{m\ge1}\mathrm H^i\left(G/H_m;M^{\sigma^m}\right),\]
		% where $M^{\sigma^m}=M^{\overline{\langle\sigma^m\rangle}}$ by continuity. Because $G/H_m$ is cyclic, \Cref{lem:compute-group-cohom-cyclic-hw1} tells us that $i>0$ has
		% \[\mathrm H^i(G;M)=\begin{cases}
		% 	\colim_{m\ge1}M^G/\op N_{G/H_m}\left(M^{\sigma^m}\right) & \text{if }i\text{ is even}, \\
		% 	\colim_{m\ge1}\ker{\op N_{G/H_m}}/(1-\sigma)M^{\sigma^m} & \text{if }i\text{ is odd}.
		% \end{cases}\]
		% We now see that we have to use \Cref{lem:compute-group-cohom-cyclic-funct-hw1} to compute the internal maps: the map between the $m$th term and the $m'$th term (where $m\mid m'$) is given by multiplication by $(m'/m)^{\floor{i/2}}$.
		% Setting $M_m\coloneqq M^{\sigma^m}$ for brevity, \Cref{lem:compute-group-cohom-cyclic-funct-hw1} tells us that the inclusion $H_{m'}\subseteq H_m$ (which exists whenever $m\mid m'$) induces a map
		% \[\mathrm H^i\left(G/H_m;M_m\right)\to\mathrm H^i\left(G/H_{m'};M_{m'}\right)\]
		% given by $(m'/m)^{\floor{i/2}}$ times the natural inclusion. For example, we see that $i\ge2$ allows us to make this map zero simply by making $m'/m$ divisible by $\#M$, so we conclude that the entire colimit $\mathrm H^i(G;M)$ vanishes.

		\item We go ahead and compute at $i=0$ and $i=1$. At $i=0$, there is nothing to do because the colimit is just $M^\mf g$ at all stages. At $i=1$, we claim that the natural inclusions $\ker{\op N_{{\mf g}/H_m}}\to M'$ induce an isomorphism
		\[\colim_{m\ge1}\frac{\ker{\op N_{{\mf g}/H_m}}}{(1-\sigma)M^{\sigma^m}}\to\frac{M'}{(1-\sigma)M}.\]
		Here are the checks on this map.
		\begin{itemize}
			\item Well-defined: the inclusions assemble into a well-defined map because the internal maps in the colimit are simply induced by inclusion. Indeed, there is no multiplication present because $i=1$.
			\item Surjective: any class $a\in M'$ will be annihilated by some $1+\sigma+\cdots+\sigma^{m-1}$. But then $\left(\sigma^m-1\right)a=0$ as well, so $a\in\ker\op N_{{\mf g}/H_m}$.
			\item Injective: suppose a class $a$ coming from $\ker{\op N_{{\mf g}/H_m}}/(1-\sigma)M^{\sigma^m}$ in the colimit goes to $0$ in $M'/(1-\sigma)M$. In particular, we are given that $a$ is annihilated by some $1+\sigma+\cdots+\sigma^{m-1}$ and also takes the form $(1-\sigma)b$. It follows that $\sigma^mb=b$ as well, so $a\in(1-\sigma)M^{\sigma^m}$, so $a$ vanishes in the colimit already.
		\end{itemize}
		While we're here, we note that if $M$ is torsion, then $M'=M$. Indeed, any given element $a\in M$ is stabilized by some open subgroup $H_m$, meaning $\sigma^ma=a$. Thus,
		\[\left(1+\sigma+\cdots+\sigma^{mn-1}\right)a=n\left(1+\sigma+\cdots+\sigma^{m-1}\right)a\]
		vanishes for some $n$ because $a$ is torsion.

		\item We handle torsion and divisibility separately.
		\begin{itemize}
			\item We show $\mathrm H^2({\mf g};M)=0$ when $M$ is torsion. Indeed, in this case, we are computing the colimit of $M^{\mf g}/\op N_{{\mf g}/H_m}\left(M^{\sigma^m}\right)$, where the transition map between the $m$th and $m'$th term is given by multiplication by $m'/m$. Because $M$ is torsion, for any given class in the $m$th term $M^{\mf g}/\op N_{{\mf g}/H_m}\left(M^{\sigma^m}\right)$, we can select $m'$ sufficiently large so that the multiplication map will kill it. We conclude that the entire colimit must vanish.

			\item We show $\mathrm H^2({\mf g};M)=0$ when $M$ is divisible. We will show that the multiplication-by-$n$ endomorphism on $\mathrm H^2({\mf g};M)$ is injective for all positive integers $n>0$, which will complete the proof because $\mathrm H^2({\mf g};M)$ is a torsion group (because it is the colimit of torsion groups). To see the claim, we note that $n\colon M\to M$ is surjective, so we have a short exact sequence
			\[0\to M[n]\to M\stackrel n\to M\to0\]
			of discrete ${\mf g}$-modules. The long exact sequence (which holds even after the colimit because the colimit is filtered) then shows that
			\[\mathrm H^2({\mf g};M[n])\to\mathrm H^2({\mf g};M)\stackrel n\to\mathrm H^2({\mf g};M)\]
			is exact, so the claim follows from the previous point.
		\end{itemize}

		\item Because $\QQ$ is divisible, we know that $\mathrm H^i(\mf g;\QQ)=0$ for $i=2$ automatically, and we show in part (e) below that $\mathrm H^i(\mf g;\QQ)=0$ for $i\ge3$ automatically as well. Further, we see that no nonzero element $\QQ$ is not annihilated by $1+\sigma+\cdots+\sigma^n$ for any $n$, so $\mathrm H^1(\mf g;\QQ)=0$ as well. We conclude that
		\[\mathrm H^i(\mf g;\QQ)=\begin{cases}
			\QQ & \text{if }i=0, \\
			0 & \text{if }i\ge1.
		\end{cases}\]

		\item Given a finite $\mf g$-module $M$, we see that $M$ is torsion, so $\mathrm H^2(\mf g;M)=0$ automatically. Additionally, we will show in part (e) below that $\mathrm H^i(\mf g;M)=0$ for $i\ge3$ automatically as well.

		It remains to actually find a finite $\mf g$-module $M$ with nonzero $\mathrm H^1(\mf g;M)$. Well, give some finite abelian group $A$ the trivial action. Then $A'=A$ and $(\sigma-1)A=0$, so $\mathrm H^1(\mf g;A)=A$, which is nonzero.
		% By \Cref{lem:compute-group-cohom-cyclic-hw1}, we know that $\mathrm H^i(G;A)=0$ for $i\ge2$ automatically. Continuing, by \Cref{rem:torsion-has-full-procyclic-h1}, we see further that $\mathrm H^1(G;A)=A$. We conclude that
		% \[\mathrm H^i(G;A)=\begin{cases}
		% 	A & \text{if }i\in\{0,1\}, \\
		% 	0 & \text{if }i\ge2.
		% \end{cases}\]

		\item We handle even $i$ and odd $i$ separately, giving the same argument twice. It is possible that one can use dimension-shifting to deduce one case from the other, but this is not a totally trivial matter because $\mf g$ is infinite.
		\begin{itemize}
			\item We show $\mathrm H^i({\mf g};M)=0$ when $i$ is odd and $i\ge3$. Write $i=2j+1$ for $j\ge1$ so that $\floor{i/2}=j$. Then
			\[\mathrm H^i({\mf g};M)=\colim_{m\ge1}\frac{\ker\op N_{{\mf g}/H_m}}{(1-\sigma)M^{\sigma^m}},\]
			and the transition maps are given by multiplication by $(m'/m)^j$. It is enough to show that any class $a\in\ker\op N_{{\mf g}/H_m}$ vanishes in the colimit. Well, $m\mathrm H^1({\mf g}/H_m;M^{\sigma^m})=0$, so we see that $ma$ must vanish in this cohomology group, so $ma=(1-\sigma)b$ for some $b\in M^{\sigma^m}$. Thus, we see that $a$ vanishes along the transition map
			\[\frac{\ker\op N_{{\mf g}/H_m}}{(1-\sigma)M^{\sigma^m}}\stackrel{m^{j}}\to\frac{\ker\op N_{{\mf g}/H_{m^2}}}{(1-\sigma)M^{\sigma^{m^2}}}\]
			because $m^{j}a=(1-\sigma)m^{j-1}b$.
			\item We show $\mathrm H^i({\mf g};M)=0$ when $i$ is even and $i\ge4$. Write $i=2j$ for $j\ge2$ so that $\floor{i/2}=j$. Then
			\[\mathrm H^i({\mf g};M)=\colim_{m\ge1}\frac{M^{\mf g}}{\op N_{{\mf g}/H_m}(M^{\sigma^m})},\]
			and the transition maps are given by multiplication by $(m'/m)^j$. Once again, it is enough to show that any class $a\in M^{\mf g}$ vanishes in the colimit. Well, find $m$ with $\sigma^ma=a$, so $m\mathrm H^2({\mf g}/H_m;M^{\sigma^m})=0$ implies that $ma=\op N_{{\mf g}/H_m}(b)$ for some $b\in M^{\sigma^m}$. Then $\op N_{{\mf g}/H_{mk}}(b)=\op N_{{\mf g}/H_k}\op N_{{\mf g}/H_m}(b)=kma$, so $a$ vanishes along the transition map
			\[\frac{M^{\mf g}}{\op N_{{\mf g}/H_m}(M^{\sigma^m})}\to\frac{M^{\mf g}}{\op N_{{\mf g}/H_{m^2}}(M^{\sigma^{m^2}})}\]
			because $m^ja=m^{j-2}\op N_{{\mf g}/H_{m^2}}(b)$.
			\qedhere
		\end{itemize}
	\end{listalph}
\end{proof}
% \begin{remark}
% 	Equivalently, when $M$ is finite, the cohomology of $M$ is computed via the two-term complex
% 	\[0\to M\stackrel{\sigma-1}\to M\to 0.\]
% \end{remark}
% \begin{remark}
% 	If $M$ is torsion, then we claim that $M'=M$. Indeed, any given element $a\in M$ is stabilized by some open subgroup $H_m$, meaning $\sigma^ma=a$. Thus,
% 	\[\left(1+\sigma+\cdots+\sigma^{mn-1}\right)a=n\left(1+\sigma+\cdots+\sigma^{m-1}\right)a\]
% 	vanishes for some $n$ because $a$ is torsion.
% \end{remark}
% \begin{example}
% 	Set $G\coloneqq\widehat{\ZZ}$ with generator $\sigma$, and give $\QQ$ the trivial action. Because $\QQ$ is divisible, \Cref{prop:procyclic-cohom} tells us that $\mathrm H^i(G;\QQ)=0$ for $i\ge2$ automatically. Further, we see that no nonzero element $\QQ$ is not annihilated by $1+\sigma+\cdots+\sigma^n$ for any $n$, so $\mathrm H^1(G;\QQ)=0$ as well. We conclude that
% 	\[\mathrm H^i(G;\QQ)=\begin{cases}
% 		\QQ & \text{if }i=0, \\
% 		0 & \text{if }i\ge1.
% 	\end{cases}\]
% \end{example}
% \begin{example}
% 	Set $G\coloneqq\widehat{\ZZ}$ with generator $\sigma$,and give the finite abelian group $A$ the trivial action. By \Cref{lem:compute-group-cohom-cyclic-hw1}, we know that $\mathrm H^i(G;A)=0$ for $i\ge2$ automatically. Continuing, by \Cref{rem:torsion-has-full-procyclic-h1}, we see further that $\mathrm H^1(G;A)=A$. We conclude that
% 	\[\mathrm H^i(G;A)=\begin{cases}
% 		A & \text{if }i\in\{0,1\}, \\
% 		0 & \text{if }i\ge2.
% 	\end{cases}\]
% \end{example}
% \begin{example}
% 	Set $G\coloneqq\widehat{\ZZ}$ with generator $\sigma$,and give the abelian group $\ZZ$ the trivial action. By \Cref{lem:compute-group-cohom-cyclic-hw1}, we know that $\mathrm H^i(G;A)=0$ for $i\ge3$ automatically; similarly, we see that no nonzero element is annihilated by $1+\sigma+\cdots+\sigma^n$ for any $n$, so $\mathrm H^1(G;\ZZ)=0$. Continuing, as in the proof of \Cref{lem:compute-group-cohom-cyclic-hw1}, we see that $\mathrm H^2(G;\ZZ)$ is the colimit of the quotients $\ZZ/m\ZZ$, where the transition maps $\ZZ/m\ZZ\to\ZZ/m'\ZZ$ are given by multiplication by $m'/m$; equivalently, this is the inclusion $\frac1m\ZZ/\ZZ\to\frac1{m'}\ZZ/\ZZ$. We conclude that
% 	\[\mathrm H^i(G;A)=\begin{cases}
% 		\ZZ & \text{if }i=0, \\
% 		\QQ/\ZZ & \text{if }i=2, \\
% 		0 & \text{otherwise}.
% 	\end{cases}\]
% \end{example}

\section{Local Cohomology}
\begin{proposition} \label{prop:local-cohom-calculations-hw1}
	In this question, $K$ is a finite extension of $\QQ_p$, $\Gamma_K\coloneqq\op{Gal}(\ov K/K)$, $M$ is a finite $\Gamma_K$-module, and we set $\mathrm H^i(M)\coloneqq\mathrm H^i(K;M)=\mathrm H^i(\Gamma_K;M)$. Then local Tate duality asserts that the cup product induces a perfect pairing between two finite abelian groups
	\[\cup\colon\mathrm H^i(K;M)\times\mathrm H^{2-i}(K;M^*)\to\QQ/\ZZ,\]
	where $M^*\coloneqq\op{Hom}(M,\mu_\infty)$ and $\mu_\infty\coloneqq\bigcup_n\mu_n$. The Euler--Poincar\'e characteristic is defined for any finite $\Gamma_K$-module $M$ as
	\[\chi(M)\coloneqq\frac{h^0(M)h^2(M)}{h^1(M)},\]
	where $h^i\coloneqq\#\mathrm H^i(M)$. Then we have a formula
	\[\chi(M)=\frac1{\#(\OO_K/n\OO_K)},\]
	where $n\coloneqq\#M$.
	\begin{listalph}
		\item How many elements does $K^\times/K^{\times2}$ have? How many different quadratic extensions does $K$ have? In general, how many elements does $K^\times/K^{\times\ell}$ have a prime $\ell$?
		\item Let $E$ be an elliptic curve over $K$ and $\ell$ a prime number (we allow $\ell=p$). Then it is known that the Weil pairing induces a $\Gamma_K$-isomorphism $E[\ell]\cong E[\ell]^*$. Show that, regardless of the reduction type of $E$, we always have
		\[\dim_{\FF_\ell}E(K)/\ell E(K)=\frac12\dim_{\FF_\ell}\mathrm H^1(K;E[\ell]).\]
		What are the possible values of $\dim_{\FF_\ell}E(K)/\ell E(K)$ if $K=\QQ_p$?
	\end{listalph}
\end{proposition}
\begin{proof}[Proof of \Cref{prop:local-cohom-calculations-hw1}(a)]
	Let $v$ be the place of $K$. Let's start by computing the size of $K^\times/K^{\times\ell}$ for primes $\ell$. By the Kummer exact sequence
	\[0\to\mu_\ell\to\mathbb G_m\stackrel\ell\to\mathbb G_m\to0\]
	and Hilbert's theorem 90, we conclude that the boundary map $K^\times/K^{\times\ell}\to\mathrm H^1(K;\mu_\ell)$ is an isomorphism. Thus, we may focus on computing the size of $\mathrm H^1(K;\mu_\ell$, for which we use the Euler characteristic. Indeed, we know that $h^1(\mu_\ell)$ is
	\[h^1(\mu_\ell)=h^0(\mu_\ell)h^2(\mu_\ell)\cdot\#(\OO_K/\ell\OO_K)\]
	after plugging in for $\chi(\mu_\ell)$. We will compute each of these terms separately.
	\begin{itemize}
		\item For $h^0(\mu_\ell)$, note $\mathrm H^0(K;\mu_\ell)=\mu_\ell(K)$.
		% We record two cases: if $v\mid\ell$, then we will simply remember that this is $\#\mu_\ell(K)$. But if $v\nmid\ell$, then Hensel's lemma tells us that solutions to $X^\ell-1$ come from the residue field $\FF_v$, so
		% \[\dim_{\FF_\ell}\mu_\ell(K)=\begin{cases}
		% 	1 & \text{if }\ell\nmid\#\FF_v-1, \\
		% 	0 & \text{otherwise}.
		% \end{cases}\]
		\item For $h^2(\mu_\ell)$, we claim that $\mu_\ell^*=\ZZ/\ell\ZZ$, which implies that $h^2(\mu_\ell^*)$ is
		\[h^0(\ZZ/\ell\ZZ)=\ell\]
		by local duality. To show the claim, recall $\mu_\ell^*=\op{Hom}(\mu_\ell,\mu_\infty)$. Any map $\mu_\ell\to\mu_\infty$ factors through $\mu_\infty[\ell]=\mu_\ell$, so $\mu_\ell^*=\op{Hom}(\mu_\ell,\mu_\ell)$. Now, a map $\mu_\ell\to\mu_\ell$ must be of the form $\zeta\mapsto\zeta^k$ for some integer $k$, so there is a surjection $\ZZ\to\mu_\ell^*$ with kernel $\ell\ZZ$. The induced map $\ZZ/\ell\ZZ\to\mu_\ell^*$ is also Galois-invariant because all the given homomorphisms $\mu_\ell\to\mu_\ell$ are automatically Galois-invariant.
		\item For $\#(\OO_K/\ell\OO_K)$, we have two cases. If $v\nmid\ell$, then $\ell$ is a unit in $\OO_K$, so this quotient is trivial. Otherwise, if $v\mid\ell$, then $\OO_K\cong\ZZ_\ell^{[K:\QQ_p]}$ as an abelian group, so this quotient is isomorphic to 
	\end{itemize}
	In total, we conclude that
	\[\#\left(K^\times/K^{\times\ell}\right)=\#\mu_\ell(K)\cdot\ell\cdot\ell^{[K:\QQ_p]\cdot1_{v\mid\ell}}.\]
	For example,
	\[\#\left(K^\times/K^{\times2}\right)=\begin{cases}
		4 & \text{if }v\nmid2, \\
		4\cdot2^{[K:\QQ_2]} & \text{if }v\mid2.
	\end{cases}\]
	Lastly, we note that the number of quadratic extensions of $K$ are in bijection with $K^\times/K^{\times2}$ by Kummer theory. (Slightly more explicitly, a class $\alpha\in K^\times/K^{\times2}$ gives rise to a quadratic extension $K(\sqrt\alpha)$, and these extensions are unique. In characteristic $0$, completing the square shows that every quadratic extension arises in this way.) Thus, the preceding equation also computes the number of quadratic extensions of $K$.
\end{proof}
For the proof of (b), we will require the following fact.
For the subsequent argument, we will need the following fact.
\begin{lemma} \label{lem:e-has-finite-index-o-hw1}
	Fix an elliptic curve $E$ over a nonarchimedean local field $K_v$. Then $E(K_v)$ admits a finite-index subgroup isomorphic to $\OO_v$.
\end{lemma}
\begin{proof}
	The proof of this result is fairly involved, so we will be sketchy; we refer to \cite[Proposition~VII.6.3]{silverman} for more details. Let $E_0(K_v)\subseteq E(K)$ denote the collection of points which reduce to a non-singular point in $E(\FF_v)$, and we let $E_1(K_v)\subseteq E(K)$ denote the collection of points which reduce to the identity of $E(\FF_v)$. 
	
	The main point is to show that $E(K_v)/E_0(K_v)$ is finite, but for now let's explain why it completes the proof. It turns out that the canonical maps
	\[0\to E_1(K_v)\to E_0(K_v)\to E(\FF_v)\to0\]
	assemble into a short exact sequence \cite[Proposition~VII.2.1]{silverman}. Thus, it is enough to show that $E_1(K_v)$ admits a finite-index subgroup isomorphic to $\OO_v$. Well, $E_1(K_v)$ is isomorphic to $G_E(\mf m_v)$, where $G_E$ is the one-dimensional formal group of $E$ \cite[Proposition~VII.2.2]{silverman}. Then the canonical filtration $G_E(\mf m_v^\bullet)$ shows that $G_E(\mf m_v^i)/G_E\left(\mf m_v^{i+1}\right)$ is finite for all $i$, and for $i$ large enough, there is a logarithm map establishing that $G_E(\mf m_v^i)$ is isomorphic to $\OO_v$. This completes the proof modulo the finiteness of $E(K_v)/E_0(K_v)$.

	It remains to show that $E(K_v)/E_0(K_v)$, for which we follow \cite[Exercise~7.6]{silverman}. Because $K_v$ is a topological field, we see that $E(K_v)\subseteq\PP^2(K_v)$ is a topological group. In fact, $E(K_v)\subseteq\PP^2(K_v)$ is a closed subset of the compact space $\PP^2(K_v)$, so $E(K_v)$ is compact. The reduction map $\PP^2(K_v)\to\PP^2(\FF_v)$ is a continuous map to a finite discrete space, so $E_0(K_v)\subseteq E(K_v)$ is an open subgroup. Compactness then forces $E_0(K_v)$ to be finite-index in $E(K_v)$.
\end{proof}
\begin{proof}[Proof of \Cref{prop:local-cohom-calculations-hw1}(b)]
	We refer to \cite[Proposition~III.8.1]{silverman} (also recorded in the notes) for the fact that the Weil pairing induces a Galois-invariant isomorphism $E[\ell]\to E[\ell]^*$. This follows from the fact that the Weil pairing $E[\ell]\times E[\ell]\to\mu_\ell$ is non-degenerate and Galois-invariant.

	We will calculate both sides of the required equality
	\[\dim_{\FF_\ell}E(K)/\ell E(K)\stackrel?=\frac12\dim_{\FF_\ell}\mathrm H^1(K;E[\ell])\]
	separately.
	\begin{itemize}
		\item We compute $\dim_{\FF_\ell}E(K)/\ell E(K)$, using \Cref{lem:e-has-finite-index-o-hw1}, which grants us an exact sequence
		\[0\to\OO_v\to E(K)\to C\to0,\]
		where $C$ is some finite abelian group. This exact sequence has an endomorphism given by multiplication by $\ell$. Applying the Snake lemma to this endomorphism yields the exact sequence
		\[0\to\OO_v[\ell]\to E(K)[\ell]\to C[\ell]\to\frac{\OO_v}{\ell\OO_v}\to\frac{E(K)}{\ell E(K)}\to\frac C{\ell C}\to0.\]
		The calculation will follow by taking dimensions of this exact sequence; for example, just the right-exact part of this sequence immediately shows us that $E(K)/\ell E(K)$ is finite. We thus have
		\[\dim_{\FF_\ell}\frac{E(K)}{\ell E(K)}-\dim_{\FF_\ell}E(K)[\ell]=\dim_{\FF_\ell}\frac{\OO_v}{\ell\OO_v}-\dim_{\FF_\ell}\OO_v[\ell]+\dim_{\FF_\ell}\frac C{\ell C}-\dim_{\FF_\ell}C[\ell].\]
		Now, we note that $\OO_v[\ell]=0$ because $K$ has characteristic $0$. Continuing, and $\#C_\ell=\#(C/\ell C)$ because the kernel and cokernel of the endomorphism $\ell\colon C\to C$ should have the same size. We are left with
		\[\dim_{\FF_\ell}\frac{E(K)}{\ell E(K)}=\dim_{\FF_\ell}\frac{\OO_v}{\ell\OO_v}+\dim_{\FF_\ell}E(K)[\ell].\]
		To compute $\OO_K/\ell\OO_K$, there are two cases: if $v\nmid\ell$, then this is trivial; otherwise, $\OO_K\cong\ZZ_\ell^{[K:\QQ_\ell]}$, so in total,
		\[\dim_{\FF_\ell}\frac{E(K)}{\ell E(K)}=\dim_{\FF_\ell}E(K)[\ell]+\begin{cases}
			0 & \text{if }v\nmid\ell, \\
			[K:\QQ_\ell] & \text{if }v\mid\ell.
		\end{cases}.\]
		% We now note that $\ell=p$ implies that $\OO_v/\ell\OO_v\cong\ZZ/\ell\ZZ$; otherwise, $\OO_v/\ell\OO_v=0$.

		\item We compute $\dim_{\FF_\ell}\mathrm H^1(K;E[\ell])$, using the Euler characteristic. Indeed, by plugging everything in, we see that
		\[h^1(E[\ell])=h^0(E[\ell])h^2(E[\ell])\cdot\#\left(\OO_K/\ell^2\OO_K\right).\]
		(Recall $E[\ell]$ has size $\ell^2$ over the algebraic closure.) Because $E[\ell]\cong E[\ell]^*$, it follows that $h^0(E[\ell])=h^2(E[\ell])$ by local duality. Thus, noting $h^0(E[\ell])=E(K)[\ell]$, we find that
		\[h^1(E[\ell])=\#E(K)[\ell]^2\cdot\#\left(\OO_K/\ell^2\OO_K\right).\]
		As in the previous point, we can calculate $\OO_K/\ell^2\OO_K$ by cases for when $v\nmid\ell$ or $v\mid\ell$, finding that
		\[\dim_{\FF_\ell}\mathrm H^1(K;E[\ell])=2\dim_{\FF_\ell}E(K)[\ell]+\begin{cases}
			0 & \text{if }v\nmid\ell, \\
			2[K:\QQ_\ell] & \text{if }v\mid\ell.
		\end{cases}.\]
	\end{itemize}
	The equality
	\[\dim_{\FF_\ell}E(K)/\ell E(K)=\frac12\dim_{\FF_\ell}\mathrm H^1(K;E[\ell])\]
	now follows by comparing the above two calculations.

	We now move to $K=\QQ_p$. The above calculation has showed that
	\[\dim_{\FF_\ell}E(\QQ_p)/\ell E(\QQ_p)=\dim_{\FF_\ell}E(\QQ_p)[\ell]+1_{p=\ell}.\]
	Now, $E(\ov\QQ_p)[\ell]$ is isomorphic to $(\ZZ/\ell\ZZ)^2$, so $E(\QQ_p)[\ell]$ has dimension in $\{0,1,2\}$. We conclude that
	\[\dim_{\FF_\ell}E(\QQ_p)/\ell E(\QQ_p)\in\begin{cases}
		\{0,1,2\} & \text{if }p\ne\ell, \\
		\{1,2,3\} & \text{if }p=\ell,
	\end{cases}\]
	as needed.
\end{proof}
\begin{remark}
	In the notes, we show the stronger statement that $\mathrm H^1(K;E[m])$ is a non-degenerate quad\-ratic space (with the pairing induced by the Weil pairing), and $E(K)/mE(K)$ embeds as a maximal isotropic subspace. Note that $m$ is not required to be prime!
\end{remark}
% \begin{proposition} \label{prop:local-is-lagrangian}
% 	Fix a local field $K$ of characteristic $0$ and a positive integer $m$.
% 	\begin{listalph}
% 		\item The Weil pairing makes $\mathrm H^1(K;E[m])$ a non-degenerate quadratic space. Explicitly, this is a quad\-ratic space over $\ZZ/m\ZZ$ if $K$ is nonarchimedean and over $\ZZ/\gcd(m,2)\ZZ$ if $K$ is archimedean.
% 		\item The submodule $E(K)/mE(K)$ of $\mathrm H^1(K;E[m])$ is maximal isotropic.
% 	\end{listalph}
% \end{proposition}
% \begin{proof}
% 	These claims follow from local duality. For (a), note the Weil pairing on $E[m]$ is perfect and Galois-invariant, so it induces an isomorphism $E[m]\cong\op{Hom}(E[m],\mu_m)$ of (finite discrete) Galois modules. Thus, we can think of the pairing on $\mathrm H^1(K;E[m])$ as fitting into the commutative diagram
% 	% https://q.uiver.app/#q=WzAsMTAsWzAsMCwiXFxtYXRocm0gSF4xKEs7RVttXSkiXSxbMSwwLCJcXG1hdGhybSBIXjEoSztFW21dKSJdLFsxLDEsIlxcbWF0aHJtIEheMShLO0VbbV0pIl0sWzAsMSwiXFxtYXRocm0gSF4xKEs7RVttXV4qKSJdLFsyLDAsIlxcbWF0aHJtIEheMihLO1xcbXVfbSkiXSxbMiwxLCJcXG1hdGhybSBIXjIoSztcXG11X20pIl0sWzMsMCwiKGNfMSxjXzIpIl0sWzQsMCwiZV9tKGNfMVxcY3VwIGNfMikiXSxbMywxLCIoZ1xcbWFwc3RvIGVfbShjXzEoZyksLSksY18yKSJdLFs0LDEsIihnLGgpXFxtYXBzdG8gZV9tKGNfMShnKSxnY18yKGgpKSJdLFswLDEsIlxcdGltZXMiLDMseyJzdHlsZSI6eyJib2R5Ijp7Im5hbWUiOiJub25lIn0sImhlYWQiOnsibmFtZSI6Im5vbmUifX19XSxbMywyLCJcXHRpbWVzIiwzLHsic3R5bGUiOnsiYm9keSI6eyJuYW1lIjoibm9uZSJ9LCJoZWFkIjp7Im5hbWUiOiJub25lIn19fV0sWzEsMiwiIiwwLHsibGV2ZWwiOjIsInN0eWxlIjp7ImhlYWQiOnsibmFtZSI6Im5vbmUifX19XSxbMCwzLCJlX20iXSxbMSw0LCJcXGN1cCJdLFsyLDUsIlxcY3VwIl0sWzQsNSwiIiwxLHsibGV2ZWwiOjIsInN0eWxlIjp7ImhlYWQiOnsibmFtZSI6Im5vbmUifX19XSxbNiw4LCIiLDEseyJzdHlsZSI6eyJ0YWlsIjp7Im5hbWUiOiJtYXBzIHRvIn19fV0sWzYsNywiIiwxLHsic3R5bGUiOnsidGFpbCI6eyJuYW1lIjoibWFwcyB0byJ9fX1dLFs4LDksIiIsMSx7InN0eWxlIjp7InRhaWwiOnsibmFtZSI6Im1hcHMgdG8ifX19XSxbNyw5LCIiLDEseyJzdHlsZSI6eyJ0YWlsIjp7Im5hbWUiOiJtYXBzIHRvIn19fV1d&macro_url=https%3A%2F%2Fraw.githubusercontent.com%2FdFoiler%2Fnotes%2Fmaster%2Fnir.tex
% 	\[\begin{tikzcd}[cramped, column sep=small]
% 		{\mathrm H^1(K;E[m])} & {\mathrm H^1(K;E[m])} & {\mathrm H^2(K;\mu_m)} & {(c_1,c_2)} & {e_m(c_1\cup c_2)} \\
% 		{\mathrm H^1(K;E[m]^*)} & {\mathrm H^1(K;E[m])} & {\mathrm H^2(K;\mu_m)} & {(g\mapsto e_m(c_1(g),-),c_2)} & {(g,h)\mapsto e_m(c_1(g),gc_2(h))}
% 		\arrow["\times"{marking, allow upside down}, draw=none, from=1-1, to=1-2]
% 		\arrow["{e_m}", from=1-1, to=2-1]
% 		\arrow["\cup", from=1-2, to=1-3]
% 		\arrow[equals, from=1-2, to=2-2]
% 		\arrow[equals, from=1-3, to=2-3]
% 		\arrow[maps to, from=1-4, to=1-5]
% 		\arrow[maps to, from=1-4, to=2-4]
% 		\arrow[maps to, from=1-5, to=2-5]
% 		\arrow["\times"{marking, allow upside down}, draw=none, from=2-1, to=2-2]
% 		\arrow["\cup", from=2-2, to=2-3]
% 		\arrow[maps to, from=2-4, to=2-5]
% 	\end{tikzcd}\]
% 	where the bottom row is perfect by local duality. Thus, the induced pairing is perfect, and we see that we actually have a quadratic space over $\ZZ/m\ZZ$ because the target of the pairing is $\mathrm H^2(K;\mu_m)\cong\ZZ/m\ZZ$ by local class field theory.

% 	% We omit the proof of (b) because it is on the homework.
% 	We now move on to (b), following \cite[Theorem~8.2]{tate-galois-cohomology}. We begin by recalling the Kummer short exact sequence
% 	\[0\to\frac{E(K)}{mE(K)}\to\mathrm H^1(K;E[m])\to\mathrm H^1(K;E)[m]\to0\]
% 	which is induced by the short exact sequence $0\to E[m]\to E\to E\to0$. Roughly speaking, the result will follow by comparing this short exact sequence with its dual. Recall that $E(K)/mE(K)$ is an isotropic subspace of $\mathrm H^1(K;E[m])$, so the Weil pairing descends to a pairing
% 	\[\frac{E(K)}{mE(K)}\times\mathrm H^1(K;E)[m]\to\mathrm H^2(K;\mu_m),\]
% 	where we are viewing $\mathrm H^1(K;E)[m]$ as the quotient of $\mathrm H^1(K;E[m])$ by $E(K)/mE(K)$.
	
% 	The main claim is that this last Weil pairing is perfect. For now, let's content ourselves with explaining why this completes the proof. We would like to show that a class $c\in\mathrm H^1(K;E[m])$ has $x\cup c=0$ for all $x\in E(K)/mE(K)$ if and only if $c\in E(K)/mE(K)$. In other words, the functional $-\cup c$ on $E(K)/mE(K)$ vanishes if and only if $c$ vanishes in the quotient $\mathrm H^1(K;E)[m]$, which is exactly the main claim.
	
% 	We now show the main claim. Let $(-)^\lor\coloneqq\op{Hom}_\ZZ(-,\QQ/\ZZ)$ denote the Pontryagin dual. Then we note that the Weil pairing induces a morphism
% 	% https://q.uiver.app/#q=WzAsMTAsWzAsMCwiMCJdLFsxLDAsIlxcZGlzcGxheXN0eWxlXFxmcmFje0UoSyl9e21FKEspfSJdLFsyLDAsIlxcbWF0aHJtIEheMShLO0VbbV0pIl0sWzMsMCwiXFxtYXRocm0gSF4xKEs7RSlbbV0iXSxbNCwwLCIwIl0sWzAsMSwiMCJdLFsxLDEsIlxcbWF0aHJtIEheMShLO0UpW21dXlxcbG9yIl0sWzIsMSwiXFxtYXRocm0gSF4xKEs7RVttXSleXFxsb3IiXSxbMywxLCJcXGxlZnQoXFxkaXNwbGF5c3R5bGVcXGZyYWN7RShLKX17bUUoSyl9XFxyaWdodCleXFxsb3IiXSxbNCwxLCIwIl0sWzAsMV0sWzEsMl0sWzIsM10sWzMsNF0sWzUsNl0sWzYsN10sWzcsOF0sWzgsOV0sWzEsNl0sWzIsN10sWzMsOF1d&macro_url=https%3A%2F%2Fraw.githubusercontent.com%2FdFoiler%2Fnotes%2Fmaster%2Fnir.tex
% 	\[\begin{tikzcd}[cramped]
% 		0 & {\displaystyle\frac{E(K)}{mE(K)}} & {\mathrm H^1(K;E[m])} & {\mathrm H^1(K;E)[m]} & 0 \\
% 		0 & {\mathrm H^1(K;E)[m]^\lor} & {\mathrm H^1(K;E[m])^\lor} & {\left(\displaystyle\frac{E(K)}{mE(K)}\right)^\lor} & 0
% 		\arrow[from=1-1, to=1-2]
% 		\arrow[from=1-2, to=1-3]
% 		\arrow[from=1-2, to=2-2]
% 		\arrow[from=1-3, to=1-4]
% 		\arrow[from=1-3, to=2-3]
% 		\arrow[from=1-4, to=1-5]
% 		\arrow[from=1-4, to=2-4]
% 		\arrow[from=2-1, to=2-2]
% 		\arrow[from=2-2, to=2-3]
% 		\arrow[from=2-3, to=2-4]
% 		\arrow[from=2-4, to=2-5]
% 	\end{tikzcd}\]
% 	of short exact sequences. We showed in (a) that the middle morphism is an isomorphism, so it follows that the left morphism is injective, and the right morphism is surjective. But the size of a finite abelian group and its Pontryagin dual are the same, so it follows that the left and right morphisms are isomorphisms, which is what we wanted to show!
% \end{proof}
% \begin{example} \label{ex:size-im-delta-v}
% 	Fix an elliptic curve $E$ over a number field $K$ which is the projective closure of $y^2=(x-a_1)(x-a_2)(x-a_3)$. Then
% 	\[\dim_{\FF_2}E(K_v)/2E(K_v)=\begin{cases}
% 		0 & \text{if }K_v=\CC, \\
% 		1 & \text{if }K_v=\RR, \\
% 		2 & \text{if }v\text{ is odd}, \\
% 		2+[K_v:\QQ_2] & \text{if }v\text{ is even}.
% 	\end{cases}\]
% \end{example}
% \begin{proof}
% 	Note the image of $E(K_v)/2E(K_v)\to\mathrm H^1(K_v;E[2])$ is maximal isotropic and therefore should have dimension equal to
% 	\[\frac12\dim_{\FF_2}\mathrm H^1(K_v;E[2])=\dim_{\FF_2}\mathrm H^1(K_v;\mu_2).\]
% 	Because $\mathrm H^1(K_v;\mu_2)=K_v^\times/K_v^{\times2}$, we are left to compute $K_v^\times/K_v^{\times2}$. In the archimedean cases, we directly see that $\CC^\times/\CC^{\times2}=1$ (because $\CC$ is algebraically closed) and $\RR^\times/\RR^{\times2}=\RR^\times/\RR^+=\{\pm1\}$.
	
% 	Otherwise, we suppose that $K$ is a finite extension of $\QQ_p$, and we claim that
% 	\[K_v^\times\cong\ZZ\times\FF_v\times\mu_{p^\infty}(K_v)\times\OO_v\]
% 	as abelian groups. To begin, note $K_v^\times\cong\ZZ\times\OO_v^\times$ by using the valuation; additionally, by modding out by $\mf p_v$, we find that $\OO_v^\times\cong\FF_v\times(1+\mf p_v)$.
	
% 	Now, recall that the exponential map $\exp\colon\mf p_v\to(1+\mf p_v)$ identifies open neighborhoods of the identity of $K_v$ and $K_v^\times$, so it follows that $\OO_v^\times$ is a finitely generated $\OO_v$-module. Because $\OO_v$ is a principal ideal domain, it follows that $\OO_v^\times$ is isomorphic to its torsion times its free part. The free part of $\OO_v^\times$ has rank $1$ because the exponential map identifies a finite-index open subgroup with $\OO_v$. Lastly, the torsion of $(1+\mf p_v)$ must be $p$-power (because $(1+\varpi)^n\equiv1+n\varpi\pmod{\mf p_v^{2m}}$ for any $\varpi\in\mf p^m$), and conversely, the $p$-power torsion of $K_v^\times$ all lives in $\OO_v^\times$ by looking at the valuation and is in fact $1\pmod{\mf p_v}$ by looking$\pmod{\mf p_v}$. The claim follows.

% 	To complete the calculation, we write
% 	\[\frac{K_v^\times}{K_v^{\times2}}\cong\frac{\ZZ}{2\ZZ}\times\frac{\FF_v^\times}{\FF_v^{\times2}}\times\frac{\mu_{p^\infty}(K_v)}{\mu_{p^\infty}(K_v)^2}\times\frac{\OO_v}{2\OO_v}.\]
% 	Continuing, we note that the kernel and cokernel of the endomorphisms $2\colon\FF_v^\times\to\FF_v^\times$ and $2\colon\mu_{2^\infty}(K_v)\to\mu_{2^\infty}(K_v)$ have the same size, so we see
% 	\[\dim_{\FF_2}\frac{K_v^\times}{K_v^{\times2}}\cong1+\dim_{\FF_2}\FF_v^\times[2]+\dim_{\FF_2}\mu_{p^\infty}(K_v)[2]+\dim_{\FF_2}\frac{\OO_v}{2\OO_v}.\]
% 	We now have two cases.
% 	\begin{itemize}
% 		\item If $v$ is odd, then $\FF_v^\times[2]\cong\ZZ/2\ZZ$ and $\mu_{p^\infty}(K_v)[2]=0$ and $\OO_v/2\OO_v=0$. In total, we get $2$ dimensions.
% 		\item If $v$ is even, then $\FF_v^\times[2]=0$ and $\mu_{p^\infty}(K_v)[2]=\mu_2(K_v)=\{\pm1\}$ and $\OO_v/2\OO_v\cong(\ZZ/2\ZZ)^{[K_v:\QQ_2]}$. In total, we get $2+[K_v:\QQ_2]$ dimensions.
% 		\qedhere
% 	\end{itemize}
% 	%  Otherwise, if $v$ is even, then $\FF_v^\times/\FF_v^{\times2}=1$, and $\OO_v/2\OO_v\cong(\ZZ/2\ZZ)^{[K_v:\QQ_2]}$; additionally, the kernel and cokernel of $2\colon\mu_{2^\infty}(K_v)\to\mu_{2^\infty}(K_v)$ have the same size. Totaling these calculations completes.
% \end{proof}
% \begin{remark}
% 	For an odd prime $\ell$, we see that $K_v^\times/K_v^{\times\ell}$ is always $\{1\}$ when $K_v\in\{\RR,\CC\}$. When $v$ is finite and $p$-adic, we similarly find that
% 	\[\dim_{\FF_\ell}\frac{K_v^\times}{K_v^{\times\ell}}=1+\dim_{\FF_\ell}\FF_v[\ell]+\dim_{\FF_\ell}\mu_{p^\infty}(K_v)[\ell]+\dim_{\FF_\ell}\frac{\OO_v}{\ell\OO_v}.\]
% 	Here are our two cases.
% 	\begin{itemize}
% 		\item If $v\nmid\ell$, then the dimension is $1+1_{\ell\mid\#\FF_v-1}+0+0$.
% 		\item If $v\mid\ell$, then the dimension is $1+0+1_{\zeta_p\in K_v}+[K_v:\QQ_p]$.
% 	\end{itemize}
% 	% If $v\nmid\ell$, then the dimension is $1+1+0+0=2$ when $\ell\mid\#\FF_v-1$ and $1+0+0+0=1$ otherwise. Otherwise, if $v\mid\ell$, then the dimension is $1+0+\dim_{\FF_\ell}\mu_{\ell^\infty}(K_v)$
% \end{remark}
% However, with a little more theory, one can say more.
% \begin{proposition}
% 	Fix an elliptic curve $E$ over field $\QQ_p$. For any prime $\ell$,
% 	\[\dim_{\FF_\ell}E(\QQ_p)/\ell E(\QQ_p)=\dim_{\FF_\ell}E(\QQ_p)[\ell]+\begin{cases}
% 		1 & \text{if }p=\ell, \\
% 		0 & \text{if }p\ne\ell.
% 	\end{cases}\]
% 	Thus, $\dim_{\FF_\ell}E(\QQ_p)/\ell E(\QQ_p)\in\{0,1,2,3\}$, where $3$ is only possible when $p=\ell$.
% \end{proposition}
% \begin{proof}
% 	Set $K_v\coloneqq\QQ_\ell$ for brevity. We will use \Cref{lem:e-has-finite-index-o-hw1}, which grants us an exact sequence
% 	\[0\to\OO_v\to E(K_v)\to C\to0,\]
% 	where $C$ is some finite abelian group. This exact sequence has an endomorphism given by multiplication by $\ell$. Applying the Snake lemma to this endomorphism yields the exact sequence
% 	\[0\to\OO_v[\ell]\to E(K_v)[\ell]\to C[\ell]\to\frac{\OO_v}{\ell\OO_v}\to\frac{E(K_v)}{\ell E(K_v)}\to\frac C{\ell C}\to0.\]
% 	The result will follow by taking dimensions of this exact sequence; for example, just the right-exact part of this sequence immediately shows us that $E(K_v)/\ell E(K_v)$ is finite. We thus have
% 	\[\dim_{\FF_\ell}\frac{E(K_v)}{\ell E(K_v)}-\dim_{\FF_\ell}E(K_v)[\ell]=\dim_{\FF_\ell}\frac{\OO_v}{\ell\OO_v}-\dim_{\FF_\ell}\OO_v[\ell]+\dim_{\FF_\ell}\frac C{\ell C}-\dim_{\FF_\ell}C[\ell].\]
% 	Now, we note that $\OO_v[\ell]=0$ because $K_v$ has characteristic $0$. Continuing, and $\#C_\ell=\#(C/\ell C)$ because the kernel and cokernel of the endomorphism $\ell\colon C\to C$ should have the same size. We are left with
% 	\[\dim_{\FF_\ell}\frac{E(K_v)}{\ell E(K_v)}=\dim_{\FF_\ell}\frac{\OO_v}{\ell\OO_v}+\dim_{\FF_\ell}E(K_v)[\ell].\]
% 	We now note that $\ell=p$ implies that $\OO_v/\ell\OO_v\cong\ZZ/\ell\ZZ$; otherwise, $\OO_v/\ell\OO_v=0$.
% \end{proof}
% \begin{lemma}
% 	Fix an elliptic curve $E$ over a field $K$, and let $m$ be a positive integer which is nonzero in $K$. Then the subspace $E(K)/mE(K)$ embedded in $\mathrm H^1(K;E[m])$ is isotropic for the pairing induced by the Weil pairing.
% \end{lemma}
% \begin{proof}
% 	This was stated in class without proof, so we will provide one. Although there are better ways to show this, we may as well work directly on the level of cocycles. Fix $P,Q\in E(K)/mE(K)$, and we want to show that the $2$-cocycle $\mathrm H^2(e_m)(\delta_P\cup\delta_Q)$ is a $2$-coboundary. Expanding out the definitions, we see that we should choose $P'$ and $Q'$ with $P=mP'$ and $Q=mQ'$, and then we want to show the map
% 	\[(\sigma,\tau)\mapsto e_m(\sigma P'-P',\sigma\tau Q'-\sigma Q')\]
% 	is a $2$-coboundary.
% 	% For this, we will use \Cref{prop:alternative-weil}, which allows us to instead show that the $2$-cocycle
% 	% \[c\colon(\sigma,\tau)\mapsto\widetilde e_m(\sigma P'-P',\sigma\tau Q'-\sigma Q')^{-1}\]
% 	% is a $2$-coboundary.
	
% 	Accordingly, we now set up some notation. If $P=Q$, then there is nothing to do, so we omit this case. For each $\sigma\in\op{Gal}(K^{\mathrm{sep}}/K)$, we define the divisors $D_{P,\sigma}\coloneqq[\sigma P']-[P']$ and $D_{Q,\sigma}\coloneqq[\sigma Q']-[Q']$. Note that any $D_{P,\sigma}$ and any $D_{Q,\tau}$ have disjoint support: $P\ne Q$ forces $\sigma P'\ne\tau Q'$ for any $\sigma$ or $\tau$ because $m\sigma P'=P$ and $m\tau Q'=Q$. We also note that $D_{P,\sigma\tau}=\sigma D_{P,\tau}+D_{P,\sigma}$ and similarly for $Q$. Continuing, we select functions $f_{P,\sigma}$ and $f_{Q,\sigma}$ so that $\op{div}f_{P,\sigma}=mD_{p,\sigma}$ and $\op{div}f_{Q,\sigma}=$. By choosing another point $S$ not in the Galois orbit of the previously defined points, we may normalize the functions $f_\bullet$ so that they evaluate to $1$ on $S$. Then we see that
% 	\[f_{P,\sigma\tau}=\sigma f_{P,\tau}\cdot f_{P,\sigma}\]
% 	because they have the same divisor and the same nonzero value at $S$; a similar statement holds for $Q$.

% 	We are now ready to calculate. To begin, recall from \cite[Exercise~3.16]{silverman} that we can compute the Weil pairing as
% 	\begin{align*}
% 		e_m(\sigma P'-P',\sigma\tau Q'-Q')e_m(\sigma P'-P',\sigma Q'-Q')^{-1}
% 		&= \frac{f_{Q,\sigma\tau}(D_{P,\sigma})}{f_{P,\sigma}(D_{Q,\sigma\tau})}\cdot\frac{f_{P,\sigma}(D_{Q,\sigma})}{f_{Q,\sigma}(D_{P,\sigma})}.
% 	\end{align*}
% 	We now recall $f_{Q,\sigma\tau}=\sigma f_\tau\cdot f_\sigma$ and $D_{Q,\sigma\tau}=\sigma D_\tau+D_\sigma$, so $c(\sigma,\tau)$ equals
% 	\[\frac{(\sigma f_{Q,\tau})(D_{P,\sigma})}{f_{P,\sigma}(\sigma D_{Q,\tau})}=\frac{\sigma f_{Q,\tau}(P')}{\sigma f_{Q,\tau}(\sigma^{-1}P')}\cdot\frac{f_{P,\sigma}(\sigma Q)}{f_{P,\sigma}(\sigma\tau Q)}.\]
% 	We will complete the proof by showing that each factor of the right-hand side is a $2$-coboundary. For example, the $1$-cochain $\sigma\mapsto f_{Q,\sigma}(P')$ produces the $1$-coboundary sending $(\sigma,\tau)$ to
% 	\[\frac{\sigma f_{Q,\tau}(P')\cdot f_{Q,\sigma}(P')}{f_{Q,\sigma\tau}(P')}=\frac{\sigma f_{Q,\tau}(P')}{\sigma f_{Q,\tau}(\sigma^{-1}P')}.\]
% 	Similarly, the $1$-cochain $\sigma\mapsto f_{P,\sigma}(\sigma Q')$ produces the $2$-coboundary sending $(\sigma,\tau)$ to
% 	\[\frac{\sigma f_{P,\tau}(\tau Q')\cdot f_{P,\sigma}(\sigma Q')}{f_{P,\sigma\tau}(\sigma\tau Q')}=\frac{\sigma f_{P,\tau}(\tau Q')\cdot f_{P,\sigma}(\sigma Q')}{\sigma f_{P,\tau}(\sigma^{-1}\sigma\tau Q)\cdot f_{P,\sigma}(\sigma\tau Q)}.\]
% 	Multiplying these two $2$-coboundaries together completes the proof.
% \end{proof}

\section{The ``Counterexample'' to Grunwald--Wang}
\begin{proposition}
	\hfill
	\begin{listalph}
		\item Show that $16$ is an eighth power in $\RR$ and all $\QQ_p$ for $p\ne2$, but not an eighth power in $\QQ_2$.
		\item Let $K=\QQ(\sqrt7)$. Show that $16$ is an eighth power in every completion $K_v$ of $K$ but not an eighth power in $K$. In particular, the localization map
		\[\mathrm H^1(K;\mu_8)\to\prod_v\mathrm H^1(K_v;\mu_8)\]
		is not injective.
	\end{listalph}
\end{proposition}
\begin{proof}
	Before showing either of the parts, we claim that $16$ is an eighth power in a field $K$ if and only if $\sqrt2\in K$ or $\sqrt{-2}\in K$ or $\sqrt{-1}\in K$. Certainly the reverse direction holds because
	\[\big(\sqrt2\big)^{8}=\left(\sqrt{-2}\right)^{8}=(1+i)^{8}=16.\]
	In the forward direction, if $16$ is an eighth power, then we see that one of $4$ or $-4$ is a fourth power. If $-4$ is a fourth power, then $-4$ is a square, so either $2=0$ (in which case $\sqrt2\in K$) or $-1$ is a square. Otherwise, $4$ is a fourth power, so one of $2$ or $-2$ is a square, as desired.
	\begin{listalph}
		\item We do this by casework on the place $v$. For example, if $v$ is archimedean, the result follows because $\sqrt2\in\RR$. Additionally, for $v=2$, we see that $v_2(16)=4$ is not divisible by $8$, so $16$ cannot possibly be an eighth power.

		It remains to handle places $v$ given by odd primes $p$. 
		% To this end, we note that
		% \[\big(\sqrt2\big)^{8}=\left(\sqrt{-2}\right)^{8}=(1+i)^{8}=16,\]
		% so for any field $K$, we see that $16$ is an eighth power in any of the extensions $K(\sqrt2)$ or $K(\sqrt{-2})$ or $K(i)$. Thus, 
		It is enough to show that $\{\sqrt2,\sqrt{-2},i\}\cap\QQ_p\ne\emp$ for each odd prime $p$. In other words, we are asking for one of the quadratics to $x^2-2$ or $x^2+2$ or $x^2-1$ to admit a root. By Hensel's lemma, it is equivalent for one of these quadratics to admit a root over $\FF_p$, which in turn is equivalent to having
		\[1\stackrel?\in\left\{\left(\frac2p\right),\left(\frac{-2}p\right),\left(\frac{-1}p\right)\right\}.\]
		To see this, we note that
		\[\left(\frac2p\right)\left(\frac{-2}p\right)\left(\frac{-1}p\right)=1,\]
		so it is not possible for all three Legendre symbols to equal $-1$!

		\item Set $K\coloneqq\QQ(\sqrt7)$ for brevity. By (a), to show that $16$ is an eighth power in all $K_v$, we only have to handle places $v$ above $2$. Note that $7$ is not a square in $\QQ_2$ (indeed, it is not a square$\pmod4$), so there is only one place $\QQ_2(\sqrt7)$ above $2$. It remains to show that $16$ is an eighth power in $\QQ_2(\sqrt7)$, which is true because $\sqrt{-1}\in\QQ_2(\sqrt7)$ (indeed, $-7\in\QQ_2^{\times2}$).

		Next, we should show that $16$ is not an eighth power in $K$. It is enough to show that none of $\sqrt{-1}$ or $\sqrt2$ or $\sqrt{-2}$ are in $K$, which holds because $K=\QQ(\sqrt7)$ is a quadratic extension avoiding all those elements. Explicitly, $K$ is totally real, so $-1$ and $-2$ cannot be squares, and we can see that there is no $a,b\in\ZZ$ for which
		\[\left(a+b\sqrt7\right)^2=a^2+7b^2+14ab\sqrt7\]
		can equal $2$: this would require $a=0$ or $b=0$, but certainly neither $2$ nor $7/2$ is a square in $\ZZ$.

		It remains to explain the last sentence. Well, we may functorially identify $\mathrm H^1(K;\mu_8)$ with $K^\times/K^{\times8}$ (by the boundary map of the Kummer exact sequence for $\mathbb G_{m,K}$), so the map
		\[\mathrm H^1(\QQ(\sqrt7);\mu_8)\to\prod_v\mathrm H^1\left(K_v;\mu_8\right)\]
		fails to be injective because the nontrivial class $16\in K^\times/K^{\times8}$ vanishes in all localizations.
		\qedhere
		% In fact, we will show that any totally real field $L$ in which $16$ is an eighth power has $\sqrt2\in L$, which will complete the proof because $\sqrt2\notin K$. To see this, if $16$ is an eighth power in $L$, then one of $4$ or $-4$ is a fourth power; however, $L$ is totally real, so $-4$ cannot be a square, so we see that $4$ must be a fourth power. Similarly, this means that one of $2$ or $-2$ is a square in $L$, but $-2$ is not a square, so $2$ is a square in $L$. Thus, $\sqrt2\in L$, as required.
	\end{listalph}
\end{proof}
% \begin{remark} \label{rem:local-to-global-not-inj}
% 	We may identify $\mathrm H^1(K;\mu_8)$ with $K^\times/K^{\times8}$, so the second part of \Cref{ex:grunwald-wang} shows that the map
% 	\[\mathrm H^1(\QQ(\sqrt7);\mu_8)\to\mathrm H^1\left(\AA_{\QQ(\sqrt7)};\mu_8\right)\]
% 	fails to be injective because the nontrivial class $16\in K^\times/K^{\times8}$ vanishes in $\mathrm H^1\big(\AA_{\QQ(\sqrt7)};\mu_8\big)$.
% \end{remark}

\section{Congruent Number Elliptic Curves at \texorpdfstring{$2$}{2}}
Consider the congruent number elliptic curve over $\QQ$ given on affine coordinates by
\[E_d\colon y^2=x(x-d)(x+d).\]
The local condition at a place $v$ can be described using the map
\[\arraycolsep=1.4pt\begin{array}{ccc}
	E_d(K_v) &\to& \left\{(\alpha,\beta,\gamma)\in(K_v^\times/K_v^{\times2})^{\oplus 3}:\alpha\beta\gamma=1\right\} \\
	(x,y) &\mapsto& (x,x-d,x+d)
\end{array}\]
for $x\notin\{0,\pm d\}$, and a similar map works for $x\in\{0,\pm d\}$ where the two nonzero coordinates determine the third via $\alpha\beta\gamma=1$. Assume that $d$ is an odd integer.
\begin{lemma} \label{lem:cong-local-condition-hw1}
	Fix everything as above. We compute information about image of $\delta_v\colon E_d(\QQ_v)/2E_d(\QQ_v)\to\mathrm H^1(\QQ_v;H)$ for each place $v$, where $H\subseteq\mu_2^{\oplus3}$ is the trace-zero hyperplane.
	\begin{listalph}
		\item If $v\nmid2d\infty$, then the image of $\delta_v$ consists of the triples $(\alpha,\beta,\gamma)$ such that $v(\alpha)=v(\beta)=v(\gamma)=0$.
		\item The image of $\delta_v$ contains the triples
		\[S\coloneqq\{(1,1,1),(-1,-d,d),(d,2,2d),(-d,-2d,2)\}.\]
		\item If $v\mid d\infty$, then the image of $\delta_v$ is $S$.
		% \item If $v=2$, the image of $\delta_v$ is $\op{span}(S\cup\{(1,5,5)\})$.
	\end{listalph}
\end{lemma}
\begin{proof}
	We show the parts in sequence.
	\begin{listalph}
		\item If $v\nmid2d\infty$, then $E_d$ has good reduction at the finite place $v$, so the image of $\delta_v$ is $\mathrm H^1_{\mathrm{ur}}(\QQ_v;H)$. The result now follows by looking coordinate-wise.
		\item The given set $S$ is precisely the image of $E_d[2]$. Indeed,
		\begin{align*}
			\delta_v(\infty) &= (1,1,1), \\
			\delta_v(0,0) &= (-1,-d,d), \\
			\delta_v(d,0) &= (d,2,2d), \\
			\delta_v(-d,0) &= (-d,-2d,2).
		\end{align*}
		\item If $v=\infty$, then we have a linearly independent set $\{(-1,-1,+1)\}$, which spans the image of $\delta_v$ by the dimension calculations of \Cref{prop:local-cohom-calculations-hw1}(a). Similarly, if $v\mid d$, then we have a linearly independent set $\{(-1,-d,d),(d,2,2d)\}$ (because $d$ is squarefree), which spans the image of $\delta_v$ by \Cref{prop:local-cohom-calculations-hw1}(a) again.
		\qedhere
	\end{listalph}
\end{proof}
% We will also need the following technical result.
\begin{proposition}% \label{lem:conic-almost-everywhere}
	Fix everything as above.
	\begin{listalph}
		\item Show that the local condition at the $2$-adic place is the three-dimensional subspace spanned by $\{(1,5,5),(-1,-d,d),(d,2,2d)\}$.
		\item Show that if the following system of equations $T_{\alpha,\beta,\gamma}$ for $\alpha,\beta,\gamma$ all positive and odd
		\[\begin{cases}
			\alpha u^2-\beta v^2=-d, \\
			\alpha u^2-\gamma w^2=d
		\end{cases}\]
		has solutions in $\QQ_v$ for all $v\ne2$, then $T_{\alpha,\beta,\gamma}$ has solutions in $\QQ_2$.
	\end{listalph}
	% Fix an odd positive squarefree integer $d$, and let $E$ be the elliptic curve over $\QQ$ which is the projective closure of $y^2=x(x-d)(x+d)$. Further, fix some triple $(\alpha,\beta,\gamma)\in\mathrm H^1(\QQ;H)$. If $(\alpha,\beta,\gamma)$ is in the image of $\delta_v\colon E_d(\QQ_v)/2E_d(\QQ_v)\to\mathrm H^1(\QQ_v;H)$ for each place $v\ne2$, then it is also in the image of $\delta_2$.% Let $T\subseteq\PP^n_\ZZ$ be a smooth projective conic. If $T$ admits solutions in $\QQ_v$ for all but one place $v_0$, then $T$ admits solutions in $\QQ_{v_0}$ as well.
\end{proposition}
\begin{proof}
	We quickly handle (a).
	\begin{enumerate}
		\item Because $-1$, $2$, and $5$ are linearly independent in $\QQ_2^\times/\QQ_2^{\times2}$, we see that the triples $(1,5,5)$, $(-1,-d,d)$, and $(d,2,2d)$ are linearly independent triples. Thus,
		\[\dim_{\FF_2}\op{span}(S\cup\{1,5,5\})=3,\]
		which also equals $\dim_{\FF_2}\im\delta_v$ by \Cref{prop:local-cohom-calculations-hw1}(a). Thus, it suffices to show that $\op{span}(S\cup\{1,5,5\})\subseteq\im\delta_v$.

		\item By \Cref{lem:cong-local-condition-hw1}, it is enough to check that $(1,5,5)\in\im\delta_v$. For this, we recall that it is equivalent to produce a nonzero solution to the system
		\[\begin{cases}
			x^2-5y^2=+dw^2, \\
			x^2-5z^2=-dw^2,
		\end{cases}\]
		in $\QQ_2$. The solubility of this system does not change if we change $d$ by an element of $\QQ_2^{\times2}$, so we may assume $d\in\{\pm1,\pm5\}$. Similarly, by symmetry, we may adjust the sign of $d$, so we may assume that $d\in\{1,5\}$. In these cases, we may set $(x,w)=(1,2)$ so that we need $5y^2\in\{-3,-19\}$ and $5z^2\in\{5,21\}$, both of which are possible.
	\end{enumerate}
	We now move on to (b). We are given a triple $(\alpha,\beta,\gamma)$ which is in the image of $\delta_v$ for each place $v\ne2$; we may as well represent $(\alpha,\beta,\gamma)$ as a triple of squarefree integers. We must show that $(\alpha,\beta,\gamma)\in\im\delta_2$.
	\begin{enumerate}
		\item Even though it is already given in the problem, we quickly explain that $\alpha$ is odd (modulo squares). Surely this is the case for $(\alpha,\beta,\gamma)\in\delta_2(E_d[2])$. Otherwise, we know $\alpha$ equals the $x$-coordinate of some $(x,y)\in E_d(\QQ_2)$, meaning
		\[y^2=x\left(x^2-d^2\right).\]
		We now compute some valuations to show that $\nu_2(x)$ is even, which means $\alpha$ is odd.
		\begin{itemize}
			\item There is nothing to do if $\nu_2(x)=0$.
			\item If $\nu_2(x)>0$, then the valuations are $2\nu_2(y)=\nu_2(x)$, so $\nu_2(x)$ is even.
			\item If $\nu_2(x)<0$, then the valuations are $2\nu_2(y)=-3\nu_2(x)$, so $\nu_2(x)$ is still even.
		\end{itemize}
	
		\item We make some reductions. By \Cref{lem:cong-local-condition-hw1}, we know that $(\alpha,\beta,\gamma)$ is unramified for $v\nmid2d\infty$, so the prime factorizations of $\alpha$, $\beta$, and $\gamma$ are supported in the prime factors of $2d$.
	
		For our next few reductions, we note that multiplying any triple by the image of $\delta(E[2])$ will not change whether it is in the image of $\delta_2$. For example, with $\alpha$ odd, we see that we may multiply by the triple $(d,2,2d)$ to force $\beta$ to be odd, which in turn forces $\gamma$ to be odd too. Multiplying by the triple $(-1,-d,d)$, we may assume that $\alpha\pmod8$ is in $\{1,5\}$. Similarly, multiplying by the triple $(1,5,5)$, we may assume that $\beta\pmod8$ is in $\{1,3\}$.
	
		\item We complete the proof using Hilbert symbols. We know that the equations
		\[\begin{cases}
			\alpha x^2-\beta y^2=-dw^2, \\
			\alpha x^2-\gamma z^2=+dw^2
		\end{cases}\]
		has solutions in each $\QQ_v$ for $v\ne2$. Thus, $(-d\alpha,d\beta)_v=(d\alpha,-d\gamma)_v=(2d\beta,-2d\gamma)_v=1$ for each $v\ne2$, so Hilbert reciprocity\footnote{In this case, it is possible to unwind the application of Hilbert reciprocity into merely applications of Quadratic reciprocity, but this language is convenient anyway.} implies that
		\[(-d\alpha,d\beta)_2=(d\alpha,-d\gamma)_2=(2d\beta,-2d\gamma)_2=1.\]
		We now show that $(\alpha,\beta,\gamma)\in\im\delta_2$ directly. We may consider this triple up to $\QQ_2^{\times2}$, meaning that we may assume $\alpha\in\{1,5\}$ and $\beta\in\{1,3\}$. For example, because everything is odd and $\alpha\in\{1,5\}$, we see that $(\alpha,d)_2=(\alpha,\beta)_2=(\alpha,\gamma)_2=1$.\footnote{In particular, $(5,-)_2$ vanishes on odds. It is not hard to show $(5,-5)_2=1$ (because $5\cdot1^2-5\cdot1^2=0^2$), and $(5,-1)_2=1$ because $5\cdot1^2-1\cdot1^2=2^2$.} As such, $(-d\alpha,d\beta)_2=(-d,\beta)_2$ and $(d\alpha,-d\gamma)_2=(d,\beta)_2$, so $(-1,\beta)_2=1$. Thus, $\beta\ne3$ is forced,\footnote{We need to show that $(-1,3)_2=-1$, which holds because $3x^2-y^2=z^2$ has no nontrivial solutions by$\pmod8$ considerations.} so $\gamma=1$ follows.
		\qedhere
	\end{enumerate}
\end{proof}

\section{A Selmer Calculation}
We begin by recalling from class how to change the local condition one place at a time.
\begin{lemma} \label{lem:basic-selmer-change-hw1}
	Fix a number field $K$ and a finite self-dual Galois module $M$ which is a vector space over $\FF_p$. Let $\mc L$ be a self-dual local condition. Letting $\Sigma$ be the singleton of a place $v_0$, we have
	\[\dim_{\FF_p}\op{Sel}_{\mc L^\Sigma}(M)-\dim_{\FF_p}\op{Sel}_{\mc L_\Sigma}(M)=\frac12\dim_{\FF_p}\mathrm H^1(K_{v_0};M).\]
\end{lemma}
\begin{proof}
	% The point is to use \Cref{ex:self-dual-local-condition}, which explains that the Weil pairing provides an isomorphism $M\cong M^*$ of Galois modules, and we are given that it also induces an identification $\mc L\cong\mc L^*$ of local conditions. %(see \Cref{lem:weil-pairing}). Furthermore, \Cref{prop:local-is-lagrangian} explains that $\mc L_v\subseteq\mathrm H^1(K_v;M)$ becomes maximal isotropic under the induced pairing, so $\mc L_v^*$ is identified with $\mc L_v$.
	%
	Recall the exact sequences
	\[0\to\op{Sel}_{\mc L_\Sigma}(M)\to\op{Sel}_{\mc L^\Sigma}(M)\to\mathrm H^1(K_{v_0};M),\]
	and
	\[0\to\op{Sel}_{(\mc L^*)_\Sigma}(M^*)\to\op{Sel}_{(\mc L^*)^\Sigma}(M^*)\to\mathrm H^1(K_{v_0};M^*).\]
	Global duality tells us that the images in the rightmost terms are orthogonal complements. Now, via the duality $M\cong M^*$ discussed in the previous paragraph, the second exact sequence is identified with the first one. We conclude that
	\[\frac{\op{Sel}_{\mc L^\Sigma}(M)}{\op{Sel}_{\mc L_\Sigma}(M)}\subseteq\mathrm H^1(K_{v_0};M)\]
	is the orthogonal complement of itself, so the result follows.
	% (We have silently used \Cref{ex:modify-dual-local-condition}.)
\end{proof}
\begin{lemma} \label{lem:change-selmer-one-place-hw1}
	Fix an elliptic curve $E$ over a number field $K$. Choose a prime $p$, and set $\mc L$ to be a local condition on $E[p]$ with $\mc L=\mc L^*$. Further, for a given place $v_0$, let $\mc L'_{v_0}\subseteq\mathrm H^1(K_{v_0};E[p])$ be some self-dual subspace disjoint from $\mc L_{v_0}$, and extend it to the local condition $\mc L'$ given by $\mc L'_v=\mc L_v$ for $v\ne v_0$.
	% \[\mc L'_v\coloneqq\begin{cases}
	% 	\mc L'_{v_0} & \text{if }v=v_0, \\
	% 	\mc L_v & \text{otherwise}.
	% \end{cases}\]
	\begin{listalph}
		\item If $\op{Sel}_{\mc L}(E[p])\to\mathrm H^1(K_{v_0};E[p])$ vanishes, then
		\[\dim_{\FF_p}\op{Sel}_{\mc L'}(E[p])=\dim_{\FF_p}\op{Sel}_{\mc L}(E[p])+\frac12\dim_{\FF_2}\mathrm H^1(K_{v_0};E[p]).\]
		\item If $\op{Sel}_{\mc L}(E[p])\to\mathrm H^1(K_{v_0};E[p])$ surjects onto $\mc L_{v_0}$, then
		\[\dim_{\FF_p}\op{Sel}_{\mc L}(E[p])=\dim_{\FF_p}\op{Sel}_{\mc L'}(E[p])-\frac12\dim_{\FF_2}\mathrm H^1(K_{v_0};E[p]).\]
	\end{listalph}
\end{lemma}
\begin{proof}
	All the hypotheses will be used, though much care will be required. Set $M\coloneqq E[p]$ and $\Sigma\coloneqq\{v_0\}$ for brevity. The main point is to chase around a pullback square. Because $\mc L_{v_0}$ and $\mc L'_{v_0}$ are disjoint maximal isotropic subspaces, we see that $\mc L+\mc L'=\mc L^\Sigma$ and $\mc L\cap\mc L'=\mc L_\Sigma$. Pulling back the intersection along $\mathrm H^1(K;M)\to\mathrm H^1(\AA_K;M)$ produces the pullback square
	% https://q.uiver.app/#q=WzAsNCxbMCwwLCJcXG9we1NlbH1fe1xcbWMgTF9cXFNpZ21hfShNKSJdLFsxLDAsIlxcb3B7U2VsfV97XFxtYyBMJ30oTSkiXSxbMCwxLCJcXG9we1NlbH1fe1xcbWMgTH0oTSkiXSxbMSwxLCJcXG9we1NlbH1fe1xcbWMgTF5cXFNpZ21hfShNKSJdLFswLDJdLFsyLDNdLFswLDFdLFsxLDNdLFswLDMsIiIsMSx7InN0eWxlIjp7Im5hbWUiOiJjb3JuZXIifX1dXQ==&macro_url=https%3A%2F%2Fraw.githubusercontent.com%2FdFoiler%2Fnotes%2Fmaster%2Fnir.tex
	\begin{equation}
		\begin{tikzcd}[cramped]
			{\op{Sel}_{\mc L_\Sigma}(M)} & {\op{Sel}_{\mc L'}(M)} \\
			{\op{Sel}_{\mc L}(M)} & {\op{Sel}_{\mc L^\Sigma}(M)}
			\arrow[from=1-1, to=1-2]
			\arrow[from=1-1, to=2-1]
			\arrow["\lrcorner"{anchor=center, pos=0.125}, draw=none, from=1-1, to=2-2]
			\arrow[from=1-2, to=2-2]
			\arrow[from=2-1, to=2-2]
		\end{tikzcd} \label{eq:pullback-of-selmers-hw1}
	\end{equation}
	of intersections inside $\mathrm H^1(K;M)$. We now show (a) and (b) separately.
	\begin{listalph}
		\item The exactness of
		\[0\to\op{Sel}_{\mc L_\Sigma}(M)\to\op{Sel}_{\mc L}(M)\to\mc L_\ell\]
		implies that the inclusion $\op{Sel}_{\mc L_\Sigma}(M)\to\op{Sel}_\mc L(M)$ is an isomorphism. Thus, the left arrow of \eqref{eq:pullback-of-selmers-hw1} is an isomorphism, so the right arrow is also an isomorphism, and the result follows from \Cref{lem:basic-selmer-change-hw1}.

		\item We are given that $\op{Sel}_{\mc L}(M)\to\mc L_\ell$ is surjective,
		% \[0\to\op{Sel}_{\mc L_\Sigma}(E[2])\to\op{Sel}_2(E)\to\mc L_\ell\]
		% is surjective on the right,
		so the exact sequence
		\[0\to\op{Sel}_{\mc L}(M)\to\op{Sel}_{\mc L^\Sigma}(M)\to\frac{\mathrm H^1(K_{v_0};M)}{\mc L_{v_0}}\]
		of maps to $0$ at the end, so the inclusion $\op{Sel}_{\mc L}(M)\to\op{Sel}_{\mc L^\Sigma}(M)$ is an isomorphism. Thus, the bottom arrow of \eqref{eq:pullback-of-selmers-hw1} is an isomorphism, so the top arrow is also an isomorphism. The claim now follows from \Cref{lem:basic-selmer-change-hw1}.
		\qedhere
	\end{listalph}
\end{proof}
\begin{proposition}
	Continue to consider the congruent number elliptic curve over $\QQ$
	\[E_d\colon y^2=(x-d)(x+d)\]
	where $d$ is an odd integer. Let $d$ be the product of an odd number of primes $p_i$ such that all $p_i$ are $5\pmod8$ and are mutually non-quadratic residues to each other. Show that
	\[\dim_{\FF_2}\op{Sel}_2(E_d)=3.\]
	% Fix a set $S$ of primes such that each pair $(\ell_1,\ell_2)$ of primes in $S$ have $\ell_1,\ell_2\equiv5\pmod8$ and $\ell_1\notin\QQ_{\ell_2}^{\times2}$. Further, suppose that $\#S$ is odd, and let $d$ be the product of the primes in $S$. Letting $E$ denote the projective closure of $y^2=x(x-d)(x+d)$, we have
	% \[\dim_{\FF_2}\op{Sel}_2(E)=3.\]
\end{proposition}
\begin{proof}
	Fix $E\coloneqq E_d$ for brevity. We will show the stronger statement that
	\[\op{Sel}_2(E)\stackrel?=\op{span}(E[2]\cup\{(1,d,d)\}).\]
	The rank claim then follows because the triples $\{(-1,-d,d),(d,2,2d),(1,d,d)\}$ form a basis. For this, we induct on $\#S$, where $\#S=1$ follows from results in class.

	For the induction, suppose we have the statement for $S$ and $d$, and we would like to show it for some set $S\cup\{\ell_1,\ell_2\}$ still satisfying the list of conditions; set $\ell\coloneqq\ell_1\ell_2$ for brevity. Accordingly, let $E$ and $E'$ be the projective closures of $y^2=x(x-d)(x+d)$ and $y^2=(x-d\ell)(x+d\ell)$. Let $\mc L$ and $\mc L'$ be the associated local conditions of $\mathrm H^1(\AA_K;H)$, where $E[2]$ and $E'[2]$ are identified with the trace-zero hyperplane $H\subseteq\mu_2^{\oplus3}$ as usual.

	Quickly, let's compare our local conditions $\mc L$ and $\mc L'$, freely using \Cref{lem:cong-local-condition-hw1}.
	\begin{itemize}
		\item For $v\nmid2d\ell\infty$, we see that $\mc L_v$ and $\mc L'_v$ both contain the unramified triples.
		\item For $v=\infty$, both are the same.
		\item For finite $v=p$ with $p\mid d$, we claim that $\mc L_v=\mc L'_v$. Well, the sizes are the same, so it is enough to just get an inclusion, so it is enough to check $\{(-1,-d\ell,d\ell),(d\ell,2,2d\ell)\}\subseteq\mc L_p$, which is equivalent to having $\{(1,\ell,\ell),(\ell,1,\ell)\}\subseteq\mc L_p$. But this is true because $\ell=\ell_1\ell_2$ is a square in $\QQ_p^\times$.
		\item For $v=2$, it is once again enough to achieve the inclusion $\{(1,\ell,\ell),(\ell,1,\ell)\}\subseteq\mc L_2$. This is true because $\ell\equiv1\pmod8$, so $\ell\in\QQ_2^{\times2}$.
		% This time, considering valuations (and the fact that $-1$ is not a square) implies that $(1,\ell,\ell)$ is either $(1,1,1)$ or $(1,5,5)$, which means that either $\ell$ or $5\ell$ is a square in $\QQ_2$. But if $5\ell$ is a square, then $(\ell,1,\ell)\notin\mc L_2$, so we must instead have $\ell$ be a square. Of course, having $\ell$ be a square is also sufficient.
		\item Lastly, for $v\in\{\ell_1,\ell_2\}$, we see that $\mc L_v$ contains unramified triples, but the only unramified triple in $\mc L'_v$ is the trivial one, so $\mc L_v\cap\mc L'_v$ is trivial.
	\end{itemize}
	Thus, we see that we are going to use \Cref{lem:change-selmer-one-place-hw1} twice. Accordingly, let $\mc L''$ be an ``intermediate'' local condition given by
	\[\mc L''_v\coloneqq\begin{cases}
		\mc L_v & \text{if }v\ne\ell_1, \\
		\mc L_v' & \text{if }v=\ell_1.
	\end{cases}\]
	Thus, $\mc L$ and $\mc L''$ differ only at the place $v=\ell_1$, and $\mc L''$ and $\mc L'$ differ only at the place $v=\ell_2$. We now have two steps.
	\begin{enumerate}
		\item We claim that $\op{Sel}_{\mc L''}(H)=\{(1,1,1),(-1,-1,1)\}$. One inclusion is not so bad: certainly $(1,1,1)\in\op{Sel}_{\mc L''}(H)$. Additionally, $(1,-1,-1)\in\op{Sel}_{\mc L''}(H)$ because it is already in $\mc L_v$ for all $v$, and $(1,-1,-1)\in\mc L_{\ell_1}$ because $(-1,-1,1)$ equals $(1,1,1)$ up to squares in $\QQ_{\ell_1}^\times$.
		% by checking place-by-place: $v\nmid2d\ell_1$ has nothing to do, $v\mid2d$ follows because $(1,d,d)\in\op{Sel}_{\mc L}(H)$ and $\ell\in\QQ_v^{\times2}$, and $v=\ell_1$ follows because this is equivalent to $(-1,-d\ell,d\ell)$ up to squares.

		For the other inclusion, it is enough to check that
		\[\dim_{\FF_2}\op{Sel}_{\mc L''}(H)=\op{Sel}_{\mc L}(H)-2.\]
		For this, we use \Cref{lem:change-selmer-one-place-hw1}(b) at the place $v_0=\ell_1$. The hypotheses on the local conditions were checked above, so it remains to check that the map $\op{Sel}_{\mc L}(H)\to\mc L_{\ell_1}$ is surjective. This holds by the calculation of \Cref{lem:cong-local-condition-hw1} by using the global triples coming from $E[2]$.

		\item We claim that $\op{Sel}_{\mc L'}(H)=\op{span}(E'[2]\cup\{(1,d\ell,d\ell)\}$, which will complete the proof. Again, one inclusion is not so bad: certainly $E'[2]$ provides elements of the Selmer group. Additionally, we once again see that $(1,d\ell,d\ell)\in\op{Sel}_{\mc L'}(H)$ by checking place-by-place: along with the checks from the previous step, we merely have to check that $(1,d\ell,d\ell)\in\mc L'_{\ell_2}$, which is true because this triple is equivalent to $(-1,-d\ell,d\ell)$ up to squares.

		For the other inclusion, we will again use ranks, noting that it is enough to check that
		\[\dim_{\FF_2}\op{Sel}_{\mc L'}(H)=\op{Sel}_{\mc L''}(H)+2,\]
		which will follow from \Cref{lem:change-selmer-one-place-hw1}(b) at the place $v_0=\ell_2$. Again, the hypotheses on the local conditions are satisfied, so it remains to check that the map $\op{Sel}_{\mc L''}(H)\to\mc L''_{\ell_2}$ is trivial. Well, from the calculation in the previous step, we know that $\op{Sel}_{\mc L''}(H)=\{(1,1,1),(-1,-1,1)\}$, and both of these elements are trivial up to squares in $\QQ_{\ell_2}^\times$.
		\qedhere
	\end{enumerate}
\end{proof}

\end{document}