% !TEX root = ../notes.tex

\documentclass[../notes.tex]{subfiles}

\begin{document}

\section{November 13}
Today we begin discussing the Bertolini--Darmon Euler system.

\subsection{The Work of Waldspurger and Bertolini--Darmon}
Thus far, we have been discussing ``rank-$1$'' elliptic curves in the case, which is the generic case where the sign of the functional equation is $-1$. There is a parallel story when the sign of the functional equation is $+1$, and the generic case should be that the elliptic curves have rank $0$.

Instead of the Gross--Zagier formula, we have Waldspurger's formula (which predates Gross--Zagier); instead of Kolyvagin's theorem, we have Bertolini--Darmon (which came after Kolyvagin's work). In total, we still find that $\op{ord}_{s=1}L(E/K,s)=0$ (for suitable $E/K$) implies the Birch--Swinnerton-Dyer conjecture.
\begin{remark}
	Earlier, we did explain that Gross--Zagier--Kolyvagin can prove the rank-$0$ case using some tricks. Namely, given that $E$ came from an elliptic curve over $\QQ$, one can choose an auxiliary quadratic extension $L$ for which $E_L$ has analytic rank $1$. Then we can appeal to the rank-$1$ situation.
\end{remark}
In fact, one can upgrade some of the above results to allow a twist by a character. In particular, given a Galois character $\chi$ of $\op{Gal}(\ov K/K)$ factoring through the Hilbert class field $H$, one can use Gross--Zagier--Kolyvagin to show that $\op{ord}_{s=1}L(\rho_E\otimes\chi,s)=1$ implies that $\op{rank}E(H)_\CC^\chi=1$ and $\#\Sha(E/H)^\chi<\infty$.
\begin{remark}
	The result of the previous paragraph works when $H$ is replaced by any ring class field.
\end{remark}
The point is that the ``tricks'' one uses to reduce the rank-$1$ case to the rank-$0$ case no longer apply because $\rho_E\otimes\chi$ has no reason to factor in general. (In fact, as long as $\chi$ does not factor through $\op{Pic}\OO_K/2\op{Pic}\OO_K$ and $E$ does not have complex multiplication, then it should not factor. The point is that we are asking when the Galois representation $\op{Ind}_\QQ^K\chi$ can factor.) Thus, we may be interested in showing the Birch--Swinnerton-Dyer conjecture in these rank-$0$ cases, which is the work of Waldspurger and Bertolini--Darmon.
\begin{theorem}[Waldspurger--Bertolini--Darmon]
	Let $E$ be an elliptic curve over $\QQ$, and let $K$ be an imaginary quadratic extension. For a Galois character $\chi$ factoring through a ring class field, if
	\[\op{ord}_{s=1}L(\op{Res}_{K/\QQ}\rho_E\otimes\chi,s)=0,\]
	then $\op{rank}(E(H))^\chi=0$ and $\#\Sha(E/H)^\chi<\infty$.
\end{theorem}
The key input is to produce a Kolyvagin system using congruences of modular forms. In the rank-$1$ case, we used Heegner points coming from modular curves. But in the rank-$0$ case, we will use (almost all) Shimura curves.

\subsection{On Shimura Curves}
Recall that we can realize $Y_0(N)$ as the quotient $\Gamma_0(N)\backslash\mc H$. This can be upgraded adelically to use the group $G=\op{PGL}_{2,\QQ}$.
\begin{notation}
	Let $G$ be a reductive group over a number field $F$. Then we define
	\[[G]\coloneqq G(F)\backslash G(\AA_F).\]
\end{notation}
\begin{example}
	For $G=\op{PGL}_{2,\QQ}$, one can define $K_\infty=\op{PSO}_2(\RR)$ and $K_f=G(\widehat\ZZ)$ so that $Y_0(1)(\CC)$ can be identified with $G(\QQ)\backslash G(\AA_\QQ)/K_\infty K_f$. The point is that $\mc H^\pm=G(\RR)/K_\infty$, and then one takes a further quotient. By modifying $K_f$, we can get different level structures $Y_0(N)$: one needs to use the compact open subgroup
	\[K_f(N)=\left\{\begin{bmatrix}
		a & b \\ c & d
	\end{bmatrix}\in G(\widehat{\ZZ}):c\equiv0\pmod N\right\}\subseteq\AA_{\QQ,f}.\]
\end{example}
Note that Heegner points are some CM points, which turn out to come from embeddings
\[\op{Res}_{F/\QQ}\mathbb G_m\into\op{PGL}_{2,\QQ}.\]
We would like for such embeddings to have $T(\AA_{\QQ,f})\cap K_f(N)$ to be a maximal compact subgroup of $T(\AA_{\QQ,f})$. Roughly speaking, this means that locally the embedding $F_v^\times\to\op{GL}_2(\QQ_p)$ should have the image of the intersection with $\op{GL}_2(\ZZ_p)$ to be given by $(\OO_F\otimes\ZZ_p)^\times$. In other words, the following diagram
% https://q.uiver.app/#q=WzAsNCxbMCwxLCIoRlxcb3RpbWVzXFxRUV9wKV5cXHRpbWVzIl0sWzEsMSwiXFxvcHtHTH1fMihcXFFRX3ApIl0sWzAsMCwiKFxcT09fRlxcb3RpbWVzXFxaWl9wKV5cXHRpbWVzIl0sWzEsMCwiXFxvcHtHTH1fMihcXFpaX3ApIl0sWzAsMV0sWzIsMF0sWzIsM10sWzMsMV1d&macro_url=https%3A%2F%2Fraw.githubusercontent.com%2FdFoiler%2Fnotes%2Fmaster%2Fnir.tex
\[\begin{tikzcd}[cramped]
	{(\OO_F\otimes\ZZ_p)^\times} & {\op{GL}_2(\ZZ_p)} \\
	{(F\otimes\QQ_p)^\times} & {\op{GL}_2(\QQ_p)}
	\arrow[from=1-1, to=1-2]
	\arrow[from=1-1, to=2-1]
	\arrow[from=1-2, to=2-2]
	\arrow[from=2-1, to=2-2]
\end{tikzcd}\]
should be a pullback when $p\mid N$.\footnote{There are some issues with $p\mid N$ here. If we are assuming that $p$ splits in $F$, then $K_f(N)_p$ is some Iwahori subgroup, and $(F\otimes\QQ_p)^\times=\QQ_p^\times\times\QQ_p^\times$, so the intersection should be $\ZZ_p^\times\times\ZZ_p^\times$. If $p$ does not split, this story is more complicated.} Now, a Heegner point basically corresponds to an embedding $T\into G$, inducing $[T]\to[G]$, which descends to
\[T(\QQ)\backslash T(\AA_{\QQ,f})/T(\widehat\ZZ)\to[G]/K_f(N),\]
where $T(\widehat\ZZ)=\prod_v\OO_{F_v}^\times$.
\begin{remark}
	It turns out that these Heegner points agree with the Heegner points from earlier in the course. For example, the corresponding elliptic curve will have CM by the given maximal orders in $F$. Roughly speaking, one has to track through the identification
	\[G(\QQ)\backslash G(\AA_\QQ)/K_\infty K_f(N)\to Y_0(N)\]
	sending $[T]/T(\widehat\ZZ)K_\infty$ to the CM points in $Y_0(N)$.
\end{remark}
This story can be generalized to quaternion algebras $B$ over $\QQ$ (i.e., four-dimensional central simple algebras over $\QQ$). Then we can set $G\coloneqq B^\times$ to be some (probably non-split) reductive group; for example, $B=M_2(\QQ)$ gives $G=\op{GL}_2(\QQ)$. Let $S$ be the set of places where $B$ is not split. By class field theory, $\#S$ is a finite set of even cardinality, and given any such set $S$, there is a unique quaternion algebra $B$ ramifying there. Indeed, this follows from considering the fundamental exact sequence
\[0\to\op{Br}\QQ\to\bigoplus_{v}\op{Br}\QQ_p\to\QQ/\ZZ\to0\]
and noting that quaternion algebras come from $\op{Br}K[2]$.
\begin{remark}
	A quadratic field $F$ embeds into $B$ if and only if all $v\in S$ are non-split in $F$. Indeed, if $F\subseteq B$, then $F_v\subseteq B_v$, so $v\notin S$. Conversely, if $v\notin S$, then one can use an explicit construction of $B$ to see that $F$ does not embed.
\end{remark}
To continue, we have to say something about automorphic representations. Fix some infinite-dimensional irreducible cuspidal automorphic representation $\pi$ of $G_B\coloneqq B^\times$. The Jacquet--Langlands correspondence implies that such $\pi$ are in bijection with irreducible automorphic cuspidal infinite-dimensional representations $\pi$ for which $\pi_v$ is discrete series at $v\in S$. Then one can base-change $\pi$ up to some automorphic representation $\pi_F$, and then one can take a tensor product $\pi_F\otimes\chi$ with any character $\chi\colon F^\times\backslash\AA_F^\times\to\CC^\times$. (We will require that $\chi|_{\AA_\QQ^\times}=\omega_\pi^{-1}$, where $\omega_\pi$ is the central character.)

We can now define an $L$-function $L(\pi_F\otimes\chi,s)$ using this automorphic representation, and there is a functional equation of the form
\[\varepsilon(\pi_F\otimes\chi,s)L(\pi_F\otimes\chi,s)=L(\pi_F^\lor\otimes\chi^\lor,s).\]
Under the assumption on the central character, this actually becomes
\[\varepsilon(\pi_F\otimes\chi,s)L(\pi_F\otimes\chi,s)=L(\pi_F\otimes\chi,s).\]
Now, it turns out that this global $\varepsilon(\pi_F\otimes\chi,s)$ can be expressed as a product of local $\varepsilon_v$s. It turns out that $\varepsilon_v=+1$ if and only if $\op{Hom}_{T(\QQ_v)}(\pi|_T\otimes\chi,\CC)\ne0$, and $\varepsilon_v=-1$ if and only if $\op{Hom}_{T(\QQ_v)}(\op{JL}_B(\pi)_v\otimes\chi,\CC)\ne0$, where $\op{JL}$ is the Jacquet--Langlands correspondences as before. Notably, this latter condition can only be satisfied with $v\in S$ already.
\begin{example}
	If $\pi_v$ is Steinberg and $\chi$ is trivial, hen one finds that $\pi_v=-1$ whenever $v$ is inert.
\end{example}
\begin{remark}
	One can show that the Heegner hypothesis (taken with $\chi=1$) implies that the global root number is $-1$.
\end{remark}

\end{document}