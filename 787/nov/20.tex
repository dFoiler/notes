% !TEX root = ../notes.tex

\documentclass[../notes.tex]{subfiles}

\begin{document}

\section{November 20}
We continue moving towards our Bertolini--Darmon Euler system.

\subsection{Uniformizations of Shimura Curves}
For our setting, $E$ is an elliptic curve of squarefree conductor $N$, and we let $f_E\in S_2(\Gamma_0(N))$ be the corresponding modular form. Now, choose an imaginary quadratic field $F$ whose discriminant is coprime to $N$, and we factor $N=N^+N^-$, where $N^+$ has the split primes and $N^-$ has the inert primes. One can compute that
\[\varepsilon(E/F)=(-1)^{\omega(N^-)+1},\]
where $\omega$ is the number of prime factors.
\begin{example}
	For example, the Heegner hypothesis has $N^-=1$, which means $\varepsilon(E/F)=-1$.
\end{example}
We now have two cases. Let's start with the case where $\omega(N^-)$ is even. Then there is a unique quaternion algebra $B$ ramified exactly at the prime factors of $N^-$, so it is split at $\RR$, and we set $G\coloneqq B^\times/\QQ^\times$ as usual. Notably, $G(\QQ_v)=\op{PGL}_2(\QQ_v)$. In this case, we use a Shimura curve $X^G(N^+)$, where $N^+$ is specifying the level. Let's explain this level: we are going to take a quotient by some subgroup $K\subseteq G(\AA_\QQ)$.
\begin{itemize}
	\item At finite places $v\nmid N^-$, we define the Iwahori subgroup
	\[\left\{\begin{bmatrix}
		a & b \\ c & d
	\end{bmatrix}\in\op{PGL}_2(\ZZ_v):c\equiv0\pmod{\varpi_v}\right\}.\]
	\item At finite places $v\mid N^-$, we note $G(\QQ_v)=B_v^\times/\QQ_v^\times$, which one can show is a compact profinite group by writing down the quaternion algebra.\footnote{In short, the kernel of the map $v\circ\mathrm{Nrd}\colon B_v^\times\to\ZZ$ is surjective with kernel $\OO_{B_v}^\times$. But the center $\QQ_v^\times$ lands in $2\ZZ$, so the quotient $\OO_{B_v}^\times/\ZZ_v^\times$ has index $2$ in $B_v^\times/\QQ_v^\times$.} We take $\OO_{B_v}^\times/\ZZ_v^\times$ to be our Iwahori subgroup.
	\item At $v=\infty$, we take the entire $M_2(\RR)$.
\end{itemize}
Then we define
\[X^B(N^+)\coloneqq G(\QQ)\backslash G(\AA_\QQ)/K_\infty K_f(N^+),\]
where $K_\infty=\op{SO}(2,\RR)$ covers all archimedean places, and $K_f(N^+)$ is the Iwahori subgroup at places dividing $N$ and is $G(\ZZ_v)$ at $v\nmid N$. Note that $K_\infty K_f(N^+)$ is some open compact set with positive finite volume, but the quotient $G(\QQ)\backslash G(\AA_\QQ)$ is compact (the only problem would be $G(\RR)$, which is compact in this case), so the total quotient $X^B(N^+)$ is discrete and compact and thus finite.

We can identify this large quotient with a quotient of $\mc H^\pm$. There is a maximal order $\OO_B\subseteq B$ which consists of the elements of $B$ which are Iwahori at $v\mid N^-$ and in $M_2(\ZZ_v)$ at $v\nmid N^-$, and it has a sub-order $\Gamma^B_0(N^+)$ which is further Iwahori at places $v\mid N^+$. Then one can check that
\[X^B(N^+)=\Gamma^B_0(N^+)^\times\backslash\mc H^\pm\]
by using the usual action of $\op{PGL}(2,\RR)$ on $\mc H^\pm$ with stabilizer $\op{SO}(2)$. (Here, $\OO_B(N^+)^\times$ can be seen to be the intersection of $K_f$ with $G(\QQ)$.) We are silently using some strong approximation here to show that there is only one orbit.
\begin{proposition}[Strong approximation]
	Let $G$ be a simple simply connected group over $\QQ$. Let $v$ be a place where $G(\QQ_v)$ is not compact. Then $G(\QQ)$ is dense in $G(\AA_\QQ^v)$.
\end{proposition}
\begin{example}
	Consider the subgroup $B^1\subseteq B^\times$ of elements of reduced norm $1$. Then it turns out that $B^1(\QQ)$ is dense in $B^1(\AA_\QQ^{v})$ as long as $B$ is not ramified at $v$. The point is that then $B^1(\QQ)K_f(N^+)$ surjects onto $B^1(\AA_\QQ^v)$.
\end{example}
\begin{remark}
	It turns out that $X^B(N^+)$ admits a model over $\ZZ$ as a moduli space of principally polarized abelian surfaces with multiplication by $\OO_B$ and some level structure. Here, having an $\mathcal O_B$-action is mildly technical: we require that the map $\iota\colon\OO_B\to\op{End}A$ to have
	\[\op{charpoly}(\iota(b);\op{Lie}A)=\op{charpoly}(b;B),\]
	where the latter characteristic polynomial is given by a reduced norm and trace.
\end{remark}
\begin{remark}
	It turns out that $N^->1$ implies that $X^B(N^+)$ is a projective curve over $\Spec\ZZ$ with good reduction away from $N$ and semistable reduction at $N$. Here, semistable reduction indicates that the reduction is a transverse union of smooth curves.
\end{remark}
\begin{theorem}[Cerednik--Drinfeld uniformization]
	Fix $N=N^+N^-$ and $B$ as above. When $v\mid N^-$, there is a uniformization $X^B(N^+)(\CC_v)=\Gamma\backslash\Omega_{\mathrm{Dr}}$, where $\Omega_{\mathrm{Dr}}$ is the Drinfeld upper-half plane $\PP^1\setminus\PP^1(\QQ_v)$.
\end{theorem}
\begin{proof}[Idea]
	One obtains this uniformization by viewing the $\CC_v$-points as
	\[G'(\QQ)\backslash\Omega_{\mathrm{Dr}}\times G(\AA_\QQ^{\infty,p})/K(N^+)^{\infty,v},\]
	where $G'$ is the group given by the quaternion algebra which is unramified exactly at the places $(S_B\setminus\{v\})\cup\{\infty\}$.
\end{proof}
\begin{remark}
	Now, $\Omega_{\mathrm{Dr}}$ has its special fiber look like transverse unions of $\PP^1$s, and the dual graph is the Bruhat--Tits tree of $\op{PGL}_2(\QQ_v)$: the vertices are given by $\ZZ_p$-lattices $\Lambda$ in $\QQ_p^{\oplus2}$ (up to scaling by $\QQ_p^\times$), where $\Lambda$ and $\Lambda'$ admit an edge if and only if $\Lambda\subseteq\Lambda'\subseteq\frac1p\Lambda$. This Bruhat--Tits tree admits a coloring $\{\pm\}$, where the color $+$ is given by matrices in $\op{PGL}_2(\QQ_p)$ with square determinant. These components turn out to be parameterized by the groupoid
	\[X^{B'}(N^+)\coloneqq G'(\QQ)\backslash G'(\AA^\infty)/K_f(N^+).\]
	The other half can be recovered by some kind of involution.
\end{remark}
\begin{example}
	Let's describe the Bruhat--Tits tree for $\op{PGL}_2(\QQ_2)$. There is a distinguished lattice $\Lambda_0=\QQ_2^2$, and then it turns out that this is a $3$-regular graph.
\end{example}

\end{document}
