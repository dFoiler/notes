% !TEX root = ../notes.tex

\documentclass[../notes.tex]{subfiles}

\begin{document}

\section{November 6}
Today, we explain Kolyvagin's proof a bit.

\subsection{Birch--Swinnerton-Dyer, \'a la Mazur--Tate}
Let's explain how one might have come up with Kolyvagin's argument. In the late 1980s, Mazur and Tate gave a variant of the Birch--Swinnerton-Dyer conjecture. Recall that the Birch--Swinnerton-Dyer conjecture asserts that
\[\op{rank}E(K)=\ord_{s=1}L(E/K,s).\]
Roughly speaking, $L(E/K,s)$ can really only be understood once we attach an automorphic representation $\pi$ of $\op{GL}_2(\AA_K)$ to $E$, and the parameter $s$ basically corresponds to evaluating an $L$-function on the automorphic representation $\pi\otimes\left|\cdot\right|^s$.
\begin{notation}
	Fix some automorphic representation $\pi$ of $\op{GL}_n(\AA_K)$, and choose some $s\in\CC$. As in the construction of Dirichlet $L$-functions, we can choose a Galois character $\chi\colon\op{Gal}(L/K)\to\CC^\times$, where $L/K$ is a finite abelian extension. Thus, the map
	\[\chi\mapsto L(\pi\otimes\chi,s)\in\CC\]
	is a functional on characters, so it produces an element $\theta_L\in\CC[\op{Gal}(L/K)]$ such that $\theta_L(\chi)=L(\pi\otimes\chi,s)$.
\end{notation}
\begin{remark}
	Thus, we get a system
	\[\{\theta_L\in\CC[\op{Gal}(L/K)]:\text{abelian }L/K\}\]
	of group elements. This could be some more algebraic variant of the $L$-function.
\end{remark}
\begin{remark}
	Now take $s$ to be the central value. Then there is a way to normalize $\theta_L$ so that it almost lies in $\ZZ[\op{Gal}(L/K)]$. Notably, we may not be able to fully get in $\ZZ[\op{Gal}(L/K)]$ due to some algebraic integers or controlled denominators. Approximately speaking, one normalizes by dividing out by some periods.
\end{remark}
\begin{remark}
	When $\pi$ comes from a weight-$2$ Hecke eigenform $f$ over $\QQ$, there is an explicit construction of the $\theta_L$s in terms of modular symbols of $f$.
\end{remark}
We would now like a way to compute a vanishing order.
\begin{definition}[vanishing order]
	Fix a group ring $R[G]$, and let $I_G$ be the kernel of the trace map $R[G]\to R$. For an element $\theta\in R[G]$, we define its \textit{vanishing order} as
	\[\ord_G\theta\coloneqq\sup\left\{r\ge0:\theta\in I^r_G\right\}.\]
\end{definition}
\begin{remark}
	Approximately speaking, this definition is analogous to saying that the order of vanishing at $s=0$ of some power series $f\in\CC[[s]]$ is
	\[\ord f=\sup\{r\ge0:f\in(s)^r\}.\]
\end{remark}
\begin{example}
	It is possible that $\ord\theta=\infty$ because the sequence $I_G\supseteq I_G^2\supseteq\cdots$ may eventually stabilize. For example, if $G$ is cyclic of order $n$ generated by $\sigma$, then
	\[\CC[G]=\CC[x]/\left((x+1)^n-1\right),\]
	where $x$ is $\sigma-1$. We thus see that $I_G=(x)$, and all its powers are also $(x)$. By summing, we see that $I_G^2=I_G$ for any finite abelian group $G$.
\end{example}
Roughly speaking, the failure of $\ord\theta$ to be interesting for $\CC[G]$ when $G$ is finite amounts to the fact that $\op{gr}R[G]$ is frequently uninteresting. To get something interesting, one usually has to take $R=\ZZ_p$.
\begin{lemma}
	Fix a finite abelian group $G$, and we work with the group ring $\ZZ_p[G]$.
	\begin{listalph}
		\item If $p\nmid\left|G\right|$, then $I_G=I_G^2$.
		\item If $G\cong\ZZ/p^m\ZZ$, then
		\[\op{gr}\ZZ_p[G]\cong\frac{\ZZ_p[x]}{\left(p^mx\right)},\]
		and this isomorphism sends $I_G$ to the principal ideal $(x)$.
	\end{listalph}
\end{lemma}
\begin{proof}
	Omitted.
\end{proof}
\begin{example}
	In (b), these rings are rather complicated. As one instance of this, much of Iwasawa theory is interested in the limit
	\[\Lambda\coloneqq\lim\ZZ_p[\ZZ/p^\bullet\ZZ].\]
\end{example}
Of course, we are not so interested in the order of a single element but rather the entire family $\{\theta_L\}$.
\begin{definition}[vanishing order]
	Fix a family $\mc F$ of finite abelian extensions $L/K$ and an elliptic curve $E$ over $K$, and choose a prime $p$. Then we define the \textit{vanishing order}
	\[\ord_{\mc F}\theta\coloneqq\inf_{L\in\mc L}\op{ord}_{\ZZ_p[\op{Gal}(L/K)]}\theta_L.\]
\end{definition}
There are many different choices one could make for $\mc L$.
\begin{example}
	For a prime $p$, we may let $\mc F_{p^\infty}$ be the family of finite abelian $p$-power extensions $L/K$ unramified outside $p$. Equivalently, these are the finite subextensions of the maximal abelian pro-$p$ extension $K_{p^\infty}$ of $K$.
\end{example}
\begin{example}
	For a prime $p$, we may let $\mc F_p$ denote the family of extensions $K(\mu_n)$ where $\gcd(n,p)=1$.
\end{example}
We are now ready to state our conjecture.
\begin{conj}[Mazur--Tate]
	Let $\mc F$ be the family of all finite abelian extensions of $\QQ$. Then for good ordinary primes $p$ for a weight-$2$ modular Hecke eigenform $f$, we have
	\[\ord_{\mc F}\theta=\op{ord}_{s=1}L(f,s).\]
\end{conj}
Very little is known about this conjecture. It is somehow analogous to the Birch--Swinnerton-Dyer conjecture. There is also a variant for the $p$-adic $L$-function: it turns out that the Mazur--Tate vanishing order is the same as the vanishing order of the $p$-adic $L$-function for the extension $\QQ\subseteq\QQ(\mu_{p^\infty})$, basically by construction of the $p$-adic $L$-function.

\subsection{Why the Derivative?}
We are now ready to explain where the Kolyvagin derivative may have come from. Kolyvagin constructs points $y_n\in E(K[n])$, and we set $G_n\coloneqq\op{Gal}(K[n]/K[1])$. As before, we define
\[\theta_n\coloneqq\sum_{\sigma\in G_n}\sigma y_n\otimes\sigma\in E(K[n])\otimes\ZZ_p[G_n]\]
to be some element in $E(K[n])\otimes\ZZ_p[G_n]$. This should encode the $L$-function. Indeed, for a character $\chi\colon G_n\to\CC^\times$, then
\[\chi^{-1}(\theta_n)=\sum_{\sigma\in G_n}\chi^{-1}(\sigma)\sigma(y_m)\in E(K[n])\otimes\CC.\]
One can check directly that this is in the $\chi$-eigenspace, and it remembers the $L$-function as follows.
\begin{theorem}[generalized Gross--Zagier]
	Fix everything as above. Then
	\[\langle\chi(\theta_n),\chi(\theta_n)\rangle_{\mathrm{NT}}=L'(E,\chi,1).\]
\end{theorem}
% \begin{remark}
% 	The Gross--Zagier formula turns out to show that
% 	\[\langle\chi(\theta_n),\chi(\theta_n)\rangle_{\mathrm{NT}}=L'(E,\chi,1),\]
% 	so $\chi(\theta_n)$ vanishing detects some first derivative, even as we vary over our $\chi$s. This explains why the elements $\theta_n$ encode the $L$-function.
% \end{remark}
\begin{notation}
	Fix everything as above, and work over the ring $\ZZ_p$ everywhere. Then we define
	\[\ord\theta_n\coloneqq\sup\{r\ge0:\theta_n\in E(K[n])\otimes I_G^r\}.\]
\end{notation}
\begin{example}
	We see that $\ord\theta_n$ vanishes if and only if $\theta_n\notin I_G$, which is equivalent to
	\[\chi_{\mathrm{triv}}(\theta_n)=\sum_{\sigma\in G_n}\sigma(y_n)\]
	failing to vanish in $E(K[n])\otimes\ZZ_p$. But this element is just the trace of $y_n$ and therefore some explicit multiple $\prod_\ell a_\ell$ of $y_1\in E(K[1])$, where $a_\ell$ is the usual Frobenius trace. For example, when $n=1$, we are asking for $y_1$ to be nonzero.
\end{example}
We are now ready to find the derivative. Continuing the previous example, suppose now that $G$ is cyclic of order $\ell$, generated by $\sigma_\ell$. Then
\[\ZZ_p[G]=\frac{\ZZ_p[x]}{\left((x+1)^{\ell+1}-1\right)},\]
and here $I_G=(x)$. Under this expansion, we write $\theta_n=\sum_{i\ge1}^{\ell+1}a_ix^i$, and it turns out that $\ord\theta_\ell=1$ if and only if $(\ell+1)\nmid a_1$ in $\ZZ_p$. This is some elementary exercise, boiling down to using the fact that $p^mx$ appears in the denominator of the associated graded ring, where $m\coloneqq\op{ord}_p(\ell+1)$. Translating this back, if we just write
\[\theta_\ell=\sum_i\lambda_i\sigma_\ell^i,\]
then $a_1$ is the derivative of $\theta_\ell$ with respect to $\theta$ and evaluating at $1$; namely, $a_1$ is $\sum_ii\lambda_i$. But one can calculate this derivative is $\mathbb D_\ell y_\ell\in E(K[\ell])$, and we see that $\ord\theta_\ell=1$ if and only if $\mathbb D_\ell y_\ell$ is nonzero in $E(K[n])\otimes\ZZ_p/(\ell+1)\ZZ_p$. And indeed, what is needed for Kolyvagin's argument was to find some nonzero elements in some Selmer groups.
\begin{remark}
	On the homework, we take higher-order derivatives. It seems plausible that it should give the same order of vanishing as we saw in the Kolyvagin system (perhaps if $p$ is larger than the order of vanishing).
\end{remark}
We close our lecture with a conjecture of Mazur.
\begin{conj}[Mazur]
	Fix some $n$, and let a prime $p$ vary. Then for almost all $\chi\colon G_{np^m}\to\CC^\times$, the element $\theta_n^\chi$ is not torsion.
\end{conj}
In other words, the $\theta_n^\chi$ should generically have vanishing order $1$. This is known due to Vatsal--Cornut, and it is the next goal of the class.

\end{document}