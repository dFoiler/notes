% !TEX root = ../notes.tex

\documentclass[../notes.tex]{subfiles}

\begin{document}

\section{November 18}
We continue discussing the Bertolini--Darmon Euler system.

\subsection{The Jacquet--Langlands Correspondence}
Fix an elliptic curve $E$ over $\mathbb Q$, and choose some imaginary quadratic field $F$. We also choose a character $\chi\colon\op{Gal}(F^{\mathrm{ab}}/F)\to\mathbb C^\times$ which factors through some ring class field. Recall that this means that $\chi$ viewed as a character on $F^\times\backslash\AA_F^\times$ factors through $F^\times\backslash\AA_F^\times/\widehat{\mathcal O}_{F,n}^\times F_\infty^\times$ for some positive integer $n$. It is equivalent to ask for the character to factor through $F^\times\backslash\AA_F^\times/\AA_{\mathbb Q}^\times$, meaning that $\chi|_{\mathbb A_{\mathbb Q}^\times}=1$. Alternatively, we could say that $\chi$ is a representation of $[T/\mathbb G_m]$, where $T=\operatorname{Res}_{F/\mathbb Q}\mathbb G_{m,F}$.
\begin{remark}
	Later, we will even require that $\chi$ is trivial, but we will allow $\chi$ to be nontrivial in order to state Waldspurger's theorem.
\end{remark}
We are interested in the following result.
\begin{theorem}[Bertolini--Darmon] \label{thm:bertolini-darmon}
	Fix everything as above. If $L(E,\chi,1)\ne0$, then $\left(E(F^{\mathrm{ab}})\otimes\mathbb C\right)^\chi=0$, and $\Sha(E/F^{\mathrm{ab}})^\chi$ is finite.
\end{theorem}
We will not bother to explain the precise meaning of $\Sha(E/F^{\mathrm{ab}})^\chi$. For now, we will focus on the case where $\chi$ is trivial.
\begin{theorem}[Bertolini--Darmon]
	Fix everything as above, and assume that $E$ does not have potential complex multiplication. If $L(E,\chi,1)\ne0$, then $\op{Sel}_p(E/F)=0$ for large primes $p$.
\end{theorem}
Our story with Kolyvagin's theorem used the Gross--Zagier formula as a key input in order to make ``geometric sense'' of the hypothesis that $L'(E,1)\ne0$. To prove the above theorem, we will use Waldspurger's formula to make ``geometric sense'' of the hypothesis that $L(E,\chi,1)\ne0$.

Thus, we need to recall Waldspurger's formula. This will require us to recall the Jacquet--Langlands correspondence for twists of $\op{PGL}_{2,\QQ}$.
\begin{notation}
	Given a quaternion algebra $B$ over $\QQ$, we let $G_B$ denote the algebraic group $B^\times/\QQ^\times$ over $\QQ$, and we let $S_B$ denote the set of places where $B$ is ramified. In other words, a place $v$ is in $S_B$ if and only if $B_v$ is a division algebra (rather than splitting as $M_2(\QQ_v)$). If $B$ is obvious from context, we will omit it.
\end{notation}
\begin{example}
	If $B=M_2(\mathbb Q)$, then $G_B=\op{PGL}_{2,\QQ}$.
\end{example}
\begin{remark}
	The set $S_B$ is always even in size. To see this, identify quaternion algebras over a field $F'$ with classes in $\op{Br} F'\coloneqq\mathrm H^2(F';\mathbb G_m)[2]$, the fundamental exact sequence
	\[0\to\op{Br}\QQ\to\bigoplus_v\op{Br}\QQ_v\to\QQ/\ZZ\to0\]
	implies that $\sum_v\op{inv}_vB=0$, so the rational number $\op{inv}_vB\in\{0,1/2\}$ may be nonzero at an even number of places. Conversely, for any finite set $S$ of places of even size, this short exact sequence shows that there is a unique $B\in\op{Br}\QQ$ with $S_B=S$.
\end{remark}
\begin{notation}
	Given a reductive group $G$, let $\mc A([G])$ denote the cuspidal automorphic representations of $G$. We also let $\mc A_0([G])$ denote the infinite-dimensional cuspidal automorphic representations.
\end{notation}
\begin{theorem}[Jacquet--Langlands]
	Fix a quaternion algebra $B$ over $\QQ$. Then the Jacquet--Langlands correspondence is an injection
	\[\op{JL}_B\colon\mc A_0([G_B])\to\mc A_0([\mathrm{PGL}_{2,\QQ}])\]
	satisfying the following properties.
	\begin{listalph}
		\item The embedding $\op{JL}_B$ has image given by the $\pi'\in\mc A_0([\mathrm{PGL}_{2,\QQ}])$ for which $\pi'_v$ is a discrete series representation for each $v\in S_B$.
		\item For each $\pi\in\mc A_0([G_B])$ and $v\notin S_B$, we have $\pi_v=\op{JL}_B(\pi)_v$.
		% \item For each $\pi\in\mc A_0([G_B])$ and $v\in S_B$, we have that $\op{JL}_B(\pi_v)$ is a discrete series representation.
	\end{listalph}
\end{theorem}
\begin{example}
	As a sanity check, we note that the correspondence is the identity when $B=M_2(\QQ)$. Indeed, the condition in (a) has no content because $S_B=\emp$.
\end{example}
% In order to use the hypothesis, we recall Waldspurger's formula. We work with the group $G$ given by $B^\times/\mathbb Q^\times$, where $B$ is some quaternion algebra over $G$. Then there is a Jacquet--Langlands correspondence between infinite-dimensional cuspidal automorphic representations on $G$ and the cuspidal automorphic representations of $\op{PGL}_2$ which are discrete series at the ramified places of $B$, and they are the same at unramified places. For example, $B$ being unramified everywhere is equivalent to $G=\op{PGL}_2$, so the correspondence has no content.
Here is an amusing application, whose proof is as hard as possible.
\begin{corollary}
	Let $B$ be the unique quaternion algebra ramified at exactly the places $\{11,\infty\}$. Define a compact open subgroup $K\subseteq G_B(\AA_\QQ)$ as follows. We have $K_\infty\subseteq B_\infty^\times/\RR^\times$ to consist of the norm-$1$ elements in the quaternion algebra $B_\infty=\HH$, which is homeomorphic to $S^3$ and therefore compact. For finite $v\ne11$, we define $K_v=\op{PGL}_2(\ZZ_\ell)$. Lastly, for $v=11$, we define $K_v\subseteq B_{11}^\times/\QQ_{11}^\times$ to be the elements with reduced norm $1$. Then the fact that the genus of $X_0(11)$ is $1$ implies that
	\[G_B(\QQ)\backslash G_B(\AA_\QQ)/K\]
	has $2$ elements.
\end{corollary}
\begin{proof}
	Quickly, observe that $[G_B]=G_B(\QQ)\backslash G_B(\AA_\QQ)$ is itself compact: indeed, there is no split torus after taking the quotient by the center. Thus, we may identify $C^\infty([G_B])$ with $C_c^\infty([G_B])$.
	
	Now, to show that $G_B(\QQ)\backslash G_B(\AA_\QQ)/K$ has two elements, we may as well show that $C_c^\infty([G_B]/K)$ is $2$-dimensional. Observe that we may identify $C_c^\infty([G_B]/K)$ with $C_c^\infty([G_B])^K$, which is useful because it allows us to introduce automorphic representations: there is a factorization of $C_c^\infty([G_B])$ as the sum of automorphic representations, which we go ahead and expand as
	\[C_c^\infty([G_B])=\bigoplus_{\dim\pi>\infty}\pi\oplus\bigoplus_{\dim\eta=1}\eta.\]
	By taking $K$-invariants, we see
	\[C_c^\infty([G_B])^K=\bigoplus_{\dim\pi>\infty}\pi^K\oplus\bigoplus_{\dim\eta=1}\eta^K.\]
	It turns out that all summands are at most one-dimensional by a multiplicity $1$ result, so it remains to count how many of our automorphic representations admit $K$-invariant vectors.
	\begin{itemize}
		\item Because they are easier, let's start with the one-dimensional representations $\eta\colon[G]\to\CC^\times$. Automorphic representations admit a factorization $\eta=\bigotimes_v\eta_v$ where $\eta_v$ is a representation $G_B(\QQ_v)\to\CC^\times$. Because the target is abelian, we see that $\eta$ factors through the abelianization, which means that it factors through $\det\colon G_B\to\mathbb G_m$, where $\det$ refers to the reduced norm at the ramified primes.
		
		On the other hand, $\dim\eta^K=1$ is equivalent to requiring that $K_v\subseteq\ker\eta_v$ for each $v$. But the construction of $K_v$ shows that the reduced norm $K_v$ surjects onto $F_v^\times$ for each $v$, so we conclude that $\eta$ must be trivial. %means that $\eta_v$ factors through the reduced norm.\todo{} One finds that the only interesting character $\eta$ here comes from the unique nontrivial quadratic Dirichlet character$\pmod{11}$.

		\item We now use the Jacquet--Langlands correspondence to understand the infinite-dimensional representations. In particular, each infinite-dimensional $\pi\subseteq C_c^\infty([G_B])$ produces some
		\[\op{JL}_B(\pi)\subseteq C_c^\infty([\mathrm{PGL}_{2,\QQ}]).\]
		Note that $\pi_\infty$ needs to be in the discrete series, so it follows that $\op{JL}_B(\pi)$ needs to correspond to a weight-$2$ cusp form. Additionally, with $\pi_v$ admitting a $\op{PGL}_2(\ZZ_v)$-fixed vector away from $\{11,\infty\}$ implies that the representation is unramified there. %\todo{}
		
		In total, we find that $\op{JL}_B(\pi)$ arises from a weight-$2$ cusp form of level dividing $11$. But $X_0(11)$ has genus $1$, so the space of its differential forms is $1$-dimensional, so the space $S_2(\Gamma_0(11))$ is one-dimensional. Thus, there is only one representation here.
	\end{itemize}
	Totaling the above two cases completes the proof.
	% Because $X_0(11)$ has genus $1$, the dimension of the space of differential forms on $X_0(11)$ is $1$, so the space $S_2(\Gamma_0(11))$ is one-dimensional.
	%
	% It turns out that the modular curve $X_0(11)$ has genus $1$. Then there is a unique cuspidal automorphic representation of $\op{PGL}_2(\mathbb A_{\mathbb Q})$ for which $\pi_\infty$ is a weight-$2$ discrete series and $\pi_{11}$ is Steinberg (up to unramified twists), and $\pi_\ell$ is unramified for $\ell\ne11$. Accordingly, we may let $B$ be the quaternion algebra ramified at exactly $\infty$ and $11$, and we let $G=B^\times/\mathbb Q^\times$; note $[G]$ is compact. Then we define our compact subgroup $K\subseteq G(\mathbb A_{\mathbb Q})$ to be $B_\infty^\times$ at the infinite places and $\mathcal O^\times_{B_v}$ for some maximal order $\mathcal O_{B_v}\subseteq B_v$ for each finite place $v$. Explicitly, for finite $v$ away from $11$, we can just take $\op{GL}_2(\mathbb Z_v)$, and at $v=11$, we can take the invertible elements in $B_v$ with reduced norm $1$. Now, the double quotient
	% \[G(\mathbb Q)\backslash G(\mathbb A_{\mathbb Q})/K\]
	% has two elements. Indeed, the space of functions on this space is $C([G])^K$, and $C([G])$ factors as a sum of the infinite-dimensional cuspidal automorphic representations plus the characters. Taking $K$-invariants means that the dimension is equal to the number of representations, which is simply the constant representation and the one arising from the Jacquet--Langlands correspondence. The uniqueness of the one-dimensional representation more or less follows because such things correspond to Dirichlet characters$\pmod{11}$.
\end{proof}
\begin{remark}
	If we wanted to actually describe the $K$-invariant infinite-dimensional $\pi$, then we now see that it is generated by the unique $K$-invariant automorphic form $\varphi\colon[G]\to\mathbb C$ which is orthogonal to the constant function. Approximately speaking, this function sends the class $[1]$ to $1$, and it sends the other class to $-1$ up to some scalar. The span of this function gives $\pi$.
\end{remark}

\subsection{Waldspurger's Formula}
We are now ready to state Waldspurger's formula.
\begin{theorem}[Waldspurger] \label{thm:waldspurger}
	Fix everything as above, now letting $T$ be the torus $\op{Res}_{F/\mathbb Q}\mathbb G_{m,F}$. Let $\chi\colon[T]\to\CC^\times$ be some character. Fix some automorphic representation $\pi'$ of $\op{PGL}_2$.
	\begin{listalph}
		\item Given some quaternion algebra $B$ with an embedding $T\into B^\times/\mathbb Q^\times$, for any $\pi^B\subseteq C([G])$ and $\varphi\in\pi^B$, we have
		\[\left|\int_{[T]}\varphi(t)\chi(t)\,dt\right|\sim L(\pi',\chi,1/2),\]
		where the $\sim$ is hiding some infinite product of local terms depending on $\varphi$, among other global constants.
		\item If $L(\pi',\chi,1/2)\ne0$, then there is a unique quaternion algebra $B$ for which $\pi^B\coloneqq\op{JL}^{-1}(\pi')$ admits some $\varphi$ with $\int_{[T]}\varphi\chi\ne0$.
		\item The unique $B$ in (b) is characterized by ramifying at exactly the places $v$ for which
		\[\op{Hom}_{T(\QQ_v)}(\pi_v^{B_v}\otimes\chi,\CC)\ne0.\]
		Equivalently, we are asking for the local root number $\varepsilon_v(\pi_v,\chi_v,1/2)$ to be $-1$.
	\end{listalph}
\end{theorem}
\begin{remark}
	As written no part of the theorem follows from any other part. However, if (a) were more explicit about $\sim$ (and in particular the relevant local factors), then a nontrivial calculation would show that (b) and (c) follow from (a).
\end{remark}
\begin{example}
	Consider $X_0(11)$ again, and let $f$ be a nonzero modular form in $S_2(\Gamma_0(11))$, which is unique up to scalar. This corresponds to the automorphic representation $\pi'=\bigotimes_v\pi_v'$ which has $\pi'_{11}=\mathrm{St}\otimes\eta$ for some unramified quadratic character $\eta\colon\QQ_{11}^\times\to\{\pm1\}$. It turns out that $L(\pi',1/2)\ne0$, but $\int_{[T]}\varphi=0$ for all $\varphi\in\pi'$, so the unique quaternion algebra of \Cref{thm:waldspurger} is the quaternion algebra $B$ with $S_B=\{11,\infty\}$.%\todo{}
\end{example}
Let's apply our currently knowledge to sketch how one can prove low-rank cases of the Birch--Swinnerton-Dyer conjecture. As usual, choose an elliptic curve $E$ of squarefree conductor $N$, and let $F$ be an imaginary quadratic field of discriminant coprime to $N$. Further, consider the representation $\pi'=\bigotimes_v\pi'_v$ given by $\pi'_v$ being unramified for $v\nmid N$ and being some unramified twist of Steinberg at each $v\mid N$.

Then our root number $\varepsilon(E/F)$ factors as
\[\varepsilon(E/F)=(-1)^{\#\{\ell\mid N:\ell\text{ is inert in }F\}+1},\]
where the extra $+1$ comes from $\infty$. Let $S$ be the set in the exponent, along with $\infty$.
\begin{example}
	Under the Heegner hypothesis, one has $S=\{\infty\}$, so $\varepsilon(E/F)=-1$.
\end{example}
We now have two cases.
\begin{itemize}
	\item Gross--Zagier--Kolyvagin: when the size of $S$ is odd, then we may find $B$ ramified exactly at the finite places in $S$, and we can do a Gross--Zagier approach to find Heegner points by using a uniformization map from a Shimura curve to $E$. The rest of the story proceeds as it does with Kolyvagin's theorem. We will spend the next few lectures explaining this story.
	\item Waldspurger--Bertolini--Darmon: when the size of $S$ is even, we may find $B$ ramified at exactly the places in $S$, and \Cref{thm:waldspurger} provides us with some $\varphi\in\op{JL}^{-1}(\pi')$ for which $L(E/F,1)$ is $\left|\int_{[T]}\varphi\right|^2$ up to some nonzero constant.
\end{itemize}
% \begin{example}
% 	 Now, choose an imaginary quadratic field $F$ of discriminant coprime to $N$. Then the
% 	\[\varepsilon(E/F)=(-1)^{\#\{\ell\mid N:\ell\text{ is inert in }F\}+1}.\]
% 	(The $+1$ is present here due to $\infty$.) Let $S$ be the set in the exponent, along with $\infty$. For example, under the Heegner hypothesis, $S=\{\infty\}$, and it follows that $\varepsilon(E/F)=-1$.
% 	\begin{itemize}
% 		\item 
% 		\item W
% 	\end{itemize}
% \end{example}
To continue our story from the first point, we still must do something with the integral
\[\int_{[T]}\varphi\ne0.\]
Now, suppose that $\varphi\colon[G]/K\to\CC$ to be primitive. The Hecke eigenvalues in this case are rational numbers, so by clearing denominators, we may assume that $\varphi$ outputs to $\ZZ$ and is primitive.
\begin{theorem}
	Fix a large prime $p$. If
	\[\int_{[T]}\varphi=\sum_{t\in\op{Pic}\OO_F}\varphi(t)\]
	does not vanish$\pmod p$, then $\op{Sel}_p(E/F)=0$.
\end{theorem}
This is akin to Kolyvagin's theorem. To produce our classes, we let $X$ be the Shimura curve attached to the quaternion algebra which is ramified at the finite places of $S$ and at some new prime $q$. Then we produce some classes in $\mathrm H^1(F;\ov\rho_E)$, which make a Kolyvagin system $\{c(q)\}_q$, and we can prove the result. The integral condition gets used because one finds $\int_{[T]}\varphi$ is nonzero$\pmod p$ if and only if $\op{loc}_qc(q)\ne0$ for each $q$.

\end{document}
