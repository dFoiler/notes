% !TEX root = ../notes.tex

\documentclass[../notes.tex]{subfiles}

\begin{document}

\section{November 25}
We continue moving towards our Bertolini--Darmon Euler system.

\subsection{Level-Raising}
Now, by Jacquet--Langlands, automorphic forms on $X^B(N^+)$ should correspond to automorphic forms on the groupoid $X^{B'}(N^+)$,
% \[X^{B'}(N^+)\coloneqq G'(\QQ)\backslash G'(\AA^\infty)/K_f(N^+),\]
where $G'$ is the group stated above. However, Jacquet--Langlands tells us that automorphic forms on the former Shimura curve arise from modular forms in $S_2(\Gamma_0(N^+N^-))$, and automorphic forms on the latter groupoid arise from $S_2(\Gamma_0(N^+N^-/p))$. We are now motivated to state the following theorem, which follows from the discussion above plus the weight spectral sequence.
\begin{theorem}[Level raising] \label{thm:level-raising}
	Choose a normalized newform $f\in S_2(\Gamma_0(N))$. Furthermore, choose primes $p$ and $\ell$ for which $a_\ell\equiv(\ell+1)\pmod p$. Then there is a normalized newform $g\in S_2(\Gamma_0(N\ell);F)$ and a prime $\mf p\subseteq\OO_F$ above $p$ for which
	\[f\equiv g\pmod{\mf p}.\]
	% provided that $\ell$ is an admissible prime for $f$, meaning that $\ell\nmid N$ and $a_p(f)\equiv\pm(p+1)\pmod\ell$.
\end{theorem}
\begin{remark} \label{rem:level-raising-condition}
	The ``level-raising'' condition more or less amounts to the reduced Galois representation $\ov\rho_f\colon\op{Gal}(\ov\QQ/\QQ)\to\op{GL}_2(\ZZ_\ell)$ having
	\[\ov\rho_f(\mathrm{Frob}_\ell)\equiv\pm\op{diag}(\ell,1)\pmod p.\]
	This is more or less a requirement to do level-raising because $\rho_g\colon\op{Gal}(\ov\QQ/\QQ)\to\op{GL}_2(\ZZ_p)$ needs to be ramified at $p$ but not too badly. For example, if $g$ should correspond to an elliptic curve $E$, which we will say for example has split multiplicative reduction, then we can use the Tate uniformization
	\[\mathbb G_m(\ov\QQ_\ell)/q_E^\ZZ\to E(\ov\QQ_\ell)\]
	to describe the $p$-adic Tate module: we have a short exact sequence
	\[0\to\mu_p\to E[p]\to\FF_p\to0\]
	of $\op{Gal}(\ov\QQ_\ell/\QQ_\ell)$-modules. It follows that $\ov\rho_{E,p}$ takes the form $\begin{bsmallmatrix}
		\chi_{\mathrm{cyclo}} & * \\ & 1
	\end{bsmallmatrix}$, which is the matrix $\begin{bsmallmatrix}
		\ell & * \\ & 1
	\end{bsmallmatrix}$ upon plugging in $\mathrm{Frob}_\ell$.
\end{remark}
\begin{remark}
	One can weaken the hypothesis that $f$ and $g$ have rational coefficients by extending $\QQ$ appropriately and choosing a prime $\mf p$ above $p$. In fact, it is conjectured that $g$ will never have rational coefficients if $p$ is large enough.
\end{remark}
\begin{example}
	Here is an example for a different weight. Then there is some $\Delta\in S_{12}(\Gamma_0(1))$, which has $q$-expansion
	\[\Delta=q\prod_{n\ge1}\left(1-q^n\right)^{24}.\]
	Then
	\[\Delta\equiv q\prod_{n\ge1}\left(1-q^n\right)^2\left(1-q^{11n}\right)^2\pmod{11},\]
	and it turns out that the right-hand side is the unique cuspidal newform in $S_{2}(\Gamma_0(11))$. Indeed, this right-hand side is $(\Delta(\tau)\Delta(11\tau))^{1/12}$ and uniqueness follows by a dimension calculation.
\end{example}

\subsection{The Construction of Classes}
We are now ready to construct the Bertolini--Darmon Kolyvagin system. Start with $f\in S_2(\Gamma_0(N))$ associated to an elliptic curve $E$ over $\QQ$, and we assume that $N$ is squarefree for simplicity. We further assume that
\[L(E/F,1)\ne0\]
as usual, so the root number is $+1$. As before, we factor $N=N^+N^-$, where $N^+$ has the primes of $N$ which split in $F$, and $N^-$ has the primes of $N$ which are inert in $F$. Last class, we defined a Shimura set $X^{B}(N^+)$, where $B$ is ramified exactly at $N^-\infty$. An embedding $F\into B$ induces an embedding $T\to G$ where $T=(\op{Res}_{F/\QQ}\mathbb G_m)/\mathbb G_{m,\QQ}$, so we have produced a map
\[[T]\to X^{B}(N^+).\]
Now, $[T]$ is $F^\times\backslash\AA_{F,f}^\times/\widehat\OO_F^\times$, which is the class group. By Waldspurger's formula, it follows that there is a function $\varphi$ on $X^B(N^+)$ for which
\[\int_{[T]}\varphi\ne0,\]
but of course, this integral is just some finite sum.

We are interested in finding some classes in $\mathrm H^1(F;V)$, where $V$ is the Galois module $E[p]$. Let's do this in the ``base case'' first. Fix a prime $p$ for which $\op{Gal}(\ov\QQ/\QQ)\to\op{GL}(E[p])$ is surjective. Choose a ``level-raising'' prime $\ell$ for which $a_\ell(f)\equiv\pm(\ell+1)\pmod p$ and $\ell\not\equiv-1\pmod p$ and $\ell$ is inert in $F$. Now, choose  $g\in S_2(\Gamma_0(N\ell))$ by \Cref{thm:level-raising} so that $\ov\rho_f=\ov\rho_g$.
\begin{remark}
	The condition that $a_\ell(f)\equiv\pm(\ell+1)$ is equivalent to having $E[p]\cong\FF_p\oplus\FF_p(1)$ as an unramified module over $\op{Gal}(\ov\QQ_\ell/\QQ_\ell)$. Indeed, it really just amounts to asking what the action of $\mathrm{Frob}_\ell$ is, which is determined by its characteristic polynomial due to the hypothesis on $p$. (This also explains why such $\ell$ exist.)
\end{remark}
\begin{remark}
	By choosing $F$, we can show that level-raising primes $\ell$ exist which are also inert in $F$. Roughly speaking, this amounts to requiring $F$ to be disjoint from $\QQ(E[p])$ so that $\ell$ being inert in $F$ is disjoint from the level-raising condition.
\end{remark}
Now, let $B_\ell$ be the quaternion algebra ramified at $N^-\ell$, so there is a map
\[\op{Jac}X^{B_\ell}(N^+)\to A_g.\]
Now, for a squarefree product $n$ of (Kolyvagin) admissible primes for $E/F$, we define the class $c(n,\ell)$ as in the Kolyvagin Euler system, where the CM point $x_n\in X^{B_\ell}(N^+)$ come from abelian surfaces with complex multiplication, which then gives a point on the Jacobian as $T_{\ell'}x_n-(\ell'+1)x_n$ where $\ell'$ is a fixed auxiliary prime. (This point then is projected onto $A_g$; note that $\ov\rho_g=\ov\rho_f=V$!)

In the above construction, it is not so important that $\ell$ is prime. Indeed, one can replace $\ell$ with any squarefree product of an odd number of our inert level-raising primes.
\begin{notation}
	Fix everything as above. Using the discussion above, we define the classes $c(n,m)\in\mathrm H^1(F;V)$, where $n$ varies over squarefree products of Kolyvagin admissible primes, and $m$ is a squarefree product of an odd number of inert level-raising inert primes.
\end{notation}
Here is a one-sided Kolyvagin system property.
\begin{theorem}[Bertolini--Darmon] \label{thm:bd-kolyagin-left}
	Fix everything as above. For each fixed $m$, the classes $\{c(n,m)\}_n$ is a Kolyvagin system in $n$, in the following sense.
	\begin{listalph}
		\item One has
		\[\op{loc}_vc(n,m)\in\begin{cases}
			\mathrm H^1_{\mathrm{ur}} & \text{if }v\nmid nm, \\
			\mathrm H^1_{\mathrm{tr}} & \text{if }v\mid nm,
		\end{cases}\]
		where $\mathrm H^1_{\mathrm{tr}}\coloneqq\mathrm H^1(F_\ell;\FF_p(1))$, where this $\FF_p(1)$ arises from the decomposition $V=\FF_p\oplus\FF_p(1)$ over $F_\ell$.
		\item Furthermore, for any $\ell'$ not dividing $n$, one of
		\[\{\op{loc}_{\ell'}c(n\ell',m),\op{loc}_{\ell'}(n,m)\}\]
		vanishes if and only if the other one does.
	\end{listalph}
\end{theorem}
\begin{remark}
	It turns out that $\mathrm H^1_{\mathrm{tr}}$ is the image of the Kummer map $\delta\colon A_g(F)\to\mathrm H^1(F;\ov\rho_g)$, which can be checked using the ideas of \Cref{rem:level-raising-condition}.
\end{remark}
There is also a Kolyvagin property in the other coordinate.
\begin{theorem}[Bertolini--Darmon] \label{thm:bd-kolyvagin-right}
	Fix everything as above. For given $m$, $\ell$, and $\ell'$, where $\ell$ and $\ell'$ are distinct inert level-raising primes away from $m$, one element of
	$\{\op{loc}_\ell c(n,m),\op{loc}_{\ell'}(n,m\ell\ell')\}$
	vanishes if and only if the other one does.
\end{theorem}
\begin{remark}
	In fact, there are preferred isomorphisms which explain the simultaneous non-vanishing properties.
\end{remark}
Now, after using the linear algebra of Kolyvagin systems, which is similar to what was done before, it only remains to use the $L$-function to show that some of these classes are non-vanishing.
\begin{theorem}[Bertolini--Darmon] \label{thm:bd-l-func}
	Fix everything as above. For any inert level-raising prime $\ell$,
	\[\op{loc}_\ell c(1,\ell)\equiv\int_{[T]}\varphi\pmod p.\]
\end{theorem}
\begin{proof}[Idea]
	We will see nothing substantial, but we will say that the key input is that there is a specialization from the components on the Shimura curve $X^{B_\ell}(N^+)$ to the Shimura set $X^B(N^+)$.
\end{proof}
\begin{remark}
	By \Cref{thm:waldspurger}, $L(E/F,1)\ne0$ implies that $\op{loc}_\ell c(1,\ell)\not\equiv0\pmod p$ for $p$ large enough.
\end{remark}
Let's explain a sketch of \Cref{thm:bertolini-darmon}. By \Cref{thm:bd-l-func},it more or less amounts to show the following.
\begin{corollary}
	Fix everything as above. If $\op{loc}_\ell c(1,\ell)\ne0$, then $\op{Sel}_p(E/F)=0$.
\end{corollary}
\begin{proof}
	Suppose we have a class $c\in\op{Sel}_p(E/F)$, which we would like to show vanishes. By a version of the Chebotarev density theorem, it is enough to show that $\op{loc}_\ell c=0$ for all inert level-raising primes $\ell$. We now compute
	\[\langle c(1,\ell),c\rangle=0.\]
	Alternatively, we can sum over all places, and it turns out that the only nonzero contribution is
	\[\langle\op{loc}_\ell c(1,\ell),\op{loc}_\ell c\rangle_\ell=0.\]
	However, $\op{loc}_\ell c(1,\ell)$ is nonzero by hypothesis and lives in $\mathrm H^1_{\mathrm{tr}}$. On the other hand, $\op{loc}_\ell c$ lives in the orthogonal subspace $\mathrm H^1_{\mathrm{ur}}$, so for the above inner product to vanish, it is required that $\op{loc}_\ell c=0$.
\end{proof}
\begin{remark}
	Even though we only used the classes $c(1,\ell)$, there are some higher-powered congruences which use higher $c(n,m)$s.
\end{remark}

\end{document}
