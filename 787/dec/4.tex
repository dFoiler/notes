% !TEX root = ../notes.tex

\documentclass[../notes.tex]{subfiles}

\begin{document}

\section{December 4}
Today, we bound Selmer groups of motives.

\subsection{Unitary Shimura Varieties}
For today, $F_0$ is a totally real number field, and we fix two (modular) elliptic curves $E_1$ and $E_2$ over $F_0$. For example, this means that one has automorphic representations $\pi_1$ and $\pi_2$ such that the Hecke eigenvalues of $\pi_{iv}$ match the $\mathrm{Frob}_v$ eigenvalues on $T_\ell E_i$ for $i\in\{1,2\}$ and $v\nmid\ell$. This notably implies that $L(\pi_i,s)=L(E_i,s)$.
\begin{theorem}[Newton--Thorne]
	Fix an elliptic curve $E$ over a totally real field $F$. Then the motives $\op{Sym}^nE$ are modular.
\end{theorem}
\begin{remark}
	More precisely, if $\pi$ is the associated automorphic representation of $\op{GL}_2(\AA_{F_0})$, then Newton and Thorne constructed the symmetric power $\op{Sym}^n\pi$ of $\op{GL}_n(\AA_{F_0})$. It is not hard to describe what this representation should look like locally at unramified $v$: here, $\op{Sym}^n\pi_v$ is the induction of some character $\chi=(\chi_1,\ldots,\chi_n)$ on the diagonal torus, and it should be given by
	\[\chi_i(\varpi_v)=\alpha_v^i\beta_v^{n-1-i}\]
	where $\alpha_v$ and $\beta_v$ are the characters arising from $\pi$ of $E$. Another way to say this is
	\[L(\op{Sym}^n\pi,s)=L(\pi,{\op{Sym}^n},s).\]
	(We have not bothered to define right-hand side, but it comes from viewing $\op{Sym}^n$ as some representation of $\op{GL}_2$.)
\end{remark}
Thus, we can translate our bounds on Selmer groups to the automorphic setting. Quickly, we note that the self-duality on $E$ extends to a self-duality on $\op{Sym}^nE$, so the automorphic form $\op{Sym}^{n-1}\pi$ should actually descend from $\op{GL}_n$ to the group $\op{GSp}_{n/2}$, at least when $n$ is even. When we extend to a CM extension $F/F_0$, there is an extra complex conjugation action, so we pass to a unitary group.
\begin{definition}[unitary group]
	Fix a CM extension $F/F_0$ and a Hermitian space $V$ over $F$, namely equip\-ped with a sesquilinear perfect pairing $V\times V\to F$. Then we define the algebraic group
	\[\op U(V)\coloneqq\{g\in\op{GL}(V):\langle gv,gv\rangle=\langle v,w\rangle\text{ for all }v,w\in V\}.\]
	This is an algebraic group over $F_0$. We define $\op{GU}(V)$ to be the corresponding similitude group, which includes a center.
\end{definition}
\begin{remark}
	One can show that $\mathrm U(n){\ov{F_0}}\cong\op{GL}(n)_{F_0}$; it is an outer form.
\end{remark}
\begin{remark}
	Localizing at an infinite place $v\mid\infty$ allows us to put the pairing on $V$ into a standard form $\op{diag}(1_p,-1_q)$, so the group is classified by the signature. We may write $\op U(p,q)$ for the corresponding real group. It has a maximal compact subgroup $\op U(p)\times\op U(q)$, and one has a Hermitian symmetric domain
	\[\mathbb D_{p,q}\coloneqq\frac{\op U(p,q)}{\op U(p)\times\op U(q)}\]
	of dimension $pq$.
\end{remark}
We are going to get a Shimura variety from this construction.
\begin{definition} \label{def:unitary-moduli}
	We describe the Shimura variety given by $\op{GU}(V)$, where $V$ is a Hermitian space for $F/\QQ$ and $\dim V=g$, and suppose that the signature at the infinite place is $(p,q)$. For a test scheme $T$ over $\OO_F$, the $T$-points of our Shimura variety $\mc M_{p,q}$ parameterizes triples $(A,\iota,\lambda)$ for which
	\begin{itemize}
		\item $A$ is an abelian scheme of dimension $g$,
		\item $\iota\colon\OO_F\to\op{End}A$ is an action with $\iota(\ov z)=\iota(z)^\dagger$,
		\item $\lambda$ is a principal polarization (inducing the Rosati involution in the previous point), and
		\item $\op{Lie}A$ (as a vector bundle) splits into $\op{Lie}_+A\oplus\op{Lie}_-A$, where the subscript denotes the two possible $\OO_F$-actions via embedding into $\ov F$. (Here, $\OO_F$ acts on $\op{Lie}A$ via $\OO_F\subseteq\OO_S$.)
	\end{itemize}
	Concretely, if $\OO_F=\ZZ[\sqrt{-D}]$, then $\op{Lie}_\pm A$ corresponds to where $\sqrt{-D}$ would act by $+\sqrt{-D}$ or $-\sqrt{-D}$, respectively. If $n=2$, one can check that we recover Shimura curves.
\end{definition}
\begin{remark}
	To unwind the Rosati involution requirement, it means that the diagram
	% https://q.uiver.app/#q=WzAsNCxbMCwwLCJBIl0sWzEsMCwiQV5cXGxvciJdLFswLDEsIkEiXSxbMSwxLCJBXlxcbG9yIl0sWzIsMywiXFxsYW1iZGEiXSxbMCwxLCJcXGxhbWJkYSJdLFswLDIsIlxcaW90YSh6KSIsMl0sWzMsMSwiXFxpb3RhKHopXlxcbG9yIiwyXV0=&macro_url=https%3A%2F%2Fraw.githubusercontent.com%2FdFoiler%2Fnotes%2Fmaster%2Fnir.tex
	\[\begin{tikzcd}[cramped]
		A & {A^\lor} \\
		A & {A^\lor}
		\arrow["\lambda", from=1-1, to=1-2]
		\arrow["{\iota(\ov z)}"', from=1-1, to=2-1]
		\arrow["\lambda", from=2-1, to=2-2]
		\arrow["{\iota(z)^\lor}"', from=2-2, to=1-2]
	\end{tikzcd}\]
	commutes.
\end{remark}
\begin{remark}
	It turns out that
	\[\mc M_{p,q}(\CC)=\bigoplus_i\frac{\mathbb D_{p,q}}{\Gamma_i},\]
	where the $\Gamma_i\subseteq\op{GU}(V)$ is some arithmetic subgroup.
\end{remark}
\begin{theorem}
	Fix a quadratic imaginary field $F=\QQ(\sqrt{-D})$. The moduli functor of \Cref{def:unitary-moduli} is smooth over $\OO_F[1/D]$ of relative dimension $pq$.
\end{theorem}
Of course, Kolyvagin's ideas want to work at controlled places of bad reduction, so we introduce an auxiliary prime $\ell$.
\begin{remark}
	Suppose that $\ell$ is inert in $F=\QQ(\sqrt{-D})$. We can similarly define $\mc M_{p,q}^{(\ell)}$ as requiring that $\ker\lambda$ be nontrivial, $\ell$-power torsion, and as small as possible. Because $\ker\lambda$ must be $\OO_F$-stable, it follows that $\#\ker\lambda\ge\ell^2$ (note $\ell$ is inert), so our requirement is actually that $\ker\lambda$ has size $\ell^2$.
\end{remark}
\begin{theorem}[Rapopport--Zink, Pappas, Go\"etz]
	Fix a quadratic imaginary field $F=\QQ(\sqrt{-D})$ and a prime $\ell$ inert in $F$. The moduli functor $\mc M_{1,n-1}^{(\ell)}$ is smooth over $\OO_F[1/D\ell]$ which admits strictly semistable reduction over $\OO_F\otimes\ZZ_\ell$.
\end{theorem}
\begin{remark}
	Here is an application of the semistable reduction at $\ell$: the special fiber $\mc M^{(\ell)}_{1,n-1}$ is the union $\mc M^\circ\cup\mc M^\bullet$ of smooth schemes, where $\mc M^\circ$ is some ``ground'' curve, and $\mc M^\bullet$ consists of adjoining some ``balloons'' $\PP^{pq-1}$s. The intersection $\mc M^\circ\cap\mc M^\bullet$ is a disjoint union of Fermat hypersurfaces in $\PP^{n-1}$. In the right coordinates, these hypersurfaces are cut out by
	\[X_1X_1^\ell+X_2X_2^\ell+\cdots+X_nX_n^\ell=0.\]
	Here, because we are over $\FF_\ell$, we may view $X_i^\ell$ as a Frobenius action, so it is preserved by $\op{GU}$ as a group over $\FF_\ell$ (using the quadratic extension $\FF_{\ell^2}/\FF_\ell$).
\end{remark}
\begin{example}
	At $n=2$, one finds that $\mc M_{(1,1)}^{(\ell)}$ is a disjoint union of some Shimura curves corresponding to some quaternion algebras $B$ which are ramified at $\ell$ (as well as $\infty$ and some primes dividing $D$).
\end{example}
Now, to make Kolyvagin's ideas work, we need to have some cohomology classes of our Shimura varieties. It turns out that we need to control the cohomology groups
\[\mathrm H^1\left(F_\ell;\mathrm H^{2i-1}(\mc M_{\ov F_\ell};\FF_p)(i)\right)\]
because we want to produce some ramified classes which are almost in the Selmer groups $\mathrm H^1_f$.

To study this, we can start by trying to understand the Galois representation $\mathrm H^{2i-1}(\mc M_{\ov F_\ell};\QQ_p)$. Because $p\ne\ell$, the Galois representation is ramified but only tamely ramified, and in fact, there are some monodromy results which explain that the action of tame ramification can be given by a single operator $N$. This operator $N$ can be computed using the ``weight'' spectral sequence
\[E_1^{pq}\Rightarrow\mathrm H^*(\mc M_{\ov F_\ell};\Lambda),\]
where $E_1^{pq}$ is given by the cohomology of $\mc M^\circ$, $\mc M^\bullet$, and $\mc M^\circ\cap\mc M^\bullet$.

\end{document}