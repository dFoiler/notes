% !TEX root = ../notes.tex

\documentclass[../notes.tex]{subfiles}

\begin{document}

\section{December 2}
Let's get going.

\subsection{Conjectures of Tate}
We are interested in making progress on the Birch and Swinnerton-Dyer conjecture in higher dimensions. This both means that we would like to work with higher-dimensional varieties and higher-dimensional Galois representations. Stating the relevant conjectures requires us to do a little more algebraic geometry.
\begin{definition}[Chow group]
	Fix a smooth projective variety $X$ over a field $k$. Then the \textit{Chow group} $\op{CH}^i_\QQ(X)$ consists of $\QQ$-linear combinations of algebraic cycles (defined over $k$) of codimension $i$ modulo rational equivalence.
\end{definition}
\begin{example}
	Let $X$ be a curve. Then $\op{CH}^1_\QQ(X)$ admits a degree map $\deg\colon\op{CH}^1_\QQ(X)\to\QQ$, and we see that
	\[\ker\deg=\op{Jac}X(\QQ)_\QQ\]
	by definition of the Jacobian as the Picard group.
\end{example}
\begin{remark}
	In general, there is a cycle class map
	\[\op{CH}^i_\QQ(X)\to\mathrm H^{2i}_{\mathrm{\acute et}}(X_{\ov k};\QQ_p)(i).\]
	Such a thing is expected to exist for any reasonable cohomology theory. By functoriality, $\op{Gal}(\ov\QQ/\QQ)$ acts on both sides, and the cycle class map is a morphism of Galois modules.
\end{remark}
\begin{remark}
	The Galois representation $\mathrm H^{2i}(X_{\ov K};\QQ_p)$ is expected to be semisimple.
\end{remark}
Here is a first conjecture of Tate.
\begin{conj}[Tate]
	Fix a variety $X$ over a number field $K$. Then the cycle class map
	\[\op{CH}^i_{\QQ_p}(X_K)\to\mathrm H^{2i}_{\mathrm{\acute et}}(X_{\ov k};\QQ_p)(i)^{\op{Gal}(\ov K/K)}\]
	is surjective.
\end{conj}
With our geometry in hand, we should define some $L$-functions.
\begin{definition}
	Fix a smooth projective variety $X$ over a number field $K$. For each $i$, we define
	\[L(\mathrm H^i_{\mathrm{\acute et}}(X_{\ov\QQ}),s)=\prod_v\det\left(1-\mathrm{Frob}_vq_v^{-s};\mathrm H^i(X_{\ov\QQ})^{I_v}\right)^{-1}.\]
	(To define the product at a place $v$, we need to choose coefficients $\QQ_p$ with $v$ and $p$ coprime.) We then have
	\[\xi_X(s)=\prod_{i=0}^{2\dim X}L(\mathrm H^i_{\mathrm{\acute et}}(X_{\ov\QQ}),s)^{(-1)^{i}}.\]
\end{definition}
\begin{remark}
	This definition of $\xi_X$ agrees with its definition as a scheme over $\ZZ$.
\end{remark}
\begin{conj}[Tate]
	Fix a smooth projective variety $X$ over a number field $K$. Then
	\[\dim\im\left(\op{CH}^i_{\QQ_p}(X)\to\mathrm H^{2i}(X_{\ov K};\QQ_p)(i)\right)=-\ord_{s=i+1}L\left(\mathrm H^{2i}(X_{\ov K}),s\right).\]
\end{conj}
\begin{example}
	When $\dim X=1$, we see that $\op{CH}^1(X)$ should surject onto $\mathrm H^2(X_\QQ)(i)\cong\QQ_p$.
\end{example}
\begin{remark}
	The expected functional equation for $L\left(\mathrm H^{2i}(X_{\ov\QQ}),s\right)$ would have center $s=i+1/2$, so this conjecture only yields orders of vanishing at non-central points.
\end{remark}
\begin{remark}
	This second conjecture of Tate would follow from the first one if one knew that $\mathrm H^{2i}(X_{\ov K})$ was semisimple and that $L(\rho,s)$ (for a $p$-adic Galois representation $\rho$) has a pole at $s=1$ if and only if $\rho$ is trivial. For example, such things are approximately known if $X$ is a power of an elliptic curve, due to potential modularity and other things.
\end{remark}
\begin{example}
	If we want to understand $X$ of the form $X=E_1^{n_1}\times E_2^{n_2}$ for $n_1,n_2\ge1$, then it turns out that we may consider merely the middle cohomology, and the interesting part comes from
	\[\op{Sym}^{n_1}\mathrm H^1(E_1;\QQ_p)\otimes\op{Sym}^{n_2}\mathrm H^1(E_2;\QQ_p)\subseteq\mathrm H^{n_1+n_2}(X;\QQ_p).\]
	Again, the conjectures are again known thanks to potential modularity results.
\end{example}
The correct generalization of the Birch and Swinnerton-Dyer conjecture would predict order of vanishing at the central point $s=i+1/2$.
\begin{conj}[Beilinson--Bloch] \label{conj:beilinson-bloch}
	Fix a smooth projective variety $X$ over a number field $K$. Then the kernel $\op{CH}^i(X)_0$ of the degree map has finite rank which equals
	\[\ord_{s=i}L\left(\mathrm H^{2i-1}(X_{\ov K}),s\right).\]
\end{conj}
\begin{example}
	If $\dim X=1$, then
	\[\op{rank}\op{Jac}X(\QQ)=\op{rank}\op{CH}^1(X)_0,\]
	so this conjecture reduces to the Birch and Swinnerton-Dyer conjecture when $X$ is an elliptic curve. It also suggests the generalization to abelian varieties.
\end{example}

\subsection{The Bloch--Kato Conjecture}
Fix a smooth projective variety $X$ over a number field $K$. There is a map
\[\op{CH}^i(X)_0\to\mathrm H^1\left(K;\mathrm H^{2i-1}(X_{\ov K};\QQ_p)(i)\right)\]
arising from a spectral sequence. Note that $\mathrm H^{2i-1}(X_{\ov K};\QQ_p)(i)$ has weight $-1$. The right-hand side is much too large because it is a Galois cohomology group for a number field. To make it smaller, we introduce a Selmer group.
\begin{definition}[Bloch--Kato Selmer group]
	Fix a $p$-adic Galois representation $V$ over a number field $K$. Then we define $\mathrm H^1_f(K;V)$ to sit in the pullback square
	% https://q.uiver.app/#q=WzAsNCxbMCwwLCJcXG1hdGhybSBIXjFfZihLO1YpIl0sWzEsMCwiXFxtYXRocm0gSF4xKEs7VikiXSxbMSwxLCJcXGRpc3BsYXlzdHlsZVxccHJvZF92XFxtYXRocm0gSF4xKEtfdjtWKSJdLFswLDEsIlxcZGlzcGxheXN0eWxlXFxwcm9kX3ZcXG1hdGhybSBIXjFfZihLX3Y7VikiXSxbMCwxXSxbMSwyXSxbMCwzXSxbMywyXSxbMCwyLCIiLDEseyJzdHlsZSI6eyJuYW1lIjoiY29ybmVyIn19XV0=&macro_url=https%3A%2F%2Fraw.githubusercontent.com%2FdFoiler%2Fnotes%2Fmaster%2Fnir.tex
	\[\begin{tikzcd}[cramped]
		{\mathrm H^1_f(K;V)} & {\mathrm H^1(K;V)} \\
		{\displaystyle\prod_v\mathrm H^1_f(K_v;V)} & {\displaystyle\prod_v\mathrm H^1(K_v;V)}
		\arrow[from=1-1, to=1-2]
		\arrow[from=1-1, to=2-1]
		\arrow["\lrcorner"{anchor=center, pos=0.125}, draw=none, from=1-1, to=2-2]
		\arrow[from=1-2, to=2-2]
		\arrow[from=2-1, to=2-2]
	\end{tikzcd}\]
	where
	\[\mathrm H^1_f(K_v;V)\coloneqq\begin{cases}
		\ker\left(\mathrm H^1(K_v;V)\to\mathrm H^1(I_v;V\right) & \text{if }v\nmid p, \\
		\ker\left(\mathrm H^1(K_v;V)\to\mathrm H^1(K_v;V\otimes_{\QQ_p}B_{\mathrm{cris}})\right) & \text{if }v\mid p.
	\end{cases}\]
\end{definition}
\begin{remark}
	In other words, $\mathrm H_f^1(K;V)\subseteq\mathrm H^1(K;V)$ consists of ``everywhere unramified'' classes. For example, for $v\nmid p$, we have
	\[\mathrm H^1_f(K_v;V)=\mathrm H^1\left(K_v^{\mathrm{unr}}/K_v;V^{I_v}\right)\]
	under inflation.
\end{remark}
\begin{remark}
	One can show that the embedding
	\[\mathrm{CH}^i(X)_0\to\mathrm H^1\left(\QQ;\mathrm H^{2i-1}(X_{\ov K})(i)\right)\]
	factors through $\mathrm H^1_f$, and it is conjectured that it is surjective. This is a difficult conjecture!
\end{remark}
\begin{example}
	If $X$ is an elliptic curve over a number field $K$, then
	\[\mathrm H^1_f(K;T_pE\otimes_{\ZZ_p}\QQ_p)\cong\left(\lim\op{Sel}_{p^\bullet}(E/K)\right)\otimes_{\ZZ_p}\QQ_p.\]
	In particular,
	\[\op{rank}\mathrm H^1_f(K;T_pE\otimes_{\ZZ_p}\QQ_p)=\op{rank}E(K)+\op{corank}\Sha(E/K)[p^\infty].\]
	Thus, this conjecture is equivalent to finiteness of $\Sha$.
\end{example}
The previous example explains that proving finiteness of $\op{CH}^i(X)_0$ is going to be rather hopeless at the current stage, thereby making \Cref{conj:beilinson-bloch} hopeless. To circumvent these difficulties, we work with $\mathrm H^1_f$ directly.
\begin{conj}[Bloch--Kato]
	Fix a $p$-adic Galois representation $V$ over a number field $K$. Then
	\[\op{rank}\mathrm H^1_f(K;V)=\op{ord}_{s=i}L(V,s).\]
\end{conj}
Let's give some examples of what is known.
\begin{theorem}[Peng~2025]
	Fix a totally real number field $F_0$ (not equal to $\QQ$), and let $F/F_0$ be a quadratic extension. Let $E/F_0$ be a modular elliptic curve which does not have CM, and set $X\coloneqq E^n$. For $i\in\{1,2,\ldots,n\}$, if
	\[L\left(\mathrm H^{2i-1}(X)(i)|_F,0\right)\ne0,\]
	then
	\[\mathrm H^1_f\left(F;\mathrm H^{2i-1}(E^n;\QQ_p)(i)\right)=0\]
	for sufficiently large primes $p$.
\end{theorem}
\begin{theorem}[Liu--Tian--Xiao--Zhang--Zhu~2022]
	Fix a totally real number field $F_0$ (not equal to $\QQ$), and let $F/F_0$ be a quadratic extension. Further, fix non-isogenous modular elliptic curves $E_1$ and $E_2$. Let $V$ be the Galois representation $\op{Sym}^n\mathrm H^1(E_1)\otimes\op{Sym}^n\mathrm H^1(E_2)(n+1)$. If
	\[L\left(V|_F,0\right)\ne0,\]
	then
	\[\mathrm H^1_f\left(F;V\right)=0\]
	for sufficiently large primes $p$.
\end{theorem}
This is basically everything which is known in very large dimension.

\end{document}