% !TEX root = ../notes.tex

\documentclass[../notes.tex]{subfiles}

\begin{document}

\section{December 9}
Today is the last lecture.

\subsection{The Weight Spectral Sequence}
Kolyvagin has explained that we are interested in situations with bad reduction, so let's start by explaining a set-up with algebraic cycles on Shimura varieties with bad reduction. To this end, let $X$ be strictly semistable over $\OO_v$, where $\OO_v$ is the ring of integers in some $\ell$-adic local field $F_v$, and let $k_v$ be the corresponding residue field. To review, we are interested in producing ramified classes in
\[\mathrm H^1\left(F;\mathrm H^{2r-1}(X_{\ov\eta};\Lambda)(r)\right),\]
where $\eta$ is the generic point, and $\Lambda$ is some ring of coefficients; we will take $\Lambda\in\{\QQ_p,\ZZ_p,\ZZ/p^n\ZZ\}$ for $p\ne\ell$.

Let's quickly review how ramification works over local fields.
\begin{remark}
	Taking reduction produces an exact sequence
	\[0\to I_v\to\op{Gal}(\ov F_v/F_v)\to\op{Gal}(\ov k_v/k_v)\to1,\]
	where $I_v$ is the inertia subgroup, and there is also an exact sequence
	\[0\to I_v^{\mathrm{wild}}\to I_v\to\prod_{\ell'\ne\ell}\ZZ_{\ell'}(1)\to0.\]
	When $X$ is strictly semistable, the action of inertia on $V\coloneqq\mathrm H^*(X_{\ov F};\Lambda)$ factors through the tame inertia, and this action is uniquely determined by some monodromy operator $N\colon V(1)\to V$ which comes from $p$-adic Hodge theory.
\end{remark}
As such, we set $V\coloneqq\mathrm H^{2r-1}(X_{\ov F_v};\Lambda)(r)$, and we let $N\colon V(1)\to V$ be the corresponding monodromy operator. Then Inflation--restriction provides an exact sequence
\[0\to\mathrm H^1_{\mathrm{ur}}(F_v;V)\to\mathrm H^1(F_v;V)\to\mathrm H^1(I_v;V)^{\op{Gal}(\ov k_v/k_v)}\to0,\]
so we define $\mathrm H^1(k_v;V^{I_v})\coloneqq\mathrm H^1_{ur}(F_v;V)$ and $\mathrm H^1_{\mathrm{sing}}(I_v;V)\coloneqq\mathrm H^1(I_v;V)^{\op{Gal}(\ov k_v/k_v)}$. We will want to understand the action of Frobenius and $N$ on each of these subspaces, for which we use the weight spectral sequence.
\begin{theorem}[Weight spectral sequence]
	Fix a variety $X$ over $F_v$ with semistable reduction, and decompose the special fiber $X_{k_v}$ into irreducible components $\bigcup_{i=1}^mY_i$, which turn out to be smooth. Set $X^{(0)}\coloneqq\bigsqcup_{i=1}^mY_i$ and more generally
	\[X^{(j)}\coloneqq\bigsqcup_{\substack{I\subseteq\{1,\ldots,m\}\\\left|I\right|=j+1}}\Bigg(\bigcap_{i\in I}Y_i\Bigg).\]
	Because $X$ is semistable, $\dim X^{(j)}=\dim X-j$. Then there is a spectral sequence of Galois representations
	\[E_1^{pq}=\bigoplus_{i}\mathrm H^{q-2i}\left(X^{(p+2i)}_{\ov k};\Lambda(r-i)\right)\Rightarrow\mathrm H^*(X_{\ov F};\Lambda(r)).\]
\end{theorem}
\begin{remark}
	The differentials are given by correspondences between the $X^{(j)}$s with an appropriate sign.
\end{remark}
\begin{remark}
	When $\Lambda=\QQ_p$, each term in $E_1^{pq}$ has Frobenius weight $q-2r$. This explains why it is called the weight spectral sequence. Note that this implies that the differentials $d_r$ automatically vanish for $r>1$ for weight reasons; in other words, the spectral sequence degenerates on $E_2$. In our application, we will further suppose that $p$ is generic so that all the Frobenius eigenvalues are distinct, so the spectral sequence continues to degenerate at $E_2$.
\end{remark}
\begin{example}
	For simplicity, suppose that $X^{(2)}=\emp$ (so that $X^{(j)}=\emp$ for $j\ge2$). Then $E_1^{pq}$ is supported on $p\in\{-1,0,+1\}$. The top of the spectral sequence looks like the following.
	% https://q.uiver.app/#q=WzAsMTUsWzIsMiwiXFxtYXRocm0gSF57MnItMn1cXGxlZnQoWF57KDApfTtcXExhbWJkYShyKVxccmlnaHQpIl0sWzIsMywicD0wIl0sWzMsMywicD0xIl0sWzEsMywicD0tMSJdLFszLDIsIlxcbWF0aHJtIEheezJyLTJ9XFxsZWZ0KFheeygxKX07XFxMYW1iZGEocilcXHJpZ2h0KSJdLFswLDIsInE9MnItMiJdLFswLDEsInE9MnItMSJdLFswLDAsInE9MnIiXSxbMiwxLCJcXG1hdGhybSBIXnsyci0xfVxcbGVmdChYXnsoMCl9O1xcTGFtYmRhKHIpXFxyaWdodCkiXSxbMiwwLCJcXG1hdGhybSBIXnsycn1cXGxlZnQoWF57KDApfTtcXExhbWJkYShyKVxccmlnaHQpIl0sWzEsMCwiXFxtYXRocm0gSF57MnItMn1cXGxlZnQoWF57KDEpfTtcXExhbWJkYShyKVxccmlnaHQpIl0sWzEsMiwiKiJdLFsxLDEsIioiXSxbMywwLCIqIl0sWzMsMSwiKiJdLFsxMCw5XSxbMCw0XV0=&macro_url=https%3A%2F%2Fraw.githubusercontent.com%2FdFoiler%2Fnotes%2Fmaster%2Fnir.tex
	\[\begin{tikzcd}[cramped,sep=tiny]
		{q=2r} & {\mathrm H^{2r-2}\left(X^{(1)}_{\ov k};\Lambda(r)\right)} & {\mathrm H^{2r}\left(X^{(0)}_{\ov k};\Lambda(r)\right)} & {*} \\
		{q=2r-1} & {*} & {\mathrm H^{2r-1}\left(X^{(0)}_{\ov k};\Lambda(r)\right)} & {*} \\
		{q=2r-2} & {*} & {\mathrm H^{2r-2}\left(X^{(0)}_{\ov k};\Lambda(r)\right)} & {\mathrm H^{2r-2}\left(X^{(1)}_{\ov k};\Lambda(r)\right)} \\
		& {p=-1} & {p=0} & {p=1}
		\arrow[from=1-2, to=1-3]
		\arrow[from=3-3, to=3-4]
	\end{tikzcd}\]
\end{example}

\subsection{The Cohomology of Shimura Curves}
Continue with the previous example, but we now assume that $X$ is a (relative) curve so that $d=1$, and we take $r=1$ because it is the only interesting part; in this case, only the five labeled groups are the nonzero ones. As such, the monodromy operator is read off of the spectral sequence by identifying
\[\mathrm H^{2r-2}\left(X^{(1)};\Lambda(r)\right)\approx\mathrm H^{2r-2}\left(X^{(1)};\Lambda(r)\right).\]
If we assume that all components of $X_{\ov k}$ are $\PP^1$s, then $\mathrm H^{2r-1}\left(X^{(0)};\Lambda(r)\right)$ vanishes. We thus see that $X_{\ov k}$ is basically a graph with $\PP^1$s intersected transversely according to $X^{(1)}$. The point is that the horizontal arrows in our spectral sequence are some maps
\[\Lambda\left[X_{\ov k}^{(1)}\right]\to\Lambda\left[\pi_0X^{(0)}_{\ov k}\right]\]
and also a map in the opposite direction. This map is described basically by considering the ``dual graph'' of $X_{\ov\eta}$, in which case the left is the edges $E$ and the right is the vertices $V$; then one can just send an edge to the formal difference of its two vertices (with a chosen correct sign via a $2$-coloring of the graph). Composing the two maps induces a map
\[\Delta\colon\CC[V]_0\to\CC[V]^0,\]
where the domain is the functions of degree $0$, and the target is the quotient of $\Lambda[V]$ by the constant functions.
\begin{theorem}
	Fix $X$ as above: $X$ is a curve over $\OO_v$ with strictly semistable reduction, $X_{\ov k}$ is connected with components isomorphic to $\PP^1$, and $X^{(2)}=\emp$.
	\begin{listalph}
		\item We have $\mathrm H^1_{\mathrm{sing}}\left(F;\mathrm H^1(X_{\ov\eta};\Lambda)(1)\right)\cong\coker\Delta$.
		\item We have $\mathrm H^1_{\mathrm{ur}}\left(F;\mathrm H^1(X_{\ov\eta};\Lambda)(1)\right)\cong\ker\Delta$.
	\end{listalph}
\end{theorem}
\begin{remark}
	There is a similar statement for more general strictly semistable $X$.
\end{remark}
\begin{example}
	One can show that everything then vanishes when $\Lambda=\QQ_p$, but one might get interesting things over $\ZZ_p$.
\end{example}
\begin{example}
	For a divisor $D$ on $X$ (over $\OO_v$) which intersects transversely with the $\PP^1$s on the reduction, it turns out that the image in $\mathrm H^1_{\mathrm{sing}}\left(F;\mathrm H^1(X_{\ov k};\Lambda(1))\right)$ is simply the image of $D$ in the reduction $\Lambda[V]^0$.
\end{example}
We are thus motivated to calculate $\Delta$ on a Shimura curve $X=X^B$, where $S(B)$ is ramified at $\ell$ (and also some other primes). Then the dual graph $(V,E)$ does admit a canonical coloring. Indeed, define $B'$ to be the ``nearby'' quaternion algebra which is instead ramified at $\infty$ and unramified at $\ell$. Then the Shimura variety for $B'$ is a Shimura set $X'$, and it turns out that $V=X'\sqcup X'$ is the desired coloring, so $\Lambda[V]=\Lambda[X']\oplus\Lambda[X']$.
\begin{proposition}
	Fix everything as in the previous paragraph. Then the Laplacian
	\[\Lambda[X']\oplus\Lambda[X']\to\Lambda[X']\oplus\Lambda[X']\]
	is given by the Hecke operator $T_\ell$ at $\ell$ as $\begin{bsmallmatrix}
		\ell+1 & T_\ell \\ T_\ell & \ell+1
	\end{bsmallmatrix}$.
\end{proposition}
\begin{proof}[Idea]
	The point is that the Hecke operator acting on the Bruhat--Tits tree basically amounts to sending a given vertex to the (formal) average of its neighbors.
\end{proof}
\begin{example}
	Over $\QQ_p$, we can see that this map is always an isomorphism.
\end{example}
\begin{remark}
	One can use this to prove level-raising of modular forms, in the admissible situation that
	\[\det\begin{bmatrix}
		\ell+1 & T_\ell \\ T_\ell & \ell+1
	\end{bmatrix}\equiv0\pmod p.\]
\end{remark}

\subsection{The Construction of Classes}
Let's now continue with the construction of our classes from Heegner cycles. Fix an imaginary quadratic field $F$ over $\QQ$, and choose a Shimura curve $X$ corresponding to a quaternion algebra $B$ as before; assume that $F$ embeds into $B$. There are now two Heegner classes which we can construct.
\begin{itemize}
	\item A Heegner point $x_E$ can directly produce a class in $\mathrm H^1(F;V)$.
	\item Alternatively, a Heegner point produces a function on $\Lambda[X']$, which produces an element in $\coker\Delta$.
\end{itemize}
It turns out that these are the same class! Now, if $\varphi$ is a Hecke eigenform attached to an elliptic curve, then tracking through this story shows that $\op{loc}_\ell x_E\ne0$ if and only if
\[\int_{[T]}\varphi\ne0\]
in $\Lambda$. This is then related to the non-vanishing of the $L$-value by Waldspurger's formula.
\begin{remark}
	One can retell much of this story on just Shimura sets, where we still have some CM cycles in our Shimura set, which produce some classes $\mc D_mx_m\in\Lambda[X']$. Analogous to the theory of Heegner points, we see that we can hit this class with $\varphi$ to produce some numbers $c(n)$ in $\Lambda$. If one could show that one of these numbers $c(n)$ was nonzero, then one would have great applications (as above), such as finiteness of $\Sha$ and so on.
\end{remark}

\end{document}