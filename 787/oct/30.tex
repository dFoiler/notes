% !TEX root = ../notes.tex

\documentclass[../notes.tex]{subfiles}

\begin{document}

\section{October 30}
Today, we will give some addenda to Kolyvagin's work.

\subsection{Loose Ends}
Recall our point $x_n$ on $X_0(N)$ given by the natural isogeny
\[\frac{\CC}{\OO_{K,n}}\onto\frac{\CC}{(\mc N\cap\OO_{K,n})^{-1}}\]
defined over $K[n]$. Last class, we needed to understand how complex conjugation acts on $x_n$.
\begin{remark} \label{rem:cm-for-ecs}
	The theory of complex multiplication explains how the Galois group acts on these elliptic curves. In short, if $\mf a$ is an invertible ideal in an order $\OO\subseteq K$, and choose some $\sigma\in\op{Gal}(K[n]/K)$, which is isomorphic to the class group of $\op{Pic}\OO_{K,n}$. Thus, $\sigma$ goes to some $\mf a_{\sigma}$, and it turns out that
	\[\sigma\cdot\left(\CC/\mf a\right)=\CC/\mf a\mf a_\sigma.\]
\end{remark}
\begin{lemma} \label{lem:atkin-lehner-on-x0}
	Fix an embedding $\ov\QQ\into\CC$, and let $\tau$ denote complex conjugation. Then
	\[\tau(x_n)=\op{AL}(\sigma x_n)\]
	for some $\sigma\in\op{Gal}(K[n]/K)$, where $\op{AL}$ denotes the Atkin--Lehner involution.
\end{lemma}
\begin{proof}
	Set $\mc N_n\coloneqq\mc N\cap\OO_{K,n}$ for brevity. Applying $\tau$ to $X_0(N)$ looks like the isogney
	\[\frac{\CC}{\OO_{K,n}}\to\frac{\CC}{\overline{\mc N}_n^{-1}}.\]
	Applying the Atkin--Lehner involution (up to suitable isogenies) gives rise to
	\[\frac{\CC}{\overline{\mc N}_n^{-1}}\onto\frac{\CC}{N^{-1}\OO_{K,n}}\]
	We would like to show that this is in the Galois orbit of $x_n$: indeed, this holds by \Cref{rem:cm-for-ecs} because multiplying the denominators by $\overline{\mc N}$ corresponds to some Galois action.
\end{proof}
This completes the proof of \Cref{thm:kolyvagin-selmer}.

\subsection{Extensions of the Theorem}
Note that we have been interested in merely the case where $c(1)\ne0$, but there is something which can be said even when $c(1)=0$.
\begin{definition}
	Fix an elliptic curve $E$ as before, and set
	\[\mc K_p\coloneqq\{c(n):n\text{ is a squarefree product of admissible primes}\}.\]
	If $p$ is implied, we may abbreviate $\mc K_p$ to $\mc K$. Then we define the \textit{vanishing order} to be
	\[\ord\mc K\coloneqq\inf\{\omega(n):c(n)\ne0\}.\]
\end{definition}
\begin{example}
	One has that $\ord\mc K=0$ if and only if $c(1)\ne0$, and $\ord\mc K\le1$ if and only if $c(\ell)\ne0$ for some nonzero admissible prime $\ell$.
\end{example}
We do not know if $\ord\mc K$ is finite in general. This is the content of the following conjecture.
\begin{conj}[Kolyvagin]
	Fix an elliptic curve $E$ as before. For sufficiently large primes $p$, we have $\mc K_p\ne0$.
\end{conj}
% \begin{remark}
% 	 In other words, we do not know if there is always some $n$ for which $c(n)\ne0$.
% \end{remark}
\begin{remark}
	\Cref{thm:kolyvagin-selmer} can be restated as showing that $\ord\mc K=0$ implies that $\dim_{\FF_p}\op{Sel}_p(E/K)=1$.
\end{remark}
\begin{theorem}[Kolyvagin] \label{thm:kolyvagin-higher}
	Fix an elliptic curve $E$ as before. Suppose that $r\coloneqq\ord\mc K$ is finite, and suppose that $c(1)\in\mathrm H^1(K;E[p])^+$ for simplicity. Then
	\[\begin{cases}
		\dim_{\FF_p}\op{Sel}_p(E/K)^{(-1)^r}=r+1, \\
		\dim_{\FF_p}\op{Sel}_p(E/K)^{(-1)^{r+1}}\le r.
	\end{cases}\]
\end{theorem}
\begin{remark}
	In fact, $\dim_{\FF_p}\op{Sel}_p(E/K)^{(-1)^{r+1}}\equiv r\pmod2$.
\end{remark}
We will prove this next class.
\begin{remark}
	One expects the total rank to be odd in this case because the root number is $-1$ under the Heegner hypothesis (which is the parity conjecture).
\end{remark}
\begin{remark}
	Professor Zhang has proven $\ord\mc K<\infty$ in the ordinary case. Naomi Sweeting has some other results.
\end{remark}
\begin{corollary}
	Fix an elliptic curve $E$ as before. Assume $\mc K\ne0$. If $\op{rank}\op{Sel}_p(E/K)=1$, then $\ord\mc K=0$, and the Birch--Swinnerton-Dyer conjecture holds.
\end{corollary}
\begin{remark}
	This is typically referred to as a ``$p$-converse'' to Gross--Zagier--Kolyvagin. In particular, one can use this to show that some proportion of elliptic curves satisfy the Birch--Swinnerton-Dyer conjecture.
\end{remark}
\begin{example}
	If $E$ has rank $1$, then it follows that $\ord\mc K\in\{1,2\}$.
\end{example}
\begin{example}
	Suppose that we wanted to check $\ord\mc K_p=0$ for a given prime $p$. This amounts to checking that $y_1\not\equiv0\pmod{pE(K)}$, which is automatic for primes $p$ large enough.
\end{example}
\begin{example}
	Suppose that we wanted to check $\ord\mc K_p=2$ for a given prime $p$. Then we would have to find admissible primes $\ell_1$ and $\ell_2$ for $p$ (which notably requires $\ell_1,\ell_2\equiv-1\pmod p$) such that $c(\ell_1\ell_2)\ne0$, which means we have to check that
	\[\mc D_{\ell_1}\mc D_{\ell_2}y_{\ell_1\ell_2}\not\equiv0\pmod{pE(K[\ell_1\ell_2])}.\]
	This is hard because $\ell_1$ and $\ell_2$ are required to vary with $p$.
\end{example}
\begin{remark}
	For primes $p$ large enough, it is known that the points $y_{pN}$ are non-torsion in $E(K[p^N])$ for $N$ large enough. This is a conjecture of Mazur, proved by Vatsal--Cornut. This will be our next topic.
\end{remark}

\end{document}