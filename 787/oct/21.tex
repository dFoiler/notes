% !TEX root = ../notes.tex

\documentclass[../notes.tex]{subfiles}

\begin{document}

\section{October 21}
Here we go.

\subsection{The Euler System Property}
Last class, we advertized a relation between our points $y_n$. We are now ready to verify this, which we call the Euler system property.
\begin{lemma}
	Fix everything as in \Cref{not:kolyvagin-setting}, and let $\ell$ be an admissible prime not dividing $n$. Further, let $\lambda_n$ and $\lambda_{n\ell}$ be primes of $K[n]$ and $K[n\ell]$, respectively, so that $\lambda_{n\ell}$ lies over $\lambda_n$ and $\lambda_n$ lies over $\ell$.
	\begin{listalph}
		\item The prime $\ell$ of $K$ is totally split in $K[n]$.
		\item The prime $\lambda_{n\ell}$ is totally ramified over $\lambda_n$.
	\end{listalph}
\end{lemma}
\begin{proof}
	We show the claims separately.
	\begin{listalph}
		\item Note that the rational prime $\ell$ is inert in $K$ because it is admissible. To see that $\ell$ is totally split in $K[n]$, we recall that Artin reciprocity provides an isomorphism
		\[K^\times\backslash\AA_{K,f}^\times/\widehat\OO_{K,n}^\times\to\op{Gal}(K[n]/K).\]
		Because $\ell\nmid n$, we see that a uniformizer at $\ell$ vanishes in $K^\times\backslash\AA_K^\times/\widehat\OO_{K,n}^\times K_\infty^\times$ because the prime $(\ell)$ in $K$ is principal and generated by an element which is coprime to $n$. We conclude that $\mathrm{Frob}_\ell|_{K[n]}$ is the identity.

		\item By the Galois-theoretic calculations of \Cref{rem:easy-galois-k-n}, we see that the extensions $K[\ell_1]$ and $K[\ell_2]$ are linearly disjoint for distinct admissible primes $\ell_1$ and $\ell_2$. Thus, we may show that $\lambda_1$ of $K[1]$ (over $\ell$) is totally ramified in $K[\ell]$; extending this to $K[n]\subseteq K[n\ell]$ then follows because ramification index can only go up in such an extension, and the extensions have the same degree.

		We are now reduced to the case $n=1$. The claim follow if we separately show that there is a unique prime $\lambda_\ell$ above $\lambda_1$ and that the inertia degree equals $1$.
		\begin{itemize}
			\item We claim that the unique prime $\lambda_\ell$ has inertial degree $1$ over $\lambda_1$. In fact, it should have inertia degree $1$ over the prime $\ell$ of $K$, which holds because uniformizers vanish in the composite
			\[K_\ell^\times\to\AA_{K,f}^\times\to\op{Gal}(K[\ell]/K).\]
			Indeed, the kernel of the last projection is $K^\times\widehat{\OO}_{K,\ell}^\times$, and we can see that a uniformizer at $\ell$ in $\AA_K^\times$ can use $K^\times$ to kill the coordinate at $\ell$, and then the remaining coordinates remain unramified and therefore in $\widehat{\OO}_{K,\ell}^\times$.

			\item We claim that there is a unique prime $\lambda_\ell$ over $\lambda_1$. For this, we have to show that the decomposition group of $\lambda_\ell$ is the full Galois group $\op{Gal}(K[\ell]/K[1])$. To this end, we note that the diagram
			% https://q.uiver.app/#q=WzAsNixbMCwwLCJLWzFdX3tcXGxhbWJkYV8xfV5cXHRpbWVzIl0sWzAsMSwiS19cXGVsbF5cXHRpbWVzIl0sWzEsMCwiXFxBQV97S1sxXX1eXFx0aW1lcyJdLFsxLDEsIlxcQUFfS15cXHRpbWVzIl0sWzIsMCwiXFxvcHtHYWx9KEtbXFxlbGxdL0tbMV0pIl0sWzIsMSwiXFxvcHtHYWx9KEtbXFxlbGxdL0spIl0sWzIsNCwiIiwwLHsic3R5bGUiOnsiaGVhZCI6eyJuYW1lIjoiZXBpIn19fV0sWzMsNSwiIiwwLHsic3R5bGUiOnsiaGVhZCI6eyJuYW1lIjoiZXBpIn19fV0sWzAsMSwiIiwwLHsibGV2ZWwiOjIsInN0eWxlIjp7ImhlYWQiOnsibmFtZSI6Im5vbmUifX19XSxbMiwzLCJcXG9wIE4iXSxbNCw1LCIiLDEseyJzdHlsZSI6eyJ0YWlsIjp7Im5hbWUiOiJob29rIiwic2lkZSI6InRvcCJ9fX1dLFswLDJdLFsxLDNdXQ==&macro_url=https%3A%2F%2Fraw.githubusercontent.com%2FdFoiler%2Fnotes%2Fmaster%2Fnir.tex
			\[\begin{tikzcd}[cramped]
				{K[1]_{\lambda_1}^\times} & {\AA_{K[1]}^\times} & {\op{Gal}(K[\ell]/K[1])} \\
				{K_\ell^\times} & {\AA_K^\times} & {\op{Gal}(K[\ell]/K)}
				\arrow[from=1-1, to=1-2]
				\arrow[equals, from=1-1, to=2-1]
				\arrow[two heads, from=1-2, to=1-3]
				\arrow["{\op N}", from=1-2, to=2-2]
				\arrow[hook, from=1-3, to=2-3]
				\arrow[from=2-1, to=2-2]
				\arrow[two heads, from=2-2, to=2-3]
			\end{tikzcd}\]
			commutes, so it is enough to show that the image of $K_\ell^\times$ in $\op{Gal}(K[\ell]/K)$ agrees with the image of $\op{Gal}(K[\ell]/K[1])$. Well, this latter Galois group is the kernel of
			\[\frac{K^\times\backslash\AA_K^\times}{\widehat{\OO}_{K,\ell}^\times}\onto\frac{K^\times\backslash\AA_K^\times}{\widehat{\OO}_K^\times},\]
			and we can see that the kernel is surjected on by $K_\ell^\times$ by examining locally at $\ell$.
			\qedhere
		\end{itemize}
	\end{listalph}
\end{proof}
\begin{proposition} \label{prop:euler-system-prop}
	Fix everything as in \Cref{not:kolyvagin-setting}, and let $\ell$ be an admissible prime not dividing $n$.
	\begin{listalph}
		\item Norm property: $\tr_{K[n\ell]/K[\ell]} y_{n\ell}=a_\ell y_n$.
		\item Congruence property: choose places $\lambda_n$ and $\lambda_{n\ell}$ of $K[n]$ and $K[n\ell]$ such that $\lambda_{n\ell}$ lives over $\lambda_n$ which lies over $\ell$. Then we may identify $\mathrm{Frob}_\ell(y_n)\pmod{\lambda_n}$ and $y_{n\ell}\pmod{\lambda_{n\ell}}$ in $E(\FF_{\ell^2})$.
	\end{listalph}
\end{proposition}
\begin{proof}
	We show the two parts separately.
	\begin{listalph}
		\item We are actually going to show the equality
		\[\tr_{K[n\ell]/K[n]}x_{n\ell}\stackrel?=T_\ell x_n\]
		of divisors on $X_0(N)$. Here, $T_\ell$ is the Hecke operator on $X_0(N)$, which on divisors is given by
		\[T_\ell([E\onto E/C])\coloneqq\sum_{D\subseteq E[\ell]}\left[\frac ED\onto\frac{E}{C+D}\right],\]
		where the sum takes over cyclic subgroups $D\subseteq E[p]$ of order $p$. From the claimed equality, we see that (a) follows by applying the uniformization $\varphi_E\colon X_0(N)\to E$. In particular, by definition of the modular form $f$ corresponding to $E$, we know that $\varphi_E(T_\ell d)=a_\ell\varphi_E(d)$ for any divisor $d$ of $X_0(N)$; the point is that $E\subseteq\op{Jac}X_0(N)$ becomes a Hecke-isotypic where the Hecke operator $T_\ell$ acts by $a_\ell(f)=a_\ell(E)$.

		It remains to show the claimed equality, which follows from the theory of complex multiplication of elliptic curves. To start, 
		\[T_\ell x_n=\sum_{D\subseteq\frac1\ell\OO_{K,n}/\OO_{K,n}}\left[\frac{\CC}{\OO_{K,n}+D}\onto\frac{\CC}{\mc N^{-1}\OO_{K,n}+D}\right].\]
		Now, the right-hand sum is over some elliptic curves with complex multiplication by $\OO_{K,n\ell}$ with level $n$ structure. The theory of complex multiplication implies that $\op{Gal}(K[n\ell]/K[n])$ acts transitively on such things, so we are done upon noting that $x_{n\ell}$ is one such elliptic curve.

		\item As in (a), we will actually show that
		\[x_{n\ell}\stackrel?\equiv\mathrm{Frob}_\ell(x_n)\]
		as elements in $\FF_{\ell^2}$; note that the residue fields here are the same because $\lambda_n$ is totally split over $\ell$ (as an inert prime of $K$), and similarly $\lambda_{n\ell}$ is totally ramified over $\ell$. Then (b) will follow by applying the uniformization $X_0(N)\to E$.

		To show the claim, we use the Eichler--Shimura relation, which tells us that
		\[T_\ell\equiv\mathrm{Frob}_\ell+\mathrm{Frob}_\ell^\intercal\pmod\ell\]
		as operators on divisors in $\op{Div}X_0(N)$. In particular, by (a), we see
		\[\tr_{K[n\ell]/K[n]}x_{n\ell}\equiv\op{Frob}_\ell x_{n}+\op{Frob}_\ell^\intercal x_{n}\pmod{\lambda_n}.\]
		It follows that one of points in the divisor $\tr_{K[n\ell]/K[n]}x_{n\ell}$ is equivalent to $\op{Frob}_\ell x_n\pmod\ell$, but the Galois group of $K[n\ell]/K[n]$ then fixes this point$\pmod{\lambda_{n\ell}}$ because $\lambda_{n\ell}$ is totally ramified over $\lambda_n$,\footnote{Explicitly, note the Galois group of $K[n\ell]/K[n]$ surjects onto the Galois group of the residue field extension, which is the trivial extension $\FF_{\ell^2}/\FF_{\ell^2}$. Thus, $\op{Gal}(K[n\ell]/K[n])$ acts trivially on $E(\FF_{\ell^2})$.} so all points in the left-hand divisor must be equivalent to $\op{Frob}_\ell x_n$. The claim follows because the point $x_{n\ell}$ is a point in the left-hand divisor.
		\qedhere
	\end{listalph}
	% Let's start with the norm property. We will actually claim that we have an isomorphism
	% \[\tr x_{n\ell}=T_\ell x_n\]
	% of divisors on $X_0(N)$, from which the result follows after projecting. (Here, $T_\ell$ is the Hecke operator.) Here, $T_\ell$ can be thought of as a correspondence $X_0(N)\from T_\ell\to X_0(N)$ which classifies $\ell$-isogenies $(A_1,A_2)\to(A_1',A_2')$, where $(A_1,A_2)$ and $(A_1',A_2')$ are points in $X_0(N)$; in particular, the projections $T_\ell\to X_0(N)$ are both degree $\ell+1$ because choosing an $\ell$-isogeny amounts to choosing a line in the two-dimensional group of $\ell$-torsion. For example, when $N=n=1$, we see that $T_\ell x_1$ is simply a sum over all $(\ell+1)$ different $\ell$-isogenous elliptic curves, but with everything defined over $\QQ$, there simply are not so many of those, so the equality follows.
	% 
	% It remains to discuss the congruence property. The point is that the trace is the same $T_\ell\pmod\ell$ by the Eichler--Shimura relation, which is then the same as $\mathrm{Frob}_\ell+\mathrm{Frob}_\ell^\intercal$. We thus get
	% \[(\ell+1)x_{n\ell}=\op{Frob}x_n+\op{Frob}^\intercal x_n\pmod\ell.\]
	% The only way to get an isomorphism on the divisors turns out to be having one of the left-hand terms be $x_{n\ell}$ and the other to be $\ell x_{n\ell}$ for degree reasons. Namely, $\op{Frob}x_n$ is a pushforward divisor, so it has degree $1$; and $\op{Frob}^\intercal x_n$ is the pullback, so it has degree $\ell$.
\end{proof}

\end{document}