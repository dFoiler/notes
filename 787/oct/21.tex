% !TEX root = ../notes.tex

\documentclass[../notes.tex]{subfiles}

\begin{document}

\section{October 21}
Here we go.

\subsection{The Euler System Property}
These classes form what is known as an Euler system. The term ``Euler'' arises because these classes bare a relation resembling Euler factors. Let's explain this.
\begin{notation}[transverse subgroup]
	Here is another way to define the transverse subgroup. Not that the extension $K[\ell]$ from $K_\ell$ can be split into unramified part $K_\ell^{\mathrm{ur}}$ and totally ramified part $K_\ell^{\mathrm{ram}}$, so we can choose some totally ramified place $\lambda'$ so that we can split the cohomology
	\[\mathrm H^1(K_\ell;E[p])=\mathrm H^1(K_\ell^{\mathrm{un}}/K_\ell;E[p])\oplus\mathrm H^1(K[\ell]_{\lambda'};E[p]),\]
	and the latter group is $\mathrm H^1_{\mathrm{tr}}$ by definition.
\end{notation}
% \begin{remark}
% 	By choosing a generator $\sigma_\ell$ of $\op{Gal}(K[\ell]/K)$, one can identify $\mathrm H^1_{\mathrm{tr}}(K_\ell;E[p])$.
% \end{remark}
% Admissibility implies the Galois action of $\op{Gal}(\ov K_\ell/K_\ell)$ acts transitively on $E[p]$, so
% \[\mathrm H^1(K_\ell;E[p])=\op{Hom}_{\mathrm{cts}}(\op{Gal}(\ov K_\ell/K),E[p]),\]
% and the unramified cohomology corresponds to homomorphisms from $\op{Gal}(K_\ell^{\mathrm{ur}}/K_\ell)$. Accordingly, we define the ``transverse'' subgroup $\mathrm H^1_{\mathrm{tr}}$ to mean homomorphisms $\op{Gal}(K[\ell]_\ell/K_\ell)\to E[p]$. On the other hand, class field theory explains that such characters have the same data as a character $K_\ell^\times\to E[p]$, which can in turn by decomposed (non-canonically!) into a character of $\ell^\ZZ$ and a character of $\OO_{K_\ell}^\times$. Characters on $\ell^\ZZ$ are unramified while characters on $\OO_{K_\ell}^\times$ are unramified, and they correspond to characters on
% \[\op{Gal}(K[\ell]_\ell/K_\ell)\cong\FF_{\ell^2}^\times/\FF_\ell^\times,\]
% which explains where the ``transverse'' notion comes from.
\begin{theorem} \label{thm:kolyvagin-system-prop}
	Define the polynomial $H_\ell(T)\coloneqq T^2-a_\ell T+\ell$ to be the Euler factor at $\ell$. Then there is a $\varphi_\ell\coloneqq-\frac1p\cdot\mathrm{Frob}_\ell\cdot H_\ell(\mathrm{Frob}_\ell)$ which induces an isomorphism
	\[\varphi_\ell\colon\mathrm H^1_{\mathrm{ur}}(K_\ell;E[p])\to\mathrm H^1_{\mathrm{tr}}(K_\ell;E[p])\]
	satisfying the following conditions.
	\begin{listalph}
		\item We have $\op{loc}_vc(n)\in\mc L_{E,v}$ for all $v\nmid n$.
		\item For $\ell\mid n$, we have $\op{loc}_\ell c(n)=\varphi_\ell(\op{loc}_\ell c(n/\ell))$.
	\end{listalph}
\end{theorem}
% \begin{remark}
% 	In the last equality in (b), we see that $\varphi_\ell$ is sending $\mathrm H^1_{\mathrm{ur}}$ to $\mathrm H^1_{\mathrm{tr}}$.
% \end{remark}
% \begin{remark}
% 	Note that $c(n)$ depends on a ``choice'' of uniformizer $\ell$ because we used it in the definition of the derivative. Similarly, $\varphi_\ell$ depended on the choice of uniformizer.
% \end{remark}
We describe now describe an isomorphism $\mathrm H^1_{\mathrm{ur}}(K_\ell;E[p])\to\mathrm H^1_{\mathrm{tr}}(K_\ell;E[p])$. Well, note that we have a composite
\[E(\FF_{\ell^2})\onto\frac{E(\FF_{\ell^2})}{pE(\FF_{\ell^2})}=\frac{E(K_\ell)}{pE(K_\ell)}\cong\mathrm H^1_{\mathrm{ur}}(K_\ell;E[p]),\]
so it will be more or less enough to exhibit a map $E(\FF_{\ell^2})\to E(\FF_{\ell^2})$. For example, we could try the element $(\ell+1)-a_\ell\mathrm{Frob}_\ell$, but of course this vanishes by the Cayley--Hamilton theorem and the fact that $\mathrm{Frob}_\ell^2$ acts by the identity. So instead we choose the element
\[\varphi_\ell\coloneqq\mathrm{Frob}_\ell\cdot\frac{(\ell+1)-a_\ell\mathrm{Frob}_\ell}p,\]
which we now see will land $E(\FF_{\ell^2})[p]$. (The Frobenius here is some extra twist which will help us later.) This mapping is an isomorphism to $E(\FF_{\ell^2})[p]$ because the admissibility of $\ell$ implies that both $E(\FF_\ell)[p^\infty]$ and $E(\FF_{\ell^2})^-[p^\infty]$ are isomorphic to $\ZZ_p/(\ell+1-a_\ell)\ZZ_p$. This latter group is then identified with $\mathrm H^1_{\mathrm{tr}}(K_\ell;E[p])$ because the Galois group $\op{Gal}(K[\ell]_{\lambda'}/K_\ell)$ is cyclic, so we can use the cohomology of cyclic groups.

We are now ready to verify the Euler system property.
\begin{proposition} \label{prop:euler-system-prop}
	Fix everything as above, and let $n$ be a product of distinct admissible primes, and let $\ell$ be an admissible prime not dividing $n$.
	\begin{listalph}
		\item Norm property: $\tr y_{n\ell}=a_\ell y_n$.
		\item Congruence property: choose a chain of places $\lambda_n$ of $K[n]$ and $\lambda_{n\ell}$ over $\ell$. Then we may identify $\mathrm{Frob}_\ell(y_n)\pmod{\lambda_n}$ and $y_{n\ell}\pmod{\lambda_{n\ell}}$ in $E(\FF_{\ell^2})$.
	\end{listalph}
\end{proposition}
\begin{proof}
	Before doing anything, we note that $\lambda_n$ is totally split over $\ell$, and similarly $\lambda_{n\ell}$ is totally ramified over $\ell$. This can all be seen from the class field theory construction of these fields.

	Let's start with the norm property. We will actually claim that we have an isomorphism
	\[\tr x_{n\ell}=T_\ell x_n\]
	of divisors on $X_0(N)$, from which the result follows after projecting. (Here, $T_\ell$ is the Hecke operator.) Here, $T_\ell$ can be thought of as a correspondence $X_0(N)\from T_\ell\to X_0(N)$ which classifies $\ell$-isogenies $(A_1,A_2)\to(A_1',A_2')$, where $(A_1,A_2)$ and $(A_1',A_2')$ are points in $X_0(N)$; in particular, the projections $T_\ell\to X_0(N)$ are both degree $\ell+1$ because choosing an $\ell$-isogeny amounts to choosing a line in the two-dimensional group of $\ell$-torsion. For example, when $N=n=1$, we see that $T_\ell x_1$ is simply a sum over all $(\ell+1)$ different $\ell$-isogenous elliptic curves, but with everything defined over $\QQ$, there simply are not so many of those, so the equality follows.

	It remains to discuss the congruence property. The point is that the trace is the same $T_\ell\pmod\ell$ by the Eichler--Shimura relation, which is then the same as $\mathrm{Frob}_\ell+\mathrm{Frob}_\ell^\intercal$. We thus get
	\[(\ell+1)x_{n\ell}=\op{Frob}x_n+\op{Frob}^\intercal x_n\pmod\ell.\]
	The only way to get an isomorphism on the divisors turns out to be having one of the left-hand terms be $x_{n\ell}$ and the other to be $\ell x_{n\ell}$ for degree reasons. Namely, $\op{Frob}x_n$ is a pushforward divisor, so it has degree $1$; and $\op{Frob}^\intercal x_n$ is the pullback, so it has degree $\ell$.
\end{proof}

\end{document}