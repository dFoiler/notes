% !TEX root = ../notes.tex

\documentclass[../notes.tex]{subfiles}

\begin{document}

\section{October 9}
Today, we begin our discussion of Euler systems. We will mostly follow Gross's article \cite{gross-kolyvagin-system}.

\subsection{The Main Theorem}
We are interested in proving the following result.
\begin{definition}[analytic rank]
	Fix a modular elliptic curve $E$ over $\QQ$. Then the \textit{analytic rank} $r_{\mathrm{an}}(E/\QQ)$ equals the order of the vanishing of the $L$-function $L(E,s)$ at $s=1$.
\end{definition}
\begin{definition}[algebraic rank]
	Fix an elliptic curve $E$ over a number field $K$. Then the \textit{algebraic rank} $r_{\mathrm{alg}}(E/K)$ is $\op{rank}E(K)$.
\end{definition}
\begin{theorem}[Gross--Zagier, Kolyvagin] \label{thm:gzk-over-q}
	Fix an elliptic curve $E$ over $\QQ$ for which $r_{\mathrm{an}}(E/\QQ)\le1$. Then $r_{\mathrm{an}}(E/\QQ)=r_{\mathrm{alg}}(E/\QQ)$ and $\Sha(E/\QQ)$ is finite.
	% \begin{listalph}
	% 	\item We have $\op{rank}E(\QQ)=r_{\mathrm{an}}(E/\QQ)$.
	% 	\item We have $\#\Sha(E/\QQ)<\infty$.
	% \end{listalph}
\end{theorem}
\begin{remark}
	For technical reasons, we will only show that $\Sha(E/\QQ)[p^\infty]=0$ for sufficiently large primes $p$.
\end{remark}
% For technical reasons, we will only be able to show the following.
% \begin{theorem}[Gross--Zagier, Kolyvagin]
% 	Fix an elliptic curve $E$ over $\QQ$ for which $r_{\mathrm{an}}(E/\QQ)\le1$.
% 	\begin{listalph}
% 		\item We have $\op{rank}E(\QQ)=r_{\mathrm{an}}(E)$.
% 		\item We have $\#\Sha(E/\QQ)[p^\infty]=0$ for sufficiently large primes $p$.
% 	\end{listalph}
% \end{theorem}
It will be helpful to fix an auxiliary imaginary quadratic extension $K\coloneqq\QQ(\sqrt d)$, which we will choose more carefully at a later time. Then we see that $L(E_K,s)=L(E,s)L\left(E^d,s\right)$.
\begin{remark}
	On the automorphic side, suppose that $E$ admits a weight $2$ Hecke eigenform $f_E$ for which $L(E,s)=L(f_E,s)$ once appropriately normalized. Letting $\eta$ be the nontrivial quadratic character of $\op{Gal}(\ov\QQ/\QQ)$ with kernel $\op{Gal}(\ov\QQ/K)$, our identity reads
	\[L(E_K,s)=L(f_E,s)L(f_E\otimes\eta,s).\]
\end{remark}
After some analytic considerations, we are able to find some $K$ for which $\op{ord}_{s=1}L(E_K,s)=1$, so we reduce to the following.
\begin{theorem}[Gross--Zagier, Kolyvagin] \label{thm:gzk-over-k}
	Fix a modular elliptic curve $E$ over $\QQ$, and choose an imaginary quadratic field $K$. Suppose the ``Heegner hypothesis'' that if $E$ has bad reduction at places over a rational prime $p$, then $p$ is split in $K$; we also require $K\notin\{\QQ(i),\QQ(\zeta_3)\}$. If $r_{\mathrm{an}}(E/K)=1$, then $r_{\mathrm{alg}}(E/K)=1$ and $\Sha(E/K)$ is finite.
\end{theorem}
\begin{remark}
	One needs to do a little work to show that \Cref{thm:gzk-over-k} implies \Cref{thm:gzk-over-q} because, after all, it is quite possible for $r_{\mathrm{alg}}(E/K)=1$ while $r_{\mathrm{alg}}(E/\QQ)=1$. This is known due to many people; we mention Waldspurger, Bump--Friedberg--Hoffstein, and Murty--Murty.
\end{remark}
\begin{remark}
	\Cref{thm:gzk-over-k} is still known without the Heegner hypothesis.
\end{remark}

\subsection{Heegner Points}
The most striking aspect of \Cref{thm:gzk-over-k} is that it is has $r_{\mathrm{alg}}(E/K)=1$ as a conclusion, which means that we will need to find a non-torsion point on $E$ using the hypothesis $r_{\mathrm{an}}(E/K)=1$. This is done using the theory of Heegner points.
\begin{definition}[modular curve]
	Fix a positive integer $N$, and define the congruence subgroup $\Gamma_0(N)$ to be
	\[\Gamma_0(N)\coloneqq\left\{\begin{bmatrix}
		a & b \\ c & d
	\end{bmatrix}\in\op{SL}_2(\ZZ):c\equiv0\pmod N\right\}.\]
	Then we define the \textit{modular curve} $Y_0(N)(\CC)$ as the quotient of $\mc H\coloneqq\{\tau\in\CC:\Im\tau>0\}$ by the action of $\Gamma_0(N)$ by M\"obius transformations. We then define $X_0(N)(\CC)$ as the completion of $Y_0(N)$ as a Riemann surface; concretely, it is the quotient of $\mc H\cup\PP^1(\QQ)$ by the action of $\Gamma_0(N)$, and the extra points in $\PP^1(\QQ)/\Gamma_0(N)$ are referred to as cusps.
\end{definition}
\begin{remark}
	There is also a ``moduli'' description of $Y_0(N)$: it is the (coarse) moduli space of isogenies $E\to E'$ whose kernel (over $\ov\QQ$) is a cyclic group of degree $N$, considered up to suitable isomorphism.
\end{remark}
\begin{definition}[geometrically modular]
	Fix an elliptic curve $E$ over $\QQ$. Then $E$ is \textit{geometrically modular} if and only if there is a non-constant map $X_0(N)\to E$.
\end{definition}
\begin{remark}
	Equivalently, we are saying that $\op{Jac}X_0(N)$ has an isogeny factor isomorphic to $E$.
\end{remark}
\begin{definition}[Heegner point]
	Fix a positive integer $N$, and suppose that an imaginary quadratic field $K/\QQ$ satisfies the ``Heegner hypothesis'' that each prime $p\mid N$ splits in $K$. Then the principal ideal $(N)$ splits into a product $\mc N\overline{\mc N}=(N)$ where $\OO_K/\mc N\cong\ZZ/N\ZZ$. Then we define the \textit{Heegner point} $x_1\in Y_0(N)$ as the element of $Y_0(N)$ corresponding to the isogeny
	\[\CC/\OO_K\to\CC/\mc N^{-1}.\]
	Note that the kernel of this map is given by $\mc N^{-1}/\OO_K$, which is isomorphic to $\OO_K/\mc N\cong\ZZ/N\ZZ$.
\end{definition}
\begin{remark}
	Let's explain where the construction of $\mc N$ comes from. By the Chinese remainder theorem, it is enough to consider the case where $N=p^\nu$ for some prime $p$, and by the Heegner hypothesis, we see $(p)=\mf p\overline{\mf p}$ in $\OO_K$. Then $\left(p^\nu\right)=\mf p^\nu\overline{\mf p}^\nu$, so we can take $\mc N\coloneqq\mf p^\nu$.
\end{remark}
\begin{remark}
	The elliptic curve $\CC/\OO_K$ has a natural action by $\OO_K$, so its endomorphisms are given by $\OO_K$. The same holds for the isogenous elliptic curve $\CC/\mc N^{-1}$.
\end{remark}
\begin{remark}
	It follows from the theory of complex multiplication of elliptic curve that $\CC/\OO_K$ descends to an elliptic curve defined over the Hilbert class field $H_K$ of $K$.
\end{remark}
\begin{definition}[Heegner point]
	Fix a geometrically modular elliptic curve $E$ of conductor $N$ over an imaginary quadratic field $K$ satisfying the ``Heegner hypothesis.'' Letting $\varphi_E\colon X_0(N)\to E$ be the modularity map. Then we define the \textit{Heegner point} $y_K\in E(K)$ as
	\[y_K=\sum_{\sigma\in\op{Gal}(H/K)}\sigma(\varphi_E(x_1)).\]
\end{definition}
This construction of $y_K$ is known to have amazing properties.
\begin{theorem}[Gross--Zagier] \label{thm:gz}
	Fix a geometrically modular elliptic curve $E$ over $\QQ$, and choose an imaginary quadratic field $K$ satisfying the ``Heegner hypothesis.'' Then the sign of the functional equation for $L(E/K,s)$ is $-1$, and
	\[L'(E/K,1)=c_E\hat h(y_K),\]
	where $c_E$ is some explicit nonzero constant, and $\hat h$ is a non-torsion point.
\end{theorem}
\begin{corollary}
	Fix a geometrically modular elliptic curve $E$ over $\QQ$, and choose an imaginary quadratic field $K$ satisfying the ``Heegner hypothesis.'' If $r_{\mathrm{an}}(E/K)=1$, then $\op{rank}E(K)\ge1$.
\end{corollary}
\begin{proof}
	If $r_{\mathrm{an}}(E/K)=1$, then $L'(E/K)\ne0$, so it follows that $\hat h(y_K)\ne0$, so this point is non-torsion by properties of the N\'eron--Tate height.
\end{proof}
\begin{remark}
	It is possible but nontrivial to relax the Heegner hypothesis in this situation. Notably, one has to replace the modular curve $X_0(N)$ with a Shimura curve whose quaternion algebra is ramified at the bad primes $p$.
\end{remark}
\begin{remark}
	In fact, one can check that $y_K$ has the expected Galois action so that it contributes to the correct $E(\QQ)$ or $E^d(\QQ)$.
\end{remark}

\subsection{Kolyvagin's Result}
We are now ready to explain what Kolyvagin proved.
\begin{theorem}[Kolyvagin]
	Fix a geometrically modular elliptic curve $E$ over $\QQ$, and choose an imaginary quadratic field $K$ satisfying the ``Heegner hypothesis.'' If $y_K$ is not a torsion point, then $r_{\mathrm{alg}}(E/K)=1$, and $\Sha(E/K)$ is finite.
\end{theorem}
Let's explain how this might be done. Suppose $y_K$ is non-torsion. Then for all but finitely many primes $p$, we see $y_K$ produces a nontrivial class in
\[E(K)/pE(K)\into\op{Sel}_p(E/K).\]
Having a nonzero class like this is surprising!
\begin{theorem}[Kolyvagin] \label{thm:kolyvagin-selmer}
	Fix a geometrically modular elliptic curve $E$ over $\QQ$, and choose an imaginary quadratic field $K$ satisfying the ``Heegner hypothesis.'' Choose a prime $p$ such that the Galois representation $\ov\rho\colon\op{Gal}(\ov\QQ/\QQ)\to\op{GL}(E[p])$ is surjective. If $y_K$ represents a nonzero class $\delta_p(y_K)$ in $\mathrm H^1(K;E[p])$, then
	\[\op{Sel}_p(E/K)=\FF_p\cdot\delta_p(y_K).\]
\end{theorem}
\begin{remark}
	Comparing \Cref{thm:gz} with the Birch--Swinnerton-Dyer conjecture, one expects that $y_K$ is divisible by $p$ if and only if $p$ divides $\#\Sha(E/K)$ multiplied by some bad local Tamagawa numbers.
\end{remark}
In particular, it follows that $\Sha(E/K)[p]=0$.
\begin{remark}
	One can sharpen the argument of \Cref{thm:kolyvagin-selmer} to work with prime-powers of $p$ instead of just $p$. This allows him to show that $\Sha(E/K)$ is finite.
\end{remark}

\end{document}