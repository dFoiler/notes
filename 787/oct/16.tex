% !TEX root = ../notes.tex

\documentclass[../notes.tex]{subfiles}

\begin{document}

\section{October 16}
Today we actually construct the Euler system.

\subsection{Admissible Primes}
As usual, we take $E$ be an elliptic curve over $\QQ$, and we let $K$ be an imaginary quadratic extension of $\QQ$ satisfying our Heegner hypothesis. We also fix a large prime $p$ for which the representation
\[\ov\rho\colon\op{Gal}(\ov\QQ/\QQ)\to\op{GL}(E[p])\]
is surjective.\footnote{This holds for sufficiently large primes $p$ if $E$ has no complex multiplication by Serre's open image theorem.} Here is one application of the large image condition.
\begin{lemma} \label{lem:large-imgae-to-no-torsion}
	Fix everything as above. If $\ov\rho\colon\op{Gal}(\ov\QQ/\QQ)\to\op{GL}(E[p])$ is surjective, then $E[p](K[n])=0$.
\end{lemma}
\begin{proof}
	The point is that $\op{Gal}(K[n]/\QQ)$ is a dihedral group. On the other hand, let $L\coloneqq\QQ(E[p])$, so we know $\op{Gal}(L/\QQ)$ is isomorphic to $\op{GL}_2(\FF_p)$. There are two cases.
	\begin{itemize}
		\item If $E[p](K[n])$ is two-dimensional, then $L\subseteq K$, so there is a surjection $\op{Gal}(K[n]/K)\to\op{GL}_2(\FF_p)$, which is not possible for a dihedral group. Indeed, $\op{SL}_2(\FF_p)$ is simple (and not cyclic) for large $p$.
		\item If $E[p](K[n])$ is one-dimensional, then $\op{Gal}(K[n]/K)$ must surject onto a subgroup of $\op{GL}_2(\FF_p)$ preserving a line. In particular, it surjects onto a Borel subgroup, which one can still see is not possible.
		\qedhere
	\end{itemize}
\end{proof}
\begin{remark} \label{rem:easy-galois-k-n}
	Fix an odd prime $\ell$ unramified in $K$. By class field theory, we see
	\[\op{Gal}(K[\ell]/K[1])\cong\begin{cases}
		\FF^\times_{\ell^2}/\FF_\ell^\times & \text{if }\ell\text{ is inert}, \\
		(\FF_\ell^\times)^2/\FF_\ell^\times & \text{if }\ell\text{ is split}.
	\end{cases}\]
	Indeed, this amounts to a local calculation of
	\[\frac{(\OO_K\otimes\ZZ_\ell)^\times}{((\ZZ+\ell\OO_K)\otimes\ZZ_\ell)^\times}.\]
	The point is that the numerator is $\FF_{\ell^2}^\times$ when $\ell$ is inert and is $(\FF_\ell^\times)^2$ when $\ell$ is split. One can then extend this calculation multiplicatively to compute the Galois group of $K[n]/K$ when $n$ is squarefree with unramified prime factors.
\end{remark}
\begin{remark}
	In the following construction, it is safe to pretend that $K[1]$ is $K$ because we are not going to do anything interesting with the class group.
\end{remark}
Last class, we constructed a point $x_n\in X_0(N)(K[n])$, which we then mapped down to $E(K[n])$.
\begin{definition}[admissible]
	Fix a prime $\ell$ which is inert in $K$ and not dividing the conductor of $E$. Then we say that $\ell$ is \textit{admissible} if and only if the following equivalent conditions hold.
	\begin{listroman}
		\item We have $\ell+1\equiv a_\ell(E)\equiv0\pmod p$, where $a_\ell(E)=\#E(\FF_\ell)-(\ell+1)$.
		\item We have $\ov\rho(\mathrm{Frob}_\ell)$ is in the conjugacy class of complex conjugation $\op{diag}(1,-1)$ in $\op{GL}(E[p])$.
		\item The characteristic polynomial of $\ov\rho(\mathrm{Frob}_\ell)$ is $X^2-1$.
		\item We have $\dim_{\FF_p}\mathrm H^1_{\mathrm{ur}}(\QQ_\ell;E[p])=\dim_{\FF_p}\mathrm H^1_{\mathrm{ur}}(\QQ_\ell;E^K[p])=1$, where $E^K$ is the quadratic twist.
	\end{listroman}
\end{definition}
\begin{remark}
	Note that (ii) and (iii) are equivalent because conjugacy classes are determined by their characteristic polynomial. Then (iii) is equivalent to (i) because the characteristic polynomial is $T^2-a_\ell T+\ell$. Lastly, because $\ell$ is inert, we note that the size of $\mathrm H^1_{\mathrm{ur}}(\QQ_\ell;E[p])$ is the same as $E(\QQ_\ell)/pE(\QQ_\ell)$, which has the same size as $E(\FF_\ell)/pE(\FF_\ell)$, which has the same size as $E[p](\kappa_\ell)$. Thus, this space is nonzero exactly when $\mathrm{Frob}_\ell$ has a fixed vector on $E[p]$. Comparing this statement with its quadratic twist shows that (iv) is equivalent to (ii).
\end{remark}
\begin{remark}
	If $\ov\rho(\mathrm{Frob}_\ell)$ is the class of complex conjugation in $\op{Gal}(K(E[p])/\QQ)$, then it is automatically inert in $K$ by restricting complex conjugation down to $K$. Thus, we see that there is a positive density of admissible primes by the Chebotarev density theorem.
\end{remark}
% \begin{remark}
% 	One can use the Chebotarev density theorem in (ii) to compute a proportion of primes which are admissible.
% \end{remark}
\begin{remark}
	Professor Zhang's assumptions are immune to complaints. In particular, the definition makes sense even when $\ell=2$ (assuming $\ell=2$ satisfies the other conditions).
\end{remark}

\subsection{The Kolyvagin System}
Our goal is to extend the class $\delta(y_n)\in\mathrm H^1(K[n];E[p])$ down to $\mathrm H^1(K;E[p])$. Well, note that there is a commutative diagram
% https://q.uiver.app/#q=WzAsNixbMCwwLCIwIl0sWzEsMCwiXFxkaXNwbGF5c3R5bGVcXGZyYWN7RShLKX17cEUoSyl9Il0sWzEsMSwiXFxkaXNwbGF5c3R5bGVcXGxlZnQoXFxmcmFje0UoS1tuXSl9e3BFKEtbbl0pfVxccmlnaHQpXntcXG9we0dhbH0oS1tuXS9LKX0iXSxbMCwxLCIwIl0sWzIsMCwiXFxtYXRocm0gSF4xKEs7RVtwXSkiXSxbMiwxLCJcXG1hdGhybSBIXjEoS1tuXTtFW3BdKV57XFxvcHtHYWx9KEtbbl0vSyl9Il0sWzAsMV0sWzEsNF0sWzQsNV0sWzMsMl0sWzIsNV0sWzEsMl1d&macro_url=https%3A%2F%2Fraw.githubusercontent.com%2FdFoiler%2Fnotes%2Fmaster%2Fnir.tex
\[\begin{tikzcd}[cramped]
	0 & {\displaystyle\frac{E(K)}{pE(K)}} & {\mathrm H^1(K;E[p])} \\
	0 & {\displaystyle\left(\frac{E(K[n])}{pE(K[n])}\right)^{\op{Gal}(K[n]/K)}} & {\mathrm H^1(K[n];E[p])^{\op{Gal}(K[n]/K)}}
	\arrow[from=1-1, to=1-2]
	\arrow[from=1-2, to=1-3]
	\arrow[from=1-2, to=2-2]
	\arrow[from=1-3, to=2-3]
	\arrow[from=2-1, to=2-2]
	\arrow[from=2-2, to=2-3]
\end{tikzcd}\]
where the vertical arrows are given by restriction. Because $E[p](K[n])=0$ by \Cref{lem:large-imgae-to-no-torsion}, we conclude that the rightmost map must restriction must be an isomorphism by the Inflation--restriction exact sequence.

Thus, we see that we really want to construct a Galois-invariant class.
\begin{definition}[derived class]
	Fix an admissible prime $\ell$ so that $\op{Gal}(K[\ell]/K)$ is isomorphic to $\FF_{\ell^2}^\times/\FF_\ell^\times$ by \Cref{rem:easy-galois-k-n}. Let $\sigma_\ell$ be a generator, and we define the \textit{derivative operator} by
	\[D_\ell\coloneqq\sum_{i=1}^\ell i\sigma_\ell^i.\]
	If $n=\prod_{\ell\in S}\ell$ is a squarefree product of admissible primes, then we define $D_n\coloneqq\prod_{\ell\in S}D_\ell$.
\end{definition}
\begin{example} \label{ex:derivative-calculation}
	We calculate $(\sigma_\ell-1)D_\ell=\sum_{i=1}^\ell i\sigma_\ell^{i+1}-\sum_{i=1}^\ell i\sigma_\ell^i=(\ell+1)-\sum_{i=0}^\ell\sigma_\ell^i$ by telescoping.
\end{example}
We would like to show that $(\sigma_\ell-1)(D_\ell y_\ell)$ is in $pE(K[n])$ because this shows that it is fixed by $\langle\sigma_\ell\rangle$.
\begin{lemma}[Norm property]
	Fix everything as above. Then we claim that
	\[\tr_{K[n\ell]/K[n]}y_{n\ell}=a_\ell y_n.\]
\end{lemma}
\begin{proof}
	Explicit calculation on the Heegner points. We will show this shortly in \Cref{prop:euler-system-prop}.
\end{proof}
This will then imply that $(\sigma_\ell-1)(D_\ell y_\ell)$ equals $(\ell+1)y_\ell-a_\ell y_1$ by \Cref{ex:derivative-calculation}. Thus, we are allowed to define
\[\mc P_n\coloneqq\sum_{\sigma\in\op{Cl}K}\sigma(D_ny_n),\]
where $\sigma(D_ny_n)$ technically has lifted $\sigma$ to an element of $\op{Gal}(K[n]/\QQ)$, but this lifting does not matter because $D_ny_n$ already had good invariance properties. Applying $\delta$, we set $c(n)\coloneqq\delta(\mc P_n)$, which is a class in $\mathrm H^1(K[n];E[p])^{\op{Gal}(K[n]/K)}$ and therefore provides us with a class in $\mathrm H^1(K;E[p])$.
\begin{definition}[Kolyvagin system]
	The \textit{Kolyvagin system} is the collection of classes $\{c(n)\}$ as $n$ varies over squarefree products of admissible primes.
\end{definition}
\begin{remark}
	Kolyvagin has conjectured that $c(n)$ is nonzero for some $n$, no matter the elliptic curve $E$. In particular, we are not assuming anything about poles of $L$-functions.
\end{remark}

% This will then show that
% \[D_\ell y_\ell\in\left(\frac{E(K[\ell])}{pE(K[\ell])}\right)^{\op{Gal}()}\]

\end{document}