% !TEX root = ../notes.tex

\documentclass[../notes.tex]{subfiles}

\begin{document}

\section{October 16}
Today we actually construct the Euler system.

\subsection{Admissible Primes}
\Cref{lem:basic-linear-alg} more or less tells us that we are interested in building classes ramified at only certain primes. (Namely, \Cref{lem:basic-linear-alg} only cares about primes of ``type $1$''.) For our story, we will want a more restrictive class.
\begin{lemma} \label{lem:admissible-grab-bag}
	Fix an elliptic curve $E$ defined over $\QQ$, and let $K$ an imaginary quadratic field, and choose an odd prime $p$ of good reduction and unramified in $K$ such that $\ov\rho\colon\op{Gal}(\ov\QQ/\QQ)\to\op{GL}(E[p])$ is surjective. For an odd rational prime $\ell\ne p$ of good reduction for both $E$ and the quadratic twist $E^K$, the following are equivalent.
	\begin{listroman}
		\item We have $\ell+1\equiv a_\ell(E)\equiv0\pmod p$, where $a_\ell(E)=(\ell+1)-\#E(\FF_\ell)$.
		\item We have $\ov\rho(\mathrm{Frob}_\ell)$ is in the conjugacy class of $\op{diag}(1,-1)$ in $\op{GL}(E[p])$.
		\item The characteristic polynomial of $\ov\rho(\mathrm{Frob}_\ell)$ is $X^2-1$.
		\item The prime $\ell$ is inert in $K$, and $\dim_{\FF_p}\mathrm H^1_{\mathrm{ur}}(\QQ_\ell;E[p])=\dim_{\FF_p}\mathrm H^1_{\mathrm{ur}}(\QQ_\ell;E^K[p])=1$, where $E^K$ is the quad\-ratic twist.
	\end{listroman}
\end{lemma}
\begin{proof}
	We start by showing that (i), (ii), and (iii) are equivalent. For example, recall that the characteristic polynomial of $\ov\rho(\mathrm{Frob}_\ell)$ can be determined by its trace and determinant, which reveals that the characteristic polynomial is
	\[x^2-a_\ell(E)x+\ell,\]
	which explains the equivalence of (i) and (iii). Further, conjugacy in $\op{GL}_2(\FF_p)$ is uniquely determined by the characteristic polynomial, so the equivalence of (ii) and (iii) follows.
	
	We now show that (ii) and (iv) are equivalent. Quickly, we check that (ii) implies that $\ell$ is inert in $K$. We must show that $\mathrm{Frob}_\ell|_K$ is complex conjugation. Well, that complex conjugation acts on $\op{GL}(E[p])$ by $\op{diag}(1,-1)$ because complex conjugation has order $2$ and determinant $-1$ (by considering the Weil pairing). In other words, $\mathrm{Frob}_\ell$ and complex conjugation have the same conjugacy class in
	\[\op{Gal}(\QQ(E[p])/\QQ)\stackrel{\ov\rho}\cong\op{GL}(E[p]).\]
	Thus, $\mathrm{Frob}_\ell$ and $\tau$ will continue to be in the same conjugacy class when extended to $\op{Gal}(\ov\QQ/\QQ)$ and then restricted to $\op{Gal}(K/\QQ)$.
	
	We now proceed with the main part of the argument. Because $E$ has good reduction at $\ell$, we see that $\mathrm H^1_{\mathrm{ur}}(\QQ_\ell;E[p])$ has the same size as $E(\QQ_\ell)/pE(\QQ_\ell)$. By passing to finite fields (as in \Cref{lem:ur-of-elliptic-curve}) we can see that this quotient has the same size as $E(\FF_\ell)/pE(F_\ell)$, which has the same size as $E(\FF_\ell)[p]$. However, $E(\FF_\ell)[p]=E[p](\FF_\ell)$ is simply the points of $E[p]$ fixed by $\mathrm{Frob}_\ell$, so we have dimension $1$ if and only if $\ov\rho(\mathrm{Frob}_\ell)$ has $+1$ as an eigenvalue.

	A similar argument handles the twist $E^K$. By definition of $E^K$, we have that the Galois representation $\ov\rho^K$ is $\ov\rho\otimes\chi_K$, where $\chi_K$ is the quadratic field corresponding to $K$. Because $\ell$ is inert in $K$ (in either (ii) or (iv)), we know that $\chi_K(\mathrm{Frob}_\ell)=-1$, so $(\ov\rho\otimes\chi_K)(\mathrm{Frob}_\ell)$ will have $+1$ as an eigenvalue if and only if $\ov\rho(\mathrm{Frob}_\ell)$ has $-1$ as an eigenvalue. The equivalence of (ii) and (iv) follows.
\end{proof}
\begin{definition}
	Fix an elliptic curve $E$ defined over $\QQ$, and let $K$ an imaginary quadratic field, and choose an odd prime $p$ of good reduction and unramified in $K$ such that $\ov\rho\colon\op{Gal}(\ov\QQ/\QQ)\to\op{GL}(E[p])$ is surjective. Then a rational prime $\ell$ is \textit{admissible} if and only if it satisfies one of the equivalent conditions of \Cref{lem:admissible-grab-bag}.
\end{definition}
\begin{remark}
	The proof of \Cref{lem:admissible-grab-bag} shows that being admissible (namely, condition (ii)) is equivalent to $\mathrm{Frob}_\ell$ being in the same conjugacy class as complex conjugation in
	\[\op{Gal}(\QQ(E[p])/\QQ)\stackrel{\ov\rho}\cong\op{GL}(E[p]).\]
	It follows from the Chebotarev density theorem that a positive proportion of rational primes $\ell$ are admissible.
\end{remark}
\begin{remark}
	It is not actually fully necessary that $\ov\rho$ is surjective. For example, it seems that the same arguments all work if the image is $\op{SL}(E[p])$. The main issue is to show that $\mathrm{Frob}_\ell$ and complex conjugation having the same conjugacy class in $\op{GL}(E[p])$ implies that they are in the same conjugacy class in the (a priori smaller) group $\op{Gal}(\QQ(E[p])/\QQ)$.
\end{remark}
\begin{remark}
	We have assumed that $p$ and $\ell$ are odd for safety, but it is not strictly necessary for now.
\end{remark}

% \subsection{Admissible Primes}
% As usual, we take $E$ be an elliptic curve over $\QQ$, and we let $K$ be an imaginary quadratic extension of $\QQ$ satisfying our Heegner hypothesis. We also fix a large prime $p$ for which the representation
% \[\ov\rho\colon\op{Gal}(\ov\QQ/\QQ)\to\op{GL}(E[p])\]
% is surjective.\footnote{This holds for sufficiently large primes $p$ if $E$ has no complex multiplication by Serre's open image theorem.} Here is one application of the large image condition.
% \begin{remark} \label{rem:easy-galois-k-n}
% 	Fix an odd prime $\ell$ unramified in $K$. By class field theory, we see
% 	\[\op{Gal}(K[\ell]/K[1])\cong\begin{cases}
% 		\FF^\times_{\ell^2}/\FF_\ell^\times & \text{if }\ell\text{ is inert}, \\
% 		(\FF_\ell^\times)^2/\FF_\ell^\times & \text{if }\ell\text{ is split}.
% 	\end{cases}\]
% 	Indeed, this amounts to a local calculation of
% 	\[\frac{(\OO_K\otimes\ZZ_\ell)^\times}{((\ZZ+\ell\OO_K)\otimes\ZZ_\ell)^\times}.\]
% 	The point is that the numerator is $\FF_{\ell^2}^\times$ when $\ell$ is inert and is $(\FF_\ell^\times)^2$ when $\ell$ is split. One can then extend this calculation multiplicatively to compute the Galois group of $K[n]/K$ when $n$ is squarefree with unramified prime factors.
% \end{remark}
% \begin{remark}
% 	In the following construction, it is safe to pretend that $K[1]$ is $K$ because we are not going to do anything interesting with the class group.
% \end{remark}
% Last class, we constructed a point $x_n\in X_0(N)(K[n])$, which we then mapped down to $E(K[n])$.
% \begin{definition}[admissible]
% 	Fix a prime $\ell$ which is inert in $K$ and not dividing the conductor of $E$. Then we say that $\ell$ is \textit{admissible} if and only if the following equivalent conditions hold.
% 	\begin{listroman}
% 		\item We have $\ell+1\equiv a_\ell(E)\equiv0\pmod p$, where $a_\ell(E)=\#E(\FF_\ell)-(\ell+1)$.
% 		\item We have $\ov\rho(\mathrm{Frob}_\ell)$ is in the conjugacy class of complex conjugation $\op{diag}(1,-1)$ in $\op{GL}(E[p])$.
% 		\item The characteristic polynomial of $\ov\rho(\mathrm{Frob}_\ell)$ is $X^2-1$.
% 		\item We have $\dim_{\FF_p}\mathrm H^1_{\mathrm{ur}}(\QQ_\ell;E[p])=\dim_{\FF_p}\mathrm H^1_{\mathrm{ur}}(\QQ_\ell;E^K[p])=1$, where $E^K$ is the quadratic twist.
% 	\end{listroman}
% \end{definition}
% \begin{remark}
% 	Note that (ii) and (iii) are equivalent because these conjugacy classes are determined by their characteristic polynomial and complex conjugation must act by $\op{diag}(1,-1)$ because it acts by a diagonalizable operator with eigenvalues in $\{+1,-1\}$ and determinant $-1$. Then (iii) is equivalent to (i) because the characteristic polynomial is $T^2-a_\ell T+\ell$. Lastly, because $\ell$ is inert, we note that the size of $\mathrm H^1_{\mathrm{ur}}(\QQ_\ell;E[p])$ is the same as $E(\QQ_\ell)/pE(\QQ_\ell)$, which has the same size as $E(\FF_\ell)/pE(\FF_\ell)$, which has the same size as $E[p](\kappa_\ell)$. Thus, this space is nonzero exactly when $\mathrm{Frob}_\ell$ has a fixed vector on $E[p]$. Comparing this statement with its quadratic twist shows that (iv) is equivalent to (ii).
% \end{remark}
% \begin{remark}
% 	If $\ov\rho(\mathrm{Frob}_\ell)$ is the class of complex conjugation in $\op{Gal}(K(E[p])/\QQ)$, then it is automatically inert in $K$ by restricting complex conjugation down to $K$. Thus, we see that there is a positive density of admissible primes by the Chebotarev density theorem.
% \end{remark}
% % \begin{remark}
% % 	One can use the Chebotarev density theorem in (ii) to compute a proportion of primes which are admissible.
% % \end{remark}
% \begin{remark}
% 	Professor Zhang's assumptions are immune to complaints. In particular, the definition makes sense even when $\ell=2$ (assuming $\ell=2$ satisfies the other conditions).
% \end{remark}

\subsection{How to Construct Classes}
We are now ready to construct the desired ramified classes. From now on, we set the following conventions unless otherwise stated.
\begin{notation} \label{not:kolyvagin-setting}
	Throughout, $E$ is an elliptic curve defined over $\QQ$, and $K$ is an imaginary quadratic field satisfying the Heegner hypothesis. Further, we fix an odd prime $p\ge5$ unramified in $K$ for which the representation $\ov\rho\colon\op{Gal}(\ov\QQ/\QQ)\to\op{GL}(E[p])$ is surjective. Lastly, we fix some positive squarefree $n$ which is a product of admissible primes.
\end{notation}
Roughly speaking, this will be done by adding ``level structure'' to the construction of our point $y_1$.
\begin{definition}[Heegner point]
	Fix coprime positive integers $N$ and $n$, and let $K$ be an imaginary quad\-ratic field satisfying the Heegner hypothesis. Then we define the \textit{Heegner point} $x_n\in X_0(N)(K[N])$ to be given by the cyclic isogeny
	\[\CC/\OO_{K,n}\to\CC/\mc N^{-1}\OO_{K,n},\]
	where $\mc N\subseteq\OO_K$ is some chosen fractional ideal with $\mc N\overline{\mc N}=n\OO_K$.
\end{definition}
\begin{remark}
	It is not totally obvious that this cyclic isogeny descends to $K[n]$. In short, this follows from the theory of complex multiplication of elliptic curves (with level structure). Approximately speaking, one can show that the Galois action of $\op{Gal}(\ov K/K)$ on the elliptic curves with complex multiplication by the order $\OO_{K,n}$ factors through the ``ring'' class group $\op{Gal}(K[n]/K)=K^\times\backslash\AA_{K,f}^\times/\widehat{\OO}^\times_{K,n}$.
\end{remark}
\begin{notation}
	Fix a modular elliptic curve $E$ over $\QQ$ of conductor $N$, and let $K$ be an imaginary quadratic field satisfying the Heegner hypothesis. Letting $\varphi_E\colon X_0(N)\to E$ denote some uniformization map, we define the point
	\[y_n\coloneqq\varphi_E(x_n).\]
\end{notation}
Thus far, we have directly generalized the construction of $y_1$ to produce a point $y_n\in E(K[n])$. Of course, we are actually interested in producing a class in $\mathrm H^1(K;E[p])$, so we will have to do something to bring down the field of definition.
\begin{remark}
	The obvious thing to do here would be to imitate the construction of $y_K$ and consider the image of $\tr_{K[n]/K}y_n$ along $\delta\colon E(K)\to\mathrm H^1(K;E[p])$. However, this will not work: the image of $\delta$ is contained in $\op{Sel}_p(E/K)$, which our target theorem (\Cref{thm:kolyvagin-selmer}) tells us should be spanned by $\delta(y_K)$. Thus, these traces will tell us nothing new about $\mathrm H^1(K;E[p])$, and they cannot provide us with ramified classes.
\end{remark}
Thus, we will have to do something more complicated than taking a trace. We will do this by moving around the following diagram.
% https://q.uiver.app/#q=WzAsMTIsWzAsMiwiMCJdLFsxLDIsIlxcZGlzcGxheXN0eWxlXFxmcmFje0UoSyl9e3BFKEspfSJdLFsyLDIsIlxcbWF0aHJtIEheMShLO0VbcF0pIl0sWzMsMiwiXFxtYXRocm0gSF4xKEs7RSlbcF0iXSxbMCwzLCIwIl0sWzEsMywiXFxkaXNwbGF5c3R5bGVcXGxlZnQoXFxmcmFje0UoS1tuXSl9e3BFKEtbbl0pfVxccmlnaHQpXntcXG9we0dhbH0oS1tuXS9LKX0iXSxbMiwzLCJcXG1hdGhybSBIXjEoS1tuXTtFW3BdKV57XFxvcHtHYWx9KEtbbl0vSyl9Il0sWzMsMywiXFxtYXRocm0gSF4xKEtbbl07RVtwXSlbcF1ee1xcb3B7R2FsfShLW25dL0spfSJdLFszLDEsIlxcbWF0aHJtIEheMShLW25dL0s7RSlbcF0iXSxbMiwxLCJcXG1hdGhybSAgSF4xKEtbbl0vSztFKSJdLFszLDAsIjAiXSxbMiwwLCIwIl0sWzIsNiwiXFxvcHtSZXN9Il0sWzMsNywiXFxvcHtSZXN9Il0sWzgsMywiXFxvcHtJbmZ9Il0sWzksMiwiXFxvcHtJbmZ9Il0sWzYsN10sWzIsM10sWzksOF0sWzUsNiwiXFxkZWx0YSJdLFsxLDIsIlxcZGVsdGEiXSxbMSw1XSxbMCwxXSxbNCw1XSxbMTAsOF0sWzExLDldXQ==&macro_url=https%3A%2F%2Fraw.githubusercontent.com%2FdFoiler%2Fnotes%2Fmaster%2Fnir.tex
\begin{equation}
	\begin{tikzcd}[cramped]
		&& 0 & 0 \\
		&& {\mathrm  H^1(K[n]/K;E)} & {\mathrm H^1(K[n]/K;E)[p]} \\
		0 & {\displaystyle\frac{E(K)}{pE(K)}} & {\mathrm H^1(K;E[p])} & {\mathrm H^1(K;E)[p]} \\
		0 & {\displaystyle\left(\frac{E(K[n])}{pE(K[n])}\right)^{\op{Gal}(K[n]/K)}} & {\mathrm H^1(K[n];E[p])^{\op{Gal}(K[n]/K)}} & {\mathrm H^1(K[n];E)[p]^{\op{Gal}(K[n]/K)}}
		\arrow[from=1-3, to=2-3]
		\arrow[from=1-4, to=2-4]
		\arrow[from=2-3, to=2-4]
		\arrow["{\op{Inf}}", from=2-3, to=3-3]
		\arrow["{\op{Inf}}", from=2-4, to=3-4]
		\arrow[from=3-1, to=3-2]
		\arrow["\delta", from=3-2, to=3-3]
		\arrow[two heads, from=3-3, to=3-4]
		\arrow[from=3-3, to=3-4]
		\arrow["{\op{Res}}", from=3-3, to=4-3]
		\arrow["{\op{Res}}", from=3-4, to=4-4]
		\arrow[from=4-1, to=4-2]
		\arrow["\delta", from=4-2, to=4-3]
		\arrow[from=4-3, to=4-4]
	\end{tikzcd} \label{eq:define-kolyvagin-classes}
\end{equation}
We currently know how to construct classes in $E(K[n])$, from which $\delta$ produces a class in $\mathrm H^1(K[n];E[p])$. Quickly, we note that the left $\op{Res}$ map above is an isomorphism.
\begin{lemma} \label{lem:large-imgae-to-no-torsion}
	Fix everything as in \Cref{not:kolyvagin-setting}. Then $E[p](K[n])=0$.
\end{lemma}
\begin{proof}
	The point is that $\op{Gal}(K[n]/\QQ)$ is a dihedral group $\ZZ/2\ZZ\ltimes A_n$ where $A_n\coloneqq\op{Gal}(K[n]/K)$ is abelian (see \Cref{rem:kn-dihedral}), but the field $L\coloneqq\QQ(E[p])$ has $\op{Gal}(L/\QQ)\cong\op{GL}_2(\FF_p)$ (thanks to $\ov\rho$). The incompatibilities between these groups will prove the result.
	\begin{itemize}
		\item If $E[p](K[n])$ is two-dimensional, then $E[p]$ is defined over $K$, so $L\subseteq K$, so there is a surjection
		\[(\ZZ/2\ZZ\ltimes A_n)\onto\op{GL}_2(\FF_p).\]
		But this is not possible is not possible for odd $p$ for group-theoretic reasons. For example, because $(\ZZ/2\ZZ\ltimes A_n)$ admits an abelian subgroup of index $2$, this implies that $\op{GL}_2(\FF_p)$ also admits an abelian subgroup $H$ of index $2$. (Certainly $\op{GL}_2(\FF_p)$ is not abelian.) Considering the kernel of $\det\colon H\to\FF_q^\times$, we see that $H\cap\op{SL}_2(\FF_p)$ is an abelian subgroup of $\op{SL}_2(\FF_p)$ of index at most $2$. By projecting, we get the same result for $\op{PSL}_2(\FF_p)$, which is simple for $p\ge5$, so we now have a contradiction.

		\item If $E[p](K[n])$ is one-dimensional, then set $W\coloneqq E[p](K[n])$ so that $\QQ(W)\subseteq K[n]$, so there is a mapping
		\[(\ZZ/2\ZZ\ltimes A_n)\onto\op{GL}_1(\FF_p).\]
		The image is quite controlled because $(\ZZ/2\ZZ\ltimes A_n)^{\mathrm{ab}}$ is a $2$-group. Indeed, this abelianization is generated by the elements $\{(1,a):a\in A\}$, all of which have order $2$.

		It follows that $\QQ(W)$ is a quadratic extension of $\QQ$. On the other hand, the image of
		\[\ov\rho\colon\op{Gal}(\QQ(E[p])/\QQ(W))\to\op{GL}_2(\FF_p)\]
		needs to be an index-$2$ subgroup of $\op{GL}_2(\FF_p)$ for size reasons, so it is impossible for the image to fix $W$. Indeed, fixing $W$ (and choosing a basis) means that the image lands in the Borel subgroup $\left\{\begin{bsmallmatrix}
			a & b \\ 0 & d
		\end{bsmallmatrix}:ad\ne0\right\}$, which has index $\frac{\left(p^2-1\right)\left(p^2-p\right)}{p(p-1)^2}=p+1$, which is larger than $2$.
		\qedhere
	\end{itemize}
\end{proof}
\begin{lemma} \label{lem:res-iso-kolyvagin}
	Fix everything as in \Cref{not:kolyvagin-setting}. Then restriction defines an isomorphism
	\[\mathrm H^1(K;E[p])\to\mathrm H^1(K[n];E[p])^{\op{Gal}(K[n]/K)}.\]
	% Fix an elliptic curve $E$ defined over $\QQ$ and an imaginary quadratic field $K$. Choose a prime $p$ for which $\op{Gal}(\ov\QQ/\QQ)\to\op{GL}(E[p])$ is surjective and a positive integer $n$ which is sad.\todo{}
\end{lemma}
\begin{proof}
	This is immediate from \Cref{lem:large-imgae-to-no-torsion} combined with the five-term exact sequence
	\[\mathrm H^1\left(K[n]/K;E[p](K[n])\right)\to\mathrm H^1(K;E[p])\to\mathrm H^1(K[n];E[p])^{\op{Gal}(K[n]/K)}\to\mathrm H^2\left(K[n]/K;E[p](K[n])\right)\]
	from \Cref{rem:inflation-restriction-spectral} because the end terms both vanish.
\end{proof}
% Choose a point $x_n\in X_0(N)(K[n])$ given by the cyclic isogeny
% \[\CC/\OO_{K,n}\to\CC/\mc N^{-1}\OO_{K,n}.\]
% Then we can define $y_n\coloneqq\varphi_E(x_n)$. Under a congruence condition, we will be able to take $y_n$ and produce a class $c(n)\in\mathrm H^1(K;E[p])$.
% \begin{remark}
% 	We should not let $c(n)$ be the trace of $y_n$: indeed, this will produce an element of $E(K)$, which we expect to be a multiple of $y_K$ already.
% \end{remark}
% Roughly speaking, if $\op{Gal}(K[n]/H)$ were cyclic, then we take the ``derivative'' of $y_n$ as $\sum_ii\sigma^{i-1}(y_n)$. Slightly more precisely, we will attempt to produce a class in
% \[\mathrm H^1(K[n];E[p])^{\op{Gal}(K[n]/K)}\]
% by taking some derivative. Then because $E[p](K[n])$ vanishes, one can show that the map
% \[\mathrm H^1(K;E[p])\to\mathrm H^1(K[n];E[p])^{\op{Gal}(K[n]/K)}\]
% is an isomorphism by the (longer) Inflation--restriction exact sequence: the next term in the exact sequence is $\mathrm H^2(K[n]/K;E[p](K[n]))$.
%
% Our goal is to extend the class $\delta(y_n)\in\mathrm H^1(K[n];E[p])$ down to $\mathrm H^1(K;E[p])$. Well, note that there is a commutative diagram
% % https://q.uiver.app/#q=WzAsNixbMCwwLCIwIl0sWzEsMCwiXFxkaXNwbGF5c3R5bGVcXGZyYWN7RShLKX17cEUoSyl9Il0sWzEsMSwiXFxkaXNwbGF5c3R5bGVcXGxlZnQoXFxmcmFje0UoS1tuXSl9e3BFKEtbbl0pfVxccmlnaHQpXntcXG9we0dhbH0oS1tuXS9LKX0iXSxbMCwxLCIwIl0sWzIsMCwiXFxtYXRocm0gSF4xKEs7RVtwXSkiXSxbMiwxLCJcXG1hdGhybSBIXjEoS1tuXTtFW3BdKV57XFxvcHtHYWx9KEtbbl0vSyl9Il0sWzAsMV0sWzEsNF0sWzQsNV0sWzMsMl0sWzIsNV0sWzEsMl1d&macro_url=https%3A%2F%2Fraw.githubusercontent.com%2FdFoiler%2Fnotes%2Fmaster%2Fnir.tex
% \[\begin{tikzcd}[cramped]
% 	0 & {\displaystyle\frac{E(K)}{pE(K)}} & {\mathrm H^1(K;E[p])} \\
% 	0 & {\displaystyle\left(\frac{E(K[n])}{pE(K[n])}\right)^{\op{Gal}(K[n]/K)}} & {\mathrm H^1(K[n];E[p])^{\op{Gal}(K[n]/K)}}
% 	\arrow[from=1-1, to=1-2]
% 	\arrow[from=1-2, to=1-3]
% 	\arrow[from=1-2, to=2-2]
% 	\arrow[from=1-3, to=2-3]
% 	\arrow[from=2-1, to=2-2]
% 	\arrow[from=2-2, to=2-3]
% \end{tikzcd}\]
% where the vertical arrows are given by restriction. To continue, we pick up a lemma.
% Because $E[p](K[n])=0$ by \Cref{lem:large-imgae-to-no-torsion}, we conclude that the rightmost map must restriction must be an isomorphism by the Inflation--restriction exact sequence.
Thus, we see that we want to construct a Galois-invariant class in $\mathrm H^1(K[n];E[p])^{\op{Gal}(K[n]/K)}$ to produce a class in $\mathrm H^1(K;E[p])$. For this, we will take a derivative.
\begin{definition}[Kolyvagin derivative]
	Fix everything as in \Cref{not:kolyvagin-setting}. Choose an admissible prime $\ell$ so that $\op{Gal}(K[\ell]/K[1])$ is isomorphic to $\FF_{\ell^2}^\times/\FF_\ell^\times$ by \Cref{rem:easy-galois-k-n}. In particular, $\op{Gal}(K[\ell]/K[1])$ is cyclic of order $\ell+1$, so we may let $\sigma_\ell$ be a generator, and we define the \textit{Kolyvagin derivative} by
	\[D_\ell\coloneqq\sum_{i=1}^\ell i\sigma_\ell^i.\]
	If $n=\prod_{\ell\in S}\ell$ is a squarefree product of admissible primes, then we define $D_n\coloneqq\prod_{\ell\in S}D_\ell$.
\end{definition}
\begin{example} \label{ex:derivative-calculation}
	The key property of $D_\ell$ is that $(\sigma_\ell-1)D_\ell=(\ell+1)-\sum_{i=0}^{\ell}\sigma_\ell^i$. Indeed, by telescoping, we see that $(\sigma_\ell-1)D_\ell$ equals
	\[\sum_{i=2}^{\ell+1} (i-1)\sigma_\ell^{i}-\sum_{i=1}^\ell i\sigma_\ell^i=\ell\sigma_\ell^{\ell+1}+1-\sum_{i=0}^{\ell+1}\sigma_\ell^i.\]
\end{example}
\begin{lemma} \label{lem:derivative-invariant}
	Fix everything as in \Cref{not:kolyvagin-setting}. Then $D_ny_n$ is fixed by $\op{Gal}(K[n]/K[1])$.
\end{lemma}
\begin{proof}
	By \Cref{rem:easy-galois-k-n}, we see that $\op{Gal}(K[n]/K[1])=\prod_{\ell\mid n}\op{Gal}(K[\ell]/K[1])$, so it is enough to check that $(\sigma_\ell-1)D_ny_n=0$ for each $\ell\mid n$. Well, in \Cref{prop:euler-system-prop} below, we will show that
	\[\tr_{K[n]/K[n/\ell]}y_{n}=a_\ell y_{n/\ell}.\]
	Thus, we see that
	\[(\sigma_\ell-1)D_ny_n=D_{n/\ell}((\ell+1)y_n-a_\ell y_{n/\ell})\]
	by \Cref{ex:derivative-calculation}, and we see that this vanishes$\pmod p$ because $\ell$ is admissible.
\end{proof}
% We would like to show that $(\sigma_\ell-1)(D_n y_n)$ is in $pE(K[n])$ for each $\ell\mid n$ because this shows that it is fixed by $\langle\sigma_\ell\rangle$. Well, in \Cref{prop:euler-system-prop} below, we will show that
% \[\tr_{K[n\ell]/K[n]}y_{n\ell}=a_\ell y_n.\]
% Thus,
% \[(\sigma_\ell-1)D_ny_n=D_{n/\ell}((\ell+1)y_n-a_\ell y_{n/\ell})\]
% by \Cref{ex:derivative-calculation}, and we see that this vanishes$\pmod p$ because $\ell$ is admissible.
% This norm propperty will then imply that $(\sigma_\ell-1)(D_\ell y_\ell)$ equals $(\ell+1)y_\ell-a_\ell y_1$ by \Cref{ex:derivative-calculation}. Thus, we are allowed to define
% \[\mc P_n\coloneqq\sum_{\sigma\in\op{Cl}K}\sigma(D_ny_n),\]
% where $\sigma(D_ny_n)$ technically has lifted $\sigma$ to an element of $\op{Gal}(K[n]/\QQ)$, but this lifting does not matter because $D_ny_n$ already had good invariance properties. Applying $\delta$, we set $c(n)\coloneqq\delta(\mc P_n)$, which is a class in $\mathrm H^1(K[n];E[p])^{\op{Gal}(K[n]/K)}$ and therefore provides us with a class in $\mathrm H^1(K;E[p])$.
\begin{definition}[Kolyvagin system]
	Fix everything as in \Cref{not:kolyvagin-setting}. Set
	\[\mc P_n\coloneqq\sum_{\sigma\in\op{Gal}(K[1]/K)}\sigma(D_ny_n),\]
	where $\sigma(D_ny_n)$ refers to a choice of lift of $\sigma$ to $K[n]$. The \textit{Kolyvagin system} is the collection of classes
	\[c(n)\coloneqq\delta(\mc P_n),\]
	where $n$ varies over squarefree products of admissible primes. This class is $\op{Gal}(K[n]/K)$-invariant by construction, so it induces a class in $\mathrm H^1(K;E[p])$ by \Cref{lem:res-iso-kolyvagin}.
\end{definition}
\begin{remark}
	The choice of lift of $\sigma\in\op{Gal}(K[1]/K)$ to $K[n]$ does not matter because $D_ny_n$ is invariant under $\op{Gal}(K[n]/K[1])$ by \Cref{lem:derivative-invariant}.
\end{remark}
% \begin{remark}
% 	Kolyvagin has conjectured that $c(n)$ is nonzero for some $n$, no matter the elliptic curve $E$. In particular, we are not assuming anything about poles of $L$-functions.
% \end{remark}

% This will then show that
% \[D_\ell y_\ell\in\left(\frac{E(K[\ell])}{pE(K[\ell])}\right)^{\op{Gal}()}\]

\end{document}