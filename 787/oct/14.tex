% !TEX root = ../notes.tex

\documentclass[../notes.tex]{subfiles}

\begin{document}

\section{October 14}
Today we will begin the proof of Kolyvagin's theorem.

\subsection{Example of Linear Algebraic Input}
The proof of \Cref{thm:kolyvagin-selmer} will be done via the theory of Euler systems. This has a linear algebraic input, and then one has to actually go and construct the Euler system.

In an abstract situation, let's suppose that we have an elliptic curve $E$ over a number field $K$, and we would like to show that $\op{Sel}_p(E/K)$ vanishes. Recall that we may view $\op{Sel}_p(E/K)$ as the global classes in $\mathrm H^1(K;E[p])$ living in some controlled local condition $\mc L$. Thus, $\op{Sel}_p(E)$ intuitively consists of unramified global classes (meaning that the places of ramification are well-controlled).

Here is the main idea.
\begin{idea}
	A Selmer group is small if and only if there are many ramified classes.
\end{idea}
Let's explain each direction of this in the simplest possible case. Because it is easier, we will begin by assuming that $\op{Sel}_p(E/K)=0$.
\begin{lemma} \label{lem:basic-linear-algebra}
	Fix an elliptic curve $E$ over a number field $K$. Further, fix a prime $p$ and a place $\ell$ of $K$ such that $p\ne\ell$ and $E$ has good reduction at $\ell$. If $\op{Sel}_p(E/K)$ vanishes and $\mc L_\ell\ne\mathrm H^1(K_\ell;E[p])$, then there is a class $c(\ell)\in\mathrm H^1(K;E[p])$ such that $\op{loc}_vc(\ell)\in\mc L_v$ if and only if $v\ne\ell$.
\end{lemma}
\begin{proof}
	Recall from \Cref{rem:selmer-les} that we have a left exact sequence
	\[0\to\op{Sel}_{\mc L}(E[p])\to\op{Sel}_{\mc L^{\{\ell\}}}(E[p])\to\frac{\mathrm H^1(K_\ell;E[p])}{\mc L_\ell}.\]
	We are hoping to construct a nonzero class in $\op{Sel}_{\mc L^{\{\ell\}}}(E[p])/\op{Sel}_{\mc L}(E[p])$, which means that we need to show that the rightmost map is nontrivial.
	
	To do this, consider the ``dual'' sequence
	\[0\to\op{Sel}_{\mc L_{\{\ell\}}}(E[p])\to\op{Sel}_{\mc L}(E[p])\to\mathrm H^1_{\mathrm{ur}}(K_\ell;E[p]),\]
	for which \Cref{thm:global-tate-application} tells us that the last images are orthogonal complements. However, the rightmost map of this sequence vanishes because $\op{Sel}_p(E/K)=0$, so the image of $\op{Sel}_{\mc L^{\{\ell\}}}(E[p])$ in $\mathrm H^1(K_\ell;E[p])/\mc L_\ell$ must be nonzero.
\end{proof}
% In particular, for each place $\ell$ where $E[p]$ is unramified and $\ell\nmid p$, we produce some class $c(\ell)\in\mathrm H^1(K;E[p])$ for which $c(\ell)\notin\mc L_\ell$. For this, we 
It turns out that the presence of these classes can go backward to produce vanishing.
\begin{definition}[type]
	Fix an elliptic curve $E$ over a number field $K$. Further, fix a prime $p$ and a place $\ell$ of $K$ such that $p\ne\ell$ and $E$ has good reduction at $\ell$. Then we define the \textit{type} of $\ell$ as
	\[\op{type}_p(\ell)\coloneqq\dim_{\FF_p}\mathrm H^1_{\mathrm{ur}}(K_\ell;E[p]).\]
\end{definition}
\begin{remark}
	We know that $\op{type}_p(\ell)\in\{0,1,2\}$.
\end{remark}
\begin{lemma} \label{lem:basic-linear-alg}
	Fix an elliptic curve $E$ over a number field $K$. Further, fix a prime $p$ and a place $\ell$ of $K$ such that $p\ne\ell$ and $E$ has good reduction at $\ell$. Suppose that the map
	\[\mathrm H^1(K;E[p])\to\prod_{\op{type}_p(\ell)=1}\mathrm H^1(K_\ell;E[p])\]
	is injective. If we have classes $\{c(\ell):\op{type}_p(\ell)=1\}$ in $\mathrm H^1(K;E[p])$ such that $\op{loc}_vc(\ell)\in\mc L_v$ for $v\ne\ell$ and $\op{loc}_vcc(\ell)$, then $\op{Sel}_p(E/K)=0$.
\end{lemma}
\begin{proof}
	Suppose we have a nonzero class $c\in\op{Sel}_p(E/K)$, and we want to show that it vanishes. By the injectivity hypothesis, it is enough to show that $\op{loc}_\ell(c)=0$ for any place $\ell$ with $\op{type}_p(\ell)=1$.

	For this, we compute the global Tate pairing of $c$ and $c(\ell)$. Functoriality of the cup product shows that the diagram
	% https://q.uiver.app/#q=WzAsNixbMCwwLCJcXG1hdGhybSBIXjEoSztFW3BdKSJdLFsxLDAsIlxcbWF0aHJtIEheMShLO0VbcF0pIl0sWzIsMCwiXFxtYXRocm0gSF4yKEs7XFxtdV9wKSJdLFswLDEsIlxcbWF0aHJtIEheMShcXEFBX0s7RVtwXSkiXSxbMSwxLCJcXG1hdGhybSBIXjEoXFxBQV9LO0VbcF0pIl0sWzIsMSwiXFxtYXRocm0gSF4yKFxcQUFfSztcXG11X3ApIl0sWzAsMSwiXFx0aW1lcyIsMyx7InN0eWxlIjp7ImJvZHkiOnsibmFtZSI6Im5vbmUifSwiaGVhZCI6eyJuYW1lIjoibm9uZSJ9fX1dLFszLDQsIlxcdGltZXMiLDMseyJzdHlsZSI6eyJib2R5Ijp7Im5hbWUiOiJub25lIn0sImhlYWQiOnsibmFtZSI6Im5vbmUifX19XSxbMCwzXSxbMSw0XSxbMSwyLCJcXGN1cCJdLFs0LDUsIlxcY3VwIl0sWzIsNV1d&macro_url=https%3A%2F%2Fraw.githubusercontent.com%2FdFoiler%2Fnotes%2Fmaster%2Fnir.tex
	\[\begin{tikzcd}[cramped, column sep=small]
		{\mathrm H^1(K;E[p])} & {\mathrm H^1(K;E[p])} & {\mathrm H^2(K;\mu_p)} \\
		{\mathrm H^1(\AA_K;E[p])} & {\mathrm H^1(\AA_K;E[p])} & {\mathrm H^2(\AA_K;\mu_p)}
		\arrow["\times"{marking, allow upside down}, draw=none, from=1-1, to=1-2]
		\arrow[from=1-1, to=2-1]
		\arrow["\cup", from=1-2, to=1-3]
		\arrow[from=1-2, to=2-2]
		\arrow[from=1-3, to=2-3]
		\arrow["\times"{marking, allow upside down}, draw=none, from=2-1, to=2-2]
		\arrow["\cup", from=2-2, to=2-3]
	\end{tikzcd}\]
	commutes, but the composite
	\[\mathrm H^2(K;\mu_p)\to\mathrm H^2(\AA_K;\mu_p)\to\ZZ/p\ZZ\]
	vanishes by global class field theory, so we see that $\langle c,c(\ell)\rangle=0$ must vanish. On the other hand, this pairing splits up into a sum of local invariants
	\[\sum_v\langle\op{loc}_v c,\op{loc}_vc(\ell)\rangle_v=0.\]
	For $v\ne\ell$, we are computing the inner product of two terms in $\mc L_{v}$, which is Lagrangian, so the pairing vanishes. Thus, we conclude that
	\[\langle\op{loc}_\ell c,\op{loc}_\ell c(\ell)\rangle=0.\]
	Thus, if we did have $\op{loc}_\ell c\ne0$, then $\op{type}_p(\ell)=1$ implies that $\op{loc}_\ell c(\ell)\in\mc L_\ell^*$, so $\op{loc}_\ell c(\ell)\in\mc L_\ell$ because this subspace is Lagrangian, which is a contradiction!
\end{proof}
\Cref{lem:basic-linear-alg} morally explains to us that the presence of (many) ramified classes will be able to bound the size of Selmer groups. As such, we will spend some time trying to construct ramified classes in $\mathrm H^1(K;E[p])$.
% The moral is that the presence of ``many'' ramified classes implies that a Selmer group vanishes.

\subsection{Ring Class Fields}
In order to construct our classes, we do some preliminary work. Our classes will start their lives over certain abelian extensions of $K$ called ring class fields.
% \Cref{lem:basic-linear-alg} morally explains to us that the presence of (many) ramified classes will be able to bound the size of Selmer groups. As such, we will spend some time trying to construct ramified classes in $\mathrm H^1(K;E[p])$. As usual, fix a modular elliptic curve $E$ defined over a field $\QQ$, so we are given a morphism $\varphi_E\colon X_0(N)\to E$. Then fix an imaginary quadratic field $K$ over $\QQ$ satisfying the Heegner hypothesis.
\begin{definition}[ring class field]
	Fix an imaginary quadratic field $K$. For each positive integer $n$, we let $K[n]$ denote the \textit{ring class field} of $K$ of conductor $n$. Explicitly, let $\mathcal O_{K,n}\subseteq\mathcal O_K$ be the order $\ZZ+\ZZ n\alpha$, where $\mathcal O_K=\ZZ[\alpha]$. Then global class field theory provides a composite
	\[\op{Gal}(K^{\mathrm{ab}}/K)\from\widehat{K^\times\backslash\AA_K^\times}\onto K^\times\backslash\AA_K^\times/\widehat{\OO}_{K,n}^\times K_\infty^\times,\]
	where $K_\infty=K\otimes_{\mathbb Q}\mathbb C$ embedded into $\AA_K$. The fixed field of the kernel of this composite is defined to be $K[n]$.
\end{definition}
It may appear more natural to use the notation $K_n$, but we are reserving such subscripts for localizations. In particular, we will have occasion to consider the ring class fields $K[\ell]$ where $\ell$ is a rational inert prime, in which case we will want to be able to distinguish them from the localization $K_\ell$ at that prime.
\begin{example}
	With $n=1$, we find that $K[1]$ is the Hilbert class field. Namely, the quotient
	\[K^\times\backslash\AA_K^\times/\widehat{\mathcal O}_K^\times K_\infty^\times=K^\times\backslash\AA_{K,f}^\times/\widehat{\mathcal O}_K^\times\]
	is the class group. Explicitly, there is a map from $\AA_{K,f}^\times$ to the class group given by sending the (finite) idele $(a_{\mf p})$ to the fractional ideal $\prod_{\mf p}\mf p^{\nu_{\mf p}(a_{\mf p})}$. This mapping is surjective, and we can see that it has kernel exactly given by $K^\times\prod_{\mf p}(K_{\mf p}^\times,\mathcal O_{\mf p}^\times)$.
\end{example}
\begin{remark}
	By construction, we see that $K[n]\subseteq K[m]$ if and only if $n\mid m$.
\end{remark}
\begin{remark} \label{rem:easy-galois-k-n}
	Let's compute $\op{Gal}(K[n]/K[1])$ for squarefree $n$. This is the kernel of the quotient map
	\[\underbrace{K^\times\backslash\AA_{K,f}^\times/\widehat{\OO}_{K,n}^\times}_{\op{Gal}(K[n]/K)}\onto\underbrace{K^\times\backslash\AA_{K,f}^\times/\widehat{\OO}_{K}^\times}_{\op{Gal}(K[1]/K)},\]
	which is just $\widehat{\OO}_{K}^\times/\widehat{\OO}_{K,n}^\times$. By the Chinese remainder theorem, this quotient is
	\[\frac{(\OO_K/n\OO_K)^\times}{(\ZZ/n\ZZ)^\times}\cong\prod_{\text{prime }\ell\mid n}\frac{(\OO_{K}\otimes\ZZ_\ell)^\times}{((\ZZ+n\OO_{K})\otimes\ZZ_\ell)^\times}.\]
	For example, if $\ell$ is split, then this quotient is $\left(\ZZ_\ell^\times\right)^2/(1+\ell\ZZ_\ell^\times)$, where $(1+\ell\ZZ_\ell^\times)$ is embedded diagonally, so the quotient is $\left(\FF_\ell^{\times}\right)^2/\FF_\ell^\times$. Similarly, if $\ell$ is inert, then this quotient is $\ZZ_{\ell^2}^\times/(1+\ell\ZZ_{\ell^2})^\times$, which is $\FF_{\ell^2}^\times/\FF_\ell^\times$.
\end{remark}
\begin{remark}
	Quickly, we check that $K[n]$ is Galois over $\QQ$. Indeed, upon embedding $\ov\QQ\into\CC$, it is enough to show that the $K[n]=\overline{K[n]}$. For this, we use the Chebotarev density theorem: observe that a prime $\mf p$ of $K$ (lying over a rational prime $p$) splits completely in $K[n]$ if and only if a uniformizer at $\mf p$ in $\AA_{K,f}^\times$ lies in $K^\times\widehat{\OO}_{K,n}^\times$. This last condition is Galois-invariant, so we see that $\mf p$ splits in $K[n]$ if and only if $\overline{\mf p}$ splits in $K[n]$, which is equivalent to $\mf p$ splitting in $\overline{K[n]}$. We conclude that $K[n]=\overline{K[n]}$ because Galois extensions of $K$ can be recognized by the totally split primes.
\end{remark}
\begin{remark} \label{rem:kn-dihedral}
	The previous remark allows us to write down a short exact sequence
	\[1\to\op{Gal}(K[n]/K)\to\op{Gal}(K[n]/\QQ)\to\op{Gal}(K/\QQ)\to1.\]
	In fact, upon choosing an embedding $\ov\QQ\into\CC$, we get a choice of complex conjugation $\tau$ on $K[n]$, which induces a splitting of this short exact sequence. Now, for each $\sigma\in\op{Gal}(K[n]/K)$, we claim that $\tau\sigma\tau=\sigma^{-1}$. It is enough to check this on Frobenius elements by the Chebotarev density theorem. Thus, we have to check that $\mathrm{Frob}_{\overline{\mf p}}\cdot\mathrm{Frob}_{\mf p}=1$. By class field theory, it is enough to note that the norm of a uniformizer at $\mf p$ in $\AA_{K,f}^\times$ lives in $\widehat{\OO}^\times_{K,n}$ for each $\mf p\nmid n$.
\end{remark}
% \begin{remark}
% 	Note that
% 	\[\bigcup_{n\ge1}K[n]\subsetneq K^{\mathrm{ab}}.\]
% 	Indeed, using class field theory, the left-hand field has Galois group given by the quotient
% 	\[\lim_nK^\times\backslash\AA_K^\times/\widehat{\OO}_{K,n}^\times K_\infty^\times\]
% 	of $\widehat{K^\times\backslash\AA_K^\times}$.
% 	% Thus, for example, we note that the intersection $\bigcap_{n\ge1}\widehat{\mathcal O}_{K,n}^\times$ is nontrivial: over a rational prime $p$, the completion looks like $(\ZZ_p+p^\bullet\mathcal O_{K_p})^\times$, which is not good enough
% \end{remark}
% Now, for each positive integer $n$, let $K[n]$ denote the ring class field of $K$ of conductor $n$, meaning that $K[n]/K$ is the abelian extension corresponding to the quotient
% \[\op{Gal}(K^{\mathrm{ab}}/K)\from\widehat{K^\times\backslash\AA_K^\times}\onto K^\times\backslash\AA_K^\times/\widehat{\OO}_{K,n}^\times K_\infty^\times,\]
% where $\widehat{\OO}_{K,n}$ denotes the profinite completion of the order $\OO_{K,n}=\ZZ+n\OO_K$. For example, $K[1]$ is the usual Hilbert class field.
% \begin{remark}
% 	In general, one expects that $K[n]/K$ to be smaller than the ray class field. Indeed, in general, ray class fields correspond to arbitrary open subgroups of $\AA_K^\times$, but ring class fields require us to have local factors which look like $\ZZ_p+p^\bullet\OO_{K_p}$ for (rational!) primes $p$. Nonetheless, we still know that $K[n]/K$
% \end{remark}

\end{document}