% !TEX root = ../notes.tex

\documentclass[../notes.tex]{subfiles}

\begin{document}

\section{October 14}
Today we will begin the proof of Kolyvagin's theorem.

\subsection{Example of Linear Algebraic Input}
The proof of \Cref{thm:kolyvagin-selmer} will be done via the theory of Euler systems. This has a linear algebraic input, and then one has to actually go and construct the Euler system.

In an abstract situation, let's suppose that we have an elliptic curve $E$ over a number field $K$, and we would like to show that $\op{Sel}_p(E/K)$ vanishes. Recall that we may view $\op{Sel}_p(E/K)$ as the global classes in $\mathrm H^1(K;E[p])$ living in some controlled local condition $\mc L$. Thus, $\op{Sel}_p(E)$ intuitively consists of unramified global classes (meaning that the places of ramification are well-controlled).

Here is the main idea.
\begin{idea}
	A Selmer group is small if and only if there are many ramified classes.
\end{idea}
Let's explain each direction of this in the simplest possible case. Because it is easier, we will begin by assuming that $\op{Sel}_p(E/K)=0$.
\begin{lemma}
	Fix an elliptic curve $E$ over a number field $K$. Further, fix a prime $p$ and a place $\ell$ of $K$ such that $p\ne\ell$ and $E$ has good reduction at $\ell$. If $\op{Sel}_p(E/K)$ vanishes and $\mc L_\ell\ne\mathrm H^1(K_\ell;E[p])$, then there is a class $c(\ell)\in\mathrm H^1(K;E[p])$ such that $\op{loc}_vc(\ell)\in\mc L_v$ if and only if $v\ne\ell$.
\end{lemma}
\begin{proof}
	Recall from \Cref{rem:selmer-les} that we have a left exact sequence
	\[0\to\op{Sel}_{\mc L}(E[p])\to\op{Sel}_{\mc L^{\{\ell\}}}(E[p])\to\frac{\mathrm H^1(K_\ell;E[p])}{\mc L_\ell}.\]
	We are hoping to construct a nonzero class in $\op{Sel}_{\mc L^{\{\ell\}}}(E[p])/\op{Sel}_{\mc L}(E[p])$, which means that we need to show that the rightmost map is nontrivial.
	
	To do this, consider the ``dual'' sequence
	\[0\to\op{Sel}_{\mc L_{\{\ell\}}}(E[p])\to\op{Sel}_{\mc L}(E[p])\to\mathrm H^1_{\mathrm{ur}}(K_\ell;E[p]),\]
	for which \Cref{thm:global-tate-application} tells us that the last images are orthogonal complements. However, the rightmost map of this sequence vanishes because $\op{Sel}_p(E/K)=0$, so the image of $\op{Sel}_{\mc L^{\{\ell\}}}(E[p])$ in $\mathrm H^1(K_\ell;E[p])/\mc L_\ell$ must be nonzero.
\end{proof}
% In particular, for each place $\ell$ where $E[p]$ is unramified and $\ell\nmid p$, we produce some class $c(\ell)\in\mathrm H^1(K;E[p])$ for which $c(\ell)\notin\mc L_\ell$. For this, we 
It turns out that the presence of these classes can go backward to produce vanishing.
\begin{definition}[type]
	Fix an elliptic curve $E$ over a number field $K$. Further, fix a prime $p$ and a place $\ell$ of $K$ such that $p\ne\ell$ and $E$ has good reduction at $\ell$. Then we define the \textit{type} of $\ell$ as
	\[\op{type}_p(\ell)\coloneqq\dim_{\FF_p}\mathrm H^1_{\mathrm{ur}}(K_\ell;E[p]).\]
\end{definition}
\begin{remark}
	We know that $\op{type}_p(\ell)\in\{0,1,2\}$.
\end{remark}
\begin{lemma}
	Fix an elliptic curve $E$ over a number field $K$. Further, fix a prime $p$ and a place $\ell$ of $K$ such that $p\ne\ell$ and $E$ has good reduction at $\ell$. Suppose that the map
	\[\mathrm H^1(K;E[p])\to\prod_{\op{type}_p(\ell)=1}\mathrm H^1(K_\ell;E[p])\]
	is injective. If we have classes $\{c(\ell):\op{type}_p(\ell)=1\}$ in $\mathrm H^1(K;E[p])$ such that $\op{loc}_vc(\ell)\in\mc L_v$ for $v\ne\ell$ and $\op{loc}_vcc(\ell)$, then $\op{Sel}_p(E/K)=0$.
\end{lemma}
\begin{proof}
	Suppose we have a nonzero class $c\in\op{Sel}_p(E/K)$, and we want to show that it vanishes. By the injectivity hypothesis, it is enough to show that $\op{loc}_\ell(c)=0$ for any place $\ell$ with $\op{type}_p(\ell)=1$.

	For this, we compute the global Tate pairing of $c$ and $c(\ell)$. Functoriality of the cup product shows that the diagram
	% https://q.uiver.app/#q=WzAsNixbMCwwLCJcXG1hdGhybSBIXjEoSztFW3BdKSJdLFsxLDAsIlxcbWF0aHJtIEheMShLO0VbcF0pIl0sWzIsMCwiXFxtYXRocm0gSF4yKEs7XFxtdV9wKSJdLFswLDEsIlxcbWF0aHJtIEheMShcXEFBX0s7RVtwXSkiXSxbMSwxLCJcXG1hdGhybSBIXjEoXFxBQV9LO0VbcF0pIl0sWzIsMSwiXFxtYXRocm0gSF4yKFxcQUFfSztcXG11X3ApIl0sWzAsMSwiXFx0aW1lcyIsMyx7InN0eWxlIjp7ImJvZHkiOnsibmFtZSI6Im5vbmUifSwiaGVhZCI6eyJuYW1lIjoibm9uZSJ9fX1dLFszLDQsIlxcdGltZXMiLDMseyJzdHlsZSI6eyJib2R5Ijp7Im5hbWUiOiJub25lIn0sImhlYWQiOnsibmFtZSI6Im5vbmUifX19XSxbMCwzXSxbMSw0XSxbMSwyLCJcXGN1cCJdLFs0LDUsIlxcY3VwIl0sWzIsNV1d&macro_url=https%3A%2F%2Fraw.githubusercontent.com%2FdFoiler%2Fnotes%2Fmaster%2Fnir.tex
	\[\begin{tikzcd}[cramped, column sep=small]
		{\mathrm H^1(K;E[p])} & {\mathrm H^1(K;E[p])} & {\mathrm H^2(K;\mu_p)} \\
		{\mathrm H^1(\AA_K;E[p])} & {\mathrm H^1(\AA_K;E[p])} & {\mathrm H^2(\AA_K;\mu_p)}
		\arrow["\times"{marking, allow upside down}, draw=none, from=1-1, to=1-2]
		\arrow[from=1-1, to=2-1]
		\arrow["\cup", from=1-2, to=1-3]
		\arrow[from=1-2, to=2-2]
		\arrow[from=1-3, to=2-3]
		\arrow["\times"{marking, allow upside down}, draw=none, from=2-1, to=2-2]
		\arrow["\cup", from=2-2, to=2-3]
	\end{tikzcd}\]
	commutes, but the composite
	\[\mathrm H^2(K;\mu_p)\to\mathrm H^2(\AA_K;\mu_p)\to\ZZ/p\ZZ\]
	vanishes by global class field theory, so we see that $\langle c,c(\ell)\rangle=0$ must vanish. On the other hand, this pairing splits up into a sum of local invariants
	\[\sum_v\langle\op{loc}_v c,\op{loc}_vc(\ell)\rangle_v=0.\]
	For $v\ne\ell$, we are computing the inner product of two terms in $\mc L_{v}$, which is Lagrangian, so the pairing vanishes. Thus, we conclude that
	\[\langle\op{loc}_\ell c,\op{loc}_\ell c(\ell)\rangle=0.\]
	Thus, if we did have $\op{loc}_\ell c\ne0$, then $\op{type}_p(\ell)=1$ implies that $\op{loc}_\ell c(\ell)\in\mc L_\ell^*$, so $\op{loc}_\ell c(\ell)\in\mc L_\ell$ because this subspace is Lagrangian, which is a contradiction!
\end{proof}
The moral is that the presence of ``many'' ramified classes implies that a Selmer group vanishes.

\subsection{The Construction of Classes}
We now turn to the construction of some classes. As usual, fix a modular elliptic curve $E$ defined over a field $\QQ$, so we are given a morphism $\varphi_E\colon X_0(N)\to E$. Then fix an imaginary quadratic field $K$ over $\QQ$ satisfying the Heegner hypothesis.

Now, for each positive integer $n$, let $K[n]$ denote the ring class field of $K$ of conductor $n$, meaning that $K[n]/K$ is the abelian extension corresponding to the quotient
\[\op{Gal}(K^{\mathrm{ab}}/K)\from\widehat{K^\times\backslash\AA_K^\times}\onto K^\times\backslash\AA_K^\times/\widehat{\OO}_{K,n}^\times K_\infty^\times,\]
where $\widehat{\OO}_{K,n}$ denotes the profinite completion of the order $\OO_{K,n}=\ZZ+n\OO_K$. For example, $K[1]$ is the usual Hilbert class field.
\begin{remark}
	In general, one expects that $K[n]/K$ to be smaller than the ray class field. Indeed, in general, ray class fields correspond to arbitrary open subgroups of $\AA_K^\times$, but ring class fields require us to have local factors which look like $\ZZ_p+p^\bullet\OO_{K_p}$ for (rational!) primes $p$. Nonetheless, we still know that $K[n]/K$
\end{remark}
Now, choose a point $x_n\in X_0(N)(K[n])$ given by the cyclic isogeny
\[\CC/\OO_{K,n}\to\CC/\mc N^{-1}\OO_{K,n}.\]
Then we can define $y_n\coloneqq\varphi_E(x_n)$. Under a congruence condition, we will be able to take $y_n$ and produce a class $c(n)\in\mathrm H^1(K;E[p])$.
\begin{remark}
	We should not let $c(n)$ be the trace of $y_n$: indeed, this will produce an element of $E(K)$, which we expect to be a multiple of $y_K$ already.
\end{remark}
Roughly speaking, if $\op{Gal}(K[n]/H)$ were cyclic, then we take the ``derivative'' of $y_n$ as $\sum_ii\sigma^{i-1}(y_n)$. Slightly more precisely, we will attempt to produce a class in
\[\mathrm H^1(K[n];E[p])^{\op{Gal}(K[n]/K)}\]
by taking some derivative. Then if $E[p](K[n])$ vanishes, one can show that the map
\[\mathrm H^1(K;E[p])\to\mathrm H^1(K[n];E[p])^{\op{Gal}(K[n]/K)}\]
is an isomorphism by the (longer) Inflation--restriction exact sequence: the next term in the exact sequence is $\mathrm H^2(K[n]/K;E[p](K[n]))$.

\end{document}