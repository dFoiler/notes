% !TEX root = ../notes.tex

\documentclass[../notes.tex]{subfiles}

\begin{document}

\section{October 23}
Here we go.

\subsection{Proof of The Euler System Property}
We are now ready to prove \Cref{thm:kolyvagin-system-prop}.
\begin{proof}[Proof of \Cref{thm:kolyvagin-system-prop}]
	Let's start by showing that $c(n)$ is in the local condition at $v$ for all $v\nmid n$. We will give two arguments.
	\begin{itemize}
		\item We work with large primes $p$. By the Kummer exact sequence
		\[0\to E(K)/pE(K)\to\mathrm H^1(K;E[p])\to\mathrm H^1(K;E)[p]\to0,\]
		it is enough to show that $c(n)$ vanishes in $\mathrm H^1(K;E)[p]$. Now, let $\mc E$ be a N\'eron model of $E$ over $\OO_v$, and it turns out that $E(K_v^{\mathrm{ur}})^\circ\subseteq E(K_v^{\mathrm{ur}})$ is a finite-index subgroup (which is equality when $E$ has good reduction), and it fits in a short exact sequence
		\[0\to E(K_v^{\mathrm{ur}})^\circ\to E(K_v^{\mathrm{ur}})\to\pi_0(\mc E_{\ov\FF_v})\to0.\]
		Now, we claim that $\mathrm H^1(K_v^{\mathrm{ur}}/K_v,E(K_v^{\mathrm{ur}})^\circ)=0$. Let's explain how to conclude, so there is an embedding $\mathrm H^1(K_v^{\mathrm{ur}}/K_v;A)\into\mathrm H^1(\FF_v;\pi_0)$, where $\pi_0$ is the (finite) component group. Thus, for large primes $p$, we see that $\mathrm H^1(K_v^{\mathrm{ur}}/K_v;E)[p]=0$ already!

		It remains to explain why $\mathrm H^1(K_v^{\mathrm{ur}}/K_v,E(K_v^{\mathrm{ur}})^\circ)=0$. In other words, we need to show that every torsor $T$ of $E(K_v^{\mathrm{ur}})$ is trivial. By Hensel's lemma, it is enough to check for a point on $T$ over finite fields, but $\mathrm H^1(\FF_v;G)=0$ already for any smooth connected group scheme $G$.

		\item We work with general $p$. Up to adding torsion, it turns out that $y_n$ is in $E(K[n])^\circ$. Roughly speaking, this is true because $\infty$ and $x_n$ reduce to the same component of $X_0(N)_{\FF_v}$. Thus, $c(n)$ lands in $\mathrm H^1(K_v^{\mathrm{ur}}/K_v;E^\circ)$, which we mentioned above will vanish.
	\end{itemize}
	Next, we show that $c(n)$ lives in $\mathrm H^1_{\mathrm{tr}}(K_\ell;E[p])$ for each $\ell\mid n$. In other words, we need to show that $c(n)$ is inflated from $\mathrm H^1_{\mathrm{tr}}(K[\ell]_{\lambda'}/K_\ell;E[p])$. We will do this by giving an explicit description of the cocycle $c(n)$. For a given point $x\in E$, we let $x/p$ denote some point over the algebraic closure for which $p\cdot(x/p)=x$. We claim that $c(n)$ can be represented by the cocycle
	\[\sigma\mapsto(\sigma-1)\frac{\mc P_n}p-\frac{(\sigma-1)\mc P_n}p.\]
	Observe that $(\sigma-1)(\mc P_n/p)$ is only defined up to some $p$-torsion, but this only adjusts the total by a $1$-coboundary; similarly, $((\sigma-1)\mc P_n)/p$ turns out to be uniquely determined as an element in $E(K[n])$ because $E(K[n])$ has no $p$-torsion of $E$. One can check that this is actually a $1$-cocycle (landing in $E[p]$!), and when restricted to $K[n]$, it turns out to be $\delta_{\mc P_n}$ becomes the second term vanishes. Having the correct restriction to $K[n]$ is enough to characterize $c(n)$, so we have proven the claim.

	We can now calculate with this cocycle to show that $\op{loc}_vc(n)\in\mathrm H^1_{\mathrm{tr}}(K_\ell;E[p])$. By Inflation--restriction, it is enough to check that the restriction of $\op{loc}_vc(n)$ vanishes when restricted to $K[\ell]_{\lambda'}$. We will show that each of the individual terms in the expansion of $c(n)$ above vanishes.
	\begin{itemize}
		\item The term $(\sigma_\ell-1)\cdot(\mc P_n/p)$ vanishes with no work.
		\item To check that the term $(\sigma-1)\mc P_n/p$ vanishes, we should show that $\mc P_n\in pE(K[\ell]_{\lambda'})$. Well, reduction shows $E(K_\lambda)/pE(K_\lambda)$ is isomorphic to $E(\FF_{\ell^2})/pE(\FF_{\ell^2})$ because $E(K_\lambda)$ is a pro-$\ell$ group, so it remains to check that $\mc P_{n/\ell}$ lands in $pE(\FF_{\ell^2})$, but this holds because $\mc P_{n/\ell}$ is in the image of the derivative operator $\mc D_\ell$, and the derivative operator has $\sigma_\ell$ acting by identity$\pmod{\lambda}$.
	\end{itemize}
	It remains to show (ii). We will work in the case where $n=\ell$ for psychological reasons. We compute each side separately, moving around the following diagram.
	% https://q.uiver.app/#q=WzAsNixbMCwwLCJcXG1hdGhybSBIXjFfe1xcbWF0aHJte3VyfX0oS19cXGVsbDtFW3BdKSJdLFsxLDAsIlxcbWF0aHJtIEheMV97XFxtYXRocm17dHJ9fShLX1xcZWxsO0VbcF0pIl0sWzEsMSwiRVtwXShLX1xcZWxsKSJdLFswLDEsIkUoS19cXGVsbCkvcEUoS19cXGVsbCkiXSxbMCwyLCJFKFxcRkZfe1xcZWxsXjJ9KS9wRShcXEZGX3tcXGVsbF4yfSkiXSxbMSwyLCJFW3BdKFxcRkZfe1xcZWxsXjJ9KSJdLFs0LDUsIlxcZnJhYzFwXFxvcHtGcm9ifV9cXGVsbCBIX1xcZWxsKFxcbWF0aHJte0Zyb2J9X3ApIl0sWzMsNF0sWzIsNV0sWzMsMiwiIiwyLHsic3R5bGUiOnsiYm9keSI6eyJuYW1lIjoiZGFzaGVkIn19fV0sWzMsMCwiXFxkZWx0YSJdLFsxLDIsIlxcc2lnbWFfXFxlbGwiXSxbMCwxLCJcXHZhcnBoaV9cXGVsbCIsMCx7InN0eWxlIjp7ImJvZHkiOnsibmFtZSI6ImRhc2hlZCJ9fX1dXQ==&macro_url=https%3A%2F%2Fraw.githubusercontent.com%2FdFoiler%2Fnotes%2Fmaster%2Fnir.tex
	\[\begin{tikzcd}[cramped]
		{\mathrm H^1_{\mathrm{ur}}(K_\ell;E[p])} & {\mathrm H^1_{\mathrm{tr}}(K_\ell;E[p])} \\
		{E(K_\ell)/pE(K_\ell)} & {E[p](K_\ell)} \\
		{E(\FF_{\ell^2})/pE(\FF_{\ell^2})} & {E[p](\FF_{\ell^2})}
		\arrow["{\varphi_\ell}", dashed, from=1-1, to=1-2]
		\arrow["{\sigma_\ell}", from=1-2, to=2-2]
		\arrow["\delta", from=2-1, to=1-1]
		\arrow[dashed, from=2-1, to=2-2]
		\arrow[from=2-1, to=3-1]
		\arrow[from=2-2, to=3-2]
		\arrow["{\frac1p\op{Frob}_\ell H_\ell(\mathrm{Frob}_p)}", from=3-1, to=3-2]
	\end{tikzcd}\]
	Here, recall that $\mathrm H^1_{\mathrm{tr}}(K_\ell;E[p])$, which is the inflation from $\mathrm H^1(K[\ell]_\lambda/K_\ell;E[p])$ is identified with $E[p](K_\ell)$ by evaluating cocycles at $\sigma_\ell$. As such, we calculate
	\[c(\ell)(\sigma_\ell)=(\sigma_\ell-1)\cdot\frac{\mc P_\ell}p-\frac{(\sigma_\ell-1)\mc P_\ell}p.\]
	It turns out that the left term vanishes$\pmod\ell$ because $\sigma_\ell$ doesn't do anything (note that we already know that $\mc P_\ell/p$ already lives in $E(\FF_{\ell^2})$). The right term is
	\[-\frac{(\ell+1)y_\ell-a_\ell y_1}p.\]
	Thus, by using \Cref{prop:euler-system-prop}, we are left with
	\[-\frac{(\ell+1)\mathrm{Frob}(y_1)-a_\ell(y_1)}p\pmod\ell\]
	living in $E[p](\FF_{\ell^2})$. We now see that this gives rise to $c(1)$.
	% Well, when $v\mid\ell$, we want to show that the restriction to $K[\ell]_{\lambda'}$ is trivial (because the transverse subgroup is the inflation from $K[\ell]$), so we need to check that the cocycle vanishes on $\sigma_\ell$: the term $(\sigma_\ell-1)\cdot(\mc P_n/p)$ vanishes, and $((\sigma_\ell-1)\mc P_\ell)/p$ is $(\ell+1)\mathrm{Frob}_\ell y_1-a_\ell y_1$, which can also be seen to vanish.
\end{proof}
\begin{remark}
	In fact, $\mathrm H^1(K_v^{\mathrm{ur}}/K_v,G(K_v^{\mathrm{ur}}))=0$ for any smooth connected group scheme $G$. As such, we see that large primes $p$ imply $\mathrm H^1(K_v^{\mathrm{ur}}/K_v;A)[p]=0$ for all $v$.
\end{remark}

\end{document}