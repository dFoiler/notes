% LTeX: enabled=false

\usepackage[margin=1in, marginparwidth=2cm]{geometry}
\usepackage[table,dvipsnames]{xcolor}
\usepackage{amsmath,amssymb,amsthm}
\usepackage{amsfonts}
\usepackage{asymptote}
\usepackage{cancel}
\usepackage{enumitem}
\usepackage{footnotebackref}
\usepackage{graphicx}
\usepackage{hyperref}
\usepackage{mathdots}
\usepackage{pgffor}
\usepackage{subfiles}
\usepackage{stmaryrd}
\usepackage{tikz-cd}
\usepackage{todonotes}
\usepackage{xparse}
\usepackage{xr}

%%%%%%%%%%%%%%%%%%%%%%%%%%%%%%%%%%%%%%%%%%%% SET-UP
\renewcommand{\familydefault}{\sfdefault}
\usepackage{cabin}
\usepackage[default]{cantarell}

\usepackage{fancyhdr}
\renewcommand{\headrulewidth}{0pt}
\fancypagestyle{contentpage}{%
	\lhead{\textit{\rightmark}}
	\cfoot{\thepage}
}
\hypersetup{
	colorlinks,
	citecolor=black,
	filecolor=black,
	linkcolor=black,
	urlcolor=black
}

% Labeling equations; I don't know where else to put this
\def\equationautorefname#1#2\null{%
	(#2\null)%
}
%%%%%%%%%%%%%%%%%%%%%%%%%%%%%%%%%%%%%%%%%%%% /SET-UP

%%%%%%%%%%%%%%%%%%%%%%%%%%%%%%%%%%%%%%%%%%%% TITLING
\usepackage{titlesec}
\newif\iftoc

% Formatting of part
\titleformat
	{\part} % command
	[display] % shape
	{\cabin\bfseries\LARGE\scshape} % format
	{\centering\LARGE Part \thepart} % label
	{10mm} % 
	{\centering\Huge} % before-code
	[
		\thispagestyle{empty}
	] % after-code
\titlespacing*{\part}{0mm}{30mm}{30mm}
\titleclass{\part}{top}
\newcommand\partbreak{\clearpage}

% Formatting of chapter
\titleformat
	{\chapter} % command
	[display] % shape
	{\cabin} % format
	{} % label
	{2in} % 
	{
		% \rule{\textwidth}{1pt}
		% \vspace{1ex}
		\raggedleft
		% \\\vspace{-22pt}
		\iftoc
			\vspace{2in}
		\else
			{\LARGE\textsc{Theme}~{\cantarell\thechapter}}\\ % I like the other numbers ...
		\fi
		\Huge\scshape\bfseries
	} % before-code
	[
		\vspace{-18pt}%
		\rule{\textwidth}{0.1pt}
		\vspace{0.0in}
	] % after-code
\titlespacing{\chapter}
	{0pt}
	{
		\iftoc
			-103pt+1in
		\else
			-127pt+1in
		\fi
	}
	{0pt}

% Formatting of parts
\titleformat
	{\section}
	{\Large\bfseries}
	{\thesection}
	{1em}
	{}
	[\label{sec:\csname thesection\endcsname}]
\setcounter{tocdepth}{1}
%%%%%%%%%%%%%%%%%%%%%%%%%%%%%%%%%%%%%%%%%%%% /TITLING

%%%%%%%%%%%%%%%%%%%%%%%%%%%%%%%%%%%%%%%%%%%% EPIGRAPH
\usepackage{epigraph}
% Thank you https://tex.stackexchange.com/a/193189
\renewcommand\textflush{flushright}

\usepackage{etoolbox}
\makeatletter
\newlength\epitextskip
\pretocmd{\@epitext}{\em}{}{}
\apptocmd{\@epitext}{\em}{}{}
\patchcmd{\epigraph}
	{\@epitext{#1}\\}
	{\vspace{-0.3in+20pt}\@epitext{#1}\\[\epitextskip]}
	{}
	{}
\makeatother

\setlength\epigraphrule{0pt}
\setlength\epitextskip{2ex}
\setlength\epigraphwidth{.6\textwidth}
\setlength\afterepigraphskip{30pt}
%%%%%%%%%%%%%%%%%%%%%%%%%%%%%%%%%%%%%%%%%%%% /EPIGRAPH

%%%%%%%%%%%%%%%%%%%%%%%%%%%%%%%%%%%%%%%%%%%% CONVENIENCE
% Various black-board things
\renewcommand{\AA}{\mathbb A}
\newcommand{\RR}{\mathbb R}
\newcommand{\ZZ}{\mathbb Z}
\newcommand{\NN}{\mathbb N}
\newcommand{\QQ}{\mathbb Q}
\newcommand{\CC}{\mathbb C}
\newcommand{\FF}{\mathbb F}
\newcommand{\OO}{\mathcal O}
\newcommand{\PP}{\mathbb P}
\newcommand{\CP}{\mathbb{CP}}

\newcommand{\e}{\varepsilon}
\newcommand{\ball}[2]{(#1-#2,\,#1+#2)}

\newcommand{\floor}[1]{\left\lfloor{#1}\right\rfloor}
\newcommand{\ceil}[1]{\left\lceil{#1}\right\rceil}
\newcommand{\norm}[1]{\left\lVert{#1}\right\rVert}
\newcommand{\diff}{\operatorname{diff }}
\newcommand{\disc}{\operatorname{disc }}
\newcommand{\ord}{\operatorname{ord}}
\newcommand{\lcm}{\operatorname{lcm}}
\newcommand{\del}{\partial}
\newcommand{\emp}{\varnothing}
\newcommand{\divides}{\,|\,}
\newcommand{\op}[1]{\operatorname{#1}}
\newcommand{\mf}[1]{\mathfrak{#1}}
\newcommand{\mc}[1]{\mathcal{#1}}

% Algebra
\newcommand{\coker}{\operatorname{coker}} % why isn't this a thing?!
\newcommand{\id}{\operatorname{id}}
\newcommand{\tr}{\operatorname{tr}}
\newcommand{\im}{\operatorname{im}}
\newcommand{\sgn}{\operatorname{sgn}}
\newcommand{\rad}{\operatorname{rad}}
\newcommand{\into}{\hookrightarrow}
\newcommand{\onto}{\twoheadrightarrow}
\newcommand{\from}{\leftarrow}

\newcommand{\limit}{\varprojlim}
\newcommand{\colimit}{\varinjlim}
\newcommand{\opp}{^\mathrm{op}}

% Type theory
\newcommand{\refl}{\op{refl}}
\newcommand{\UU}{\mathcal{U}}

% Complex Analysis
\renewcommand{\Re}{\operatorname{Re}}
\renewcommand{\Im}{\operatorname{Im}}

% Category theory
\DeclareFontFamily{U}{dmjhira}{}
\DeclareFontShape{U}{dmjhira}{m}{n}{ <-> dmjhira }{}

\DeclareRobustCommand{\yo}{\text{\usefont{U}{dmjhira}{m}{n}\symbol{"48}}}

\newenvironment{solution}{
\begin{proof}[Solution]}{\end{proof}}
%%%%%%%%%%%%%%%%%%%%%%%%%%%%%%%%%%%%%%%%%%%% /CONVENIENCE

%%%%%%%%%%%%%%%%%%%%%%%%%%%%%%%%%%%%%%%%%%%% DANGERS
% Thank you https://tex.stackexchange.com/a/604048
\newcommand{\nirideasymbol}{%
	\begin{tikzpicture}[baseline=(x.base)]
		\draw[rounded corners=.01em] (-.05em,-1.07em)rectangle(.05em,.78em);
		\draw[fill=white,rounded corners=1.3] (0,.75em)--(.75em,0)--(0,-.75em)--(-.75em,0)--cycle;
		\draw[line width=0.2mm, line cap=round](-.4em,-1.07em)--(.4em,-1.07em);
		\node(x) at (0,0em) {};
		\node at (0,0em) {{\cabin\textbf{!}}};
	\end{tikzpicture}%
}
\newcommand{\nirwarnsymbol}{%
	\begin{tikzpicture}[baseline=(x.base)]
		\draw[rounded corners=.01em] (-.05em,-1.07em)rectangle(.05em,.78em);
		\draw[fill=white,rounded corners=1.3] (0,.75em)--(.75em,0)--(0,-.75em)--(-.75em,0)--cycle;
		\draw[line width=0.2mm, line cap=round](-.4em,-1.07em)--(.4em,-1.07em);
		\node(x) at (0,0em) {};
		% Thank you https://tex.stackexchange.com/a/262510
		\draw[
			line cap=but,
			line join=round,
			x=.5em,
			line width=0.5mm,
			y=1*(height("Z")-\pgflinewidth)*(1-sin(10)),
			rotate=-10,
			rounded corners=1.5pt,
		](-0.57, 0.57) -- (0.57, 0.57) -- (-0.57, -0.57) -- (0.57, -0.57);
	\end{tikzpicture}%
}
%%%%%%%%%%%%%%%%%%%%%%%%%%%%%%%%%%%%%%%%%%%% /DANGERS

%%%%%%%%%%%%%%%%%%%%%%%%%%%%%%%%%%%%%%%%%%%% MARGINS
\usepackage{marginnote}
% Thank you https://tex.stackexchange.com/a/472882
% Makes marginnotes always appear on th eleft, apparently
%
\makeatletter
\long\def\@mn@@@marginnote[#1]#2[#3]{%
	\begingroup
		\ifmmode\mn@strut\let\@tempa\mn@vadjust\else
			\if@inlabel\leavevmode\fi
			\ifhmode\mn@strut\let\@tempa\mn@vadjust\else\let\@tempa\mn@vlap\fi
		\fi
		\@tempa{%
			\vbox to\z@{%
				\vss
				\@mn@margintest
				\if@reversemargin\if@tempswa
						\@tempswafalse
					\else
						\@tempswatrue
				\fi\fi

					\llap{%
						\vbox to\z@{\kern\marginnotevadjust\kern #3
							\vbox to\z@{%
								\hsize\marginparwidth
								\linewidth\hsize
								\kern-\parskip
								%\mn@parboxrestore
								\marginfont\raggedleftmarginnote\strut\hspace{\z@}%
								\ignorespaces#1\endgraf
								\vss
							}%
							\vss
						}%
						\if@mn@verbose
							\PackageInfo{marginnote}{xpos seems to be \@mn@currxpos}%
						\fi
						\begingroup
							\ifx\@mn@currxpos\relax\else\ifx\@mn@currpos\@empty\else
									\kern\@mn@currxpos
							\fi\fi
							\ifx\@mn@currpage\relax
								\let\@mn@currpage\@ne
							\fi
							\if@twoside\ifodd\@mn@currpage\relax
									\kern-\oddsidemargin
								\else
									\kern-\evensidemargin
								\fi
							\else
								\kern-\oddsidemargin
							\fi
							\kern-1in
						\endgroup
						\kern\marginparsep
					}%
			}%
		}%
	\endgroup
}
\makeatother
%
% Mostly for todonotes
\renewcommand{\marginpar}{\marginnote}
%%%%%%%%%%%%%%%%%%%%%%%%%%%%%%%%%%%%%%%%%%%% /MARGINS

%%%%%%%%%%%%%%%%%%%%%%%%%%%%%%%%%%%%%%%%%%%% LISTS
% Putting these into an environment decreases clicking
\newlist{listalph}{enumerate}{1}
\setlist[listalph,1]{label=(\alph*)}
\newlist{listroman}{enumerate}{1}
\setlist[listroman,1]{label=(\roman*)}
%%%%%%%%%%%%%%%%%%%%%%%%%%%%%%%%%%%%%%%%%%%% /LISTS

%%%%%%%%%%%%%%%%%%%%%%%%%%%%%%%%%%%%%%%%%%%% LISTINGS
\usepackage{courier}
\usepackage{listings}
\lstset{basicstyle=\ttfamily,breaklines=true}
%%%%%%%%%%%%%%%%%%%%%%%%%%%%%%%%%%%%%%%%%%%% /LISTINGS

%%%%%%%%%%%%%%%%%%%%%%%%%%%%%%%%%%%%%%%%%%%% THM BOXES
% See http://texdoc.net/texmf-dist/doc/latex/thmtools/thmtools.pdf
\renewcommand{\qedsymbol}{$\blacksquare$}
\usepackage{thmtools,thm-restate}

\usepackage[framemethod=TikZ]{mdframed}
% Fixing mdframed skip below
% See https://tex.stackexchange.com/a/292090/143086
\usepackage[framemethod=TikZ]{mdframed}
\usepackage{xpatch}
\makeatletter
\xpatchcmd{\endmdframed}
	{\aftergroup\endmdf@trivlist\color@endgroup}
	{\endmdf@trivlist\color@endgroup\@doendpe}
	{}{}
\makeatother

\definecolor{thmblue}{RGB}{225, 250, 250}
\declaretheoremstyle[
	mdframed={
		backgroundcolor=thmblue,
		linecolor=blue,
		rightline=false,
		topline=false,
		bottomline=false,
		linewidth=2pt,
		innertopmargin=5pt,
		innerbottommargin=8pt,
		innerleftmargin=8pt,
		leftmargin=-2pt,
		skipbelow=2pt,
		nobreak
	}
]{thmblue}
% I want to label things by chapter, but not all things I write have chapter
\ifx\thechapter\undefined
	\declaretheorem[style=thmblue,name=Theorem]{thm}
\else
	\declaretheorem[style=thmblue,name=Theorem,within=chapter]{thm}
\fi
\declaretheorem[style=thmblue,name=Theorem,sibling=thm]{theorem}
\declaretheorem[style=thmblue,name=Proposition,sibling=thm]{prop}
\declaretheorem[style=thmblue,name=Proposition,sibling=thm]{proposition}

\definecolor{probblue}{RGB}{100, 150, 220}
\declaretheoremstyle[
	mdframed={
		backgroundcolor=thmblue,
		linecolor=probblue,
		rightline=false,
		topline=true,
		bottomline=false,
		linewidth=2pt,
		innertopmargin=5pt,
		innerbottommargin=8pt,
		innerleftmargin=8pt,
		leftmargin=-2pt,
		skipbelow=2pt,
		nobreak
	}
]{probblue}
\declaretheorem[style=probblue,name=Problem,numberwithin=section]{prob}

\declaretheoremstyle[
	mdframed={
		backgroundcolor=thmblue,
		linecolor=probblue,
		rightline=false,
		topline=false,
		bottomline=false,
		linewidth=2pt,
		innertopmargin=5pt,
		innerbottommargin=8pt,
		innerleftmargin=8pt,
		leftmargin=-2pt,
		skipbelow=2pt,
		nobreak
	}
]{lemblue}
\declaretheorem[style=lemblue,name=Lemma,sibling=thm]{lem}
\declaretheorem[style=lemblue,name=Lemma,sibling=thm]{lemma}
\declaretheorem[style=lemblue,name=Corollary,sibling=thm]{cor}
\declaretheorem[style=lemblue,name=Corollary,sibling=thm]{corollary}
\declaretheorem[style=lemblue,name=Exercise,sibling=thm]{exercise}
\declaretheorem[style=lemblue,name=Exercise,sibling=thm]{exe}

\definecolor{thmred}{RGB}{250, 220, 220}
\declaretheoremstyle[
	mdframed={
		backgroundcolor=thmred,
		linecolor=red,
		rightline=false,
		topline=false,
		bottomline=false,
		linewidth=2pt,
		innertopmargin=5pt,
		innerbottommargin=8pt,
		innerleftmargin=8pt,
		leftmargin=-2pt,
		skipbelow=2pt,
		nobreak
	}
]{thmred}
\declaretheorem[style=thmred,name=Conjecture,sibling=thm]{conj}
\declaretheorem[style=thmred,name=Question,sibling=thm]{ques}
\declaretheorem[style=thmred,name=Convention,sibling=thm]{conv}
\declaretheorem[style=thmred,name=Convention,sibling=thm]{convention}

\declaretheorem[style=thmred,name=Idea,sibling=thm]{ideahelper}
\NewDocumentEnvironment{idea}{o}
	{
		\IfNoValueTF{#1}%
			{\begin{ideahelper}\phantom{}\marginnote{\nirideasymbol}[-3pt]}%
			{\begin{ideahelper}[#1]\phantom{}\marginnote{\nirideasymbol}[-3pt]}%
	}
	{
		\end{ideahelper}
	}

\declaretheorem[style=thmred,name=Warning,sibling=thm]{warnhelper}
\NewDocumentEnvironment{warn}{o}
	{
		\IfNoValueTF{#1}%
			{\begin{warnhelper}\phantom{}\marginnote{\nirwarnsymbol}[-3pt]}%
			{\begin{warnhelper}[#1]\phantom{}\marginnote{\nirwarnsymbol}[-3pt]}%
	}
	{
		\end{warnhelper}
	}

\definecolor{thmlightgreen}{RGB}{230, 255, 234}
\definecolor{thmdarkgreen}{RGB}{60, 110, 60}
\declaretheoremstyle[
	mdframed={
		backgroundcolor=thmlightgreen,
		linecolor=thmdarkgreen,
		rightline=false,
		topline=false,
		bottomline=false,
		linewidth=2pt,
		innertopmargin=5pt,
		innerbottommargin=8pt,
		innerleftmargin=8pt,
		leftmargin=-2pt,
		skipbelow=2pt,
		nobreak
	}
]{thmgreen}
\declaretheorem[style=thmgreen,name=Axiom,sibling=thm]{ax}
\declaretheorem[style=thmgreen,name=Axiom,sibling=thm]{axiom}
\declaretheorem[style=thmgreen,name=Inventory,sibling=thm]{inv}
\declaretheorem[style=thmgreen,name=Inventory,sibling=thm]{inventory}
\declaretheorem[style=thmgreen,name=Definition,sibling=thm]{definition}
\declaretheorem[style=thmgreen,name=Definition,sibling=thm]{defi}

\definecolor{exdarkgreen}{RGB}{120, 180, 120}
\declaretheoremstyle[
	mdframed={
		backgroundcolor=thmlightgreen,
		linecolor=exdarkgreen,
		rightline=false,
		topline=false,
		bottomline=false,
		linewidth=2pt,
		innertopmargin=5pt,
		innerbottommargin=8pt,
		innerleftmargin=8pt,
		leftmargin=-2pt,
		skipbelow=2pt,
		nobreak
	}
]{exgreen}
\declaretheorem[style=exgreen,name=Example,sibling=thm]{ex}
\declaretheorem[style=exgreen,name=Example,sibling=thm]{example}
\declaretheorem[style=exgreen,name=Non-Example,sibling=thm]{nex}
\declaretheorem[style=exgreen,name=Non-Definition,sibling=thm]{ndefi}

\definecolor{remlightbrown}{RGB}{252,240,235}
\definecolor{remdarkbrown}{RGB}{180, 133, 75}
\declaretheoremstyle[
	mdframed={
		backgroundcolor=remlightbrown,
		linecolor=remdarkbrown,
		rightline=false,
		topline=false,
		bottomline=false,
		linewidth=2pt,
		innertopmargin=5pt,
		innerbottommargin=8pt,
		innerleftmargin=8pt,
		leftmargin=-2pt,
		skipbelow=2pt,
		nobreak
	}
]{rembrown}
\declaretheorem[style=rembrown,name=Remark,sibling=thm]{remark}
\declaretheorem[style=rembrown,name=Quote,sibling=thm]{quot}
%%%%%%%%%%%%%%%%%%%%%%%%%%%%%%%%%%%%%%%%%%%% /THM BOXES

%%%%%%%%%%%%%%%%%%%%%%%%%%%%%%%%%%%%%%%%%%%% QUIVER
% *** quiver ***
% A package for drawing commutative diagrams exported from https://q.uiver.app.
%
% This package is currently a wrapper around the `tikz-cd` package, importing necessary TikZ
% libraries, and defining a new TikZ style for curves of a fixed height.
%
% Version: 1.2.0
% Authors:
% - varkor (https://github.com/varkor)
% - AndréC (https://tex.stackexchange.com/users/138900/andr%C3%A9c)

% NIR: this is causing errors, so I killed it
% \NeedsTeXFormat{LaTeX2e}
% \ProvidesPackage{quiver}[2021/01/11 quiver]

% `tikz-cd` is necessary to draw commutative diagrams.
\RequirePackage{tikz-cd}
% `amssymb` is necessary for `\lrcorner` and `\ulcorner`.
\RequirePackage{amssymb}
% `calc` is necessary to draw curved arrows.
\usetikzlibrary{calc}
% `pathmorphing` is necessary to draw squiggly arrows.
\usetikzlibrary{decorations.pathmorphing}

% A TikZ style for curved arrows of a fixed height, due to AndréC.
\tikzset{curve/.style={settings={#1},to path={(\tikztostart)
	.. controls ($(\tikztostart)!\pv{pos}!(\tikztotarget)!\pv{height}!270:(\tikztotarget)$)
	and ($(\tikztostart)!1-\pv{pos}!(\tikztotarget)!\pv{height}!270:(\tikztotarget)$)
	.. (\tikztotarget)\tikztonodes}},
	settings/.code={\tikzset{quiver/.cd,#1}
		\def\pv##1{\pgfkeysvalueof{/tikz/quiver/##1}}},
	quiver/.cd,pos/.initial=0.35,height/.initial=0}

% TikZ arrowhead/tail styles.
\tikzset{tail reversed/.code={\pgfsetarrowsstart{tikzcd to}}}
\tikzset{2tail/.code={\pgfsetarrowsstart{Implies[reversed]}}}
\tikzset{2tail reversed/.code={\pgfsetarrowsstart{Implies}}}
% TikZ arrow styles.
\tikzset{no body/.style={/tikz/dash pattern=on 0 off 1mm}}
%%%%%%%%%%%%%%%%%%%%%%%%%%%%%%%%%%%%%%%%%%%% /QUIVER

%%%%%%%%%%%%%%%%%%%%%%%%%%%%%%%%%%%%%%%%%%%% FITCH
% Macros for Fitch-style natural deduction. 
% Author: Peter Selinger, University of Ottawa
% Created: Jan 14, 2002
% Modified: Feb 8, 2005
% Version: 0.5
% Copyright: (C) 2002-2005 Peter Selinger
% Filename: fitch.sty
% Documentation: fitchdoc.tex
% URL: http://quasar.mathstat.uottawa.ca/~selinger/fitch/

% License:
%
% This program is free software; you can redistribute it and/or modify
% it under the terms of the GNU General Public License as published by
% the Free Software Foundation; either version 2, or (at your option)
% any later version.
%
% This program is distributed in the hope that it will be useful, but
% WITHOUT ANY WARRANTY; without even the implied warranty of
% MERCHANTABILITY or FITNESS FOR A PARTICULAR PURPOSE. See the GNU
% General Public License for more details.
%
% You should have received a copy of the GNU General Public License
% along with this program; if not, write to the Free Software Foundation, 
% Inc., 59 Temple Place, Suite 330, Boston, MA 02111-1307, USA.

% USAGE EXAMPLE:
% 
% The following is a simple example illustrating the usage of this
% package.  For detailed instructions and additional functionality, see
% the user guide, which can be found in the file fitchdoc.tex.
% 
% \[
% \begin{nd}
%   \hypo{1}  {P\vee Q}   
%   \hypo{2}  {\neg Q}                         
%   \open                              
%   \hypo{3a} {P}
%   \have{3b} {P}        \r{3a}
%   \close                   
%   \open
%   \hypo{4a} {Q}
%   \have{4b} {\neg Q}   \r{2}
%   \have{4c} {\bot}     \ne{4a,4b}
%   \have{4d} {P}        \be{4c}
%   \close                             
%   \have{5}  {P}        \oe{1,3a-3b,4a-4d}                 
% \end{nd}
% \]

{\chardef\x=\catcode`\*
\catcode`\*=11
\global\let\nd*astcode\x}
\catcode`\*=11

% References

\newcount\nd*ctr
\def\nd*render{\expandafter\ifx\expandafter\nd*x\nd*base\nd*x\the\nd*ctr\else\nd*base\ifnum\nd*ctr<0\the\nd*ctr\else\ifnum\nd*ctr>0+\the\nd*ctr\fi\fi\fi}
\expandafter\def\csname nd*-\endcsname{}

\def\nd*num#1{\nd*numo{\nd*render}{#1}\global\advance\nd*ctr1}
\def\nd*numopt#1#2{\nd*numo{$#1$}{#2}}
\def\nd*numo#1#2{\edef\x{#1}\mbox{$\x$}\expandafter\global\expandafter\let\csname nd*-#2\endcsname\x}
\def\nd*ref#1{\expandafter\let\expandafter\x\csname nd*-#1\endcsname\ifx\x\relax%
  \errmessage{Undefined natdeduction reference: #1}\else\mbox{$\x$}\fi}
\def\nd*noop{}
\def\nd*set#1#2{\ifx\relax#1\nd*noop\else\global\def\nd*base{#1}\fi\ifx\relax#2\relax\else\global\nd*ctr=#2\fi}
\def\nd*reset{\nd*set{}{1}}
\def\nd*refa#1{\nd*ref{#1}}
\def\nd*aux#1#2{\ifx#2-\nd*refa{#1}--\def\nd*c{\nd*aux{}}%
  \else\ifx#2,\nd*refa{#1}, \def\nd*c{\nd*aux{}}%
  \else\ifx#2;\nd*refa{#1}; \def\nd*c{\nd*aux{}}%
  \else\ifx#2.\nd*refa{#1}. \def\nd*c{\nd*aux{}}%
  \else\ifx#2)\nd*refa{#1})\def\nd*c{\nd*aux{}}%
  \else\ifx#2(\nd*refa{#1}(\def\nd*c{\nd*aux{}}%
  \else\ifx#2\nd*end\nd*refa{#1}\def\nd*c{}%
  \else\def\nd*c{\nd*aux{#1#2}}%
  \fi\fi\fi\fi\fi\fi\fi\nd*c}
\def\ndref#1{\nd*aux{}#1\nd*end}

% Layer A

% define various dimensions (explained in fitchdoc.tex):
\newlength{\nd*dim} 
\newdimen\nd*depthdim
\newdimen\nd*hsep
\newdimen\ndindent
\ndindent=1em
% user command to redefine dimensions
\def\nddim#1#2#3#4#5#6#7#8{\nd*depthdim=#3\relax\nd*hsep=#6\relax%
\def\nd*height{#1}\def\nd*thickness{#8}\def\nd*initheight{#2}%
\def\nd*indent{#5}\def\nd*labelsep{#4}\def\nd*justsep{#7}}
% set initial dimensions
\nddim{4.5ex}{3.5ex}{1.5ex}{1em}{1.6em}{.5em}{2.5em}{.2mm}

\def\nd*v{\rule[-\nd*depthdim]{\nd*thickness}{\nd*height}}
\def\nd*t{\rule[-\nd*depthdim]{0mm}{\nd*height}\rule[-\nd*depthdim]{\nd*thickness}{\nd*initheight}}
\def\nd*i{\hspace{\nd*indent}} 
\def\nd*s{\hspace{\nd*hsep}}
\def\nd*g#1{\nd*f{\makebox[\nd*indent][c]{$#1$}}}
\def\nd*f#1{\raisebox{0pt}[0pt][0pt]{$#1$}}
\def\nd*u#1{\makebox[0pt][l]{\settowidth{\nd*dim}{\nd*f{#1}}%
	\addtolength{\nd*dim}{2\nd*hsep}\hspace{-\nd*hsep}\rule[-\nd*depthdim]{\nd*dim}{\nd*thickness}}\nd*f{#1}}

% Lists

\def\nd*push#1#2{\expandafter\gdef\expandafter#1\expandafter%
	{\expandafter\nd*cons\expandafter{#1}{#2}}}
\def\nd*pop#1{{\def\nd*nil{\gdef#1{\nd*nil}}\def\nd*cons##1##2%
		{\gdef#1{##1}}#1}}
\def\nd*iter#1#2{{\def\nd*nil{}\def\nd*cons##1##2{##1#2{##2}}#1}}
\def\nd*modify#1#2#3{{\def\nd*nil{\gdef#1{\nd*nil}}\def\nd*cons##1##2%
		{\advance#2-1 ##1\advance#2 1 \ifnum#2=1\nd*push#1{#3}\else%
			\nd*push#1{##2}\fi}#1}}

\def\nd*cont#1{{\def\nd*t{\nd*v}\def\nd*v{\nd*v}\def\nd*g##1{\nd*i}%
		\def\nd*i{\nd*i}\def\nd*nil{\gdef#1{\nd*nil}}\def\nd*cons##1##2%
		{##1\expandafter\nd*push\expandafter#1\expandafter{##2}}#1}}

% Layer B

\newcount\nd*n
\def\nd*beginb{\begingroup\nd*reset\gdef\nd*stack{\nd*nil}\nd*push\nd*stack{\nd*t}%
	\begin{array}{l@{\hspace{\nd*labelsep}}l@{\hspace{\nd*justsep}}l}}
\def\nd*resumeb{\begingroup\begin{array}{l@{\hspace{\nd*labelsep}}l@{\hspace{\nd*justsep}}l}}
\def\nd*endb{\end{array}\endgroup}
\def\nd*hypob#1#2{\nd*f{\nd*num{#1}}&\nd*iter\nd*stack\relax\nd*cont\nd*stack\nd*s\nd*u{#2}&}
\def\nd*haveb#1#2{\nd*f{\nd*num{#1}}&\nd*iter\nd*stack\relax\nd*cont\nd*stack\nd*s\nd*f{#2}&}
\def\nd*havecontb#1#2{&\nd*iter\nd*stack\relax\nd*cont\nd*stack\nd*s\nd*f{\hspace{\ndindent}#2}&}
\def\nd*hypocontb#1#2{&\nd*iter\nd*stack\relax\nd*cont\nd*stack\nd*s\nd*u{\hspace{\ndindent}#2}&}

\def\nd*openb{\nd*push\nd*stack{\nd*i}\nd*push\nd*stack{\nd*t}}
\def\nd*closeb{\nd*pop\nd*stack\nd*pop\nd*stack}
\def\nd*guardb#1#2{\nd*n=#1\multiply\nd*n by 2 \nd*modify\nd*stack\nd*n{\nd*g{#2}}}

% Layer C

\def\nd*clr{\gdef\nd*cmd{}\gdef\nd*typ{\relax}}
\def\nd*sto#1#2#3{\gdef\nd*typ{#1}\gdef\nd*byt{}%
  \gdef\nd*cmd{\nd*typ{#2}{#3}\nd*byt\\}}
\def\nd*chtyp{\expandafter\ifx\nd*typ\nd*hypocontb\def\nd*typ{\nd*havecontb}\else\def\nd*typ{\nd*haveb}\fi}
\def\nd*hypoc#1#2{\nd*chtyp\nd*cmd\nd*sto{\nd*hypob}{#1}{#2}}
\def\nd*havec#1#2{\nd*cmd\nd*sto{\nd*haveb}{#1}{#2}}
\def\nd*hypocontc#1{\nd*chtyp\nd*cmd\nd*sto{\nd*hypocontb}{}{#1}}
\def\nd*havecontc#1{\nd*cmd\nd*sto{\nd*havecontb}{}{#1}}
\def\nd*by#1#2{\ifx\nd*x#2\nd*x\gdef\nd*byt{\mbox{#1}}\else\gdef\nd*byt{\mbox{#1, \ndref{#2}}}\fi}

% multi-line macros
\def\nd*mhypoc#1#2{\nd*mhypocA{#1}#2\\\nd*stop\\}
\def\nd*mhypocA#1#2\\{\nd*hypoc{#1}{#2}\nd*mhypocB}
\def\nd*mhypocB#1\\{\ifx\nd*stop#1\else\nd*hypocontc{#1}\expandafter\nd*mhypocB\fi}
\def\nd*mhavec#1#2{\nd*mhavecA{#1}#2\\\nd*stop\\}
\def\nd*mhavecA#1#2\\{\nd*havec{#1}{#2}\nd*mhavecB}
\def\nd*mhavecB#1\\{\ifx\nd*stop#1\else\nd*havecontc{#1}\expandafter\nd*mhavecB\fi}
\def\nd*mhypocontc#1{\nd*mhypocB#1\\\nd*stop\\}
\def\nd*mhavecontc#1{\nd*mhavecB#1\\\nd*stop\\}

\def\nd*beginc{\nd*beginb\nd*clr}
\def\nd*resumec{\nd*resumeb\nd*clr}
\def\nd*endc{\nd*cmd\nd*endb}
\def\nd*openc{\nd*cmd\nd*clr\nd*openb}
\def\nd*closec{\nd*cmd\nd*clr\nd*closeb}
\let\nd*guardc\nd*guardb

% Layer D

% macros with optional arguments spelled-out
\def\nd*hypod[#1][#2]#3[#4]#5{\ifx\relax#4\relax\else\nd*guardb{1}{#4}\fi\nd*mhypoc{#3}{#5}\nd*set{#1}{#2}}
\def\nd*haved[#1][#2]#3[#4]#5{\ifx\relax#4\relax\else\nd*guardb{1}{#4}\fi\nd*mhavec{#3}{#5}\nd*set{#1}{#2}}
\def\nd*havecont#1{\nd*mhavecontc{#1}}
\def\nd*hypocont#1{\nd*mhypocontc{#1}}
\def\nd*base{undefined}
\def\nd*opend[#1]#2{\nd*cmd\nd*clr\nd*openb\nd*guard{#1}#2}
\def\nd*close{\nd*cmd\nd*clr\nd*closeb}
\def\nd*guardd[#1]#2{\nd*guardb{#1}{#2}}

% Handling of optional arguments.

\def\nd*optarg#1#2#3{\ifx[#3\def\nd*c{#2#3}\else\def\nd*c{#2[#1]{#3}}\fi\nd*c}
\def\nd*optargg#1#2#3{\ifx[#3\def\nd*c{#1#3}\else\def\nd*c{#2{#3}}\fi\nd*c}

\def\nd*five#1{\nd*optargg{\nd*four{#1}}{\nd*two{#1}}}
\def\nd*four#1[#2]{\nd*optarg{0}{\nd*three{#1}[#2]}}
\def\nd*three#1[#2][#3]#4{\nd*optarg{}{#1[#2][#3]{#4}}}
\def\nd*two#1{\nd*three{#1}[\relax][]}

\def\nd*have{\nd*five{\nd*haved}}
\def\nd*hypo{\nd*five{\nd*hypod}}
\def\nd*open{\nd*optarg{}{\nd*opend}}
\def\nd*guard{\nd*optarg{1}{\nd*guardd}}

\def\nd*init{%
	\let\open\nd*open%
	\let\close\nd*close%
	\let\hypo\nd*hypo%
	\let\have\nd*have%
	\let\hypocont\nd*hypocont%
	\let\havecont\nd*havecont%
	\let\by\nd*by%
	\let\guard\nd*guard%
	\def\ii{\by{$\Rightarrow$I}}%
	\def\ie{\by{$\Rightarrow$E}}%
	\def\Ai{\by{$\forall$I}}%
	\def\Ae{\by{$\forall$E}}%
	\def\Ei{\by{$\exists$I}}%
	\def\Ee{\by{$\exists$E}}%
	\def\ai{\by{$\wedge$I}}%
	\def\ae{\by{$\wedge$E}}%
	\def\ai{\by{$\wedge$I}}%
	\def\ae{\by{$\wedge$E}}%
	\def\oi{\by{$\vee$I}}%
	\def\oe{\by{$\vee$E}}%
	\def\ni{\by{$\neg$I}}%
	\def\ne{\by{$\neg$E}}%
	\def\be{\by{$\bot$E}}%
	\def\nne{\by{$\neg\neg$E}}%
	\def\r{\by{R}}%
}

\newenvironment{nd}{\begingroup\nd*init\nd*beginc}{\nd*endc\endgroup}
\newenvironment{ndresume}{\begingroup\nd*init\nd*resumec}{\nd*endc\endgroup}

\catcode`\*=\nd*astcode

% End of file fitch.sty
%%%%%%%%%%%%%%%%%%%%%%%%%%%%%%%%%%%%%%%%%%%% /FITCH