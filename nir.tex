\usepackage[margin=1in]{geometry}
\usepackage[dvipsnames]{xcolor}
\usepackage{amsmath,amssymb,amsthm}
\usepackage{amsfonts}
\usepackage{asymptote}
\usepackage{cancel}
\usepackage{enumitem}
\usepackage{footnotebackref}
\usepackage{graphicx}
\usepackage{hyperref}
\usepackage{mathdots}
\usepackage{pgffor}
\usepackage{subfiles}
\usepackage{tikz-cd}
\usepackage{todonotes}
\usepackage{xr}

%%%%%%%%%%%%%%%%%%%%%%%%%%%%%%%%%%%%%%%%%%%% SET-UP
\renewcommand{\familydefault}{\sfdefault}
\usepackage{cabin}
\usepackage[default]{cantarell}

\usepackage{fancyhdr}
\renewcommand{\headrulewidth}{0pt}
\fancypagestyle{contentpage}{%
	\lhead{\textit{\rightmark}}
	\cfoot{\thepage}
}
\hypersetup{
	colorlinks,
	citecolor=black,
	filecolor=black,
	linkcolor=black,
	urlcolor=black
}
%%%%%%%%%%%%%%%%%%%%%%%%%%%%%%%%%%%%%%%%%%%% /SET-UP

%%%%%%%%%%%%%%%%%%%%%%%%%%%%%%%%%%%%%%%%%%%% TITLING
\usepackage{titlesec}
\newif\iftoc
\titleformat
	{\chapter} % command
	[display] % shape
	{\cabin\bfseries\LARGE} % format
	{} % label
	{0pt} % 
	{
		\rule{\textwidth}{1pt}
		\vspace{1ex}
		\centering
		\\\vspace{-22pt}
		\iftoc
			\scshape
		\else
			\textsc{Theme}~{\cantarell\thechapter}\scshape:~ % I like the other numbers ...
		\fi
	} % before-code
	[
		\vspace{-0.5ex}%
		\rule{\textwidth}{0.3pt}
	] % after-code
\titlespacing{\chapter}
	{0pt}
	{
		\iftoc
			-103pt+1in
		\else
			-127pt+1in
		\fi
	}
	{40pt}
\titleformat
	{\section}
	{\Large\bfseries}
	{\thesection}
	{1em}
	{}
	[\label{sec:\csname thesection\endcsname}]
\setcounter{tocdepth}{1}
%%%%%%%%%%%%%%%%%%%%%%%%%%%%%%%%%%%%%%%%%%%% /TITLING

%%%%%%%%%%%%%%%%%%%%%%%%%%%%%%%%%%%%%%%%%%%% CONVENIENCE
\renewcommand{\AA}{\mathbb A}
\newcommand{\RR}{\mathbb R}
\newcommand{\ZZ}{\mathbb Z}
\newcommand{\NN}{\mathbb N}
\newcommand{\QQ}{\mathbb Q}
\newcommand{\CC}{\mathbb C}
\newcommand{\FF}{\mathbb F}
\newcommand{\OO}{\mathcal O}
\newcommand{\PP}{\mathbb P}
\newcommand{\CP}{\mathbb{CP}}
\newcommand{\e}{\varepsilon}
\newcommand{\ball}[2]{(#1-#2,\,#1+#2)}

\newcommand{\floor}[1]{\left\lfloor{#1}\right\rfloor}
\newcommand{\ceil}[1]{\left\lceil{#1}\right\rceil}
\newcommand{\norm}[1]{\left\lVert{#1}\right\rVert}
\newcommand{\diff}{\operatorname{diff }}
\newcommand{\disc}{\operatorname{disc }}
\newcommand{\ord}{\operatorname{ord}}
\newcommand{\lcm}{\operatorname{lcm}}
\newcommand{\del}{\partial}
\newcommand{\emp}{\varnothing}
\newcommand{\divides}{\,|\,}
\newcommand{\op}[1]{\operatorname{#1}}
\newcommand{\mf}[1]{\mathfrak{#1}}
\newcommand{\mc}[1]{\mathcal{#1}}

% Algebra
\newcommand{\coker}{\operatorname{coker}} % why isn't this a thing?!
\newcommand{\id}{\operatorname{id}}
\newcommand{\tr}{\operatorname{tr}}
\newcommand{\im}{\operatorname{im}}
\newcommand{\sgn}{\operatorname{sgn}}
\newcommand{\into}{\hookrightarrow}
\newcommand{\onto}{\twoheadrightarrow}
\newcommand{\from}{\leftarrow}

\newcommand{\limit}{\varprojlim}
\newcommand{\colimit}{\varinjlim}

% Type theory
\newcommand{\refl}{\op{refl}}
\newcommand{\UU}{\mathcal{U}}

\newenvironment{solution}{
\begin{proof}[Solution]}{\end{proof}}
%%%%%%%%%%%%%%%%%%%%%%%%%%%%%%%%%%%%%%%%%%%% /CONVENIENCE

%%%%%%%%%%%%%%%%%%%%%%%%%%%%%%%%%%%%%%%%%%%% EPIGRAPH
\usepackage{epigraph}
% Thank you https://tex.stackexchange.com/a/193189
\renewcommand\textflush{flushright}

\usepackage{etoolbox}
\makeatletter
\newlength\epitextskip
\pretocmd{\@epitext}{\em}{}{}
\apptocmd{\@epitext}{\em}{}{}
\patchcmd{\epigraph}{\@epitext{#1}\\}{\@epitext{#1}\\[\epitextskip]}{}{}
\makeatother

\setlength\epigraphrule{0pt}
\setlength\epitextskip{2ex}
\setlength\epigraphwidth{.8\textwidth}
%%%%%%%%%%%%%%%%%%%%%%%%%%%%%%%%%%%%%%%%%%%% /EPIGRAPH

%%%%%%%%%%%%%%%%%%%%%%%%%%%%%%%%%%%%%%%%%%%% LISTS
% Putting these into an environment decreases clicking
\newlist{listalph}{enumerate}{1}
\setlist[listalph,1]{label=(\alph*)}
\newlist{listroman}{enumerate}{1}
\setlist[listroman,1]{label=(\roman*)}
%%%%%%%%%%%%%%%%%%%%%%%%%%%%%%%%%%%%%%%%%%%% /LISTS

%%%%%%%%%%%%%%%%%%%%%%%%%%%%%%%%%%%%%%%%%%%% LISTINGS
\usepackage{courier}
\usepackage{listings}
\lstset{basicstyle=\ttfamily,breaklines=true}
%%%%%%%%%%%%%%%%%%%%%%%%%%%%%%%%%%%%%%%%%%%% /LISTINGS

%%%%%%%%%%%%%%%%%%%%%%%%%%%%%%%%%%%%%%%%%%%% THM BOXES
% See http://texdoc.net/texmf-dist/doc/latex/thmtools/thmtools.pdf
\renewcommand{\qedsymbol}{$\blacksquare$}
\usepackage{thmtools,thm-restate}

\usepackage[framemethod=TikZ]{mdframed}
% Fixing mdframed skip below
% See https://tex.stackexchange.com/a/292090/143086
\usepackage[framemethod=TikZ]{mdframed}
\usepackage{xpatch}
\makeatletter
\xpatchcmd{\endmdframed}
	{\aftergroup\endmdf@trivlist\color@endgroup}
	{\endmdf@trivlist\color@endgroup\@doendpe}
	{}{}
\makeatother

\definecolor{thmblue}{RGB}{225, 250, 250}
\declaretheoremstyle[
	mdframed={
		backgroundcolor=thmblue,
		linecolor=blue,
		rightline=false,
		topline=false,
		bottomline=false,
		linewidth=2pt,
		innertopmargin=5pt,
		innerbottommargin=8pt,
		innerleftmargin=8pt,
		leftmargin=-2pt,
		skipbelow=2pt,
		nobreak
	}
]{thmblue}
\declaretheorem[style=thmblue,name=Theorem]{thm}
\declaretheorem[style=thmblue,name=Theorem,sibling=thm]{theorem}
\declaretheorem[style=thmblue,name=Proposition,sibling=thm]{prop}
\declaretheorem[style=thmblue,name=Proposition,sibling=thm]{proposition}

\definecolor{probblue}{RGB}{100, 150, 220}
\declaretheoremstyle[
	mdframed={
		backgroundcolor=thmblue,
		linecolor=probblue,
		rightline=false,
		topline=true,
		bottomline=false,
		linewidth=2pt,
		innertopmargin=5pt,
		innerbottommargin=8pt,
		innerleftmargin=8pt,
		leftmargin=-2pt,
		skipbelow=2pt,
		nobreak
	}
]{probblue}
\declaretheorem[style=probblue,name=Problem,numberwithin=section]{prob}

\declaretheoremstyle[
	mdframed={
		backgroundcolor=thmblue,
		linecolor=probblue,
		rightline=false,
		topline=false,
		bottomline=false,
		linewidth=2pt,
		innertopmargin=5pt,
		innerbottommargin=8pt,
		innerleftmargin=8pt,
		leftmargin=-2pt,
		skipbelow=2pt,
		nobreak
	}
]{lemblue}
\declaretheorem[style=lemblue,name=Lemma,sibling=thm]{lem}
\declaretheorem[style=lemblue,name=Lemma,sibling=thm]{lemma}
\declaretheorem[style=lemblue,name=Corollary,sibling=thm]{cor}
\declaretheorem[style=lemblue,name=Corollary,sibling=thm]{corollary}
\declaretheorem[style=lemblue,name=Exercise,sibling=thm]{exercise}
\declaretheorem[style=lemblue,name=Exercise,sibling=thm]{exe}

\definecolor{thmred}{RGB}{250, 220, 220}
\declaretheoremstyle[
	mdframed={
		backgroundcolor=thmred,
		linecolor=red,
		rightline=false,
		topline=false,
		bottomline=false,
		linewidth=2pt,
		innertopmargin=5pt,
		innerbottommargin=8pt,
		innerleftmargin=8pt,
		leftmargin=-2pt,
		skipbelow=2pt,
		nobreak
	}
]{thmred}
\declaretheorem[style=thmred,name=Idea,sibling=thm]{idea}
\declaretheorem[style=thmred,name=Conjecture,sibling=thm]{conj}
\declaretheorem[style=thmred,name=Question,sibling=thm]{ques}
\declaretheorem[style=thmred,name=Warning,sibling=thm]{warn}

\definecolor{thmlightgreen}{RGB}{230, 255, 234}
\definecolor{thmdarkgreen}{RGB}{60, 110, 60}
\declaretheoremstyle[
	mdframed={
		backgroundcolor=thmlightgreen,
		linecolor=thmdarkgreen,
		rightline=false,
		topline=false,
		bottomline=false,
		linewidth=2pt,
		innertopmargin=5pt,
		innerbottommargin=8pt,
		innerleftmargin=8pt,
		leftmargin=-2pt,
		skipbelow=2pt,
		nobreak
	}
]{thmgreen}
\declaretheorem[style=thmgreen,name=Definition,sibling=thm]{defi}
\declaretheorem[style=thmgreen,name=Definition,sibling=thm]{definition}
\declaretheorem[style=thmgreen,name=Axiom,sibling=thm]{ax}
\newenvironment{axiom}{\begin{ax}}{\end{ax}}

\definecolor{exdarkgreen}{RGB}{120, 180, 120}
\declaretheoremstyle[
	mdframed={
		backgroundcolor=thmlightgreen,
		linecolor=exdarkgreen,
		rightline=false,
		topline=false,
		bottomline=false,
		linewidth=2pt,
		innertopmargin=5pt,
		innerbottommargin=8pt,
		innerleftmargin=8pt,
		leftmargin=-2pt,
		skipbelow=2pt,
		nobreak
	}
]{exgreen}
\declaretheorem[style=exgreen,name=Example,sibling=thm]{ex}
\declaretheorem[style=exgreen,name=Example,sibling=thm]{example}
\declaretheorem[style=exgreen,name=Non-Example,sibling=thm]{nex}
\declaretheorem[style=exgreen,name=Non-Definition,sibling=thm]{ndefi}

\definecolor{remlightbrown}{RGB}{246, 240, 228}
\definecolor{remdarkbrown}{RGB}{180, 133, 75}
\declaretheoremstyle[
	mdframed={
		backgroundcolor=remlightbrown,
		linecolor=remdarkbrown,
		rightline=false,
		topline=false,
		bottomline=false,
		linewidth=2pt,
		innertopmargin=5pt,
		innerbottommargin=8pt,
		innerleftmargin=8pt,
		leftmargin=-2pt,
		skipbelow=2pt,
		nobreak
	}
]{rembrown}
\declaretheorem[style=rembrown,name=Remark,sibling=thm]{remark}
%%%%%%%%%%%%%%%%%%%%%%%%%%%%%%%%%%%%%%%%%%%% /THM BOXES

%%%%%%%%%%%%%%%%%%%%%%%%%%%%%%%%%%%%%%%%%%%% QUIVER
% *** quiver ***
% A package for drawing commutative diagrams exported from https://q.uiver.app.
%
% This package is currently a wrapper around the `tikz-cd` package, importing necessary TikZ
% libraries, and defining a new TikZ style for curves of a fixed height.
%
% Version: 1.2.0
% Authors:
% - varkor (https://github.com/varkor)
% - AndréC (https://tex.stackexchange.com/users/138900/andr%C3%A9c)

% NIR: this is causing errors, so I killed it
% \NeedsTeXFormat{LaTeX2e}
% \ProvidesPackage{quiver}[2021/01/11 quiver]

% `tikz-cd` is necessary to draw commutative diagrams.
\RequirePackage{tikz-cd}
% `amssymb` is necessary for `\lrcorner` and `\ulcorner`.
\RequirePackage{amssymb}
% `calc` is necessary to draw curved arrows.
\usetikzlibrary{calc}
% `pathmorphing` is necessary to draw squiggly arrows.
\usetikzlibrary{decorations.pathmorphing}

% A TikZ style for curved arrows of a fixed height, due to AndréC.
\tikzset{curve/.style={settings={#1},to path={(\tikztostart)
	.. controls ($(\tikztostart)!\pv{pos}!(\tikztotarget)!\pv{height}!270:(\tikztotarget)$)
	and ($(\tikztostart)!1-\pv{pos}!(\tikztotarget)!\pv{height}!270:(\tikztotarget)$)
	.. (\tikztotarget)\tikztonodes}},
	settings/.code={\tikzset{quiver/.cd,#1}
		\def\pv##1{\pgfkeysvalueof{/tikz/quiver/##1}}},
	quiver/.cd,pos/.initial=0.35,height/.initial=0}

% TikZ arrowhead/tail styles.
\tikzset{tail reversed/.code={\pgfsetarrowsstart{tikzcd to}}}
\tikzset{2tail/.code={\pgfsetarrowsstart{Implies[reversed]}}}
\tikzset{2tail reversed/.code={\pgfsetarrowsstart{Implies}}}
% TikZ arrow styles.
\tikzset{no body/.style={/tikz/dash pattern=on 0 off 1mm}}
%%%%%%%%%%%%%%%%%%%%%%%%%%%%%%%%%%%%%%%%%%%% /QUIVER