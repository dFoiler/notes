% !TEX root = ../notes.tex

\documentclass[../notes.tex]{subfiles}

\begin{document}

\section{December 4}
We are almost done with the course, so let's do some bonus stuff.

\subsection{Character Theory}
Let's explain how to compute a $q$-deformation of the dimension.
\begin{defihelper}[$q$-dimension] \nirindex{Q-dimension@$q$-dimension}
	Fix a finite-dimensional representation $V$ of a semisimple Lie algebra $\mf g$. Then we define the \textit{$q$-dimension} by
	\[\dim_qV\coloneqq\tr_V\left(q^{2\rho}\right)=\sum_{\lambda\in P_+}\dim V[\lambda]q^{2(\rho,\lambda)}.\]
\end{defihelper}
\begin{remark}
	This is the image of the character $\chi_V$ under the homomorphism $\CC[P]\to\CC\left[q,q^{-1}\right]$ given by $\lambda\mapsto e^{2(\lambda,\rho)}$.
\end{remark}
\begin{remark}
	Taking $\lambda=1$ recovers the usual dimension.
\end{remark}
Let's use the Weyl character formula.
\begin{theorem}
	Fix a semisimple complex Lie algebra $\mf g$. For $\lambda\in P_+$, we have
	\[\dim_qL_\lambda=\prod_{\alpha\in\Phi_+}\frac{q^{(\lambda+\rho,\alpha)}-q^{-(\lambda+\rho,\alpha)}}{q^{(\rho,\alpha)}-q^{-(\rho,\alpha)}}.\]
\end{theorem}
\begin{proof}
	By the Weyl character formula, we get
	\[\dim_qL_\lambda=\frac{\displaystyle\sum_{w\in W}\varepsilon(w)q^{2(w(\lambda+\rho),\rho)}}{\displaystyle\prod_{\alpha\in\Phi_+}q^{(\alpha,\rho)}-q^{-(\alpha,\rho)}},\]
	so it remains to rearrange the numerator. Well, using the $W$-invariance of the $(\cdot,\cdot)$, we see that the numerator rearranges into
	\[\pi_{\lambda+\rho}\Bigg(\sum_{w\in W}\varepsilon(w)e^{2(\rho)}\Bigg),\]
	where $\pi_{\lambda+\rho}\colon\CC[P]\to\CC\left[q,q^{-1}\right]$ sends $e^\mu\mapsto e^{2(\lambda+\rho,\mu)}$. We are now done by taking $\lambda=0$ in the Weyl character formula.
\end{proof}
For example, we can take $q\to1$ in the above theorem to deduce that
\[\dim L_\lambda=\prod_{\alpha\in\Phi_+}\frac{(\lambda+\rho,\alpha)}{(\rho,\alpha)},\]
where we are notably using the fact that
\[\lim_{q\to1}\frac{q^n-q^{-n}}{q^m-q^{-m}}=\frac nm.\]
While we're here, we note that our characters form a basis of a suitable space, analogous to the situation for finite groups.
\begin{theorem}
	Fix a complex semisimple Lie algebra $\mf g$. Then the characters $\{\chi_\lambda\}_{\lambda\in P_+}$  form a basis of $\CC[P]^W$.
\end{theorem}
\begin{proof}
	To begin, we note that there is a basis of $\CC[P]^W$ given by
	\[m_\lambda\coloneqq\sum_{\mu\in W\lambda}e^\mu,\]
	where $W\lambda$ refers to the $W$-orbit of $\lambda$. Now, the construction of $L_\lambda$ tells us that its highest weights come from $m_\lambda$, so we can write
	\[\chi_\lambda=m_\lambda+\sum_{\mu<\lambda}c_\mu^\lambda m_\mu\]
	for some coefficients $c_\mu^\lambda$. In particular, we see that the change-of-basis matrix from $\{m_\lambda\}$ to $\{\chi_\lambda\}$ is upper-triangular.
\end{proof}
\begin{remark}
	As a corollary, we note that expressing a character $\chi_V$ of a representation $V$ recovers its decomposition into irreducible representations.
\end{remark}

\subsection{Back to \texorpdfstring{$\mf{sl}_n$}{sl n}}
Let's do an example: let's compute the characters of $\mf{sl}_n$. Upon identifying $P$ with $\ZZ^n/\ZZ(1,\ldots,1)$, we can identify $\CC[P]$ with $\CC[x_1,\ldots,x_n]/(x_1\cdots x_n-1)$. The $S_n$-action then permutes the entries of $P$ and hence the $x_\bullet$s in $\CC[P]$.
\begin{proposition}
	The Weyl denominator identity for $\mf{sl}_n$ asserts.
	\[\prod_{i<j}(x_i-x_j)=\sum_{w\in S_n}\op{sgn}(w)x_{w(1)}^{n-1}x_{w(2)}^{n-2}\cdots x_{w(n)}^0.\]
\end{proposition}
\begin{proof}
	The Weyl denominator identity takes $\lambda=0$ in the Weyl character formula to find
	\[e^\rho\prod_{\alpha\in\Phi_+}\left(1-e^{-\alpha}\right)=\sum_{w\in W}\varepsilon(w)e^{w\rho}.\]
	By summing the positive roots, we find that $\rho=(n-1,n-2,\ldots,1,0)$. Plugging everything in completes the proof.
\end{proof}
\begin{remark}
	Notably, the right-hand side is the Vandermonde determinant
	\[\det\begin{bmatrix}
		x_1^{n-1} & \cdots & 1 \\
		\vdots & \ddots & \vdots \\
		x_n^{n-1} & \cdots & 1
	\end{bmatrix}.\]
	Thus, we have recovered the evaluation of the Vandermonde determinant.
\end{remark}
A similar computation is now able to show that $\lambda=(\lambda_1,\ldots,\lambda_n)$ has
\[\chi_\lambda=\frac{\det\begin{bmatrix}
	x_1^{\lambda_1+n-1} & \cdots & x_1^{\lambda_n} \\
	\vdots & \ddots & \vdots \\
	x_n^{\lambda_1+n-1} & \cdots & x_n^{\lambda_n}
\end{bmatrix}}{\det\begin{bmatrix}
	x_1^{n-1} & \cdots & x_1^{0} \\
	\vdots & \ddots & \vdots \\
	x_n^{n-1} & \cdots & x_n^{0}
\end{bmatrix}}\]
Note that it is not even totally obvious that this is a symmetric polynomial at all, but indeed it does, and we even know that these characters make a basis of the symmetric polynomials. These are known as the Schur polynomials, and they are denoted $s_\lambda$.
\begin{example}
	Take $\lambda=(m,0,\ldots,0)$. Then $\chi_\lambda$ is
	\[\sum_{\substack{1\le i_1,\ldots,i_m\le m\\i_1+\cdots+i_m=m}}x_1^{i_1}\cdots x_m^{i_m},\]
	which we see immediately allows us to read off the weights of $\op{Sym}^mV$.
\end{example}
\begin{example}
	Take $\lambda=(1,\ldots,1,0,\ldots,0)$ with $m$ ones. Then $s_\lambda$ is the $m$th elementary symmetric polynomial
	\[\sum_{\substack{1\le i_1,\ldots,i_m\le m\\i_1+\cdots+i_m=m\\i_1,\ldots,i_m\text{ distinct}}}x_1^{i_1}\cdots x_m^{i_m}.\]
\end{example}
\begin{remark}
	One knows that there are integers $c_{\lambda\mu}^\nu$ such that
	\[L_\lambda\otimes L_\mu=\bigoplus_{\nu\in P_+}c_{\lambda\mu}^\nu,\]
	which corresponds to an equality
	\[s_\lambda s_\mu=\sum_{\nu\in P_+}c_{\lambda\mu}^\nu s_\nu.\]
	The coefficients $c_{\lambda\mu}^\nu$ are called Littlewood--Richardson coefficients.
\end{remark}

\subsection{The Harish-Chandra Isomorphism}
As usual, let $\mf g$ be a semisimple Lie algebra. We recall that $\op{sym}$ defines an isomorphism
\[(S\mf g)^G\to Z(U\mf g),\]
where $G$ is the connected simply-connected Lie group with Lie algebra $\mf g$. We would like a better understanding of the center $Z\mf g\coloneqq Z(U\mf g)$ because they correspond to intertwining operators on the representations of $\mf g$; for example, in the theory of finite-dimensional representations, we found great utility out of merely the Casimir element. It is believable that perhaps having more elements in the center could have other applications.

For each $\lambda\in P_+$, we note that $Z\mf g$ acts by scalars on $L_\lambda$, so we get a character $\chi_\lambda\colon Z\mf g\to\CC$. However, we note that this central character only depends on the monomials in $U\mf h$, so there is a projection
\[U\mf g\to U\mf h.\]
In this way, we get an algebra homomorphism $Z\mf g\to S\mf h$. It is a theorem that this restricts to an isomorphism $Z\mf g\to(S\mf h)^W$, where the $W$-action is given by $w\cdot f(\lambda)\coloneqq f\left(w^{-1}(\lambda+\rho)-\rho\right)$.
\begin{example}
	One can use this isomorphism to compute $Z\mf{sl}_2$, recovering that it is generated by the Casimir element. There is also a way to construct ``higher Casimir'' elements in $Z\mf{sl}_n$, which is computed to be the symmetric polynomials in $\CC[x_1,\ldots,x_n]$ moduli $e_1=x_1+\cdots+x_n$.
\end{example}
\begin{remark}
	In general, one expects $\CC[\mf h^\lor]^W$ to be a polynomial algebra.
\end{remark}

\end{document}