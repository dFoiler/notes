% !TEX root = ../notes.tex

\documentclass[../notes.tex]{subfiles}

\begin{document}

\section{November 8}
Today we continue talking about abstract root systems.

\subsection{Positive and Simple Roots}
For our discussion, fix a reduced root system $\Phi\subseteq E$. We now fix some $t\in E$ which is not orthogonal to any element in $\Phi$, which is possible because these orthogonality lines are only cutting out some positive codimension subspace of $E$. A choice of $t$ allows us to sign the roots.
\begin{definition}[positive]
	Fix a root system $\Phi\subseteq E$. Given $t\in E$ which is not orthogonal to any element in $\Phi$, we say that a root $\alpha\in\Phi$ is \textit{positive} if and only if $(t,\alpha)>0$ and \textit{negative} if and only if $(t,\alpha)<0$. The set of positive roots is denoted $\Phi_+$, and the set of negative roots is denoted $\Phi_-$. Note that we are omitting the choice of $t$ from our language and our notation because it will never change.
\end{definition}
\begin{example}
	Consider $\Phi$ of type $A_{n-1}$. Then note that $(t,e_i-e_j)=0$ if and only if $t_i-t_j=0$, so we may choose $t$ so that $t_1>\cdots>t_n$. Then one finds that $e_i-e_j\in\Phi_+$ if and only if $i<j$ by a direct computation:
	\[(t,e_i-e_j)=t_i-t_j.\]
	By adjusting the ordering of the entries in $t$, one can see that the choice of polarizations of $\Phi$ are in bijection with $S_n$.
\end{example}
Having a notion of positive allows us to find the smallest roots.
\begin{definition}[simple]
	Fix a reduced root system $\Phi\subseteq E$. A positive root $\alpha$ is \textit{simple} if and only if it cannot be written as the sum of positive roots.
\end{definition}
\begin{remark}
	Note that every positive root $\alpha$ can be written as the finite sum of simple roots. This is by induction on $(t,\alpha)>0$. If $\alpha$ is already simple, then there is nothing to do. Otherwise, we can write $\alpha=\beta+\gamma$ where $\beta$ and $\gamma$ are both positive, and then we see that $0<(t,\beta),(t,\gamma)<(t,\alpha)$, so we are done by the inductive hypothesis.
\end{remark}
Simple roots will form the core of the entire root system, as we will soon see.
\begin{lemma}
	Fix a reduced root system $\Phi\subseteq E$. If $\alpha$ and $\beta$ are distinct simple roots, then $(\alpha,\beta)\le0$.
\end{lemma}
\begin{proof}
	Suppose for the sake of contradiction that we had $(\alpha,\beta)>0$. Then $(\alpha,-\beta)<0$, so $\alpha-\beta\in\Phi$. If this is a positive root, then $\alpha=(\alpha-\beta)+\beta$, so $\alpha$ fails to be simple; if this is a negative root, then $\beta=(\beta-\alpha)+\alpha$, so $\beta$ fails to be simple. In all cases, we derive a contradiction.
\end{proof}
\begin{theorem}
	Fix a reduced root system $\Phi\subseteq E$, and let $\Delta$ be the collection of simple roots. Then $\Delta$ is a basis of $\Phi$.
\end{theorem}
\begin{proof}
	We already know that $\Delta$ spans positive roots, so it spans all roots by taking signs, so it spans $E$ because $E=\op{span}\Phi$. It remains to show that $\Delta$ is linearly independent. For brevity, set $\Delta=\{\alpha_1,\ldots,\alpha_m\}$, and suppose that we have some relation
	\[\sum_{i\in I}c_i\alpha_i=\sum_{j\in J}c_j\alpha_j,\]
	where each $c_\bullet$ is positive and $I\cap J=\emp$. Now, because our inner product is positive-definite, we see that
	\[0<\Bigg(\sum_{i\in I}c_i\alpha_i,\sum_{i\in I}c_i\alpha_i\Bigg)=\Bigg(\sum_{i\in I}c_i\alpha_i,\sum_{j\in J}c_j\alpha_j\Bigg)=\sum_{i,j}c_ic_j(\alpha_i,\alpha_j)\le0,\]
	where we have used the lemma in the last inequality.
\end{proof}
\begin{example}
	Consider $\Phi$ of type $A_{n-1}$. Then one can show that the collection
	\[\{e_i-e_{i+1}:1\le i<n\}\]
	is a collection of simple roots. We can see that this is a basis of $E$, which is the subspace of $\QQ^n$ cut out by the coordinates summing to $0$.
\end{example}
\begin{remark}
	Note that it follows that any root in $\Phi$ can be written uniquely as an integer linear combination of simple roots in $\Delta$. Further, the coefficients are all nonnegative if the root is positive, and the coefficients are nonpositive if the root is negative.
\end{remark}
The previous remark permits the following definition.
\begin{definition}[height]
	Fix a reduced root system $\Phi\subseteq E$, and let $\Delta$ be a collection of simple roots. Then the \textit{height} of a positive root $\alpha\in\Phi$ is the sum $\sum_{\beta\in\Delta}n_\beta$, where the $n_\beta$s satisfy
	\[\alpha=\sum_{\beta\in\Delta}n_\beta\beta.\]
\end{definition}
\begin{example}
	For any positive root $e_i-e_j$ where $i<j$, we see that
	\[e_i-e_j=\sum_{k=i}^{j-1}(e_k-e_{k+1}),\]
	so the height of this root is $j-i$.
\end{example}
As an aside, we note that there is a dual notion for everything we have done so far.
\begin{definition}
	Fix a root system $\Phi\subseteq E$. Then we define the \textit{dual root system} $\Phi^\lor\subseteq E^\lor$ to be
	\[\left\{\alpha^\lor:\alpha\in\Phi\right\}.\]
\end{definition}
\begin{remark}
	One can check that $\Phi^\lor$ is a root system when $\Phi$ is, and one can check that $\Phi^\lor$ is reduced when $\Phi$ is. Note that a choice of $t\in E$ produces a dual $t^\lor\in E^\lor$ given by the inner product, and one can check that positive (and simple) roots of $\Phi$ correspond to positive (and simple) roots of $\Phi^\lor$.
\end{remark}
\begin{remark}
	Now that everything is proven to be integral, we note that we could more or less just work with the following lattices.
	\begin{itemize}
		\item There is a root lattice $\ZZ[\Phi]\subseteq E$.
		\item There is a coroot lattice $\ZZ[\Phi^\lor]\subseteq E^\lor$.
		\item There is a weight lattice
		\[\{\lambda\in E:(\lambda,\alpha^\lor)\in\ZZ\text{ for all }\alpha^\lor\in\Phi^\lor\}.\]
		\item There is a coweight lattice
		\[\{\lambda^\lor\in E^\lor:\lambda^\lor(\alpha)\in\ZZ\text{ for all }\alpha\in\Phi\}.\]
	\end{itemize}
\end{remark}

\end{document}