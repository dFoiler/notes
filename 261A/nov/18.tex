% !TEX root = ../notes.tex

\documentclass[../notes.tex]{subfiles}

\begin{document}

\section{November 18}
Today we continue talking about Cartan matrices.

\subsection{Dynkin Diagrams}
We now use Cartan matrices to define Dynkin diagrams. We take our $r$ simple roots $\{\alpha_1,\ldots,\alpha_r\}$, and we note that
\[\frac{(\alpha_i,\alpha_j)^2}{(\alpha_i,\alpha_i)(\alpha_j,\alpha_j)}\]
equals $4\cos\varphi$ where $\varphi$ is the angle between $\alpha_i$ and $\alpha_j$. In particular, we see that this must live in $\{0,1,2,3\}$.

Now, to define our Dynkin diagram, we begin with $r$ vertices labeled $\alpha_1,\ldots,\alpha_r$. Then we connect $\alpha_i$ and $\alpha_j$ by $4\cos\varphi$ edges, and we put an arrow pointing to the shorter root when this value is not $1$. Here are some examples.
\begin{example}
	Here is the Dynkin diagram for $A_1\sqcup A_1$.
	% https://q.uiver.app/#q=WzAsMixbMCwwLCJcXGJ1bGxldCJdLFsxLDAsIlxcYnVsbGV0Il1d&macro_url=https%3A%2F%2Fraw.githubusercontent.com%2FdFoiler%2Fnotes%2Fmaster%2Fnir.tex
	\[\begin{tikzcd}
		\bullet & \bullet
	\end{tikzcd}\]
\end{example}
\begin{example}
	Here is the Dynkin diagram for $A_2$.
	% https://q.uiver.app/#q=WzAsMixbMCwwLCJcXGJ1bGxldCJdLFsxLDAsIlxcYnVsbGV0Il0sWzAsMSwiIiwwLHsic3R5bGUiOnsiaGVhZCI6eyJuYW1lIjoibm9uZSJ9fX1dXQ==&macro_url=https%3A%2F%2Fraw.githubusercontent.com%2FdFoiler%2Fnotes%2Fmaster%2Fnir.tex
	\[\begin{tikzcd}
		\bullet & \bullet
		\arrow[no head, from=1-1, to=1-2]
	\end{tikzcd}\]
\end{example}
\begin{example}
	Here is the Dynkin diagram for $B_2$.
	% https://q.uiver.app/#q=WzAsMixbMCwwLCJcXGJ1bGxldCJdLFsxLDAsIlxcYnVsbGV0Il0sWzAsMSwiIiwwLHsibGV2ZWwiOjJ9XV0=&macro_url=https%3A%2F%2Fraw.githubusercontent.com%2FdFoiler%2Fnotes%2Fmaster%2Fnir.tex
	\[\begin{tikzcd}
		\bullet & \bullet
		\arrow[Rightarrow, from=1-1, to=1-2]
	\end{tikzcd}\]
\end{example}
\begin{example}
	Here is the Dynkin diagram for $G_2$.
	% https://q.uiver.app/#q=WzAsMixbMCwwLCJcXGJ1bGxldCJdLFsxLDAsIlxcYnVsbGV0Il0sWzAsMSwiIiwwLHsibGV2ZWwiOjN9XV0=&macro_url=https%3A%2F%2Fraw.githubusercontent.com%2FdFoiler%2Fnotes%2Fmaster%2Fnir.tex
	% https://q.uiver.app/#q=WzAsMixbMCwwLCJcXGJ1bGxldCJdLFsxLDAsIlxcYnVsbGV0Il0sWzAsMSwiIiwwLHsibGV2ZWwiOjJ9XSxbMCwxLCIiLDIseyJzdHlsZSI6eyJoZWFkIjp7Im5hbWUiOiJub25lIn19fV1d&macro_url=https%3A%2F%2Fraw.githubusercontent.com%2FdFoiler%2Fnotes%2Fmaster%2Fnir.tex
	\[\begin{tikzcd}
		\bullet & \bullet
		\arrow[Rightarrow, from=1-1, to=1-2]
		\arrow[no head, from=1-1, to=1-2]
	\end{tikzcd}\]
\end{example}
The point is that the Dynkin diagram encodes all the information about the Cartan matrix, so it also classifies our root system. This graph is just some piece of combinatorics, so a significant amount of effort allows us to classify them. There are four infinite families, as follows.
\begin{itemize}
	\item Type $A_n$.
	% https://q.uiver.app/#q=WzAsNCxbMCwwLCJcXGJ1bGxldCJdLFsxLDAsIlxcYnVsbGV0Il0sWzIsMCwiXFxjZG90cyJdLFszLDAsIlxcYnVsbGV0Il0sWzAsMSwiIiwwLHsic3R5bGUiOnsiaGVhZCI6eyJuYW1lIjoibm9uZSJ9fX1dLFsxLDIsIiIsMCx7InN0eWxlIjp7ImhlYWQiOnsibmFtZSI6Im5vbmUifX19XSxbMiwzLCIiLDAseyJzdHlsZSI6eyJoZWFkIjp7Im5hbWUiOiJub25lIn19fV1d&macro_url=https%3A%2F%2Fraw.githubusercontent.com%2FdFoiler%2Fnotes%2Fmaster%2Fnir.tex
	\[\begin{tikzcd}
		\bullet & \bullet & \cdots & \bullet
		\arrow[no head, from=1-1, to=1-2]
		\arrow[no head, from=1-2, to=1-3]
		\arrow[no head, from=1-3, to=1-4]
	\end{tikzcd}\]
	\item Type $B_n$.
	% https://q.uiver.app/#q=WzAsNSxbMCwwLCJcXGJ1bGxldCJdLFsxLDAsIlxcYnVsbGV0Il0sWzIsMCwiXFxjZG90cyJdLFszLDAsIlxcYnVsbGV0Il0sWzQsMCwiXFxidWxsZXQiXSxbMCwxLCIiLDAseyJzdHlsZSI6eyJoZWFkIjp7Im5hbWUiOiJub25lIn19fV0sWzEsMiwiIiwwLHsic3R5bGUiOnsiaGVhZCI6eyJuYW1lIjoibm9uZSJ9fX1dLFsyLDMsIiIsMCx7InN0eWxlIjp7ImhlYWQiOnsibmFtZSI6Im5vbmUifX19XSxbMyw0LCIiLDAseyJsZXZlbCI6Mn1dXQ==&macro_url=https%3A%2F%2Fraw.githubusercontent.com%2FdFoiler%2Fnotes%2Fmaster%2Fnir.tex
	\[\begin{tikzcd}
		\bullet & \bullet & \cdots & \bullet & \bullet
		\arrow[no head, from=1-1, to=1-2]
		\arrow[no head, from=1-2, to=1-3]
		\arrow[no head, from=1-3, to=1-4]
		\arrow[Rightarrow, from=1-4, to=1-5]
	\end{tikzcd}\]
	The root at the end is $e_n$.
	\item Type $C_n$.
	% https://q.uiver.app/#q=WzAsNSxbMCwwLCJcXGJ1bGxldCJdLFsxLDAsIlxcYnVsbGV0Il0sWzIsMCwiXFxjZG90cyJdLFszLDAsIlxcYnVsbGV0Il0sWzQsMCwiXFxidWxsZXQiXSxbMCwxLCIiLDAseyJzdHlsZSI6eyJoZWFkIjp7Im5hbWUiOiJub25lIn19fV0sWzEsMiwiIiwwLHsic3R5bGUiOnsiaGVhZCI6eyJuYW1lIjoibm9uZSJ9fX1dLFsyLDMsIiIsMCx7InN0eWxlIjp7ImhlYWQiOnsibmFtZSI6Im5vbmUifX19XSxbNCwzLCIiLDIseyJsZXZlbCI6Mn1dXQ==&macro_url=https%3A%2F%2Fraw.githubusercontent.com%2FdFoiler%2Fnotes%2Fmaster%2Fnir.tex
	\[\begin{tikzcd}
		\bullet & \bullet & \cdots & \bullet & \bullet
		\arrow[no head, from=1-1, to=1-2]
		\arrow[no head, from=1-2, to=1-3]
		\arrow[no head, from=1-3, to=1-4]
		\arrow[Rightarrow, from=1-5, to=1-4]
	\end{tikzcd}\]
	The root at the end is $2e_n$.
	\item Type $D_n$.
	% https://q.uiver.app/#q=WzAsNixbMCwxLCJcXGJ1bGxldCJdLFsxLDEsIlxcYnVsbGV0Il0sWzIsMSwiXFxjZG90cyJdLFszLDEsIlxcYnVsbGV0Il0sWzQsMCwiXFxidWxsZXQiXSxbNCwyLCJcXGJ1bGxldCJdLFswLDEsIiIsMCx7InN0eWxlIjp7ImhlYWQiOnsibmFtZSI6Im5vbmUifX19XSxbMSwyLCIiLDAseyJzdHlsZSI6eyJoZWFkIjp7Im5hbWUiOiJub25lIn19fV0sWzIsMywiIiwwLHsic3R5bGUiOnsiaGVhZCI6eyJuYW1lIjoibm9uZSJ9fX1dLFszLDQsIiIsMCx7InN0eWxlIjp7ImhlYWQiOnsibmFtZSI6Im5vbmUifX19XSxbMyw1LCIiLDAseyJzdHlsZSI6eyJoZWFkIjp7Im5hbWUiOiJub25lIn19fV1d&macro_url=https%3A%2F%2Fraw.githubusercontent.com%2FdFoiler%2Fnotes%2Fmaster%2Fnir.tex
	\[\begin{tikzcd}
		&&&& \bullet \\
		\bullet & \bullet & \cdots & \bullet \\
		&&&& \bullet
		\arrow[no head, from=2-1, to=2-2]
		\arrow[no head, from=2-2, to=2-3]
		\arrow[no head, from=2-3, to=2-4]
		\arrow[no head, from=2-4, to=1-5]
		\arrow[no head, from=2-4, to=3-5]
	\end{tikzcd}\]
	\item Type $E_6$.
	% https://q.uiver.app/#q=WzAsNixbMCwxLCJcXGJ1bGxldCJdLFsxLDEsIlxcYnVsbGV0Il0sWzIsMSwiXFxidWxsZXQiXSxbNCwxLCJcXGJ1bGxldCJdLFszLDEsIlxcYnVsbGV0Il0sWzIsMCwiXFxidWxsZXQiXSxbMCwxLCIiLDAseyJzdHlsZSI6eyJoZWFkIjp7Im5hbWUiOiJub25lIn19fV0sWzEsMiwiIiwwLHsic3R5bGUiOnsiaGVhZCI6eyJuYW1lIjoibm9uZSJ9fX1dLFsyLDQsIiIsMCx7InN0eWxlIjp7ImhlYWQiOnsibmFtZSI6Im5vbmUifX19XSxbNCwzLCIiLDAseyJzdHlsZSI6eyJoZWFkIjp7Im5hbWUiOiJub25lIn19fV0sWzUsMiwiIiwwLHsic3R5bGUiOnsiaGVhZCI6eyJuYW1lIjoibm9uZSJ9fX1dXQ==&macro_url=https%3A%2F%2Fraw.githubusercontent.com%2FdFoiler%2Fnotes%2Fmaster%2Fnir.tex
	\[\begin{tikzcd}
		&& \bullet \\
		\bullet & \bullet & \bullet & \bullet & \bullet
		\arrow[no head, from=1-3, to=2-3]
		\arrow[no head, from=2-1, to=2-2]
		\arrow[no head, from=2-2, to=2-3]
		\arrow[no head, from=2-3, to=2-4]
		\arrow[no head, from=2-4, to=2-5]
	\end{tikzcd}\]
	\item Type $E_7$.
	% https://q.uiver.app/#q=WzAsNyxbMCwxLCJcXGJ1bGxldCJdLFsxLDEsIlxcYnVsbGV0Il0sWzIsMSwiXFxidWxsZXQiXSxbNCwxLCJcXGJ1bGxldCJdLFszLDEsIlxcYnVsbGV0Il0sWzIsMCwiXFxidWxsZXQiXSxbNSwxLCJcXGJ1bGxldCJdLFswLDEsIiIsMCx7InN0eWxlIjp7ImhlYWQiOnsibmFtZSI6Im5vbmUifX19XSxbMSwyLCIiLDAseyJzdHlsZSI6eyJoZWFkIjp7Im5hbWUiOiJub25lIn19fV0sWzIsNCwiIiwwLHsic3R5bGUiOnsiaGVhZCI6eyJuYW1lIjoibm9uZSJ9fX1dLFs0LDMsIiIsMCx7InN0eWxlIjp7ImhlYWQiOnsibmFtZSI6Im5vbmUifX19XSxbNSwyLCIiLDAseyJzdHlsZSI6eyJoZWFkIjp7Im5hbWUiOiJub25lIn19fV0sWzMsNiwiIiwwLHsic3R5bGUiOnsiaGVhZCI6eyJuYW1lIjoibm9uZSJ9fX1dXQ==&macro_url=https%3A%2F%2Fraw.githubusercontent.com%2FdFoiler%2Fnotes%2Fmaster%2Fnir.tex
	\[\begin{tikzcd}
		&& \bullet \\
		\bullet & \bullet & \bullet & \bullet & \bullet & \bullet
		\arrow[no head, from=1-3, to=2-3]
		\arrow[no head, from=2-1, to=2-2]
		\arrow[no head, from=2-2, to=2-3]
		\arrow[no head, from=2-3, to=2-4]
		\arrow[no head, from=2-4, to=2-5]
		\arrow[no head, from=2-5, to=2-6]
	\end{tikzcd}\]
	\item Type $E_*$.
	% https://q.uiver.app/#q=WzAsOCxbMCwxLCJcXGJ1bGxldCJdLFsxLDEsIlxcYnVsbGV0Il0sWzIsMSwiXFxidWxsZXQiXSxbNCwxLCJcXGJ1bGxldCJdLFszLDEsIlxcYnVsbGV0Il0sWzIsMCwiXFxidWxsZXQiXSxbNSwxLCJcXGJ1bGxldCJdLFs2LDEsIlxcYnVsbGV0Il0sWzAsMSwiIiwwLHsic3R5bGUiOnsiaGVhZCI6eyJuYW1lIjoibm9uZSJ9fX1dLFsxLDIsIiIsMCx7InN0eWxlIjp7ImhlYWQiOnsibmFtZSI6Im5vbmUifX19XSxbMiw0LCIiLDAseyJzdHlsZSI6eyJoZWFkIjp7Im5hbWUiOiJub25lIn19fV0sWzQsMywiIiwwLHsic3R5bGUiOnsiaGVhZCI6eyJuYW1lIjoibm9uZSJ9fX1dLFs1LDIsIiIsMCx7InN0eWxlIjp7ImhlYWQiOnsibmFtZSI6Im5vbmUifX19XSxbMyw2LCIiLDAseyJzdHlsZSI6eyJoZWFkIjp7Im5hbWUiOiJub25lIn19fV0sWzYsNywiIiwwLHsic3R5bGUiOnsiaGVhZCI6eyJuYW1lIjoibm9uZSJ9fX1dXQ==&macro_url=https%3A%2F%2Fraw.githubusercontent.com%2FdFoiler%2Fnotes%2Fmaster%2Fnir.tex
	\[\begin{tikzcd}
		&& \bullet \\
		\bullet & \bullet & \bullet & \bullet & \bullet & \bullet & \bullet
		\arrow[no head, from=1-3, to=2-3]
		\arrow[no head, from=2-1, to=2-2]
		\arrow[no head, from=2-2, to=2-3]
		\arrow[no head, from=2-3, to=2-4]
		\arrow[no head, from=2-4, to=2-5]
		\arrow[no head, from=2-5, to=2-6]
		\arrow[no head, from=2-6, to=2-7]
	\end{tikzcd}\]
	\item Type $F_4$.
	% https://q.uiver.app/#q=WzAsNCxbMCwwLCJcXGJ1bGxldCJdLFsxLDAsIlxcYnVsbGV0Il0sWzIsMCwiXFxidWxsZXQiXSxbMywwLCJcXGJ1bGxldCJdLFswLDEsIiIsMCx7InN0eWxlIjp7ImhlYWQiOnsibmFtZSI6Im5vbmUifX19XSxbMSwyLCIiLDAseyJsZXZlbCI6Mn1dLFsyLDMsIiIsMCx7InN0eWxlIjp7ImhlYWQiOnsibmFtZSI6Im5vbmUifX19XV0=&macro_url=https%3A%2F%2Fraw.githubusercontent.com%2FdFoiler%2Fnotes%2Fmaster%2Fnir.tex
	\[\begin{tikzcd}
		\bullet & \bullet & \bullet & \bullet
		\arrow[no head, from=1-1, to=1-2]
		\arrow[Rightarrow, from=1-2, to=1-3]
		\arrow[no head, from=1-3, to=1-4]
	\end{tikzcd}\]
	\item Type $G_2$.
	% https://q.uiver.app/#q=WzAsMixbMCwwLCJcXGJ1bGxldCJdLFsxLDAsIlxcYnVsbGV0Il0sWzAsMSwiIiwwLHsibGV2ZWwiOjJ9XSxbMCwxLCIiLDIseyJzdHlsZSI6eyJoZWFkIjp7Im5hbWUiOiJub25lIn19fV1d&macro_url=https%3A%2F%2Fraw.githubusercontent.com%2FdFoiler%2Fnotes%2Fmaster%2Fnir.tex
	\[\begin{tikzcd}
		\bullet & \bullet
		\arrow[Rightarrow, from=1-1, to=1-2]
		\arrow[no head, from=1-1, to=1-2]
	\end{tikzcd}\]
\end{itemize}
Let's sketch some part of the proof of the classification.
\begin{enumerate}
	\item There are no cycles. If there was a cycle consisting of the roots $\alpha_1,\ldots,\alpha_m$, then one can compute that the vector
	\[\alpha=\alpha_1+\cdots+\alpha_m\]
	has $(\alpha,\alpha)=0$, but this is impossible because $(\cdot,\cdot)$ is positive-definite.
	\item There are no vertices of degree at least $4$. Indeed, suppose $\beta$ is adjacent to $\{\alpha_1,\alpha_2,\alpha_3,\alpha_4\}$. Then one finds that
	\[\gamma\coloneqq2\beta+(\alpha_1+\alpha_2+\alpha_3+\alpha_4)\]
	has $(\gamma,\gamma)=0$, which is the desired contradiction.
	\item There is at most one vertex of degree $3$. Otherwise, we have a subgraph which looks like the following.
	% https://q.uiver.app/#q=WzAsOCxbMCwwLCJcXGJldGFfMSJdLFswLDIsIlxcYmV0YV80Il0sWzEsMSwiXFxhbHBoYV8xIl0sWzIsMSwiXFxhbHBoYV8yIl0sWzMsMSwiXFxjZG90cyJdLFs0LDEsIlxcYWxwaGFfbSJdLFs1LDAsIlxcYmV0YV8yIl0sWzUsMiwiXFxiZXRhXzMiXSxbMCwyLCIiLDAseyJzdHlsZSI6eyJoZWFkIjp7Im5hbWUiOiJub25lIn19fV0sWzEsMiwiIiwyLHsic3R5bGUiOnsiaGVhZCI6eyJuYW1lIjoibm9uZSJ9fX1dLFsyLDMsIiIsMix7InN0eWxlIjp7ImhlYWQiOnsibmFtZSI6Im5vbmUifX19XSxbMyw0LCIiLDIseyJzdHlsZSI6eyJoZWFkIjp7Im5hbWUiOiJub25lIn19fV0sWzQsNSwiIiwyLHsic3R5bGUiOnsiaGVhZCI6eyJuYW1lIjoibm9uZSJ9fX1dLFs1LDYsIiIsMix7InN0eWxlIjp7ImhlYWQiOnsibmFtZSI6Im5vbmUifX19XSxbNSw3LCIiLDIseyJzdHlsZSI6eyJoZWFkIjp7Im5hbWUiOiJub25lIn19fV1d&macro_url=https%3A%2F%2Fraw.githubusercontent.com%2FdFoiler%2Fnotes%2Fmaster%2Fnir.tex
	\[\begin{tikzcd}
		{\beta_1} &&&&& {\beta_2} \\
		& {\alpha_1} & {\alpha_2} & \cdots & {\alpha_m} \\
		{\beta_4} &&&&& {\beta_3}
		\arrow[no head, from=1-1, to=2-2]
		\arrow[no head, from=2-2, to=2-3]
		\arrow[no head, from=2-3, to=2-4]
		\arrow[no head, from=2-4, to=2-5]
		\arrow[no head, from=2-5, to=1-6]
		\arrow[no head, from=2-5, to=3-6]
		\arrow[no head, from=3-1, to=2-2]
	\end{tikzcd}\]
	This time, we find that $\gamma=2(\alpha_1+\cdots+\alpha_m)+(\beta_1+\cdots+\beta_4)$ has $(\gamma,\gamma)=0$.
	\item We are now able to say that our Dynkin diagram looks like the following.
	% https://q.uiver.app/#q=WzAsMTAsWzAsMywiXFxkZWx0YV8xIl0sWzEsMywiXFxjZG90cyJdLFsyLDMsIlxcZGVsdGFfe3ItMX0iXSxbMywzLCJcXGFscGhhIl0sWzMsMiwiXFxiZXRhX3twLTF9Il0sWzMsMSwiXFx2ZG90cyJdLFszLDAsIlxcYmV0YV8xIl0sWzQsMywiXFxnYW1tYV97cS0xfSJdLFs1LDMsIlxcY2RvdHMiXSxbNiwzLCJcXGdhbW1hXzEiXSxbNiw1LCIiLDAseyJzdHlsZSI6eyJoZWFkIjp7Im5hbWUiOiJub25lIn19fV0sWzUsNCwiIiwwLHsic3R5bGUiOnsiaGVhZCI6eyJuYW1lIjoibm9uZSJ9fX1dLFs0LDMsIiIsMCx7InN0eWxlIjp7ImhlYWQiOnsibmFtZSI6Im5vbmUifX19XSxbMCwxLCIiLDAseyJzdHlsZSI6eyJoZWFkIjp7Im5hbWUiOiJub25lIn19fV0sWzEsMiwiIiwwLHsic3R5bGUiOnsiaGVhZCI6eyJuYW1lIjoibm9uZSJ9fX1dLFsyLDMsIiIsMSx7InN0eWxlIjp7ImhlYWQiOnsibmFtZSI6Im5vbmUifX19XSxbOSw4LCIiLDEseyJzdHlsZSI6eyJoZWFkIjp7Im5hbWUiOiJub25lIn19fV0sWzgsNywiIiwxLHsic3R5bGUiOnsiaGVhZCI6eyJuYW1lIjoibm9uZSJ9fX1dLFs3LDMsIiIsMSx7InN0eWxlIjp7ImhlYWQiOnsibmFtZSI6Im5vbmUifX19XV0=&macro_url=https%3A%2F%2Fraw.githubusercontent.com%2FdFoiler%2Fnotes%2Fmaster%2Fnir.tex
	\[\begin{tikzcd}
		&&& {\beta_1} \\
		&&& \vdots \\
		&&& {\beta_{p-1}} \\
		{\delta_1} & \cdots & {\delta_{r-1}} & \alpha & {\gamma_{q-1}} & \cdots & {\gamma_1}
		\arrow[no head, from=1-4, to=2-4]
		\arrow[no head, from=2-4, to=3-4]
		\arrow[no head, from=3-4, to=4-4]
		\arrow[no head, from=4-1, to=4-2]
		\arrow[no head, from=4-2, to=4-3]
		\arrow[no head, from=4-3, to=4-4]
		\arrow[no head, from=4-5, to=4-4]
		\arrow[no head, from=4-6, to=4-5]
		\arrow[no head, from=4-7, to=4-6]
	\end{tikzcd}\]
	We now define $\beta\coloneqq\sum_ii\beta_i$ and $\gamma\coloneqq\sum_i\gamma_i$ and $\delta\coloneqq\sum_i\delta_i$, and we find that
	\[\left(\alpha,\frac{\beta}{\left|\beta\right|}\right)^2+\left(\alpha,\frac{\gamma}{\left|\gamma\right|}\right)^2+\left(\alpha,\frac{\delta}{\left|\delta\right|}\right)^2<(\alpha,\alpha)\]
	by the ambient geometry. Now, a direct computation shows that $(\alpha,\beta)=-p+1$ and $(\beta,\beta)=p(p-1)$, and we have something similar for the remaining. Thus, we have the inequality
	\[\frac{(p-1)^2}{p(p-1)}+\frac{(q-1)^2}{q(q-1)}+\frac{(r-1)^2}{r(r-1)}<2,\]
	which rearranges into
	\[\frac1p+\frac1q+\frac1r>1.\]
	If any of these vanish, we get type $A_n$. If we have $(2,2,n)$, then we get type $D_{n+2}$. The remaining solutions are $(2,3,3)$ or $(2,3,4)$ or $(2,3,5)$, which are types $E_6$ or $E_7$ or $E_8$ respectively.
	
	Let's give an argument that this classifies the solutions $(p,q,r)$: certainly one of the variables must be $2$ because $\frac13+\frac13+\frac13=1$, so say $p=2$. Then we are solving $\frac1q+\frac1r>\frac12$. One of $\{q,r\}$ must be $2$ or $3$ because $\frac14+\frac14=\frac12$, so say we have $q\in\{2,3\}$. If $q=2$, then $r$ can be anything. If $q=3$, then we see that $r\in\{3,4,5\}$, as required.
\end{enumerate}

\subsection{Serre Relations}
We would like to take a root system and then produce a Lie algebra from a root system.
\begin{theorem}
	Fix an algebraically closed field $F$ of characteristic $0$. Let $\mf h\subseteq\mf g$ be a Cartan subalgebra of a semisimple Lie algebra $\mf g$ over $F$, and let $\Phi\subseteq\mf h^\lor$ be the root system. Further, let $(\cdot,\cdot)$ be a non-degenerate invariant symmetric bilinear form on $\mf g$. Fix a polarization $\Phi=\Phi_+\sqcup\Phi_-$, and let $\Delta=\{\alpha_1,\ldots,\alpha_r\}$ be the simple roots.
	\begin{listalph}
		\item The subspaces $\mf n_\pm=\bigoplus_{\alpha\in\Phi_\pm}\mf g_\alpha$ are subalgebras, and we have $\mf g=\mf h\oplus\mf n_+\oplus\mf n_+$ (as vector spaces).
		\item Choose $e_i\in\mf g_{\alpha_i}$ and $f_i\in\mf g_{\alpha_i}$ such that $(e_i,f_i)=2/(\alpha_i,\alpha_i)$, and set $h_i\coloneqq[e_i,f_i]$ for each $i$. Then $\{e_1,\ldots,e_r\}$ generate $\mf n_+$ and $\{f_1,\ldots,f_r\}$ generate $\mf n_-$, and $\{h_1,\ldots,h_r\}$ forms a basis of $\mf h$.
		\item Set $a_{ij}\coloneqq\langle\alpha_i^\lor,\alpha_j\rangle$ to be the entries of the Cartan matrix. Then the elements $e_i,f_i,h_i$ satisfy the root decomposition relations
		\[\begin{cases}
			[h_i,h_j]=0, \\
			[h_i,e_j]=a_{ij}e_j, \\
			[h_i,f_j]=-a_{ij}f_j, \\
			[e_i,f_j]=1_{i=j}h_i,
		\end{cases}\]
		and the Serre relations
		\[\begin{cases}
			\op{ad}_{e_i}^{1-a_{ij}}e_j=0, \\
			\op{ad}_{f_i}^{1-a_{ij}}f_j=0.
		\end{cases}\]
	\end{listalph}
\end{theorem}
\begin{proof}
	Here we go.
	\begin{listalph}
		\item This is direct from the root decomposition.
		\item The fact that the $h_\bullet$s form a basis of $\mf h_\bullet$ follows from the fact that the simple roots form a basis of $\mf h^\lor$. Now, let $\mf n_+'\subseteq\mf n_+$ be the subalgebra spanned by the $e_\bullet$s. We would like to show that $\mf n_+'=\mf n_+$; an analogous argument shows that the $f_\bullet$s span $\mf n_-$. Well, $\mf n_+'$ admits a root decomposition as $\bigoplus_{\alpha\in\Phi_+'}\mf g_\alpha$, where $\Phi_+'\subseteq\Phi_+$ is some subset. Supposing this is not an equality for the sake of contradiction, we choose $\alpha\in\Phi_+\setminus\Phi_+'$ of minimal height. Then for each $\mf g_{\alpha-\alpha_i}\in\mf n_+'$, we see that we must have $[e_i,\mf g_{\alpha-\alpha_i}]=0$ in order to actually have a subalgebra. Now, choosing nonzero $x\in\mf g_{-\alpha}$, we note that any $y\in\mf g_{\alpha-\alpha_i}$ must have
		\[0=(x,[e_i,y])=([x,e_i],y)=0,\]
		so $[x,e_i]=0$ follows. In particular, we use the fact that we have a representation over $(\mf{sl}_2)_i$ to note that $x$ becomes a vector of the highest weight, which eventually gives us our contradiction.

		\item The root decomposition relations require no explanation except for $[e_i,f_j]=0$ when $i\ne j$. Well, $[e_i,f_j]\in\mf g_{\alpha_i-\alpha_j}$, but $\alpha_i-\alpha_j$ is not a root (it is neither negative nor positive!), so $[e_i,f_j]=0$. 

		For the last two relations, we will content ourselves with only showing the first because they are symmetric. View $f_j$ as an element of $\mf g$, viewed as a representation of $(\mf{sl}_2)_i$. Notably, $e_i\cdot f_j=[e_i,f_j]=0$ implies that $f_j$ is expected to be a highest weight vector. To make this rigorous, we consider the subrepresentation $U$ generated by $f_j$ and $\op{ad}_{f_i}(f_j)$ and $\op{ad}_{f_i}(f_j)$. Note that $\op{ad}_{h_i}(f_j)=-a_{ij}$, so the weight of $f_j$ is $-a_{ij}$, so the relation follows by the representation theory of $\mf{sl}_2$.
		\qedhere
	\end{listalph}
\end{proof}
\begin{remark}
	It turns out that this forms a full list of relations satisfied by the given basis, but it is not obvious. Explicitly, it turns out that the given relations define a semisimple Lie algebra with root system $\Phi$, but we will not prove this here. However, it is not obvious that the given relations even form a finite-dimensional Lie algebra!
\end{remark}
The main point is that our simple Lie algebras are in bijection with Dynkin diagrams.

\end{document}