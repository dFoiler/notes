% !TEX root = ../notes.tex

\documentclass[../notes.tex]{subfiles}

\begin{document}

\section{November 20}
Today we continue talking about the representation theory of semisimple Lie algebras.

\subsection{Verma Modules}
To continue our classification, we will need to produce representations with a given highest weight. For this, we pass to infinite-dimensional representations, and we will take a quotient only later.
\begin{definition}
	Fix a semisimple Lie algebra $\mf g$ with Cartan subalgebra $\mf h\subseteq\mf g$. Given $\lambda\in\mf h^\lor$, we define the \textit{Verma module} $M_\lambda$ as the corresponding maximal highest weight representation. Explicitly, we define $M_\lambda$ as the quotient of $U\mf g$ by the ideal generated by $h-\lambda(h)$ and the $e_i$.
\end{definition}
\begin{remark}
	Here is another construction. Define $\mf n_+$ as the sum of the positive weight spaces of $\mf g$, and let $\mf b$ be the sum of $\mf h$ and $\mf n_+$. Then note that $\mf b$ can act on $\CC_\lambda$ by $\lambda$, and one can induce this representation from $U\mf b$ to $U\mf g$. Namely,
	\[M_\lambda=U\mf g\otimes_{U\mf b}\CC_\lambda.\]
	Note that the PBW theorem tells us $U\mf g=U\mf n_-\otimes U\mf b$, where $U\mf n_-$ is a free (right) $U\mf b$-module, so we get left with $U\mf n_-\otimes_\CC\CC_\lambda$. For example, we see that $M_\lambda$ has a basis given by vectors of the form $uv_\lambda$ where $u$ is some monomial in $\mf n_-$ and $v_\lambda$ is a fixed nonzero vector in $\CC_\lambda$. This construction also explains that the weights of $M_\lambda$ are exactly the weights bounded above by $\lambda$.
\end{remark}
Here are some more properties of Verma modules.
\begin{lemma}
	Fix a semisimple Lie algebra $\mf g$ with Cartan subalgebra $\mf h\subseteq\mf g$. Given $\lambda\in\mf h^\lor$ and representation $V$, then $\op{Hom}_{\mf g}(M_\lambda,V)$ is in bijection with singular vectors in $V$ of weight $\lambda$. The bijection sends $f\colon M_\lambda\to V$ to $f(v_\lambda)$, where $v_\lambda\in M_\lambda$ is a fixed vector of weight $\lambda$.
\end{lemma}
\begin{proof}
	This is simply Frobenius reciprocity. Indeed,
	\begin{align*}
		\op{Hom}_{\mf g}(M_\lambda,V) &= \op{Hom}_{U\mf g}\left(\op{Ind}_{U\mf b}^{U\mf g}\CC_\lambda,V\right) \\
		&= \op{Hom}_{U\mf b}(\CC_\lambda,V),
	\end{align*}
	and this map sends $f$ to $f(v_\lambda)$, where the point is that morphisms at the bottom are determined precisely by where they send $v_\lambda$, and one can check that the singularity condition is exactly what is dictated by being a $U\mf b$-morphism: the weight comes from the action of $\mf h\subseteq\mf b$, and the singularity comes from the (trivial) action of $\mf n_+\subseteq\mf b$.
\end{proof}
\begin{remark}
	As a consequence, we see that any representation $V$ with a unique highest weight $\lambda$ is a quotient of $M_\lambda$. Indeed, the point is to check that the induced map $M_\lambda\to V$ is surjective, which is not hard to check.
\end{remark}
For our next result, we want the following lemma.
\begin{lemma}
	Fix a semisimple Lie algebra $\mf g$ with Cartan subalgebra $\mf h\subseteq\mf g$. If $V$ is a $U\mf g$-module admitting a weight decomposition, then the same is true for any subquotient.
\end{lemma}
\begin{proof}
	If $U\subseteq V$ is a subrepresentation, then it admits a weight decomposition with $U[\lambda]\coloneqq U\cap V[\lambda]$; it is a little tricky to check that we actually produce a decomposition, but we will not write it out in detail here. Then the quotient $V/U$ also admits a weight decomposition with weight space $V[\lambda]/U[\lambda]$.
\end{proof}
\begin{corollary}
	Any highest weight $U\mf g$-module admits a weight decomposition.
\end{corollary}
\begin{proof}
	Note $M_\lambda$ admits a weight decomposition, so the result follows from the lemma.
\end{proof}
Let's see a couple examples.
\begin{example}
	For $\mf g=\mf{sl}_2(\CC)$, given a weight $\lambda$, we see that $M_\lambda$ has the weights
	\[\{\lambda,\lambda-2,\lambda-4,\ldots\},\]
	where the corresponding vectors are $f^\bullet v_\lambda$.
\end{example}
\begin{example}
	For $\mf g=\mf{sl}_3(\CC)$, given a weight $\lambda$, we see that $M_\lambda$ has the weights
	\[\{\lambda-k\alpha_1-\ell\alpha_2:k,\ell\ge0\}.\]
	Notably, these weight spaces may have higher dimension. For example, $f_1f_2v_\lambda$ and $[f_1,f_2]v_\lambda$ both produce the weight $\lambda-\alpha_1-\alpha_2$.
\end{example}
In general, perhaps we want to compute the character of $M_\lambda$, which we see is
\[\chi_{M_\lambda}=\sum_{\beta\in Q_+}\dim M_\lambda[\lambda-\beta]e^{\lambda-\beta}.\]
Well, suppose we have a partition of $\beta$ into a sum of positive roots $\gamma_1+\cdots+\gamma_k$, we produce a vector $f_{\gamma_1}\cdots f_{\gamma_k}v_\lambda$ in the weight space $M_\lambda[\lambda-\beta]$. Now, the generating function for this construction turns out to be
\[\prod_{\alpha\in\Phi_+}\left(1+e^{-\alpha}+e^{-2\alpha}+\cdots\right)=\sum_\beta\#\{\text{partitions of }\beta\}e^{-\beta}.\]
Thus, we find that
\[\chi_{M_\lambda}=\frac{e^\lambda}{\prod_{\alpha\in\Phi_+}\left(1-e^{-\alpha}\right)}.\]

\subsection{Classifying the Highest Weights}
Anyway, the point is that we are interested in quotients of $M_\lambda$.
\begin{lemma}
	Fix a semisimple Lie algebra $\mf g$ with Cartan subalgebra $\mf h\subseteq\mf g$. For a weight $\lambda$, the Verma module $M_\lambda$ has a unique maximal proper subspace.
\end{lemma}
\begin{proof}
	Note that $N\subseteq M_\lambda$ is proper if and only if it fails to contain the vector $v_\lambda\in M_\lambda[v_\lambda]$. Thus, we can just take the sum of all submodules $N\subseteq M_\lambda$ such that $N[\lambda]=0$.
\end{proof}
\begin{corollary}
	Fix a semisimple Lie algebra $\mf g$ with Cartan subalgebra $\mf h\subseteq\mf g$. For a weight $\lambda$, the Verma module $M_\lambda$ has a unique irreducible quotient, which we denote $L_\lambda$.
\end{corollary}
\begin{proof}
	This follows by taking the quotient by the unique maximal proper subspace given by the lemma.
\end{proof}
As a result, we see that each irreducible finite-dimensional representation of $\mf g$ can be seen in one of the $L_\lambda$s. We are left with the problem of determining which weights $\lambda$ produce a finite-dimensional $L_\lambda$. For this, we pick up the following definition.
\begin{definition}[dominant integral]
	Fix a semisimple Lie algebra $\mf g$ with Cartan subalgebra $\mf h\subseteq\mf g$. A weight $\lambda\in\mf h^\lor$ is a \textit{dominant integral} weight if and only if $\langle\lambda,\alpha^\lor\rangle\in\ZZ^+$ for all $\alpha\in\Phi_+$.
\end{definition}
On the homework, we show that the highest weights of finite-dimensional irreducible representations are all dominant. Our end goal is to show the following result.
\begin{theorem}
	Fix a semisimple Lie algebra $\mf g$ with Cartan subalgebra $\mf h\subseteq\mf g$. For weight $\lambda$, the irreducible representation $L_\lambda$ is finite-dimensional if and only if $\lambda$ is dominant integral.
\end{theorem}
We begin with a lemma.
\begin{lemma}
	For dominant integral weight $\lambda$, define $\lambda_i\coloneqq\alpha_i^\lor(\lambda)$ for each simple root $\alpha_i$. Then
	\[f_i^{\lambda_i+1}v_\lambda=0.\]
\end{lemma}
\begin{proof}
	We use the representation theory of $\mf{sl}_2(\CC)_i$. For example, being singular grants $e_iv_\lambda=0$ and $h_iv_\lambda=\lambda_iv_\lambda$. A computation in $\mf{sl}_2(\CC)_i$ shows that
	\[e_if_i^kv_\lambda=k(\lambda_i-k+1)f^{k-1}v_\lambda,\]
	so for example $e_if_i^{\lambda_i+1}v_\lambda=0$. One can also check that $e_jf^{\lambda_i+1}v_\lambda=0$ for all $j\ne i$, so the vector $f_i^{\lambda_i+1}v_\lambda$ must be singular. However, it has weight smaller than $v_\lambda$, which violates the irreducibility of $L_\lambda$.
\end{proof}
We will complete the proof next class.

\end{document}