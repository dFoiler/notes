% !TEX root = ../notes.tex

\documentclass[../notes.tex]{subfiles}

\begin{document}

\section{November 15}
Today we continue talking about Weyl chambers.

\subsection{More on Weyl Chambers}
Here are a couple more facts.
\begin{corollary}
	Fix a root system $\Phi\subseteq E$. The Weyl group is generated by simple reflections.
\end{corollary}
\begin{proof}
	By moving the positive Weyl chamber transitively, we see that each $\alpha\in\Phi$ has $L_\alpha=wL_{\beta}$ for some $\beta\in\Delta$. This implies that $s_\alpha=ws_\beta w^{-1}$ for some $w$ which is the product of simple reflections, so we are done.
\end{proof}
\begin{corollary}
	Fix a reduced root system $\Phi\subseteq E$ with collection of simple roots $\Delta$. Then $\Phi=W(\Phi)\Delta$.
\end{corollary}
\begin{proof}
	The construction of the previous corollary has $\alpha=\pm w(\beta)$, so we are done.
\end{proof}
\begin{remark}
	This implies that a root system $\Phi$ can be recovered from its collection of simple roots: if $\Delta$ is the collection of simple roots of some root system, then the Weyl group $S$ is recovered from the group generated by the simple reflections, and the root system is recovered as $W(\Delta)$.
\end{remark}
It will be helpful to pick up a notion of length for the Weyl element, to aid our combinatorics.
\begin{definition}[length]
	Fix a polarized reduced root system $\Phi$ with polarization given by $t\in E$ and simple roots $\Delta$. For $w\in W$, its \textit{length} $\ell(w)$ is the number of root hyperplanes separating $C_+$ and $w(C_+)$. Here, a hyperplane $L_\alpha$ separates two Weyl chambers if and only if they lie on opposite sides of $L_\alpha$, meaning that $t\in C_+$ and $t'\in w(C_+)$ has $(t,\alpha)(t',\alpha)<0$. One can unwind this definition as
	\[\ell(w)=\#\{\alpha\in\Phi_+:(t,w\alpha)<0\}.\]
\end{definition}
\begin{remark}
	Note that $\ell(w)=\ell\left(w^{-1}\right)$ by a direct counting argument.
\end{remark}
\begin{example}
	If $w=s_\alpha$ is a simple reflection, then $\ell(w)=1$ because $L_{\alpha}$ is the only hyperplane separating $C_+$ and $s_\alpha(C_+)$.
\end{example}
To continue our discussion, we take a lemma.
\begin{lemma}
	Fix a polarized reduced root system $\Phi\subseteq E$, and define
	\[\rho\coloneqq\frac12\sum_{\alpha\in\Phi_+}\alpha.\]
	Then $\langle\rho,\alpha_i^\lor\rangle=2\frac{(\rho,\alpha_i)}{(\alpha_i,\alpha_i)}=1$ for each simple root in $\{\alpha_1,\ldots,\alpha_n\}$.
\end{lemma}
\begin{proof}
	One finds that $s_i(\rho)=\rho-\alpha_i$ because $s_i$ can only flip the positive root $\alpha_i$ and must leave the other fixed (because they live in $C_+$). This then finishes by a computation.
\end{proof}
Anyway, here is the point of defining length.
\begin{theorem}
	Fix a polarized reduced root system $\Phi\subseteq E$ with simple reflections $\{s_1,\ldots,s_n\}$. For $w\in W(\Phi)$, let
	\[w=s_{i_1}\cdots s_{i_\ell}\]
	be an expression for $w$ as a product of simple reflections with minimal possible length. Then $\ell=\ell(w)$.
\end{theorem}
\begin{proof}
	For each $j\in\{0,\ldots,\ell\}$, we define $C_j\coloneq s_{i_1}\cdots s_{i_j}C_+$. Then $C_{j}$ and $C_{j+1}$ are adjacent separated only by $L_{s_{i_1}\cdots s_{i_j}\alpha_{j+1}}$. This yields $\ell(w)\le\ell$ because we have shown that $C_+$ and $wC_+$ are separated by at most $\ell$ walls.

	To show the other inequality, consider the straight-line path from a point in $C_+$ to a point in $w(C_+)$, which pass through exactly $\ell(w)$ hyperplanes. Taking the hyperplanes that this path intersects, we see that
	\[w={s_{i_1}}\cdots s_{j_{\ell(w)}},\]
	so we see that $\ell\le\ell(w)$ by the minimality of $\ell$.
\end{proof}
We are now allowed to define a reduced product.
\begin{definition}[reduced]
	Fix a polarized reduced root system $\Phi\subseteq E$. For $w\in W(\Phi)$, an expansion
	\[w=s_{i_1}\cdots s_{i_\ell}\]
	of $w$ into a product of simple reflections is \textit{reduced} if and only if $\ell=\ell(w)$.
\end{definition}
And here are some nice corollaries.
\begin{corollary}
	Fix a reduced root system $\Phi\subseteq E$. The Weyl group acts simply transitively on the collection of Weyl chambers.
\end{corollary}
\begin{proof}
	Choose a polarization of $\Phi$, which amounts to choosing some $C_+$. We already know our action is transitive, so it remains to show that the action is free. Well, if $w(C_+)=C_+$, we would like to show that $w$ is the identity. Well, $w(C_+)=C_+$ implies that $\ell(w)=0$, so $w$ can be written as a product of zero simple reflections, so $w$ is the identity.
\end{proof}
\begin{corollary}
	Fix a polarized reduced root system $\Phi\subseteq E$. Let $C_-\coloneqq-C_+$ be the negative Weyl chamber, and choose the unique $w\in W$ such that $w(C_+)=C_-$. Then $\ell(w_0)=\#\Phi_+$, and for any $w'\ne w$, we have $\ell(w')<\ell(w)$.
\end{corollary}
\begin{proof}
	Certainly $\ell(w_0)=\#\Phi_+$ is maximal by definition of $\ell(w_0)$. This is unique by being simply transitive because $C_-$ is unique as being separated from $C_+$ by all root hyperplanes.
\end{proof}

\subsection{Cartan Matrices}
Our overall goal is to classify root systems. We will only classify reducible ones for the following reason.
\begin{definition}[irreducible]
	A root system $\Phi\subseteq E$ is \textit{reducible} if and only if it can be written as $\Phi=\Phi_1\sqcup\Phi_2$ where $\Phi_1$ and $\Phi_2$ are perpendicular. Otherwise, $\Phi$ is \textit{irreducible}.
\end{definition}
\begin{remark}
	If $\Phi$ is a polarized reduced root system with simple roots $\Delta$, then we note that the reducibility can be seen on the level of the simple roots. If $\Phi$ reduces as $\Phi_1\sqcup\Phi_2$ with simple roots $\Delta_1$ and $\Delta_2$, then $\Delta=\Delta_1\sqcup\Delta_2$ by uniqueness of having the simple roots (e.g., as a basis). Conversely, if one can decompose $\Delta=\Delta_1\sqcup\Delta_2$ with $\Delta_1$ and $\Delta_2$ orthogonal, then we can recover $\Phi_1$ and $\Phi_2$ as Weyl group orbits $W(\Delta_1)\Delta_1$ and $W(\Delta_2)\Delta_2$. One sees that $W(\Phi)=W(\Delta_1)\times W_2(\Delta_2)$, so
	\[\Phi=W(\Delta_1\sqcup\Delta_2)=W(\Delta_1)\sqcup W(\Delta_2),\]
	allowing us to check that $W(\Delta_1)$ and $W(\Delta_2)$ are in fact root systems witnessing the decomposition of $\Phi$.
\end{remark}
\begin{remark}
	The above remark shows that any root system can be written uniquely as the disjoint union of irreducible ones.
\end{remark}
We are now ready to define the Cartan matrix.
\begin{definition}[Cartan matrix]
	Fix a polarized reduced root system $\Phi\subseteq E$ with simple roots $\Delta$. Then the \textit{Cartan matrix} $A(\Phi)$ is the matrix with entries
	\[A_{ij}\coloneqq\langle\alpha_i^\lor,\alpha_j\rangle=2\frac{(\alpha_i,\alpha_j)}{(\alpha_i,\alpha_i)}.\]
\end{definition}
\begin{remark}
	The Cartan matrix fully determines the root system. Indeed, these inner products are enough to define the simple reflections on the level of the simple roots. The simple roots form a basis, so one can apply as many simple reflections as possible to eventually build the full root system.
\end{remark}
\begin{remark}
	By definition of the root system, we see that $A_{ii}=2$.
\end{remark}
\begin{remark}
	Simple roots have obtuse angles, so $A_{ij}$ is a nonpositive integer for $i\ne j$.
\end{remark}
\begin{remark}
	The matrix $\{(\alpha_i,\alpha_i)a_{ij}\}$ can be checked to be a positive-definite symmetric matrix.
\end{remark}

\end{document}