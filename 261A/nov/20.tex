% !TEX root = ../notes.tex

\documentclass[../notes.tex]{subfiles}

\begin{document}

\section{November 20}
There's nobody here today! We finished class by proving the Serre relations.

\subsection{Highest Weights}
We now begin discussing the representation theory of semisimple Lie algebras. On the homework, we will prove the following.
\begin{lemma} \nirindex{weight}
	Fix a finite-dimensional representation $V$ of a semisimple Lie algebra $\mf g$. Let $\mf h\subseteq\mf g$ be a Cartan subalgebra.
	\begin{listalph}
		\item The action of $\mf h$ on $V$ is semisimple.
		\item We have a weight decomposition
		\[V=\bigoplus_{\lambda\in\mf h^\lor}V[\lambda],\]
		where $\lambda\in\mf h^\lor$ such that $V[\lambda]\ne0$ is called a weight; we note this set by $P(V)$. For each weight $\lambda$, one has $\alpha^\lor(\lambda)\in\ZZ$ for all roots $\alpha$ of $\mf g$.
	\end{listalph}
\end{lemma}
The above lemma lets us define a character theory.
\begin{definition}[character]
	Fix a finite-dimensional representation $V$ of a semisimple Lie algebra $\mf g$. Then we define the \textit{character} of $V$ as an element $\chi_V$ in the polynomial algebra $\ZZ\left[\{e^\lambda:\lambda\in P\}\right]$ defined by
	\[\chi_V\coloneqq\sum_{\substack{\lambda\in P\\V[\lambda]\ne0}}\dim V[\lambda]e^\lambda.\]
\end{definition}
\begin{remark}
	One can check that $\chi_{V\oplus W}=\chi_V+\chi_W$ and $\chi_{V\otimes W}=\chi_V\chi_W$ and $\chi_{V^\lor}=\overline{\chi_V}$ (where conjugation sends $e^\lambda\mapsto e^{-\lambda}$).
\end{remark}
\begin{remark}
	One can check that this agrees with our notion of characters for $\mf{sl}_2$.
\end{remark}
\begin{remark}
	We have not yet shown that characters determine the representation up to isomorphism, but this will turn out to be the same.
\end{remark}
It will be helpful for us to have an ordering on the weights.
\begin{definition}
	Fix a semisimple Lie algebra $\mf g$, and provide a polarization of its root system. Given two elements $\lambda,\mu$ in $P$, we say that $\lambda\ge\mu$ if and only if $\lambda-\mu$ is a nonnegative linear combination of positive roots.
\end{definition}
Anyway, let's continue with our discussion of highest weights. We would like an intrinsic way to check for vectors corresponding to highest weights.
\begin{definition}[singular]
	Fix a semisimple Lie algebra $\mf g$ with Cartan subalgebra $\mf h$ and the usual root decomposition. A vector $v$ in a representation $V$ is \textit{singular} if and only if $e_iv=0$ for all $e_i$ corresponding to a simple root.
\end{definition}
\begin{remark}
	For reducibility reasons, one cannot expect that singular vectors always have the highest weight. In particular, if $v$ is the highest weight vector for some $V$, and $w$ is the highest weight vector for some $W$, then both $v$ and $w$ are singular for $V\oplus W$, but is it possible for one to no longer actually have the highest weight for $V\oplus W$.
\end{remark}
\begin{remark}
	One can check that any finite-dimensional representation has a singular vector.
\end{remark}
\begin{remark}
	Given a singular vector $v_\lambda\in V[\lambda]$, supposing that $V$ is generated by $v_\lambda$ (for example, if $V$ is irreducible), then we find that $V=U\mf g\cdot v_\lambda$. But $\mf n_+$ and $\mf h$ do not act in an interesting way on $v_\lambda$, so we find that $V=U\mf n_-\cdot v_\lambda$, so the weights of $V$ are seen to all be bounded above by $\lambda$.
\end{remark}
\begin{definition}[highest weight]
	Fix a semisimple Lie algebra $\mf g$. A representation $V$ is called a \textit{highest weight representation} if and only if it is generated by a singular vector.
\end{definition}
\begin{remark}
	One can check that irreducible representations $V$ have only one singular vector up to scalar, and the weight $\lambda$ is larger than all other weights of $V$, so this is a genuine highest weight vector. Additionally, one can check that $\lambda\in P\cap\overline{C_+}$: one has $\alpha^\lor(\lambda)\ge0$ for all positive roots $\alpha$.
\end{remark}

\end{document}