% !TEX root = ../notes.tex

\documentclass[../notes.tex]{subfiles}

\begin{document}

\section{November 4}
Last class was cancelled due to a room lockout. Today we continue talking about regular elements.

\subsection{Cartan Subalgebras via Regularity}
We now use regularity for fun and profit.
\begin{proposition}
	Fix a semisimple Lie algebra $\mf g$ over an algebraically closed field $F$ of characteristic $0$.
	\begin{listalph}
		\item For any regular semisimple element $X$, its centralizer $C(X)$ is a Cartan subalgebra of $\mf g$.
		\item For any Cartan subalgebra $\mf h$, we have $\mf h=C(X)$ for some regular semisimple element $X$.
	\end{listalph}
\end{proposition}
\begin{proof}
	We quickly address (b). The lemma from last class shows that $\mf h$ contains a regular semisimple element $X$, so $\mf h\subseteq C(X)$. However, $\dim\mf h=\op{rank}\mf g=\dim C(X)$, so equality follows.

	We spend the rest of our time focusing on (a). Note that $\CC X$ is a toral subalgebra of $\mf g$, so we get a root decomposition
	\[\bigoplus_{\lambda\in\CC}\mf g_\lambda,\]
	where $\mf g_\lambda$ is the eigenspace of $\op{ad}_X$ with eigenvalue $\lambda$. For example, $\mf g_0=C(X)$ is a reductive subalgebra with dimension $\dim C(X)=\op{rank}\mf g$.

	Quickly, we claim that $C(X)$ is nilpotent. It is enough to check that each $Y\in C(X)$ has $\op{ad}_Y\colon C(X)\to C(X)$ a nilpotent operator. Well, note that $X+tY\in C(X)$ has ${\op{ad}_{X+tY}}|_{\mf g/\mf g_0}$ invertible for small values of $t$ (because it is invertible at $t=0$), so the null space of $X+tY$ has $n(X+tY)\subseteq C(X)$; on the other hand, the rank bound requires $n(X+tY)\ge\op{rank}\mf g=\dim C(X)$, so equality follows. Because $X$ acts by $0$ on $C(X)$, we thus conclude that ${\op{ad}_{X+tY}}|_{C(X)}$ is nilpotent for small $t$, so ${\op{ad}_Y}|_{C(X)}$ is also nilpotent.

	We are now ready to claim that $\mf g_0=C(X)$ is a Cartan subalgebra.
	\begin{itemize}
		\item Abelian: we already know that it is reductive and nilpotent, so it must be abelian.
		\item Semisimple: for any $Y\in\mf g_0$, we want to show that $Y_n=0$. Well, $\op{ad}_{Y_n}$ is nilpotent, and $\mf g_0$ is abelian, so ${\op{ad}_{Y_n}}\circ\op{ad}_Z$ is nilpotent for any $Z\in\mf g_0$, so the trace of this composite vanishes. But the trace of this composite is the Killing form, which is non-degenerate on $\mf g_0$, so $Y_n=0$ follows.
		\item Cartan: note that $\mf g_0$ is now toral, so place it in some Cartan subalgebra $\mf g_0\subseteq\mf h$. But then the commutator of $\mf h$ is contained in $C(X)=\mf g_0$, so equality follows.
		\qedhere
	\end{itemize}
\end{proof}
\begin{corollary}
	Fix a semisimple Lie algebra $\mf g$ over an algebraically closed field $F$ of characteristic $0$.
	\begin{listalph}
		\item Any regular element is semisimple.
		\item Any regular element is contained in a unique Cartan subalgebra.
	\end{listalph}
\end{corollary}
\begin{proof}
	For (a), note that any regular $X$ will have $\op{ad}X_s=(\op{ad}X)_s$, so regularity of $X$ implies regularity of $X_s$. But $C(X_s)$ is then a Cartan subalgebra by the proposition, and $X\in C(X_s)$, so $X$ is semisimple!

	Suppose $X$ is regular and contained in a Cartan subalgebra $\mf h$. Then $C(X)$ is a Cartan subalgebra, but $\mf h\subseteq C(X)$, so equality follows by maximality upon comparing dimensions.
\end{proof}
Here is the main application of our discussion of regularity; note that this result's statement does not mention regularity at all!
\begin{theorem}
	Fix a semisimple Lie algebra $\mf g$ over an algebraically closed field $F$ of characteristic $0$. Then any two Cartan subalgebras are conjugate in the following sense: if $G$ is a connected Lie algebra with $\op{Lie}G=\mf g$ (such as $G=\op{Aut}(\mf g)^\circ$), then any two Cartan subalgebras $\mf h_1$ and $\mf h_2$ have $g\in G$ such that $\mf h_2=\op{Ad}_g\mf h_2$.
\end{theorem}
\begin{proof}
	Note that Cartan subalgebras of $\mf g$ all look like commutators of regular elements, which will be how we profit. As such, we define an equivalence relation on $\mf g^{\mathrm{reg}}$ by saying that $X\sim Y$ if and only if the Cartan subalgebras $C(X)$ and $C(Y)$ are conjugate; we will not bother to check that this is an equivalence relation.

	The main claim is that the equivalence class of some $X\in\mf g^{\mathrm{reg}}$ contains an open neighborhood of $X$. Well, any element in $C(X)$ of course lives in the same equivalence class; further, we see that $C(\op{Ad}_gX)=\op{Ad}_gC(X)$ by a quick computation, so we conclude that the equivalence class fully consists of the elements of the form $\op{Ad}_gY$ for any $Y\in C(X)$ and $g\in G$. However, \Cref{lem:regularity-check-cartan} checked that
	\[\left\{\op{Ad}_gY\in\mf g^{\mathrm{reg}}:Y\in C(X),g\in G\right\}\]
	is open.

	We now finish the proof: all equivalence classes are open in $\mf g^{\mathrm{reg}}$, so by taking complements, we see that they are also all closed. Because $\mf g^{\mathrm{reg}}$ is connected, we conclude that there must be only one equivalence class.
\end{proof}
\begin{remark}
	Next class we will start talking about root systems abstractly. We will want to work over $\RR$ throughout instead of $\CC$, which we note is legal basically because the root system has integral eigenvalues, so everything can be done on the level of a lattice. For example, one can check that there is a decomposition $\mf h=\mf h_\RR\oplus i\mf h_\RR$ (because these cannot really intersect after we have an integral lattice), and the Killing form is positive-definite on $\mf h_\RR$ (because $K$ is basically a sum of square of eigenvalues of $\mf h$).
\end{remark}

\end{document}