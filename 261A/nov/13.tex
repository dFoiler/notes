% !TEX root = ../notes.tex

\documentclass[../notes.tex]{subfiles}

\begin{document}

\section{November 13}
We began class talking about our various lattices.

\subsection{Various Lattices}
Now that everything is proven to be integral, we note that we could more or less just work with the following lattices.
\begin{itemize}
	\item There is a root lattice $Q\coloneqq\ZZ[\Phi]\subseteq E$.
	\item There is a coroot lattice $Q^\lor\coloneqq\ZZ[\Phi^\lor]\subseteq E^\lor$.
	\item There is a weight lattice
	\[P\coloneqq\{\lambda\in E:(\lambda,\alpha^\lor)\in\ZZ\text{ for all }\alpha^\lor\in\Phi^\lor\}.\]
	In particular, the dual of the coroot $(-,\alpha^\lor)$ is referred to as a fundamental weight. Note that $Q\subseteq P$ because $(\alpha,\beta^\lor)\in\ZZ$ always.
	\item There is a coweight lattice
	\[\{\lambda^\lor\in E^\lor:\lambda^\lor(\alpha)\in\ZZ\text{ for all }\alpha\in\Phi\}.\]
\end{itemize}
\begin{example}
	Take $\Phi$ of type $A_1$. Then we have a single root $\alpha$ given by $h_\alpha\in\mf{sl}_2(\CC)$. Then $Q=\ZZ\alpha$, but $P=\frac12\ZZ\alpha$ because $\left(\frac12\alpha,\alpha^\lor\right)=1$.
\end{example}
\begin{example}
	Take $\Phi\subseteq E$ of type $A_{n-1}$. Then $Q$ is the collection of vectors in $\ZZ^n$ with sum $0$. In contrast, we find that $P$ consists of the vectors $(\lambda_1,\ldots,\lambda_n)\in E$ with the constraint $\lambda_i-\lambda_j\in\ZZ$ for all $i$ and $j$ by trying to pair elements of $E$ with $(e_i-e_j)\in\Phi$. As such, we see that $P$ has translates of $Q$ by vectors $(t,\ldots,t)$ for $t\in\QQ$, but summing to $0$ requires $t\in\frac12n\ZZ/\ZZ$. We find $P/Q=\ZZ/n\ZZ$.
\end{example}
\begin{remark}
	Essentially by construction, we see that the Weyl group has an action on all of our lattices. Namely, $Q$ is spanned by roots, so the Weyl group of course acts on $Q$, and it acts on $Q^\lor$ and then $P$ by taking enough duals.
\end{remark}

\subsection{Weyl Chambers}
Given some $t\in E$, one produces a polarization $\Phi=\Phi_+\sqcup\Phi_-$. However, we see that this choice of decomposition only depends on the choice of signs $(t,\alpha)$. As such, one can imagine moving $t$ around until hitting a ``wall''
\[L_\alpha=\{t\in E:(t,\alpha)=0\}\]
where a sign change (and thus change in polarization) can occur.
\begin{definition}[Weyl chamber]
	Fix a root system $\Phi\subseteq E$. A \textit{Weyl chamber} is a connected component of
	\[E_\RR\setminus\bigcup_{\alpha\in\Phi}L_\alpha,\]
	where $L_\alpha$ is the wall $\{t\in E:(t,\alpha)=0\}$.
\end{definition}
The point is that $t,t'\in E_\RR$ live in the same Weyl chamber if and only if there is a path connecting the two, which implies that they produce the same polarization. Conversely, if they produce the same polarization, then one checks that they satisfy all the same inequalities when paired with roots in $\Phi$, which is exactly a Weyl chamber. More precisely, given some $\varepsilon\colon\Phi\to\{\pm1\}$, we may define
\[C_\varepsilon\coloneqq\{t\in E_\RR:\varepsilon(\alpha)(t,\alpha)>0\},\]
and then we see that $C_\varepsilon$ is either empty or equal to a Weyl chamber. Of course, it is always possible to be empty because one cannot always solve this system of inequalities.

Here are some geometric facts.
\begin{lemma}
	Fix a root system $\Phi\subseteq E$, and let $C\subseteq E_\RR$ be a Weyl chamber.
	\begin{listalph}
		\item The closure $\ov C$ is a convex cone.
		\item The boundary $\del C$ is a union of codimension one faces $F$ which live in exactly one ``wall'' $L_\alpha$.
	\end{listalph}
\end{lemma}
\begin{proof}
	Checking (a) is direct: we have to check that $\ov C$ is closed under nonnegative linear combination, which is direct. Checking (b) is just some geometry.
\end{proof}
As such, given a polarization of $\Phi$, we may define the positive Weyl chamber
\[C_+\coloneqq\{t\in E:(t,\alpha)>0\text{ for all }\alpha\in\Phi_+\}.\]
One can check that the walls of $C_+$ are given exactly by the simple roots of this polarization: certainly
\[C_+=\{t\in E:(t,\alpha)\text{ for }\alpha\in\Delta\}\]
by taking positive linear combinations of simple roots to get positive roots. From here, one can check that we have $L_\alpha$s for $\alpha\in\Delta$ as our walls.

Unsurprisingly, we have a Weyl group action.
\begin{proposition}
	Fix a root system $\Phi\subseteq E$. The Weyl group acts transitively on the set of Weyl chambers.
\end{proposition}
\begin{proof}
	It is not hard to check that the Weyl group action on $E$ descends to an action on the Weyl chambers simply by staring at our inequalities. Thus, the difficulty comes from checking the transitivity. Well, note that if $C$ and $C'$ share an adjacent face $L_\alpha$, then $s_\alpha(C)=C'$, which again we can see by writing out the relevant inequalities.

	Thus, given any two Weyl chambers $C$ and $C'$, we want to show that we can connect them by a sequence of Weyl chambers which share a face. Well, choose sufficiently generic $t\in C$ and $t'\in C'$ so that the line segment connecting $t$ to $t'$ intersects walls (but never more than one at once). Ordering which chambers this line segment goes through, we get a sequence of Weyl chambers $C=C_0,C_1,\ldots,C_n=C'$ each of which share a face. Applying the previous paragraph inductively completes the proof.
\end{proof}
\begin{corollary}
	Fix a root system $\Phi\subseteq E$. Given two polarizations $\Phi=\Phi_+\sqcup\Phi_-$ and $\Phi=\Phi_+'\sqcup\Phi_-'$ giving simple roots $\Delta\subseteq\Phi$ and $\Delta'\subseteq\Phi'$, there is an element $w\in W(\Phi)$ such that $w\Delta=\Delta'$.
\end{corollary}
\begin{proof}
	The walls of the Weyl chamber corresponding to $C_+$ get permuted to the walls of the Weyl chamber corresponding to $C_+'$.
\end{proof}
We now reduce to simple reflections, which are reflections belonging to simple roots.
\begin{lemma}
	Fix a polarized root system $\Phi\subseteq E$. For every Weyl chambers $C$, there are simple reflections $s_1,\ldots,s_n$ such that
	\[C=s_1\cdots s_n(C).\]
\end{lemma}
\begin{proof}
	Choose generic $t\in C$ and $t'\in C_+$, and suppose the line segment connecting $t$ to $t'$ passes through $m$ walls. If $m=1$, there is nothing to do, so we assume $m>1$ and induct. Well, we can move $C$ to an adjacent chamber via a reflection which is simple with respect to some polarization, but a Weyl element conjugates this reflection to one simple with respect to the polarization given by $C_+$. So we are done.
\end{proof}

\end{document}