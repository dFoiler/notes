% !TEX root = ../notes.tex

\documentclass[../notes.tex]{subfiles}

\begin{document}

\section{August 30}

Today we finish our review of smooth manifolds. Once again, we refer to \cite{elber-diff-top} for a few more details and \cite{lee-manifolds} for many more details.
\begin{notation}
	We will use the word \textit{regular} to refer to one of the regularity conditions $C^k$, smooth, real analytic, or complex analytic. We may abbreviate complex analytic to ``complex'' when no confusion is possible. We use the field $\FF$ to denote the ``ground field,'' which is $\CC$ when considering the complex analytic case and $\RR$ otherwise.
\end{notation}

\subsection{Smooth Manifolds}
We now define a regular manifold.
\begin{definition}[regular manifold]
	A \textit{regular manifold} of dimension $n$ is a pair $(M,\mc A)$ of a topological manifold $M$ and a maximal regular atlas $\mc A$; a chart is called regular if and only if it is in $\mc A$. We will eventually suppress the $\mc A$ from our notation as much as possible.
\end{definition}
The reason for using a maximal atlas is to ensure that it is more or less unique.
\begin{remark}
	Here is perhaps a more ``canonical'' way to deal with atlas confusion. One can say that two regular atlases $\mc A_1$ and $\mc A_2$ are compatible if and only if the transition maps between them are also regular; this is the same as saying that $\mc A_1\cup\mc A_2$ is regular. Compatibility forms an equivalence relation, and each equivalence class $[\mc A]$ has a unique maximal element, which one can explicitly define as
	\[\mc A_{\mathrm{max}}\coloneqq\{(U,\varphi):\mc A\text{ and }(U,\varphi)\text{ are compatible}\}.\]
	This explains why it is okay to just work with maximal atlases.
\end{remark}
\begin{example}
	One can give the topological manifold $\RR^2$ many non-equivalent complex structures. For example, one has the usual choice of $\RR^2\cong\CC$, but one can also make $\RR^2$ homeomorphic to $B(0,1)\subseteq\CC$.
\end{example}
\begin{example}
	There are ``exotic'' smooth structures on $S^7$.
\end{example}
\begin{example}
	Given regular manifolds $(X,\mc A)$ and $(Y,\mc B)$, one can form the product manifold $X\times Y$. It should have maximal atlas compatible with the atlas
	\[\{(U\times V,\varphi\times\psi):(U,\varphi)\in\mc A\text{ and }(V,\psi)\in\mc B\}.\]
\end{example}

\subsection{Regular Functions}
With any class of objects, we should have morphisms.
\begin{definition}
	A function $f\colon X\to Y$ of regular manifolds $(X,\mc A)$ and $(Y,\mc B)$ is regular if and only if any $p\in X$ has a choice of charts $(U,\varphi)\in\mc A$ and $(V,\psi)\in\mc B$ such that $p\in U$ and $f(U)\subseteq V$ and the composite
	\[\varphi(U)\stackrel\varphi\from U\stackrel f\to V\stackrel\psi\to\psi(V)\]
	is a regular function between open subsets of Euclidean space.
\end{definition}
\begin{remark}
	One can replace the single choice of charts above with any choice of charts satisfying $p\in U$ and $f(U)\subseteq V$.
\end{remark}
\begin{remark}
	Here is another way to state this: for any open $V\subseteq Y$ and smooth function $h\colon V\to\FF$, the composite
	\[f^{-1}(U)\stackrel f\to V\stackrel h\to\FF\]
	succeeds in being smooth (in any local coordinates).
\end{remark}
\begin{definition}[diffeomorphism]
	A \textit{diffeomorphism} of regular manifolds $(X,\mc A)$ and $(Y,\mc B)$ is a regular map $f\colon X\to Y$ with regular inverse.
\end{definition}
\begin{remark}
	Alternatively, one can say that the charts in $\mc A$ and the charts in $\mc B$ are in natural bijection via $f$. Checking that these notions align is not too hard.
\end{remark}
The above definition of regular map is a little rough to handle, so let's break it down into pieces.
\begin{definition}[local coordinates]
	Fix a regular manifold $(X,\mc A)$ of dimension $n$. Then a system of \textit{local coordinates} around some point $p\in X$ is a choice of regular chart $(U,\varphi)\in\mc A$ for which $\varphi(p)=0$. From here, our local coordinates $(x_1,\ldots,x_n)$ are the composite of $\varphi$ with a coordinate projection to the ground field. (In the complex analytic case, we want the ground field to be $\CC$; otherwise, the ground field is $\RR$.)
\end{definition}
Now, we are able to see that a function $f\colon X\to\RR$ is regular if and only if it becomes regular in local coordinates. One can even define regularity with respect to a subset of $X$.

Regularity allows us to produce lots of manifolds, as follows.
\begin{theorem}
	Given regular maps $f_1,\ldots,f_m\colon X\to\FF$, the subset
	\[\left\{p\in\FF^n:f_1(p)=\cdots=f_m(p)=0\text{ and }\{df_1(p),\ldots,df_n(p)\}\text{ are linearly independent}\right\}\]
	is a manifold of dimension $n-m$.
\end{theorem}
\begin{proof}[Sketch]
	This is more or less by the implicit function theorem; for the $\FF=\RR$ cases, one can essentially follow \cite[Corollary~5.14]{lee-manifolds}.
\end{proof}
\begin{example}
	The function $\RR^{n+1}\to\RR$ given by $(x_0,\ldots,x_n)\mapsto x_0^2+\cdots+x_n^2$ is real analytic and sufficiently regular at the value $1$, which establishes that $S^n$ defined in \Cref{ex:sphere} succeeds at being a real analytic manifold.
\end{example}
Functions to $\FF$ have a special place in our hearts, so we take the following notation.
\begin{notation}
	Give a regular manifold $X$ and any open subset $U\subseteq X$, we let $\OO_X(U)$ denote the set of regular functions $U\to\FF$
\end{notation}
\begin{remark}
	One can check that the data $\OO_X$ assembles into a sheaf. Namely, an inclusion $U\subseteq V$ produces restriction maps $\OO_X(V)\to\OO_X(U)$.
\end{remark}
\begin{remark}
	Once we have all of our regular functions out of $X$, we note that some Yoneda-like philosophy explains that the sheaf of $X$ determines its full regular structure. Here is an explicit statement: given a manifold $X$ and two maximal regular atlases $\mc A_1$ and $\mc A_2$ determining sheaves of regular functions $\OO_1$ and $\OO_2$, having $\OO_1=\OO_2$ forces $\mc A_1=\mc A_2$. Indeed, it is enough to show the inclusion $\mc A_1\subseteq\mc A_2$, so suppose $(U,\varphi)$ is a regular chart in $\mc A_1$. Then the corresponding local coordinates $(x_1,\ldots,x_n)$ all succeed at being regular for $\mc A_1$, so they are smooth functions in $\OO_1$, so they live in $\OO_2$ also, so $(U,\varphi)$ will succeed at being a regular local diffeomorphism for $\mc A_2$ and hence be a regular chart.
\end{remark}
Sheaf-theoretic notions tell us that we should be interested in germs.
\begin{definition}[germ]
	Fix a point $p$ on a regular manifold $X$. A \textit{germ} of a regular function $f\in\OO_X(U)$ (where $p\in U$) is the equivalence class of functions $g\in\OO_X(V)$ (for a possibly different open subset $V$ containing $p$) such that $f|_{U\cap V}=g|_{U\cap V}$. The collection of equivalence classes is denoted $\OO_{X,p}$ and is called the stalk at $p$.
\end{definition}

\end{document}