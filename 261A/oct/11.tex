% !TEX root = ../notes.tex

\documentclass[../notes.tex]{subfiles}

\begin{document}

\section{October 11}
I missed class due to a minor leg injury. I thank Justin for access to his notes.

\subsection{Ideals and Commutants}
We will spend some time focusing on Lie algebras in their own right, instead of studying their representations. As such, we pick up where we left off in \cref{subsec:lie-subalg} on this topic. Here are a few ways to build Lie ideals.
\begin{lemma}
	Fix a Lie algebra $\mf g$, and let $\{I_\alpha\}_{\alpha\in\lambda}$ be a collection of Lie ideals. Then the sum
	\[\sum_{\alpha\in\lambda}I_\alpha\coloneqq\left\{\sum_{\alpha\in\lambda}X_\alpha:X_\alpha\in I_\alpha\text{ and }X_\alpha=0\text{ for all but finitely many }\alpha\in\lambda\right\}\]
	is also a Lie ideal.
\end{lemma}
\begin{proof}
	Linear algebra tells us that $\sum_{\alpha\in\lambda} I_\alpha$ is at least a subspace. Now, for any $X\in\mf g$ and $\sum_{\alpha\in\lambda} X_\alpha\in\sum_{\alpha\in\lambda} I_\alpha$, we see that
	\[\left[X,\sum_{\alpha\in\lambda}X_\alpha\right]=\sum_{\alpha\in\lambda}[X,X_\alpha],\]
	where we are allowed to use the bilinearity of the bracket here because $\sum_{\alpha\in\lambda}X_\alpha$ is actually a finite sum. Now, $[X,X_\alpha]\in I_\alpha$ for each $\alpha$ because $I_\alpha$ is a Lie ideal, so we conclude that $\left[X,\sum_{\alpha\in\lambda}X_\alpha\right]\in\sum_{\alpha\in\lambda}I_\alpha$.
\end{proof}
\begin{lemma} \label{lem:commutator-ideal}
	Fix Lie ideals $I$ and $J$ of a Lie algebra $\mf g$. Then
	\[[I,J]\coloneqq\op{span}\left\{[X,Y]:X\in I,Y\in J\right\}\]
	is a Lie ideal of $\mf g$. In fact, $[I,J]\subseteq I\cap J$.
\end{lemma}
\begin{proof}
	We have taken a span of some vectors, so $[I,J]$ is certainly a subspace. To check that it is a Lie ideal, we must check that $[W,Z]\in[I,J]$ for any $W\in\mf g$ and $Z\in[I,J]$. Well, $[W,-]\colon\mf g\to\mf g$ is a linear map, so the pre-image of $[I,J]$ is a linear subspace; to check that this pre-image contains $[I,J]$, it is thus enough to check that the pre-image contains a spanning subset, for which we use the elements of the form $[X,Y]$ where $X\in I$ and $Y\in J$. As such, we want to see $[W,[X,Y]]\in[I,J]$, for which we use the Jacobi identity to write
	\[[W,[X,Y]]=-[X,[Y,W]]-[Y,[W,X]]=-[X,[Y,W]]+[[W,X],Y].\]
	Now, $[Y,W]\in J$ because $J$ is a Lie ideal, and $[W,X]\in I$ because $I$ is a Lie ideal, so we see that $-[X,[Y,W]]+[[W,X],Y]\in[I,J]$.

	Lastly, to check that $[I,J]\subseteq I\cap J$, we note that the latter is a Lie ideal by \Cref{rem:intersect-lie-ideals} and hence a subspace, so it is enough to check the inclusion on a spanning subset of $[I,J]$, for which we take the elements of the form $[X,Y]=-[Y,X]$ for $X\in I$ and $Y\in J$. Well, $[X,Y]\in I$ and $-[Y,X]\in J$ because $I$ and $J$ are Lie ideals, so we are done.
\end{proof}
We will get a lot of utility out of the above lemma. For example, we can make the following definition.
\begin{definition}[commutator]
	The \textit{commutator} of a Lie algebra $\mf g$ is the Lie ideal $[\mf g,\mf g]$.
\end{definition}
\begin{lemma}
	Fix a Lie algebra $\mf g$. For any Lie ideal $I\subseteq\mf g$, the quotient $\mf g/I$ is abelian if and only if $I$ contains $[\mf g,\mf g]$. For example, $\mf g/[\mf g,\mf g]$ is abelian.
	\begin{listalph}
		\item The quotient $\mf g/[\mf g,\mf g]$ is an abelian Lie algebra.
		\item 
	\end{listalph}
\end{lemma}
\begin{proof}
	Note that the last sentence follows from the previous one (take $I=[\mf g,\mf g]$), so it only remains to prove the second sentence. We show the two implications separately.
	\begin{itemize}
		\item Suppose $I\subseteq\mf g$, and we will show $\mf g/I$ is abelian. Well, for any $X+I,Y+I\in\mf g/I$, we compute
		\[[X+I,Y+I] = [X,Y]+I,\]
		which we note equals $0+I$ because $[X,Y]\in[\mf g,\mf g]\subseteq I$.
		\item Suppose $\mf g/I$ is abelian, and we will show $[\mf g,\mf g]\subseteq I$. Well, $[\mf g,\mf g]$ and $I$ are both subspaces, so it is enough to check that a spanning subset of $[\mf g,\mf g]$ is contained in $I$, for which we use the elements of the form $[X,Y]$. Then we see that
		\[[X,Y]+I=[X+I,Y+I]=0+I,\]
		where the last equality holds because $\mf g/I$ is abelian. Thus, $[X,Y]\in I$, as required.
		\qedhere
	\end{itemize}
\end{proof}
\begin{exe}
	Fix a field $F$. Then $[\mf{gl}_n(F),\mf{gl}_n(F)]=\mf{sl}_n(F)$.
\end{exe}
\begin{proof}
	Here are our inclusions.
	\begin{itemize}
		\item To show $[\mf{gl}_n(F),\mf{gl}_n(F)]=\mf{sl}_n(F)$, it is enough to check that a spanning subset of $[\mf{gl}_n(F),\mf{gl}_n(F)]$ lives in $\mf{sl}_n(F)$, for which we note that elements of the form $[X,Y]$ have $\tr[X,Y]=\tr XY-\tr YX=0$ and thus $[X,Y]\in\mf{sl}_n(F)$.
		\item To show $\mf{sl}_n(F)\subseteq[\mf{gl}_n(F),\mf{gl}_n(F)]$, we should show that $[\mf{gl}_n(F),\mf{gl}_n(F)]$ contains a spanning subset of $\mf{sl}_n(F)$. Let $E_{ij}$ be the matrix with a $1$ in the $(i,j)$ component and a $0$ everywhere else. Then
		\[[E_{ij},E_{ji}]=E_{ij}E_{ji}-E_{ji}E_{ij}=E_{ii}-E_{jj},\]
		and
		\[[E_{ii}-E_{jj},E_{ij}]=E_{ii}E_{ij}-E_{jj}E_{ij}-E_{ij}E_{ii}+E_{ij}E_{jj}=E_{ij}-0-0+E_{ij}=2E_{ij}\]
		for any $i\ne j$. These elements span $\mf{sl}_n(F)$ (we have any off-diagonal entry, and a traceless diagonal matrix can be written as a sum of $(E_{ii}-E_{nn})$s), so we conclude.
		\qedhere
	\end{itemize}
\end{proof}
As with group theory, commutators allow us to define solvability.
\begin{definition}[derived series]
	Fix a Lie algebra $\mf g$. Then the \textit{derived series} is a sequence $\left\{D^i\mf g\right\}_{i\ge0}$ of Lie ideals defined inductively by $D^0\mf g\coloneqq\mf g$ and
	\[D^{i+1}\mf g\coloneqq\left[D^i\mf g,D^i\mf g\right].\]
\end{definition}
\begin{remark} \label{rem:derived-series-descends}
	Inductively applying \Cref{lem:commutator-ideal} shows that each $D^i\mf g$ is in fact a Lie ideal. Furthermore, we see \Cref{lem:commutator-ideal} implies each $i\ge0$ has
	\[D^{i+1}\mf g=\left[D^i\mf g,D^i\mf g\right]\subseteq D^i\mf g\cap D^i\mf g=D^i\mf g.\]
\end{remark}
\begin{lemma}
	Fix a Lie algebra $\mf g$. Then the following conditions are equivalent.
	\begin{listalph}
		\item $D^m\mf g=0$ for some $m$.
		\item $D^m\mf g=0$ for all sufficiently large $m$.
		\item There is a descending chain
		\[\mf g=\mf g^0\supseteq\mf g^1\supseteq\cdots\supseteq\mf g^m=0\]
		of Lie ideals such that $\mf g/\mf g_{i+1}$ is abelian.
	\end{listalph}
\end{lemma}
\begin{proof}
	We show our implications separately.
	\begin{itemize}
		\item Note that (a) implies (b) because $D^{i+1}\mf g\subseteq D^i\mf g$ for each $i\ge0$ (by \Cref{rem:derived-series-descends}), so $D^m\mf g=0$ implies (inductively) that $D^{m+i}\mf g=0$ for all $i\ge0$.
		\item We show that (b) implies (c) by taking $\mf g^i\coloneqq D^i\mf g$ for all $i\in\{0,\ldots,m\}$ for some sufficiently large $m$. This is a descending chain of Lie ideals by \Cref{rem:derived-series-descends}, and it has $\mf g^m=0$ by hypothesis. Lastly, we note that
		\[\mf g^i/\mf g^{i+1}=\mf g^i/\left[\mf g^i,\mf g^i\right]\]
		is abelian by \Cref{lem:commutator-ideal}.
		\item We show that (c) implies (a). We claim that $D^i\mf g\subseteq\mf g^i$ for each $i\in\{0,\ldots,m\}$, which will imply that $D^m\mf g=0$ and thus complete the proof. Well, we show the claim by induction, for which the base case of $i=0$ has nothing to show. For the inductive step, we note that $\mf g^i/\mf g^{i+1}$ being abelian implies that
		\[\mf g^{i+1}\subseteq\left[\mf g^i,\mf g^i\right].\]
		The spanning subset of commutators defining $\left[\mf g^i,\mf g^i\right]$ is a subset of those defining $D^{i+1}\mf g=\left[D^i\mf g,D^i\mf g\right]$ by the inductive hypothesis, so we conclude that $\mf g^{i+1}\subseteq D^{i+1}\mf g$.
		\qedhere
	\end{itemize}
\end{proof}
\begin{definition}[solvable]
	A Lie algebra $\mf g$ is \textit{solvable} if and only if 
\end{definition}
\begin{remark}
	One can see that $\mf g$ is solvable if and only if $\mf g\otimes_F\ov F$ is solvable.
\end{remark}
One can check that sums of ideals are ideals. Also, one sees that $\mf g/[\mf g,\mf g]$ is the maximal abelian quotient: if $I\subseteq\mf g$ is an ideal with $\mf g/I$ abelian, then we must have $[\mf g,\mf g]\subseteq I$.
\begin{example}
	One can check that $[\mf{gl}_n,\mf{gl}_n]\subseteq\mf{sl}_n$ because the trace of $XY-YX$ is zero for any $X,Y\in\mf{gl}_n$. In fact, this is an equality, which one can check by hand.
\end{example}
These commutants provide a derived series: we define $\left\{D^i\mf g\right\}_{i\ge0}$ inductively by $D^0\mf g\coloneqq\mf g$ and
\[D^{i+1}\mf g\coloneqq\left[D^i\mf g,D^i\mf g\right]\]
for all $i\ge0$. This derived series plays the role of derived series in group theory. For example, one can use this to define solvability.
\begin{proposition}
	Fix a Lie algebra $\mf g$. Then the following are equivalent.
	\begin{listalph}
		\item $D^n\mf g=0$ for $n$ sufficiently large.
		\item There exists a sequence of subalgebras
		\[\mf g=\mf a^0\supseteq\mf a^1\supseteq\cdots\supseteq\mf a^k=0\]
		such that $\mf a^{i+1}$ is an ideal in $\mf a^i$ with abelian quotient.
		\item For every $n$ sufficiently large and sequence of elements $\{x_1,\ldots,x_{2^n}\}\subseteq\mf g$, the $n$-fold commutator
		\[[\cdots[[x_1,x_2],[x_3,x_4]],\cdots]\]
		vanishes.
	\end{listalph}
\end{proposition}
\begin{proof}
	The equivalence of (a) and (c) has no content. Note that (a) implies (b) because one may take $\mf a^i=D^i\mf g$. One achieves (b) implies (a) by showing that $\mf a^i\supseteq D^i\mf g$ inductively.
\end{proof}

\end{document}