% !TEX root = ../notes.tex

\documentclass[../notes.tex]{subfiles}

\begin{document}

\section{October 11}
I missed class due to a minor leg injury. I thank Justin for access to his notes.

\subsection{Ideals and Commutants}
One can check that sums of ideals are ideals. Also, one sees that $\mf g/[\mf g,\mf g]$ is the maximal abelian quotient: if $I\subseteq\mf g$ is an ideal with $\mf g/I$ abelian, then we must have $[\mf g,\mf g]\subseteq I$.
\begin{example}
	One can check that $[\mf{gl}_n,\mf{gl}_n]\subseteq\mf{sl}_n$ because the trace of $XY-YX$ is zero for any $X,Y\in\mf{gl}_n$. In fact, this is an equality, which one can check by hand.
\end{example}
These commutants provide a derived series: we define $\left\{D^i\mf g\right\}_{i\ge0}$ inductively by $D^0\mf g\coloneqq\mf g$ and
\[D^{i+1}\mf g\coloneqq\left[D^i\mf g,D^i\mf g\right]\]
for all $i\ge0$. This derived series plays the role of derived series in group theory. For example, one can use this to define solvability.
\begin{proposition}
	Fix a Lie algebra $\mf g$. Then the following are equivalent.
	\begin{listalph}
		\item $D^n\mf g=0$ for $n$ sufficiently large.
		\item There exists a sequence of subalgebras
		\[\mf g=\mf a^0\supseteq\mf a^1\supseteq\cdots\supseteq\mf a^k=0\]
		such that $\mf a^{i+1}$ is an ideal in $\mf a^i$ with abelian quotient.
		\item For every $n$ sufficiently large and sequence of elements $\{x_1,\ldots,x_{2^n}\}\subseteq\mf g$, the $n$-fold commutator
		\[[\cdots[[x_1,x_2],[x_3,x_4]],\cdots]\]
		vanishes.
	\end{listalph}
\end{proposition}
\begin{proof}
	The equivalence of (a) and (c) has no content. Note that (a) implies (b) because one may take $\mf a^i=D^i\mf g$. One achieves (b) implies (a) by showing that $\mf a^i\supseteq D^i\mf g$ inductively.
\end{proof}

\end{document}