% !TEX root = ../notes.tex

\documentclass[../notes.tex]{subfiles}

\begin{document}

\section{October 5}
Today we classify the representations of $\mf{sl}_2(\CC)$.

\subsection{Some Representations of \texorpdfstring{$\mf{sl}_2(\CC)$}{sl2(C)}}
% Note that \Cref{ex:sl-reduces} tells us that all representations are completely reducible, but we will provide a purely algebraic proof of this (namely, avoiding the integration theory presented in that subsection).
%
We begin by choosing a basis of $\mf{sl}_2(\CC)$. By \Cref{ex:sl}, this Lie algebra consists of the traceless $2\times2$ matrices and is three-dimensional, so we define
\[e\coloneqq\begin{bmatrix}
	0 & 1 \\ 0 & 0
\end{bmatrix},\qquad f\coloneqq\begin{bmatrix}
	0 & 0 \\
	1 & 0
\end{bmatrix},\qquad\text{and}\qquad h\coloneqq\begin{bmatrix}
	1 & 0 \\
	0 & -1
\end{bmatrix}.\]
Note that $\{e,f,h\}\subseteq\mf{sl}_2(\CC)$ are certainly linearly independent because they occupy disjoint coordinates of these matrices, so we have found a basis. For future use, it will be useful to record the commutator relations
\begin{align*}
	[e,f] &= ef-fe = \begin{bmatrix}
		0 & 1 \\
		0 & 0
	\end{bmatrix}\begin{bmatrix}
		0 & 0 \\
		1 & 0
	\end{bmatrix}-\begin{bmatrix}
		0 & 0 \\
		1 & 0
	\end{bmatrix}\begin{bmatrix}
		0 & 1 \\
		0 & 0
	\end{bmatrix}=\begin{bmatrix}
		1 & 0 \\ 0 & -1
	\end{bmatrix}=h, \\
	[h,e] &= he-eh = \begin{bmatrix}
		1 & 0 \\
		0 & -1
	\end{bmatrix}\begin{bmatrix}
		0 & 1 \\
		0 & 0
	\end{bmatrix}-\begin{bmatrix}
		0 & 1 \\
		0 & 0
	\end{bmatrix}\begin{bmatrix}
		1 & 0 \\
		0 & -1
	\end{bmatrix} = \begin{bmatrix}
		0 & 2 \\
		0 & 0
	\end{bmatrix} = 2e, \\
	[h,f] &= hf-fh = (eh-he)^\intercal = -[h,e]^\intercal = -2f.
\end{align*}
These commutator relations can be frequently be extending inductively. Here are a couple examples.
\begin{example} \label{ex:sl2-hf-power}
	As an example of something we can compute, suppose $\rho\colon\mf{sl}_2(\CC)\to\mf{gl}(V)$ is some representation. Then we claim that
	\[\rho(h)\circ\rho(f)^m=\rho(f)^m\circ(\rho(h)-2m{\id_V})\]
	for any $m\ge0$. Indeed, for $m=0$, there is nothing to say. For an inductive step, we note
	\[\rho(h)\rho(f)^{m+1}=\rho(h)\rho(f)^m\rho(f)=\rho(f)^m\circ(\rho(h)-2m{\id_V})\rho(f)=\rho(f)^{m+1}(\rho(h)-2(m+1){\id_V}),\]
	where the key point is that $\rho(h)\rho(f)=\rho(f)(\rho(h)-2{\id_V})$ by the commutator relations.
\end{example}
\begin{example}
	Replacing $f$ with $e$ everywhere in \Cref{ex:sl2-hf-power} proves that
	\[\rho(h)\circ\rho(f)^m=\rho(f)^m\circ(\rho(h)+2m{\id_V}),\]
	where the point is that $\rho(h)\rho(e)=\rho(e)(\rho(h)+2{\id_V})$ by the commutator relations.
\end{example}
We are going to find many irreducible representations of $\mf{sl}_2(\CC)$. We will do this fairly geometrically in two ways, starting with $\op{GL}_2(\CC)$. On one hand, we can start with the standard representation $\rho_{\mathrm{std}}\colon\op{SL}_2(\CC)\subseteq\op{GL}_2(\CC)$ and then define $\rho_n\coloneqq\op{Sym}^n\rho_{\mathrm{std}}$ for all $n\ge0$. On the other hand, we can provide a more geometric construction: 
Note that $\op{SL}_2(\CC)$ acts on polynomials $\CC[x,y]$ by $(\rho(g)p)(x,y)\coloneqq p\left((x,y)g\right)$. Quickly, we note that this is at least a linear action.
\begin{itemize}
	\item Identity: $(\rho(1)p)(x,y)=p(x,y)$.
	\item Associative: $(\rho(gh)p)(x,y)=p\left((x,y)gh\right)=(\rho(g)\rho(h)p)(x,y)$.
	\item Linear: note $\rho(g)(ap+bq)(x,y)=ap\left((x,y)g\right)+bq\left((x,y)g\right)=(a\rho(g)p+b\rho(g)q)(x,y)$.
\end{itemize}
Note that $\CC[x,y]$ is a ring graded by (total) degree, so we let $V_n\subseteq\CC[x,y]$ be the subspace of $\CC[x,y]$ spanned by the degree-$n$ monomials. (Namely, $V_n$ consists of $0$ and the homogeneous polynomials of degree $n$.) Then we note that
\[\rho\left(\begin{bmatrix}
	a & b \\ c & d
\end{bmatrix}\right)\left(x^pq^p\right)=(ax+cy)^p(bx+dy)^q\]
continues to homogeneous of degree $p+q$, so the $\op{SL}_2(\CC)$-action on $\CC[x,y]$ stabilizes the subspaces $V_n$. Thus, we have produced representations $\rho_n'\colon\op{SL}_2(\CC)\to\op{GL}(V_n)$. Giving $V_n$ the basis of monomials and expanding about the above formula $(ax+cy)^p(bx+dy)^q$, we see that the matrix coefficients of $\rho_n'(g)$ are polynomial in the matrix coefficients of $g$, so $\rho_n'(g)$ is indeed a regular representation.
\begin{lemma} \label{lem:sl2-poly-is-sym-power}
	Fix notation as above, and let $\rho_{\mathrm{std}}\colon\op{SL}_2(\CC)\subseteq\op{GL}_2(\CC)$. Then $\rho_n'\cong\op{Sym}^n\rho_{\mathrm{std}}$ for all $n\ge0$.
\end{lemma}
\begin{proof}
	Let $\{v_x,v_y\}$ be the standard basis of $\op{SL}_2(\CC)$. Then we note that $\rho_n'$ has basis given by the monomials $\left\{x^py^q\right\}_{p+q=n}$, and $\op{Sym}^n\rho_{\mathrm{std}}$ has basis given by $\left\{v_x^{\otimes p}v_y^{\otimes q}\right\}_{p+q=n}$ (by some linear algebra).\footnote{Here, $v_x^{\otimes p}v_y^{\otimes q}$ is the average over all permutations of the vector $v_x^{\otimes p}\otimes v_y^{\otimes q}$. Certainly the permutations of these vectors provide a basis of $\left(\CC^2\right)^{\otimes n}$, and taking averages over all permutations of the basis shows that we have in fact given a basis.}
	
	With this in mind, we define a vector space isomorphism $\varphi\colon\op{Sym}^n\CC^2\to V_n$ by $\varphi\colon v_x^pv_y^q\mapsto x^py^q$. To check that this is $\op{SL}_2(\CC)$-invariant, it is enough to check on a basis because $G$-invariance is a linear condition, so we note that $g\coloneqq\begin{bsmallmatrix}
		a & b \\ c & d
	\end{bsmallmatrix}$ has
	\begin{align*}
		\varphi\left(g\cdot v_x^pv_y^q\right) &= \varphi\left((av_x+cv_y)^p(bv_x+dv_y)^q\right) \\
		&= (ax+by)^p(bx+dy)^q \\
		&= g\cdot x^py^q \\
		&= g\cdot\varphi\left(v_x^pv_y^q\right),
	\end{align*}
	as required.
\end{proof}
Some $V_\bullet$s already have geometric incarnations.
\begin{example}
	We see that $V_0$ is the trivial representation, and $V_1$ is the standard representation.
\end{example}
\begin{lemma}
	The representations $\rho_2\colon\mf{sl}_2(\CC)\to\mf{gl}(V_2)$ and $\op{ad}_\bullet\colon\mf{sl}_2(\CC)\to\mf{gl}(\mf{sl}_2(\CC))$ are isomorphic.
\end{lemma}
\begin{proof}
	We will construct an explicit isomorphism $\mf{sl}_2(\CC)\to V_2$ of representations. For this, we compare the actions of $\rho_2\colon\mf{sl}_2(\CC)\to\mf{gl}(V_2)$ and $\op{ad}\colon\mf{sl}_2(\CC)\to\mf{gl}(\mf{sl}_2(\CC))$.
	\begin{itemize}
		\item We compute $\rho_2\colon\mf{sl}_2(\CC)\to\mf{gl}(V_2)$. Note that $\mf{sl}_2(\CC)$ acts component-wise on $V_2=\op{Sym}^2V\subseteq V\otimes V$. Thus, we see that
		\[X(vw)=X\cdot\frac12\left(v\otimes w+w\otimes v\right)=\frac12(Xv\otimes w+v\otimes W+Xw\otimes v+w\otimes Xv)=(Xv)w+v(Xw).\]
		In particular, using the ordered basis $\left\{v_x^2,v_y^2,v_xv_y\right\}$, we compute
		\begin{align*}
			e\left(v_x^2\right) &= 2v_x\cdot e(v_x) = 0, \\
			e\left(v_y^2\right) &= 2v_y\cdot e(v_y) = 2v_xv_y, \\
			e(v_xv_y) &= v_x\cdot e(v_y)+e(v_x)\cdot v_y=v_x^2,
		\end{align*}
		and
		\begin{align*}
			f\left(v_x^2\right) &= 2v_x\cdot f(v_x) = 2v_xv_y, \\
			f\left(v_y^2\right) &= 2v_y\cdot f(v_y) = 0, \\
			f(v_xv_y) &= v_x\cdot f(v_y)+f(v_x)\cdot v_y = v_y^2,
		\end{align*}
		and
		\begin{align*}
			h\left(v_x^2\right) &= 2v_x\cdot h(v_x) = 2v_x^2, \\
			h\left(v_y^2\right) &= 2v_y\cdot h(v_y) = -2v_y^2, \\
			h(v_xv_y) &= v_x\cdot h(v_y)+h(v_x)\cdot v_y = 0
		\end{align*}
		Thus,
		\[\rho_2(e)=\begin{bmatrix}
			0 & 0 & 1 \\
			0 & 0 & 0 \\
			0 & 2 & 0
		\end{bmatrix},\qquad\rho_2(f)=\begin{bmatrix}
			0 & 0 & 0 \\
			0 & 0 & 1 \\
			2 & 0 & 0
		\end{bmatrix},\qquad\text{and}\qquad\rho_2(h)=\begin{bmatrix}
			2 & 0 & 0 \\
			0 & -2 & 0 \\
			0 & 0 & 0
		\end{bmatrix}.\]
		\item We compute $\op{ad}\colon\mf{sl}_2(\CC)\to\mf{gl}(\mf{sl}_2(\CC))$, where we recall that $\op{ad}_X(Y)=[X,Y]$. Using the commutator relations already computed in the previous homework, we use the ordered basis $\{e,f,h\}$ to see that
		\[{\op{ad}_e}=\begin{bmatrix}
			0 & 0 & -2 \\
			0 & 0 & 0 \\
			0 & 1 & 0
		\end{bmatrix},\qquad{\op{ad}_f}=\begin{bmatrix}
			0 & 0 & 0 \\
			0 & 0 & 2 \\
			-1 & 0 & 0
		\end{bmatrix},\qquad\text{and}\qquad{\op{ad}_h}=\begin{bmatrix}
			2 & 0 & 0 \\
			0 & -2 & 0 \\
			0 & 0 & 0
		\end{bmatrix}.\]
	\end{itemize}
	With this in mind, we define $\varphi\colon V_2\to\mf{sl}_2(\CC)$ by $\varphi\left(v_x^2\right)\coloneqq-2e$ and $\varphi\left(v_y^2\right)=2f$ and $\varphi\left(v_xv_y\right)=h$. Then we want to check that $\varphi(\rho_2(X)v)=\op{ad}_X\varphi(v)$ for all $X\in\mf{sl}_2(\CC)$ and $v\in V_2$ This condition is linear in $X$ and $v$, so we can check it on bases, where we see that we are asking to check that $\varphi\circ\rho_2(X)\circ\varphi^{-1}={\op{ad}_X}$ for any $X$. However, translating everything into matrices, we see that this comes down to the computations
	\begin{align*}
		\begin{bmatrix}
			-2 \\ & 2 \\ & & 1
		\end{bmatrix}\begin{bmatrix}
			0 & 0 & 1 \\
			0 & 0 & 0 \\
			0 & 2 & 0
		\end{bmatrix}\begin{bmatrix}
			-2 \\ & 2 \\ & & 1
		\end{bmatrix}^{-1} &= \begin{bmatrix}
			0 & 0 & -2 \\
			0 & 0 & 0 \\
			0 & 1 & 0
		\end{bmatrix}, \\
		\begin{bmatrix}
			-2 \\ & 2 \\ & & 1
		\end{bmatrix}\begin{bmatrix}
			0 & 0 & 0 \\
			0 & 0 & 1 \\
			2 & 0 & 0
		\end{bmatrix}\begin{bmatrix}
			-2 \\ & 2 \\ & & 1
		\end{bmatrix}^{-1} &= \begin{bmatrix}
			0 & 0 & 0 \\
			0 & 0 & 2 \\
			-1 & 0 & 0
		\end{bmatrix}, \\
		\begin{bmatrix}
			-2 \\ & 2 \\ & & 1
		\end{bmatrix}\begin{bmatrix}
			2 & 0 & 0 \\
			0 & -2 & 0 \\
			0 & 0 & 0
		\end{bmatrix}\begin{bmatrix}
			-2 \\ & 2 \\ & & 1
		\end{bmatrix}^{-1} &= \begin{bmatrix}
			2 & 0 & 0 \\
			0 & -2 & 0 \\
			0 & 0 & 0
		\end{bmatrix},
	\end{align*}
	as required.
\end{proof}
Thus, for the rest of today, we will just work with $\rho_n'$s due to its more concrete description, but we will identify the notations $\rho_n$ and $\rho_n'$. We now induce a representation $(d\rho_n)_1\colon\mf{sl}_2(\CC)\to\mf{gl}(V_n)$. We quickly run a few computations; here and throughout the rest of the computations, if we ever find negative exponents, then the relevant expression actually vanishes.
\begin{itemize}
	\item We claim that $(d\rho_n)_1(e)=x\del_y$ as operators on $V_n$. It is enough to check the equality on the basis of monomials of $V_n$, so we choose $(p,q)$ such that $p+q=n$. Then we note $\exp(te)=\sum_{k=0}^\infty\frac1{n!}t^ne^n=\begin{bsmallmatrix}
		1 & t \\ & 1
	\end{bsmallmatrix}$ because $e^2=0$, so we may compute
	\begin{align*}
		(d\rho_n)_1(e)\left(x^py^q\right) &= \frac d{dt}\rho_n(\exp(te))\left(x^py^q\right)\bigg|_{t=0} \\
		&= \frac d{dt}\begin{bmatrix}
			1 & t \\ & 1
		\end{bmatrix}\left(x^py^q\right)\bigg|_{t=0} \\
		&= \frac d{dt}x^p(tx+y)^q\bigg|_{t=0} \\
		&= qx^{p+1}(tx+y)^{q-1}\bigg|_{t=0} \\
		&= qx^{p+1}y^{q-1} \\
		&= x\del_x\left(x^py^q\right).
	\end{align*}
	\item We claim that $(d\rho_n)_1(f)=y\del_x$. Indeed, this follows by switching the roles of $x$ and $y$ everywhere in the previous computation, which effectively exchanges the matrices $e$ and $f$ by interchanging the ordered basis.
	\item We claim that $h=x\del_x-y\del_y$ as operators on $V_n$. Again, it is enough to check the equality on the basis of monomials of $V_n$, so we choose $(p,q)$ such that $p+q=n$. Then we note that
	\[\exp(th)=\sum_{k=0}^\infty\frac1{n!}t^nh^n=\sum_{k=0}^\infty\frac1{n!}\begin{bmatrix}
		t^n \\ & (-t)^n
	\end{bmatrix}=\begin{bmatrix}
		e^t \\ & e^{-t}
	\end{bmatrix},\]
	so we may compute
	\begin{align*}
		(d\rho_n)_1(h)\left(x^py^q\right) &= \frac d{dt}\rho_n(\exp(th))\left(x^py^q\right)\bigg|_{t=0} \\
		&= \frac d{dt}\begin{bmatrix}
			e^t \\ & e^{-t}
		\end{bmatrix}\left(x^py^q\right)\bigg|_{t=0} \\
		&= \frac d{dt}\left(e^tx\right)^p\left(e^{-t}y\right)^q\bigg|_{t=0} \\
		&= \frac d{dt}e^{(p-q)t}x^py^q\bigg|_{t=0} \\
		&= (p-q)x^py^q \\
		&= (x\del_x-y\del_y)\left(x^py^q\right).
	\end{align*}
\end{itemize}
The rest of the argument forgets about $\op{SL}_2(\CC)$, so we will replace the notation $(d\rho_n)_1$ with just $d\rho_n\colon\mf{sl}_2(\CC)\to\mf{gl}(V_n)$. For brevity, we also let $v_{pq}$ be the basis monomial $x^py^q\in\CC[x,y]$, for any $p,q\ge0$. In particular, the above computations have found that
\begin{equation}
	\begin{cases}
		\rho_n(e)v_{pq}=qv_{p+1,q-1}, \\
		\rho_n(f)v_{pq}=pv_{p-1,q+1}, \\
		\rho_n(h)v_{pq}=(p-q)v_{pq}.
	\end{cases} \label{eq:sl2-poly-action}
\end{equation}
In particular, $\rho_n(h)$ acts diagonally on $V_n$ with eigenbasis given by the monomials. We now begin running some checks.
\begin{lemma} \label{lem:sl2-get-irreps}
	Fix notation as above. Then the representation $\rho_n\colon\mf{sl}_2(\CC)\to\mf{gl}(V_n)$ is irreducible.
\end{lemma}
\begin{proof}
	Fix some nonzero subrepresentation $W\subseteq V_n$, and we want to show that $W=V_n$. Well, $W$ must have some nonzero vector $v\coloneqq\sum_{p+q=n}a_{pq}v_{pq}$. Supposing that $p'$ is the largest index for $p$ such that $a_{pq}\ne0$, we see that applying $e$ enough times to $v$ allows us to assume that $v$ is a scalar multiple of $v_{n0}$. (In particular, $\rho_n(e)^{p'}v$ is a scalar multiple of $v_{n0}$.) Thus, $v_{n0}\in W$. But then we can apply $f$ inductively to see that
	\[\rho_n(f)^kv_{n0}=n\rho_n(f)^{k-1}v_{n-1,1}=\cdots=n(n-1)\cdots(n-k+1)v_{n-k,k},\]
	so we see that $v_{n-k,k}\in W$ for each $k\ge0$. Thus, $W$ contains the monomial basis of $V_n$, so $W=V_n$.
\end{proof}

\subsection{Irreducible Representations of \texorpdfstring{$\mf{sl}_2(\CC)$}{sl2(C)}}
We now classify the irreducible representations of $\mf{sl}_2(\CC)$. An interesting property of the above computations, and one used in the above argument, is that $\rho_n(e)$ and $\rho_n(f)$ are nilpotent operators on $V_n$. We will eventually show that $\{\rho_n\}_{n\ge0}$ lists all irreducible representations of $\mf{sl}_2(\CC)$, so we expect this property to be true in general. Our next step is to prove it.
\begin{lemma} \label{lem:sl2-e-f-nilpotent}
	Let $\rho\colon\mf{sl}_2(\CC)\to\mf{gl}(V)$ be a complex representation. Then $\rho(e)$ and $\rho(f)$ are nilpotent operators on $V$.
\end{lemma}
\begin{proof}
	Note that \eqref{eq:sl2-poly-action} tells us that we may hope to extract monomials as eigenvectors of $h$. Thus, we employ the following trick: we let $\sigma(h)$ denote the collection of eigenvalues of $h$ (which is finite because $\dim V<\infty$), and then we let $V[\lambda]$ be the generalized eigenspace for $\rho(h)$ of the eigenvalue $\lambda\in\sigma(h)$. Thus, we have a decomposition
	\[V=\bigoplus_{\lambda\in\sigma(h)}V[\lambda].\]
	Now, the commutator relations imply that $\rho(h)\rho(e)=\rho(e)\rho(h)+2\rho(e)$ and $\rho(h)\rho(f)=\rho(f)\rho(h)-2\rho(f)$, so
	\[(\rho(h)-(\lambda+2))^d\circ\rho(e)=\rho(e)\circ(\rho(h)-\lambda)^n\qquad\text{and}\qquad(\rho(h)-(\lambda-2))^d\circ\rho(f)=\rho(f)\circ(\rho(h)-\lambda)^n\]
	for all $d\ge0$ and $\lambda\in\CC$. Thus, taking $d$ large enough, we see that $\rho(e)\colon V[\lambda]\to V[\lambda+2]$ and $\rho(f)\colon V[\lambda]\to V[\lambda-2]$. Because $V[\lambda]\ne0$ for only finitely many $\lambda$s, we see that $\rho(e)^n$ and $\rho(f)^n$ must be zero for $n$ large enough.  For example, any $n>\left|\sigma(h)\right|$ will do because then any $\lambda\in\sigma(h)$ and hence $V[\lambda]=0$ will have $\lambda+2k\notin\sigma(h)$ for some $k\in\{0,\ldots,n\}$, implying that $\rho(e)^n\colon V[\lambda]\to V[\lambda+2n]$ vanishes for all $\lambda$. (A similar argument with $+$ replaced by $-$ shows that $\rho(f)^n$ vanishes.)
\end{proof}
For technical reasons, we note that the action of $h$ on the $V_n$s is particularly simple: it diagonalizes. Some algebra with the commutator relations is able to show this in a special case.
\begin{lemma} \label{lem:sl2-h-diagonalizes}
	Let $\rho\colon\mf{sl}_2(\CC)\to\mf{gl}(V)$ be a complex representation, and set $U\coloneqq\ker\rho(e)$. Then $\rho(h)$ preserves $U$, and $\rho(h)\colon U\to U$ diagonalizes with nonnegative integer eigenvalues.
\end{lemma}
\begin{proof}
	We will omit $\rho$ from our actions for brevity. We proceed in steps. Throughout, we fix some $v\in U$.
	\begin{enumerate}
		\item We quickly check that $h(U)\subseteq U$. Indeed, for $v\in U$, we see that $ehv=(he-2e)v=(h-2)ev=0$, so $hv\in U$.
		\item For any $m\ge1$, we claim that
		\begin{equation}
			ef^mv\stackrel?=f^{m-1}m(h-m+1)v. \label{eq:sl2-e-f-power}
		\end{equation}
		Indeed, for $m=1$, we recall that $ef=fe+h$, so $ev=0$ proves the conclusion. For the inductive step, we take $m\ge1$ and note
		\begin{align*}
			ef^{m+1}v &= (fe+h)f^mv \\
			&= \left(fef^m+hf^m\right)v \\
			&\stackrel*= \left(f^mm(h-m+1)+f^m(h-2m)\right)v \\
			&= f^m\left((m+1)h-m^2-m\right)v \\
			&= f^m(m+1)(h-m)v,
		\end{align*}
		where $\stackrel*=$ holds by \Cref{ex:sl2-hf-power}.
		\item Using the previous step, we conclude that
		\[e^mf^mv=e^{m-1}f^{m-1}m(h-(m-1))v.\]
		Because $h$ preserves $U$, we see that $m(h-m+1)v\in U$ as well. Thus, we may apply the previous step inductively to see that $m\ge0$ has
		\[e^mf^mv\stackrel?=m!h(h-1)\cdots(h-(m-1))v.\]
		Indeed, $m=0$ has nothing to prove, and for the inductive step, we simply use the previous step.
		\item We complete the proof. By \Cref{lem:sl2-e-f-nilpotent}, we know that there exists $m\ge0$ such that $\rho(h)^m\colon V\to V$ is the zero operator. For this $m$, we see
		\[0=e^mf^mv=m!h(h-1)\cdots(h-(m-1))v\]
		for all $v\in V$. Thus, the minimal polynomial of $h$ divides $T(T-1)\cdots(T-(m-1))$ and in particular has no repeated roots, so linear algebra implies that $h$ acts diagonally with eigenvalues in $\{0,1,\ldots,m-1\}$.
		\qedhere
	\end{enumerate}
\end{proof}
% \begin{remark}
% 	In fact, the proof of \Cref{lem:sl2-h-diagonalizes} tells us that the eigenvalues of $\rho(h)$ are nonnegative integers, but we will not need this.
% \end{remark}
We now use the eigenvectors we have access to build some subrepresentations by hand.
\begin{lemma} \label{lem:sl2-map-from-sym-power}
	Fix a nonzero complex representation $\rho\colon\mf{sl}_2(\CC)\to\mf{gl}(V)$. Suppose that there is a nonzero eigenvector $v\in\ker\rho(e)$ for $\rho(h)$ with eigenvalue $n\in\ZZ_{\ge0}$. Then there exists an embedding $\varphi\colon V_n\to V$ of representations such that $\varphi\left(x^n\right)=v$.
\end{lemma}
\begin{proof}
	We proceed in steps.
	\begin{enumerate}
		\item %Set $U\coloneqq\ker\rho(e)$, which is nonzero because $\rho(e)\colon V\to V$ is nilpotent by \Cref{lem:sl2-e-f-nilpotent}. Let $v\in U$ be a nonzero eigenvector of $\rho(h)$ with eigenvalue $n\in\ZZ_{\ge0}$ (using \Cref{lem:sl2-h-diagonalizes}).
		We begin by making a motivational remark. With $\varphi$ as a guide and staring at \eqref{eq:sl2-poly-action}, we expect $v$ to be a monomial of the form $x^{n+q}y^q$ because it is an eigenvector for $\rho(h)$ with eigenvalue $n$, and $\rho(e)v=0$ suggests that we should have $\varphi(v)=x^n$.

		\item We now find other monomials. Namely, note $\op{span}\{v\}$ is not yet stable under the action of $\mf{sl}_2(\CC)$ because $v$ need not be an eigenvector for $\rho(f)$. Thus, we define
		\[v_q\coloneqq\rho(f)^qv\]
		for $q\ge0$ and $v_{-1}\coloneqq0$, which \eqref{eq:sl2-poly-action} suggests should behave like our monomials with $\varphi(v_q)=n(n-1)\cdots(n-q+1)x^{p-q}y^q$. Indeed, for $q\ge0$, we have the relations
		\[\begin{cases}
			\rho(e)v_q=q(n-q+1)v_{q-1}, \\
			\rho(f)v_q=v_{q+1}, \\
			\rho(h)v_q=(n-2q)v_q.
		\end{cases}\]
		Here, the relation for $\rho(e)$ follows from \eqref{eq:sl2-e-f-power}, and the relation for $\rho(h)$ follows from \Cref{ex:sl2-hf-power}. In particular, we see that $\rho(e)v_{n+1}=0$, so $v_{n+1}\in U$, but then $\rho(h)$ acts on $v_{n+1}$ with negative eigenvalue $-2$, so $v_{n+1}=0$ is forced by \Cref{lem:sl2-h-diagonalizes}; then the $\rho(f)$ relation gives $v_q=0$ for $q>n$.

		\item We construct the map $\varphi$. For notational ease, we begin by fixing our collection of monomials by defining
		\[w_q\coloneqq\frac1{n(n-1)\cdots(n-q+1)}v_q\]
		for $q\in\{0,\ldots,n\}$ and $v_q=0$ for $q\in\ZZ\setminus\{0,\ldots,n\}$, where now we expect $\varphi(w_q)=x^{n-q}y^q$. Indeed, for $q\in\{0,\ldots,n\}$, we have the relations, we have the relations
		\[\begin{cases}
			\rho(e)w_q=qw_{q-1}, \\
			\rho(f)w_q=(n-q)w_{q+1}, \\
			\rho(h)w_q=(n-2q)w_q.
		\end{cases}\]
		We now may compare the above relations with \eqref{eq:sl2-poly-action} to see that $\varphi\colon V_n\to V$ defined by $\varphi\left(x^py^q\right)\coloneqq w_q$ preserves the $\mf{sl}_2(\CC)$-action. Indeed, we want to check that $\varphi(\rho_n(X)v)=\rho(X)(\varphi(v))$ for any $X\in\mf{sl}_2(\CC)$ and $v\in V_n$, for which it suffices to check on the bases $\{e,f,h\}\subseteq\mf{sl}_2(\CC)$ and $\left\{x^py^q\right\}_{p+q=n}\subseteq V_n$; these checks are immediate by comparing the above relations with \eqref{eq:sl2-poly-action}.

		\item It remains to check that $\varphi$ is an embedding. We provide two ways of doing this.
		\begin{itemize}
			\item Note that $\{w_0,\ldots,w_n\}\subseteq V_n$ is a linearly independent set because these vectors have distinct eigenvalues for $\rho(h)$. Thus, $\varphi$ sends a basis to a linearly independent subset of $V_n$ and hence must be injective.
			\item Note $\varphi$ is nonzero because $\varphi\left(x^n\right)=w_0=v$ is nonzero. Because $V_n$ is irreducible by \Cref{lem:sl2-get-irreps}, we conclude that $\varphi$ is injective by \Cref{prop:schur-lemma}.
			\qedhere
		\end{itemize}
	\end{enumerate}
\end{proof}
At long last, here is our classification result.
\begin{theorem} \label{thm:sl2-classify-irreps}
	Let $V_1$ be the standard representation $\rho_1\colon\mf{sl}_2(\CC)\subseteq\mf{gl}_2(\CC)$, and define $V_n\coloneqq\op{Sym}^nV_1$ for each $n\ge0$ so that we have representations $\rho_n\colon\mf{sl}_2(\CC)\to\mf{gl}(V_n)$. Then
	\[\{V_n:n\ge0\}\]
	is exactly the set of irreducible representations of $\mf{sl}_2(\CC)$, and these are all distinct.
\end{theorem}
\begin{proof}
	By \Cref{lem:sl2-get-irreps}, we see that the $V_\bullet$s are irreducible, and they are all distinct because their dimensions are all distinct: $\dim V_n=n+1$.
	
	It remains to check that these are the only irreducible representations. Well, pick up some irreducible representation $\rho\colon\mf{sl}_2(\CC)\to\mf{gl}(V)$. Now, we use \Cref{lem:sl2-map-from-sym-power} to get some $n\in\ZZ_{\ge0}$ and a nonzero map $\varphi\colon V_n\to V$ of representations. (Note that the existence of the required $v$ is satisfied by \Cref{lem:sl2-h-diagonalizes}.) Because $V_n$ and $V$ are both irreducible, \Cref{prop:schur-lemma} implies that $\varphi$ is an isomorphism, so $V\cong V_n$ is one of the $V_\bullet$s.
\end{proof}

\subsection{Complete Reducibility for \texorpdfstring{$\mf{sl}_2(\CC)$}{sl2(C)}}
In this subsection, we provide a purely algebraic proof for the complete reducibility of representations of $\mf{sl}_2(\CC)$. Namely, we are avoiding the integration theory used in \Cref{ex:sl-reduces}. Technically, this subsection can thus be skipped, but it is instructive because the methods used in this subsection will reappear when we want to show that the complex representations of general semisimple algebras are completely reducible.

Note that the key to the proof of \Cref{ex:sl-reduces} was the ability to take averages in order to produce some invariant maps. (In particular, we needed to provide an invariant Hermitian form.) Our substitute for being able to take averages is to use the ``Casimir'' operator
\[C\coloneqq ef-fe+\frac12h^2,\]
which (suitably interpreted) is always an invariant map. Let's check this.
\begin{lemma} \label{lem:sl2-casimir-invariant}
	Let $\rho\colon\mf{sl}_2(\CC)\to\mf{gl}(V)$ be some representation. Then
	\[\rho(C)\coloneqq\rho(e)\circ\rho(f)-\rho(f)\circ\rho(e)+\frac12\rho(h)\circ\rho(h)\]
	is an $\mathfrak{sl}_2(\CC)$-invariant morphism $V\to V$.
\end{lemma}
\begin{proof}
	Being $\mf{sl}_2(\CC)$-invariant is linear in $\mf{sl}_2(\CC)$ and thus can be checked on the standard basis of $\mf{sl}_2$, which is a purely formal computation with the commutator. We will drop the $\rho$s everywhere for brevity. We compute
	\begin{align*}
		Ce &= efe+fee+\frac12hhe \\
		&= e(ef-h)+(ef-h)e+\frac12h(eh+2e) \\
		&= eef-eh+efe+\frac12heh \\
		&= eef-eh+efe+\frac12(eh+2e)h \\
		&= eef+efe+\frac12eh^2 \\
		&= eC.
	\end{align*}
	Similarly,
	\begin{align*}
		Cf &= eff+fef+\frac12hhf \\
		&= (fe+h)f+f(fe+h)+\frac12h(fh-2f) \\
		&= fef+f(fe+h)+\frac12hfh \\
		&= fef+f(fe+h)+\frac12(fh-2f)h \\
		&= fef+ffe+\frac12fhh \\
		&= fC.
	\end{align*}
	Lastly,
	\begin{align*}
		Ch &= efh+feh+\frac12hhh \\
		&= e(hf+2f)+f(he-2e)+\frac12hhh \\
		&= ehf+fhe+2(ef-fe)+\frac12hhh \\
		&= (he-2e)f+(hf+2f)e+2(ef-fe)+\frac12hhh \\
		&= hef+hfe+\frac12hhh \\
		&= hC.
		\qedhere
	\end{align*}
\end{proof}
\begin{remark}
	It may appear that our definition of $C$ came out of nowhere, but it turns out that all operators $V\to V$ which can be written as a polynomial in $\{e,f,h\}$ is actually a polynomial in $C$. We will prove this later once we have talked about the universal enveloping algebra, which is the correct context to talk about polynomials in $\{e,f,h\}$.
\end{remark}
Importantly, $C$ provides a basis-free way to distinguish between the irreducible representations $V_\bullet$.
\begin{lemma} \label{lem:sl2-casimir-irrep-scalar}
	The operator $\rho_n(C)\colon V_n\to V_n$ equals the scalar operator $\frac{n(n+2)}2\id_{V_n}$.
\end{lemma}
\begin{proof}
	Because $V_n$ is irreducible by \Cref{lem:sl2-get-irreps}, and $\rho_n(C)\colon V_n\to V_n$ is $\mf{sl}_2(\CC)$-invariant by \Cref{lem:sl2-casimir-invariant}, \Cref{prop:schur-lemma} tells us that $\rho(C)$ is a scalar operator, so it only remains to compute what this scalar is. Well, we may just compute $\rho(C)$ on the vector $x^n$: omitting the $\rho_n$s everywhere for brevity, we see
	\begin{align*}
		C\left(x^n\right) &= \left(ef+fe+\frac12h^2\right)\left(x^n\right) \\
		&= ef\left(x^n\right)+fe\left(x^n\right)+\frac12h^2\left(x^n\right) \\
		&= nx^n+0+\frac12n^2x^n \\
		&= \frac{n(n+2)}2x^n,
	\end{align*}
	as required.
\end{proof}
\begin{remark}
	One does not need to use \Cref{lem:sl2-casimir-invariant} and \Cref{prop:schur-lemma} here; instead, one can simply compute $\rho_n(C)$ on the entire basis $\left\{x^py^q\right\}_{p+q=n}$ to see what the scalar should be.
\end{remark}
We are now ready for our theorem.
\begin{theorem} \label{thm:sl2-reduces}
	The complex representations of $\mf{sl}_2(\CC)$ are completely reducible.
\end{theorem}
\begin{proof}
	Let $\rho\colon\mf{sl}_2(\CC)\to\mf{gl}(V)$ be any representation, which we want to be completely reducible. We will induct on $\dim V$, where the base case of $\dim V\in\{0,1\}$ has no content because $\dim V=0$ makes $V$ the empty sum of irreducibles, and $\dim V=1$ makes $V$ already irreducible. For the inductive step, we proceed in steps.
	\begin{enumerate}
		\item We reduce to the case where $V$ is indecomposable: indeed, if $V$ is the direct sum of two nonzero representations $V_1\oplus V_2$, then $\dim V_1,\dim V_2<\dim V$, so the induction promises that $V_1$ and $V_2$ are the direct sum of irreducible representations, so $V=V_1\oplus V_2$ is the direct sum of irreducible representations.

		For the rest of the argument, we thus may assume that $V$ is indecomposable.

		\item We reduce to the case where $\rho(C)$ has a single generalized eigenvalue on $V$. Let $\sigma(C)$ be the collection of eigenvalues of $V$, which is finite because $V$ is finite-dimensional. Then we have a decomposition
		\[V=\bigoplus_{\mu\in\sigma(C)}V[\mu]\]
		into generalized eigenspaces, where $V[\mu]$ is the generalized eigenspace with eigenvalue $\mu$.
		
		We claim that each $V[\mu]$ is a subrepresentation, implying that we must have $V=V[\mu]$ because $V$ is indecomposable, completing this step. Well, $V[\mu]$ is the kernel of $(\rho(C)-\mu{\id_V})^d$ for some perhaps large $d$. Because $\rho(C)$ is $\mf{sl}_2(\CC)$-invariant by \Cref{lem:sl2-casimir-invariant}, we conclude that $(\rho(C)-\mu{\id_V})^d$ is as well, so $V[\mu]\subseteq V$ becomes a subrepresentation by \Cref{ex:ker-subrep-alg}.

		\item We place $V$ into a controlled short exact sequence. Let $W\subseteq V$ be an irreducible subrepresentation; to construct one, we can just take a minimal nonzero subrepresentation. We want to show that $V=W$. Note \Cref{thm:sl2-classify-irreps} tells us that $W\cong V_n$ for some $n$, which means that $\rho(C)$ will act on $W$ and thus $V$ as the scalar $\frac{n(n+2)}2$ by \Cref{lem:sl2-casimir-irrep-scalar}. Now, we have a short exact sequence
		\[0\to V_n\to V\to V/V_n\to0.\]
		Because $\dim V/V_n<\dim V$, it must also be completely reducible. However, $\rho(C)$ acts as the scalar $\frac{n(n+2)}2$ on all irreducible subrepresentations of $V$, so using \Cref{lem:sl2-casimir-irrep-scalar} backward tells us that $V_n$ is the only permitted irreducible subrepresentation. Thus, $V/V_n\cong V_n^{\oplus(m-1)}$ for some $m\ge1$, and we are given the short exact sequence
		\begin{equation}
			0\to V_n\to V\to V_n^{\oplus(m-1)}\to0. \label{eq:sl2-decompose-ses}
		\end{equation}

		\item We construct morphisms $V_n\to V$, which will eventually produce an isomorphism $V_n^{\oplus m}\to V$. We will use \Cref{lem:sl2-map-from-sym-power}, which requires some eigenvectors of $\rho(h)$. Well, let $\sigma(h)$ be the eigenvalues of $\rho(h)$, so we get a decomposition
		\[V=\bigoplus_{\lambda\in\sigma(h)}V[\lambda]\]
		into generalized eigenspaces, where $V[\lambda]$ is the generalized eigenspace for $\rho(h)$ with eigenvalue $\lambda$. Let $\lambda_0$ be an eigenvalue with maximal real part. As in the proof of \Cref{lem:sl2-e-f-nilpotent}, we know that $\rho(e)\colon V[\lambda]\to V[\lambda+2]$, so the maximality of $\lambda_0$ implies that $\rho(e)\colon V[\lambda_0]\to V[\lambda_0+2]$ must be the zero map.
		
		Thus, $V[\lambda_0]\subseteq\ker\rho(e)$, so \Cref{lem:sl2-h-diagonalizes} tells us that $\rho(h)$ actually acts diagonally on $V[\lambda_0]$, and $\lambda_0=n'$ for some $n'\in\ZZ_{\ge0}$. Each vector in $V[n']$ provides an embedding $V_{n'}\to V$, but we know that all irreducible subrepresentations of $V$ are $V_n$, so $n'=n$. Quickly, we see that \eqref{eq:sl2-decompose-ses} tells us that $\dim V[n]=m\dim V_n[n]=m$, where $\dim V_n[n]=1$ because $V_n[n]=\op{span}\left\{x^n\right\}$.

		We now note that \Cref{lem:sl2-map-from-sym-power} takes each $u\in V[n]$ and gives an embedding $\varphi\colon V_n\to V$ such that $\varphi\left(x^n\right)=u$.

		\item We construct an isomorphism $V_n^{\oplus m}\to V$, which completes the proof because it shows that $V$ is completely reducible.\footnote{Because $V$ is assumed to be indecomposable, we actually know that $m=1$, but we do not need this to conclude the proof.} Let $\{u_1,\ldots,u_m\}$ be a basis of $V[n]$, which produces embeddings $\varphi_1,\ldots,\varphi_n\colon V_n\to V$ such that $\varphi_i\left(x^n\right)=u_i$ for each $i$. Now, $\bigoplus_i\varphi_i\colon V_n^{\oplus m}\to V$ will be the required morphism.

		We would like to show that $\bigoplus_i\varphi_i$ is an isomorphism. Because $\dim V=\dim V_n^{\oplus m}$ by \eqref{eq:sl2-decompose-ses}, it is enough to check that $\bigoplus_i\varphi_i$ is injective. Because each $\varphi_i$ commutes with the action of $h$, it is enough to check that $\bigoplus_i\varphi_i[\lambda]$ is injective for each $\lambda\in V_n^{\oplus m}[\lambda]=V_n[\lambda]^{\oplus m}$.
		
		We already have the injectivity for $\lambda=n$ because $\left\{\varphi_i\left(x^n\right)\right\}_{1\le i\le m}$ is a basis of $V[n]$. We will reduce all of our injectivity checks to this one. Well, recall from \eqref{eq:sl2-poly-action} that $\rho_n(h)$ diagonalizes with eigenvectors $\{-n,-n+2,\ldots,n-2,n\}$, so we are really trying to show that $\bigoplus_i\varphi_i[n-2j]$ is injective for $j\in\{0,2,\ldots,2n\}$. We will induct on $j$, where we already discussed the case of $j=0$. For the inductive step, take $j\in\{0,\ldots,2n-2\}$ and note that
		% https://q.uiver.app/#q=WzAsNCxbMCwwLCJWX25bbi0yai0yXV57XFxvcGx1cyBtfSJdLFsxLDAsIlZfbltuLTJqXV57XFxvcGx1cyBtfSJdLFswLDEsIlZbbi0yai0yXSJdLFsxLDEsIlZbbi0yal0iXSxbMCwxLCJcXHJob19uKGUpIl0sWzAsMiwiXFxiaWdvcGx1c19pXFx2YXJwaGlfaSIsMl0sWzIsMywiXFxyaG8oZSkiXSxbMSwzLCJcXGJpZ29wbHVzX2lcXHZhcnBoaV9pIl1d&macro_url=https%3A%2F%2Fraw.githubusercontent.com%2FdFoiler%2Fnotes%2Fmaster%2Fnir.tex
		\[\begin{tikzcd}
			{V_n[n-2j-2]^{\oplus m}} & {V_n[n-2j]^{\oplus m}} \\
			{V[n-2j-2]} & {V[n-2j]}
			\arrow["{\rho_n(e)}", from=1-1, to=1-2]
			\arrow["{\bigoplus_i\varphi_i}"', from=1-1, to=2-1]
			\arrow["{\bigoplus_i\varphi_i}", from=1-2, to=2-2]
			\arrow["{\rho(e)}", from=2-1, to=2-2]
		\end{tikzcd}\]
		commutes, where the horizontal maps are well-defined by the argument of the previous step. By the induction, we may assume that the right map has trivial kernel, and \eqref{eq:sl2-poly-action} tells us that the top map is an isomorphism. Thus, the composite map from the top-left to the bottom-right has trivial kernel, so the left map must have trivial kernel, as required.
		\qedhere
	\end{enumerate}
\end{proof}

\end{document}