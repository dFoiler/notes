% !TEX root = ../notes.tex

\documentclass[../notes.tex]{subfiles}

\begin{document}

\section{October 14}
Today we continue with our structure theory.

\subsection{Some Semisimple Lie Algebras}
We continue with our study of $\op{rad}(\mf g)$.
\begin{proposition}
	Fix an algebraically closed field $F$ of characteristic $0$. If $V$ is an irreducible representation of a Lie algebra $\mf g$ (over $F$), then $\op{rad}(\mf g)$ acts on $V$ by scalars, and $[\mf g,\op{rad}(\mf g)]$ vanishes.
\end{proposition}
\begin{proof}
	By \Cref{thm:lie}, we get a common eigenvector $v\in V$ for $\op{rad}(\mf g)$, so there is a linear functional $\lambda\colon\op{rad}(\mf g)\to F$ so that $Xv=\lambda(X)v$ for all $X\in\op{rad}(\mf g)$. Now, for any $X\in\mf g$, we define
	\[\mf g_X\coloneqq\op{rad}(\mf g)+FX.\]
	Note that $\mf g_X$ is a subalgebra because $\op{rad}(\mf g)$ is a Lie ideal. Now, for any $a\in\op{rad}(\mf g)$, we see that
	\[aX^nv=\lambda(a)X^nv+\sum_{i=1}^nc_iX^{n-i}v\]
	for some constants $c_1,\ldots,c_n\in F$ where $c_i\coloneqq\lambda([X,a]\ldots)$. As such, we  may let $W$ be the span of the $x^\bullet v$s, so $W$ is stable under $\mf g_X$, and $a$ only has the eigenvalue $\lambda$. Thus, for all $[X,a]\in[\mf g,\op{rad}(\mf g)]$, we see $\lambda([X,a])=0$ because $[X,a]$ on $W$ has vanishing trace. Thus, we actually see that $aXv=\lambda(a)Xv$. As such, the $\lambda$-eigenspace $V_\lambda$ of $V$ is actually stable under all $X\in\mf g$, so $V_\lambda=V$ by the irreducibility.
\end{proof}
We are now ready to define reductive.
\begin{definition}[reductive]
	A Lie algebra $\mf g$ is reductive if and only if $\op{rad}(\mf g)=\mf z(\mf g)$.
\end{definition}
\begin{remark}
	Note $\mf g$ is reductive if and only if $[\mf g,\op{rad}(\mf g)]=0$. Indeed, this will imply that $\op{rad}(\mf g)\subseteq\mf z(\mf g)$, but of course $\mf z(\mf g)$ is solvable, so the other inclusion holds as well.
\end{remark}
\begin{remark}
	Intuitively, one can say that being reductive means being a direct sum of semisimple and center.
\end{remark}
To check that certain Lie algebras are semisimple or reductive, it will be helpful to have access to invariant inner products.
\begin{definition}[invariant]
	Fix a Lie algebra $\mf g$ over a field $F$. A bilinear form $\langle-,-\rangle$ on $\mf g$ is \textit{invariant} if and only if
	\[B([X,Y],Z)=B(X,[Y,Z])\]
	for all $X,Y,Z\in\mf g$.
\end{definition}
\begin{example}
	For any representation $\rho\colon\mf g\to\mf{gl}(V)$ of a Lie algebra $\mf g$, the form
	\[B_V(X,Y)\coloneqq\tr(\rho(X)\rho(Y)).\]
	(Technically, we ought to write $B_\rho$, but we will write $B_V$ or even $B$ when confusion cannot arise.)
\end{example}
As usual, inner products allow us to take complements.
\begin{proposition}
	Fix a symmetric invariant bilinear form $B$ on a Lie algebra $\mf g$. For any ideal $I\subseteq\mf g$, the orthogonal complement
	\[I^\perp\coloneqq\{X\in\mf g:\langle X,Y\rangle=0\text{ for all }Y\in I\}\]
	is also a Lie ideal.
\end{proposition}
\begin{proof}
	Check it.
\end{proof}
\begin{proposition}
	Fix a Lie algebra $\mf g$. Suppose there is a representation $\rho\colon\mf g\to\mf{gl}(V)$ for which $B_\rho$ is non-degenerate. Then $\mf g$ is reductive.
\end{proposition}
\begin{proof}
	Let $\{V_1,\ldots,V_n\}$ be the irreducible factors of $V$, counted with multiplicity. Then $\rho$ can be upper-triangularized appropriately to see that
	\[B_V=\sum_{i=1}^kB_{V_i}.\]
	Now, for each $X\in[\mf g,\op{rad}(\mf g)]$, we see that $\rho_{V_i}(X)=0$, so $B_{V_i}(X,Y)=0$ for any $Y\in\mf g$, so $X=0$ because $B_V$ is non-degenerate.
\end{proof}
\begin{example}
	For any classical Lie algebra $\mf g\subseteq\mf{gl}(V)$, the standard representation $\rho\colon\mf g\to\mf{gl}(V)$ has $B_\rho$ non-degenerate. Thus, $\mf g$ is reductive. If further $\mf z(\mf g)=0$, then we see that $\mf g$ is semisimple. Perhaps one should be worried about positive characteristic.
\end{example}

\subsection{Cartan Criteria}
We now define a special invariant form.
\begin{definition}[Killing form]
	Fix a Lie algebra $\mf g$. Then the \textit{Killing form} is the invariant form
	\[B_{\mf g}(X,Y)\coloneqq\tr({\op{ad}_X}\circ{\op{ad}_Y}).\]
	We will write $K_{\mf g}$ for this form or simply $K$ if no confusion can arise.
\end{definition}
We will prove two theorems about this.
\begin{theorem}[Cartan criterion of solvability]
	Fix a Lie algebra $\mf g$ over a field $F$ of characteristic $0$. Then $\mf g$ is solvable if and only if $[\mf g,\mf g]\subseteq\ker K$.
\end{theorem}
\begin{theorem}[Cartan criterion of semisimplicity]
	Fix a Lie algebra $\mf g$ over a field $F$ of characteristic $0$. Then $\mf g$ is semisimple if and only if $K$ is non-degenerate.
\end{theorem}
The point is that $K$ ``detects'' being solvable and semisimple. We will require a notion of Jordan decomposition.
\begin{proposition}
	Fix a perfect field $F$, and let $V$ be a finite-dimensional $F$-vector space. Then any $A\in\mf{gl}(V)$ can be written uniquely in the form
	\[A=A_s+A_n\]
	for $A_s,A_n\in\mf{gl}(V_{\ov F})$ satisfying the following.
	\begin{itemize}
		\item $A_s$ is diagonalizable over $\ov F$.
		\item $A_n$ is nilpotent.
		\item $A_sA_n=A_nA_s$.
	\end{itemize}
	It turns out that $A_s,A_n\in\mf{gl}(V)$ and that $A_s$ can be expressed as a polynomial in $A$.
\end{proposition}
\begin{proof}
	We begin by finding $A_s$ over an algebraic closure. Here, we let $\{\lambda_1,\ldots,\lambda_e\}\subseteq\ov F$ be the roots of the characteristic polynomial $\chi_A(T)$ of $A$, where $\lambda_i$ occurs with multiplicity $m_i$. Then we note $\ov F[T]$ is a principal ideal domain with maximal given by $\{(T-\lambda):\lambda\in\ov F\}$, so we may decompose
	\[V_{\ov F}\cong\frac{\ov F[T]}{(\chi_A(T))}\simeq\bigoplus_{i=1}^e\frac{\ov F[T]}{(T-\lambda_i)^{m_i}},\]
	where $V_{\ov F}$ has been given the structure of an $\overline F[T]$-module via $T\mapsto A$. Now, the Chinese remainder theorem grants us a polynomial $P$ such that
	\[P(T)\equiv\lambda_i\pmod{(T-\lambda_i)^{m_i}}\]
	for each $i$. In particular, we see that
	\[P(A)-\lambda_i{\id_V}\equiv(A-\lambda_i{\id_V})^{m_i}Q_i(A)\]
	for some polynomial $Q_i$. In particular, evaluating in $\mf{gl}(V)$, we see that $P(A)$ acts as $\lambda\id_V$ on $V[\lambda]$ for each $\lambda_i$, so $A_s\coloneqq P(A)$ is semisimple, and one can check that $A_n\coloneqq A-A_s$ has all eigenvalues equal to $0$ and hence is nilpotent.

	We now argue that our decomposition is unique. Suppose we have another such decompositions $A=A_s'+A_n'$. Then $A$ commuting with $A_s'$ means that $A_s=P(A)$ commutes with $A_s'$, so $A_s$ and $A_s'$ can be simultaneously diagonalized. As such, $A_n$ and $A_n'$ can commute by taking the differences, so we see that
	\[A_s-A_s'=A_n'-A_n\]
	is a matrix which is both diagonalizable and nilpotent and hence must be zero. The uniqueness allows us to see that $A_s,A_n\in\mf{gl}(V)$ because $A=A_s+A_n$ implies that $A=\sigma(A_s)+\sigma(A_n)$ for all $\sigma\in\op{Gal}(\ov F/F)$, so $A_s=\sigma(A_s)$ and $A_n=\sigma(A_n)$ for each $\sigma$.
\end{proof}

\end{document}