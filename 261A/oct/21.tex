% !TEX root = ../notes.tex

\documentclass[../notes.tex]{subfiles}

\begin{document}

\section{October 21}
Today we begin talking about representations of semisimple Lie algebras.

\subsection{More on Derivations}
Here is a nice aside.
\begin{proposition} \label{prop:ss-derivations-are-inner}
	Fix a semisimple Lie algebra $\mf g$ over a field $F$ of characteristic $0$. Then the map $\op{ad}_\bullet\colon\mf g\to\op{Der}(\mf g)$ is a bijection.
\end{proposition}
\begin{proof}
	We checked earlier that $\op{ad}_\bullet$ at least outputs to derivations. It is injective because the kernel is the abelian Lie ideal $\mf z(\mf g)$, which is trivial because $\mf g$ is semisimple. It remains to check surjectivity.

	For example, we check that $\mf g\subseteq\op{Der}(\mf g)$ is a Lie ideal. Indeed, for any $X\in\mf g$ and $a\in\op{Der}(\mf g)$, we would like to check that $[a,{\op{ad}_X}]$ is still of the form $\op{ad}_\bullet$. Well, for any $Y\in\mf g$, we compute
	\[[a,{\op{ad}_X}](Y)=a([X,Y])-[X,a(Y)]=[a(X),Y]=\op{ad}_{a(X)}Y,\]
	where we have used the Jacobi identity.

	The moral of the story is that the Killing form $K$ of $\op{Der}(\mf g)$ will restrict to $\mf g$ as its own Killing form $K_{\mf g}$. Because $\mf g$ is semisimple, we know that $K|_{\mf g}$ is thus non-degenerate. Now, let $I\coloneqq\mf g^{\perp}$, which we would like to vanish. Well, $I\subseteq\op{Der}(\mf g)$ is an ideal satisfying $[I,\mf g]=0$, which means that any $a\in I$ and $X\in\mf g$ makes $[a,{\op{ad}_X}]=\op{ad}_{a(X)}$ (see above) vanish, meaning $a(X)=0$ by injectivity, so $a=0$.
\end{proof}
\begin{corollary}
	Fix a semisimple Lie algebra $\mf g$ over a field $F\in\{\RR,\CC\}$. Then the Lie algebra of the group $G\coloneqq\op{Aut}(\mf g)$ is $\mf g$.
\end{corollary}
\begin{proof}
	We already know that $\op{Aut}(\mf g)$ has Lie algebra $\op{Der}(\mf g)$.
\end{proof}
This provides a clean way to produce a Lie group for a semisimple Lie algebra.

\subsection{Motivating \texorpdfstring{$H^1$}{ H1}}
We are going to show that representations of semisimple Lie algebras are completely reducible. For this, as in \Cref{thm:sl2-reduces}, the key point is to show that irreducible subrepresentations have a complement. Thus, for our Lie algebra $\mf g$, we would like to know when a short exact sequence
\[0\to U\to V\to W\to0\]
of representations will split into a direct sum $V=U\oplus W$. Well, we can surely decompose $V=U\oplus W$ as vector spaces, but this need not make the $\mf g$-action-commute: in general, there will be some function $a\colon\mf g\to\op{Hom}_F(W,U)$ such that
\[\rho_V(X)(u,w)=(\rho_U(X)u+a(X)w,\rho_W(w))\]
by properties of the short exact sequence, and one can check that $a$ is a linear map satisfying
\[([X,Y](u,w)-XY+YX)(u,w)=(a([X,Y]),0),\]
so one finds that
\[a([X,Y])=[X,a(Y)]+[a(X),Y],\]
where we interpret $[X,a(Y)]$ as $\rho_U(X)\circ a(Y)-a(Y)\circ\rho_V(X)$ and similar for $[a(X),Y]$. This notation is actually okay because one can see that $[X,a]$ is the natural action of $\mf g$ on $\op{Hom}_F(W,U)$.

This motivates the following definition.
\begin{defihelper}[$Z^1(\mf g,E)$] \nirindex{Z1@$Z^1(\mf g,E)$}
	Fix a representation $E$ of a Lie algebra $\mf g$. Then the \textit{$1$-cocycle} group $Z^1(\mf g,E)$ consists of the group of morphisms $a\colon\mf g\to E$ satisfying
	\[a([X,Y])=X\cdot a(Y)-Y\cdot a(X).\]
\end{defihelper}
\begin{example} \label{ex:cocycle-trivial-rep}
	One sees that $Z^1(\mf g,F)$ consists of functionals $\mf g\to F$ which vanish on commutators, so this space is simply $(\mf g/[\mf g,\mf g])^\lor$. For example, if $\mf g$ is semisimple, then $Z^1(\mf g,F)$ vanishes.
\end{example}
\begin{example}
	One sees that $Z^1(\mf g,\mf g)$ consists exactly of the derivations $\op{Der}(\mf g)$.
\end{example}
Thus far we have showed that any extension produces a cocycle. In fact, one can check that a $1$-cocycle $a\in Z^1(\mf g,\op{Hom}_F(W,U))$ produces a representation $\rho_a\colon\mf g\to\mf{gl}(U\oplus W)$ by
\[\rho_a(X)(u,w)\coloneqq(\rho_U(X)u+a(X)w,\rho_W(X)w)\]
which sits in the short exact sequence
\[0\to U\to U\oplus W\to W\to0.\]
In the sequel, we may call this $\rho_a$ by simply $V_a$. Of course, if $a=0$, then the short exact sequence splits, but this condition is too strong for our purposes.

To set ourselves up, recall that a morphism of extensions $U$ and $W$ is a diagram of the following form.
% https://q.uiver.app/#q=WzAsMTAsWzAsMCwiMCJdLFsxLDAsIlUiXSxbMiwwLCJWX2EiXSxbMiwxLCJWX2IiXSxbMywwLCJXIl0sWzQsMCwiMCJdLFswLDEsIjAiXSxbMSwxLCJVIl0sWzMsMSwiVyJdLFs0LDEsIjAiXSxbNiw3XSxbNywzXSxbMyw4XSxbOCw5XSxbMCwxXSxbMSwyXSxbMiw0XSxbNCw1XSxbMSw3LCIiLDEseyJsZXZlbCI6Miwic3R5bGUiOnsiaGVhZCI6eyJuYW1lIjoibm9uZSJ9fX1dLFs0LDgsIiIsMSx7ImxldmVsIjoyLCJzdHlsZSI6eyJoZWFkIjp7Im5hbWUiOiJub25lIn19fV0sWzIsM11d&macro_url=https%3A%2F%2Fraw.githubusercontent.com%2FdFoiler%2Fnotes%2Fmaster%2Fnir.tex
\[\begin{tikzcd}
	0 & U & {V_a} & W & 0 \\
	0 & U & {V_b} & W & 0
	\arrow[from=1-1, to=1-2]
	\arrow[from=1-2, to=1-3]
	\arrow[Rightarrow, no head, from=1-2, to=2-2]
	\arrow[from=1-3, to=1-4]
	\arrow[from=1-3, to=2-3]
	\arrow[from=1-4, to=1-5]
	\arrow[Rightarrow, no head, from=1-4, to=2-4]
	\arrow[from=2-1, to=2-2]
	\arrow[from=2-2, to=2-3]
	\arrow[from=2-3, to=2-4]
	\arrow[from=2-4, to=2-5]
\end{tikzcd}\]
We are interested in classifying the extensions up to isomorphism. Thus far, we have found a way to list out all extensions as $V_\bullet$, but we still need to check when they are isomorphic. Well, if $f\colon V_a\to V_b$ is an isomorphism of short exact sequences, then following around the diagram means that
\[f(u,w)=(u+\varphi(w),w)\]
for some $\varphi\colon W\to U$. Note that any such morphism is automatically an isomorphism of vector spaces: its inverse is $(u,w)\mapsto(u-\varphi(w),w)$. Anyway, to check that $\varphi$ is a morphism of representations, we must check that $f\circ\rho_a(X)=\rho_b(X)\circ f$ for all $X\in\mf g$, which upon expansion yields
\[(Xu+X\varphi w+b(X)w,Xw)\stackrel?=(Xu+a(X)w+\varphi Xw,Xw),\]
so we see that we end up asking for
\[a(X)-b(X)\stackrel?=X\varphi-\varphi X=[X,\varphi].\]
Thus, we see that $V_a\cong V_b$ if and only if $a-b$ lives in the space of homomorphisms of the form $X\mapsto(X\varphi-\varphi X)$ for some $\varphi\in\op{Hom}_F(W,U)$. This motivates the following definition.
\begin{defihelper}[$B^1(\mf g,E)$] \nirindex{B1@$B^1(\mf g,E)$}
	Fix a representation $E$ of a Lie algebra $\mf g$. Then the \textit{$1$-coboundary} group $B^1(\mf g,E)$ consists of the group of morphisms $a\colon\mf g\to E$ for which there exists a vector $v\in E$ such that
	\[a(X)=X\cdot v.\]
\end{defihelper}
\begin{example}
	We can compute that $B^1(\mf g,\mf g)$ consists of the maps $a\colon\mf g\to\mf g$ of the form $a=\op{ad}_Y$ for some $Y\in\mf g$. Thus, $B^1(\mf g,\mf g)$ consists of the inner derivations.
\end{example}
\begin{remark}
	One can check that $B^1(\mf g,E)\subseteq Z^1(\mf g,E)$.
\end{remark}
Thus, we have the following definition.
\begin{defihelper}[$H^1(\mf g,E)$] \nirindex{H1@$H^1(\mf g,E)$}
	Fix a representation $E$ of a Lie algebra $\mf g$. Then the \textit{first cohomology group} $H^1(\mf g,E)$ is the quotient
	\[H^1(\mf g,E)\coloneqq\frac{Z^1(\mf g,E)}{B^1(\mf g,E)}.\]
	For convenience, we also define $\op{Ext}^1_\mf g(W,U)\coloneqq H^1(\mf g,\op{Hom}(W,U))$.
\end{defihelper}
\begin{example}
	Above we showed that $\op{Ext}^1_{\mf g}(W,U)$ classifies extensions
	\[0\to U\to V\to W\to0\]
	of representations.
\end{example}
\begin{example}
	Thus, we see that \Cref{prop:ss-derivations-are-inner} has showed that
	\[H^1(\mf g,\mf g)=0.\]
\end{example}
\begin{remark}
	There is a general procedure to define $H^n(\mf g,E)$ for any $n\ge0$. It can be constructed as the cohomology of the ``Chevalley complex'' defined by
	\[0\to E\to\op{Hom}_k(\mf g,E)\to\op{Hom}_k\left(\op{Alt}^2\mf gE\right)\to\cdots,\]
	where
	\[df(X_1\land\cdots\land X_k)\coloneqq\sum_{i=1}^k(-1)^{i-1}X_i f(X_1\land\cdots\widehat{X_i}\land\cdots X_k)-\sum_{i,j=1}^k(-1)^{i-j+1}f([X_i,X_j]\land\cdots\land\land\widehat{X_i}\land\widehat{X_j}).\]
	Alternatively, we can simply see this as the $\op{Ext}$ groups for the ring $U\mf g$. For example, one can compute that $H^0(\mf g,E)$ consists of the $\mf g$-fixed points of $E$.
\end{remark}

\end{document}