% !TEX root = ../notes.tex

\documentclass[../notes.tex]{subfiles}

\begin{document}

\section{October 9}
Today we continue discussing the universal enveloping algebra.

\subsection{Gradings and Filtrations}
We would like to ``temper'' the infinite-dimensional representation $\op{ad}_\bullet\mf g\to\mf{gl}(U\mf g)$ given by \Cref{lem:ad-action-ug}. For $T\mf g$, one has a grading.
\begin{definition}[graded algebra]
	Fix a field $F$. A \textit{grading} on an $F$-algebra $A$ is a decomposition $A=\bigoplus_{i=0}^\infty A_i$ where each $A_\bullet\subseteq A$ is a subspace, and
	\[A_i\cdot A_j\subseteq A_{i+j}\]
	for all $i,j\ge0$. An $F$-algebra equipped with a grading is a \textit{graded algebra}. An element of some $A_i$ is \textit{homogeneous}.
\end{definition}
\begin{example}
	Any $F$-algebra $A$ has a trivial grading given by $A_0=A$ and $A_i=0$ for all $i>0$. Indeed, $A_i\cdot A_j\subseteq A_{i+j}$ has $A_iA_j=0$ unless $i=j=0$, in which case we see that the inclusion reads $A_0A_0=A=A_0$.
\end{example}
\begin{example} \label{ex:grade-tensor-algebra}
	Fix a vector space $\mf g$. Then the universal tensor algebra $T\mf g$ is graded by $T\mf g=\bigoplus_{i\ge0}\mf g^{\otimes i}$. Indeed, $\mf g^{\otimes i}\subseteq T\mf g$ has
	\[\mf g^{\otimes i}\cdot\mf g^{\otimes j}\to\mf g^{\otimes(i+j)}\]
	for all $i,j\ge0$ by definition of our multiplication.
\end{example}
Gradings sometimes descend quotients.
\begin{lemma} \label{lem:homogeneous-ideal-quotient}
	Fix a graded $F$-algebra $A=\bigoplus_{i=0}^\infty A_i$. If $I\subseteq A$ is a two-sided ideal generated by homogeneous elements, then we have a decomposition
	\[\frac AI=\bigoplus_{i=0}^\infty\frac{A_i}{I\cap A_i}\]
	making $A/I$ into a graded $F$-algebra.
\end{lemma}
\begin{proof}
	Note that we have maps $A_i\subseteq A\onto A/I$ with kernel $I\cap A_i$, so we induce injections $A_i/(I\cap A_i)\into A/I$. Summing over all $i$, we get an $F$-linear map
	\[\bigoplus_{i=0}^\infty\frac{A_i}{I\cap A_i}\to\frac AI.\]
	We claim that this map is an isomorphism of $F$-vector spaces providing the required grading. Here are our checks.
	\begin{itemize}
		\item Surjective: note that any $a+I\in A/I$ can decompose $a=\sum_{i=0}^\infty a_i$ where $a_i\in A_i$ for each $i$, so $(a_i+(I\cap A_i))_i$ maps to $\sum_i(a_i+I)=a+I$.

		\item Injective: suppose that $(a_i+(I\cap A_i))_i$ maps to $0$ in $A/I$. Namely, we see that
		\[\sum_{i=0}^\infty a_i\in I,\]
		which we must prove implies $a_i\in I$ for each $i$. It suffices to show that
		\[I\stackrel?=\left\{\sum_{i=0}^\infty a_i\in A:a_i\in(I\cap A_i)\text{ for all }i\right\}.\]
		Let the right-hand side be $J$. Certainly, $J\subseteq I$: note $a_i\in I$ for each $i$ implies that $\sum_{i=0}^\infty a_i\in I$ because $I$ is an ideal. To show $I\subseteq J$, we use the hypothesis that $I$ is generated by homogeneous elements.\footnote{This is the only place where we need this hypothesis, which explains why this check must be somewhat difficult.} In other words, $I$ is generated by the subsets $\{I\cap A_i\}_{i=0}^\infty$, which are all contained in $J$, so we will know that $I\subseteq J$ as soon as we check that $J$ is actually a two-sided ideal.
		
		Well, the construction of $J$ implies that $J$ is certainly an $F$-subspace, so it remains to do the ideal checks. By symmetry, it will be enough to just check that $J$ is a left ideal. Thus, we want to check that $AJ\subseteq J$. By linearity of this condition (and because $J$ is already an $F$-subspace), it is enough to merely check this on a spanning subset of $A$, for which we take the homogeneous elements. Namely, it is enough to check that $A_iJ\subseteq J$ for each $i$, so pick up some $b_j\in A_j$ and $\sum_ia_i\in J$, and we see
		\[b_j\cdot\sum_{i=0}^\infty a_i=\sum_{i=0}^\infty b_ja_i.\]
		Now, $b_ja_i\in A_{i+j}$ by the grading, and $b_ja_i\in I$ because $I$ is an ideal, so we see $b_j\cdot\sum_{i=0}^\infty a_i\in J$ follows.

		\item Grading: we claim that $\frac{A_i}{I\cap A_i}\cdot\frac{A_j}{I\cap A_j}\subseteq\frac{A_{i+j}}{I\cap A_{i+j}}$. Well, for any $a_i+(I\cap A_i)$ and $a_j+(I\cap A_j)$, we see that the product has $(a_ia_j+I)\in A/I$, which comes from $a_ia_J+(I\cap A_{i+j})=A_{i+j}/(I\cap A_{i+j})$.
		\qedhere
	\end{itemize}
\end{proof}
\begin{example}
	Let $\mf g$ be an $F$-vector space. Then $S\mf g$ is the quotient of $T\mf g$ by the ideal generated by the homogeneous elements $X\otimes Y-Y\otimes X$, so \Cref{lem:homogeneous-ideal-quotient} tells us that the grading on $T\mf g$ descends to a grading on $S\mf g$.
\end{example}
However, the grading $T\mf g$ need not descend to the quotient $U\mf g$. For example, given $X,Y\in\mf g$ which we expect to degree $1$, we have
\[XY-YX=[X,Y],\]
but the left-hand side is expected to live in degree $2$ while the right-hand side is expected to live in degree $1$. Of course, if $\mf g$ is abelian, then everything here vanishes, so this is okay, but in general, there is no reason to believe that this would be okay.

Thus, we want a version of grading which descends along quotients, which leads to the following notion.
\begin{definition}[filtered algebra]
	Fix a field $F$. A \textit{filtration} on an $F$-algebra $A$ is a sequence of ascending $F$-subspaces
	\[0=\mc F_{-1}A\subseteq\mc F_0A\subseteq\mc F_1A\subseteq\cdots\subseteq\mc F_{n-1}A\subseteq\mc F_nA\subseteq\mc F_{n+1}A\subseteq\cdots\]
	such that $A=\bigcup_{i\ge0}\mc F_iA$, and $\mc F_iA\cdot\mc F_jA\to\mc F_{i+j}A$ for all $i,j\ge0$.
\end{definition}
Here's a sanity check.
\begin{lemma} \label{lem:grade-to-filtration}
	Fix a graded $F$-algebra $A=\bigoplus_{i=0}^\infty A_i$. Then
	\[\mc F_jA\coloneqq\bigoplus_{i=0}^jA_i\]
	provides a filtration of $A$.
\end{lemma}
\begin{proof}
	Here are our checks.
	\begin{itemize}
		\item Certainly each $\mc F_jA$ is an $F$-subspace.
		\item The construction implies $\mc F_jA\subseteq\mc F_jA\oplus A_{j+1}=\mc F_{j+1}A$ for each $j$.
		\item Note $\bigcup_{j\ge9}\mc F_jA=\bigoplus_{i\ge0}A_i=A$. Explicitly, any $a\in A$ can be decomposed into $a=\bigcup_{i\ge0}a_i$ with $a_i\in A_i$ for each $i$; but then there is the largest $n$ for which $a_n\ne0$, and we see $a\in\bigoplus_{i=0}^na_i=\mc F_nA$.
		\item We check that $\mc F_iA\cdot\mc F_jA\subseteq\mc F_{i+j}A$. Everything here is linear, so is enough to check this on spanning subsets of $\mc F_iA$ and $\mc F_jA$, for which we use the homogeneous elements. Well, choose some $a_k\in A_k\subseteq\mc F_iA$ and $a_\ell\in A_\ell\subseteq\mc F_jA$ where $k\le i$ and $\ell\le j$. Then the product $a_ka_\ell$ lives in $A_{k+\ell}$, which is contained in $\mc F_{i+j}A$ because $k+\ell\le i+j$.
		\qedhere
	\end{itemize}
\end{proof}
The up-shot of filtrations is that they do descend to quotients.
\begin{lemma} \label{lem:quotient-filtration}
	Fix a filtered $F$-algebra $A$ with filtration $\{\mc F_iA\}_{i=0}^\infty$. If $I\subseteq I$ is any two-sided ideal, then $A/I$ is a filtered $F$-algebra with filtration
	\[\left\{\frac{\mc F_iA}{I\cap\mc F_iA}\right\}_{i=0}^\infty.\]
\end{lemma}
\begin{proof}
	Here are our checks.
	\begin{itemize}
		\item Well-defined: note that the inclusions $\mc F_iA\subseteq A$ turn into inclusions $\frac{\mc F_iA}{I\cap\mc F_iA}\subseteq\frac AI$ because $I\cap\mc F_iA$ is the kernel of the composite $\mc F_iA\subseteq A\onto A/I$.
		\item Covers: we claim that
		\[\bigcup_{i=0}^\infty\frac{\mc F_iA}{I\cap\mc F_iA}=\frac AI.\]
		Well, any $a+I$ has $a\in\mc F_iA$ for some $i$, so $a+I$ is in the image of $a+(I\cap\mc F_iA)\in\frac{\mc F_iA}{I\cap\mc F_iA}$.
		\item Filtration: for any $i$ and $j$, we must check that $\frac{\mc F_iA}{I\cap\mc F_iA}\cdot\frac{\mc F_jA}{I\cap\mc F_jA}\to\frac{\mc F_{i+j}A}{I\cap\mc F_{i+j}A}$. Well, for any $a_i+(I\cap\mc F_iA)$ and $a_j+(I\cap\mc F_jA)$ where $a_i\in\mc F_iA$ and $a_j\in\mc F_jA$, we see that $a_ia_j\in\mc F_{i+j}A$, so the product
		\[(a_i+(I\cap\mc F_iA))\cdot(a_j+(I\cap\mc F_jA))\]
		is $(a_ia_j+I)\in A/I$, which comes from $(a_ia_j+(I\cap\mc F_{i+j}A))$, as needed.
		\qedhere
	\end{itemize}
\end{proof}
\begin{example} \label{ex:filter-ug}
	The grading on $T\mf g$ of \Cref{ex:grade-tensor-algebra} produces a filtration via \Cref{lem:grade-to-filtration}, which then descends to the quotient $U\mf g$ via \Cref{lem:quotient-filtration}. We will denote the induced filtration by $\{\mc F_iU\mf g\}_{i=0}^\infty$. Notably, looking at the induced quotient filtration of \Cref{lem:quotient-filtration}, we see that $\mc F_iU\mf g$ is simply the image of $\mc F_iT\mf g$ in $U\mf g$. (This formally follows from the well-definedness check in \Cref{lem:quotient-filtration}.) Importantly, each graded component of $T\mf g$ is finite-dimensional, so $\mc F_iT\mf g$ is finite-dimensional, so its image $\mc F_iU\mf g$ is also finite-dimensional.
\end{example}
Here is a concrete consequence of the intuition that the filtration ``tempers'' $U\mf g$.
\begin{lemma}
	Fix a Lie algebra $\mf g$ over a field $F$. For $A\mf g\in\{T\mf g,U\mf g,S\mf g\}$, the adjoint action descends to a Lie algebra morphism $\op{ad}_\bullet\colon\mf g\to\mf{gl}(\mc F_iA\mf g)$ for each $i$.
\end{lemma}
\begin{proof}
	The main point is to check that $\op{ad}_X\colon A\mf g\to A\mf g$ restricts to $\op{ad}_X\colon\mc F_iA\mf g\to\mc F_iA\mf g$ for each $i\ge0$ and $X\in\mf g$. Well, it suffices to check this on a spanning subset of $\mc F_iA\mf g$, for which we use pure tensors of length less than or equal to $i$. Namely, for any $Y_1\cdots Y_j\in A\mf g$ for $Y_1,\ldots,Y_j\in\mf g$, we see that
	\[\op{ad}_X(Y_1\cdots Y_j) = \sum_{k=1}^jY_1\cdots Y_{k-1}[X,Y_k]Y_{k+1}\cdots Y_j\]
	by definition of $\op{ad}_\bullet$ (or alternatively by repeated application of the product rule). Now, we see that the summand $Y_1\cdots Y_{k-1}[X,Y_k]Y_{k+1}\cdots Y_j$ lives in the image of $\mf g^{\otimes j}\subseteq\mc F_i T\mf g\onto\mc F_i A\mf g$, so we conclude that $\op{ad}_X$ does in fact restrict.

	We now describe the other checks. Note $\op{ad}_X$ was linear on $A\mf g$, so it continues to be linear on $\mc F_iA\mf g$. Additionally, the equalities
	\[{\op{ad}_{c_1X_1+c_2X_2}}=c_1{\op{ad}_{X_1}}+c_2{\op{ad}_{X_2}}\]
	and
	\[{\op{ad}_{[X,Y]}}={\op{ad}_X}\circ{\op{ad}_Y}-{\op{ad}_Y}\circ{\op{ad}_X}\]
	held as equalities of linear maps $A\mf g\to A\mf g$, so they will continue to be true on their restrictions to $\mc F_iA\mf g$. This completes the proof that $\op{ad}_\bullet\colon\mf g\to\mf{gl}(\mc F_iA\mf g)$ is a representation of Lie algebras.
\end{proof}
The statement of \Cref{lem:grade-to-filtration} tells us that we may hope to recover a grading from a filtration by using the quotients $\mc F_{j+1}A/\mc F_jA$. This leads to the following definition.
\begin{definition}[associated graded algebra]
	Fix a filtered $F$-algebra $A$ with filtration $\{\mc F_iA\}_{i=0}^\infty$. Then the \textit{associated graded algebra} is
	\[\op{gr}A\coloneqq\bigoplus_{i=0}^\infty\frac{\mc F_{i}A}{\mc F_{i-1}A}.\]
	Here, $\mc F_{-1}A$ is understood to be $0$.
\end{definition}
Here are the checks on this definition.
\begin{lemma}
	Fix a filtered $F$-algebra $A$ with filtration $\{\mc F_iA\}_{i=0}^\infty$. Then $\op{gr}A$ is a graded $F$-algebra with grading given by its definition.
\end{lemma}
\begin{proof}
	Here are our checks.
	\begin{itemize}
		\item Note that $\op{gr}A$ is certainly a vector space over $F$ as the sum of vector spaces over $F$.
		\item We must define a multiplication for $\op{gr}A$. We are trying to define an $F$-bilinear map $\op{gr}A\times\op{gr}A\to\op{gr}A$, so we are trying to define a linear map $\op{gr}A\otimes_F\op{gr}A\to\op{gr}A$. Swapping the tensor product and sum, it is enough to define a map
		\[\frac{\mc F_iA}{\mc F_{i-1}A}\otimes_F\frac{\mc F_jA}{\mc F_{j-1}A}\to\frac{\mc F_{i+j}A}{\mc F_{i+j-1}A}\subseteq\op{gr}A\]
		for $i,j\ge0$. For this, we note the multiplication on $A$ defines an $F$-bilinear map $\mc F_iA\times\mc F_jA\to\mc F_{i+j}A$, but $\mc F_{i-1}A\mc F_jA\subseteq\mc F_{i+j-1}A$ and $\mc F_iA\mc F_{j-1}A\subseteq\mc F_{i+j-1}A$ means that our map descends to an $F$-bilinear map
		\[\frac{\mc F_iA}{\mc F_{i-1}A}\times\frac{\mc F_jA}{\mc F_{j-1}A}\to\frac{\mc F_{i+j}A}{\mc F_{i+j-1}A},\]
		so this descends to a map on the tensor product, as needed. Note that this definition of multiplication automatically respects the ``grading'' in the definition of $\op{gr}A$.
		\item Thus, it only remains that we actually have a well-defined multiplication. Well, the multiplication is $F$-bilinear by its construction, so it satisfies the usual distributivity and $F$-linear requirements. It remains to check associativity. Well, the associativity condition $(ab)c=a(bc)$ corresponds to the vanishing of some multilinear functional on $(\op{gr}A)^3$, which corresponds to the vanishing of some functional on $(\op{gr}A)^{\otimes3}$, which can be checked on the spanning subset of homogeneous elements. As such, we pick up three homogeneous elements $a_i+\mc F_{i-1}A$ and $a_j+\mc F_{j-1}A$ and $a_k+\mc F_{k-1}A$ and use the above definition to compute
		\begin{align*}
			& ((a_i+\mc F_{i-1}A)(a_j+\mc F_{j-1}A))(a_k+\mc F_{k-1}A) \\
			={}& (a_ia_j+\mc F_{i+j-1}A)(a_k+\mc F_{k-1}A) \\
			={}& (a_ia_ja_k+\mc F_{i+j+k-1}A) \\
			={}& (a_i+\mc F_{i-1}A)(a_ja_k+\mc F_{j+k-1}A) \\
			={}& (a_i+\mc F_{i-1}A)((a_j+\mc F_{j-1}A)(a_k+\mc F_{k-1}A)),
		\end{align*}
		as required.
		\qedhere
	\end{itemize}
\end{proof}
Before ending our discussion, we note the following property of $\op{gr}A$.
\begin{lemma} \label{lem:zero-divisor-gr}
	Fix a filtered $F$-algebra $A$ with filtration $\{\mc F_iA\}_{i=0}^\infty$. If $\op{gr}A$ has no zero divisors, then $A$ has no zero divisors.
\end{lemma}
\begin{proof}
	We proceed by contraposition. Suppose $A$ has zero divisors so that we have nonzero elements $a,b\in A$ such that $ab=0$. Let $i$ be the smallest nonnegative element $a\in\mc F_iA$ but $a\notin\mc F_{i-1}A$. (If $i=0$, then certainly $a\notin\mc F_{i-1}A$ because $\mc F_{i-1}A=0$, and $a\ne0$.) Similarly, we find $j\ge0$ such that $b\in\mc F_jA$ but $b\notin\mc F_{j-1}A$. Then $a+\mc F_{i-1}A$ and $b+\mc F_{j-1}A$ are nonzero elements in $\op{gr}A$ which multiply to $0+\mc F_{i+j-1}A$, so $\op{gr}A$ has zero divisors.
\end{proof}
% \begin{remark}
% 	One can check that $\op{gr}A$ is a graded algebra. Conversely, given a graded algebra $A$ with grading $A=\bigoplus_{i=1}^\infty A_i$, one can produce a filtered algebra with the filtration
% 	\[\mc F_iA\coloneqq\bigoplus_{j\le i}A_j.\]
% \end{remark}
% \begin{remark}
% 	Fix a filtered algebra $A$. If $\op{gr}A$ has no zero divisors, then one can check that $A$ also has no zero divisors.
% \end{remark}
% Now, the point is that $T\mf g=\bigoplus_{i\ge0}\mf g^{\otimes i}$ is graded, but we cannot expect $U\mf g$ to be graded because the relation $x\otimes y-y\otimes x=[x,y]$ is a relation between the second and first degree! But we can still pass to the filtration: let $\mc F_iU\mf g$ simply be the image of the natural grading $\mc F_iT\mf g$ under the surjection $T\mf g\onto U\mf g$.
% \begin{remark}
% 	In contrast, one can check that $S\mf g$ is not just filtered but graded!
% \end{remark}

\subsection{The Poincar\'e--Birkoff--Witt Theorem}
To understand the theorem of this subsection, we make the following observation.
\begin{lemma} \label{lem:pbw-span}
	Fix a Lie algebra $\mf g$ over a field $F$ with basis $\{X_1,\ldots,X_n\}$. Then the ordered monomials
	\[\left\{X_1^{d_1}\cdots X_n^{d_n}:a_1,\ldots,a_n\ge0\right\}\]
	span $U\mf g$.
\end{lemma}
\begin{proof}
	The point is to use the equality $XY-YX=[X,Y]$ in $U\mf g$ to slowly reorder unordered monomials, one transposition at a time. Let's be more explicit. The monomials $X_{i_1}\otimes\cdots\otimes X_{i_m}$ spans $\mf g^{\otimes m}$ (here, $i_1,\ldots,i_m\in\{1,\ldots,n\}$), so they will span $T\mf g$ as $m$ varies. Thus, it is enough to show that any such monomial $X_{i_1}\cdots X_{i_m}\in U\mf g$ lives in the span of the above ordered monomials. We will do this via a nested induction. To begin, we induct on $m$, for which the base cases $m=0$ and $m=1$ have no content because the monomial is already ordered. Thus, for the inductive step, we may assume the result for any monomial with less than $m$ terms.
	
	Next up, we note that $i_\bullet\colon\{1,\ldots,m\}\to\{1,\ldots,n\}$ is some function, so we choose some permutation $\sigma\in S_m$ so that
	\[i_{\sigma(1)}\le i_{\sigma(2)}\le\cdots\le i_{\sigma(m)}.\]
	Now, $\sigma\in S_m$ can be written as a product of transpositions of the form $(j,j+1)$, say $\sigma=(j_1,j_1+1)\cdots(j_\ell,j_\ell+1)$. We next induct on $\ell$, where the base case of $\ell=0$ has no content because it means that $\sigma$ is the identity so that the $i_\bullet$s are already ordered so that our monomial is already ordered. For the inductive step, we pick up our monomial $X_{i_1}\cdots X_{i_m}$ with $\sigma=(j_1,j_1+1)\cdots(j_\ell,j_\ell+1)$ where $\ell>0$, and we note that
	\begin{align*}
		X_{i_1}\cdots X_{i_m} &= X_{i_1}\cdots X_{i_{j_\ell}}X_{i_{j_\ell+1}}\cdots X_{i_m} \\
		&\stackrel*= X_{i_1}\cdots\big(X_{i_{j_\ell+1}}X_{i_{j_\ell}}+[X_{i_{j_\ell}},X_{i_{j_\ell+1}}]\big)\cdots X_{i_m} \\
		&= X_{i_1}\cdots X_{i_{j_\ell-1}}X_{i_{j_\ell+1}}X_{i_{j_\ell}}X_{i_{j_\ell+2}}\cdots X_{i_m}+X_{i_1}\cdots X_{i_{j_\ell-1}}[X_{i_{j_\ell}},X_{i_{j_\ell+1}}]X_{i_{j_\ell+2}}\cdots X_{i_m}.
	\end{align*}
	Here, the key step is the application of $XY-YX=[X,Y]$ in $U\mf g$ in the equality $\stackrel*=$. Anyway, the second term has fewer than $m$ terms, so the first inductive hypothesis implies that it is the span of our ordered monomials. The first term still has $m$ terms, but the permutation $\sigma'\in S_m$ required to reorder the $i_\bullet$s can simply be taken to be $(j_1,j_1+1)\cdots(j_{\ell-1},j_{\ell-1}+1)$, which has length smaller than $\ell$, so the second inductive hypothesis implies that it too is in the span of our ordered monomials.
\end{proof}
\begin{remark}
	The above nested induction can be turned into an algorithm: simply use the relation $XY-YX=[X,Y]$ to gradually reorder the consecutive terms of any monomial until it becomes ordered. The complications in this algorithm is that the ``error term'' $[X,Y]$ spawn lower-order monomials which must be dealt with recursively.
\end{remark}
\begin{remark} \label{rem:spit-error-term}
	Let's note a consequence of the proof of \Cref{lem:pbw-span}: in $U\mf g$, the difference
	\[X_{i_1}\cdots X_{i_m}-X_{\sigma(i_1)}\cdots X_{\sigma(i_m)}\]
	is a linear combination of ordered monomials of degree than smaller $m$. Indeed, the exact same inductive argument as in the proof of \Cref{lem:pbw-span} shows this, where the point is that the produced error term
	\[X_{i_1}\cdots X_{i_{j_\ell-1}}[X_{i_{j_\ell}},X_{i_{j_\ell+1}}]X_{i_{j_\ell+2}}\cdots X_{i_m}\]
	has fewer than $m$ monomials after expanding $[X_{i_{j_\ell}},X_{i_{j_\ell+1}}]$.
\end{remark}
With the lemma in mind, here is our statement of the theorem.
\begin{theorem} \label{thm:pbw}
	Fix a Lie algebra $\mf g$ over a field $F$ with basis $\{X_1,\ldots,X_n\}$. Then the ordered monomials
	\[\left\{X_1^{d_1}\cdots X_n^{d_n}:a_1,\ldots,a_n\ge0\right\}\]
	are linearly independent in $U\mf g$.
\end{theorem}
\begin{proof}
	We omit the proof of this theorem for now because it is somewhat tedious.\todo{}
\end{proof}
\Cref{thm:pbw} has important consequences for the structure of $U\mf g$. Let's see some.
\begin{corollary}
	Fix a Lie algebra $\mf g$ over a field $F$. Then the inclusion $\mf g\to U\mf g$ is actually injective.
\end{corollary}
\begin{proof}
	Let $\{X_1,\ldots,X_n\}$ be a basis of $\mf g$. Then \Cref{thm:pbw} implies that the monomials $\{X_1,\ldots,X_n\}\subseteq U\mf g$ are linearly independent. In particular, we see that any nonzero $X\in\mf g$ can be expanded as $\sum_{i=1}^na_iX_i$, where $a_i\ne0$ for some $i$, meaning that
	\[\sum_{i=1}^na_iX_i\in U\mf g\]
	is also nonzero by the aforementioned linear independence. Thus, the map $\mf g\to U\mf g$ has trivial kernel.
\end{proof}
\begin{corollary}
	Fix a Lie algebra $\mf g$ over a field $F$ with Lie subalgebras $\mf g_1,\ldots,\mf g_n\subseteq\mf g$ such that we have a vector space decomposition $\mf g=\mf g_1\oplus\cdots\oplus\mf g_n$. Then the ``multiplication'' map $\mu\colon U\mf g_1\otimes_F\cdots\otimes_F U\mf g_n\to U\mf g$, given by
	\[\mu(a_1\otimes\cdots\otimes a_n)\coloneqq a_1\cdots a_n\]
	on pure tensors, is an isomorphism of vector spaces.
\end{corollary}
\begin{proof}
	The point is that $\mu$ should send a basis of ordered monomials to a basis of ordered monomials. Quickly, observe that functoriality produces maps $U\mf g_\bullet\to U\mf g$ from the inclusions $\mf g_\bullet\subseteq\mf g$. As such, we may note that the map $U\mf g_1\times\cdots\times U\mf g_n\to U\mf g$
	\[(a_1,\ldots,a_n)\mapsto a_1\cdots a_n\]
	is $F$-multilinear because $U\mf g$ is an $F$-algebra, so we have indeed induced a unique morphism $U\mf g_1\otimes\cdots\otimes U\mf g_n\to U\mf g$ of vector spaces over $F$.

	It remains to check that this map is an isomorphism. Well, it is enough to check that it sends a basis. For this, we give each $\mf g_i$ a basis $\{X_{i1},\ldots,X_{in_i}\}$ so that the concatenation of these bases provides a basis of $\mf g$. Then the combination of \Cref{lem:pbw-span} and \Cref{thm:pbw} tells us that the ordered monomials
	\[X_{i1}^{d_{i1}}\cdots X_{in_i}^{d_{in_i}}\]
	form a basis of $\mf g_i$ for each $i$, so by taking the tensor product, we see that the ordered monomials
	\[X_{11}^{d_{11}}\cdots X_{1n_1}^{d_{1n_1}}\otimes\cdots\otimes X_{n1}^{d_{n1}}\cdots X_{nn_n}^{d_{nn_n}}\]
	form a basis of $U\mf g_1\otimes_F\cdots\otimes_F U\mf g_n$. On the other hand, the elements
	\[\left(X_{11}^{d_{11}}\cdots X_{1n_1}^{d_{1n_1}}\right)\cdots\left(X_{n1}^{d_{n1}}\cdots X_{nn_n}^{d_{nn_n}}\right)\]
	are the ordered monomials of $U\mf g$, so they also form a basis. Thus, $\mu$ sends a basis to a basis, so we are done.
\end{proof}
\begin{remark}
	The above proposition does not require the $\mf g_\bullet$s to commute within $\mf g$. Namely, it is not at all required that the vector space decomposition is actually a Lie algebra decomposition.
\end{remark}
% \begin{remark}
% 	Suppose we use the more general construction of $U\mf g$ as given in \Cref{rem:general-ug}. If we want the induced map $\mf g\to U\mf g$ to be injective, there are some requirements.
% 	\begin{itemize}
% 		\item Skew-symmetry: note 
% 	\end{itemize}
% \end{remark}
We next move on to results which sharpen \Cref{thm:pbw} in various ways. To begin, we will exhibit an isomorphism between $S\mf g$ and $\op{gr}U\mf g$. Let's begin with a couple lemmas.
\begin{lemma} \label{lem:gr-ug-commutative}
	Fix a Lie algebra $\mf g$ over a field $F$. Then $\op{gr}U\mf g$ is a commutative $F$-algebra.
\end{lemma}
\begin{proof}
	The point is to use \Cref{rem:spit-error-term} after making enough reductions. We would like to check that $ab=ba$ for all $a,b\in\op{gr}U\mf g$. This condition corresponds to an equality of some multilinear maps $\op{gr}U\mf g\times\op{gr}U\mf g\to\op{gr}U\mf g$, so it can be checked on spanning subsets of $\op{gr}U\mf g$. Namely, we may assume that $a\in\mc F_kU\mf g/\mc F_{k-1}U\mf g$ and $b\in\mc F_\ell U\mf g/\mc F_{\ell-1}U\mf g$. Now, by definition of $\mc F_kU\mf g$ in \Cref{ex:filter-ug} means that elements of $\mc F_kU\mf g$ can be written as homogeneous polynomials of degree $n$ in a basis $\{X_1,\ldots,X_n\}$ of $\mf g$. Thus, using the multilinearity, we may assume that $a\coloneqq X_{i_1}\cdots X_{i_k}$ and $b\coloneqq X_{j_1}\cdots X_{j_\ell}$ are monomials. We would like to show that $ab-ba=0$, which corresponds to showing that
	\[ab-ba\in\mc F_{k+\ell-1}U\mf g,\]
	which means that $ab-ba$ must be a linear combination of monomials of degree less than $k+\ell$. Well, let $\sigma,\tau\in S_{k+\ell}$ be permutations which order the monomials $ab$ and $ba$, which will produce the same ordered monomial $X_1^{d_1}\cdots X_n^{d_n}$ from both $ab$ and $ba$ (because we are reordering the same multiset $\{i_1,\ldots,i_k\}\sqcup\{j_1,\ldots,j_\ell\}$). But now \Cref{rem:spit-error-term} tells us that $ab-X_1^{d_1}\cdots X_n^{d_n}$ and $ba-X_1^{d_1}\cdots X_n^{d_n}$ both live in $\mc F_{k+\ell-1}U\mf g$, so $ab-ba\in\mc F_{k+\ell-1}U\mf g$, as required.
\end{proof}
\begin{proposition} \label{prop:gr-ug-is-sg}
	Fix a Lie algebra $\mf g$ over a field $F$ with basis $\{X_1,\ldots,X_n\}$. Then the map $S\mf g\to\op{gr}U\mf g$ given by
	\[X_1^{d_1}\cdots X_n^{d_n}\mapsto X_1^{d_1}\cdots X_n^{d_n}\in\frac{\mc F_{d_1+\cdots+d_n}U\mf g}{\mc F_{d_1+\cdots+d_n-1}U\mf g}\]
	is an isomorphism of graded $F$-algebras.
\end{proposition}
\begin{proof}
	Quickly, we note that there is a unique $F$-linear map $S\mf g\to\op{gr}U\mf g$ because ordered monomials form a basis of $S\mf g$: recall from \Cref{ex:sg} that $S\mf g=F[X_1,\ldots,X_n]$ is a commutative polynomial ring.
	
	To show that such an $F$-algebra morphism exists, we use the universal property of such polynomial rings is that an $F$-algebra map out of $S\mf g$ is determined exactly by choosing where the elements $\{X_1,\ldots,X_n\}$ go, provided the target is a commutative $F$-algebra. Thus, we define $\varphi\colon S\mf g\to\op{gr}U\mf g$ by defining $\varphi(X_i)\coloneqq X_i$ for each $i$. (Note $\op{gr}U\mf g$ is commutative by \Cref{lem:gr-ug-commutative}.) Here are the required checks on this map.
	\begin{itemize}
		\item On ordered monomials $X_1^{d_1}\cdots X_n^{d_n}$, we see that
		\[\varphi\left(X_1^{d_1}\cdots X_n^{d_n}\right)=\varphi(X_1)^{d_1}\cdots\varphi(X_n)^{d_n}=X_1^{d_1}\cdots X_n^{d_n},\]
		which we note lives in the degree-$d$ piece $\mc F_{d_1+\cdots+d_n}U\mf g/\mc F_{d_1+\cdots+d_n-1}U\mf g$ because each $X_i\in\op{gr}U\mf g$ has degree $1$.
		\item Graded: we claim that if $p(X_1,\ldots,X_n)\in S\mf g$ is in the degree-$d$ graded piece, then $\varphi(p)\in\op{gr}U\mf g$ is in the degree-$d$ graded piece as well. The graded pieces are $F$-linear subspaces, so we may check this on a basis of the degree-$d$ graded piece of $S\mf g$, for which we use the ordered monomials $X_1^{d_1}\cdots X_n^{d_n}$ of total degree $d$. But then the previous check already verified that $\varphi\left(X_1^{d_1}\cdots X_n^{d_n}\right)$ lands in the correct graded piece.
		\item Surjective: we use \Cref{lem:pbw-span}. It is enough to check that the spanning subset of homogeneous elements of $\op{gr}U\mf g$ are in the image of $\varphi$. Namely, we want a component $\mc F_kU\mf g/\mc F_{k-1}U\mf g$ to be in $\im\varphi$. However, $\mc F_kU\mf g$ consists of the sum of homogeneous polynomials in $\{X_1,\ldots,X_n\}$ of degree $k$, so it is enough to check that homogeneous polynomials in $\{X_1,\ldots,X_n\}$ of degree $k$ in $\mc F_kU\mf g/\mc F_{k-1}U\mf g$ are in $\im\varphi$. (Explicitly, $\mc F_{k-1}U\mc g$ kills homogeneous polynomials of degree less than $k$.) However, such homogeneous polynomials are spanned by the monomials
		\[X_1^{d_1}\cdots X_n^{d_n}\]
		of total degree $k$, which we note equals $\varphi\left(X_1^{d_1}\cdots X_n^{d_n}\right)\in\mc F_kU\mf g/\mc F_{k-1}$ and thus is in the image of $\varphi$.
		\item Injective: we use \Cref{thm:pbw} to show $\ker\varphi=0$. Suppose that some polynomial $p(X_1,\ldots,X_n)\in S\mf g$ vanishes in $\op{gr}U\mf g$. Splitting up $p$ up into graded pieces by the degree, the fact that $\varphi$ preserves grading allows us to assume that $p$ is homogeneous of degree $d$. Thus, say that
		\[\sum_{\substack{(d_1,\ldots,d_n)\in\NN^n\\d_1+\cdots+d_n=d}}a_{(d_1,\ldots,d_n)}X_1^{d_1}\cdots X_n^{d_n}\in\ker\varphi,\]
		and we want to check that this element actually vanishes. Well, unwrapping the definition of $\varphi$, we see that
		\[\sum_{\substack{(d_1,\ldots,d_n)\in\NN^n\\d_1+\cdots+d_n=d}}a_{(d_1,\ldots,d_n)}X_1^{d_1}\cdots X_n^{d_n}\in\mc F_{d_1+\cdots+d_n-1}U\mf g.\]
		Namely, we have a homogeneous polynomial of degree $d$ which can be written as an $F$-linear combination of homogeneous polynomials of strictly smaller degree, so we may write
		\[\sum_{\substack{(d_1,\ldots,d_n)\in\NN^n\\d_1+\cdots+d_n=d}}a_{(d_1,\ldots,d_n)}X_1^{d_1}\cdots X_n^{d_n}=\sum_{\substack{(d_1,\ldots,d_n)\in\NN^n\\d_1+\cdots+d_n<d}}a_{(d_1,\ldots,d_n)}X_1^{d_1}\cdots X_n^{d_n}\]
		for some new coefficients $a_{(d_1,\ldots,d_n)}$. However, the monomials $X_1^{d_1}\cdots X_n^{d_n}$ over all total degrees are linearly independent in $U\mf g$ by \Cref{thm:pbw}, so we see that all coefficients $a_{(d_1,\ldots,d_n)}$ must vanish, so in particular $p=0$.
		\qedhere
	\end{itemize}
\end{proof}
\begin{corollary}
	Fix a Lie algebra $\mf g$ over a field $F$. Then the algebra $U\mf g$ has no zero divisors.
\end{corollary}
\begin{proof}
	By \Cref{lem:zero-divisor-gr}, it is enough to check that $\op{gr}U\mf g$ has no zero divisors. But \Cref{prop:gr-ug-is-sg} tells us that $\op{gr}U\mf g$ is isomorphic to $S\mf g$ as rings, and the commutative polynomial ring $S\mf g$ certainly does not have zero divisors.
\end{proof}
\Cref{prop:gr-ug-is-sg} may be unsatisfying because it requires us to pass through $\op{gr}U\mf g$ even though it is really $U\mf g$ which interests us. Of course, we cannot expect $U\mf g$ and $S\mf g$ to be isomorphic as $F$-algebras because $U\mf g$ is not commutative in general (indeed, $XY-YX=[X,Y]$ only vanishes when $\mf g$ is abelian). However, we can get some structure preserved.
\begin{proposition} \label{prop:sg-is-ug}
	Fix a Lie algebra $\mf g$ over a field $F$. Define $\op{sym}\colon S\mf g\to U\mf g$ by
	\[\op{sym}(Y_1\cdots Y_k)\coloneqq\frac1{k!}\sum_{\sigma\in S_k}Y_{\sigma(1)}\cdots Y_{\sigma(k)}\]
	for any $Y_1,\ldots,Y_k\in\mf g$. Then $\op{sym}$ is a well-defined isomorphism of $\mf g$-modules, where we are using the adjoint action by $\mf g$.
\end{proposition}
\begin{proof}
	This is on the homework.
	% We run our checks in sequence. Let $\{X_1,\ldots,X_n\}$ be a basis of $\mf g$.
	% \begin{itemize}
	% 	\item Unique: note that $S\mf g$ is spanned as an $F$-vector space by monomials of the form $Y_1\cdots Y_k$, so $\op{sym}$ is certainly uniquely defined by the given data.
	% 	\item Well-defined: we begin by defining a map $\op{sym}\colon T\mf g\to U\mf g$ to satisfy the given equation, and then we will take a quotient by the ideal $I$ which is the kernel of linear map $T\mf g\onto S\mf g$. Well, to define $\op{sym}\colon T\mf g\to U\mf g$ as a linear map, it is enough to define it on the summands $\mf g^{\otimes k}$ of $T\mf g$, so we note that the map
	% 	\[(Y_1,\ldots,Y_k)\mapsto\frac1{k!}\sum_{\sigma\in S_k}Y_{\sigma(1)}\cdots Y_{\sigma(k)}\]
	% 	is an $F$-multilinear map $\mf g^k\to U\mf g$ because multiplication in $U\mf g$ is $F$-linear. Thus, we induce a map $\op{sym}\colon\mf g^{\otimes k}\to U\mf g$ such that
	% 	\[(Y_1\otimes\cdots\otimes Y_k)\mapsto\frac1{k!}\sum_{\sigma\in S_k}Y_{\sigma(1)}\cdots Y_{\sigma(k)}.\]
	% 	We now descend this to a map $S\mf g\to U\mf g$. It is not immediately enough to check this on generators of $I$ as an ideal because $\op{sym}$ is not a morphism of $F$-algebras. Thus, we must check the result on generators of $I$ as an abelian group: for $a,b\in T\mf g$ and $Y_1,Y_2\in\mf g$, we must check that
	% 	\[\op{sym}(a(Y_1\otimes Y_2-Y_2\otimes Y_1))\stackrel?=0,\]
	% 	where there is also a symmetric statement for right multiplication, which will have a sufficiently symmetric proof. %(Technically, we must also check that $\op{sym}(Y_1\otimes Y_2-Y_1\otimes Y_2)=0$, which is true by a direct expansion.)
	% 	Upon fixing $Y_1$ and $Y_2$, we see that we are checking some linear map $T\mf g\to U\mf g$ vanishes. Thus, we may check this when $a$ is homogeneous and even a pure tensor $a\coloneqq Y_3\otimes\cdots\otimes Y_k$ for some $k\ge2$ (permitting the empty pure tensor). As such, we compute
	% 	\begin{align*}
	% 		\op{sym}(a(Y_1\otimes Y_2)) &= \op{sym}(Y_3\otimes\cdots\otimes Y_k\otimes Y_1\otimes Y_2) \\
	% 		&= \frac1{k!}\sum_{\sigma\in S_k}Y_{\sigma(3)}\cdots Y_{\sigma(k)}Y_{\sigma(1)}Y_{\sigma(2)} \\
	% 		&\stackrel*= \frac1{k!}\sum_{\sigma\in S_k}Y_{\sigma(3)}\cdots Y_{\sigma(k)}Y_{\sigma(2)}Y_{\sigma(1)} \\
	% 		&= \op{sym}(Y_3\otimes\cdots\otimes Y_k\otimes Y_2\otimes Y_1) \\
	% 		&= \op{sym}(a(Y_2\otimes Y_1)),
	% 	\end{align*}
	% 	where the key point is that $\stackrel*=$ is merely a re-indexing of the sum to multiply $\sigma\in S_k$ by an additional transposition.
	%
	% 	\item Surjective: it is enough to surject onto $\mc F_iU\mf g$ for each $i\ge0$ because $U\mf g$ is a filtered $F$-algebra. For this, we will induct on $i$, where the base case of $i=-1$ has no content because $\mc F_{-1}U\mf g=0$. For the inductive step, we may assume that $\mc F_iU\mf g$ is already in the image, and we want to check that $\mc F_{i+1}U\mf g$ is in the image. By \Cref{lem:pbw-span}, it is enough to merely achieve the ordered monomials. Now, by \Cref{rem:spit-error-term}, we see that
	% 	\[\op{sym}\left(X_1^{a_1}\cdots X_n^{a_n}\right)\equiv X_1^{a_1}\cdots X_n^{a_n}\pmod{\mc F_iU\mf g}\]
	% 	if the total degree of $X_1^{a_1}\cdots X_n^{a_n}$ is $i+1$: all summands in $\op{sym}\left(X_1^{a_1}\cdots X_n^{a_n}\right)$ can be reordered to return to the ordered monomial $X_1^{a_1}\cdots X_n^{a_n}$ via \Cref{rem:spit-error-term} (up to elements of $\mc F_iU\mf g$). Thus, we see that all ordered monomials of total degree $i+1$ are in the image, which completes this check.
	%
	% 	\item Injective: we will have to use \Cref{thm:pbw}. Suppose that $p(X_1,\ldots,X_n)\in S\mf g$ is a nonzero polynomial. We will show that $\op{sym}(p)\ne0$. For example, if $p$ is constant, then of course $\op{sym}(p)=p$ is nonzero. Otherwise, let the total degree of $p$ be $i+1$ where $i\ge0$, and we will show $\op{sym}(p)\notin\mc F_iU\mf g$. Note that any terms in $p(X_1,\ldots,X_n)$ of total degree $i$ go to elements of $\mc F_iU\mf g$ by the construction of $\op{sym}$, so we may throw these terms to assume that $p(X_1,\ldots,X_n)$ is homogeneous of degree $i+1$. As such, we write
	% 	\[p(X_1,\ldots,X_n)=\sum_{\substack{(d_1,\ldots,d_n)\in\NN^n\\d_1+\cdots+d_n=i+1}}a_{(d_1,\ldots,d_n)}X_1^{d_1}\cdots X_n^{d_n}.\]
	% 	Now, we again use \Cref{rem:spit-error-term} to reorder our monomials$\pmod{\mc F_iU\mf g}$ so that
	% 	\[\op{sym}(p)\equiv\sum_{\substack{(d_1,\ldots,d_n)\in\NN^n\\d_1+\cdots+d_n=i+1}}a_{(d_1,\ldots,d_n)}X_1^{d_1}\cdots X_n^{d_n}\pmod{\mc F_iU\mf g}.\]
	% 	Thus, adding in terms of degree at most $i$, we see
	% 	\[\op{sym}(p)=\sum_{\substack{(d_1,\ldots,d_n)\in\NN^n\\d_1+\cdots+d_n\le i+1}}a_{(d_1,\ldots,d_n)}X_1^{d_1}\cdots X_n^{d_n}.\]
	% 	However, $p$ is nonzero, so some coefficient on the right-hand side is nonzero, so the right-hand side is nonzero because ordered monomials are linearly independent (by \Cref{thm:pbw}!), so $\op{sym}(p)\ne0$.
	%
	% 	\item $\mf g$-module: for any $X\in\mf g$, we must check that $\op{sym}(\op{ad}_Xp)=\op{ad}_X\op{sym}(p)$ for any $p\in S\mf g$. This is an equality of linear functions $S\mf g\to U\mf g$, so it is enough to check it on a spanning subset of $S\mf g$, so we will choose the spanning subset of pure tensors. As such, for $Y_1,\ldots,Y_k\in\mf g$, we compute
	% 	\[\op{sym}(\op{ad}_X(Y_1\cdots Y_k)) = \op{sym}\Bigg(\sum_{i=1}^kY_1\cdots Y_{i-1}[X,Y_i]Y_{i+1}\cdots Y_k\Bigg).\]
	% 	To move $\op{sym}$ inside, we need some notation: for each $i$, set $Y^i_j$ to be $Y_j$ if $i\ne j$ and $[X,Y_i]$ if $i=j$. Then
	% 	\begin{align*}
	% 		\op{sym}(\op{ad}_X(Y_1\cdots Y_k)) &= \sum_{i=1}^k\op{sym}\left(Y_1^i\cdots Y_k^i\right) \\
	% 		&= \frac1{k!}\sum_{i=1}^k\sum_{\sigma\in S_k}Y_{\sigma(1)}^i\cdots Y_{\sigma(k)}^i \\
	% 		&= \frac1{k!}\sum_{\sigma\in S_k}\sum_{i=1}^kY_{\sigma(1)}^i\cdots Y_{\sigma(k)}^i.
	% 	\end{align*}
	% 	We now acknowledge that the inner sum simply equals $\op{ad}_X\left(Y_{\sigma(1)}\cdots Y_{\sigma(k)}\right)$ after some rearranging of summands, so
	% 	\begin{align*}
	% 		\op{sym}(\op{ad}_X(Y_1\cdots Y_k)) &= \frac1{k!}\sum_{\sigma\in S_k}\op{ad}_X\left(Y_{\sigma(1)}\cdots Y_{\sigma(k)}\right) \\
	% 		&= \op{ad}_X\bigg(\frac1{k!}\sum_{\sigma\in S_k}Y_{\sigma(1)}\cdots Y_{\sigma(k)}\bigg) \\
	% 		&= \op{ad}_X(\op{sym}(Y_1\cdots Y_k)),
	% 	\end{align*}
	% 	as required.
	% 	\qedhere
	% \end{itemize}
\end{proof}
\begin{corollary}
	Fix a Lie algebra $\mf g$ over a field $F$. The map $\op{sym}\colon S\mf g\to U\mf g$ of \Cref{prop:sg-is-ug} restricts to an isomorphism
	\[(S\mf g)^{\mf g}\to Z(U\mf g).\]
\end{corollary}
\begin{proof}
	Because $\op{sym}$ is an isomorphism of $\mf g$-modules, we see that $\mf g$ will restrict to an isomorphism of $\mf g$-invariants. On one hand, the $\mf g$-invariants of $S\mf g$ make the space $(S\mf g)^{\mf g}$. On the other hand, we claim $(U\mf g)^{\mf g}=Z(U\mf g)$, which will complete the proof. In one direction, note $a\in Z(U\mf g)$ implies that $Xa=aX$ for all $X\in\mf g$, so $\op{ad}_X(a)=0$ by \Cref{lem:ad-action-ug}. In the other direction, if $a\in(U\mf g)^{\mf g}$, we merely know $\op{ad}_X(a)=Xa-aX$ vanishes for all $X\in\mf g$. Thus, we define
	\[C(a)\coloneqq\{b\in U\mf g:ab=ba\}.\]
	Note that $C(a)\subseteq U\mf g$ is the kernel of the linear map $b\mapsto ab-ba$, so it is a linear subspace. Additionally, $C(a)$ is closed under multiplication: if $b,b'\in C(a)$, then $abb'=bab'=bb'a$, so $bb'\in C(a)$. Now, we would like to check that $C(a)=U\mf g$, for which it is enough to check that $C(a)$ contains monomials (because these span $U\mf g$), for which it is enough to check that $\mf g\subseteq C(a)$ (because monomials are products of elements of $\mf g$).
\end{proof}
% \begin{remark}
% 	As a consequence, we see that the invariant subspace of $U\mf g$ by the adjoint action is the same as for $S\mf g$, which of course is $Z(U\mf g)$. For example, if $G$ is connected with $\mf g=\op{Lie}G$, then one sees that these are also the $G$-invariants.
% \end{remark}
% \begin{example}
% 	One can check that the Casimir element $C=ef+fe+\frac12h^2$ lives in the center of $Z(\mf{sl}_2)$. In fact, one can show that $Z(\mf{sl}_2)$ is generated by $C$.
% \end{example}

\subsection{Ideals and Commutants}
One can check that sums of ideals are ideals. Also, one sees that $\mf g/[\mf g,\mf g]$ is the maximal abelian quotient: if $I\subseteq\mf g$ is an ideal with $\mf g/I$ abelian, then we must have $[\mf g,\mf g]\subseteq I$.
\begin{example}
	One can check that $[\mf{gl}_n,\mf{gl}_n]\subseteq\mf{sl}_n$ because the trace of $XY-YX$ is zero for any $X,Y\in\mf{gl}_n$. In fact, this is an equality, which one can check by hand.
\end{example}
These commutants provide a derived series: we define $\left\{D^i\mf g\right\}_{i\ge0}$ inductively by $D^0\mf g\coloneqq\mf g$ and
\[D^{i+1}\mf g\coloneqq\left[D^i\mf g,D^i\mf g\right]\]
for all $i\ge0$. This derived series plays the role of derived series in group theory. For example, one can use this to define solvability.
\begin{proposition}
	Fix a Lie algebra $\mf g$. Then the following are equivalent.
	\begin{listalph}
		\item $D^n\mf g=0$ for $n$ sufficiently large.
		\item There exists a sequence of subalgebras
		\[\mf g=\mf a^0\supseteq\mf a^1\supseteq\cdots\supseteq\mf a^k=0\]
		such that $\mf a^{i+1}$ is an ideal in $\mf a^i$ with abelian quotient.
		\item For every $n$ sufficiently large and sequence of elements $\{x_1,\ldots,x_{2^n}\}\subseteq\mf g$, the $n$-fold commutator
		\[[\cdots[[x_1,x_2],[x_3,x_4]],\cdots]\]
		vanishes.
	\end{listalph}
\end{proposition}
\begin{proof}
	The equivalence of (a) and (c) has no content. Note that (a) implies (b) because one may take $\mf a^i=D^i\mf g$. One achieves (b) implies (a) by showing that $\mf a^i\supseteq D^i\mf g$ inductively.
\end{proof}

\end{document}