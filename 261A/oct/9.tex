% !TEX root = ../notes.tex

\documentclass[../notes.tex]{subfiles}

\begin{document}

\section{October 9}
Today we continue discussing the universal enveloping algebra.

\subsection{More on the Universal Enveloping Algebra}
Last time we showed that $\mf g$ acts on $U\mf g$ by derivations. In fact, for any $Z\in\mf g$, one can check that
\[\op{ad}_Z(a)=aZ-Za\]
for all $a\in U\mf g$.
\begin{remark}
	One can also have $\mf g$ act on the symmetric algebra $S\mf g$, which is the quotient of $T\mf g$ by the two-sided ideal generated by the elements $x\otimes y-y\otimes x$. The point is that we have produced some interesting infinite-dimensional representations.
\end{remark}
Note that there is a natural filtration on $T\mf g$ which can sometimes ``temper'' our infinite-dimensional representations. Here is the definition.
\begin{definition}[filtered]
	An algebra $A$ is \textit{filtered} if and only if there exists a filtration
	\[0=\mc F_0A\subseteq\mc F_1A\subseteq\cdots\subseteq\]
	such that $A=\bigcup_{i\ge0}\mc F_iA$, and the multiplication in $A$ sends $\mc F_iA\otimes\mc F_jA\to\mc F_{i+j}A$.
\end{definition}
A filtration permits us to pay close attention to the quotients.
\begin{definition}[associated graded algebra]
	Fix a filtered algebra $A$ with filtration $\{\mc F_iA\}_{i\ge0}$. Then the \textit{associated graded algebra} is
	\[\op{gr}A\coloneqq\bigoplus_{i\ge1}\frac{\mc F_iA}{\mc F_{i+1}A}.\]
\end{definition}
\begin{remark}
	One can check that $\op{gr}A$ is a graded algebra. Conversely, given a graded algebra $A$ with grading $A=\bigoplus_{i=1}^\infty A_i$, one can produce a filtered algebra with the filtration
	\[\mc F_iA\coloneqq\bigoplus_{j\le i}A_j.\]
\end{remark}
\begin{remark}
	Fix a filtered algebra $A$. If $\op{gr}A$ has no zero divisors, then one can check that $A$ also has no zero divisors.
\end{remark}
Now, the point is that $T\mf g=\bigoplus_{i\ge0}\mf g^{\otimes i}$ is graded, but we cannot expect $U\mf g$ to be graded because the relation $x\otimes y-y\otimes x=[x,y]$ is a relation between the second and first degree! But we can still pass to the filtration: let $\mc F_iU\mf g$ simply be the image of the natural grading $\mc F_iT\mf g$ under the surjection $T\mf g\onto U\mf g$.
\begin{remark}
	In contrast, one can check that $S\mf g$ is not just filtered but graded!
\end{remark}
The definition of the filtration quickly implies that $\mc F_iU\mf g\cdot\mc F_jU\mf g\subseteq\mc F_{i+j}U\mf g$ and thus $[\mc F_iU\mf g,\mc F_jU\mf g]\subseteq\mc F_{i+j-1}U\mf g$. One can also see that $\op{gr}U\mf g$ is then commutative, and there is a canonical map $S\mf g\to\op{gr}U\mf g$.
\begin{theorem}[Poincar\'e--Birkoff--Witt]
	Fix a Lie algebra $\mf g$. Then the canonical map $S\mf g\to\op{gr}U\mf g$ is an isomorphism.
\end{theorem}
\begin{remark}
	Intuitively, one can say that ordered monomials are linearly independent in $U\mf g$.
\end{remark}
This is an important theorem, but we will not prove it because it is somewhat tedious.

The remarks above tell us that the map is surjective and well-defined, so the main content is in the injectivity of the map. Here are some corollaries.
\begin{corollary}
	The map $\mf g\to U\mf g$ is injective.
\end{corollary}
\begin{remark}
	In fact, defining $U\mf g$ for an arbitrary bilinear map $[-,-]\colon\mf g\otimes\mf g\to\mf g$. It turns out that the induced map $\mf g\to U\mf g$ will succeed at being injective if and only if $[-,-]$ is skew-symmetric and satisfies a Jacobi identity. For example, we need
	\[x\otimes x-x\otimes x=[x,x]\]
	to vanish.
\end{remark}
\begin{corollary}
	The algebra $U\mf g$ has no zero divisors.
\end{corollary}
One may a map directly between $U\mf g$ and $S\mf g$ instead of passing to $\op{gr}$.
\begin{proposition}
	The map $\op{sym}\colon S\mf g\to U\mf g$ defined by
	\[\op{sym}(x_1\cdots x_p)\coloneqq\frac1{p!}\sum_{\omega\in S_p}X_{\sigma(1)}\cdots x_{\sigma(p)}\]
	is an isomorphism of $\mf g$-modules (via the adjoint action).
\end{proposition}
\begin{remark}
	As a consequence, we see that the invariant subspace of $U\mf g$ by the adjoint action is the same as for $S\mf g$, which of course is $Z(U\mf g)$. For example, if $G$ is connected with $\mf g=\op{Lie}G$, then one sees that these are also the $G$-invariants.
\end{remark}
\begin{example}
	One can check that the Casimir element $C=ef+fe+\frac12h^2$ lives in the center of $Z(\mf{sl}_2)$. In fact, one can show that $Z(\mf{sl}_2)$ is generated by $C$.
\end{example}

\subsection{Ideals and Commutants}
One can check that sums of ideals are ideals. Also, one sees that $\mf g/[\mf g,\mf g]$ is the maximal abelian quotient: if $I\subseteq\mf g$ is an ideal with $\mf g/I$ abelian, then we must have $[\mf g,\mf g]\subseteq I$.
\begin{example}
	One can check that $[\mf{gl}_n,\mf{gl}_n]\subseteq\mf{sl}_n$ because the trace of $XY-YX$ is zero for any $X,Y\in\mf{gl}_n$. In fact, this is an equality, which one can check by hand.
\end{example}
These commutants provide a derived series: we define $\left\{D^i\mf g\right\}_{i\ge0}$ inductively by $D^0\mf g\coloneqq\mf g$ and
\[D^{i+1}\mf g\coloneqq\left[D^i\mf g,D^i\mf g\right]\]
for all $i\ge0$. This derived series plays the role of derived series in group theory. For example, one can use this to define solvability.
\begin{proposition}
	Fix a Lie algebra $\mf g$. Then the following are equivalent.
	\begin{listalph}
		\item $D^n\mf g=0$ for $n$ sufficiently large.
		\item There exists a sequence of subalgebras
		\[\mf g=\mf a^0\supseteq\mf a^1\supseteq\cdots\supseteq\mf a^k=0\]
		such that $\mf a^{i+1}$ is an ideal in $\mf a^i$ with abelian quotient.
		\item For every $n$ sufficiently large and sequence of elements $\{x_1,\ldots,x_{2^n}\}\subseteq\mf g$, the $n$-fold commutator
		\[[\cdots[[x_1,x_2],[x_3,x_4]],\cdots]\]
		vanishes.
	\end{listalph}
\end{proposition}
\begin{proof}
	The equivalence of (a) and (c) has no content. Note that (a) implies (b) because one may take $\mf a^i=D^i\mf g$. One achieves (b) implies (a) by showing that $\mf a^i\supseteq D^i\mf g$ inductively.
\end{proof}

\end{document}