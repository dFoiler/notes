% !TEX root = ../notes.tex

\documentclass[../notes.tex]{subfiles}

\begin{document}

\section{October 21}
Today there is a sub. We are going to continue talking about root decompositions.

\subsection{More on Root Decompositions}
As usual, $\mf g$ is semisimple, and we give it a Cartan subalgebra $\mf h\subseteq\mf g$. Then recall the decomposition
\[\mf g=\mf h\oplus\bigoplus_{\alpha\in\Phi}\mf g_{\alpha},\]
where $\Phi$ is our root system. One can show that Cartan subalgebras are unique up to conjugation, which shows that the root system $\Phi$ is essentially independent of the choice of $\mf h$.

Quickly, we note that the study of root systems immediately goes down to simple Lie algebras.
\begin{proposition}
	Fix a semisimple Lie algebra $\mf g$ over an algebraically closed field $F$ of characteristic $0$. Write $\mf g$ as a sum $\mf g_1\oplus\cdots\oplus\mf g_n$ of simple Lie algebras.
	\begin{listalph}
		\item For each $i$, choose a Cartan subalgebra $\mf h_i\subseteq\mf g_i$, producing a root system $\Phi_i$. Then $\mf h_1\oplus\cdots\oplus\mf h_n$ is a Cartan subalgebra $\mf g$, and the corresponding root system $\Phi$ is disjoint union $\Phi_1\sqcup\cdots\sqcup\Phi_n$.
		\item Any Cartan subalgebra of $\mf g$ is a direct sum of Cartan subalgebras of the $\mf g_\bullet$s.
	\end{listalph}
\end{proposition}
\begin{proof}
	Note that (a) has little content: one simply has to compute the centralizer to see that $\mf h_1\oplus\cdots\oplus\mf h_n$ is Cartan, which is doable because the factors have split up $\mf g$ into a direct sum already. (Of course this sum is abelian.) For (b), let $\mf h$ be some Cartan algebra of $\mf g$, and we set $\mf h_i$ to be $\op{pr}_i(\mf h)$. One can check then that $\mf h_i$ equals its own centralizer in $\mf g_i$ by lifting to $\mf h$; thus, $\mf h$ is contained in the Cartan subalgebra
	\[\mf h_1\oplus\cdots\oplus\mf h_n,\]
	so equality follows because Cartan subalgebras are maximal.
\end{proof}
We set some notation for future use. Fix a semisimple Lie algebra $\mf g$ over an algebraically closed field $F$ of characteristic $0$, and let $\mf h\subseteq\mf g$ be a Cartan subalgebra. Let $B$ be an invariant non-degenerate bilinear form on $\mf g$, which induces an isomorphism $\mf h\to\mf h^\lor$ denoted $H_\alpha\mapsto\alpha$ for any $\alpha\in\mf h^\lor$. Explicitly, we have that
\[B(\alpha,\beta)=\langle H_\alpha,\beta\rangle=B^\lor(H_\alpha,H_\beta)\]
for any $\alpha,\beta\in\mf h^\lor$.
\begin{lemma}
	Fix notation as above. For any $e\in\mf g_{-\alpha}$ and $f\in\mf g_{-\alpha}$, we have
	\[[e,f]=(e,f)H_\alpha.\]
\end{lemma}
\begin{proof}
	It is enough to compute $([e,f],h)$ for any $h\in\mf h$ and compare it with $H_\alpha$. Well, by the invariance, we see that
	\[B([e,f],h)=-B(e,[h,f])=\langle h,\alpha\rangle B(e,f)=B(e,f)B(h,H_\alpha),\]
	so we get the result by the non-degeneracy of $B$ on $\mf h$.
\end{proof}
\begin{lemma}
	Fix notation as above. If $\alpha\in\Phi$, then $B(\alpha,\alpha)\ne0$.
\end{lemma}
\begin{proof}
	Choose $e\in\mf g_\alpha$ and $f\in\mf g_{-\alpha}$ so that $B(e,f)\ne0$, which is possible because $B$ restricts to a non-degenerate form on $\mf g_\alpha\times\mf g_{-\alpha}$. Now, define $h\coloneqq[e,f]=B(e,f)H_\alpha$ (via the previous lemma).  Now, consider the algebra $\mf a$ generated by $\{e,f,h\}$. Notably,
	\[[h,e]=\langle h,\alpha\rangle e=B(\alpha,\alpha)B(e,f)e,\]
	and
	\[[h,f]=-\langle h,\alpha\rangle f=-B(\alpha,\alpha)B(e,f)f.\]
	Thus, supposing for the sake of contradiction that $B(\alpha,\alpha)=0$, then we see that our Lie algebra $\mf a$ is solvable (!), so Lie's theorem allows us to choose a basis in $\mf g$ such that the adjoint operators ${\op{ad}_e},{\op{ad}_f},{\op{ad}_h}$ are upper-triangular. Because $h=[e,f]$, we see that $h$ is strictly upper-triangular and hence nilpotent. On the other hand, $h\in\mf h$ lives in a Cartan subalgebra, so $h$ is also semisimple, so we conclude $h=0$, so $B(e,f)H_\alpha=h$ vanishes, so $B(e,f)$ vanishes, which is a contradiction to its construction!
\end{proof}
\begin{proposition}
	Fix notation as above, and choose $e\in\mf g_\alpha$ and $f\in\mf g_{-\alpha}$ such that $B(e,f)=2/B(\alpha,\alpha)$. Then let
	\[h_\alpha\coloneqq\frac{2H_\alpha}{B(\alpha,\alpha)}.\]
	Then $\langle h_\alpha,\alpha\rangle=2$, and the elements $\{e,f,h_\alpha\}$ induce an embedding $\mf{sl}_2(F)\to\mf g$, whose image we denote by $\mf{sl}_2(F)_\alpha$.
\end{proposition}
\begin{proof}
	One checks the commutator relations everywhere (as we did in the previous lemma) to see that we have indeed defined an embedding.
\end{proof}
\begin{remark}
	One can check that the subalgebra $\mf{sl}_2(F)_\alpha$ is independent of the choices we made. For example, we can reduce to the case where this is simple, and then one can check directly (by adjusting everything by scalar) that $h_\alpha$ is independent of $B$, and then we see that the choices of $e$ and $f$ also do not adjust the end product $\mf{sl}_2(F)_\alpha$.
\end{remark}
This copy of $\mf{sl}_2(F)_\alpha$ allows us to reduce some study of $\mf g$ back down to $\mf{sl}_2$, which we understand well.
\begin{lemma}
	Fix notation as above, and choose some $\alpha\in\Phi$. Then
	\[V\coloneqq\CC h_\alpha\oplus\bigoplus_{\substack{k\in\ZZ\\k\ne0}}\mf g_{k\alpha}\]
	is an irreducible representation of $\mf{sl}_2(F)_\alpha$.
\end{lemma}
\begin{proof}
	Choose $e$ and $f$ as usual. Note that $[\mf g_{k\alpha},e]\subseteq\mf g_{(k+1)\alpha}$, and $[e,\mf g_\alpha]\subseteq F\mf h_\alpha$, and one has similar results for $f$. Then we see that $V$ is a representation of $\mf{sl}_2(F)_\alpha$, and one can explicitly compute its weight decomposition as having $V[k]=0$ when $k$ is odd and $V[2k]=\mf g_{k\alpha}$ for nonzero $k$, and $V[0]=Fh_\alpha$ is one-dimensional. Our classification of irreducible representations of $\mf{sl}_2(F)_\alpha$ allows us to conclude from staring at the $0$-dimensional subspace.
\end{proof}
\begin{remark}
	By looking at $\mf g_\alpha\subseteq V$, we see that $\mf g_\alpha$ must be one-dimensional by the classification of irreducible representations of $\mf{sl}_2$.
\end{remark}
Let's prove a few more things.
% \begin{lemma}
% 	Fix notation as above. For any $\alpha\in\Phi$, if $k\alpha\in\Phi$ for $k\in\ZZ$, then $k\in\{\pm1\}$.
% \end{lemma}
% \begin{proof}
	
% \end{proof}
\begin{lemma}
	Fix notation as above. Then $\Phi$ spans $\mf h^\lor$.
\end{lemma}
\begin{proof}
	Find $h\in\mf h$ such that $\langle h,\alpha\rangle=0$ for all $\alpha\in\Phi$, and we will show that $h=0$; this implies that $\Phi^\perp=0$ and hence that $\Phi$ spans $\mf h$. But then our root decomposition forces $\op{ad}_h$ to vanish, so $h$ is in the center of $\mf g$, so $h=0$ because $\mf g$ is semisimple.
\end{proof}
\begin{lemma}
	Fix notation as above. For any two roots $\alpha$ and $\beta$, we have
	\[\frac{2B(\alpha,\beta)}{(\alpha,\alpha)}\in\ZZ.\]
\end{lemma}
\begin{proof}
	View $\mf g$ as a representation of $\mf{sl}_2(F)_\alpha$. Then all weight spaces are known to be given by integers, so we note that the elements of $\mf g_\beta$ have weight given by $\langle h_\alpha,\beta\rangle$, which is the number in question.
\end{proof}

\end{document}