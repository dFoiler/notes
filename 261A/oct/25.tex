% !TEX root = ../notes.tex

\documentclass[../notes.tex]{subfiles}

\begin{document}

\section{October 21}
Today we begin discussing root decompositions.

\subsection{Jordan Decomposition for Semisimple Lie Algebras}
Fix a Lie algebra $\mf g$ over a field $F$. We begin with some motivational discussion. We would like to make sense of decomposing $\mf g$ into eigenspaces; here is one such avatar of this.
\begin{lemma} \label{lem:decompose-eigenspace-lie-alg}
	Fix a Lie algebra $\mf g$ over an algebraically closed field $F$. For any $X\in\mf g$, let $\mf g_\lambda$ be the generalized eigenspace for the operator $\op{ad}_X\colon\mf g\to\mf g$ with eigenvalue $\lambda$. Then the decomposition
	\[\mf g=\bigoplus_{\lambda\in F}\mf g_\lambda\]
	has $[\mf g_\lambda,\mf g_\mu]\subseteq\mf g_{\lambda+\mu}$ for any $\lambda,\mu\in F$.
\end{lemma}
\begin{proof}
	Choose $Y\in\mf g_\lambda$ and $Z\in\mf g_\mu$ so that we want to show $({\op{ad}_X}-(\lambda+\mu))^\bullet[Y,Z]$ vanishes for large enough power. Well, selecting some $N>0$, we see
	\[({\op{ad}_X}-(\lambda+\mu))^N=\sum_{a+b+c=N}\binom{N}{a,b,c}{\op{ad}_X^a}(-\lambda)^b(-\mu)^n,\]
	which when applied to $[Y,Z]$ will rearrange into
	\[\sum_{k+\ell=N}\binom Nk\left[({\op{ad}_X}-\lambda)^kY,({\op{ad}_X}-\mu)^\ell Z\right].\]
	This will vanish for $N$ large enough, so we are done.
\end{proof}
In order to make this decomposition a diagonalization, we pick up the following definitions.
\begin{defihelper}[semisimple, nilpotent] \nirindex{semisimple} \nirindex{nilpotent}
	Fix a Lie algebra $\mf g$. Then $X\in\mf g$ is \textit{semisimple} if and only if $\op{ad}_X\colon\mf g\to\mf g$ is a semisimple operator. Further, $X\in\mf g$ is \textit{nilpotent} if and only if $\op{ad}_X\colon\mf g\to\mf g$ is nilpotent.
\end{defihelper}
\begin{remark}
	Note $X\in\mf g$ is both semisimple and nilpotent if and only if $\op{ad}_X$ vanishes, which is equivalent to $X\in\mf z(\mf g)$.
\end{remark}
The Jordan decomposition grants the following.
\begin{proposition} \label{prop:ss-jordan}
	Fix a Lie algebra $\mf g$ over a perfect field $F$. Then each $X\in\mf g$ admits a unique decomposition
	\[X=X_s+X_n\]
	where $X_s$ is semisimple, $X_n$ is nilpotent, and $[X_s,X_n]=0$. In fact, if $Y\in\mf g$ has $[X,Y]=0$, then $[X_s,Y]=0$.
\end{proposition}
\begin{proof}
	Because $\mf g$ is semisimple, we may embed $\mf g\subseteq\mf{gl}(\mf g)$ via $\op{ad}_\bullet$. Now, in $\mf{gl}(\mf g)$, we do have a decomposition $X=X_s+X_n$ where $X_s,X_n\in\mf{gl}(\mf g)$ are semisimple and nilpotent (respectively) and satisfy $X_sX_n=X_nX_s$ (so that $[X_s,X_n]=0$), and we know that this is the only possible decomposition. Furthermore, we see that $X_s$, being a polynomial in $X$ in $\mf{gl}(\mf g)$, implies that $[X,Y]=0$ forces $[X_s,Y]=0$.
	
	It remains to check that these elements actually live in $\mf g$. Well, for each $\lambda\in\mf g$, define $\mf g_\lambda$ as in \Cref{lem:decompose-eigenspace-lie-alg}, and our construction of $X_s$ yields $[X_s,Y]=\lambda Y$ for each $Y\in\mf g_\lambda$. Thus, the conclusion of \Cref{lem:decompose-eigenspace-lie-alg} allows us to check that $\op{ad}_{X_s}\colon\mf g\to\mf g$ is a derivation! However, all derivations come from $\mf g$ because $\mf g$ is semisimple, so $X_s\in\mf g$ after all. Thus, $X_n=X-X_s$ lives in $\mf g$ as well, so we are done.
\end{proof}
\begin{remark}
	For $\mf{sl}_n(F)$, one can check that the above Jordan decomposition coincides with the usual one.
\end{remark}
\begin{corollary}
	Suppose $\mf g$ is a nonzero semisimple Lie algebra over a perfect field $F$. Then there exists a nonzero semisimple element $X$.
\end{corollary}
\begin{proof}
	If not, \Cref{prop:ss-jordan} forces all elements of $\mf g$ to be nilpotent, so $\mf g$ is nilpotent, so $\mf g=0$ because $\mf g$ is semisimple.
\end{proof}

\subsection{The Root Decomposition}
We are now interested in repeating our eigenvalue decomposition for semisimple elements.
\begin{definition}[toral]
	Fix a Lie algebra $\mf g$. A subalgebra $\mf h\subseteq\mf g$ is \textit{toral} if and only if $\mf h$ is abelian, and all elements of $\mf h$ are semisimple.
\end{definition}
\begin{example}
	For $\mf{sl}_n(F)$, we see that there is a subalgebra $\mf h$ of the diagonal matrices.
\end{example}
We now decompose with respect to toral subalgebras.
\begin{proposition} \label{prop:toral-root-decomposition}
	Fix a semisimple Lie algebra $\mf g$ over an algebraically closed field $F$, and let $\mf h\subseteq\mf g$ be a toral subalgebra. Suppose $B$ is a non-degenerate symmetric bilinear form on $\mf g$.
	\begin{listalph}
		\item For each functional $\alpha\colon\mf h\to F$, let $\mf g_\alpha$ be the eigenspace of $\mf g$ with the corresponding eigenvalue $\alpha$. Then
		\[\mf g=\bigoplus_{\alpha\in\mf h^\lor}\mf g_\alpha.\]
		\item For each $\alpha,\beta\in\mf h^\lor$, we have $[\mf g_\alpha,\mf g_\beta]\subseteq\mf g_{\alpha+\beta}$.
		\item We have $B(\mf g_\alpha,\mf g_\beta)\ne0$ if and only if $\alpha+\beta=0$.
		\item The bilinear form $B$ is non-degenerate when restricted to $\mf g_\alpha\times\mf g_{-\alpha}$ for any $\alpha\in\mf h^\lor$.
	\end{listalph}
\end{proposition}
\begin{proof}
	Note (a) and (b) are immediate from \Cref{lem:decompose-eigenspace-lie-alg}, where we use the fact that $\mf h$ has semisimple elements in order to diagonalize, and we use the fact that $\mf h$ is abelian to simultaneously diagonalize. For (c), we choose $Y\in\mf g_\alpha$ and $Z\in\mf g_\beta$ and $X\in\mf h$, and we see
	\[\alpha(X)B(Y,Z)=B(\op{ad}_XY,Z)=B(Y,\op{ad}_XZ)=\beta(X)B(Y,Z)\]
	by the invariance of the bilinear form. Thus, $(\alpha+\beta)(X)=0$ for all $X\in\mf h$, or $B(Y,Z)=0$ for all $Y\in\mf g_\alpha$ and $Z\in\mf g_\beta$, which proves (c). Now, (d) follows because $B$ must be non-degenerate, and (c) tells us that $B$ vanishes except on the given subspaces.
\end{proof}
\begin{remark}
	Note that $\mf g_0$ is the commutator of $\mf h$ by its definition.
\end{remark}
\begin{remark}
	In fact, we claim that $\mf g_0\subseteq\mf g$ is reductive because the invariant bilinear form $B$ restricts to be non-degenerate on $\mf g$.
\end{remark}
\begin{remark}
	Furthermore, we see that $X\in\mf g_0$ will imply that $X_s\in\mf g_0$ because $X$ commutes with $\mf h$ implies that $X_s$ commutes with $\mf g$; thus, we are also able to say that $X_n\in\mf h$.
\end{remark}
In order to profit the most from our toral subalgebra, we would like for it to be large.
\begin{definition}[Cartan subalgebra]
	Fix a Lie algebra $\mf g$. Then a toral subalgebra $\mf h\subseteq\mf g$ is \textit{Cartan} if and only if $\mf h$ equals its own centralizer.
\end{definition}
\begin{example}
	One can check by hand that the diagonal torus of $\mf{sl}_n(F)$ is Cartan.
\end{example}
Let's check that such things exist.
\begin{lemma}
	Fix a semisimple Lie algebra $\mf g$ over a perfect field $F$ of characteristic $0$. Then a toral subalgebra $\mf h\subseteq\mf g$ is a Cartan subalgebra if and only if $\mf h$ is a maximal toral subalgebra.
\end{lemma}
\begin{proof}
	We show our implications separately. Let $C(\mf h)$ be the centralizer of $\mf h$.
	\begin{itemize}
		\item If $\mf h$ is Cartan, then we note that $\mf h$ is a maximal torus. Indeed, if we have a toral subalgebra $\mf h'$ such that $\mf h\subseteq\mf h'$, then $\mf h'$ is contained in the centralizer of $\mf h$, so $\mf h'\subseteq\mf h$ because $\mf h$ is Cartan.

		\item Suppose that $\mf h$ is a maximal toral subalgebra, and choose some $X$ commuting with $\mf h$, and we want to show that $X\in\mf h$. Well, \Cref{prop:ss-jordan} implies $X_s$ also commutes with $\mf h$, so $\mf h+FX_s$ is an abelian algebra consisting of semisimple elements (these elements commute, so sums of them will be simultaneously diagonalizable, so this algebra still has semisimple elements), so $X_s\in\mf h$ by the maximality of $\mf h$.

		Now, $C(\mf h)$ is reductive by an above remark, and we see that all $X\in C(\mf h)$ will have $\op{ad}_{X_s}$ vanish on $C(\mf h)$, so $\op{ad}_X$ must be nilpotent on $C(\mf h)$. Thus, $C(\mf h)$ is nilpotent as well by Engel's theorem, so $C(\mf h)$ must be abelian.

		We would like to show that $C(\mf h)$ is furthermore consisting of semisimple elements, which will complete the proof by the ambient maximality. Well, for any nilpotent $X\in C(\mf h)$, we see that any $Y\in C(\mf h)$ has $[Y,X]=0$ by the commutativity, so the composite ${\op{ad}_X}\circ{\op{ad}_Y}$ is nilpotent as an operator $\mf g\to\mf g$ (because $X$ is nilpotent, and these operators commute), so
		\[\tr_{\mf g}({\op{ad}_X}\circ{\op{ad}_Y})=0,\]
		so $X$ is in the kernel of the Killing form of $C(\mf h)$, so $X=0$ because the Killing form is non-degenerate.
		\qedhere
	\end{itemize}
\end{proof}
We are now ready to state our root decompositions.
\begin{corollary}[root decomposition]
	Fix a semisimple Lie algebra $\mf g$ over an algebraically closed field $F$ of characteristic $0$ and choose a Cartan subalgebra $\mf h\subseteq\mf g$. Let $\Phi$ be the collection of nonzero $\alpha\in\mf h^\lor$ such that $\mf g_\alpha\ne0$. Then
	\[\mf g=\mf h\oplus\bigoplus_{\alpha\in\Phi}\mf g_\alpha.\]
\end{corollary}
\begin{proof}
	Immediate from \Cref{prop:toral-root-decomposition}.
\end{proof}
\begin{definition}[root system]
	Fix a semisimple Lie algebra $\mf g$ over an algebraically closed field $F$ of characteristic $0$ and choose a Cartan subalgebra $\mf h\subseteq\mf g$. Then the collection $\Phi$ of nonzero $\alpha\in\mf h^\lor$ such that $\mf g_\alpha\ne0$ is called the \textit{root system}.
\end{definition}
\begin{example}
	For $\mf{sl}_n(F)$, we choose $\mf h$ to be the diagonal subalgebra. Let $e_i\in\mf h^\lor$ be the projection onto the $(i,i)$ coordinate. Then we compute that $\op{ad}_X(E_{ij})=(X_i-X_j)E_{ij}$ for any $X\in\mf h$, so our root system consists of the functionals $e_i-e_j$ for each $i\ne j$.
\end{example}

\end{document}