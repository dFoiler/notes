% !TEX root = ../notes.tex

\documentclass[../notes.tex]{subfiles}

\begin{document}

\section{October 30}
We continue discussing properties of our root decompositions.

\subsection{Even More on Root Decompositions}
As usual, $\mf g$ is semisimple, and we give it a Cartan subalgebra $\mf h\subseteq\mf g$. Then recall the decomposition
\[\mf g=\mf h\oplus\bigoplus_{\alpha\in\Phi}\mf g_{\alpha},\]
where $\Phi$ is our root system. Last time, we argued that
\[V_{\alpha,\beta}=\bigoplus_{k\in\ZZ}\mf g_{\beta+k\alpha}\]
is an irreducible representation of $\mf{sl}_2(F)_\alpha$. (We actually showed this for $\beta=0$, but the same proof goes through once we establish that one of the weight spaces is one-dimensional.) This will allow us to define a reflection operator.
\begin{lemma} \label{lem:reflect-root-system}
	Fix notation as above, and choose some $\alpha\in\Phi$. Then define $s_\alpha\colon\mf h^\lor\to\mf h^\lor$ by
	\[s_\alpha(\lambda)\coloneqq\lambda-2\frac{B(\lambda,\lambda)}{B(\alpha,\alpha)}\alpha.\]
	Then $s_\alpha$ maps $\Phi\to\Phi$.
\end{lemma}
\begin{proof}
	Given $\beta\in\Phi$, we want to check that $s_\alpha(\beta)\in\Phi$. Well, the point is that $s_\alpha(\beta)$ is $\beta$ minus some number of copies $m\coloneqq2B(\beta,\beta)/B(\alpha,\alpha)$ of $\alpha$. To show that $s_\alpha(\beta)$ lives in $\Phi$, we should find a nonzero vector in $\mf g_{s_\alpha(\beta)}$. For example, if $m\ge0$, we choose nonzero $X\in\mf g_\beta$ and then find that $f_\alpha^mX\in\mf g_{\alpha-m\beta}$, so we get our root. The point is that $m$ is small enough so that we don't escape our nonzero weight spaces: notably, $g_\beta$ already has weight $m$ with respect to our action of $\mf{sl}_2(F)_\alpha$, so the space with weight $-m$ is nonzero!
\end{proof}

\subsection{Regular Elements}
We now embark on a technical condition to understand ``generic'' semisimple elements, like diagonal matrices in $\mf{sl}_n(F)$ with distinct eigenvalues. The point is that we may hope to produce Cartan subalgebras as the centralizer of such a generic element. We begin with a few definitions to define the notion of regularity.
\begin{definition}[nullity]
	Fix a Lie algebra $\mf g$. Given $X\in\mf g$, we define the \textit{nullity} $n(X)$ as the multiplicity of $0$ as a generalized eigenvalue of the operator $\op{ad}_X\colon\mf g\to\mf g$.
\end{definition}
\begin{remark}
	Note that $n(X)\ge1$ always because $\op{ad}_X(X)=0$.
\end{remark}
\begin{remark}
	Equivalently, we can say that $n(X)=\dim\ker X_s$ (over a perfect field), where $X_s$ is the semisimple part of the Jordan decomposition.
\end{remark}
\begin{definition}[rank]
	Fix a Lie algebra $\mf g$. Its \textit{rank} is
	\[\op{rank}\mf g\coloneqq\min_{X\in\mf g}n(X).\]
\end{definition}
\begin{remark}
	Because $n(X)$ is bounded by $1$, we see $\op{rank}\mf g\ge1$ always.
\end{remark}
\begin{example}
	For some $X\in\mf {gl}_n(F)$ with eigenvalues $\{\lambda_1,\ldots,\lambda_n\}$, one sees that $\op{ad}_X$ diagonalizes with respect to the basis of elementary matrices, so $\op{ad}_X$ has eigenvalues
	\[\{\lambda_i-\lambda_j:1\le i,j\le n\}.\]
	Thus, we see that the smallest number of $0$s we can get is $n$, which happens when all eigenvalues are distinct. A similar argument shows $\op{rank}\mf{sl}_n(F)=n-1$.
\end{example}
This allows us to define regularity.
\begin{definition}[regular semisimple]
	Fix a Lie algebra $\mf g$. An element $X\in\mf g$ is \textit{regular semisimple} if and only if $n(X)=\op{rank}\mf g$. We denote this set by $\mf g^{\mathrm{reg}}$.
\end{definition}
\begin{remark}
	Usually, $X\in\mf g$ is called regular when $\dim\ker\op{ad}_X$ equals $\op{rank}\mf g$. This may allow some nonzero nilpotent elements: for example, the element of $\mf{sl}_n(F)$ with one Jordan block with generalized eigenvalue $0$ will have the correct value of $\dim\ker\op{ad}_X$.
\end{remark}
\begin{example}
	From our previous example, we see that $X\in\mf{gl}_n(F)$ is regular semisimple if and only if all its eigenvalues are distinct (thus forcing $X$ to be semisimple).
\end{example}
To justify that $\mf g^{\mathrm{reg}}$ consists of generic elements, we show that it is Zariski open.
\begin{lemma}
	Fix a finite-dimensional Lie algebra $\mf g$ over an algebraically closed field $F$. Then $\mf g^{\mathrm{reg}}$ is open and dense in $\mf g$. If $F=\CC$, then it is also connected.
\end{lemma}
\begin{proof}
	We will find a polynomial equation cutting out $\mf g\setminus\mf g^{\mathrm{reg}}$. Well, for any $X\in\mf g$, we let $P_X(T)$ be the characteristic polynomial of $\op{ad}_X$, which we note has coefficients
	\[a_n(X)T^n+\cdots+a_0(X)\]
	which are polynomials in $X$. Indeed, the adjoint representation $\op{ad}_\bullet\colon\mf g\to\mf{gl}(\mf g)$ is some linear map and hence polynomial, so the coefficients of $\op{ad}_X$ are linear in $X\in\mf g$, so the characteristic polynomial has coefficients which are polynomial in $X\in\mf g$. Now, $n(X)$ is equal to the multiplicity of $0$ as a root of $P_X(T)$. In other words, setting $r\coloneqq\op{rank}\mf g$, we see that all coefficients after $a_r(X)$ always vanish, and $a_r(X)\ne0$ if and only if $X$ is regular semisimple.

	It remains to show that the nonzero locus of the polynomial $a_r$ is connected, open, and dense. Well, it is open by continuity of polynomials, and it is dense because a polynomial vanishing on a full open ball vanishes on infinitely many points (even after fixing all but one coordinate) and hence must vanish. For connectivity, choose any $X$ and $Y$ and consider the complex line $\ell$ connecting $X$ to $Y$. Then the locus where $a_r$ is nonzero on $\ell$ is simply $\ell$ minus some number of points, but one can choose a path through the two-dimensional space $\CC$ (living inside the plane $\ell$) avoiding these finite number of points.
\end{proof}
We now begin talking about Cartan subalgebras.
\begin{lemma} \label{lem:regularity-check-cartan}
	Fix a semisimple Lie algebra $\mf g$ over an algebraically closed field $F$, and let $\mf h$ be a Cartan subalgebra.
	\begin{listalph}
		\item $\dim\mf h=\op{rank}\mf g$.
		\item $\mf h\cap\mf g^{\mathrm{reg}}$ consists of the set of elements $X\in\mf h$ such that $\alpha(X)\ne0$ for all $\alpha\in\Phi$.
	\end{listalph}
\end{lemma}
\begin{proof}
	Here we go.
	\begin{listalph}
		\item Choose a connected Lie group $G$ with $\op{Lie}G=\mf g$; for example, we showed earlier that $G=\op{Aut}(\mf g)^\circ$ will do. Consider the action map of $\psi\colon G\times\mf h\to\mf g$ given by $g\cdot X\coloneqq\op{Ad}_g(X)$. Taking the differential of $\psi$ at some $(1,X)\in G\times\mf h$, we see that
		\[d\psi_{(1,X)}(Y,Z)=\frac d{dt}\op{Ad}_e(tY(X+tZ))=[Y,X]+Z\]
		for $Y\in\mf g$ and $Z\in\mf h$, upon computing the derivatives separately in each component. Notably, the kernel of this differential consists of the elements of the form $(Y,-[Y,X])$, where $Y\in\mf g$ satisfies $[Y,X]\in\mf h$. For example, consider the restriction of the action to
		\[V\coloneqq\{X\in\mf h:\langle X,\alpha\rangle\ne0\text{ for all }\alpha\in\Phi\}.\]
		Then any $X\in V$ has commutator equal to exactly $\mf h$ by the root decomposition.
		
		We now claim that $\ker d_{(1,X)}\psi|_{G\times V}$ is exactly $C(x)$, which will complete the proof by a dimension computation. Well, using the Killing form as a non-degenerate bilinear form, we see that
		\[K([Y,X],Z)=K(Y,[X,Z])=0,\]
		so $[Y,X]\in\mf h$ if and only if $[Y,X]=0$. In particular, we see that the differential $d_{(1,X)}\psi|_{G\times V}$ is surjective (one can use the above computation of the differential plus the root decomposition combined with the fact that $[\mf g_\alpha,X]=\mf g_\alpha$), so the image of $\psi|_{G\times V}$ contains an open neighborhood $U_X$ of $X$ in $\mf g$.

		Now, because $\mf g^{\mathrm{reg}}$ is open and dense, so find some $Y$ in the intersection of the form $Y=\op{Ad}_g(X)$. Now, $n(Y)=\op{rank}\mf g$, but an automorphism $\op{Ad}_g$ tells us $n(Y)=n(X)=\dim C(X)=\dim\mf h$.

		\item Choose some $X\in\mf h$. Then $n(X)$ equals the dimension of $C(X)$ (we are asking for the kernel of the semisimple operator $\op{ad}_X$), which is $\dim\mf h$ plus the number of $\alpha\in\Phi$ such that $\alpha(X)=0$ because this is equivalent to $[X,\mf g_\alpha]=0$. (Notably, the action of $\op{ad}_X$ already diagonalizes along the root decomposition by its construction, so we can check for kernel of $\op{ad}_X$ on each weight space individually!)
		\qedhere
	\end{listalph}
\end{proof}

\end{document}