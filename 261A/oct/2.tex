% !TEX root = ../notes.tex

\documentclass[../notes.tex]{subfiles}

\begin{document}

\section{October 2}
Today we will continue talking about representations.

\subsection{Schur's Lemma}
The following result is our first interesting result about representations.
\begin{proposition}[Schur's lemma] \label{prop:schur-lemma}
	Fix representations $V$ and $W$ (over a field $F$) over a Lie group $G$ or Lie algebra $\mf g$, denoted $R$.
	\begin{listalph}
		\item If $V$ is irreducible, then any nonzero morphism $\varphi\colon V\to W$ is injective.
		\item If $W$ is irreducible, then any nonzero morphism $\varphi\colon V\to W$ is surjective.
		\item If $V$ and $W$ are both irreducible, then any nonzero morphism $\varphi\colon V\to W$ is an isomorphism.
		\item If $V$ and $W$ are both irreducible, then the endomorphism algebra $\op{End}_R(V)$ is a finite-dimensional division algebra over $F$. In particular, if $F$ is algebraically closed, then the map $F\cong\op{End}_R(V)$ defined by $\lambda\mapsto\lambda\id_V$ is a ring isomorphism.
	\end{listalph}
\end{proposition}
\begin{proof}
	Here we go.
	\begin{listalph}
		\item Note that $\ker\varphi\subseteq V$ is a subrepresentation by \Cref{ex:ker-subrep-grp,ex:ker-subrep-alg}. Thus, either $\ker\varphi=0$ in which case $\varphi$ is injective, or $\ker\varphi=V$ in which case $\varphi=0$.
		\item Note that $\im\varphi\subseteq V$ is a subrepresentation by \Cref{ex:image-subrep,ex:image-subrep-alg}. Thus, either $\im\varphi=0$ in which case $\varphi=0$, or $\im\varphi=W$ in which case $\varphi$ is surjective.
		\item This follows by combining the previous two parts with \Cref{rem:bijective-morphism-reps}.
		\item Note that $\op{End}_R(V)$ is certainly an algebra (possibly non-commutative). Part (c) explains that all non\-zero elements have inverses, so this algebra becomes a division algebra. It remains to check the claim when $F$ is algebraically closed. In fact, we show that any morphism $\varphi\colon V\to V$ must be a scalar, which will complete the proof because it shows that the natural map
		\[k\to\op{End}_R(V)\]
		given by $c\mapsto c\id_V$ is an isomorphism.\footnote{Certainly this is a ring map, and it is injective because $V$ is nonzero, so we are really interested in showing that this map is surjective.} Note that $\varphi$ will have an eigenvector $v$ with eigenvalue $\lambda$. Then $\varphi-\lambda\id_V$ is a morphism with a nontrivial kernel, so it must be the zero map because it is not an isomorphism! Thus, we conclude that $\varphi=\lambda\id_V$ is a scalar.
		\qedhere
	\end{listalph}
\end{proof}
This result (and in particular (d)) is important enough to warrant its own subsection. To explain why, here are some interesting corollaries.
\begin{corollary} \label{cor:center-of-lie}
	Fix an algebraically closed field $F$.
	\begin{listalph}
		\item For any injective irreducible representation $\rho\colon G\to\op{GL}(V)$ of a regular Lie group $G$, the center of $G$ is
		\[Z(G)=\{g\in G:\rho(g)=\lambda{\id_V}\text{ for some }\lambda\in\CC\}.\]
		\item For any injective representation $\rho\colon\mf g\to\mf{gl}(V)$ of a Lie algebra $\mf g$ over a field $F$. Then the center of $\mf g$ is
		\[\mf z(\mf g)=\{X\in\mf g:\rho(X)=\lambda{\id_V}\text{ for some }\lambda\in k\}.\]
	\end{listalph}
\end{corollary}
\begin{proof}
	The point is that living in the center implies commuting with the ambient action, which \Cref{prop:schur-lemma} explains implies the element must be a scalar. The injectivity of the representations implies that this characterizes the center.
	\begin{listalph}
		\item In one direction, if $g\in G$ has $\rho(g)=\lambda\id_V$ for some $\lambda\in\CC$, then any $h\in G$ has
		\begin{align*}
			\rho\left(hgh^{-1}\right) &= \rho(h)\rho(g)\rho(h)^{-1} \\
			&= \rho(h)\circ{\lambda\id_V}\circ\rho(h)^{-1} \\
			&= \lambda\rho(h)\rho(h)^{-1} \\
			&= \lambda\id_V \\
			&= \rho(g),
		\end{align*}
		so injectivity of $\rho$ implies that $hgh^{-1}=g$; thus, $g\in Z(G)$.

		Conversely, suppose $g\in Z(G)$. Then $\rho(g)\colon V\to V$ is an operator on an irreducible representation of $G$. In fact, $\rho(g)$ commutes with the action of $G$: for any $h\in G$, we see that
		\[\rho(g)\circ\rho(h)=\rho(gh)=\rho(hg)=\rho(h)\circ\rho(g)\]
		because $g\in Z(G)$. Thus, \Cref{prop:schur-lemma} implies that $\rho(g)=\lambda\id_V$ for some $\lambda\in\CC$.

		\item In one direction, if $X\in\mf g$ has $\rho(X)=\lambda\id_V$ for some $\lambda\in\CC$, then any $Y\in\mf g$ has
		\begin{align*}
			\rho([X,Y]) &= \rho(X)\circ\rho(Y)-\rho(Y)\circ\rho(X) \\
			&= \lambda\rho(Y)-\lambda\rho(Y) \\
			&= 0,
		\end{align*}
		so the injectivity of $\rho$ implies that $[X,Y]=0$; thus, $X\in\mf z(\mf g)$.
		
		Conversely, suppose $X\in\mf z(\mf g)$. Then $\rho(X)\colon V\to V$ is an operator on an irreducible representation of $\mf g$ which commutes with the $\mf g$ action: any $Y\in\mf g$ has
		\[\rho(X)\circ\rho(Y)-\rho(Y)\circ\rho(X)=\rho([X,Y])=\rho(0)=0.\]
		Thus, \Cref{prop:schur-lemma} implies that $\rho(X)=\lambda\id_V$ for some $\lambda\in\CC$.
		\qedhere
	\end{listalph}
\end{proof}
\begin{example}
	By \Cref{cor:center-of-lie}, we see that $Z({\op{GL}_n(\FF)})$ consists of scalar matrices. One can do similar computations for all the classical groups.
\end{example}
\begin{example}
	Note that $Z(\mf{sl}_n(\FF))=0$ for $n\ge2$. (If $n=1$, then $\mf{sl}_1(\FF)=0$ already.) Indeed, the main point is that the standard representation $\mf{sl}_n(\FF)\subseteq\op{gl}_n(\FF)$ is irreducible. Well, for any nonzero subrepresentation $V\subseteq\FF^n$, say $v\in V\setminus\{0\}$, and we may assume that $v=e_1$ upon changing basis. Now, for any $w\in\FF^n$, we see that there is a traceless matrix $X\in\mf{sl}_n(\FF)$ such that $Xv=w$, thus proving that $w\in V$, so $W=\RR^n$. Applying this irreducible representation to \Cref{cor:center-of-lie}, we conclude that
	\[\mf{sl}_n(\FF)=\{\lambda1_n\in\mf{sl}_n(\FF):\lambda\in\FF\}=0\]
	because $\tr\lambda1_n=0$ requires $\lambda=0$.
\end{example}
\begin{corollary} \label{cor:abelian-irreps}
	Fix an abelian Lie group $G$ or Lie algebra $\mf g$, denoted $R$. Then all irreducible complex representations are one-dimensional.
\end{corollary}
\begin{proof}
	Let $V$ be an irreducible complex representation of $R$ with structure morphism $\rho$. Then for any $g\in R$, we see that $\rho(g)\colon V\to V$ is an operator commuting with the action of $G$: for any $h\in R$, we see that
	\[\rho(g)\circ\rho(h)=\rho(h)\circ\rho(g)\]
	because $R$ is abelian. (Each case with $R$ requires a slightly different argument, but the conclusion is the same: both equal $\rho(gh)=\rho(hg)$ when $R=G$, and the difference equals $\rho([g,h])=0$ when $R=\mf g$.)
	
	Thus, \Cref{prop:schur-lemma} implies that $\rho(g)$ is a scalar operator $\lambda_g\id_V$ for each $g\in G$. In particular, for any nonzero vector $v\in V$, we see that $\rho(g)$ acts a scalar on $\op{span}\{v\}$ and hence preserves this subspace. Thus, $\op{span}\{v\}$ is a nonzero subrepresentation of $V$, forcing $V=\op{span}\{v\}$ by irreducibility.
\end{proof}
Let's compute the representations of some abelian groups/algebras.
\begin{example} \label{ex:irreps-ga}
	The complex representations of the abelian Lie algebra $\FF=\mf{gl}(\FF)$ are just arbitrary $\CC$-vector spaces $V$ with a chosen endomorphism by $\rho\colon\mf{gl}(\FF)\to\mf{gl}(V)$. In particular, the irreducible Lie algebra representations $\rho\colon\mf{gl}(\FF)\to\mf{gl}(V)$ have $V=\CC$ by \Cref{cor:abelian-irreps}, and then we see we are just asking for a linear map $\FF\to\CC$.
	
	Now, applying the equivalence of \Cref{prop:rep-theory-to-lie-alg}, we construct the representation $\rho_\lambda\colon\FF\to\op{GL}(\CC)$ by $\rho_\lambda(t)(v)\coloneqq\exp(\lambda tv)$ (for any $\lambda\in\CC$), and we note that $d(\rho_\lambda)_e\colon\FF\to\mf{gl}(\CC)$ is the multiplication-by-$\lambda$ irreducible representation of the previous paragraph. In particular, the equivalence of categories establishes these as our irreducible representations of $\FF$.
\end{example}
\begin{example} \label{ex:irreps-gm-r}
	Consider the real Lie group $G\coloneqq\RR^\times$. Note that $\RR^\times\cong\{\pm1\}\times\RR^+$ by the multiplication map, and $\RR^+\cong\RR$ by taking the exponential. Now, this Lie group is abelian, so all irreducible representations are one-dimensional, so we can classify irreducible representations $\rho\colon G\to\op{GL}(\CC)$ as
	\[\op{Hom}_{\op{LieGrp}}(\{\pm1\}\times\RR,\CC^\times)=\op{Hom}_{\op{LieGrp}}(\{\pm1\},\CC^\times)\times\op{Hom}_{\op{LieGrp}}(\RR,\CC^\times)\]
	by tracking the universal property of the product (for both manifolds and groups). Now, $\op{Hom}(\{\pm1\},\CC)$ is just looking for elements of $\CC^\times$ of order dividing $2$, which we know are only $\{\pm1\}$. Continuing, we note $\op{Hom}_{\op{LieGrp}}(\RR,\CC^\times)$ was classified as the maps $t\mapsto\exp(\lambda t)$ for some $\lambda\in\CC$ in \Cref{ex:irreps-ga} (because such representations must be irreducible by virtue of being one-dimensional). As such, we see that $\op{Hom}_{\op{LieGrp}}(\RR^\times,\CC^\times)$ consists of the maps $t\mapsto\op{sgn}(t)^\varepsilon\left|t\right|^\lambda$ for some $\varepsilon\in\{0,1\}$ and $\lambda\in\CC$.
\end{example}
\begin{example}
	Consider the real Lie group $S^1$ equipped with the projection $\pi\colon\RR\to S^1$ given by $\pi(t)\coloneqq e^{2\pi it}$. Then $\pi$ is a smooth surjection with kernel $\ker\pi=\ZZ$. Thus, a representation $\rho\colon S^1\to\op{GL}(\CC)$ (as usual, all irreducible representations are $1$-dimensional by \Cref{cor:abelian-irreps}) induces a representation $\widetilde\rho\colon\RR\to\op{GL}(\CC)$ as $\widetilde\rho\coloneqq\rho\circ\pi$. Now, \Cref{ex:irreps-ga} tells us that $\widetilde\rho(t)=\exp(t\lambda)\in\CC^\times$ for some $\lambda\in\CC$. However, $\widetilde\rho$ must have $\ker\pi=\ZZ$ in its kernel, so $\exp(\lambda)=1$, so $\lambda=2\pi in$ for some $n\in\ZZ$. Going back through $\pi$, we thus see that $\rho(z)=z^n$ for some $n\in\ZZ$, and we can check that these are all in fact polynomial (and hence smooth) representations $S^1\to\CC^\times=\op{GL}(\CC)$.
\end{example}
% \begin{example}
% 	We see that $\CC^\times\cong S^1\times\RR$ as a real Lie group, so the complex irreducible representations are simply given by $\rho\left(re^{i\theta}\right)\coloneqq r^\lambda e^{ni\theta}$ for some $\lambda\in\CC$ and $n\in\ZZ$.
% \end{example}
One can upgrade \Cref{prop:schur-lemma} for arbitrary representations.
\begin{corollary} \label{cor:compute-hom-by-irreps}
	Fix a Lie group $G$ or Lie algebra $\mf g$, denoted $R$. Fix complex completely reducible representations
	\[V=\bigoplus_{i=1}^nV_i^{\oplus m_i}\qquad\text{ and }W\cong\bigoplus_{i=1}^nV_i^{\oplus n_i},\]
	where the set $\{V_i\}_{i=1}^n$ consists of pairwise non-isomorphic complex irreducible representations. Then
	\[\op{Hom}_R(V,W)=\bigoplus_{i=1}^n\CC^{n_i\times m_i}.\]
\end{corollary}
\begin{proof}
	Finite sums move outside $\op{Hom}$ by the universal properties involved, so
	\begin{align*}
		\op{Hom}_R(V,W) &= \op{Hom}_R\Bigg(\bigoplus_{i=1}^nV_i^{\oplus m_i},\bigoplus_{i=1}^nV_i^{\oplus n_i}\Bigg) \\
		&= \bigoplus_{i,j=1}^n\op{Hom}_R\left(V_i^{\oplus n_i},V_j^{\oplus m_i}\right) \\
		&= \bigoplus_{i,j=1}^n\op{Hom}_R(V_i,V_j)^{n_i\times m_i}.
	\end{align*}
	Now, \Cref{prop:schur-lemma} tells us that $\op{Hom}_R(V_i,V_j)=\CC1_{i=j}$, so the result follows.
\end{proof}
\begin{corollary} \label{cor:unique-irrep-decomp}
	Fix a Lie group $G$ or Lie algebra $\mf g$. If $V$ is a completely reducible complex representation, then $V$ has a unique decomposition into irreducibles up to isomorphism and permutation of the factors.
\end{corollary}
\begin{proof}
	Because $V$ is finite-dimensional, any two decompositions of $V$ into irreducibles will end up using only finitely many irreducible components, which we can list out as $\{V_1,\ldots,V_n\}$. Then we are given two decompositions
	\[\bigoplus_{i=1}^nV_i^{\oplus m_i}\cong V\cong\bigoplus_{i=1}^nV_i^{\oplus n_i}\]
	for nonnegative integers $m_\bullet$s and $n_\bullet$s. We want to check that $m_i=n_i$ for each $i$. Well, for each $i$, \Cref{cor:compute-hom-by-irreps} implies that
	\[\dim\op{Hom}_G(V_i,V)=\dim\op{Hom}_G\Bigg(V_i,\bigoplus_{i=1}^nV_i^{\oplus m_i}\Bigg)=m_i\]
	and similarly $\dim\op{Hom}_G(V_i,V)=n_i$, so $m_i=n_i$ follows.
\end{proof}

\subsection{The Unitarization Trick}
We would like tools to show that all representations are completely reducible. One place to start is with unitary representations.
\begin{definition}
	Fix a Lie group $G$ or Lie algebra $\mf g$, denoted $R$. Then a representation $V$ of $R$ is \textit{unitary} if and only if it has a positive-definite Hermitian inner product $\langle\cdot,\cdot\rangle$ commuting with the $R$-action. More precisely, we have the following.
	\begin{itemize}
		\item If $R$ is a Lie group, then we want $\langle gv,gw\rangle=\langle v,w\rangle$ for all $g\in G$ and $v,w\in V$.
		\item If $R$ is a Lie algebra, then we want $\langle Xv,w\rangle+\langle v,Xw\rangle=0$ for all $X\in\mf g$ and $v,w\in V$.
	\end{itemize}
	For a general bilinear form $\langle-,-\rangle$ on a representation $V$, we say that $\langle-,-\rangle$ is \textit{invariant} if and only if the appropriate condition above is satisfied.
\end{definition}
Here's a quick coherence check for the definition.
\begin{lemma} \label{lem:descend-invariant-inner-prod}
	Fix a Lie group $G$ with Lie algebra $\mf g$, and let $\rho\colon G\to\op{GL}(V)$ be a complex representation inducing a representation $d\rho_e\colon\mf g\to\mf{gl}(V)$. Suppose that $V$ has an $\mathbb R$-bilinear product $\langle-,-\rangle\colon V\times V\to\CC$ (possibly Hermitian).
	\begin{listalph}
		\item For any $X\in\mf g$, we have
		\[\frac d{dt}\langle\exp(tX)v,\exp(tX)w\rangle=\langle Xv,w\rangle+\langle v,Xw\rangle.\]
		\item If $\langle-,-\rangle$ is invariant for $G$, then it is invariant for $\mf g$.
	\end{listalph}
\end{lemma}
\begin{proof}
	Here we go.
	\begin{listalph}
		\item This is essentially the product rule. In $V$, we compute
		\begin{align*}
			\frac d{dt}\langle\exp(tX)v,\exp(tX)w\rangle &= \lim_{t\to0}\frac{\langle\exp(tX)v,\exp(tX)w\rangle-\langle v,w\rangle}{t} \\
			&= \lim_{t\to0}\frac{\langle\exp(tX)v,\exp(tX)w\rangle-\langle\exp(tX)v,w\rangle}{t}+\lim_{t\to0}\frac{\langle\ v,w\rangle-\langle\exp(tX)v,w\rangle}{t} \\
			&= \lim_{t\to0}\left\langle\exp(tX)v,\frac{\exp(tX)w-w}t\right\rangle+\lim_{t\to0}\left\langle\frac{\exp(tX)v-v}t,w\right\rangle.
		\end{align*}
		Now, linearity of the bilinear product implies its continuity, so we can bring the limit inside the bilinear products. Doing so and using \Cref{rem:induced-lie-alg-rep} shows that these limits evaluate to $\langle Xv,w\rangle+\langle v,Xw\rangle$.
		\item If $\rho$ is unitary, then $\langle\exp(tX)v,\exp(tX)w\rangle=\langle v,w\rangle$ always, so the derivative in (a) always vanishes, so $\langle Xv,w\rangle+\langle v,Xw\rangle=0$ always.
		\qedhere
	\end{listalph}
\end{proof}
% Inner products have the usual consequences for duality.
% \begin{lemma}
% 	Let $V$ be a unitary representation of a Lie group $G$ or Lie algebra $\mf g$, denoted $R$, and let $\rho$ be the structure morphism. Then the map $\overline\rho\to\rho^\lor$ given by $v\mapsto\langle-,v\rangle$ is an isomorphism of representations.
% \end{lemma}
% \begin{proof}
% 	Here are our checks. Let this map be denoted $i_\bullet\colon V\to V^\lor$.
% 	\begin{itemize}
% 		\item We note that $i_v\in V^\lor$ always: for any $a,a'\in\CC$ and $w,w'\in V$, we see that $i_v(aw+a'w')=ai_v(w)+a'i_v(w')$ because $\langle-,-\rangle$ is Hermitian.
% 		\item Linear: for any $a,a'\in\CC$ and $v,v',w\in V$ we compute
% 		\[\]
% 	\end{itemize}
% \end{proof}
We take a moment to acknowledge that, as usual, inner products have applications to duality.
\begin{lemma} \label{lem:dualize-unitary-rep}
	Fix a complex vector space $V$, and recall that we can define a complex vector space $\ov V$ as having the conjugate action. If $V$ has Hermitian inner product $\langle-,-\rangle$, then the map $\ov V\to\overline{V^\lor}$ given by $v\mapsto\langle-,v\rangle$ is an isomorphism of vector spaces. If $V$ is a unitary representation over a real Lie group or algebra, then this map is also an isomorphism  of representations.
\end{lemma}
\begin{proof}
	Well, for any $v\in V$, we see that $\langle-,v\rangle$ is a linear operator $V\to\CC$ because $\langle-,-\rangle$ is Hermitian. In fact, this gives an $\mathbb R$-linear map $i_\bullet\colon\ov V\to V^\lor$ defined by $i_v\coloneqq\langle-,v\rangle_\bullet$, and it has trivial kernel because $v$ nonzero implies that $\langle v,v\rangle>0$. Thus, our linear map $V\to V^\lor$ is a vector space isomorphism in light of the fact that $\dim V=\dim V^\lor$. Lastly, we note that this upgrades to an isomorphism of $\CC$-vector spaces because
	\[\langle-,a\ov v\rangle=\ov a\langle-,v\rangle.\]
	Now, if $V$ is a unitary representation, we need to check that this isomorphism is invariant.
	\begin{itemize}
		\item If $V$ is a representation of a group $G$, then we note that
		\[i_{gv}(w)=\langle gv,w\rangle=\left\langle v,g^{-1}w\right\rangle=(gi_v)(w).\]
		\item If $V$ is a representation of a Lie algebra $\mf g$, then we note that
		\[i_{Xv}(w)=\langle Xv,w\rangle=-\langle v,Xw\rangle=(Xi_v)(w).\]
	\end{itemize}
	The above checks complete the proof.
\end{proof}
\begin{lemma} \label{lem:dualize-inner-prod-rep}
	Fix a complex vector space $V$. If $V$ has a non-degenerate inner product $\langle-,-\rangle$, then the map $V\to V^\lor$ given by $v\mapsto\langle-,v\rangle$ is an isomorphism of vector spaces. If $\langle-,-\rangle$ is invariant over a Lie group or algebra, then this map is also an isomorphism of representations.
\end{lemma}
\begin{proof}
	The first paragraph of \Cref{lem:dualize-unitary-rep} (with no over-lines) proves the first sentence. The rest of that proof verbatim shows the second sentence.
\end{proof}
As another coherence check, we note that the choice of invariant product is more or less unique.
\begin{proposition}
	Fix a complex irreducible representation $V$ of a Lie group $G$. Then there is at most one invariant Hermitian form on $V$, up to a positive scalar.
\end{proposition}
\begin{proof}
	Fixing some invariant Hermitian forms $\langle\cdot,\cdot\rangle_1$ and $\langle\cdot,\cdot\rangle_2$ of a representation $V$, then we claim that we can find a function $\varphi\colon V\to V$ such that
	\[\langle v,w\rangle_1=\langle\varphi(v),w\rangle_2\]
	for all $v,w\in V$. Indeed, $\varphi$ is simply the composite of the representation isomorphisms $\ov V\cong V^\lor\cong\ov V$ provided by \Cref{lem:dualize-unitary-rep}.

	In particular, taking conjugates, we see that $\varphi$ is an automorphism of an irreducible representation, so \Cref{prop:schur-lemma} implies that $\varphi=\lambda\id_V$ for some $\lambda\in\CC$. In particular, we conclude that
	\[\langle-,-\rangle_1=\lambda\langle-,-\rangle_2.\]
	It remains to check that $\lambda$ is a positive real number. Well, choose a nonzero vector $v\in V$, and then we see that $\lambda=\langle v,v\rangle_1/\langle v_v\rangle_2>0$.
\end{proof}
Anyway, here is the reason for defining the notion of unitary.
\begin{proposition} \label{prop:unitary-complement}
	Let $V$ be a unitary representation of a Lie group $G$ or Lie algebra $\mf g$, denoted $R$. If $W\subseteq V$ is a subrepresentation, then so
	\[W^\perp\coloneqq\{v\in V:\langle v,w\rangle=0\text{ for all }w\in W\}.\]
	In fact, $V=W\oplus W^\perp$ as representations.
\end{proposition}
\begin{proof}
	We run our checks in sequence.
	\begin{itemize}
		\item We claim that $W^\perp\subseteq V$ is a subrepresentation. Well, for each $v\in V$, we note that $w\mapsto\langle w,v\rangle$ is a linear map $V\to\CC$ because $\langle-,-\rangle$ is Hermitian, so $W^\perp=\bigcap_{w\in W}\ker\langle-,w\rangle$ is a linear subspace. To see that this is a subrepresentation, we pick up some $v\in W^\perp$, and we want to show that $gv\in W^\perp$ for any $g\in R$. To continue, we do casework on $R$.
		\begin{itemize}
			\item If $R=G$, then note that
			\[\langle gv,w\rangle=\left\langle v,g^{-1}w\right\rangle=0\]
			for all $w\in W$ because $g^{-1}w\in W$ as well.
			\item If $R=\mf g$, then set $X=g$ and note that
			\[\langle Xv,w\rangle=-\langle v,Xw\rangle=0\]
			for all $w\in W$ because $Xw\in W$ as well.
		\end{itemize}

		\item We claim that the summation map $W\oplus W^\perp\to V$ is an isomorphism. Because $W$ and $W^\perp$ are both subspaces of $V$, we certainly have a linear summation map $W\oplus W^\perp\to V$, so it is merely a matter of checking that we have an isomorphism.
		\begin{itemize}
			\item Trivial kernel: suppose that $(w,v)\in W\oplus W^\perp$ has $w+v=0$. Then $w=-v$ lives in $W\cap W^\perp$. In particular, $\langle w,w\rangle=0$, which implies $w=0$ (and hence $v=0$) because $\langle-,-\rangle$ is Hermitian and non-degenerate.
			\item Surjective: by a dimension count, it is now enough to check that $\dim V\le\dim W+\dim W^\perp$.\footnote{Explicitly, the image of $W+W^\perp\subseteq V$ has dimension $\dim W+\dim W'$ because the summation map already has trivial kernel by the previous point.} Well, the presence of an inner product allows us to begin with an orthonormal basis $\{e_1,\ldots,e_k\}$ of $W$ and then extend it to an orthonormal basis $\{e_{k+1},\ldots,e_n\}$ of $V$. However, the condition of being orthonormal implies that $\{e_{k+1},\ldots,e_n\}\subseteq W^\perp$, so this orthonormal subset provides a lower bound
			\[\dim W^\perp\ge n-k=\dim V-\dim W,\]
			as required.
			\qedhere
		\end{itemize}
	\end{itemize}
\end{proof}
\begin{corollary}[unitarization trick] \label{cor:unitary-reduces}
	Let $V$ be a unitary representation of a Lie group $G$ or Lie algebra $\mf g$. Then $V$ is completely reducible.
\end{corollary}
\begin{proof}
	We induct on $\dim V$ by using \Cref{prop:unitary-complement}. If $\dim V=0$, then $V=0$, so $V$ is the empty sum of irreducible representations. Otherwise, for our inductive step, take $\dim V>0$. If $V$ is irreducible already, then there is nothing to do. Otherwise, let $W\subseteq V$ be a proper nontrivial subrepresentation of $V$, and then \Cref{prop:unitary-complement} implies that $V\cong W\oplus W^\perp$. Now, $0<\dim W,\dim W^\perp<\dim V$, so $W$ and $W^\perp$ are unitary representations with strictly smaller dimension than $V$, so $W$ and $W^\perp$ are completely irreducible, so $V$ is also completely irreducible (by taking the sum of the decompositions for $W$ and $W^\perp$).
\end{proof}
\begin{example}
	Let $\mf g$ be an abelian complex Lie algebra. Then the adjoint representation $\op{ad}_\bullet\colon\mf g\to\mf{gl}(\mf g)$ is unitary no matter what Hermitian inner product $\langle-,-\rangle$ we give $\mf g$: indeed, we see that
	\[\langle\op{ad}_XY,Z\rangle+\langle Y,\op{ad}_XZ\rangle=\langle[X,Y],Z\rangle+\langle Y,[X,Z]\rangle=0+0=0\]
	by \Cref{prop:commutator-by-adjoint}. Thus, the adjoint representation is completely reducible by \Cref{cor:unitary-reduces}.
\end{example}
% \begin{example}
% 	If $G$ is a compact Lie group, then any representation $V$ of $G$ can be made unitary and hence is completely irreducible. Indeed, we need to define an invariant Hermitian form on $V$. Well, start with any Hermitian form $\langle\cdot,\cdot\rangle$, and then we fix it by defining
% 	\[\langle v,w\rangle_G\coloneqq\int_G\langle gv,gw\rangle\,dg,\]
% 	where $dg$ is an invariant top differential form on $G$. One can check that this is an invariant Hermitian form.
% \end{example}
% \begin{remark}
% 	Continue with $G$ a finite group. It may look like we have a lot of choice in our invariant Hermitian form, but in fact they are only unique up to positive scalar. Indeed, 
% \end{remark}
% \begin{example}
% 	Fix an abelian Lie algebra $\mf g$. Then any Hermitian form on a representation of $\mf g$ is automatically invariant, so we see that all representations are completely irreducible.
% \end{example}
% \begin{example}
% 	We see that $\op{SL}_n(\CC)$ has compact real form $\op{SU}_n$, so we see that the representation theory of $\op{SL}_2(\CC)$ must be completely irreducible.\todo{}
% \end{example}

\subsection{Compact Lie Groups}
The main application of \Cref{cor:unitary-reduces} is to compact groups. To explain this, we need some notion of an integration theory. Fix a regular Lie group $G$ of dimension $n$ with Lie algebra $\mf g$. \Cref{rem:parallelize-lie-group} provides a parallelization of $TG\cong G\times\mf g$ by right-invariant vector fields. Choosing a right-invariant global frame $\{\xi_1,\ldots,\xi_n\}$ of $TG$, we define
\[\omega\coloneqq\xi_1\land\cdots\land\xi_n\]
to be a right-invariant top-degree differential form in $\Omega^nG=\land^nTG$. Then differential topology explains how to integrate regular compactly supported functions $f\colon G\to\FF$ against $\omega$. In particular, tracking through all the definitions, one finds that
\[\int_G(R_gf)\,\omega=\int_Gf\,\omega\]
for any $g\in G$ and $f\colon G\to\FF$. Indeed, integration is linear, so we may assume that $f$ is supported in a single chart $(U,\varphi)$ of $G$. Letting the coordinates be $\varphi=(x_1,\ldots,x_n)$, we see that $\omega=r(x)\,dx_1\land\cdots\land dx_n$ for some regular function $r\colon U\to\FF$. Then
\[\int_Gf\,\omega=\int_Uf(x)r(x)\,dx_1\land\cdots\land dx_n.\]
However, the $G$-invariance of $\omega$ implies that we can translate everything by $g$ to get the same value of the integral, which is the desired conclusion.

The point of having an integration theory is that we are able to take ``averages.'' The following is our main application.
\begin{proposition} \label{prop:compact-reduces}
	Fix a complex representation $V$ of a compact Lie group $G$. Then there is an invariant Hermitian inner product $\langle-,-\rangle$ on $V$. Thus, \Cref{cor:unitary-reduces} implies that representations of a compact Lie group are completely reducible.
\end{proposition}
\begin{proof}
	Begin with any Hermitian inner product $\langle-,-\rangle_0$ on $V$. Then we define $\langle-,-\rangle\colon V\times V\to\CC$ by
	\[\langle v,w\rangle\coloneqq\int_G\langle gv,gw\rangle_0\,\omega,\]
	where $\omega$ is a right-invariant top differential form on $G$, scaled so that $\int_G\omega=1$.\footnote{Because $G$ is compact, we can cover it in finitely many charts to conclude that $\int_G\omega$ is finite, and then we can scale $\omega$ by this integral to conclude that we can choose $\omega$ so that $\int_G\omega=1$.} Note that $G$ being compact implies that the integral certainly converges; notably, the function $g\mapsto\langle gv,gw\rangle$ is smooth because $\langle-,-\rangle$ is bilinear, and the representation is regular.

	We now claim that $\langle-,-\rangle$ is the required invariant Hermitian inner product.
	\begin{itemize}
		\item Conjugate-symmetric: for any $v,w\in V$, we note that
		\[\langle w,v\rangle=\int_G\langle gw,gv\rangle_0\,\omega=\overline{\int_G\langle gv,gw\rangle_0\,\omega}=\overline{\langle v,w\rangle}.\]
		\item Linear: for any $v,v',w\in V$ and $a,a'\in\CC$, we note that
		\begin{align*}
			\langle av+a'v',w\rangle &= \int_G\langle g(av+a'v'),gw\rangle_0\,\omega \\
			&= a\int_G\langle gv,gw\rangle_0\,\omega+a'\int_G\langle gv',gw\rangle_0\,\omega \\
			&= a\langle v,w\rangle+a'\langle v',w\rangle.
		\end{align*}
		\item Non-degenerate: for any $v\in V$, we note that the function $G\to\CC$ given by $g\mapsto\langle gv,gv\rangle_0$ is a function which is always positive because $\langle-,-\rangle_0$ is Hermitian. Because $G$ is compact, this function must have a minimum value $m_v>0$, so we conclude that
		\[\langle v,v\rangle=\int_G\langle gv,gv\rangle\,\omega\ge m_v>0.\]
		\item Invariant: for any $v\in V$ and $h\in G$, we note that
		\[\langle hv,hv\rangle=\int_G\langle ghv,ghv\rangle\,\omega\stackrel*=\int_G\langle gv,gv\rangle\,\omega=\langle v,v\rangle,\]
		where $\stackrel*=$ holds because $\omega$ is right-invariant.
	\end{itemize}
	The above checks complete the proof.
\end{proof}
Here is an interesting example.
\begin{example} \label{ex:sl-reduces}
	The group $\op{SU}_n$ is a compact real Lie group, so all its representations are completely reducible by \Cref{prop:compact-reduces}. In fact, it is simply connected, so $\op{Rep}_\CC({\op{SU}_n})=\op{Rep}_\CC(\mf{su}_n)$ by \Cref{prop:rep-theory-to-lie-alg}. However, $\mf{su}_n$ is also a real form of $\mf{sl}_n(\CC)$ by \Cref{ex:su-real-form-sl}, so we can use \Cref{rem:rep-theory-complexification} to note that
	\[\op{Rep}_\CC(\mf{su}_n)=\op{Rep}_\CC(\mf{su}_n\otimes_\RR\CC)=\op{Rep}_\CC(\mf{sl}_n(\CC))=\op{Rep}_\CC(\mf{sl}_n(\RR)\otimes_\RR\CC)=\op{Rep}_\CC({\op{SL}_n(\RR)}).\]
	Thus, the complex representations of $\op{SL}_n(\RR)$ are also completely reducible!
\end{example}

\end{document}