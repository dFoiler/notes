% !TEX root = ../notes.tex

\documentclass[../notes.tex]{subfiles}

\begin{document}

\section{October 2}
Today we will continue talking about representations.

\subsection{Schur's Lemma}
The following result is our first interesting result about representations.
\begin{proposition}[Schur's lemma] \label{prop:schur-lemma}
	Fix representations $V$ and $W$ (over a field $k$) over a Lie group $G$ or Lie algebra $\mf g$, denoted $R$.
	\begin{listalph}
		\item If $V$ is irreducible, then any nonzero morphism $\varphi\colon V\to W$ is injective.
		\item If $W$ is irreducible, then any nonzero morphism $\varphi\colon V\to W$ is surjective.
		\item If $V$ and $W$ are both irreducible, then any nonzero morphism $\varphi\colon V\to W$ is an isomorphism.
		\item If $V$ and $W$ are both irreducible, then the endomorphism algebra $\op{End}_R(V)$ is a finite-dimensional division algebra over $k$. In particular, if $k$ is algebraically closed, then the map $k\cong\op{End}_R(V)$ defined by $\lambda\mapsto\lambda\id_V$ is a ring isomorphism.
	\end{listalph}
\end{proposition}
\begin{proof}
	Here we go.
	\begin{listalph}
		\item Note that $\ker\varphi\subseteq V$ is a subrepresentation by \Cref{ex:ker-subrep-grp,ex:ker-subrep-alg}. Thus, either $\ker\varphi=0$ in which case $\varphi$ is injective, or $\ker\varphi=V$ in which case $\varphi=0$.
		\item Note that $\im\varphi\subseteq V$ is a subrepresentation by \Cref{ex:image-subrep,ex:image-subrep-alg}. Thus, either $\im\varphi=0$ in which case $\varphi=0$, or $\im\varphi=W$ in which case $\varphi$ is surjective.
		\item This follows by combining the previous two parts with \Cref{rem:bijective-morphism-reps}.
		\item Note that $\op{End}_R(V)$ is certainly an algebra (possibly non-commutative). Part (c) explains that all non\-zero elements have inverses, so this algebra becomes a division algebra. It remains to check the claim when $k$ is algebraically closed. In fact, we show that any morphism $\varphi\colon V\to V$ must be a scalar, which will complete the proof because it shows that the natural map
		\[k\to\op{End}_R(V)\]
		given by $c\mapsto c\id_V$ is an isomorphism.\footnote{Certainly this is a ring map, and it is injective because $V$ is nonzero, so we are really interested in showing that this map is surjective.} Note that $\varphi$ will have an eigenvector $v$ with eigenvalue $\lambda$. Then $\varphi-\lambda\id_V$ is a morphism with a nontrivial kernel, so it must be the zero map because it is not an isomorphism! Thus, we conclude that $\varphi=\lambda\id_V$ is a scalar.
		\qedhere
	\end{listalph}
\end{proof}
This result (and in particular (d)) is important enough to warrant its own subsection. To explain why, here are some interesting corollaries.
\begin{corollary} \label{cor:center-of-lie}
	Fix an algebraically closed field $k$.
	\begin{listalph}
		\item For any injective irreducible representation $\rho\colon G\to\op{GL}(V)$ of a regular Lie group $G$, the center of $G$ is
		\[Z(G)=\{g\in G:\rho(g)=\lambda{\id_V}\text{ for some }\lambda\in\CC\}.\]
		\item For any injective representation $\rho\colon\mf g\to\mf{gl}(V)$ of a Lie algebra $\mf g$ over a field $k$. Then the center of $\mf g$ is
		\[\mf z(\mf g)=\{X\in\mf g:\rho(X)=\lambda{\id_V}\text{ for some }\lambda\in k\}.\]
	\end{listalph}
\end{corollary}
\begin{proof}
	The point is that living in the center implies commuting with the ambient action, which \Cref{prop:schur-lemma} explains implies the element must be a scalar. The injectivity of the representations implies that this characterizes the center.
	\begin{listalph}
		\item In one direction, if $g\in G$ has $\rho(g)=\lambda\id_V$ for some $\lambda\in\CC$, then any $h\in G$ has
		\begin{align*}
			\rho\left(hgh^{-1}\right) &= \rho(h)\rho(g)\rho(h)^{-1} \\
			&= \rho(h)\circ{\lambda\id_V}\circ\rho(h)^{-1} \\
			&= \lambda\rho(h)\rho(h)^{-1} \\
			&= \lambda\id_V \\
			&= \rho(g),
		\end{align*}
		so injectivity of $\rho$ implies that $hgh^{-1}=g$; thus, $g\in Z(G)$.

		Conversely, suppose $g\in Z(G)$. Then $\rho(g)\colon V\to V$ is an operator on an irreducible representation of $G$. In fact, $\rho(g)$ commutes with the action of $G$: for any $h\in G$, we see that
		\[\rho(g)\circ\rho(h)=\rho(gh)=\rho(hg)=\rho(h)\circ\rho(g)\]
		because $g\in Z(G)$. Thus, \Cref{prop:schur-lemma} implies that $\rho(g)=\lambda\id_V$ for some $\lambda\in\CC$.

		\item In one direction, if $X\in\mf g$ has $\rho(X)=\lambda\id_V$ for some $\lambda\in\CC$, then any $Y\in\mf g$ has
		\begin{align*}
			\rho([X,Y]) &= \rho(X)\circ\rho(Y)-\rho(Y)\circ\rho(X) \\
			&= \lambda\rho(Y)-\lambda\rho(Y) \\
			&= 0,
		\end{align*}
		so the injectivity of $\rho$ implies that $[X,Y]=0$; thus, $X\in\mf z(\mf g)$.
		
		Conversely, suppose $X\in\mf z(\mf g)$. Then $\rho(X)\colon V\to V$ is an operator on an irreducible representation of $\mf g$ which commutes with the $\mf g$ action: any $Y\in\mf g$ has
		\[\rho(X)\circ\rho(Y)-\rho(Y)\circ\rho(X)=\rho([X,Y])=\rho(0)=0.\]
		Thus, \Cref{prop:schur-lemma} implies that $\rho(X)=\lambda\id_V$ for some $\lambda\in\CC$.
		\qedhere
	\end{listalph}
\end{proof}
\begin{example}
	By \Cref{cor:center-of-lie}, we see that $Z({\op{GL}_n(\FF)})$ consists of scalar matrices. One can do similar computations for all the classical groups.
\end{example}
\begin{example}
	Note that $Z(\mf{sl}_n(\FF))=0$ for $n\ge2$. (If $n=1$, then $\mf{sl}_1(\FF)=0$ already.) Indeed, the main point is that the standard representation $\mf{sl}_n(\FF)\subseteq\op{gl}_n(\FF)$ is irreducible. Well, for any nonzero subrepresentation $V\subseteq\FF^n$, say $v\in V\setminus\{0\}$, and we may assume that $v=e_1$ upon changing basis. Now, for any $w\in\FF^n$, we see that there is a traceless matrix $X\in\mf{sl}_n(\FF)$ such that $Xv=w$, thus proving that $w\in V$, so $W=\RR^n$. Applying this irreducible representation to \Cref{cor:center-of-lie}, we conclude that
	\[\mf{sl}_n(\FF)=\{\lambda1_n\in\mf{sl}_n(\FF):\lambda\in\FF\}=0\]
	because $\tr\lambda1_n=0$ requires $\lambda=0$.
\end{example}
\begin{corollary} \label{cor:abelian-irreps}
	Fix an abelian Lie group $G$ or Lie algebra $\mf g$, denoted $R$. Then all irreducible complex representations are one-dimensional.
\end{corollary}
\begin{proof}
	Let $V$ be an irreducible complex representation of $R$ with structure morphism $\rho$. Then for any $g\in R$, we see that $\rho(g)\colon V\to V$ is an operator commuting with the action of $G$: for any $h\in R$, we see that
	\[\rho(g)\circ\rho(h)=\rho(h)\circ\rho(g)\]
	because $R$ is abelian. (Each case with $R$ requires a slightly different argument, but the conclusion is the same: both equal $\rho(gh)=\rho(hg)$ when $R=G$, and the difference equals $\rho([g,h])=0$ when $R=\mf g$.)
	
	Thus, \Cref{prop:schur-lemma} implies that $\rho(g)$ is a scalar operator $\lambda_g\id_V$ for each $g\in G$. In particular, for any nonzero vector $v\in V$, we see that $\rho(g)$ acts a scalar on $\op{span}\{v\}$ and hence preserves this subspace. Thus, $\op{span}\{v\}$ is a nonzero subrepresentation of $V$, forcing $V=\op{span}\{v\}$ by irreducibility.
\end{proof}
Let's compute the representations of some abelian groups/algebras.
\begin{example}
	The complex representations of the abelian Lie algebra $\FF=\mf{gl}(\FF)$ are just arbitrary $\FF$-vector spaces $V$ with a chosen endomorphism by $\rho\colon\mf{gl}(\FF)\to\mf{gl}(V)$. Thus, we can compute that the complex irreducible representations of the Lie group $\FF$ are all $\CC$ (by \Cref{cor:abelian-irreps}) with action given by $\rho(t)v\coloneqq\exp(\lambda t)v$ for some $\lambda\in\CC$.
\end{example}
\begin{example}
	Note that $\RR^\times\cong\{\pm1\}\times\RR$ by taking the exponential, so the representations are given by $\rho(t)v\coloneqq\pm t^\lambda v$ for some $\lambda\in\CC$.
\end{example}
\begin{example}
	Note that the representations of $S^1$ are all going to be induced by a Lie algebra representation of $\RR$, and we can check that the only representations which are actually representations of $S^1$ are of the form $\rho(z)v\coloneqq z^nv$ for some $n\in\ZZ$.
\end{example}
\begin{example}
	We see that $\CC^\times\cong S^1\times\RR$ as a real Lie group, so the complex irreducible representations are simply given by $\rho\left(re^{i\theta}\right)\coloneqq r^\lambda e^{ni\theta}$ for some $\lambda\in\CC$ and $n\in\ZZ$.
\end{example}
One can upgrade \Cref{prop:schur-lemma} for arbitrary representations.
\begin{corollary}
	Fix a Lie group $G$ or Lie algebra $\mf g$, denoted $R$. Fix completely reducible representations
	\[V=\bigoplus_{i=1}^nV_i^{\oplus m_i}\qquad\text{ and }W\cong\bigoplus_{i=1}^nV_i^{\oplus n_i},\]
	where the set $\{V_i\}_{i=1}^n$ consists of pairwise non-isomorphic complex irreducible representations. Then
	\[\op{Hom}_R(V,W)=\bigoplus_{i=1}^n\CC^{n_i\times m_i}.\]
\end{corollary}
\begin{proof}
	Purely formal.\todo{}
\end{proof}

\subsection{The Unitarization Trick}
We would like tools to show that all representations are completely reducible. One place to start is with unitary representations.
\begin{definition}
	Fix a Lie group $G$ or Lie algebra $\mf g$, denoted $R$. Then a representation $V$ of $R$ is \textit{unitary} if and only if it has a positive-definite Hermitian inner product $\langle\cdot,\cdot\rangle$ commuting with the $R$-action. More precisely, we have the following.
	\begin{itemize}
		\item If $R$ is a Lie group, then we want $\langle gv,gw\rangle=\langle v,w\rangle$ for all $g\in G$ and $v,w\in V$.
		\item If $R$ is a Lie algebra, then we want $\langle Xv,w\rangle+\langle v,Xw\rangle=0$ for all $X\in\mf g$ and $v,w\in V$.
	\end{itemize}
\end{definition}
\begin{remark}
	As a quick sanity check, we note that if $\mf g=\op{Lie}G$, then $V$ being a unitary representation of $G$ implies that the induced representation of $\mf g$ is also unitary.
\end{remark}
And here is our result.
\begin{proposition}
	Any unitary representation is completely irreducible.
\end{proposition}
\begin{proof}
	By induction, it is enough to check that any subrepresentation of a representation has a complement. Well, the orthogonal complement will work.
\end{proof}
\begin{example}
	If $G$ is a compact Lie group, then any representation $V$ of $G$ can be made unitary and hence is completely irreducible. Indeed, we need to define an invariant Hermitian form on $V$. Well, start with any Hermitian form $\langle\cdot,\cdot\rangle$, and then we fix it by defining
	\[\langle v,w\rangle_G\coloneqq\int_G\langle gv,gw\rangle\,dg,\]
	where $dg$ is an invariant top differential form on $G$. One can check that this is an invariant Hermitian form.
\end{example}
% \begin{remark}
% 	Continue with $G$ a finite group. It may look like we have a lot of choice in our invariant Hermitian form, but in fact they are only unique up to positive scalar. Indeed, 
% \end{remark}
\begin{example}
	Fix an abelian Lie algebra $\mf g$. Then any Hermitian form on a representation of $\mf g$ is automatically invariant, so we see that all representations are completely irreducible.
\end{example}
\begin{example}
	We see that $\op{SL}_n(\CC)$ has compact real form $\op{SU}_n$, so we see that the representation theory of $\op{SL}_2(\CC)$ must be completely irreducible.\todo{}
\end{example}
The above two examples may look like we have a lot of freedom in our choice of invariant form, but we do not.
\begin{proposition}
	Fix a complex irreducible representation $V$ of a Lie group $G$. Then there is at most one invariant Hermitian form on $V$, up to a positive scalar.
\end{proposition}
\begin{proof}
	Fixing some invariant Hermitian forms $\langle\cdot,\cdot\rangle_1$ and $\langle\cdot,\cdot\rangle_2$ of a representation $V$, then one can find a function $\varphi\colon V\to V$ such that
	\[\langle v,w\rangle_1=\langle \varphi(v),w\rangle_2\]
	for all $v,w\in V$. One can check that $\varphi$ is linear and in fact $G$-invariant, so \Cref{prop:schur-lemma} tells us that $\varphi=\lambda\id_V$ for some $\lambda\in\CC$. Thus, we see that $\langle\cdot,\cdot\rangle_1=\lambda\langle\cdot,\cdot\rangle_2$, so we see that $\lambda>0$ by plugging in any nonzero vector.
\end{proof}

\end{document}