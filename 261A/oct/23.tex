% !TEX root = ../notes.tex

\documentclass[../notes.tex]{subfiles}

\begin{document}

\section{October 23}
Today we show that representations of semisimple Lie algebras are completely reducible.

\subsection{Complete Reducibility}
Complete reducibility basically amounts to showing that any short exact sequence
\[0\to U\to V\to W\to0\]
of $\mf g$-representation splits, which last time we checked is equivalent to asking for $\op{Ext}^1(W,U)$ to vanish. Thus, we are after a result for vanishing of cohomology, so we pick up some theory around cohomology.
\begin{lemma} \label{lem:les-h1}
	Fix a Lie algebra $\mf g$. Then a short exact sequence
	\[0\to U\to V\to W\to0\]
	of representations of $\mf g$ gives rise to a longer exact sequence
	\[0\to U^\mf g\to V^\mf g\to W^\mf g\stackrel\delta\to H^1(\mf g,U)\to H^1(\mf g,V)\to H^1(\mf g,W).\]
\end{lemma}
\begin{proof}
	Label our maps by
	\[0\to U\stackrel\alpha\to V\stackrel\beta\to W\to0.\]
	In the sequel, we may view the embedding $\alpha\colon U\to V$ as an identification.

	The exactness of
	\[0\to U^\mf g\to V^\mf g\to W^\mf g\]
	follows because this is simply a restriction of the short exact sequence. Explicitly, exactness at $V^\mf g$ has a little content: for some $v\in V^\mf g$ which is in the kernel of the map $V^\mf g\to W^\mf g$, we know that there is some $u\in U$ such that $\alpha\colon u\mapsto v$; however, this is a morphism of representations, so $X\alpha\colon u\mapsto Xv$ vanishes for all $X\in\mf g$, so the injectivity of $\alpha$ requires $Xu=0$ for all $X$, so $u\in U^\mf g$.

	Our next step is define the map $\delta$. Well, we take some $w\in W^\mf g$. We need to get all the way to $U$, so we begin by pulling this element back to some $v\in V$. Then we define the $1$-cocycle $c_v\colon\mf g\to U$ by
	\[c_v(X)\coloneqq Xv,\]
	which we note will vanish when mapped to $W$ (because $w\in W^\mf g$) and hence can be identified with an element of $U$. Additionally, we note that this is actually a $1$-cocycle because
	\[c_v([X,Y])=[X,Y]v=XYv-YXv=Xc_v(Y)-Yc_v(X).\]
	Quickly, note that this map $w\mapsto c_v$ is well-defined up to cohomology class: namely, if we choose a different $v'$ lifting $w$, then the difference $c_v-c_{v'}$ is a coboundary in $B^1(\mf g,U)$. Namely, there is some $u\in U$ such that $v'=u+v$, and we can compute that
	\[c_{v'}(X)=c_v(X)+Xu,\]
	and the mapping $u\mapsto Xu$ is a $1$-cobounary.

	We now check the remaining exactness points.
	\begin{itemize}
		\item Exact at $W^\mf g$: note that any $v\in V^\mf g$ has $\beta(v)$ lifting to $v\in V^\mf g$, which has $c_v=0$, so $\delta(\beta(v))=0$. On the other hand, for any $c\in H^1(\mf g,U)$ which vanishes in $H^1(\mf g,V)$, we are being told that $c$ is a $1$-coboundary in $H^1(\mf g,V)$, so there exists $v\in V$ such that $c(X)=Xv$ for all $v$, so $c=c_v=\delta(\beta(v))$ is in the image from $W^\mf g$.
		\item Exact at $H^1(\mf g,V)$ follows by restriction of the short exact sequence again. In one direction, any $c\in H^1(\mf g,U)$ vanishes in $H^1(\mf g,W)$ because $\alpha(\beta(c))(X)$ vanishes always. In the other direction, if $c\in H^1(\mf g,V)$ vanishes under $\beta$, then $c(X)\in U$ for all $X\in\mf g$, so $c$ actually defines an element of $H^1(\mf g,U)$.
		\qedhere
	\end{itemize}
\end{proof}
We are after some vanishing result for $H^1$, as follows.
\begin{theorem} \label{thm:ss-vanishing-h1}
	Fix a semisimple Lie algebra $\mf g$ over a field $F$ of characteristic $0$. For any representation $V$ of $\mf g$, we have $H^1(\mf g,V)$.
\end{theorem}
\begin{proof}
	We proceed in steps.
	\begin{enumerate}
		\item We begin by reducing to the case where $V$ is an irreducible representation. This is by induction on $\dim V$. Indeed, suppose we have the result for irreducible representations. For any representation $V$, find an irreducible subrepresentation $U\subseteq V$. Then \Cref{lem:les-h1} produces the exact sequence
		\[H^1(\mf g,U)\to H^1(\mf g,V)\to H^1(\mf g,V/U).\]
		The left term vanishes because $U$ is irreducible, and the right term vanishes because $V/U$ has smaller dimension than $V$, so we may apply an inductive hypothesis. Thus, the middle term vanishes.

		\item For the rest of the proof, we will assume that $V$ is an irreducible representation; in fact, $H^1(\mf g,F)=0$ by \Cref{ex:cocycle-trivial-rep}, so we may assume that $V\ne F$. Imitating the idea of \Cref{thm:sl2-reduces}, we are interested in $Z(U\mf g)$. Well, suppose that there is $c\in Z(U\mf g)$ such that $\rho_V(c)$ is a nonzero scalar $\lambda\id_V$ but $\rho_F(c)=0$. Then we claim that $H^1(\mf g,V)=\op{Ext}^1_{\mf g}(F,V)$ vanishes. Well, suppose that we have some extension
		\[0\to V\stackrel\alpha\to W\stackrel\beta\to F\to0\]
		which we would like to split. To make this split, we will eventually define a function $s\colon F\to W$ splitting $\beta$, but for this, we should look for $s(1)$. Well, by scaling and using $c$ suitably, we can find $w\in W$ such that $\beta(w)=1$. Now, we note that we can adjust $W$ by some $w$ by some $\lambda^{-1}cw$ in order to further get $cw=0$.

		We now use $w$ to define the required splitting. Define $s\colon F\to W$ by $s(1)\coloneqq w$. Notably, $Fw\subseteq W$ is a subrepresentation: for any $X\in\mf g$, we see that $\beta(Xw)=X\beta(w)=0$, so $Xw\in V$, but then $\lambda Xw=cXw=Xcw=0$, so $Xw=0$. Thus, $s$ is the required splitting.

		\item We are now on the hunt for the desired $c\in Z(U\mf g)$. Let $B_V(x,y)\coloneqq\tr\rho_V(X)\rho_V(Y)$. Note that $B_V$ needs to be nonzero on the image of $\mf g\to\mf{gl}(V)$: \Cref{lem:get-cartan-solvability} would then imply that the image of $\mf g$ would be solvable, which is a problem because $\mf g$ has no nonzero solvable quotients (as these would lift to solvable ideals of $\mf g$), meaning that $V$ is the trivial representation.

		Now, $\mf g$ will split into $K\oplus\mf g'$, where $K$ is the kernel of $B_V$. Then we can select a basis $\{x_1,\ldots,x_n\}$ of $\mf g'$, and then $B_V$ provides a dual basis $\{x_1^\lor,\ldots,x_n^\lor\}$ of $\mf g'$. We now define
		\[C\coloneqq\sum_{i=1}^nx_ix_i^\lor,\]
		which is some element of $U\mf g$. One can check that this does not depend on the choice of basis and lives in $Z(U\mf g)$, and in fact $C|_F=0$. Further, we see that $\tr\rho_V(C)=\dim\mf g$ because $B_V(x_i,x_i^\lor)=1$, so $\rho_V(C)\ne0$; further, one can check that $C$ is a scalar when passing to the algebraic closure, so $C$ must just be a scalar. Thus, we are done by the previous step.
		\qedhere
	\end{enumerate}
\end{proof}
From the theorem, we get the desired result.
\begin{corollary}
	Fix a semisimple Lie algebra $\mf g$ over an algebraically closed field $F$ of characteristic $0$. Then any representation $V$ of $\mf g$ is completely reducible.
\end{corollary}
\begin{proof}
	Induct on $\dim V$. For $\dim V\in\{0,1\}$, there is nothing to do. For the inductive step, find an irreducible subrepresentation $U\subseteq V$, and then we see that the short exact sequence
	\[0\to U\to V\to(V/U)\to0\]
	must split by \Cref{thm:ss-vanishing-h1}, so complete reducibility for $V$ follows from the inductive hypothesis applied to $U$ and $V/U$.
\end{proof}
We can use this result to prove the Levi decomposition for reductive groups.
\begin{corollary}[Levi decomposition]
	A reductive Lie algebra $\mf g$ over an algebraically closed field $F$ of characteristic $0$ is the direct sum of an abelian and semisimple Lie algebra.
\end{corollary}
\begin{proof}
	We want to show that the exact sequence
	\[0\to\mf z(\mf g)\to\mf g\to\mf g_{\mathrm{ss}}\to0\]
	splits. Well, the action of $\mf g$ on $\mf z(\mf g)$ actually descends to make $\mf z(\mf g)$ into a representation of $\mf g_{\mathrm{ss}}$. Thus, this is a short exact sequence of representations of $\mf g_{\mathrm{ss}}$, so this splits as a sequence of representations of $\mf g_{\mathrm{ss}}$. But then we only need to add the center back in to see that this implies the splitting of Lie algebras.
\end{proof}

\end{document}