% !TEX root = ../notes.tex

\documentclass[../notes.tex]{subfiles}

\begin{document}

\section{October 14}
Today we finish up some structure theory of Lie algebras.

\subsection{Engel's Theorem}
Last time we proved the following theorem.
\begin{theorem}[Lie]
	If $\mf g$ is a complex solvable Lie algebra, then any irreducible representation $V$ of $\mf g$ is $1$-dimensional. For any representation of $V$ of $\mf g$, there is a basis in which $\mf g$ acts by upper-triangular matrices.
\end{theorem}
Today we begin by proving Engel's theorem. Here are some lemmas.
\begin{lemma}
	Fix a Lie algebra $\mf g$ over a field $F$, and let $\rho\colon\mf g\to\mf{gl}(V)$ be a representation into an $F$-vector space $V$.
	\begin{listalph}
		\item If all elements of $\mf g$ act by nilpotent operators, then there is nonzero $v\in V$ such that
		\[Xv=0\]
		for all $X\in\mf g$.
		\item There exists a basis of $V$ in which all matrices in $\mf g$ are strictly upper-triangular.
	\end{listalph}
\end{lemma}
\begin{proof}
	Note that (b) follows from (a) inductively, so we focus on proving (a). For this, we induct on $\dim\mf g$, where the case of $\dim\mf g=0$ means that $\mf g$ vanishes, so there is nothing to do.

	To induct downwards, we would like to find a Lie ideal of $\mf g$. Well, we claim that any maximal proper subalgebra $\mf h$ is a Lie ideal of codimension $1$. Of course, $\mf h$ has codimension $1$ because otherwise we could add a vector to it, violating maximality. We will Now, consider the adjoint action of $\mf g$ on $\mf g$, which we note will descend to an adjoint action of $\mf h$ on $\mf g/\mf h$. Notably, all $X\in\mf h$ are nilpotent (this is true for $\mf g\subseteq\mf{gl}(V)$ even), so one can check that the operator $\op{ad}_X\colon\mf g\to\mf g$ is nilpotent by some repeated applications, so $\op{ad}_X\colon(\mf g/\mf h)\to(\mf g/\mf h)$ is a nilpotent operator, so the inductive hypothesis tells us that we can find nonzero $Y+\mf h\in\mf g/\mf h$ such that $\op{ad}_X(Y)\in\mf h$ for all $X$. Thus, we see that we can write $\mf g=\mf h+kY$ is a subalgebra, so we conclude that $\mf h$ is an ideal of codimension $1$.

	For the inductive step, we let $W$ by the collection of $\mf h$-invariants of $W$. The inductive hypothesis applied to $\mf h$ tells us that $V^{\mf h}$ is nonzero. Note that $W$ is in fact a subrepresentation for $\mf g$ because any $X\in\mf h$ and $w\in W$ has
	\[XYw+YXw+[X,Y]w=0,\]
	so $Yw$ is fixed by $\mf h$, and we are done. To complete the proof, we choose some nonzero $w\in W$ and apply $Y$ enough times until $Y^{k-1}w\ne0$ but $Y^kw=0$. Then $Y^{k-1}w$ works.
\end{proof}
\begin{remark}
	This result works over any field $F$.
\end{remark}
\begin{definition}
	An element $X\in\mf g$ is nilpotent if and only if $\op{ad}_X\colon\mf g\to\mf g$ is a nilpotent operator.
\end{definition}
\begin{corollary}
	A finite-dimensional complex Lie algebra $\mf g$ is nilpotent if and only if each $X\in\mf g$ is nilpotent.
\end{corollary}
\begin{proof}
	Of course, if $\mf g$ is nilpotent, then each element is nilpotent by considering some adjoint composite with the lower central series. Conversely, if each element of $\mf g$ is nilpotent, then the lemma above allows us to strictly upper-triangularize our operators $\op{ad}_X$, which means that they are nilpotent.
\end{proof}

\subsection{Semisimple Lie Algebras}
We now give a central definition of this subject: semisimple.
\begin{lemma}
	Fix a Lie algebra $\mf g$. Then $\mf g$ contains a maximal solvable Lie ideal containing all solvable Lie ideals.
\end{lemma}
\begin{proof}
	The main point it to show that the sum of two solvable ideals is again solvable, for which we note that we have short exact sequences
	\[0\to I\to I+J\to\frac{I+J}I\to0\]
	and
	\[0\to (I\cap J)\to J\to\frac{I+J}I\to0,\]
	so we are able to conclude that $I+J$ is solvable. Everything in sight is finite-dimensional, so we can just sum over all solvable ideals to get the required ideal.
\end{proof}
The lemma allows us to define semisimple.
\begin{defihelper}[radical, simple, semisimple] \nirindex{radical} \nirindex{simple} \nirindex{semisimple}
	Fix a Lie algebra $\mf g$.
	\begin{listalph}
		\item The \textit{radical} $\op{rad}\mf g$ of $\mf g$ is the maximal solvable ideal of $\mf g$.
		\item $\mf g$ is \textit{semisimple} if and only if $\op{rad}\mf g=0$.
		\item $\mf g$ is \textit{simple} if and only if it is not abelian and the only ideals of $\mf g$ are $0$ and $\mf g$.
	\end{listalph}
\end{defihelper}
Some remarks are in order to make sure that these definitions make sense.
\begin{remark}
	One can see that $\mf g$ is simple if and only if $\op{ad}_\bullet\colon\mf g\to\mf{gl}(\mf g)$ is an irreducible representation.
\end{remark}
\begin{remark}
	If $\mf g$ is simple and $\dim\mf g>1$, then $[\mf g,\mf g]=\mf g$. Indeed, $[\mf g,\mf g]$ is some Lie ideal, and it must be nonzero: if $[\mf g,\mf g]=0$, then $\mf g$ is abelian, which is not permitted.
\end{remark}
\begin{remark}
	If $\mf g$ is simple, then $\mf g$ is semisimple. Indeed, if $\mf g$ is simple, then the only Lie ideals available are $0$ and $\mf g$. However, $\mf g$ is not solvable because $[\mf g,\mf g]=\mf g$.
\end{remark}
\begin{remark}
	We claim that
	\[\op{rad}(\mf g_1\oplus\mf g_2)=\op{rad}\mf g_1\oplus\op{rad}\mf g_2.\]
	This implies that the direct sum of semisimple Lie algebras is semisimple.
\end{remark}
\begin{example}
	The Lie algebras $\mf{sl}_2(\CC)$ and $\mf{so}_3(\CC)$ are simple. This can be checked directly. We will show later that $\mf{sl}_n(\CC)$ and $\mf{so}_n(\CC)$ and $\mf{sp}_{2n}(\CC)$ are all semisimple.
\end{example}
In general, one can always reduce our Lie algebras to semisimple ones.
\begin{lemma}
	The Lie algebra $\mf g_{\mathrm{ss}}\coloneqq\mf g/\op{rad}\mf g$ is semisimple. In fact, if $\mf h\subseteq\mf g$ is solvable and has $\mf g/\mf h$ semisimple, then $\mf h=\op{rad}\mf g$.
\end{lemma}
\begin{proof}
	Solvable ideals lift from the quotient. Namely, $I\subseteq\mf g/\op{rad}\mf g$ being solvable implies that its pre-image in $\mf g$ is solvable, so it will be contained in the radical. For the second part, we note that $\mf h\subseteq\op{rad}\mf g$, so $\mf g/\mf h\onto\mf g/\op{rad}\mf g$.
\end{proof}
\begin{remark}
	It is a theorem that $\op{char}F=0$ makes the short exact sequence
	\[0\to\op{rad}\mf g\to\mf g\to\mf g_{\mathrm{ss}}\to0\]
	splits. We will not prove this today.
\end{remark}
\begin{example}
	One can show that the group of rigid motions preserving orientation is a semidirect product of rotation group $\op{SO}_3(\RR)$ with the translation group $\RR^3$.
\end{example}

\end{document}