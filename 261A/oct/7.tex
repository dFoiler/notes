% !TEX root = ../notes.tex

\documentclass[../notes.tex]{subfiles}

\begin{document}

\section{October 7}
Today we finish classifying the representations of $\mf{sl}_2$.

\subsection{}
Recall that the Casimir operator $C\coloneqq ef+fe+\frac12h^2$ will commute with the action of $\op{sl}_2$ in any representation. We are using $C$ to show that any representation is sum of the $V_n$s. By an induction, we may assume that we are in the situation where we have an extension
\[0\to V_n\to W\to V_n^{\oplus(m-1)}\to0,\]
and we need to check that $W\cong V_n^{\oplus m}$. For this, we note that we can split $W$ as
\[W=\bigoplus_{\lambda\in\CC}W(\lambda),\]
where $W(\lambda)$ is the generalized eigenspace for $\lambda$. However, one checks that there is a highest weight $n$ of $W$, implying we can write
\[W=\bigoplus_{i=0}^nW(n-2i).\]
Notably, the commutator relations imply that $eW(\lambda)=W(\lambda+2)$, so $W(n)\subseteq\ker e|_W$.

We complete our classification by showing that $h$ acts diagonally on $\ker e$. Well, for any $u$, we note that $f^mu=0$ for some $m$ large enough (by looking at weights), but the commutator relations find that
\[e^mf^mu=e^{m-1}f^{m-1}m(h-m+1)u=\cdots=m!h(h-1)\cdots(h-(m+1))u.\]
Thus, the polynomial $h(h-1)\cdots(h-(m+1))$ acts by $0$ on $\ker e$, so the minimal polynomial of $h$ has no repeated roots, so $h$ acts diagonally.

To apply the fact that $h$ acts diagonally on $W(n)\subseteq\ker e$, we note that $W(n-2i)$ is simply $W(n)$ shifted over by $f$ some number of times, so we complete. Here are some corollaries.
\begin{corollary}
	The element $h$ acts diagonally on any finite-dimensional complex representation of $\mf{sl}_2$.
\end{corollary}
\begin{corollary}
	Let $N\colon V\to V$ be a nilpotent operator on a finite-dimensional vector space $V$. Then there is a unique (up to isomorphism) way to make $V$ into a representation of $\mf{sl}_2$ so that $N=e|_V$.
\end{corollary}
\begin{proof}
	Put $N$ into Jordan normal form. Then the Jordan blocks communicate how $e$ should operate. The commutator relations imply how the rest of $\mf{sl}_2$ should operate.
\end{proof}
We would also like to understand tensor products.
\begin{definition}
	Fix a finite-dimensional complex representation $V$ of $\mf{sl}_2$. Then the \textit{character} of $V$ is
	\[\chi_V(z)\coloneqq\sum_{m\in\ZZ}\dim V(m)z^m.\]
\end{definition}
\begin{remark}
	This relates to the usual character by noting that $\chi_V\left(e^t\right)=\tr_V(\exp(th))$.
\end{remark}
One can check that $\chi_{V\oplus W}=\chi_V+\chi_W$ and $\chi_{V\otimes W}=\chi_V\chi_W$ by a direct computation with the eigenspaces. One can also compute that
\[\chi_n(z)\coloneqq\chi_{V_n}(z)=z^n+z^{n-2}+\cdots+z^{2-n}+z^{-n}=\frac{z^{n+1}-z^{-n-1}}{z-z^{-1}}.\]
Thus, we see that these characters are linearly independent (say, over $\QQ$), so we can recover any representation by writing it as a sum of the characters $\chi_{V_n}$. (One can algorithmically remove highest-order terms in order to get our decomposition.)

This character theory allows us to prove the following result.
\begin{theorem}
	One has
	\[V_m\otimes V_n=\bigoplus_{0\le i\le\min\{m,n\}}=V_{\left|m-n\right|+2i}.\]
\end{theorem}
\begin{proof}
	Decompose the characters.
\end{proof}
\begin{example}
	One can compute directly that
	\[\chi_2\chi_3=\chi_5+\chi_3+\chi_1.\]
\end{example}
\begin{example}
	One finds that $\chi_n^2=\chi_0+\cdots+\chi_{2n}$.
\end{example}

\subsection{The Universal Enveloping Algebra}
We have shown that the category $\op{Rep}_\CC\mf g$ is abelian for any Lie algebra $\mf g$, so we may expect to be able to realize this category as a category of modules over some (possibly non-commutative) ring. This is the role of the universal enveloping algebra.

To start, we would have the universal algebra
\[T\mf g\coloneqq\bigoplus_{k=0}^\infty\mf g^{\otimes k},\]
which is some graded ring, where the multiplication $\mf g^{\otimes k}\otimes\mf g^{\otimes\ell}\to\mf g^{\otimes(k+\ell)}$ is given by concatenation. We now take a quotient to remember the Lie bracket.
\begin{definition}[universal enveloping algebra]
	We define the \textit{universal enveloping algebra} $U\mf g$ as the quotient of $T\mf g$ by the two-sided ideal $I\mf g$ generated by the elements
	\[x\otimes y-y\otimes x-[x,y].\]
\end{definition}
Here are some properties of this definition.
\begin{proposition}
	Fix a Lie algebra $\mf g$.
	\begin{listalph}
		\item For any associative algebra $A$, we have
		\[\op{Hom}_{\mathrm{LieAlg}}(\mf g,A)\simeq\op{Hom}_{\mathrm{Alg}}(U\mf g,A).\]
		\item The category $\op{Rep}_k\mf g$ is equivalent to the category $\op{Mod}(U\mf g)$.
	\end{listalph}
\end{proposition}
In order to actually compute $U\mf g$, one can fix a basis $\{x_1,\ldots,x_n\}$ of $\mf g$ and then note that $T\mf g$ will be the free polynomial ring in these variables, so we can take a quotient to recover $U\mf g$.
\begin{example}
	If $\mf g$ is abelian, then we see that $U\mf g=k[x_1,\ldots,x_n]$.
\end{example}
\begin{example}
	We see that
	\[U(\mf{sl}_2)=\frac{\CC\langle e,f,h\rangle}{(ef-fe-h,he-eh-2e,hf-fh+2f)}.\]
\end{example}
As we had with our Lie algebra, we note that $U\mf g$ can act on itself in some interesting ways. For example, it has the usual left multiplication structures, but there is also an adjoint action.
\begin{lemma}
	The action of $\mf g$ on $T\mf g$ by derivations descends to $U\mf g$.
\end{lemma}
\begin{proof}
	For any $Z\in\mf g$, we want to check that $\op{ad}_Z(I)\subseteq I$. Well, it is enough to check this on generators of $I$, for which we compute
	\begin{align*}
		\op{ad}_Z(X\otimes Y-Y\otimes X-[X,Y]) &= ([Z,X]\otimes Y-Y\otimes[Z,X])+(X\otimes[Z,Y]-[Z,Y]\otimes X)-[Z,[X,Y]],
	\end{align*}
	and we are done after applying the Jacobi identity and rearranging.
\end{proof}

\end{document}