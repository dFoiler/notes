% !TEX root = ../notes.tex

\documentclass[../notes.tex]{subfiles}

\begin{document}

\section{October 7}
Today we finish classifying the representations of $\mf{sl}_2$.

\subsection{Applications for Representations of \texorpdfstring{$\mf{sl}_2(\CC)$}{sl2(C)}}
% Recall that the Casimir operator $C\coloneqq ef+fe+\frac12h^2$ will commute with the action of $\op{sl}_2$ in any representation. We are using $C$ to show that any representation is sum of the $V_n$s. By an induction, we may assume that we are in the situation where we have an extension
% \[0\to V_n\to W\to V_n^{\oplus(m-1)}\to0,\]
% and we need to check that $W\cong V_n^{\oplus m}$. For this, we note that we can split $W$ as
% \[W=\bigoplus_{\lambda\in\CC}W(\lambda),\]
% where $W(\lambda)$ is the generalized eigenspace for $\lambda$. However, one checks that there is a highest weight $n$ of $W$, implying we can write
% \[W=\bigoplus_{i=0}^nW(n-2i).\]
% Notably, the commutator relations imply that $eW(\lambda)=W(\lambda+2)$, so $W(n)\subseteq\ker e|_W$.

% We complete our classification by showing that $h$ acts diagonally on $\ker e$. Well, for any $u$, we note that $f^mu=0$ for some $m$ large enough (by looking at weights), but the commutator relations find that
% \[e^mf^mu=e^{m-1}f^{m-1}m(h-m+1)u=\cdots=m!h(h-1)\cdots(h-(m+1))u.\]
% Thus, the polynomial $h(h-1)\cdots(h-(m+1))$ acts by $0$ on $\ker e$, so the minimal polynomial of $h$ has no repeated roots, so $h$ acts diagonally.

% To apply the fact that $h$ acts diagonally on $W(n)\subseteq\ker e$, we note that $W(n-2i)$ is simply $W(n)$ shifted over by $f$ some number of times, so we complete.
Let's discuss some applications of \Cref{thm:sl2-classify-irreps,thm:sl2-reduces}. To begin, we can upgrade the diagonal action of \Cref{lem:sl2-h-diagonalizes}.
\begin{corollary} \label{cor:sl2-h-diagonalizes-better}
	Fix any complex representation $\rho\colon\mf{sl}_2(\CC)\to\mf{gl}(V)$. Then $\rho(h)\colon V\to V$ acts diagonally with eigenvalues in $\ZZ$.
\end{corollary}
\begin{proof}
	By \Cref{thm:sl2-reduces}, it is enough to check this for irreducible representations $V$. By \Cref{thm:sl2-classify-irreps}, we see that $V\cong\op{Sym}^nV_{\mathrm{std}}$, where $V_{\mathrm{std}}$ is the standard representation. Then \Cref{lem:sl2-poly-is-sym-power} explains that these can be realized as polynomial representations, from which the required diagonalization of $\rho(h)$ follows from its computation on the monomial basis given in \eqref{eq:sl2-poly-action}.
\end{proof}
\begin{corollary}[Jacobson--Morozov]
	For any nilpotent operator $N\colon V\to V$ on a finite-dimensional complex vector space $V$, there exists a (unique up to isomorphism) structure of $\mf{sl}_2(\CC)$-representation on $V$ such that $e\mapsto N$. More precisely, we have the following. 
	\begin{listalph}
		\item There exists a representation $\rho\colon\mf{sl}_2(\CC)\to\mf{gl}(V)$ such that $\rho(e)=N$.
		\item If $\rho^1,\rho^2\colon\mf{sl}_2(\CC)\to\mf{gl}(V)$ have $\rho^1(e)=\rho^2(e)$, then $\rho^1\cong\rho^2$.
	\end{listalph}
\end{corollary}
\begin{proof}
	We will proceed with the claims separately.
	\begin{listalph}
		\item By giving $V$ a basis, we may identify it with $\CC^d$; using the Jordan normal form, we are able to choose this basis so that $V$ is the direct sum of Jordan blocks of the form
		\[J_n=\underbrace{\begin{bmatrix}
			0 & 1 \\
			  & 0 & 1 \\
			  &   & \ddots & \ddots \\
			  &   &        &        & 0 & 1 \\
			  &   &        &        &   & 0
		\end{bmatrix}}_{n+1}\in \CC^{(n+1)\times(n+1)}.\]
		Decomposing $V=V_1\oplus\cdots\oplus V_m$ so that $N$ decomposes into these Jordan blocks as $N=J_{n_1}\oplus\cdots\oplus J_{n_m}$, we see that we may assume that $N=J_n$ for some $n$: if we can find $\rho_i\colon\mf{sl}_2(\CC)\to\mf{gl}(V_i)$ such that $\rho_i(e)=J_{n_i}$ for each $i$, then $\rho\coloneqq\rho_1\oplus\cdots\oplus\rho_m$ will be a representation $\rho\colon\mf{sl}_2(\CC)\to\mf{gl}(V)$ satisfying $\rho(e)=N$.

		We are thus reduced to the case where $N=J_n$ for some $n\ge0$; note then that $\dim V=n+1$, so we expect to be able to take $\rho=\rho_n$. Let $\{v_0,\ldots,v_n\}$ be the given basis of $V$, which we will adjust to fit $\rho(e)=J_n$. With this in mind, define $\varphi\colon V\to V_n$ by $\varphi(v_q)\coloneqq q!^{-1}x^{n-q}y^q$; this sends a basis to a basis, so $\varphi$ is an isomorphism of vector spaces. Further, we claim that $\varphi\circ N=\rho_n(e)\circ\varphi$: it is enough to check this on the basis $\{v_0,\ldots,v_n\}$, so we use \eqref{eq:sl2-poly-action} to compute $\varphi(Nv_0)=0=\rho_n(e)x^n$ and
		\begin{align*}
			\varphi(Nv_q) &= \varphi(v_{q-1}) \\
			&= (q-1)!^{-1}x^{n-q+1}y^{q-1} \\
			&= \rho_n(e)\left(q!^{-1}x^{n-q}y^q\right) \\
			&= \rho_n(e)\varphi(v_q)
		\end{align*}
		for $q\ge1$. Thus, we may define $\rho\colon\mf{sl}_2(\CC)\to\mf{gl}(V)$ by $\rho(X)\coloneqq\varphi^{-1}\circ\rho_n(X)\circ\varphi$ for all $X\in\mf{sl}_2(\CC)$. Adjusting by conjugating $\varphi$ makes it so that $\rho$ succeeds by a representation, and we checked that $\varphi\circ N=\rho_n(e)\circ\varphi$, so $\rho(e)=N$, which is what we wanted.

		\item We proceed directly. We will read the structure of $\rho^1$ and $\rho^2$ directly off of $N$. Fix some $i\in\{1,2\}$. By using \Cref{thm:sl2-classify-irreps,thm:sl2-reduces}, we may decompose
		\[\rho^i\cong\bigoplus_{n\ge0}\rho_n^{\oplus a_n^i}\]
		for some nonnegative integers $a_{n}^i\ge0$. We will show that $a_{m}^1=a_{m}^2$ for each $m\ge0$, which will complete the proof by comparing the two decompositions.
		
		For this, we use the dimension of $\ker\rho^i(e)^m$ for various $m\ge0$. In particular, \eqref{eq:sl2-poly-action} gives
		\[\dim\ker\rho_n(e)^m=\dim\op{span}\left\{x^n,yx^{n-1},\ldots,x^{n-m+1}y^{m-1}\right\}=\min\{m,n+1\},\]
		so
		\[\dim\ker N^m=\dim\ker\rho^i(e)^m=\sum_{n\ge0}\dim\ker\left((\rho_n(e)^{\oplus a_n^i}\right)^m=\sum_{n\ge0}a_n^i\max\{m,n+1\}.\]
		We now use this to read off the values of $a_n^i$: for any $m\ge1$, we see
		\[\sum_{n=0}^{m-1}a_n^i=\sum_{n\ge0}a_n^i\max\{m+1,n+1\}-\sum_{n\ge0}a_n^i\max\{m,n+1\}=\dim\ker N^{m+1}-\dim\ker N^m,\]
		so
		\[a_m^i=\sum_{n=0}^{m}a_n^i-\sum_{n=0}^{m-1}a_n^i=\dim\ker N^{m+2}-\dim\ker N^m.\]
		Thus, $a_m^i$ is independent of $i$, so $a_m^1=a_m^2$ for all $m\ge0$.
		\qedhere
	\end{listalph}
\end{proof}
\begin{example}
	We may hope that $\rho_1=\rho_2$ on the nose, but this is not true in general. For example, one can use an inner automorphism of $\rho$ fixing $\rho(e)$ to produce an isomorphic representation $\rho'\colon\mf{sl}_2(\CC)\to V$ with $\rho(e)=\rho'(e)$. Concretely, take $V=\CC^2$ and $N\coloneqq\begin{bsmallmatrix}
		0 & 1 \\ 0 & 0
	\end{bsmallmatrix}$. Then we could define $\rho=\rho_{\mathrm{std}}$ and
	\[\rho'(X)\coloneqq\begin{bmatrix}
		1 & 1 \\ 0 & 1
	\end{bmatrix}\rho(X)\begin{bmatrix}
		1 & 1 \\ 0 & 1
	\end{bmatrix}^{-1}\]
	so that $\rho'(e)=\rho(e)$ while $\rho'(h)\ne\rho(h)$.
\end{example}

\subsection{Character Theory of \texorpdfstring{$\mf{sl}_2(\CC)$}{sl2(C)}}
By analogy with the representation theory of finite groups, we may want a notion of characters for representations of $\mf{sl}_2(\CC)$. Our classification allows to do this cleanly.
\begin{definition}
	Fix a finite-dimensional complex representation $\rho\colon\mf{sl}_2(\CC)\to V$. For any $n\in\ZZ$, let $V[n]$ be the eigenspace of $\rho(h)$ with eigenvalue $n$. Then the \textit{character} of $V$ is the rational polynomial
	\[\chi_\rho(T)\coloneqq\sum_{n\in\ZZ}\dim V[n]T^n.\]
	This is a polynomial in $\ZZ\left[T,T^{-1}\right]$ because only finitely many of the $V[n]$ may be nonzero because $\dim V$ is finite. We will write $\chi_V$ for $\chi_\rho$ when no confusion is possible.
\end{definition}
\begin{remark}
	This definition does not lose any information by merely considering the given $V[n]$s because \Cref{cor:sl2-h-diagonalizes-better} tells us that $\rho(h)$ diagonalizes with integral eigenvalues.
\end{remark}
\begin{remark}
	To relate this definition with the characters of finite groups, we claim
	\[\chi_V\left(e^t\right)\stackrel?=\tr\exp(t\rho(h)).\]
	Indeed, using \Cref{cor:sl2-h-diagonalizes-better}, we may write $\rho(h)=\op{diag}(n_1,\ldots,n_d)$ where $\{n_1,\ldots,n_d\}$ are integers. (Technically, we do not need to know that the action is diagonal for the subsequent argument.) Then
	\[\tr\exp(t\rho(h)) = \tr\exp(\op{diag}(tn_1,\ldots,tn_d))=\sum_{i=1}^de^{tn_1}=\sum_{i=1}^d\left(e^t\right)^{n_i}.\]
	Grouping the $n_\bullet$s by multiplicity, we conclude that this equals $\chi_V\left(e^t\right)$: note $\dim V[n]=\#\{i:n_i=n\}$.
\end{remark}
\begin{example} \label{ex:sl2-poly-chars}
	Using \eqref{eq:sl2-poly-action}, we see that $n\ge0$ has
	\[\chi_{V_n}(T)=T^{-n}+T^{-n+2}+\cdots+T^{n-2}+T^n=\frac{T^{n+1}-T^{-n-1}}{T-T^{-1}}.\]
	For example, we see that $\chi_{V_n}(T)$ is the first of the $\chi_n$s with nonzero coefficient on $T^n$, so the collection $\{\chi_n:n\ge0\}$ is $\CC$-linearly independent. Explicitly, any nontrivial expression $\sum_{n\ge0}a_n\chi_{V_n}$ (with $a_n=0$ for all but finitely many $n$) will have some largest $N$ for which $a_N\ne 0$, but then the monomial $a_NT^N$ lives in $\sum_{n\ge0}a_n\chi_{V_n}$, so $\sum_{n\ge0}a_n\chi_{V_n}\ne0$.
\end{example}
Here are some easy checks on our characters.
\begin{lemma} \label{lem:sl2-decompose-char}
	Fix complex representations $\rho_V\colon\mf{sl}_2(\CC)\to\mf{gl}(V)$ and $\rho_W\colon\mf{sl}_2(\CC)\to\mf{gl}(W)$.
	\begin{listalph}
		\item We have $\chi_{V\oplus W}=\chi_V+\chi_W$.
		\item We have $\chi_{V\otimes W}=\chi_V\cdot\chi_W$.
		\item We have $\chi_{V^\lor}(T)=\chi_V\left(T^{-1}\right)$.
	\end{listalph}
\end{lemma}
\begin{proof}
	These checks are purely formal.
	\begin{listalph}
		\item Because $\rho_{V\oplus W}(h)=\rho_V(h)\oplus\rho_W(h)$, we can split up our eigenspaces for $n\in\ZZ$ by
		\[(V\oplus W)[n]=V[n]\oplus W[n],\]
		so
		\begin{align*}
			\chi_{V\oplus W}(T) &= \sum_{n\in\ZZ}\dim(V\oplus W)[n]T^n \\
			&= \sum_{n\in\ZZ}\dim(V[n]\oplus W[n])T^n \\
			&= \sum_{n\in\ZZ}\dim V[n]T^n+\sum_{n\in\ZZ}\dim W[n]T^n \\
			&= \chi_V(T)+\chi_W(T).
		\end{align*}
		\item Let $\{v_1,\ldots,v_k\}$ and $\{w_1,\ldots,w_\ell\}$ be eigenbases for the operators $\rho_V(h)\colon V\to V$ and $\rho_W(h)\colon W\to W$ with eigenvalues $\{\lambda_1,\ldots,\lambda_k\}$ and $\{\mu_1,\ldots,\mu_\ell\}$, respectively. Then $\{v_i\otimes w_j\}$ is a basis for $V\otimes W$, and in fact it is an eigenbasis for $\rho_{V\otimes W}(h)$: note
		\begin{align*}
			\rho_{V\otimes W}(h)(v_i\otimes w_j) &= \rho_V(h)v_i\otimes w_j+v_i\otimes\rho_W(h)w_j \\
			&= \lambda_iv_i\otimes w_j+w_i\otimes\mu_jw_j \\
			&= (\lambda_i+\mu_j)(v_i\otimes w_j).
		\end{align*}
		Thus, for any $z\in\CC$, we see that
		\[\dim(V\otimes W)[n]=\#\{(i,j):\lambda_i+\mu_j=n\},\]
		so
		\begin{align*}
			\chi_{V\otimes W}(T) &= \sum_{n\in\ZZ}\dim(V\otimes W)[n]T^n \\
			&= \sum_{n\in\ZZ}\#\{(i,j):\lambda_i+\mu_j=n\}T^n \\
			&= \sum_{n\in\ZZ}\sum_{a+b=z}(\dim V[a]\dim W[b])T^n \\
			&= \sum_{a,b\in\ZZ}(\dim V[a]\dim W[b])T^{a+b} \\
			&= \Bigg(\sum_{a\in\ZZ}\dim V[a]T^a\Bigg)\Bigg(\sum_{b\in\ZZ}\dim V[b]T^b\Bigg) \\
			&= \chi_V(T)\chi_W(T),
		\end{align*}
		\item As in (b), let $\{v_1,\ldots,v_k\}$ be an eigenbasis for the operator $\rho_V(h)\colon V\to V$ with eigenvalues $\{\lambda_1,\ldots,\lambda_k\}$. Then we claim that the dual basis $\left\{v_1^\lor,\ldots,v_k^\lor\right\}$ is an eigenbasis for $\rho_{V^\lor}(h)$: for any $v_i^\lor$ and $v_j$, we see
		\begin{align*}
			\left(\rho_{V^\lor}(h)v_i^\lor\right)(v_j) &= -v_i^\lor\left(\rho_V(h)v_j\right) \\
			&= -\lambda_jv_i^\lor(v_j) \\
			&= -\lambda_j1_{i=j},
		\end{align*}
		so $\rho_{V^\lor}(h)v_i^\lor=-\lambda_iv_i^\lor$. Thus, gathering multiplicities, we see that $\dim V[n]=\dim V^\lor[-n]$ for any $n\in\ZZ$, so
		\begin{align*}
			\chi_{V^\lor}(T)=\sum_{n\in\ZZ}\dim V^\lor[n]T^n \\
			&= \sum_{n\in\ZZ}\dim V[-n]T^n \\
			&= \sum_{n\in\ZZ}\dim V[n]\left(T^{-1}\right)^n \\
			&= \chi_V\left(T^{-1}\right),
		\end{align*}
		as required.
		\qedhere
	\end{listalph}
\end{proof}
Importantly, we can use characters to determine representations.
\begin{proposition} \label{prop:sl2-char-determines}
	Fix complex representations $\rho_V\colon\mf{sl}_2(\CC)\to\mf{gl}(V)$ and $\rho_W\colon\mf{sl}_2(\CC)\to\mf{gl}(W)$. If $\chi_V=\chi_W$, then $V\cong W$.
\end{proposition}
\begin{proof}
	By \Cref{thm:sl2-classify-irreps,thm:sl2-reduces}, we have decompositions
	\[V\cong\bigoplus_{n\ge0}V_n^{\oplus a_n}\qquad\text{and}\qquad W\cong\bigoplus_{n\ge0}V_n^{\oplus b_n}.\]
	We will show that $a_n=b_n$ for all $n$, which will complete the proof upon comparing the decompositions. Well, \Cref{lem:sl2-decompose-char} tells us that
	\begin{align*}
		0 &= \chi_V(T)-\chi_W(T) \\
		&= \sum_{n\ge0}a_n\chi_{V_n}(T)-\sum_{n\ge0}b_n\chi_{V_n}(T) \\
		&= \sum_{n\ge0}(a_n-b_n)\chi_{V_n}(T).
	\end{align*}
	This relation is enough to imply $a_n=b_n$ for all $n$ by the linear independence given in \Cref{ex:sl2-poly-chars}.
\end{proof}
\begin{example} \label{ex:sl2-dual-reps}
	We claim that $V\cong V^\lor$ for any complex representation $\rho_V\colon\mf{sl}_2(\CC)\to\mf{gl}(V)$. By \Cref{prop:sl2-char-determines}, we may check this on characters. Using the complete reducibility of \Cref{thm:sl2-reduces} with \Cref{lem:sl2-decompose-char}, it is enough to check this for irreducible $V$ (notably, $(V\oplus W)^\lor\cong V^\lor\oplus W^\lor$). Thus, \Cref{thm:sl2-classify-irreps} lets us assume that $V=V_n$ for some $n\ge0$, so we are left to show that
	\[\chi_{V_n}(T)\stackrel?=\chi_{V_n}\left(T^{-1}\right)\]
	by \Cref{lem:sl2-decompose-char}. This is true by the explicit computation of \Cref{ex:sl2-poly-chars}.
\end{example}
\begin{example} \label{ex:sl2-tensor-example}
	We claim that $V_2\otimes V_3\cong V_1\oplus V_3\oplus V_5$. By \Cref{prop:sl2-char-determines}, it is enough to show an equality of characters. For this, we use \Cref{ex:sl2-poly-chars} with \Cref{lem:sl2-decompose-char} to see
	\begin{align*}
		\chi_{V_2\otimes V_3}(T) &= \left(T^{-2}+1+T^2\right)\left(T^{-3}+T^{-1}+T+T^3\right) \\
		&= T^{-5}+2T^{-3}+3T^{-1}+3T+2T^3+T^5 \\
		&= \chi_{V_5}(T)+\chi_{V_3}(T)+\chi_{V_1}(T) \\
		&= \chi_{V_1\oplus V_3\oplus V_5}(T).
	\end{align*}
\end{example}
Here is the general case of \Cref{ex:sl2-tensor-example}.
% One can check that $\chi_{V\oplus W}=\chi_V+\chi_W$ and $\chi_{V\otimes W}=\chi_V\chi_W$ by a direct computation with the eigenspaces. One can also compute that
% \[\chi_n(z)\coloneqq\chi_{V_n}(z)=z^n+z^{n-2}+\cdots+z^{2-n}+z^{-n}=\frac{z^{n+1}-z^{-n-1}}{z-z^{-1}}.\]
% Thus, we see that these characters are linearly independent (say, over $\QQ$), so we can recover any representation by writing it as a sum of the characters $\chi_{V_n}$. (One can algorithmically remove highest-order terms in order to get our decomposition.)
% This character theory allows us to prove the following result.
\begin{proposition}[Clebsch--Gordan rule]
	Let $V_1$ be the standard representation $\rho_1\colon\mf{sl}_2(\CC)\subseteq\mf{gl}_2(\CC)$, and define $V_n\coloneqq\op{Sym}^nV_1$ for each $n\ge0$ so that we have representations $\rho_n\colon\mf{sl}_2(\CC)\to\mf{gl}(V_n)$. Then
	\[V_m\otimes V_n\cong\bigoplus_{i=0}^{\min\{m,n\}}V_{\left|m-n\right|+2i}.\]
\end{proposition}
\begin{proof}
	By symmetry, we may assume that $m\le n$. We generalize the argument of \Cref{ex:sl2-tensor-example}. By \Cref{prop:sl2-char-determines}, it is enough to compare the characters of both sides, for which we use \Cref{lem:sl2-decompose-char} with \Cref{ex:sl2-poly-chars}. Now, we compute
	\begin{align*}
		\chi_{V_m\otimes V_n}(T) &= \chi_{V_m}(T)\chi_{V_n}(T) \\
		&= \frac{T^{m+1}-T^{-m-1}}{T-T^{-1}}\cdot\frac{T^{n+1}-T^{-n-1}}{T-T^{-1}}, \\
		\chi_{\bigoplus_{i=0}^{m}V_{n-m+2i}}(T) &= \sum_{i=0}^m\chi_{V_{n-m+2i}}(T) \\
		&= \sum_{i=0}^m\frac{T^{n-m+2i+1}-T^{-(n-m-2i+1)}}{T-T^{-1}}.
	\end{align*}
	Thus, it remains to show the combinatorial identity
	\[\frac{T^{m+1}-T^{-m-1}}{T-T^{-1}}\cdot\frac{T^{n+1}-T^{-n-1}}{T-T^{-1}}\stackrel?=\sum_{i=0}^m\frac{T^{n-m+2i+1}-T^{-(n-m+2i+1)}}{T-T^{-1}}.\]
	Multiplying both sides by $T-T^{-1}$, we would like to show that
	\[\frac{\left(T^{m+1}-T^{-m-1}\right)\left(T^{n+1}-T^{-n-1}\right)}{T-T^{-1}}\stackrel?=\sum_{i=0}^m\left(T^{n-m+2i+1}-T^{-(n-m+2i+1)}\right).\]
	Well,
	\[\sum_{i=0}^mT^{\pm\left(n-m+2i+1\right)}=\frac{T^{\pm(n+m+2+1)}-T^{\pm(n-m+1)}}{T^{\pm2}-1},\]
	so
	\begin{align*}
		\sum_{i=0}^m\left(T^{n-m+2i+1}-T^{-(n-m+2i+1)}\right) &= \frac{T^{(n+m+2+1)}-T^{(n-m+1)}}{T^{2}-1}-\frac{T^{-(n+m+2+1)}-T^{-(n-m+1)}}{T^{-2}-1} \\
		&= \frac{T^{n+m+2}-T^{n-m}}{T-T^{-1}}-\frac{T^{-n-m-2}-T^{-n+m}}{T^{-1}-T} \\
		&= \frac{T^{n+m+2}-T^{n-m}-T^{-n+m}+T^{-n-m-2}}{T-T^{-1}} \\
		&= \frac{\left(T^{m+1}-T^{-m-1}\right)\left(T^{n+1}-T^{-n-1}\right)}{T-T^{-1}},
	\end{align*}
	as desired.
\end{proof}
% \begin{example}
% 	One can compute directly that
% 	\[\chi_2\chi_3=\chi_5+\chi_3+\chi_1.\]
% \end{example}
% \begin{example}
% 	One finds that $\chi_n^2=\chi_0+\cdots+\chi_{2n}$.
% \end{example}

\subsection{Dual Representations of \texorpdfstring{$\mf{sl}_2(\CC)$}{sl2(C)}}
\Cref{ex:sl2-dual-reps} showed that $V_n\cong V_n^\lor$ for each $n\ge0$. We would like to be able to provide an explicit such map. Well, recalling \Cref{lem:dualize-inner-prod-rep}, we see that we would like to give an invariant bilinear inner product. Here is the result.
\begin{proposition}
	Let $V_1$ be the standard representation $\rho_1\colon\mf{sl}_2(\CC)\subseteq\mf{gl}_2(\CC)$, and define $V_n\coloneqq\op{Sym}^nV_1$ for each $n\ge0$ so that we have representations $\rho_n\colon\mf{sl}_2(\CC)\to\mf{gl}(V_n)$. Then $V_n$ admits an $\mathfrak{sl}_2(\CC)$-invariant inner product $\langle-,-\rangle$ which is symmetric when $n$ is even and skew-symmetric when $n$ is odd.
\end{proposition}
\begin{proof}
	We begin by defining the bilinear form. Set $V\coloneqq V_{\mathrm{std}}$ for brevity. Now, for any $n\ge0$, we define $\langle-,-\rangle\colon V_n\times V_n\to\CC$ on by
	\[\big\langle(v_1,\ldots,v_n),(w_1,\ldots,w_n)\big\rangle\coloneqq\prod_{i=1}^n\det\begin{bmatrix}
		| & | \\
		v_i & w_i \\
		| & |
	\end{bmatrix}.\]
	Because determinants (and products of determinants) are multilinear, we see that this produces a multilinear map $V^{\otimes n}\otimes V^{\otimes n}\to\CC$. Thus, we have produced a bilinear form. We now run checks in sequence.
	\begin{enumerate}
		\item Note that this bilinear form on $V^{\otimes n}$ is $\op{SL}_2(\CC)$-invariant, which implies that it will be $\mf{sl}_2(\CC)$-invariant upon passing to the differential representation by \Cref{lem:descend-invariant-inner-prod}. Well, for any $g\in\op{SL}_2(\CC)$, it is enough to check the invariance on pure tensors by the ambient bilinearity, so we compute
		\begin{align*}
			& \big\langle g(v_1\otimes\cdots\otimes v_n),g(w_1\otimes\cdots\otimes w_n)\big\rangle \\
			={}& \big\langle (gv_1\otimes\cdots\otimes gv_n),(gw_1\otimes\cdots\otimes gw_n)\big\rangle \\
			={}& \prod_{i=1}^n\det\begin{bmatrix}
				| & | \\
				gv_i & gw_i \\
				| & |
			\end{bmatrix} \\
			={}& \prod_{i=1}^n\det g\begin{bmatrix}
				| & | \\
				v_i & w_i \\
				| & |
			\end{bmatrix} \\
			={}& (\underbrace{\det g}_1)^n\prod_{i=1}^n\begin{bmatrix}
				| & | \\
				v_i & w_i \\
				| & |
			\end{bmatrix} \\
			={}& \big\langle g(v_1\otimes\cdots\otimes v_n),g(w_1\otimes\cdots\otimes w_n)\big\rangle \\
			={}& \big\langle (v_1\otimes\cdots\otimes v_n),(w_1\otimes\cdots\otimes w_n)\big\rangle,
		\end{align*}
		as required.
		\item We now restrict $\langle-,-\rangle$ to the subrepresentation $\op{Sym}^nV\subseteq V^{\otimes n}$. As such, we see that
		\begin{align*}
			\langle v_1\cdots v_n,w_1\cdots w_n\rangle &= \frac1{(n!)^2}\sum_{\sigma,\tau\in S_n}\prod_{i=1}^n\det\begin{bmatrix}
				| & | \\
				v_{\sigma i} & w_{\tau i} \\
				| & |
			\end{bmatrix} \\
			&= \frac1{n!}\sum_{\sigma\in S_n}\prod_{i=1}^n\det\begin{bmatrix}
				| & | \\
				v_{\sigma i} & w_{i} \\
				| & |
			\end{bmatrix}
		\end{align*}
		by rearranging. Before doing the non-degeneracy check, we verify that $\langle-,-\rangle$ is symmetric when $n$ is even and skew-symmetric when $n$ is odd. This condition is linear in all the coordinates, so it is enough to check the (skew-)symmetry on the spanning set of tensors of the form $v_1\cdots v_n$. Well,
		\begin{align*}
			\langle v_1\cdots v_n,w_1\cdots w_n\rangle &= \frac1{n!}\sum_{\sigma\in S_n}\prod_{i=1}^n\det\begin{bmatrix}
				| & | \\
				v_{\sigma i} & w_{i} \\
				| & |
			\end{bmatrix} \\
			&= \frac1{n!}\sum_{\sigma\in S_n}\prod_{i=1}^n-\det\begin{bmatrix}
				| & | \\
				w_{\sigma i} & v_{i} \\
				| & |
			\end{bmatrix} \\
			&= (-1)^n\frac1{n!}\sum_{\sigma\in S_n}\prod_{i=1}^n\det\begin{bmatrix}
				| & | \\
				w_{i} & v_{\sigma i} \\
				| & |
			\end{bmatrix} \\
			&= (-1)^n\langle v_1\cdots v_n,w_1\cdots w_n\rangle,
		\end{align*}
		as required.
		\item We now check that $\langle-,-\rangle$ is non-degenerate. We will do this by computing it on a basis. Let $V$ have basis $\{v_x,v_y\}$, and then we note that $\op{Sym}^nV$ has basis $\{v_x^pv_y^q\}_{p+q=n}$. For bookkeeping reasons, for each pair $(p,q)$, define the function $v_{pq}\colon\{1,\ldots,n\}\to V$ by
		\[v_{pq}(i)\coloneqq\begin{cases}
			v_x & \text{if }i \le p, \\
			v_y & \text{if }i > p.
		\end{cases}\]
		In particular, $v_x^pv_y^q=v_{pq}(1)\cdots v_{pq}(n)$. We now choose two basis vectors $v_x^pv_y^q$ and $v_x^{p'}v_y^{q'}$ and compute
		\begin{align*}
			\left\langle v_x^pv_y^q,v_x^{p'}v_y^{q'}\right\rangle &= \left\langle v_{pq}(1)\cdots v_{pq}(n),v_{p'q'}(1)\cdots v_{p'q'}(n)\right\rangle \\
			&= \frac1{n!}\sum_{\sigma\in S_n}\prod_{i=1}^n\det\begin{bmatrix}
				| & | \\
				v_{pq}(\sigma i) & v_{p'q'}(i) \\
				| & |
			\end{bmatrix} \\
			&= \frac1{n!}\sum_{\sigma\in S_n}\prod_{i=1}^{p'}\det\begin{bmatrix}
				| & | \\
				v_{pq}(\sigma i) & v_x \\
				| & |
			\end{bmatrix}\prod_{i=p'+1}^{n}\det\begin{bmatrix}
				| & | \\
				v_{pq}(\sigma i) & v_y \\
				| & |
			\end{bmatrix} \\
		\end{align*}
		Now, the only way for a product of these determinants to not vanish is to have $v_{pq}(\sigma i)=v_y$ for $i\le p'$ and $v_{pq}(\sigma i)=v_x$ for $i>p'$. In particular, by counting the number of $v_x$s and $v_y$s, we see that all terms of the sum vanish unless $(p,q)=(q',p')$. In the case where $(p,q)=(q',p')$, we have
		\begin{align*}
			\left\langle v_x^pv_y^q,v_x^qv_y^p\right\rangle &= \frac1{n!}\sum_{\sigma\in S_n}\prod_{i=1}^{q}\det\begin{bmatrix}
				| & | \\
				v_{pq}(\sigma i) & v_x \\
				| & |
			\end{bmatrix}\prod_{i=q+1}^{n}\det\begin{bmatrix}
				| & | \\
				v_{pq}(\sigma i) & v_y \\
				| & |
			\end{bmatrix},
		\end{align*}
		so we see that each nonzero term will evaluate to $(-1)^q$, and the number of nonzero terms is the number of $\sigma\in S_n$ such that $\sigma$ carries $\{1,\ldots,q\}$ to $\{p+1,\ldots,n\}$. There are $p!q!$ total such permutations, so we conclude that
		\[\left\langle v_x^pv_y^q,v_x^qv_y^p\right\rangle=(-1)^q\frac{p!q!}{n!}\ne0.\]
		Thus, we see that the matrix given by the bilinear form on the basis $\left\{v_x^pv_y^q\right\}_{p+q=n}$ is anti-diagonal with all nonzero anti-diagonal entries, so it is invertible. Thus, the relevant bilinear form is non-degen\-erate.
		\qedhere
	\end{enumerate}
\end{proof}

\subsection{The Universal Enveloping Algebra}
We have shown that the category $\op{Rep}_\CC\mf g$ is abelian for any Lie algebra $\mf g$, so we may expect to be able to realize this category as a category of modules over some (possibly non-commutative) ring. This is the role of the universal enveloping algebra.

To start, we begin with a universal algebra which does not remember the Lie bracket.
\begin{definition}[universal tensor algebra]
	Fix a vector space $\mf g$ over a field $F$. Then the \textit{universal tensor algebra} is the vector space
	\[T\mf g\coloneqq\bigoplus_{k=0}^\infty\mf g^{\otimes k}.\]
	We turn $T\mf g$ into an $F$-algebra by defining multiplication $T\mf g\otimes_F T\mf g\to T\mf g$ on the components $\mf g^{\otimes k}\otimes\mf g^{\otimes\ell}\to\mf g^{\otimes(k+\ell)}$ by the natural ``concatenation'' isomorphism. We will let $\iota_{\mf g}\colon\mf g\to T\mf g$ denote the canonical map.
\end{definition}
\begin{example}
	Fix a basis $\{X_1,\ldots,X_n\}$ of some $\mf g$. Then $T\mf g$ is (by its definition) the free (non-commu\-tative!) polynomial algebra $F\langle X_1,\ldots,X_n\rangle$.
\end{example}
Quickly, we note that this construction is functorial.
\begin{lemma} \label{lem:tensor-algebra-functor}
	Fix a field $F$. The universal tensor algebra defines a functor $\mathrm{Vec}(F)\to\mathrm{Alg}(F)$.
\end{lemma}
\begin{proof}
	To begin, note that a linear map $f\colon\mf g\to\mf h$ of $F$-vector spaces induces an $F$-algebra map $Tf\colon T\mf g\to T\mf h$. Namely, note there is certainly an $F$-linear map $f\colon\mf g^{\otimes k}\to\mf h^{\otimes k}$ on the components of $T\mf g$ by functoriality of the tensor product (more precisely, note that the map $\mf g^k\to\mf h^{\otimes k}$ given by $(v_1,\ldots,v_k)\mapsto f(v_1)\otimes\cdots\otimes f(v_k)$ if $F$-multilinear), so we get a linear map $Tf\colon T\mf g\to T\mf h$. Then we check that $Tf$ is an algebra map: it is enough to check the vanishing on a spanning subset of $T\mf g$, for which we see that pure tensors span by the definition of $T\mf g$, so it is enough to compute
	\begin{align*}
		Tf((v_1\otimes\cdots\otimes v_k)\cdot(w_1\otimes\cdots\otimes w_\ell)) &= f(v_1)\otimes\cdots\otimes f(v_k)\otimes f(w_1)\otimes\cdots\otimes f(w_\ell) \\
		&= Tf(v_1\otimes\cdots\otimes v_k\otimes w_1\otimes\cdots\otimes w_\ell).
	\end{align*}
	Quickly, we note that $Tf$ is in fact the unique map making the diagram
	% https://q.uiver.app/#q=WzAsNCxbMCwwLCJcXG1mIGciXSxbMSwwLCJcXG1mIGgiXSxbMCwxLCJUXFxtZiBnIl0sWzEsMSwiVFxcbWYgaCJdLFswLDEsImYiXSxbMCwyLCJcXGlvdGFfe1xcbWYgZ30iLDJdLFsxLDMsIlxcaW90YV97XFxtZiBofSJdLFsyLDMsIlRmIl1d&macro_url=https%3A%2F%2Fraw.githubusercontent.com%2FdFoiler%2Fnotes%2Fmaster%2Fnir.tex
	\[\begin{tikzcd}
		{\mf g} & {\mf h} \\
		{T\mf g} & {T\mf h}
		\arrow["f", from=1-1, to=1-2]
		\arrow["{\iota_{\mf g}}"', from=1-1, to=2-1]
		\arrow["{\iota_{\mf h}}", from=1-2, to=2-2]
		\arrow["Tf", from=2-1, to=2-2]
	\end{tikzcd}\]
	commute. Above we checked that $Tf$ is a well-defined algebra morphism, and its definition gives $Tf(\iota_{\mf g}v)=\iota-{\mf h}f(v)$, so the diagram commutes. For the uniqueness, note that any algebra morphism $Tf$ is determined by a spanning subset of $T\mf g$, for which we can take the pure tensors; however, being an algebra morphism then requires that
	\[Tf(v_1\otimes\cdots\otimes v_k)=f(v_1)\otimes\cdots\otimes f(v_k)\]
	on pure tensors, fully determining $Tf$.

	It remains to run some functoriality checks.
	\begin{itemize}
		\item Identity: the diagram
		% https://q.uiver.app/#q=WzAsNCxbMCwwLCJcXG1mIGciXSxbMSwwLCJcXG1mIGciXSxbMCwxLCJUXFxtZiBnIl0sWzEsMSwiVFxcbWYgZyJdLFswLDEsIlxcaWRfe1xcbWYgZ30iXSxbMiwzLCJcXGlkX3tUXFxtZiBnfSJdLFswLDIsIlxcaW90YV97XFxtZiBnfSIsMl0sWzEsMywiXFxpb3RhX3tcXG1mIGd9Il1d&macro_url=https%3A%2F%2Fraw.githubusercontent.com%2FdFoiler%2Fnotes%2Fmaster%2Fnir.tex
		\[\begin{tikzcd}
			{\mf g} & {\mf g} \\
			{T\mf g} & {T\mf g}
			\arrow["{\id_{\mf g}}", from=1-1, to=1-2]
			\arrow["{\iota_{\mf g}}"', from=1-1, to=2-1]
			\arrow["{\iota_{\mf g}}", from=1-2, to=2-2]
			\arrow["{\id_{T\mf g}}", from=2-1, to=2-2]
		\end{tikzcd}\]
		commutes, so the uniqueness of $T\id_{\mf g}$ requires $T{\id_{\mf g}}=\id_{T\mf g}$.
		\item Associativity: for morphisms $f\colon\mf g\to\mf h$ and $g\colon\mf h\to\mf k$, we see that the diagram
		% https://q.uiver.app/#q=WzAsNixbMCwwLCJcXG1mIGciXSxbMSwwLCJcXG1mIGgiXSxbMiwwLCJcXG1mIGsiXSxbMCwxLCJUXFxtZiBnIl0sWzEsMSwiVFxcbWYgaCJdLFsyLDEsIlRcXG1mIGsiXSxbMCwxLCJmIl0sWzEsMiwiZyJdLFswLDMsIlxcaW90YV97XFxtZiBnfSIsMl0sWzEsNCwiXFxpb3RhX3tcXG1mIGh9IiwyXSxbMiw1LCJcXGlvdGFfe1xcbWYga30iXSxbMyw0LCJUZiIsMl0sWzQsNSwiVGciLDJdXQ==&macro_url=https%3A%2F%2Fraw.githubusercontent.com%2FdFoiler%2Fnotes%2Fmaster%2Fnir.tex
		\[\begin{tikzcd}
			{\mf g} & {\mf h} & {\mf k} \\
			{T\mf g} & {T\mf h} & {T\mf k}
			\arrow["f", from=1-1, to=1-2]
			\arrow["{\iota_{\mf g}}"', from=1-1, to=2-1]
			\arrow["g", from=1-2, to=1-3]
			\arrow["{\iota_{\mf h}}"', from=1-2, to=2-2]
			\arrow["{\iota_{\mf k}}", from=1-3, to=2-3]
			\arrow["Tf"', from=2-1, to=2-2]
			\arrow["Tg"', from=2-2, to=2-3]
		\end{tikzcd}\]
		commutes, so the uniqueness of $T(g\circ f)$ forces $T(g\circ f)=Tg\circ Tf$.
		\qedhere
	\end{itemize}
\end{proof}
Here is the universal property.
\begin{lemma} \label{lem:tensor-universal}
	Fix a vector space $\mf g$ over a field $F$, and let $\iota\colon\mf g\to T\mf g$ be the natural map.
	\begin{listalph}
		\item For any $F$-algebra $A$, restriction provides a natural bijection
		\[\op{Hom}_{\op{Alg}(F)}(T\mf g,A)\to\op{Hom}_F(\mf g,A).\]
		\item The category $\op{Mod}_F(T\mf g)$ is equivalent to the category $\op{Mod}_F(\mf g)$, where a $\mf g$-module is an $F$-vector space $V$ with a morphism $\mf g\to\mf{gl}(V)$. The functor $\op{Mod}_F(T\mf g)\to\op{Mod}_F(\mf g)$ sends algebra morphisms $\varphi\colon T\mf g\to\mf{gl}(V)$ to linear morphisms $(\varphi\circ\iota)\colon\mf g\to\mf{gl}(V)$.
	\end{listalph}
\end{lemma}
\begin{proof}
	We run our checks separately.
	\begin{listalph}
		\item Of course, one can restrict an $F$-algebra morphism $T\mf g\to A$ to a morphism $\mf g\to A$ via the inclusion $\iota\colon\mf g\subseteq T\mf g$. Here are our checks on this construction.
		\begin{itemize}
			\item Linear: given $a_1,a_2\in F$ and $\varphi_1,\varphi_2\colon T\mf g\to A$, we see that
			\[(a_1\varphi_1+a_2\varphi_2)\circ\iota=a_1(\varphi\circ\iota)+a_2(\varphi\circ\iota)\]
			because all maps in sight are linear.

			\item Injective: it is enough to show that we have trivial kernel, so suppose that $\varphi\colon T\mf g\to A$ has $\varphi\circ\iota=0$, and we want to check that $\varphi=0$. Well, $T\mf g$ is spanned by its components $\mf g^{\otimes k}$, so it is enough to check that $\varphi|_{\mf g^{\otimes k}}$. Further, $\mf g^{\otimes k}$ is spanned by pure tensors, so it is enough to check that the linear map $\varphi$ vanishes on pure tensors in $\mf g^{\otimes k}$. Well, any pure tensor looks like $v_1\otimes\cdots\otimes v_k$ for some vectors $v_1,\ldots,v_k\in\mf g$, for which we note
			\[\varphi(v_1\otimes\cdots\otimes v_k)=\varphi(v_1)\cdots\varphi(v_k)=0\cdots0=0.\]

			\item Surjective: given any linear map $\psi\colon\mf g\to A$, we must extend it to an algebra map $\varphi\colon T\mf g\to A$. We begin by defining the linear map $\varphi$, and then we will show that it is actually a map of $F$-algebras. Well, it is enough to define $\varphi$ on each of the components $\mf g^{\otimes k}$ and then take the direct sum; thus, we need to define an $F$-multilinear map $\varphi\colon\mf g^k\to A$, for which we take
			\[\varphi(v_1,\ldots,v_k)\coloneqq\psi(v_1)\cdots\psi(v_k).\]
			Because $A$ is an $F$-algebra, this map is in fact $F$-multilinear, so we descend to a linear map $\varphi\colon\mf g^{\otimes k}\to A$, which we can then sum together to produce an $F$-linear map $\varphi\colon T\mf g\to A$.

			It remains to check that $\varphi$ is actually multiplicative. Well, the condition that $\varphi(x)\varphi(y)-\varphi(xy)=0$ for all $x,y\in T\mf g$ is equivalent to the vanishing of the corresponding bilinear functional $T\mf g\times T\mf g\to A$. Thus, we are checking if some linear functional $T\mf g\otimes_F T\mf g\to A$ vanishes, which we can check on a spanning subset of $T\mf g\otimes_F T\mf g$. Well, we see that $T\mf g$ is spanned by pure tensors of the form $(v_1\otimes\cdots\otimes v_k)$, so it suffices to compute that
			\[\varphi(v_1\otimes\cdots\otimes v_k)\varphi(w_1\otimes\cdots\otimes w_\ell)=\psi(v_1)\cdots\psi(v_k)\psi(w_1)\cdots\psi(w_\ell)=\varphi((v_1\otimes\cdots\otimes v_k)\cdot(w_1\otimes\cdots\otimes w_\ell)),\]
			as required.
			
			\item Natural in $A$: we claim that the given morphism is natural in $A$. Indeed, we note that any $F$-algebra map $f\colon A\to B$ makes the diagram
			% https://q.uiver.app/#q=WzAsOCxbMCwwLCJcXG9we0hvbX1fe1xcb3B7QWxnfShGKX0oVFxcbWYgZyxBKSJdLFsxLDAsIlxcb3B7SG9tfV9GKFxcbWYgZyxBKSJdLFswLDEsIlxcb3B7SG9tfV97XFxvcHtBbGd9KEYpfShUXFxtZiBnLEIpIl0sWzEsMSwiXFxvcHtIb219X0YoXFxtZiBnLEIpIl0sWzIsMCwiXFx2YXJwaGkiXSxbMywwLCIoXFx2YXJwaGlcXGNpcmNcXGlvdGEpIl0sWzIsMSwiKGZcXGNpcmNcXHZhcnBoaSkiXSxbMywxLCIoZlxcY2lyY1xcdmFycGhpXFxjaXJjXFxpb3RhKSJdLFswLDIsImYiLDJdLFsxLDMsImYiXSxbMCwxLCJcXGlvdGEiXSxbMiwzLCJcXGlvdGEiXSxbNCw2LCIiLDAseyJzdHlsZSI6eyJ0YWlsIjp7Im5hbWUiOiJtYXBzIHRvIn19fV0sWzYsNywiIiwwLHsic3R5bGUiOnsidGFpbCI6eyJuYW1lIjoibWFwcyB0byJ9fX1dLFs1LDcsIiIsMix7InN0eWxlIjp7InRhaWwiOnsibmFtZSI6Im1hcHMgdG8ifX19XSxbNCw1LCIiLDIseyJzdHlsZSI6eyJ0YWlsIjp7Im5hbWUiOiJtYXBzIHRvIn19fV1d&macro_url=https%3A%2F%2Fraw.githubusercontent.com%2FdFoiler%2Fnotes%2Fmaster%2Fnir.tex
			\[\begin{tikzcd}
				{\op{Hom}_{\op{Alg}(F)}(T\mf g,A)} & {\op{Hom}_F(\mf g,A)} & \varphi & {(\varphi\circ\iota)} \\
				{\op{Hom}_{\op{Alg}(F)}(T\mf g,B)} & {\op{Hom}_F(\mf g,B)} & {(f\circ\varphi)} & {(f\circ\varphi\circ\iota)}
				\arrow["\iota", from=1-1, to=1-2]
				\arrow["f"', from=1-1, to=2-1]
				\arrow["f", from=1-2, to=2-2]
				\arrow[maps to, from=1-3, to=1-4]
				\arrow[maps to, from=1-3, to=2-3]
				\arrow[maps to, from=1-4, to=2-4]
				\arrow["\iota", from=2-1, to=2-2]
				\arrow[maps to, from=2-3, to=2-4]
			\end{tikzcd}\]
			commute.

			\item Natural in $\mf g$: we claim that the given morphism is natural in $\mf g$. Indeed, for any linear map $f\colon\mf g\to\mf h$, we note that the diagram
			% https://q.uiver.app/#q=WzAsOCxbMCwwLCJcXG9we0hvbX1fe1xcb3B7QWxnfShGKX0oVFxcbWYgZyxBKSJdLFswLDEsIlxcb3B7SG9tfV97XFxvcHtBbGd9KEYpfShUXFxtZiBoLEEpIl0sWzEsMCwiXFxvcHtIb219X3tGfShcXG1mIGcsQSkiXSxbMSwxLCJcXG9we0hvbX1fRihcXG1mIGgsQSkiXSxbMiwxLCJcXHZhcnBoaSJdLFsyLDAsIlxcdmFycGhpXFxjaXJjIFRmIl0sWzMsMSwiXFx2YXJwaGlcXGNpcmNcXGlvdGFfe1xcbWYgaH0iXSxbMywwLCJcXHZhcnBoaVxcY2lyYyhUZlxcY2lyY1xcaW90YV97XFxtZiBnfSkiXSxbMSwwLCJUZiJdLFszLDIsImYiLDJdLFswLDIsIlxcaW90YV97XFxtZiBnfSJdLFsxLDMsIlxcaW90YV97XFxtZiBofSJdLFs0LDYsIiIsMCx7InN0eWxlIjp7InRhaWwiOnsibmFtZSI6Im1hcHMgdG8ifX19XSxbNiw3LCIiLDAseyJzdHlsZSI6eyJ0YWlsIjp7Im5hbWUiOiJtYXBzIHRvIn19fV0sWzQsNSwiIiwyLHsic3R5bGUiOnsidGFpbCI6eyJuYW1lIjoibWFwcyB0byJ9fX1dLFs1LDcsIiIsMix7InN0eWxlIjp7InRhaWwiOnsibmFtZSI6Im1hcHMgdG8ifX19XV0=&macro_url=https%3A%2F%2Fraw.githubusercontent.com%2FdFoiler%2Fnotes%2Fmaster%2Fnir.tex
			\[\begin{tikzcd}
				{\op{Hom}_{\op{Alg}(F)}(T\mf g,A)} & {\op{Hom}_{F}(\mf g,A)} & {\varphi\circ Tf} & {\varphi\circ(Tf\circ\iota_{\mf g})} \\
				{\op{Hom}_{\op{Alg}(F)}(T\mf h,A)} & {\op{Hom}_F(\mf h,A)} & \varphi & {\varphi\circ\iota_{\mf h}}
				\arrow["{\iota_{\mf g}}", from=1-1, to=1-2]
				\arrow[maps to, from=1-3, to=1-4]
				\arrow["Tf", from=2-1, to=1-1]
				\arrow["{\iota_{\mf h}}", from=2-1, to=2-2]
				\arrow["f"', from=2-2, to=1-2]
				\arrow[maps to, from=2-3, to=1-3]
				\arrow[maps to, from=2-3, to=2-4]
				\arrow[maps to, from=2-4, to=1-4]
			\end{tikzcd}\]
			commutes because $\iota_{\mf h}\circ f=\iota_{\mf g}\circ Tf$.
		\end{itemize}

		\item Here are the checks on this functor. Throughout, $V$ and $W$ denote $F$-vector spaces with structure morphisms $\varphi_V\colon T\mf g\to\mf{gl}(V)$ and $\varphi_W\colon T\mf g\to\mf{gl}(W)$ if they are in $\op{Mod}_F(T\mf g)$ and with structure morphisms $\psi_V\colon\mf g\to\mf{gl}(V)$ and $\psi_W\colon\mf g\to\mf{gl}(W)$ if they are in $\op{Mod}_F(\mf g)$.
		\begin{itemize}
			\item Functorial: if $f\colon V\to W$ is a map in $\op{Mod}_F(T\mf g)$, we claim that this is also a map in $\op{Mod}_F(\mf g)$. Indeed, we are being given that
			\[f\circ\varphi_V(X)=\varphi_W(X)\circ f\]
			for any $X\in T\mf g$. Restricting our attention to $X\in\mf g$, we see that
			\[f\circ(\varphi_V\circ\iota)(X)=(\varphi_W\circ\iota)(X)\circ f\]
			for any $X\in\mf g$, so we are done.
			
			Because the induced maps
			\[\op{Hom}_{T\mf g}(V,W)\to\op{Hom}_{\mf g}(V,W)\]
			are just inclusion maps, we see that our mapping is automatically functorial and faithful.

			\item Full: given $V,W\in\op{Mod}_{T\mf g}(A)$, we need to show that any morphism $f\colon V\to W$ preserved by $\mf g$ is fully preserved by $T\mf g$. Explicitly, we are given that
			\[f\circ\varphi_V(X)=\varphi_W(X)\circ f\]
			for any $X\in\mf g$, which we would like to extend to all $T\mf g$. However, the condition that $f\circ\varphi_V(v)=\varphi_W(v)\circ f$ for all $v$ is linear in $v\in T\mf g$, so we may check it on a spanning set. Pure tensors span $T\mf g$, so we may assume that $v=v_1\otimes\cdots\otimes v_k$ for some $v_1,\ldots,v_k\in\mf g$, for which we note that
			\begin{align*}
				f\circ\varphi_V(v_1\otimes\cdots\otimes v_k) &= f\circ\varphi_V(v_1)\circ\cdots\circ\varphi_V(v_k) \\
				&\stackrel*=\varphi_W(v_1)\circ\cdots\circ\varphi_W(v_k)\circ f \\
				&= \varphi(v_1\otimes\cdots\otimes v_k)\circ f,
			\end{align*}
			where $\stackrel*=$ holds by applying the hypothesis inductively.

			\item Essentially surjective: given any linear map $\psi\colon\mf g\to\mf{gl}(V)$, we need to extend it to an algebra map $\varphi\colon T\mf g\to\mf{gl}(V)$. This is the content of (a).
			\qedhere
		\end{itemize}
	\end{listalph}
\end{proof}
We now take a quotient to remember the Lie bracket.
\begin{definition}[universal enveloping algebra]
	Fix a Lie algebra $\mf g$ over a field $F$. We define the \textit{universal enveloping algebra} $U\mf g$ as the quotient of $T\mf g$ by the two-sided ideal $L\mf g$ generated by the elements
	\[\{X\otimes Y-Y\otimes X-[X,Y]:X,Y\in\mf g\}.\]
	We continue to denote the natural linear map $\mf g\to U\mf g$ by $\iota_{\mf g}$, though we frequently omit writing this morphism for brevity (and will as such treat elements of $\mf g$ as already living in $U\mf g$).
\end{definition}
As before, we quickly note that this construction is functorial.
\begin{lemma}
	Fix a field $F$. The universal enveloping algebra defines a functor $\op{LieAlg}(F)\to\op{Alg}(F)$.
\end{lemma}
\begin{proof}
	We begin by defining $U$ on morphisms. We claim that there is a unique morphism $Uf$ making the diagram
	% https://q.uiver.app/#q=WzAsNCxbMCwwLCJcXG1mIGciXSxbMSwwLCJcXG1mIGgiXSxbMCwxLCJVXFxtZiBnIl0sWzEsMSwiVVxcbWYgaCJdLFswLDEsImYiXSxbMCwyLCJcXGlvdGFfe1xcbWYgZ30iLDJdLFsyLDMsIlVmIl0sWzEsMywiXFxpb3RhX3tcXG1mIGh9Il1d&macro_url=https%3A%2F%2Fraw.githubusercontent.com%2FdFoiler%2Fnotes%2Fmaster%2Fnir.tex
	\[\begin{tikzcd}
		{\mf g} & {\mf h} \\
		{U\mf g} & {U\mf h}
		\arrow["f", from=1-1, to=1-2]
		\arrow["{\iota_{\mf g}}"', from=1-1, to=2-1]
		\arrow["{\iota_{\mf h}}", from=1-2, to=2-2]
		\arrow["Uf", from=2-1, to=2-2]
	\end{tikzcd}\]
	commute. For the uniqueness, we note that the required morphism, being a morphism of algebras, will be defined on a generating subset of $U\mf g$. The subset $\mf g\subseteq T\mf g$ generates, and $T\mf g\onto U\mf g$ surjects, so it is enough to define $Uf$ on $\mf g$. But the diagram dictates $Uf(\iota_{\mf g}X)=\iota_{\mf h}f(X)$, so $Uf$ is uniquely determined.
	
	For existence, we note that we already have a morphism $Tf\colon T\mf g\to T\mf h$ of algebras, which we would like to descend to a quotient morphism $Uf\colon U\mf g\to U\mf h$. For this, it is enough to check that $Tf(L\mf g)\subseteq L\mf h$. Because $Tf$ is an algebra morphism, it is enough to check the inclusion on generators of $L\mf g$, for which we note that the elements $X\otimes Y-Y\otimes X-[X,Y]$ generate, so we compute
	\begin{align*}
		Tf(X\otimes Y-Y\otimes X-[X,Y]) &= f(X)\otimes f(Y)-f(Y)\otimes f(X)-f([X,Y]) \\
		&= f(X)\otimes f(Y)-f(Y)\otimes f(X)-[f(X),f(Y)]
	\end{align*}
	is a generating element of $L\mf h$, so we are done. We now note that the functoriality checks for $U\mf g$ are identical to the functoriality checks for $T\mf g$ done at the end of the proof of \Cref{lem:tensor-algebra-functor}.
\end{proof}
Here is the universal property.
\begin{lemma}
	Fix a Lie algebra $\mf g$ over a field $F$, and let $\iota\colon\mf g\to U\mf g$ be the natural map.
	\begin{listalph}
		\item For any $F$-algebra $A$, restriction provides a natural bijection
		\[\op{Hom}_{\mathrm{Alg}(F)}(U\mf g,A)\to\op{Hom}_{\mathrm{LieAlg(F)}}(\mf g,A).\]
		\item The category $\op{Mod}_F(U\mf g)$ is equivalent to the category $\op{Mod}_F(\mf g)$.
	\end{listalph}
\end{lemma}
\begin{proof}
	We run our checks separately.
	\begin{listalph}
		\item We begin by checking that the given map is well-defined. Namely, if $\varphi\colon U\mf g\to A$ is a morphism of algebras, then $(\varphi\circ\iota_{\mf g})\colon\mf g\to A$ is a morphism of Lie algebras. Well, for any $X,Y\in\mf g$, we compute
		\begin{align*}
			(\varphi\circ\iota_{\mf g})([X,Y]) &= \varphi(\iota_{\mf g}[X,Y]) \\
			&\stackrel*= \varphi(X\otimes Y-Y\otimes X) \\
			&= \varphi(X)\varphi(Y)-\varphi(Y)\varphi(X),
		\end{align*}
		where the key step is $\stackrel*=$ where we used the construction of $U\mf g$.
		
		The linearity, injectivity, and naturality checks are now all exactly the same as in \Cref{lem:tensor-universal} (merely replace $T$ with $U$ throughout), so it only remains to check surjectivity. Namely, given a Lie algebra morphism $\psi\colon\mf g\to A$, we must extend it to an algebra map $\ov\varphi\colon U\mf g\to A$. Well, \Cref{lem:tensor-universal} provides some algebra map $\varphi\colon T\mf g\to A$, which we would like to show descends to the quotient to give a morphism $\ov\varphi\colon U\mf g\to A$. To descend to the quotient, we want to check that $L\mf g\subseteq\ker\varphi$, for which we note that it is really enough to show that a generating subset of $L\mf g$ is contained in $\ker\varphi$. For this, we compute
		\[\varphi(X\otimes Y-Y\otimes X-[X,Y]) = \varphi(X)\varphi(Y)-\varphi(Y)\varphi(X)-\varphi([X,Y])\]
		vanishes because $\varphi\colon\mf g\to A$ is a morphism of Lie algebras.

		\item The exact same proof as in \Cref{lem:tensor-universal} goes through after replacing $T$ with $U$ throughout.
		\qedhere
	\end{listalph}
\end{proof}
In order to actually compute $U\mf g$, one can fix a basis $\{X_1,\ldots,X_n\}$ of $\mf g$ and then note that $T\mf g$ will be the free polynomial ring in these variables, so we can take a quotient to recover $U\mf g$.
\begin{example}
	If $\mf g$ is any vector space, we can upgrade $\mf g$ to a Lie algebra with an abelian Lie bracket (namely, $[X,Y]=0$ for all $X,Y\in\mf g$). Giving $\mf g$ the usual basis, we get the polynomial ring
	\[S\mf g\coloneqq U\mf g=\frac{T\mf g}{(X_i\otimes X_j-X_j\otimes X_i:1\le i,j\le n)}=F[X_1,\ldots,X_n].\]
	Even if $\mf g$ is a Lie algebra, we can forget about its Lie bracket and replace it with the abelian Lie bracket to produce a commutative $F$-algebra $S\mf g$.
\end{example}
\begin{example}
	We see that
	\[U(\mf{sl}_2(\CC))=\frac{\CC\langle e,f,h\rangle}{(ef-fe-h,he-eh-2e,hf-fh+2f)}.\]
\end{example}
We close this initial discussion of $U\mf g$ by noting that $U\mf g$ has an adjoint action.
\begin{lemma}
	Fix a Lie algebra $\mf g$ over a field $F$. For $A\mf g\in\{T\mf g,U\mf g,S\mf g\}$, there are unique Lie algebra morphisms $\op{ad}_\bullet\colon\mf g\to\mf{gl}(A\mf g)$ satisfying the following.
	\begin{itemize}
		\item $\op{ad}_X(\iota_{\mf g}Y)=\iota_{\mf g}([X,Y])$ for all $X,Y\in\mf g$.
		\item Leibniz rule: $\op{ad}_X(ab)=(\op{ad}_Xa)b+a(\op{ad}_Xb)$.
	\end{itemize}
	For $U\mf g$, we have $\op{ad}_X(a)=(\iota_{\mf g}X)a-a(\iota_{\mf g}X)$.
\end{lemma}
\begin{proof}
	Throughout, we write $A\mf g$ to denote either $U\mf g$ or $S\mf g$ when the proof works for both. We begin by checking the uniqueness of $\op{ad}_\bullet$. This means that we must define $\op{ad}_X\colon A\mf g\to A\mf g$ for each $X\in\mf g$. For this, it is enough to show that the definition of $\op{ad}_X$ is determined by a spanning subset of $U\mf g$. Well, the surjection $T\mf g\onto A\mf g$ tells us that pure tensors $Y_1\otimes\cdots\otimes Y_k$ span $A\mf g$ because they span $A\mf g$. We claim that the definition of $\op{ad}_X$ is uniquely defined on these pure tensors from the given rules for each $k$, which we prove by induction on $k$. For $k=0$, there is nothing to do because $\op{ad}_X(0)=0$. For the inductive step, we note that the Leibniz rule requires
	\[\op{ad}_X(Y_1\otimes\cdots\otimes Y_k\otimes Y_{k+1}) = \op{ad}_X(Y_1\otimes\cdots\otimes Y_k)\iota_{\mf g}Y_{k+1}+(Y_1\otimes\cdots\otimes Y_k)\iota_{\mf g}\op{ad}_X(Y_{k+1}),\]
	and the right-hand side is uniquely determined by the inductive hypothesis.

	% We now show that $\op{ad}_\bullet\colon \mf g\to\mf{gl}(U\mf g)$ defined by $\op{ad}_X(a)\coloneqq Xa-aX$ is a morphism of Lie algebras satisfying the required properties.
	% \begin{itemize}
	% 	\item We claim that $\op{ad}_X\colon U\mf g\to U\mf g$ is well-defined. The given map $a\mapsto Xa-aX$ is certainly well-defined as a map $T\mf g\to U\mf g$, so it remains to check that $L\mf g\subseteq\op{ad}_X$. Well, it is enough to check this on generators of $L\mf g$, for which we note that
	% 	\begin{align*}
	% 		\op{ad}_X(Y\otimes Z-Z\otimes Y-[Y,Z]) &= \op{ad}_X(Y\otimes Z)-\op{ad}_X(Z\otimes Y)-\op{ad}_X([Y,Z]) \\
	% 		&= Y\otimes\op{ad}_X(Z)+\op{ad}_XY\otimes Z-\op{ad}_XZ\otimes Y-Z\otimes\op{ad}_XY-\op{ad}_X([Y,Z]) \\
	% 		&= Y\otimes[X,Z]+[X,Y]\otimes Z-[X,Z]\otimes Y-Z\otimes [X,Y]-[X,[Y,Z]] \\
	% 		&= (Y\otimes[X,Z]-[X,Z]\otimes Y-[Y,[X,Z]]) \\
	% 		&\qquad+([X,Y]\otimes Z-Z\otimes [X,Y]-[[X,Y],Z]),
	% 	\end{align*}
	% 	where the last equality has used the Jacobi identity. (We have abbreviated our $\iota_{\mf g}$s above for brevity.)
	% 	\item We claim that $\op{ad}_X$ is $F$-linear: for any $c_1,c_2\in F$ and $a_1,a_2\in U\mf g$, we see
	% 	\begin{align*}
	% 		\op{ad}_X(c_1a_2+c_2a_2) &= X(c_1a_1+c_2a_2)-(c_1a_1+c_2a_2)X \\
	% 		&= c_1(Xa_1-a_1X)+c_2(Xa_2-a_2X) \\
	% 		&= c_1\op{ad}_X(a_1)+c_2\op{ad}_X(a_2).
	% 	\end{align*}
	% 	\item We claim that $\op{ad}_\bullet$ is $F$-linear: for any $c_1,c_2\in F$ and $X_1,X_2\in\mf g$ and $a\in U\mf g$, we see
	% 	\begin{align*}
	% 		\op{ad}_{c_1X_1+c_2X_2}(a) &= (c_1X_1+c_2X_2)a-a(c_1X_1+c_2X_2) \\
	% 		&= c_1(X_1a-aX_1)+c_2(X_2a-aX_2) \\
	% 		&= (c_1{\op{ad}_{X_1}}+c_2{\op{ad}_{X_2}})(a).
	% 	\end{align*}
	% 	\item We claim that $\op{ad}_\bullet$ is a morphism of Lie algebras: well, for any $X,Y\in\mf g$ and $a\in U\mf g$, we compute
	% 	\begin{align*}
	% 		({\op{ad}_X}\circ{\op{ad}_Y}-{\op{ad}_Y}\circ{\op{ad}_X})(a) &= \op{ad}_X(\op{ad}_Ya)-\op{ad}_Y(\op{ad}_Xa) \\
	% 		&= \op{ad}_X(Ya-aY)-\op{ad}_Y(Xa-aX) \\
	% 		&= X(Ya-aY)-(Ya-aY)X-Y(Xa-aX)+(Xa-aX)Y \\
	% 		&= (XYa-XaY)-(YaX-aYX)-(YXa-YaX)+(XaY-aXY) \\
	% 		&= (XY-YX)a-a(XY-YX) \\
	% 		&\stackrel*= [X,Y]a-a[X,Y] \\
	% 		&= \op{ad}_{[X,Y]}(a),
	% 	\end{align*}
	% 	where $\stackrel*=$ is true because we are working in $U\mf g$.
	% 	\item Note that $\op{ad}_X(Y)=XY-YX=[X,Y]$ because we are working in $U\mf g$.
	% 	\item Leibniz rule: we see
	% 	\begin{align*}
	% 		\op{ad}_X(ab) &= X(ab)-(ab)X \\
	% 		&= (Xa-aX)b+a(Xb-bX) \\
	% 		&= (\op{ad}_Xa)b+a(\op{ad}_Xb).
	% 	\end{align*}
	% \end{itemize}
	It remains to show that there exists a Lie algebra morphism $\op{ad}_\bullet\colon\mf g\to\mf{gl}(A\mf g)$ satisfying the given properties. Well, we begin by defining the map on the level of $T\mf g$. To merely define the map, it is enough to define it on the components $\mf g^{\otimes k}$, so we need an $F$-multillinear map $\mf g^k\to\mf g^{\otimes k}$, for which we take motivation from the Leibniz rule to write
	\[\op{ad}_X(Y_1,\ldots, Y_k)\coloneqq\sum_{i=1}^kY_1\otimes\cdots\otimes[X,Y_i]\otimes\cdots\otimes X_k\]
	for any $Y_1,\ldots,Y_k\in\mf g$. This map is of course $F$-multilinear because the tensor product and Lie bracket are both multilinear, so we have induced a map $\mf g^{\otimes k}\to\mf g^{\otimes k}$, which then sums to a map $\op{ad}_X\colon T\mf g\to T\mf g$. Here are some checks on this action.
	\begin{itemize}
		\item We note that $\op{ad}_X\colon T\mf g\to T\mf g$ is $F$-linear by its construction.
		\item For any $Y\in\mf g$, we note that $\op{ad}_XY=[X,Y]$ by construction.
		\item Leibniz rule: we claim that $\op{ad}_X(ab)=(\op{ad}_Xa)b+a(\op{ad}_Xb)$ for any $X\in\mf g$ and $a,b\in T\mf g$. This corresponds to an equality of $F$-bilinear maps $T\mf g\times T\mf g\to T\mf g$, so the equality can be checked on a spanning subset of $T\mf g\otimes_FT\mf g$. For this, we note that $T\mf g$ has a spanning subset of pure tensors, so it is enough to compute
		\begin{align*}
			\op{ad}_X((Y_1\otimes\cdots\otimes Y_k)(Z_1\otimes\cdots\otimes Z_\ell)) &= \op{ad}_X((Y_1\otimes\cdots\otimes Y_k)\otimes(Z_1\otimes\cdots\otimes Z_\ell)) \\
			&= \sum_{i=1}^k(Y_1\otimes\cdots\otimes[X,Y_i]\otimes\cdots\otimes Y_k)\otimes(Z_1\otimes\cdots\otimes Z_\ell) \\
			&\qquad+\sum_{j=1}^\ell(Y_1\otimes\cdots\otimes Y_k)\otimes(Z_1\otimes\cdots\otimes[X,Z_j]\otimes\cdots\otimes Z_\ell) \\
			&= \op{ad}_X(Y_1\otimes\cdots\otimes Y_k)(Z_1\otimes\cdots\otimes Z_\ell) \\
			&\qquad+(Y_1\otimes\cdots\otimes Y_k)\op{ad}_X(Z_1\otimes\cdots\otimes Z_\ell),
		\end{align*}
		as required.
		\item Lie algebra morphism: we claim that ${\op{ad}_{[X,Y]}}={\op{ad}_X}\circ{\op{ad}_Y}-{\op{ad}_Y}\circ{\op{ad}_X}$. This corresponds to equality of morphisms $T\mf g\to T\mf g$, so we can check it on a spanning subset of $T\mf g$, for which we use the pure tensors. As such, we compute
		\begin{align*}
			& ({\op{ad}_X}\circ{\op{ad}_Y}-{\op{ad}_Y}\circ{\op{ad}_X})(Z_1\otimes\cdots\otimes Z_k) \\
			={}& \op{ad}_X\op{ad}_Y(Z_1\otimes\cdots\otimes Z_k)+\op{ad}_Y\op{ad}_X(Z_1\otimes\cdots\otimes Z_k) \\
			={}& \sum_{i=1}^k\op{ad}_X(Z_1\otimes\cdots\otimes[Y,Z_i]\otimes\cdots\otimes Z_k) \\
			&\quad-\sum_{j=1}^k\op{ad}_Y(Z_1\otimes\cdots\otimes[X,Z_i]\otimes\cdots\otimes Z_k) \\
			={}& \sum_{i=1}^k(Z_1\otimes\cdots\otimes[X,[Y,Z_i]]\otimes\cdots\otimes Z_k) \\
			&\quad+\sum_{\substack{1\le i,j\le k\\i\ne j}}^k(Z_1\otimes\cdots\otimes[Y,Z_i]\otimes\cdots\otimes[X,Z_j]\otimes\cdots\otimes Z_k) \\
			&\quad-\sum_{j=1}^k(Z_1\otimes\cdots\otimes[Y,[X,Z_i]]\otimes\cdots\otimes Z_k) \\
			&\quad-\sum_{\substack{1\le i,j\le k\\i\ne j}}^k(Z_1\otimes\cdots\otimes[Y,Z_i]\otimes\cdots\otimes[X,Z_j]\otimes\cdots\otimes Z_k) \\
			&{}= \sum_{i=1}^k(Z_1\otimes\cdots\otimes([X,[Y,Z_i]]-[Y,[X,Z_i]])\otimes\cdots\otimes Z_k) \\
			&{}\stackrel*= \sum_{i=1}^k(Z_1\otimes\cdots\otimes[[X,Y],Z_i]\otimes\cdots\otimes Z_k) \\
			&{}= \op{ad}_{[X,Y]}(Z_1\otimes\cdots\otimes Z_k),
		\end{align*}
		where we have used the Jacobi identity at $\stackrel*=$.
	\end{itemize}
	We now descend this definition of $\op{ad}_X$ to a map $A\mf g\to A\mf g$ for $A\mf g\in\{U\mf g,S\mf g\}$. Let $I$ be the kernel of the natural projection $T\mf g\to A\mf g$, and we want to check that $\op{ad}_X(I)\subseteq I$. Note that it suffices to check this on generators of $I$: if $J\subseteq I$ is the subspace such that $\op{ad}_X(a)\subseteq I$ for each $a\in J$, we claim that $J$ is an ideal. Indeed, we note that $J$ is certainly a linear subspace (it is the pre-image of a subspace under a linear map), and for any $a\in T\mf g$ and $b\in J$, we see that $ab\in I$ and
	\[\op{ad}_X(ab)=\op{ad}_X(a)b+a\underbrace{\op{ad}_X(b)}_{\in I}\in I,\]
	so $J$ is closed under multiplication by $T\mf g$. Thus, to check that $J=I$, it is enough to check that $J$ contains the generators of $I$.
	\begin{itemize}
		\item In the case where $A\mf g=S\mf g$, we see that the elements $Y\otimes Z-Z\otimes Y$ generate $I$, so we compute that
		\begin{align*}
			\op{ad}_X(Y\otimes Z-Z\otimes Y) &= ([X,Y]\otimes Z+Y\otimes[X,Z])-([X,Z]\otimes Y-Z\otimes[X,Y]) \\
			&= ([X,Y]\otimes Z-Z\otimes[X,Y])+(Y\otimes[X,Z]-[X,Z]\otimes Y)
		\end{align*}
		lives in $I$.
		\item In the case where $A\mf g=U\mf g$, we see that the elements $Y\otimes Z-Z\otimes Y-[Y,Z]$ generate $I$, so we compute that
		\begin{align*}
			\op{ad}_X(Y\otimes Z-Z\otimes Y-[Y,Z]) &= Y\otimes[X,Z]+[X,Y]\otimes Z-[X,Z]\otimes Y-Z\otimes [X,Y]-[X,[Y,Z]] \\
			&= (Y\otimes[X,Z]-[X,Z]\otimes Y-[Y,[X,Z]]) \\
			&\qquad+([X,Y]\otimes Z-Z\otimes [X,Y]-[[X,Y],Z]),
		\end{align*}
	\end{itemize}
	Now, the checks that $\op{ad}_\bullet\colon\mf g\to\mf{gl}(A\mf g)$ is a Lie algebra morphism satisfying the required properties follow because $\op{ad}_X\colon A\mf g\to A\mf g$ is a quotient of the map $\op{ad}_X\colon T\mf g\to T\mf g$. In particular, all the checks we needed to do amounted to checking some equalities of functions from some tensor power of $T\mf g$ to $T\mf g$, and these maps all quotient appropriately down to $A\mf g$.

	Lastly, we must check that $\op{ad}_X(a)=Xa-aX$ in the case where $A\mf g=U\mf g$. As usual, we note that we are checking the equality of some linear maps $Y\mf g\to U\mf g$, so it is enough to check this on a spanning subset of $U\mf g$, for which we use the pure tensors: we compute
	\begin{align*}
		\op{ad}_X(Y_1\cdots Y_k) &= \sum_{i=1}^k(Y_1\cdots[X,Y_i]\cdots Y_k) \\
		&= \sum_{i=1}^k(Y_1\cdots Y_{i-1}XY_iY{i+1}\cdots Y_k-Y_1\cdots Y_{i-1}Y_iXY{i+1}\cdots Y_k) \\
		&= X(Y_1\cdots Y_k)-(Y_1\cdots Y_k)X,
	\end{align*}
	where the last equality uses the observation that the given sum telescopes.
\end{proof}
% As we had with our Lie algebra, we note that $U\mf g$ can act on itself in some interesting ways. For example, it has the usual left multiplication structures, but there is also an adjoint action.
% \begin{lemma}
% 	The action of $\mf g$ on $T\mf g$ by derivations descends to $U\mf g$.
% \end{lemma}
% \begin{proof}
% 	For any $Z\in\mf g$, we want to check that $\op{ad}_Z(I)\subseteq I$. Well, it is enough to check this on generators of $I$, for which we compute
% 	\begin{align*}
% 		\op{ad}_Z(X\otimes Y-Y\otimes X-[X,Y]) &= ([Z,X]\otimes Y-Y\otimes[Z,X])+(X\otimes[Z,Y]-[Z,Y]\otimes X)-[Z,[X,Y]],
% 	\end{align*}
% 	and we are done after applying the Jacobi identity and rearranging.
% \end{proof}

\end{document}