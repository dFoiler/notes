% !TEX root = ../notes.tex

\documentclass[../notes.tex]{subfiles}

\begin{document}

\section{October 7}
Today we finish classifying the representations of $\mf{sl}_2$.

\subsection{Applications for Representations of \texorpdfstring{$\mf{sl}_2(\CC)$}{sl2(C)}}
% Recall that the Casimir operator $C\coloneqq ef+fe+\frac12h^2$ will commute with the action of $\op{sl}_2$ in any representation. We are using $C$ to show that any representation is sum of the $V_n$s. By an induction, we may assume that we are in the situation where we have an extension
% \[0\to V_n\to W\to V_n^{\oplus(m-1)}\to0,\]
% and we need to check that $W\cong V_n^{\oplus m}$. For this, we note that we can split $W$ as
% \[W=\bigoplus_{\lambda\in\CC}W(\lambda),\]
% where $W(\lambda)$ is the generalized eigenspace for $\lambda$. However, one checks that there is a highest weight $n$ of $W$, implying we can write
% \[W=\bigoplus_{i=0}^nW(n-2i).\]
% Notably, the commutator relations imply that $eW(\lambda)=W(\lambda+2)$, so $W(n)\subseteq\ker e|_W$.

% We complete our classification by showing that $h$ acts diagonally on $\ker e$. Well, for any $u$, we note that $f^mu=0$ for some $m$ large enough (by looking at weights), but the commutator relations find that
% \[e^mf^mu=e^{m-1}f^{m-1}m(h-m+1)u=\cdots=m!h(h-1)\cdots(h-(m+1))u.\]
% Thus, the polynomial $h(h-1)\cdots(h-(m+1))$ acts by $0$ on $\ker e$, so the minimal polynomial of $h$ has no repeated roots, so $h$ acts diagonally.

% To apply the fact that $h$ acts diagonally on $W(n)\subseteq\ker e$, we note that $W(n-2i)$ is simply $W(n)$ shifted over by $f$ some number of times, so we complete.
Let's discuss some applications of \Cref{thm:sl2-classify-irreps,thm:sl2-reduces}. To begin, we can upgrade the diagonal action of \Cref{lem:sl2-h-diagonalizes}.
\begin{corollary} \label{cor:sl2-h-diagonalizes-better}
	Fix any complex representation $\rho\colon\mf{sl}_2(\CC)\to\mf{gl}(V)$. Then $\rho(h)\colon V\to V$ acts diagonally with eigenvalues in $\ZZ$.
\end{corollary}
\begin{proof}
	By \Cref{thm:sl2-reduces}, it is enough to check this for irreducible representations $V$. By \Cref{thm:sl2-classify-irreps}, we see that $V\cong\op{Sym}^nV_{\mathrm{std}}$, where $V_{\mathrm{std}}$ is the standard representation. Then \Cref{lem:sl2-poly-is-sym-power} explains that these can be realized as polynomial representations, from which the required diagonalization of $\rho(h)$ follows from its computation on the monomial basis given in \eqref{eq:sl2-poly-action}.
\end{proof}
\begin{corollary}[Jacobson--Morozov]
	For any nilpotent operator $N\colon V\to V$ on a finite-dimensional complex vector space $V$, there exists a (unique up to isomorphism) structure of $\mf{sl}_2(\CC)$-representation on $V$ such that $e\mapsto N$. More precisely, we have the following. 
	\begin{listalph}
		\item There exists a representation $\rho\colon\mf{sl}_2(\CC)\to\mf{gl}(V)$ such that $\rho(e)=N$.
		\item If $\rho^1,\rho^2\colon\mf{sl}_2(\CC)\to\mf{gl}(V)$ have $\rho^1(e)=\rho^2(e)$, then $\rho^1\cong\rho^2$.
	\end{listalph}
\end{corollary}
\begin{proof}
	We will proceed with the claims separately.
	\begin{listalph}
		\item By giving $V$ a basis, we may identify it with $\CC^d$; using the Jordan normal form, we are able to choose this basis so that $V$ is the direct sum of Jordan blocks of the form
		\[J_n=\underbrace{\begin{bmatrix}
			0 & 1 \\
			  & 0 & 1 \\
			  &   & \ddots & \ddots \\
			  &   &        &        & 0 & 1 \\
			  &   &        &        &   & 0
		\end{bmatrix}}_{n+1}\in \CC^{(n+1)\times(n+1)}.\]
		Decomposing $V=V_1\oplus\cdots\oplus V_m$ so that $N$ decomposes into these Jordan blocks as $N=J_{n_1}\oplus\cdots\oplus J_{n_m}$, we see that we may assume that $N=J_n$ for some $n$: if we can find $\rho_i\colon\mf{sl}_2(\CC)\to\mf{gl}(V_i)$ such that $\rho_i(e)=J_{n_i}$ for each $i$, then $\rho\coloneqq\rho_1\oplus\cdots\oplus\rho_m$ will be a representation $\rho\colon\mf{sl}_2(\CC)\to\mf{gl}(V)$ satisfying $\rho(e)=N$.

		We are thus reduced to the case where $N=J_n$ for some $n\ge0$; note then that $\dim V=n+1$, so we expect to be able to take $\rho=\rho_n$. Let $\{v_0,\ldots,v_n\}$ be the given basis of $V$, which we will adjust to fit $\rho(e)=J_n$. With this in mind, define $\varphi\colon V\to V_n$ by $\varphi(v_q)\coloneqq q!^{-1}x^{n-q}y^q$; this sends a basis to a basis, so $\varphi$ is an isomorphism of vector spaces. Further, we claim that $\varphi\circ N=\rho_n(e)\circ\varphi$: it is enough to check this on the basis $\{v_0,\ldots,v_n\}$, so we use \eqref{eq:sl2-poly-action} to compute $\varphi(Nv_0)=0=\rho_n(e)x^n$ and
		\begin{align*}
			\varphi(Nv_q) &= \varphi(v_{q-1}) \\
			&= (q-1)!^{-1}x^{n-q+1}y^{q-1} \\
			&= \rho_n(e)\left(q!^{-1}x^{n-q}y^q\right) \\
			&= \rho_n(e)\varphi(v_q)
		\end{align*}
		for $q\ge1$. Thus, we may define $\rho\colon\mf{sl}_2(\CC)\to\mf{gl}(V)$ by $\rho(X)\coloneqq\varphi^{-1}\circ\rho_n(X)\circ\varphi$ for all $X\in\mf{sl}_2(\CC)$. Adjusting by conjugating $\varphi$ makes it so that $\rho$ succeeds by a representation, and we checked that $\varphi\circ N=\rho_n(e)\circ\varphi$, so $\rho(e)=N$, which is what we wanted.

		\item We proceed directly. We will read the structure of $\rho^1$ and $\rho^2$ directly off of $N$. Fix some $i\in\{1,2\}$. By using \Cref{thm:sl2-classify-irreps,thm:sl2-reduces}, we may decompose
		\[\rho^i\cong\bigoplus_{n\ge0}\rho_n^{\oplus a_n^i}\]
		for some nonnegative integers $a_{n}^i\ge0$. We will show that $a_{m}^1=a_{m}^2$ for each $m\ge0$, which will complete the proof by comparing the two decompositions.
		
		For this, we use the dimension of $\ker\rho^i(e)^m$ for various $m\ge0$. In particular, \eqref{eq:sl2-poly-action} gives
		\[\dim\ker\rho_n(e)^m=\dim\op{span}\left\{x^n,yx^{n-1},\ldots,x^{n-m+1}y^{m-1}\right\}=\min\{m,n+1\},\]
		so
		\[\dim\ker N^m=\dim\ker\rho^i(e)^m=\sum_{n\ge0}\dim\ker\left((\rho_n(e)^{\oplus a_n^i}\right)^m=\sum_{n\ge0}a_n^i\max\{m,n+1\}.\]
		We now use this to read off the values of $a_n^i$: for any $m\ge1$, we see
		\[\sum_{n=0}^{m-1}a_n^i=\sum_{n\ge0}a_n^i\max\{m+1,n+1\}-\sum_{n\ge0}a_n^i\max\{m,n+1\}=\dim\ker N^{m+1}-\dim\ker N^m,\]
		so
		\[a_m^i=\sum_{n=0}^{m}a_n^i-\sum_{n=0}^{m-1}a_n^i=\dim\ker N^{m+2}-\dim\ker N^m.\]
		Thus, $a_m^i$ is independent of $i$, so $a_m^1=a_m^2$ for all $m\ge0$.
		\qedhere
	\end{listalph}
\end{proof}
\begin{example}
	We may hope that $\rho_1=\rho_2$ on the nose, but this is not true in general. For example, one can use an inner automorphism of $\rho$ fixing $\rho(e)$ to produce an isomorphic representation $\rho'\colon\mf{sl}_2(\CC)\to V$ with $\rho(e)=\rho'(e)$. Concretely, take $V=\CC^2$ and $N\coloneqq\begin{bsmallmatrix}
		0 & 1 \\ 0 & 0
	\end{bsmallmatrix}$. Then we could define $\rho=\rho_{\mathrm{std}}$ and
	\[\rho'(X)\coloneqq\begin{bmatrix}
		1 & 1 \\ 0 & 1
	\end{bmatrix}\rho(X)\begin{bmatrix}
		1 & 1 \\ 0 & 1
	\end{bmatrix}^{-1}\]
	so that $\rho'(e)=\rho(e)$ while $\rho'(h)\ne\rho(h)$.
\end{example}

\subsection{Character Theory of \texorpdfstring{$\mf{sl}_2(\CC)$}{sl2(C)}}
By analogy with the representation theory of finite groups, we may want a notion of characters for representations of $\mf{sl}_2(\CC)$. Our classification allows to do this cleanly.
\begin{definition}
	Fix a finite-dimensional complex representation $\rho\colon\mf{sl}_2(\CC)\to V$. For any $n\in\ZZ$, let $V[n]$ be the eigenspace of $\rho(h)$ with eigenvalue $n$. Then the \textit{character} of $V$ is the rational polynomial
	\[\chi_\rho(T)\coloneqq\sum_{n\in\ZZ}\dim V[n]T^n.\]
	This is a polynomial in $\ZZ\left[T,T^{-1}\right]$ because only finitely many of the $V[n]$ may be nonzero because $\dim V$ is finite. We will write $\chi_V$ for $\chi_\rho$ when no confusion is possible.
\end{definition}
\begin{remark}
	This definition does not lose any information by merely considering the given $V[n]$s because \Cref{cor:sl2-h-diagonalizes-better} tells us that $\rho(h)$ diagonalizes with integral eigenvalues.
\end{remark}
\begin{remark}
	To relate this definition with the characters of finite groups, we claim
	\[\chi_V\left(e^t\right)\stackrel?=\tr\exp(t\rho(h)).\]
	Indeed, using \Cref{cor:sl2-h-diagonalizes-better}, we may write $\rho(h)=\op{diag}(n_1,\ldots,n_d)$ where $\{n_1,\ldots,n_d\}$ are integers. (Technically, we do not need to know that the action is diagonal for the subsequent argument.) Then
	\[\tr\exp(t\rho(h)) = \tr\exp(\op{diag}(tn_1,\ldots,tn_d))=\sum_{i=1}^de^{tn_1}=\sum_{i=1}^d\left(e^t\right)^{n_i}.\]
	Grouping the $n_\bullet$s by multiplicity, we conclude that this equals $\chi_V\left(e^t\right)$: note $\dim V[n]=\#\{i:n_i=n\}$.
\end{remark}
\begin{example} \label{ex:sl2-poly-chars}
	Using \eqref{eq:sl2-poly-action}, we see that $n\ge0$ has
	\[\chi_{V_n}(T)=T^{-n}+T^{-n+2}+\cdots+T^{n-2}+T^n=\frac{T^{n+1}-T^{-n-1}}{T-T^{-1}}.\]
	For example, we see that $\chi_{V_n}(T)$ is the first of the $\chi_n$s with nonzero coefficient on $T^n$, so the collection $\{\chi_n:n\ge0\}$ is $\CC$-linearly independent. Explicitly, any nontrivial expression $\sum_{n\ge0}a_n\chi_{V_n}$ (with $a_n=0$ for all but finitely many $n$) will have some largest $N$ for which $a_N\ne 0$, but then the monomial $a_NT^N$ lives in $\sum_{n\ge0}a_n\chi_{V_n}$, so $\sum_{n\ge0}a_n\chi_{V_n}\ne0$.
\end{example}
Here are some easy checks on our characters.
\begin{lemma} \label{lem:sl2-decompose-char}
	Fix complex representations $\rho_V\colon\mf{sl}_2(\CC)\to\mf{gl}(V)$ and $\rho_W\colon\mf{sl}_2(\CC)\to\mf{gl}(W)$.
	\begin{listalph}
		\item We have $\chi_{V\oplus W}=\chi_V+\chi_W$.
		\item We have $\chi_{V\otimes W}=\chi_V\cdot\chi_W$.
		\item We have $\chi_{V^\lor}(T)=\chi_V\left(T^{-1}\right)$.
	\end{listalph}
\end{lemma}
\begin{proof}
	These checks are purely formal.
	\begin{listalph}
		\item Because $\rho_{V\oplus W}(h)=\rho_V(h)\oplus\rho_W(h)$, we can split up our eigenspaces for $n\in\ZZ$ by
		\[(V\oplus W)[n]=V[n]\oplus W[n],\]
		so
		\begin{align*}
			\chi_{V\oplus W}(T) &= \sum_{n\in\ZZ}\dim(V\oplus W)[n]T^n \\
			&= \sum_{n\in\ZZ}\dim(V[n]\oplus W[n])T^n \\
			&= \sum_{n\in\ZZ}\dim V[n]T^n+\sum_{n\in\ZZ}\dim W[n]T^n \\
			&= \chi_V(T)+\chi_W(T).
		\end{align*}
		\item Let $\{v_1,\ldots,v_k\}$ and $\{w_1,\ldots,w_\ell\}$ be eigenbases for the operators $\rho_V(h)\colon V\to V$ and $\rho_W(h)\colon W\to W$ with eigenvalues $\{\lambda_1,\ldots,\lambda_k\}$ and $\{\mu_1,\ldots,\mu_\ell\}$, respectively. Then $\{v_i\otimes w_j\}$ is a basis for $V\otimes W$, and in fact it is an eigenbasis for $\rho_{V\otimes W}(h)$: note
		\begin{align*}
			\rho_{V\otimes W}(h)(v_i\otimes w_j) &= \rho_V(h)v_i\otimes w_j+v_i\otimes\rho_W(h)w_j \\
			&= \lambda_iv_i\otimes w_j+w_i\otimes\mu_jw_j \\
			&= (\lambda_i+\mu_j)(v_i\otimes w_j).
		\end{align*}
		Thus, for any $z\in\CC$, we see that
		\[\dim(V\otimes W)[n]=\#\{(i,j):\lambda_i+\mu_j=n\},\]
		so
		\begin{align*}
			\chi_{V\otimes W}(T) &= \sum_{n\in\ZZ}\dim(V\otimes W)[n]T^n \\
			&= \sum_{n\in\ZZ}\#\{(i,j):\lambda_i+\mu_j=n\}T^n \\
			&= \sum_{n\in\ZZ}\sum_{a+b=z}(\dim V[a]\dim W[b])T^n \\
			&= \sum_{a,b\in\ZZ}(\dim V[a]\dim W[b])T^{a+b} \\
			&= \Bigg(\sum_{a\in\ZZ}\dim V[a]T^a\Bigg)\Bigg(\sum_{b\in\ZZ}\dim V[b]T^b\Bigg) \\
			&= \chi_V(T)\chi_W(T),
		\end{align*}
		\item As in (b), let $\{v_1,\ldots,v_k\}$ be an eigenbasis for the operator $\rho_V(h)\colon V\to V$ with eigenvalues $\{\lambda_1,\ldots,\lambda_k\}$. Then we claim that the dual basis $\left\{v_1^\lor,\ldots,v_k^\lor\right\}$ is an eigenbasis for $\rho_{V^\lor}(h)$: for any $v_i^\lor$ and $v_j$, we see
		\begin{align*}
			\left(\rho_{V^\lor}(h)v_i^\lor\right)(v_j) &= -v_i^\lor\left(\rho_V(h)v_j\right) \\
			&= -\lambda_jv_i^\lor(v_j) \\
			&= -\lambda_j1_{i=j},
		\end{align*}
		so $\rho_{V^\lor}(h)v_i^\lor=-\lambda_iv_i^\lor$. Thus, gathering multiplicities, we see that $\dim V[n]=\dim V^\lor[-n]$ for any $n\in\ZZ$, so
		\begin{align*}
			\chi_{V^\lor}(T)=\sum_{n\in\ZZ}\dim V^\lor[n]T^n \\
			&= \sum_{n\in\ZZ}\dim V[-n]T^n \\
			&= \sum_{n\in\ZZ}\dim V[n]\left(T^{-1}\right)^n \\
			&= \chi_V\left(T^{-1}\right),
		\end{align*}
		as required.
		\qedhere
	\end{listalph}
\end{proof}
Importantly, we can use characters to determine representations.
\begin{proposition} \label{prop:sl2-char-determines}
	Fix complex representations $\rho_V\colon\mf{sl}_2(\CC)\to\mf{gl}(V)$ and $\rho_W\colon\mf{sl}_2(\CC)\to\mf{gl}(W)$. If $\chi_V=\chi_W$, then $V\cong W$.
\end{proposition}
\begin{proof}
	By \Cref{thm:sl2-classify-irreps,thm:sl2-reduces}, we have decompositions
	\[V\cong\bigoplus_{n\ge0}V_n^{\oplus a_n}\qquad\text{and}\qquad W\cong\bigoplus_{n\ge0}V_n^{\oplus b_n}.\]
	We will show that $a_n=b_n$ for all $n$, which will complete the proof upon comparing the decompositions. Well, \Cref{lem:sl2-decompose-char} tells us that
	\begin{align*}
		0 &= \chi_V(T)-\chi_W(T) \\
		&= \sum_{n\ge0}a_n\chi_{V_n}(T)-\sum_{n\ge0}b_n\chi_{V_n}(T) \\
		&= \sum_{n\ge0}(a_n-b_n)\chi_{V_n}(T).
	\end{align*}
	This relation is enough to imply $a_n=b_n$ for all $n$ by the linear independence given in \Cref{ex:sl2-poly-chars}.
\end{proof}
\begin{example} \label{ex:sl2-dual-reps}
	We claim that $V\cong V^\lor$ for any complex representation $\rho_V\colon\mf{sl}_2(\CC)\to\mf{gl}(V)$. By \Cref{prop:sl2-char-determines}, we may check this on characters. Using the complete reducibility of \Cref{thm:sl2-reduces} with \Cref{lem:sl2-decompose-char}, it is enough to check this for irreducible $V$ (notably, $(V\oplus W)^\lor\cong V^\lor\oplus W^\lor$). Thus, \Cref{thm:sl2-classify-irreps} lets us assume that $V=V_n$ for some $n\ge0$, so we are left to show that
	\[\chi_{V_n}(T)\stackrel?=\chi_{V_n}\left(T^{-1}\right)\]
	by \Cref{lem:sl2-decompose-char}. This is true by the explicit computation of \Cref{ex:sl2-poly-chars}.
\end{example}
\begin{example} \label{ex:sl2-tensor-example}
	We claim that $V_2\otimes V_3\cong V_1\oplus V_3\oplus V_5$. By \Cref{prop:sl2-char-determines}, it is enough to show an equality of characters. For this, we use \Cref{ex:sl2-poly-chars} with \Cref{lem:sl2-decompose-char} to see
	\begin{align*}
		\chi_{V_2\otimes V_3}(T) &= \left(T^{-2}+1+T^2\right)\left(T^{-3}+T^{-1}+T+T^3\right) \\
		&= T^{-5}+2T^{-3}+3T^{-1}+3T+2T^3+T^5 \\
		&= \chi_{V_5}(T)+\chi_{V_3}(T)+\chi_{V_1}(T) \\
		&= \chi_{V_1\oplus V_3\oplus V_5}(T).
	\end{align*}
\end{example}
Here is the general case of \Cref{ex:sl2-tensor-example}.
% One can check that $\chi_{V\oplus W}=\chi_V+\chi_W$ and $\chi_{V\otimes W}=\chi_V\chi_W$ by a direct computation with the eigenspaces. One can also compute that
% \[\chi_n(z)\coloneqq\chi_{V_n}(z)=z^n+z^{n-2}+\cdots+z^{2-n}+z^{-n}=\frac{z^{n+1}-z^{-n-1}}{z-z^{-1}}.\]
% Thus, we see that these characters are linearly independent (say, over $\QQ$), so we can recover any representation by writing it as a sum of the characters $\chi_{V_n}$. (One can algorithmically remove highest-order terms in order to get our decomposition.)
% This character theory allows us to prove the following result.
\begin{proposition}[Clebsch--Gordan rule]
	Let $V_1$ be the standard representation $\rho_1\colon\mf{sl}_2(\CC)\subseteq\mf{gl}_2(\CC)$, and define $V_n\coloneqq\op{Sym}^nV_1$ for each $n\ge0$ so that we have representations $\rho_n\colon\mf{sl}_2(\CC)\to\mf{gl}(V_n)$. Then
	\[V_m\otimes V_n\cong\bigoplus_{i=0}^{\min\{m,n\}}V_{\left|m-n\right|+2i}.\]
\end{proposition}
\begin{proof}
	By symmetry, we may assume that $m\le n$. We generalize the argument of \Cref{ex:sl2-tensor-example}. By \Cref{prop:sl2-char-determines}, it is enough to compare the characters of both sides, for which we use \Cref{lem:sl2-decompose-char} with \Cref{ex:sl2-poly-chars}. Now, we compute
	\begin{align*}
		\chi_{V_m\otimes V_n}(T) &= \chi_{V_m}(T)\chi_{V_n}(T) \\
		&= \frac{T^{m+1}-T^{-m-1}}{T-T^{-1}}\cdot\frac{T^{n+1}-T^{-n-1}}{T-T^{-1}}, \\
		\chi_{\bigoplus_{i=0}^{m}V_{n-m+2i}}(T) &= \sum_{i=0}^m\chi_{V_{n-m+2i}}(T) \\
		&= \sum_{i=0}^m\frac{T^{n-m+2i+1}-T^{-(n-m-2i+1)}}{T-T^{-1}}.
	\end{align*}
	Thus, it remains to show the combinatorial identity
	\[\frac{T^{m+1}-T^{-m-1}}{T-T^{-1}}\cdot\frac{T^{n+1}-T^{-n-1}}{T-T^{-1}}\stackrel?=\sum_{i=0}^m\frac{T^{n-m+2i+1}-T^{-(n-m+2i+1)}}{T-T^{-1}}.\]
	Multiplying both sides by $T-T^{-1}$, we would like to show that
	\[\frac{\left(T^{m+1}-T^{-m-1}\right)\left(T^{n+1}-T^{-n-1}\right)}{T-T^{-1}}\stackrel?=\sum_{i=0}^m\left(T^{n-m+2i+1}-T^{-(n-m+2i+1)}\right).\]
	Well,
	\[\sum_{i=0}^mT^{\pm\left(n-m+2i+1\right)}=\frac{T^{\pm(n+m+2+1)}-T^{\pm(n-m+1)}}{T^{\pm2}-1},\]
	so
	\begin{align*}
		\sum_{i=0}^m\left(T^{n-m+2i+1}-T^{-(n-m+2i+1)}\right) &= \frac{T^{(n+m+2+1)}-T^{(n-m+1)}}{T^{2}-1}-\frac{T^{-(n+m+2+1)}-T^{-(n-m+1)}}{T^{-2}-1} \\
		&= \frac{T^{n+m+2}-T^{n-m}}{T-T^{-1}}-\frac{T^{-n-m-2}-T^{-n+m}}{T^{-1}-T} \\
		&= \frac{T^{n+m+2}-T^{n-m}-T^{-n+m}+T^{-n-m-2}}{T-T^{-1}} \\
		&= \frac{\left(T^{m+1}-T^{-m-1}\right)\left(T^{n+1}-T^{-n-1}\right)}{T-T^{-1}},
	\end{align*}
	as desired.
\end{proof}
% \begin{example}
% 	One can compute directly that
% 	\[\chi_2\chi_3=\chi_5+\chi_3+\chi_1.\]
% \end{example}
% \begin{example}
% 	One finds that $\chi_n^2=\chi_0+\cdots+\chi_{2n}$.
% \end{example}

\subsection{Dual Representations of \texorpdfstring{$\mf{sl}_2(\CC)$}{sl2(C)}}
\Cref{ex:sl2-dual-reps} showed that $V_n\cong V_n^\lor$ for each $n\ge0$. We would like to be able to provide an explicit such map. Well, recalling \Cref{lem:dualize-inner-prod-rep}, we see that we would like to give an invariant bilinear inner product. Here is the result.
\begin{proposition}
	Let $V_1$ be the standard representation $\rho_1\colon\mf{sl}_2(\CC)\subseteq\mf{gl}_2(\CC)$, and define $V_n\coloneqq\op{Sym}^nV_1$ for each $n\ge0$ so that we have representations $\rho_n\colon\mf{sl}_2(\CC)\to\mf{gl}(V_n)$. Then $V_n$ admits an $\mathfrak{sl}_2(\CC)$-invariant inner product $\langle-,-\rangle$ which is symmetric when $n$ is even and skew-symmetric when $n$ is odd.
\end{proposition}
\begin{proof}
	We begin by defining the bilinear form. Set $V\coloneqq V_{\mathrm{std}}$ for brevity. Now, for any $n\ge0$, we define $\langle-,-\rangle\colon V_n\times V_n\to\CC$ on by
	\[\big\langle(v_1,\ldots,v_n),(w_1,\ldots,w_n)\big\rangle\coloneqq\prod_{i=1}^n\det\begin{bmatrix}
		| & | \\
		v_i & w_i \\
		| & |
	\end{bmatrix}.\]
	Because determinants (and products of determinants) are multilinear, we see that this produces a multilinear map $V^{\otimes n}\otimes V^{\otimes n}\to\CC$. Thus, we have produced a bilinear form. We now run checks in sequence.
	\begin{enumerate}
		\item Note that this bilinear form on $V^{\otimes n}$ is $\op{SL}_2(\CC)$-invariant, which implies that it will be $\mf{sl}_2(\CC)$-invariant upon passing to the differential representation by \Cref{lem:descend-invariant-inner-prod}. Well, for any $g\in\op{SL}_2(\CC)$, it is enough to check the invariance on pure tensors by the ambient bilinearity, so we compute
		\begin{align*}
			& \big\langle g(v_1\otimes\cdots\otimes v_n),g(w_1\otimes\cdots\otimes w_n)\big\rangle \\
			={}& \big\langle (gv_1\otimes\cdots\otimes gv_n),(gw_1\otimes\cdots\otimes gw_n)\big\rangle \\
			={}& \prod_{i=1}^n\det\begin{bmatrix}
				| & | \\
				gv_i & gw_i \\
				| & |
			\end{bmatrix} \\
			={}& \prod_{i=1}^n\det g\begin{bmatrix}
				| & | \\
				v_i & w_i \\
				| & |
			\end{bmatrix} \\
			={}& (\underbrace{\det g}_1)^n\prod_{i=1}^n\begin{bmatrix}
				| & | \\
				v_i & w_i \\
				| & |
			\end{bmatrix} \\
			={}& \big\langle g(v_1\otimes\cdots\otimes v_n),g(w_1\otimes\cdots\otimes w_n)\big\rangle \\
			={}& \big\langle (v_1\otimes\cdots\otimes v_n),(w_1\otimes\cdots\otimes w_n)\big\rangle,
		\end{align*}
		as required.
		\item We now restrict $\langle-,-\rangle$ to the subrepresentation $\op{Sym}^nV\subseteq V^{\otimes n}$. As such, we see that
		\begin{align*}
			\langle v_1\cdots v_n,w_1\cdots w_n\rangle &= \frac1{(n!)^2}\sum_{\sigma,\tau\in S_n}\prod_{i=1}^n\det\begin{bmatrix}
				| & | \\
				v_{\sigma i} & w_{\tau i} \\
				| & |
			\end{bmatrix} \\
			&= \frac1{n!}\sum_{\sigma\in S_n}\prod_{i=1}^n\det\begin{bmatrix}
				| & | \\
				v_{\sigma i} & w_{i} \\
				| & |
			\end{bmatrix}
		\end{align*}
		by rearranging. Before doing the non-degeneracy check, we verify that $\langle-,-\rangle$ is symmetric when $n$ is even and skew-symmetric when $n$ is odd. This condition is linear in all the coordinates, so it is enough to check the (skew-)symmetry on the spanning set of tensors of the form $v_1\cdots v_n$. Well,
		\begin{align*}
			\langle v_1\cdots v_n,w_1\cdots w_n\rangle &= \frac1{n!}\sum_{\sigma\in S_n}\prod_{i=1}^n\det\begin{bmatrix}
				| & | \\
				v_{\sigma i} & w_{i} \\
				| & |
			\end{bmatrix} \\
			&= \frac1{n!}\sum_{\sigma\in S_n}\prod_{i=1}^n-\det\begin{bmatrix}
				| & | \\
				w_{\sigma i} & v_{i} \\
				| & |
			\end{bmatrix} \\
			&= (-1)^n\frac1{n!}\sum_{\sigma\in S_n}\prod_{i=1}^n\det\begin{bmatrix}
				| & | \\
				w_{i} & v_{\sigma i} \\
				| & |
			\end{bmatrix} \\
			&= (-1)^n\langle v_1\cdots v_n,w_1\cdots w_n\rangle,
		\end{align*}
		as required.
		\item We now check that $\langle-,-\rangle$ is non-degenerate. We will do this by computing it on a basis. Let $V$ have basis $\{v_x,v_y\}$, and then we note that $\op{Sym}^nV$ has basis $\{v_x^pv_y^q\}_{p+q=n}$. For bookkeeping reasons, for each pair $(p,q)$, define the function $v_{pq}\colon\{1,\ldots,n\}\to V$ by
		\[v_{pq}(i)\coloneqq\begin{cases}
			v_x & \text{if }i \le p, \\
			v_y & \text{if }i > p.
		\end{cases}\]
		In particular, $v_x^pv_y^q=v_{pq}(1)\cdots v_{pq}(n)$. We now choose two basis vectors $v_x^pv_y^q$ and $v_x^{p'}v_y^{q'}$ and compute
		\begin{align*}
			\left\langle v_x^pv_y^q,v_x^{p'}v_y^{q'}\right\rangle &= \left\langle v_{pq}(1)\cdots v_{pq}(n),v_{p'q'}(1)\cdots v_{p'q'}(n)\right\rangle \\
			&= \frac1{n!}\sum_{\sigma\in S_n}\prod_{i=1}^n\det\begin{bmatrix}
				| & | \\
				v_{pq}(\sigma i) & v_{p'q'}(i) \\
				| & |
			\end{bmatrix} \\
			&= \frac1{n!}\sum_{\sigma\in S_n}\prod_{i=1}^{p'}\det\begin{bmatrix}
				| & | \\
				v_{pq}(\sigma i) & v_x \\
				| & |
			\end{bmatrix}\prod_{i=p'+1}^{n}\det\begin{bmatrix}
				| & | \\
				v_{pq}(\sigma i) & v_y \\
				| & |
			\end{bmatrix} \\
		\end{align*}
		Now, the only way for a product of these determinants to not vanish is to have $v_{pq}(\sigma i)=v_y$ for $i\le p'$ and $v_{pq}(\sigma i)=v_x$ for $i>p'$. In particular, by counting the number of $v_x$s and $v_y$s, we see that all terms of the sum vanish unless $(p,q)=(q',p')$. In the case where $(p,q)=(q',p')$, we have
		\begin{align*}
			\left\langle v_x^pv_y^q,v_x^qv_y^p\right\rangle &= \frac1{n!}\sum_{\sigma\in S_n}\prod_{i=1}^{q}\det\begin{bmatrix}
				| & | \\
				v_{pq}(\sigma i) & v_x \\
				| & |
			\end{bmatrix}\prod_{i=q+1}^{n}\det\begin{bmatrix}
				| & | \\
				v_{pq}(\sigma i) & v_y \\
				| & |
			\end{bmatrix},
		\end{align*}
		so we see that each nonzero term will evaluate to $(-1)^q$, and the number of nonzero terms is the number of $\sigma\in S_n$ such that $\sigma$ carries $\{1,\ldots,q\}$ to $\{p+1,\ldots,n\}$. There are $p!q!$ total such permutations, so we conclude that
		\[\left\langle v_x^pv_y^q,v_x^qv_y^p\right\rangle=(-1)^q\frac{p!q!}{n!}\ne0.\]
		Thus, we see that the matrix given by the bilinear form on the basis $\left\{v_x^pv_y^q\right\}_{p+q=n}$ is anti-diagonal with all nonzero anti-diagonal entries, so it is invertible. Thus, the relevant bilinear form is non-degen\-erate.
		\qedhere
	\end{enumerate}
\end{proof}

\subsection{The Universal Enveloping Algebra}
We have shown that the category $\op{Rep}_\CC\mf g$ is abelian for any Lie algebra $\mf g$, so we may expect to be able to realize this category as a category of modules over some (possibly non-commutative) ring. This is the role of the universal enveloping algebra.

To start, we would have the universal algebra
\[T\mf g\coloneqq\bigoplus_{k=0}^\infty\mf g^{\otimes k},\]
which is some graded ring, where the multiplication $\mf g^{\otimes k}\otimes\mf g^{\otimes\ell}\to\mf g^{\otimes(k+\ell)}$ is given by concatenation. We now take a quotient to remember the Lie bracket.
\begin{definition}[universal enveloping algebra]
	We define the \textit{universal enveloping algebra} $U\mf g$ as the quotient of $T\mf g$ by the two-sided ideal $I\mf g$ generated by the elements
	\[x\otimes y-y\otimes x-[x,y].\]
\end{definition}
Here are some properties of this definition.
\begin{proposition}
	Fix a Lie algebra $\mf g$.
	\begin{listalph}
		\item For any associative algebra $A$, we have
		\[\op{Hom}_{\mathrm{LieAlg}}(\mf g,A)\simeq\op{Hom}_{\mathrm{Alg}}(U\mf g,A).\]
		\item The category $\op{Rep}_k\mf g$ is equivalent to the category $\op{Mod}(U\mf g)$.
	\end{listalph}
\end{proposition}
In order to actually compute $U\mf g$, one can fix a basis $\{x_1,\ldots,x_n\}$ of $\mf g$ and then note that $T\mf g$ will be the free polynomial ring in these variables, so we can take a quotient to recover $U\mf g$.
\begin{example}
	If $\mf g$ is abelian, then we see that $U\mf g=k[x_1,\ldots,x_n]$.
\end{example}
\begin{example}
	We see that
	\[U(\mf{sl}_2)=\frac{\CC\langle e,f,h\rangle}{(ef-fe-h,he-eh-2e,hf-fh+2f)}.\]
\end{example}
As we had with our Lie algebra, we note that $U\mf g$ can act on itself in some interesting ways. For example, it has the usual left multiplication structures, but there is also an adjoint action.
\begin{lemma}
	The action of $\mf g$ on $T\mf g$ by derivations descends to $U\mf g$.
\end{lemma}
\begin{proof}
	For any $Z\in\mf g$, we want to check that $\op{ad}_Z(I)\subseteq I$. Well, it is enough to check this on generators of $I$, for which we compute
	\begin{align*}
		\op{ad}_Z(X\otimes Y-Y\otimes X-[X,Y]) &= ([Z,X]\otimes Y-Y\otimes[Z,X])+(X\otimes[Z,Y]-[Z,Y]\otimes X)-[Z,[X,Y]],
	\end{align*}
	and we are done after applying the Jacobi identity and rearranging.
\end{proof}

\end{document}