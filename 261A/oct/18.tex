% !TEX root = ../notes.tex

\documentclass[../notes.tex]{subfiles}

\begin{document}

\section{October 18}
Today we prove the Cartan criteria.

\subsection{Cartan Criteria}
We now define a special invariant form.
\begin{definition}[Killing form]
	Fix a Lie algebra $\mf g$. Then the \textit{Killing form} is the invariant form
	\[B_{\mf g}(X,Y)\coloneqq\tr({\op{ad}_X}\circ{\op{ad}_Y}).\]
	We will write $K_{\mf g}$ for this form or simply $K$ if no confusion can arise.
\end{definition}
We will prove two theorems about this.
\begin{theorem}[Cartan criterion of solvability] \label{thm:cartan-solvable}
	Fix a Lie algebra $\mf g$ over a field $F$ of characteristic $0$. Then $\mf g$ is solvable if and only if $[\mf g,\mf g]\subseteq\ker K$.
\end{theorem}
\begin{theorem}[Cartan criterion of semisimplicity] \label{thm:cartain-semisimple}
	Fix a Lie algebra $\mf g$ over a field $F$ of characteristic $0$. Then $\mf g$ is semisimple if and only if $K$ is non-degenerate.
\end{theorem}
We begin with a lemma.
\begin{lemma} \label{lem:get-cartan-solvability}
	Fix an algebraically closed field $F$ of characteristic $0$, and choose some Lie subalgebra $\mf g\subseteq\mf{gl}_n(F)$. Suppose that all $X\in[\mf g,\mf g]$ and $Y\in\mf g$ have
	\[\tr XY=0.\]
	Then $\mf g$ is solvable.
\end{lemma}
\begin{proof}
	To show $\mf g$ is solvable, it is enough to show that $[\mf g,\mf g]$ is nilpotent, so we may hope to show that all the eigenvalues of any element of $[\mf g,\mf g]$ has vanishing eigenvalues. (We are using Engel's theorem.)
	
	Well, pick up some $X\in[\mf g,\mf g]$, and we let $\{\lambda_1,\ldots,\lambda_n\}$ be the eigenvalues of $X$ counted with multiplicity. We will show that
	\[\Lambda\coloneqq\op{span}_\QQ\{\lambda_1,\ldots,\lambda_n\}\]
	is the $0$ vector space, for which we will show that all of its linear functionals $\varphi\colon\Lambda\to\QQ$ vanish. Well, we extend $\varphi$ to an operator on $V$ by acting on each generalized eigenspace $V[\lambda_i]$ by $\varphi(\lambda_i)$. Then we can compute $\op{ad}\varphi$ is diagonalizable with eigenvalues $\varphi(\lambda_i-\lambda_j)$; similarly, $\op{ad}X_s$ is diagonalizable with eigenvalues $\lambda_i-\lambda_j$. As such, we can choose a polynomial $Q(T)\in F[T]$ such that $Q(\lambda_i-\lambda_j)=\varphi(\lambda_i-\lambda_j)$ for each $\lambda_i-\lambda_j$. In particular, we see that
	\[\op{ad}b=Q(\op{ad}X_s).\]
	Similarly, we know that there is a polynomial $P$ such that $\op{ad}X_s=P(\op{ad}X)$. Thus, we see that
	\[\op{ad}b=(Q(P(\op{ad}X))).\]
	As an example computation, note that $P(0)=Q(0)=0$ because $0$ is an eigenvalue. Now, $X\in[\mf g,\mf g]$ can be written as $\sum_{i=1}^m[Y_i,Z_i]$, so on one hand,
	\[\tr(\varphi\circ X)=\sum_{i=1}^m\varphi(\lambda_i)n_i,\]
	where $n_i=\dim V[\lambda_i]$. On the other hand,
	\[\tr\Bigg(\varphi\circ\sum_{i=1}^m[Y_i,Z_i]\Bigg)=\tr\Bigg(\sum_{i=1}^m\op{ad}_\varphi(Y_i)Z_i\Bigg),\]
	which we see must vanish because $R(0)=0$. Thus, we get the condition
	\[\sum_{i=1}^mn_i\varphi(\lambda_i)\lambda_i=0,\]
	so applying $\varphi$ again allows us to conclude that $\varphi=0$.
\end{proof}
We are now ready to prove \Cref{thm:cartan-solvable}.
\begin{proof}[Proof of \Cref{thm:cartan-solvable}]
	There are two implications.
	\begin{itemize}
		\item If $\mf g$ is solvable, then we must show that $[\mf g,\mf g]$ lives in the kernel of $K$. Well, for all $X\in\mf g$, we note that solvability implies that $\op{ad}_X$ is strictly upper-triangular, so $[\mf g,\mf g]$ continues to be strictly upper-triangular (by Lie's theorem), so $K(x,y)=0$ whenever $X\in[\mf g,\mf g]$.
		\item Suppose that $[\mf g,\mf g]$ lives in the kernel of $K$. Then $\im\op{ad}$ inside $\mf{gl}(\mf g)$ is solvable by the above lemma, so $\mf g/\mf z(\mf g)\cong\im\op{ad}$ is solvable, but $\mf z(\mf g)\subseteq\mf g$ is a solvable ideal, so we conclude.
		\qedhere
	\end{itemize}
\end{proof}
And here is the proof of \Cref{thm:cartain-semisimple}.
\begin{proof}[Proof of \Cref{thm:cartain-semisimple}]
	There are two implications.
	\begin{itemize}
		\item If $\mf g$ is semisimple, then the kernel of $K$ is an ideal $I$ of $\mf g$. However, $K|_I=K_{\mf g}|_I$ will vanish,\footnote{This property holds for general ideals.} so $I$ is solvable as discussed, so $I=0$, so $K_{\mf g}$ is non-degenerate.
		\item If $K$ is non-degenerate, then $\mf g$ is reductive by a result from last class. However, $\op{rad}(\mf g)=\mf z(\mf g)$ would be contained in the kernel of $K$, so we see that $\mf z(\mf g)=0$, so $\mf g$ is in fact semisimple.
		\qedhere
	\end{itemize}
\end{proof}
\begin{corollary}
	Fix a semisimple Lie algebra $\mf g$ over a field $F$ of characteristic $0$. Then $\mf g$ is semisimple if and only if $\mf g\otimes_F\ov F$ is semisimple.
\end{corollary}
\begin{proof}
	The Killing form is stable under field extension, so this is immediate from \Cref{thm:cartain-semisimple}.
\end{proof}
% \begin{remark}
% 	It is false that simple is preserved by base change. For example, a complex Lie algebra $\mf g$ considered as a real Lie algebra can succeed at being simple, but when you base-change it back to $\CC$ it will split into two pieces $\mf g\oplus\overline{\mf g}$.
% \end{remark}
\begin{remark}
	It is false that being simple is preserved by restriction: any simple Lie algebra $\mf g$ over $\CC$ has $\mf g|_{\RR}$ split into two Lie algebras (given by the ``realification'' Lie ideal, defined by the fixed points of the conjugation action).
\end{remark}
% \begin{remark}
% 	However, it is true that 
% \end{remark}
\begin{corollary}
	Fix a semisimple Lie algebra $\mf g$ over a field $F$ of characteristic $0$. For any ideal $I\subseteq\mf g$, there exists an ideal $J\subseteq\mf g$ such that $\mf g=I\oplus J$.
\end{corollary}
\begin{proof}
	Let $J\coloneqq I^\perp$ be the orthogonal complement of $I$ with respect to the Killing form $K$ on $\mf g$. One can check that $J$ is an ideal, and $I\cap J=0$ because $K$ is non-degenerate (namely, $K$ vanishes on $I\cap J$, so $I\cap J$ is solvable, so $I\cap J=0$ because $\mf g$ is semisimple).
\end{proof}
\begin{corollary}
	Fix a semisimple Lie algebra $\mf g$ over a field $F$ of characteristic $0$. Then $\mf g$ is a direct sum of simple Lie algebras.
\end{corollary}
\begin{proof}
	Induct with the previous corollary. Namely, if $\mf g$ fails to be simple, we can decompose it into two smaller pieces.
\end{proof}
Here is a more powerful version of the above result.
\begin{proposition}
	Fix a semisimple Lie algebra $\mf g$ over a field $F$ of characteristic $0$, and write $\mf g=\mf g_1\oplus\cdots\oplus\mf g_k$ as a sum of simple Lie algebras. Then any ideal $I\subseteq\mf g$ is of the form
	\[\bigoplus_{i\in S}\mf g_i,\]
	where $S\subseteq\{1,\ldots,k\}$ is some subset.
\end{proposition}
\begin{proof}
	Induct on $k$. If $k\in\{0,1\}$, there is nothing to do. For the induction, write $\mf g=\mf h\oplus\mf g_{k+1}$. Then consider the projection $\pi_k\colon\mf g\to\mf g_{k+1}$. There are two cases.
	\begin{itemize}
		\item If $\pi_k(I)=0$, then $I\subseteq\mf h$, so we are done by the inductive hypothesis.
		\item If $\pi_k(I)=\mf g_k$, then we note that $[\mf g_{k+1},I]=\mf g_{k+1}$ because $\mf g_{k+1}$ is simple, so $I=I'\oplus\mf g_{k+1}$ for some other ideal $I'$, for which we again use the inductive hypothesis.
		\qedhere
	\end{itemize} 
\end{proof}
\begin{corollary}
	Any ideal in a semisimple Lie algebra is semisimple. Any quotient of a semisimple Lie algebra is semisimple.
\end{corollary}

\end{document}