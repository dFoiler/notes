% !TEX root = ../notes.tex

\documentclass[../notes.tex]{subfiles}

\begin{document}

\section{September 16}
Today we will talk about Lie algebras of classical groups.

\subsection{The Exponential Map: The Classical Case}
Let's work through our examples by hand. Recall that our classical groups are our subgroups of $\op{GL}_n(\FF)$ cut out by equations involving $\det$ and preserving a bilinear/sesquilinear form (symmetric, symplectic, or Hermitian).
\begin{example}
	We show that $\op{GL}_n(\FF)$ is a Lie group over $\FF$ and compute its Lie algebra.
\end{example}
\begin{proof}
	Note $\op{GL}_n(\FF)$ is an open submanifold of $\op M_n(\FF)\cong\FF^{n\times n}$. Matrix multiplication and inversion are rational functions of the coordinates and hence smooth, so $\op{GL}_n(\FF)$ succeeds at being a Lie group. Lastly, we see that being open implies that our tangent space is
	\[T_e\op{GL}_n(\FF)\cong T_e\op M_n(\FF)\cong\FF^n,\]
	as required.
\end{proof}
We will postpone the remaining computations until we discuss the exponential.
For these computations, we want the exponential map.
\begin{definition}[exponential]
	For $X\in\mf{gl}_n(\FF)$, we define the exponential map $\exp\colon\mf{gl}_n(\FF)\to\op{GL}_n(\FF)$ by
	\[\exp(X)\coloneqq\sum_{k=0}^\infty\frac{X^k}{k!}.\]
	Note that $\exp$ is an isomorphism at the identity, so the Inverse function theorem provides a smooth ``local'' inverse $\log(1_n+X)$ defined in an open neighborhood of $1_n$. In fact, one can formally compute that
	\[\log(1_n+X)=\sum_{k=1}^\infty(-1)^{k+1}\frac{X^k}k.\]
\end{definition}
We run a few small checks.
\begin{remark}
	Note $\exp(0)=1$. In fact, one can check that $d\exp_0(A)=A$ for any $A\in\mf{gl}_n(\FF)$ by taking the derivative term by term.
\end{remark}
\begin{remark} \label{rem:manipulate-in-exp}
	We also see that $\exp\left(AXA^{-1}\right)=A\exp(X)A^{-1}$ and $\exp(X^\intercal)=\exp(X)^\intercal$ and $\exp(X^\dagger)=\exp(X)^\dagger$ by a direct expansion.
\end{remark}
What's important about $\exp$ is the following multiplicative property.
\begin{lemma}
	Fix $X,Y\in\mf{gl}_n(\FF)$ which commute. Then
	\[\exp(X+Y)=\exp(X)\exp(Y).\]
\end{lemma}
\begin{proof}
	We check this in formal power series. Because everything in sight converges, this is safe. The main point is to just expand everything. Indeed,
	\begin{align*}
		\exp(X+Y) &= \sum_{k=0}^\infty\frac{(X+Y)^k}{k!} \\
		&= \sum_{k=0}^\infty\frac1{k!}\Bigg(\sum_{a+b=k}\binom kaX^aY^b\Bigg) \\
		&= \sum_{a,b=0}^\infty\frac1{(a+b)!}\cdot\frac{(a+b)!}{a!b!}X^aY^b \\
		&= \sum_{a,b=0}^\infty\frac{X^a}{a!}\cdot\frac{Y^b}{b!} \\
		&= \exp(X)\exp(Y),
	\end{align*}
	as required.
\end{proof}
\begin{remark}
	For fixed $X$, the previous point implies that the map $\FF\to\op{GL}_n(\FF)$ given by $t\mapsto\exp(tX)$ is a Lie group homomorphism. (Smoothness is automatic by smoothness of $\exp$.) The image of this map is called the ``$1$-parameter subgroup'' generated by $X$.
\end{remark}
\begin{remark}
	Taking inverses shows that $\log(XY)=\log X+\log Y$ for $X$ and $Y$ close enough to the identity.
\end{remark}
Here is another check which is a little more interesting.
\begin{lemma} \label{lem:det-exp-exp-tr}
	Fix $X\in\mf{gl}_n(\FF)$. Then
	\[\det\exp(X)=\exp(\tr X).\]
\end{lemma}
\begin{proof}
	The computations do not change if we extend the base field, so we may work over $\CC$ everywhere. Thus, we may assume that $X$ is upper-triangular by conjugating (see \Cref{rem:manipulate-in-exp}) say with diagonal entries $\{d_1,\ldots,d_n\}$. Now, for any $k\ge0$, any $X^k$ continues to be upper-triangular with diagonal entries $\left\{d_1^k,\ldots,d_n^k\right\}$. Thus, we see that $\exp(X)$ is upper-triangular with diagonal entries $\left\{\exp(d_1),\ldots,\exp(d_n)\right\}$, so
	\begin{align}
		\det\exp(X) &= \exp(d_1)\cdots\exp(d_n) \\
		&= \exp(d_1+\cdots+d_n) \\
		&= \exp(\tr X),
	\end{align}
	as required.
\end{proof}

\subsection{The Classical Groups}
For our classical groups, we will show the following result.
\begin{theorem}
	For each classical group $G\subseteq\op{GL}_n(\FF)$, there will exist a vector subspace $\mf g\subseteq\mf{gl}_n(\FF)$ (which can be identified with $T_eG$ via the embedding $G\subseteq\op{GL}_n(\FF)$) and open neighborhoods of the identity $U\subseteq\op{GL}_n(\FF)$ and $\mf u\subseteq\mf g$ such that $\exp\colon(U\cap G)\to(\mf u\cap\mf g)$ is a local isomorphism.
\end{theorem}
Before engaging with the examples, we note the following corollary.
\begin{corollary}
	For each classical group $G$, we see that $G$ is a Lie group with $T_eG=\mf g$ and $\dim G=\dim\mf g$.
\end{corollary}
\begin{proof}
	It suffices to provide a slice chart of the identity for $G\subseteq\op{GL}_n(\FF)$; we then get slice charts everywhere by translation. Well, 
\end{proof}
Let's now proceed with our examples.
We begin with some general remarks.
\begin{lemma} \label{lem:special-tangent-space}
	Let $G\subseteq\op{GL}_n(\FF)$ be a closed Lie subgroup, and let $SG\coloneqq\{g\in G:\det g=1\}$. We show $SG$ is a Lie subgroup and compute $T_1SG\subseteq T_1G$ as
	\[T_1SG=\{g\in T_1G:\tr g=0\}.\]
\end{lemma}
\begin{proof}
	Let $G$ act on $\FF$ by $\mu\colon G\times\FF\to\FF$ by $\mu(g,c)\coloneqq(\det g)c$. Note that $\mu$ is a polynomial and hence regular, so this is a regular action upon checking that $\mu(1,c)=c$ and $\mu(g,\mu(h,c))=\mu(gh,c)$, which hold because $\det$ is a homomorphism.
	
	Now, the stabilizer of $1\in\FF$ consists of the $g\in G$ such that $(\det g)\cdot1=1$, which is equivalent to $\det g=1$ and hence equivalent to $g\in SG$. Thus, $SG\subseteq G$ is a closed Lie subgroup with
	\[T_1SG(\FF)=\{v\in T_1G:(d{\det})_1(v)=0\},\]
	where $\det\colon G\to\FF$ is the determinant map. To compute $(d{\det})_1(v)$, we identify $T_1G\subseteq T_1\op{GL}_n(\FF)=T_1M_n(\FF)\cong M_n(\FF)$; then for any $X\in M_n(\FF)$, we note that the path $\gamma\colon\RR\to M_n(\RR)$ defined by $\gamma(t)\coloneqq 1+tX$ has $\gamma(0)=1$ and $\gamma'(0)=X$, so
	\[(d{\det})_1(X)=(d{\det})_1(\gamma'(0))=(\det\circ\gamma)'(0)=\frac d{dt}\det(1+tX)\bigg|_{t=0}.\]
	Thus, we are interested in the linear terms of the polynomial $\det(1+tX)$. Now, writing $X$ out in coordinates as $X=[X_{ij}]_{1\le i,j\le n}$ and setting $A_{ij}=1_{i=j}+tX_{ij}$, we note
	\[\det(1+tX)=\det A=\sum_{\sigma\in S_n}\op{sgn}(\sigma)A_{1\sigma(1)}\cdots A_{n\sigma(n)}.\]
	Now, the only way a summand can produce linear terms is if there is at most one non-diagonal entry $A_{ij}$, which of course forces all entries to be diagonal. Thus,
	\[\frac d{dt}\det(1+tX)\bigg|_{t=0}=\frac d{dt}(1+tX_{11})\cdots(1+tX_{nn})\bigg|_{t=0}\stackrel*=(X_{11}+\cdots+X_{nn})=\tr X,\]
	where $\stackrel*=$ holds by an expansion of the terms looking for linear terms. Thus,
	\[T_1SG=\{X\in T_1G:\tr X=0\}.\]
\end{proof}
\begin{lemma} \label{lem:bilinear-tangent-space}
	Let $J\in M_n(\FF)$ be some matrix, and let $(-)^*$ denote either of the involutions $(-)^\intercal$ or $(-)^\dagger$. Then one has the subgroup
	\[O_J(\FF)\coloneqq\{g\in\op{GL}_n(\FF):g^*Jg=J\}.\]
	We claim that $O_J(\FF)\subseteq\op{GL}_n(\FF)$ is a closed Lie subgroup (though if $(-)^*=(-)^\dagger$ and $\FF=\CC$, then $O_J(\FF)$ is a group over $\RR$) and compute that
	\[T_1O_J(\FF)=\{X\in M_n(\FF):X^*J+JX=0\}.\]
\end{lemma}
\begin{proof}
	Indeed, let $\op{GL}_n(\FF)$ act on $M_n(\FF)$ by $\mu(g,A)\coloneqq g^*Ag$. This (right!) action is polynomial and hence regular (with the previous parenthetical in mind), and we can check that it is an action because $\mu(1,A)=A$ and $\mu(g,\mu(h,A))=g^*h^*Ahg=\mu(hg,A)$.

	Now, the stabilizer of $J\in M_n(\FF)$ is precisely $O_J(\FF)$ by definition, so $O_J(\FF)\subseteq\op{GL}_n(\FF)$ is in fact a closed Lie subgroup. We also go ahead and compute $T_1O_J(\FF)$. Letting $f\colon\op{GL}_n(\FF)\to M_n(\FF)$ be defined by $f(g)\coloneqq g^*Jg$, we see that
	\[T_1O_J(\FF)=\ker df_1,\]
	so we want to compute $df_1$. As usual, we identify $T_1G\subseteq T_1\op{GL}_n(\FF)=T_1M_n(\FF)\cong M_n(\FF)$; then for any $X\in M_n(\FF)$, we note that the path $\gamma\colon\RR\to M_n(\RR)$ defined by $\gamma(t)\coloneqq 1+tX$ has $\gamma(0)=1$ and $\gamma'(0)=X$, so
	\[df_1(X)=df_1(\gamma'(0))=(f\circ\gamma)'(0)=\frac d{dt}f(1+tX)\bigg|_{t=0}.\]
	Thus, we go ahead and compute
	\[f(1+tX)=(1+tX)^*J(1+tX)=J+t(X^*J+JX)+t^2X^*JX,\]
	so
	\[df_1(X)=\frac d{dt}f(1+tX)\bigg|_{t=0}=X^*J+JX.\]
	Thus,
	\[T_1O_J(\FF)=\{X\in M_n(\FF):X^*J+JX=0\},\]
	as required.
\end{proof}
We now execute our computations in sequence.
\begin{listalph}
	\item Using the preceding remarks, we see that
	\[T_1\op U_{p,q}(\CC)=\{X\in M_n(\CC):X^*B_{p,q}+B_{p,q}X=0\},\]
	where $B_{p,q}=\begin{bsmallmatrix}
		1_p \\ & -1_q
	\end{bsmallmatrix}$ is a diagonal matrix. We now continue as in (c). Set $X\coloneqq\begin{bsmallmatrix}
		A & B \\ C & D
	\end{bsmallmatrix}$ to have the appropriate dimensions, and then we compute
	\begin{align*}
		X^*B_{p,q}+B_{p,q}X &= \begin{bmatrix}
			A^* & C^* \\ B^* & D^*
		\end{bmatrix}\begin{bmatrix}
			1 \\ & -1
		\end{bmatrix}+\begin{bmatrix}
			1 \\ & -1
		\end{bmatrix}\begin{bmatrix}
			A & B \\ C & D
		\end{bmatrix} \\
		&= \begin{bmatrix}
			A^* & -C^* \\ B^* & -D^*
		\end{bmatrix}+\begin{bmatrix}
			A & B \\ -C & -D
		\end{bmatrix} \\
		&= \begin{bmatrix}
			A^*+A & B-C^* \\ B^*-C & -D^*-D
		\end{bmatrix}.
	\end{align*}
	In particular, this will vanish if and only if $A$ and $D$ are skew-Hermitian and $B=C^*$, so
	\begin{align*}
		T_1\op U_{p,q}(\CC) &= \left\{\begin{bmatrix}
			A & B \\ B^* & D
		\end{bmatrix}:A\in M_p(\CC),D\in M_q(\CC),A=-A^*,D=-D^*\right\}, \\
		T_1\op U_n(\CC) &= \left\{A\in M_n(\CC):A=-A^*\right\}.
	\end{align*}
	Now, the space of $p\times p$ skew-Hermitian matrices $A$ (namely, satisfying $A=-A^*$) is forced to have imaginary diagonal, and then the remaining entries are uniquely determined by their values strictly above the diagonal. Thus, the real dimension of this space is $p+p(p-1)=p^2$. We conclude that
	\begin{align*}
		\dim_\RR\op U_{p,q}(\CC) &= p^2+2pq+q^2=n^2, \\
		\dim_\RR\op U_n(\CC) &= n^2.
	\end{align*}
	From here, we address $\op{SU}$ by recalling that
	\[T_1\op{SU}_{p,q}(\RR)=\{X\in T_1\op U_{p,q}(\CC):\tr X\}.\]
	In particular,
	\begin{align*}
		T_1\op{SU}_{p,q}(\CC) &= \left\{\begin{bmatrix}
			A & B \\ B^* & D
		\end{bmatrix}:A\in M_p(\CC),D\in M_q(\CC),A=-A^*,D=-D^*,\tr A+\tr D=0\right\}, \\
		T_1\op{SU}_n(\CC) &= \left\{A\in M_n(\CC):A=-A^*,\tr A=0\right\}.
	\end{align*}
	Now, $\tr$ continues is real and actually surjects onto $\RR$ for $n\ge1$, even for our family of matrices above (for example, for any real number $r$, the matrix $\op{diag}(r,0,\ldots,0)$ has trace $r$ and lives in the above families). Thus, the kernel has dimension one smaller than the total space, giving
	\begin{align*}
		\dim_\RR\op{SU}_{p,q}(\CC) &= n^2-1, \\
		\dim_\RR\op{SU}_n(\CC) &= n^2-1.
	\end{align*}
\end{listalph}
\begin{example}
	We will show that
	\[\op{SL}_n(\FF)\coloneqq\{A\in\op{GL}_n(\FF):\det A=1\}\]
	is a Lie group over $\FF$ and compute its Lie algebra to find $\dim_\FF\op{SL}_n(\FF)=n^2-1$.
\end{example}
\begin{proof}
	We use \Cref{lem:special-tangent-space}. We see that
	\[{T_1\op{SL}_n(\FF)=\{X\in M_n(\FF):\tr X=0\}}.\]
	(Note $T_1\op{GL}_n(\FF)=T_1M_n(\FF)\cong M_n(\FF)$.) As such, we note that $\tr\colon M_n(\FF)\to\FF$ is surjective (for $n\ge1$), so $\dim_\FF T_1\op{SL}_n(\FF)=\dim_\FF\ker\tr=\dim_\FF M_n(\FF)-1=n^2-1$.
\end{proof}
We now begin our computations for bilinear forms.
\begin{example}
	Let $B\coloneqq1_n$ be the standard bilinear form. We will show that
	\[\op O_n(\FF)\coloneqq\left\{A\in\op{GL}_n(\FF):ABA^\intercal B\right\}\]
	is a Lie group over $\FF$ and compute its Lie algebra to find $\dim_\FF\op O_n(\FF)=\frac12n(n-1)$.
\end{example}
\begin{proof}
	We use \Cref{lem:bilinear-tangent-space}. We see that
	\[T_1\op O_n(\FF)=\left\{X\in M_n(\FF):X^\intercal+X=0\right\},\]
	which is the space of alternating matrices. Thus, we see that the diagonal of $X\in T_1\op O_n(\FF)$ vanishes, and the remaining entries are determined by the values strictly above the diagonal, of which there are $\frac12n(n-1)$. Thus, $\dim\op O_n(\FF)=\frac12n(n-1)$.
\end{proof}
\begin{example}
	We will show that
	\[\op{SO}_n(\FF)\coloneqq\{A\in\op{O}_n(\FF):\det A=1\}\]
	is a Lie group over $\FF$ and compute its Lie algebra to find $\dim_\FF\op{SO}_n(\FF)=\frac12n(n-1)$.
\end{example}
\begin{proof}
	Using \Cref{lem:special-tangent-space}, we see that
	\[T_1\op{SO}_n(\FF)=\{X\in\op O_n(\FF):\tr X=0\}.\]
	However, alternating matrices already have vanishing traces, so $T_1\op{SO}_n(\FF)$ is simply the full space of alternating matrices, giving $\dim_\FF\op{SO}_n(\FF)=\frac12n(n-1)$.
\end{proof}
Over $\RR$, there are more bilinear forms.
\begin{example}
	Let $B_{p,q}\coloneqq1_p\oplus1_q$ where $n=p+q$. We will show that
	\[\op O_{p,q}(\RR)\coloneqq\left\{A\in\op{GL}_n(\RR):AB_{p,q}A^\intercal B_{p,q}\right\}\]
	is a Lie group over $\RR$ and compute its Lie algebra to find $\dim_\RR\op O_{p,q}(\RR)=\frac12n(n-1)$.
\end{example}
\begin{proof}
	By \Cref{lem:bilinear-tangent-space}, we see that
	\[T_1\op O_{p,q}(\RR)=\left\{X\in M_n(\RR):X^\intercal B_{p,q}+B_{p,q}X=0\right\},\]
	where $B_{p,q}=\begin{bsmallmatrix}
		1_p \\ & -1_q
	\end{bsmallmatrix}$ is a diagonal matrix. To compute the dimension of this space of matrices, we set $X\coloneqq\begin{bsmallmatrix}
		A & B \\ C & D
	\end{bsmallmatrix}$ to have the appropriate dimensions, and then we compute
	\begin{align*}
		X^\intercal B_{p,q}+B_{p,q}X &= \begin{bmatrix}
			A^\intercal & C^\intercal \\ B^\intercal & D^\intercal
		\end{bmatrix}\begin{bmatrix}
			1 \\ & -1
		\end{bmatrix}+\begin{bmatrix}
			1 \\ & -1
		\end{bmatrix}\begin{bmatrix}
			A & B \\ C & D
		\end{bmatrix} \\
		&= \begin{bmatrix}
			A^\intercal & -C^\intercal \\ B^\intercal & -D^\intercal
		\end{bmatrix}+\begin{bmatrix}
			A & B \\ -C & -D
		\end{bmatrix} \\
		&= \begin{bmatrix}
			A^\intercal+A & B-C^\intercal \\ B^\intercal-C & -D^\intercal-D
		\end{bmatrix}.
	\end{align*}
	In particular, this will vanish if and only if $A$ and $D$ are both alternating, and $B=C^\intercal$, yielding
	\[T_1\op O_{p,q}(\RR)=\left\{\begin{bmatrix}
		A & B \\ B^\intercal & D
	\end{bmatrix}:A\in M_p(\RR)\text{ and }D\in M_q(\RR)\text{ are alternating}\right\}.\]
	Thus, the dimension of our space is
	\begin{align*}
		\dim_\RR\op O_{p,q}(\RR) &= \underbrace{\frac12p(p-1)}_A+\underbrace{pq}_{B=C^\intercal}+\underbrace{\frac12q(q-1)}_{D} \\
		&= \frac12\left(p^2+2pq+q^2-p-q\right) \\
		&= \frac12(p+q)(p+q-1) \\
		&= \frac12n(n-1),
	\end{align*}
	where the dimension computations for (the spaces of) $A$ and $D$ are as in.
\end{proof}
\begin{example}
	We will show that
	\[\op{SO}_{p,q}(\RR)\coloneqq\{A\in\op{O}_{p,q}(\FF):\det A=1\}\]
	is a Lie group over $\RR$ and compute its Lie algebra to find $\dim_\RR\op{SO}_{p,q}(\RR)=\frac12n(n-1)$.
\end{example}
\begin{proof}
	We use \Cref{lem:special-tangent-space}, we note that
	\[T_1\op{SO}_{p,q}(\RR):\{X\in T_1\op O_{p,q}(\RR):\tr X=0\},\]
	but our description of $X=\begin{bsmallmatrix}
		A & B \\ C & D
	\end{bsmallmatrix}$ has $A$ and $D$ alternating, so $\tr X=\tr A+\tr D=0$. Thus, we see $T_1\op{SO}_{p,q}(\RR)=T_1\op O_{p,q}(\RR)$, so the above description of tangent space and dimension go through.
\end{proof}
\begin{example}
	Let $\Omega_{2n}\coloneqq\begin{bsmallmatrix}
		0_n & -1_n \\
		1_n & 0_n
	\end{bsmallmatrix}$ be the standard symplectic form. We will show that
	\[\op{Sp}_{2n}(\FF)\coloneqq\left\{A\in\op{GL}_n(\FF):A\Omega A^\intercal=A\right\}\]
	is a Lie group over $\FF$ and compute its Lie algebra to find $\dim_\FF\op{Sp}_{2n}(\FF)=2n^2+n$.
\end{example}
\begin{proof}
	By \Cref{lem:bilinear-tangent-space}, we see that
	\[T_1\op{Sp}_{2n}(\FF)=\{X\in M_n(\FF):X^\intercal\Omega+\Omega X=0\},\]
	where $\Omega=\begin{bmatrix}
		& -1_n \\ 1_n
	\end{bmatrix}$ is alternating. As usual, we set $X\coloneqq\begin{bsmallmatrix}
		A & B \\ C & D
	\end{bsmallmatrix}$ to have the appropriate dimensions, and we compute
	\begin{align*}
		X^\intercal\Omega+\Omega X &= \begin{bmatrix}
			A^\intercal & C^\intercal \\ B^\intercal & D^\intercal
		\end{bmatrix}\begin{bmatrix}
			& -1 \\ 1
		\end{bmatrix}+\begin{bmatrix}
			& -1 \\ 1
		\end{bmatrix}\begin{bmatrix}
			A & B \\ C & D
		\end{bmatrix} \\
		&= \begin{bmatrix}
			C^\intercal & -A^\intercal \\
			D^\intercal & -B^\intercal
		\end{bmatrix}+\begin{bmatrix}
			-C & -D \\
			A & B
		\end{bmatrix} \\
		&= \begin{bmatrix}
			C^\intercal-C & -D-A^\intercal \\
			A+D^\intercal & B-B^\intercal
		\end{bmatrix}.
	\end{align*}
	Thus, we see that
	\[T_1\op{Sp}_{2n}(\FF)=\left\{\begin{bmatrix}
		A & B \\ C & -A^\intercal
	\end{bmatrix}:A,B,C\in M_n(\FF),B=B^\intercal,C=C^\intercal\right\},\]
	and our dimension is
	\[\dim_\FF\op{Sp}_{2n}(\FF)=\underbrace{n^2}_A+\underbrace{\frac12n(n+1)}_B+\underbrace{\frac12n(n+1)}_C=2n^2+n,\]
	where we compute the dimension of space of symmetric matrices exactly analogously to the case of alternating matrices, except now the diagonal is permitted to be nonzero.
\end{proof}
Lastly, we handle Hermitian forms.
% \begin{example}
% 	We will show that
% 	\[\op U_n(\CC)\coloneqq\left\{A\in\op{GL}_n(\CC):AA^\dagger=1_n\right\}\]
% 	is a Lie group over $\RR$ and compute its Lie algebra.
% \end{example}
% \begin{example}
% 	We will show that
% 	\[\op{SU}_n(\CC)\coloneqq\left\{A\in\op{U}_n(\CC):\det A=1_n\right\}\]
% 	is a Lie group over $\RR$ and compute its Lie algebra.
% \end{example}
% And because we are over $\RR$, we have the following extra forms.
\begin{example}
	Let $B_{p,q}\coloneqq1_p\oplus1_q$ where $n=p+q$. We will show that
	\[\op U_{p,q}(\CC)\coloneqq\left\{A\in\op{GL}_n(\FF):AB_{p,q}A^\dagger B_{p,q}\right\}\]
	is a Lie group over $\RR$ and compute its Lie algebra to find $\dim_\RR\op U_{p,q}(\CC)=n^2$.
\end{example}
\begin{proof}
	Using \Cref{lem:bilinear-tangent-space}, we see that
	\[T_1\op U_{p,q}(\CC)=\{X\in M_n(\CC):X^*B_{p,q}+B_{p,q}X=0\},\]
	where $B_{p,q}=\begin{bsmallmatrix}
		1_p \\ & -1_q
	\end{bsmallmatrix}$ is a diagonal matrix. We now continue as in (c). Set $X\coloneqq\begin{bsmallmatrix}
		A & B \\ C & D
	\end{bsmallmatrix}$ to have the appropriate dimensions, and then we compute
	\begin{align*}
		X^*B_{p,q}+B_{p,q}X &= \begin{bmatrix}
			A^* & C^* \\ B^* & D^*
		\end{bmatrix}\begin{bmatrix}
			1 \\ & -1
		\end{bmatrix}+\begin{bmatrix}
			1 \\ & -1
		\end{bmatrix}\begin{bmatrix}
			A & B \\ C & D
		\end{bmatrix} \\
		&= \begin{bmatrix}
			A^* & -C^* \\ B^* & -D^*
		\end{bmatrix}+\begin{bmatrix}
			A & B \\ -C & -D
		\end{bmatrix} \\
		&= \begin{bmatrix}
			A^*+A & B-C^* \\ B^*-C & -D^*-D
		\end{bmatrix}.
	\end{align*}
	In particular, this will vanish if and only if $A$ and $D$ are skew-Hermitian and $B=C^*$, so
	\begin{align*}
		T_1\op U_{p,q}(\CC) &= \left\{\begin{bmatrix}
			A & B \\ B^* & D
		\end{bmatrix}:A\in M_p(\CC),D\in M_q(\CC),A=-A^*,D=-D^*\right\}, \\
		T_1\op U_n(\CC) &= \left\{A\in M_n(\CC):A=-A^*\right\}.
	\end{align*}
	Now, the space of $p\times p$ skew-Hermitian matrices $A$ (namely, satisfying $A=-A^*$) is forced to have imaginary diagonal, and then the remaining entries are uniquely determined by their values strictly above the diagonal. Thus, the real dimension of this space is $p+p(p-1)=p^2$. We conclude that
	\[\dim_\RR\op U_{p,q}(\CC) = p^2+2pq+q^2=n^2,\]
	as required.
\end{proof}
\begin{example} \label{ex:su}
	We will show that
	\[\op{SU}_{p,q}(\CC)\coloneqq\left\{A\in\op{SU}_{p,q}(\CC):\det A=1_n\right\}\]
	is a Lie group over $\RR$ and compute its Lie algebra to find $\dim_\RR\op{SU}_{p,q}(\CC)=n^2-1$.
\end{example}
\begin{proof}
	By \Cref{lem:special-tangent-space}, we see
	\[T_1\op{SU}_{p,q}(\RR)=\{X\in T_1\op U_{p,q}(\CC):\tr X\}.\]
	In particular,
	\begin{align*}
		T_1\op{SU}_{p,q}(\CC) &= \left\{\begin{bmatrix}
			A & B \\ B^* & D
		\end{bmatrix}:A\in M_p(\CC),D\in M_q(\CC),A=-A^*,D=-D^*,\tr A+\tr D=0\right\}, \\
		T_1\op{SU}_n(\CC) &= \left\{A\in M_n(\CC):A=-A^*,\tr A=0\right\}.
	\end{align*}
	Now, $\tr$ continues is real and actually surjects onto $\RR$ for $n\ge1$, even for our family of matrices above (for example, for any real number $r$, the matrix $\op{diag}(r,0,\ldots,0)$ has trace $r$ and lives in the above families). Thus, the kernel has dimension one smaller than the total space, giving
	\[\dim_\RR\op{SU}_{p,q}(\CC) = n^2-1,\]
	as required.
\end{proof}

\end{document}