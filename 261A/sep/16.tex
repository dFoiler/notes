% !TEX root = ../notes.tex

\documentclass[../notes.tex]{subfiles}

\begin{document}

\section{September 16}
Today we will talk about Lie algebras of classical groups.

\subsection{The Exponential Map: The Classical Case}
Let's work through our examples by hand. Recall that our classical groups are our subgroups of $\op{GL}_n(\FF)$ cut out by equations involving $\det$ and preserving a bilinear/sesquilinear form (symmetric, symplectic, or Hermitian).
\begin{example}
	We show that $\op{GL}_n(\FF)$ is a Lie group over $\FF$ and compute its Lie algebra.
\end{example}
\begin{proof}
	Note $\op{GL}_n(\FF)$ is an open submanifold of $\op M_n(\FF)\cong\FF^{n\times n}$. Matrix multiplication and inversion are rational functions of the coordinates and hence smooth, so $\op{GL}_n(\FF)$ succeeds at being a Lie group. Lastly, we see that being open implies that our tangent space is
	\[T_e\op{GL}_n(\FF)\cong T_e\op M_n(\FF)\cong\FF^n,\]
	as required.
\end{proof}
We will postpone the remaining computations until we discuss the exponential.
For these computations, we want the exponential map.
\begin{definition}[exponential]
	For $X\in\mf{gl}_n(\FF)$, we define the exponential map $\exp\colon\mf{gl}_n(\FF)\to\op{GL}_n(\FF)$ by
	\[\exp(X)\coloneqq\sum_{k=0}^\infty\frac{X^k}{k!}.\]
	Note that $\exp$ is an isomorphism at the identity, so the Inverse function theorem provides a smooth ``local'' inverse $\log(1_n+X)$ defined in an open neighborhood of $1_n$. In fact, one can formally compute that
	\[\log(1_n+X)=\sum_{k=1}^\infty(-1)^{k+1}\frac{X^k}k.\]
\end{definition}
We run a few small checks.
\begin{remark}
	Note $\exp(0)=1$. In fact, one can check that $d\exp_0(A)=A$ for any $A\in\mf{gl}_n(\FF)$ by taking the derivative term by term.
\end{remark}
\begin{remark} \label{rem:manipulate-in-exp}
	We also see that $\exp\left(AXA^{-1}\right)=A\exp(X)A^{-1}$ and $\exp(X^\intercal)=\exp(X)^\intercal$ and $\exp(X^\dagger)=\exp(X)^\dagger$ by a direct expansion.
\end{remark}
What's important about $\exp$ is the following multiplicative property.
\begin{lemma}
	Fix $X,Y\in\mf{gl}_n(\FF)$ which commute. Then
	\[\exp(X+Y)=\exp(X)\exp(Y).\]
\end{lemma}
\begin{proof}
	We check this in formal power series. Because everything in sight converges, this is safe. The main point is to just expand everything. Indeed,
	\begin{align*}
		\exp(X+Y) &= \sum_{k=0}^\infty\frac{(X+Y)^k}{k!} \\
		&= \sum_{k=0}^\infty\frac1{k!}\Bigg(\sum_{a+b=k}\binom kaX^aY^b\Bigg) \\
		&= \sum_{a,b=0}^\infty\frac1{(a+b)!}\cdot\frac{(a+b)!}{a!b!}X^aY^b \\
		&= \sum_{a,b=0}^\infty\frac{X^a}{a!}\cdot\frac{Y^b}{b!} \\
		&= \exp(X)\exp(Y),
	\end{align*}
	as required.
\end{proof}
\begin{remark}
	For fixed $X$, the previous point implies that the map $\FF\to\op{GL}_n(\FF)$ given by $t\mapsto\exp(tX)$ is a Lie group homomorphism. (Smoothness is automatic by smoothness of $\exp$.) The image of this map is called the ``$1$-parameter subgroup'' generated by $X$.
\end{remark}
\begin{remark}
	Taking inverses shows that $\log(XY)=\log X+\log Y$ for $X$ and $Y$ close enough to the identity.
\end{remark}
Here is another check which is a little more interesting.
\begin{lemma} \label{lem:det-exp-exp-tr}
	Fix $X\in\mf{gl}_n(\FF)$. Then
	\[\det\exp(X)=\exp(\tr X).\]
\end{lemma}
\begin{proof}
	The computations do not change if we extend the base field, so we may work over $\CC$ everywhere. Thus, we may assume that $X$ is upper-triangular by conjugating (see \Cref{rem:manipulate-in-exp}) say with diagonal entries $\{d_1,\ldots,d_n\}$. Now, for any $k\ge0$, any $X^k$ continues to be upper-triangular with diagonal entries $\left\{d_1^k,\ldots,d_n^k\right\}$. Thus, we see that $\exp(X)$ is upper-triangular with diagonal entries $\left\{\exp(d_1),\ldots,\exp(d_n)\right\}$, so
	\begin{align}
		\det\exp(X) &= \exp(d_1)\cdots\exp(d_n) \\
		&= \exp(d_1+\cdots+d_n) \\
		&= \exp(\tr X),
	\end{align}
	as required.
\end{proof}

\subsection{The Classical Groups}
For our classical groups, we will show the following result.
\begin{theorem}
	For each classical group $G\subseteq\op{GL}_n(\FF)$, there will exist a vector subspace $\mf g\subseteq\mf{gl}_n(\FF)$ (which can be identified with $T_eG$ via the embedding $G\subseteq\op{GL}_n(\FF)$) and open neighborhoods of the identity $U\subseteq\op{GL}_n(\FF)$ and $\mf u\subseteq\mf g$ such that $\exp\colon(U\cap G)\to(\mf u\cap\mf g)$ is a local isomorphism.
\end{theorem}
Before engaging with the examples, we note the following corollary.
\begin{corollary}
	For each classical group $G$, we see that $G$ is a Lie group with $T_eG=\mf g$ and $\dim G=\dim\mf g$.
\end{corollary}
\begin{proof}
	It suffices to provide a slice chart of the identity for $G\subseteq\op{GL}_n(\FF)$; we then get slice charts everywhere by translation. Well, 
\end{proof}
Let's now proceed with our examples.
\begin{example}
	We will show that
	\[\op{SL}_n(\FF)\coloneqq\{A\in\op{GL}_n(\FF):\det A=1\}\]
	is a Lie group over $\FF$ and compute its Lie algebra.
\end{example}
\begin{proof}
	The main point is to classify which $A\in\im\exp$ actually live in our subgroup. Well, $A=\exp(X)$ has $\det A=1$ if and only if $\tr X=0$ by \Cref{lem:det-exp-exp-tr}, so we see that our Lie subalgebra is
	\[\mf{sl}_n(\FF)\coloneqq\{X\in\mf{gl}_n(\FF):\tr X=0\}.\]
	The result follows.
\end{proof}
We now begin our computations for bilinear forms.
\begin{example}
	Let $B\coloneqq1_n$ be the standard bilinear form. We will show that
	\[\op O_n(\FF)\coloneqq\left\{A\in\op{GL}_n(\FF):ABA^\intercal B\right\}\]
	is a Lie group over $\FF$ and compute its Lie algebra.
\end{example}
\begin{proof}
	As before, writing $A=\exp(X)$, we see that $ABA^\intercal=B$ is equivalent to $BA^\intercal B^{-1}=A^{-1}$, which is equivalent to
	\[\exp\left(BXB^{-1}\right)=\exp(-X),\]
	so we are asking for $BX^\intercal B^{-1}=-X$, as required.
\end{proof}
\begin{example}
	We will show that
	\[\op{SO}_n(\FF)\coloneqq\{A\in\op{O}_n(\FF):\det A=1\}\]
	is a Lie group over $\FF$ and compute its Lie algebra.
\end{example}
Over $\RR$, there are more bilinear forms.
\begin{example}
	Let $B_{p,q}\coloneqq1_p\oplus1_q$ where $n=p+q$. We will show that
	\[\op O_{p,q}(\FF)\coloneqq\left\{A\in\op{GL}_n(\FF):AB_{p,q}A^\intercal B_{p,q}\right\}\]
	is a Lie group over $\FF$ and compute its Lie algebra.
\end{example}
\begin{example}
	We will show that
	\[\op{SO}_{p,q}(\FF)\coloneqq\{A\in\op{O}_{p,q}(\FF):\det A=1\}\]
	is a Lie group over $\FF$ and compute its Lie algebra.
\end{example}
\begin{example}
	Let $\Omega_{2n}\coloneqq\begin{bsmallmatrix}
		0_n & -1_n \\
		1_n & 0_n
	\end{bsmallmatrix}$ be the standard symplectic form. We will show that
	\[\op{Sp}_{2n}(\FF)\coloneqq\left\{A\in\op{GL}_n(\FF):A\Omega A^\intercal=A\right\}\]
	is a Lie group over $\FF$ and compute its Lie algebra.
\end{example}
Lastly, we handle Hermitian forms.
\begin{example}
	We will show that
	\[\op U_n(\CC)\coloneqq\left\{A\in\op{GL}_n(\CC):AA^\dagger=1_n\right\}\]
	is a Lie group over $\RR$ and compute its Lie algebra.
\end{example}
\begin{example}
	We will show that
	\[\op{SU}_n(\CC)\coloneqq\left\{A\in\op{U}_n(\CC):\det A=1_n\right\}\]
	is a Lie group over $\RR$ and compute its Lie algebra.
\end{example}
And because we are over $\RR$, we have the following extra forms.
\begin{example}
	Let $B_{p,q}\coloneqq1_p\oplus1_q$ where $n=p+q$. We will show that
	\[\op U_{p,q}(\FF)\coloneqq\left\{A\in\op{GL}_n(\FF):AB_{p,q}A^\dagger B_{p,q}\right\}\]
	is a Lie group over $\FF$ and compute its Lie algebra.
\end{example}
\begin{example}
	We will show that
	\[\op{SU}_{p,q}(\CC)\coloneqq\left\{A\in\op{SU}_{p,q}(\CC):\det A=1_n\right\}\]
	is a Lie group over $\RR$ and compute its Lie algebra.
\end{example}
All the above examples are similar to the orthogonal group case. I will add the computations in after doing the homework.

\end{document}