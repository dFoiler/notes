% !TEX root = ../notes.tex

\documentclass[../notes.tex]{subfiles}

\begin{document}

\section{September 27}

Today we continue our discussion of the fundamental theorems of Lie theory.

\subsection{Distributions and Foliations}
Here is the main definition.
\begin{definition}[distribution]
	Fix a regular manifold $M$. Then a \textit{$k$-dimensional distribution $\mc D$} on $X$ is a $k$-dimensional (local) subbundle $\mc D\subseteq TX$.
\end{definition}
\begin{remark}
	Locally at a point $p\in M$, we can think about $\mc D_p$ as being spanned by $k$ linearly independent differentials which spread out over a neighborhood.
\end{remark}
\begin{definition}[integrable]
	A distribution $\mc D$ of dimension $k$ on a regular manifold $M$ is \textit{integrable} if and only if each $p\in M$ has a regular chart $(U,\varphi)$ with local coordinates $\varphi=(x_1,\ldots,x_n)$ such that
	\[D|_U=\op{span}\left\{\frac\del{\del x_1},\ldots,\frac\del{\del x_k}\right\}.\]
\end{definition}
Here is a more coordinate-free check for being integrable.
\begin{definition}[foliation]
	A distribution $\mc D$ of dimension $k$ on a regular manifold $M$ is a \textit{foliation} if and only if $p\in M$ has an ``integral'' immersed submanifold $S_p\subseteq M$, meaning that $T_qS_p=D_q$ for all $q\in S_p$.
\end{definition}
Foliations give rise to partitions of the manifold, called leaves.
\begin{definition}[leaf]
	Fix a foliation $\mathcal D$ of rank $k$ on a smooth manifold $M$. Given $p\in M$, a \textit{leaf} of $\mathcal D$ is the collection of points $q\in M$ such that there is a path $\gamma$ connecting $p$ and $q$ with $\gamma'(t)\in\mathcal D_{\gamma(t)}$ for all $t$. 
\end{definition}
Note that the leaves of $\mathcal D$ are connected and partition $M$.
\begin{example}
	Orbits of the action of $\mathbb R$ on $\mathbb R^2/\mathbb Z^2$ by $r\colon(x,y)\mapsto(x+r,y)$ are leaves.
\end{example}
\begin{example}
	For a fiber bundle with connected fibers, the leaves are fibers.
\end{example}
\begin{example}
	For any connected closed Lie subgroup $H$ of a regular Lie group $G$, the quotient $G\onto G/H$ is a fiber bundle and hence produces leaves given by $H$.
\end{example}
\begin{example}
	If $\mathcal D$ is a vector field (i.e., has dimension $1$), then the leaves are integral submanifolds, which are the integral curves.
\end{example}
Anyway, here is our main theorem.
\begin{theorem}[Frobenius] \label{thm:frobenius}
	A distribution $\mathcal D$ on a smooth manifold $M$ is integrable if and only if $\mathcal D$ is closed under the Lie bracket.
\end{theorem}
\begin{proof}
	The forward direction is not so bad: note $\mathcal D$ being integrable means that each $p\in M$ has an open neighborhood where $\mathcal D$ is just given by tangent spaces, and vector fields living in tangent spaces will be preserved by the Lie bracket. For the converse, see \cite[Theorem~19.12]{lee-manifolds}. It proceeds by induction.
\end{proof}

\subsection{Sketches of the Fundamental Theorems}
We begin with \Cref{thm:lie-1}.
\begin{proof}[Proof of \Cref{thm:lie-1}]
	The main point is to produce the reverse map producing Lie subgroups from Lie subalgebras. As such, fix some Lie subalgebra $\mf h\subseteq\mf g$. For each $X\in\mf g$, let $\xi_X$ be the corresponding left-invariant vector field. We then let
	\[\mathcal D^{\mf h}\coloneqq\op{span}\{\xi_Y:Y\in\mf h\}.\]
	Here are some checks on $\mathcal D^{\mf h}$.
	\begin{itemize}
		\item Quickly, we claim that $\mathcal D^{\mf h}$ is integrable. By \Cref{thm:frobenius}, it is enough to check that $\mathcal D^{\mf h}$ is closed under the bracket. This is a matter of computation: for two vector fields $\sum_if_i\xi_{Y_i}$ and $\sum_jg_j\xi_{Y_j}$ contained in $\mathcal D^{\mf h}$, we find
		\begin{align*}
			\left[\sum_if_i\xi_{Y_i},\sum_jg_j\xi_{Y_j}\right] &= \sum_{i,j}\left(f_i(\xi_{Y_i}g_j)\xi_{X_i}-g_j(\xi_{Y_j}f_i)\xi_{h_i}\right) \\
			&= \cdots.
		\end{align*}
		\item In fact, we note that $\mathcal D^{\mf h}$ is left-invariant.
	\end{itemize}
	We now let $S_g$ be the integral submanifold corresponding to $g\in G$, and we note that we can take $H\coloneqq S_1$ to complete the proof.
\end{proof}
We now proceed with \Cref{thm:lie-2}.
\begin{proof}[Proof of \Cref{thm:lie-2}]
	Injectivity follows from \Cref{cor:get-morphism-from-algebra}, so we merely need to get the surjectivity. The point is to pass to the graph in order to produce morphisms when we already know how to produce objects (via \Cref{thm:lie-1}).
	
	Fix some homomorphism $\psi\colon\mf g\to\mf h$ of Lie algebras. Well, define $\theta\colon\mf g\to\mf g\oplus\mf k$ by $\theta(X)\coloneqq(X,\psi(X))$; note that this is still a Lie algebra homomorphism because it is the sum of Lie algebra homomorphisms. Now, $\im\theta$ is a Lie subalgebra of $\op{Lie}(G\times H)=\mf g\times\mf h$ by \Cref{lem:subalgebra-checks}, so \Cref{thm:lie-1} tells us that we can find some subgroup connected $\Gamma\subseteq G\times H$ with
	\[\op{Lie}\Gamma=\im\theta.\]
	Let $\op{pr}_1\colon\Gamma\to G$ and $\op{pr}_2\colon\Gamma\to H$ be the projections. Note that $d({\op{pr}_1})_e\circ\theta={\op{id}_{\mathfrak g}}$ by definition of $\theta$, and $\theta\circ d({\op{pr}_1})_e=\id_{\op{Lie}\Gamma}$ by construction of $\Gamma$. Thus, we see that $d(\op{pr}_1)_e$ is a bijection and hence a local isomorphism; in particular, $\op{pr}_1\colon G\to\Gamma$ must be a covering space map, so we conclude that $\op{pr}_1$ is actually an isomorphism. We thus recover a map
	\[G\stackrel{\op{pr}_1}\from\Gamma\stackrel{\op{pr}_2}\to H\]
	which is $\psi$ on the level of Lie algebras, as required.
\end{proof}
We will largely omit the proof of \Cref{thm:lie-3}. It follows from strong structure theory of Lie algebras. For example, one wants the following result.
\begin{theorem}[Ado]
	Any finite-dimensional Lie algebra $\mf g$ has a faithful representation. In other words, there exists a finite-dimensional vector space $V$ and an injective Lie algebra homomorphism $\mf g\into\mf{gl}(V)$.
\end{theorem}
We will not show this, but we remark on a special case.
\begin{remark}
	Suppose that $\mf g$ is a Lie algebra with $\mf z(\mf g)=0$. Then the adjoint representation $\op{ad}_\bullet\colon\mf g\to\mf{gl}(\mf g)$ given by $X\mapsto[X,-]$ is a faithful representation.
\end{remark}
With this in hand, we can prove \Cref{thm:lie-3}.
\begin{proof}[Proof of \Cref{thm:lie-3}]
	By passing to the universal cover of the connected component, it suffices to produce some regular Lie group $G$ with $\op{Lie}G=\mf g$. Well, embed $\mf g\subseteq\mf{gl}(V)$ for some finite-dimensional vector space $V$, and then we are done by \Cref{thm:lie-1} after noticing $\mf{gl}(V)=\op{Lie}\op{GL}(V)$.
\end{proof}

\subsection{Complexifications}
In the sequel, we will want to focus on Lie algebras of $\CC$ instead of $\RR$. For this, we make the following definition.
\begin{definition}[complexification]
	Fix a real Lie algebra $\mf g$. Then we define the \textit{complexification} as
	\[\mf g_\CC\coloneqq\mf g\otimes_\RR\CC.\]
	Then $\mf g_\CC$ is a Lie algebra over $\CC$.
\end{definition}
% \begin{remark}
% 	Let's explain why $\mf g_\CC$ is a Lie algebra over $\CC$.
% \end{remark}
\begin{example}
	One sees that $\mf{gl}_n(\RR)_\CC$ is simply $\mf{gl}_n(\CC)$. However, $\mf{gl}_n(\CC)$ is also $\mf u_n(\CC)_\CC$ after some care.
\end{example}
\begin{example}
	One sees that $\mf{so}_{k,\ell}(\RR)_\CC$ is just $\mf{so}_\CC(\CC)$.
\end{example}
\begin{definition}[complexification]
	Fix a simply connected real Lie group $H$ over $\RR$. Then we let $G$ be the unique simply connected complex Lie group $G$ such that
	\[\op{Lie}G=\op{Lie}H\otimes_\RR\CC.\]
	Note that $G$ certainly exists by \Cref{thm:lie-3}.
\end{definition}
\begin{example}
	Note that $\op{SL}_2(\RR)$ has a two-sheeted cover, which on the level of Lie algebras is given by $\mf{sl}_2(\RR)\otimes_\RR\CC$. 
\end{example}
With care, one is able to go in the reverse direction.
\begin{definition}[real form]
	Fix a connected complex Lie group $G$. A real Lie subgroup $H\subseteq G$ is a \textit{real form} of $G$ such that the natural map
	\[\op{Lie}H\otimes_\RR\CC\to\op{Lie}G\]
	is an isomorphism of Lie algebras over $\CC$.
\end{definition}
\begin{example}
	We see that $\mf u_n(\CC)\subseteq\mf{gl}_n(\CC)$ is a real form coming from a compact Lie group!
\end{example}
\begin{remark}
	It is not technically obvious that real forms always exist. One thing one may try for simply connected $G$ is to take the fixed points of the map $G\to G$ induced by the complex conjugation morphism $\mf g\to\mf g$.
\end{remark}

\end{document}
