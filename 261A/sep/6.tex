% !TEX root = ../notes.tex

\documentclass[../notes.tex]{subfiles}

\begin{document}

\section{September 6}
Last time we defined a Lie group. Today and for the rest of the course, we will study them.

\subsection{Examples of Lie Groups}
Here are some examples of Lie groups and isomorphisms.
\begin{example}
	For our field $\FF$, the $\FF$-vector space $\FF^n$ is a Lie group over $\FF$.
\end{example}
\begin{example} \label{ex:finite-lie-group}
	Any finite (or countably infinite) group given the discrete topology becomes a real and complex Lie group.
\end{example}
\begin{example}
	The groups $\RR^\times$ and $\RR^+$ (under multiplication) are real Lie groups. In fact, one has an isomorphism $\{\pm1\}\times\RR^+\to\RR^\times$ of real Lie groups given by $(\varepsilon,r)\mapsto\varepsilon r$.
\end{example}
\begin{example}
	The group $S^1\coloneqq\{z\in\CC:\left|z\right|=1\}$ (under multiplication) is a real Lie group.
\end{example}
\begin{example}
	The group $\CC^\times$ is a real Lie group. In fact, one has an isomorphism $S^1\times\RR^+\to\CC^\times$ of real Lie groups given again by $(\varepsilon,r)\mapsto\varepsilon r$.
\end{example}
\begin{example}
	Over our field $\FF$, the set $\op{GL}_n(\FF)$ of invertible $n\times n$ matrices is a Lie group. Indeed, it is an open subset of $\FF^{n^2}$ and thus a manifold, and one can check that the inverse and multiplication maps are rational and hence smooth.
\end{example}
\begin{example}
	Consider the collection of matrices
	\[{\op{SU}_2}\coloneqq\left\{A\in\op{GL}_2(\CC):\det A=1\text{ and }AA^\dagger=1_2\right\},\]
	where $A^\dagger$ is the conjugate transpose. Then $\op{SU}_2$ is an embedded submanifold of $\op{GL}_2(\CC)$ (cut out by the given equations) and also a subgroup. By writing out $A=\begin{bsmallmatrix}
		a & b \\ c & d
	\end{bsmallmatrix}$, one can write out our equations on the coefficients as
	\[\begin{cases}
		ad-bc=1, \\
		a\ov a+b\ov b=1, \\
		a\ov c+b\ov d=0, \\
		c\ov c+d\ov d=1.
	\end{cases}\]
	In particular, we see that the vector $(a,b)\in\CC^2$ is orthogonal to the vector $(\ov c,\ov d)$, so we can solve for this line as providing some $\lambda\in\CC$ such that $(\ov c,\ov d)=\lambda(-b,a)$. But then the determinant condition requires $\lambda=1$ from $\left|a\right|^2+\left|b\right|^2=1$. By expanding out $a=w+ix$ and $b=y+iz$, one finds that $\op{SU}_2$ is diffeomorphic to $S^3$.
\end{example}
The classical groups provide many examples of Lie groups over our field $\FF$.
\begin{itemize}
	\item One has $\op{GL}_n(\FF)$ and $\op{SL}_n(\FF)$, which are subsets of matrices cut out by the conditions $\det A\ne0$ and $\det A=1$, respectively.
	\item Orthogonal: fix a non-degenerate symmetric $2$-form $\Omega$ on $\FF^n$. One can always adjust our basis of $\FF^n$ so that $\Omega$ is diagonal, and by adjusting our basis by squares, we may assume that $\Omega$ has only $+1$ or $-1$s on the diagonal. If $\FF=\CC$, we can in fact assume that $\Omega=1_n$, and then we find that our group is
	\[\op O_n(\CC)\coloneqq\left\{A:A^\intercal A=1_n\right\}.\]
	Otherwise, if $\FF=\RR$, then our adjustment (and rearrangement) of the basis allows us to assume that $\Omega$ takes the form $\Omega_{k,n-k}\coloneqq\op{diag}(+1,\ldots,+1,-1,\ldots,-1)$ with $k$ copies of $+1$ and $n-k$ copies of $-1$, and we define
	\[\op O_{k,n-k}(\RR)\coloneqq\left\{A:A^\intercal\Omega A=\Omega\right\}\]
	\item Special orthogonal: one can add the condition that $\det A=1$ to all the above orthogonal groups, which makes the special orthogonal groups.
	\item Symplectic: for $\FF^{2n}$, one can fix a non-degenerate symplectic $2$-form $\Omega$. It turns out that, up to basis, we find that $\Omega=\begin{bsmallmatrix}
		0 & -1_n \\ 1_n & 0
	\end{bsmallmatrix}$, and we define
	\[\op{Sp}_{2n}(\FF)\coloneqq\left\{A:A^\intercal\Omega A=1_{2n}\right\}.\]
	\item Unitary: using the non-degenerate Hermitian forms, we can similarly define
	\[\op U_n(\CC)\coloneqq\left\{A:A^\dagger A=1_n\right\}\]
	is a real Lie group. (Conjugation is not complex analytic, so this is not a complex Lie group!)
\end{itemize}

\subsection{Connected Components}
We will want to focus on connected Lie groups in this class, so we spend a moment describing why one might hope that this is a reasonable reduction. The main point is that it is basically infeasible to classify finite groups, and allowing for disconnected Lie groups forces us to include all these groups in our study by \Cref{ex:finite-lie-group}.

Quickly, recall our notions of connectivity; we refer to \cite[Appendix~A.1]{elber-top} for details.
\begin{definition}[connected]
	A topological space $X$ is \textit{disconnected} if and only if there exists disjoint nonempty open subsets $U,V\subseteq X$ covering $X$. If there exists no such pair of open subsets, then $X$ is \textit{connected}; in other words, the only subsets of $X$ which are both open and closed are $\emp$ and $X$.
\end{definition}
\begin{definition}[connected component]
	Given a topological space $X$ and a point $p\in X$, the \textit{connected component} of $p\in X$ is the union of all connected subspaces of $X$ containing $p$.
\end{definition}
\begin{remark}
	One can check that the connected component is in fact connected and is thus the maximal connected subspace.
\end{remark}
\begin{definition}[path-connected]
	A topological space $X$ is \textit{path-connected} if and only if any two points $p,q\in X$ have some (continuous) path $\gamma\colon[0,1]\to X$ such that $\gamma(0)=p$ and $\gamma(1)=q$.
\end{definition}
\begin{definition}
	Given a topological space $X$ and a point $p\in X$, the \textit{path-connected component} of $p\in X$ is the collection of all $q\in X$ for which there is a path connecting $p$ and $q$.
\end{definition}
\begin{remark}
	One can check that having a path connecting two points of $X$ is an equivalence relation on the points of $X$. Then the path-connected components are the equivalence classes for this equivalence relation. From this, one can check that the path-connected components are the maximal path-connected subsets of a topological space.
\end{remark}
One has the following lemmas.
\begin{lemma}
	Fix a topological space $X$.
	\begin{listalph}
		\item If $X$ is path-connected, then $X$ is connected.
		\item If $X$ is a connected topological $n$-manifold, then $X$ is path-connected.
	\end{listalph}
\end{lemma}
\begin{proof}
	Part (a) is \cite[Lemma~A.16]{elber-top}. Part (b) is \cite[Proposition~1.39]{elber-diff-top}.
\end{proof}
\begin{lemma} \label{lem:top-ivt}
	Fix a continuous surjection $f\colon X\to Y$ of topological spaces. If $X$ is connected, then $Y$ is connected.
\end{lemma}
\begin{proof}
	This is \cite[Lemma~A.8]{elber-top}.
\end{proof}
Anyway, we are now equipped to return to our discussion of Lie groups.
\begin{lemma} \label{lem:components-of-lie-group}
	Fix a Lie group $G$, and let $G^\circ\subseteq G$ be the connected component of the identity $e\in G$. For any $g\in G$, we see that $gG^\circ$ is the connected component of $g$.
\end{lemma}
\begin{proof}
	Certainly $gG^\circ$ is a connected subset containing $g$ by \Cref{lem:top-ivt} (note multiplication is continuous), so it is contained in the connected component of $g$. On the other hand, any connected subset $U$ around $g$ must have $g^{-1}U$ be a connected subset around $e$, so $g^{-1}U\subseteq G^\circ$, so $U\subseteq gG^\circ$. In particular, the connected component of $g$ is also contained in $gG^\circ$.
\end{proof}
\begin{proposition}
	Fix a Lie group $G$, and let $G^\circ\subseteq G$ be the connected component of the identity $e\in G$.
	\begin{listalph}
		\item Then $G^\circ$ is a normal subgroup of $G$.
		\item The quotient $\pi_0(G)\coloneqq G/G^\circ$ given the quotient topology from the surjection $G\onto\pi_0(G)$ is a discrete countable group.
	\end{listalph}
\end{proposition}
\begin{proof}
	We show the parts independently.
	\begin{listalph}
		\item We check this in parts.
		\begin{itemize}
			\item Of course $G^\circ$ is a subgroup: it contains the identity, and the images of the maps $i\colon G^\circ\to G$ and $m\colon G^\circ\times G^\circ\to G$ must land in connected subsets of $G$ containing the identity by \Cref{lem:top-ivt}, so we see that $G^\circ$ is contained 
			\item We now must check that $G^\circ$ is normal. Fix some $g\in G$, and we want to show that $gG^\circ g^{-1}\subseteq G^\circ$. Well, define the map $G^\circ\to G$ by $a\mapsto gag^{-1}$, which we note is continuous because multiplication and inversion are continuous. \Cref{lem:top-ivt} tells us that the image must be connected, and we see $e\mapsto e$, so the image must actually land in $G^\circ$.
		\end{itemize}
		\item One knows that $\pi_0(G)$ is a group because $G^\circ$ is normal, and it is discrete because connected components are both closed and open in $G$, so the corresponding points are closed and open in $\pi_0(G)$. (We are implicitly using \Cref{lem:components-of-lie-group}.) This is countable because a separated topological space must have countably many connected components.
		\qedhere
	\end{listalph}
\end{proof}
\begin{remark}
	One can restate the above result as providing a short exact sequence
	\[1\to G^\circ\to G\to\pi_0(G)\to1\]
	of Lie groups. In this way, we can decompose our study of $G$ into connected Lie groups and discrete countable groups. In this course, we will ignore studying discrete countable groups because they are too hard.
\end{remark}

\end{document}