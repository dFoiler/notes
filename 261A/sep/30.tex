% !TEX root = ../notes.tex

\documentclass[../notes.tex]{subfiles}

\begin{document}

\section{September 30}
Today we will talk about representations.

\subsection{Representations}
Fix a ground field $k$ which is an extension of $\FF$. To review, recall that a representation of a regular Lie group $G$ is a morphism $\rho_V\colon G\to\op{GL}(V)$ of Lie groups; given the data of only the $k$-vector space $V$, we will assume that the representation is called $\rho_V$. A morphism $\varphi\colon V\to W$ of representations is one respecting the $G$-actions: we require $\varphi$ to be linear and satisfying
\[\rho_W(g)\circ\varphi=\varphi\circ\rho_V(g)\]
for all $g\in G$. The category here is called $\mathrm{Rep}_k(G)$.

Similarly, for a Lie algebra $\mf g$, a representation is a morphism $\rho_V\colon\mf g\to\mf{gl}(V)$ of Lie algebras. A morphism $\varphi\colon V\to W$ of representations is one respecting the $G$-action again: again, we need
\[\rho_W(g)\circ\varphi=\varphi\circ\rho_V(g).\]
The category here is called $\mathrm{Rep}_k(\mf g)$.
\begin{remark} \label{rem:bijective-morphism-reps}
	As a quick aside, we note that a bijective morphism $\varphi\colon V\to W$ will be an isomorphism. Indeed, the inverse map $\psi\colon W\to V$ is an isomorphism of vector spaces by linear algebra, and we see that it is invariant under our action as follows: for any $w\in W$ and operator $g$ in $G$ or $\mf g$, write $w=\varphi(v)$ for some unique $v\in V$ so that
	\[\psi(gw)=\psi(g\varphi(v))=\psi(\varphi(gv))=gv=g\psi(w).\]
\end{remark}
Note that if $\mf g=\op{Lie}G$, then we have a functor taking $\rho\colon G\to\op{GL}(V)$ to $d\rho_e\colon\mf g\to\mf{gl}(V)$. Let's explain this.
\begin{lemma} \label{lem:rep-theory-to-lie-alg}
	Fix a regular Lie group $G$ with Lie algebra $\mf g$.
	\begin{listalph}
		\item One has a functor $F\colon\mathrm{Rep}(G)\to\mathrm{Rep}(\mf g)$ sending a representation $\rho\colon G\to\op{GL}(V)$ to $d\rho_e\colon\mf g\to\mf{gl}(V)$.
		\item The functor $F$ is faithful.
	\end{listalph}
\end{lemma}
\begin{proof}
	For (a), we explain that $F\colon\mathrm{Rep}(G)\to\mathrm{Rep}(\mf g)$ is a functor. Here is the data.
	\begin{itemize}
		\item On objects, we send $\rho\colon G\to\op{GL}(V)$ to the map $d\rho_e\colon\mf g\to\mf{gl}(V)$, which we know is a morphism of Lie algebras because $\rho$ is a group homomorphism.

		\item Further, we send morphisms $\varphi\colon V\to W$ of $G$-representations (namely, satisfying $\varphi\circ\rho_V(g)=\rho_W(g)\circ\varphi$ for all $g\in G$) to the morphism $d\varphi_0\colon V\to W$, which of course can be identified with the original map because $\varphi$ is linear. For this to make sense, we should check that $\varphi\colon V\to W$ preserves the $\mf g$-action if it preserves the $G$-action. Well, for $X\in\mf g$ and $v\in V$, we must check that
		\[\varphi(d(\rho_V)_e(X)v)\stackrel?=d(\rho_W)_e(X)\varphi(v).\]
		Well, define $\gamma\colon\FF\to G$ by $\gamma(t)\coloneqq\exp(tX)$. Then we note that linear maps (such as evaluation at $v$) pass through derivatives by their definition as a limit, so
		\begin{align*}
			\varphi(d(\rho_V)_e(X)v) &= \varphi(d(\rho_V)_e(\gamma'(0))v) \\
			&= \varphi\left((\rho_V\circ\gamma)'(0)v\right) \\
			&= \varphi\left(\frac d{dt}\rho_V\circ\gamma(t)\bigg|_{t=0}v\right) \\
			&= \frac d{dt}\varphi(\rho_V(\gamma(t))(v))\bigg|_{t=0} \\
			&= \frac d{dt}\rho_W(\gamma(t))(\varphi(v))\bigg|_{t=0} \\
			&= (\rho_W\circ\gamma)'(0)(\varphi(v)) \\
			&= d\rho_e(X)(\varphi(v)),
		\end{align*}
		as required.
	\end{itemize}
	Here are the coherence checks.
	\begin{itemize}
		\item Identity: note that the identity map $\id_V\colon V\to V$ on $G$-representations (which is the identity linear map) gets sent to the identity linear map $V\to V$ on $\mf g$-representations.
		\item Associativity: for morphisms $\varphi\colon V\to V'$ and $\varphi'\colon V'\to V''$ of $G$-representations, we note that we get the exact same maps out as $\mf g$-representations, so $\psi\circ\varphi$ as a $G$-representation gets sent to $F(\psi\circ\varphi)=\psi\circ\varphi=F\psi\circ F\varphi$.
	\end{itemize}
	The previous point has given us our functor, so we now need to check that it is faithful for (b). Well, a $G$-invariant map $\varphi\colon V\to W$ goes to the same map $\varphi\colon V\to W$ as a $\mf g$-representation by definition of $F$. Thus, given two maps $\varphi_1,\varphi_2\colon V\to W$ of $G$-representations, we see that $F\varphi_1=F\varphi_2$ implies that
	\[\varphi_1=F\varphi_1=F\varphi_2=\varphi_2,\]
	as required.
\end{proof}
\begin{remark} \label{rem:induced-lie-alg-rep}
	It will be helpful to remember in the sequel that
	\[d\rho_e(X)v=\frac d{dt}\rho(\exp(tX))v\bigg|_{t=0},\]
	which was proved in the argument above. Note that this derivative makes sense because it takes place in some Euclidean space.
\end{remark}

\subsection{Operations on Representations}
We present some operations on representations of $G$ and $\mf g$. Note that these should always be related by the ambient functor $\mathrm{Rep}(G)\to\mathrm{Rep}(\mf g)$ which is an equivalence when $G$ is simply connected by \Cref{prop:rep-theory-to-lie-alg}. As basic examples, here are some trivial representations.
\begin{lemma}
	Fix a Lie group $G$ and Lie algebra $\mf g$ and a vector space $V$.
	\begin{listalph}
		\item We can make $V$ into a ``trivial'' $G$-representation by $\rho_V(g)=\id_V$ for all $g\in G$.
		\item We can make $V$ into a ``trivial'' $\mf g$-representation by $\rho_V(X)\coloneqq0$ for all $X\in\mf g$.
		\item Suppose $\mf g=\op{Lie}G$. Making $V$ into a trivial $G$-representation, we see that $F(V)$ is the trivial $\mf g$-representation.
	\end{listalph}
\end{lemma}
\begin{proof}
	Here we go.
	\begin{listalph}
		\item We have indeed defined a homomorphism $G\to\op{GL}(V)$ because this is the trivial homomorphism. It is also regular because constant maps are regular.
		\item We have indeed defined a map $\rho_V\colon\mf g\to\mf{gl}(V)$, and it is linear map of vector spaces. It remains to check that we have defined a map of Lie algebras, for which we note that
		\[[\rho_V(X),\rho_V(Y)]=\rho_V(X)\circ\rho_V(Y)-\rho_V(Y)\circ\rho_V(X)=0=\rho_V([X,Y]).\]
		\item Fix the trivial representation as $\rho\colon G\to\op{GL}(V)$. Then the induced map $d\rho_1\colon\mf g\to\mf{gl}(V)$ is given by
		\[d\rho_1(X)v=\frac d{dt}\rho(\exp(-tX))v\bigg|_{t=0},\]
		but of course $\rho(\exp(-tX))v=v$ for all $t\in\RR$, so this derivative vanishes. Thus, $d\rho_1\colon\mf g\to\mf{gl}(V)$ is the zero map, as required.
		\qedhere
	\end{listalph}
\end{proof}
\begin{example}
	We always have the trivial representation on the zero-dimensional vector space.
\end{example}
As something else easy to do, we note that there are complex conjugate representations.
\begin{lemma}
	Fix a Lie group $G$ and Lie algebra $\mf g$.
	\begin{listalph}
		\item Given a representation $V\in\op{Rep}_\CC(G)$, we can make the complex conjugate vector space $\ov V$ into a representation of $\ov G$ by
		\[\rho_{\ov V}(g)(\ov v)\coloneqq\overline{\rho_V(g)\ov v}.\]
		\item Given a representation $V\in\op{Rep}_\CC(\mf g)$, we can make the complex conjugate vector space $\ov V$ into a representation of $\ov{\mf g}$ by
		\[\rho_{\ov V}(X)(v)\coloneqq\overline{\rho_V(X)v}.\]
		\item Suppose $\mf g=\op{Lie}(G)$. Given a representation $V\in\op{Rep}_\CC(G)$, then $F\ov V=\ov{FV}$ as representations in $\op{Rep}_\CC(\mf g)$.
	\end{listalph}
\end{lemma}
\begin{proof}
	Here we go.
	\begin{listalph}
		\item For each $g\in G$, we note that $\rho_{\ov V}(g)\colon\ov V\to\ov V$ is $\CC$-linear: for any $a,a'\in\CC$ and $\ov v,\ov v'\in\ov V$, we see
		\begin{align*}
			\rho_{\ov V}(g)(a\ov v+a'\ov v') &= \rho_{\ov V}(g)(\ov{\ov av+\ov a'v'}) \\
			&= \ov{\rho_V(g)(\ov{\ov av+\ov a'v'})} \\
			&= \ov{\ov a\rho_V(g)v+\ov a'\rho_V(g)v'} \\
			&= a\rho_{\ov V}(g)(\ov v)+a'\rho_{\ov V}(g)\ov v'.
		\end{align*}
		To show that we have defined a group homomorphism, we see that
		\[\rho_{\ov V}(gh)\ov v=\overline{\rho_V(gh)v}=\overline{\rho_V(g)\rho_V(h)v}=\rho_{\ov V}(g)\rho_{\ov V}(h)\ov v.\]

		Lastly, we note that the map $\rho_{\ov V}\colon G\to\op{GL}(\ov V)$ is a regular map by expanding it on a basis: upon picking a $\CC$-basis of $V$ (which is also a $\CC$-basis of $\ov V$), we see that the matrix $\rho_{\ov V}(g)$ is simply the complex conjugate of the matrix of $\rho(g)$, which will continue to be a regular map after keeping track of all of our conjugations.

		\item The same check as in (a) explains that $\rho_{\ov V}(X)$ is at least a $\CC$-linear map for all $X\in\mf g$. This map is also of course linear in $X$ given by the linearity of $\rho_V$. Lastly, this is a homomorphism of Lie algebras by taking the conjugate of the identity
		\[\rho_V([X,Y])=\rho_V(X)\rho_V(Y)-\rho_V(Y)-\rho_V(X).\]
		
		\item Simply take the conjugate everywhere in sight.
		\qedhere
	\end{listalph}
\end{proof}
To begin doing something with content, we handle direct sums.
\begin{lemma}
	Fix a Lie group $G$ and Lie algebra $\mf g$.
	\begin{listalph}
		\item Given representations $V,W\in\mathrm{Rep}_k(G)$, we can make $V\otimes W$ into a representation of $G$ via the coordinate-wise action
		\[\rho_{V\oplus W}(g)(v\oplus w)\coloneqq\rho_V(g)v\oplus\rho_W(g)w.\]
		\item Given representations $V,W\in\mathrm{Rep}_k(\mf g)$, we can make $V\oplus W$ into a representation of $G$ via the coordinate-wise  action
		\[\rho_{V\oplus W}(X)(v\oplus w)=\rho_V(X)v\oplus\rho_W(X)w.\]
		\item Suppose $\mf g=\op{Lie}(G)$. Given representations $V,W\in\mathrm{Rep}_k(G)$, then $F(V\otimes W)$ is the direct sum representation in $\op{Rep}_k(\mf g)$.
	\end{listalph}
\end{lemma}
\begin{proof}
	Here we go.
	\begin{listalph}
		\item By taking the direct sum of the homomorphisms $\rho_V\colon G\to\op{GL}(V)$ and $\rho_W\colon G\to\op{GL}(W)$, we obtain a regular homomorphism $G\to\op{GL}(V)\oplus\op{GL}(W)$. To finish, we note that $\op{GL}(V)\oplus\op{GL}(W)$ embeds into $\op{GL}(V\oplus W)$ by sending $(\varphi,\psi)$ to the linear map $V\oplus W\to V\oplus W$ acting by $(\varphi,\psi)$ on the coordinates. To see that this last map is a regular homomorphism, we note that fixing an ordered basis of both $V$ and $W$ allows us to identify these $\op{GL}$ groups with invertible matrices, in which case our map is given by
		\[(A,B)\mapsto\begin{bmatrix}
			A \\ & B
		\end{bmatrix}.\]
		In particular, this map is regular in coordinates and hence regular; one can check that it is a homomorphism directly because $(A,B)\cdot(A',B')$ goes to the block-diagonal matrix $\op{diag}(AA',BB')$. In total, we have obtained a composite of regular homomorphisms $G\to\op{GL}(V)\oplus\op{GL}(W)\to\op{GL}(V\oplus W)$.

		\item We will simply proceed directly. We define a map $\rho\colon\mf g\to\mf{gl}(V\oplus W)$ by
		\[\rho(X)\coloneqq\begin{bmatrix}
			\rho_V(X) \\ & \rho_W(X)
		\end{bmatrix},\]
		where we are thinking about endomorphisms of $V\oplus W$ in the above block-diagonal format. Linearity of $\rho_V$ and $\rho_W$ gives linearity of $\rho$. To check the bracket, we compute
		\begin{align*}
			[\rho(X),\rho(Y)] &= \begin{bmatrix}
				\rho_V(X) \\ & \rho_W(X)
			\end{bmatrix}\begin{bmatrix}
				\rho_V(Y) \\ & \rho_W(Y)
			\end{bmatrix}-\begin{bmatrix}
				\rho_V(Y) \\ & \rho_W(Y)
			\end{bmatrix}\begin{bmatrix}
				\rho_V(X) \\ & \rho_W(X)
			\end{bmatrix} \\
			&= \begin{bmatrix}
				\rho_V(X)\circ\rho_V(Y)-\rho_V(Y)\circ\rho_V(X) \\
				& \rho_W(X)\circ\rho_W(Y)-\rho_W(Y)\circ\rho_W(X)
			\end{bmatrix} \\
			&= \begin{bmatrix}
				[\rho_V(X),\rho_V(Y)] \\ & [\rho_W(X),\rho_W(Y)]
			\end{bmatrix} \\
			&= \begin{bmatrix}
				\rho_V([X,Y]) \\ & \rho_W([X,Y])
			\end{bmatrix} \\
			&= \rho([X,Y]).
		\end{align*}

		\item This is a direct computation. Given the representations $\rho_V\colon G\to\op{GL}(V)$ and $\rho_W\colon G\to\op{GL}(W)$ with direct sum $\rho_{V\oplus W}$, we need to compute the direct sum of the representations $d\rho_V$ and $d\rho_W$. Well, for any $X\in\mf g$ and $(v,w)\in V\oplus W$, we note that evaluation at $(v,w)$ is a linear map and hence passes through derivative computations (in Euclidean space!), so
		\begin{align*}
			d\rho_{V\oplus W}(X)(v,w) &= \frac d{dt}\rho_{V\oplus W}(\exp(tX))\bigg|_{t=0}(v,w) \\
			&= \frac d{dt}\begin{bmatrix}
				\rho_V(\exp(tX)) \\ & \rho_W(\exp(tX))
			\end{bmatrix}\begin{bmatrix}
				v \\ w
			\end{bmatrix}\bigg|_{t=0} \\
			&= \frac d{dt}\begin{bmatrix}
				\rho_V(\exp(tX))v \\ & \rho_W(\exp(tX))w
			\end{bmatrix}\bigg|_{t=0}.
		\end{align*}
		Now, because we are in a Euclidean space, we can compute the derivative on each coordinate separately, which we see to be $\op{diag}(d\rho_V(X),d\rho_W(X))$, as needed.
		\qedhere
	\end{listalph}
\end{proof}
Next we handle the tensor product.
\begin{lemma}
	Fix a Lie group $G$ and Lie algebra $\mf g$.
	\begin{listalph}
		\item Given representations $V,W\in\mathrm{Rep}_k(G)$, we can make $V\otimes W$ into a representation of $G$ via the coordinate-wise action
		\[\rho_{V\otimes W}(g)(v\otimes w)\coloneqq\rho_V(g)v\otimes\rho_W(g)w.\]
		\item Given representations $V,W\in\mathrm{Rep}_k(\mf g)$, we can make $V\otimes W$ into a representation of $G$ via the product rule action
		\[\rho_{V\otimes W}(X)(v\otimes w)=\rho_V(X)v\otimes w+v\otimes\rho_W(X)w.\]
		\item Suppose $\mf g=\op{Lie}(G)$. Given representations $V,W\in\mathrm{Rep}_k(G)$, then $F(V\otimes W)$ is the tensor product representation in $\op{Rep}_k(\mf g)$.
	\end{listalph}
\end{lemma}
\begin{proof}
	Here we go.
	\begin{listalph}
		\item For each $g\in G$, we need to provide a bilinear map $\rho(g)\colon(V\times W)\to(V\otimes W)$, for which we take
		\[\rho(g)(v,w)\coloneqq\rho_V(g)v\otimes\rho_W(g)w.\]
		Linearity of $\rho_V(g)$ and $\rho_W(g)$ (and properties of the tensor product) verify that we have in fact defined a bilinear map, so we have in fact defined a map $G\to\op{End}(V\otimes W)$. Here are our checks to make this map a representation.
		\begin{itemize}
			\item Group action: for the identity check, we note that
			\[\rho(e)(v\otimes w)=(v\otimes w)\]
			for any pure tensor $v\otimes w\in V\otimes W$. Thus, because maps out of $V\otimes W$ are determined by their action on pure tensors, we see that $\rho(e)=\id$. Similarly, for $g,h\in G$, we see that
			\[\rho(gh)(v\otimes w)=(\rho_V(g)\rho_V(h)v\otimes\rho_W(g)\rho_W(h)w)=\rho(g)\rho(h)(v\otimes w),\]
			so $\rho(gh)$ and $\rho(g)\circ\rho(h)$ are equal on pure tensors and hence equal as maps $V\otimes W\to V\otimes W$.
			\item Regular: we expand everything on a basis. Fix a basis $\{e_1,\ldots,e_m\}$ of $V$ and $\{f_1,\ldots,f_m\}$ on $W$ so that $\{e_i\otimes f_j\}_{i,j}$ is a basis of $V\otimes W$; let $\op{pr}_\bullet$ be the appropriate projection whenever it appears. The previous step verifies that we have a group homomorphism $\rho\colon G\to\op{GL}(V\otimes W)$, which we must now show to be regular. Notably, the matrix coefficients $\rho(g)_{i_1j_1,i_2j_2}$ of $\rho(g)$ are now computable as
			\[\op{pr}_{i_2j_2}\rho(g)(e_{i_1}\otimes f_{j_1})=\op{pr}_{i_2j_2}(\rho_V(g)e_{i_1}\otimes\rho_W(g)f_{j_1})=\rho_V(g)_{i_1i_2}\rho_W(g)_{j_1j_2},\]
			which is a product of regular functions and hence regular. Thus, $\rho$ is regular on coordinates and hence regular. 
		\end{itemize}

		\item For each $X$, we need to provide a bilinear map $\rho(X)\colon(V\otimes W)\to(V\otimes W)$, for which we take
		\[\rho(X)(v,w)\coloneqq\rho_V(X)v\otimes w+v\otimes\rho_W(X)w.\]
		Linearity of $\rho_V(X)$ and $\rho_W(X)$ (and properties of the tensor product) verify that we have in fact defined a bilinear map, so we have in fact defined a map $\mf g\to\mf{gl}(V\otimes W)$. Here are our checks to make this a representation.
		\begin{itemize}
			\item Linear: for $a,b\in\FF$ and $X,Y\in\mf g$, we should check that $\rho(aX+bY)=a\rho(X)+b\rho(Y)$. Because pure tensors span $V\otimes W$, it is enough to check this equality on pure tensors, for which we compute
			\begin{align*}
				\rho(aX+bY)(v\otimes w) &= \rho_V(aX+bY)v\otimes w+v\otimes\rho_W(aX+bY)w \\
				&= a(\rho_V(X)v\otimes w+v\otimes\rho_W(X)w)+b(\rho_V(Y)v\otimes w+v\otimes\rho_W(Y)w) \\
				&= (a\rho(X)+b\rho(Y))(v\otimes w).
			\end{align*}
			\item Lie bracket: for $X,Y\in\mf g$, we need to check $[\rho(X),\rho(Y)]=\rho([X,Y])$. It is enough to check this on pure tensors, for which we compute
			\begin{align*}
				[\rho(X),\rho(Y)](v\otimes w) &= (\rho(X)\rho(Y)-\rho(Y)\rho(X))(v\otimes w) \\
				&= \rho(X)\rho(Y)(v\otimes w)-\rho(Y)\rho(X)(v\otimes w) \\
				&= \rho_V(X)\rho_V(Y)v\otimes w-\rho_V(Y)v\otimes\rho_W(X)w \\
				&\qquad-\rho_V(X)v\otimes\rho_W(Y)+v\otimes\rho_W(X)\rho_W(Y)w \\
				&\qquad-\rho_V(Y)\rho_V(X)v\otimes w+\rho_V(X)v\otimes\rho_W(Y)w \\
				&\qquad+\rho_V(Y)v\otimes\rho_W(X)w-v\otimes\rho_W(Y)\rho_W(X)w \\
				&= (\rho_V(X)\rho_V(Y)-\rho_V(Y)\rho_V(X))v\otimes w \\
				&\qquad+v\otimes(\rho_W(X)\rho_W(Y)-\rho_W(Y)\rho_W(X))w \\
				&= \rho([X,Y])(v\otimes w).
			\end{align*}
		\end{itemize}

		\item This is a direct computation. Given the representations $\rho_V\colon G\to\op{GL}(V)$ and $\rho_W\colon G\to\op{GL}(W)$, we would like to compute $(d\rho_{V\otimes W})_e(X)\in\mf{gl}(V\otimes W)$ for some $X\in\mf g$. Well, it is enough to compute this on pure tensors $v\otimes w$, for which we note that evaluation is a linear map and hence can be moved inside a derivative in the computation
		\begin{align*}
			(d\rho_{V\otimes W})_e(X)(v\otimes w) &= \frac d{dt}\rho_{V\otimes W}(\exp(tX))\bigg|_{t=0}(v\otimes w) \\
			&= \frac d{dt}\rho_{V\otimes W}(\exp(tX))(v\otimes w)\bigg|_{t=0} \\
			&= \frac d{dt}\rho_V(\exp(tX))v\otimes\rho_W(\exp(tX))w\bigg|_{t=0} \\
			&= \frac d{dt}(1+td\rho_V(X)+\cdots)v\otimes\rho_W(1+td\rho_W(X)+\cdots)w\bigg|_{t=0} \\
			&= d\rho_V(X)v\otimes w+v\otimes d\rho_W(X)w,
		\end{align*}
		as required. Notably, we expanded our the Taylor series in order to the computation of the derivative at $t=0$, but one can also indirectly apply some product rule after working more explicitly with coordinates.
		\qedhere
	\end{listalph}
\end{proof}
\begin{remark}
	By induction, we see that we can also define a tensor representation
	\[V_1\otimes\cdots\otimes V_k\]
	for any finite number of representations $V_1,\ldots,V_k$. One can compute the actions by simply extending the above ones to more terms inductively.
\end{remark}
\begin{example} \label{ex:twist-by-character}
	We explain how to twist by a character.
	\begin{itemize}
		\item Fix a regular Lie group $G$. Given a representation $\rho\colon G\to\op{GL}(V)$ and a character $\chi\colon G\to\op{GL}_1(\FF)$, we see that we have a representation $\chi\otimes\rho$ on $\FF\otimes V$. However, $\FF\otimes V$ can be identified with $V$ by the map $c\otimes v\mapsto cv$ (on pure tensors), so we have really defined a representation $\chi\rho\colon G\to\op{GL}(V)$ given by
		\[(\chi\rho)(g)\coloneqq\chi(g)\rho(g).\]
		\item Fix a Lie algebra $\mf g$. Given a representation $\rho\colon\mf g\to\mf{gl}(V)$ and a character $\chi\colon\mf g\to\mf{gl}(\FF)$, we again see that we have a representation $\chi\otimes\rho$ on $\FF\otimes V$. Identifying $\FF\otimes V$ with $V$ as before, we see that we have defined a representation $\chi\rho\colon G\to\op{GL}(V)$ by
		\[(\chi\rho)(X)=\chi(X)+\rho(X).\]
	\end{itemize}
\end{example}
We also have $\op{Hom}$ sets.
\begin{lemma} \label{lem:internal-hom-rep-theory}
	Fix a Lie group $G$ and Lie algebra $\mf g$.
	\begin{listalph}
		\item Given representations $V,W\in\mathrm{Rep}_k(G)$, we can make $\op{Hom}(V,W)$ into a representation of $G$ via
		\[\rho_{\op{Hom}(V,W)}(g)\varphi\coloneqq\rho_W(g)\cdot\varphi\circ\rho_V(g)^{-1}.\]
		\item Given representations $V,W\in\mathrm{Rep}_k(\mf g)$, we can make $\op{Hom}(V,W)$ into a representation of $G$ via
		\[\rho_{\op{Hom}(V,W)}(X)\varphi\coloneqq\rho_W(X)\circ\varphi-\varphi\circ\rho_V(X).\]
		\item Suppose $\mf g=\op{Lie}(G)$. Given representations $V,W\in\mathrm{Rep}_k(G)$, then $F(\op{Hom}(V,W))$ is the corresponding in $\op{Rep}_k(\mf g)$.
	\end{listalph}
\end{lemma}
\begin{proof}
	Here we go.
	\begin{listalph}
		\item Given finite-dimensional representations $\rho_V\colon G\to\op{GL}(V)$ and $\rho_W\colon G\to\op{GL}(W)$, we explain how to build a representation $\rho\colon G\to\op{Hom}(V,W)$. Indeed, for $g\in G$ and $\varphi\in\op{Hom}(V,W)$, define
		\[(\rho(g)\varphi)(v)\coloneqq\rho_W(g)\varphi\left(\rho_V(g)^{-1}v\right).\]
		In other words, $\rho(g)\varphi=\rho_W(g)\circ\varphi\circ\rho(g)^{-1}$. Here are our checks.
		\begin{itemize}
			\item Group action: for the identity check, we note
			\[\rho(e)\varphi=\rho_W(e)\circ\varphi\circ\rho_V(e)^{-1}={\id_W}\circ\varphi\circ{\id_V^{-1}},\]
			as required. For the associativity check, we choose $g,h\in G$ and note
			\[\rho(g)\rho(h)\varphi=\rho_W(g)\circ\rho_W(h)\circ\varphi\circ\rho(h)^{-1}\circ\rho(g)^{-1}=\rho(gh)\varphi.\]

			\item Regular: it is enough to show that we have given a regular map $G\times\op{Hom}(V,W)\to\op{Hom}(V,W)$ by considering component-wise formulations of matrix entries. Well, our map is simply the composite
			\[\arraycolsep=1.4pt\begin{array}{cccccccc}
				G\times\op{Hom}(V,W) &\to& \op{GL}(W)\times\op{Hom}(V,W)\times\op{GL}(V) &\mapsto& \op{Hom}(V,W) \\
				(g,\varphi) &\mapsto& \left(\rho_W(g),\varphi,\rho_V(g)^{-1}\right) &\mapsto& \rho_W(g)\circ\varphi\circ\rho_V(g)^{-1}
			\end{array}\]
			which is regular as the composite of (products of) regular maps. For example, the last map is regular because it is simply matrix multiplication, which is polynomial on coordinates and hence regular.
		\end{itemize}

		\item For any two Lie algebra representations $V$ and $W$ of $\mf g$, we note that $\op{Hom}(V,W)$ also has a Lie algebra representation structure given by
		\[\rho_{\op{Hom}(V,W)}(X)\varphi\coloneqq\rho_W(X)\circ\varphi-\varphi\circ\rho_V(X).\]

		Anyway, we now run our checks. Certainly $\rho_{\op{Hom}(V,W)}(X)$ is a linear map $\op{Hom}(V,W)\to\op{Hom}(V,W)$ (namely, our construction is linear in $\varphi$) because composition distributes over addition. Additionally, our construction is linear in $X$ because $\rho_W\colon\mf g\to\mf{gl}(W)$ and $\rho_V\colon\mf g\to\mf{gl}(V)$ should be linear. Lastly, we must check preservation of the bracket of our map $\mf g\to\mf{gl}(\op{Hom}(V,W))$. Well, given $\varphi,\psi\in\op{Hom}(V,W)$ and $X,Y\in\mf g$, we compute
		\begin{align*}
			[\rho_{\op{Hom}(V,W)}(X),\rho_{\op{Hom}(V,W)}(Y)](\varphi) &= \rho_{\op{Hom}(V,W)}(X)\circ\rho_{\op{Hom}(V,W)}(Y)(\varphi)\\
			&\qquad-\rho_{\op{Hom}(V,W)}(Y)\circ\rho_{\op{Hom}(V,W)}(X)(\varphi) \\
			&= \rho_{\op{Hom}(V,W)}(X)(Y\circ\varphi-\varphi\circ Y) \\
			&\qquad-\rho_{\op{Hom}(V,W)}(Y)(X\circ\varphi-\varphi\circ X) \\
			&= X\circ Y\circ\varphi-Y\circ\varphi\circ X-X\circ\varphi\circ Y+\varphi\circ Y\circ X \\
			&\qquad-Y\circ X\circ\varphi+X\circ\varphi\circ Y+Y\circ\varphi\circ X-\varphi\circ X\circ Y \\
			&= (X\circ Y-Y\circ X)\circ\varphi-\varphi\circ(X\circ Y-Y\circ X) \\
			&= \rho_W([X,Y])\circ\varphi-\varphi\circ\rho_V([X,Y]) \\
			&= \rho_{\op{Hom}(V,W)}([X,Y])(\varphi),
		\end{align*}
		where we have frequently but not always omitted our $\rho_V$s and $\rho_W$s.

		\item This is a direct computation. If $\varphi\colon V\to W$ were already a morphism of $G$-representations, then the action of (b) is simply $d\rho_{\op{Hom}(V,W)}(X)(\varphi)$: indeed, the action should be
		\begin{align*}
			d\rho_{\op{Hom}(V,W)}(X)(\varphi) &= \frac d{dt}\rho_{\op{Hom}(V,W)}(\exp(tX))\varphi\bigg|_{t=0} \\
			&= \frac d{dt}\rho_W(\exp(tX))\circ\varphi\circ\rho_V(\exp(-tX))\bigg|_{t=0} \\
			&= \frac d{dt}(1+d\rho_W(tX)+\cdots)\circ\varphi\circ(1-d\rho_W(tX)+\cdots)\bigg|_{t=0} \\
			&= d\rho_W(X)\circ\varphi-\varphi\circ d\rho_V(X).
		\end{align*}
		As usual, $+\cdots$ denotes higher-order terms which cannot affect our derivative. Notably, we are using the fact that the linear term of a Taylor expansion (into some Euclidean space) is given by the derivative.
		\qedhere
	\end{listalph}
\end{proof}
\begin{example}
	Taking $W=k$ to be the trivial representation, we obtain duals as a special case of \Cref{lem:internal-hom-rep-theory}.
\end{example}
Here is also a good notion of subobjects.
\begin{definition}[suprepresentation]
	Fix a regular Lie group or Lie algebra. A \textit{subrepresentation} of a representation is a subspace preserved by the $G$-action.
\end{definition}
\begin{remark}
	Let's make this notion more precise.
	\begin{itemize}
		\item For a regular Lie group $G$, we see that a subspace $U\subseteq V$ preserved by the $G$-action on a representation $\rho_V\colon G\to\op{GL}(V)$ means that we can restrict the linear action map $G\times U\to V$ to an action $G\times U\to U$. Thus, we do indeed have a regular map $\rho_U\colon G\to\op{GL}(U)$ by computing coordinates of matrices component-wise, and the natural inclusion map $U\into V$ is a morphism in $\op{Rep}_k(G)$.
		\item For a Lie algebra $\mf g$, we see that a subspace $U\subseteq V$ preserved by the $\mf g$-action on a representation $\rho_V\colon\mf g\to\mf{gl}(V)$ means that we can restrict this linear map to $\rho_U\colon\mf g\to\mf{gl}(U)$. Notably, the Lie bracket of $\mf{gl}(U)$ is more or less the restriction of the Lie bracket on $\mf{gl}(V)$, so $\rho_U$ continues to be a Lie algebra representation, and we see that the natural inclusion $U\into V$ is a morphism in $\op{Rep}_k(\mf g)$.
	\end{itemize}
\end{remark}
\begin{example} \label{ex:ker-subrep-grp}
	Let $\varphi\colon V\to W$ be a morphism in $\op{Rep}_k(G)$. Then $\ker\varphi$ is a subrepresentation of $V$. Indeed, $\ker\varphi\subseteq V$ is certainly a linear subspace, and for the $G$-invariance, we note that any $v\ker\varphi$ has
	\[\varphi(\rho_V(g)v)=\rho_W(g)(\varphi(v))=0\]
	for any $g\in G$, so $\rho_V\colon G\to\op{GL}(V)$ restricts to a subrepresentation $\rho_{\ker\varphi}\colon G\to\op{GL}(\ker\varphi)$. (More precisely, we have restricted our regular action $G\times\ker\varphi\to V$ to a regular action $G\times\ker\varphi\to\ker\varphi$.)
\end{example}
\begin{example} \label{ex:ker-subrep-alg}
	Let $\varphi\colon V\to W$ be a morphism in $\op{Rep}_k(\mf g)$. Again, we see that $\ker\varphi\subseteq V$ is a subrepresentation for essentially the same reason: certainly $\ker\varphi\subseteq V$ is a linear subspace, and $v\in\ker\varphi$ has $\varphi(Xv)=X(\varphi(v))=0$ for any $X\in\mf g$, so $\ker\varphi$ is closed under the $G$-action.
\end{example}
\begin{example} \label{ex:image-subrep}
	Let $\varphi\colon V\to W$ be a morphism in $\op{Rep}_k(G)$. Then $\im\varphi$ is a subrepresentation of $W$. Again, it is certainly a linear subspace, and it is preserved by the $G$-action because any $g\in G$ and $\varphi(v)\in\im\varphi$ has
	\[\rho_W(g)(\varphi(v))=\varphi(\rho_V(g)v)\in\im\varphi.\]
\end{example}
\begin{example} \label{ex:image-subrep-alg}
	Let $\varphi\colon V\to W$ be a morphism in $\op{Rep}_k(\mf g)$. Then $\im\varphi$ is a subrepresentation of $W$. As usual, we have a linear subspace, and it is fixed by the $G$-action because $X\in\mf g$ and $\varphi(v)\in\im\varphi$ has $X\cdot\varphi(v)=\varphi(X\cdot v)\in\im\varphi$.
\end{example}
Invariants provide an important example of subrepresentations.
\begin{definition}[invariants]
	Fix a regular Lie group $G$ or Lie algebra $\mf g$.
	\begin{itemize}
		\item We denote the $G$-invariants of a representation $V\in\op{Rep}_k(G)$ by
		\[V^G\coloneqq\{v\in V:\rho_V(g)v=v\text{ for all }g\in G\}.\]
		\item We denote the $\mf g$-invariants of a representation $V\in\op{Rep}_k(\mf g)$ by
		\[V^{\mf g}\coloneqq\{v\in V:\rho_V(g)v=0\text{ for all }X\in\mf g\}.\]
	\end{itemize}
\end{definition}
\begin{remark}
	We won't bother to check that invariants provide subrepresentations right now. It follows from the more general \Cref{lem:weight-decomposition}.
\end{remark}
\begin{example}
	Note that $\op{Hom}(V,W)^G=\op{Hom}_G(V,W)$ for $G$-representations $V$ and $W$. Indeed, a linear map $\varphi\in\op{Hom}(V,W)$ is fixed by the $G$-action if and only if
	\[g^{-1}\cdot\varphi(g\cdot v)=\left(g^{-1}\varphi\right)(v)=\varphi(v)\]
	for all $g\in G$, which of course rearranges into $\varphi$ being $G$-equivariant.
\end{example}
\begin{example}
	Note again note that $\op{Hom}(V,W)^{\mf g}=\op{Hom}_{\mf g}(V,W)$ for this Lie algebra representation structure. Namely, we can see that $X\cdot\varphi(v)=\varphi(X\cdot v)$ for any $X$ and $v$ if and only if $X\varphi=0$.
\end{example}
\begin{example}
	Let $V$ be a vector space, and fix a nonnegative integer $k\ge0$. Then $S_k$ acts on the tensor power $V^{\otimes k}$ by permuting the coordinates. Explicitly, permuting the coordinates provides a bilinear map $V^k\to V^{\otimes k}$, so it extends to a linear map $V^{\otimes k}\to V^{\otimes k}$ defined by
	\[\sigma\colon (v_1\otimes\cdots\otimes v_k)\mapsto(v_{\sigma1}\otimes\cdots\otimes v_{\sigma k})\]
	for any pure tensor. We won't bother to check that this is actually a group action, though it is not a lengthy check. The fixed points of this $S_k$-action is the symmetric power $\op{Sym}^k(V)$.
\end{example}
Here is a more general notion of invariants.
\begin{lemma} \label{lem:weight-decomposition}
	Fix a regular Lie group $G$ or Lie algebra $\mf g$, and let $X^*(G)$ and $X^*(\mf g)$ denote the set of regular homomorphisms $G\to\FF^\times$ and $\mf g\to\mf{gl}(\FF)$, respectively. For the statements, select $R\in\{G,\mf g\}$.
	\begin{listalph}
		\item Fix some $\chi\in X^*(R)$. For any representation $V$, the subspace
		\[V^\chi\coloneqq\{v\in V:\rho_V(r)v=\chi(r)v\text{ for all }r\in R\}\]
		is a subrepresentation of $V$.
		\item For distinct characters $\chi_1,\ldots,\chi_k\in X^*(R)$ and any representation $V$, the subspaces $V^{\chi_1},\ldots,V^{\chi_k}$ are linearly disjoint.
	\end{listalph}
\end{lemma}
\begin{proof}
	Here we go.
	\begin{listalph}
		\item Note that $V^\chi$ is the kernel of the family of linear maps $V\to V$ defined by $\{v\mapsto\rho_V(r)v-\chi(r)v\}_{r\in R}$, so $V^\chi$ is the intersection of linear subspaces and hence a linear subspace. To see that $V^\chi$ is preserved by the $G$-action, we note that any $v\in V^\chi$ and $r\in R$ will have $\rho_V(r)v\in V^\chi$: for any $s\in R$, we see
		\[\rho_V(s)\rho_V(r)=\chi(r)\rho_V(s)v=\chi(s)\chi(r)v=\chi(s)\rho_V(r)v.\]

		\item Suppose for the sake of contradiction that there exists a nontrivial relation $v_1+\cdots+v_k=0$ where $v_i\in V^{\chi_i}$ for $i\in\{1,\ldots,k\}$. By possibly making $k$ smaller, we may assume that all the $v_\bullet$s are nonzero, and in fact, we may assume that there does not exist such a relation with fewer than $k$ characters $\chi_1,\ldots,\chi_k\in X^*(R)$. Now, if $k=1$, then we are simply asserting that $v_1=0$, so there is nothing to say.

		Otherwise, we may assume that $k>1$. Then there is $r\in R$ such that $\chi_k(r)\ne\chi_1(r)$, and we see that multiplying our relation by $\rho_V(r)$ produces the equation
		\[\chi_1(r)v_1+\cdots+\chi_k(r)v_k=0.\]
		But now we can subtract this relation from $\chi_k(r)v_k+\cdots+\chi_k(r)v_k=0$, which produces a strictly smaller relation with at least one term $(\chi_1(r)-\chi_k(r))v$, which is a contradiction to the minimality of our relation.
		\qedhere
	\end{listalph}
\end{proof}
\begin{remark}
	If $G$ is a finite group acting on a vector space $V$, and $\chi$ is a character of $G$, then we can define an operator $\pi_\chi\colon V\to V$ by
	\[\pi_\chi(v)\coloneqq\frac1{\left|G\right|}\sum_{g\in G}\chi^{-1}(g)\rho(g)v.\]
	We have the following checks on $\pi_\chi$.
	\begin{itemize}
		\item Note $\pi_\chi$ is a linear map (as the sum of linear maps).
		\item By rearranging the sum, we see that $\rho_V(h)\pi_\chi(v)=\chi(h)\pi_\chi(v)$ for any $h\in G$, so $\im\pi_\chi\subseteq V^\chi$.
		\item On the other hand, if $v\in V^\chi$ already, then $\pi_\chi(v)$ is just a sum with $\left|G\right|$ copies of $v$, so $\pi_\chi$ fixes $V^\chi$ pointwise.
	\end{itemize}
	In conclusion, we see that $\im\pi_\chi=V^\chi$ by \Cref{ex:image-subrep}. This is an alternate way to see that $V^\chi$ is a subrepresentation.
\end{remark}
\begin{example}
	Suppose $V$ is a representation of a regular Lie group $G$ or Lie algebra $\mf g$. Given some nonnegative integer $k$, we recall that $S_k$ acts on $V$. Thus, for a character $\chi$ of $S_k$, we note that the map $\pi_\chi\colon V^{\otimes k}\to V^{\otimes k}$ is a projection. In fact, $\pi_\chi$ respects the ambient action on $V$.
	\begin{itemize}
		\item In the case of a Lie group $G$, we see that
		\[\rho_{V^{\otimes k}}(g)\pi_\chi(v_1\otimes\cdots\otimes v_k)=\frac1{k!}\sum_{\sigma\in S_k}\chi(\sigma)(\rho_V(g)v_{\sigma1}\otimes\cdots\otimes\rho_V(g)v_{\sigma k})=\pi_\chi\rho_{V^{\otimes k}}(g)(v_1\otimes\cdots\otimes v_k),\]
		so the equality $\rho_{V^{\otimes k}}(g)\circ\pi_\chi=\pi_\chi\circ\rho_{V^{\otimes k}}(g)$ follows by linearity.
		\item In the case of Lie algebra $\mf g$, we see that
		\begin{align*}
			\rho_{V^{\otimes k}}(X)\pi_\chi(v_1\otimes\cdots\otimes v_k) &= \frac1{k!}\sum_{\sigma\in S_k}\chi(\sigma)\rho_{V^{\otimes k}}(X)(v_{\sigma1}\otimes\cdots\otimes v_{\sigma k}) \\
			&= \frac1{k!}\sum_{\sigma\in S_k}\chi(\sigma)(Xv_{\sigma1}\otimes \otimes\cdots\otimes v_{\sigma k}+\cdots+v_{\sigma1}\otimes\cdots\otimes Xv_{\sigma k}) \\
			&= \pi_\chi\pi_{V^{\otimes k}}(g)(v_1\otimes\cdots\otimes v_k).
		\end{align*}
	\end{itemize}
	Thus, $\left(V^{\otimes k}\right)^\chi\subseteq V^{\otimes k}$ continues to be a subrepresentation in all cases by \Cref{ex:image-subrep}. When $\chi=1$, this is the symmetric power representation $\op{Sym}^k(V)$. When $\chi=\op{sgn}$, this is the alternating representation $\op{Alt}^k(V)$.
\end{example}
% With subrepresentations, we can take quotients in the usual way.
% \begin{lemma}
% 	Fix a Lie group $G$ and Lie algebra $\mf g$.
% 	\begin{enumerate}
% 		\item Given a morphism $\varphi\colon V\to W$ of representations $V,W\in\mathrm{Rep}_k(G)$, we can make $\coker\varphi$ into a representation of $G$ via the action
% 		\[\rho_{V/W}(g)(v+\im\varphi)\coloneqq\rho_V(v)+.\]
% 		\item Given a representation $V,W\in\mathrm{Rep}_k(\mf g)$, we can make $V/W$ into a representation of $G$ via the product rule action
% 		\[\rho_{V\otimes W}(X)(v\otimes w)=\rho_V(X)v\otimes w+v\otimes\rho_W(X)w.\]
% 		\item Suppose $\mf g=\op{Lie}(G)$. Given a representation $V,W\in\mathrm{Rep}_k(G)$, then $F(V\otimes W)$ is the tensor product representation in $\op{Rep}_k(\mf g)$.
% 	\end{enumerate}
% \end{lemma}

% \subsection{Symmetric and Alternating Powers}
% As another interesting operation that one can do on representations, we consider the symmetric and alternating powers.
% \begin{lemma}
% 	Fix a representation $V$ and a nonnegative integer $k\ge0$. Then the discrete group $S_k$ acts on $V^{\otimes k}$.
% \end{lemma}
% Now, for $k\ge2$, we see $S_k$ has two notable characters: one has the trivial character and the sign character. As such, we can have a subrepresentation of fixed by the trivial character, which is called $\op{Sym}^kV$; and we have a subrepresentation fixed by the sign character, which is called $\op{Alt}^kV$. One can check that these are all actually subrepresentations!

\subsection{Lie's Theorems for Representation Theory}
We now discuss how to pass the representation theory for $G$ to the representation theory of $\mf g$. We want the following lemma.
\begin{lemma} \label{lem:invariants-lie-group-algebra}
	Fix a regular Lie group $G$ with Lie algebra $\mf g$. For a representation $V\in\op{Rep}_k(G)$, give $V$ the natural $\mf g$-action via $d\rho$. Further, fix some character $\chi\colon G\to\op{GL}_1(\FF)$ inducing a character $d\chi\colon\mf g\to\mf{gl}(\FF)$.
	\begin{listalph}
		\item We always have $V^\chi\subseteq V^{d\chi}$.
		\item If $G$ is connected, then $V^\chi=V^{d\chi}$.
	\end{listalph}
\end{lemma}
\begin{proof}
	Quickly, we reduce to the case where $\chi=1$ and thus $d\chi=0$ (because $\chi$ is constant). By \Cref{ex:twist-by-character}, we may consider the representation $\chi^{-1}\rho$. On one hand, we see that $v\in V^\chi$ if and only if $\left(\chi^{-1}\rho\right)(g)v=v$ for all $g\in G$; on the other hand, we see similarly that $v\in V^{d\chi}$ if and only if $d\left(\chi^{-1}\rho\right)(X)v=v$ for all $X\in\mf g$. Thus, for our arguments, we will take $\chi=1$ so that we may consider $V^G$ and $V^{\mf g}$.
	\begin{listalph}
		\item For $v\in V^G$, we must show that $v\in V^{\mf g}$. Well, fix any $X\in\mf g$, and we would like to show that $d\rho_e(X)(v)=v$. For this, define the path $\gamma\colon\FF\to\mf g$ given by $\gamma(t)\coloneqq\exp(tX)$ so that $\gamma(0)=e$ and $\gamma'(0)=X$. Then
		\[d\rho_e(X)=d\rho_e(\gamma'(0))=(\rho\circ\gamma)'(0)\in\mf{gl}(V)\]
		by the chain rule, where this last derivative makes technical sense because we are outputting to a Euclidean space. To compute $(\rho\circ\gamma)'(0)(v)$, we note that applying an endomorphism in $\mf{gl}(V)$ to a vector $v\in V$ is a linear map, and linear maps pass through the definition of the derivative, so we find that
		\begin{align*}
			(\rho\circ\gamma)'(0)(v) &= \frac d{dt}(\rho\circ\gamma)(t)\bigg|_{t=0}(v) \\
			&= \frac d{dt}(\rho\circ\gamma)(t)(v)\bigg|_{t=0} \\
			&= \frac d{dt}\rho(\exp(tX))(v)\bigg|_{t=0} \\
			&\stackrel*= \frac d{dt}v\bigg|_{t=0} \\
			&= 0,
		\end{align*}
		where $\stackrel*=$ holds because $v\in V^G$.
		\item We already showed one inclusion in (a), so now we just have to show that any $v\in V^{\mf g}$ is fully fixed by $G$. Well, let $H\subseteq G$ be the subgroup of $G$ stabilizing $v$, which we know to be a closed Lie subgroup. In fact, by our more precise isomorphism theorem, we know that its Lie algebra $\mf h$ can be described by
		\[\mf h=\{X\in\mf g:(\rho_*X)_v=0\}.\]
		However, we can compute
		\[(\rho_*X)_vf=\frac d{dt}f(\exp(-tX)v)\bigg|_{t=0}=df_p((\rho\circ\exp)'(0)v)=df_p(d\rho_e(X)v)\]
		for any germ $f$, but this derivative is of course $0$ because $d\rho_e(X)v=0$ for all $X\in\mf g$ by assumption. Thus, $\mf h=\mf g$, so the exponential map $\exp\colon\mf h\to G$ will be a local diffeomorphism. In particular, $H$ contains in an open neighborhood of the identity, so $H$ must equal $G$ because $G$ is connected. Thus, $v\in V^G$.
		\qedhere
	\end{listalph}
\end{proof}
And here is our main result.
\begin{proposition} \label{prop:rep-theory-to-lie-alg}
	Fix a regular Lie group $G$ with Lie algebra $\mf g$. Recall the functor $F\colon\mathrm{Rep}(G)\to\mathrm{Rep}(\mf g)$ sending a representation $\rho\colon G\to\op{GL}(V)$ to $d\rho_e\colon\mf g\to\mf{gl}(V)$.
	\begin{listalph}
		\item If $G$ is connected, then $F$ is fully faithful.
		\item If $G$ is connected and simply connected, then $F$ is essentially surjective and hence an equivalence.
	\end{listalph}
\end{proposition}
\begin{proof}
	Here we go.
	\begin{listalph}
		\item Suppose that $G$ is connected, and we want to show that $F$ is fully faithful. In \Cref{lem:rep-theory-to-lie-alg}, we showed that $F$ is faithful, so we now must show that $F$ is full. Well, for $G$-representations $V$ and $W$, we must show that any linear map $\varphi\in\op{Hom}_{\mf g}(V,W)$ is in fact $G$-invariant. Well, we simply note that
		\[\op{Hom}_{\mf g}(V,W)=\op{Hom}(V,W)^{\mf g},\]
		which equals $\op{Hom}(V,W)^G=\op{Hom}_G(V,W)$ by \Cref{lem:invariants-lie-group-algebra}, so we are done.

		\item Suppose that $G$ is connected and simply connected. We want to show that $F$ is essentially surjective in order to finish the proof that $F$ is an equivalence of categories. Well, fix a representation of $\mf g$ given by some Lie algebra homomorphism $\ov\rho\colon\mf g\to\mf{gl}(V)$. Then \Cref{thm:lie-2} tells us that the differential provides a bijection
		\[F\colon\op{Hom}_{\mathrm{LieGrp}}(G,\op{GL}(V))\cong\op{Hom}_{\mathrm{LieAlg}}(\mf g,\mf{gl}(V))\]
		because $G$ is simply connected. In particular, there is a Lie algebra homomorphism $\rho\colon G\to\op{GL}(V)$ such that $\ov\rho=F\rho$, as required.
		\qedhere
	\end{listalph}
\end{proof}
\begin{remark}
	Given any connected Lie group $G$ with universal cover $\widetilde G$, one can attempt to recover the representation theory of $G$ from $\widetilde G$ via the short exact sequence in \Cref{rem:universal-cover-ses}.
\end{remark}
\begin{remark} \label{rem:rep-theory-complexification}
	For any Lie algebra $\mf g$ over $\RR$, one sees that $\op{Rep}_\CC(\mf g)=\op{Rep}_\CC(\mf g_\CC)$. (Another perspective is that we can reduce the complex representation theory of a complex Lie algebra to a real form.) To see this, note that any morphism $\mf g\to\mf{gl}(V)$ where $V$ is a complex vector space canonically upgrades to a map $\mf g\otimes_\RR\CC\to\mf{gl}(V)$ by taking the tensor product with the canonical inclusion $\CC\to\mf{gl}(V)$ given by $c\mapsto c\id_V$. In the reverse direction, any representation $\mf g\otimes_\RR\CC\to\mf{gl}(V)$ can simply forget about the $\CC$ factor to define a representation $\mf g\to\mf{gl}(V)$. This remark does not have the space or motivation to check that we have actually defined an equivalence.
\end{remark}

\subsection{Decomposing Representations}
With direct sums, we have notions of irreducibility.
\begin{definition}[indecomposable]
	Fix a representation $V$. Then $V$ is \textit{indecomposable} if and only if any direct sum decomposition $V=V_1\oplus V_2$ must have $V_1=0$ or $V_2=0$.
\end{definition}
\begin{definition}[irreducible]
	Fix a representation $V$. Then $V$ is \textit{irreducible} if and only if $V$ is nonzero, and any subrepresentation $U\subseteq V$ has $U=0$ or $U=V$.
\end{definition}
\begin{example}
	The standard representation $V$ of $\op{GL}(V)$ is irreducible. Indeed, any nonzero subrepresentation $U\subseteq V$ has a nonzero vector $v\in U$. But then the orbit of $v$ under $\op{GL}(V)$ is $V\setminus\{0\}$, so $U$ must contain $V\setminus\{0\}$, so $U=V$.

	It will also turn out that $\op{Sym}^k(V)$ and $\op{Alt}^k(V)$ are irreducible representations of $\op{GL}(V)$, but this is not so obvious. We will be able to show this with more ease later in the course.
\end{example}
These notions are related but not the same.
\begin{remark}
	Any irreducible representation $V$ is indecomposable. Indeed, writing $V=V_1\oplus V_2$ has $V_1\subseteq V$, so $V_1=0$ or $V_1=V$.
\end{remark}
\begin{example}
	Consider the representation $\rho\colon\CC\to\op{GL}_2(\CC)$ given by
	\[\rho(x)\coloneqq\begin{bmatrix}
		1 & x \\ 0 & 1
	\end{bmatrix}.\]
	Then $\op{span}\{e_1\}$ is a nontrivial proper subrepresentation of $\rho$ because $\rho(e_1)=e_1$; thus, $\rho$ fails to be irreducible.
	
	However, $\rho$ is indecomposable! Indeed, our vector space is two-dimensional, so a nontrivial decomposition of $\rho$ into $\rho_1\oplus\rho_2$ must have underlying vector spaces $V_1$ and $V_2$ with dimensions $\dim V_1=\dim V_2=1$. But the action of $\CC$ on $\CC$ must be linear, so $V_1$ and $V_2$ must be eigenspaces. As such, we can see from the definition of $\rho$ that all eigenvalues are $1$, so $\rho_1$ and $\rho_2$ would have to be the trivial representation, meaning that $\rho$ would have to be the sum of trivial representations and hence trivial, which is false because $\rho(1)\ne\id$.
\end{example}
Do note that there is something that we can always do for our decomposition, but it is not always as satisfying as a direct sum.
\begin{remark}[Jordan--Holder]
	For any representation $V$, one can always find a filtration
	\[0=V_0\subseteq V_1\subseteq\cdots\subseteq V_{k-1}\subseteq V_k=V\]
	where each quotient $V_i/V_{i-1}$ is irreducible. Indeed, we can proceed by induction on $\dim V$. As a base case, $\dim V=1$ has nothing to do because $V$ is irreducible for dimension reasons.
	
	If $V$ is already irreducible, then our filtration is $0\subseteq V$. Otherwise, $V$ is not irreducible, so we can find a nontrivial proper subrepresentation $V'\subseteq V$; choosing a minimal such representation (by dimension) must have $V'$ irreducible. Then we can apply the inductive hypothesis to $V/V'$ (which has smaller dimension than $V$) to build the required filtration.
\end{remark}
In particular, filtrations means that we would have to build representations by short exact sequences, which may be difficult to handle especially when iterated.

We would like to decompose representations into irreducible parts because dealing with filtrations is difficult.
\begin{definition}[completely reducible]
	A representation $V$ is \textit{completely reducible} if and only if it is the direct sum of irreducible representations.
\end{definition}
\begin{remark}
	Technically, we have not required that the decomposition into irreducibles is unique. This is the content of \Cref{cor:unique-irrep-decomp}.
\end{remark}

\end{document}