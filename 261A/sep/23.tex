% !TEX root = ../notes.tex

\documentclass[../notes.tex]{subfiles}

\begin{document}

\section{September 23}
Today we continue talking about vector fields.

\subsection{Vector Fields on Lie Groups}
Let's return to Lie groups.
\begin{lemma} \label{lem:left-invariant-by-germ}
	Fix a regular Lie group $G$. A vector field $\xi$ on $G$ is left-invariant if and only if
	\[\xi(f\circ L_g)=\xi f\circ L_g\]
	for any germ $f$ defined in a neighborhood of $g$.
\end{lemma}
\begin{proof}
	We show the two implications separately.
	\begin{itemize}
		\item If $\xi$ is left-invariant, then $\xi_{gh}=(dL_g)_h(\xi_h)$ for any $g,h\in G$. Thus, for any $h\in G$, we see that
		\begin{align*}
			(\xi f\circ L_g)(h) &= \xi_{gh}f \\
			&= ((dL_g)_h\xi_h)f \\
			&= \xi_h(f\circ L_g),
		\end{align*}
		as required.
		\item Suppose $\xi(f\circ L_g)=\xi f\circ L_g$ for any $f$. Then plugging in the identity tells us that
		\[\xi_gf=(\xi f\circ L_g)(e)=\xi_e(f\circ L_g)=((dL_g)_e(\xi_e))(f).\]
		Thus, $\xi_g=(dL_g)_e\xi_e$, as required.
		\qedhere
	\end{itemize}
\end{proof}
\begin{lemma} \label{lem:vector-field-lie-group-action}
	Fix a left-invariant vector field $\xi$ on a regular Lie group $G$. Then for a germ $f$ at a point $g\in G$, one has
	\[\xi_g f=\frac d{dt}f(g\exp(t\xi_e))\bigg|_{t=0}.\]
\end{lemma}
\begin{proof}
	This is more or less the chain rule. For our $g\in G$, \Cref{lem:left-invariant-by-germ} tells us that
	\[\xi_gf=\xi_e(f\circ L_g).\]
	Now, the path $\gamma\colon\FF\to G$ given by $\gamma(t)\coloneqq\exp(t\xi_e)$ has $\gamma'(0)=\xi_e$, so
	\[\xi_e(f\circ L_g)=d(f\circ L_g\circ\gamma)'(0)=\frac d{dt}f(g\exp(t\xi_e))\bigg|_{t=0},\]
	as required.
\end{proof}
\begin{proposition}
	Fix a regular Lie group $G$ with Lie algebra $\mf g$. Then the collection of left-invariant vector fields $\op{Vect}^L(G)$ is a Lie subalgebra of $\op{Vect}(G)$ which is isomorphic to $\mf g$.
\end{proposition}
\begin{proof}
	By \Cref{rem:lie-algebra-to-invariant}, one certainly has an isomorphism $\op{Vect}^L(G)\to\mf g$ given by $\xi\mapsto\xi_e$, with inverse given by $X\mapsto\xi_X$, where $\xi_X$ is the vector field $\xi_X(g)\coloneqq dL_g(X)$. Now, by \Cref{lem:left-invariant-by-germ}, $\xi$ is left-invariant if and only if
	\[\xi(f\circ L_g)=\xi f\circ L_g\]
	for any germ $f$ defined in a neighborhood of $g$. Thus, we see that $\op{Vect}^L(G)$ is preserved by the commutator of $\op{Vect}(G)$.
	
	It remains to check that our isomorphism with $\mf g$ is a morphism of Lie algebras. Fix $X,Y\in\mf g$, and we would like to show that $[\xi_X,\xi_Y]=\xi_{[X,Y]}$. It is enough to check this equality after mapping back down to $\mf g$, so we want to check that $[\xi_X,\xi_Y]_e=[X,Y]$. This is a direct computation: by \Cref{lem:vector-field-lie-group-action}, any germ $f$ at $e$ has
	\begin{align*}
		[\xi_X,\xi_Y]_ef &= \frac d{dt}\big(\xi_Yf(\exp(tX))-\xi_Xf(\exp(tY))\big)\bigg|_{t=0} \\
		&= \frac{\del^2}{\del s\del t}\frac d{ds}\big(f(\exp(tX)\exp(sY))-f(\exp(tY)\exp(sX))\big)\bigg|_{(s,t)=(0,0)} \\
		&= \frac{\del^2}{\del s\del t}\bigg(f\exp\left(tX+sY+\frac12st[X,Y]+\cdots\right)-f\exp\left(tX+sY-\frac12st[X,Y]+\cdots\right)\bigg)\bigg|_{(s,t)=(0,0)}.
	\end{align*}
	Now, one can imagine taking a Taylor series expansion of $f\circ\exp\colon\mf g\to\RR$ in terms of $Z$, in which we see that the above derivative will only depend on the $st$ term of the relevant expansion. More precisely, write $(f\circ\exp)(Z)=f(e)+\lambda(Z)+Q(Z)+C(Z)$, where $\lambda$ is linear, $Q$ is quadratic, and $C$ has vanishing first- and second-order derivatives. Then, after cancellation within $\lambda$, we see that
	\begin{align*}
		[\xi_X,\xi_Y]_ef &= \frac{\del^2}{\del s\del t}st\lambda(st[X,Y])\bigg|_{(s,t)=(0,0)} \\
		&\qquad+\frac{\del^2}{\del s\del t}Q\left(tX+sY+\frac12st[X,Y]+\cdots\right)\bigg|_{(s,t)=(0,0)} \\
		&\qquad+\frac{\del^2}{\del s\del t}Q\left(tX+sY-\frac12st[X,Y]+\cdots\right)+\cdots\bigg|_{(s,t)=(0,0)},
	\end{align*}
	where $+\cdots$ denotes higher-order terms which will not affect the current derivative (for example, containing $C$). Now, the linear terms inside $Q$ will produce cancelling terms after expansion, so the only term we are left to care about is
	\[[\xi_X,\xi_Y]_ef=\lambda([X,Y])=\frac d{dt}(f\circ\exp)(t[X,Y])\bigg|_{t=0}=(\xi_{[X,Y]})_ef,\]
	as required.
\end{proof}

\subsection{Group Actions via Lie Algebras}
In general, if $G$ acts on a regular manifold $M$ via the action $a\colon G\times X\to X$, one can define an action of $\mf g$ on $\op{Vect}(X)$ by analogy with \Cref{lem:vector-field-lie-group-action}.
\begin{definition}
	Fix a left action $a\colon G\times M\to M$ of a regular Lie group $G$ on a regular manifold $M$. Then we define $a_*\colon\mf g\to\op{Vect}(M)$ by
	\[(a_*X)_pf\coloneqq\frac d{dt}f(a(\exp(-tX),p))\bigg|_{t=0}\]
	for any $p\in M$ and germ $f$ at $p$.
\end{definition}
\begin{remark}
	Let's explain the sign in the above definition: the action of $G$ on $M$ induces a natural action of $G$ on the regular functions $\OO(M)$ by $(g\cdot f)(p)\coloneqq f\left(g^{-1}\cdot p\right)$. It is this action of $G$ on $\OO(M)$ which motivates the above definition.
\end{remark}
Let's run our checks on this definition.
\begin{lemma}
	Let $a\colon G\times M\to M$ be an action of a regular Lie group $G$ on a regular manifold $M$. Then that $a_*$ is a homomorphism of Lie algebras.
\end{lemma}
\begin{proof}
	We run our many checks in sequence. Throughout, $p,q\in M$ and $X,Y\in\mf g$ and $s,t\in\FF$ and $f$ and $g$ are regular functions on an open neighborhood of $p$.
	\begin{enumerate}
		\item For any regular function $f$ defined in an open neighborhood of a point $p\in M$, we claim that
		\[da_{(e,p)}(-X,0)(f_p)\stackrel?=(a_*X)_p(f).\]
		This is a matter of computation. Define $\gamma\colon\FF\to G\times M$ by $\gamma(t)\coloneqq(\exp(-tX),p)$. Then we see that $\gamma'(0)=(-X,0)$ by definition of $\exp$. Thus, using the chain rule, we see that
		\begin{align*}
			da_{(e,p)}(-X,0)(f_p) &= da_{(e,p)}(-X,0)(f_p) \\
			&= d(f\circ a)_{(e,p)}(-X,0) \\
			&= d(f\circ a)_{(e,p)}(\gamma'(0)) \\
			&= (f\circ a\circ\gamma)'(0) \\
			&= \frac d{dt}(f\circ a\circ\gamma)(t)\bigg|_{t=0} \\
			&= \frac d{dt}f(a(\exp(-tX),p))\bigg|_{t=0} \\
			&= (a_*X)_p(f),
		\end{align*}
		as required.

		\item We check that $(a_*X)_p$ is a derivation $T_pM$. This follows essentially immediately from the previous step. We enumerate the checks for clarity.
		\begin{itemize}
			\item Note that $(a_*X)_p(f)$ only depends on the germ $f_p$ because it equals $da_{(e,p)}(-X,0)(f)$, and
			\[da_{(e,p)}(-X,0)\in T_pM\]
			only depends on the germ $f_p$. Thus, we may redefine $(a_*X)_p$ as taking germs as input.\footnote{One can also check this directly: regular local functions $f$ and $g$ with $f_p=g_p$ has $f_p-g_p$ vanish in a neighborhood of $p$, permitting us to compute $(a_*X)_p(f)=(a_*X)_p(g)$.}
			\item Now, we see that $(a_*X)_p$ is a function taking input as germs at $p$ and outputting elements of $\FF$; in particular, it equals the differential $da_{(e,p)}(-X,0)$, so $(a_*X)_p$ immediately becomes a linear map and satisfies the product rule, making it a derivation.
		\end{itemize}

		\item We check that $a_*X$ is a vector field. Thus, far we know that we have $(a_*X)_p\in T_pM$ for each $p\in M$, so we have a section $a_*X\colon M\to TM$. It remains to check that $a_*X$ is smooth. Well, the first step tells us that $(a_*X)_p=da_{(e,p)}(-X,0)$, so we see that $a_*X$ equals the composite
		\[\arraycolsep=1.4pt\begin{array}{cccccccccccc}
			M &\to& TM &\to& TG\times TM &\simeq& T(G\times M) &\stackrel{da}\to& TM \\
			p &\mapsto& (p,0) &\mapsto& ((e,-X),(p,0)) &\mapsto& ((e,p),(-X,0)) &\mapsto& da_{(e,p)}(-X,0)
		\end{array}\]
		of smooth maps, so $a_*X\colon M\to TM$ is smooth.

		% \item We check that $(a_*X)_p$ depends only on the germ $f_p$. Indeed, suppose that $f$ and $g$ are functions on a neighborhood of $p$ with $f_p=g_p$. Then $f-g$ vanishes in an open neighborhood $U$ of $p$; note that the function $t\mapsto a(\exp(-tX),p)$ is continuous because it is a composite of regular maps, so we have some $\varepsilon>0$ such that $\left|t\right|<\varepsilon$ such that $t\mapsto a(\exp(-tX),p)$ outputs to $U$. (Note that this also verifies that the definition of $(a_*X)_p(f)$ makes sense for functions $f$ only defined on open neighborhoods of the point $p$.) Thus,
		% \begin{align*}
		% 	(a_*X)_p(f)-(a_*X)_p(g) &= \frac d{dt}f(a(\exp(-tX),p))\bigg|_{t=0}-\frac d{dt}g(a(\exp(-tX),p))\bigg|_{t=0} \\
		% 	&= \frac d{dt}(f-g)(a(\exp(-tX),p))\bigg|_{t=0} \\
		% 	&= 0
		% \end{align*}
		% because our composite vanishes in an open neighborhood of $0$. Thus, $(a_*X)_p(f)=(a_*X)_p(g)$, as required.

		% \item We check that $(a_*X)_p$ is $\FF$-linear. Choose $c,d\in\FF$ and germs $f_p$ and $g_p$ with representative functions $f$ and $g$ (respectively). Then we compute
		% \begin{align*}
		% 	(a_*X)_p(cf+dg) &= \frac d{dt}(cf+dg)(a(\exp(-tX),p))\bigg|_{t=0} \\
		% 	&= c\frac d{dt}f(a(\exp(-tX),p))\bigg|_{t=0}+d\frac d{dt}g(a(\exp(-tX),p))\bigg|_{t=0} \\
		% 	&= c(a_*X)_p(f)+d(a_*X)_p(g),
		% \end{align*}
		% as required.

		% \item We check that $(a_*X)_p$ is a derivation in $T_pM$. For this, it remains to check the Leibniz rule, so choose germs $f_p$ and $g_p$ with representative functions $f$ and $g$, and set $b(t)\coloneqq a(\exp(-tX),p)$ to be a regular function $\FF\to M$. Then we can simply re-prove the product rule by writing
		% \begin{align*}
		% 	(a_*X)_p(fg) &= \frac d{dt}(fg)(a(\exp(-tX),p))\bigg|_{t=0} \\
		% 	&= \lim_{t\to0}\frac{f(b(t))g(b(t))-f(p)g(p)}t \\
		% 	&= \lim_{t\to0}\left(f(b(t))\cdot\frac{g(b(t))-g(p)}t+g(p)\cdot\frac{f(b(t))-f(p)}{t}\right) \\
		% 	&= f(p)\frac d{dt}g(b(t))\bigg|_{t=0}+g(p)\frac d{dt}f(b(t))\bigg|_{t=0} \\
		% 	&= f(p)(a_*X)(g)+g(p)(a_*X)(f),
		% \end{align*}
		% as required.

		% \item We check that $a_*X$ is a regular vector field. Thus far, we know that we have a well-defined section $M\to TM$ given by $p\mapsto(a_*X)_p\in T_pM$. It remains to check that the section is regular. Regularity can be checked locally around each point $p\in M$, so we choose some regular chart $(U,\varphi)$ around $p$, and we use $\varphi$ to build coordinate functions $x_1,\ldots,x_n\colon U\to\FF$. Then these standard coordinates provide a basis for each $T_qM$ for $q\in U$, so we know that there are functions $(a_*X)_1,\ldots,(a_*X)_n\colon M\to\FF$
		% \[(a_*X)_q=\sum_{k=1}^n(a_*X)_k(q)\frac{\del}{\del x_k}\bigg|_q.\]
		% To check that $(a_*X)_q$ is regular on $U$, it thus suffices to check that the functions $(a_*X)_k$ are regular. In particular, plugging in the germ $(x_k)_q$ defined on $U$, we see
		% \[(a_*X)_q(x_k)=(a_*X)_k(q).\]
		% Thus, it suffices to check the map $U\to\FF$ defined by $q\mapsto(a_*X)_q(f)$ is regular for a function $f$ defined in an open neighborhood of $p$. For this, we note that the function $\FF\times M\to\FF$ defined by $(t,q)\mapsto f(a(\exp(-tX),q))$ is a composite of regular maps and hence regular. Thus, taking the derivative with respect to $t$, we see that $q\mapsto(a_*X)_q(f)$ is still regular, as required.

		\item Thus far, we know that we have a well-defined map $a_*\colon\mf g\to\op{Vect}(X)$. It remains to check that this is a homomorphism of Lie algebras. We begin by checking that it is $\FF$-linear. Well, for $X,Y\in\mf g$ and $c,d\in\FF$, we are asking to check that $a_*(cX+dY)=ca_*X+da_*Y$. For this, we check the equality of derivations at some point $p\in M$, for which the first step verifies
		\begin{align*}
			a_*(cX+dY)_p &= da_{(e,p)}(-cX-dY,0) \\
			&= c\cdot da_{(e,p)}(-X,0)+d\cdot da_{(e,p)}(-Y,0) \\
			&= (ca_*X+da_*Y)_p,
		\end{align*}
		as required.
		
		% To make our lives easier, we show homogeneity and additivity separately.

		% For homogeneity, choose $X\in\mf g$ and $c\in\RR$, and we want to check that $a_*(cX)=ca_*(X)$. For this, choose $p\in M$ and a germ $f_p$ represented by some function $f$ defined in an open neighborhood of $p$, and we note
		% \[(a_*(cX))_p(f)=\frac d{dt}f(a(\exp(-tcX),p))\bigg|_{t=0}.\]
		% Simply by applying the chain rule (from single-variable calculus), we see that
		% \[\frac d{dt}f(a(\exp(-tcX),p))\bigg|_{t=0}=\frac d{d(ct)}f(a(\exp(-tcX),p))\bigg|_{ct=0}\cdot\frac{d(ct)}{dt}\bigg|_{t=0}=c\cdot a_*(X)_p(f),\]
		% as required.

		% For additivity, choose $X,Y\in\mf g$, and we want to check that $a_*(X+Y)=a_*X+a_*Y$. Well, choose $p\in M$ and a germ $f_p$ represented by some function $f$ defined in an open neighborhood of $p$.\todo{}
		
		\item We check that $a_*$ is a homomorphism of Lie algebras. We already know that $a_*\colon\mf g\to\op{Vect}(M)$ is linear (and we know that everything in sight is a Lie algebra from class), so it only remains to check that $a_*$ preserves the Lie bracket. Explicitly, we would like to show that $a_*[X,Y]=[a_*X,a_*Y]$ for given $X,Y\in\mf g$. For this, we choose a germ $f_p$ represented by a regular function $f$ defined in an open neighborhood of $p$. %We may further suppose that $f$ is analytic: in fact, to test the equality $a_*[X,Y]_pf=[a_*X,a_*Y]_pf$, we could merely assume that $f$ is a coordinate function in a neighborhood of $p$.
		
		To run our computations, we employ a trick motivated by one in the Etingof book. Namely, define $F\colon\mf g\to\FF$ by $F(Z)\coloneqq f(a(\exp(Z),p))$. Now, we compute
		\begin{align*}
			(a_*X)_p(a_*Yf) &= \frac d{dt}(a_*Yf)(a(\exp(-tX),p))\bigg|_{t=0} \\
			&= \frac d{dt}\frac d{ds}f\big(a(\exp(-sY),a(\exp(-tX),p))\big)\bigg|_{s=0}\bigg|_{t=0} \\
			&= \frac d{dt}\frac d{ds}f\big(a(\exp(-sY)\exp(-tX),p)\big)\bigg|_{s=0}\bigg|_{t=0} \\
			&= \frac d{dt}\frac d{ds}f\left(a\left(\exp\left(-sY-tX+\frac12st[Y,X]+\cdots\right),p\right)\right)\bigg|_{s=0}\bigg|_{t=0} \\
			&= \frac {\del^2}{\del s\del t}F\left(-sY-tX+\frac12st[Y,X]+\cdots\right)\bigg|_{(s,t)=(0,0)} \\
			&= \frac {\del^2}{\del s\del t}F\left(-tX-sY-\frac12st[X,Y]+\cdots\right)\bigg|_{(s,t)=(0,0)}.
		\end{align*}
		By reversing the roles of $X$ and $Y$ in the above argument, we see that
		\[(a_*Y)_p(a_*Xf)=\frac {\del^2}{\del s\del t}F\left(-tX-sY+\frac12st[X,Y]+\cdots\right)\bigg|_{(s,t)=(0,0)}.\]
		Thus, we see that we want to compute some particular derivatives of $F$. Now, $f$ is regular, so $F$ is a regular function $\mf g\to\FF$, so it will be approximately equal its Taylor expansion in a neighborhood of $0$ as
		\[F(Z)=f(0)+\lambda(Z)+Q(Z)+\cdots,\]
		where $\lambda$ is a linear functional, $Q$ is a quadratic form, and $+\cdots$ refers to higher-order terms (with vanishing first- and second-derivatives). Plugging everything in and expanding, we see that
		\begin{align*}
			(a_*X)_p(a_*Yf)-(a_*Y)_p(a_*Xf) &= \frac{\del^2}{\del s\del t}-st\lambda([X,Y])\bigg|_{(s,t)=(0,0)} \\
			&\qquad+\frac{\del^2}{\del s\del t}Q\left(-tX-sY+\frac12st[X,Y]+\cdots\right)\bigg|_{(s,t)=(0,0)} \\
			&\qquad-\frac{\del^2}{\del s\del t}Q\left(-tX-sY-\frac12st[X,Y]+\cdots\right)+\cdots\bigg|_{(s,t)=(0,0)},
		\end{align*}
		where $+\cdots$ continues to denote higher-order terms, but now we see that we are only going to care about $-\lambda([X,Y])$ when computing $\frac\del{\del s\del t}\big|_{(s,t)=(0,0)}$. (Notably, the last two terms cancel out as a derivative of $Q(-tX-sY+\cdots)-Q(-tX-sY+\cdots)$.) But now we see that
		\[(a_*X)_p(a_*Yf)-(a_*Y)_p(a_*Xf)=\lambda(-[X,Y])=\frac d{dt}F(-t[X,Y])\bigg|_{t=0}=a_*[X,Y]_pf,\]
		as required.
		\qedhere
	\end{enumerate}
\end{proof}
We can now prove the Orbit--stabilizer theorem (\Cref{thm:orbit-stabilizer-lie-algebra}) in the following more precise form.
\begin{theorem}[Orbit--stabilizer] \label{thm:orbit-stabilizer-lie-algebra}
	Fix a left action $a\colon G\times M\to M$ of a regular Lie group $G$ on a regular manifold $M$. Fix some $p\in M$.
	\begin{listalph}
		\item For all $p\in M$, the stabilizer $G_p$ is a closed Lie subgroup with Lie algebra
		\[\op{Lie}G_p=\{X\in\mf g:(a_*X)_p=0\}.\]
		\item The induced map $G/G_p\to M$ given by $g\mapsto g\cdot p$ is an injective immersion. In particular, the orbit $Go$ is an immersed submanifold.
		\item If the induced map $G/G_p\to M$ is an embedding, then $G/G_p$ is diffeomorphic to $Gp$.
	\end{listalph}
\end{theorem}
\begin{proof}
	We begin with the proof of (a), which we do in steps.
	\begin{enumerate}
		\item Set
		\[\mf g_p\coloneqq\{X\in\mf g:(a_*X)_p=0\}\]
		for brevity. We claim that $\mf g_p\subseteq\mf g$ is a Lie subalgebra. Certainly $X\mapsto (a_*X)_p$ is a linear map $\mf g\to\op{Vect}(M)\to T_pM$, so $\mf g_p$ is a linear subspace.
		
		It remains to check that $\mf g_p$ is preserved by the bracket. Fix $X,Y\in\mf g_p$, and we want to check $[X,Y]\in\mf g_p$. Well, because $a_*$ is a homomorphism of Lie algebras, we see
		\[a_*[X,Y]_pf=\underbrace{(a_*X)_p}_0(a_*Yf)-\underbrace{(a_*Y)_p}_0(a_*Xf)=0\]
		for any germ $f$ at $p$. Thus, $a_*[X,Y]=0$, so $[X,Y]\in\mf g_p$.

		\item For $X\in\mf g_p$, we check that $\exp(X)\in G_p$. Indeed, we claim the two curves $\gamma_1(t)\coloneqq \exp(-tX)\cdot p$ and $\gamma_2(t)\coloneqq p$ are both integral curves for $a_*X$ with the same initial condition at $0$. This completes the check because it implies that $\exp(X)\cdot p=\gamma_1(-1)=\gamma_2(-1)=p$ by uniqueness of integral curves.
		
		To prove the claim, we note that $\gamma_2$ is constant, so there is nothing to check there. For $\gamma_1$, we must check that
		\[\gamma_2'(t)\stackrel?=(a_*X)_{\gamma_2(t)}\]
		in $T_{\gamma_2(t)}M$. To check this, we pass through an arbitrary germ $f$ to see that
		\[\gamma_2'(t)f=(f\circ\gamma_2)'(t)=\frac d{ds}f(\exp(-sX-tX)\cdot p)\bigg|_{s=0},\]
		and
		\[(a_*X)_{\gamma_2(t)}f=\frac d{ds}f(\exp(-tX-sX)\cdot p)\bigg|_{s=0},\]
		as required.
		
		\item We attempt to control $\mf g/\mf g_p$. Choose a complement $\mf u$ of $\mf g_p\subseteq\mf g$ so that $\mf g=\mf g_p\oplus\mf u$. (We do not require that $\mf u$ is a Lie subalgebra, despite the font.) Then the map $f\colon\mf u\to T_pM$ given by $Z\mapsto(a_*Z)_p$ has kernel $\mf g_p\cap\mf u=0$ and hence is injective. Thus, the Implicit function theorem tells us that the map $F\colon\mf u\to M$ given by $v\mapsto\exp(-V)\cdot p$ must be an injective immersion for small $v$ because $df_p(V)=dF_p(V)$.
		
		Instead of using the Implicit function theorem, we can argue using local diffeomorphisms as follows: fix a basis $\{e_1,\ldots,e_k\}$ of $\mf u$, and extend the linearly independent set $\{dF_p(e_1),\ldots,df_p(e_k)\}\subseteq T_pM$ to a basis $\{dF_p(e_1),\ldots,dF_p(e_k)\}\sqcup\{e_{k+1}',\ldots,e_m'\}$. Then define a local map $\widetilde F\colon\mf u\times\FF^{m-k}\to M$ by
		\[\widetilde F(a_1e_1+\cdots+a_me_m)=F(a_1e_1+\cdots+a_ke_k)+a_{k+1}e'_{k+1}+\cdots+a_me_m',\]
		where the addition on the right-hand side is defined in a local chart of $M$ around $p$. (Technically, $\widetilde F$ is only defined in a neighborhood of $0\in\mf u$.) Then $\widetilde F$ is a local diffeomorphism at $0$ by construction, so $F$ is an injective immersion in this same neighborhood of $0$.

		\item We construct a slice chart for $G_p\subseteq G$ at the identity, which will complete the proof (a) by \Cref{lem:one-slice-subgroup}. Note that the map $\exp^\oplus\colon\mf g_p\oplus\mf u\to G$ given by $(V,X)\mapsto\exp(V)\exp(X)$ is a local diffeomorphism at $0$ (because the differential is simply the identity by checking what happens on each piece $\mf g_p$ and $\mf u$ separately). Thus, for $g\in G$ sufficiently close to $e$, we can write $g$ uniquely as in the image of $e$ and thus as $g=\exp(V)\exp(X)$ where $V\in\mf u$ and $X\in\mf g_p$. Now, we see that $g\in G_p$ if and only if $\exp(V)\in G_p$, which for small enough $V$ is equivalent to $V\in\mf g_p$ by the previous step.

		In total, we have constructed a very small open neighborhood $U\subseteq\mf g_p\oplus\mf u$ of the identity such that $e|_U$ is a diffeomorphism onto its image $\exp^\oplus(U)\subseteq G$ and
		\[G_p\cap\exp^\oplus(U)=\{(V,X)\in\mf g_p\oplus\mf u:V=0\},\]
		which is a slice chart.

		% \item Now, one sees that $\mf u\times\mf g_p\to G$ given by the exponential is a local diffeomorphism. Thus, for a small open neighborhood $U$ of $G$, one finds that $g\in G_x$ can be written uniquely as $\exp(v)\exp(Z)$, so $g\in G_x$ if and only if $v=0$, which provides a slice chart. Thus, $G_p$ is an embedded submanifold with the described Lie subalgebra.
		
		% \item For any $Z\in\mf g_p$, we claim that $\exp(-tZ)p=p$ for all $t$. Indeed, both sides are integral curves for $Z$, so the uniqueness of integral curves completes the proof.
	\end{enumerate}
	We now proceed with (b). Let $\ov\varphi$ denote the induced map $G/G_p\to M$ given by $\ov\varphi(g)\coloneqq g\cdot p$, which we want to see is an injective immersion. Injectivity follows by definition of $G_p$: if $\varphi(g_1)=\varphi(g_2)$, then $g_1\cdot p=g_2\cdot p$, so $g_1^{-1}g_2\in G_p$, so $g_1G_p=g_2G_p$. Being an immersion more or less follows from the proof. By translation, it suffices to show that $d\ov\varphi_e$ is injective.\footnote{Once $d\ov\varphi_e$ is injective, we note that $\ov\varphi\circ L_g=L_g\circ\ov\varphi$ (where the first $L_g$ is a map $G\to G$ and the second is a map $M\to M$, but both are diffeomorphisms), so $d\ov\varphi_g\circ d(L_g)_e=d(L_g)_{p}\circ d\ov\varphi_e$ verifies that $d\ov\varphi_g$ is injective.} Well, the Lie algebra of $G/G_p$ is the quotient $\mf g/\mf g_p$ by \Cref{thm:quotient-group}, which is isomorphic to $\mf u$ by construction of $\mf u$. But we know that the action map is injective on $\mf u$ by the third step above, so we are done.

	Lastly, we note that (c) follows immediately from (b) because embeddings are diffeomorphic onto their images by the uniqueness of the smooth structure of embedded submanifolds.
\end{proof}
\begin{remark} \label{rem:isomorphism-theorem}
	We also remark that \Cref{thm:isomorphism} follows quickly from the above result. Indeed, let $G$ act on $H$ via the homomorphism $\varphi\colon G\to H$: $g\cdot h\coloneqq\varphi(g)h$. Then the stabilizer of any $h\in H$ is given by $\ker\varphi$, proving $\ker\varphi$ is in fact a closed Lie subgroup. Now, passing to $\ov\varphi$ as in the above proof shows that $G/\ker\varphi\to\im\varphi$ is an injective immersion.
\end{remark}
% \begin{remark}
% 	Let's discuss Fix any $\FF$-vector space $V$, possibly infinite. 
% \end{remark}

\end{document}