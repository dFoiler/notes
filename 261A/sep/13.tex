% !TEX root = ../notes.tex

\documentclass[../notes.tex]{subfiles}

\begin{document}

\section{September 13}
Today we continue our discussion of coverings.

\subsection{The Universal Cover}
There is more or less one covering space which produces all the other ones.
\begin{definition}[universal cover]
	Fix a path-connected topological space $X$. Then a covering space $p\colon Y\to X$ is the \textit{universal cover} if and only if $Y$ is connected and simply connected.
\end{definition}
We now discuss an action of $\pi_1(X,b)$ on covering spaces in order to better understand this universal cover. Fix a covering space $p\colon Y\to X$ and a basepoint $x\in X$. Then we note that $\pi_1(X,x)$ acts on the fiber $p^{-1}(\{x\})$ as follows: for any $[\gamma]\in\pi_1(X,x)$ and $\widetilde x\in p^{-1}(\{x\})$, we define $\widetilde\gamma\colon[0,1]\to Y$ by lifting the path $\gamma\colon[0,1]\to X$ up to $Y$ so that $\widetilde\gamma(0)=\widetilde x$; then
\[[\gamma]\cdot\widetilde x\coloneqq\widetilde\gamma(1).\]
One can check that this action is well-defined (namely, it does not depend on the representative $\gamma$ and does provide a group action). Here are some notes.
\begin{itemize}
	\item If $Y$ is path-connected, then the action is transitive: and $\widetilde x,\widetilde x'\in p^{-1}(\{x\})$ admit a path $\widetilde\gamma\colon[0,1]\to Y$ with $\widetilde\gamma(0)=\widetilde x$ and $\widetilde\gamma(1)=\widetilde x'$, so $\gamma\coloneqq p\circ\widetilde\gamma$ has
	\[[\gamma]\cdot\widetilde x\coloneqq\widetilde x'\]
	by construction of $\gamma$.
	\item If $Y$ is simply connected, then this action is also free. Indeed, choose two paths $\gamma_1,\gamma_2\colon[0,1]\to X$ representing classes in $\pi_1(X,x)$. Now, suppose that $[\gamma_1]\cdot\widetilde x=[\gamma_2]\cdot\widetilde x$ for each $\widetilde x\in p^{-1}(\{x\})$, and we will show that $[\gamma_1]=[\gamma_2]$. Well, choosing lifts $\widetilde\gamma_1$ and $\widetilde\gamma_2$, the hypothesis implies that they have the same endpoints. Thus, because $Y$ is simply connected, we know $\widetilde\gamma_1\sim\widetilde\gamma_2$. We now see that $\gamma_1\sim\gamma_2$ by composing the homotopy witnessing $\widetilde\gamma_1\sim\widetilde\gamma_2$ with $p$.
\end{itemize}
The conclusion is that $p^{-1}(\{x\})$ is in bijection with $\pi_1(X,x)$ when $p\colon Y\to X$ is the universal cover. Here are some examples.
\begin{example}
	One has a covering space $p\colon S^n\to\RP^n$ given by
	\[(x_0,\ldots,x_n)\mapsto[x_0:\ldots:x_n].\]
	For $n\ge2$, we know $S^n$ is simply connected, so it will be the universal cover, and we are able to conclude that $\pi_1(\RP^n)$ is isomorphic to a fiber of $p$, which has two elements, so $\pi_1(\RP^n)\cong\ZZ/2\ZZ$.
\end{example}
\begin{example}
	One has a covering space $p\colon\RR\to S^1$ given by $p(t)\coloneqq e^{2\pi it}$. We can see that $\RR$ is simply connected (it's convex), so this is a universal covering. This at least tells us that $\pi_1\left(S^1\right)$ is countable, and one can track through the group law through the above bijections to see that actually $\pi_1\left(S^1\right)\cong\ZZ$.
\end{example}
\begin{example}
	One can show that $\pi_1(\CC\setminus\{z_1,\ldots,z_n\})$ is the free group on $n$ generators, basically corresponding to how one goes around each point.
\end{example}
Now, in the context of our Lie groups, we get the following result.
\begin{theorem} \label{thm:universal-cover-lie}
	Fix a regular Lie group $G$, and let $p\colon\widetilde G\to G$ be the universal cover.
	\begin{listalph}
		\item Then $\widetilde G$ has the structure of a regular Lie group.
		\item The projection $p$ is a homomorphism of Lie groups.
		\item The kernel $\ker p\subseteq\widetilde G$ is discrete, central, and isomorphic to $\pi_1(G,e)$. In particular, $\pi_1(G,e)$ is commutative.
	\end{listalph}
\end{theorem}
\begin{proof}
	Here we go.
	\begin{listalph}
		\item \Cref{rem:covering-manifold} tells us that $\widetilde G$ is a regular manifold, so it really only remains to exhibit the group structure. We will content ourselves with merely describing the group structure. Fix any $\widetilde e\in p^{-1}(\{e\})$, which will be our identity.
		
		Now, $\widetilde G$ is simply connected, so $\widetilde G\times\widetilde G$ is also simply connected. Thus, \Cref{rem:covering-manifold} explains that the composite
		\[\widetilde G\times\widetilde G\to G\times G\stackrel m\to G\]
		will lift to a unique map to the universal cover as a map $\widetilde m$ making the following diagram commute.
		% https://q.uiver.app/#q=WzAsOCxbMCwwLCJcXHdpZGV0aWxkZSBHXFx0aW1lc1xcd2lkZXRpbGRlIEciXSxbMCwxLCJHXFx0aW1lcyBHIl0sWzEsMCwiXFx3aWRldGlsZGUgRyJdLFsxLDEsIkciXSxbMiwwLCIoXFx3aWRldGlsZGUgZSxcXHdpZGV0aWxkZSBlKSJdLFsyLDEsIihlLGUpIl0sWzMsMSwiZSJdLFszLDAsIlxcd2lkZXRpbGRlIGUiXSxbMCwyLCJcXHdpZGV0aWxkZSBtIl0sWzEsMywibSJdLFswLDEsIihwLHApIiwyXSxbMiwzLCJwIl0sWzQsNywiIiwwLHsic3R5bGUiOnsidGFpbCI6eyJuYW1lIjoibWFwcyB0byJ9fX1dLFs3LDYsIiIsMCx7InN0eWxlIjp7InRhaWwiOnsibmFtZSI6Im1hcHMgdG8ifX19XSxbNSw2LCIiLDIseyJzdHlsZSI6eyJ0YWlsIjp7Im5hbWUiOiJtYXBzIHRvIn19fV0sWzQsNSwiIiwyLHsic3R5bGUiOnsidGFpbCI6eyJuYW1lIjoibWFwcyB0byJ9fX1dXQ==&macro_url=https%3A%2F%2Fraw.githubusercontent.com%2FdFoiler%2Fnotes%2Fmaster%2Fnir.tex
		\[\begin{tikzcd}
			{\widetilde G\times\widetilde G} & {\widetilde G} & {(\widetilde e,\widetilde e)} & {\widetilde e} \\
			{G\times G} & G & {(e,e)} & e
			\arrow["{\widetilde m}", from=1-1, to=1-2]
			\arrow["{(p,p)}"', from=1-1, to=2-1]
			\arrow["p", from=1-2, to=2-2]
			\arrow[maps to, from=1-3, to=1-4]
			\arrow[maps to, from=1-3, to=2-3]
			\arrow[maps to, from=1-4, to=2-4]
			\arrow["m", from=2-1, to=2-2]
			\arrow[maps to, from=2-3, to=2-4]
		\end{tikzcd}\]
		One can construct the inverse map similarly by lifting the map $\widetilde G\stackrel p\to G\stackrel i\to G$ to a map to $\widetilde G$ sending $\widetilde e\mapsto\widetilde e$. Uniqueness of lifting will guarantee that we satisfy the group law.
		\item We see that $p$ is a homomorphism by construction of $\widetilde m$ above.
		\item This is on the homework.
		\qedhere
	\end{listalph}
\end{proof}
\begin{example}
	Recall that we have the fiber bundle $\op{SO}_n(\RR)\to S^{n-1}$ with fiber $\op{SO}_{n-1}(\RR)$. Thus, the long exact sequence in homotopy groups produces
	\[\pi_2\left(S^{n-1}\right)\to\pi_1(\op{SO}_{n-1}(\RR))\to\pi_1(\op{SO}_n(\RR))\to\pi_1\left(S^{n-1}\right)\to\pi_0\left(\op{SO}_{n-1}(\RR)\right).\]
	Now, for $n\ge4$, one has that $\pi_2\left(S^{n-2}\right)=\pi_1\left(S^{n-1}\right)=1$, so we have $\pi_1\left(\op{SO}_{n-1}(\RR)\right)\cong\pi_1\left(\op{SO}_n(\RR)\right)$. One can check that $\op{SO}_3(\RR)\cong\RP^3$, so we see that
	\[\pi_1\left(\op{SO}_n(\RR)\right)\cong\ZZ/2\ZZ\]
	for $n\ge4$. The universal (double) cover of $\op{SO}_n(\RR)$ is called $\op{Spin}_n$, and \Cref{thm:universal-cover-lie} explains that we have a short exact sequence
	\[1\to\ZZ/2\ZZ\to{\op{Spin}_n}\to\op{SO}_n(\RR)\to1.\]
\end{example}
\begin{example}
	More concretely, one can show that $\op{SU}_2(\CC)$ has an action on $\RR^3$ preserving distances and orientation, so we get a homomorphism ${\op{SU}_2(\CC)}\to{\op{SO}_3(\RR)}$. One can check that this map is surjective with kernel isomorphic to $\ZZ/2\ZZ$.
\end{example}
In general, \Cref{thm:universal-cover-lie} explains that we have a short exact sequence
\[1\to\pi_1(G)\to\widetilde G\to G\to1\]
for any regular Lie group $G$, so it does not cost us too much to pass from $G$ to $\widetilde G$, allowing us to assume that the Lie groups we study are simply connected. (Note that even though $\pi_1(G)$ is discrete, the short exact sequence does not split: $\widetilde G$ succeeds at being connected.)

\subsection{Vector Fields}
Fix a regular manifold $X$ of dimension $n$. We may be interested in thinking about all our tangent spaces at once.
\begin{definition}[tangent bundle]
	Fix a regular manifold $X$. Then we define the \textit{tangent bundle} as
	\[TX\coloneqq\{(x,v):v\in T_xX\}.\]
	Note that there is a natural projection map $TX\to X$ by $(x,v)\mapsto x$.
\end{definition}
\begin{remark}
	Locally on a chart $(U,\varphi)$ of $X$, we see that $\varphi$ provides coordinates $(x_1,\ldots,x_n)$ on $U$, so one has a bijection
	\[U\times\RR^n\to TU\]
	by sending $(x,\del/\del x_i)\mapsto\left(\varphi^{-1}(x),d\varphi^{-1}_x(\del/\del x_i)\right)$ (In the future, we may conflate $d\varphi^{-1}_x(\del/\del x_i)$ with $\del/\del x_i$). This provides a chart for $TU$, and one can check that these charts are smoothly compatible by an explicit computation using the smooth compatibility of charts on $X$. The point is that $TX\to X$ is a vector bundle of rank $n$.
\end{remark}
Vector bundles are interesting because of their sections.
\begin{definition}[vector field]
	Fix a regular manifold $X$. Then a \textit{vector field} on $X$ is a smooth section $\sigma\colon X\to TX$ of the natural projection map $TX\to X$.
\end{definition}
\begin{remark}
	Locally on a chart $(U,\varphi)$ with coordinates $(x_1,\ldots,x_n)$, we see that we can think about a vector field $\sigma$ locally as
	\[\sigma(x)\coloneqq\sum_{i=1}^n\sigma_i(x)\frac{\del}{\del x_i}\bigg|_x,\]
	where the smoothness of $\sigma$ enforces the $\sigma_\bullet$s to be smooth. Changing coordinates to $(U',\varphi')$ with a coordinate expansion $\sigma(x)=\sum_i\sigma_i'(x)\frac{\del}{\del x_i'}$, one can change bases using the Jacobian of $\varphi'\circ\varphi^{-1}$ to find that
	\[\sigma_i'(x)=\sum_{j=1}^n\frac{\del x_i'}{\del x_j'}\sigma_j(x).\]
	Anyway, the point is that we can define a vector field locally on these coordinates and then going back and checking that we have actually defined something that will glue smoothly up to $X$.
\end{remark}
The reason we care so much about tangent spaces in this class is because they give rise to our Lie algebras, whose representations are somehow our main focus.
\begin{definition}[Lie algebra]
	Fix a Lie group $G$. Then the \textit{Lie algebra} of $G$ is the vector space
	\[\mf g\coloneqq T_eG.\]
\end{definition}
It is somewhat difficult to find structure in this tangent space immediately, so we note that $T_eG$ is isomorphic with another vector space.
\begin{definition}[invariant vector field]
	Fix a Lie group $G$. Then a vector field $\xi\colon G\to TG$ is \textit{left-invariant} if and only if
	\[\xi(gx)=dL_g(\xi(x))\]
	for any $x,g\in G$. One can define \textit{right-invariant} analogously.
\end{definition}
\begin{remark}
	We claim that the vector space of left-invariant vector fields is isomorphic to $\mf g$. Here are our maps.
	\begin{itemize}
		\item Given a left-invariant vector field $\xi$, one can produce the tangent vector $\xi(e)\in\mf g$.
		\item Given some $\xi(e)\in\mf g$, we define
		\[\xi(g)\coloneqq dL_g(\xi(e))\in T_gG.\]
		It is not difficult to check that $\xi\colon G\to TG$ is at least a section of the natural projection $TG\to G$. We omit the check that $\xi$ is smooth because it is somewhat involved.
	\end{itemize}
\end{remark}
\begin{remark}
	As an aside, we note that the produced left-invariant vector fields parallelizes $G$ after providing a basis of $\mf g$; in particular, one has a canonical isomorphism $TG\cong G\times\mf g$. One can actually show that $TG$ is a Lie group with Lie group structure given by functoriality of the tangent bundle applied to the group operations of $G$, and one finds that $TG\cong G\rtimes\mf g$, where $G$ acts on $\mf g$ by the adjoint action.
\end{remark}
Next class we will go back and argue that our classical groups are actually Lie groups and compute their Lie algebras.

\end{document}