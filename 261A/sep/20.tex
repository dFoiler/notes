% !TEX root = ../notes.tex

\documentclass[../notes.tex]{subfiles}

\begin{document}

\section{September 20}
Today we continue discussing the Lie bracket.

\subsection{The Adjoint Action}
Fix a regular Lie group $G$ with Lie algebra $\mf g$. Here is the standard example of a ``Lie algebra representation.''
\begin{notation}
	Fix a regular Lie group $G$ with Lie algebra $\mf g$. Note that the map $(d{\op{Ad}_g})_1\colon G\to\op{GL}(\mf g)$ is smooth, so we can consider the differential of this map, which we label $\op{ad}\colon\mf g\to\mf{gl}(\mf g)$.
\end{notation}
Here are some checks on this map.
\begin{proposition} \label{prop:commutator-by-adjoint}
	Fix a regular Lie group $G$ with Lie algebra $\mf g$.
	\begin{listalph}
		\item For $X,Y\in\mf g$, we have $\op{ad}_X(Y)=[X,Y]$.
		\item For $X\in\mf g$, we have ${\op{Ad}_{\exp(X)}}=\exp({\op{ad}_X})$ as operators $\mf g\to\mf g$.
	\end{listalph}
\end{proposition}
\begin{proof}
	Here we go.
	\begin{listalph}
		\item By definition of our differential action, we have
		\[(d{\op{Ad}_g})_1(Y)=\frac d{dt}g\exp(tY)g^{-1}\bigg|_{t=0}\]
		for any $g\in G$ and $Y\in\mf g$. We would like to compute a derivative of this map with respect to $g$ (at the identity). As such, we plug in $g=\exp(sX)$ to compute
		\begin{align*}
			\op{ad}_X(Y) &= \frac d{ds}(d{\op{Ad}_{\exp(sX)}})_1(Y)\bigg|_{s=0} \\
			&= \frac d{ds}\frac d{dt}\exp(sX)\exp(tY)\exp(-sX)\bigg|_{t=0}\bigg|_{s=0} \\
			&\stackrel*= \frac d{ds}\frac d{dt}\exp(tY+st[X,Y]+\cdots)\bigg|_{t=0}\bigg|_{s=0},
		\end{align*}
		where in $\stackrel*=$ we have used the definition of our bracket. Upon expanding out $\exp$ as a series, we see that the lower-order terms are $1+tY+st[X,Y]+\cdots$ (everything higher is at least quadratic) for small enough $s$ and $t$, so the derivative evaluates to $[X,Y]$.
		\item This follows immediately from \Cref{lem:exp-on-homs} upon setting $\varphi=(d{\op{Ad}_\bullet})_1$.
		\qedhere
	\end{listalph}
\end{proof}
Here is an example computation of what all this adjoint business looks like for $\op{GL}_n$, more directly than appealing to the bracket.
\begin{lemma} \label{lem:adjoint-rep-gln}
	Identify $T\op{GL}_n(\FF)$ with $\op{GL}_n(\FF)\times\mf{gl}_n(\FF)$ via left-invariant vector fields. For $X\in\mf {gl}_n(\FF)$, we have
	\[\begin{cases}
		dL_g(X)=gX, \\
		dR_g(X)=Xg^{-1}, \\
		d\op{Ad}_g(X)=gXg^{-1}.
	\end{cases}\]
\end{lemma}
\begin{proof}
	Set $G\coloneqq\op{GL}_n(\FF)$ and $\mf g\coloneqq\mf{gl}_n(\FF)$. Note that the adjoint is the composite of $L_g$ and $R_g$, so the last equality follows from the first two. For the first equality, we are computing the differential of the maps $L_g,R_g\colon G\to G$ at some $h\in G$. Well, $L_g$ and $R_g$ actually extend to perfectly fine linear maps $M_n(\FF)\to M_n(\FF)$, and the differential of any linear map is simply itself upon identifying the tangent spaces of $M_n(\FF)$ with itself, so we conclude that $dL_g(X)=gX$ and $dR_g(X)=Xg^{-1}$, as required.
\end{proof}
\begin{lemma} \label{lem:adjoint-hom}
	Fix a homomorphism $\varphi\colon G\to H$ of Lie groups with Lie algebras $\mf g$ and $\mf h$ respectively. For any $g\in G$ and $X\in\mf g$, we have
	\[(d{\op{Ad}_{\varphi(g)}})_e(d\varphi_1(X))=d\varphi_1((d{\op{Ad}_g})_e(X)).\]
\end{lemma}
\begin{proof}
	Simply take the differential (at $1$) of the equation ${\op{Ad}_{\varphi(g)}}\circ\varphi=\varphi\circ\op{Ad}_g$, which is true because $\varphi$ is a homomorphism.
\end{proof}
\begin{example}
	Given any embedding $G\subseteq\op{GL}_n(\FF)$ with Lie algebra $\mf g\subseteq\mf{gl}_n(\FF)$, we can use \Cref{lem:adjoint-hom} to compute the adjoint action on $\mf g$ by conjugation (via \Cref{lem:adjoint-rep-gln})!
\end{example}
\begin{proposition} \label{lem:lie-adjoint-gln}
	Let $(d{\op{Ad}_\bullet})_1\colon\op{GL}_n(\FF)\to\op{GL}(\mf{gl}_n(\FF))$ denote the adjoint representation. Then
	\[\op{ad}_X(Y)=XY-YX.\]
\end{proposition}
\begin{proof}
	To parse the symbols, we note that $(d(d{\op{Ad}_\bullet})_1)_1\colon\mf{gl}_n(\FF)\to\op{End}(\mf{gl}_n(\FF))$, so the statement at least makes sense. Now, given $X\in\mf{gl}_n(\FF)$, define $\gamma\colon\FF\to M_n(\FF)$ by $\gamma(t)\coloneqq 1+tX$. Then $\gamma'(0)=X$. As such,
	\[(d(d{\op{Ad}_\bullet})_1)_1(X)=(d(d{\op{Ad}_\bullet})_1)_1(\gamma'(0))=((d{\op{Ad}_\bullet})_1\circ\gamma)'(0).\]
	In particular, plugging in some $Y\in\mf{gl}_n(\FF)$, we may use \Cref{lem:adjoint-rep-gln} to compute that
	\begin{align*}
		\frac d{dt}((d{\op{Ad}_\bullet})_1\circ\gamma)(t)(Y)\bigg|_{t=0} &= \frac d{dt}(d{\op{Ad}_{1+tX}})_1(Y)\bigg|_{t=0} \\
		&= \frac d{dt}(1+tX)Y(1+tX)^{-1}\bigg|_{t=0} \\
		&= \frac d{dt}(1+tX)Y\left(1-tX+t^2X^2+\cdots\right)\bigg|_{t=0} \\
		&= XY-YX,
	\end{align*}
	where the series expansion takes $t$ small enough for the series to converge. (For example, one can take $t$ small enough so that all eigenvalues of $tX$ are less than $1$.)
\end{proof}
\begin{example}
	Given any embedding $G\subseteq\op{GL}_n(\FF)$ with Lie algebra $\mf g\subseteq\mf{gl}_n(\FF)$, we note that the action of $G$ on $\mf g$ actually extends to an action of $G$ on $\mf{gl}_n(\FF)$ (still by conjugation) which happens to stabilize $\mf g$. Then the action $G\to\op{GL}(\mf{gl}_n(\FF))$ is a restriction of the adjoint action $\op{GL}_n(\FF)\to\op{GL}(\mf{gl}_n(\FF))$ given by conjugation still, whose differential action $\mf{gl}_n(\FF)\to\mf{gl}(\mf{gl}_n(\FF))$ we computed above to be given by $\op{ad}_X\colon Y\mapsto XY-YX$. This restricts back to the subspace $\mf g\subseteq\mf{gl}_n(\FF)$ (via the inclusion $G\subseteq\op{GL}_n(\FF)$), where we know that the action must happen to stabilize $\mf g\subseteq\mf{gl}_n(\FF)$. The point is that we have computed our adjoint representation $\mf g\to\mf{gl}(\mf g)$ is given by the commutator. (Alternatively, one can redo the computation of the above proof.)
\end{example}

\subsection{Lie Algebras}
Here is a standard consequence of this theory.
\begin{proposition}[Jacobi identity] \label{prop:jacobi}
	Fix a Lie group $G$ with Lie algebra $\mf g$. Then we have the Jacobi identity
	\[[X,[Y,Z]]+[Y,[Z,X]]+[Z,[X,Y]]=0.\]
\end{proposition}
\begin{proof}
	Doing some rearranging with \Cref{prop:commutator-by-adjoint} (and the skew-symmetry), we see that this is equivalent to plugging $Z$ into the identity
	\[{\op{ad}_{[X,Y]}}\stackrel?={\op{ad}_X}\circ{\op{ad}_Y}-{\op{ad}_Y}\circ{\op{ad}_X}.\]
	To verify this, we note that the right-hand side is $[{\op{ad}_X},{\op{ad}_Y}]$, where the commutator is taken in $\mf{gl}(\mf g)$. Thus, we are trying to show that the adjoint preserves a commutator, which we do as follows: recall that $\op{Ad}_\bullet\colon G\to\op{GL}(\mf g)$ is a morphism of Lie groups, meaning that the differential map $\op{ad}$ preserves the commutator by \Cref{prop:lie-bracket-functoriality}.
\end{proof}
The Jacobi identity is important enough to earn the following definition.
\begin{definition}[Lie algebra]
	Fix a field $k$. Then a \textit{Lie algebra} is a $k$-vector space $\mf g$ equipped with a bilinear form $[-,-]\colon\mf g\times\mf g\to\mf g$ satisfying the following.
	\begin{listalph}
		\item Skew-symmetric: $[X,X]=0$ for all $X\in\mf g$.
		\item Jacobi identity: for any $X,Y,Z\in\mf g$, we have
		\[[X,[Y,Z]]+[Y,[Z,X]]+[Z,[X,Y]]=0.\]
	\end{listalph}
	A morphism of Lie algebras is a $k$-linear morphism preserving the forms.
\end{definition}
\begin{definition}[commutative]
	A Lie algebra $\mf g$ is \textit{commutative} if and only if $[X,Y]=0$ for all $X,Y\in\mf g$.
\end{definition}
\begin{example}
	For any $k$-algebra $A$, we produce a Lie bracket on $A$ given by
	\[[X,Y]\coloneqq XY-YX.\]
	This map is of course linear in both $X$ and $Y$ (because multiplication is $k$-linear in a $k$-algebra), and $[X,X]=X^2-X^2=0$. Lastly, to see the Jacobi identity, we expand:
	\begin{align*}
		[X,[Y,Z]]+[Y,[Z,X]]+[Z,[X,Y]] &= [X,YZ-ZY]+[Y,ZX-XZ]+[Z,XY-YX] \\
		&= X(YZ-ZY)-(YZ-ZY)X \\
		&\qquad+Y(ZX-XZ)-(ZX-XZ)Y \\
		&\qquad+Z(XY-YX)-(XY-YX)Z \\
		&= 0.
	\end{align*}
	For example, one can take $A$ to be $\op{End}_k(V)$ for some $k$-vector space $V$; this produces the Lie algebra $\mf{gl}(V)$.
\end{example}
\begin{example}
	Given a regular Lie group $G$, the tangent space at the identity $\mf g$ is a Lie algebra according to the above definition.
\end{example}
The above example upgrades into a functor.
\begin{proposition} \label{prop:lie-group-to-lie-algebra}
	Fix a regular Lie group $G$. For any morphism of Lie groups $\varphi\colon G_1\to G_2$, the differential $d\varphi_e\colon T_eG_1\to T_eG_2$ is a (functorial) morphism of Lie algebras. In fact, if $G_1$ and $G_2$ is connected, the induced map
	\[\op{Hom}_{\mathrm{LieGrp}}(G_1,G_2)\to\op{Hom}_{\mathrm{Lie}(k)}(T_eG_1,T_eG_2)\]
	is injective. In other words, the functor $G\to T_eG$ from connected Lie groups to Lie algebras is faithful.
\end{proposition}
\begin{proof}
	The differential being a homomorphism of Lie algebras follows from \Cref{prop:lie-bracket-functoriality}. Functoriality follows from the corresponding functoriality for differentials of more general smooth maps. The injectivity follows from \Cref{cor:get-morphism-from-algebra}.
\end{proof}
\begin{remark}
	It turns out that the functor above is also full, though we are not in a position to show this yet.
\end{remark}

\subsection{Subalgebras}
Lie algebras are interesting enough to study on their own right, but we now note that we have sufficient motivation from \Cref{prop:lie-group-to-lie-algebra}.
\begin{defihelper}[subalgebra, ideal] \nirindex{subalgebra} \nirindex{ideal}
	Fix a Lie algebra $\mf g$.
	\begin{itemize}
		\item A \textit{Lie subalgebra} $\mf h\subseteq\mf g$ is a subspace closed under the Lie bracket of $\mf g$; note that $\mf h$ continues to be a Lie algebra.
		\item A \textit{Lie ideal} is a subalgebra $\mf h\subseteq\mf g$ with the stronger property that
		\[[X,Y]\in\mf h\]
		for any $X\in\mf h$ and $Y\in\mf g$.
	\end{itemize}
\end{defihelper}
\begin{definition}[representation]
	A \textit{representation} of a Lie algebra $\mf g$ over a field $k$ is a morphism $\mf g\to\mf{gl}(V)$ for some vector space $V$ over $k$.
\end{definition}
Here is how these things relate back to Lie groups.
\begin{proposition}
	Fix a regular Lie subgroup $H$ of a regular Lie group $G$. Let their Lie algebras be $\mf h$ and $\mf g$, respectively.
	\begin{listalph}
		\item Then $\mf h\subseteq\mf g$ is a Lie subalgebra.
		\item If $H$ is normal in $G$, then $\mf h$ is an ideal of $\mf g$.
		\item If $G$ and $H$ are connected, and $\mf h$ is an ideal of $\mf g$, then $H$ is normal in $G$.
	\end{listalph}
\end{proposition}
\begin{proof}
	Here we go.
	\begin{listalph}
		\item Certainly $\mf h\subseteq\mf g$ is a subspace, so we want to check that $[X,Y]\in\mf h$ for $X,Y\in\mf h$, where the target bracket is taken in $\mf g$. Consider the embedding $\varphi\colon H\to G$ so that $\mf h=\im d\varphi_0$. Thus, we use \Cref{prop:lie-bracket-functoriality} to see that
		\[d\varphi_0([X,Y])=[d\varphi_0(X),d\varphi_0(Y)].\]
		Thus, for any $X,Y\in\im d\varphi_0$, we see that $[X,Y]\in\im d\varphi_0$, as required.

		\item For any $X\in\mf g$ and $Y\in\mf h$, we want to check that $[X,Y]\in\mf h$. By \Cref{prop:commutator-by-adjoint}, we are asking to check that $\op{ad}_X(Y)\in\mf h$. Well, for any $g\in G$, we see that $gHg^{-1}\subseteq H$, so the adjoint $\op{Ad}_g\colon G\to G$ restricts to $\op{Ad}_g\colon H\to H$. In particular, by taking the differential, we see that the adjoint $(d{\op{Ad}_\bullet})_1\colon G\to\op{GL}(\mf g)$ restricts to $(d{\op{Ad}_\bullet})_1\colon G\to\op{GL}(\mf h)$. (Namely, $(d{\op{Ad}_g})_1(Y)\in\mf h$ for any $Y\in\mf h$.) Taking the differential of this, we see that we get our map $\op{ad}_\bullet\colon\mf g\to\mf{gl}(\mf h)$, meaning that $\op{ad}_X(Y)\in\mf h$ for any $X\in\mf g$ and $Y\in\mf h$.
		
		\item Recall from \Cref{prop:commutator-by-adjoint} that
		\[\op{Ad}_{\exp(X)}(Y)=\exp(\op{ad}_XY).\]
		Thus, for any $X\in\mf g$, we see that $\op{Ad}_{\exp(X)}$ is an operator $\mf h\to\mf h$. Thus, for $g\in G$ sufficiently close to the identity, we see that $\op{Ad}_g(Y)\in\mf h$ for $Y\in\mf h$. Taking the exponential, \Cref{prop:exp-adjoint} tells us that $ghg^{-1}\in H$ for $g\in G$ and $h\in H$ both sufficiently close to the identity.
		
		Concretely, we get an open neighborhood $U$ of the identity of $G$ such that $ghg^{-1}\in H$ for any $g\in U$ and $h\in H\cap U$. Now, the subset of $G$ normalizing $U\cap H$ is a subgroup of $G$ containing $U$, so we see that it must be all of $G$ because $G$ is connected. Then the subset of $H$ normalized by $G$ is again a subgroup of $H$ containing $U\cap H$, so we see that it must be all of $H$ because $H$ is connected. Thus, $H$ is normal in $G$.
		\qedhere
	\end{listalph}
\end{proof}
Here is some motivation for our definition of ideal.
\begin{lemma} \label{lem:subalgebra-checks}
	Fix a morphism $\varphi\colon\mf g\to\mf h$ of Lie algebras.
	\begin{listalph}
		\item The kernel $\ker\varphi\subseteq\mf g$ is a Lie ideal.
		\item The image $\im\varphi\subseteq\mf h$ is a Lie subalgebra.
	\end{listalph}
\end{lemma}
\begin{proof}
	Here we go.
	\begin{listalph}
		\item For any $X\in\ker\varphi$ and $Y\in\mf g$, we need to check that $[X,Y]\in\ker\varphi$. Well,
		\[\varphi([X,Y])=[\varphi(X),Y]=[0,Y]=0\]
		by the bilinearity of $[-,-]$.
		\item For any $X,Y\in\im\varphi$, we must check that $[X,Y]\in\im\varphi$. Well, find $X_0,Y_0\in\mf g$ such that $X=\varphi(X_0)$ and $Y=\varphi(Y_0)$, and then we see that
		\[[X,Y]=[\varphi(X_0),\varphi(Y_0)]=\varphi([X_0,Y_0])\]
		is in the image of $\varphi$, as required.
		\qedhere
	\end{listalph}
\end{proof}
\begin{remark} \label{rem:intersect-lie-ideals}
	Fix a collection $\{\mf g_\alpha\}_{\alpha\in\kappa}$ of Lie ideals of $\mf g$. Then we claim that the intersection $\bigcap_{\alpha\in\kappa}\mf g_\alpha$ is still a Lie ideal of $\mf g$. Indeed, for any $X\in\bigcap_{\alpha\in\kappa}\mf g_\alpha$ and $Y\in\mf g$, we see that $X\in\mf g_\alpha$ and hence $[X,Y]\in\mf g_\alpha$ for all $\alpha\in\kappa$; thus, $[X,Y]\in\bigcap_{\alpha\in\kappa}\mf g_\alpha$.
\end{remark}
\begin{lemma}
	Fix a Lie ideal $\mf h$ of a Lie algebra $\mf g$. Then the quotient space $\mf g/\mf h$ is a Lie algebra with bracket given by
	\[[X+\mf h,Y+\mf h]_{\mf g/\mf h}\coloneqq[X,Y]_{\mf g}+\mf h.\]
\end{lemma}
\begin{proof}
	The main issue is checking that the bracket is well-defined. Well, if $X,Y\in\mf g$ and $X',Y'\in\mf h$, we must check that
	\[[X+X',Y+Y']+\mf h\stackrel?=[X,Y]+\mf h,\]
	where the bracket is taken in $\mf g$. This is a matter of expanding with the bilinearity: note
	\begin{align*}
		[X+X',Y+Y'] &= [X+X',Y]+[X+X',Y'] \\
		&= [X,Y]+[X',Y]+[X,Y']+[X',Y'],
	\end{align*}
	and now we see that the last three terms live in $\mf h$ because $\mf h\subseteq\mf g$ is an ideal.

	Now, note that we have a canonical surjective linear map $\pi\colon\mf g\onto\mf g/\mf h$ which satisfies
	\[\pi([X,Y])=[\pi(X),\pi(Y)].\]
	Thus, the bilinearity, skew-symmetry, and Jacobi identity for $\mf g/\mf h$ are immediately inherited from the corresponding checks on $\mf g$. Rigorously, perhaps one should note that (for example) the Jacobi identity corresponds to showing that some linear functional on $(\mf g/\mf h)^3$ vanishes; however, this linear functional can be checked to vanish on the level of $\mf g^3$.
\end{proof}
\begin{proposition}
	Fix a morphism $\varphi\colon\mf g\to\mf h$ of Lie algebras. Then the induced quotient map
	\[\ov\varphi\colon\mf g/\ker\varphi\to\im\varphi\]
	is an isomorphism.
\end{proposition}
\begin{proof}
	Linear algebra implies that $\ov\varphi$ is already an isomorphism of vector spaces. Thus, it merely remains to check that $\ov\varphi$ is a morphism of Lie algebras. Well, for $X,Y\in\mf g=$, we see
	\begin{align*}
		\ov\varphi([X+\ker\varphi,Y+\ker\varphi]) &= \ov\varphi([X,Y]+\ker\varphi) \\
		&= \varphi([X,Y])+\ker\varphi \\
		&= [\varphi(X),\varphi(Y)]+\ker\varphi \\
		&= [\ov\varphi(X),\ov\varphi(Y)],
	\end{align*}
	as required.
\end{proof}

\subsection{Lie Algebra of a Vector Field}
One can in general provide a Lie algebra of a vector field. Fix a regular vector field $\xi\colon X\to TX$ on a regular manifold $X$. For any regular function $f$ on an open subset $U\subseteq X$, we may define
\[(\xi f)(x)\coloneqq\xi_x(f_x),\]
where we recall that $\xi_x\in T_xX$ is some derivation which outputs a number when fed a germ $f_x$. The point is that $\xi f$ is itself a regular function $X\to\FF$! We are now able to define a bracket.
\begin{proposition}
	Fix a regular manifold $X$. Given vector fields $\xi,\eta\colon X\to TX$, we define the Lie bracket
	\[[\xi,\eta]\coloneqq\xi\eta-\eta\xi.\]
	Then $[-,-]$ is a Lie bracket.
\end{proposition}
\begin{proof}
	At each $x\in X$, we have certainly defined a map taking regular functions $f$ on $X$ and outputting an element of $\FF$ given by
	\[[\xi,\eta]_x(f)\coloneqq\xi_x(\eta f)_x-\eta_x(\xi f)_x.\]
	This is certainly linear in $f$ because $\xi$ and $\eta$ are. Further, the value of $[\xi,\eta]_x(f)$ only depends on the germ $f_x$ because having $f_x=g_x$ for functions $f$ and $g$ implies $(f-g)_x=0_x$, and then $\eta(f-g)$ and $\xi(f-g)$ both vanish in a neighborhood of $x$, so $[\xi,\eta]_x(f-g)=0$.
	
	It remains to check the product rule. Well, for regular functions $f$ and $g$ and some $y\in X$, we compute
	\[(\eta fg)(y)=\eta_y(f_yg_y)=f(y)\eta_y(g_y)+g(y)\eta_y(g_y)=(f\cdot\eta g+g\cdot\eta f)(y),\]
	and a similar computation works for $\xi$. Thus,
	\begin{align*}
		\xi(\eta fg)(x) &= \xi(f\eta g+g\eta f)(x) \\
		&= \xi(f\eta g)(x)+\xi(g\eta f)(x) \\
		&= f(x)\xi(\eta g)(x)+(\xi f)(x)(\eta g)(x)+g(x)\xi(\eta f)(x)+(\xi g)(x)(\eta f)(x),
	\end{align*}
	and a similar computation holds for $\eta(\xi fg)(x)$. Thus, we see that
	\begin{align*}
		[\xi,\eta]_x(fg) &= f(x)\xi(\eta g)(x)+(\xi f)(x)(\eta g)(x)+g(x)\xi(\eta f)(x)+(\xi g)(x)(\eta f)(x) \\
		&\qquad-\left(f(x)\eta(\xi g)(x)+(\eta f)(x)(\xi g)(x)+g(x)\eta(\xi f)(x)+(\eta g)(x)(\xi f)(x)\right) \\
		&= f(x)[\xi,\eta]_xg+g(x)[\xi,\eta]_xf
	\end{align*}
	after sufficient cancellation and rearranging.
\end{proof}
\begin{example}
	Fix regular functions $f$ and $g$ on some open subset of $U\subseteq\RR^m$, and let $x_i$ and $x_j$ be two coordinates. Then we compute
	\[\left[f\frac\del{\del x_i},g\frac\del{\del x_j}\right]=f\frac{\del g}{\del x_i}\frac\del{\del x_j}-g\frac{\del f}{\del x_j}\frac\del{\del x_i}.\]
\end{example}
\begin{proof}
	Fixing some $p\in U$ and regular germ $h$, we see
	\begin{align*}
		\left[f\frac\del{\del x_i},g\frac\del{\del x_j}\right]_p(h) &= f(p)\frac\del{\del x_i}g\frac{\del h}{\del x_j}\bigg|_p-g(p)\frac\del{\del x_j}f\frac{\del h}{\del x_i}\bigg|_p \\
		&= f(p)\frac{\del g}{\del x_i}\bigg|_p\frac{\del h}{\del x_j}\bigg|_p+f(p)g(p)\frac{\del h}{\del x_i\del x_j}\bigg|_p-g(p)\frac{\del f}{\del x_j}\bigg|_p\frac{\del h}{\del x_i}\bigg|_p-f(p)g(p)\frac{\del h}{\del x_i\del x_j}\bigg|_p \\
		&= f(p)\frac{\del g}{\del x_i}\bigg|_p\frac{\del h}{\del x_j}\bigg|_p-g(p)\frac{\del f}{\del x_j}\bigg|_p\frac{\del h}{\del x_i}\bigg|_p,
	\end{align*}
	as required.
\end{proof}
\begin{remark}
	In local coordinates in some chart $(U,\varphi)$ with $\varphi=(x_1,\ldots,x_m)$ of our regular manifold $M$, one can write vector fields as
	\[\xi=\sum_{i=1}^ma_i\frac{\del}{\del x_i}\qquad\text{and}\qquad\eta=\sum_{i=1}^mb_i\frac\del{\del x_i},\]
	where $a_i$ and $b_i$ are regular functions. Then one can expand the bilinearity to see that
	\[[\xi,\eta]=\sum_{i,j=1}^m\left(a_j\frac{\del b_i}{\del x_j}-b_j\frac{\del a_i}{\del x_j}\right)\frac\del{\del x_i}.\]
	Indeed, after applying bilinearity, the main point is to compute $\left[f\frac\del{\del x},g\frac\del{\del y}\right]$ for regular functions $f$ and $g$ and coordinates $x$ and $y$, which we did in the previous example.
\end{remark}
\begin{remark}
	For example, if $\xi$ and $\eta$ are tangent to a regular submanifold $N\subseteq M$ of dimension $k$, then $[\xi,\eta]$ continues to be tangent. One can check this using a local slice chart, where the condition that $\xi$ is tangent to $Y$ is equivalent to having $a_i=0$ for $i>k$. Combining this with the computation of the previous remark completes the argument.
\end{remark}

\end{document}