% !TEX root = ../notes.tex

\documentclass[../notes.tex]{subfiles}

\begin{document}

\section{September 11}
Today we talk more about homogeneous spaces.

\subsection{Homogeneous Spaces}
Let's see some applications of \Cref{thm:orb-stab}.
\begin{example} \label{ex:transitive-group-action}
	Suppose a regular Lie group $G$ acts smoothly and transitively on a regular manifold $X$. For each $x\in X$, we see that $G/G_x\to X$ is a bijective immersion. In particular, Sard's theorem implies that $\dim G/G_x=\dim X$, so we conclude that this map is in fact a bijective local diffeomorphism, which of course is just a diffeomorphism. Thus, \Cref{thm:quotient-group} tells us that the map $G\to X$ given by $g\mapsto gx$ is a fiber bundle with fiber $G_x$.
\end{example}
The above situation is so nice that it earns a name.
\begin{definition}[homogeneous space]
	Fix a regular Lie group $G$ acting smoothly and transitively on a regular manifold $X$. If the action of $G$ is transitive, we say that $X$ is a \textit{homogeneous space} of $G$.
\end{definition}
Here are many examples.
\begin{example}
	Continuing from \Cref{ex:so3-action}, we recall that $\op{SO}_3(\RR)$ acts on $S^2$. In fact, one can check that the stabilizer of any $x\in S^3$ is isomorphic to $S^1$, so \Cref{ex:transitive-group-action} tells us that $\op{SO}_3(\RR)\to S^2$ is a fiber bundle with fiber $S^1$. In general, we find that $\op{SO}_n(\RR)\to S^n$ is a fiber bundle with fiber $\op{SO}_{n-1}(\RR)$.\todo{}
\end{example}
\begin{example}
	The group $\op{SU}_2$ acts on $\CP^1$ by matrix multiplication. We see that the stabilizer of some line in $\CP^1$ consists of the matrices in $\op{SU}_2$ with a nonzero eigenvector on the line. For example, using the computation of \Cref{ex:su2}, we see that trying to stabilizer $[1:0]$ gives rise to the matrices $\begin{bsmallmatrix}
		a & -b \\
		\ov b & \ov a
	\end{bsmallmatrix}$ where we require $b=0$. Thus, we see that our stabilizer is isomorphic to $\op U_1$. In particular, our orbits are compact immersed submanifolds of $\CP^1$ of dimension $\dim{\op{SU}_2}-\dim{\op U_1}=\dim\CP^1$, so the action must be transitive in order for orbits to be closed and the correct dimension.
\end{example}
\begin{example}
	One can check that $\op{SU}_n$ acts on $S^{2n-1}\subseteq\CC^n$ with stabilizer isomorphic to $\op{SU}_{n-1}$.\todo{}
\end{example}
\begin{example}[flag varieties]
	Let $\mc F_n$ be the set of ``flags'' of $\FF^n$, which is an ascending chain of subspaces
	\[0=V_0\subsetneq V_1\subsetneq\cdots\subsetneq V_{n-1}\subsetneq V_n=\FF^n.\]
	Then we see that $\op{GL}_n(\FF)$ acts on $\mc F_n$ by matrix multiplication. On the homework, we check that this action is transitive with stabilizer (of the standard flag $\{\op{span}_\FF(e_1,\ldots,e_i)\}_{i=0}^n$) given by the matrix subgroup $B_n(\FF)\subseteq\op{GL}_n(\FF)$ of upper-triangular matrices. Thus, we see that we can realize $\mc F_n$ as the manifold quotient $\op{GL}_n(\FF)/B_n(\FF)$, providing a manifold structure.
\end{example}
\begin{example}[Grassmannians]
	Let $\op{Gr}_k(\FF^n)$ be the set of vector subspaces $V\subseteq\FF^n$ of dimension $k$. Then we see that $\op{GL}_n(\FF)$ acts transitively on $\op{Gr}_k(\FF^n)$ with stabilizer of $\op{span}(e_1,\ldots,e_k)$ given by matrices of the form
	\[\begin{bmatrix}
		A & B \\
		0 & D
	\end{bmatrix},\]
	where $A\in\FF^{k\times k}$ and $B\in\FF^{k\times(n-k)}$ and $D\in\FF^{(n-k)\times(n-k)}$. Thus, we can realize $\op{Gr}_k(\FF^n)$ as the manifold quotient of $\op{GL}_n(\FF)$, providing a manifold structure.
\end{example}
\begin{example}
	There are many regular actions of $G$ on itself.
	\begin{itemize}
		\item Regular left: define our action $R_\ell\colon G\times G\to G$ by $(g,x)\mapsto gx$.
		\item Regular right: define our action $R_r\colon G\times G\to G$ by $(g,x)\mapsto xg^{-1}$.
		\item Adjoint: define our action $\op{Ad}\colon G\times G\to G$ by $(g,x)\mapsto gxg^{-1}$. (This action is not transitive in general!) Notably, $\op{Ad}_g(1)=1$, so this also provides an action of $G$ on $T_eG$.
	\end{itemize}
\end{example}

\subsection{Covering Spaces}
It will help to recall some theory around covering spaces. See \cite{elber-alg-top} for (some) more detail about this theory or \cite{hatcher} for (much) more detail.
\begin{definition}[covering space]
	A \textit{covering space} is a fibration $p\colon Y\to X$ with discrete fiber $S$. The \textit{degree} of $p$ equals $\#S$.
\end{definition}
In more words, we are asking for each $x\in X$ to have an open neighborhood $U$ such that the restriction $p^{-1}(U)\to U$ is homeomorphic (over $U$) to $\bigsqcup_{s\in S}p^{-1}(U)\to U$ for some discrete set $S$.
\begin{remark}
	If $X$ is a regular manifold and $\deg p<\infty$, then $Y$ is also a regular manifold. Indeed, being a manifold is checked locally, so one can find neighborhoods as in the previous remark to witness the manifold structure.
\end{remark}
We are interested in paths in topological spaces, but there are too many. To make this set smaller, we consider it up to homotopy.
\begin{definition}[homotopy]
	Fix a topological space $X$. Two paths $\gamma_0,\gamma_1\colon[0,1]\to X$ are \textit{homotopic} relative to their endpoints if and only if there is a continuous map $H_\bullet\colon[0,1]^2\to X$ such that $H_0=\gamma_0$ and $H_1=\gamma_1$ and $H_s(0)=\gamma_0(0)=\gamma_1(0)$ and $H_s(1)=\gamma_0(1)=\gamma_1(1)$ for all $s$. The map $H$ is called a \textit{homotopy}.
\end{definition}
\begin{definition}[simply connected]
	A topological space $X$ is \textit{simply connected} if and two paths with the same endpoints are homotopic relative to those endpoints.
\end{definition}
\begin{example}
	One can check that $S^1$ fails to be simply connected because the path going around the circle is not homotopic to the constant path.
\end{example}
It is important to know that one can lift paths.
\begin{theorem}
	Fix a covering space $p\colon Y\to X$. Fix some point $x\in X$ and a path $\gamma\colon[0,1]\to X$ with $\gamma(0)=x$. Then each $\widetilde x\in p^{-1}(\{x\})$ has a unique path $\widetilde\gamma\colon[0,1]\to Y$ such that $\widetilde\gamma(0)=\widetilde x$ and making the following diagram commute.
	% https://q.uiver.app/#q=WzAsMyxbMCwwLCJbMCwxXSJdLFsxLDAsIlxcd2lkZXRpbGRlIFgiXSxbMSwxLCJYIl0sWzAsMSwiXFx3aWRldGlsZGVcXGdhbW1hIl0sWzEsMiwicCJdLFswLDIsIlxcZ2FtbWEiLDJdXQ==&macro_url=https%3A%2F%2Fraw.githubusercontent.com%2FdFoiler%2Fnotes%2Fmaster%2Fnir.tex
	\[\begin{tikzcd}
		{[0,1]} & {\widetilde X} \\
		& X
		\arrow["{\widetilde\gamma}", from=1-1, to=1-2]
		\arrow["\gamma"', from=1-1, to=2-2]
		\arrow["p", from=1-2, to=2-2]
	\end{tikzcd}\]
\end{theorem}
\begin{remark}
	One can further check that having two homotopic paths $\gamma_1\sim\gamma_2$ downstairs produce homotopic paths $\widetilde\gamma_1\sim\widetilde\gamma_2$. 
\end{remark}
\begin{remark}
	More generally, fix a simply connected topological space $Z$. Then given a map $f\colon Z\to X$ and a choice of $\widetilde x\in p^{-1}(\{x\})$ and $z\in f^{-1}(\{x\})$, there will be a unique lift $\widetilde f\colon Z\to Y$ such that $\widetilde f(z)=\widetilde x$. In short, given any $z'\in Z$, find a path connecting $z$ and $z'$, send this path into $X$ and then lift it up to $Y$. Because $Z$ is simply connected (and the above theorem), the choice of path from $z$ to $z'$ does not really matter.
\end{remark}
Anyway, we now define our collection of paths.
\begin{definition}[fundamental group]
	Fix a point $x$ of a topological space $X$. Then the set of paths both of whose endpoints are $x$ forms a monoid with operation given by composition (i.e., concatenation). If we take the quotient of this monoid by homotopy classes of paths, then we get a group of path homotopy classes, which we call $\pi_1(X,x)$. This is the \textit{fundamental group}.
\end{definition}
\begin{remark}
	For any two $x,y\in X$ in the same path-connected component, the path $\alpha\colon[0,1]\to X$ connecting $x$ to $y$ produces an isomorphism $\pi_1(X,x)\cong\pi_1(X,y)$ by $\gamma\mapsto\alpha\cdot\gamma\cdot\alpha^{-1}$, where $\cdot$ denotes path composition.
\end{remark}
\begin{remark}
	The above remark allows us to verify that $X$ is simply connected if and only $\pi_1(X,x)$ is trivial for all $x$. In fact, we only have to check this for one $x$ in each path-connected component.
\end{remark}
There is more or less one covering space which produces all the other ones.
\begin{definition}[universal cover]
	Fix a path-connected topological space $X$. Then a covering space $p\colon Y\to X$ is the \textit{universal cover} if and only if $Y$ is connected and simply connected.
\end{definition}

\end{document}