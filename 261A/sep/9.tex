% !TEX root = ../notes.tex

\documentclass[../notes.tex]{subfiles}

\begin{document}

\section{September 9}
Today we talk more about subgroups and coverings.

\subsection{Closed Lie Subgroups}
Arbitrary subgroups of Lie groups may not inherit a manifold structure, so we add an adjective to acknowledge this.
\begin{definition}[closed Lie subgroup]
	Fix a Lie group $G$. A \textit{closed Lie subgroup} is a subgroup $H\subseteq G$ which is also an embedded submanifold.
\end{definition}
\begin{remark}
	On the homework, we will show that closed Lie subgroups are in fact closed subsets of $G$. It is a difficult theorem (which we will not prove nor use in this class) that being a closed subset and a subgroup implies that it is an embedded submanifold.
\end{remark}
Here are some checks on subgroups.
\begin{lemma} \label{lem:neighborhood-generates}
	Fix a connected topological group $G$. Given an open neighborhood $U$ of $G$ of the identity $e\in G$, the group $G$ is generated by $U$ is all of $G$.
\end{lemma}
\begin{proof}
	Let $H$ be the subgroup generated by $U$. For each $h\in H$, we see that $hU\subseteq H$, which is an open neighborhood (see \Cref{lem:mult-diffeo}), so $H\subseteq G$ is open. However, we also see that
	\[G=\bigsqcup_{[g]}gH,\]
	where $[g]$ varies over representatives of cosets. Thus, $G\setminus H$ is again the union of open subsets, so $H$ is also closed, so $G=H$ because $G$ is connected.
\end{proof}
\begin{lemma}
	Fix a homomorphism $f\colon G_1\to G_2$ of connected Lie groups. If $df_e\colon T_eG_1\to T_eG_2$ is surjective, then $f$ is surjective.
\end{lemma}
\begin{proof}
	By translating around (by \Cref{lem:mult-diffeo}), we see that $f$ is a submersion. (Explicitly, for each $g\in G_1$, we see that $R_{f(g)}\circ f$ must continue to be a submersion at the identity, but this equals $f\circ R_g$, so $f$ is a submersion at $g$ too.) Because submersions are open \cite[Proposition~4.28]{lee-manifolds}, we see that $f$ being a submersion means that its image is an open subgroup of $G_2$, which is all of $G_2$ by \Cref{lem:neighborhood-generates}.
\end{proof}
Here is a check to be a closed Lie subgroup.
\begin{lemma} \label{lem:one-slice-subgroup}
	Fix a regular Lie group $G$ of dimension $n$. A subgroup $H\subseteq G$ is a closed Lie subgroup of dimension $k$ if and only if there is a single regular chart $(U,\varphi)$ with $e\in U$ such that
	\[U\cap H=\{g\in U:\varphi_{k+1}(g)=\cdots=\varphi_n(g)=0\}\]
	for some.
\end{lemma}
\begin{proof}
	We have constructed a slice chart for the identity $e\in H$. We will translate this slice chart around to produce a slice chart for arbitrary $h\in H$, which will complete the proof by \cite[Theorem~5.8]{lee-manifolds}. In particular, for any $h\in H$, we know that left translation $L_{h^{-1}}\colon G\to G$ is a diffeomorphism, so the composite
	\[hU\stackrel{L_{h^{-1}}}\to U\stackrel{\varphi}\to\varphi(U)\]
	continues to be a chart of $G$ with $h\in hU$. Furthermore, we see that $g\in hU$ lives in $H$ if and only if $L_{h^{-1}}g\in U\cap H$, which by hypothesis is equivalent to
	\[\varphi_{k+1}\left(L_{h^{-1}}g\right)=\cdots=\varphi_n\left(L_{h^{-1}}g\right)=0.\]
	Thus, we have constructed the desired slice chart.
\end{proof}
We may want some more flexibility with our subgroups.
\begin{example}
	Fix an irrational number $\alpha\in\RR$. Then there is a Lie group homomorphism $f\colon\RR\to(\RR/\ZZ)^2$ defined by $f(t)\coloneqq(t,\alpha t)$. One can check that $\im f\subseteq(\RR/\ZZ)^2$ is a dense subgroup, but it is not closed!
\end{example}
So we have the following definition.
\begin{definition}[Lie subgroup]
	Fix a Lie group $G$. A \textit{Lie subgroup} is a subgroup $H\subseteq G$ which is an immersed submanifold.
\end{definition}

\subsection{Quotient Groups}
Along with subgroups, we want to be able to take quotients.
\begin{definition}[fiber bundle]
	A \textit{fiber bundle} with fiber $F$ on a smooth manifold $X$ is a surjective continuous map $\pi\colon Y\to X$ such that there is an open cover $\mc U$ of $X$ and (local) homeomorphisms making the following diagram commute for all $U\in\mc U$.
	% https://q.uiver.app/#q=WzAsMyxbMCwwLCJcXHBpXnstMX0oVSkiXSxbMiwwLCJGXFx0aW1lcyBVIl0sWzEsMSwiVSJdLFsxLDIsIlxcb3B7cHJ9XzIiXSxbMCwxLCJcXHNpbSJdLFswLDIsIlxccGkiLDJdXQ==&macro_url=https%3A%2F%2Fraw.githubusercontent.com%2FdFoiler%2Fnotes%2Fmaster%2Fnir.tex
	\[\begin{tikzcd}
		{\pi^{-1}(U)} && {F\times U} \\
		& U
		\arrow["\sim", from=1-1, to=1-3]
		\arrow["\pi"', from=1-1, to=2-2]
		\arrow["{\op{pr}_2}", from=1-3, to=2-2]
	\end{tikzcd}\]
\end{definition}
Fiber bundles are the correct way to discuss quotients.
\begin{theorem} \label{thm:quotient-group}
	Fix a closed Lie subgroup $H$ of a Lie group $G$.
	\begin{listalph}
		\item Then $G/H$ is a manifold of dimension $\dim G-\dim H$ equipped with a quotient map $q\colon G\onto G/H$.
		\item In fact, $q$ is a fiber bundle with fiber $H$.
		\item If $H$ is normal in $G$, then $G/H$ is actually a Lie group (with the usual group structure).
		\item We have $T_e(G/H)\cong T_eG/T_eH$.
	\end{listalph}
\end{theorem}
\begin{proof}
	We construct the manifold structure on $G/H$ as follows: for each $g\in G$, we produce a coset $\ov g\in G/H$, which we note as $q^{-1}(\ov g)=gH$. Now, $gH\subseteq G$ is an embedded submanifold because $H$ is (we are using \Cref{lem:mult-diffeo}), so one can locally find a submanifold $M\subseteq G$ around $g$ intersecting $gH$ transversally, meaning that
	\[T_gG=T_gM\oplus T_g(gH).\]
	By shrinking $M$, we can ensure that the above map continues to be an isomorphism in a neighborhood of $g$, so the multiplication map $M\times H\to UH$ is a diffeomorphism. Now, $MH$ is an open neighborhood of $g\in G$, and $M$ projects down to an open subset of $G/H$, so $M\cong q(\ov M)$ provides our chart.

	Now, (a) and (d) follows by inspection of the construction. We see that (b) follows because we built our projection map $G\onto G/H$ so that it locally looks like $U\times H\onto\ov U$, so we get our fiber bundle. Lastly, (d) follows by the equality $T_gG=T_gM\oplus T_g(gH)$.
\end{proof}
\begin{remark}
	Writing the above out in detail would take several pages; see \cite[Theorem~21.10]{lee-manifolds}.
\end{remark}
Access to quotients permits an isomorphism theorem, which we will prove later when we have talked a bit about Lie algebras.
\begin{theorem}[Isomorphism] \label{thm:isomorphism}
	Fix a Lie group homomorphism $f\colon G_1\to G_2$.
	\begin{listalph}
		\item The kernel $\ker f$ is a normal closed Lie subgroup of $G_1$.
		\item The quotient $G_1/\ker f$ is a Lie subgroup of $G_2$.
		\item The image $\im f$ is a Lie subgroup of $G_2$. If $\im f$ is further closed, then $G_1/\ker f\to\im f$ is an isomorphism of Lie subgroups.
	\end{listalph}
\end{theorem}

\subsection{Actions}
Groups are known by their actions, so let's think about how our actions behave.
\begin{definition}[action]
	Fix a Lie group $G$ and regular manifold $X$. Then a \textit{regular action} of $G$ on $X$ is a regular map $\alpha\colon G\times X\to X$ satisfying the usual constraints, as follows.
	\begin{listalph}
		\item Identity: $\alpha(e,x)\coloneqq x$.
		\item Composition: $\alpha(g,\alpha(h,x))=\alpha(gh,x)$.
	\end{listalph}
\end{definition}
This allows us to define the usual subsets.
\begin{defihelper}[orbit, stabilizer] \nirindex{orbit} \nirindex{stabilizer}
	Fix a regular action of a Lie group $G$ on a regular manifold $X$.
	\begin{listalph}
		\item The \textit{orbit} of $x\in X$ is the subspace $Gx\coloneqq\{gx:g\in G\}$.
		\item The \textit{stabilizer} of $x\in X$ is the subgroup $G_x\coloneqq\{g\in G:gx=x\}$.
	\end{listalph}
\end{defihelper}
Here are some examples.
\begin{example}
	The group $\op{GL}_n(\FF)$ acts on the vector space $\FF^n$.
\end{example}
\begin{example} \label{ex:so3-action}
	The group $\op{SO}_3(\RR)$ preserves distances in its action on $\RR^3$, so its action descends to an action on $S^2$.
\end{example}
Representations are special kinds of actions.
\begin{definition}[representation]
	Fix a Lie group $G$ over $\FF$. Then a \textit{representation} of $G$ is the (regular) linear action of $G$ on a finite-dimensional vector space $V$ over $\FF$; namely, the map $v\mapsto g\cdot v$ for each $g\in G$ must be a linear map $V\to V$. A \textit{homomorphism} of representations $V$ and $W$ is a linear map $A\colon V\to W$ such that $A(gv)=g(Av)$. These objects and morphisms make the category $\op{Rep}_{\FF}(G)$.
\end{definition}
\begin{remark}
	Equivalently, we may ask for the induced map $G\to\op{GL}(V)$, given by sending $g\in G$ to the map $v\mapsto gv$, to be a Lie group homomorphism.
\end{remark}
\begin{remark}
	The category $\op{Rep}_{\FF}(G)$ comes with many nice operations.
	\begin{listalph}
		\item Duals: given a representation $\pi\colon G\to\op{GL}(V)$, we can induce a $G$-action on $V^*\coloneqq\op{Hom}(V,\FF)$ by
		\[\big((\pi^*g)(v^*)\big)(v)\coloneqq v^*\left(g^{-1}v\right).\]
		(Here, $\pi^*g$ should be a map $V^*\to V^*$, so it takes a linear functional $v^*\in V^*$ as input and produces the linear functional $(\pi^*g)(v^*)$ as output.)
		\item Tensor products: given representations $\pi\colon G\to\op{GL}(V)$ and $\pi'\colon G\to\op{GL}(V')$, we can induce a $G$-action on $V\otimes W$ by
		\[\big((\pi\otimes\pi')(g\otimes g')\big)(v\otimes v')\coloneqq\pi(g)v\otimes\pi'(g')v'.\]
		\item Quotients: given representations $\pi\colon G\to\op{GL}(V)$ and $\pi'\colon G\to\op{GL}(V')$, where $V\subseteq V'$ is a $G$-representation, then we can induce a $G$-action on $V'/V$ by
		\[\pi'(g)(v'+V)\coloneqq gv'+V.\]
	\end{listalph}
	One can check that these operations make $\op{Rep}_{\FF}(G)$ into a symmetric monoidal abelian category.
\end{remark}
Returning to group actions on manifolds, we remark that \Cref{thm:isomorphism} can be seen as a version of the Orbit--stabilizer theorem.
\begin{theorem}[Orbit--stabilizer] \label{thm:orb-stab}
	Fix a regular action of a Lie group $G$ on a regular manifold $X$. Further, fix $x\in X$.
	\begin{listalph}
		\item The orbit $Gx$ is an immersed submanifold of $X$.
		\item The stabilizer $G_x$ is a closed Lie subgroup of $G$.
		\item The quotient map $f\colon G/G_x\to X$ given by $g\mapsto gx$ is an injective immersion.
		\item If $Gx$ is an embedded submanifold, then the map $f$ of (c) is a diffeomorphism.
	\end{listalph}
\end{theorem}

\end{document}