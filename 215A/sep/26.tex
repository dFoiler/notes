% !TEX root = ../notes.tex

\documentclass[../notes.tex]{subfiles}

\begin{document}

\section{September 26}

Today we discuss free groups and graphs.

\subsection{Spanning Tree}
For technical reasons, it will be helpful to rigorize give graphs a CW topology.
\begin{definition}[graph]
	A \textit{graph} is a $1$-dimensional CW complex $X$ built as follows: the vertices are $X^0$, and the edges are built by taking two vertices $v_1,v_2\in X^0$ and connecting them by an edge $e_\alpha$ with $\del e_\alpha=\{v_1,v_2\}$.
\end{definition}
\begin{remark}
	A graph $X$ with a vertex $v\in X$ of infinite degree fails to be locally compact. Indeed, any open neighborhood of $v$ will intersect infinitely many edges, which is not contained in any compact set because one can build an open cover with an open set from each of the individual edges, from which no finite subcover is possible to construct.
\end{remark}
\begin{definition}[subgraph]
	A \textit{subgraph} is a closed CW subcomplex of a graph.
\end{definition}
Trees are the simplest graphs.
\begin{definition}[tree]
	A \textit{tree} is a contractible graph. A subtree $T$ of a graph $X$ is \textit{maximal} or \textit{spanning} if and only if $T^0=X^0$.
\end{definition}
\begin{example}
	The highlighted edges make a maximal subtree of the following graph.
	% https://q.uiver.app/#q=WzAsMTQsWzEsMCwiXFxidWxsZXQiXSxbMiwwLCJcXGJ1bGxldCJdLFszLDAsIlxcYnVsbGV0Il0sWzQsMCwiXFxidWxsZXQiXSxbNSwwLCJcXGJ1bGxldCJdLFsxLDEsIlxcYnVsbGV0Il0sWzIsMSwiXFxidWxsZXQiXSxbMywxLCJcXGJ1bGxldCJdLFs0LDEsIlxcYnVsbGV0Il0sWzUsMSwiXFxidWxsZXQiXSxbMCwwLCJcXGNkb3RzIl0sWzAsMSwiXFxjZG90cyJdLFs2LDAsIlxcY2RvdHMiXSxbNiwxLCJcXGNkb3RzIl0sWzEwLDAsIiIsMCx7ImNvbG91ciI6WzAsNjAsNjBdLCJzdHlsZSI6eyJoZWFkIjp7Im5hbWUiOiJub25lIn19fV0sWzAsMSwiIiwwLHsiY29sb3VyIjpbMCw2MCw2MF0sInN0eWxlIjp7ImhlYWQiOnsibmFtZSI6Im5vbmUifX19XSxbMSwyLCIiLDAseyJjb2xvdXIiOlswLDYwLDYwXSwic3R5bGUiOnsiaGVhZCI6eyJuYW1lIjoibm9uZSJ9fX1dLFsyLDMsIiIsMCx7ImNvbG91ciI6WzAsNjAsNjBdLCJzdHlsZSI6eyJoZWFkIjp7Im5hbWUiOiJub25lIn19fV0sWzMsNCwiIiwwLHsiY29sb3VyIjpbMCw2MCw2MF0sInN0eWxlIjp7ImhlYWQiOnsibmFtZSI6Im5vbmUifX19XSxbNCwxMiwiIiwwLHsiY29sb3VyIjpbMCw2MCw2MF0sInN0eWxlIjp7ImhlYWQiOnsibmFtZSI6Im5vbmUifX19XSxbMTEsNSwiIiwwLHsiY29sb3VyIjpbMCw2MCw2MF0sInN0eWxlIjp7ImhlYWQiOnsibmFtZSI6Im5vbmUifX19XSxbNSw2LCIiLDAseyJjb2xvdXIiOlswLDYwLDYwXSwic3R5bGUiOnsiaGVhZCI6eyJuYW1lIjoibm9uZSJ9fX1dLFs2LDcsIiIsMCx7ImNvbG91ciI6WzAsNjAsNjBdLCJzdHlsZSI6eyJoZWFkIjp7Im5hbWUiOiJub25lIn19fV0sWzcsOCwiIiwwLHsiY29sb3VyIjpbMCw2MCw2MF0sInN0eWxlIjp7ImhlYWQiOnsibmFtZSI6Im5vbmUifX19XSxbOCw5LCIiLDAseyJjb2xvdXIiOlswLDYwLDYwXSwic3R5bGUiOnsiaGVhZCI6eyJuYW1lIjoibm9uZSJ9fX1dLFs5LDEzLCIiLDAseyJjb2xvdXIiOlswLDYwLDYwXSwic3R5bGUiOnsiaGVhZCI6eyJuYW1lIjoibm9uZSJ9fX1dLFs0LDksIiIsMSx7InN0eWxlIjp7ImhlYWQiOnsibmFtZSI6Im5vbmUifX19XSxbMyw4LCIiLDEseyJzdHlsZSI6eyJoZWFkIjp7Im5hbWUiOiJub25lIn19fV0sWzIsNywiIiwxLHsiY29sb3VyIjpbMCw2MCw2MF0sInN0eWxlIjp7ImhlYWQiOnsibmFtZSI6Im5vbmUifX19XSxbMSw2LCIiLDEseyJzdHlsZSI6eyJoZWFkIjp7Im5hbWUiOiJub25lIn19fV0sWzAsNSwiIiwxLHsic3R5bGUiOnsiaGVhZCI6eyJuYW1lIjoibm9uZSJ9fX1dXQ==&macro_url=https%3A%2F%2Fraw.githubusercontent.com%2FdFoiler%2Fnotes%2Fmaster%2Fnir.tex
	\[\begin{tikzcd}
		\cdots & \bullet & \bullet & \bullet & \bullet & \bullet & \cdots \\
		\cdots & \bullet & \bullet & \bullet & \bullet & \bullet & \cdots
		\arrow[color={rgb,255:red,214;green,92;blue,92}, no head, from=1-1, to=1-2]
		\arrow[color={rgb,255:red,214;green,92;blue,92}, no head, from=1-2, to=1-3]
		\arrow[color={rgb,255:red,214;green,92;blue,92}, no head, from=1-3, to=1-4]
		\arrow[color={rgb,255:red,214;green,92;blue,92}, no head, from=1-4, to=1-5]
		\arrow[color={rgb,255:red,214;green,92;blue,92}, no head, from=1-5, to=1-6]
		\arrow[color={rgb,255:red,214;green,92;blue,92}, no head, from=1-6, to=1-7]
		\arrow[color={rgb,255:red,214;green,92;blue,92}, no head, from=2-1, to=2-2]
		\arrow[color={rgb,255:red,214;green,92;blue,92}, no head, from=2-2, to=2-3]
		\arrow[color={rgb,255:red,214;green,92;blue,92}, no head, from=2-3, to=2-4]
		\arrow[color={rgb,255:red,214;green,92;blue,92}, no head, from=2-4, to=2-5]
		\arrow[color={rgb,255:red,214;green,92;blue,92}, no head, from=2-5, to=2-6]
		\arrow[color={rgb,255:red,214;green,92;blue,92}, no head, from=2-6, to=2-7]
		\arrow[no head, from=1-6, to=2-6]
		\arrow[no head, from=1-5, to=2-5]
		\arrow[color={rgb,255:red,214;green,92;blue,92}, no head, from=1-4, to=2-4]
		\arrow[no head, from=1-3, to=2-3]
		\arrow[no head, from=1-2, to=2-2]
	\end{tikzcd}\]
\end{example}
We have the following result on trees.
\begin{proposition}
	Any connected graph $X$ contains a maximal tree. In fact, any subtree can be extended to a maximal tree.
\end{proposition}
\begin{proof}
	We begin by fixing some subtree $X_0\subseteq X$. Then to construct $X_{n+1}$ from $X_n$, we look at the set of vertices adjacent to a vertex in $X_n$, and we add exactly one edge to $X_{n+1}$ to add in all these vertices. Each added edge maintains being contractible, and adding them all in at once will continue to be contractible; explicitly, $X_{n+1}$ has a deformation retract back to $X_n$ and will therefore be contractible by induction.
	
	Eventually the union $T$ of $X_0\subseteq X_1\subseteq X_2\subseteq\cdots$ will hit every vertex: note $X$ is connected and locally path-connected hence path-connected, so it follows that any two vertices can be connected by a path, which may only hit finitely many vertices and edges along its path by compactness of the interval.\footnote{Hitting infinitely many vertices or edges implies that the image of $[0,1]$ has an infinite discrete closed subset (choose a single point from each vertex and from each hit edge), violating compactness.} Thus, $T$ is the desired subtree.
\end{proof}
\begin{remark}
	One needs some form of the axiom of choice to achieve the above result because we may be making infinitely many choices in the construction of $X_{n+1}$ from $X_n$.
\end{remark}

\subsection{Fundamental Groups of Graphs}
Having spanning trees allows us to compute fundamental groups. Fix a spanning tree $T\subseteq X$. Fix a basepoint $x_0\in T$. Then each edge $e_\alpha$ of $X\setminus T$ produces a loop based at $x_0$: if $e_\alpha$ connects $v_1$ and $v_2$, then we have a loop going from $x_0$ to $v_1$ (through $T$) to $v_2$ (through $e_\alpha$) and back to $x_0$ (through $T$ again). These loops generate the fundamental group.
\begin{proposition} \label{prop:fund-group-of-graph}
	Fix a connected graph $X$ with spanning tree $T$. Then $\pi_1(X)$ is a free group with basis $[e_\alpha]$ where $e_\alpha$ is an edge of $X\setminus T$.
\end{proposition}
\begin{proof}
	The quotient map $X\onto X/T$ is a homotopy equivalence because $T$ is contractible (it's a tree). However, $X/T$ now only has a single vertex $x_0$, and we see that each edge $e_\alpha$ of $X\setminus T$ then goes down to a loop at $x_0$. Thus, $X/T$ is $S^1$ wedged with itself once for each edge in $X\setminus T$, so the result follows.
\end{proof}
Our work allows us the following application.
\begin{lemma}
	Every covering space of a graph $X$ is itself a graph whose vertices and edges as pre-images.
\end{lemma}
\begin{proof}
	Let $p\colon\widetilde X\to X$ be a covering space. Set vertices of $\widetilde X$ to be $p^{-1}\left(X^0\right)$, and our edges are similarly given by pre-images because $p$ is locally a homeomorphism, we see that $\widetilde X$ has the desired topology.
\end{proof}
\begin{theorem}
	Any subgroup of a free group is free.
\end{theorem}
\begin{proof}
	A free group $F$ generated by $\kappa$ generators is the fundamental group of the graph $X\coloneqq\left(S^1\right)^\kappa$. Then any subgroup $F'\subseteq F$ arises from the fundamental group of the covering space $p\colon\widetilde X\to X$, and the lemma tells us that $\widetilde X$ is a graph, so its fundamental group is in fact also free by \Cref{prop:fund-group-of-graph}.
\end{proof}
The above result is quite nice: it is quite non-obvious that this result should be true purely from the algebra, but the topology makes it easier to attack.
\begin{remark}
	There is an algorithm (due to Reidemeister--Schreier) to find a generating set for finite-index subgroups of a free group.
\end{remark}

\subsection{\texorpdfstring{$K(G,1)$s}{ K(G,1) s}}
We have the following definition.
\begin{defihelper}[$K(G,1)$] \nirindex{K(G,1)@$K(G,1)$}
	Fix a group $G$. A path-connected topological space $X$ is a $K(G,1)$ if and only if $\pi_1(X)\cong G$, and $X$ has a contractible universal cover.
\end{defihelper}
It turns out that $K(G,1)$ is unique up to homotopy equivalence. Here are some examples.
\begin{example}
	The space $\mathbb{RP}^\infty$ is a $K(\ZZ/2\ZZ,1)$. The fundamental group can be computed by seeing that the universal cover is $S^\infty\onto\mathbb{RP}^\infty$. Let's see that $S^\infty$ is in fact contractible: the map $(x_1,x_2,x_3,\ldots)\mapsto(0,x_1,x_2,\ldots)$ defines an embedding $i\colon S^\infty\to S^\infty$. However, $i$ has a linear homotopy to $\id$ given by
	\[f_t(x_1,x_2,\ldots)\coloneqq\frac{(1-t)(x_1,x_2,\ldots)+t(0,x_1,\ldots)}{\norm{(1-t)(x_1,x_2,\ldots)+t(0,x_1,\ldots)}},\]
	and then $i$ has a linear homotopy to a constant map by
	\[g_t(x_1,x_2,\ldots)\coloneqq\frac{(1-t)(1,0,\ldots)+t(0,x_1,\ldots)}{\norm{(1-t)(1,0,\ldots)+t(0,x_1,\ldots)}}.\]
	(Note we needed the inclusion $i$ because the linear combination $(1-t)(1,0,\ldots)+t(x_1,x_2,\ldots)$ goes through the origin if we use the point $(x_1,x_2,\ldots)=(-1,0,\ldots)$.)
\end{example}
\begin{example}
	The space $S^\infty/(\ZZ/m\ZZ)$ is a $K(\ZZ/m\ZZ,1)$. Here, $\ZZ/m\ZZ$ acts on $S^\infty$ by having $1\in\ZZ/m\ZZ$ be pointwise multiplication by $e^{2\pi i/m}$. The covering space is still $S^\infty$, which is contractible by the previous example.
\end{example}
\begin{example}
	Fix a closed, connected subspace $K\subseteq S^3$ (thought of as a knot). If $G\coloneqq\pi_1\left(S^3\setminus K\right)$, then $S^3\setminus K$ is a $K(G,1)$; this is a result to Papakyriakopoulos (yes, this name is hard to spell). Note that having $S^3$ is important; otherwise, if $K$ is bounded, we could just place a large box around $K\subseteq\RR^3$, and it is not possible to contract this box in $\RR^3\setminus K$. Instead, we want to contract it in $S^3$ by passing to the point at infinity.
\end{example}
\begin{example}
	Let $X_G$ be a $K(G,1)$, and let $X_H$ be a $K(H,1)$, and we assume that both are CW complexes. Then $X_G\times X_H$ (given the product topology!) becomes a $K(G\times H,1)$ because the universal cover of $X_G\times X_H$ is the product of the universal covers, which will then remain contractible.
\end{example}
Taking a product of $K(\ZZ/m\ZZ,1)$s, we see that there is a $K(G,1)$ for a finitely generated abelian group $G$. One can in fact give a $K(G,1)$ for any group $G$, though this trickier. Let's see this. The following notions will be helpful.
\begin{definition}[simplex]
	An \textit{$n$-simplex} is constructed by taking affinely linearly independent vectors $v_0,\ldots,v_n\in\RR^m$ (i.e., the set $\{v_1-v_0,\ldots,v_n-v_0\}$ is linearly independent---note that this condition is independent of rearranging the $v_\bullet$) and setting
	\[[v_0,v_1,\ldots,v_n]\coloneqq\left\{\sum_{i=0}^nt_iv_i:0\le t_i\text{ for each }i\text{ and }\sum_{i=1}^nt_i=1\right\}.\]
	Namely, $[v_0,v_1,\ldots,v_n]$ is the convex hull of the $v_\bullet$; a \textit{face} of this $n$-simplex is an $(n-1)$-simplex of the form $[v_0,\ldots,\widehat v_i,\ldots,v_n]$ attained by deleting one of the vertices $v_i$. Then the boundary of the $n$-simplex is
	\[\del[v_0,\ldots,v_n]\coloneqq\bigsqcup_{i=1}^n[v_0,\ldots\widehat v_i,\ldots,v_n],\]
	and the interior is defined in the obvious way.
\end{definition}
\begin{defihelper}[$\Delta$-complex] \nirindex{Delta-complex@$\Delta$-complex}
	A \textit{$\Delta$-complex} is a CW complex $X$ such that the cells $e^n_\alpha$ are homeomorphic to $(\Delta^n)^\circ$, where we require that the attaching maps $\varphi_\alpha\colon\del\Delta^n\to X^{n-1}$ restricts to a face $\varphi_\alpha|_{[v_0,\ldots,\widehat v_i,\ldots,v_n]}$ is an attaching map $\varphi_\beta\colon\Delta^{n-1}\to X^{n-1}$ for some $\beta$.
\end{defihelper}
\begin{example}[dunce cap] \label{ex:dunce-cap-delta-complex}
	Glue the following $2$-simplex to a $1$-simplex following the arrows.
	\begin{center}
		\begin{asy}
			unitsize(1cm);
			pair A = dir(90);
			pair B = dir(90+120);
			pair C = dir(90-120);
			draw(A -- B, arrow=EndArrow(Relative(0.5)));
			draw(A -- C, arrow=EndArrow(Relative(0.5)));
			draw(B -- C, arrow=EndArrow(Relative(0.5)));
			dot(A ^^ B ^^ C);
		\end{asy}
	\end{center}
	This is weird, but we allow it.
\end{example}
We now describe $K(G,1)$ for a general group $G$. We begin by constructing the universal cover $EG$, which will be a $\Delta$-complex. The vertices of $EG$ are elements of $G$. Then the $n$-simplices of $EG$ (for $n\ge1$) are simply $[g_0,\ldots,g_n]$ attached to the $(n-1)$-simplices $[g_0,\ldots,\widehat g_i,\ldots,g_n]$ in the obvious way.
\begin{example}
	Take $G$ to be the trivial group. Then we have a single $n$-simplex $[e,\ldots,e]$ for each $n$. For example, the two-simplex $[e,e]$ is attaching at its ends to a single vertex. Then the $[e,e,e]$ is attaching its edges to the loops as in \Cref{ex:dunce-cap-delta-complex}.
\end{example}
\begin{example}
	Take $G$ to be $\ZZ/2\ZZ=\{0,1\}$. Then we have $2^{n+1}$ total $n$-simplices.
\end{example}
Note that $G$ acts freely on $EG$ by multiplication of the vertices, so we produce a covering space $EG\to BG$, where $BG\coloneqq EG/G$. We claim that $EG$ is contractible, which will complete our construction with $BG$ as our $K(G,1)$. Indeed, inside any $n$-simplex $[g_0,\ldots,g_n]$, we embed it into $[e,g_0,\ldots,g_n]$ and then use the linear homotopy to the identity $e$. This will be well-defined with respect to our gluing, so we have indeed produced contraction.

\end{document}