% !TEX root = ../notes.tex

\documentclass[../notes.tex]{subfiles}

\begin{document}

\section{September 14}

We're talking about covering spaces today.

\subsection{Examples of Covering Spaces}
Our goal is to generalize the method we used to compute $\pi_1\left(S^1\right)$. Let's recall our definition.
\coveringspacedef*
\begin{example}
	The map $p\colon\RR\to S^1$ given by $t\mapsto e^{2\pi it}$ is a covering space map. Here, we are viewing $S^1$ as $\{z\in\CC:\left|z\right|=1\}$. The point is that, for any $e^{2\pi i\theta}\in S^1$, we have
	\[p^{-1}\left(S^1\setminus\left\{e^{2\pi i\theta}\right\}\right)=\bigsqcup_{n\in\ZZ}(\theta,\theta+2\pi).\]
\end{example}
\begin{nex}
	The map $p\colon(0,2)\to S^1$ given by $t\mapsto e^{2\pi it}$ is not a covering space map. For example, any open interval $U$ around $1\in S^1$ will have pre-image by $p$ looking like $(0,\varepsilon)\sqcup(1-\varepsilon,1+\varepsilon)\sqcup(2-\varepsilon,2)$, and $(0,\varepsilon)$ is not mapped homeomorphically to our $U\subseteq S^1$.
\end{nex}
\begin{example}
	The map $f\colon\CC^\times\to\CC^\times$ given by $f\colon z\mapsto z^n$ for a positive integer $n$ is a covering space map. Roughly speaking, for any ray $\ell$ through the origin in $\CC$, one can define $\log\colon(\CC\setminus\ell)\to\CC$, which allows us to define an $n$th root $\sqrt[n]w\coloneqq\exp\left(\frac1n\log w\right)$ on $\CC\setminus\ell$; this makes $\CC\setminus\ell$ into an evenly covered subset, so we are a covering space upon letting $\ell$ vary.
\end{example}
\begin{example}
	Fix a topological space $X$ and a discrete set $E$. Then of course $p\colon X\times E\to X$ is a covering space: indeed, $X$ is an evenly covered subset. In fact, if $p\colon\widetilde X\to X$ is a covering space map where $X$ is evenly covered, then the definition of $p$ requires $\widetilde X\cong\bigsqcup_{\alpha\in\lambda}X_\alpha$
\end{example}
\begin{example}
	Map $p\colon S^\infty\to\mathbb{RP}^\infty$ by sending $x\in S^\infty$ to the corresponding line in $\mathbb{RP}^\infty$. More precisely, embed some $S^n\subseteq S^\infty$ into $\RR^{n+1}$ and then take lines down to $\mathbb{RP}^n$. Notably, $p(x)=p(-x)$ for each $x$ (and conversely $p(x)=p(y)$ if and only if $\RR x=\RR y$ if and only if $x=\pm y$), so $p$ is $2$-to-$1$. One can check that $p$ is a covering space map by looking on the level of cell complexes: the pre-image of the interior of the unique $n$-cell $\left(e^n\right)^\circ\subseteq\mathbb{RP}^\infty$ is the disjoint union of the interior of the two $n$-cells of $S^\infty$. More precisely, the $n$-cell $e^n_i$ inside $\mathbb{RP}^n$ given by
	\[\{[x_0:\cdots:x_{i-1}:1:x_{i+1}:\cdots:x_n]:x_0,\ldots,x_{i-1},x_{i+1},\ldots,x_n\in\RR\}\]
	is evenly covered in the map $S^n\to\mathbb{RP}^n$. One can extend this idea up to $\mathbb{RP}^\infty$ to conclude: let $e_i$ be the above subset except we do not terminate at $x_n$, and then $e_i$ is covered by the open subsets $e_{i,\pm}\subseteq S^\infty$ defined as
	\[e_{i,\pm}=\left\{(x_0,x_1,\ldots)\in S^\infty:x_i\text{ has sign }\pm\right\}.\]
\end{example}
Let's do a few examples on $S^1\lor S^1$.
\begin{example} \label{ex:two-to-one-s1-wedge-s1}
	We examine $2$-fold (i.e., $2$-to-$1$) covers of $S^1\lor S^1$. There is the trivial one with two copies of $S^1\lor S^1$. As another example, note that $S^1\lor S^1\lor S^1$ loop around $S^1\lor S^1$ twice: the first $S^1$ goes around the first $S^1$, then half of the second $S^1$ goes around the second $S^1$, then the third $S^1$ goes around the first $S^1$ around. Here is an image.
	\begin{center}
		\begin{asy}
			unitsize(1cm);
			draw((-3,0)..(-2,1)..(-1,0)..(-2,-1)..cycle, red, arrow=Arrow);
			draw((0,-1)..(-1,0)..(0,1), blue, arrow=Arrow);
			draw((0,1)..(1,0)..(0,-1), blue, arrow=Arrow);
			draw((3,0)..(2,-1)..(1,0)..(2,1)..cycle, red, arrow=Arrow);
		\end{asy}
	\end{center}
	It turns out that, with one more, these are all the $2$-to-$1$ covering maps, which can be seen by finding index-$2$ subgroups of $\pi_1\left(S^1\lor S^1\right)=\ZZ*\ZZ$, as we will soon see.
\end{example}
\begin{example}
	Consider the grid $\ZZ\times\RR\cup\RR\times\ZZ$. This is then a covering space of $S^1\lor S^1$ by sending the $\ZZ\times\RR$ to traverse one of the circles $S^1$ and the $\RR\times\ZZ$ to traverse the other circle of $S^1$. More generally, it turns out that covering spaces are exactly graphs where every vertex has degree $4$, which we can see by coloring the edges red and blue so that each vertex has exactly two red edges and two blue edges; then choosing an Euler cycle provides the needed covering space. The previous example is one way to do this. Here is another example of such a graph, with marked Euler cycle.
	% https://q.uiver.app/#q=WzAsOCxbMSwyLCJcXGJ1bGxldCJdLFsxLDEsIlxcYnVsbGV0Il0sWzIsMSwiXFxidWxsZXQiXSxbMiwyLCJcXGJ1bGxldCJdLFszLDIsIlxcYnVsbGV0Il0sWzIsMywiXFxidWxsZXQiXSxbMCwyLCJcXGJ1bGxldCJdLFsyLDAsIlxcYnVsbGV0Il0sWzAsMSwiNSIsMSx7ImNvbG91ciI6WzAsNjAsNjBdLCJzdHlsZSI6eyJoZWFkIjp7Im5hbWUiOiJub25lIn19fSxbMCw2MCw2MCwxXV0sWzEsMiwiNiIsMSx7ImNvbG91ciI6WzAsNjAsNjBdLCJzdHlsZSI6eyJoZWFkIjp7Im5hbWUiOiJub25lIn19fSxbMCw2MCw2MCwxXV0sWzIsMywiMTAiLDEseyJjb2xvdXIiOlsyNDAsNjAsNjBdLCJzdHlsZSI6eyJoZWFkIjp7Im5hbWUiOiJub25lIn19fSxbMjQwLDYwLDYwLDFdXSxbMywwLCIxMSIsMSx7ImNvbG91ciI6WzI0MCw2MCw2MF0sInN0eWxlIjp7ImhlYWQiOnsibmFtZSI6Im5vbmUifX19LFsyNDAsNjAsNjAsMV1dLFs0LDMsIjE0IiwxLHsiY29sb3VyIjpbMCw2MCw2MF0sInN0eWxlIjp7ImhlYWQiOnsibmFtZSI6Im5vbmUifX19LFswLDYwLDYwLDFdXSxbNSwzLCIxMyIsMSx7ImNvbG91ciI6WzAsNjAsNjBdLCJzdHlsZSI6eyJoZWFkIjp7Im5hbWUiOiJub25lIn19fSxbMCw2MCw2MCwxXV0sWzYsMSwiMyIsMSx7ImNvbG91ciI6WzI0MCw2MCw2MF0sInN0eWxlIjp7ImhlYWQiOnsibmFtZSI6Im5vbmUifX19LFsyNDAsNjAsNjAsMV1dLFs3LDEsIjIiLDEseyJjb2xvdXIiOlsyNDAsNjAsNjBdLCJzdHlsZSI6eyJoZWFkIjp7Im5hbWUiOiJub25lIn19fSxbMjQwLDYwLDYwLDFdXSxbNSwwLCIxMiIsMSx7ImNvbG91ciI6WzI0MCw2MCw2MF0sInN0eWxlIjp7ImhlYWQiOnsibmFtZSI6Im5vbmUifX19LFsyNDAsNjAsNjAsMV1dLFs1LDYsIjE2IiwxLHsiY29sb3VyIjpbMjQwLDYwLDYwXSwic3R5bGUiOnsiaGVhZCI6eyJuYW1lIjoibm9uZSJ9fX0sWzI0MCw2MCw2MCwxXV0sWzQsMiwiOSIsMSx7ImNvbG91ciI6WzI0MCw2MCw2MF0sInN0eWxlIjp7ImhlYWQiOnsibmFtZSI6Im5vbmUifX19LFsyNDAsNjAsNjAsMV1dLFs0LDcsIjgiLDEseyJjb2xvdXIiOlsyNDAsNjAsNjBdLCJzdHlsZSI6eyJoZWFkIjp7Im5hbWUiOiJub25lIn19fSxbMjQwLDYwLDYwLDFdXSxbNSw0LCIxNSIsMSx7ImNvbG91ciI6WzAsNjAsNjBdLCJzdHlsZSI6eyJoZWFkIjp7Im5hbWUiOiJub25lIn19fSxbMCw2MCw2MCwxXV0sWzYsMCwiNCIsMSx7ImNvbG91ciI6WzAsNjAsNjBdLCJzdHlsZSI6eyJoZWFkIjp7Im5hbWUiOiJub25lIn19fSxbMCw2MCw2MCwxXV0sWzcsMiwiNyIsMSx7ImNvbG91ciI6WzAsNjAsNjBdLCJzdHlsZSI6eyJoZWFkIjp7Im5hbWUiOiJub25lIn19fSxbMCw2MCw2MCwxXV0sWzYsNywiMSIsMSx7ImN1cnZlIjotMiwiY29sb3VyIjpbMCw2MCw2MF0sInN0eWxlIjp7ImhlYWQiOnsibmFtZSI6Im5vbmUifX19LFswLDYwLDYwLDFdXV0=&macro_url=https%3A%2F%2Fraw.githubusercontent.com%2FdFoiler%2Fnotes%2Fmaster%2Fnir.tex
	\[\begin{tikzcd}
		&& \bullet \\
		& \bullet & \bullet \\
		\bullet & \bullet & \bullet & \bullet \\
		&& \bullet
		\arrow["5"{description}, color={rgb,255:red,214;green,92;blue,92}, no head, from=3-2, to=2-2]
		\arrow["6"{description}, color={rgb,255:red,214;green,92;blue,92}, no head, from=2-2, to=2-3]
		\arrow["10"{description}, color={rgb,255:red,92;green,92;blue,214}, no head, from=2-3, to=3-3]
		\arrow["11"{description}, color={rgb,255:red,92;green,92;blue,214}, no head, from=3-3, to=3-2]
		\arrow["14"{description}, color={rgb,255:red,214;green,92;blue,92}, no head, from=3-4, to=3-3]
		\arrow["13"{description}, color={rgb,255:red,214;green,92;blue,92}, no head, from=4-3, to=3-3]
		\arrow["3"{description}, color={rgb,255:red,92;green,92;blue,214}, no head, from=3-1, to=2-2]
		\arrow["2"{description}, color={rgb,255:red,92;green,92;blue,214}, no head, from=1-3, to=2-2]
		\arrow["12"{description}, color={rgb,255:red,92;green,92;blue,214}, no head, from=4-3, to=3-2]
		\arrow["16"{description}, color={rgb,255:red,92;green,92;blue,214}, no head, from=4-3, to=3-1]
		\arrow["9"{description}, color={rgb,255:red,92;green,92;blue,214}, no head, from=3-4, to=2-3]
		\arrow["8"{description}, color={rgb,255:red,92;green,92;blue,214}, no head, from=3-4, to=1-3]
		\arrow["15"{description}, color={rgb,255:red,214;green,92;blue,92}, no head, from=4-3, to=3-4]
		\arrow["4"{description}, color={rgb,255:red,214;green,92;blue,92}, no head, from=3-1, to=3-2]
		\arrow["7"{description}, color={rgb,255:red,214;green,92;blue,92}, no head, from=1-3, to=2-3]
		\arrow["1"{description}, color={rgb,255:red,214;green,92;blue,92}, curve={height=-12pt}, no head, from=3-1, to=1-3]
	\end{tikzcd}\]
\end{example}
\begin{example}
	We can take a subgroup of $\ZZ*\ZZ$ to produce a covering space of $S^1\lor S^1$. As an example, take the subgroup generated by $ab$ and $b^{-1}ab$. Reading off these generators produces a graph as follows.
	% https://q.uiver.app/#q=WzAsNCxbMSwxLCJcXGJ1bGxldCJdLFsyLDEsIlxcYnVsbGV0Il0sWzEsMCwiXFxidWxsZXQiXSxbMCwwLCJcXGJ1bGxldCJdLFswLDEsIiIsMCx7ImN1cnZlIjoxLCJjb2xvdXIiOlswLDYwLDYwXX1dLFsxLDAsIiIsMCx7ImN1cnZlIjoxLCJjb2xvdXIiOlsyNDAsNjAsNjBdfV0sWzAsMiwiIiwwLHsiY29sb3VyIjpbMjQwLDYwLDYwXX1dLFsyLDMsIiIsMCx7ImNvbG91ciI6WzAsNjAsNjBdfV0sWzAsMywiIiwwLHsiY29sb3VyIjpbMjQwLDYwLDYwXX1dXQ==&macro_url=https%3A%2F%2Fraw.githubusercontent.com%2FdFoiler%2Fnotes%2Fmaster%2Fnir.tex
	\[\begin{tikzcd}
		\bullet & \bullet \\
		& \bullet & \bullet
		\arrow[color={rgb,255:red,214;green,92;blue,92}, curve={height=6pt}, from=2-2, to=2-3]
		\arrow[color={rgb,255:red,92;green,92;blue,214}, curve={height=6pt}, from=2-3, to=2-2]
		\arrow[color={rgb,255:red,92;green,92;blue,214}, from=2-2, to=1-2]
		\arrow[color={rgb,255:red,214;green,92;blue,92}, from=1-2, to=1-1]
		\arrow[color={rgb,255:red,92;green,92;blue,214}, from=2-2, to=1-1]
	\end{tikzcd}\]
	In general, we basically fold edges together to make relations. For example, the multiple outgoing blue edges should be folded together.
\end{example}
\begin{example}
	There is an infinite tree where each vertex has degree $4$. A coloring of the edges produces a ``Cayley graph'' $C_2$, which will turn out to be the universal covering space once we define such a notion. It turns out to be maximal in the sense that it covers any path-connected cover of $S^1\lor S^1$.
\end{example}

\subsection{Lifting with Covering Spaces}
We will want the following result.
\begin{proposition} \label{prop:covering-lifts-homotopy}
	Covering spaces have the homotopy lifting property. In other words, given a covering space $p\colon\widetilde X\to X$, a ``homotopy'' $f_\bullet\colon Y\times I\to X$ with a given lift $\widetilde f_0\colon Y\to\widetilde X$ will lift uniquely to $\widetilde f_\bullet\colon Y\times I\to\widetilde X$ agreeing with $X$.
\end{proposition}
\begin{proof}
	This is direct from \Cref{prop:fibration-prop}.
\end{proof}
\begin{corollary}
	Fix a covering space $p\colon(\widetilde X,\widetilde x_0)\to (X,x_0)$. Then $\pi_1(p)\colon\pi_1(\widetilde X,\widetilde x_0)\to\pi_1(X,x)$ is injective.
\end{corollary}
\begin{proof}
	Fix some loop $\widetilde f_0\colon I\to\widetilde X$ in the kernel of $\pi_1(p)$. Then there is a homotopy $f_\bullet\colon I\times I\to X$ from $\widetilde f_0$ to the constant path, which by \Cref{prop:covering-lifts-homotopy} will lift uniquely to a homotopy $\widetilde f_\bullet\colon I\times I\to\widetilde X$ agreeing on $\widetilde f_0$. Now $p\circ\widetilde f_1$ is constant, so looking locally at $\widetilde x_0$, we conclude that $\widetilde f_1$ is constant, so $\widetilde f_0$ is homotopic to the constant map and hence vanishes in $\pi_1(\widetilde X,\widetilde x_0)$.
\end{proof}

\end{document}