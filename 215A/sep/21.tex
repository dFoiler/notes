% !TEX root = ../notes.tex

\documentclass[../notes.tex]{subfiles}

\begin{document}

\section{September 21}

We continue to cover spaces.

\subsection{Deck Transformations}
Let $X$ be a path-connected, locally path-connected, and semilocally simply-connected space with universal cover $\widetilde X\to X$. We would like to use the universal cover to produce intermediate covering maps.
\begin{definition}[deck transformation]
	Let $X$ be a path-connected, locally path-connected, and semi-locally simply-connected space with cover $p\colon\widetilde X\to X$. A homeomorphism $f\colon\widetilde X\to\widetilde X$ such that $p=p\circ f$ is called a \textit{deck transformation}.
\end{definition}
In our set-up, let $G$ be the group of deck transformations of the universal cover $\widetilde X\to X$. Then $G\cong\pi_1(X)$. Let's explain why. Fix a basepoint $\widetilde x_0\in\widetilde X$ lying over $x_0\in X_0$. The point is that a deck transformation is uniquely determined by where it sends $\widetilde x_0$ by how path-lifting works. So a deck transformation $f\colon\widetilde X\to\widetilde X$ produces a path from $\widetilde x_0$ to $f(\widetilde x_0)$ (which is unique up to homotopy class because $\widetilde X$ is simply-connected), and then mapping this down to $p$ produces an element of $\pi_1(G,x_0)$. And conversely a loop in $\pi_1(X,x_0)$ lifts to a path up in $\widetilde X\to\widetilde X$ sending $\widetilde x_0\mapsto f(\widetilde x_0)$, and there is a unique automorphism $f\colon\widetilde X\to\widetilde X$ sending $\widetilde x_0$ to the right place.\footnote{At this point, it is perhaps clearer to use the direct construction of $\widetilde X$ as homotopy classes of paths starting at $x_0$.}
\begin{remark}
	More generally, if $(Y,y_0)\to (X,x_0)$ is any covering space, one has a bijection between $\pi_1(X,x_0)/\pi_1(Y,y_0)$ and points in the fiber of $x_0$.
\end{remark}
Extending the above discussion, we have the following result.
\begin{theorem}
	Fix a path-connected covering space $p\colon(Y,yx_0)\to (X,x_0)$, and let $G$ be the group of deck transformations. Then $X$ is homeomorphic to $Y/G$ in the natural way if and only if $\pi_1(Y,y_0)$ is a normal subgroup of $\pi_1(X,x_0)$. In this case, $G\cong\pi_1(X)/\pi_1(Y)$.
\end{theorem}
\begin{proof}
	Let $(\widetilde X,\widetilde x_0)\to(X,x_0)$ be the universal cover. Then the universal property allows us to factor as follows.
	% https://q.uiver.app/#q=WzAsMyxbMCwwLCIoXFx3aWRldGlsZGUgWCxcXHdpZGV0aWxkZSB4XzApIl0sWzEsMCwiKFkseV8wKSJdLFsxLDEsIihYLHhfMCkiXSxbMSwyLCJwIl0sWzAsMSwiciJdLFswLDIsInEiLDJdXQ==&macro_url=https%3A%2F%2Fraw.githubusercontent.com%2FdFoiler%2Fnotes%2Fmaster%2Fnir.tex
	\[\begin{tikzcd}
		{(\widetilde X,\widetilde x_0)} & {(Y,y_0)} \\
		& {(X,x_0)}
		\arrow["p", from=1-2, to=2-2]
		\arrow["r", from=1-1, to=1-2]
		\arrow["q"', from=1-1, to=2-2]
	\end{tikzcd}\]
	Now, for all $y\in p^{-1}(\{x_0\})$, we see that $r^{-1}(\{y\})$ will correspond to a coset of $\pi_1(\widetilde X)$ in $\pi_1(X)$ via the discussion with the universal cover; looping over $y$, we produce a bijection with points in the fiber of $q^{-1}(\{x_0\})$. Normality of the subgroup then follows because the action of $G$ here is trying to act on cosets.\todo{What?}
\end{proof}
A less careful version of this discussion lets us work with more general subgroups.
\begin{proposition}
	Let $X$ be a path-connected, locally path-connected, and semilocally simply connected space with universal cover $p\colon\widetilde X\to X$. For any subgroup $H\subseteq\pi_1(X,x_0)$, the quotient space $\widetilde X/H$ is a covering space of $X$ and has fundamental group $H$.
\end{proposition}
\begin{proof}
	Track through the above discussion without focusing on the group being normal.
\end{proof}

\subsection{Attempts for Universal Covers}
We are interested in the universal covering space construction having the lifting property. For our purposes, we will assume that our topological space $(X,x_0)$ which is locally path-connected, and we can still just try to define $\widetilde X$ as the set of homotopy classes of paths starting at $x_0$. Then the topology is defined by building a sub-base as follows: for open path-connected subsets $V\subseteq X$, one defines an open set around $[\gamma]$ with $\gamma(0)=x_0$ and $\gamma(1)=x$ by
\[\widetilde V\coloneqq\{[\gamma\cdot\gamma']:\gamma'\text{ is a path }I\to V\text{ such that }\gamma'(0)=x_0\text{ and }\gamma'(1)=y\}.\]
Let's see some examples.
\begin{example}
	Let's apply this to the earring $E$. One can show that this construction produces an open map $p\colon\widetilde E\to E$, but it is not a covering space. Nonetheless, $\widetilde E$ is path-connected, locally path-connected, and simply-connected, and it has the unique path-lifting property. Indeed, for any locally path-connected map $f\colon Y\to E$ where $f_*\pi_1(Y)\subseteq\pi_1(X)$ is trivial, the map $f$ factors through $p$ uniquely.

	One might want to try to draw $\widetilde E$, but this is hard: for example, with $e\in E$ the vertex of the earring, one has $p^{-1}(\{e\})$ uncountable, and $E$ is an $\mathbb R$-tree, meaning any two points has a unique path connecting them.
\end{example}
\begin{example}
	Let $X\coloneqq\prod_{i\in\NN}S^1$. By how the product topology works, this remains path-connected (as the product of path-connected spaces) but is not semilocally simply-connected because any open set contains at least one $S^1$, which fails to be simply-connected. Nonetheless, the map $\RR\to S^1$ remains continuous, so there is a map
	\[\prod_{i\in\NN}\RR\to\prod_{i\in\NN}S^1\]
	which behaves like a covering space.
\end{example}
\begin{example}
	CW-complexes $X$ are locally contractible and hence locally path-connected and locally simply-connected. Thus, our construction provides a universal covers for connected CW complexes. For simplicity, we work with the $2$-skeleton $X^{(2)}$, which encodes all $\pi_1$-information anyway, and we will focus on constructing $\widetilde X$. One can show that the covering space of a CW-complex remains a CW-complex because one can lift sufficiently small evenly covered cells to produce a CW-structure on the covering space. Looking at how $X^{(2)}$ is constructed by adding $2$-cells to produce quotients, we see that $\widetilde{X}^{(1)}$ corresponds to the kernel of $\pi_1(X^{(1)})\to\pi_1(X)$, which by van Kampen is the normal subgroup generated by $2$-cells as $\pi_1(\del e_\bullet^2)$ for the various $e_\bullet^2$.
\end{example}
\begin{example}
	Fix coprime positive integers $p$ and $q$, and construct the lens space $L(p,q)$ by taking the quotient of $D^3$ by dividing an equator $S^1$ into $q$ pieces and then gluing the top and bottom hemisphere after rotating by $2\pi q/p$. Equivalently, one can view this as $S^3/(\ZZ/p\ZZ)$, where the action is given by $k\cdot(z_1,z_2)\coloneqq\left(\zeta_p^kz_1,\zeta_{qk}z_2\right)$. One sees that $L(p,q)$ is a CW-complex with $1$-skeleton given by $S^1$ and two-skeleton by attaching $D^2$ and identifying $z$ with $\zeta_qz$ for each $z$.
	\begin{itemize}
		\item $\widetilde{L(p,q)}^{(1)}$ is $S^1$ again, but it is viewed as the $p$-fold cover of $S^1$.
		\item $\widetilde{L(p,q)}^{(2)}$ is $p$ disks glued at their boundaries.
		\item $\widetilde{L(p,q)}$ fills in these disks with $3$-balls.
	\end{itemize}
\end{example}
\begin{example}
	Suppose $X^{(1)}=\biglor_SS^1$, then $\pi_1\left(X^{(1)}\right)$ is the free group on $S$ as letters. Each attached $2$-cell to $X^{(1)}$ gives a relation for $G\coloneqq\pi_1\left(X^{(2)}\right)$. Now, $\widetilde{X^{(2)}}^{(1)}$ turns out to be Cayley graph of $G$, and its $0$-skeleton is in bijection with $G$, where edges are given by group elements in the natural way.

	Let's be more explicit: for any generating set $S\subseteq G$, let $N$ be the kernel of the surjection $F(S)\onto G$, and then we can view our Cayley graph via some covering space quotient.
\end{example}
For the next few examples, we have the following definition.
\begin{definition}
	We say that a CW-complex $X$ is $K(G,1)$ if and only if it has fundamental group $G$ and has contractible $\widetilde X$.
\end{definition}
It turns out that $K(G,1)$ is unique up to homotopy equivalence, so it allows us to talk more canonically about the group $G$ via topology. Here are some examples.
\begin{example}
	Note $K(\ZZ,1)=S^1$ because $\widetilde{S^1}=\RR$ is contractible.
\end{example}
\begin{example}
	Note $K(\ZZ/2\ZZ,1)=\mathbb{RP}^\infty$ because $S^\infty=\widetilde{\mathbb{RP}^\infty}$ is contractible.
\end{example}
\begin{example}
	We see $S^1\times S^1=K(\ZZ^2,1)$ because the universal cover of $S^1\times S^1$ is the contractible space $\RR^2$. Of course, we can take arbitrary powers and products like this.
\end{example}

\end{document}