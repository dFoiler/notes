% !TEX root = ../notes.tex

\documentclass[../notes.tex]{subfiles}

\begin{document}

\section{September 28}

Today we talk about graphs of groups.
\begin{remark}
	Problem 1.B.9 on the homework needs to assume that the edge maps are injective.
\end{remark}

\subsection{Using
 Classifying Spaces}
Given a group $G$, last time we constructed a contractible $\Delta$-complex $EG$, and from there we built $BG\coloneqq EG/G$, and we argued that $BG$ is a $K(G,1)$ because the action of $G$ on $EG$ was free, making $\pi_1(BG)=\pi_1(EG/G)=G$. Though huge, the $EG$ and $BG$ construction are nice because they are functorial: a homomorphism $\varphi\colon G\to H$ of groups produces a continuous map $E\varphi\colon EG\to EH$ by moving the vertices (which continuously will send simplices to simplices), and this commutes with the group actions on both spaces, so we produce a map $BG\colon BG\to BH$. Explicitly, $B\varphi([g])=[\varphi(g)]$, so
\[B\varphi([g_1,\ldots,g_n])=[\varphi(g_1),\ldots,\varphi(g_n)],\]
and this map is preserved by the group actions because
\[B\varphi(g\cdot[g'])=B\varphi([gg'])=[\varphi(gg')]=\varphi(g)\cdot[\varphi(g')]=\varphi(g)\cdot B\varphi([g']),\]
so there is a quotient down to a map $E\varphi\colon EG\to EH$.

One might now hope that we can produce a map $K(\varphi,1)\colon K(G,1)\to K(H,1)$, but for this to make sense, we need to know that $K(G,1)$ is well-defined in some sense.
\begin{theorem} \label{thm:k-g-1-uniq-homotopy}
	The homotopy type of a CW-complex $K(G,1)$ is uniquely determined by $G$.
\end{theorem}
The main input to the theorem is the following functoriality result.
\begin{proposition} \label{prop:k-g-1-up}
	Fix a connected CW-complex $X$, and let $Y$ be a $K(G,1)$. Then any homomorphism $\varphi\colon\pi_1(X,x_0)\to\pi_1(Y,y_0)$ is induced by a map $\Phi\colon(X,x_0)\to(Y,y_0)$ which is unique up to homotopy (relative to basepoints).
\end{proposition}
\begin{proof}
	We construct $X\to Y$ inductively. Map $X^0$ to $y_0$. As in our discussion of graphs, choose a spanning tree $T$ of $X^1$, and we see that each edge $e$ of $X^{1}\setminus T$ determines a generator $[e]$ of $\pi_1(X^1)$, and we map these down to the corresponding generator in $\pi_1(Y,y_0)$ as required by $\varphi$.

	By way of example, we can take $X=S^1\times S^1$ to be the torus, mapping the two generators of $\pi_1$ to $1\in\ZZ$. Then may extend $\Phi$ on the vertices to $X^2$ linearly via this triangulation (check up in the covering space to be told how to do this), viewing things as simplices. One can then keep going up to higher $X^n$ by continuing to go linearly, noting that the effect on the fundamental group is now not doing anything.

	For the uniqueness, suppose we have two maps $\Phi,\Psi\colon(X,x_0)\to(Y,y_0)$. This will essentially follow from the homotopy extension property. If they induce the same map $\pi_1(\Phi)=\pi_1(\Psi)$, then we move them up to the universal cover, and the convex combinations as described in the previous paragraph are forced and homotopic (linearly), where we are essentially using contractability of our universal cover. One needs to do this by induction on the skeletons: there is a homotopy on the $0$-skeleton by moving, there is a homotopy on the $1$-skeleton because they have the same $\pi_1$, there is a homotopy on the $2$-skeleton because the relations are the same, and from here one inducts upwards.
\end{proof}
We can now prove \Cref{thm:k-g-1-uniq-homotopy}.
\begin{proof}[Proof of \Cref{thm:k-g-1-uniq-homotopy}]
	One has identities relating fundamental groups on two $K(G,1)$s, so one produces maps in both directions by the proposition, and then the composition of these maps (in both directions) are homotopy equivalent to identity maps by uniqueness of these maps up to homotopy.
\end{proof}
Classifying spaces allow one to classify principal bundles with fiber given by a particular group. For example, the annulus $A$ provides a double-cover of the M\"obius strip $M$, so we see that this double-cover corresponds to $2\ZZ\subseteq\ZZ$. (Note the M\"obius strip has a deformation retraction to $S^1$, so the fundamental groups are the same.) Now, each fiber has a $(\ZZ/2\ZZ)$-action, and mapping $M\to\mathbb{RP}^\infty$ (given by the surjection $\ZZ\onto\ZZ/2\ZZ$ and using the $K(\ZZ/2\ZZ,1)$ universal property), we see that the composite $A\to M\to\mathbb{RP}^\infty$ is now trivial on $\pi_1$, so we induce a map making the following diagram commute.
% https://q.uiver.app/#q=WzAsNCxbMCwxLCJNIl0sWzEsMSwiXFxtYXRoYmJ7UlB9XlxcaW5mdHkiXSxbMSwwLCJTXlxcaW5mdHkiXSxbMCwwLCJBIl0sWzMsMCwiIiwwLHsic3R5bGUiOnsiaGVhZCI6eyJuYW1lIjoiZXBpIn19fV0sWzAsMV0sWzIsMSwiIiwyLHsic3R5bGUiOnsiaGVhZCI6eyJuYW1lIjoiZXBpIn19fV0sWzMsMiwiIiwyLHsic3R5bGUiOnsiYm9keSI6eyJuYW1lIjoiZGFzaGVkIn19fV1d&macro_url=https%3A%2F%2Fraw.githubusercontent.com%2FdFoiler%2Fnotes%2Fmaster%2Fnir.tex
\[\begin{tikzcd}
	A & {S^\infty} \\
	M & {\mathbb{RP}^\infty}
	\arrow[two heads, from=1-1, to=2-1]
	\arrow[from=2-1, to=2-2]
	\arrow[two heads, from=1-2, to=2-2]
	\arrow[dashed, from=1-1, to=1-2]
\end{tikzcd}\]
Namely, this map is given by tracking fibers through on the map $M\to\mathbb{RP}^\infty$.

More generally, if we have a covering space $\widetilde X\to X$, where $G$ acts freely and transitively (as deck transformations), then $G=\pi_1(X)/\im\pi_1(p)$, so maps $\pi_1(X)\onto G$ will be given by maps $X\to K(G,1)$ via the above construction. So $K(G,1)$ in some sense allows us to classify these covering spaces $\widetilde X\to X$, which is of interest. Indeed, one can go the other direction: given a map $\varphi\colon X\to K(G,1)$, we pull back the bundle $p\colon EG\to K(G,1)$ to $X$ to produce the necessary covering space. Namely, set
\[\widetilde X\coloneqq\{(x,y)\in X\times EG:\varphi(x)=p(y)\}\subseteq X\times EG,\]
and one can check that the induced map $\widetilde X\to EG$ is continuous, and the map $\widetilde X\to X$ is a covering space map where $G$ is acting on the fibers via $EG$.

\subsection{Graphs of Groups}
Fix a connected directed graph $\Gamma$, and for each vertex $v\in\Gamma^0$, we place a group $G_v$, and for each edge $e\in\Gamma^1$ connecting $v$ to $w$, we place a homomorphism $\varphi_e\colon G_v\to G_w$. This will be our set-up for this subsection.

We are going to build a classifying space $B\Gamma$ for this graph by putting a classifying space $BG_v$ (which is a CW-complex) at each vertex and attaching these along vertices with the mapping cylinders $MB\varphi_e$ for each $B\varphi_e\colon BG_v\to BG_w$. Notably, $B\varphi_e$ can always be constructed by \Cref{prop:k-g-1-up}. We will be interested in $\pi_1(B\Gamma)$. Note that $\pi_1(B\Gamma)$ does not depend on the choices of $BG-v$ and $B\varphi_e$ because these things are all well-defined up to homotopy.
\begin{example}
	Consider the following graph.
	% https://q.uiver.app/#q=WzAsNCxbMCwwLCJHX3t2XzF9Il0sWzIsMCwiR197dl8yfSJdLFsxLDEsImUiXSxbMSwyLCJHX3t2XzN9Il0sWzIsMF0sWzIsMV0sWzIsM11d&macro_url=https%3A%2F%2Fraw.githubusercontent.com%2FdFoiler%2Fnotes%2Fmaster%2Fnir.tex
	\[\begin{tikzcd}
		{G_{v_1}} && {G_{v_2}} \\
		& e \\
		& {G_{v_3}}
		\arrow[from=2-2, to=1-1]
		\arrow[from=2-2, to=1-3]
		\arrow[from=2-2, to=3-2]
	\end{tikzcd}\]
	Now, $K(e,1)$ is just a point, so the corresponding $B\Gamma$ is just a wedge product, so van Kampen tells us that this is $G_{v_1}*G_{v_2}*G_{v_3}$.
\end{example}
\begin{example}
	Consider the following graph.
	% https://q.uiver.app/#q=WzAsMyxbMSwwLCJcXFpaIl0sWzIsMCwiXFxaWiJdLFswLDAsIlxcWloiXSxbMCwxLCJwIl0sWzAsMiwicSIsMl1d&macro_url=https%3A%2F%2Fraw.githubusercontent.com%2FdFoiler%2Fnotes%2Fmaster%2Fnir.tex
	\[\begin{tikzcd}
		\ZZ & \ZZ & \ZZ
		\arrow["p", from=1-2, to=1-3]
		\arrow["q"', from=1-2, to=1-1]
	\end{tikzcd}\]
	Applying van Kampen to the resulting $B\Gamma$, we get a group presentation of $\left\langle a,b:a^p=b^q\right\rangle$. If $p=q=2$, one can squint very hard and see a Klein bottle as we are in some sense attaching two M\"obius strips.
\end{example}
\begin{example}
	Consider the following graph.
	% https://q.uiver.app/#q=WzAsMixbMCwwLCJDIl0sWzIsMCwiQSJdLFswLDEsIlxcdmFycGhpXzEiLDAseyJjdXJ2ZSI6LTJ9XSxbMCwxLCJcXHZhcnBoaV8yIiwyLHsiY3VydmUiOjJ9XV0=&macro_url=https%3A%2F%2Fraw.githubusercontent.com%2FdFoiler%2Fnotes%2Fmaster%2Fnir.tex
	\[\begin{tikzcd}
		C && A
		\arrow["{\varphi_1}", curve={height=-12pt}, from=1-1, to=1-3]
		\arrow["{\varphi_2}"', curve={height=12pt}, from=1-1, to=1-3]
	\end{tikzcd}\]
	This looks like $\pi_1(B\Gamma)=\left\langle A,t:t\varphi_2(c)t^{-1}=\varphi_1(c)\text{ for }c\in C\right\rangle$, again by some van Kampen argument.
\end{example}
Anyway, here is our main theorem.
\begin{theorem}
	Fix everything as above, and further assume that the $\varphi_e$ maps are injective. Then $B\Gamma$ is a $K(G,1)$ where $G\coloneqq\pi_1(B\Gamma)$, and the maps $\pi_1(BG_v)\to\pi_1(B\Gamma)$ are injective.
\end{theorem}
\begin{proof}
	Start with a specific edge $B\varphi_e\colon BG_v\to BG_w$. Then the $MB\varphi_e$ connecting the two will lift to connect $EG_v$ and $EG_w$ by checking each ``end'' of this cylinder. We now build upwards via a tree to slowly encompass the entire graph. Being path-connected implies that this inductive process will union out to give us a legitimate ``tree of spaces'' connecting all the groups. Now, each vertex group $G_v$ successfully acts on $EG_v$ and then goes on to act on the mapping cylinders adjacent, so we have the right fundamental group. And we can see by reversing the inductive constructive process that we can deformation retract our mapping cylinders away to show that our covering space is contractible.
\end{proof}

\end{document}