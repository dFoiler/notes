% !TEX root = ../notes.tex

\documentclass[../notes.tex]{subfiles}

\begin{document}

\section{September 12}

Let's wrap up some loose ends. People are doing pretty well on the homeworks, but please cite theorems and so on to be rigorous.

\subsection{The Fundamental Group of a Torus Knot}
Let's give a few applications of van Kampen's theorem.
\begin{example}
	Let $K\subseteq\RR^n$ be a compact subset for $n\ge3$. Embed $\RR^n\subseteq S^n$ by stereographic projection, and van Kampen shows that
	\[\pi_1\left(\RR^n\setminus K\right)\cong\pi_1\left(S^n\setminus K\right).\]
	More precisely, we have $S^n$ sitting inside $S^{n-1}\times\RR$ (place $K$ inside a large ball, and we can continuously deform any loop in $\RR^n\setminus K$ into this large ball), and the $\pi_1$ arising from this $\RR$ cannot help you.
\end{example}
\begin{example}[torus knots]
	Fix positive integers $m,n\in\ZZ$ bigger than $1$ with $\gcd(m,n)=1$. Define the torus knot $K_{m,n}\subseteq T^2$ (where $T^2=S^1\times S^1=\RR^2/\ZZ^2$) as the image of the line $my=nx$; alternatively, it is the image of the map $t\mapsto(mt,nt)$. For example, here is $K_{3,2}$ sitting inside the square $\RR^2/\ZZ^2$.
	\begin{center}
		\begin{asy}
			unitsize(1cm);
			draw((0,0)--(1,2.0/3), red);
			draw((0,2.0/3)--(0.5,1), red);
			draw((0.5,0)--(1,1.0/3), red);
			draw((0,1.0/3)--(1,1), red);
			draw((0,0)--(1,0)--(1,1)--(0,1)--cycle);
		\end{asy}
	\end{center}
	We compute $\pi_1\left(\RR^3\setminus K_{m,n}\right)$.
\end{example}
\begin{proof}
	Professor Agol seems to prefer the ``Clifford torus'' thought of as
	\[T^2=\left\{(z_1,z_2):\left|z_1\right|=\left|z_2\right|=1/\sqrt2\right\}.\]
	This sits inside $S^3=\left\{(z_1,z_2):\left|z_1\right|^2+\left|z_2\right|^2=1\right\}$. Anyway, we begin by giving us some breathing room. Define the ``thickening'' of $K$ as
	\[A\coloneqq\left\{(z_1,z_2):\left|z_1\right|<\frac1{\sqrt2}+\varepsilon\right\}\setminus\left\{\left(rz^m,\sqrt{1-r^2}z^n\right):z\in S^1,\frac1{\sqrt2}\le r\le\frac1{\sqrt2}+\varepsilon\right\}\]
	(namely, $A$ is the torus thickened in such a way that it carries the subtraction of $K_{m,n}$) and in the other way as
	\[B\coloneqq\left\{(z_1,z_2):\left|z_2\right|<\frac{1\sqrt2}+\varepsilon\right\}\setminus\left\{(\sqrt{1-r^2}z^m,rz^n):\frac1{\sqrt2}\le r\le\frac1{\sqrt2}+\varepsilon\right\}.\]
	Intersecting, we see that $A\cap B$ is $(S^1\times S^1)\setminus K_{m,n}$ thickened by $(-\varepsilon,\varepsilon)$, which we note can be deformed to $(S^1\times\RR)\times(-\varepsilon,\varepsilon)$, which has fundamental group $\ZZ$. Notably, $\pi_1(A)\cong\ZZ$ and $\pi_1(B)\cong\ZZ$ by deforming them carefully to the circle $S^1$, so our fundamental group is going to be $(\ZZ*\ZZ)/\ZZ$ by van Kampen.

	However, we need to compute the image of $\pi_1(A\cap B)$ in $\pi_1(A)*\pi_1(B)$. In the retraction of $A$ down to a circle says that the image in $\pi_1(A)$ is by multiplication by $n$, and similarly going to $B$ is multiplication by $m$. We conclude that our fundamental group is
	\[\left\langle a,b:a^n=b^m\right\rangle.\]
	As an aside, we note that $S^3\setminus K_{m,n}$ will have a deformation retract back to $K_{m,n}$ shifted upwards by some amount (for example, see the diagram and imagine a copy of $K_{m,n}$ shifted up by some small $\varepsilon>0$). Anyway, for $m,n>1$ we can see that the center of the above group is $\left\langle a^n\right\rangle$, so modding out by the center yields $\ZZ/m\ZZ*\ZZ/n\ZZ$. In total, we are able to distinguish the torus knots $S^3\setminus K_{m,n}$ from each other.\footnote{Alternatively, the abelianization $\pi_1(S^3\setminus K_{m,n})$ is the free group with $(m-1)(n-1)$ generators, and the abelianization of $(\ZZ/m\ZZ)*(\ZZ/n\ZZ)$ is $mn$, from which we can read off $m$ and $n$.} To deal with the signs of $m$ and $n$, we need a notion of isotopy to distinguish a knot from its ``mirror image.''
\end{proof}

\subsection{The Fundamental Group of a CW Complex}
Let's move on from knots and compute the fundamental group of some cell complexes.
\begin{example}
	Let $X$ be a connected graph (i.e., a $1$-dimensional CW complex), then $\pi_1(X)$ is homotopy equivalent to a wedge of circles, which has fundamental group $\ZZ^{*r}$ for some $r$, which is the free group on $r$ letters.
\end{example}
Now, if $Y$ is a connected CW complex, then $\pi_1(Y^1)$ is a free group. Then $\pi_1(Y^2)$ might be complicated, but let's imagine computing $\pi_1(Y^3)$. The point is that we take some ball $e^3_\alpha\cong D^2$ and attach it via some $\varphi_\alpha\colon\del D^3\to Y^2$.

To compute the fundamental group of this, we cover $Y^2\sqcup_{\varphi_\alpha}e_\alpha^3$ by $A\coloneqq Y^2\cup_{\varphi_\alpha}(e^3_\alpha\setminus\{x\})$ and $B=(e_3^\alpha)^\circ$ (Here, $x$ is some point in the interior.) Notably, the intersection is simply $S^2\times\RR$, which is trivial, so we conclude that the attachment $e_\alpha^3$ did nothing to our fundamental group by van Kampen. Applying this argument inductively (perhaps transfinitely), we see that $\pi_1(Y^3)=\pi_1(Y^2)$. One can continue upwards to conclude that $\pi_1(Y)=\pi_1(Y^2)$.\footnote{One can do this without transfinite induction by working with a single loop and arguing about homotopy equivalence. The point is that a single loop (and in fact a single homotopy) can is compact and therefore only cares about finitely many cells.}

Now, let's say that we actually want to compute $\pi_1(Y^2)$. To do so, we note that we have a surjection $\pi_1(Y^1)\to\pi_1(Y^2)$ given by the inclusion (any loop can be deformed off the $2$-skeleton to the $1$-skeleton). Now, for each $2$-cell $e_\alpha^2$ attached via $\varphi_\alpha\colon\del e_\alpha^2\to Y^1$, we choose a path $\gamma_\alpha\colon I\to Y^1$ so that $\gamma_\alpha(0)=y$ and $\gamma_\alpha(1)=\varphi_\alpha(0)$ and then find that
\[\gamma_\alpha\cdot\varphi_\alpha\cdot\overline\gamma_\alpha\]
ought to be in the kernel of our projection. An argument shows that these elements will generate the needed kernel. One can show this by an analogous argument to the above: the point is that the attachment of $e_\alpha^2$ kills basically exactly the loop given above and nothing else, so we can use an inductive argument to conclude.
\begin{remark}
	One can use this result to show that any group $G$ arises as the fundamental group of a CW complex of dimension $2$. Roughly speaking, the point is that any group is the quotient of a free group, and the above argument allows us to dictate relations, provided that we are sufficiently careful.
\end{remark}
\begin{example}
	Fix a positive integer $g$. Define $S_g$ by starting with a $4g$-gon and attaching the edges. Namely, for $n<3$, an $n\pmod4$ edge is identified with the next over $n+2\pmod4$ edge in the opposite direction. Roughly speaking, after some manipulation, one finds that $S_g$ ought to be a $g$-hole torus. Using the above argument, one finds that $\pi_1(S_g)$ is generated by $2g$ generators $a_1,\ldots,a_g,b_1,\ldots,b_g$ modded out by the relations $a_ib_ia_i^{-1}b_i^{-1}$ for each $i$. In particular, the abelianization of $\pi_1(S_g)$ has all the commutators, so we get $\ZZ^{2g}$. Thus, $\pi_1$ distinguishes our surfaces.
\end{example}
\begin{example}[projective space]
	We note $\pi_1(\mathbb{RP}^\infty)\cong\pi_1(\mathbb{RP}^2)$ because the higher cells cannot help you in the fundamental group. Further, we see $\pi_1(\mathbb{RP}^2)$ is a disk with semicircles identified in the opposite direction, which we can see from the above argument is simply $\ZZ/2\ZZ$.
\end{example}
\begin{example}[lens space]
	Fix positive integers $p$ and $q$ with $\gcd(p,q)=1$. Take $S^2$ and divide the equator into $p$ circles, and we glue the top hemisphere to the bottom hemisphere by gluing after a $2\pi p/q$ rotation. The space has fundamental group $\ZZ/p\ZZ$. Indeed, our $1$-skeleton is the equator, and the $a^p$ comes from how we attached our disks together.
\end{example}
Next time we will talk about covering spaces.

\end{document}