% !TEX root = ../notes.tex

\documentclass[../notes.tex]{subfiles}

\begin{document}

\section{August 24}

It begins.

\subsection{Logistics}
Here are the logistical notes.
\begin{itemize}
	\item The professor is Ian Agol, whose office is Evans 921. Office hours are Tuesdays after class, Monday at 3PM, Wednesday at 9AM, or by appointment.
	\item There is a \href{https://bcourses.berkeley.edu/courses/1528186/}{\texttt{bCourses}}.
	\item Homework will be weekly, and it will make up the entire grade.
	\item The prerequisites are Math 113 and 202A or equivalent. From point-set topology in particular we will want notions of compactness, connectedness, metric spaces, and a few topologies like the identification topology with respect to a continuous map.
\end{itemize}

\subsection{Overview}
We will cover chapters 0--3 of \cite{hatcher}.
\begin{itemize}
	\item Chapter 0 consists of ``geometric notions.'' Particularly important are the notion of homotopy and CW complexes.
	\item Chapter 1 is on fundamental groups.
	\item Chapter 2 is on homology. This is an abelian extension of fundamental groups.
	\item Chapter 3 is on cohomology. Poincar\'e duality relates cohomology with homology.
\end{itemize}
Chapter 4 is typically covered in Math 215B, on homotopy theory.

Let's talk a bit about the interests of the course. Topology as a whole is interested in ``spaces up to deformation.'' In this class, deformation will mean homotopy mostly, but there are finer notions of interest like homeomorphism. As for the spaces, we will focus on spaces which are locally homogeneous in some sense, like manifolds (which are locally homeomorphic to $\RR^n$). These notions come up naturally throughout mathematics; for example, integrals of holomorphic functions are roughly independent of path chosen. Poincar\'e himself was interested in differential equations, whose configuration spaces could be manifolds.

In this class, we will attach invariants to our topological spaces to be able to understand how to differentiate between our spaces (up to deformation). We focus on the following invariants.
\begin{itemize}
	\item Fundamental groups and covering spaces. This has a close tie to Galois theory, an analogy made process by the \'etale fundamental group in algebraic geometry.
	\item Cohomology. The origins are from complex analysis and Stokes's theorem, but cohomology itself has vast generalizations and manifestations throughout mathematics, leading to the field of homological algebra. However, there are applications to algebraic geometry, number theory, and so on. The most notable application here is the proof of the Weil conjectures.
	\item Higher homotopy groups. Our approach will not begin with this viewpoint, but it is possible.
\end{itemize}

\subsection{Homotopy and Homotopy Type}
Let's jump in chapter 0.
\begin{notation}
	We set $I\coloneqq[0,1]$ for convenience.
\end{notation}
\begin{definition}[deformation retract]
	Fix a subspace $A$ of a topological space $X$. Then a \textit{deformation retract} is a family of functions $f_\bullet\colon X\times I\to X$ where $f_0=\id_X$ and $\im f_1=A$ and $f_t|_A=\id_A$ for all $t\in I$.
\end{definition}
\begin{example}[mapping cylinder]
	Fix a continuous function $f\colon X\to Y$. Then the \textit{mapping cylinder} $M_f$ is the space $(X\times I)\sqcup Y$ quotiented by $(x,1)\sim f(x)$. Then $M_f$ has a deformation retraction to $Y$ by $f_t(x)\coloneqq(x,t)$. Visually, we have attached $Y$ to a thickening of $X$.
\end{example}
\begin{example} \label{ex:mobius-by-cylinder}
	Define $f\colon S^1\to S^1$ by $f(z)\coloneqq z^2$. Then $M_f$ has $S^1$ on one domain side and $S^1$ covered twice on the target side. With a little deformation, this is a M\"obius strip. Approximately speaking, one should cut the cylinder in half and then rearrange. One can see that the M\"obius strip deformation retracts to $S^1$ by squishing the width of the cylinder to the central line.
\end{example}
A deformation retract is a special case of a homotopy. Here is the definition of a homotopy.
\begin{definition}[homotopy]
	Two continuous maps $f_0,f_1\colon X\to Y$ are \textit{homotopic} if and only if there is a continuous function $F_\bullet\colon X\times I\to Y$ such that $F_0=f_0$ and $F_1=f_1$. Here, $F$ is called a \textit{homotopy}, and we write $f_0\sim f_1$.
\end{definition}
\begin{example}
	A subspace $A\subseteq X$ has a deformation retract if and only if $\id_X$ is homotopic to some $r\colon X\to X$ with $\im r=A$ and $r|_A=\id_A$. Indeed, the deformation retract is exactly the needed homotopy.
\end{example}
Homotopy allows us to define homotopy equivalence.
\begin{definition}[homotopy equivalence]
	A continuous map $f\colon X\to Y$ is a \textit{homotopy equivalence} if and only if there is a continuous map $g\colon Y\to X$ such that $(g\circ f)\sim\id_X$ and $(f\circ g)\sim\id_Y$. We then say that $X$ and $Y$ have the same \textit{homotopy type} and write $X\simeq Y$.
\end{definition}
\begin{remark}
	It is not enough to merely require $(g\circ f)\sim\id_X$. For example, let $X\coloneqq\{x\}$ be a point.
\end{remark}
\begin{remark}
	One can check directly that $\sim$ is an equivalence relation on spaces. The main check here is that one can compose homotopies.
\end{remark}
We will often find that our algebraic invariants are only able to detect homotopy equivalence, which is why homotopy equivalence will be so important to us.
\begin{example}
	\Cref{ex:mobius-by-cylinder} shows that the M\"obius strip is homotopic to $S^1$.
\end{example}
\begin{example}
	More generally, a deformation retract is a homotopy equivalence. The inverse homotopy equivalence is the inclusion.\todo{}
\end{example}
\begin{example}[dunce cap]
	Take the disc $D^2$ and glue the edges together as follows: mark three points $A$, $B$, and $C$, and glue $AB$ to $AC$ to $CB$ (in those orientations). Then the resulting space is homotopic to a point.
\end{example}
We have a special name for being homotopic to a point.
\begin{definition}[contractible]
	A topological space $X$ is \textit{contractible} if and only if it is homotopic to a point.
\end{definition}
These notions allow us to define a homotopy category, whose objects are homotopy classes of topological spaces and morphisms are continuous maps. In some sense, our algebraic invariants are trying to distinguish between objects in this category. It turns out that this category is not concrete, meaning that there is no way to realize its objects as sets reasonably. Approximately speaking, this means that there can be no canonical representing topological space for each homotopy class, but topologists try anyway.
\begin{remark}
	There are a number of results called ``topological rigidity'' theorems which give homeomorphism $X\cong Y$ given merely $X\simeq Y$ and some extra hypotheses. For example, this holds for closed surfaces by a classification result.
\end{remark}
\begin{example}
	Attach two $S^1$s by a line to make a space $X$, and attach them along an edge to make a space $Y$. These spaces are homotopic, but they are not homeomorphic (removing a point from $X$ may disconnect it, but this is not the case for $Y$).
\end{example}

\subsection{CW Complexes}
Here is our definition.
\begin{definition}[CW complex]
	Let $X^0$ be a discrete set of points, and define $X^n$ inductively by $X^{n+1}\coloneqq X^n\cup\{e_\alpha^{n+1}\}$, where $\varphi_\alpha\colon\del e_\alpha^{n+1}\to X^n$ is a homeomorphism telling us how to union. Here, $e_\alpha^{n+1}$ is a copy of the $n$-ball $B^n$, so the $\varphi_\alpha$ are explaining how to identify the edges.
\end{definition}
\begin{example}
	Here is a CW complex.
	\begin{center}
		\begin{asy}
			unitsize(1cm);
			dot("$x_0$",(0,0),W);
			dot("$x_1$",(1,0),E);
			draw((0,0)--(1,0));
			draw(circle((-1,0),1));
			draw(circle((2,0),1));
		\end{asy}
	\end{center}
	Namely, $X^0=\{x_0,x_1\}$, and $X^1$ is the edges.
\end{example}
\begin{example}
	Take a point $\{*\}$ for $X^0$, and define $\varphi_n$ to be some loop based on $\{*\}$. Then the resulting space is some infinite union of circles intersecting at $\{*\}$. Notably, this space is not compact and in fact should not even be embedded into the plane or $\RR^3$ because such an embedding is unlikely to be a homeomorphism.
\end{example}
\begin{example}
	The sphere $S^n\coloneqq D^n/\del D^n$ is a CW structure with only two cells: it is $e^0\cup e^n$. Notably, the CW structure here has $X^0=X^1=\cdots=X^{n-1}$.
\end{example}
\begin{example}
	Alternatively, one can define $S^n$ inductively as follows: take $S^0$ to be two points, and define $S^{n+1}$ to be $S^n$ as an equator unioned with two $(n+1)$-cells making hemispheres attached to the equator. One can then define $S^\infty$ to be the union of all the $S^\bullet$ where we identify $S^n\into S^{n+1}$ via the equator. This is a CW complex of infinite dimension. It turns out that $S^\infty$ is contractible, though $S^n$ is not for any finite $n$.
\end{example}
\begin{example}
	Define real projective space $\mathbb{RP}^n$ as the set of vectors $x\in\RR^{n+1}\setminus\{0\}$ where we identify $x$ with $\lambda x$ for any $\lambda\in\RR^\times$. Notably, by setting the last coordinate equal to $0$, we expect to get $\mathbb{RP}^{n-1}$. But if the last coordinate is equal to zero, we can scale it uniquely to $1$, and then the remaining coordinates may vary arbitrarily. In total, we find
	\[\mathbb{RP}^n=\mathbb{RP}^{n-1}\sqcup\RR^{n-1}.\]
	Thus, we get the cell structure $\mathbb{RP}^n=e^0\cup e^1\cup\cdots\cup e^n$.
\end{example}
\begin{remark}
	The CW structure is not unique. For example, one can separate out edges by putting a point in the middle of them.
\end{remark}
One can show that the CW complex is compact if and only if it has finitely many cells.

\end{document}