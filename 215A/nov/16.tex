% !TEX root = ../notes.tex

\documentclass[../notes.tex]{subfiles}

\begin{document}

\section{November 16}

Today we will discuss the cup product.

\subsection{The Cup Product}
In the discussion that follows, we choose coefficients in a ring $R$, which we will assume is commutative and has unity and so on.
\begin{definition}
	Fix a topological space $X$ and a ring $R$. Given $\varphi\in C^k(X;R)$ and $\psi\in C^\ell(X;R)$, we define the \textit{cup product} as the cochain $(\varphi\cup\psi)\in C^{k+\ell}(X;R)$
	\[(\varphi\cup\psi)(\sigma)=\varphi(\sigma|_{[v_0,\ldots,v_k]})\psi(\sigma|_{[v_k,\ldots,v_n]}),\]
	where $\sigma\colon[v_0,\ldots,v_n]\to X$ (with $n=k+\ell$) is a singular simplex.
\end{definition}
Extending linearly, we see that this defines a map
\[C^k(X;R)\otimes_RC^\ell(X;R)\to C^{k+\ell}(X).\]
For example, we would like this to agree with the boundary map.
\begin{lemma} \label{lem:diff-on-cocycle}
	Fix a topological space $X$ and a ring $R$. Given $\varphi\in C^k(X;R)$ and $\psi\in C^\ell(X;R)$, we have
	\[\del(\varphi\cup\psi)=\del\varphi\cup\psi+(-1)^k\varphi\cup\del\psi,\]
	where $\del$ is the boundary map.
\end{lemma}
\begin{proof}
	We check on a single singular simplex $\sigma\colon[v_0,\ldots,v_n]\to X$, where $n=k+\ell$. Indeed, we directly compute
	\begin{align*}
		(\del\varphi\cup\psi)(\sigma) &= \del\varphi(\sigma|_{[v_0,\ldots,v_{k+1}]})\psi(\sigma|_{[v_{k+1},\ldots,v_{n+1}}) \\
		&= \sum_{i=0}^{k+1}(-1)^i\varphi(\sigma|_{[v_0,\ldots,\widehat v_i,\ldots,v_{k+1}]})\psi(\sigma|_{[v_{k+1},\ldots,v_{n+1}}),
	\end{align*}
	and
	\begin{align*}
		(-1)^k(\varphi\cup\del\psi)(\sigma) &= (-1)^k\varphi(\sigma|_{[v_0,\ldots,v_k]})\del\psi(\sigma|_{[v_k,\ldots,v_{n+1}]}) \\
		&= \sum_{i=k}^{n+1}(-1)^i\varphi(\sigma|_{[v_0,\ldots,v_k]})\psi(\sigma|_{[v_k,\ldots,\widehat v_i,\ldots,v_{n+1}]}),
	\end{align*}
	and
	\begin{align*}
		\del(\varphi\cup\psi)(\sigma) &= \sum_{i=0}^{n+1}(-1)^i(\varphi\cup\psi)(\sigma|_{[v_0,\ldots,\widehat v_i,\ldots,v_{n+1}]}) \\
		&= \sum_{i=0}^{k}(-1)^i\varphi(\sigma|_{[v_0,\ldots,\widehat v_i,\ldots,v_{k+1}]})\psi(\sigma|_{[v_{k+1},\ldots,v_{n+1}]}) \\
		&\qquad+\sum_{i=k+1}^{n+1}(-1)^i\varphi(\sigma|_{[v_0,\ldots,v_{k}]})\psi(\sigma|_{[v_{k+1},\ldots,\widehat v_i,\ldots,v_{n+1}]}).
	\end{align*}
	Collecting the terms completes the proof; notably, the last term of the $(\del\varphi\cup\psi)$ sum cancels with the first term of the $(-1)^k(\varphi\cup\del\psi)$ sum, making the total number of terms agree.
\end{proof}
\begin{corollary}
	Fix a topological space $X$ and a ring $R$. Given $\varphi\in C^k(X;R)$ and $\psi\in C^\ell(X;R)$, if $\varphi$ and $\psi$ are cocycles, then so is $\varphi\cup\psi$.
\end{corollary}
\begin{proof}
	Note that $\del\varphi=0$ and $\del\psi=0$, so the result follows from \Cref{lem:diff-on-cocycle}.
\end{proof}
\begin{corollary}
	Fix a topological space $X$ and a ring $R$. Given $\varphi\in C^k(X;R)$ and $\psi\in C^\ell(X;R)$, if $\varphi$ or $\psi$ is a coboundary and the other is a cocycle, then so is $\varphi\cup\psi$.
\end{corollary}
\begin{proof}
	For example, if $\varphi$ is a coboundary and $\psi$ is a cocycle, then write $\varphi=\del\varphi_0$, so
	\[\varphi\cup\del=\del\varphi_0\cup\psi=\del(\varphi_0\cup\psi)-(-1)^k\varphi_0\cup\underbrace{\del\psi}_0=\del(\varphi_0\cup\psi).\]
	The other argument is similar.
\end{proof}
The point now is that we have induced a multiplication structure
\[\cup\colon H^k(X;R)\otimes_RH^\ell(X;R)\to H^{k+\ell}(X;R)\]
because we send cocycles to cocycles and coboundaries to coboundaries; a direct computation shows that $\cup$ is associative and distributes over addition, so we are in fact getting a graded ring structure, perhaps without unity and perhaps not commutative. Let's see some examples.
\begin{example}
	Consider the genus-$2$ surface $M$, which can be visualized as an octagon with every other edge identified in the opposite orientation. Now, there is a map $H^1(M)\times H^1(M)\to H^2(M)$. Recall that we computed $H_i(M)$ is always free abelian, with ranks $1$, $4$, and $1$ in degrees $0$, $1$, and $2$. Then \Cref{thm:univ-coef} tells us that we may identify $H^i(M)$ with $\op{Hom}_\ZZ(H_i(M),\ZZ)$; for example, distinguish generators $a_1$, $a_2$, $b_1$, and $b_2$ for $H_1(M)$, and then we let the corresponding indicators (i.e., the dual basis) be $\alpha_1$, $\alpha_2$, $\beta_1$, and $\beta_2$. This produces a cocycle ``for free'' by \Cref{thm:univ-coef}, but one can also just check it directly. For example, up to labeling the edges appropriately, $[\alpha_1]\cup[\beta_1]$ will be nonzero on a single $2$-simplex by a direct computation; a dual computation (with opposing signs) explains that $[\beta_1]\cup[\alpha_1]$ is exactly the negative of this.
\end{example}
Cup products also come with a naturality.
\begin{prop}
	Fix a continuous map $f\colon X\to Y$. Then the maps $H^\bullet(f)$ induce a homomorphism of graded rings
	\[H^\bullet(f)\colon H^\bullet(Y;R)\to H^\bullet(X;R).\]
\end{prop}
\begin{proof}
	This is a direct computation. Of course $H^\bullet(f)$ is already additive, so it only remains to show that it is multiplicative. As usual, we fix some $\varphi\in C^k(X;R)$ and $\psi\in C^\ell(Y;R)$ along with some singular simplex $\sigma\colon\Delta^n\to X$ where $n\coloneqq k+\ell$. Then
	\begin{align*}
		H_n(f)(\varphi\cup\psi)(\sigma) &= (\varphi\cup\psi)(f\circ\sigma) \\
		&= \varphi(f\circ\sigma|_{[v_0,\ldots,v_k]})\psi(f\circ\sigma|_{[v_k,\ldots,v_n]}) \\
		&= H_n(f)(\varphi)(\sigma|_{[v_0,\ldots,v_k]})H_n(f)(\psi|_{[v_k,\ldots,v_n]}) \\
		&= H_n(f)(\varphi)\cup H_n(f)(\psi),
	\end{align*}
	as desired.
\end{proof}
\begin{remark}
	Let's take a moment to provide a geometric interpretation of the cup product if two $1$-cocycles $\alpha,\beta\in H^1(X;\ZZ)$. By \Cref{thm:univ-coef}, we are computing the product of two elements of
	\[H^1(X;\ZZ)=\op{Hom}_\ZZ(H_1(X),\ZZ)=\op{Hom}(\pi_1(X),\ZZ),\]
	where in the last equality we have used the fact that $\pi_1(X)^{\mathrm{ab}}=H_1(X)$. So we may choose maps $a\colon X\to S^1$ and $b\colon X\to S^1$ such that the induced maps $\pi_1(X)\to\pi_1(\ZZ)$ and $\pi_1(X)\to\pi_1(\ZZ)$ are simply $\alpha$ and $\beta$. Now, taking the product of $a\times b$ produces a map $X\to S^1\times S^1$ for which $\pi_1(a\times b)$ projects down to $\alpha$ and $\beta$. From here, an explicit computation (using the above result) can show $\alpha\cup\beta$ is simply given by $\pi_1(a\times b)(x\cup y)$ where $x,y\in H^2\left(S^1\times S^1\right)$ are the generators by the edges of the corresponding square diagram.
\end{remark}
\begin{remark}
	We take a moment to note that there are relative cup products
	\[H^k(X,A;R)\times H^\ell(X,B;R)\to H^{k+\ell}(X,A\cup B;R).\]
	The point is that $\varphi$ vanishing on $A$ and $\psi$ vanishing on $B$ will have $\varphi\cup\psi$ vanishing on their union by \Cref{lem:diff-on-cocycle}. From here, we note that we then get another graded ring structure on $H^\bullet(X,A;R)$. 
\end{remark}
\begin{example}
	We can compute that $H^\bullet\left(\mathbb{RP}^n,\FF_2\right)$ is isomorphic to $\FF_2[x]/\left(x^{n+1}\right)$ where $x$ has degree $1$. Similarly, we can compute that $H^\bullet\left(\mathbb{RP}^\infty,\FF_2\right)$ is isomorphic to $\FF_2[x]$ where $x$ has degree $1$. 
\end{example}
\begin{proof}
	We will work on $\mathbb{RP}^n$ only. The point is that $\mathbb{RP}^n$ can be given a triangulation by viewing it as $S^n/{\sim}$ where $\sim$ is the antipodal equivalence relation. Now, taking joins via $*$, we note that $S^n=S^0*\cdots*S^n$, which provides $S^n$ with a simplicial structure. Explicitly, realizing $S^0$ on an axis of $\RR^{n+1}$, we see that we can view $S^n$ as sitting inside $\RR^{n+1}$ as connecting the dots of the form $(0,\ldots,\pm1,\ldots,0)$; then modding out by $\sim$, we receive a simplicial structure on $\mathbb{RP}^n$ with $n+1$ vertices.

	We now acknowledge that \Cref{thm:univ-coef} tells us that $H^k(\mathbb{RP}^n,\FF_2)=\FF_2$ for $0\le k\le n$, so we are at least correct on the level of abelian groups. It remains to compute the cup product, where we must show that a generator of $H^k(\mathbb{RP}^n,\FF_2)$ cupped with a generator of $H^\ell(\mathbb{RP}^n,\FF_2)$ produces a generator of $H^{k+\ell}(\mathbb{RP}^n,\FF_2)$. One must make some choice of generator, so we choose a generator of $H^1(\mathbb{RP}^n,\FF_2)$ by being $1$ on each edge meeting the plane $x_0+\cdots+x_n=0$ and $0$ elsewhere. Then we compute the cup product with itself a few times to conclude.
\end{proof}
\begin{remark}
	One can similarly compute for $\mathbb{CP}^n$ and $\mathbb{CP}^\infty$; the cohomology turns out to be the same ``ring'' but with the generator $x$ in degree $2$ so that the ring is in fact commutative.
\end{remark}

\end{document}