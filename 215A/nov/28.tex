% !TEX root = ../notes.tex

\documentclass[../notes.tex]{subfiles}

\begin{document}

\section{November 28}

Today we will continue talking about the cup product. Homework has been posted.

\subsection{More on Projective Space}
Let's make a few remarks on projective space. Last time we computed the cohomology ring of $H^\bullet\left(\mathbb{RP}^n,\FF_2\right)$ fairly explicitly as $\FF_2[\alpha]/\left(\alpha^{n+1}\right)$, and by taking the direct limit with $n\to\infty$, we get $H^\bullet\left(\mathbb{RP}^\infty,\FF_2\right)\cong\FF_2[\alpha]$. We note that we can recover $H^\bullet\left(\mathbb{RP}^\infty,\ZZ\right)$ in the following way. The map $\ZZ\to\FF_2$ induces a map on cellular cohomology chain complexes as follows.
% https://q.uiver.app/#q=WzAsMTIsWzUsMCwiMCJdLFs1LDEsIjAiXSxbNCwxLCJcXEZGXzIiXSxbMywxLCJcXEZGXzIiXSxbMiwxLCJcXEZGXzIiXSxbMSwxLCJcXEZGXzIiXSxbMCwxLCJcXGNkb3RzIl0sWzAsMCwiXFxjZG90cyJdLFsxLDAsIlxcWloiXSxbMiwwLCJcXFpaIl0sWzMsMCwiXFxaWiJdLFs0LDAsIlxcWloiXSxbMCwxMV0sWzExLDEwLCIwIiwyXSxbMTAsOSwiMiIsMl0sWzksOCwiMCIsMl0sWzgsNywiMiIsMl0sWzEsMl0sWzIsMywiMCIsMl0sWzMsNCwiMCIsMl0sWzQsNSwiMCIsMl0sWzUsNiwiMCIsMl0sWzExLDJdLFsxMCwzXSxbOSw0XSxbOCw1XV0=&macro_url=https%3A%2F%2Fraw.githubusercontent.com%2FdFoiler%2Fnotes%2Fmaster%2Fnir.tex
\[\begin{tikzcd}
	\cdots & \ZZ & \ZZ & \ZZ & \ZZ & 0 \\
	\cdots & {\FF_2} & {\FF_2} & {\FF_2} & {\FF_2} & 0
	\arrow[from=1-6, to=1-5]
	\arrow["0"', from=1-5, to=1-4]
	\arrow["2"', from=1-4, to=1-3]
	\arrow["0"', from=1-3, to=1-2]
	\arrow["2"', from=1-2, to=1-1]
	\arrow[from=2-6, to=2-5]
	\arrow["0"', from=2-5, to=2-4]
	\arrow["0"', from=2-4, to=2-3]
	\arrow["0"', from=2-3, to=2-2]
	\arrow["0"', from=2-2, to=2-1]
	\arrow[from=1-5, to=2-5]
	\arrow[from=1-4, to=2-4]
	\arrow[from=1-3, to=2-3]
	\arrow[from=1-2, to=2-2]
\end{tikzcd}\]
By computing the cohomology, we see that the ring map $H^\bullet\left(\mathbb{RP}^\infty,\ZZ\right)\to H^\bullet\left(\mathbb{RP}^\infty,\FF_2\right)$ is an isomorphism in positive degree, from which we can pull back to get
\[H^\bullet\left(\mathbb{RP}^\infty,\ZZ\right)\cong\frac{\ZZ[\alpha]}{(2\alpha)}\]
where $\alpha$ has degree $2$. There is a similar computation for $\mathbb{CP}^\infty$.

\subsection{More on Cup Products}
Let's go ahead and prove that the cup product is graded commutative.
\begin{proposition}
	Fix pairs $(X,A)$ and $(X,B)$ with classes $\alpha\in H^k(X,A;R)$ and $\beta\in H^\ell(X,A;R)$. Then
	\[\alpha\cup\beta=(-1)^{k\ell}(\beta\cup\alpha).\]
\end{proposition}
\begin{remark}
	Roughly speaking, one expects this anticommutativity from differential geometry and in particular the anticommutativity of the wedge product for differential forms.
\end{remark}
\begin{proof}
	We take $A=B=\emp$; the general case can be derived from this. The main point is that trying to compute $\beta\cup\alpha$ wants us to reverse $[v_0,\ldots,v_n]$ to $[v_n,\ldots,v_0]$. Thus, given a singular $n$-simplex $\sigma\colon[v_0,\ldots,v_n]\to X$, we will define $\overline\sigma\colon[v_0,\ldots,v_n]\to X$ by extending $\overline\sigma(v_i)\coloneqq v_{n-i}$ linearly. One can then extend $\sigma\mapsto\overline\sigma$ linearly to all $n$-cycles. However, we kind of are introducing a sign here because $(v_0,\ldots,v_n)\mapsto(v_n,\ldots,v_0)$ is a permutation of sign $\varepsilon_n\coloneqq(-1)^{n(n+1)/2}$ by explicitly computing the number of needed transpositions, so we will actually define $\rho\colon C_\bullet(X)\to C_\bullet(X)$ by extending
	\[\rho(\sigma)\coloneqq\varepsilon_n\overline\sigma\]
	linearly to all $n$-cycles. By a short computation with boundaries, we see that $\rho$ is actually a chain map, and we note that $\rho$ squares to the identity. In fact, one can write down an explicit chain homotopy from $\rho$ to the identity; alternatively, one can use the Eilenberg--Steenrod axioms in order to show that ``switching the vertices'' of all our $n$-simplices is producing a naturally isomorphic cohomology theory.

	From here, we can now compute
	\begin{align*}
		(\rho^*\varphi\cup\rho^*\psi)(\sigma) &= \varphi(\varepsilon_k\sigma|_{[v_k,\ldots,v_0]})\psi(\varepsilon_\ell\sigma|_{[v_n,\ldots,v_k]}) \\
		\rho^*(\varphi\cup\psi)(\sigma) &= \varepsilon_{k+\ell}\psi(\sigma|_{[v_n,\ldots,v_k]})\varphi(\sigma|_{[v_k,\ldots,v_0]}),
	\end{align*}
	where $n=k+\ell$. Passing to cohomology will make $\rho^*$ be the identity as discussed above, so the above equalities imply $\varepsilon_k[\varphi]\cup\varepsilon_\ell[\psi]=\varepsilon_{k+\ell}([\psi]\cup[\varphi])$, which is the result upon counting our signs.
\end{proof}
\begin{remark}
	For a surface $\Sigma$, we note that the cup product induces an antisymmetric form
	\[H^1(\Sigma;\QQ)\otimes_\QQ H^1(\Sigma;\QQ)\to H^2(\Sigma;\QQ)\cong\QQ,\]
	which shows up in differential geometry.
\end{remark}

\subsection{The K\"unneth Formula}
Given two graded rings $R$ and $S$, we can form a graded ring $R\otimes S$ in the usual way. However, due to our graded anticommutativity, we will require that our multiplication introduce the sign
\[(r\otimes s)(r'\otimes s')=(-1)^{(\deg s)(\deg r')}(rr'\otimes ss')\]
to account for switching $s$ and $r'$.
\begin{example}
	Take the graded polynomial ring $\Lambda_R[\alpha_1,\ldots,\alpha_n]$ where the $\alpha_i$ have degree $2i+1$. Note that $\alpha_i^2=0$ for each $\alpha_i$. One sees that
	\[H^\bullet\left(S^n;\ZZ\right)\cong\Lambda_\ZZ[\alpha_n]\]
	by an explicit computation.
\end{example}
With this in mind, we define the cross product.
\begin{definition}[cross product]
	Fix spaces $X$ and $Y$. Then we define the \textit{cross product} $\times\colon H^\bullet(X;R)\otimes H^\bullet(Y;R)\to H^\bullet(X\times Y;R)$ by extending
	\[(\alpha\times\beta)\coloneqq p_X(\alpha)\otimes p_Y(\beta)\]
	$R$-linearly to the entire tensor product.
\end{definition}
\begin{remark}
	One can recover the cup product from the cross product by using the diagonal embedding $\Delta\colon X\to X\times X$. Then the composite
	\[H^\bullet(X;R)\otimes H^\bullet(X;R)\stackrel\times\to H^\bullet(X\times X;R)\stackrel\times\Delta\to H^\bullet(X;R)\]
	is equal to the cup product. Indeed, the main point is that $\Delta$ composed with either projection is simply the identity.
\end{remark}
\begin{remark}
	In fact one can directly define a cross product by defining a chain map
	\[C_\bullet(X)\otimes C_\bullet(Y)\to C_\bullet(X\times Y)\]
	by taking two singular simplices $\sigma_X\colon\Delta^k\to X$ and $\sigma_Y\colon\Delta^\ell\to Y$ by producing a map $\Delta^k\times\Delta^\ell\to X\times Y$, essentially by viewing everything as a cube.
\end{remark}
The construction of the graded anticommutativity above assures that $\times$ is in fact a ring homomorphism. Indeed, we compute
\begin{align*}
	(\alpha\times\beta)\cup(\alpha'\cup\beta') &= p_X(\alpha)\cup p_Y(\beta)\cup p_X(\alpha')\cup p_Y(\beta') \\
	&= (-1)^{(\deg\alpha')(\deg\beta)}p_X(\alpha)\cup p_X(\alpha')\cup p_Y(\beta)\cup p_Y(\beta') \\
	&= (-1)^{(\deg\alpha')(\deg\beta)}p_X(\alpha\cup\alpha')\cup p_Y(\beta\cup\beta') \\
	&= (-1)^{(\deg\alpha')(\deg\beta)}(\alpha\cup\alpha')\times(\beta\cup\beta').
\end{align*}
In simple cases, the cross product map defines an isomorphism.
\begin{theorem}
	Fix CW-complexes $X$ and $Y$. If $H^\ell(Y;R)$ is a finitely generated free $R$-module for all $\ell$, then
	\[\times\colon H^\bullet(X;R)\otimes H^\bullet(Y;R)\to H^\bullet(X\times Y;R)\]
	is an isomorphism.
\end{theorem}
\begin{proof}
	We use the Eilenberg--Steenrod axioms. Define the cohomology theories
	\begin{align*}
		h^n(X,A) \coloneqq{}& \bigoplus_{i+j=n}H^i(X,A;R)\otimes_R H^{j}(Y;R) \\
		k^n(X,A) \coloneqq{}& H^n(X\times Y,A\times Y;R).
	\end{align*}
	Note that there is a natural transformation $\mu\colon h^n\to k^n$ given by the cross product. Now, one can check that $h^n$ and $k^n$ are both cohomology theories, and $\mu_n$ is an isomorphism on the point, so one can see purely formally that $\mu_n$ will be an isomorphism on all CW pairs $(X,A)$.

	Let's give a few of the details here.
	\begin{itemize}
		\item We note that $\mu$ is natural in the topological spaces automatically, and it is also natural in the excision long exact sequence by an explicit computation of the boundary maps.
		\item Being an isomorphism on the point extends to all CW complexes approximately as follows: one gets contractible spaces immediately, and then the wedge sum axiom allows us to get the skeleton to any finite-dimensional CW-complex. Then cellular homology allows us to get an isomorphism for any finite-dimensional CW-complex. One then gets the general case by taking some kind of limit.
		\item The axioms for $h^\bullet$ and $k^\bullet$ are checked rather immediately from the axioms for $H^\bullet$.
		\qedhere
	\end{itemize}
\end{proof}
Let's give a quick application to division rings.
\begin{proposition}
	If $D$ is a finite-dimensional division $\RR$-algebra, then $\dim_\RR K$ is a power of $2$.
\end{proposition}
\begin{proof}
	Say $D=\RR^n$, and we want to show that $n$ is a power of $2$; take $n\ge2$. Now, define $g\colon S^{n-1}\times S^{n-1}\to S^{n-1}$ by
	\[g(x,y)\coloneqq\frac{x\cdot y}{\left|x\cdot y\right|},\]
	where the point is that $x\cdot y$ is always nonzero when $x$ and $y$ are nonzero because $D$ is a division algebra. Now, having $(-x)y=-(xy)=x(-y)$ implies that $g(-x,y)=-g(x,y)=g(x,-y)$, so we descend to a map
	\[\overline g\colon\mathbb{RP}^{n-1}\times\mathbb{RP}^{n-1}\to\mathbb{RP}^{n-1}.\]
	This then produces a ring homomorphism
	\[H^\bullet\left(\mathbb{RP}^{n-1},\FF_2\right)\to H^\bullet\left(\mathbb{RP}^{n-1},\FF_2\right)\otimes H^\bullet\left(\mathbb{RP}^{n-1},\FF_2\right).\]
	Let the generators (in degree $1$) of the above three rings be $\gamma$, $\alpha$, and $\beta$, respectively. A topological computation reveals that $\gamma$ goes to $\alpha+\beta$, but then having $\gamma^n=0$ will force $(\alpha+\beta)^n$ to vanish, upon which expanding by the binomial theorem will enforce $n$ to be a power of $2$.
\end{proof}

\end{document}