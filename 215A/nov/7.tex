% !TEX root = ../notes.tex

\documentclass[../notes.tex]{subfiles}

\begin{document}

\section{November 7}

We're falling behind, but everything will be okay.

\subsection{Homology and the Fundamental Group}
Throughout, $X$ is path-connected.
\begin{proposition}[Hurewicz]
	Let $X$ be a path-connected space with basepoint $x_0\in X$. Then there is a natural map $h\colon\pi_1(X,x_0)\to H_1(X)$.
\end{proposition}
\begin{proof}
	Consider the path $\alpha\colon S^1\to X$. This will induce a map
	\[H_1(\alpha)\colon H_1\left(S^1\right)\to H_1(X),\]
	but $H_1\left(S^1\right)$ is isomorphic to $\ZZ$ generated by the counterclockwise loop. So we define $h(\alpha)$ as going to $H_1(\alpha)(1)$. Homotopic maps define the same map on homology so $h$ is defined up to homotopy class in $\pi_1(X,x_0)$.
	\begin{remark}
		An alternate way to think about this map is by viewing $S^1$ as $\Delta^1$ with endpoints identified, so $\alpha$ produces a singular chain $\Delta^1\to X$, and with the endpoints identified this is actually a singular cycle, so $[\alpha]$ is a genuine class in $H_1(X)$. Note that this agrees with the above definition by tracking through what the map $H_1\left(S^1\right)\to H_1(X)$ actually is: we send the generating singular cycle $\Delta^1\to S^1$ to the map $\Delta^1\to S^1\stackrel\alpha\to X$.
	\end{remark}
	Lastly, we should probably check that our map is a homomorphism. Fix $\alpha,\beta\colon S^1\to X$. Note that the composite $\alpha\cdot\beta$ can simply be thought of as a map $S^1\lor S^1\to X$. But now this looks like the composite
	\[S^1\to S^1\lor S^1\to X,\]
	which on homology is the map $H_1\left(S^1\right)\to H_1\left(S^1\right)\oplus H_1\left(S^1\right)\to H_1(X)$, which goes $1\mapsto(1,1)\mapsto h(\alpha)+h(\beta)$.
\end{proof}
\begin{theorem} \label{thm:pi1-ab}
	Let $X$ be a path-connected space with basepoint $x_0\in X$. Then $h$ descends to an isomorphism
	\[\pi_1(X,x_0)^{\mathrm{ab}}\to H_1(X).\]
\end{theorem}
\begin{proof}
	Note $\ker h$ contains the commutator of $\pi_1(X,x_0)$ because the image is an abelian group, so $h$ does descend to some morphism $\pi_1(X,x_0)^{\mathrm{ab}}\to H_1(X)$. Next, we check that $h$ is surjective: it suffices to show that any cycle lives in the image of $h$, so consider some cycle $z=\sum_in_i\sigma_i$. We may assume that $n_i\in\{\pm1\}$ for each $i$, and because $\del z$ must vanish, if any $\sigma_i$ is not immediately a loop, we may find some $\sigma_j$ which connects to $\sigma_i$ to cancel out the endpoints; this then allows us to replace $\sigma_i$ with $\sigma_i\cdot\sigma_j$ upon removing $\sigma_j$. Continuing this process finitely, we may assume that our cycle is a sum of loops. But now each of these loops is in the image of $h$ by translating them to have basepoint at $x_0$, where the translation is legal because this corresponds to adding a loop which goes directly forwards and backwards (which is of course homologically trivial).

	It remains to check injectivity. The point is that two-dimensional homology classes are represented by surfaces. Namely, a $1$-cycle is trivial if and only if it is represented by loops which are the boundaries of $\Delta^2$, making a similar argument to the one we gave above. However, such a boundary is a product of commutators because of how these oriented surfaces behave. Essentially, one glues together these $2$-cycles to build a surface that embeds into $X$ with boundary equal to the loop, and then one can apply a homotopy of the loop through this surface to trivialize it.
\end{proof}

\subsection{Applications of Homology}
Here is a nice result which we will use for some applications.
\begin{proposition} \label{prop:embed-sphere-homology}
	We have the following.
	\begin{listalph}
		\item Upon embedding $D^k\subseteq S^n$, we have $\widetilde H_i\left(S^n\setminus D^k\right)=0$ for all $i$.
		\item If $S$ is a subspace of $S^n$ homeomorphic to $S^k$ for $0\le k\le n$, we have
		\[\widetilde H_i\left(S^n\setminus S\right)\cong\begin{cases}
			\ZZ & \text{if }i=n-k-1, \\
			0 & \text{else}.
		\end{cases}\]
	\end{listalph}
\end{proposition}
Here are some nice applications.
\begin{corollary}[Jordan curve]
	Any embedding $f\colon S^1\to S^2$ separates $S^2$ into two path-connected components.
\end{corollary}
\begin{proof}
	Namely, $\widetilde H_0\left(S^2\setminus S^1\right)\cong\ZZ$, so $S^2\setminus S^1$ must have two connected components.
\end{proof}
\begin{example}
	One has
	\[\widetilde H_i\left(S^3\setminus S^1\right)\cong\begin{cases}
		\ZZ & \text{if }i=1, \\
		0 & \text{else}.
	\end{cases}\]
	We verified this by hand on the homework by providing $\pi_1\left(S^3\setminus S^1\right)$ with a presentation, from which the computation for $H_1$ follows by \Cref{thm:pi1-ab}. Note that this is potentially surprising because there are some pretty horrible embeddings $h\colon S^1\to S^3$; for example, $\pi_1\left(S^3\setminus h\left(S^1\right)\right)$ need not be well-behaved.
\end{example}
Anyway, let's show \Cref{prop:embed-sphere-homology}.
\begin{proof}[Proof of \Cref{prop:embed-sphere-homology}]
	We show our parts separately.
	\begin{listalph}
		\item We induct on $k$. Identify $D^k$ with its image in $S^n$, for convenience. If $k=0$, then we are looking at $S^n$ minus a point, which is homeomorphic to $\RR^n$, which is contractible and hence has trivial homology. We now apply the induction. Let $h\colon I^k\to D^k$ be a homeomorphism, for convenience. We would like to use Mayer--Vietoris, so we define
		\[A\coloneqq S^n\setminus h\left(I^{k-1}\times[0/1,2]\right)\qquad\text{and}\qquad B\coloneqq S^n\setminus h\left(I^{k-1}\times[1/2,1]\right).\]
		Then $A\cap B=S^n\setminus D^k$ is the desired space, and $A\cup B=S^n\setminus h\left(I^{k-1}\times\{1/2\}\right)$ is of lower dimension and so has vanishing homology by the induction. We now may apply \Cref{thm:mv-seq}, which tells us that
		\[\widetilde H_i\left(S^n\setminus D^k\right)\cong\widetilde H_i(A)\oplus\widetilde H_i(B).\]
		Now, any nontrivial cycle in $\widetilde H_i\left(S^n\setminus D^k\right)$ would imply require nontrivial cycle in $\widetilde H_i\left(A\right)$ or $\widetilde H_i\left(B\right)$. But now $A$ and $B$ are just some version of $S^n\setminus D^k$ again, so we may continue this subdivision process, and having a nontrivial cycle requires a nontrivial cycle in $\widetilde H_i\left(S^n\setminus h\left(I^{k-1}\times J\right)\right)$ for smaller and smaller intervals $J$, which will eventually converge to a unique point $x$ in all of these intervals $J$.
		
		We now complete by a compactness argument. Namely, $\alpha$ viewed as a cycle of $S^n\setminus\{x\}$ must be trivial, so we can write $\alpha=\del\beta$ for some $(i+1)$-cycle $\beta$, and because $\beta$ is the union of compact sets, it will live in some $S^n\setminus h\left(I^{k-1}\times J\right)$ for one of these vary small intervals $J$ (because the union of the $S^n\setminus h\left(I^{k-1}\times J\right)$s is $S^n\setminus\{x\}$), so the equation $\alpha=\del\beta$ must actually hold in one of the homology groups $\widetilde H_i\left(S^n\setminus h\left(I^{k-1}\times J\right)\right)$s, which is a contradiction.

		\item This is also an induction on $k$. For $k=0$, we note that $S^n\setminus S^0$ is $\RR^n$ minus a point, which is $S^{n-1}\times\RR$, which has exactly the correct homology by contracting away the $\RR$. To complete the proof, one does some Mayer--Vietoris argument. Namely, write $S^n$ as the hemispheres $D_1^k$ and $D_2^k$, which union to $S$ and have intersection some space $S'$ homeomorphic to $S^{k-1}$, from which \Cref{thm:mv-seq} produces
		\[\widetilde H_{i+1}\left(S^n\setminus D_1^k\right)\oplus\widetilde H_{i+1}\left(S^n\setminus D_2^k\right)\to \widetilde H_{i+1}\left(S^n\setminus S'\right)\to\widetilde H_{i}\left(S^n\setminus S\right)\to \widetilde H_{i}\left(S^n\setminus D_1^k\right)\oplus\widetilde H_i\left(S^n\setminus D_2^k\right).\]
		The left and right terms vanish by (a), so we get an isomorphism of our homology groups, from which the result follows by induction.
		\qedhere
	\end{listalph}
\end{proof}
Here is a surprising application to algebra.
\begin{theorem}
	The rings $\RR$ and $\CC$ are the only finite-dimensional commutative division $\RR$-algebras.
\end{theorem}
\begin{proof}
	Suppose $\RR^n$ has been given a commutative division ring structure. There is a map $f\colon S^{n-1}\to S^{n-1}$ by sending $x\mapsto x^2/\left|x^2\right|$, where $x^2$ refers to the multiplication structure; namely, $x^2\ne0$ when $x\ne0$ because $\RR^n$ is a division ring. Further, the product is multilinear and hence extends linearly from a basis, so it is essentially a linear map $\RR^n\times\RR^n\to\RR^n$ and hence is continuous, so $f$ is a continuous map. We also note that $f(-x)=f(x)$, so we in fact achieve a map
	\[\overline f\colon\mathbb{RP}^{n-1}\to S^{n-1}.\]
	We also note that $\overline f$ is injective because $\RR^n$ is commutative: having $x^2=(\alpha y)^2$ implies that $(x-\alpha y)(x+\alpha y)=0$ by commutativity, so $x=\pm\alpha y$, so $x$ and $y$ are the same point in $\mathbb{RP}^{n-1}$.

	Now, one can show that an injective continuous map from a compact manifold to a connected manifold (both of the same dimension) must be surjective and hence a homeomorphism. So $\overline f$ is a homeomorphism when $n\ge2$, but $\mathbb{RP}^{n-1}$ and $S^{n-1}$ fails to be a homeomorphism for $n>2$ because (say) they have different fundamental groups.

	So we are left with the cases $n=1$ and $n=2$. When $n=1$, there is nothing to say because it is an $\mathbb R$-algebra already and hence must be $\RR$. Lastly, one must show that a $2$-dimensional commutative division $\RR$-algebra must be $\CC$, which is just algebra and hence omitted.
\end{proof}
Here is a more topological application.
\begin{corollary}[Borsuk--Ulam]
	Each map $g\colon S^n\to\RR^n$ must have a point $x\in S^n$ such that $g(x)=g(-x)$.
\end{corollary}
In the case of $n=1$, this is some kind of intermediate value theorem. We will prove this in the general case via cohomology later.

\end{document}