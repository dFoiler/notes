% !TEX root = ../notes.tex

\documentclass[../notes.tex]{subfiles}

\begin{document}

\section{November 2}

Today let's discuss the axioms for homology.

\subsection{More on Mayer--Vietoris}
We continue our discussion of Mayer--Vietoris.
\mvseq*
\begin{proof}
	Note that we have the short exact sequence of simplices
	\[0\to C_\bullet(A\cap B)\to C_\bullet(A)\oplus C_n(B)\to C_\bullet(A)+C_\bullet(B)\to0,\]
	where the left map is $x\mapsto(x,-x)$ and the right map is $(x,y)\mapsto x+y$. Notably, this is exact because the kernel of the map $C_n(A)\oplus C_n(B)\to C_n(A)+C_n(B)$ is simply $C_n(A)\cap C_n(B)$, but the only way to have an $n$-chain land in both $A$ and in $B$ is for it to land in $A\cap B$, so $C_n(A\cap B)=C_n(A)\cap C_n(B)$ follows. Further, the inclusion $C_n(A)+C_n(B)\subseteq C_n(X)$ is a chain homotopy equivalence by \Cref{prop:homology-is-local} because $X$ is covered by $\{A,B\}$. So we have a long exact sequence in homology, which is the desired one upon noting that
	\[H_n(C_\bullet(A)\oplus C_\bullet(B))=H_n(A)\oplus H_n(B)\qquad\text{and}\qquad H_n(C_\bullet(A)+C_\bullet(B))=H_n(C_\bullet(X))=H_n(X),\]
	where the left equality is because $H_n$ is additive, and the right equality is by the chain homotopy equivalence as just discussed.
\end{proof}
\begin{example} \label{ex:take-suspension}
	We compute the homology of the suspension $SX=CX\cup_XCX$. Well, let $A$ be some open neighborhood around the left $CX$, and let $B$ be some open neighborhood around the left $CX$. Rigorously, if $SX$ is $X\times[-1,1]$ where we collapse $X\times\{-1\}$ and $X\times\{1\}$, then $A\coloneqq X\times[-1,0.1)$ and $B\coloneqq X\times(-0.1,1]$ will do. Then $A\cap B$ is homotopic to $X$, but $A$ and $B$ are both contractible to $CX$ and thus to a point, so \Cref{thm:mv-seq} tells us that
	\[0\to H_n(SX)\to\underbrace{\widetilde H_{n-1}(A\cap B)}_{H_n(X)}\to0\]
	is exact, so $H_n(SX)=H_{n-1}(X)$ follows. Approximately speaking, the geometric content here is that we can turn an $(n-1)$-cycle (made out of some simplices) and bring it up to an $n$-cycle by taking its cone.
\end{example}
\begin{example}
	Consider the torus knot $K_{n,m}$ of \Cref{ex:torus-knot}, and set $X\coloneqq S^3\setminus K_{n,m}$. Now one can choose $A$ to be the space outside the torus and $B$ to be the space inside the torus so that $X$ is covered by $A\cap B$, and both $A$ and $B$ include the boundary. However, $A$ and $B$ can both be contracted to $S^1$, as can their intersection, so
	\[\widetilde H_n(A)=\widetilde H_n(B)=\widetilde H_n(A\cap B)=\begin{cases}
		\ZZ & \text{if }n=1, \\
		0 & \text{if }n\ne1.
	\end{cases}\]
	However, $H_1(C)\to H_1(A)$ is multiplication by $p$ by because $C$ winds around $p$ times around the torus in one direction that way by construction, and similarly $H_1(C)\to H_1(B)$ is multiplication by $q$. The point is that \Cref{thm:mv-seq} yields
	\[\widetilde H_n(C)\to\widetilde H_n(A)\oplus\widetilde H_n(B)\to\widetilde H_n(X)\to0,\]
	so (for example) $\widetilde H_1(X)=\ZZ/pq\ZZ$, and the other homology will vanish.
\end{example}
\begin{example}
	Suppose $X=A\cup B$ where $X$ is a finite CW-complex, and $A$ and $B$ and $A\cap B$ are homotopic to finite CW-complexes. Then we claim $\chi(X)=\chi(A)+\chi(B)-\chi(A\cap B)$. Indeed, \Cref{thm:mv-seq} tells us that we have an exact sequence
	\[\cdots\to H_n(A\cap B)\to H_n(A)\oplus H_n(B)\to H_n(X)\to H_{n-1}(A\cap B)\to\cdots.\]
	Taking alternating sum of ranks, the total sum must vanish, so we conclude that
	\[\sum_{n=0}^\infty(-1)^n\op{rank}_\ZZ H_n(X)+\sum_{n=0}^\infty(-1)^n\op{rank}_\ZZ H_n(A\cap B)=\sum_{n=0}^\infty(-1)^n\op{rank}_\ZZ H_n(A)+\sum_{n=0}^\infty(-1)^n\op{rank}_\ZZ H_n(B),\]
	which is what we wanted.
\end{example}

\subsection{More on Homology with Coefficients}
As usual, $G$ is an abelian group, and $X$ is a space, and we recall $C_\bullet(X;G)\coloneqq C_\bullet(X)\otimes_\ZZ\ZZ[G]$. The arguments we made for $G=\ZZ$ generalize immediately; for example,
\[H_n(X,G)=\begin{cases}
	G & \text{if }n=0, \\
	0 & \text{if }n>0.
\end{cases}\]
We are able to define simplicial, singular, and cellular homology theories all in the same way, but we now allow coefficients in $\ZZ[G]$ instead of merely $\ZZ$. One complication is in computing the boundary map for the cellular chain complex, for which we need to understand how to compute the degree of a map. So we have the following result.
\begin{lemma} \label{lem:deg-with-coeffs}
	If $f\colon S^n\to S^n$ is a map of degree $m$, then the map $H_n(f)\colon H_n(S^n;G)\to H_n(S^n;G)$ is multiplication by $m$.
\end{lemma}
We will prove this as a result of naturality in $G$.
\begin{remark}
	Notably, we are allowing for multiplication by $m$ to be zero here!
\end{remark}
Anyway, here is our notion of naturality.
\begin{lemma}
	Fix a pair $(X,A)$. A homomorphism of groups $\varphi\colon G\to H$ induces a homomorphism of chain complexes
	\[C_n(\varphi)\colon C_n(X,A;G)\to C_n(X,A;H)\]
	which is functorial.
\end{lemma}
\begin{proof}
	The map is simply given by passing the coefficients in $C_n(X,A;G)$ through $\varphi$. It will commute with the boundary morphism of $C_n(X,A;G)$ and of $C_n(X,A;G)$ by a direct check.
\end{proof}
Having a map of chain complexes will thus induce a map on homology, allowing us to prove \Cref{lem:deg-with-coeffs}.
\begin{proof}[Proof of \Cref{lem:deg-with-coeffs}]
	Fix any $g\in G$ representing a class in $H_n\left(S^n;G\right)$. Then consider the map $\varphi\colon\ZZ\to G$ sending $\varphi(1)\coloneqq g$, and the above naturality tells us that the following diagram commutes.
	% https://q.uiver.app/#q=WzAsOCxbMCwwLCJIX24oU15uO1xcWlopIl0sWzEsMCwiSF9uKFNebjtcXFpaKSJdLFswLDEsIkhfbihTXm47RykiXSxbMSwxLCJIX24oU15uO0cpIl0sWzIsMCwiMSJdLFsyLDEsImciXSxbMywxLCJtZyJdLFszLDAsIm0iXSxbMCwxLCJIX24oZikiXSxbMCwyLCJcXHZhcnBoaSIsMl0sWzEsMywiXFx2YXJwaGkiXSxbMiwzLCJIX24oZikiXSxbNCw3LCIiLDAseyJzdHlsZSI6eyJ0YWlsIjp7Im5hbWUiOiJtYXBzIHRvIn19fV0sWzUsNiwiIiwwLHsic3R5bGUiOnsidGFpbCI6eyJuYW1lIjoibWFwcyB0byJ9fX1dLFs0LDUsIiIsMSx7InN0eWxlIjp7InRhaWwiOnsibmFtZSI6Im1hcHMgdG8ifX19XSxbNyw2LCIiLDEseyJzdHlsZSI6eyJ0YWlsIjp7Im5hbWUiOiJtYXBzIHRvIn19fV1d&macro_url=https%3A%2F%2Fraw.githubusercontent.com%2FdFoiler%2Fnotes%2Fmaster%2Fnir.tex
	\[\begin{tikzcd}
		{H_n(S^n;\ZZ)} & {H_n(S^n;\ZZ)} & 1 & m \\
		{H_n(S^n;G)} & {H_n(S^n;G)} & g & mg
		\arrow["{H_n(f)}", from=1-1, to=1-2]
		\arrow["\varphi"', from=1-1, to=2-1]
		\arrow["\varphi", from=1-2, to=2-2]
		\arrow["{H_n(f)}", from=2-1, to=2-2]
		\arrow[maps to, from=1-3, to=1-4]
		\arrow[maps to, from=2-3, to=2-4]
		\arrow[maps to, from=1-3, to=2-3]
		\arrow[maps to, from=1-4, to=2-4]
	\end{tikzcd}\]
	This is exactly what we wanted to prove.
\end{proof}
So we can compute our cellular chain complex boundary maps in the usual way.
\begin{example}
	Fix a field $F$, and we will compute the homology on $\mathbb{RP}^n$. Our discussion with lens spaces in \Cref{ex:lens-homology} produces a chain complex
	\[0\to F\to F\to F\to\cdots\to F\to F\to 0\]
	where the maps alternate being doubling or zero. So if $\op{char}F=2$, then all these maps are the zero map, so we get $H_k(\mathbb{RP}^nF;F)\cong F$ for $0\le k\le n$. And if $\op{char}F\ne2$, then multiplication by $2$ is an isomorphism, so we get $H_k(\mathbb{RP}^n;F)=F$ at only $k\in\{0,n\}$ where $n$ is odd. One can check that the Euler characteristic is zero in odd dimensions and one in even dimensions. Of course, a similar computation will work for more arbitrary lens spaces $L_m(\ell_1,\ldots,
	\ell_n)$, where the point is that multiplication by $m$ as a map $F\to F$ is zero if $\op{char}F\mid m$ and is an isomorphism otherwise.
\end{example}

\subsection{Axioms for Homology}
To give some perspective, let's provide a version of the Eilenberg--Steenrod axioms for reduced homology theories for CW-complexes. Namely, for each integer $n\in\ZZ$, we want a functor $\widetilde h_n$ from the category of CW-complexes to $\op{AbGrp}$. We now add in the following extra conditions.
\begin{enumerate}
	\item If $f$ is homotopic to $g$, then $\widetilde h_\bullet(f)=\widetilde g_\bullet(g)$.
	\item For each CW-pair $(X,A)$, we have a long exact sequence
	\[\cdots\to\widetilde h_n(A)\to\widetilde h_n(X)\to\widetilde h_n(X/A)\to\widetilde h_{n-1}(A)\to\cdots.\]
	This long exact sequence is functorial in the pair $(X,A)$.
	\item If $X=\bigwedge_\alpha X_\alpha$ with inclusions $i_\alpha\colon X_\alpha\to X$, then the induced map on homology
	\[\bigoplus_{\alpha}\widetilde h_n(X_\alpha)\to\widetilde h_n(X)\]
	is an isomorphism.
	\item Take $X$ to be a point. Then we have $\widetilde h_n(X)=0$ for any integer $n$.
\end{enumerate}
One can show that the above axioms are sufficient to fully pin down homology as singular homology. However, a relaxation of the dimension axiom is able to produce more exotic homology theories.
\begin{example}
	For example, there is some homology arising from cobordism of manifolds: consider maps of manifolds to our space $X$ modulo cobordism, where two maps $f_1\colon M_1\to X$ and $f_2\colon M_2\to X$ are equivalent modulo cobordism if and only if there is some $F\colon N\to X$ such that $\del N=M_1\sqcup M_2$ and $F|_{M_i}=f_i$ for each $i$.
\end{example}
\begin{remark}
	Generally speaking, homology theories provide functors from the homotopy category of topological spaces to the category of graded abelian groups. Isomorphisms between homology theories (perhaps on a subcategory of the homotopy category) amount to natural isomorphisms between these functors. Namely, in all the situations above, we were able to produce isomorphisms between our homology theories essentially on the level of chain complexes, which promises that the induced isomorphisms on the level of homology would be natural. We also remark that changing coefficients is natural.
\end{remark}

\end{document}