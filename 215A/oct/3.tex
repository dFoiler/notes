% !TEX root = ../notes.tex

\documentclass[../notes.tex]{subfiles}

\begin{document}

\section{October 3}

The homeworks will now get a little longer.

\subsection{Homology for \texorpdfstring{$\Delta$}{ Delta}-Complexes}
Let's recall our construction of $\Delta$-complexes.
\begin{definition}[simplex]
	We define the $n$-simplex
	\[\Delta^n\coloneqq\Bigg\{(t_0,t_1,\ldots,t_n)\in[0,1]^{n+1}:\sum_{k=0}^nt_k=1\Bigg\}.\]
	The $i$th face $\Delta_i^{n-1}\subseteq\Delta^n$ consists of the points with $t_i=0$. An \textit{orientation} of the simplex consists of an ordering of the vertices modulo the action of $A_{n+1}$ on the vertices $\{0,1,\ldots,n\}$.
\end{definition}
The orientation basically indicates which vertices are ``small'' and which are ``large.''
\begin{defihelper}[$\Delta$-complex] \nirindex{Delta-complex@$\Delta$-complex}
	A \textit{$\Delta$-complex} is a CW-complex $X$ with maps $\sigma_\alpha\colon\Delta^n\to X$ satisfying the following properties.
	\begin{itemize}
		\item Interiors: the map $\sigma_\alpha$ is injective on the interior of $\Delta^n$.
		\item Faces: the map $\sigma_\alpha$ restricted to the face $\Delta_i^{n-1}$ is simply another map $\sigma_\beta\colon\Delta^{n-1}\to X$.
		\item Continuity: if $A\subseteq X$ is open, then $\sigma_\alpha^{-1}(A)$ is open in $\Delta^n$ for each $\sigma_\alpha$.
	\end{itemize}
\end{defihelper}
Given a $\Delta$-complex $X$, orientations will tend to extend uniquely to $X$.
\begin{example}
	We provide an orientation on the torus $T^2$.
	\begin{center}
		\begin{asy}
			unitsize(1.5cm);
			draw((0,0)--(1,0), Arrow(Relative((0.6))));
			draw((1,0)--(1,1), Arrow(Relative((0.6))));
			draw((0,0)--(0,1), Arrow(Relative(0.6)));
			draw((0,1)--(1,1), Arrow(Relative(0.6)));
			draw((0,0)--(1,1), Arrow(Relative(0.6)));
		\end{asy}
	\end{center}
\end{example}
Note that the diagonal arrow cannot go the other way to have an orientation because this would create a loop!
\begin{example}
	We provide an orientation on the projective plane $\PP^2$.
	\begin{center}
		\begin{asy}
			unitsize(1.5cm);
			draw((0,0)--(1,0), Arrow(Relative((0.6))));
			draw((1,1)--(1,0), Arrow(Relative((0.6))));
			draw((0,0)--(0,1), Arrow(Relative(0.6)));
			draw((1,1)--(0,1), Arrow(Relative(0.6)));
			draw((0,0)--(1,1), Arrow(Relative(0.6)));
		\end{asy}
	\end{center}
\end{example}
We would like to define homology. For this, we have a notion of a chain.
\begin{definition}[chain]
	Fix a $\Delta$-complex $X$ with maps $\sigma_\alpha\colon\Delta^n\to X$. Then we define \textit{chains} $\Delta_n(X)$ to be the formal sums
	\[\Delta_n(X)\coloneqq\Bigg\{\sum_\alpha n_\alpha\sigma_\alpha:n_\alpha\in\ZZ\Bigg\},\]
	and then we define the \textit{chain map} $\del_n\colon\Delta_n(X)\to\Delta_{n-1}(X)$ given by
	\[\del_n(\sigma_\alpha)\coloneqq\sum_{i=0}^n(-1)^i\sigma_\alpha|_{\Delta_i^{n-1}}.\]
\end{definition}
The point of the signs in the definition of $\del_n$ is to have the correct orientation. For example, suppose we want to go ``around'' $\Delta^2$ as in this diagram.
\begin{center}
	\begin{asy}
		unitsize(2cm);
		draw((0,0)--(1,0), Arrow(Relative(0.6)));
		draw((1,0)--(1,1), Arrow(Relative(0.6)));
		draw((0,0)--(1,1), Arrow(Relative(0.6)));
		label("$\Delta^1_0$", (0.5,0), S);
		label("$\Delta^1_2$", (1,0.5), E);
		label("$\Delta^1_1$", (0.5,0.5), NW);
	\end{asy}
\end{center}
One now has the following check.
\begin{proposition}
	Fix a $\Delta$-complex $X$. For any positive integer $n$, we have $\del_{n-1}\circ\del_n=0$.
\end{proposition}
\begin{proof}
	Direct computation. It suffices to show this for $\Delta^n$ because $\Delta_n(X)$ is freely generated by images of this $\Delta^n$. And for $\Delta^n$, the point is that our signs are going to cancel:
	\begin{align*}
		(\del_{n-1}\circ\del_n)\left(\Delta^n\right) &= \del_{n-1}\Bigg(\sum_{i=0}^n(-1)^i\Delta^{n-1}_i\Bigg) \\
		&= \sum_{i=0}^{n}(-1)^i\del_{n-1}\left(\Delta^{n-1}_i\right).
	\end{align*}
	Now, for some notation, writing out the vertices $\Delta^n$ as $\{0,1,\ldots,n\}$, we write $\Delta^n=[0,1,\ldots,n]$ so that $\Delta^{n-1}_i=[0,\ldots,\widehat i,\ldots,n]$, so w are looking at
	\begin{align*}
		(\del_{n-1}\circ\del_n)\left(\Delta^n\right) &= \sum_{i=0}^{n}(-1)^i\del_{n-1}([0,\ldots,\widehat i,\ldots,n]) \\
		&= \sum_{i=0}^{n}\Bigg(\sum_{j=0}^{i-1}(-1)^i(-1)^j[0,\ldots,\widehat j,\ldots,\widehat i,\ldots,n]+\sum_{j=i+1}^{n}(-1)^i(-1)^{j+1}[0,\ldots,\widehat i,\ldots,\widehat j,\ldots,n]\Bigg) \\
		&= \sum_{j<i}(-1)^{i+j}[0,\ldots,\widehat j,\ldots,\widehat i,\ldots,n]-\sum_{i<j}(-1)^{i+j}[0,\ldots,\widehat i,\ldots,\widehat j,\ldots,n] \\
		&= 0,
	\end{align*}
	as desired.
\end{proof}
We are now ready to define homology.
\begin{definition}[simplicial homology]
	Fix a $\Delta$-complex $X$. Then we define $\Delta(X)$ to be the graded module $\bigoplus_{n=0}^\infty\Delta_n(X)$, and we define the \textit{$n$th homology group} as
	\[H_n^\Delta(X)=H_n(\Delta(X))\coloneqq\frac{\ker\del_n}{\im\del_{n+1}}.\]
	For notation, we set $Z_n(X)\coloneqq\ker\del_n$ to be \textit{$n$-cycles} and $B_n(X)\coloneqq\im\del{n+1}$ to be \textit{$n$-boundaries}. Then $H_n(\Delta(X))=Z_n(X)/B_n(X)$, so we are measuring cycles which are not boundaries, which approximately is finding holes.
\end{definition}
Note that we have not shown that $H_\bullet$ does not depend on the choice of $\Delta$-structure, which is why we are marking our $H_n^\Delta$ by $\Delta$, but we will do this in due time.
\begin{example}
	Give $S^1$ a $\Delta$-complex structure by attaching both endpoints of $\Delta^1$ together at some vertex $v$ as an edge $e$.
	\begin{itemize}
		\item We see $H_0^\Delta\left(S^1\right)$ is $\ker\del_0/\im\del_1$, but $\im\del_1=0$ because we are looking at $\del_1(e)=v-v=0$. However, $\ker\del_0$ is simply all $\ZZ v$, so we have $\ZZ$.
		\item We see $H_1^\Delta\left(S^1\right)$ is $\ker\del_1/\im\del_2$, and then $\del_1=\ZZ e$ as shown in the previous point, but $\im\del_2=0$ because there is nothing to map, so we have $\ZZ$.
	\end{itemize}
	We note that all the higher homology groups vanish because there is nothing to compute.
\end{example}
\begin{example}
	Give $T^2$ the $\Delta$-complex as described earlier. We expect to have a two-dimensional hole and two one-dimensional holes. We compute some homology.
\end{example}
\begin{proof}
	\begin{center}
		\begin{asy}
			unitsize(2cm);
			draw((0,0)--(1,0), Arrow(Relative((0.6))));
			draw((1,1)--(1,0), Arrow(Relative((0.6))));
			draw((0,0)--(0,1), Arrow(Relative(0.6)));
			draw((1,1)--(0,1), Arrow(Relative(0.6)));
			draw((0,0)--(1,1), Arrow(Relative(0.6)));
			label("$a$", (0.5,0), S);
			label("$b$", (0,0.5), W);
			label("$c$", (0.5,0.5), NW);
			label("$U$", (1/4,3/4));
			label("$L$", (3/4,1/4));
			dot("$v$", (0,0), W);
		\end{asy}
	\end{center}
	Now, $\del_2(U)=b-c+a$ and $\del_2(L)=a-c+b$, which is the same, so $\ker\del_2$ is generated by $U-L$. Now, $\im\del_3=0$ (there is nothing to compute), so $H_2^\Delta\left(T^2\right)\cong\ZZ$. As for $H_1^\Delta$, we note that $\del_1$ is identically zero because there is only a single vertex, so $\ker\del_1=\ZZ a+\ZZ b+\ZZ c$, so $H_1^\Delta\left(T^2\right)=\ker\del_1/\im\del_2\cong\ZZ a\oplus\ZZ b$.
\end{proof}
\begin{example}
	Give $\PP^2$ the $\Delta$-complex as described earlier. We compute some homology.
\end{example}
\begin{proof}
	Here is our structure.
	\begin{center}
		\begin{asy}
			unitsize(2cm);
			draw((0,0)--(1,0), Arrow(Relative((0.6))));
			draw((1,1)--(1,0), Arrow(Relative((0.6))));
			draw((0,0)--(0,1), Arrow(Relative(0.6)));
			draw((1,1)--(0,1), Arrow(Relative(0.6)));
			draw((0,0)--(1,1), Arrow(Relative(0.6)));
			label("$a$", (0.5,0), S);
			label("$b$", (0,0.5), W);
			label("$c$", (0.5,0.5), NW);
			label("$U$", (1/4,3/4));
			label("$L$", (3/4,1/4));
			dot("$v$", (0,0), W);
			dot("$w$", (1,0), E);
		\end{asy}
	\end{center}
	Here are our computations.
	\begin{itemize}
		\item We see $H_0^\Delta\left(\PP^2\right)=\ZZ v\oplus\ZZ w/(\ZZ(v-w))\cong\ZZ$, where the point is that $\del_1(c)=0$ and $\del_1(a)=w-v$ and $\del_1(b)=v-w$.
		\item Next up, we compute $\del_2(U)=b-a+c$ and $\del_2(L)=a-b+c$, so $\del_2$ is injective, so $H_2^\Delta\left(\PP^2\right)=0$. Further, we note $\ker\del_1=\ZZ c\oplus\ZZ(a-b)$, and we have $\del_2(U+L)=2c$ and $\del_2(U-L)=2a-2b$, so we have
		\[H_1^\Delta\left(\PP^2\right)=\frac{\ZZ c\oplus\ZZ(a-b)}{\ZZ(2c)\oplus\ZZ(2a-2b)\oplus\ZZ(a-b+c)}\cong\frac\ZZ{2\ZZ},\]
		finishing.
		\qedhere
	\end{itemize}
\end{proof}
\begin{example}
	We note that $\del\Delta^{n+1}\cong S^n=\Delta^n$, so we can give $S^n$ a natural $\Delta$-complex structure. Then we can compute that $H_n^\Delta\left(\del\Delta^{n+1}\right)\cong\ZZ$, where the point is that $\del_{n+1}\left(\Delta^{n+1}\right)$ does provide a cycle, and all cycles are generated in this way.
\end{example}

\subsection{Singular Homology}
Let's define singular homology now.
\begin{definition}[singular simplex]
	Fix a topological space $X$. A \textit{singular $n$-simplex} is simply a map $\sigma\colon\Delta^n\to X$ to a topological space, with no other requirements. We define our $n$-chains $C_n(X)$ to be the $\ZZ$-linear formal sums of such $\sigma$s, and we define our chain maps $\del_n\colon C_n(X)\to C_{n-1}(X)$ given in the usual way by
	\[\del_n(\sigma)\coloneqq\sum_{i=0}^n(-1)^i\sigma|_{\Delta_i^{n-1}}.\]
\end{definition}
As before, one can do the exact same proof to show that $\del_n\circ\del_{n+1}=0$, and so we may define homology.
\begin{definition}[singular homology]
	Fix a topological space $X$. Then we define $S(X)$ to be the $\Delta$-complex with exactly one $n$simplex $\Delta^n_\sigma$ for each singular $n$-simplex $\sigma\colon\Delta^n\to X$, attached via faces. Then we define $H_n(X)$ to be the $n$th homology on $S(X)$ of the chain
	\[\cdots\to C_{n+1}(X)\to C_n(X)\to C_{n-1}(X)\to\cdots.\]
\end{definition}
Dealing with $S(X)$ is a little annoying. By allowing for repetitions, we may assume that all our $\ZZ$-coefficients are actually $1$. For $n=1$, one can realize these as oriented loops, and for $n=2$, we can think of these as maps of oriented surfaces.

\end{document}