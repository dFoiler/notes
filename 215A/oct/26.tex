% !TEX root = ../notes.tex

\documentclass[../notes.tex]{subfiles}

\begin{document}

\section{October 26}

Today we discuss cellular homology.

\subsection{Cellular Homology}
Let's attempt to compute the homology of a CW-complex.
\begin{lemma} \label{lem:towards-cell-hom}
	Fix a CW-complex $X$ and indices $k$ and $n$.
	\begin{listalph}
		\item $\widetilde H_k\left(X^n/X^{n-1}\right)=0$ if $k\ne n$.
		\item $H_n\left(X^n/X^{n-1}\right)$ is free abelian if $k=n$, with a basis given by the $n$-cells.
		\item $H_k\left(X^n\right)=0$ if $k>n$.
		\item The inclusion $i\colon X^n\to X$ induces an isomorphism $H_n(i)\colon H_k\left(X^n\right)\to H_k(X)$ if $k<n$ and is a surjection if $k=n$.
	\end{listalph}
\end{lemma}
\begin{proof}
	This is similar to what we saw with $\Delta$-complexes.
	\begin{listalph}
		\item We see
		\[H_k\left(X^n,X^{n-1}\right)\cong\widetilde H_k\left(X^n/X^{n-1}\right)=\widetilde H_k\left(\biglor S^n\right),\]
		and we know the homology of $S^n$ already.
		\item This follows by induction on $n$ and using (a). The base case is that $H_k\left(X^0\right)=0$ for $k>0$. The long exact sequence provides
		\[H_k\left(X^{n-1}\right)\to H_k\left(X^n\right)\to H_{k-1}\left(X^n,X^{n-1}\right).\]
		The left term vanishes by the inductive hypothesis, and the right term vanishes by (a), so the middle term will also vanish.
		\item A similar exact sequence as in (b) shows that $H_k(X^n)\to H_k\left(X^{n+1}\right)$ is an isomorphism if $k<n$ and surjective when $k=n$. Indeed, we simply write down
		\[H_{k+1}\left(X^{n+1},X^n\right)\to H_k(X^{n})\to H_k\left(X^{n+1}\right)\to H_k\left(X^{n+1},X^{n}\right)\]
		to achieve the result. If $X$ is finite-dimensional, we are done because $X=X^n$ for some $n$ large enough. In the infinite-dimensional case, we use the fact that
		\[H_k(X)=\colimit H_k(X^n)\]
		because any cycle or boundary lives in some fixed chain. So we get this result purely algebraically.
		\qedhere
	\end{listalph}
\end{proof}
We now build a complex from $X$ using its skeletons. For each $n$, we acknowledge that we have maps $H_{n+1}\left(X^{n+1},X^n\right)\to H_n(X^n)$ and $H_n(X^n)\to H_n\left(X^n,X^{n-1}\right)$ induced by some long exact sequences, so we get a map $H_{n+1}\left(X^{n+1},X^n\right)\to H_n\left(X^n,X^{n-1}\right)$ via composition. So we have a sequence
\[\cdots\to H_{n+1}\left(X^{n+1},X^n\right)\to H_n\left(X^n,X^{n-1}\right)\to H_{n-1}\left(X^{n-1},X^{n-2}\right)\to\cdots.\]
Quickly, we claim that this is a chain complex. Indeed, the main point is that the composition of two consecutive maps amounts to a long composition
\[H_{n+1}\left(X^{n+1},X^n\right)\to H_n(X^n)\to H_n\left(X^n,X^{n-1}\right)\to H_{n-1}\left(X^{n-1}\right)\to H_{n-1}\left(X^{n-1},X^{n-2}\right).\]
However, the composite of the middle three maps must vanish by the relevant long exact sequence. So we are allowed to make the following definition.
\begin{definition}[cellular homology]
	Fix a CW-complex $X$. Then the \textit{cellular homology groups} $H_n^{CW}(X)$ is the homology of the chain complex
	\[\cdots\to H_{n+1}\left(X^{n+1},X^n\right)\to H_n\left(X^n,X^{n-1}\right)\to H_{n-1}\left(X^{n-1},X^{n-2}\right)\to\cdots.\]
\end{definition}
Of course, we would like to see that this is independent of the chosen CW-structure. In fact, we have the following result.
\begin{proposition}
	Fix a CW-complex $X$. For all $n$, we have $H_n^{CW}(X)=H_n(X)$.
\end{proposition}
\begin{proof}
	Draw the following very large diagram.
	% https://q.uiver.app/#q=WzAsNyxbMCwyLCJIX3tuKzF9XFxsZWZ0KFhee24rMX0sWF5uXFxyaWdodCkiXSxbMiwyLCJIX25cXGxlZnQoWF5uLFhee24tMX1cXHJpZ2h0KSJdLFsxLDEsIkhfblxcbGVmdChYXm5cXHJpZ2h0KSJdLFswLDAsIkhfblxcbGVmdChYXntuLTF9XFxyaWdodCkiXSxbMiwwLCJIX25cXGxlZnQoWF57bisxfSxYXm5cXHJpZ2h0KSJdLFszLDMsIkhfe24tMX1cXGxlZnQoWF57bi0xfVxccmlnaHQpIl0sWzQsMiwiSF97bi0xfVxcbGVmdChYXntuLTF9LFhee24tMn1cXHJpZ2h0KSJdLFswLDIsIlxcZGVsX3tuKzF9Il0sWzAsMSwiZF97bisxfSJdLFszLDJdLFsyLDEsImpfbiJdLFsyLDRdLFsxLDUsIlxcZGVsX24iLDJdLFs1LDYsImpfe24tMX0iLDJdLFsxLDYsImRfbiJdXQ==&macro_url=https%3A%2F%2Fraw.githubusercontent.com%2FdFoiler%2Fnotes%2Fmaster%2Fnir.tex
	\[\begin{tikzcd}[column sep=small]
		{H_n\left(X^{n-1}\right)} && {H_n\left(X^{n+1},X^n\right)} \\
		& {H_n\left(X^n\right)} \\
		{H_{n+1}\left(X^{n+1},X^n\right)} && {H_n\left(X^n,X^{n-1}\right)} && {H_{n-1}\left(X^{n-1},X^{n-2}\right)} \\
		&&& {H_{n-1}\left(X^{n-1}\right)}
		\arrow["{\del_{n+1}}", from=3-1, to=2-2]
		\arrow["{d_{n+1}}", from=3-1, to=3-3]
		\arrow[from=1-1, to=2-2]
		\arrow["{j_n}", from=2-2, to=3-3]
		\arrow[from=2-2, to=1-3]
		\arrow["{\del_n}"', from=3-3, to=4-4]
		\arrow["{j_{n-1}}"', from=4-4, to=3-5]
		\arrow["{d_n}", from=3-3, to=3-5]
	\end{tikzcd}\]
	Now, by the lemma, we see that $H_n\left(X^{n+1}\right)=H_n(X)$, and we know this is isomorphic to $H_n\left(X^n\right)/\im\del_{n+1}$. The diagram above has $H_n(X^{n-1})=0$, so $j_n$ is injective, and similarly, $j_{n-1}$ is injective. As such, we see $H_n\left(X^n\right)/\im\del_{n+1}$ is isomorphic to
	\[\frac{\im j_n}{\im(j_n\circ\del_{n+1})}.\]
	Again, because $j_{n-1}$ is injective, it follows that $\ker d_n=\ker\del_n$, which we know by the long exact sequence is the image of $j_n$, so the numerator is $\ker d_n$. Similarly, we know that the denominator is $\im\ d_{n+1}$ by definition, so we are done.
\end{proof}
\begin{example}
	If $X$ has some $r$ number of $n$-cells, then $H_n(X)$ is a subgroup of the free abelian group $H_n(X^n,X^{n-1})$ on $r$ generators, so $\ker d_n$ is a free abelian group of at most $r$ generators, so the quotient $H_n^{CW}(X)$ is an abelian group on at most $r$ generators as well.
\end{example}
\begin{example}
	Take $X=\mathbb{CP}^n$. This has exactly one cell in each even dimension. So \Cref{lem:towards-cell-hom} tells us that the cellular homology sequence has every other term equal to $\ZZ$ up to $2n$, so
	\[H_i\left(\mathbb{CP}^n\right)=\begin{cases}
		\ZZ & \text{if }i\in\{0,2,\ldots,2n\}, \\
		0 & \text{else}.
	\end{cases}\]
\end{example}
We would like to use $H_\bullet^{CW}$ to actually compute some homology groups, but for this we need to be able to compute the boundary maps $d_\bullet$.
\begin{proposition}
	Fix a CW-complex $X$. The boundary map $d_n\colon H_n\left(X^n,X^{n-1}\right)\to H_{n-1}\left(X^{n-1},X^{n-2}\right)$ sends some $n$-cell $e_n^\alpha$ representing a class in $H_n\left(X^n,X^{n-1}\right)$ to
	\[\sum_\beta d_{\alpha\beta}e_\beta^{n-1},\]
	where $d_{\alpha\beta}$ is the degree of the composite $\Delta_{\alpha\beta}$
	\[\underbrace{S_\alpha^{n-1}}_{\del e_\alpha^n}\to X^{n-1}/X^{n-2}\to\underbrace{S^{n-1}_\beta}_{e_\beta^{n-1}}.\]
	Here, the second map is induced via the retraction $q_\beta$ of $X^{n-1}/X^{n-2}$ onto $S^{n-1}_\beta$, squishing $X^{n-1}\setminus e_\beta^{n-1}$ to a point.
\end{proposition}
\begin{proof}
	Let $\Phi_\alpha\colon D^n_\alpha\to X^n$ denote the embedding of this $n$-cell, and $\varphi_\alpha\colon\del D^n_\alpha\to X^{n-1}$ denote the attaching map. We now draw the following very large diagram.
	% https://q.uiver.app/#q=WzAsNixbMCwwLCJIX24oRF5uX1xcYWxwaGEsXFxkZWwgRF5uX1xcYWxwaGEpIl0sWzEsMCwiSF97bi0xfShcXGRlbCBEXm5fXFxhbHBoYSkiXSxbMiwwLCJIX3tuLTF9KGVfXFxiZXRhXntuLTF9L1xcZGVsIGVee24tMX1fXFxiZXRhLCopIl0sWzAsMSwiSF9uXFxsZWZ0KFhebixYXntuLTF9XFxyaWdodCkiXSxbMSwxLCJIX3tuLTF9XFxsZWZ0KFhee24tMX0sWF57bi0yfVxccmlnaHQpIl0sWzIsMSwiSF97bi0xfShYXntuLTF9L1hee24tMn0sKikiXSxbMCwzLCJcXFBoaV9cXGFscGhhIl0sWzEsNCwiXFx2YXJwaGlfXFxhbHBoYSJdLFswLDEsIlxcZGVsIl0sWzMsNCwiZF9uIl0sWzEsMiwiXFxEZWx0YV97XFxhbHBoYVxcYmV0YX0iXSxbNCw1LCJxIl0sWzUsMiwicV9cXGJldGEiXV0=&macro_url=https%3A%2F%2Fraw.githubusercontent.com%2FdFoiler%2Fnotes%2Fmaster%2Fnir.tex
	\[\begin{tikzcd}
		{H_n(D^n_\alpha,\del D^n_\alpha)} & {H_{n-1}(\del D^n_\alpha)} & {H_{n-1}\left(e_\beta^{n-1}/\del e^{n-1}_\beta,*\right)} \\
		{H_n\left(X^n,X^{n-1}\right)} & {H_{n-1}\left(X^{n-1},X^{n-2}\right)} & {H_{n-1}(X^{n-1}/X^{n-2},*)}
		\arrow["{\Phi_\alpha}", from=1-1, to=2-1]
		\arrow["{\varphi_\alpha}", from=1-2, to=2-2]
		\arrow["\del", from=1-1, to=1-2]
		\arrow["{d_n}", from=2-1, to=2-2]
		\arrow["{\Delta_{\alpha\beta}}", from=1-2, to=1-3]
		\arrow["q", from=2-2, to=2-3]
		\arrow["{q_\beta}", from=2-3, to=1-3]
	\end{tikzcd}\]
	Here, $q$ and $q_\beta$ are the relevant quotient maps. Then one tracks around the relevant diagram and sums over all $\beta$ to achieve the result. In particular, $q_\beta$ detects the coordinate of $e_\beta^{n-1}$ in $d_n(e_n^\alpha)$, and $e_n^\alpha$ is the image of a generator of $H_n(D^n_\alpha,\del D^n_\alpha)$ passed through $\Phi_\alpha$. So the top composite tells us what the coordinate of $e_\beta^{n-1}$ in $d_n(e^n_\alpha)$ should look like, which we see is the degree of $\Delta_{\alpha\beta}$, as needed. (Note that $\del$ above is an isomorphism because the relevant long exact sequence has the terms before and after the homology of a disk, which vanishes because disks are contractible.)
\end{proof}
\begin{example} \label{ex:sigma-2-hom}
	Consider the surface $\Sigma_2$ produced by identifying opposite ends of an octagon. This has one vertex, four edges, and one face, so our cellular homology chain complex is
	\[0\to\ZZ\to\ZZ^4\to\ZZ\to0.\]
	Using the above formula, we see that each edge in $\ZZ^4$ goes to $0$ (the main point is that we are taking the edge and doing a signed sum of its boundary, but the boundary points have been identified), so we verify that $H_0(\Sigma_2)=\ZZ$. Next, for the face $e^2$ generating the left $\ZZ$, one checks that the identified edges are in such a way that the differential again vanishes, so $H_1(\Sigma_2)=\ZZ^4$ and $H_2(\Sigma_2)=\ZZ$. All higher homology vanishes.
\end{example}
\begin{example} \label{ex:four-rp2-hom}
	Consider the surface $X$ produced by identifying adjacent edges of an octagon. There is still one vertex, four edges, and one face, so our cellular homology chain complex is
	\[0\to\ZZ\to\ZZ^4\to\ZZ\to0.\]
	For the same reason as in the previous example, one sees that $\ZZ^4\to\ZZ$ is the zero map, verifying $H_0(\Sigma_2)=\ZZ$. Computing using the boundary formula, we see that $d_2\colon\ZZ\to\ZZ^4$ is the diagonal map multiplied by $2$. So $H_2(\Sigma_2)=0$ because $d_2$ is injective, and $H_1(\Sigma_2)=\ZZ^4/(2,2,2,2)\ZZ$. One can see this group is $\ZZ^3\oplus(\ZZ/2\ZZ)$, where the point is that we have given $\ZZ^4$ a new basis given by $(1,0,0,0)$ and $(0,1,0,0)$ and $(0,0,1,0)$ and $(1,1,1,1)$.
\end{example}

\subsection{Euler Characteristic}
Fix a finite CW-complex $X$.
\begin{definition}[Euler characteristic]
	Fix a finite CW-complex $X$. Then the \textit{Euler characteristic} $\chi(X)$ is the alternating sum
	\[\sum_{n\ge0}(-1)^nc_n,\]
	where $c_n$ is the number of $n$-cells of $X$.
\end{definition}
A priori, $\chi(X)$ depends on the CW-structure of $X$, but we can remove this dependency.
\begin{proposition} \label{prop:euler-char-by-hom}
	Fix a finite CW-complex $X$. Then
	\[\chi(X)=\sum_{n\ge0}(-1)^n\op{rank}H_n(X).\]
	Here, $\op{rank}H_n(X)$ is the number of $\ZZ$-summands in the finitely generated abelian group $H_n(X)$
\end{proposition}
Alternatively, the rank is $\dim_\QQ(H_n(X)\otimes_\ZZ\QQ)$, where the point is that tensoring by $\QQ$ deletes the torsion. We will want the following result.
\begin{lemma}
	Fix a short exact sequence
	\[0\to A\to B\to C\to 0\]
	of finitely generated abelian groups. Then $\op{rank}B=\op{rank}A+\op{rank}C$.
\end{lemma}
\begin{proof}
	Tensor with $\QQ$ and then use the corresponding fact for dimensions of $\QQ$-vector spaces, which is proven directly by counting bases.
\end{proof}
We are now ready to prove \Cref{prop:euler-char-by-hom}.
\begin{proof}[Proof of \Cref{prop:euler-char-by-hom}]
	More generally, suppose we have a finite chain complex
	\[0\to C_k\to C_{k-1}\to\cdots\to C_1\to C_0\to0\]
	with boundary maps $d_n\colon C_n\to C_{n-1}$, and we let $\chi(C_\bullet)$ denote the sum
	\[\chi(C_\bullet)\coloneqq\sum_{n\ge0}\op{rank}H_n(C_\bullet).\]
	Note that we have the short exact sequence
	\[0\to\ker d_k\to C_k\to\im d_k\to0,\]
	so we find that $\op{rank}C_k=\op{rank}\ker d_k+\op{rank}\im d_k=\op{rank}H_k(C_\bullet)+\op{rank}\im d_k$. In particular, we find that, trying to reduce ourselves down to
	\[0\to\frac{C_{k-1}}{\im d_k}\to C_{k-2}\to\cdots\to C_1\to C_0\to0,\]
	we have
	\[\sum_{n\ge0}(-1)^n\op{rank}C_n=\sum_{n\ge0}(-1)^n\op{rank}H_n(C_\bullet).\]
	Anyway, for our application, we take $C_k=H_k\left(X^k,X^{k-1}\right)$ to be our cellular homology chain complex, so it follows $H_n(C_\bullet)=H_n(X)$ and $\op{rank}C_n$ is the number of $n$-cells. This completes the proof.
\end{proof}
\begin{remark}
	As a nice corollary, we see that $\chi(X)$ is homotopy invariant, which is not so obvious. Notably, this allows us to define $\chi(X)$ whenever $X$ is homotopy equivalent to a CW-complex.
\end{remark}

\subsection{Homology with Coefficients}
If $A$ is any abelian group, we can define simplicial homology with coefficients in $A$ simply using by replacing simplicial chains $C_n(X)$ with $C_n(X;A)\coloneqq C_n(X)\otimes_\ZZ A$, which can be intuitively thought of as the free $A$-module with basis given by singular complexes $\sigma_\bullet\colon\Delta^n\to X$. Notably, if $A$ is in fact a ring, then these are $R$-modules, so if $A$ is a field, these are $F$-vector spaces!
\begin{example}
	Let $R$ be a ring, and consider the surface $\Sigma_2$ from earlier. Then the same computation as in \Cref{ex:sigma-2-hom} reveals
	\[H_i(\Sigma_2;R)=\begin{cases}
		R & \text{if }i\in\{0,2\}, \\
		R^4 & \text{if }i=1, \\
		0 & \text{else}.
	\end{cases}\]
\end{example}
\begin{example}
	Consider the space $X$ constructed in \Cref{ex:four-rp2-hom} and work with coefficients in $\FF_2$. Then the same computation as in the example tells us that the relevant cellular homology sequence
	\[0\to\FF_2\to\FF_2^4\to\FF_2\to0\]
	has differentials equal to $0$! So the homology changes.
\end{example}

\end{document}