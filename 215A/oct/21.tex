% !TEX root = ../notes.tex

\documentclass[../notes.tex]{subfiles}

\begin{document}

\section{October 19}

We continue with some applications of homology.

\subsection{Applications of Degree}
Let's give a few fun applications of the degree.
\begin{proposition}
	Fix an integer $n$. Then $S^n$ has a continuous vector field nonzero everywhere if and only if $n$ is odd.
\end{proposition}
\begin{proof}
	Quickly, recall that a vector field is a function assigning a tangent vector to each point. Namely, for each $x\in S^n$, there is a tangent plane $T_xS^n\subseteq\RR^{n+1}$ consisting of the vectors $y\in\RR^{n+1}$ such that $(y-x)\cdot x=0$. Shifting down by $x$, we may as well as say that $T_xS^n$ intersects the origin, and so we are asking for a continuous map $f\colon S^n\to\RR^{n+1}$ such that $f(x)\cdot x=0$ for each $x\in S^n$.

	For example, there is a continuous vector field nonzero everywhere on $S^1$ given by $(x,y)\mapsto(y,-x)$. More generally, if $n$ is odd, then $S^n\subseteq\RR^{n+1}$ can have its coordinates enumerated by $(x_1,y_1,\ldots,x_n,y_n)$, and we have a continuous vector field given by
	\[(x_1,y_1,\ldots,x_n,y_n)\mapsto(-y_1,x_1,\ldots,-y_n,x_n).\]
	Thus, if $n$ is odd, we have a nonzero continuous vector field.

	For the other direction, suppose we have a nonzero continuous vector field $f\colon S^n\to\RR^{n+1}$. Applying a deformation retraction, we may assume that $f$ actually maps $S^n\to S^n$. But then $f$ maps a vector to a perpendicular vector, so it has no fixed points, so we have a homotopy to the antipodal map, so $\deg f=(-1)^{n+1}$. On the other hand, $f$ is homotopic to the identity by simply following the vector field backwards to the original point. So $\deg f=1$ also, so $n$ must be odd.
\end{proof}
\begin{remark}
	Colloquially, this is the hairy ball theorem: there is no way to comb the hair of a ball $S^2\subseteq\RR^3$.
\end{remark}
\begin{remark}
	A more interesting question one can ask is for which $n$ do there exist $n$ pointwise orthogonal vector fields which vanish nowhere. This is equivalent to saying that the tangent bundle $TS^n$ is trivializable. We discussed how to do this for $S^1$, and there is a similar process for $S^3$ (viewing $\RR^4$ as the underlying vector space for a quaternion algebra) as well as $S^7$ (using the octonions). It turns out that these are the only such $n$.
\end{remark}
\begin{proposition}
	Let $n$ be an even integer. Then $\ZZ/2\ZZ$ is the only group which can act freely on $S^n$.
\end{proposition}
\begin{proof}
	Suppose $G$ is a group acting freely on $S^n$. Then we show that $G$ has an injection into $\ZZ/2\ZZ$. Note that each $g\in G$ must act by a homeomorphism on $S^n$ because it has inverse given by $g^{-1}$, so the action of $g$ must be surjective, so we see that $\deg g\in\{\pm1\}$. Because $\deg$ is multiplicative, this is actually a homomorphism $\deg\colon G\to\{\pm1\}$. We argue that this map is injective, which will complete the proof.

	Well, suppose $g\ne e$ for some $g\in G$, and we show that $\deg g=-1$. To see this, note that having a free action implies that $g$ has no fixed points, so as usual $g$ is homotopic to the antipodal map, so $\deg g=(-1)^{n+1}=-1$.
\end{proof}
\begin{remark}
	Of course $\ZZ/2\ZZ$ acts on any $S^n$ because the antipodal map $x\mapsto-x$ has order $2$.
\end{remark}
\begin{remark}
	For odd spheres, the story is more complicated. We have classified all the groups which act linearly on spheres, but we don't know all the actions explicitly.
\end{remark}

\subsection{Local Degree}
Take $n>0$. Let's discuss a way to compute degree via a ``signed point count.'' Given a map $f\colon S^n\to S^n$, we can try to look locally at some point $y\in\im f$ and attempt to count the number of points in the pre-image of $f$. Signed appropriately, this will turn into the degree. For example, if we are looking at (say) differentiable maps $f\colon S^1\to S^1$, counting signed by direction turns into the winding number.

Explicitly, fix $y\in\im f$ such that the fiber $f^{-1}(\{y\})$ is finite, whose points we number off as $\{x_1,\ldots,x_n\}$. By choosing a radius less than half of the smallest distance between any two $x_\bullet$s, we may fix disjoint open neighborhoods $U_i$ around each $x_i$. We now draw the following rather large diagram.
% https://q.uiver.app/#q=WzAsOCxbMSwwLCJIX25cXGxlZnQoU15uLFNeblxcc2V0bWludXNcXHt4XFx9XFxyaWdodCkiXSxbMiwwLCJIX25cXGxlZnQoVV9pLFVfaVxcc2V0bWludXNcXHt4XFx9XFxyaWdodCkiXSxbMSwxLCJIX25cXGxlZnQoU15uXFxyaWdodCkiXSxbMywwLCJIX24oU15uLFNeblxcc2V0bWludXNcXHt5XFx9KSJdLFszLDEsIkhfblxcbGVmdChTXm5cXHJpZ2h0KSJdLFsyLDEsIkhfblxcbGVmdChTXm4sU15uXFxzZXRtaW51cyBmXnstMX0oXFx7eVxcfSlcXHJpZ2h0KSJdLFs0LDEsIlxcWloiXSxbMCwxLCJcXFpaIl0sWzAsMSwiMSJdLFswLDIsIjIiLDJdLFsxLDMsImYiXSxbMyw0LCIyIiwyXSxbMSw1XSxbMiw1XSxbNywyLCIiLDEseyJsZXZlbCI6Miwic3R5bGUiOnsiaGVhZCI6eyJuYW1lIjoibm9uZSJ9fX1dLFs0LDYsIiIsMSx7ImxldmVsIjoyLCJzdHlsZSI6eyJoZWFkIjp7Im5hbWUiOiJub25lIn19fV0sWzUsMCwiZiIsMV0sWzUsMywiZiIsMV1d&macro_url=https%3A%2F%2Fraw.githubusercontent.com%2FdFoiler%2Fnotes%2Fmaster%2Fnir.tex
\[\begin{tikzcd}
	& {H_n\left(S^n,S^n\setminus\{x\}\right)} & {H_n\left(U_i,U_i\setminus\{x\}\right)} & {H_n(S^n,S^n\setminus\{y\})} \\
	\ZZ & {H_n\left(S^n\right)} & {H_n\left(S^n,S^n\setminus f^{-1}(\{y\})\right)} & {H_n\left(S^n\right)} & \ZZ
	\arrow["1", from=1-2, to=1-3]
	\arrow["2"', from=1-2, to=2-2]
	\arrow["f", from=1-3, to=1-4]
	\arrow["2"', from=1-4, to=2-4]
	\arrow[from=1-3, to=2-3]
	\arrow[from=2-2, to=2-3]
	\arrow[Rightarrow, no head, from=2-1, to=2-2]
	\arrow[Rightarrow, no head, from=2-4, to=2-5]
	\arrow["f"{description}, from=2-3, to=1-2]
	\arrow["f"{description}, from=2-3, to=1-4]
\end{tikzcd}\]
To begin, we note excision by $S^n\setminus U_i$ implies that the $1$ arrow is an isomorphism. Because $S^n\setminus\{*\}$ is contractible for any point $*$, we see that the $2$ arrows are isomorphisms. We are now equipped to make the following definition.
\begin{definition}[local degree]
	Fix everything as above. Then the \textit{local degree} $\deg f|_{x_i}$ is the degree of the induced map $H_n(S^n)\to H_n(S^n)$ as above.
\end{definition}
\begin{proposition}
	Fix everything as above. Then
	\[\deg f=\sum_{i=1}^n\deg f|_{x_i}.\]
\end{proposition}
\begin{proof}
	We basically take direct sums of our large diagram, as follows.
	% https://q.uiver.app/#q=WzAsNixbMSwwLCJcXGRpc3BsYXlzdHlsZVxcYmlnb3BsdXNfe2k9MX1ebkhfbihVX2ksVV9pXFxzZXRtaW51c1xce3hfaVxcfSkiXSxbMCwwLCJIXm5cXGxlZnQoU15uLFNeblxcc2V0bWludXMgZl57LTF9KFxce3lcXH0pXFxyaWdodCkiXSxbMiwwLCJIX25cXGxlZnQoU15uLFNeblxcc2V0bWludXNcXHt5XFx9XFxyaWdodCkiXSxbMSwxLCJcXGRpc3BsYXlzdHlsZVxcYmlnb3BsdXNfe2k9MX1ebkhfbihTXm4sU15uXFxzZXRtaW51c1xce3hfaVxcfSkiXSxbMSwyLCJIXm4oU15uKSJdLFsyLDIsIkhebihTXm4pIl0sWzQsM10sWzQsNSwiZiJdLFs1LDIsIiIsMCx7ImxldmVsIjoyLCJzdHlsZSI6eyJoZWFkIjp7Im5hbWUiOiJub25lIn19fV0sWzAsM10sWzEsMCwiIiwyLHsibGV2ZWwiOjIsInN0eWxlIjp7ImhlYWQiOnsibmFtZSI6Im5vbmUifX19XSxbMCwyLCJcXGJpZ29wbHVzIGZfaSJdXQ==&macro_url=https%3A%2F%2Fraw.githubusercontent.com%2FdFoiler%2Fnotes%2Fmaster%2Fnir.tex
	\[\begin{tikzcd}
		{H^n\left(S^n,S^n\setminus f^{-1}(\{y\})\right)} & {\displaystyle\bigoplus_{i=1}^nH_n(U_i,U_i\setminus\{x_i\})} & {H_n\left(S^n,S^n\setminus\{y\}\right)} \\
		& {\displaystyle\bigoplus_{i=1}^nH_n(S^n,S^n\setminus\{x_i\})} \\
		& {H^n(S^n)} & {H^n(S^n)}
		\arrow[from=3-2, to=2-2]
		\arrow["f", from=3-2, to=3-3]
		\arrow[Rightarrow, no head, from=3-3, to=1-3]
		\arrow[from=1-2, to=2-2]
		\arrow[Rightarrow, no head, from=1-1, to=1-2]
		\arrow["{\bigoplus f_i}", from=1-2, to=1-3]
	\end{tikzcd}\]
	By excision to delete everything outside the $U_\bullet$s, we see that the top-left arrow is an isomorphism. Then the vertical rectangle commutes by tracking through how $H_n(S^n)\cong\ZZ$ goes around (this is really the diagram we drew above the definition), so we are done because the vertical maps are all isomorphisms.
\end{proof}
\begin{remark}
	Any map is homotopic to a map with finite fibers somewhere, so this local degree check can usually be carried through. Explicitly, cover $S^n$ by convex balls, such as the hemispheres
	\[H_i^\pm\coloneqq\{(x_0,\ldots,x_n):\pm x_i>0\}.\]
	Now, for $f\colon S^n\to S^n$, do a barycentric subdivision repeatedly until the diameter is smaller than the Lebesgue number of the cover $f^{-1}(H_i^\pm)$: i.e., we want a cover of $S^n$ such that each point in one of the covering sets lands inside some hemisphere. Then we can ``straighten'' the map $f$ inside one of the convex hemispheres to make the map $f$ piecewise affine. So the size of our fibers is bounded by the number of simplices of $f$.
\end{remark}
\begin{remark}
	In fact, one can show that two maps $f,g\colon S^n\to S^n$ are homotopic if and only if $\deg f=\deg g$, which allows us to strengthen the above result.
\end{remark}
Let's use this to show that any degree is achievable.
\begin{example}
	For $n=1$, the map $S^1\to S^1$ given by $z\mapsto z^k$ has degree $k$.
\end{example}
We now go up from $n=1$.
\begin{proposition}
	Fix a map $f\colon S^n\to S^n$. Then the suspension map $Sf\colon S^{n+1}\to S^{n+1}$ has $\deg Sf=\deg f$.
\end{proposition}
\begin{proof}
	The main concern is that we must go up in the dimension of our homology groups, for which we want to use the long exact sequence. Note that we have a map $Cf\colon(CS^n,S^n\times\{0\})\to(CS^n,S^n\times\{0\})$, so the quotient space is $S^n$. Naturality of our long exact sequences now produces the following commutative diagram.
	% https://q.uiver.app/#q=WzAsMTAsWzAsMCwiSF97bisxfShDU15uKSJdLFsxLDAsIkhfe24rMX0oQ1NebixTXm4pIl0sWzIsMCwiSF97bisxfShTXntuKzF9KSJdLFszLDAsIkhfbihTXm4pIl0sWzQsMCwiSF9uKENTXm4pIl0sWzAsMSwiSF97bisxfShDU15uKSJdLFsxLDEsIkhfe24rMX0oQ1NebixTXm4pIl0sWzIsMSwiSF97bisxfShTXntuKzF9KSJdLFszLDEsIkhfbihTXm4pIl0sWzQsMSwiSF9uKENTXm4pIl0sWzAsMV0sWzEsMiwiIiwwLHsibGV2ZWwiOjIsInN0eWxlIjp7ImhlYWQiOnsibmFtZSI6Im5vbmUifX19XSxbNiw3LCIiLDAseyJsZXZlbCI6Miwic3R5bGUiOnsiaGVhZCI6eyJuYW1lIjoibm9uZSJ9fX1dLFs1LDZdLFsyLDMsIlxcZGVsIl0sWzcsOCwiXFxkZWwiXSxbMyw0XSxbOCw5XSxbMCw1LCJDZiJdLFsxLDYsIkNmIl0sWzIsNywiU2YiXSxbNCw5LCJDZiJdLFszLDgsImYiXV0=&macro_url=https%3A%2F%2Fraw.githubusercontent.com%2FdFoiler%2Fnotes%2Fmaster%2Fnir.tex
	\[\begin{tikzcd}
		{H_{n+1}(CS^n)} & {H_{n+1}(CS^n,S^n)} & {H_{n+1}(S^{n+1})} & {H_n(S^n)} & {H_n(CS^n)} \\
		{H_{n+1}(CS^n)} & {H_{n+1}(CS^n,S^n)} & {H_{n+1}(S^{n+1})} & {H_n(S^n)} & {H_n(CS^n)}
		\arrow[from=1-1, to=1-2]
		\arrow[Rightarrow, no head, from=1-2, to=1-3]
		\arrow[Rightarrow, no head, from=2-2, to=2-3]
		\arrow[from=2-1, to=2-2]
		\arrow["\del", from=1-3, to=1-4]
		\arrow["\del", from=2-3, to=2-4]
		\arrow[from=1-4, to=1-5]
		\arrow[from=2-4, to=2-5]
		\arrow["Cf", from=1-1, to=2-1]
		\arrow["Cf", from=1-2, to=2-2]
		\arrow["Sf", from=1-3, to=2-3]
		\arrow["Cf", from=1-5, to=2-5]
		\arrow["f", from=1-4, to=2-4]
	\end{tikzcd}\]
	Here, $S^n$ has been embedded into $CS^n$ via the copy in the code, and the point is that the quotient $CS^n/S^n$ is simply $SS^n=S^{n+1}$. All terms on the ends vanish because $CS^n$ is contractible, so $\del$ is an isomorphism, so the proof is complete.
\end{proof}
\begin{remark}
	If a map $f\colon S^n\to S^n$ is differentiable at a point $x$, then an exercise we did on the homework allows us to compute $\deg f|_x$ as $\det Df_x$. Indeed, $f$ is locally linear at $x$, so we choose the corresponding neighborhood where $f$ is homotopic to a linear map, and the degree of linear maps was computed on the homework.
\end{remark}

\end{document}