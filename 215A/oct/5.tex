% !TEX root = ../notes.tex

\documentclass[../notes.tex]{subfiles}

\begin{document}

\section{October 5}

We continue our discussion of homology.

\subsection{Basic Homology Facts}
Let's continue working with our singular homology because it is a little more canonical. To begin, it suffices to look at path-connected spaces.
\begin{proposition}
	Fix a topological space $X$ with path-connected components $X_\alpha$ for $\alpha\in\pi_0(X)$. Then
	\[H_n(X)\cong\bigoplus_{\alpha\in\pi_0(X)}H_n(X_\alpha)\]
\end{proposition}
\begin{proof}
	Note that
	\[C_\bullet(X)\cong\bigoplus_{\alpha\in\pi_0(X)}C_\bullet(X_\alpha)\]
	because any map $\Delta^n\to X$ must land in a single path-connected component. We can see that this provides an isomorphism of chain complexes, so the isomorphism in homology follows.
\end{proof}
\begin{proposition}
	Fix a nonempty path-connected topological space $X$. Then $H_0(X)\cong\ZZ$.
\end{proposition}
\begin{proof}
	Let $\varepsilon\colon C_0(X)\to\ZZ$ be the map given by sending
	\[\sum_\sigma\alpha_\sigma\sigma\mapsto\sum_\sigma\alpha_\sigma.\]
	Intuitively, some $\sigma\colon\Delta^0\to X$ is just marking a point of $X$. Now, when $X$ is path-connected, we see that $\im\del_1=\ker\varepsilon$. Note that $\ker\varepsilon$ is generated by differences $p-q$ for points $p,q\in X$. So to get these differences, note that for any two points $p,q\in X$, we have a path $f\colon\Delta^1\to X$ with $f(0)=q$ and $f(1)=p$, so $\del_1(f)=p-q$, as needed. So we see that
	\[H^0(X)\cong\frac{C_0(X)}{\im\del_1}=\frac{C_0(X)}{\ker\varepsilon}\cong\ZZ,\]
	as needed.
\end{proof}
\begin{remark}
	The above points we are checking go under the ``Eilenberg--Steenrod axioms.''
\end{remark}
\begin{proposition}
	If $X$ is a point, then $H_n(X)=0$ for $n>0$.
\end{proposition}
\begin{proof}
	We do this computation by hand. Notably, for each $n$, there is a unique $n$-simplex $\sigma_n\colon\Delta^n\to X$ sending everyone to the point. Then we note
	\[\del\sigma_n=\sum_{i=0}^n(-1)^i\sigma_{n-1}=\begin{cases}
		0 & \text{if }n\text{is odd}, \\
		\sigma_{n-1} & \text{if }n\text{ is even}.
	\end{cases}\]
	Thus, our chain complex looks like
	\[\cdots\cong\underbrace{C_3(X)}_{\ZZ\sigma_3}\stackrel0\to\underbrace{C_2(X)}_{\sigma_2}\cong\underbrace{C_1(X)}_{\ZZ\sigma_1}\stackrel0\to\underbrace{C_0(X)}_{\ZZ\sigma_0}\to 0.\]
	At odd degrees $2n+1$, we have $\ker\del_{2n+1}=C_{2n+1}(X)=\im\del_{2n+2}$, so homology vanishes; at even degrees $ \ker\del_{2n}=0=\im\del_{2n+1}$, so homology still vanishes.
\end{proof}
The following technical definition will be helpful, mostly for functoriality reasons.
\begin{definition}[reduced homology]
	Fix a topological space $X$, and let $\varepsilon\colon C_0(X)\to\ZZ$ be the augmentation map. Then we define
	\[\widetilde H_0(X)=\frac{\ker\varepsilon}{\im\del_1},\]
	and $\widetilde H_n(X)=H_n(X)$ for $n>0$. In particular, $\widetilde H_0(\{p\})=0$.
\end{definition}

\subsection{Functoriality of Homology}
Note that $H_n$ is in fact a functor.
\begin{proposition}
	Fix a continuous map $f\colon X\to Y$. Then there is an induced map $H_\bullet(f)\colon H_\bullet(X)\to H_\bullet(Y)$.
\end{proposition}
\begin{proof}
	Post-composition will send some $\sigma\colon\Delta^n\to X$ to some $(f\circ\sigma)\colon\Delta^n\to Y$. This extends to a map of chain complexes
	\[C_\bullet(f)\colon C_\bullet(X)\to C_\bullet(Y),\]
	so we induce a map on homology. Rigorously, one notes that $(f\circ-)$ commutes with $\del$: one checks that
	% https://q.uiver.app/#q=WzAsOCxbMCwwLCJDX3tufShYKSJdLFsxLDAsIkNfe24tMX0oWCkiXSxbMCwxLCJDX3tufShZKSJdLFsxLDEsIkNfe24tMX0oWSkiXSxbMywwLCJcXHNpZ21hIl0sWzMsMSwiKGZcXGNpcmNcXHNpZ21hKSJdLFs0LDAsIlxcZGlzcGxheXN0eWxlXFxzdW1fe2k9MH1eblxcc2lnbWF8X3tcXERlbHRhX3tuLTF9Xml9Il0sWzQsMSwiXFxkaXNwbGF5c3R5bGVcXHN1bV97aT0wfV5uKGZcXGNpcmNcXHNpZ21hKXxfe1xcRGVsdGFfe24tMX1eaX0iXSxbMCwxLCJcXGRlbF9uIl0sWzIsMywiXFxkZWxfbiJdLFswLDIsIihmXFxjaXJjLSkiLDJdLFsxLDMsIihmXFxjaXJjLSkiXSxbNCw2LCIiLDAseyJzdHlsZSI6eyJ0YWlsIjp7Im5hbWUiOiJtYXBzIHRvIn19fV0sWzYsNywiIiwwLHsic3R5bGUiOnsidGFpbCI6eyJuYW1lIjoibWFwcyB0byJ9fX1dLFs1LDcsIiIsMix7InN0eWxlIjp7InRhaWwiOnsibmFtZSI6Im1hcHMgdG8ifX19XSxbNCw1LCIiLDIseyJzdHlsZSI6eyJ0YWlsIjp7Im5hbWUiOiJtYXBzIHRvIn19fV1d&macro_url=https%3A%2F%2Fraw.githubusercontent.com%2FdFoiler%2Fnotes%2Fmaster%2Fnir.tex
	\[\begin{tikzcd}
		{C_{n}(X)} & {C_{n-1}(X)} && \sigma & {\displaystyle\sum_{i=0}^n\sigma|_{\Delta_{n-1}^i}} \\
		{C_{n}(Y)} & {C_{n-1}(Y)} && {(f\circ\sigma)} & {\displaystyle\sum_{i=0}^n(f\circ\sigma)|_{\Delta_{n-1}^i}}
		\arrow["{\del_n^X}", from=1-1, to=1-2]
		\arrow["{\del_n^Y}", from=2-1, to=2-2]
		\arrow["{(f\circ-)}"', from=1-1, to=2-1]
		\arrow["{(f\circ-)}", from=1-2, to=2-2]
		\arrow[maps to, from=1-4, to=1-5]
		\arrow[maps to, from=1-5, to=2-5]
		\arrow[maps to, from=2-4, to=2-5]
		\arrow[maps to, from=1-4, to=2-4]
	\end{tikzcd}\]
	commutes, and this is enough to induce a map on the homology upon checking what lives in what kernels and images. Let's explain this: to begin, we note that $C_n(f)$ maps $\ker\del_n^X\to\ker\del_n^Y$ because $\del_n^Y(C_n(f)(\alpha))=C_n(f)(\del_n^X(\alpha))=0$. Similarly, we note that $C_n(f)$ maps $\im\del_{n+1}^X\to\im\del_{n+1}^Y$ because $C_n(f)(\del_{n+1}^X(\alpha))=\del_{n+1}^Y(C_n(f)(\alpha))$. Thus, we get to produce a map
	\[H_n(f)\colon\underbrace{\frac{\ker\del_n^X}{\im\del_{n+1}^Y}}_{H_n(X)}\to\underbrace{\frac{\ker\del_n^Y}{\im\del_{n+1}^Y}}_{H_n(Y)},\]
	as needed.
\end{proof}
\begin{remark}
	As usual, one can check the usual functoriality checks such as that $H_\bullet(f\circ g)=H_\bullet(f)\circ H_\bullet(g)$ and $H_\bullet({\id_X})=\id_{H_\bullet(X)}$. These facts follow directly from the definition of $H_\bullet$.
\end{remark}
More generally, the above proof establishes the following result.
\begin{proposition}
	Fix a homomorphism $f\colon\left(C,\del^C\right)\to\left(D,\del^D\right)$ of chain complexes such that $\del^C\circ f=f\circ\del^D$. Then $f$ induces a natural map on homology.
\end{proposition}
\begin{proof}
	This is the last half of the proof of the above proposition.
\end{proof}
We are now ready to show homotopy invariance. This will follow from the following result.
\begin{theorem} \label{thm:homotopy-on-homology-maps}
	Fix homotopic maps $f,g\colon X\to Y$ of topological spaces. Then $H_n(f)=H_n(g)$.
\end{theorem}
\begin{proof}
	The point is to construct a ``chain homotopy'' between the maps $H_n(f)$ and $H_n(g)$. Let $F_\bullet\colon X\times I\to Y$ be the needed homotopy from $f$ to $g$ with $F_0=f$ and $F_1=g$. Then any singular simplex $\sigma\colon\Delta^n\to X$ will induce a map $(F_\bullet\circ\sigma)\colon\Delta^n\times I\to Y$ with $(F_0\circ\sigma)=(f\circ\sigma)$ and $(F_1\circ\sigma)=(g\circ\sigma)$. Technically, $F\circ\sigma$ is not a singular chain, but it is somewhat close.

	The goal is as follows: for any chain $[c]\in C_n(X)$, we would like to produce a chain $[d]\in C_{n+1}(X)$ such that $[\del d]=[f(c)]-[g(c)]$, and this will show that $H_n(f)=H_n(g)$. For this, we would like to make $\Delta^n\times I$ more like a simplex, so we triangulate it in a way which will be compatible with restricting to faces (and hence compatible with $\del$).

	As a warm-up, let's explain how to triangulate $I^{n+1}=[0,1]^{n+1}$. This is a cube with vertices of the form $(x_0,\ldots,x_n)$ where $x_\bullet\{0,1\}$ for each $x_\bullet$. Now, for each $\sigma\in S_{n+1}$, we choose the $(n+1)$-simplex given by
	\[\Delta_\sigma\coloneqq\{(x_0,\ldots,x_n):x_{\sigma(0)}\le x_{\sigma(1)}\le\cdots\le x_{\sigma(n)}\}.\]
	Notably, every face will be homeomorphic to $I^n$, and we roughly respect rearranging the coordinates (it just moves simplices around), though reflections will reverse the orientation of the simplex; also, there are $(n+1)!$ total simplices. Summing, we see that $I^n$ is triangulated as
	\[\sum_{\sigma\in S_{n+1}}(-1)^{\op{sgn}\sigma}\Delta_\sigma.\]
	Now, each simplex contains $(0,\ldots,0)$ to $(1,\ldots,1)$, and one can read off $\sigma$ by noting the simplex has a unique monotonic path along the vertices of the cube from $(0,\ldots,0)$ to $(1,\ldots,1)$.

	We now return to note that $\Delta^n\times I=\Delta^n\times\Delta^1$ embeds into $(\Delta^1)^n=I^n$, so we may triangulate $\Delta^n\times I$ as a $\Delta$-subcomplex. Explicitly, we see that we are essentially choosing our monotonic path as having its first $i+1$ vertices in $\Delta^n\times\{0\}$ and its last $n-i+1$ vertices in $\Delta^n\times\{1\}$. Anyway, for this chosen $\Delta$-complex structure on $\Delta^n\times I$, there is a ``prism operator,'' we get something
	\[\rho_n\coloneqq\sum_i(-1)^i[v_0,\ldots,v_i,w_i,\ldots,w_n],\]
	where the vertices of $\Delta^n\times\{0\}$ are given by $v_0,\ldots,v_n$, and the vertices of $\Delta^n\times\{1\}$ are given by $w_0,\ldots,w_n$. Taking faces, we see that
	\[\del\rho_n=[v_0,\ldots,v_n]-[w_0,\ldots,w_n]+\sum_i(-1)^iF_i\circ\rho_{n-1},\]
	where $F_i$ corresponds to the $i$th face. But by construction of $\rho_\bullet$ and our $\Delta$-complex structure, it follows that this summation is merely $\rho_{n-1}\circ\del$, so we get the inductive equation
	\[\del\rho_n=[v_0,\ldots,v_n]-[w_0,\ldots,w_n]+\rho_{n-1}\del.\]
	Applying $F$, we get the needed chain homotopy: given a singular simplex $\sigma\colon\Delta^n\to X$, we define
	\[P(\sigma)\coloneqq (F\circ\sigma)(\rho_n),\]
	which is a map $P\colon C_n(X)\to C_{n+1}(Y)$, and the relation tells us that
	\[\del\circ P=C_\bullet(g)-C_\bullet(f)-P\circ\del,\]
	so upon going down to homology, we are done.
\end{proof}
\begin{remark}
	Here is an intuitive argument, using the notation of the first paragraph of the above proof. As a reduction step, we let $i_0\colon X\to X\times I$ and $i_1\colon X\to X\times I$ be the embeddings so that $i_t(a)\coloneqq(a,t)$. Now, $f=F\circ i_0$ and $g=F\circ i_1$, so by functoriality, it is enough to check that $H_n(i_0)=H_n(i_1)$. Thus, we may as well assume that $Y$ is $X\times I$ and that $f$ and $g$ are $i_0$ and $i_1$ respectively. At this point, the result is somewhat intuitive because one should be able to continuously deform $i_0\circ\sigma$ to $i_1\circ\sigma$ for any $\sigma\colon\Delta^n\to X$. However, it is mildly difficult to make this argument precise.
\end{remark}
\begin{corollary}
	Fix a homotopy equivalence $f\colon X\to Y$. Then $H_n(f)\colon H_n(X)\to H_n(Y)$ is an isomorphism.
\end{corollary}
\begin{proof}
	This follows from functoriality. Let $g\colon Y\to X$ be the inverse homotopy equivalence for $f$. Then
	\[H_n(f)\circ H_n(g)=H_n(f\circ g)\stackrel*=H_n({\id_Y})=\id_{H_n(Y)},\]
	where $\stackrel*=$ follows from \Cref{thm:homotopy-on-homology-maps}. A symmetric argument shows that $H_n(g)\circ H_n(f)=\id_{H_n(X)}$, so $H_n(f)$ is an isomorphism with inverse given by $H_n(g)$.
\end{proof}

\end{document}