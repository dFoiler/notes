% !TEX root = ../notes.tex

\documentclass[../notes.tex]{subfiles}

\begin{document}

\section{October 19}

Here we go.

\subsection{Excision for Fun and Profit}
Let's use excision to compute homology of some spaces.
\begin{proposition}
	Fix pointed topological spaces $(X_\alpha,x_\alpha)$ for $\alpha\in\lambda$, and let $X$ denote the wedge sum of these spaces. Then the induced map
	\[\bigoplus_{\alpha\in\lambda}\widetilde H_n(X_\alpha)\to\widetilde H_n(X)\]
	is an isomorphism.
\end{proposition}
\begin{proof}
	Apply \Cref{prop:quot-on-homology} to the good pair given by the disjoint union of the $X_\alpha$s and the disjoint union of the $x_\alpha$s.
\end{proof}
\begin{proposition}
	Fix nonempty open subsets $U\subseteq\RR^m$ and $V\subseteq\RR^n$ which are homeomorphic. Then $m=n$.
\end{proposition}
\begin{proof}
	Fix some $x\in U$. Then find an open ball $B(x,r)\subseteq U$, so excision tells us that
	\[\widetilde H_\bullet(U,U\setminus\{x\})=\widetilde H_\bullet(B(x,r),B(x,r)\setminus\{x\}).\]
	This is then isomorphic to $\widetilde H_\bullet(\RR^m,\RR^m\setminus\{0\})$ by using an isomorphism $B(x,r)\cong\RR^m$.
	
	Now, we claim that $\widetilde H_\bullet(\RR^m,\RR^m\setminus\{0\})$ is $\ZZ$ if $k=m$ and $0$ otherwise, which will complete the proof because it allows us to read off $m$ from $U$. This follows from the long exact sequence
	\[\underbrace{\widetilde H_k(\RR^m)}_0\to\widetilde H_k(\RR^m,\RR^m\setminus\{0\})\to\widetilde H_{k-1}(\RR^m\setminus\{0\})\to\underbrace{\widetilde H_{k-1}(\RR^n)}_0.\]
	Now, $\widetilde H_{k-1}(\RR^m\setminus\{0\})$ was computed in \Cref{ex:homology-sn}, so the result follows.
\end{proof}

\subsection{Functoriality of Long Exact Sequences}
Let's prove a few things.
\begin{proposition}
	Fix a map of pairs $f\colon(X,A)\to(Y,B)$. Then this induces a morphism of long exact sequences as follows.
	% https://q.uiver.app/#q=WzAsMTIsWzEsMCwiSF9uKEEpIl0sWzIsMCwiSF9uKFgpIl0sWzMsMCwiSF9uKFgsQSkiXSxbNCwwLCJIX3tuLTF9KEEpIl0sWzUsMCwiXFxjZG90cyJdLFswLDAsIlxcY2RvdHMiXSxbMSwxLCJIX24oQikiXSxbMiwxLCJIX24oWSkiXSxbMywxLCJIX24oWSxCKSJdLFs0LDEsIkhfe24tMX0oQikiXSxbNSwxLCJcXGNkb3RzIl0sWzAsMSwiXFxjZG90cyJdLFs1LDBdLFswLDFdLFsxLDJdLFsyLDNdLFszLDRdLFsxMSw2XSxbNiw3XSxbNyw4XSxbOCw5XSxbMCw2XSxbMSw3XSxbMiw4XSxbMyw5XSxbOSwxMF1d&macro_url=https%3A%2F%2Fraw.githubusercontent.com%2FdFoiler%2Fnotes%2Fmaster%2Fnir.tex
	\[\begin{tikzcd}
		\cdots & {H_n(A)} & {H_n(X)} & {H_n(X,A)} & {H_{n-1}(A)} & \cdots \\
		\cdots & {H_n(B)} & {H_n(Y)} & {H_n(Y,B)} & {H_{n-1}(B)} & \cdots
		\arrow[from=1-1, to=1-2]
		\arrow[from=1-2, to=1-3]
		\arrow[from=1-3, to=1-4]
		\arrow[from=1-4, to=1-5]
		\arrow[from=1-5, to=1-6]
		\arrow[from=2-1, to=2-2]
		\arrow[from=2-2, to=2-3]
		\arrow[from=2-3, to=2-4]
		\arrow[from=2-4, to=2-5]
		\arrow[from=1-2, to=2-2]
		\arrow[from=1-3, to=2-3]
		\arrow[from=1-4, to=2-4]
		\arrow[from=1-5, to=2-5]
		\arrow[from=2-5, to=2-6]
	\end{tikzcd}\]
\end{proposition}
\begin{proof}
	Commutativity of all squares not involving the boundary map is automatic because $H_n$ is a functor. Anyway, the point is that we actually have a homomorphism of short exact sequences of chain complexes as follows.
	% https://q.uiver.app/#q=WzAsMTAsWzAsMCwiMCJdLFsxLDAsIkNfXFxidWxsZXQoQSkiXSxbMiwwLCJDX1xcYnVsbGV0KFgpIl0sWzMsMCwiQ19cXGJ1bGxldChYKS9DX1xcYnVsbGV0KEEpIl0sWzQsMCwiMCJdLFswLDEsIjAiXSxbMSwxLCJDX1xcYnVsbGV0KEIpIl0sWzIsMSwiQ19cXGJ1bGxldChZKSJdLFszLDEsIkNfXFxidWxsZXQoWSkvQ19cXGJ1bGxldChCKSJdLFs0LDEsIjAiXSxbMCwxXSxbMSwyXSxbMiwzXSxbMyw0XSxbNSw2XSxbNiw3XSxbNyw4XSxbOCw5XSxbMSw2XSxbMiw3XSxbMyw4XV0=&macro_url=https%3A%2F%2Fraw.githubusercontent.com%2FdFoiler%2Fnotes%2Fmaster%2Fnir.tex
	\[\begin{tikzcd}
		0 & {C_\bullet(A)} & {C_\bullet(X)} & {C_\bullet(X)/C_\bullet(A)} & 0 \\
		0 & {C_\bullet(B)} & {C_\bullet(Y)} & {C_\bullet(Y)/C_\bullet(B)} & 0
		\arrow[from=1-1, to=1-2]
		\arrow[from=1-2, to=1-3]
		\arrow[from=1-3, to=1-4]
		\arrow[from=1-4, to=1-5]
		\arrow[from=2-1, to=2-2]
		\arrow[from=2-2, to=2-3]
		\arrow[from=2-3, to=2-4]
		\arrow[from=2-4, to=2-5]
		\arrow[from=1-2, to=2-2]
		\arrow[from=1-3, to=2-3]
		\arrow[from=1-4, to=2-4]
	\end{tikzcd}\]
	One sees that this diagram commutes for any given $n$ because the left square commutes by functoriality of $C_\bullet$, and the right morphism is simply taking the cokernel. So the result will now follow from the following piece of homological algebra.
\end{proof}
\begin{proposition}
	Fix a morphism of short exact sequences of chain complexes
	% https://q.uiver.app/#q=WzAsMTAsWzAsMCwiMCJdLFsxLDAsIlxcbWF0aGNhbCBBX1xcYnVsbGV0JyJdLFsyLDAsIlxcbWF0aGNhbCBBX1xcYnVsbGV0Il0sWzMsMCwiXFxtYXRoY2FsIEFfXFxidWxsZXQnJyJdLFs0LDAsIjAiXSxbMCwxLCIwIl0sWzEsMSwiXFxtYXRoY2FsIEJfXFxidWxsZXQnIl0sWzIsMSwiXFxtYXRoY2FsIEJfXFxidWxsZXQiXSxbMywxLCJcXG1hdGhjYWwgQl9cXGJ1bGxldCcnIl0sWzQsMSwiMCJdLFswLDFdLFsxLDJdLFsyLDNdLFszLDRdLFs1LDZdLFs2LDddLFs3LDhdLFs4LDldLFsxLDZdLFsyLDddLFszLDhdXQ==&macro_url=https%3A%2F%2Fraw.githubusercontent.com%2FdFoiler%2Fnotes%2Fmaster%2Fnir.tex
	\[\begin{tikzcd}
		0 & {\mathcal A_\bullet'} & {\mathcal A_\bullet} & {\mathcal A_\bullet''} & 0 \\
		0 & {\mathcal B_\bullet'} & {\mathcal B_\bullet} & {\mathcal B_\bullet''} & 0
		\arrow[from=1-1, to=1-2]
		\arrow[from=1-2, to=1-3]
		\arrow[from=1-3, to=1-4]
		\arrow[from=1-4, to=1-5]
		\arrow[from=2-1, to=2-2]
		\arrow[from=2-2, to=2-3]
		\arrow[from=2-3, to=2-4]
		\arrow[from=2-4, to=2-5]
		\arrow[from=1-2, to=2-2]
		\arrow[from=1-3, to=2-3]
		\arrow[from=1-4, to=2-4]
	\end{tikzcd}\]
	Then there is a morphism of induced long exact sequences as follows.
	% https://q.uiver.app/#q=WzAsMTIsWzEsMCwiSF9uKFxcbWF0aGNhbCBBJ19cXGJ1bGxldCkiXSxbMCwwLCJcXGNkb3RzIl0sWzIsMCwiSF9uKFxcbWF0aGNhbCBBX1xcYnVsbGV0KSJdLFszLDAsIkhfbihcXG1hdGhjYWwgQV9cXGJ1bGxldCcnKSJdLFs0LDAsIkhfe24tMX0oXFxtYXRoY2FsIEFfXFxidWxsZXQnKSJdLFs1LDAsIlxcY2RvdHMiXSxbMCwxLCJcXGNkb3RzIl0sWzEsMSwiSF9uKFxcbWF0aGNhbCBCJ19cXGJ1bGxldCkiXSxbMiwxLCJIX24oXFxtYXRoY2FsIEJfXFxidWxsZXQpIl0sWzMsMSwiSF9uKFxcbWF0aGNhbCBCJydfXFxidWxsZXQpIl0sWzQsMSwiSF97bi0xfShcXG1hdGhjYWwgQidfXFxidWxsZXQpIl0sWzUsMSwiXFxjZG90cyJdLFsxLDBdLFswLDJdLFsyLDNdLFszLDRdLFs0LDVdLFs2LDddLFs3LDhdLFs4LDldLFs5LDEwXSxbMTAsMTFdLFswLDddLFsyLDhdLFszLDldLFs0LDEwXV0=&macro_url=https%3A%2F%2Fraw.githubusercontent.com%2FdFoiler%2Fnotes%2Fmaster%2Fnir.tex
	\[\begin{tikzcd}
		\cdots & {H_n(\mathcal A'_\bullet)} & {H_n(\mathcal A_\bullet)} & {H_n(\mathcal A_\bullet'')} & {H_{n-1}(\mathcal A_\bullet')} & \cdots \\
		\cdots & {H_n(\mathcal B'_\bullet)} & {H_n(\mathcal B_\bullet)} & {H_n(\mathcal B''_\bullet)} & {H_{n-1}(\mathcal B'_\bullet)} & \cdots
		\arrow[from=1-1, to=1-2]
		\arrow[from=1-2, to=1-3]
		\arrow[from=1-3, to=1-4]
		\arrow[from=1-4, to=1-5]
		\arrow[from=1-5, to=1-6]
		\arrow[from=2-1, to=2-2]
		\arrow[from=2-2, to=2-3]
		\arrow[from=2-3, to=2-4]
		\arrow[from=2-4, to=2-5]
		\arrow[from=2-5, to=2-6]
		\arrow[from=1-2, to=2-2]
		\arrow[from=1-3, to=2-3]
		\arrow[from=1-4, to=2-4]
		\arrow[from=1-5, to=2-5]
	\end{tikzcd}\]
\end{proposition}
\begin{proof}
	Again, the commutativity of any square not involving the boundary is automatic. So it remains to check commutativity of the boundary square
	% https://q.uiver.app/#q=WzAsNCxbMCwwLCJIX24oXFxtYXRoY2FsIEFfXFxidWxsZXQnJykiXSxbMSwwLCJIX3tuLTF9KFxcbWF0aGNhbCBBX1xcYnVsbGV0JykiXSxbMCwxLCJIX24oXFxtYXRoY2FsIEInJ19cXGJ1bGxldCkiXSxbMSwxLCJIX3tuLTF9KFxcbWF0aGNhbCBCJ19cXGJ1bGxldCkiXSxbMCwxXSxbMiwzXSxbMCwyXSxbMSwzXV0=&macro_url=https%3A%2F%2Fraw.githubusercontent.com%2FdFoiler%2Fnotes%2Fmaster%2Fnir.tex
	\[\begin{tikzcd}
		{H_n(\mathcal A_\bullet'')} & {H_{n-1}(\mathcal A_\bullet')} \\
		{H_n(\mathcal B''_\bullet)} & {H_{n-1}(\mathcal B'_\bullet)}
		\arrow[from=1-1, to=1-2]
		\arrow[from=2-1, to=2-2]
		\arrow[from=1-1, to=2-1]
		\arrow[from=1-2, to=2-2]
	\end{tikzcd}\]
	which can be done directly. Well, choose $[\alpha'']\in H_n(\mathcal A_\bullet'')$ where $\alpha\in\mathcal A''_n$, and we track it through the diagram.
	\begin{itemize}
		\item Along the top, we pull $\alpha''$ back to some $\alpha\in\mathcal A_n$, take boundary down to $\del\alpha\in\mathcal A_{n-1}$, and then we find $\alpha'\in\mathcal A_{n-1}'$ such that $\alpha'\mapsto\del\alpha$. This is then passed through the map $\mathcal A_{n-1}'\to\mathcal B_{n-1}'$.
		\item Along the bottom, we push $\alpha''$ to some $\beta''\in\mathcal B_n''$. Now we compute the boundary. We need to pull $\beta''$ back to some $\beta\in\mathcal B_n$, but we might as well use the image of $\alpha\in\mathcal A_n$. Then we take boundary down to $\del\beta$, which we might as well take as the image of $\del\alpha$. Then we find $\beta'\in\mathcal A_{n-1}'$ such that $\beta'\mapsto\del\beta$, but again, we may as well take the image of $\alpha'$.
	\end{itemize}
	The above computation completes the proof.
\end{proof}
This sort of naturality allows us to derive an equivalence between simplicial and singular homology; as a corollary, this will imply that the simplicial homology is invariant under the chosen $\Delta$-complex structure. We will purely formally use the axioms we have built.
\begin{proposition}
	Fix a $\Delta$-complex $X$ a subcomplex $A\subseteq X$. Then $(X,A)$ is a good pair, and there is an isomorphism
	\[H_n^\Delta(X,A)\to H_n(X,A).\]
\end{proposition}
\begin{proof}
	Checking that $(X,A)$ is a good pair follows from the case of a CW-complex, which can be checked by manually finding the needed open neighborhood of all the cells. We now proceed in many steps.
	\begin{enumerate}
		\item To begin, note that there is at least an embedding $C^\Delta_\bullet(X,A)\to C_\bullet(X,A)$ always. Our goal is to show that the induced map on homology is an isomorphism.
		\item Take $A=\emp$ and $X$ is a point. Then we manually computed both sides are isomorphic to $\ZZ$ at degree $0$ and no nonzero higher homology.
		\item Take $A=\emp$ and $X$ is some set of points. Then we take disjoint unions (which cohere for both of our homology theories) to conclude.
		\item Take $A=\emp$ and $X$ a finite-dimensional $\Delta$-complex. Let's say $X$ is $k$-dimensional so that $X=X^{(k)}$. Then we use the previous piece of homological algebra to produce a morphism of long exact sequences as follows.
		% https://q.uiver.app/#q=WzAsMTIsWzAsMCwiXFxjZG90cyJdLFsxLDAsIkhfbl5cXERlbHRhXFxsZWZ0KFheeyhrLTEpfVxccmlnaHQpIl0sWzEsMSwiSF9uXFxsZWZ0KFheeyhrKX0sWF57KGstMSl9XFxyaWdodCkiXSxbMCwxLCJcXGNkb3RzIl0sWzIsMCwiSF9uXlxcRGVsdGFcXGxlZnQoWF57KGspfVxccmlnaHQpIl0sWzMsMCwiSF9uXlxcRGVsdGFcXGxlZnQoWF57KGspfSxYXnsoay0xKX1cXHJpZ2h0KSJdLFs0LDAsIkhfe24tMX1eXFxEZWx0YVxcbGVmdChYXnsoay0xKX1cXHJpZ2h0KSJdLFsyLDEsIkhfblxcbGVmdChYXnsoayl9XFxyaWdodCkiXSxbMywxLCJIX25cXGxlZnQoWF57KGspfSxYXnsoay0xKX1cXHJpZ2h0KSJdLFs0LDEsIkhfe24tMX1cXGxlZnQoWF57KGstMSl9XFxyaWdodCkiXSxbNSwwLCJcXGNkb3RzIl0sWzUsMSwiXFxjZG90cyJdLFswLDFdLFsxLDRdLFs0LDVdLFs1LDZdLFs2LDEwXSxbMywyXSxbMiw3XSxbNyw4XSxbOCw5XSxbOSwxMV0sWzYsOV0sWzUsOF0sWzQsN10sWzEsMl1d&macro_url=https%3A%2F%2Fraw.githubusercontent.com%2FdFoiler%2Fnotes%2Fmaster%2Fnir.tex
		\[\begin{tikzcd}
			\cdots & {H_n^\Delta\left(X^{(k-1)}\right)} & {H_n^\Delta\left(X^{(k)}\right)} & {H_n^\Delta\left(X^{(k)},X^{(k-1)}\right)} & {H_{n-1}^\Delta\left(X^{(k-1)}\right)} & \cdots \\
			\cdots & {H_n\left(X^{(k)},X^{(k-1)}\right)} & {H_n\left(X^{(k)}\right)} & {H_n\left(X^{(k)},X^{(k-1)}\right)} & {H_{n-1}\left(X^{(k-1)}\right)} & \cdots
			\arrow[from=1-1, to=1-2]
			\arrow[from=1-2, to=1-3]
			\arrow[from=1-3, to=1-4]
			\arrow[from=1-4, to=1-5]
			\arrow[from=1-5, to=1-6]
			\arrow[from=2-1, to=2-2]
			\arrow[from=2-2, to=2-3]
			\arrow[from=2-3, to=2-4]
			\arrow[from=2-4, to=2-5]
			\arrow[from=2-5, to=2-6]
			\arrow[from=1-5, to=2-5]
			\arrow[from=1-4, to=2-4]
			\arrow[from=1-3, to=2-3]
			\arrow[from=1-2, to=2-2]
		\end{tikzcd}\]
		By induction, the leftmost and rightmost arrows are isomorphisms. Now, we show that the right middle morphism is an isomorphism by hand, which forces the remaining map to be an isomorphism by the Five lemma (see \Cref{prop:five-lemma} below). Well, note that $\Delta_n(X^{(k)},X^{(k-1)})$ is zero for $n\ne k$ and free abelian group with basis given by the $k$-simplices for $k=n$. (For $n<k$, everything is in $X^{(k-1)}$, and for $n>k$, there is nothing there to begin with.) As such, the same will be true for $H_n^\Delta(X^{(k)},X^{(k-1)})$. On the other hand, consider the maps
		\[\frac{\bigsqcup_\alpha\Delta^k_\alpha}{\bigsqcup_\alpha\del\Delta^k_\alpha}\to\frac{X^{(k)}}{X^{(k-1)}}\]
		which is a homeomorphism and thus an isomorphism on singular homology. So our singular homology is again we are again zero for $n\ne k$ and when $n=k$ we have the same presentation as before via a computation of $H_n(\Delta^k,\del\Delta^k)$ done in \Cref{ex:rel-homology-sn}.
		\item Let $A$ be empty and $X$ be an infinite-dimensional complex. Then we note $H_n^\Delta(X^{(n+1)})=H_n^\Delta(X)$ because all the relevant $\Delta$-complexes for $H_n^\Delta(X)$ will come from $X^{(n+1)}$. So by the previous step, this is $H_n(X^{(n+1)})$. For the other side,
		\[\colimit H_n(X^{(k)})=H_n(X)\]
		because the computation of $H_n(X)$ can only ever use finitely many simplices from $X^k$. (The map is also injective because anything $[\alpha]\in H_n(X^{(k)})$ landing in the trivial class of $H_n(X)$ will be the boundary of some chain, but then this chain can be witnesses again by some $X^{(\ell)}$ for perhaps different but still finite $\ell$.) This colimit completes our argument because $H_n(X^{(k)})$ has been dealt with in the finite case.
		\item For $A$ nonempty, we simply use the induced morphism of long exact sequences given as follows.
		% https://q.uiver.app/#q=WzAsMTQsWzAsMCwiXFxjZG90cyJdLFszLDAsIkhfbl5cXERlbHRhKFgsQSkiXSxbMywxLCJIX24oWCxBKSJdLFsxLDAsIkhfbl5cXERlbHRhKEEpIl0sWzIsMCwiSF9uXlxcRGVsdGEoWCkiXSxbNCwwLCJIX3tuLTF9XlxcRGVsdGEoQSkiXSxbNSwwLCJIX3tuLTF9XlxcRGVsdGEoWCkiXSxbMSwxLCJIX24oQSkiXSxbMiwxLCJIX24oWCkiXSxbNCwxLCJIX3tuLTF9KEEpIl0sWzUsMSwiSF97bi0xfShYKSJdLFswLDEsIlxcY2RvdHMiXSxbNiwwLCJcXGNkb3RzIl0sWzYsMSwiXFxjZG90cyJdLFswLDNdLFszLDRdLFs0LDFdLFsxLDVdLFs1LDZdLFs2LDEyXSxbMTEsN10sWzcsOF0sWzgsMl0sWzIsOV0sWzksMTBdLFsxMCwxM10sWzMsN10sWzQsOF0sWzEsMl0sWzUsOV0sWzYsMTBdXQ==&macro_url=https%3A%2F%2Fraw.githubusercontent.com%2FdFoiler%2Fnotes%2Fmaster%2Fnir.tex
		\[\begin{tikzcd}
			\cdots & {H_n^\Delta(A)} & {H_n^\Delta(X)} & {H_n^\Delta(X,A)} & {H_{n-1}^\Delta(A)} & {H_{n-1}^\Delta(X)} & \cdots \\
			\cdots & {H_n(A)} & {H_n(X)} & {H_n(X,A)} & {H_{n-1}(A)} & {H_{n-1}(X)} & \cdots
			\arrow[from=1-1, to=1-2]
			\arrow[from=1-2, to=1-3]
			\arrow[from=1-3, to=1-4]
			\arrow[from=1-4, to=1-5]
			\arrow[from=1-5, to=1-6]
			\arrow[from=1-6, to=1-7]
			\arrow[from=2-1, to=2-2]
			\arrow[from=2-2, to=2-3]
			\arrow[from=2-3, to=2-4]
			\arrow[from=2-4, to=2-5]
			\arrow[from=2-5, to=2-6]
			\arrow[from=2-6, to=2-7]
			\arrow[from=1-2, to=2-2]
			\arrow[from=1-3, to=2-3]
			\arrow[from=1-4, to=2-4]
			\arrow[from=1-5, to=2-5]
			\arrow[from=1-6, to=2-6]
		\end{tikzcd}\]
		Everything but the middle morphism is an isomorphism by the previous steps, so we complete by the Five lemma again (see \Cref{prop:five-lemma}).
	\end{enumerate}
\end{proof}
Professor Agol then proceeded to prove the five lemma. I have copy-pasted a proof using the Snake lemma from a previous homework below.
\begin{proposition} \label{prop:five-lemma}
	Consider a commutative diagram of $R$-modules and homomorphisms such that each row is exact.
	% https://q.uiver.app/?q=WzAsMTAsWzAsMCwiTV8xIl0sWzEsMCwiTV8yIl0sWzIsMCwiTV8zIl0sWzMsMCwiTV80Il0sWzQsMCwiTV81Il0sWzAsMSwiTl8xIl0sWzEsMSwiTl8yIl0sWzIsMSwiTl8zIl0sWzMsMSwiTl80Il0sWzQsMSwiTl81Il0sWzAsMV0sWzEsMl0sWzIsM10sWzMsNF0sWzUsNl0sWzYsN10sWzcsOF0sWzgsOV0sWzAsNSwiZl8xIiwyXSxbMSw2LCJmXzIiLDJdLFsyLDcsImZfMyIsMl0sWzMsOCwiZl80IiwyXSxbNCw5LCJmXzUiLDJdXQ==
	\[\begin{tikzcd}
		{M_1} & {M_2} & {M_3} & {M_4} & {M_5} \\
		{N_1} & {N_2} & {N_3} & {N_4} & {N_5}
		\arrow[from=1-1, to=1-2]
		\arrow[from=1-2, to=1-3]
		\arrow[from=1-3, to=1-4]
		\arrow[from=1-4, to=1-5]
		\arrow[from=2-1, to=2-2]
		\arrow[from=2-2, to=2-3]
		\arrow[from=2-3, to=2-4]
		\arrow[from=2-4, to=2-5]
		\arrow["{f_1}"', from=1-1, to=2-1]
		\arrow["{f_2}"', from=1-2, to=2-2]
		\arrow["{f_3}"', from=1-3, to=2-3]
		\arrow["{f_4}"', from=1-4, to=2-4]
		\arrow["{f_5}"', from=1-5, to=2-5]
	\end{tikzcd}\]
	\begin{enumerate}[label=(\alph*)]
		\item If $f_1$ is surjective and $f_2,f_4$ are monomorphisms, then $f_3$ is a monomorphism.
		\item If $f_5$ is a monomorphism and $f_2,f_4$ are surjective, then $f_3$ is surjective.
	\end{enumerate}
\end{proposition}
\begin{proof}
	Label the diagram as follows.
	% https://q.uiver.app/?q=WzAsMTAsWzAsMCwiTV8xIl0sWzEsMCwiTV8yIl0sWzIsMCwiTV8zIl0sWzMsMCwiTV80Il0sWzQsMCwiTV81Il0sWzAsMSwiTl8xIl0sWzEsMSwiTl8yIl0sWzIsMSwiTl8zIl0sWzMsMSwiTl80Il0sWzQsMSwiTl81Il0sWzAsMSwiYV8xIl0sWzEsMiwiYV8yIl0sWzIsMywiYV8zIl0sWzMsNCwiYV80Il0sWzUsNiwiYl8xIiwyXSxbNiw3LCJiXzIiLDJdLFs3LDgsImJfMyIsMl0sWzgsOSwiYl80IiwyXSxbMCw1LCJmXzEiLDJdLFsxLDYsImZfMiIsMl0sWzIsNywiZl8zIiwyXSxbMyw4LCJmXzQiLDJdLFs0LDksImZfNSIsMl1d
	\[\begin{tikzcd}
		{M_1} & {M_2} & {M_3} & {M_4} & {M_5} \\
		{N_1} & {N_2} & {N_3} & {N_4} & {N_5}
		\arrow["{a_1}", from=1-1, to=1-2]
		\arrow["{a_2}", from=1-2, to=1-3]
		\arrow["{a_3}", from=1-3, to=1-4]
		\arrow["{a_4}", from=1-4, to=1-5]
		\arrow["{b_1}"', from=2-1, to=2-2]
		\arrow["{b_2}"', from=2-2, to=2-3]
		\arrow["{b_3}"', from=2-3, to=2-4]
		\arrow["{b_4}"', from=2-4, to=2-5]
		\arrow["{f_1}"', from=1-1, to=2-1]
		\arrow["{f_2}"', from=1-2, to=2-2]
		\arrow["{f_3}"', from=1-3, to=2-3]
		\arrow["{f_4}"', from=1-4, to=2-4]
		\arrow["{f_5}"', from=1-5, to=2-5]
	\end{tikzcd}\]
	Very quickly, we claim that we can induce the following diagram with exact rows.
	% https://q.uiver.app/?q=WzAsOCxbMSwwLCJNXzIiXSxbMiwwLCJNXzMiXSxbMywwLCJhXzNNXzMiXSxbNCwwLCIwIl0sWzAsMSwiMCJdLFsxLDEsIk5fMi9iXzFOXzEiXSxbMiwxLCJOXzMiXSxbMywxLCJOXzQiXSxbMCwxLCJhXzIiXSxbMSwyLCJhXzMiXSxbMiwzXSxbNCw1XSxbNSw2LCJiXzIiLDJdLFs2LDcsImJfMyIsMl0sWzAsNSwiXFxvdmVybGluZXtmXzJ9IiwyXSxbMSw2LCJmXzMiLDJdLFsyLDcsIlxcb3ZlcmxpbmV7Zl80fSIsMl1d
	\[\begin{tikzcd}
		& {M_2} & {M_3} & {a_3M_3} & 0 \\
		0 & {N_2/b_1N_1} & {N_3} & {N_4}
		\arrow["{a_2}", from=1-2, to=1-3]
		\arrow["{a_3}", from=1-3, to=1-4]
		\arrow[from=1-4, to=1-5]
		\arrow[from=2-1, to=2-2]
		\arrow["{b_2}"', from=2-2, to=2-3]
		\arrow["{b_3}"', from=2-3, to=2-4]
		\arrow["{\overline{f_2}}"', from=1-2, to=2-2]
		\arrow["{f_3}"', from=1-3, to=2-3]
		\arrow["{\overline{f_4}}"', from=1-4, to=2-4]
	\end{tikzcd}\]
	Here, $\overline{f_2}$ is induced as the composite of $M_2\stackrel{f_2}{\to}N_2\onto N_2/b_1N_1$; and $\overline{f_2}$ is induced as the restriction of $M_4\stackrel{f_4}\to N_4$ to $a_3M_3.$ We also note that $a_3:M_3\to a_3M_3$ is well-defined because $a_3$ outputs into its image; $b_2:N_2/b_1N_1\to N_3$ is well-defined because $b_1N_1=\im b_1\subseteq\ker b_2$ by the exactness of the original diagram.

	We now check the exactness of the rows.
	\begin{itemize}
		\item Exact at $M_3$: we still have $\im a_2=\ker a_3$ by exactness of the original diagram.
		\item Exact at $a_3M_3$: we note that $a_3:M_3\to a_3M_3$ is surjective by definition of $a_3M_3.$
		\item Exact at $N_2/b_1N_1$: we note that $\ker b_2=\im b_1$ by exactness of the original diagram, so $b_2:N_2/\im b_1\to N_3$ has trivial kernel.
		\item Exact at $N_3$: we still have $\im b_2=\ker b_3$ by exactness of the original diagram.
	\end{itemize}
	We now attack the parts of the problem individually.
	\begin{enumerate}[label=(\alph*)]
		\item The trick is to claim that we have the following commutative diagram with exact rows, where $\widetilde{f_2}$ and $\overline f_4$ are monic.
		% https://q.uiver.app/?q=WzAsOCxbMSwwLCJNXzIvYV8xTV8xIl0sWzIsMCwiTV8zIl0sWzMsMCwiYV8zTV8zIl0sWzQsMCwiMCJdLFswLDEsIjAiXSxbMSwxLCJOXzIvYl8xTl8xIl0sWzIsMSwiTl8zIl0sWzMsMSwiTl80Il0sWzAsMSwiYV8yIl0sWzEsMiwiYV8zIl0sWzIsM10sWzQsNV0sWzUsNiwiYl8yIiwyXSxbNiw3LCJiXzMiLDJdLFswLDUsIlxcb3ZlcmxpbmV7Zl8yfSIsMl0sWzEsNiwiZl8zIiwyXSxbMiw3LCJcXG92ZXJsaW5le2ZfNH0iLDJdXQ==
		\[\begin{tikzcd}
			& {M_2/a_1M_1} & {M_3} & {a_3M_3} & 0 \\
			0 & {N_2/b_1N_1} & {N_3} & {N_4}
			\arrow["{a_2}", from=1-2, to=1-3]
			\arrow["{a_3}", from=1-3, to=1-4]
			\arrow[from=1-4, to=1-5]
			\arrow[from=2-1, to=2-2]
			\arrow["{b_2}"', from=2-2, to=2-3]
			\arrow["{b_3}"', from=2-3, to=2-4]
			\arrow["{\widetilde{f_2}}"', from=1-2, to=2-2]
			\arrow["{f_3}"', from=1-3, to=2-3]
			\arrow["{\overline{f_4}}"', from=1-4, to=2-4]
		\end{tikzcd}\]
		We start by showing that the map $\widetilde{f_2}:M_2/a_1M_1\to N_2/b_1N_1$ is actually well-defined with trivial kernel. It suffices to show that the composite $M_2\stackrel{f_2}\to N_2\onto N_2/b_1N_1$ has kernel $a_1M_1.$

		Well, $\alpha$ lives in the kernel of the composite if and only if $f_2\alpha\in b_1N_1$ if and only if $f_2\alpha\in b_1(f_1M_1)$ (because $f_1$ is surjective) if and only if $f_2\alpha\in\im (b_1\circ f_1)$ if and only if $f_2\alpha\in\im (f_2\circ a_1)$ (by commutativity) if and only if $f_2\alpha\in f_2(\im a_1)$ if and only if $\alpha\in\im a_1$ (because $f_2$ is monic and hence injective). So indeed,
		\[\ker(M_2\to N_2\onto N_2/b_1N_1)=\im(M_1\to M_2),\]
		which is what we needed to show that $M_2/a_1M_1\into N_2/b_1N_1$ is well-defined and monic.

		We now note that the rows of the diagram are exact. The only modified point here is exactness at $M_3,$ which now must accommodate for $M_2/a_1M_1\to M_3.$ This map is well-defined because $\im a_1\subseteq\ker a_2$ by exactness of the original diagram, and we are exact at $M_3$ because
		\[\ker(M_3\to a_3M_3)=\im(M_2\to M_3)=\im(M_2/a_1M_1\to M_3)\]
		because modding in the domain does not alter the image.

		To finish, we note that $\widetilde{f_2}$ and $\overline{f_4}$ being monic imply that $f_3$ is monic by Lang III.14 part (a).
		% because it is the restriction of the monic function $f_4:M_4\to N_4.$ Anyways, the Snake lemma now gives us the exact sequence
		% \[\ker\widetilde{f_2}\to\ker f_3\to\ker\overline{f_4},\]
		% but the left and right are $0$ because $\widetilde{f_2}$ and $\overline{f_4}$ are monic by previous discussion. So $\ker f_3=0$ is forced by \autoref{lem:zeroseq}, making $f_3$ monic.

		\item Similarly, the trick is to claim that we have the following commutative diagram with exact rows, where $\overline{f_2}$ and $\widetilde{f_4}$ are surjective.
		% https://q.uiver.app/?q=WzAsOCxbMSwwLCJNXzIiXSxbMiwwLCJNXzMiXSxbMywwLCJhXzNNXzMiXSxbNCwwLCIwIl0sWzAsMSwiMCJdLFsxLDEsIk5fMi9iXzFOXzEiXSxbMiwxLCJOXzMiXSxbMywxLCJiXzNOXzMiXSxbMCwxLCJhXzIiXSxbMSwyLCJhXzMiXSxbMiwzXSxbNCw1XSxbNSw2LCJiXzIiLDJdLFs2LDcsImJfMyIsMl0sWzAsNSwiXFxvdmVybGluZXtmXzJ9IiwyXSxbMSw2LCJmXzMiLDJdLFsyLDcsIlxcd2lkZXRpbGRle2ZfNH0iLDJdXQ==
		\[\begin{tikzcd}
			& {M_2} & {M_3} & {a_3M_3} & 0 \\
			0 & {N_2/b_1N_1} & {N_3} & {b_3N_3}
			\arrow["{a_2}", from=1-2, to=1-3]
			\arrow["{a_3}", from=1-3, to=1-4]
			\arrow[from=1-4, to=1-5]
			\arrow[from=2-1, to=2-2]
			\arrow["{b_2}"', from=2-2, to=2-3]
			\arrow["{b_3}"', from=2-3, to=2-4]
			\arrow["{\overline{f_2}}"', from=1-2, to=2-2]
			\arrow["{f_3}"', from=1-3, to=2-3]
			\arrow["{\widetilde{f_4}}"', from=1-4, to=2-4]
		\end{tikzcd}\]
		We quickly note that the map $N_3\to b_3N_3$ is well-defined because $b_3$ always outputs to $b_3N_3$ by definition. We also note that the top row is exact as checked earlier, and the only perturbation to the bottom row is exactness at $N_3,$ which holds because the kernel of $b_3$ has not changed and will still be $\im b_2.$

		Next we show that $\widetilde{f_4}$ is well-defined. For this, we need to show that the image of $f_4:M_4\to N_4$ under the restriction to $a_3M_3\to N_4$ will always output to $N_4.$ Well, we see
		\[f_4(\im a_3)=\im(f_4\circ a_3)=\im(b_3\circ f_3)\subseteq\im b_3,\]
		so we are indeed safe.

		We now note that $\overline f_2:M_2\to N_2/b_1N_1$ is surjective because it is the composite of the surjective maps $f_2:M_2\to N_2$ and $N_2\onto N_2/b_1N_1.$ (Any element of $N_2/b_1N_1$ can be pulled back to a representative in $N_2,$ which can then be pulled back along $f_2$ to a representative in $M_2.$)

		Further, we claim that $\widetilde{f_4}$ is surjective. Well, find any $\beta\in\im b_3$ that we want to hit. Because $f_4$ is surjective, there exists $\alpha\in M_4$ such that $f_4\alpha=\beta,$ and we will show that $\alpha\in a_3M_3,$ which will be enough.
		
		Indeed, $f_4\alpha\in\im b_3$ if and only if $f_4\alpha\in\ker b_4$ (by exactness) if and only if $\alpha\in\ker (b_4\circ f_4)$ if and only if $\alpha\in\ker (f_5\circ a_4)$ (by commutativity) if and only if $a_4\alpha\in\ker f_5$ if and only if $a_4\alpha=0$ ($f_5$ is monic) if and only if $\alpha\in\ker a_4$ if and only if $\alpha\in\im a_3$ (by exactness).

		In total, the fact that $\overline{f_2}$ and $\widetilde{f_4}$ are surjective implies that $f_3$ is surjective by Lang III.14 part (b).
		% the Snake lemma gives us the exact sequence
		% \[\coker\overline f_2\to\coker f_3\to\coker\widetilde{f_4},\]
		% but the left and right are $0$ because $\overline f_2$ and $\widetilde{f_4}$ are both surjective. Thus, $\coker f_3$ is zero by \autoref{lem:zeroseq}, so $f_3$ is surjective. (We note $\coker f_3=N_3/\im f_3,$ so $\coker f_3=0$ implies that $N_3=\im f_3.$)
		\qedhere
	\end{enumerate}
\end{proof}

\subsection{Degrees}
As an application, let's talk a bit about degrees. For example, any map $f\colon S^n\to S^n$ induces a map on homology $H_n(S^n)\to H_n(S^n)$. This is a map $\ZZ\to\ZZ$, so it will have to be multiplication by some integer $d$ (independent of the choice of isomorphism $H_n(S^n)\cong\ZZ$), which is called the degree of $f$.
\begin{example}
	The degree of $\id_{S^n}$ is $1$.
\end{example}
\begin{example}
	Suppose $f$ is not surjective. Then $\deg f=0$. The point is that $f$ lands in $S^n$ minus a point, which contracts to a point, so the image of $f$ factors through $H_n(S^n\setminus\{*\})=0$.
\end{example}
\begin{remark}
	Fix $f,g\colon S^n\to S^n$. If $f\sim g$, then $\deg f=\deg g$ because homotopic maps produce the same map on homology.
\end{remark}
\begin{remark}
	Fix $f,g\colon S^n\to S^n$. We have $\deg(f\circ g)=(\deg f)(\deg g)$ by tracking through the composite maps as $\ZZ\to\ZZ\to\ZZ$.
\end{remark}
\begin{example}
	If $f\colon S^n\to S^n$ is a continuous bijection, then it is a homeomorphism and so has an inverse map, so $\deg f$ must be a unit in $\ZZ$, so $\deg f\in\{\pm1\}$.
\end{example}
\begin{example}
	The degree of a reflection $f\colon S^n\to S^n$ is $-1$. Namely, let $\Delta_1^n$ denote the top hemisphere and $\Delta_2^n$ denote the bottom hemisphere, and we see that $f$ flips $\Delta_1^n$ and $\Delta_2^n$. Noting that $H_n(S^n)$ is generated by $\Delta_1^n-\Delta_2^n$ (one can track through the boundary maps to show this or see it directly on simplicial homology), the fact that $\deg f=-1$ follows.
\end{example}
\begin{example}
	The degree of the antipodal map $x\mapsto-x$ is $(-1)^{n+1}$ because it is a composite of $(n+1)$ reflections.
\end{example}
\begin{example}
	Suppose $f\colon S^n\to S^n$ has no fixed points. Then one can find a homotopy from $f$ to the antipodal map because the ``straight-line'' path from $f(x)$ to $-x$ fails to go through the origin.
\end{example}

\end{document}