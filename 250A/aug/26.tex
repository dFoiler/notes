








\subsection{Logistics}
Course website is \href{https://bcourses.berkeley.edu/courses/1504926}{\texttt{bcourses.berkeley.edu/courses/1504926}}.

\subsection{Group Talk}
Recall the definition.
\begin{defi}[Group, concrete]
    A group $G$ is the set of symmetries of something.
\end{defi}
``Something'' here is quite vague, but it's generally something like a graph or vector space or group.

Also, ``symmetry'' is quite vague, but it's somewhat intuitive: we are more or less asking for structure-preserving mappings. Namely, our ``something'' is a set $X,$ we are asking for our group $G$ to be a structure-preserving maps in $\op{Sym}(X).$ In practice, what ``structure-preserving'' means is clear.

There is also an abstract definition of groups.
\begin{defi}[Group, abstract]
    A group is a set $G$ with an operation $*:G\times G\to G$ which satisfies the following properties for any $a,b,c\in G.$
    \begin{itemize}
        \item Associative: $a*(b*c)=(a*b)*c.$
        \item Identity: there is an identity $e\in G$ such that $a*e=e*a=a.$
        \item Inverses: there is an $a^{-1}$ such that $a*a^{-1}=a^{-1}*a=e.$
    \end{itemize}
\end{defi}
We might ask how the abstract and concrete definitions interplay.

For example, suppose we have a concrete group $G.$ Then we can recover our abstract group by having the binary operation be composition. Association holds because function application is associative; $\id$ is our identity; inverses exists because symmetries are bijective.

In the other direction, it's less obvious how we take an abstract group to a concrete one.
\begin{ques}[Cayley] \label{ques:cayley}
    Given an abstract group $G,$ can we realize $G$ as the symmetries of some object $X$?
\end{ques}
To make this rigorous, we should talk about group actions so that ``some object'' can be rigorized.
\begin{defi}[Group action]
    Fix $G$ a(n abstract) group and $S$ a set. We say that $G$ (left) acts on a set $S$ if we have a map $\cdot:G\times S\to S$ which satisfies the following.
    \begin{itemize}
        \item Identity: $es=s$ for any $s\in S.$
        \item Associativity: $(gh)\cdot s=g\cdot(h\cdot s)$ for any $g,h\in G.$
    \end{itemize}
\end{defi}
Above we have technically defined a left group action; right group actions are defined analogously.

To answer \autoref{ques:cayley}, we let $G$ act on $S:=G$ (as the set) where the group action is defined by left multiplication. This is indeed an action, where the identity and associative laws follow from the corresponding laws in a group. This implies that we have a map
\[G\to\op{Sym}(G).\]
In fact, this is injective because its kernel is trivial: the only map taking $e\mapsto e$ is $e$ itself by the identity law.

How can we restrict $\op{Sym}(S)$ so that this is injection is also surjective? To add extra structure to $S,$ we equip $S$ with a right $G$-action.\footnote{For these keeping score, we now have three copies of $G$: we have $S=G,$ as well as actions of $G$ on $S$ on the left and right.} The key observation, now, is that the left action on $S$ by $G$ preserves the right action. Namely, if we have $g_\ell,g_r\in G$ and $s\in G,$ then
\[(g_\ell s)g_r=g_\ell(sg_r)\]
by associativity. In other words, we can multiply on the left or right in any order.
\begin{warn}
    We do not need to have that the left action preserves the left action. Namely, this is asking for $g\cdot (hs)=h\cdot (gs),$ which need not be true for non-abelian groups.
\end{warn}
So in fact we have the restriction that
\[G\hookrightarrow\op{Sym}_{G\text{-right}}(S).\]
We claim this is surjective. Indeed, suppose $\sigma:S\to S$ is a bijection such that $\sigma(sg)=(\sigma s)g$ for any $g\in G.$ Then we claim $\sigma$ is multiplication by $\sigma e\in G.$ Indeed,
\[\sigma s=\sigma(es)=(\sigma e)s,\]
which is what we wanted. So we have the following.
\begin{thm}[Cayley]
    Any abstract group $G$ is the group of symmetries of some mathematical object.
\end{thm}

\subsection{Representation Talk}
Perhaps we would like a more natural object than the group acting on literally itself. It turns out that we can also realize any group $G$ as the symmetry group of a graph. We'll do this for finite groups.
\begin{thm}
    Any finite group $G$ is the symmetry group of some finite graph.
\end{thm}
\begin{proof}
    We again do a little bit of cheating. We set our graph $S$ to have vertices labeled by $G.$ Next we color the edges of the graph according to the group action. Here is an example graph for $\ZZ/4\ZZ.$
    % https://q.uiver.app/?q=WzAsNCxbMiwwLCIxIl0sWzIsMiwiMiJdLFswLDIsIjMiXSxbMCwwLCIwIl0sWzMsMCwiMSIsMSx7ImN1cnZlIjotMSwiY29sb3VyIjpbMCw2MCw2MF19LFswLDYwLDYwLDFdXSxbMCwxLCIxIiwxLHsiY3VydmUiOi0xLCJjb2xvdXIiOlswLDYwLDYwXX0sWzAsNjAsNjAsMV1dLFsxLDIsIjEiLDEseyJjdXJ2ZSI6LTEsImNvbG91ciI6WzAsNjAsNjBdfSxbMCw2MCw2MCwxXV0sWzIsMywiMSIsMSx7ImN1cnZlIjotMSwiY29sb3VyIjpbMCw2MCw2MF19LFswLDYwLDYwLDFdXSxbMCwzLCIzIiwxLHsiY3VydmUiOi0xLCJjb2xvdXIiOlsxMjAsNjAsNjBdfSxbMTIwLDYwLDYwLDFdXSxbMSwwLCIzIiwxLHsiY3VydmUiOi0xLCJjb2xvdXIiOlsxMjAsNjAsNjBdfSxbMTIwLDYwLDYwLDFdXSxbMiwxLCIzIiwxLHsiY3VydmUiOi0xLCJjb2xvdXIiOlsxMjAsNjAsNjBdfSxbMTIwLDYwLDYwLDFdXSxbMywyLCIzIiwxLHsiY3VydmUiOi0xLCJjb2xvdXIiOlsxMjAsNjAsNjBdfSxbMTIwLDYwLDYwLDFdXSxbMywxLCIyIiwxLHsiY3VydmUiOjEsImNvbG91ciI6WzE4MCw2MCw2MF19LFsxODAsNjAsNjAsMV1dLFsxLDMsIjIiLDEseyJjdXJ2ZSI6MSwiY29sb3VyIjpbMTgwLDYwLDYwXX0sWzE4MCw2MCw2MCwxXV0sWzIsMCwiMiIsMSx7ImN1cnZlIjoxLCJjb2xvdXIiOlsxODAsNjAsNjBdfSxbMTgwLDYwLDYwLDFdXSxbMCwyLCIyIiwxLHsiY3VydmUiOjEsImNvbG91ciI6WzE4MCw2MCw2MF19LFsxODAsNjAsNjAsMV1dXQ==
    \[\begin{tikzcd}[row sep=huge,column sep=huge]
    	0 \arrow["0"{description}, loop, distance=2em, in=180, out=90] && 1 \arrow["0"{description}, loop, distance=2em, in=90, out=0] \\
    	\\
    	3 \arrow["0"{description}, loop, distance=2em, in=270, out=180] && 2 \arrow["0"{description}, loop, distance=2em, in=0, out=270]
    	\arrow["1"{description}, color={rgb,255:red,214;green,92;blue,92}, curve={height=-6pt}, from=1-1, to=1-3]
    	\arrow["1"{description}, color={rgb,255:red,214;green,92;blue,92}, curve={height=-6pt}, from=1-3, to=3-3]
    	\arrow["1"{description}, color={rgb,255:red,214;green,92;blue,92}, curve={height=-6pt}, from=3-3, to=3-1]
    	\arrow["1"{description}, color={rgb,255:red,214;green,92;blue,92}, curve={height=-6pt}, from=3-1, to=1-1]
    	\arrow["3"{description}, color={rgb,255:red,92;green,214;blue,92}, curve={height=-6pt}, from=1-3, to=1-1]
    	\arrow["3"{description}, color={rgb,255:red,92;green,214;blue,92}, curve={height=-6pt}, from=3-3, to=1-3]
    	\arrow["3"{description}, color={rgb,255:red,92;green,214;blue,92}, curve={height=-6pt}, from=3-1, to=3-3]
    	\arrow["3"{description}, color={rgb,255:red,92;green,214;blue,92}, curve={height=-6pt}, from=1-1, to=3-1]
    	\arrow["2"{description}, color={rgb,255:red,92;green,214;blue,214}, curve={height=6pt}, from=1-1, to=3-3]
    	\arrow["2"{description}, color={rgb,255:red,92;green,214;blue,214}, curve={height=6pt}, from=3-3, to=1-1]
    	\arrow["2"{description}, color={rgb,255:red,92;green,214;blue,214}, curve={height=6pt}, from=3-1, to=1-3]
    	\arrow["2"{description}, color={rgb,255:red,92;green,214;blue,214}, curve={height=6pt}, from=1-3, to=3-1]
    \end{tikzcd}\]
    So we have a colored graph, and we can check that the only symmetries of this graph corresponds to the action of $G$ itself: once we decide which vertex we should take $0$ to, the preservation of each colored arrow forces the other vertices. (For example, if we take $0$ to $2,$ there is only one red edge which was going out from $0,$ and there is only one edge currently going out from $2,$ so we have to send $1$ to $3$ as well. The other vertices are similar.)
    
    Now we would like to get rid of the colors and directions of the graph. For example, we might take a directed edge and add markers along the edge to ensure the symmetries are well-defined. For example, here is one way we could add markers to the graph of $\ZZ/4\ZZ.$
    \begin{center}
        \begin{asy}
            size(7cm);
            real R = 6;
            pair[] ps = {R*dir(45), R*dir(45+90), R*dir(45+180), R*dir(45+270)};
            for(int i = 0; i < 4; ++i)
            {
                    pair v11 = 3*dir(190-20+90*i);
                    pair v12 = dir(180+90*i);
                    pair vertical = dir(90*(i+1));
                    // + 1
                    draw(ps[i] -- ps[i]+v11 -- ps[i]+v11+v12 -- ps[(i+1)%4], red);
                    draw(ps[i]+v11 -- ps[i]+v11+vertical, red);
                    draw(ps[i]+v11+v12 -- ps[i]+v11+v12+vertical, red);
                    for(int j = 0; j <= 3; ++j)
                            dot(ps[i] + v11 + j*vertical/3);
                    dot(ps[i]+v11+v12); dot(ps[i]+v11+v12+vertical);
                    // + 3
                    draw(ps[(i+1)%4] -- ps[(i+1)%4]-v11 -- ps[(i+1)%4]-v11-v12 -- ps[i], green);
                    draw(ps[(i+1)%4]-v11 -- ps[(i+1)%4]-v11-vertical, green);
                    draw(ps[(i+1)%4]-v11-v12 -- ps[(i+1)%4]-v11-v12-vertical, green);
                    for(int j = 0; j <= 5; ++j)
                            dot(ps[(i+1)%4]-v11-j*vertical/5);
                    dot(ps[(i+1)%4]-v11-v12); dot(ps[(i+1)%4]-v11-v12-vertical);
                    // + 2
                    pair v21 = 3*dir(270-45-10+90*i);
                    pair v22 = dir(270-45+90*i);
                    pair oblique = dir(270-45+90*i-90);
                    draw(ps[i] -- ps[i]+v21 -- ps[i]+v21+v22 -- ps[(i+2)%4], blue);
                    draw(ps[i]+v21 -- ps[i]+v21+oblique, blue);
                    draw(ps[i]+v21+v22 -- ps[i]+v21+v22+oblique, blue);
                    for(int j = 0; j <= 4; ++j)
                            dot(ps[i]+v21+j*oblique/4);
                    dot(ps[i]+v21+v22); dot(ps[i]+v21+v22+oblique);
                    // + 0
                    pair v01 = dir(15+90*i);
                    pair v02 = dir(75+90*i);
                    oblique = dir(45+90*i);
                    draw(ps[i] -- ps[i]+v01 -- ps[i]+v02 -- cycle);
                    draw(ps[i]+v01 -- ps[i]+v01+oblique);
                    draw(ps[i]+v02 -- ps[i]+v02+oblique);
                    for(int j = 0; j <= 2; ++j)
                            dot(ps[i]+v02+j*oblique/2);
                    dot(ps[i]+v01); dot(ps[i]+v01+oblique);
            }
            for(int i = 0; i < 4; ++i)
            {
                    // draw point
                    fill(circle(ps[i],0.3), black);
            }
        \end{asy}
    \end{center}
    Here the colors are added for clarity though the graph is actually uncolored. The point is that we should need to send the big vertices to other big vertices and fake ``colored and directed'' edges to ones that match.
\end{proof}

These are called Cayley graphs. Let's do some examples now.
\begin{ex}
    Here is a rectangle.
    \begin{center}
        \begin{asy}
            size(4cm);
            draw((0,0)--(2,0)--(2,1)--(0,1)--cycle);
        \end{asy}
    
    \end{center}
    Our symmetries are do-nothing $e,$ flip vertically $v,$ flip horizontally $h,$ and rotate 180 degrees $r.$ Our Cayley graph looks like this.
    % https://q.uiver.app/?q=WzAsNCxbMCwwLCJcXGJ1bGxldCJdLFsyLDAsIlxcYnVsbGV0Il0sWzAsMSwiXFxidWxsZXQiXSxbMiwxLCJcXGJ1bGxldCJdLFswLDEsImgiLDAseyJzdHlsZSI6eyJ0YWlsIjp7Im5hbWUiOiJhcnJvd2hlYWQifX19XSxbMiwzLCJoIiwwLHsic3R5bGUiOnsidGFpbCI6eyJuYW1lIjoiYXJyb3doZWFkIn19fV0sWzEsMywidiIsMCx7InN0eWxlIjp7InRhaWwiOnsibmFtZSI6ImFycm93aGVhZCJ9fX1dLFswLDIsInYiLDAseyJzdHlsZSI6eyJ0YWlsIjp7Im5hbWUiOiJhcnJvd2hlYWQifX19XSxbMiwxLCIiLDEseyJzdHlsZSI6eyJ0YWlsIjp7Im5hbWUiOiJhcnJvd2hlYWQifX19XSxbMCwzLCJyIiwyLHsic3R5bGUiOnsidGFpbCI6eyJuYW1lIjoiYXJyb3doZWFkIn19fV1d
    \begin{center}
        \begin{tikzcd}
        	\bullet && \bullet \\
        	\bullet && \bullet
        	\arrow["h"{description}, tail reversed, from=1-1, to=1-3]
        	\arrow["h"{description}, tail reversed, from=2-1, to=2-3]
        	\arrow["v"{description}, tail reversed, from=1-3, to=2-3]
        	\arrow["v"{description}, tail reversed, from=1-1, to=2-1]
        	\arrow[tail reversed, from=2-1, to=1-3]
        	\arrow["r"{description}, tail reversed, from=1-1, to=2-3]
        \end{tikzcd}
    \end{center}
\end{ex}

To round things out, we note that group theory is roughly about the following.
\begin{itemize}
    \item We want to classify all groups. For example, what kinds of groups act on crystals?
    \item Given a group $G,$ we want to see what are the interesting things that groups act on. In general, these are permutation representations, but we are often just interested in linear representations acting on vector spaces. For example, there is some story here in physics.
    
    (We can also represent groups via their multiplication table. Professor Borcherds does not like these.)
\end{itemize}

\subsection{Maps of Groups}
Here is our motivating question.
\begin{ques}
    When are two groups the same?
\end{ques}
For example, we might have $G_1$ be the symmetry group of a rectangle and $G_2$ the set of elements $\langle a,b:a^2=b^2=ab^{-1}b^{-1}\rangle,$ then these are in fact the same: name $a$ the horizontal flip and $b$ the vertical flip. Then we see we have a bijective, structure-preserving map from $G_1\to G_2.$
\begin{defi}[Homomorphism]
    A map of groups $\varphi:G_1\to G_2$ is a \textit{homomorphism} if and only if
    \[\varphi(gh)=\varphi(g)\varphi(h)\]
    for any $g,h\in G_1.$ We can check this implies $\varphi(e_1)=e_2$ and $\varphi(g^{-1})=\varphi(g)^{-1}.$
\end{defi}
Then an isomorphism is a bijective homomorphism.
\begin{defi}
    An map of groups $\varphi:G_1\to G_2$ is an isomorphism if and only if it is a bijective homomorphism.
\end{defi}
Let's give some examples.
\begin{ex}
    Consider $\exp:(\RR,+)\to(\RR^\times,\times).$ This is a homomorphism because $\exp(a+b)=\exp(a)\times\exp(b).$ However, this is not an isomorphism because it does not hit negative elements.
\end{ex}
\begin{ex}
    Fix $G=\ZZ/4\ZZ$ and $H=(\ZZ/5\ZZ)^\times.$ Then we have the isomorphism by sending
    \[1\in G\longmapsto2\in H.\]
    In other words, $\varphi(k)=2^k.$ We can check this really is a bijection.
\end{ex}

\subsection{Lagrange's Theorem}
Let's list all groups.
\begin{enumerate}
    \item There is only one group of order $1$: it has to be trivial.
    \item There is only one group of order $2$: we need an identity and a non-identity element, which has to square to the identity.
    \item For order $3,$ we introduce Lagrange's theorem.
\end{enumerate}
\begin{thm}[Lagrange] \label{thm:lagrange}
    The order of an element $g$ of a group $G$ divides $\#G.$
\end{thm}
We won't prove this yet.
\begin{enumerate}
    \setcounter{enumi}2
    \item Now, with Lagrange's theorem, we note that a non-identity element needs to have order bigger than one but dividing $3$ and so must be three. So there is an element of order $3,$ so it must be cyclic.
\end{enumerate}
We remark that this same argument gives the following.
\begin{prop}
    Suppose $G$ is a group of prime order. Then $G\cong\ZZ/p\ZZ.$
\end{prop}
Let's prove Lagrange's theorem then.
\begin{proof}[Proof of \autoref{thm:lagrange}]
    The point is that the order of $g$ is the size of the subgroup $\langle g\rangle.$ So we show the more general statement as follows.
    \begin{thm}[Lagrange, II]
        Fix $H\subseteq G$ a subgroup of a group. Then $\#H\mid\#G.$
    \end{thm}
    \begin{proof}
        We need to study the geometric meaning of a subgroup $H.$ Well, suppose $G$ is the group of symmetries of some object $S$ and pick up a point $p\in S.$ Then we could set $\op{Stab}(s)$ to be the set of elements fixing $s\in S.$ For example, for an icosahedron, there is a $\ZZ/5\ZZ$ fixing a vertex, there is a $\ZZ/3\ZZ$ fixing a face, and so on.
        
        So we can realize subgroups as stabilizers of subsets. We would like the converse: given a subgroup $H,$ we would like a set $S$ with a $G$-action such that $H$ is the stabilizer of some subset of $S.$
        
        To make the problem easier, suppose that the $G$-action is transitive so that it lives in one orbit. Namely, fixing $s_0\in S,$ we have a function $G\to S$ by
        \[g\longmapsto gs_0.\]
        We would like for $gs_0=s_0$ to be equivalent to $g\in H$ for our particular subgroup $H.$ Quickly, we note that $gs_0=g's_0$ if and only if $g(g')^{-1}s_0=s_0$ if and only if $g(g')^{-1}\in H$ if and only if $g\in g'H$ if and only if $gH=g'H.$
        
        This suggests a construction of our set $S$ as $G/H,$ the set of \textit{cosets} $\{gH:g\in G\},$ or the equivalence classes as given above. We do have to check that $g\in g'H$ is an equivalence relation (say $\sim$), however. We will not be detailed about this.
        \begin{itemize}
            \item Note $g\sim g$ because $e\in H.$
            \item Note $g\sim g'$ implies $g'\sim g$ because $H$ has inverses.
            \item Note $g\sim g'$ and $g'\sim g''$ implies $g\sim g''$ because $H$ is associative.
        \end{itemize}
        \begin{remark}
            If we work with monoids, this is no longer an equivalence relation because of the lack of inverses.
        \end{remark}
        In fact, we can check that equivalence classes have the same size: if we have two cosets $g_1H$ and $g_2H,$ then we have a bijection $g_1H\to g_2H$ by $g_1h\mapsto g_2g_1^{-1}\cdot g_1h.$ (We will not check that this is bijective here, but it is at least injective, and it has an inverse, so it is.)
        
        So we have $G$ act on $G/H$ by left multiplication. Any two of these equivalence classes have the same size, so they all have size $\#(eH)=\#H,$ so we see that the order of $G$ is $\#H$ times the number of classes $\#(G/H)=:[G:H].$ So indeed, $\#H\mid\#G.$
    \end{proof}
    This completes the proof.
\end{proof}
We remark that we also have the following.
\begin{prop}
    If $G$ acts transitively on a set $S,$ then we see $\#S=\#G/\#\op{Stab}(s_0)$ for any chosen $s_0\in S,$
\end{prop}
\begin{proof}
    This follows from the above proof: consider the (surjective) map $G\twoheadrightarrow S$ defined by $g\mapsto gs_0.$ This is actually defined up to coset of $\op{Stab}(s_0)$ because we have that $g_1s_0=g_2s_0$ if and only if $g_2^{-1}g_1\in\op{Stab}(s_0)$ if and only if $g_1\op{Stab}(s_0)=g_2\op{Stab}(s_0).$ So we actually have a bijection $G/\op{Stab}(s_0)\twoheadrightarrow S,$ which is the result.
\end{proof}
\begin{ex}
    How many rotations of an icosahedron are there? Well, take $H$ to be the subgroup fixing a vertex. By spinning along a vertex, there are $5$ such rotations fixing the subgroup, and there are $12$ total vertices, so there are $60$ total rotations here.
\end{ex}
Let's see some other applications of Lagrange's theorem.
\begin{prop}[Fermat's little]
    Fix $x\in(\ZZ/p\ZZ)^\times.$ Then $x^{p-1}\equiv1.$
\end{prop}
\begin{proof}
    Well, the order of $(\ZZ/p\ZZ)^\times$ is $p-1,$ so the order of $x$ divides $p-1,$ from which the result follows.
\end{proof}
More generally, we have the following.
\begin{prop}[Euler's totient]
    Fix $x\in(\ZZ/m\ZZ)^\times.$ Then $x^{\varphi(m)}\equiv1.$
\end{prop}
\begin{proof}
    The point is that the order of $x$ has to divide the order of $\#(\ZZ/m\ZZ)^\times=\varphi(m).$ So the result follows.
\end{proof}