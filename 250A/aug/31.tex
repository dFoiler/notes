\documentclass[../notes.tex]{subfiles}

\begin{document}









\subsection{Groups of Order Four}
Let's continue our list of groups. Let's work with groups $G$ of orders $4.$ All elements must have order dividing $4.$
\begin{itemize}
	\item If there's an element of order $4,$ we are cyclic.
	\item Otherwise, all non-identity elements have order $2.$ Note that we know this group is abelian already! Indeed, for $a,b\in G,$ we see $(ab)^2=e$ implies that $abab=e,$ so 
	\[ab=aababb=ba.\]
	Note that this is very special for $4$; it is not the case that if all groups have order dividing $3.$
	
	Well, now that we are abelian, we see $G$ is a vector space over $\FF_2,$ which we can check by hand, and size reasons force us to have $G\cong(\FF_2)^2$ by choosing a suitable basis.
\end{itemize}
So we have the following.
\begin{thm}
	We have exactly two groups of order $4,$ up to isomorphism.
\end{thm}
\begin{proof}
	Above we showed that all groups of order $4$ are isomorphic to either $\ZZ/4\ZZ$ or $(\ZZ/2\ZZ)^2.$ Note that these are different because $\ZZ/4\ZZ$ has an element of order $4,$ though $(\ZZ/2\ZZ)^2$ does not.
\end{proof}

\subsection{Product Groups}
We remark that $\FF_2^2$ is an example of a product.
\begin{defi}[Product groups]
	Given two groups $G,H$ we can define the \textit{product} group $G\times H$ of pairs $(g,h)$ where $g\in G$ and $h\in H.$ Here, multiplication is defined componentwise.
\end{defi}
\begin{ex}
	For any field $k,$ we have that $k^n$ is a product group, for any positive integer $n.$
\end{ex}
\begin{ex}
	We have that $\CC^\times\cong\RR_{>0}\times S^1.$ This is merely saying that we can represent nonzero complex numbers uniquely by $z=re^{i\theta}\mapsto(r,\theta).$ Here is the image.
	\begin{center}
		\begin{asy}
			unitsize(1.5cm);
			draw((-1.5,0)--(1.5,0));
			draw((0,-1.5)--(0,1.5));
			draw(circle((0,0),1), gray);
			pair p = (2,1) * 0.7;
			dot("$z={\color{blue}r}e^{i{\color{red}\theta}}$", p, N);
			draw(arc((0,0),0.5,0,30-3), red);
			draw((0,0)--p, blue);
			label("\color{red}$\theta$", (0,0), 7*dir(15));
			label("\color{blue}$r$", p/2, dir(30+90));
		\end{asy}
	\end{center}
\end{ex}
\begin{ex}
	We have that $\ZZ/6\ZZ\cong\ZZ/2\ZZ\times\ZZ/3\ZZ,$ which is an instance of the Chinese remainder theorem.
\end{ex}
We can generalize the previous example.
\begin{prop} \label{prop:groupcrt}
	More generally, we have that $\ZZ/mn\ZZ\cong\ZZ/m\ZZ\times\ZZ/n\ZZ$ when $\gcd(m,n)=1.$
\end{prop}
\begin{remark}
	This does not hold for $m=n=2.$
\end{remark}
\begin{proof}[Proof of \autoref{prop:groupcrt}]
	This follows from the mapping
	\[\ZZ/mn\ZZ\to\ZZ/m\ZZ\times\ZZ/n\ZZ\]
	by taking $[k]_{mn}\to([k]_m,[k]_n).$ We can check that this is homomorphic by hand. This is injective because if $k\equiv0\pmod m$ and $k\equiv0\pmod n,$ then $m,n\mid k,$ so $mn\mid k$ because $\gcd(m,n)=1,$ so $k\equiv0\pmod{mn}.$ Then this map is surjective for size reasons, giving our isomorphism.
\end{proof}
\begin{ex}
	Consider the group of rotations of the various platonic solids. They have orders as follows.
	\begin{itemize}
		\item Tetrahedron: $12.$
		\item Cube: $24.$
		\item Octahedron: $24.$
		\item Icosahedron: $60.$
		\item Dodecahedron: $60.$
	\end{itemize}
	If we add in reflections, the number of these objects doubles, and in fact the bottom four are a product of $\ZZ/2\ZZ$ times the group of rotations. As for why, the added $\ZZ/2\ZZ$ comes from the reflection which inverts the entire figure, sending a vertex to its opposite. (This inversion is not a rotation because it has determinant $-1,$ when thought of as a matrix over $\RR^3.$)
\end{ex}
\begin{ex}
	Consider the set of all roots of unity in $\CC.$ These can be written explicitly as
	\[U_\infty=\left\{e^{2i\pi q}:q\in\QQ\right\}.\]
	We can decompose this into
	\[U_\infty\cong\{z:z\text{ has order a power of }2\}\times\{z:z\text{ has odd order}\}.\]
\end{ex}
Note that we can also take infinite products of groups, but sometimes that's too strong.
\begin{defi}[Sum]
	Given an infinite collection of groups $\{G_\alpha\}_{\alpha\in\lambda},$ we define the sum group
	\[\bigoplus_{\alpha\in\lambda}G_\alpha=\left\{\{g_\alpha\}_{\alpha\in\lambda}\in \prod_{\alpha\in\lambda}G_\alpha:g_\alpha=e_\alpha\text{ with finitely many exceptions}\right\}.\]
	This is a subgroup of the big product group.
\end{defi}
\begin{ex}
	We can check that
	\[U_\infty=\bigoplus_{p\text{ prime}}\{z:z\text{ has order a power of }p\}.\]
	This proof essentially boils down to the Chinese remainder theorem.
\end{ex}
\begin{ex}
	By unique prime factorization, we see that
	\[\QQ^\times\cong\{\pm1\}\oplus\bigoplus_{p\text{ prime}}\langle p\rangle,\]
	where $\langle p\rangle=p^\ZZ$ consists of the powers of $p.$
\end{ex}

\subsection{Groups of Orders Five and Six}
For groups of order $5,$ they are cyclic. Here is an exercise, for fun.
\begin{ques}
	Find a graph whose automorphism groups is $\ZZ/5\ZZ.$
\end{ques}
For groups of order $6,$ we note that we we have two obvious groups already:
\begin{itemize}
	\item We have $\ZZ/6\ZZ,$ which is $\ZZ/2\ZZ\times\ZZ/3\ZZ.$
	\item We have $S_3,$ the permutation group on three letters.
\end{itemize}
\begin{remark}
	Additionally, we see that $S_3$ is our first example of a nonabelian group! We see that
	\[(12)(23)=(123)\qquad\text{but}\qquad(23)(12)=(132).\]
	So this also shows that $S_3$ is not abelian.
\end{remark}
These are not isomorphic because $S_6$ is not cyclic. Alternatively, we can draw out our subgroup chart; here is the chart for $\ZZ/6\ZZ.$
% https://q.uiver.app/?q=WzAsNCxbMSwwLCJcXFpaLzZcXFpaIl0sWzAsMSwiXFxaWi8yXFxaWiJdLFsxLDIsIlxcbGFuZ2xlIGVcXHJhbmdsZSJdLFsyLDEsIlxcWlovM1xcWloiXSxbMiwzXSxbMywwXSxbMiwxXSxbMSwwXV0=
\[\begin{tikzcd}
	& {\ZZ/6\ZZ} \\
	{\ZZ/2\ZZ} && {\ZZ/3\ZZ} \\
	& {\langle e\rangle}
	\arrow[no head, from=3-2, to=2-3]
	\arrow[no head, from=2-3, to=1-2]
	\arrow[no head, from=3-2, to=2-1]
	\arrow[no head, from=2-1, to=1-2]
\end{tikzcd}\]
And we could write out the subgroup table of $S_3,$ and find that there are lots of subgroups of order $2.$
% https://q.uiver.app/?q=WzAsNixbMiwwLCJTXzMiXSxbMCwxLCJcXHtlLCgxLDIpXFx9Il0sWzEsMSwiXFx7ZSwoMSwzKVxcfSJdLFsyLDEsIlxce2UsKDIsMylcXH0iXSxbNCwyLCJcXHtlLCgxMjMpLCgxMzIpXFx9Il0sWzIsMywiXFx7ZVxcfSJdLFsxLDAsIiIsMCx7InN0eWxlIjp7ImhlYWQiOnsibmFtZSI6Im5vbmUifX19XSxbMiwwLCIiLDIseyJzdHlsZSI6eyJoZWFkIjp7Im5hbWUiOiJub25lIn19fV0sWzMsMCwiIiwyLHsic3R5bGUiOnsiaGVhZCI6eyJuYW1lIjoibm9uZSJ9fX1dLFs0LDAsIiIsMix7InN0eWxlIjp7ImhlYWQiOnsibmFtZSI6Im5vbmUifX19XSxbNSw0LCIiLDIseyJzdHlsZSI6eyJoZWFkIjp7Im5hbWUiOiJub25lIn19fV0sWzUsMywiIiwxLHsic3R5bGUiOnsiaGVhZCI6eyJuYW1lIjoibm9uZSJ9fX1dLFs1LDIsIiIsMSx7InN0eWxlIjp7ImhlYWQiOnsibmFtZSI6Im5vbmUifX19XSxbNSwxLCIiLDEseyJzdHlsZSI6eyJoZWFkIjp7Im5hbWUiOiJub25lIn19fV1d
\[\begin{tikzcd}
	&& {S_3} \\
	{\{e,(1,2)\}} & {\{e,(1,3)\}} & {\{e,(2,3)\}} \\
	&&&& {\{e,(123),(132)\}} \\
	&& {\{e\}}
	\arrow[no head, from=2-1, to=1-3]
	\arrow[no head, from=2-2, to=1-3]
	\arrow[no head, from=2-3, to=1-3]
	\arrow[no head, from=3-5, to=1-3]
	\arrow[no head, from=4-3, to=3-5]
	\arrow[no head, from=4-3, to=2-3]
	\arrow[no head, from=4-3, to=2-2]
	\arrow[no head, from=4-3, to=2-1]
\end{tikzcd}\]
In fact, the subgroup table of $S_3$ has ``lots'' of subgroups of order $2.$ What's going on? These are an instance of ``non-normal subgroups.''

\subsection{Normal Subgroups and Quotients}
Our motivation here is the following question.
\begin{ques}
	Given groups $H\subseteq G,$ can we define a group $G/H$? more precisely? Can we have a (surjective) homomorphism $\varphi:G\to G/H$ with kernel exactly $H$?
\end{ques}
We can write this as a short exact sequence
\[1\to H\to G\to G/H\to 1.\]
In general, we can define an exact sequence.
\begin{defi}[Exactness]
	Given a sequence of maps
	\[A\to B\to C,\]
	we say that this is \textit{exact at $B$} if the image of $A\to B$ is the kernel of $B\to C.$
\end{defi}
This lets us define short exact.
\begin{defi}[Short exact sequence]
	We define a short exact sequence as an exact sequence of the form
	\[1\to A\to B\to C\to 1.\]
	Namely, $A\to B$ is injective, $B\to C$ is surjective, and the image of $A\to B$ is the kernel of $B\to C.$
\end{defi}
Anyways, let's return to talking about our question. We are hoping that we have a well-defined map. So suppose that $g_1$ and $g_2$ have the same image in $G/H$: this is equivalent to $g_1h=g_2$ for some $h\in H$ by rearranging $\varphi(g_1)=\varphi(g_2).$ So we define left cosets.
\begin{defi}[Left cosetes]
	We define $G/H$ as the set of \textit{left cosets} $\{gH:g\in G\}.$ Note that we are not claiming this is a group in general.
\end{defi}
We hope that our group law is
\[g_1H\cdot g_2H=(g_1g_2)H.\]
However, this might not be well-defined! The issue is that, for any $h\in H,$ we also need
\[g_1hH\cdot g_2H=(g_1hg_2)H.\]
Note that this is free for abelian groups, for $hg_2=g_2h,$ so we can move the $h$ over. However, we can weaken this condition to merely requiring $hg_2=g_2h'$ for some $h',$ which is equivalent to $g_2hg_2^{-1}\in H$ for each $h\in H.$
\begin{defi}[Normal]
	We say that a subgroup $H\subseteq G$ is normal if, for each $g\in G,$ we have that $gHg^{-1}=H.$ (Actually, $gHg^{-1}\subseteq H$ is good enough here.)
\end{defi}
\begin{prop}
	Fix $H$ a normal subgroup of $G.$ Then $G/H$ is a group.
\end{prop}
\begin{proof}
	The main check is that $G/H$ has well-defined multiplication. Indeed, if $g_1H=g_1'H$ and $g_2H=g_2'H,$ then $g_1=g_1'h_1$ and $g_2=g_2'h_2$ for some $H_1,h_2\in H$ so that
	\[g_1H\cdot g_2H=(g_1g_2)H=(g_1'h_1g_2'h_2)H=g_1'H\cdot g_2'\underbrace{(g_2')^{-1}h_1g_2'h_2}_{\in H}H=g_1'H\cdot g_2'H.\]
	From here, checking that $G/H$ is actually a group is inherited more or less directly from $G$ because $G\to G/H$ is homomorphic and surjective.
\end{proof}
\begin{ex}
	The subgroup $\{e,(123),(132)\}\subseteq S_3$ is normal. For example, for any $\sigma\in S_3,$ we can check that
	\[\sigma(123)\sigma^{-1}=(\sigma1,\sigma2,\sigma3\}\in\{e,(123),(132)\}\]
	because $\sigma$ is a permutation. This normal subgroup gives us the exact sequence
	\[1\to\ZZ/3\ZZ\to S_3\to \ZZ/2\ZZ\to 1\]
	because the quotient $S_3/\{e,(123)(132)\}$ has order $2$ and must be $\ZZ/2\ZZ.$
\end{ex}
\begin{nex}
	The subgroup $\{e,(12)\}\subseteq S_3$ is \textit{not} a normal subgroup. Indeed, we can just check that
	\[(13)(12)(13)=(23)\notin\{e,(12)\}.\]
	However, we can check that conjugating $H=\{e,(12)\}$ by $g\in(23),$ we have that $H$ is conjugate to $gHg^{-1}=\{e,(23)\}.$
\end{nex}
As a side remark, we note that the left cosets equal the right ones for normal subgroups: any coset $gH$ can be written as a right coset by writing it as $gH=gHg^{-1}g=Hg$ by normality.

However, for non-normal subgroups, there are dangers.
\begin{ex}
	Again take $H:=\{e,(12)\}\subseteq S_3.$ Our left cosets are
	\[\begin{cases}
		H=\{e,(12)\}, \\
		(123)H=\{(123),(13)\}, \\
		(132)H=\{(132),(23)\}.
	\end{cases}\]
	However, our right cosets are
	\[\begin{cases}
		H=\{e,(12)\}, \\
		H(123)=\{(123),(23)\}, \\
		H(132)=\{(132),(13)\}.
	\end{cases}\]
\end{ex}

\subsection{Cauchy's Theorem}
Let's use this an excuse to introduce some theorems. Here is a motivating question.
\begin{ques}
	Suppose that $d\mid\#G$ for a group $G.$ Is there an element of order $d$?
\end{ques}
Well, of course not: $\ZZ/2\ZZ\times\ZZ/2\ZZ$ has order $4$ but does not have an element of order $4.$ However, we have the following.
\begin{thm}[Cauchy]
	Suppose that $p$ is a prime dividing the order of a group $G.$ Then there is an element of order $p.$
\end{thm}
\begin{proof}
	We do casework on if $G$ is abelian.
	\begin{remark}
		Trying to prove something for groups $G$ by doing casework on $G$ abelian vs. $G$ nonabelian is like trying to prove something for objects $O$ in the universe by doing casework on if $O$ is a banana or $O$ is not a banana. But here we go.
	\end{remark}
	\begin{itemize}
		\item If $G$ is abelian, we start by picking up $a\in G\setminus\{e\}.$ (If $G=\{e\},$ there is nothing to show.) Then we can raise $a$ to a power to kill all the primes in its order except for, say, $q.$ If $p=q,$ then we are done.
		
		Otherwise, we can look at $G/\langle a\rangle,$ where this quotient is good because our groups is abelian. Then this has order $\#G/q,$ which is still divisible by $p$ because $q\ne p.$ So induction can give us a coset $b\langle a\rangle\in G/\langle a\rangle$ of order $p.$
		
		However, $b^p\in\langle a\rangle$ is either the identity or some generic element of $\langle a\rangle,$ but certainly $b^{pq}=e.$ The order cannot be $1$ ($b\langle a\rangle\in G/\langle a\rangle$ has order $p$), nor can it be $q$ (this would force $b^q\in\langle a\rangle,$ but $p\nmid q$), so the order is either $p$ or $pq.$ If $p,$ we are done; if $pq,$ then $b^q$ has order $p.$
		
		\item If $G$ is nonabelian, then we again have two cases: if $G$ has a proper subgroup of order divisible by $p,$ then we can do induction to finish. Otherwise, all proper subgroups have order not divisible by $p$ with index $[G:H]$ always divisible by $p.$
		
		Now the trick is to look at the action of $G$ on $G$ by conjugation, and split up the action into orbits, which are conjugacy classes. Explicitly,
		\[Gg=\left\{ghg^{-1}:h\in G\right\}.\]
		Now we check that the size of any orbit $Gg$ is $\#G/\#\op{Stab}(g)$ by the Orbit-stabilizer theorem. But this is always divisible by $p,$ except when $\#\op{Stab}(g)=G$ because I said so.
		
		To finish, we do the class equation by hand. We see that
		\[G=\bigcup_{Gg}Gg\label{eq:classeq}\tag{$*$}\]
		because we are partitioning by the action. The left-hand side has size divisible by $p,$ and the right-hand orbits are all divisible by $p$ except for elements $h\in G$ such that $ghg^{-1}=h$ for all $g\in G.$ This gives us the following definition.
	\end{itemize}
	\begin{defi}[Center]
		For $g$ a group, we define $Z(G)=\{g\in G:ghg^{-1}=h\}.$ In other words, $gh=hg$ for each $g\in Z(G)$ and $h\in G,$ so $Z(G)$ commutes with everyone.
	\end{defi}
	\begin{itemize}
		\item[] Finishing up the proof, we see that \hyperref[eq:classeq]{$(*)$} reads as
		\[\#G=Z(G)+\sum_{\substack{Gg\\\#Gg>1}}\#Gg\]
		after taking sizes, and everything here is divisible by $p$ except $Z(G),$ requiring that $Z(G)$ has size divisible by $p.$ But now $Z(G)$ is an abelian subgroup (everything commutes by definition), so it has an element of order $p,$ finishing.
		
		Alternatively, $Z(G)$ is a proper subgroup (proper because $G$ is nonabelian) with order divisible by $p,$ which is a contradiction to our assumption that $G$ has no proper subgroups with order divisible by $p.$
		\qedhere
	\end{itemize}
\end{proof}
\begin{remark} \label{rem:inddivp}
	The above argument actually shows that if all proper subgroups have index divisible by $p,$ then $Z(G)$ is divisible by $p.$
\end{remark}
\begin{remark}
	There are many ways for a group $G$ to act on itself.
	\begin{itemize}
		\item There is a left action, by $g\cdot h=gh.$.
		\item There is a trivial action: $g\cdot h=h.$
		\item There is a right action: $g\cdot h=hg^{-1}.$ (Note the inverse is required for associativity reasons.)
		\item There is the conjugacy action: $g\cdot h=ghg^{-1}.$
	\end{itemize}
	Then there are the corresponding right actions.
\end{remark}

Let's use this to classify groups of order $6.$
\begin{prop}
	There are only two non-isomorphic groups of order $6,$ which are $\ZZ/6\ZZ$ and $S_3.$
\end{prop}
\begin{proof}
	Fix $G$ of order $6.$ Then we are promised an element $a$ of order $3$ and an element $b$ of order $2.$ Well, we claim that $\langle a\rangle$ is normal. More generally, we have the following.
	\begin{lem}
		Fix $H\subseteq G$ a subgroup of index $2.$ Then $H$ is normal.
	\end{lem}
	\begin{proof}
		Indeed, for any $g\in H,$ we see that $gHg^{-1}=H$ for free. Otherwise, when $g\in G\setminus H,$ we have that $gH$ and $Hg$ must both be disjoint from $H$ while having size $H$ (recall $[G:H]=2$), so $gH=Hg=G\setminus H.$ In particular, $gH=Hg$ implies $gHg^{-1}=H$ still.
	\end{proof}
	Thus, we have a short exact sequence
	\[1\to\underbrace{\ZZ/3\ZZ}_{\langle a\rangle}\to G\to\ZZ/2\ZZ\to 1.\]
	\begin{remark}
		Filling in the middle here need not be unique, even in basic cases. For example, we have a short exact sequence
		\[1\to\ZZ/2\ZZ\to G\to\ZZ/2\ZZ\to 1\]
		where $G=(\ZZ/2\ZZ)^2$ or $\ZZ/4\ZZ.$
	\end{remark}
	Regardless, we simply do this by hand. We have the following definition.
	\begin{defi}[Split short exact sequence]
		The short exact sequence
		\[1\to A\to B\to C\to 1\]
		\textit{splits} if $B$ has a subgroup $C_B$ isomorphic to $C$ lifting $B\to C.$
	\end{defi}
	In particular, we see that
	\[1\to\underbrace{\ZZ/3\ZZ}_{\langle a\rangle}\to G\to\ZZ/2\ZZ\to 1.\]
	splits because $G$ does have a subgroup $\langle b\rangle$ isomorphic to $\ZZ/2\ZZ.$ The point is that $\langle b\rangle$ acts on $\langle a\rangle$ by conjugation because $\langle a\rangle$ is normal (this is the restriction of $G\to\op{Aut}(\langle a\rangle)$ to from $G$ to $\langle b\rangle$). So we have induced an action of $\ZZ/2\ZZ$ on $\ZZ/3\ZZ,$ but we only have a few automorphisms of $\ZZ/3\ZZ,$ so we are forced to have one of
	\[\begin{cases}
		b a b^{-1}=a, \\
		b a b^{-1}=a^2.
	\end{cases}\]
	So we have the group presentations
	\[\begin{cases}
		G=\left\langle a^3=1,b^2=1,b a b^{-1}=a\right\rangle\cong\ZZ/2\ZZ\times\ZZ/3\ZZ, \\
		G=\left\langle a^3=1,b^2=1,b a b^{-1}=a^2\right\rangle\cong S_3.
	\end{cases}\]
	The last group is isomorphic to $S_3$ by taking $a=(123)$ and $b=(12),$ say.
\end{proof}

\subsection{Semidirect Products}
What's happening with our split short exact sequences is semidirect products.
\begin{defi}[Semidirect products]
	Suppose that $A$ and $C$ are groups such that $A$ has a $C$-action. (In other words, there is a homomorphism $C\to\op{Aut}(A).$) Then we define $G$ as the \textit{semidirect product} if we can form the short exact sequence
	\[1\to A\to G\to C\to 1\]
	such that $G$ has (isomorphic copies of) $A$ as a normal subgroup and $C$ as another subgroup.
\end{defi}
We should actually exhibit our semidirect product. We have the following.
\begin{prop} \label{prop:semidirect}
	Fix $A$ and $C$ as above. We define the semidirect product $G=A\times C$ as a set, with multiplication defined by
	\[(a_1,c_1)(a_2,c_2)=(a_1(c_1\cdot a_2),c_1c_2),\]
	well $c_1\cdot a_2$ refers to the $C$-action on $A.$
\end{prop}
\begin{remark}
	Let's try to motivate this multiplication. Informally, we want the action of $C$ on $A$ to be conjugation so that $A$ stands a pretty good chance of being normal, and we want to be able to think of pairs $(a,c)$ as actual products $ac.$ These forces combine to let us write
	\begin{align*}
		(a_1,c_1)(a_2,c_2) &= a_1c_1a_2c_2 \\
		&= a_1c_1a_2\left(c_1^{-1}c_1\right)c_2 \\
		&= a_1\left(c_1a_2c_1^{-1}\right)(c_1c_2) \\
		&= \left(a_1(c_1\cdot a_2),c_1c_2\right).
	\end{align*}
\end{remark}
\begin{proof}[Proof of \autoref{prop:semidirect}]
	We have to check that this is a group, which can be checked by force. We run down the properties because some of this more subtle than it appears.
	\begin{itemize}
		\item Associativity in the second coordinate is inherited from $C.$ Associativity in the first coordinate comes from writing
		\[\big((a_1,c_1)(a_2,c_2)\big)(a_3,c_3)=(a_1(c_1\cdot a_2),c_1c_2)(a_3,c_3)=(a_1(c_1\cdot a_2)(c_1c_2\cdot a_3),\bullet),\]
		and comparing it with
		\[(a_1,c_1)\big((a_2,c_2)(a_3,c_3)\big)=(a_1,c_1)(a_2(c_2\cdot a_3),\bullet)=(a_1c_1\cdot(a_2(c_2\cdot a_3)),\bullet).\]
		These are equal because our $C$-action is inducing a homomorphism $C\to\op{Aut}(A).$
		\item Our identity element is $(e,e).$
		\item Our inverse element is $(a,c)^{-1}=\left(c^{-1}\cdot a^{-1},c^{-1}\right).$ On one side,
		\[(a,c)\left(c^{-1}\cdot a^{-1},c^{-1}\right)=\left(a\left(c\cdot c^{-1}a^{-1}\right),e\right)=(e,e).\]
		On the other side,
		\[\left(c^{-1}\cdot a^{-1},c^{-1}\right)(a,c)=\left(\left(c^{-1}\cdot a^{-1}\right)\left(c^{-1}a\right),e\right)=\left(c^{-1}\cdot\left(a^{-1}a\right),e\right)=(e,e),\]
		where we again used that the $C$-action is inducing a homomorphism $C\to\op{Aut}(A).$
	\end{itemize}
	Now we will check that the short exact sequence
	\[1\to A\to G\to C\to 1\]
	splits, as well as that $A$ is normal in $G.$ We have the following to check.
	\begin{itemize}
		\item Exact at $A$: the map $A\to G$ is injective, defined by $a\mapsto(a,e).$ It's not hard to see that this is homomorphic.
		\item Exact at $C$: the map $G\to C$ is surjective, defined by $(a,c)\mapsto c.$ This is homomorphic because the second coordinate of $A\times C$ is merely multiplication.
		\item Exact at $G$: the map $A\to G$ surjects onto points of the form $\{(a,e):a\in A\},$ and the kernel of $G\to C$ is exactly the points such that $(a,c)\mapsto c=e,$ which is again $\{(a,e):a\in A\}.$ So $\im(A\to G)=\ker(G\to C).$
		\item We split: The subgroup $\{(e,c):c\in C\}$ is isomorphic to $C$ and lifts our $G\to C$ projection, so the given short exact sequence splits.
		\item $A$ is normal: We need to show that $A_G:=\{(a,e):a\in A\}$ is normal in $G.$ It is enough to note that, for any $(a_0,e)\in A_G$ and $(a,c)\in G,$ we have
		\begin{align*}
			(a,c)(a_0,e)(a,c)^{-1} &= (a,c)(a_0,e)\left(c^{-1}a^{-1},c^{-1}\right) \\
			&= (\text{garbage},c)\left(c^{-1}a^{-1},c^{-1}\right) \\
			&= \left(\text{more garbage},cc^{-1}\right) \\
			&= (\text{more garbage},e).\qedhere
		\end{align*}
	\end{itemize}
\end{proof}
\begin{ex}
	We have that $S_3$ is the semidirect product of $\ZZ/3\ZZ$ by $\ZZ/2\ZZ,$ notated $\ZZ/\ZZ\rtimes\ZZ/2\ZZ.$ Notice that the construction of ``semidirect'' takes more data than is provided by $\ZZ/3\ZZ$ and $\ZZ/2\ZZ$: we also need to know the action.
\end{ex}
Let's do some more examples.
\begin{ex}
	Take the set of all linear functions $x\mapsto ax+b,$ where our multiplication is composition. We can check that we have a normal subgroup $x\mapsto x+b,$ and its quotient group is isomorphic to $x\mapsto ax.$
\end{ex}
\begin{ex}
	The Poincar\'e group consists of the automorphisms of space-time. It has a normal subgroup consisting of translations through space-time, and the quotient is the ``Lorentz group'' all rotations of space-time which preserve the metric $t^2-x^2-y^2-z^2=0.$
\end{ex}

\end{document}