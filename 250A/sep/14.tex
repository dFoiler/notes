% !TEX root = ../notes.tex













% https://www.asofterworld.com/index.php?id=695
If you can't stand the heat, turn the A/C on.

\subsection{Free Abelian Groups}
This is going to be our last lecture on group theory, and it will be on free groups. A free groups on generators $a,b,c$ is the ``largest possible'' group generated by those elements.

As an exercise, let's talk about free abelian groups. The free abelian group on generators $\{a_k\}_{k=1}^n$ can more or less be tracked by the sums
\[\sum_{k=1}^nm_ka_k\]
for integers $\{m_k\}_{k=1}^n.$ Indeed, a group containing $\{a_k\}_{k=1}^n$ must have the above elements, and being abelian, we can always coerce a word in the above form. All of these elements are different, so we get the direct sum
\[\bigoplus_{k=1}^n\ZZ a_k\cong\ZZ^n,\]
where the isomorphism consists of coordinate-extraction. With this in mind, we take this as our definition of the free abelian group.
\begin{definition}[Free abelian group]
	Given letters $\{a_k\}_{k=1}^n,$ we define the free abelian group $F$ on the letters $\{a_k\}_{k=1}^n$ as the group
	\[\bigoplus_{k=1}^n\ZZ a_k.\]
\end{definition}
Let's prove some things; it turns out that free abelian groups are quite nice. We start with the universal property because it's nicer than the actual definition we gave.
\begin{proposition}[Universal property of free abelian groups]
	Fix $F$ the free abelian group generated by $\{a_k\}_{k=1}^n.$ Then, given an abelian group $G$ with elements $\{g_k\}_{k=1}^n,$ there is a unique group homomorphism $\varphi:F\to G$ such that $\varphi:a_k\mapsto g_k$ for each $k.$
\end{proposition}
\begin{proof}
	On one hand, certainly if $\varphi$ exists, then, for any $\sum_{k=1}^nm_ka_k\in F,$ then we have
	\[\varphi\left(\sum_{k=1}^nm_ka_k\right)=\sum_{k=1}^nm_k\varphi(a_k)=\sum_{k=1}^nm_kg_k,\]
	so $\varphi$ has only one option for where it can send all the elements.

	In the other direction, we claim that
	\[\varphi\left(\sum_{k=1}^nm_ka_k\right):=\sum_{k=1}^nm_kg_k\]
	actually defines a group homomorphism. This is well-defined because every element of $F$ has a unique representation as $\sum_{k=1}^nm_ka_k$ (by definition of the direct sum). This is homomorphic because we can sum
	\begin{align*}
		\varphi\left(\sum_{k=1}^nm_ka_k+\sum_{k=1}^nm_k'a_k\right) &= \varphi\left(\sum_{k=1}^n(m_k+m_k')a_k\right) \\
		&= \sum_{k=1}^n(m_k+m_k')g_k \\
		&= \sum_{k=1}^nm_kg_k+\sum_{k=1}^nm_k'g_k,
	\end{align*}
	where the last equality holds because $G$ is abelian. (Note this is the only place where we used that $G$ is abelian.)
\end{proof}
This gives the following properties with ease.
\begin{proposition} \label{prop:freeabprops}
	The following are true.
	\begin{enumerate}[label=(\alph*)]
		\item The rank of a free abelian group is well-defined.
		\item Any subgroup of a free abelian group on $n$ generators is free abelian on at most $n$ generators.
	\end{enumerate}
\end{proposition}
\begin{proof}
	We do these one at a time.
	\begin{enumerate}[label=(\alph*)]
		\item Essentially, for $n_1\ne n_2,$ we want to show that $\ZZ^{n_1}\not\cong\ZZ^{n_2}.$ To show this, we look at the number of homomorphisms from $\ZZ^n\to\ZZ/2\ZZ.$ Each homomorphism can be tracked by if it sends each generator to $1$ or $0,$ so there are $2^n$ of these.
		
		Explicitly, there is a unique homomorphism from the free abelian group on $n$ letters
		\[\varphi:\bigoplus_{k=1}^n\ZZ\alpha_k\to\ZZ/2\ZZ\]
		for each function $f:\{\alpha_k\}_{k=1}^n\to\ZZ/2\ZZ,$ by the universal property. Because each homomorphism also gives rise to a function $f,$ we see that the number of homomorphisms is equal to the number of functions $f,$ so there are $2^n$ total homomorphisms.

		To finish, we see that the free abelian group on $n_1$ letters and the free abelian group on $n_2$ letters being isomorphic implies that the number of homomorphisms to $\ZZ/2\ZZ$ is equal, so $2^{n_1}=2^{n_2},$ so $n_1=n_2.$

		\item This proof is the same in spirit to the classification of finitely generated abelian groups.\footnote{In fact, this property can be used to show the Classification of finitely generated abelian groups, but I am having trouble going the other direction.} It suffices to show that any subgroup $G\subseteq\ZZ^n$ is free abelian on at most $n$ letters.
		
		We first show that $G$ is finitely generated by $n$ vectors; this is by induction. The idea is to project onto the last coordinate, yielding the subgroup
		\[G_n:=\{k_n:(k_1,k_2,\ldots,k_n)\in G\}\subseteq\ZZ,\]
		which must take the form $d_n\ZZ$ for some $d_n\in\ZZ.$ Fixing $(d_1,\ldots,d_n)\in G_n$ its vector, then we can embed $G/(d_1,\ldots,d_n)\ZZ$ into $\ZZ^{n-1},$ so $G/(d_1,\ldots,d_n)$ can be generated by $n-1$ vectors by induction, so $G$ can be generated by at most $n$ vectors.

		So suppose that $G$ is generated by the row vectors of
		\[\begin{bmatrix}
			d_{11} & \cdots & d_{1n} \\
			\vdots & \ddots & \vdots \\
			d_{n1} & \cdots & d_{n n}
		\end{bmatrix}.\]
		We can check again that the row and column operations seen in the Classification of finitely generated abelian groups do not change the actual group structure (the check is identical, so we won't do it here), so we can reduce our matrix to one that looks like
		\[\begin{bmatrix}
			m_{1} & 0 & \cdots & 0 \\
			0 & m_{2} & \cdots & 0 \\
			\vdots & \vdots & \ddots & \vdots \\
			0 & 0 & \cdots & m_{n}
		\end{bmatrix}\]
		so that $G\cong\bigoplus_{k=1}^nm_k\ZZ,$ finishing the proof.
		\qedhere
	\end{enumerate}
\end{proof}

\subsection{Free Nonabelian Groups}
So let's look at free nonabelian groups. We want the biggest possible group generated by the elements. Observe that it's not even obvious that such a thing exists!
\begin{proposition}
	The free group $F$ on $\{a_k\}_{k=1}^n$ exists and is a group.
\end{proposition}
\begin{proof}
	Let $S$ be the set of all possible words (empty allowed) whose letters are in $\{a_k\}_{k=1}^n$ or $\{a_k^{-1}\}_{k=1}^n,$ and we simply mod this out by all relations which give the group axioms. For example, we should mod out by the relation that
	\[(ab)(cc)=(a(bc))c\]
	and all of its friends. To be explicit, we define the equivalence relation $\equiv$ on $S$ defined as follows.
	\begin{itemize}
		\item Inverse: if we have $w=w_1\ell\ell^{-1}w_2$ for some words $w_1,w_2$ and letter $\ell\in\{a_k\}_{k=1}^N\cup\{a_k^{-1}\}_{k=1}^n,$ then $w\equiv w_1w_2.$
		\item Well-defined concatenation: if $w_1\equiv w_2$ and $v_1\equiv v_2,$ then $w_1v_1\equiv w_2v_2.$
	\end{itemize}
	To be rigorous, we could do something like declare $S$ a graph where the above two rules define edges; then $\equiv$ consists of equivalence classes of vertices, where two vertices are in the same equivalence if there is a finite path connecting them. We now check the group axioms by hand.
	\begin{remark}
		It is almost obvious, but it's not obvious that it's obvious.
	\end{remark}
	\begin{itemize}
		\item We make our group law concatenation. It is well-defined because our equivalence class forced it to be.
		\item Associativity: given $w_1,w_2,w_3,$ then concatenating $w_1w_2$ with $w_3$ is the same as concatenating $w_1$ with $w_2w_3.$
		\item Identity: our identity is the empty string because concatenating the empty string does nothing.
		\item Inverse: given a string $w=\prod_{k=1}^N\ell_k$ for letters $\ell_k,$ the inverse law implies
		\[\left(\prod_{k=1}^N\ell_k\right)\left(\prod_{k=1}^N\ell_{N+1-k}^{-1}\right)\]
		is the empty string; formally we would do an induction here, but we won't bother.
		\qedhere
	\end{itemize}
\end{proof}
Even though we have defined the free group as being equivalence classes of words, we will liberally call the elements of the free group ``words'' and refer to specific representatives.

In reality, the easiest way to handle the free group is by universal property.
\begin{proposition}[Universal property of the free group]
	Given a group $G$ with elements $\{g_k\}_{k=1}^n,$ the free group $F$ on $\{a_k\}_{k=1}^n$ has a unique map $\varphi:F\to G$ such that $\varphi(a_k)=g_k.$
\end{proposition}
\begin{proof}
	Again, the uniqueness of this map is the easier part: given a word $w=\prod_{k=1}^Na_k^{\varepsilon_k}$ for $\varepsilon_k\in\{\pm1\},$ we see that $\varphi$ being a homomorphism forces
	\[\varphi(w)=\varphi\left(\prod_{k=1}^Na_k^{\varepsilon_k}\right)=\prod_{k=1}^N\varphi(a_k)^{\varepsilon_k}=\prod_{k=1}^Ng_k^{\varepsilon_k},\]
	so indeed, $\varphi$ is forced. It remains to show that
	\[\varphi\left(\prod_{k=1}^Na_k^{\varepsilon_k}\right):=\prod_{k=1}^Ng_k^{\varepsilon_k}\]
	is actually a group homomorphism. Namely, we have to show that $\varphi$ is well-defined and a homomorphism.
	\begin{itemize}
		\item We show that $\varphi$ satisfies $\varphi(wv)=\varphi(w)\varphi(v)$ for words $w$ and $v,$ where now we are not treating $w$ and $v$ as equivalence classes but as actual words. Well, writing $w=\prod_{k=1}^Na_k^{\alpha_k}$ and $v=\prod_{k=N+1}^Ma_k^{\alpha_k}$ (where we have continued our indexing implicitly), we see that
		\[\varphi(wv)=\varphi\left(\prod_{k=1}^Na_k^{\alpha_k}\cdot\prod_{k=N+1}^Ma_k^{\alpha_k}\right)=\varphi\left(\prod_{k=1}^Ma_k^{\alpha_k}\right)=\prod_{k=1}^Mg_k^{\alpha_k}=\prod_{k=1}^Ng_k^{\alpha_k}\cdot\prod_{k=N+1}^Mg_k^{\alpha_k}=\varphi(w)\varphi(v).\]
		\item We show that $\varphi$ is well-defined. Because two elements are equal if and only if we can finitely apply the inverse and well-defined concatenation laws to make them term-wise equal, it suffices to show that $\varphi$ is well-defined up to one application of each of these (and finish by induction).

		For the inverse law, we show that $\varphi(w_1a_k^{\varepsilon}a_k^{-\varepsilon}w_2)=\varphi(w_1w_2)$ for some letter $a_k^\varepsilon.$ We know that $\varphi$ satisfies the homomorphism property for words, so we may freely write
		\[\varphi(w_1a_k^{\varepsilon}a_k^{-\varepsilon}w_2)=\varphi(w_1)g_k^\varepsilon g_k^{-\varepsilon}\varphi(w_2)=\varphi(w_1)\varphi(w_2)=\varphi(w_1w_2).\]
		For the well-defined concatenation law, we show that $w_1\equiv w_2$ and $v_1\equiv v_2$ implies that $\varphi(w_1v_1)=\varphi(w_2v_2).$ Well, by induction (say on the maximum word length among $w_1,w_2,v_1,v_2$), we may suppose that $\varphi(w_1)=\varphi(w_2)$ and $\varphi(v_1)=\varphi(v_2)$ so that
		\[\varphi(w_1v_1)=\varphi(w_1)\varphi(v_1)=\varphi(w_2)\varphi(v_2)=\varphi(w_2v_2).\]
		This finishes.
		\qedhere
	\end{itemize}
\end{proof}
Note that the same proof as above works for other algebraic structures. So we can also define free rings, free algebras, and so on as the ``universal object'' by all possible words and modding out by all relations.
\begin{warn}
	As a warning, there are no ``free fields.''
\end{warn}
Here are two reasons why there are no ``free fields.''
\begin{itemize}
	\item The core problem here is that the inverse function $a\mapsto a^{-1}$ is necessary but not defined everywhere, so fields aren't as nice as algebraic structures, in the sense that an algebraic structure should have operations defined everywhere with some relations. This makes ``all possible words'' in the above argument somewhat difficult.
	\item We can actually prove that there is no free functor for $\text{Fld}.$ In short, all morphisms are injective, so if our free object on $n$ elements is to have a map into $\FF_2,$ then our free object must inject into $\FF_2$ and be $\FF_2.$ But then there is no map $\FF_2$ into $\FF_3.$
\end{itemize}
So free abelian groups are nicely behaved. Let's move on to the nonabelian case.

\subsection{Reduced Words}
Free groups are pretty huge; what do they look like? For example, is it nontrivial? This is not immediately obvious because our construction was complicated in the sense that there were a lot of equivalence relations. So let's try to bound the size of our group.

To upper-bound the size of our group, we note that every element is a word in the letters $a_\bullet$ and $a_{\bullet}^{-1},$ but this is somewhat inefficient because we can immediately cancel $a_1a_1^{-1}$ and its friends. So we actually count with the following definition.
\begin{definition}[Reduced words]
	Let $F$ be the free group on $\{a_k\}_{k=1}^n.$ Then we define \textit{reduced} words as words in $F$ which do not contain $\ell\ell^{-1}$ for some letter $\ell\in\{a_k\}_{k=1}^n\cup\{a_k^{-1}\}_{k=1}^n,$ which is our upper bound.
\end{definition}
In particular, we see that every word in the free group has at least one representation as a reduced word simply by removing all $\ell\ell^{-1}$ substrings.

It feels like reduced words cannot collide, but it is nontrivial to prove this. Well, suppose that we have two reduced words $w_1$ and $w_2$ which are not equal term-wise so that we want to show $w_1w_2^{-1}\ne e.$ In other words, we have that $w_1w_2^{-1}$ does not immediately reduce to the identity ($w_1$ and $w_2$ are not term-wise equal), and we want $w_1w_2^{-1}.$

So it suffices to show the following lemma.
\begin{lemma} \label{lem:reduced}
	Fix $F$ the free group on $\{a_k\}_{k=1}^n.$ Then all nontrivial reduced words $w$ cannot collapse to $e.$
\end{lemma}
The idea here is to use the universal property. Let's give some examples of things that we can do, just to get the feeling for our power.
\begin{itemize}
	\item Let's show that $a_k\ne e.$ Well, we can map $a_k\mapsto1$ in $\ZZ$ and $a_\ell\mapsto0$ in $\ZZ$ so that $a_k\mapsto1,$ which is not the identity, so $a_k\ne e.$
	\item Let's show that $a_k\ne a_\ell$ for $k\ne\ell.$ Well, we map $a_k\mapsto1$ and $a_\ell\mapsto0$ again, and the map any other generator $a_\bullet\mapsto0.$ Then we see that $a_ka_\ell\mapsto1$ and is not the identity, so $a_ka_\ell^{-1}\ne e.$
\end{itemize}
In other words, we just showed that the map from our set of generators to the group is injective; I'm glad we got that squared away. We continue.
\begin{itemize}
	\item We show that $a_1^2a_2a_1^{-1}a_2^{-1}$ is nontrivial. Well, send $a_1\mapsto1$ and $a_2\mapsto0$ in $\ZZ$ and our word gets sent to $1\ne0.$
	\item In general, if a word $w$ has an unequal number of $a_k$ and $a_k^{-1}$ letters, then we can send $a_k\mapsto1\in\ZZ$ and all other generators to $0.$ Then $w$ gets sent to some nontrivial integer.
\end{itemize}
In some sense, this last condition is the best we can do by mapping to abelian groups, for abelian groups will always send elements with an equal number of $\ell$ letters and $\ell^{-1}$ letters to the identity.

Now that we've gotten a feeling for the universal property, let's jump into the general case.
\begin{proof}[Proof of \autoref{lem:reduced}]
	Suppose that $w$ is a word of length $N$; we map $F$ into $S_{N+1}.$ For concreteness, here is an example of our idea.
	\begin{example}
		Let's show that $aaba^{-1}b^{-1}a^{-1}$ is nontrivial in the free group on $\{a,b\}.$ The idea is that we want to give permutations $a$ and $b$ which satisfy the following movement.
		% https://q.uiver.app/?q=WzAsNyxbMCwwLCIxIl0sWzEsMCwiMiJdLFsyLDAsIjMiXSxbMywwLCI0Il0sWzQsMCwiNSJdLFs1LDAsIjYiXSxbNiwwLCI3Il0sWzEsMCwiYSJdLFsyLDEsImIiXSxbMywyLCJhIl0sWzMsNCwiYiIsMl0sWzQsNSwiYSIsMl0sWzUsNiwiYSIsMl1d
		\[\begin{tikzcd}
			1 & 2 & 3 & 4 & 5 & 6 & 7
			\arrow["a", from=1-2, to=1-1]
			\arrow["b", from=1-3, to=1-2]
			\arrow["a", from=1-4, to=1-3]
			\arrow["b"', from=1-4, to=1-5]
			\arrow["a"', from=1-5, to=1-6]
			\arrow["a"', from=1-6, to=1-7]
		\end{tikzcd}\]
		The point is that $aaba^{-1}b^{-1}a^{-1}$ will surely get sent to a nontrivial permutation now: $aaba^{-1}b^{-1}a^{-1}1=7.$ Actually exhibiting $a$ and $b$ a matter of extending the constraints
		\[\begin{tikzcd}
			1 & 2 & 3 & 4 & 5 & 6 & 7
			\arrow["a", from=1-2, to=1-1]
			\arrow["a", from=1-4, to=1-3]
			\arrow["a"', from=1-5, to=1-6]
			\arrow["a"', from=1-6, to=1-7]
		\end{tikzcd}\]
		to a permutation $a$ and the constraints
		\[\begin{tikzcd}
			1 & 2 & 3 & 4 & 5 & 6 & 7
			\arrow["b", from=1-3, to=1-2]
			\arrow["b"', from=1-4, to=1-5]
		\end{tikzcd}\]
		to a permutation $b.$ There are lots of ways to do this.
	\end{example}
	In general, fix our nontrivial reduced word $w=\prod_{k=1}^Na_k^{\varepsilon_k},$ where $\varepsilon_k\in\{\pm1\}.$ Then we would like to send $a_k$ to a permutation so that we can have the following computation.
	% https://q.uiver.app/?q=WzAsNixbMCwwLCIxIl0sWzEsMCwiMiJdLFsyLDAsIjMiXSxbMywwLCJcXGNkb3RzIl0sWzQsMCwiTi0xIl0sWzUsMCwiTiJdLFswLDEsImFfTl57XFx2YXJlcHNpbG9uX059Il0sWzEsMiwiYV97Ti0xfV57XFx2YXJlcHNpbG9uX3tOLTF9fSJdLFsyLDNdLFszLDRdLFs0LDUsImFfMV57XFx2YXJlcHNpbG9uXzF9Il1d
	\[\begin{tikzcd}
		1 & 2 & 3 & \cdots & {N-1} & N
		\arrow["{a_N^{\varepsilon_N}}", from=1-1, to=1-2]
		\arrow["{a_{N-1}^{\varepsilon_{N-1}}}", from=1-2, to=1-3]
		\arrow[from=1-3, to=1-4]
		\arrow[from=1-4, to=1-5]
		\arrow["{a_1^{\varepsilon_1}}", from=1-5, to=1-6]
	\end{tikzcd}\]
	(A forward arrow of $a_k^{-1}$ is intended to mean a backward arrow for $a_k.$) Here, $w$ will be sent to a nontrivial permutation, namely sending $1$ to $N.$ It remains to show that we can actually extend the above constraints to actual permutations.

	There are some obstructions to extending our constraints. For example, if we every end up with the following constraints, we immediately violate injectivity and cannot be a permutation.
	% https://q.uiver.app/?q=WzAsNixbMCwwLCJcXGJ1bGxldCJdLFsxLDAsIlxcYnVsbGV0Il0sWzIsMCwiXFxidWxsZXQiXSxbMywwLCJcXGJ1bGxldCJdLFs0LDAsIlxcYnVsbGV0Il0sWzUsMCwiXFxidWxsZXQiXSxbMCwxXSxbMiwxXV0=
	\[\begin{tikzcd}
		\bullet & \bullet & \bullet
		\arrow[from=1-1, to=1-2]
		\arrow[from=1-3, to=1-2]
	\end{tikzcd}\]
	Similarly, the following constraints cannot even make a function.
	% https://q.uiver.app/?q=WzAsNixbMCwwLCJcXGJ1bGxldCJdLFsxLDAsIlxcYnVsbGV0Il0sWzIsMCwiXFxidWxsZXQiXSxbMywwLCJcXGJ1bGxldCJdLFs0LDAsIlxcYnVsbGV0Il0sWzUsMCwiXFxidWxsZXQiXSxbMSwwXSxbMSwyXV0=
	\[\begin{tikzcd}
		\bullet & \bullet & \bullet
		\arrow[from=1-2, to=1-1]
		\arrow[from=1-2, to=1-3]
	\end{tikzcd}\]
	However, these are the only obstructions to extending a permutation.\footnote{ Showing this is annoying. Lacking the given obstructions, any constraint arrow $k\to k+1$ must either have no arrow into $k$ or an arrow into $k,$ but no arrow out of $k$; similar holds for $k+1.$ Essentially this means that all of our constraints look like disjoint ``chains'' $x\to x+1\cdots\to\cdots\to y-1\to y$ (possibly backwards). We extend this to a permutation by fixing any element not in a chain and sending $y$ to $x.$} Further, neither of these obstructions in our constraints for a particular letter: having
	\[\begin{tikzcd}
		\bullet & \bullet & \bullet
		\arrow["a_k", from=1-1, to=1-2]
		\arrow["a_k"', from=1-3, to=1-2]
	\end{tikzcd}\]
	would mean that $w$ contains $a_ka_k^{-1},$ violating that $w$ is reduced. And having
	\[\begin{tikzcd}
		\bullet & \bullet & \bullet
		\arrow["a_k"', from=1-2, to=1-1]
		\arrow["a_k", from=1-2, to=1-3]
	\end{tikzcd}\]
	would mean that $w$ contains $a_k^{-1}a_k,$ again violating that $w$ is reduced. So indeed, we can extend our constraints on $\{a_k\}_{k=1}^n$ to actual permutations on $S_{n+1},$ finishing.
\end{proof}
So we see that all of our talk about reduced words has given us the following way to look at the free group.
\begin{proposition}
	Elements of the free group are in bijection with reduced words.
\end{proposition}
\begin{proof}
	This follows from the above discussion.
\end{proof}
As a bonus, we get the following.
\begin{definition}[Residually finite]
	A group $G$ is \textit{residually finite} if, for each $g\in G\setminus\{e\},$ there exists a finite group $H$ and a homomorphism $\varphi:G\to H$ such that $\varphi(g)\ne e.$
\end{definition}
\begin{proposition}
	If $g\in F$ the free group on $\{a_k\}_{k=1}^n,$ then if $g\ne e,$ then we can find some finite group $H$ such that $g$ does not go to $e\in H.$
\end{proposition}
\begin{proof}
	Indeed, in the proof of \autoref{lem:reduced}, we showed that we can take $H=S_{n+1}.$
\end{proof}
\begin{nex}
	The rational numbers $\QQ$ is residually finite: given any finite group $H$ with a map $\varphi:\QQ\to H,$ we claim that $\varphi$ is the trivial map. Indeed, for any element $\tfrac nm,$ we claim that $\varphi(\tfrac nm)=e,$ for
	\[\varphi\left(\frac nm\right)=\varphi\left(\#H\cdot\frac n{m\#H}\right)=\varphi\left(\frac n{m\#H}\right)^{\#H}=e,\]
	where the last equality is by Lagrange's theorem.
\end{nex}
\begin{remark}
	There is a terrible way to define the free group on reduced words by brute force defining the multiplication law on reduced words, simply cancelling out neighbors. This gets bad when trying to check associativity: for example, we have to keep track of how to associate
	\[(ab)(b^{-1})(a^{-1}b).\]
\end{remark}
While we're here, we present an alternate proof of \autoref{lem:reduced}. The idea is to find a set which $G$ acts on and then show that every element acts nontrivially. We choose a Cayley graph; as a warning they are large graphs. For example, here is what our Cayley graph looks like for the free group $F_2$ generated by $\{a,b\}.$
\begin{itemize}
	\item We start by writing down vertices for each word consisting of only $a$s, connecting $a^k\to a^{k+1}$ by a directed red edge, as follows.
	% https://q.uiver.app/?q=WzAsNyxbMywwLCJlIl0sWzQsMCwiYSJdLFsyLDAsImFeey0xfSJdLFs1LDAsImFeMiJdLFsxLDAsImFeey0yfSJdLFs2LDAsIlxcY2RvdHMiXSxbMCwwLCJcXGNkb3RzIl0sWzAsMSwiIiwwLHsiY29sb3VyIjpbMCw2MCw2MF19XSxbMiwwLCIiLDAseyJjb2xvdXIiOlswLDYwLDYwXX1dLFsxLDMsIiIsMCx7ImNvbG91ciI6WzAsNjAsNjBdfV0sWzQsMiwiIiwwLHsiY29sb3VyIjpbMCw2MCw2MF19XSxbMyw1LCIiLDAseyJjb2xvdXIiOlswLDYwLDYwXX1dLFs2LDQsIiIsMCx7ImNvbG91ciI6WzAsNjAsNjBdfV1d
	\[\begin{tikzcd}
		\cdots & {a^{-2}} & {a^{-1}} & e & a & {a^2} & \cdots
		\arrow[color={rgb,255:red,214;green,92;blue,92}, from=1-4, to=1-5]
		\arrow[color={rgb,255:red,214;green,92;blue,92}, from=1-3, to=1-4]
		\arrow[color={rgb,255:red,214;green,92;blue,92}, from=1-5, to=1-6]
		\arrow[color={rgb,255:red,214;green,92;blue,92}, from=1-2, to=1-3]
		\arrow[color={rgb,255:red,214;green,92;blue,92}, from=1-6, to=1-7]
		\arrow[color={rgb,255:red,214;green,92;blue,92}, from=1-1, to=1-2]
	\end{tikzcd}\]
	\item Next, for each $a^k,$ we add in vertices $a^kb^\ell,$ where our edges are directed blue edges connecting $a^kb^{\ell}\to a^kb^{\ell+1},$ as follows.
	\[\begin{tikzcd}
		& \vdots & \vdots & \vdots & \vdots & \vdots \\
		& {a^{-2}b^2} & {a^{-1}b^2} & {b^2} & {ab^2} & {a^2b^2} \\
		& {a^{-2}b} & {a^{-1}b} & b & ab & {a^2b} \\
		\cdots & {a^{-2}} & {a^{-1}} & e & a & {a^2} & \cdots \\
		& {a^{-2}b^{-1}} & {a^{-1}b^{-1}} & {b^{-1}} & {ab^{-1}} & {a^2b^{-1}} \\
		& \vdots & \vdots & \vdots & \vdots & \vdots
		\arrow[draw={rgb,255:red,214;green,92;blue,92}, from=4-4, to=4-5]
		\arrow[draw={rgb,255:red,214;green,92;blue,92}, from=4-3, to=4-4]
		\arrow[draw={rgb,255:red,214;green,92;blue,92}, from=4-5, to=4-6]
		\arrow[draw={rgb,255:red,214;green,92;blue,92}, from=4-2, to=4-3]
		\arrow[draw={rgb,255:red,214;green,92;blue,92}, from=4-6, to=4-7]
		\arrow[draw={rgb,255:red,214;green,92;blue,92}, from=4-1, to=4-2]
		\arrow[color={rgb,255:red,92;green,92;blue,214}, from=4-4, to=3-4]
		\arrow[color={rgb,255:red,92;green,92;blue,214}, from=3-4, to=2-4]
		\arrow[color={rgb,255:red,92;green,92;blue,214}, from=4-5, to=3-5]
		\arrow[color={rgb,255:red,92;green,92;blue,214}, from=2-4, to=1-4]
		\arrow[color={rgb,255:red,92;green,92;blue,214}, from=2-5, to=1-5]
		\arrow[color={rgb,255:red,92;green,92;blue,214}, from=3-5, to=2-5]
		\arrow[color={rgb,255:red,92;green,92;blue,214}, from=2-6, to=1-6]
		\arrow[color={rgb,255:red,92;green,92;blue,214}, from=4-6, to=3-6]
		\arrow[color={rgb,255:red,92;green,92;blue,214}, from=3-6, to=2-6]
		\arrow[color={rgb,255:red,92;green,92;blue,214}, from=4-3, to=3-3]
		\arrow[color={rgb,255:red,92;green,92;blue,214}, from=4-2, to=3-2]
		\arrow[color={rgb,255:red,92;green,92;blue,214}, from=3-2, to=2-2]
		\arrow[color={rgb,255:red,92;green,92;blue,214}, from=2-2, to=1-2]
		\arrow[color={rgb,255:red,92;green,92;blue,214}, from=2-3, to=1-3]
		\arrow[color={rgb,255:red,92;green,92;blue,214}, from=3-3, to=2-3]
		\arrow[color={rgb,255:red,92;green,92;blue,214}, from=6-2, to=5-2]
		\arrow[color={rgb,255:red,92;green,92;blue,214}, from=5-2, to=4-2]
		\arrow[color={rgb,255:red,92;green,92;blue,214}, from=5-3, to=4-3]
		\arrow[color={rgb,255:red,92;green,92;blue,214}, from=6-3, to=5-3]
		\arrow[color={rgb,255:red,92;green,92;blue,214}, from=5-4, to=4-4]
		\arrow[color={rgb,255:red,92;green,92;blue,214}, from=6-4, to=5-4]
		\arrow[color={rgb,255:red,92;green,92;blue,214}, from=5-5, to=4-5]
		\arrow[color={rgb,255:red,92;green,92;blue,214}, from=6-5, to=5-5]
		\arrow[color={rgb,255:red,92;green,92;blue,214}, from=5-6, to=4-6]
		\arrow[color={rgb,255:red,92;green,92;blue,214}, from=6-6, to=5-6]
	\end{tikzcd}\]
	\item Then at each $a^kb^\ell,$ we add in another red line using the same joining rules. This roughly looks like the following; to avoid crowding, we choose one small part.
	% https://q.uiver.app/?q=WzAsMTAsWzIsMiwiZSJdLFsyLDEsImIiXSxbMywxLCJiYSJdLFs0LDAsIlxcaWRkb3RzIl0sWzEsMSwiYmFeey0xfSJdLFsyLDAsIlxcdmRvdHMiXSxbMiwzLCJcXHZkb3RzIl0sWzAsMCwiXFxkZG90cyJdLFs0LDIsIlxcY2RvdHMiXSxbMCwyLCJcXGNkb3RzIl0sWzAsMSwiIiwwLHsiY29sb3VyIjpbMjQwLDYwLDYwXX1dLFsxLDIsIiIsMCx7ImNvbG91ciI6WzAsNjAsNjBdfV0sWzIsMywiIiwwLHsiY29sb3VyIjpbMCw2MCw2MF19XSxbMSw1LCIiLDAseyJjb2xvdXIiOlsyNDAsNjAsNjBdfV0sWzYsMCwiIiwwLHsiY29sb3VyIjpbMjQwLDYwLDYwXX1dLFs3LDQsIiIsMCx7ImNvbG91ciI6WzAsNjAsNjBdfV0sWzQsMSwiIiwwLHsiY29sb3VyIjpbMCw2MCw2MF19XSxbMCw4LCIiLDAseyJjb2xvdXIiOlswLDYwLDYwXX1dLFs5LDAsIiIsMCx7ImNvbG91ciI6WzAsNjAsNjBdfV1d
	\[\begin{tikzcd}
		\ddots && \vdots && \iddots \\
		& {b a^{-1}} & b & b a \\
		\cdots && e && \cdots \\
		&& \vdots
		\arrow[color={rgb,255:red,92;green,92;blue,214}, from=3-3, to=2-3]
		\arrow[color={rgb,255:red,214;green,92;blue,92}, from=2-3, to=2-4]
		\arrow[color={rgb,255:red,214;green,92;blue,92}, from=2-4, to=1-5]
		\arrow[color={rgb,255:red,92;green,92;blue,214}, from=2-3, to=1-3]
		\arrow[color={rgb,255:red,92;green,92;blue,214}, from=4-3, to=3-3]
		\arrow[color={rgb,255:red,214;green,92;blue,92}, from=1-1, to=2-2]
		\arrow[color={rgb,255:red,214;green,92;blue,92}, from=2-2, to=2-3]
		\arrow[color={rgb,255:red,214;green,92;blue,92}, from=3-3, to=3-5]
		\arrow[color={rgb,255:red,214;green,92;blue,92}, from=3-1, to=3-3]
	\end{tikzcd}\]
	\item Then we can continue adding layers to the above graph, building a giant monstrosity recursively.
\end{itemize}
With this example in mind, here is the general case.
\begin{proof}[Proof of \autoref{lem:reduced}]
	We build our graph $X$ as with vertices which are reduced words and add a directed edge $w_1\to w_2$ of color $k$ if and only if $w_1a_k=w_2.$ In particular, an outgoing edge implies that the length of the word is strictly increasing; rigorously, we would build $X$ recursively as in the example (to make sure $X$ is a tree), but we will not bother here.
	
	Then the action of $g\in G$ on $X$ consists of sending vertices $v\in X$ to $gv$; it's not hard to check that this is in $\op{Aut}(X),$ and in fact $g\ne e$ yields a nontrivial element of $\op{Aut}(X)$ because the empty word is taken to $g$ in $X.$
\end{proof}
This sort of process turns out to be easier for free abelian groups. For example, the Cayley graph for the free abelian group on $\{a,b\}$ looks like the following. (Add dimensions with more letters.)
% https://q.uiver.app/?q=WzAsMjEsWzIsMiwiXFxidWxsZXQiXSxbMywyLCJcXGJ1bGxldCJdLFs0LDIsIlxcY2RvdHMiXSxbMSwyLCJcXGJ1bGxldCJdLFswLDIsIlxcY2RvdHMiXSxbMiwxLCJcXGJ1bGxldCJdLFsyLDMsIlxcYnVsbGV0Il0sWzMsMywiXFxidWxsZXQiXSxbMSwzLCJcXGJ1bGxldCJdLFswLDMsIlxcY2RvdHMiXSxbNCwzLCJcXGNkb3RzIl0sWzAsMSwiXFxjZG90cyJdLFsxLDEsIlxcYnVsbGV0Il0sWzMsMSwiXFxidWxsZXQiXSxbNCwxLCJcXGNkb3RzIl0sWzEsNCwiXFx2ZG90cyJdLFsyLDQsIlxcdmRvdHMiXSxbMyw0LCJcXHZkb3RzIl0sWzMsMCwiXFx2ZG90cyJdLFsyLDAsIlxcdmRvdHMiXSxbMSwwLCJcXHZkb3RzIl0sWzAsMSwiYSIsMCx7ImNvbG91ciI6WzAsNjAsNjBdfSxbMCw2MCw2MCwxXV0sWzEsMiwiIiwwLHsiY29sb3VyIjpbMCw2MCw2MF19XSxbMywwLCIiLDAseyJjb2xvdXIiOlswLDYwLDYwXX1dLFs0LDMsIiIsMCx7ImNvbG91ciI6WzAsNjAsNjBdfV0sWzAsNSwiYiIsMCx7ImNvbG91ciI6WzI0MCw2MCw2MF19LFsyNDAsNjAsNjAsMV1dLFs2LDAsIiIsMCx7ImNvbG91ciI6WzI0MCw2MCw2MF19XSxbNiw3LCIiLDAseyJjb2xvdXIiOlswLDYwLDYwXX1dLFs3LDEsIiIsMCx7ImNvbG91ciI6WzI0MCw2MCw2MF19XSxbOCw2LCIiLDAseyJjb2xvdXIiOlswLDYwLDYwXX1dLFs4LDMsIiIsMCx7ImNvbG91ciI6WzI0MCw2MCw2MF19XSxbOSw4LCIiLDAseyJjb2xvdXIiOlswLDYwLDYwXX1dLFs3LDEwLCIiLDAseyJjb2xvdXIiOlswLDYwLDYwXX1dLFszLDEyLCIiLDAseyJjb2xvdXIiOlsyNDAsNjAsNjBdfV0sWzEsMTMsIiIsMCx7ImNvbG91ciI6WzI0MCw2MCw2MF19XSxbMTEsMTIsIiIsMCx7ImNvbG91ciI6WzAsNjAsNjBdfV0sWzEyLDUsIiIsMCx7ImNvbG91ciI6WzAsNjAsNjBdfV0sWzUsMTMsIiIsMCx7ImNvbG91ciI6WzAsNjAsNjBdfV0sWzEzLDE0LCIiLDAseyJjb2xvdXIiOlswLDYwLDYwXX1dLFsxNSw4LCIiLDAseyJjb2xvdXIiOlsyNDAsNjAsNjBdfV0sWzE2LDYsIiIsMCx7ImNvbG91ciI6WzI0MCw2MCw2MF19XSxbMTcsNywiIiwwLHsiY29sb3VyIjpbMjQwLDYwLDYwXX1dLFsxMywxOCwiIiwwLHsiY29sb3VyIjpbMjQwLDYwLDYwXX1dLFs1LDE5LCIiLDAseyJjb2xvdXIiOlsyNDAsNjAsNjBdfV0sWzEyLDIwLCIiLDAseyJjb2xvdXIiOlsyNDAsNjAsNjBdfV1d
\[\begin{tikzcd}
	& \vdots & \vdots & \vdots \\
	\cdots & \bullet & \bullet & \bullet & \cdots \\
	\cdots & \bullet & \bullet & \bullet & \cdots \\
	\cdots & \bullet & \bullet & \bullet & \cdots \\
	& \vdots & \vdots & \vdots
	\arrow["a", color={rgb,255:red,214;green,92;blue,92}, from=3-3, to=3-4]
	\arrow[color={rgb,255:red,214;green,92;blue,92}, from=3-4, to=3-5]
	\arrow[color={rgb,255:red,214;green,92;blue,92}, from=3-2, to=3-3]
	\arrow[color={rgb,255:red,214;green,92;blue,92}, from=3-1, to=3-2]
	\arrow["b", color={rgb,255:red,92;green,92;blue,214}, from=3-3, to=2-3]
	\arrow[color={rgb,255:red,92;green,92;blue,214}, from=4-3, to=3-3]
	\arrow[color={rgb,255:red,214;green,92;blue,92}, from=4-3, to=4-4]
	\arrow[color={rgb,255:red,92;green,92;blue,214}, from=4-4, to=3-4]
	\arrow[color={rgb,255:red,214;green,92;blue,92}, from=4-2, to=4-3]
	\arrow[color={rgb,255:red,92;green,92;blue,214}, from=4-2, to=3-2]
	\arrow[color={rgb,255:red,214;green,92;blue,92}, from=4-1, to=4-2]
	\arrow[color={rgb,255:red,214;green,92;blue,92}, from=4-4, to=4-5]
	\arrow[color={rgb,255:red,92;green,92;blue,214}, from=3-2, to=2-2]
	\arrow[color={rgb,255:red,92;green,92;blue,214}, from=3-4, to=2-4]
	\arrow[color={rgb,255:red,214;green,92;blue,92}, from=2-1, to=2-2]
	\arrow[color={rgb,255:red,214;green,92;blue,92}, from=2-2, to=2-3]
	\arrow[color={rgb,255:red,214;green,92;blue,92}, from=2-3, to=2-4]
	\arrow[color={rgb,255:red,214;green,92;blue,92}, from=2-4, to=2-5]
	\arrow[color={rgb,255:red,92;green,92;blue,214}, from=5-2, to=4-2]
	\arrow[color={rgb,255:red,92;green,92;blue,214}, from=5-3, to=4-3]
	\arrow[color={rgb,255:red,92;green,92;blue,214}, from=5-4, to=4-4]
	\arrow[color={rgb,255:red,92;green,92;blue,214}, from=2-4, to=1-4]
	\arrow[color={rgb,255:red,92;green,92;blue,214}, from=2-3, to=1-3]
	\arrow[color={rgb,255:red,92;green,92;blue,214}, from=2-2, to=1-2]
\end{tikzcd}\]
So this free abelian group has Cayley graph which fits nicely in Euclidean space. On the other hand, the free group fits nicely on hyperbolic space, which itself does not fit nicely on Euclidean space.
\begin{remark}
	In some sense, the free group is approximately the size of hyperbolic space in the same way that the free abelian group is approximately the size of Euclidean space.
\end{remark}

\subsection{Free Groups as Fundamental Groups}
Let's talk about some properties of free groups, in the same way we talked about free abelian groups. For example, we still have the following.
\begin{proposition}
	The rank of a free group is well-defined.
\end{proposition}
\begin{proof}
	Again, given a free group on $n$ letters named $F,$ exactly the same argument as in \autoref{prop:freeabprops} shows that there are $2^n$ homomorphisms $F\to\ZZ/2\ZZ.$ It follows that two free groups are isomorphic if and only if they are generated by sets of the same cardinality.
\end{proof}
However, rank does not behave the way we might want it to. For example, intuitively the free group on three elements ought to be larger than the free group on two elements, and indeed, for abelian groups, there is no injective homomorphism $\ZZ^3\to\ZZ^2.$\footnote{The basis ``vectors'' of $\ZZ^3$ go to some three ``vectors'' in $\ZZ^2,$ but any three vectors in $\ZZ^2\subseteq\QQ^2$ have a nontrivial $\QQ$-linear relation for dimension reasons, which can be lifted to a nontrivial $\ZZ$-linear relation by clearing denominators.} But life is not so good with general free groups.
\begin{restatable}{proposition}{ftwohasfthree} \label{prop:f2big}
	There is an injective homomorphism from the free group on three letters to the free group on two letters.
\end{restatable}
Regardless, we will still be able to show the following, albeit with more effort.
\begin{restatable}{theorem}{freesubgrp} \label{thm:freesub}
	Any subgroup of a free group is free.
\end{restatable}
The main idea of this theorem is to show that $G$ is free if and only if $G$ is the fundamental group of some graph. So we begin by defining the fundamental group.
\begin{definition}[Circuit]
	Given a graph $X,$ a \textit{circuit} is an alternating sequence of vertices and adjacent edges of $X,$ say $x_1e_1x_2e_2\ldots e_{n-1}x_n,$ such that $x_n=x_1.$
\end{definition}
\begin{warn}
	We are going to direct our edges but be sloppy about it: every edge $e:v\to w$ will have a designated inverse edge $e^{-1}:w\to v.$ Because our circuits are also tracking the endpoints where we move, this does not matter most of the time, but it does matter for loops. The short version is that we need to keep track of ``which'' way we move along loops but no other edges, so I will not keep careful track of which way we move along non-loop edges (and so may write $e=e^{-1}$ when $v\ne w$).
\end{warn}
\begin{definition}[Homotopy]
	Fix a graph $X$ and a basepoint $x_0\in X.$ We define the equivalence class $\equiv$ on circuits starting and ending at $x_0\in X$ by asserting that, if $x_0x_1\cdots x_nx_0$ is a circuit, then
	\[x_0e_0x_1e_1\cdots x_ne_nx_0\equiv x_0e_0x_1e_1\cdots x_k{\color{red}eye^{-1}}x_ke_k\cdots x_n,\tag{$*$}\]
	and closing this under the requirements to be an equivalence relation. (Any two paths with that can be translated into each other using a finite number of these moves are equivalent.) Here $e^{-1}$ explicitly means moving backwards along the edge $e.$

	If two circuits are equivalent under $\equiv,$ then we say that they are \textit{homotopic}. Also, for brevity, we will call a move of the form $(*)$ a ``back-and-forth move.''
\end{definition}
Essentially, two paths are homotopic if there is some ``back and forth'' steps we can optimize out to make the paths equivalent to each other. As an example, consider the following graph $X.$
% https://tikzcd.yichuanshen.de/#N4Igdg9gJgpgziAXAbVABwnAlgFyxMJZARgBoAmAXVJADcBDAGwFcYkQAPAfQAYQBfUuky58hFOVLFqdJq3bcAzAKEgM2PASKSqNBizaJOXYiuEaxRMjxn75R7gBYzakZvHIeU23MPHyAjIwUADm8ESgAGYAThAAtkheIDgQSJIgABYw9FDskGBsNIz0AEYwjAAKbpZGWGDYsCB6vuwwvC4x8UiKNClIjjRZOXkEhbIGrVwArE0gxWWV1Vq19ViNglGxCYgDyamISUO5Rvlj8+VVFssgdQ1jdn5tARsgndtJfYhkmdnH4KOzB6TZRFUoXJbiG6rdaqN7dXr7HpzCAQNBWAAcXkiTDgMBk50WV0htzWYyOIwKgJaRjaADYOls0gimcjURisTi8aCFpdRNcSY1Br8KfdqSA2qYXnCvsydkLhicAdzwUT2ALRRMaVxnPxKPwgA
\[\begin{tikzcd}
	& x_4 \arrow[ld, "e_5" description, no head] \arrow["e_6" description,loop, distance=2em, in=125, out=55] &                                                                            \\
x_2 \arrow[rd, "e_2" description, no head] &                                                                                                                   & x_3 \arrow[ll, "e_4" description, no head]                                 \\
	& x_0 \arrow[r, "e_0" description, no head] \arrow[ru, "e_3" description, no head]                                  & x_1 \arrow["e_1" description, loop, distance=2em, in=125, out=55]
\end{tikzcd}\]
In this graph, the circuits $x_0e_2x_2e_4x_3e_3x_0$ is homotopic with $x_0e_2x_2{\color{red}e_5x_4e_5}x_2e_4x_3e_3x_0$ but is not homotopic with $x_0e_2x_2{\color{red}e_5x_4e_6x_4e_5}x_2e_4x_3e_3x_0.$

Now, our fundamental group is more or less the same thing as seen in algebraic topology, but we will define this algebraically.
\begin{definition}[Fundamental group]
	Suppose that $X$ is a connected graph, with loops permitted but not multiple edges, with a particular basepoint $x_0\in X.$ Then the \textit{fundamental group} of $(X,x_0),$ notated $\pi_1(X,x_0),$ consists of the circuits around $x_0$ up to homotopy.

	Given a circuit $C$ starting and ending at $x_0,$ we will denote its homotopy equivalence class by $[C].$
\end{definition}
Note that we have not actually made our fundamental group into a group yet. Our composition law will be composition of circuits (follow the first circuit; then follow the second), but it might feel like we don't need homotopy for this. It turns out that homotopy is what makes this group law a group, giving us inverses.
\begin{lemma}
	Fix $X$ a connected graph and $x_0\in X$ a basepoint. Then $\pi_1(X,x_0)$ is a group.
\end{lemma}
\begin{proof}
	As promised, our group law is composition of circuits: write down the first circuit, subtract its last $x_0$ to avoid duplicates, and then write down the second circuit. We check the group conditions by hand.
	\begin{itemize}
		\item We show that the group law is well-defined: given circuits $C_1\equiv C_2$ and $D_1\equiv D_2,$ we have to show that $C_1D_1=C_2D_2.$ Well, it requires finitely many back-and-forth moves to turn $C_1$ into $C_2$ and finitely many moves to turn $D_1$ into $D_2,$ so it still requires finitely many moves to turn $C_1D_1$ into $C_2D_2.$
		\item Associative: given circuits $C_1,C_2,C_3,$ we see that concatenating $C_1$ with $C_2$ first and then with $C_3$ is the same total string as concatenating $C_2$ with $C_3$ and then concatenating $C_1$ at the front.
		\item Identity: our identity is the do-nothing circuit ``$x_0.$'' Concatenating with it does nothing.
		\item Inverse: reversing a circuit $C:=x_0e_0x_1e_1\cdots x_ne_nx_0$ to $C^{-1}:=x_0e_n^{-1}x_n\cdots e_1^{-1}x_1e_1^{-1}x_0$ gives its inverse. Indeed, concatenating gives
		\begin{align*}
		C\cdot C^{-1} &\equiv x_0e_0x_1e_1\cdots x_{n-1}e_{n-1}x_n{\color{red}e_n x_0e_n^{-1}} x_ne_{n-1}^{-1}x_{n-1}\cdots e_1^{-1}x_1e_0^{-1}x_0 \\
		&\equiv x_0e_0x_1e_1\cdots x_{n-1}{\color{red}e_{n-1}x_ne_{n-1}^{-1}}x_{n-1}\cdots e_1^{-1}x_1e_0^{-1}x_0 \\
		&\equiv \cdots \\
		&\equiv x_0{\color{red}e_0x_1e_0^{-1}}x_0 \\
		&\equiv x_0,
		\end{align*}
		which is our identity.
		\qedhere
	\end{itemize}
\end{proof}
We now begin moving towards our description of fundamental groups.
\begin{lemma} \label{lem:freeisfun}
	The free group $F$ on $\{a_k\}_{k=1}^n$ is a fundamental group.
\end{lemma}
\begin{proof}
	The idea is to assign $a_k$ to a loop around the basepoint so that reduced words roughly correspond to unique homotopy classes. Explicitly, we choose a graph $X$ with one vertex $x_0$ and $n$ different loops named $a_1,\ldots,a_k.$ This looks like the following.
	% https://tikzcd.yichuanshen.de/#N4Igdg9gJgpgziAXAbVABwnAlgFyxMJZABgBpiBdUkANwEMAbAVxiRAA8B9YkAXypAwoAc3hFQAMwBOEALZIyIHBAXUGECGiJkJjODAEM6AIxgMACplz5CiEFjDZYIavWas7dTgEYXIABYwdFBskGCsvKSSMvKIisqqIOqaRACcOnoGaiZmlth4BGwOTqyujCxsXgBMfoHBoQQRUSDScokJcWoaWijeABwZDPqGORZWBbb2jljOZe6VnADMtUEhdmFN0W2dSio7yT3IVQDsg8PZpmP5NkXTs7TlHiAAOs8AxlAQOAjUdWvgjT4FF4QA
	\[\begin{tikzcd}
		x_0 \arrow["a_1" description, no head, loop, distance=2em, in=305, out=235] \arrow["a_2" description, no head, loop, distance=2em, in=35, out=325] \arrow["a_3" description, no head, loop, distance=2em, in=125, out=55] \arrow["\cdots" description, no head, loop, distance=2em, in=215, out=145]
	\end{tikzcd}\]
	We would like to show that $\pi(X,x_0)\cong F,$ where $F$ is the free group on $\{a_k\}_{k=1}^n.$ Because any circuit on $X$ must have $x_0$ as its only vertex, so all of our circuits look like
	\[x_0\ell_0 x_0\ell_1\cdots x_n\ell_nx_0\]
	for some letters $\ell_\bullet\in\{a_k\}_{k=1}^n\cup\{a_k^{-1}\}_{k=1}^n.$ Now, the point is that we have a function $\varphi$ from circuits to $F$ by
	\[\varphi:x_0\ell_0 x_0\ell_1\cdots x_n\ell_nx_0\longmapsto\ell_0\ell_1\cdots\ell_n.\]
	We can show that this is a group isomorphism $\pi_1(X,x_0)\to F,$ which we do by hand.
	\begin{itemize}
		\item We can show that $\varphi$ satisfies $\varphi(C_1C_2)=\varphi(C_1)\varphi(C_2)$ on circuits $C_1,C_2.$ This follows directly from the fact $C_1C_2$ corresponds to concatenation, as does $\varphi(C_1)\varphi(C_2).$
		\item The main obstruction is showing that $\varphi$ is well-defined. By induction, it suffices to show that if two circuits differ by a single back-and-forth move give the same element of $F.$ Well, by the previous part, we can take
		\[\varphi(x_0\ell_0x_0\ell_1\cdots x_k{\color{red}\ell x_0\ell^{-1}}x_k\ell_k\cdots x_0)\]
		to
		\[\varphi(x_0\ell_0x_0\ell_1\cdots x_k)\varphi(x_k{\color{red}\ell x_0\ell^{-1}}x_k)\varphi(x_k\ell_k\cdots x_0),\]
		which is indeed $\varphi(x_0\ell_0x_0\ell_1\cdots x_k)\varphi(x_k\ell_k\cdots x_0)=\varphi(x_1\ell_0\cdots x_n).$
		\item Surjectivity is relatively apparent: pull a word of $w\in F$ directly backwards by punctuating it with $x_0.$
		\item Injectivity is also difficult. In short, we can reduce a circuit by removing any stray $\ell x_0\ell^{-1}$ terms by a back-and-forth move. Then nontrivial homotopy classes will map to nontrivial reduced words, so nontrivial homotopy classes are not in the kernel. So the kernel is trivial.
		\qedhere
	\end{itemize}
\end{proof}
And here is the other direction.
\begin{lemma}
	Any fundamental group of a finite, connected graph is a free group. In fact, given a finite, connected graph and basepoint $(X,x_0)$ and a spanning tree $T,$ then $\pi_1(X,x_0)$ is isomorphic to the free group on the edges of $X\setminus T.$
\end{lemma}
\begin{proof}
	The idea is to contract edges by homotopy. For example, we can take
	% https://tikzcd.yichuanshen.de/#N4Igdg9gJgpgziAXAbVABwnAlgFyxMJZARgBoAmAXVJADcBDAGwFcYkQAPAfQAYQBfUuky58hFOVLFqdJq3bcAzAKEgM2PASKSqNBizaJOXYiuEaxRMjxn75R7gBYzakZvHIeU23MPHyAjIwUADm8ESgAGYAThAAtkheIDgQSJIgABYw9FDskGBsNIz0AEYwjAAKbpZGWGDYsCB6vuwwvC4x8UiKNClIjjRZOXkEhbIGrVwArE0gxWWV1Vq19ViNglGxCYgDyamISUO5Rvlj8+VVFssgdQ1jdn5tARsgndtJfYhkmdnH4KOzB6TZRFUoXJbiG6rdaqN7dXr7HpzCAQNBWAAcXkiTDgMBk50WV0htzWYyOSDAzEYjGaEyMbQAbB0tmkEazkaiMVicXjQQtLqJriTGoNfhSqTTxvYQG1TC84V82TtRcMTgC+eCiexhfcWvSuM5+JR+EA
	\[\begin{tikzcd}
		& x_4 \arrow[ld, "e_5" description, no head] \arrow["e_6" description, loop, distance=2em, in=125, out=55] &                                                                   \\
	x_2 \arrow[rd, "e_2" description, no head] &                                                                                                          & x_3 \arrow[ll, "e_4" description, no head]                        \\
		& x_0 \arrow[r, "e_0" description, no head] \arrow[ru, "e_3" description, no head]                         & x_1 \arrow["e_1" description, loop, distance=2em, in=125, out=55]
	\end{tikzcd}\]
	to
	% https://tikzcd.yichuanshen.de/#N4Igdg9gJgpgziAXAbVABwnAlgFyxMJZARgBpiBdUkANwEMAbAVxiRAA8B9ABhAF9S6TLnyEUAJlLcqtRizZcAzP0EgM2PASKTK1es1aIOnYiqEbRRblJn75RruP4yYUAObwioAGYAnCAC2SNYgOBBIkiAAFjB0UGyQYKzUDHQARjAMAArCmmIgWGDYsCB6coYgMDxmIH6BSIrUYcHUMXEJBMkgqRnZuZZGhcVddhVVTgI+-kGIIc2IZNGx8UaJI+VsVcop6Zk5FlqDRVglk7XTEU3hiJEMEBBoRMQAHNbejHAwMj17-YcFxxKrWWSDATAYDDKBk2JhqdRmi3mjSW7VWnVK3V2fQO+SGJ3W0KMVQALHCLohkUiUvdHigXm8Pl8dr19iJ-nigbJCZVOAA2Zx8IA
	\[\begin{tikzcd}
		x_2=x_4 \arrow[rd, "e_2" description, no head] \arrow["e_6" description, loop, distance=2em, in=125, out=55] &                                                                                  & x_3 \arrow[ll, "e_4" description, no head]                        \\
		& x_0 \arrow[r, "e_0" description, no head] \arrow[ru, "e_3" description, no head] & x_1 \arrow["e_1" description, loop, distance=2em, in=125, out=55]
	\end{tikzcd}\]
	by contracting along the edge $e_5.$ The main thing we need to show is that the fundamental group does not change after contracting along a non-loop edge.

	Indeed, given a graph $X$ with basepoint $x_0\in X,$ and non-loop edge $e$ connecting $v_1$ and $v_2,$ we construct the contracted graph $X/e$ by deleting $e$ and declaring $v_1=v_2.$ It remains to show $\pi_1(X,x_0)\cong\pi_1(X/e,x_0).$ We construct $\varphi:\pi_1(X,x_0)\to\pi_1(X/e,x_0)$ by taking a circuit
	\[x_0e_0x_1e_1\cdots x_n\]
	and deleting any occurrence of $e$ while replacing each $v_2$ with a $v_1.$ We need to show that $\varphi$ is a homomorphism; this is generally annoying but visually makes sense, so we outline.
	\begin{itemize}
		\item Homomorphic on paths: we won't be rigorous about this. Essentially, concatenating two paths and then contracting the paths about $e$ to some reduced word is the same as contracting the paths about $e$ first and then concatenating.
		\item Well-defined: we can use the fact we are homomorphic on paths. Any back-and-forth move can be isolated from the rest of the circuit in the same way as in the proof of \autoref{lem:freeisfun} so that $\varphi$ is well-defined up to back-and-forth moves. Thus, $\varphi$ is well-defined up to homotopy.
		\item Surjective: any path on $X/e$ can be lifted to a path of $X$ by, roughly speaking, just following the path in $X/e$ in $X.$ Any time we hit $v_1$ or $v_2$ in $X/e,$ the next edge in $X$ in the path might not adjacent to our current $v_1$ or $v_2,$ but it will be adjacent to one of $\{v_1,v_2\},$ so we can use $e$ to cross between with no repercussions from $\varphi.$
		\item Injective: essentially, it suffices to show that applying a back-and-forth move to a circuit in $X/e$ does not affect its lift described to $X$ described in the surjectivity. Because our lift essentially just follows the circuit in $X/e$ with minor adjustments around $e,$ a back-and-forth move will lift directly to a back-and-forth move, so the lift is well-defined up to homotopy.
	\end{itemize}

	To finish, we fix $T$ a spanning tree of $X.$ Recursively applying contraction along the edges of $T$ will eventually\footnote{Technically, we have to show that the edges of the spanning tree never become loops when contracted. Well, contraction really just declares vertices equal, so the only way to have a loop would be to have a circuit in our spanning tree.} leave us with a single vertex and $\#E(X)-\#E(T)$ loops around our basepoint. So we see $\pi_1(X,x_0)$ is isomorphic to the free group on $\#E(X)-\#E(T)$ letters from \autoref{lem:freeisfun}.
\end{proof}
\begin{remark}
	Technically, we can extend the above argument to work for all connected graphs, but this requires more technical effort. Essentially, given a connected graph $X$ and spanning tree $T,$ we can mod $X$ by the entire tree $T$ in one blow. I am under the impression that the same arguments that work for a single edge generalize.
\end{remark}
We now show our theorem.
\freesubgrp*
\begin{proof}[Proof of \autoref{thm:freesub}]
	Suppose that $G\subseteq F$ is a subgroup of a free group. Then we define the graph $X$ whose points are the cosets in $F/G$ and the edges are the actions of the generators of $F.$ Note that $F$ acts transitively on $F/G,$ so $X$ is connected; we choose $eG$ as our basepoint.

	We now claim that $G$ is $\pi_1(X,eG),$ which will be sufficient because fundamental groups are free. Essentially, the idea is that we can map words $w=\prod_{k=1}^N\ell_k\in G$ to the circuit of $X$ starting at $eG$ and following $\ell_k$ as edges:
	\[eG\stackrel{\ell_N}\to\ell_NeG\stackrel{\ell_{N-1}}\to\ell_{N-1}\ell_NeG\stackrel{\ell_{N-2}}\to\ell_{N-2}\ell_{N-1}\ell_NG\to\cdots.\]
	Then we see that the last coset we hit in the circuit is $wG,$ so the path we make is a circuit if and only if $w\in G.$ We briefly talk through the checks to show we have an isomorphism. Call this map from words to paths $\varphi.$
	\begin{itemize}
		\item Well-defined: we know that every word can be reduced to a unique reduced representative by simply recursively removing $\ell\ell^{-1}$ subword, so it suffices to show that introducing an $\ell\ell^{-1}$ does not change the homotopy class of the output. But introducing $\ell\ell^{-1}$ means inserting
		\[\cdots\to gG{\color{red}\stackrel{\ell^{-1}}\to\ell^{-1}gG\stackrel\ell\to\ell\ell^{-1}gG}\to\cdots,\]
		where we see that this is just a back-and-forth move and therefore does nothing.
		\item Homomorphic: both group laws are concatenation, and we concatenate before or after.
		\item Surjective: all of our circuits take the form
		\[eG\stackrel{\ell_N}\to\ell_NeG\stackrel{\ell_{N-1}}\to\ell_{N-1}\ell_NeG\stackrel{\ell_{N-2}}\to\ell_{N-2}\ell_{N-1}\ell_NG\to\cdots\stackrel{\ell_1}\to\underbrace{\left(\prod_{k=1}^N\ell_k\right)}_{=:w}G\]
		for some word $w,$ where $w\in G$ so that $wG=G.$ It follows that $w\in G$ maps to this circuit.
		\item Injective: in the well-defined point, we showed that back-and-forth moves correspond to removing $\ell\ell^{-1}$ substrings of words, so it follows that the inverse map introduced in the surjective point is also well-defined.
		\qedhere
	\end{itemize}
\end{proof}

\subsection{Applications of Fundamental Groups}
We are going to actually use this spanning tree contraction algorithm, so we give an example of this algorithm. We start with $X$ as above with the designated red spanning tree.
\[\begin{tikzcd}
	& x_4 \arrow[ld, "e_5" description, no head, color={rgb,255:red,214;green,92;blue,92}] \arrow["e_6" description, loop, distance=2em, in=125, out=55] &                                                                   \\
x_2 \arrow[rd, "e_2" description, no head] &                                                                                                          & x_3 \arrow[ll, "e_4" description, no head, color={rgb,255:red,214;green,92;blue,92}]                        \\
	& x_0 \arrow[r, "e_0" description, no head, color={rgb,255:red,214;green,92;blue,92}] \arrow[ru, "e_3" description, no head, color={rgb,255:red,214;green,92;blue,92}]                         & x_1 \arrow["e_1" description, loop, distance=2em, in=125, out=55]
\end{tikzcd}\]
Contracting along $e_5$ gives $X/e_5.$
% https://tikzcd.yichuanshen.de/#N4Igdg9gJgpgziAXAbVABwnAlgFyxMJZARgBpiBdUkANwEMAbAVxiRAA8B9ABhAF9S6TLnyEUAJlLcqtRizZcAzP0EgM2PASKTK1es1aIOnYiqEbRRblJn75RruP4yYUAObwioAGYAnCAC2SNYgOBBIkiAAFjB0UGyQYKzUDHQARjAMAArCmmIgWGDYsCB6coYgMDxmIH6BSIrUYcHUMXEJBMkgqRnZuZZGhcVddhVVTgI+-kGIIc2IZNGx8UaJI+VsVcop6Zk5FlqDRVglk7XTEU3hiJEMEBBoRMQAHNbejHAwMj17-YcFxxKrWWSDATAYDDKBk2JhqdRmi3mjSW7VWnVK3V2fQO+SGJ3W0KMVQALHCLohkUiUvdHigXm8Pl8dr19iJ-nigbJCZVOAA2Zx8IA
\[\begin{tikzcd}
	x_2 \arrow[rd, "e_2" description, no head] \arrow["e_6" description, loop, distance=2em, in=125, out=55] &                                                                                  & x_3 \arrow[ll, "e_4" description, no head, color={rgb,255:red,214;green,92;blue,92}]                        \\
																											 & x_0 \arrow[r, "e_0" description, no head, color={rgb,255:red,214;green,92;blue,92}] \arrow[ru, "e_3" description, no head, color={rgb,255:red,214;green,92;blue,92}] & x_1 \arrow["e_1" description, loop, distance=2em, in=125, out=55]
\end{tikzcd}\]
Contracting along $e_4$ gives $X/\{e_4,e_5\}.$
% https://tikzcd.yichuanshen.de/#N4Igdg9gJgpgziAXAbVABwnAlgFyxMJZABgBoBGAXVJADcBDAGwFcYkQAPAfWJAF9S6TLnyEU5CtTpNW7buX6CQGbHgJEyxKQxZtEnLgCZ+UmFADm8IqABmAJwgBbJGRA4ISCSAAWMelHZIMDYaRnoAIxhGAAVhNTEQLDBsWBAaHVl9GB5FWwdnRC93T1CICDQicgAOMhsmOBgpMMiYuNF2JJSQnz8AxDBmRkZ0mT0QbIUBPKckQxpixFdff0CCbsiwPoBaAGZXDLHs41CIqNjVdv1OrFSaOG8sGxxPKZB7GcQ5tw9P0vLKmqkOqMBpNU6tC7qK7JG7dA7sbIANlyb3ys3mP1cGz6exOLXOIihiRht2kugRXB2aR6K30QW690ez0QW3IfEofCAA
\[\begin{tikzcd}
	x_2 \arrow[d, "e_2" description, no head, bend right, shift right] \arrow["e_6" description, loop, distance=2em, in=125, out=55] \arrow[d, "e_3" description, no head, bend left, shift left, color={rgb,255:red,214;green,92;blue,92}] &                                                                   \\
	x_0 \arrow[r, "e_0" description, no head, color={rgb,255:red,214;green,92;blue,92}]  & x_1 \arrow["e_1" description, loop, distance=2em, in=125, out=55]
\end{tikzcd}\]
Contracting along $e_3$ gives $X/\{e_3,e_4,e_5\}.$
% https://tikzcd.yichuanshen.de/#N4Igdg9gJgpgziAXAbVABwnAlgFyxMJZABgBpiBdUkANwEMAbAVxiRAA8B9YkAX1PSZc+QigCM5KrUYs2XMXykwoAc3hFQAMwBOEALZIyIHBCQSQACxh0obSGFbUGdAEYwGABSF4CbLGGxYEGp6ZlZEEBhuPgEQHX0zahNEkAYICDQiMQAOMk1GOBgpZzdPbxE-AKwg6isbJDAmBgYQmXDIzgV+LV0DRCNk-qd0zPFc0nyGQuLXdy9sH1EQf0DHaTC2KIA2GJ6EoeNTA7SMogAmAHY8gqKnWbKFioiV6rXQ2Qios8VeIA
\[\begin{tikzcd}
	x_0 \arrow[r, "e_0" description, no head, color={rgb,255:red,214;green,92;blue,92}] \arrow["e_6" description, loop, distance=2em, in=125, out=55] \arrow["e_2" description, loop, distance=2em, in=215, out=145] & x_1 \arrow["e_1" description, loop, distance=2em, in=125, out=55]
\end{tikzcd}\]
Lastly, contracting along $e_0$ gives $X/\{e_0,e_3,e_4,e_5\}.$
% https://tikzcd.yichuanshen.de/#N4Igdg9gJgpgziAXAbVABwnAlgFyxMJZABgBpiBdUkANwEMAbAVxiRAA8B9YkAXypAwoAc3hFQAMwBOEALZIyIHBAXUGECGiIBGABxkJjODAEM6AIxgMACplz5CiEFjDZYIavWasnMTgDY+UkkZeURFZVUQdU0iACYAdgMjEzULK1tsPAI2FzdWT0YWNj84oJC5KMjwtQ0tFABOZIZjU3SbO2zHZ1csd0LvEs5tPgpeIA
\[\begin{tikzcd}
	x_0 \arrow["e_6" description, loop, distance=2em, in=125, out=55] \arrow["e_2" description, loop, distance=2em, in=215, out=145] \arrow["e_1" description, loop, distance=2em, in=35, out=325]
\end{tikzcd}\]
So we see that $\pi_1(X,x_0)$ is isomorphic to the free group on $\{x_0e_1x_0,x_0e_6x_0,x_0e_2x_0\},$ which were exactly the edges in $X$ minus the spanning tree. We can even track this backwards to find generators of $\pi_1(X,x_0).$ We will omit the edges to prevent overcrowding.
\begin{itemize}
	\item We have that $\pi_1(X/\{e_0,e_3,e_4,e_5\},x_0)$ is generated by $\{e_1,e_6,e_2\}.$
	\item We have that $\pi_1(X/\{e_3,e_4,e_5\},x_0)$ is generated by $\{e_0e_1e_0^{-1},e_6,e_2\}$ by lifting along $e_0.$
	\item We have that $\pi_1(X/\{e_4,e_5\},x_0)$ is generated by $\{e_0e_1^{-1}e_0^{-1},e_3e_2,e_3e_6e_3^{-1}\}$ by lifting along $e_3.$
	\item We have that $\pi_1(X/e_5,x_0)$ is generated by $\{e_0e_1^{-1}e_0^{-1},e_3e_4e_2,e_3e_4e_6e_4^{-1}e_3^{-1}\}$ by lifting along $e_4.$
	\item We have that $\pi_1(X,x_0)$ is generated by $\{e_0e_1^{-1}e_0^{-1},e_3e_4e_2,e_3e_4e_5e_6e_5^{-1}e_4^{-1}e_3^{-1}\}$ by lifting along $e_4.$
\end{itemize}

Let's do a more sophisticated example of this algorithm. We show the following.
\ftwohasfthree*
\begin{proof}
	We use the construction of graphs from the proof of \autoref{thm:freesub}. Fix $F$ the free group generated by the two letters $a$ and $b,$ and we study subgroups $G$ of index $2.$ Using the construction from \autoref{thm:freesub}, we make a graph with two vertices and edges dictated by the action of $\{a,b\}$ on these vertices.

	The case we care about is when $a$ and $b$ swaps both of the vertices, as follows. Call this graph $X$; we label the $a$ and $b$ by subscripts for clarity.
	% https://tikzcd.yichuanshen.de/#N4Igdg9gJgpgziAXAbVABwnAlgFyxMJZABgBpiBdUkANwEMAbAVxiRBgHEQBfU9TXPkIoATOSq1GLNnS7cJMKAHN4RUADMAThAC2SMiBwQkARmoAjGGChIAzAYZ1LDAAoC8BNljDZYIavTMrIggdDx8IFq6+tRGphZWNogALACc1I7ObtgewiDevqwBUsEg5uEa2nqIZobGiA5OMK7uQl4+WH7FQTL+ZYl2xLyV0TWx9QaW1khpGU0tOW0hBZ1Fkj0h5fLcQA
	\[\begin{tikzcd}
		eG \arrow[rr, "a_1" description, bend left=24, color={rgb,255:red,214;green,92;blue,92}] \arrow[rr, "b_1" description, bend left=49] &  & aG \arrow[ll, "a_2" description, bend left=24] \arrow[ll, "b_2" description, bend left=49]
	\end{tikzcd}\]
	We need to contract along something to make this a flower graph, so we contract along the red $a_1,$ which gives the following graph $X/a_1.$
	% https://tikzcd.yichuanshen.de/#N4Igdg9gJgpgziAXAbVABwnAlgFyxMJZABgBpiBdUkANwEMAbAVxiRAB12AjJhhmHCAC+VEDCgBzeEVAAzAE4QAtkjIgcEVdQYQIaIgCYA7GVmM4MUQzpcYDAAqZc+QohBYw2WCGr1mrNzphUjlFFUQ1DS0QHT0iAEYADlNzS20bO0dsPAI2Dy9WX0YWNi5g0OVoqIjtXX0UAE4UhgsrDIcnHNd3TyxvIv9S4QohIA
	\[\begin{tikzcd}
		eG \arrow["a_2" description, loop, distance=2em, in=215, out=145] \arrow["b_1" description, loop, distance=2em, in=125, out=55] \arrow["b_2" description, loop, distance=2em, in=35, out=325]
	\end{tikzcd}\]
	So we see that $X/a_1$ is freely generated by $\{a_2,b_1,b_2\},$ so pulling back along $a_1$ we have that $X$ is freely generated by $\left\{a_1a_2,b_1a_1^{-1},a_1b_2\right\}.$ In other words, we can check that the subgroup generated by $\left\{a^2,ba^{-1},ab\right\}$ is free on those generators, finishing.
\end{proof}
Note that the above discussion can extend to classify all subgroups of index two of $F.$ We simply have to do casework on what the possible (connected) graphs on two vertices are. We will not do this computation here.

% In general, if we want to find the fundamental group of some graph the algorithm takes a spanning tree of our graph, and our generators of the subgroup correspond to the edges not in the tree.
% \begin{example}
%     Consider the map $F_2\to S_3$ by $a\mapsto(12)$ and $b\mapsto(23).$ What is the kernel? It's not the group generated by $a^2,b^2,(ab)^3$ (which are in the kernel, of course), but this is somewhat complex. Well, let's draw the graph of what the generators are doing.
%     % https://q.uiver.app/?q=WzAsNixbMSwwLCJcXGJ1bGxldCJdLFsyLDAsIlxcYnVsbGV0Il0sWzAsMSwiXFxidWxsZXQiXSxbMCwyLCJcXGJ1bGxldCJdLFszLDEsIlxcYnVsbGV0Il0sWzMsMiwiXFxidWxsZXQiXSxbMiwzLCIiLDAseyJvZmZzZXQiOjJ9XSxbMywyLCJhIiwwLHsib2Zmc2V0IjoyfV0sWzAsMSwiIiwwLHsib2Zmc2V0IjoyfV0sWzEsMCwiYSIsMix7Im9mZnNldCI6Mn1dLFs0LDUsIiIsMCx7Im9mZnNldCI6Mn1dLFs1LDQsImEiLDAseyJvZmZzZXQiOjJ9XSxbMiwwLCJiIiwwLHsib2Zmc2V0IjotMn1dLFswLDIsIiIsMSx7Im9mZnNldCI6LTJ9XSxbMSw0LCJiIiwwLHsib2Zmc2V0IjotMn1dLFs0LDEsIiIsMSx7Im9mZnNldCI6LTJ9XSxbNSwzLCIiLDEseyJvZmZzZXQiOi0yfV0sWzMsNSwiYiIsMCx7Im9mZnNldCI6LTJ9XV0=
%     \[\begin{tikzcd}
%         & \bullet & \bullet \\
%         \bullet &&& \bullet \\
%         \bullet &&& \bullet
%         \arrow[shift right=2, from=2-1, to=3-1]
%         \arrow["a", shift right=2, from=3-1, to=2-1]
%         \arrow[shift right=2, from=1-2, to=1-3]
%         \arrow["a"', shift right=2, from=1-3, to=1-2]
%         \arrow[shift right=2, from=2-4, to=3-4]
%         \arrow["a", shift right=2, from=3-4, to=2-4]
%         \arrow["b", shift left=2, from=2-1, to=1-2]
%         \arrow[shift left=2, from=1-2, to=2-1]
%         \arrow["b", shift left=2, from=1-3, to=2-4]
%         \arrow[shift left=2, from=2-4, to=1-3]
%         \arrow[shift left=2, from=3-4, to=3-1]
%         \arrow["b", shift left=2, from=3-1, to=3-4]
%     \end{tikzcd}\]
%     Again, we can find a spanning tree and can find some set of $7$ elements which generate our kernel. To start off, $b^2,baab,bab^{-1}b^{-1}ab$ are subgroups.
% \end{example}
% As a last example, let's have $F_2$ act on $\ZZ$ with $a:n\mapsto n+1$ and $b:n\mapsto n.$ This looks like the following.
% % https://tikzcd.yichuanshen.de/#N4Igdg9gJgpgziAXAbVABwnAlgFyxMJZABgBpiBdUkANwEMAbAVxiRAB12AjJhhmHCAC+pdJlz5CKAIzkqtRizacefAcNEgM2PASIAmOdXrNWiDt179BIsTslEAzEYWnlltTc3aJelABYXEyVzTgBjKAgcBCF5GCgAc3giUAAzACcIAFskMhAcCCRpWxAM7KLqAqR9ErKcxEN8wsRHWsz65yakfzbyxDyq-uoGCAg0IjJUxjgYeQY6LhgGAAVxXSkQdKwEgAtBah2YOig2SDBWXvrZLsRrkbGJ0imGGbmFpdX7P03tvZADo4ncxnC6aOrVSrNRr3cYoSbTWbDd4rNYOcxbXb7ECHY6nAigtLtJCdQadGGPZ6vJGLFFfDYYv4A3HA-HCChCIA
% \[\begin{tikzcd}
%     \bullet \arrow[r] \arrow[no head, loop, distance=2em, in=305, out=235] & \bullet \arrow[r] \arrow[no head, loop, distance=2em, in=305, out=235] & \bullet \arrow[r] \arrow[no head, loop, distance=2em, in=305, out=235] & \bullet \arrow[r] \arrow[no head, loop, distance=2em, in=305, out=235] & \cdots
% \end{tikzcd}\]
% This is generated by $b,$ $aba^{-1},$ $a^2ba^{-2},$ and so on.