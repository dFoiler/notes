\documentclass[../notes.tex]{subfiles}

\begin{document}

% !TEX root = ../notes.tex








You feel like you're going to have a bad time.

\subsection{Groups of Order \texorpdfstring{$2^n$}{}}
Last lecture we noticed that groups of order $16$ were rather a mess. In general, it turns out that groups of order higher powers of $2$ are even worse.
\[\begin{array}{c|c}
	n & \text{number of groups of order }2^n \\\hline
	4 & 14 \\
	6 & 267 \\
	10 & 49487365422
\end{array}\]
It turns out that the number of groups of order $p^n$ is a roughly $p^{(2/27)n^3},$ which is frankly huge; not even the $n=10$ case fully captures the enormity of having a cube in an exponential. There's an entire book for groups of order $2^n$ for $n\le6.$
\begin{remark}
	It turns out that the vast majority of groups of order less than some bound are going to be $2$-groups; see \href{https://math.stackexchange.com/q/241369/869257}{this MathExchange thread}. The next most common are groups of order $3\cdot2^n,$ then $5\cdot2^n.$ In general, classifying these is quite boring.
\end{remark}

\subsection{Classification of Finitely Generated Abelian Groups}
Today we'll prove \autoref{thm:finab}. Recall the statement.
\begin{thm}[Classification of finitely generated abelian groups]
	Any finitely generated abelian group is a product of cyclic groups.
\end{thm}
\begin{proof}
	Fix our group $G,$ and fix generators $\{g_1,\ldots,g_m\}.$ We will write the group operation of $G$ additively. There might be a list of relations among these elements; we list all relations, which gives us a large system of equations
	\[\left\{\begin{array}{ccccccc}
		a_{11}g_1 & + & \cdots & + & a_{1m}g_m & = & 0 \\
		a_{21}g_1 & + & \cdots & + & a_{2m}g_m & = & 0 \\
				  &   & \vdots
	\end{array}\right.\]
	We will abbreviate this system to the (unaugmented) matrix
	\[\begin{bmatrix}
		a_{11} & \cdots & a_{1m} \\
		a_{21} & \cdots & a_{2m} \\
		\vdots & \ddots & \vdots
	\end{bmatrix}\]
	We would like to simplify this to be diagonal; more precisely, because the above matrix need not be square (in fact, it might have countably infinite height), we want nonzero elements off the diagonal.
	
	So, roughly speaking, we want to row-reduce. Here are our row operations; these correspond to moving around our relations.
	\begin{itemize}
		\item We can swap rows. Effectively, swapping row $k$ with row $\ell$ turns the system
		\[\left\{\begin{array}{ccccccc}
			a_{11}g_1 & + & \cdots & + & a_{1m}g_m & = & 0 \\
			a_{21}g_1 & + & \cdots & + & a_{2m}g_m & = & 0 \\
					  &   & \vdots \\
			a_{k1}g_1 & + & \cdots & + & a_{km}g_m & = & 0 \\
					  &   & \vdots \\
			a_{\ell1}g_1 & + & \cdots & + & a_{\ell m}g_m & = & 0 \\
					  &   & \vdots
		\end{array}\right.\]
		into
		\[\left\{\begin{array}{ccccccc}
			a_{11}g_1 & + & \cdots & + & a_{1m}g_m & = & 0 \\
			a_{21}g_1 & + & \cdots & + & a_{2m}g_m & = & 0 \\
					  &   & \vdots \\
			a_{\ell1}g_1 & + & \cdots & + & a_{\ell m}g_m & = & 0 \\
					  &   & \vdots \\
			a_{k1}g_1 & + & \cdots & + & a_{km}g_m & = & 0 \\
					  &   & \vdots
		\end{array}\right.\]
		Merely rearranging the relations does not change the structure of the group.
		
		\item We can negate a row. Because negation of a row is an involution, this doesn't change the underlying structure.
		
		\item We can add two rows. Adding row $k$ to row $\ell,$ we see that we are essentially saying that the system
		\[\left\{\begin{array}{ccccccc}
			a_{11}g_1 & + & \cdots & + & a_{1m}g_m & = & 0 \\
			a_{21}g_1 & + & \cdots & + & a_{2m}g_m & = & 0 \\
					  &   & \vdots \\
			a_{k1}g_1 & + & \cdots & + & a_{km}g_m & = & 0 \\
					  &   & \vdots \\
			a_{\ell1}g_1 & + & \cdots & + & a_{\ell m}g_m & = & 0 \\
					  &   & \vdots
		\end{array}\right.\]
		implies
		\[\left\{\begin{array}{ccccccc}
			a_{11}g_1 & + & \cdots & + & a_{1m}g_m & = & 0 \\
			a_{21}g_1 & + & \cdots & + & a_{2m}g_m & = & 0 \\
					  &   & \vdots \\
			a_{k1}g_1 & + & \cdots & + & a_{km}g_m & = & 0 \\
					  &   & \vdots \\
			(a_{\ell1}+a_{k1})g_1 & + & \cdots & + & (a_{km}+a_{\ell m})g_m & = & 0 \\
					  &   & \vdots
		\end{array}\right.\]
		which is true. Also, the converse (the second system implies the first) holds by subtraction, so these do yield the same group.
		
		\item By induction, we can actually add any integer multiple of a row to another row.
	\end{itemize}
	Here are our column operations; these correspond to moving around our generators.
	\begin{itemize}
		\item We can swap columns. Effectively, swapping column $k$ with column $\ell$ turns the system
		\[\left\{\begin{array}{ccccccccccccccc}
			a_{11}g_1 & + & \cdots & + & a_{1k}g_k & + & \cdots & + & a_{1\ell}g_\ell & + & \cdots & + & a_{1m}g_m & = & 0 \\
			a_{21}g_1 & + & \cdots & + & a_{2k}g_k & + & \cdots & + & a_{2\ell}g_\ell & + & \cdots & + & a_{2m}g_m & = & 0 \\
					  &   &        &   &           &   & \vdots
		\end{array}\right.\]
		into the system
		\[\left\{\begin{array}{ccccccccccccccc}
			a_{11}g_1 & + & \cdots & + & a_{1\ell}g_k & + & \cdots & + & a_{1k}g_\ell & + & \cdots & + & a_{1m}g_m & = & 0 \\
			a_{21}g_1 & + & \cdots & + & a_{2\ell}g_k & + & \cdots & + & a_{2k}g_\ell & + & \cdots & + & a_{2m}g_m & = & 0 \\
					  &   &        &   &           &   & \vdots
		\end{array}\right.\]
		We note that this is the same as taking $(g_k,g_\ell)\mapsto(g_\ell,g_k),$ and rearranging the generators does not alter the structure of the group.
		
		\item We can negate a column, say $k.$ Effectively, this turns the system
		\[\left\{\begin{array}{ccccccccccccc}
			a_{11}g_1 & + & \cdots & + & a_{1k}g_k & + & \cdots & + & a_{1m}g_m & = & 0 \\
			a_{21}g_1 & + & \cdots & + & a_{2k}g_k & + & \cdots & + & a_{2m}g_m & = & 0 \\
					  &   &        &   &           &   & \vdots
		\end{array}\right.\]
		into
		\[\left\{\begin{array}{ccccccccccccc}
			a_{11}g_1 & + & \cdots & + & a_{1k}(-g_k) & + & \cdots & + & a_{1m}g_m & = & 0 \\
			a_{21}g_1 & + & \cdots & + & a_{2k}(-g_k) & + & \cdots & + & a_{2m}g_m & = & 0 \\
					  &   &        &   &           &   & \vdots
		\end{array}\right.\]
		Because inversion is an involution, we see that that the exchange of generators $g_k\mapsto-g_k$ does not change the group structure.
		
		\item We can add two columns, say $k$ to $\ell.$ Effectively, this turns the system
		\[\left\{\begin{array}{ccccccccccccccc}
			a_{11}g_1 & + & \cdots & + & a_{1k}g_k & + & \cdots & + & a_{1\ell}g_\ell & + & \cdots & + & a_{1m}g_m & = & 0 \\
			a_{21}g_1 & + & \cdots & + & a_{2k}g_k & + & \cdots & + & a_{2\ell}g_\ell & + & \cdots & + & a_{2m}g_m & = & 0 \\
					  &   &        &   &           &   & \vdots
		\end{array}\right.\]
		to
		\[\left\{\begin{array}{ccccccccccccccc}
			a_{11}g_1 & + & \cdots & + & a_{1k}(g_k-g_\ell) & + & \cdots & + & (a_{1\ell}+a_{1k})g_\ell & + & \cdots & + & a_{1m}g_m & = & 0 \\
			a_{21}g_1 & + & \cdots & + & a_{2k}(g_k-g_\ell) & + & \cdots & + & (a_{2\ell}+a_{2k})g_\ell & + & \cdots & + & a_{2m}g_m & = & 0 \\
					  &   &        &   &           &   & \vdots
		\end{array}\right.\]
		So we have taken the generator $g_k$ to $g_k-g_\ell,$ which is a reversible process and hence does not actually change the group structure. (We could construct an isomorphism if we wanted.
		
		\item By induction, we can actually add any integer multiple of a row to another row.
	\end{itemize}
	To ``row-reduce,'' we do row and column operations. Here are the steps.
	\begin{enumerate}
		\item Consider the smallest we can make $a_{11}$ by applying row and column operations while keeping $a_{11}$ nonnegative. We have two cases.
		\begin{itemize}
			\item If $a_{11}=0,$ then we can apply operations to make the entire matrix vanish, so $G\cong\ZZ^n.$ Indeed, if there is a nonnegative entry anywhere, then we can swap that entry to $a_{11}.$
			\item Otherwise $a_{11}>0.$ Currently our matrix looks like the following.
			\[\begin{bmatrix}
				a_{11} & a_{12} & \cdots & a_{1m} \\
				a_{21} & a_{22} & \cdots & a_{2m} \\
				\vdots & \vdots & \ddots & \vdots
			\end{bmatrix}\]
			We claim that $a_{11}\mid a_{k1}$ for each $k.$ Indeed, if $a_{11}\nmid a_{k1},$ then we can write $a_{k1}=qa_{11}+r$ for some $r<a_{11},$ so subtracting $q$ times the $k$\text{th} row from the first makes $a_{11}$ smaller. So subtracting $a_{k1}/a_{11}$ times the first row from the $k$\text{th} row gives the matrix
			\[\begin{bmatrix}
				a_{11} & a_{12} & \cdots & a_{1m} \\
				0 & a_{22} & \cdots & a_{2m} \\
				\vdots & \vdots & \ddots & \vdots
			\end{bmatrix},\]
			where the first column is all $0.$ Similarly, 
			
			The same division algorithm argument shows that $a_{11}\mid a_{1k}$ for each $k.$ So subtracting $a_{1k}/a_{11}$ times the first column from the $k$\text{th} column gives the matrix
			\[\begin{bmatrix}
				a_{11} & 0 & \cdots & 0 \\
				0 & a_{22} & \cdots & a_{2m} \\
				\vdots & \vdots & \ddots & \vdots
			\end{bmatrix}.\]
		\end{itemize}
		
		\item Then we simply repeat the process to the smaller matrix
		\[\begin{bmatrix}
			a_{22} & \cdots & a_{2m} \\
			\vdots & \ddots & \vdots
		\end{bmatrix}\]
		inside of our larger matrices. Note that applying row and column operations (which are swaps, negations, or additions) will not affect the $0$s surrounding this sub-matrix.
		
		So again, making $a_{22}$ as small as possible and repeat the previous step lets us assert a matrix of the form
		\[\begin{bmatrix}
			a_{11} & 0 & 0 & \cdots & 0 \\
			0 & a_{22} & 0 & \cdots & 0 \\
			0 & 0      & a_{33} & \cdots & a_{3m} \\
			\vdots & \vdots & \vdots & \ddots & \vdots
		\end{bmatrix}\]
		and inductively continue
	\end{enumerate}
	Once we are done with this process, we get the matrix of relations where all terms off the diagonal are $0.$ This looks like the system
	\[\left\{\begin{array}{ccccccccccccccc}
		a_{11}g_1 & + & 0g_2 & + & \cdots & + & 0g_k & + & \cdots & + & 0g_m & = & 0 \\
		0g_1 & + & a_{12}g_2 & + & \cdots & + & 0g_k & + & \cdots & + & 0g_m & = & 0 \\
			 &   &           &   &        &   & \vdots \\
		0g_1 & + & 0g_2 & + & \cdots & + & a_{kk}g_k & + & \cdots & + & 0g_m & = & 0 \\
			 &   &           &   &        &   & \vdots \\
		0g_1 & + & 0g_2 & + & \cdots & + & 0g_k & + & \cdots & + & a_{mm}g_m & = & 0 \\
	\end{array}\right.\]
	It follows that each generator $g_k$ has the sole relation $a_kg_k=0$ (with possibly $a_k=0$), so $g_k\mapsto1$ yields an isomorphism $\langle g_k\rangle\to\ZZ/a_k\ZZ$
	\[G\cong(\ZZ/a_1\ZZ)\times(\ZZ/a_2\ZZ)\times\cdots.\qedhere\]
\end{proof}
\begin{remark}
	We can actually guarantee that $a_{11}\mid a_{22}\mid\cdots.$ Indeed, otherwise we could use our row reduction to apply the division algorithm dividing $a_{22}$ by $a_{11},$ thus making $a_{11}$ smaller.
\end{remark}

\subsection{Groups of Order 17 and 18}
Groups of order $17$ are cyclic because $17$ is prime.

Let's talk about groups $G$ of order $18.$ We see $G$ they have a subgroup $H_9$ of order $9$ by Sylow, which must be normal, so $G\cong H_9\rtimes\ZZ/2\ZZ,$ where the $\ZZ/2\ZZ$ appears because it is our Sylow $2$-subgroup. We have the following cases.
\begin{itemize}
	\item If $H_9=\ZZ/9\ZZ,$ we see that $\ZZ/2\ZZ$ only has the trivial action or the inversion action on $\ZZ/9\ZZ.$
	\item For $H_9=(\ZZ/3\ZZ)^2,$ the trick is to view $(\ZZ/3\ZZ)^2$ as a vector space over $\FF_3$ of dimension $3.$ In particular, we are looking for maps $\ZZ/2\ZZ\to\op{GL}_2(\FF_3),$ which aside from the trivial map correspond to order-$2$ elements of $\op{GL}_2(\FF_3).$ We can now do this by hand.
\end{itemize}
One of these groups turns out to be more interesting. Namely, there is a $\ZZ/2\ZZ$-action on $(\ZZ/3\ZZ)^2$ by switching the two copies of $\ZZ/3\ZZ$; this corresponds to the order-$2$ matrix
\[\begin{bmatrix}0 & 1 \\ 1 & 0\end{bmatrix}\in\op{GL}_2(\FF_3).\]
In some sense, what is happening is that we have a nontrivial $\ZZ/2\ZZ$-action on $\ZZ/2\ZZ$-indexed sequences of $\ZZ/3\ZZ.$ We can generalize this construction.
\begin{defi}[Wreath products]
	Pick up two groups $G$ and $H$ and a set $\Omega$ with an $H$-action. (By default, we take $\Omega=H.$) Then we define
	\[\op{Mor}(\Omega,G)=\{f:\Omega\to G\},\]
	which is a group by with (say) pointwise operation: $(fg)(x)=f(x)g(x).$ This has an $H$-action defined by
	\[h\cdot f(x)=f(hx)\]
	for $h\in H,$ $f\in\op{Mor}(\Omega,G),$ and $x\in\Omega.$ So we define the \textit{wreath product} $G\wr_\Omega H:=\op{Mor}(\Omega,G)\rtimes H.$
\end{defi}
At a high level, what is happening is that we have a list of symmetries of $G$ (indexed by $\Omega$), but this list itself has symmetries we want to keep track of (which is the $H$-action on $\Omega$). A perhaps more concrete way to look at $\op{Mor}(\Omega,G)$ is as sequences $\{g_\omega\}_{\omega\in\Omega}$ in $G$ indexed by $\Omega.$ Here, the group operation is component-wise, and the $H$-action on $\Omega$ essentially induces a rearranging of the sequence.

Another quick fact that can we see straight from the definition is that
\[\#(G\wr_\Omega H)=\#\big(\op{Mor}(\Omega,G)\rtimes H\big)=\#\op{Mor}(\Omega,G)\cdot\#H=\#G^{\#\Omega}\cdot\#H.\]
Anyways, let's do some examples.
\begin{prop}
	We have $\ZZ/2\ZZ\wr\ZZ/2\ZZ\cong D_8.$
\end{prop}
\begin{proof}
	For concreteness, we fix $G=H=\ZZ/2\ZZ$ so that we are computing $G\wr H.$ To say that $D_8$ is the semidirect product of $\op{Mor}(H,G)$ and $H$ is to say that $D_8$ can fit in the short exact sequence
	\[1\to\op{Mor}(H,G)\to D_8\to H\to1,\]
	where $\op{Mor}(H,G)$ is normal in $D_8,$ $D_8\to H$ has a pull-back into $D_8,$ and we also have a prescribed conjugation action of $H$ on $\op{Mor}(H,G).$ We construct these manually from $D_8$ as follows.
	\begin{center}
		\begin{asy}
			unitsize(1cm);
			draw((1,1) -- (1,-1) -- (-1,-1) -- (-1,1) -- cycle);
			draw((-1.3, -1.3) -- (1.3,1.3), red+dashed); label("\color{red}$v$", (1.3,1.3), NE);
			draw((1.3, -1.3) -- (-1.3,1.3), magenta+dashed); label("\color{magenta}$w$", (-1.3,1.3), NW);
			draw((0,-1.3) -- (0,1.3), blue+dashed); label("\color{blue}$h$", (0,1.3), N);
		\end{asy}
	\end{center}
	Each of $v,w,h$ refer to the reflection over the prescribed line.
	\begin{enumerate}[label=(\alph*)]
		\item We claim that we want $\op{Mor}(H,G)\cong\langle v,w\rangle.$
		
		Note that $\op{Mor}(H,G)$ consists of $\ZZ/2\ZZ$-indexed sequences of $\ZZ/2\ZZ,$ so these are effectively ordered pairs $(\ZZ/2\ZZ)^2$ where the group law is component-wise. So to check that $\op{Mor}(H,G)\cong\langle v,w\rangle,$ it suffices to say that $v$ and $w$ both have order $2,$ as does $vw$ (which is the $180^\circ$ rotation), so indeed, $\op{Mor}(H,G)$ is an $\FF_2$-vector space with $4$ elements.
		
		We also note that $\langle v,w\rangle$ is normal in $D_8$ because it is index $8/4=2.$
		
		\item We claim that we want $H\cong\langle h\rangle.$ These are isomorphic because $h$ has order $2.$ We also see that $h\in D_8\setminus\langle v,w\rangle,$ so $h\langle v,w\rangle\ne\langle v,w\rangle,$ meaning that we do indeed have the short exact sequence
		\[1\to\langle v,w\rangle\to D_8\to\langle h\rangle\to 1.\]
		
		\item Lastly, we need to check that the $H$-action on $\op{Mor}(H,G)$ matches what it should be. Applying force, there are only two cases to check.
		\begin{itemize}
			\item We note that, given $\{a_0,b_1\}\in\op{Mor}(H,G),$ $0\cdot\{a_0,b_1\}=\{a_{0+0},b_{1+0}\}=\{a_0,b_1\},$ so the action by $0\in H$ is trivial; indeed, the action of $e$ on $\langle v,w\rangle$ by conjugation is trivial.
			\item We note that, given $\{a_0,b_1\}\in\op{Mor}(H,G),$ $1\cdot\{a_0,b_1\}=\{a_{0+1},b_{1+1}\}=\{b_0,a_1\},$ so the action by $0\in H$ swaps; indeed, the action of $h$ on $\langle v,w\rangle$ by conjugation swaps $v$ and $w,$ for $hvh^{-1}=w$ and $hwh^{-1}=v.$
			\qedhere
		\end{itemize}
	\end{enumerate}
\end{proof}
Wreath products also show up naturally as symmetry groups of rooted trees. Here is the key lemma.
\begin{prop} \label{prop:wreathtree}
	Fix $T_0$ a rooted tree with symmetry group $\op{Sym}T_0.$ Then the symmetry group of the tree
	% https://q.uiver.app/?q=WzAsNixbMiwwLCJcXGJ1bGxldCJdLFswLDEsIlQiXSxbMSwxLCJUIl0sWzIsMSwiXFxjZG90cyJdLFszLDEsIlQiXSxbNCwxLCJUIl0sWzAsMV0sWzAsMl0sWzAsNF0sWzAsNV1d
	\[\begin{tikzcd}
		&& \bullet \\
		T_1 & T_2 & \cdots & T_{n-1} & T_n
		\arrow[from=1-3, to=2-1, no head]
		\arrow[from=1-3, to=2-2, no head]
		\arrow[from=1-3, to=2-4, no head]
		\arrow[from=1-3, to=2-5, no head]
	\end{tikzcd}\]
	made of a root and $n$ copies of $T_0=T_1=T_2=\cdots=T_n$ is $\op{Sym}T_0\wr_{\{1,\ldots,n\}}S_n$
\end{prop}
Note that the wreath product is now actually taking in a named $\Omega=\{1,\ldots,n\}$ parameter. The action of $S_n$ on $\Omega$ is by permuting, of course.
\begin{proof}
	Name the big tree $T$ and let $[n]:=\{1,\ldots,n\}$ for brevity. The main idea is that there are two steps to choose a symmetry $T.$
	\begin{enumerate}
		\item Pick a symmetry of each of the $n$ copies of $T_0.$ This more or less corresponds to an ordered sequence $\{\sigma_k\}_{k=1}^n$ of elements in $\op{Sym}(T_0),$ which is the same thing as picking up an element of $\op{Mor}([n],\op{Sym}T_0).$
		\item Pick a way to rearrange the copies of $T_0$ itself. This corresponds to picking a permutation $\sigma\in S_n.$
	\end{enumerate}
	These steps combine into something which is believably $\op{Mor}([n],\op{Sym}T_0)\rtimes S_n=\op{Sym}T_0\wr_{[n]}S_n.$ We now rigorize this but not by too much because I don't hate myself. We want to build a split short exact sequence
	\[1\to\op{Mor}([n],\op{Sym}T_0)\to\op{Sym}T\to S_n\to1\]
	with prescribed $S_n$-action on $\op{Mor}([n],\op{Sym}T_0).$
	\begin{enumerate}[label=(\alph*)]
		\item We claim that $N:=\{\sigma\in\op{Sym}T:\sigma T_k\subseteq T_k\text{ for each }k\}$ is normal in $\op{Sym}T$ and isomorphic to $\op{Mor}([n],\op{Sym}T_0).$
		\begin{itemize}
			\item We show that $N$ is normal in $\op{Sym}T.$ Indeed, for any $\tau\in\op{Sym}T$ and $\sigma\in N,$ then we have to check that
			\[\left(\tau^{-1}\sigma\tau\right)(T_k),\]
			for any of the subtrees $T_k.$ Well, because the $T_k$ are rooted trees, we see that $\tau$ must move the entire subtree $T_k$ to some other tree $T_\ell$ wholesale. So any vertex $t\in T_k$ has $\tau t\in T_\ell,$ so $\sigma\tau t\in T_\ell,$ and $\tau^{-1}\sigma\tau t\in T_k,$ finishing.
			\item We show that $N\cong\op{Mor}([n],\op{Sym}T_0).$ We construct our map $\varphi:N\to\op{Mor}([n]\to\op{Sym}T_0)$ by restriction: 
			\[\varphi(\sigma)\mapsto\{\sigma|_{T_k}\}_{k=1}^n.\]
			This is homomorphic because look at it. It has an inverse map by taking $\{\sigma_k\}_{k=1}^n$ to the symmetry of $\op{Sym}T$ which applies $\sigma_k$ to $T_k.$ It follows that $\varphi$ is an isomorphism.
		\end{itemize}
		
		\item We note that there is an embedding $S_n$ into $\op{Sym}T$ by sending $\sigma\in S_n$ to the permutation which merely permutes the $\{T_k\}_{k=1}^n.$ We claim that we can set $H$ to be the image of this permutation. (Technically, we have to fix a standard equality of the $T_k$ to $T_0$ and then state that our elements of $H$ do not alter this, but we will not bother.) We get $H\cong S_n$ for free.
		
		We also see that there is a map
		\[\op{Sym}T\to S_n\]
		by viewing $\sigma\in\op{Sym}T$ as a permutation of the $\{T_k\}_{k=1}^n.$ Here we again use the fact that a symmetry of $T$ must send a subtree $T_k$ to a $T_\ell$ wholesale.
		
		We see that the image of $H$ fully covers $S_n$ because $H$ describes all the ways we can rearrange the $\{T_k\}_{k=1}^n.$ Further, we see that the kernel of this map consists of the maps which fix each tree in place, which is exactly $N.$ So we indeed have the split short exact sequence
		\[1\to N\to\op{Sym}T\to H\to1.\]
		In particular, we do have $\op{Sym}T=\op{Mor}([n],\op{Sym}T_0)\rtimes S_n.$
		
		\item It remains to check that the $H$-action on $N$ (by conjugation: $h\cdot x=h^{-1}xh$) matches the $S_n$-action on $\op{Mor}([n],\op{Sym}T_0).$ Fix $h\in H$ corresponding to $\sigma\in S_n$ and $g\in N$ corresponding to $\{g|_{T_k}\}_{k=1}^n\in\op{Mor}([n],\op{Sym}T_0).$ On one hand,
		\[\sigma\cdot\{g|_{T_k}\}_{k=1}^n=\{g|_{T_{\sigma k}}\}_{k=1}^n.\tag{$*$}\]
		On the other hand, for any $t\in T_{k}$ in any subtree $T_{k},$ we see $ht\in T_{\sigma k},$ so $g$ will behave like $g|_{T_{\sigma k}}$ on $ht,$ which then gets sent back to $g|_{T_{\sigma k}}t$ after another $h^{-1}.$
		
		So indeed, $h^{-1}gh$ restricts to $g|_{T_{\sigma k}}$ on each $T_k,$ which matches $(*).$
		\qedhere
	\end{enumerate}
\end{proof}
That was a lot of work, so here is a nice corollary.
\begin{cor}
	Fix $T$ a complete binary rooted tree with $n+1$ levels for $n>0.$ Then
	\[\op{Sym}T\cong\underbrace{S_2\wr S_2\wr\cdots\wr S_2}_n,\]
	where $\wr$ is left-associative. It follows (by induction) that there are $2^{2^{n}-1}$ total symmetries.
\end{cor}
\begin{proof}
	We induct. For $n=1,$ we have a complete binary tree with two levels, which looks like the following.
	% https://q.uiver.app/?q=WzAsMyxbMiwwLCJcXGJ1bGxldCJdLFswLDEsIlxcYnVsbGV0Il0sWzQsMSwiXFxidWxsZXQiXSxbMCwxXSxbMCwyXV0=
	\[\begin{tikzcd}
		&& \bullet \\
		\bullet &&&& \bullet
		\arrow[from=1-3, to=2-1, no head]
		\arrow[from=1-3, to=2-5, no head]
	\end{tikzcd}\]
	This has symmetry group $S_2,$ which is our base case. For the inductive step, we fix $T_0$ the completed binary rooted tree with $n+1$ levels and construct the completed binary rooted tree with $n+2$ levels as follows.
	\[\begin{tikzcd}
		&& \bullet \\
		T_0 &&&& T_0
		\arrow[from=1-3, to=2-1, no head]
		\arrow[from=1-3, to=2-5, no head]
	\end{tikzcd}\]
	By \autoref{prop:wreathtree}, we see that the symmetry group of the big tree is $\op{Sym}T_0\wr S_2,$ which is what we wanted.
\end{proof}
\begin{remark}
	Technically we may permit the $n=0$ as the base of our induction, which is the tree with only a root.
\end{remark}
% \begin{ex}
%     The symmetry group of the following huge rooted tree is $S_3\wr_{[3]} S_3.$
%     % https://q.uiver.app/?q=WzAsMTMsWzQsMCwiXFxidWxsZXQiXSxbNCwxLCJcXGJ1bGxldCJdLFs3LDEsIlxcYnVsbGV0Il0sWzEsMSwiXFxidWxsZXQiXSxbMCwyLCJcXGJ1bGxldCJdLFsxLDIsIlxcYnVsbGV0Il0sWzIsMiwiXFxidWxsZXQiXSxbMywyLCJcXGJ1bGxldCJdLFs0LDIsIlxcYnVsbGV0Il0sWzUsMiwiXFxidWxsZXQiXSxbNiwyLCJcXGJ1bGxldCJdLFs3LDIsIlxcYnVsbGV0Il0sWzgsMiwiXFxidWxsZXQiXSxbMCwxXSxbMCwyLCIiLDIseyJzdHlsZSI6eyJoZWFkIjp7Im5hbWUiOiJub25lIn19fV0sWzAsMywiIiwyLHsic3R5bGUiOnsiaGVhZCI6eyJuYW1lIjoibm9uZSJ9fX1dLFszLDQsIiIsMix7InN0eWxlIjp7ImhlYWQiOnsibmFtZSI6Im5vbmUifX19XSxbMyw1LCIiLDIseyJzdHlsZSI6eyJoZWFkIjp7Im5hbWUiOiJub25lIn19fV0sWzMsNiwiIiwyLHsic3R5bGUiOnsiaGVhZCI6eyJuYW1lIjoibm9uZSJ9fX1dLFsxLDcsIiIsMCx7InN0eWxlIjp7ImhlYWQiOnsibmFtZSI6Im5vbmUifX19XSxbMSw4LCIiLDAseyJzdHlsZSI6eyJoZWFkIjp7Im5hbWUiOiJub25lIn19fV0sWzEsOSwiIiwwLHsic3R5bGUiOnsiaGVhZCI6eyJuYW1lIjoibm9uZSJ9fX1dLFsyLDEwLCIiLDIseyJzdHlsZSI6eyJoZWFkIjp7Im5hbWUiOiJub25lIn19fV0sWzIsMTEsIiIsMix7InN0eWxlIjp7ImhlYWQiOnsibmFtZSI6Im5vbmUifX19XSxbMiwxMiwiIiwyLHsic3R5bGUiOnsiaGVhZCI6eyJuYW1lIjoibm9uZSJ9fX1dXQ==
%     \[\begin{tikzcd}
%     	&&&& \bullet \\
%     	& \bullet &&& \bullet &&& \bullet \\
%     	\bullet & \bullet & \bullet & \bullet & \bullet & \bullet & \bullet & \bullet & \bullet
%     	\arrow[no head, from=1-5, to=2-5]
%     	\arrow[no head, from=1-5, to=2-8]
%     	\arrow[no head, from=1-5, to=2-2]
%     	\arrow[no head, from=2-2, to=3-1]
%     	\arrow[no head, from=2-2, to=3-2]
%     	\arrow[no head, from=2-2, to=3-3]
%     	\arrow[no head, from=2-5, to=3-4]
%     	\arrow[no head, from=2-5, to=3-5]
%     	\arrow[no head, from=2-5, to=3-6]
%     	\arrow[no head, from=2-8, to=3-7]
%     	\arrow[no head, from=2-8, to=3-8]
%     	\arrow[no head, from=2-8, to=3-9]
%     \end{tikzcd}\]
%     In particular, we see that there are $6!^4$ total symmetries.
% \end{ex}
Here are some more miscellaneous examples of the wreath product.
\begin{ex}
	In electrodynamics, it turns out that the symmetry group is also a wreath product. Space-time acts as $\RR^4,$ and the Gauge group is $S^1.$ The symmetry group consists of (smooth) functions $\RR^4\to S^1,$ on which the Poincare group acts.
\end{ex}
% I don't understand this example
\begin{ex}
	We can also have the group of symmetries of an $n$-dimensional cube. Fix its vertices are $(\pm1,\pm1,\cdots).$ The symmetries are various inversions and permutations of coordinates, so our group of symmetries is $(\ZZ/2\ZZ)^n\wr_{[n]}S_n$ using similar logic as in \autoref{prop:wreathtree}.
\end{ex}

\subsection{Groups of Order 24}
We're going to skip over groups of order $19,$ $20,$ $21,$ $22,$ and $23,$ and we're not even going to fully classify groups of order $24.$ But let's sketch this; fix $G$ of order $24.$
\begin{enumerate}
	\item If there is a normal Sylow $3$-subgroup, then this is a semidirect product. There are many possibilities here for what is acting on our Sylow $3$-subgroup.
	\item Otherwise, the number of Sylow $3$-subgroup is $1\pmod3$ and divides $24$ and hence must divide $8$ and hence must be $4.$ The trick is for $G$ to act by conjugation of $G$ on its Sylow $4$-subgroups, which gives a homomorphism
	\[G\to S_4.\]
	What is the kernel? Well, it has order dividing $24,$ and in fact it has order $1,2,3,$ or $6$ because $G$ acts transitively on the Sylow $3$-subgroups.
	\begin{enumerate}
		\item If the order is $1,$ we get $G\cong S_4.$
		\item If the order is $2,$ we get the binary tetrahedral group. Namely, we can realize $A_4$ as the group of rotations of a tetrahedron, which we can pull back along the short exact sequence
		\[1\to\{\pm1\}\to S^3\to\op{SO}_3(\RR)\to 1.\]
	\end{enumerate}
	We can work out the other cases if we want, but we won't here.
\end{enumerate}

\subsection{Symmetric Groups}
While we're here, let's use this as a discussion to talk about symmetric groups. Recall the following definition.
\begin{defi}[Symmetric group]
	The symmetric group $S_n$ consists of the permutations of $\{1,\ldots,n\}.$
\end{defi}
To talk about the conjugacy classes, we note that we can write any permutation as a product of cycles by tracking the orbits of single elements. It turns out that the structure we need is the notion of the ``cycle shape.''
\begin{defi}[Cycle shape]
	Fix $\sigma\in S_n.$ We say that $\sigma$ has ``cycle shape
	\[1^{n_1}2^{n_2}3^{n_3}\cdots\]
	if and only if the cycle decomposition of $\sigma$ has exactly $n_k$ $k$-cycles. For example, note that $n_k=0$ for $k>n$ and that $\sum_{k=1}^nn_kk=n.$ We will not show that cycle shape is well-defined.
\end{defi}
We have the following proposition.
\begin{prop}
	Any two permutations $\sigma$ and $\tau$ of $S_n$ are conjugate if and only if they have the same cycle shape.
\end{prop}
\begin{proof}
	In one direction, suppose that $\sigma$ and $\tau$ have the same cycle shape. We'll give the proof idea: conjugation ``renames'' the elements that $\sigma$ and $\tau$ are acting on. This is clearer with an example.
	\begin{ex}
		Take $\sigma=(123)(45)(6)$ and $\tau=(425)(16)(3).$ The point is that, for any $x\in S_n,$
		\[x(123)(45)(6)x^{-1}=x(123)x^{-1}\cdot x(45)x^{-1}\cdot x(6)x^{-1}=(x1,x2,x3)(x4,x5)(x6).\]
		So we can make this equal to $\tau$ by setting $x1=4,$ $x2=2,$ $x3=4,$ $x4=1,$ $x5=6,$ and $x3=6.$ Visually, we see that $x$ is the vertical map in the following diagram.
		% https://q.uiver.app/?q=WzAsMTIsWzEsMSwiMiJdLFswLDAsIigxIl0sWzEsMCwiMiJdLFsyLDAsIjMpIl0sWzMsMCwiKDQiXSxbNCwwLCI1KSJdLFs1LDAsIig2KSJdLFswLDEsIig0Il0sWzIsMSwiNSkiXSxbMywxLCIoMSJdLFs0LDEsIjYpIl0sWzUsMSwiKDMpIl0sWzEsN10sWzIsMF0sWzMsOF0sWzQsOV0sWzUsMTBdLFs2LDExXV0=
		\[\begin{tikzcd}
			{(1} & 2 & {3)} & {(4} & {5)} & {(6)} \\
			{(4} & 2 & {5)} & {(1} & {6)} & {(3)}
			\arrow[from=1-1, to=2-1]
			\arrow[from=1-2, to=2-2]
			\arrow[from=1-3, to=2-3]
			\arrow[from=1-4, to=2-4]
			\arrow[from=1-5, to=2-5]
			\arrow[from=1-6, to=2-6]
		\end{tikzcd}\]
	\end{ex}
	This idea generalizes into the proof. Indeed, the main trick is the following lemma.
	\begin{lem}
		Fix a cycle $(a_1,a_2,\ldots,a_k)\in S_n$ and some $\sigma\in S_n.$ Then
		\[\sigma(a_1,a_2,\ldots,a_k)\sigma^{-1}=(\sigma a_1,\sigma a_2,\ldots,\sigma a_k).\]
	\end{lem}
	\begin{proof}
		This is by brute force. The main thing to check is that $\left(\sigma(a_1,a_2,\ldots,a_k)\sigma^{-1}\right)(\sigma a_\ell)=\sigma a_{\ell+1}.$
	\end{proof}
	Now dissolve our given permutations $\sigma$ and $\tau$ into a cycle decompositions
	\[\sigma=\prod_{k=1}^n\prod_{\ell=1}^{n_k}(a_{k,\ell1},a_{k,\ell2},\ldots,a_{k,\ell k})\qquad\text{and}\qquad\prod_{k=1}^n\prod_{\ell=1}^{n_k}(k,b_{\ell1},b_{k,\ell2},\ldots,b_{k,\ell k}).\]
	(Here we have organized the cycle decomposition by cycle length.) Our conjugating element $x$ takes $a_{k,ij}$ to $b_{k,ij}$; because all elements of $\{1,\ldots,n\}$ appear in the $a_{k,ij}$ and $b_{k,ij},$ this $x$ is surjective map $\{1,\ldots,n\}\to\{1,\ldots,n\},$ so $x\in S_n.$ So we merely check
	\[x\sigma x^{-1}=\prod_{k=1}^n\prod_{\ell=1}^{n_k}x(a_{k,\ell1},a_{k,\ell2},\ldots,a_{k,\ell k})x^{-1}=\prod_{k=1}^n\prod_{\ell=1}^{n_k}(b_{k,\ell1},b_{k,\ell2},\ldots,b_{k,\ell k})=\tau,\]
	which finishes this direction of the proof.
	
	In the other direction, we show that all conjugates of $\sigma$ have the same cycle shape. Well, fix the cycle shape of $\sigma$ by
	\[\sigma=\prod_{k=1}^n\prod_{\ell=1}^{n_k}(a_{k,\ell1},a_{k,\ell2},\ldots,a_{k,\ell k}).\]
	Then, for any permutation $x\in S_n,$ we can compute the conjugate
	\[x\sigma x^{-1}=\prod_{k=1}^n\prod_{\ell=1}^{n_k}x(a_{k,\ell1},a_{k,\ell2},\ldots,a_{k,\ell k})x^{-1}=\prod_{k=1}^n\prod_{\ell=1}^{n_k}(xa_{k,\ell1},xa_{k,\ell2},\ldots,xa_{k,\ell k}).\]
	Because $x$ is injective, and the $a_{k,ij}$ appear exactly once in the original cycle decomposition, we see that the $xa_{k,ij}$ still make a valid cycle decomposition of $x\sigma x^{-1}.$ However, then the given cycle decomposition forces the cycle shape of $x\sigma x^{-1}$ to match $\sigma,$ finishing.
\end{proof}

We can also ask how many possible values of $x$ there are which conjugate $\sigma$ into $\tau.$ Essentially, we are having $S_n$ act on the conjugacy class of $\sigma$ by conjugation and asking how many elements take $\sigma$ to $\tau$; by Orbit-stabilizer logic, it suffices to count $\#\op{Stab}\sigma.$ The image is that we are roughly asking how many ways we can rewrite
\[\sigma=\underbrace{(*)\cdots(*)}_{n_1~1\text{-cycles}}\underbrace{(**)\cdots(**)}_{n_2~2\text{-cycles}}\underbrace{(***)\cdots}_{\cdots}.\]
Formally, we have the terrible cycle decomposition
\[\sigma=\prod_{k=1}^n\prod_{\ell=1}^{n_k}(a_{k,\ell1},a_{k,\ell2},\ldots,a_{k,\ell k}).\]
Well, for each $(a_{1,\ell1})$ of these $1$-cycles, there are $n_1!$ ways we can rearrange them. Then there are $n_2!$ ways to rearrange the $2$-cycles, but then each individually $2$-cycle has two ways to rearrange it internally, so we get $2^{n_2}n_2!.$ Continuing, we get
\[\prod_{k=1}^nk^{n_k}n_k!\]
total permutations stabilizing $\sigma.$ With this in mind, we can compute the number of elements of the conjugacy class of $\sigma$ is
\[\frac{\#S_n}{\#\op{Stab}(\sigma)}=\frac{n!}{\prod_{k=1}^nk^{n_k}n_k!}\]
by the Orbit-stabilizer theorem.
\begin{ex}
	Let's work this out for $S_4.$ We get the following table.
	\[\begin{array}{c|cc}
		\text{cycle shape} & \text{centralizer of element} & \text{size of class} \\\hline
		4 & 4 & 6 \\
		31 & 3 & 8 \\
		2^2 & 2^2\cdot2!=8 & 3 \\
		1^22 & 2!\cdot2=4 & 6 \\
		1^4 & 4!=24 & 1
	\end{array}\]
	We can check the union of our conjugacy classes has $24$ elements.
\end{ex}

\subsection{Solvability}
We can also asking about the normal subgroups of $S_n.$ Of course, there is $\{\id\}$ and $S_n,$ but there is also the alternating group.
\begin{defi}[Alternating]
	Fix the determinant
	\[\Delta=\prod_{1\le k<\ell\le n}(x_\ell-x_k).\]
	Then $S_n$ acts on $\Delta$ by permuting the coordinates, and the orbit of $\Delta$ is $\{\pm\Delta\}$ because it can only add signs to each factor (and indeed, we can check by hand that transpositions do add signs). We define $A_n=\op{Stab}(\Delta).$ Note that because the orbit is fully $\{\pm\Delta\},$ we have $\#A_n=\#S_n/\#\{\pm\Delta\}=n!/2$ by the Orbit-stabilizer theorem.
\end{defi}
These are almost all the normal subgroups of $S_n.$ We have the following.
\begin{prop} \label{prop:normsn}
	The normal subgroups of $S_n$ are the following.
	\begin{itemize}
		\item $1.$
		\item $S_n.$
		\item $A_n.$
		\item For $n=4,$ we also have the normal subgroup $\{\id,(12)(34),(13)(24),(14)(23)\}.$ Namely, it just so happens that the set $\{(12)(34),(13)(24),(14)(23)\}$ is a conjugacy class in $S_4$ (those are all the elements with cycle type $2^2$) so this subgroup is a union of conjugacy classes, by magic.
	\end{itemize}
\end{prop}
\begin{proof}
	We omit this proof because it is actually nontrivially annoying to classify the normal subgroups of $S_n.$ The key trick is that any normal nontrivial subgroup has an element $\sigma\in S_n\setminus\{e\}$; then a commutator $\tau\sigma\tau\sigma^{-1}$ for $\tau$ some transposition will force the normal subgroup to have a three-cycle, which forces the normal subgroup to contain $A_n.$
\end{proof}
This shows that $S_4$ is solvable.
\begin{defi}[Solvable]
	A group $G$ is \textit{solvable} if we can construct an ascending sequence of normal subgroups
	\[\{\id\}=G_0\subseteq G_1\subseteq G_2\subseteq\cdots\subseteq G_n=G\]
	such that each quotient $G_{k+1}/G_k$ is cyclic.
\end{defi}
\begin{remark}
	We can weaken the quotient condition to abelian, but it doesn't matter that much.
\end{remark}
\begin{ex}
	For $S_4,$ we have
	\[\{\id\}\subseteq\{(12)(34)\}\subseteq\{\id,(12)(34),(13)(24),(14)(23)\}\subseteq A_4\subseteq S_4.\]
	These quotients have order $2,2,3,2$ respectively, so they are cyclic because their order is prime.
\end{ex}
This notion of solvable will come up again in Galois theory; it ``turns out'' that $S_5$ is not solvable, and this has to do with non-solvability of the quintic by radicals. Perhaps stranger, the weird exception for $n=4$ in \autoref{prop:normsn} is why quartics are solvable by radicals.
\begin{defi}[Simple]
	A group is called \textit{simple} if it has no proper, nontrivial, normal subgroups.
\end{defi}
\begin{ex}
	Consider $A_5,$ the set of rotations of the icosahedron, of which there are $3\cdot20=60$ elements by the Orbit-stabilizer theorem. (Each of the $20$ vertex has $3$ rotations fixing it.) Let's write out its conjugacy classes.
	\begin{itemize}
		\item We have $\id,$ which is a conjugacy class of size $1.$
		
		\item Face symmetries: we can rotate a face by $2\pi/3,$ of which there are $20$ elements, of order $3.$
		
		Note that rotating a face by $4\pi/3$ is the same as rotating its opposite face by $2\pi/3,$ so we don't count this symmetry.
		
		\item Edge symmetries: we can rotate an edge by $\pi/2,$ of which there are $30$ elements, of order $2.$ However, flipping over an edge is the same as flipping over the opposite edge, so there are only $15$ here.
		
		\item Vertex symmetries: we can rotate a vertex by $2\pi/5,$ of which there are $12$ elements, of order $5.$ We can also rotate a vertex by $4\pi/5,$ of which there are $12$ elements, of order $5.$
		
		Note that rotating by $6\pi/5$ is rotating the opposite vertex by $4\pi/5,$ so we don't count it; similarly, we don't count rotation by $8\pi/5.$
	\end{itemize}
	We can check that $1+20+15+12+12=60,$ so these are all our conjugacy classes. Now suppose we have a normal subgroup. It must be a sum of the above conjugacy classes and have size dividing $60,$ but it turns out that the only ways to do this are to have size $1$ or $60.$
	
	It follows that $A_5$ have no proper, nontrivial, normal subgroups, so $A_5$ is simple and not solvable.
\end{ex}
\begin{ex}
	We also have that $\ZZ/p\ZZ$ is simple for $p$ prime.
\end{ex}
It turns out that all simple groups of order less than $60$ are of the form $\ZZ/p\ZZ$ for $p$ prime. The hard cases here are $48$ or $56.$ The point is that $A_5$ is the first group we have some trouble understanding.

This gives us the following way to study groups. For any group $G,$ we can find some maximal chain of normal subgroups
\[\{e\}\subseteq G_0\subseteq G_1\subseteq\cdots\subseteq G_n=G\]
such that $G_k$ is normal in $G_{k+1}$ and $G_{k+1}/G_k$ is simple. So we have two problems.
\begin{enumerate}
	\item Find all simple groups.
	\item Find all ways to take the above chain and combine the simple groups into larger groups.
\end{enumerate}
The second question is hopeless: for example, if we just have ten copies of $\ZZ/2\ZZ,$ then we are back to trying to classify groups of order $2^{10}$ again, which is very sad. The first question does have an answer: there are $18$ infinite families of simple groups and $26$ some exceptions.
\begin{remark}
	Nobody actually knows how long the proof of the classification of simple groups is. It's probably somewhere between ten or twenty thousand pages. It has not been computer verified because it's too long and hard.
\end{remark}

\subsection{Miscellaneous Group Theory}
There are some interesting groups of order $120.$ There are three groups built from $\ZZ/2\ZZ$ and $A_5.$
\begin{itemize}
	\item We can take $\ZZ/2\ZZ\times A_5.$
	\item We can also take $S_5$ which has a normal subgroup $A_5$ and quotient $\ZZ/2\ZZ.$
	\item We also have the binary icosahedron group, created by pulling back $A_5$ as the symmetries of the icosahedron as a subgroup of $\op{SO}_3(\RR)$ along
	\[1\to\{\pm1\}\to S^3\to\op{SO}_3(\RR)\to 1.\]
\end{itemize}
These groups are different by counting the number of elements which square to $e$; for example, the binary icosahedron group has exactly one element of order $2$ from $\{\pm1\},$ yielding $2$ elements. In contrast, $\ZZ/2\ZZ\times A_5$ has $2\cdot\binom54\cdot3=30$ such elements, and $S_5$ has more.
\begin{remark}
	It turns out that binary icosahedron group $G$ shows up in algebraic topology. It turns out that $S^3/G$ is a three-dimensional manifold $M$ with $\pi_1(M)=G$ and first homology group the maximal abelian group of $G,$ which is trivial. This motivated the Poincare conjecture: if $\pi_1$ vanishes for a $3$-manifold, then it must be a $3$-sphere. (This is not true if the first homology group vanishes, as shown.)
\end{remark}
While we're here, we record the following rather amusing (and surprisingly difficult) result.
\begin{exe}
	We show that there are no simple groups of order $120$.
\end{exe}
\begin{proof}
	We proceed in steps.
	\begin{enumerate}
		\item We claim that either $G$ is not simple or the number $n_5$ of Sylow $5$-subgroups is $6$. Well, we know that $n_5\equiv1\pmod5$ and $n_5\mid24$, so $n_5\in\{1,6\}$. If $n_5=1$, then the Sylow $5$-subgroup is normal, so we are done.

		Thus, we may assume that $n_5=6$.

		\item We claim that either $G$ is not simple or $G$ embeds into $S_6$. For this, let $\mathrm{Syl}_5(G)$ be the set of Sylow $5$-subgroups. Then $G$ acts by conjugation on $\mathrm{Syl}_5(G)$, so we have an action map
		\[\varphi\colon G\to\mathrm{Sym}(\mathrm{Syl}_5(G)).\]
		This map is nontrivial because the action of $G$ on the Sylow $5$-subgroups is transitive. Thus, if $\ker\varphi$ is nontrivial, then it is a nontrivial proper normal subgroup of $G$, and we are done. As such, we may assume that $\varphi$ is injective, thereby providing an embedding of $G$ into $S_6$.

		\item We claim that $A_6$ has no subgroup $H$ of order $120$. Indeed, this would imply that $[A_6:H]=3$, which is quite small. To use this, we consider the action map
		\[\psi\colon A_6\to\mathrm{Sym}(A_6/H).\]
		We claim that $\ker\psi$ becomes a proper nontrivial normal subgroup of $A_6$, which contradicts the simplicity of $A_6$. Well, the action is transitive, so $\ker\psi$ is proper, and $\ker\psi$ must be nontrivial because $\left|A_6\right|<\left|S_3\right|$.

		\item We complete the proof. Thus far, we know that $G$ is either not simple or embeds into $S_6$, but the previous step implies that $G\not\subseteq A_6$. As such, we claim that $H\coloneqq G\cap A_6$ is a nontrivial proper normal subgroup of $G$, which will complete the proof. Certainly $H\subseteq G$ is normal because $A_6\subseteq S_6$ is normal, and $H$ is proper because $G\not\subseteq A_6$.

		It remains to show that $H$ is nontrivial. Well, let $P$ be a Sylow $5$-subgroup of $G$, and we claim that $P\subseteq H$. It is enough to check that $H\subseteq A_6$. For this, we note that $P$ is also a Sylow $5$-subgroup of $S_6$. Letting $P'$ be a Sylow $5$-subgroup of $A_6$, we see that $P$ and $P'$ must be conjugate, so because $A_6\subseteq S_6$ is normal, we conclude that $P\subseteq A_6$. This completes the proof.
		\qedhere
	\end{enumerate}
\end{proof}
After the cyclic groups and the alternating groups, the next simple group comes up as the symmetry group of the Fano plane, which is the following finite geometry. (Here, the unit circle is a ``line.'')
\begin{center}
	\begin{asy}
		unitsize(1cm);
		path p;
		for(int i = 0; i < 3; ++i)
		{
			p = p -- 2 * dir(90 + 120*i);
			dot(2 * dir(90 + 120*i));
			dot(dir(30 + 120*i));
			draw(dir(30 + 120*i) -- 2*dir(210+120*i));
		}
		draw(p -- cycle);
		draw(circle((0,0),1));
		dot((0,0));
	\end{asy}
\end{center}
This group also turns out to be $\op{GL}_3(\FF_2)$ because the Fano plane is the the projective plane over $\FF_2.$ Alternatively, this group is $\op{PSL}_2(\FF_7)\cong\op{SL}_2(\FF_7)/\{\pm I\}.$ There is no good reason why we should expect these groups to be isomorphic, but they are.

\end{document}