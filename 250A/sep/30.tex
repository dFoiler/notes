% !TEX root = ../notes.tex

















Let's just get to the point.

\subsection{Prime Ideals}
We're having some kind of introduction to commutative algebra today.
\begin{warn}
	Rings in this lecture are commutative with identity, except when explicitly said otherwise.
\end{warn}
Let's start by talking about prime ideals.
\begin{definition}[Prime]
	An ideal $I$ of a ring $R$ is \textit{prime} if and only if $R/I$ is an integral domain.
\end{definition}
Recall that an integral domain means that $ab=0$ implies $a=0$ or $b=0$ in $R,$ as well as $1\ne0.$\footnote{Here, $0=1$ is equivalent to $R=\{0\}$: if $R=\{0\},$ then $1=0$; if $1=0,$ then $r=1r=0r=0.$} Similarly, we have the following.
\begin{definition}[Maximal]
	An ideal $I$ of a ring $R$ is \textit{maximal} if and only if $R/I$ is a field.
\end{definition}
So we get the following for free.
\begin{proposition}
	All maximal ideals are prime.
\end{proposition}
\begin{proof}
	Fix $I$ maximal in $R.$ Then $R/I$ is a field and hence an integral domain.
\end{proof}
These are not the usual definitions, but Professor Borcherds likes looking at quotients. Here is the usual definition of maximal.
\begin{prop} \label{prop:defi2}
	An ideal $I$ of a ring $R$ is maximal if and only if $I$ is maximal among the set of proper ideals.
\end{prop}
\begin{proof}
	We have two implications. Note the condition $R/I\ne\{0\}$ is equivalent to $R\ne I.$
	\begin{itemize}
		\item If $I$ is maximal among the set of ideals, we claim that $R/I$ is a field. Indeed, for any nonzero coset $a+I\in R/I$ (i.e., $a+I\ne0$), we note that $a\notin I$ implies
		\[I\subsetneq(a)+I.\]
		But then $(a)+I=R$ by maximality, so there exists $b\in R$ and $j\in I$ such that $1=ab+j.$ But then
		\[(a+I)(b+I)=(ab)+I=(1-j)+I=1+I,\]
		so we have found a multiplicative inverse for $a+I.$
		\item If $R/I$ is a field, we claim that $I$ is maximal among the set of ideals. Suppose $J\supsetneq I$ is an ideal properly containing $I,$ and we show $J=R.$ Then there exists $a\in J\setminus I,$ and we note that $a\notin I$ implies $a+I\ne0+I,$ so there is $b\in R$ such that
		\[(a+I)(b+I)=1+I.\]
		In particular, $ab=1+j$ for some $j\in I\subseteq J.$ Then
		\[1=ab-j\in J\]
		by closure, so it follows $J=R.$
		\qedhere
	\end{itemize}
\end{proof}
And here is the usual definition for prime.
\begin{proposition}
	An ideal $I$ of a ring $R$ is prime if and only if $I\ne R$ and $ab\in I$ implies $a\in I$ or $b\in I.$
\end{proposition}
\begin{proof}
	The condition is really rephrasing that $R/I$ is an integral domain.
	\begin{itemize}
		\item The condition that $R/I=\{0\}$ is equivalent to $a+I=0+I$ for all $a\in R$ is equivalent to $a\in I$ for all $r\in R$ is equivalent to $I=R.$ By contraposition, $R/I\ne\{0\}$ is equivalent to $I$ being proper.
		\item The condition that $R/I$ is an integral domain is equivalent to ``$(a+I)(b+I)=0+I$ implies $a+I=0+I$ or $b+I$'' is equivalent to ``$ab\in I$ implies $a\in I$ or $b\in I,$'' which is the listed condition.
		\qedhere
	\end{itemize}
\end{proof}
Note that there is some bad terminology here. Note that $p\in R$ is a prime element in an integral domain implies that $(p)$ is a prime ideal, which we can check easily.\footnote{Note $p\mid a$ is equivalent to $a\in(p).$ So we see that ``$p\mid ab$ implies $p\mid a$ or $p\mid b$'' is equivalent to ``$ab\in(p)$ implies $a\in(p)$ or $b\in (p).$'' Additionally, $p$ not a unit is equivalent to $(p)\ne R.$} However, the converse is not true.
\begin{warn}
	In $R$ an integral domain, the ideal $(0)$ is prime: if $ab=0,$ then $a=0$ or $b=0$ because $R$ is an integral domain. However, $0$ is not a prime element by convention, even though it generates a prime ideal.
\end{warn}
Such is life.

\subsection{The Spectrum}
Suppose that, for a ring map $\varphi:R\to S,$ suppose we have a maximal ideal $\mf m\subseteq S.$ Then is $\varphi^{-1}\mf m$ maximal? Well, no; here is an example.
\begin{example}
	Consider the embedding $\varphi:\ZZ\into\QQ.$ Then $(0)$ is maximal in $\QQ,$ but $\varphi^{-1}((0))=(0)$ is not maximal in $\ZZ.$
\end{example}
This is somewhat annoying. What is happening here is that we have the composite
\[R\stackrel\varphi\to S\onto S/\mf m,\]
and we were hoping that the image $R/\varphi^{-1}\mf m$ would be a field. But in general, subrings of fields are not fields, so have no reason to expect that $R/\varphi^{-1}\mf m$ to be a field.

But subrings of fields do have to be integral domains, so we get the following.
\begin{lemma} \label{prop:contraspec}
	Fix $\varphi:R\to S$ a ring map with $\mf p\subseteq S$ prime in $S.$ Then $\varphi^{-1}\mf p$ is prime.
\end{lemma}
\begin{proof}
	We essentially repeat the above argument. Consider the composite map $\overline\varphi$ defined by
	\[R\stackrel\varphi\to S\onto S/\mf p.\]
	Now, $r\in R$ is in the kernel of the composite $\overline\varphi$ if and only if $\varphi(r)\in\mf p$ if and only if $r\in\varphi^{-1}\mf p.$ Thus, we see
	\[R/\varphi^{-1}\mf p\cong\im\overline\varphi\subseteq S/\mf p.\]
	So now it suffices to check that a subring of an integral domain is an integral domain. Well, if $A\subseteq B$ are rings with $B$ an integral domain, then note that $a_1a_2=0$ for $a_1,a_2\in A$ implies that $a_1=0$ or $a_2=0$ in $B$ and hence in $A.$ This finishes.
\end{proof}
So we have a functor from rings to sets that takes $R$ to its set of prime ideals, which we denote $\op{Spec}R.$ Then we take morphisms $\varphi:R\to S$ to the map
\[\varphi^{-1}:\op{Spec}S\to\op{Spec}R,\]
and everything here is functorial.
\begin{proposition}
	The function $\op{Spec}:\op{Ring}\to\op{Set}$ taking $R\mapsto\op{Spec}R$ and $\varphi:R\to S$ to $\varphi^{-1}:\op{Spec}R\to\op{Spec}S$ is a contravariant functor.
\end{proposition}
\begin{proof}
	We see \autoref{prop:contraspec} implies that $\varphi^{-1}:\op{Spec}S\to\op{Spec}R$ is well-defined, so $\op{Spec}$ is well-defined. It remains to show that $\op{Spec}$ is functorial.
	\begin{itemize}
		\item Note that $\id:R\to R$ goes to $\id^{-1}:\op{Spec}R\to\op{Spec}R,$ which is the identity map again because $\id^{-1}\mf p=\mf p.$
		\item Now suppose that $\varphi:R\to S$ and $\gamma:S\to T$ are maps of rings. Then we need to show that, given $\mf p\in\op{Spec}T,$ we have
		\[\varphi^{-1}\left(\gamma^{-1}\mf p\right)=\left(\gamma\circ\varphi\right)^{-1}(\mf p).\]
		Well, $r\in\varphi^{-1}\left(\gamma^{-1}\mf p\right)$ if and only if $\varphi(r)\in\gamma^{-1}\mf p$ if and only if $\gamma(\varphi(r))\in\mf p$ if and only if $(\gamma\circ\varphi)(r)\in\mf p$ if and only if $r\in(\gamma\circ\varphi)^{-1}\mf p.$
		\qedhere
	\end{itemize}
\end{proof}
\begin{remark}
	Grothendieck was the one who suggested that prime ideals would be better than maximal ideals because of the above functor.
\end{remark}

\subsection{Zariski Talk}
It turns out that we can make the spectrum into a topological space. So let's think about topological spaces. If $X$ is a topological space, we can take the set
\[\CC(X):=\op{Mor}_{\op{Top}}(X,\CC)\]
of continuous functions $X\to\CC.$ This is a ring with pointwise addition and multiplication; can we achieve all rings like this?

The answer is no, but let's try to make the answer closer to yes. Here is our question.
\begin{ques}
	Suppose $R$ is a ring. How can we realize a space $X$ so that $R$ is the ring of continuous functions from that space to (say) $\CC$?
\end{ques}
Namely, fix $R$ any ring, and we will try to find a space $X$ with $R=\CC(X).$ Well, we claim that any $x\in X$ induces a ring homomorphism $\op{ev}_x:R\to\CC$ by taking $\op{ev}_x:r\mapsto rx.$ This kernel has some nice properties.
\begin{exercise}
	Fix $X$ a topological space and $R:=\CC(X).$ Then, given $x\in X,$  we have $\op{ev}_x:r\mapsto rx$ is a ring homomorphism, and the ideal $\ker\op{ev}_x$ is a maximal ideal of $X.$
\end{exercise}
\begin{proof}
	Showing that $\op{ev}_x$ is a ring homomorphism comes down to checking the properties.
	\begin{itemize}
		\item Given $r,s\in R,$ we have $\op{ev}_x(r+s)=(r+s)x=rx+sx=\op{ev}_xr+\op{ev}_xs$ by definition of the ring addition.
		\item Given $r,s\in R,$ we have $\op{ev}_x(rs)=(rs)x=rx\cdot sx=\op{ev}_xr\cdot\op{ev}_xs$ again by definition of the ring multiplication.
		\item Lastly, the multiplicative identity in $R$ is the $1:z\mapsto1$ map, which goes to $\op{ev}_x1=1.$
	\end{itemize}
	We now check that $\ker\op{ev}_x$ is maximal. For this, we study the quotient
	\[R/\ker\op{ev}_x\cong\im\op{ev}_x\subseteq\CC.\]
	So we need to show that $\im\op{ev}_x$ is a field. We will only outline this because this is not a topology class; note it is an integral domain because it is a subring of $\CC.$

	So we have left to exhibit inverses in $\op{ev}_x.$ Essentially, we need to know that $z\in\im\op{ev}_x\setminus\{0\},$ then $z^{-1}\in\im\op{ev}_x\setminus\{0\}.$ However, $r\in\im\op{ev}_x\setminus\{0\}$ implies that there exists $r\in R$ such that $rx=z,$ but then
	\[\left(\frac r{z^2}\right)(z)=\frac1z,\]
	so indeed, $1/z\in\im\op{ev}_x.$ The point we need to rigorize is that $r/z^2$ is actually a continuous function, which I assert without proof.
\end{proof}
If the topology on $R$ is good, then it turns out $\ker\op{ev}_x$ will also be a closed, maximal ideal. I don't want to define a topology on $\CC(X),$ so I won't bother elaborating on this.

Conversely, if we were to give $R$ a nice enough topology, we could check that (closed) maximal ideals correspond to some $x\in X.$ So in some sense, we could imagine recovering $X$ as the set of closed, maximal ideals of $R.$
\begin{remark}
	Non-closed ideals make Professor Borcherds nervous. We want ideals to have quotients, and taking quotients by non-closed spaces make the quotient space not Hausdorff, which is sad.
\end{remark}
Now, what is the topology on $X$? Well, what kinds of open sets can we generate from $\CC(X)$? The point is that we have access to continuous functions, so, say, $\CC\setminus\{0\}$ is an open set, which makes
\[f ^{-1}\left(\CC\setminus\{0\}\right)\]
an open set for any $f \in R.$ These sets will turn out to make a perfectly fine basis for a topology; again details ignored.

Now, points $x\in X$ correspond with maximal ideals $\mf m\subseteq R,$ essentially behaving like kernels of special functions, so our basis element of $\{x:f  x\ne0\},$ which corresponds to $\{\mf m:f \notin\mf m\}$ upon associating each $x\in X$ with $\mf m=\ker\op{ev}_x.$ So here is our topology: on the set of maximal ideals $\op{MaxSpec}R,$ we define the topology to have a basis of open sets given by
\[D_f :=\{\mf m\in\op{MaxSpec}R:f \notin\mf m\},\]
where $f $ is some element of $R.$

But this topology has the ``concrete'' $\CC(X)$ part in sight! So we can do this more generally, still working with maximal ideals, creating a topology $\op{MaxSpec}R$ out of the maximal ideals of our ring $R.$ But we want to work with prime ideals
\begin{definition}[Zariski]
	Fix $R$ a ring. Then we define the \textit{Zariski} topology on $\op{Spec}R$ to have open sets defined by the basis elements
	\[\overline V(f ):=\{\mf p\in\op{Spec}R:f \notin\mf p\}\]
	for any particular $f \in R.$ The closed set $V(f )$ might be called the ``vanishing set'' of $f .$
\end{definition}
It's not too hard to check that the set $\overline V(f )$ actually forms a basis. Indeed, given any two $\overline V(f _1)$ and $\overline V(f _2),$ we can check that
\[\overline V(f _1f _2)=\overline V(f _1)\cap\overline V(f _2).\]
In the analogy, this is saying that if $f _1$ and $f _2$ both fail to vanish at a point, then $f _1f _2$ fails as well. Anyways, this comes down to checking that $f _1f _2\notin\mf p$ if and only if $f _1\notin\mf p$ and $f _2\notin\mf p,$ which is true: forwards because $\mf p$ is an ideal and backwards because $\mf p$ is prime.

Anyways, for all of our hard work, we get the following.
\begin{proposition}
	The function $\op{Spec}:\op{Ring}\to\op{Top}$ taking $R\mapsto\op{Spec}R$ and $\varphi:R\to S$ to $\varphi^{-1}:\op{Spec}S\to\op{Spec}R$ is a contravariant functor.
\end{proposition}
\begin{proof}
	We have shown that $R\mapsto\op{Spec}R$ is well-defined, but we do not yet know that $\varphi^{-1}:\op{Spec}S\to\op{Spec}R$ is a continuous function given $\varphi:R\to S$ is a ring homomorphism. It suffices to check that the pre-image of a basis element $\overline V(f)$ is open. Namely, we want to show that
	\[\left(\varphi^{-1}\right)^{-1}\big(\overline V(f)\big)\]
	is open. Now, fix a prime $\mf q\in\op{Spec}S$ with $\mf q\in\left(\varphi^{-1}\right)^{-1}\big(\overline V(f)\big).$ This condition is equivalent to $\varphi^{-1}\mf q\in\overline V(f)$ is equivalent to $f\notin\varphi^{-1}\mf q$ is equivalent to $\varphi f\notin\mf q.$ So we find that
	\[\left(\varphi^{-1}\right)^{-1}\big(\overline V(f)\big)=\{\mf q:\varphi f\notin\mf q\}=\overline V(\varphi f).\]
	This finishes.
\end{proof}

\subsection{Making Ideals}
Anyways, let's do an example.
\begin{example}
	In the zero ring, we might want $(0)$ to be a prime ideal, but it is not proper. So $\op{Spec}\{0\}=\emp.$ Thankfully, trivialities correspond to trivialities.
\end{example}
This is somewhat troublesome: are there any maximal ideals? The way that maximal ideals are usually constructed is by Zorn's lemma.
\begin{axiom}[Zorn's lemma] \label{ax:zorn}
	Fix $X$ a nonempty partially ordered set. Further, suppose that any totally ordered subset has an upper bound in $X.$ In other words, for any ascending chain
	\[a_0\le a_1\le a_2\le\cdots,\]
	there exists $a\in X$ with $a_\bullet\le a$ for each $a_\bullet.$ Then $X$ has a maximal element.
\end{axiom}
Be careful with what ``maximal'' means; these need not be unique.
\begin{definition}[Maximal]
	An element $m$ of a partially ordered set is \textit{maximal} if and only if $m\le x$ implies $m=x$ for $x\in X.$
\end{definition}
Zorn's lemma requires the axiom of choice, which is somewhat annoying. Roughly speaking, the proof is as follows.
\begin{proof}[Proof of \autoref{ax:zorn}]
	The idea is to apply ``transfinite induction.''
	\begin{itemize}
		\item We can start with any $a_0\in X.$
		\item If $a_0$ is maximal, we are done; otherwise, we can find $a_1>a_0.$
		\item If $a_1$ is maximal, we are done; otherwise, we can find $a_2>a_1.$
		\item Then we can continue down the line, and if we never find our element, we have an ascending chain
		\[a_0<a_1<a_2<\ldots,\]
		which gives some $a_\omega$ bigger than everyone by the ascending chain condition on $X.$
		\item If $a_\omega$ is maximal, we are done; otherwise, we can find $a_{\omega+1}>a_{\omega}.$
		\item If $a_{\omega+1}$ is maximal, we are done; otherwise we can find $a_{\omega+2}>a_{\omega+1}.$
		\item This process could theoretically continue to all ordinals, adding $1$ to the index and using the ascending chain condition to overcome limit ordinals. However, there is an absolute limit: there are ordinals with size larger than $\#X,$ so the process will have to stop before then.
	\end{itemize}
	We note that this requires making infinitely many choices, which is where the axiom of choice is required.
\end{proof}
Anyways, we can now show that maximal ideals exist.
\begin{prop} \label{prop:maxexist}
	Fix $R$ a nonzero ring. Then $R$ has a maximal ideal.
\end{prop}
\begin{proof}
	We use Zorn's lemma on the collection $\mathcal P$ of proper ideals so that a maximal ideal will be the same as maximal element in this partially ordered set. We see $\mathcal P$ is nonempty because $(0)\ne R$ is a proper ideal.
	
	So now we check the ascending chain condition on $\mathcal P.$ Suppose that we have an ascending chain of proper ideals
	\[I_1\subseteq I_2\subseteq I_3\subseteq\cdots.\]
	Then we can check that
	\[I:=\bigcup_{k=1}^\infty I_k\]
	is an ideal\footnote{Given $a,b\in I$ and $r\in R,$ we can find $N$ such that $a,b\in I_N.$ Then $ra\in I_N\subseteq I$ and $a+b\in I_N\subseteq I$ because $I_N$ is an ideal.} and is proper because $1\notin I$ because $1\notin I_k$ for any $k\in\NN.$ So $I$ is an upper bound for our chain in $\mathcal P.$
	
	To finish, we see that Zorn's lemma gives an ideal $\mf m$ which is maximal among the set of proper ideals, which is maximal by \autoref{prop:defi2}.
\end{proof}
We can actually do a little better than this. Here are some variations.
\begin{proposition}
	Given any proper $R$-ideal $J,$ we can find a maximal ideal $\mf m$ containing $J.$
\end{proposition}
\begin{proof}
	Note that \autoref{prop:maxexist} was essentially a proof that we can find a maximal ideal containing the ideal $(0),$ so we essentially repeat the argument from there, verbatim, instead using the partially ordered set $\mathcal P_J$ of proper ideals which contain $J.$ This collection is nonempty because it contains $J.$
\end{proof}
For the next variation, we pick up the following definition.
\begin{defi}
	Fix $R$ a ring. A subset $S\subseteq R$ is \textit{multiplicative} if and only if $1\in S,$ and $S$ is closed under multiplication. In other words, $x,y\in S$ implies $xy\in S.$
\end{defi}
\begin{proposition}
	Suppose $S\subseteq R$ satisfies $1\in S$ and is closed under multiplication. Then any maximal element among the proper ideals disjoint from $S$ is prime, and such elements exist if $0\notin S.$
\end{proposition}
\begin{proof}
	To show that such an element exists, we use Zorn's lemma on the collection $\mathcal P_S$ of proper ideals disjoint from $S.$ Essentially the same argument as in \autoref{prop:maxexist} will again work here: $\mathcal P_S$ is nonempty because $0\notin S,$ and for any ascending chain
	\[I_1\subseteq I_2\subseteq I_3\subseteq\cdots\]
	has upper bound given by
	\[I:=\bigcup_{k=1}^\infty I_k,\]
	which is proper and disjoint from $S$ because each of the $I_N$ are.

	Now, take $\mf p$ in the set of proper ideals disjoint from $S.$ To show that $\mf p$ is prime, suppose that $a\notin\mf p$ and $b\notin\mf p.$ Then $(a)+\mf p$ and $(b)+\mf p$ are ideals properly containing $\mf p$ and hence must intersect $S$ by maximality. But now
	\[\big((a)+\mf p\big)\big((b)+\mf p\big)=(ab)+(a)\mf p+(b)\mf p+\mf p^2\supseteq(ab)+\mf p\]
	will also contain an element of $S,$ so we must have $ab\notin\mf p$ because $\mf p\cap S=\emp.$ This finishes.
\end{proof}
In general, any maximal ideal of some collection of ideals tend to be prime.

\subsection{Examples of the Spectrum}
Anyways, let's do more examples of the spectrum.
\begin{example}
	For $R$ a field, the only ideals are $0$ or $R,$ so $\op{Spec}R=\{(0)\},$ which is extraordinarily nice.
\end{example}
\begin{example}
	For $R=\CC[x],$ we note that $\CC[x]$ is a principal ideal domain, so any ideal takes the form $(f)$ for some $f\in\CC[x].$ This ideal will be prime if and only if $f=0$ or $f$ is an irreducible polynomial by unique prime factorization. But over $\CC,$ irreducibles only look like $c(x-\alpha)$ for some $c,\alpha\in\CC,$ so we see
	\[\op{Spec}\CC[x]=\{(0)\}\cup\{(x-\alpha):\alpha\in\CC\}.\]
	Note that we can correspond $(x-\alpha)\in\CC[x]$ with $\alpha\in\CC$ in some sense.
\end{example}
The above example seems to recover the complex plane from $\op{Spec}\CC[x],$ but the topology on $\op{Spec}\CC[x]$ is very bad. Recall that our basis consisted of closed sets
\[V(f)=\{\mf p\in\op{Spec}\CC[x]:f\in\mf p\},\]
for various $f\in\CC[x].$ Unravelling this for our example, we see that $f\in(0)$ is equivalent to $f=0,$ and $f\in(x-\alpha)$ is equivalent to $x-\alpha\mid f$ is equivalent to $f(\alpha)=0.$ So our vanishing sets are as follows.
\begin{itemize}
	\item If $f=c\prod_{k=1}^n(x-\alpha_k)$ for $c,\{\alpha_k\}_{k=1}^n\subseteq\CC,$ then $V(f)=\{(x-\alpha_1),\ldots,(x-\alpha_n)\},$ which corresponds to some finite set of points in $\CC.$
	\item If $f\equiv0,$ then $V(f)=\op{Spec}\CC[x],$ which corresponds to all of $\CC.$
	\item If $f\equiv c$ for $c\in\CC^\times,$ then $V(f)=\emp.$
\end{itemize}
In particular, we have the cofinite topology, which isn't even Hausdorff. In general, most of our spectrums are sad like this.
\begin{example}
	For $R=\ZZ,$ we have
	\[\op{Spec}\ZZ=\{(0)\}\cup\{(p):p\text{ prime}\}.\]
	Again, let's think about the topology. Well, given $n\in\ZZ,$ a prime $p$ ``vanishes'' at $n$ if and only if $n\notin(n),$ which corresponds to $p\nmid n.$ So we find that our vanishing sets are empty ($n=1$), some finite set of primes ($n=\prod p_\bullet$), or everything ($n=0$).
\end{example}
\begin{remark}
	In the above examples, we tend to have sane primes in addition to the weird prime ideal $(0)$ which lives inside of all the other primes. This is an example of a ``generic'' point.
\end{remark}

\subsection{Localization}
Our first example of localization is as follows.
\begin{example}
	Take $\ZZ$ and force all nonzero elements to have inverses by making it $\QQ.$
\end{example}
This turns out to be a very useful, general operation. Here is our idea.
\begin{idea}
	Fix $R$ a ring with $S$ a subset, we want to define a ring $S^{-1}R$ to be $R$ where the elements of $S$ have inverses.
\end{idea}
This can be done, essentially, by taking $S$ and looking at the quotient
\[R\left[S^{-1}\right]=\frac{R\left[\{t_s\}_{s\in S}\right]}{\left(\{(st_s-1)\}_{s\in S}\right)}.\]
This is a very nice, universal way of forcing inverses, but we have no idea what it looks like. For example, is the canonical embedding $R\into R\left[S^{-1}\right]$ is injective? Well, it often isn't. Is $R\left[S^{-1}\right]$ even nonzero? These questions are not at all obvious because we have a huge ring modded out by a huge ideal.

\subsubsection{Ignoring Zero-Divisors}
So we want a little more control over our inverses, in the same way that we did for $\QQ.$ So let's try to imitate $\QQ.$ Here is our first attempt:
\begin{defi}[Localization for integral domains]
	Fix $R$ an integral domain and $S$ a multiplicative subset not containing $0.$ Then we define the \textit{localization} $S^{-1}R$ to be the set of pairs $(r,s)$ (denoted $r/s$) modulo the equivalence relation $r_1/s_1\equiv r_2/s_2$ if and only if $r_1s_2=r_2s_1.$ Then we define addition by
	\[\frac{r_1}{s_1}+\frac{r_2}{s_2}:=\frac{r_1s_2+r_2s_1}{s_1s_2}\qquad\text{and}\qquad\frac{r_1}{s_1}\cdot\frac{r_2}{s_2}:=\frac{r_1r_2}{s_1s_2}.\]
	We note that the multiplication in the denominator is why we want $S$ to be a multiplicative subset.
\end{defi}
There are lots of things to check here; we will outline the things we have to check.
\begin{lemma}
	Fix $R$ a ring and $S$ a multiplicative subset not containing any zero-divisors. The relation $\equiv$ above on $R\times S$ is an equivalence relation.
\end{lemma}
\begin{proof}
	We check the conditions one at a time.
	\begin{itemize}
		\item Reflexive: for $r/s\in S^{-1}R,$ we note that $r/s\equiv r/s$ is implied by $rs=rs.$
		\item Symmetric: note that $r_1/s_1\equiv r_2/s_2$ implies $r_1s_2=r_2s_1$ implies $r_2s_1=r_1s_2$ implies $r_2/s_2\equiv r_1/s_2.$
		\item Transitive: note that $r_1/s_1\equiv r_2/s_2$ and $r_2/s_2\equiv r_3/s_3$ implies $r_1s_2=r_2s_1$ and $r_2s_3=r_3s_2.$ Then
		\[s_2\cdot r_1s_3=(r_1s_2)s_3=(r_2s_1)s_3=(r_2s_3)s_1=(r_3s_2)s_1=s_2\cdot r_3s_1,\]
		so we are done after applying the cancellation law to get rid of the $s_2.$
		\qedhere
	\end{itemize}
\end{proof}
\begin{remark}
	We had to use that $R$ is an integral domain in the transitivity check to cancel $s_2.$ Namely, transitivity need not be transitive when $S$ has zero-divisors. For example, in $R=\ZZ/12\ZZ,$ we can take $S=\{1,2,4,8\}$ so that
	\[\frac32=\frac64=\frac34,\]
	but $\frac32\ne\frac34.$
\end{remark}
\begin{lemma}
	Fix $R$ an integral domain and $S$ a multiplicative subset not containing any zero-divisors. Then $S^{-1}R$ is a ring.
\end{lemma}
\begin{proof}
	This is identical to the proof that $\QQ$ is a ring: we apply brute force to each of the checks to be a ring, and they all work. We won't write them out here.
\end{proof}
\begin{lemma}
	Fix $R$ an integral domain and $S$ a multiplicative subset not containing any zero-divisors. Then the canonical map $R\into S^{-1}R$ is injective.
\end{lemma}
\begin{proof}
	Suppose $r_1,r_2\in R$ have $r_1/1=r_2/1.$ Then $r_1=1r_1=1r_2=r_2$ by definition.
\end{proof}

\subsubsection{Not Ignoring Zero-Divisors}
What if our multiplicative set $S$ does have zero-divisors? To deal with the problem, we set
\[I:=\{a\in R:as=0\text{ for some }s\in S\}.\]
We quickly check that $I$ is an ideal.
\begin{itemize}
	\item If $a_1,a_2\in I,$ then $a_1s_1=0$ and $a_2s_2=0$ for some $s_1,s_2\in S.$ Then
	\[(a_1a_2)(s_1+s_2)=(a_1s_1)s_2+(a_2s_2)s_1=0s_2+0s_1=0.\]
	\item If $a\in I$ and $r\in R,$ then $as=0$ for some $s\in S,$ so $(ar)s=a(rs)=a0=0.$
\end{itemize}
The point is that, in $R/I,$ the set $S/I$ has no zero-divisors because we have decided to kill all problematic elements: if $(r+I)(s+I)=I,$ then $rs\in I,$ so there is $s'\in S$ such that $rss'=0,$ so $r(ss')=0,$ so $r\in I$ because $S$ is multiplicative. So now we can define
\[S^{-1}R:=(S/I)^{-1}(R/I).\]

This construction took two steps: take the quotient and then localize. However, it is possible to do this by making a trickier equivalence relation.
\begin{defi}[Localization]
	Fix $R$ a ring and $S$ a multiplicative subset. Then we define the \textit{localization} $S^{-1}R$ to be the set of pairs $(r,s)$ (denoted $r/s$) modulo the equivalence relation $r_1/s_1\equiv r_2/s_2$ if and only if $sr_1s_2=sr_2s_1$ for some $s\in S.$ Then we define addition by
	\[\frac{r_1}{s_1}+\frac{r_2}{s_2}:=\frac{r_1s_2+r_2s_1}{s_1s_2}\qquad\text{and}\qquad\frac{r_1}{s_1}\cdot\frac{r_2}{s_2}:=\frac{r_1r_2}{s_1s_2}.\]
	We note that the multiplication in the denominator is why we want $S$ to be a multiplicative subset.
\end{defi}
We check that the trickier equivalence relation creates the same localization as $(S/I)^{-1}(R/I).$
\begin{proposition} \label{prop:eqvlocal}
	Fix $R,S,I$ as above. Then, given pairs $(r_1,s_1),(r_2,s_2)\in R\times S$ has the following equivalent relations.
	\begin{itemize}
		\item We write $(r_1,s_1)\equiv(r_2,s_2)$ if and only if there exists $s\in S$ such that $sr_1s_2=sr_2s_1.$
		\item We write $(r_1,s_1)\equiv'(r_2,s_2)$ if and only if $(r_1+I)(s_2+I)=(r_2+I)(s_1+I).$
	\end{itemize}
	In particular, $\equiv$ is an equivalence relation because $\equiv'$ is.
\end{proposition}
\begin{proof}
	Suppose that $(r_1,s_1)\equiv(r_2,s_2),$ which is equivalent to the existence of $s\in S$ such that $sr_1s_2=sr_2s_1.$ Then this is equivalent to
	\[s(r_1s_2-r_2s_1)=0,\]
	which is equivalent to $r_1s_2-r_2s_1\in I.$ Moving arithmetic into $R/I,$ we have $r_1s_2-r_2s_1\in I$ is equivalent to
	\[(r_1+I)(s_2+I)=(r_2+I)(s_1+I)\]
	after some rearranging, and this is equivalent to $(r_1,s_1)\equiv'(r_2,s_2).$
\end{proof}
So, for example, we get the following for free.
\begin{lemma}
	Fix $R,S,I$ as above. Then $S^{-1}R$ is a well-defined ring where the map $R\to S^{-1}R$ has kernel $I.$
\end{lemma}
\begin{proof}
	\autoref{prop:eqvlocal} tells us that $S^{-1}R$ is in bijection with $(S/I)^{-1}(R/I)$ by sending
	\[\frac rs\longmapsto\frac{r+I}{s+I}.\]
	(Explicitly, the equivalence relations on $R\times S$ are the same, so we are getting the same equivalence classes.) Because the addition and multiplication laws are defined in the same way, we in fact have that $S^{-1}R\cong(S/I)^{-1}(R/I).$
	
	So our work with $S$ not containing zero-divisors tells us that $S^{-1}R$ is a ring for free, and the kernel of $R\to S^{-1}R$ is the kernel of the composite map
	\[R\onto R/I\to (S/I)^{-1}(R/I).\]
	Here the first map has kernel $I$ and the second map has trivial kernel, so the composite's kernel is $I.$
\end{proof}
\begin{remark}
	Professor Borcherds does not like the one-step construction because the equivalence relation is somewhat unintuitive.
\end{remark}

\subsection{Localizing at a Prime}
Let's try and use localization for something. Given a space $X,$ we can look back at $\CC(X)$ and note that we actually have lots of possible functions: for each open set $U\subseteq X,$ we can define the continuous functions $U\to\CC$ as the space $\CC(U).$ This has some nice properties.
\begin{itemize}
	\item Given two open sets $U_1\subseteq U_2,$ we can take functions in $\CC(U_2)$ and ``restrict'' them to $\CC(U_1).$ This gives us a function $\op{Res}_{U_1,U_2}:\CC(U_2)\to\CC(U_1).$
	\item Given continuous functions on a family $\{U_\alpha\}_{\alpha\in\lambda}$ of open sets such that the continuous functions behave nicely with restriction, we can build a larger continuous function on $\bigcup_{\alpha\in\lambda}U_\alpha.$
	\item Given an open cover of $U$ named $\{U_\alpha\}_{\alpha\in\lambda},$ then if two continuous functions are identically equal on each $U_\alpha,$ then they are equal on $U.$
\end{itemize}
In other words, $\CC(-)$ is a sheaf of rings, but we won't use this.

Now, back in the analogy, take $R$ a ring with $\op{Spec}R$ the spectrum, as usual. Now, take $U$ to be an open subset of $\op{Spec}R,$ and we want to imagine what the analogue of $\CC(U)$ should be. To start, we should take $U$ as a basis element $\overline V(f).$

Let's check what $\overline V(f)$ means back in our example. Well, if $U$ is the open sets where a fixed function $f\in\CC(U)$ doesn't vanish, then really the only truly obvious function we have added here is $f^{-1}.$ So in general, we define the ring of $U=\overline V(f)$ to be
\[R\left(f^{-1}\right),\]
where we are just inverting out by that function $f.$ One might think that we've added different functions with the extra power to invert at a point, but, algebraically speaking, we don't have access to these.
\begin{remark}
	The secret to making algebraic geometry easier is to ignore all open sets which are not the basis elements $\overline V(f).$
\end{remark}
Anyways, here is an example.
\begin{example}
	Suppose we live in $\ZZ,$ but we want to kill the prime $2$ because $2$ has been really messing up your day. Well, the idea is to take the open set
	\[U=\overline V(2)=\{\mf p:2\notin\mf p\}\]
	so that we get $\ZZ[1/2],$ effectively killing the prime $(2).$
\end{example}
\begin{example}
	Conversely, suppose we live in $\ZZ,$ and we want to focus on $2$ alone. We start by ignoring $3,5,7,$ which means we want the open set
	\[\overline V(105)=\{\mf p:105\notin\mf p\}=\{\mf p:3\notin\mf p\text{ and }5\notin\mf p\text{ and }7\notin\mf p\},\]
	and we get $\ZZ[1/3,1/5,1/7].$ To keep killing more primes, we take the direct limit of this process for all open sets $U$ containing $2.$ At the end of this process, we get
	\[\ZZ[1/3,1/5,1/7,\ldots],\]
	which is called the localization of $\ZZ$ at the prime of $(2).$
\end{example}
The above example can be generalized.
\begin{definition}[Localization at a prime]
	Fix $R$ a ring and $\mf p$ a prime. Then $S:=R\setminus\mf p$ is multiplicative, so we define $R_\mf p:=S^{-1}R$ to be the \textit{localization at $\mf p$}.
\end{definition}