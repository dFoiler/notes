








You are filled with determination.

\subsection{Groups of Order 10}
Last time we classified groups of order $9.$ So let's do groups $G$ of order $10.$ Well, Cauchy's theorem promises subgroups $\langle a\rangle$ and $\langle b\rangle$ of order $5$ and $2$ respectively. But now $[G:\langle a\rangle]=2$ is normal, so we have the short exact sequence
\[1\to\langle a\rangle\to G\to \langle b\rangle\to 1,\]
so in particular, it follows that $G$ is the semidirect product of $\ZZ/5\ZZ\rtimes\ZZ/2\ZZ.$ So we can determine $G$ entirely based off of the $\ZZ/2\ZZ$-actions on $\ZZ/5\ZZ.$

This isn't very hard because $\op{Aut}(\ZZ/5\ZZ)=(\ZZ/5\ZZ)^\times\cong\ZZ/4\ZZ,$ but $\ZZ/4\ZZ$ only has two elements of order two, so the only $\ZZ/2\ZZ$-actions on $\ZZ/5\ZZ$ are $\id$ and $x\mapsto x^{-1}.$ So we get two groups of order $10,$ defined by
\[\begin{cases}
	b a b^{-1}=a, \\
	b a b^{-1}=a^{-1}.
\end{cases}\]
The first case is abelian and hence is $\ZZ/5\ZZ\times\ZZ/2\ZZ\cong\ZZ/10\ZZ.$ The second case is nonabelian, and we know a nonabelian group of order $10,$ namely $D_{10},$ so this must be that group.
\begin{defi}[Dihedral group]
	The dihedral group $D_{2n}$ is the group of symmetries of a regular $n$-gon.
\end{defi}
We remark that this logic can be generalized.
\begin{prop}
	Let $G$ be a group of order $2p$ for $p$ prime. Then $G\cong\ZZ/(2p)\ZZ$ or $G\cong D_{2p}.$
\end{prop}

\subsection{Dihedral Groups and Involutions}
Let's look at some dihedral groups.
\begin{itemize}
	\item $D_4$ is the group of symmetries of a line, which is $\ZZ/2\ZZ\times\ZZ/2\ZZ.$ This should really be imagined as the symmetries of a rectangle, where one of the sides is very thin. Here we have highlighted the horizontal and vertical flips.
	\begin{center}
		\begin{asy}
			unitsize(1cm);
			draw((-1,-0.1)--(1,-0.1)--(1,0.1)--(-1,0.1)--cycle);
			draw((-1.5,0) -- (1.5,0), red+dashed);
			draw((0,-0.3)--(0,0.3), blue+dashed);
		\end{asy}
	\end{center}
	\item $D_6$ is the group of symmetries of a triangle, which is $S_3$ because reflections can transpose any two vertices. For example, the following reflection transposes the bottom two vertices.
	\begin{center}
		\begin{asy}
			unitsize(1cm);
			path p;
			for(int i = 0; i < 3; ++i)
			{
				p = p -- dir(90 + 120*i);
				dot(dir(90 + 120*i));
			}
			draw((0,1.3) -- (0,-0.8), blue+dashed);
			draw(p -- cycle);
		\end{asy}
	\end{center}
	\item $D_8$ is the group of symmetries of a square.
	\item In general $D_{2n}$ is the group of symmetries of a regular $n$-gon.
\end{itemize}
In general $D_{2n}=\ZZ/n\ZZ\rtimes\ZZ/2\ZZ,$ where $\ZZ/n\ZZ$ is a rotation and $\ZZ/2\ZZ$ is a reflection. Here's the picture for $D_{20}=\left\langle a^2=b^{10}=e,aba^{-1}=b^{-1}\right\rangle.$
\begin{center}
	\begin{asy}
		unitsize(1.3cm);
		int n = 10;
		path p;
		for(int i = 0; i < n; ++i)
		{
			p = p -- dir(360.0/n * i);
			dot(dir(360.0/n * i));
		}
		draw((0,1.3) -- (0,-1.3), red+dashed);
		label("\color{red}$a$", (0,-1.3), S);
		draw(arc((0,0), 1.2, (360/n)*(int)(n/4), (360/n)*(int)(n/4+1)), arrow=EndArrow, p=blue);
		draw((0,0) -- 1.3 * dir( (360/n)*(int)(n/4) ), blue);
		draw((0,0) -- 1.3 * dir( (360/n)*(int)(n/4+1) ), blue);
		label("\color{blue}$b$", 1.3 * dir( (360/n)*(int)(n/4) ), E);
		draw(p -- cycle);
		for(int i = 0; i < n; ++i)
		{
			dot(dir(360.0/n * i));
		}
	\end{asy}
\end{center}
Also observe that each $D_4,D_8,D_{12},$ and so on all have nontrivial center, namely rotation by $180^\circ.$ (In the group presentation $D_{4k}=\left\langle a^2=b^{2k}=e,aba^{-1}=b^{-1}\right\rangle,$ this is $Z(D_{4k})=\{e,b^k\}.$)

Let's continue talking about order-$2$ elements, or ``involutions.'' In $D_2,D_6,D_{10},$ and so on all have their order-$2$ are conjugate. Indeed, all order-$2$ elements are reflections (there is no $180^\circ$ rotation), which can all be rotated into each other, and this rotation corresponds to conjugation. Here is the image of rotating a translation for $D_{18},$ the symmetries of a nonagon.
\begin{center}
	\begin{asy}
		unitsize(1.3cm);
		// centering abuse
		draw((-4,0)--(4,0), white);
		int n = 9;
		path p;
		for(int i = 0; i < n; ++i)
		{
			p = p -- dir(360.0/n * i + 90);
			dot(dir(360.0/n * i + 90));
		}
		draw((0,1.3) -- (0,-1.3), red+dashed);
		label("\color{red}$a$", (0,-1.3), S);
		draw(arc((0,0), 1.2, 90-2*(360/n), 90-(360/n)), arrow=EndArrow, p=blue);
		draw((0,0) -- 1.3 * dir( 90-2*(360/n) ), blue);
		draw((0,0) -- 1.3 * dir( 90-(360/n) ), blue);
		label("\color{blue}$b$", 1.3 * dir( 90-2*(360/n) ), E);
		draw(1.3 * dir(90 + 360/n) -- -1.3*dir(90+360/n), dashed+purple);
		label("{\color{blue}$b$}{\color{red}$a$}{\color{blue}$b^{-1}$}", -1.3*dir(90+360/n), E);
		draw(p -- cycle);
		for(int i = 0; i < n; ++i)
		{
			dot(dir(360.0/n * i + 90));
		}
	\end{asy}
\end{center}
Regardless, $D_{4n}$ only has three types of involutions: a $180^\circ$ rotation, reflection where the line goes through a vertex, and reflection where the line goes through the midpoint of a side. Here's a picture for the three types in $D_{20},$ the symmetries of a decagon.
\begin{center}
	\begin{asy}
		unitsize(1.3cm);
		// centering abuse
		draw((-4,0)--(4,0), white);
		int n = 10;
		path p;
		for(int i = 0; i < n; ++i)
		{
			p = p -- dir(360.0/n * i);
			dot(dir(360.0/n * i));
		}
		draw((0,1.3) -- (0,-1.3), red+dashed);
		draw(arc((0,0), 1.2, (360/n)*(int)(n/4), (360/n)*(int)(n/4+n/2)), arrow=EndArrow, p=blue);
		draw((0,0) -- 1.3 * dir( (360/n)*(int)(n/4) ), blue);
		draw((0,0) -- 1.3 * dir( (360/n)*(int)(n/4+n/2) ), blue);
		draw(1.3 * dir((360/n)*(int)(n/2-1)) -- -1.3 * dir((360/n)*(int)(n/2-1)), orange+dashed);
		draw(p -- cycle);
		for(int i = 0; i < n; ++i)
		{
			dot(dir(360.0/n * i));
		}
	\end{asy}
\end{center}
(To reiterate, in $D_{4n+2},$ there is no $180^\circ$ rotation, and reflections all go through both a vertex and a side.) In particular, having the $180^\circ$ rotation in the center implies that there is a nontrivial element which commutes with all of our order-$2$ elements.

This property turns out to answer the following question.
\begin{ques}
	Is there a general property which holds for all groups of finite order but fails for some groups of infinite order?
\end{ques}
Well, of course, ``the group is finite'' is some property, but this is not what we want. Namely, we want our property to be stated in terms of group theory: we only want to use group elements, their multiplication structure, and first-order logic.
\begin{nex}
	Here is something which doesn't work: $\forall g\in G,\exists n\in\ZZ, g^n=e.$ This doesn't work because $n\in\ZZ$ is invalid.
\end{nex}
However, this can be done. For example,
\[\forall g,h,\big[gg=hh=e\land g\ne e\ne h\big]\implies\big[(\exists g_0: g_0hg_0^{-1}=g\lor(\exists g_0\ne e,gg_0=g_0g\land hg_0=g_0h)\big].\tag{$*$}\]
In other words, any two elements of order-$2$ are conjugate or have a nontrivial third element commuting with both of them. We see $(*)$ works for any finite group because $g$ and $h$ will generate a dihedral group (generated by the ``reflection'' $g$ and the ``rotation'' $gh$), and we checked that this statement holds for dihedral groups.

However, $(*)$ fails for the ``infinite'' dihedral group as symmetries of $\ZZ,$ which is $\ZZ\rtimes\ZZ/2\ZZ.$ To be explicit, we can imagine this as the group of symmetries of the number line.
\begin{center}
	\begin{asy}
		unitsize(1cm);
		for(int i = -3; i <= 3; ++i)
		{
			dot((i,0));
			label("$"+string(i)+"$", (i,0), S);
		}
		draw((-3.5,0) -- (3.5,0), Arrows);
	\end{asy}
\end{center}
We can check that $g:x\mapsto-x$ and $h:x\mapsto1-x$ are neither conjugate nor commuting with a third element\footnote{They aren't conjugate because $g$ has one fixed point while $h$ has none. They don't commute with any third element by brute force: our group is generated by $x\mapsto x+1$ and $x\mapsto-x,$ so all elements take the form $g_0:x\mapsto\pm x+n$ for some $n\in\ZZ.$ We see $(gg_0)(1)=(g_0hg)(1)$ implies $n=0,$ so the only nontrivial option for $g_0$ is $g$ itself, but $gh\ne hg.$} even though $g^2=h^2=\id,$ so indeed, $(*)$ fails here.

Something else funny about dihedral groups is that some of these split as products.
\begin{itemize}
	\item $D_4\cong\ZZ/2\ZZ\times\ZZ/2\ZZ,$ generated by the vertical and horizontal flips.
	\item $D_{12}\cong D_6\times\ZZ/2\ZZ.$ Here, this is the symmetries of a hexagon, but inside the hexagon we can draw a triangle.
	\begin{center}
		\begin{asy}
			unitsize(1cm);
			path p;
			path q;
			path r;
			for(int i = 0; i < 6; ++i)
			{
				p = p -- dir(60*i);
				if(i % 2 == 1)
					q = q -- dir(60*i);
				else
					r = r -- dir(60*i);
			}
			draw(p -- cycle);
			draw(q -- cycle, blue);
			draw(r -- cycle, orange);
			for(int i = 0; i < 6; ++i)
			{
				dot(dir(60*i));
			}
		\end{asy}
	\end{center}
	We see that we can write $D_{12}$ as the symmetries of the blue triangle, times perhaps a $180^\circ$ rotation mapping the blue triangle to the orange triangle. This turns out to create a direct product, commuting because the $180^\circ$ rotation is in the center.
	\item In general, $D_{8k+4}\cong D_{2k}\times\ZZ/2\ZZ$ by generalizing the above argument.
\end{itemize}
\begin{remark}
	The above product decomposition does not work for (say) $D_{16}.$ If we tried, we would get the following two squares.
	\begin{center}
		\begin{asy}
			unitsize(1cm);
			path p;
			path q;
			path r;
			for(int i = 0; i < 8; ++i)
			{
				p = p -- dir(45*i);
				if(i % 2 == 1)
					q = q -- dir(45*i);
				else
					r = r -- dir(45*i);
			}
			draw(p -- cycle);
			draw(q -- cycle, blue);
			draw(r -- cycle, orange);
			for(int i = 0; i < 8; ++i)
			{
				dot(dir(45*i));
			}
		\end{asy}
	\end{center}
	The issue is that the $180^\circ$ rotation that is supposed to send the blue square to the orange one already lives in the symmetries of the embedded square. Perhaps we could map the blue square to the red one in some other way, but this would lose being a direct product because $180^\circ$ is the only element of the center.
\end{remark}
\begin{remark}
	It's not hard to see that any group generated by two elements of order $2$ are either abelian or dihedral. However, for two elements of order three turns into a terrible mess: for example, we can achieve any finite simple group. So elements of order two are nice.
\end{remark}

\subsection{Groups of Orders 11 and 12}
Groups of order $11$ are cyclic because $11$ is prime. So let's just jump into $12.$ There are five of them; here are some obvious ones:
\[\ZZ/6\ZZ\times\ZZ/\ZZ,\qquad S_3\times\ZZ/2\ZZ,\qquad\ZZ/12\ZZ,\qquad A_4.\]
We note that $\ZZ/3\ZZ\times\ZZ/4\ZZ\cong\ZZ/12\ZZ$ and $D_{12}\cong S_3\times\ZZ/2\ZZ,$ so these are already in our list. So where's the fifth group? Well, let's find out.

\subsection{Sylow Time}
Let's return to our attempt at reversing Lagrange.
\begin{ques} \label{ques:sylow}
	We know that $H\subseteq G$ as groups implies $\#H\mid\#G.$ But if we have $n\mid\#G,$ then is there a subgroup of order $n$?
\end{ques}
The answer turns out to be no: $A_4$ has no subgroups of order $6.$ (Again, we can check this by hand.)

However, we can salvage \autoref{ques:sylow}: it turns out to be true if $n$ is a prime power, which is a special case of the Sylow theorems.
\begin{remark}
	Nobody actually knows how to pronounce ``Sylow.'' There's no point trying to pronounce it correctly because no matter how hard you try, a Norwegian will smile patronizingly at you and tell you you're wrong.
\end{remark}
Here's the statement.
\begin{thm}[Sylow, I] \label{thm:sylow1}
	Fix $p^{\nu_p(n)}$ the largest prime power of $p$ dividing $n:=\#G<\infty.$ Then there is a subgroup of order $p^{\nu_p(n)}.$
\end{thm}
\begin{defi}[Sylow subgroups]
	The subgroups in \autoref{thm:sylow1} are called Sylow $p$-subgroups.
\end{defi}
\begin{proof}
	There are two possibilities.
	\begin{itemize}
		\item If there are some subgroups which have proper subgroups with index not divisible by $p,$ then we can just induct on one of these subgroups because their orders will also be divisible by $p^{\nu_p(n)}.$
		\item Otherwise, all proper subgroups have index divisible by $p.$ But we saw in \autoref{rem:inddivp} that this implies that $Z(G)$ has order divisible by $p.$ So Cauchy's theorem gives us an element $g\in Z(G)$ with order $p.$ In particular, we have the short exact sequence
		\[1\to\langle g\rangle\to G\to G/\langle g\rangle\to 1.\]
		Indeed, because $\langle g\rangle$ is in the center, $\langle g\rangle$ is normal, so $G/\langle g\rangle$ is actually a group. To finish, we use induction to get a Sylow $p$-subgroup of $G/\langle g\rangle,$ and we can pull this backwards along the modulo by $g$ map to get a subgroup of $G$ of the correct order.
		\qedhere
	\end{itemize}
\end{proof}
\begin{remark}
	The end of this proof uses the fact that pre-images of subgroups are subgroups. To see why this is true, fix $\varphi:G\to H$ a group homomorphism and $B\subseteq H$ a subgroup. Then $A:=\varphi^{-1}(B)$ contains $e\in\varphi^{-1}(\{e\}),$ is closed under multiplication ($\varphi(a_1),\varphi(a_2)\in B\implies\varphi(a_1a_2)\in B$), and is closed under inversion ($\varphi(a)\in B\implies\varphi(a^{-1})=\varphi(a)^{-1}\in B$).
\end{remark}

\subsection{Back to Groups of Order 12}
Let's return to groups of order $12$; fix $G$ with $\#G=12.$ From \autoref{thm:sylow1}, we see that $G$ has a subgroup of order $3$ and a group of order $4.$ We would like a normal subgroup; do we have one? Well, let's do casework.
\begin{enumerate}[label=(\alph*)]
	\item If our subgroup $H_3$ of order $3$ is normal, then $G=\ZZ/3\ZZ\rtimes H_4,$ where $H_4$ is our Sylow $2$-subgroup.
	\item If $H_3$ is not normal, then it has four conjugates, by another Sylow theorem we will prove later. This gives us $4\cdot2$ elements of order $3,$ so we have exactly four elements left over, which must be our $H_4$ of order $4,$ and we see that $H_4$ is normal because we can only have one of them.
	
	So in this case, we see that $G=H_4\rtimes\ZZ/3\ZZ$ by letting an $H_3\cong\ZZ/3\ZZ$ act on our Sylow $2$-subgroup.
\end{enumerate}
We now work out our cases separately.

\subsubsection{Case (b)}
We start with (b). We have the following table.
\[\begin{array}{c|cc}
	 & \text{trivial }\ZZ/3\ZZ\text{-action} & \text{nontrivial }\ZZ/3\ZZ\text{-action}  \\\hline
	(\ZZ/2\ZZ)^2 &  \\
	\ZZ/4\ZZ & 
\end{array}\]
When $\ZZ/3\ZZ$ acts trivially, our group is abelian, so the top left is $(\ZZ/2\ZZ)^2\times\ZZ/3\ZZ,$ and the bottom left is $\ZZ/12\ZZ.$
\[\begin{array}{c|cc}
	 & \text{trivial }\ZZ/3\ZZ\text{-action} & \text{nontrivial }\ZZ/3\ZZ\text{-action}  \\\hline
	(\ZZ/2\ZZ)^2 & (\ZZ/2\ZZ)^2\times\ZZ/3\ZZ \\
	\ZZ/4\ZZ & \ZZ/12\ZZ
\end{array}\]
We now work on the right column. We need to consider a nontrivial map $\ZZ/3\ZZ\to\op{Aut}\left((\ZZ/2\ZZ)^2\right).$ Writing our $(\ZZ/2\ZZ)^2$ as $\{e,a_1,a_2,a_3\}\subseteq G,$ we see that an element of $\op{Aut}\left((\ZZ/2\ZZ)^2\right)$ must fix $e$ and hence essentially be a permutation in $S_3$ on $\{a_1,a_2,a_3\}.$ It turns out that these are all actually automorphisms.\footnote{I think the most ``pure thought'' way to see this is to view $(\ZZ/2\ZZ)^2$ as a $\ZZ$-module, so we see that it is actually a $\FF_2$-vector space, so $\op{Aut}(\left(\ZZ/2\ZZ)^2\right)=\op{Hom}_\ZZ\left((\ZZ/2\ZZ)^2,(\ZZ/2\ZZ)^2\right)=\op{GL}_2(\FF_2).$ As described previously, we have an injection $\op{Aut}(\left(\ZZ/2\ZZ)^2\right)\hookrightarrow S_3,$ but $\#\op{GL}_2(\FF_2)=6,$ so $\op{Aut}(\left(\ZZ/2\ZZ)^2\right)\cong S_3.$}

Now, if $\ZZ/3\ZZ\to\op{Aut}\left((\ZZ/2\ZZ)^2\right)\cong S_3$ is to be nontrivial, then we need to send our order-$3$ element (which we name conjugation by $b$) to a three-cycle in $S_3.$ By switching around our elements, it doesn't matter which one, so we have the restrictions
\[\left\langle b^3=a_1^2=a_2^2=a_3^2=a_1a_2a_3^{-1}=e,\quad ba_1b^{-1}=a_2,\quad ba_2b^{-1}=a_3,\quad ba_3b^{-1}=a_1\right\rangle,\]
or after doing some reduction,
\[\left\langle b^3=a^2=(ab)^3=e\right\rangle.\]
This turns out to be $A_4$ because we can take $a=(12)(34)$ and $b=(123).$ (We can check by hand that there are twelve elements in the above group presentation.) So our table looks like the following.
\[\begin{array}{c|cc}
	 & \text{trivial }\ZZ/3\ZZ\text{-action} & \text{nontrivial }\ZZ/3\ZZ\text{-action}  \\\hline
	(\ZZ/2\ZZ)^2 & (\ZZ/2\ZZ)^2\times\ZZ/3\ZZ & A_4\\
	\ZZ/4\ZZ & \ZZ/12\ZZ
\end{array}\]
Lastly we can have $\ZZ/3\ZZ$ act on $\ZZ/4\ZZ.$ However, $\op{Aut}(\ZZ/4\ZZ)\cong(\ZZ/4\ZZ)^\times\cong(\ZZ/2\ZZ)$ has no order-$3$ elements, so this is impossible. So in total, we have the following table.
\[\begin{array}{c|cc}
	 & \text{trivial }\ZZ/3\ZZ\text{-action} & \text{nontrivial }\ZZ/3\ZZ\text{-action}  \\\hline
	(\ZZ/2\ZZ)^2 & (\ZZ/2\ZZ)^2\times\ZZ/3\ZZ & A_4\\
	\ZZ/4\ZZ & \ZZ/12\ZZ & \text{impossible}
\end{array}\]

\subsubsection{Case (a)}
In part (a), we have the following table.
\[\begin{array}{c|cc}
	 & \text{trivial action} & \text{nontrivial action}  \\\hline
	\text{by }\ZZ/4\ZZ &  \\
	\text{by }(\ZZ/2\ZZ)^2 &  \\
\end{array}\]
Here everything is acting on our Sylow $3$-subgroup $\ZZ/3\ZZ.$ In the left column, our group is abelian, so we can fill these out.
\[\begin{array}{c|cc}
	 & \text{trivial action} & \text{nontrivial action}  \\\hline
	\text{by }\ZZ/4\ZZ & \ZZ/12\ZZ & \\
	\text{by }(\ZZ/2\ZZ)^2 & (\ZZ/2\ZZ)^2\times\ZZ/3\ZZ &  \\
\end{array}\]
In the bottom right, we have $(\ZZ/2\ZZ)^2$ acting nontrivially on $\ZZ/3\ZZ.$ Again, there is only nontrivial automorphism of $\ZZ/3\ZZ,$ so we had better send some element of $(\ZZ/2\ZZ)^2$ there. Without loss of generality, we send $(1,0)$ to the nontrivial automorphism; then exactly one of $(1,1)$ or $(0,1)$ will be nontrivial as well, so we'll say $(0,1)$ is nontrivial. In total, we have the presentation
\[\left\langle a_1^2=a_2^2=b^3=e,\quad a_1ba_1^{-1}=a_2ba_2^{-1}=b^2\right\rangle.\]
This turns out to be $\ZZ/2\ZZ\times S_3.$ For example, we can take $a_1\mapsto(1,\id)$ and $a_2\mapsto(1,(12))$ and $b\mapsto(0,(123)).$ So our table so far is the following.
\[\begin{array}{c|cc}
	 & \text{trivial action} & \text{nontrivial action}  \\\hline
	\text{by }\ZZ/4\ZZ & \ZZ/12\ZZ & \\
	\text{by }(\ZZ/2\ZZ)^2 & (\ZZ/2\ZZ)^2\times\ZZ/3\ZZ &  \ZZ/2\ZZ\times S_3 \\
\end{array}\]
In the top right, we have $\ZZ/4\ZZ$ acting nontrivially on $\ZZ/3\ZZ.$ Well, $\op{Aut}(\ZZ/3\ZZ)\cong\ZZ/2\ZZ,$ letting $\ZZ/4\ZZ\cong\langle a\rangle\subseteq G$ and $\ZZ/3\ZZ\cong\langle b\rangle\subseteq G,$ we are forced into $aba^{-1}=b^{-1}.$ So we have the following presentation.
\[\left\langle a^4=b^3=e,\quad aba^{-1}=b^2\right\rangle.\]
And here, we call it quits, having more or less classified all groups of order $12.$
\begin{remark}
	This last group is hard to visualize. It turns out to be a binary dihedral group. Namely, we recall that our binary dihedral groups were defined as pull backs from $\op{SO}_3(\RR)$ in the short exact sequence
	\[1\to\{\pm1\}\to S^3\to\op{SO}_3(\RR)\to 1.\]
	Namely, we take $D_6\cong S_3\subseteq\op{SO}_3(\RR)$ as the group of symmetries of a triangle (which is in $\op{SO}_3(\RR)$ by placing the triangle in $3$-space) and pull it back into $S^3$ to get $G_6.$
\end{remark}

\subsection{Back to Sylow}
Let's go back and prove that one fact that told us $H_3\subseteq G$ has four subgroups of order $3$ if not normal. We will go through the Sylow theorems one at a time, though we will not do this in order.
\begin{thm}[Sylow, III(a)] \label{thm:sylow3a}
	Fix $G$ a finite group and $p$ prime. Then the number $n_p$ of Sylow $p$-subgroups is $1\pmod p.$
\end{thm}
\begin{proof}
	We show that $n_p\equiv1\pmod p.$ The following lemma is the meat of the argument.
	\begin{lem}
		Fix $G$ a finite group with Sylow $p$-subgroups named $S_1$ and $S_2.$ Then $S_1$ does not normalize $S_2.$
	\end{lem}
	\begin{proof}
		Note that if $S_1$ normalizes $S_2$ for Sylow $p$-subgroups $S_1$ and $S_2,$ then $S_1S_2$ is a subgroup of order which is larger than $p^{\nu_p(n)},$ which is a contradiction.
		
		For completeness, we check that $S_1S_2$ a subgroup if $S_1$ normalizes $S_2.$ Well, for $g_1,g_2\in S_1$ and $h_1,g_2\in S_2,$ then
		\[(g_1h_1)(g_2h_2)=(g_1g_2)(g_1^{-1}h_1g_1h_2)\in S_1S_2\]
		because $S_1$ normalizes $S_2.$ This gives us closure under multiplication. For closure under inversion, we see $g\in S_1$ and $h\in S_2$ has $(gh)^{-1}=h^{-1}g^{-1}=g^{-1}gh^{-1}g^{-1}\in S_1S_2.$
	\end{proof}
	
	To finish, fix a Sylow $p$-subgroup $S.$ We check the action of $S$ on all Sylow $p$-subgroups by conjugation. One orbit is $\{S\}$ because $S$ should certainly fix itself. Otherwise, all orbits have at least one element because $S$ doesn't normalize any other Sylow $p$-subgroup.
	
	However, each orbit size not $1$ still must divide $\#S$ by Orbit-stabilizer, so the size of each orbit which isn't $\{S\}$ must be divisible by $p.$ It follows that the sum of the sizes of all orbits is $1\pmod p,$ where the $1$ comes from $\{S\}.$
\end{proof}
\begin{thm}[Sylow, II]
	Fix $G$ a finite group and $p$ prime. Then all Sylow $p$-subgroups are conjugate.
\end{thm}
\begin{proof}
	% I don't understand this argument, so I'm going to sub in the one that I understand
	% Suppose for the sake of contradiction that Sylow $p$-subgroups $S$ and $T$ are not conjugate. Then look at the action of $T$ on $S$ by conjugation; simply borrowing the above argument we see that the number of conjugates of $T$ is $1\pmod p,$ so the number of $T$-conjugates of any Sylow $p$-subgroup is divisible by $p,$ implying that the number of Sylow $p$-subgroups is divisible by $p,$ which is a contradiction.
	Fix $S$ and $T$ two Sylow $p$-subgroups so that we want to show they are conjugate. The trick is to let $T$ act on the left cosets $G/S$ of $S.$ We note that we can use the end of the previous argument on this action to note that the number of fixed points by this action equals the number size-$1$ orbits, which is
	\[[G:S]\pmod p\]
	after adding in the sizes of all the other orbits (which divide $T$ and hence are divisible by $p$). Because $[G:S]\not\equiv0\pmod p,$ there is some fixed point; namely, for some $gP\in G/P,$ we have $tgP=gP$ for any $t\in T.$ It follows $T\subseteq gPg^{-1},$ so $T=gPg^{-1}$ for size reasons. This finishes.
\end{proof}
\begin{thm}[Sylow, III(b)] \label{thm:sylow3b}
	Fix $G$ a finite group and $p$ prime. Then the number $n_p$ of Sylow $p$-subgroups divides $\#G.$
\end{thm}
\begin{proof}
	Lastly, because all Sylow $p$-subgroups are conjugate, we see that their number is
	\[\frac{\#G}{\#\{g\in G:g\text{ normalizes some }S\}},\]
	which divides $\#G.$ Namely, this is just the Orbit-stabilizer theorem because there is only orbit, so the size of this orbit is the index in $G.$
\end{proof}
\begin{remark}
	We used \autoref{thm:sylow3a} in our classification of groups of order $12$: the number of Sylow $3$-subgroups needed to be $1\pmod3$ and divide into $12$ and hence must have been $1$ or $4.$
\end{remark}

As a consequence of the Sylow theorems, because all Sylow subgroups are conjugate, it follows that they are isomorphic. However, this is not true in general.
\begin{ex}
	The group $\ZZ/2\ZZ\times\ZZ/2\ZZ$ has non-conjugate subgroups of order $2.$ Namely, it has more than one subgroup of order $2,$ and $\ZZ/2\ZZ\times\ZZ/2\ZZ$ is abelian, so these are normal. (However, there is an outer automorphism connecting them.)
\end{ex}
\begin{ex}
	The group $D_8$ has subgroups isomorphic to $\ZZ/4\ZZ$ and $\ZZ/2\ZZ\times\ZZ/2\ZZ,$ which are not even isomorphic though they both have order $4.$
\end{ex}
So it is somewhat surprising that looking at the largest power of $p$ forces the $p$-subgroups to be isomorphic.

\subsection{Nilpotent Groups}
Recall that if $\#G=p^n>1,$ then we know that $G$ has nontrivial center. We showed this by using the class equation. This lets us fix $Z_1:=Z(G)$ and we note that $G_1:=G/Z_1$ again has order a power of $p$; if it is trivial, we declare we are done, and otherwise we can fix $Z_2$ to be the pre-image of $Z(G/Z_1)$ in $G$ and look at $G_2:=G/Z_2.$ Then we can just continue this inductively.

Note that killing the center by modding might still have a center afterwards, so this process isn't trivial.
\begin{ex}
	With $Q_8,$ we have center $\{\pm1\},$ but $G/\{\pm1\}$ has center $\ZZ/2\ZZ\times\ZZ/2\ZZ$ because $Q_8/\{\pm1\}$ is abelian.
\end{ex}
What's happening is that we get the sequence
\[\{e\}=Z_0\subseteq Z_1\subseteq Z_2\subseteq\cdots\subseteq Z_{n-1}\subseteq Z_n=G,\]
where $G_0=\{e\}$ and $Z_{k+1}/Z_k=Z(G/Z_k).$
\begin{defi}[Nilpotent]
	Call a group $G$ \textit{nilpotent} if $G$ has such a chain.
\end{defi}
\begin{nex}
	The group $S_3$ is not nilpotent. Indeed, $S_3$ has trivial center, so the chain of taking centers never descends.
\end{nex}
It turns out that nilpotence and Sylow subgroups are connected.
\begin{prop}
	The following are equivalent for a finite group $G.$
	\begin{enumerate}[label=(\alph*)]
		\item $G$ is nilpotent.
		\item All proper subgroups $H$ have normalizer strictly bigger than $H.$
		\item All Sylow $p$-subgroups are normal.
		\item $G$ is a product of groups order a power of prime.
	\end{enumerate}
\end{prop}
\begin{proof}
	We take these one at a time, in varying amounts of detail.
	\begin{itemize}
		\item We show that (a) implies (b). We show the contrapositive: suppose that we have a proper subgroup $H\subsetneq G$ such that $N(H)=H,$ and we show that $G$ is not nilpotent.

		The idea is to use $H$ to bound the subgroup chain. Indeed, we show that $Z_k\subseteq H\subsetneq G$ for each $k,$ which keeps $G$ from being nilpotent. Certainly this is true for $Z_0=\{e\}.$

		For the inductive step, take $Z_k\subseteq H.$ Now, $g\in Z_{k+1}$ implies that $gZ_k\in Z(G/Z_k)$ implies that
		\[(gh)Z_k=gZ_k\cdot hZ_k=hZ_k\cdot gZ_k=(hg)Z_k\]
		for any $h\in G.$ In particular, we see that $ghg^{-1}h^{-1}\in Z_k\subseteq H$ for any $h\in G.$
		
		Now, taking $h\in H,$ we see that $ghg^{-1}\in H$ for each $h\in H,$ so $g\in N(H)=H$! So indeed, $g\in H,$ from which it follows that $Z_{k+1}\subseteq H,$ completing our induction.

		\item We show that (b) implies (c). Fix $P$ a Sylow $p$-subgroup so that we want to show $P$ is normal in $G.$ Well, it suffices to show that $N(P)=G.$

		The main claim is that $N(N(P))=N(P).$ From this it will follow that $N(P)$ is not a proper subgroup, forcing $N(P)=G.$ Certainly $N(P)\subseteq N(N(P)),$ so we spend our time with $N(N(P))\subseteq N(P).$

		Well, fixing any $g\in G,$ we see that $g\in N(N(P))$ implies
		\[N\left(gPg^{-1}\right)=gN(P)g^{-1}=N(P).\]
		In particular, $P\subseteq N(P)=N\left(gPg^{-1}\right).$
		
		But $gPg^{-1}$ is normal in $N\left(gPg^{-1}\right)$ while both $gPg^{-1}$ and $P$ are both Sylow $p$-subgroups of $N\left(gPg^{-1}\right)$ (compare the powers of $p$). Because Sylow $p$-subgroups are all conjugate, it follows that
		\[P=gPg^{-1}\]
		because $gPg^{-1}$ is normal in $N\left(gPg^{-1}\right).$ So indeed, $g\in N(P).$

		\item We show that (c) implies (d). This follows from the homework; fix $H_1,\ldots,H_n$ our Sylow $p$-subgroups.
		
		The main claim is that, for $N_1$ and $N_2$ normal subgroups with trivial intersection, we have
		\[N_1N_2\cong N_1\times N_2.\]
		Our isomorphism is defined by $N_1\times N_2\to N_1N_2$ with
		\[\varphi:(n_1,n_2)\mapsto n_1n_2.\]
		We see $\varphi$ is homomorphic because $n_1n_2=n_2n_1$ for any $n_1\in N_1,n_2\in N_2$ because $n_1n_2n_2^{-1}n_1^{-1}\in N_1\cap N_2=\{e\}.$ We see $\varphi$ is surjective by definition of $N_1N_2.$ Lastly, we see that $\varphi$ has trivial kernel because $n_1n_2=e$ implies that $n_1=n_2^{-1}\in N_1\cap N_2=\{e\}$ implies $n_1=n_2=e.$

		Now, we can simply inductively say
		\[H_1H_2\cdots H_n\cong H_1\times H_2\cdots H_n\cong H_1\times H_2\times H_3\cdots H_n\cong\cdots\cong H_1\times H_2\times\cdots\times H_n.\]
		This induction works because $H_{k+1}H_{k+2}\cdots H_n$ is a normal subgroup always\footnote{Note $gH_{k+1}H_{k+2}\cdots H_ng^{-1}=gH_{k+1}g^{-1}\cdot gH_{k+2}g^{-1}\cdots gH_ng^{-1}=H_1H_2\cdots H_n.$} and has order coprime to $H_k$ (and hence trivial intersection) because the prime-power orders separate.

		\item We show that (d) implies (a). The point that (d) means that $G$ is the product of its Sylow $p$-subgroups, and we know that $p$-groups are nilpotent from the above discussion. It follows that $G$ is also nilpotent by attaching the chains together; we will not be rigorous about this because I cannot be bothered.
		\qedhere
	\end{itemize}
\end{proof}

The point is that nilpotent groups are the ones which are the product of groups of prime-power order, which seems very nice. However, it turns out that there are lots of groups of prime-power order.

\subsection{Groups of Order 13, 14, and 15}
We see that $13$ is prime, so all groups are cyclic. As for $14,$ it's twice a prime, so it's either cyclic or $D_{14}.$

So let's look at groups of order $15.$
\begin{prop}
	Suppose $G$ is a group with $\#G=pq$ with $p<q$ primes. Then $G=\ZZ/q\ZZ\rtimes\ZZ/p\ZZ.$ If $q\not\equiv1\pmod p,$ then $G$ is cyclic.
\end{prop}
\begin{proof}
	The number of Sylow $q$-subgroups is $1\pmod q$ and divides $pq,$ so it must be $1.$ So our Sylow $q$-subgroup is normal, which forces $G\cong\ZZ/q\ZZ\rtimes\ZZ/p\ZZ.$
	
	In particular, if $q\not\equiv1\pmod p,$ then the action of $\ZZ/p\ZZ$ on $\ZZ/q\ZZ$ must be trivial because $\op{Aut}(\ZZ/q\ZZ)\cong\ZZ/(q-1)\ZZ$ has no order-$p$ elements.
\end{proof}
\begin{ex}
	With $q=5$ and $p=3,$ we see that $5\not\equiv1\pmod 3,$ so any group of order $15$ is cyclic.
\end{ex}
\begin{ex}
	With $q=7$ and $p=3,$ we do have a nonabelian group of order $\#G=21,$ though it is still $\ZZ/7\ZZ\rtimes\ZZ/3\ZZ.$ We can represent this group by
	\[\left\{\begin{bmatrix}a & b \\ 0 & 1\end{bmatrix}:a,b\in\FF_7\text{ and }a\in\{1,2,4\}\right\}.\]
	This is closed under multiplication because $a\in\{1,2,4\}$ is the same thing as $a\in(\FF_7^\times)^{\times2}.$
\end{ex}

\subsection{Groups of Order 16}
We're not going to classify all groups of order $16$ because it is a mess. However, we can list them. We have the following cases.
\begin{itemize}
	\item Abelian: we have $\ZZ/16\ZZ,$ $\ZZ/8\ZZ\times\ZZ/2\ZZ,$ $\ZZ/4\ZZ\times\ZZ/4\ZZ,$ $\ZZ/4\ZZ\times(\ZZ/2\ZZ)^2,$ and $(\ZZ/2\ZZ)^4.$
	\item There are $4$ cyclic subgroups of order $8$: There is a generalized quaternion group, which is binary dihedral. Otherwise, there is an element $a$ of order eight and an element $b$ of order $2.$ Then we have the cases $bab^{-1}\in\{a,a^3,a^5,a^7\}.$ Not all of these even have names.
	\item There are products: $Q_8\times\ZZ/2\ZZ$ and $D_8\times\ZZ/2\ZZ.$
	\item There are semidirect products: $\ZZ/4\ZZ\rtimes\ZZ/4\ZZ$ and $(\ZZ/2\ZZ)^2\rtimes\ZZ/4\ZZ.$ Also there is $(\ZZ/2\ZZ\times\ZZ/4\ZZ)\rtimes\ZZ/2\ZZ.$ This is sometimes called the Pauli group because it is the group generated by the Pauli matrices.
\end{itemize}
So yes, this list is rather a mess. It turns out that as we add more powers of $2,$ it just gets worse. It's just that $2$-groups and $p$-groups in general have terrible structure.

\subsection{Classification of Finitely Generated Abelian Groups}
So we gave up on classifying all groups of order $16,$ but we can classify the abelian ones.
\begin{thm}[Classification of finitely generated abelian groups] \label{thm:finab}
	Any finitely generated abelian group is a product of cyclic groups.
\end{thm}
\begin{remark}
	This is not unique because, for example, $\ZZ/6\ZZ\cong\ZZ/3\ZZ\times\ZZ/2\ZZ.$ However, we can make this unique by forcing
	\[G\cong\bigoplus_{k=1}^N\ZZ/n_k\ZZ\]
	with $n_1\mid n_2\mid\cdots\mid n_N$ or by forcing the $n_\bullet$ to be prime-powers. Either of these gives us uniqueness, though using prime powers is only unique up to ordering the prime powers.
\end{remark}
\begin{ex}
	We can classify all groups of order $p^5.$ This comes down to writing down all the permutations of $5,$ which are
	\[\begin{array}{c}
		5,\qquad 4+1,\qquad 3+2,\qquad 3+1+1, \\
		2+2+1,\qquad 2+1+1+1+1,\qquad 1+1+1+1+1+1.
	\end{array}\]
	Each partition gives us a group as follows.
	\[\begin{array}{c}
		\ZZ/p^5\ZZ,\qquad\ZZ/p^4\ZZ\times\ZZ/p\ZZ,\qquad\ZZ/p^3\ZZ\times\ZZ/p^2\ZZ,\qquad\ZZ/3\ZZ\times(\ZZ/p\ZZ)^2, \\
		\left(\ZZ/p^2\ZZ\right)^2\times\ZZ/p\ZZ,\qquad\ZZ/p^2\ZZ\times(\ZZ/p\ZZ)^3,\qquad(\ZZ/p\ZZ)^5.
	\end{array}\]
\end{ex}
Next lecture we will prove \autoref{thm:finab}.