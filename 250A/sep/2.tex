\documentclass[../notes.tex]{subfiles}

\begin{document}

% !TEX root = ../main.tex









Why do I hear boss music?

\subsection{Groups of Order 8}
Last time we classified all groups of order $6.$ Note that groups of order $7$ are cyclic because $7$ is prime.

So let's look at order $8.$ Fix $G$ a group of order $8.$ Note that our orders are all in $\{1,2,4,8\}.$ If there's an element of order $8,$ are cyclic, so we may ignore this order. So we have two possibilities.
\begin{itemize}
	\item If all elements have order $2,$ then we see that all elements commute (again, $abab=e$ implies $ab=ba$), so $G$ is a vector space over $\FF_2,$ so we are $G\cong\FF_2^3\cong(\ZZ/2\ZZ)^3$ by size reasons.
	\item Otherwise there is at least one element of order $4.$ Calling this element $a\in G,$ then we have an order-$4$ subgroup $\langle a\rangle,$ which is index $2$ and hence normal. So, as usual, we get a short exact sequence
	\[1\to\underbrace{\ZZ/4\ZZ}_{\langle a\rangle}\to G\to\ZZ/2\ZZ\to 1.\]
	So we have another extension problem to fill in $G.$ Some possibilities for $G$ include $\ZZ/4\ZZ\times\ZZ/2\ZZ$ or $\ZZ/8\ZZ$ (even though we don't care about this case currently), but perhaps there are others.
\end{itemize}
The point of our short exact sequence is that we have a $\ZZ/2\ZZ$-action on $\langle a\rangle$ by conjugation because $\langle a\rangle$ is abelian: given any coset $b\langle a\rangle\in G/\langle a\rangle,$ the action of $b$ on $\langle a\rangle$ only depends on the coset.

So we need to understand the actions of $\ZZ/2\ZZ$ on $\langle a\rangle.$ Well, $\langle a\rangle\cong\ZZ/4\ZZ$ only has the automorphisms $\id$ and $a^k\mapsto a^{-k}.$ Now, fix $b\in G\setminus\langle a\rangle$ so that we know
\[\begin{cases}
	bab^{-1}=a, & \text{or}\\
	bab^{-1}=a^{-1}.
\end{cases}\]
However, we note that for $b\in G/\langle a\rangle,$ we see that $b^2$ needs to be in $\langle a\rangle,$ so $b^2\in\{1,a,a^2,a^3\},$ but in fact $b^2=a$ and $b^2=a^3$ are the same by taking $a\mapsto a^{-1}.$ This gives us lots of cases, which we tabulate.
\[\begin{array}{c|cc}
			& bab^{-1}=a & bab^{-1}=a^{-1}\\\hline
	b^2=e   & ? & ? \\
	b^2=a   & ? & ?\\
	b^2=a^2 & ? & ?
\end{array}\]
We note that $bab^{-1}=a$ forces our group to be abelian because $a$ and $b$ generate. We now go through these in sequence.
\begin{itemize}
	\item The case of $b^2=e$ gives us $G\cong\langle b\rangle\times\langle a\rangle\cong\ZZ/2\ZZ\times\ZZ/4\ZZ.$
	\item The case $b^2=a$ gives $\ZZ/8\ZZ$ ($b$ has order $8$).
	\item In the last case, we see that $(ba)^2=1,$ so $ba\mapsto b$ throws us into the abelian with $b^2=e$ case, so we have $\ZZ/2\ZZ\times\ZZ/4\ZZ$ again.
\end{itemize}
\begin{remark}
	The case of $b^2=e$ makes the short exact sequence
	\[1\to\langle a\rangle\to G\to G/\langle a\rangle\to 1\]
	split with $\langle b\rangle$ as our lift of $G/\langle a\rangle.$
\end{remark}
So here is the table so far.
\[\begin{array}{c|cc}
			& bab^{-1}=a & bab^{-1}=a^{-1}\\\hline
	b^2=1   & \ZZ/2\ZZ\times\ZZ/4\ZZ & ?\\
	b^2=a   & \ZZ/8\ZZ & ?\\
	b^2=a^2 & \ZZ/2\ZZ\times\ZZ/4\ZZ & ?
\end{array}\]
Now we start looking at our nonabelian groups.
\begin{itemize}
	\item The case of $b^2=e$ is our \textit{split} case, which is $\ZZ/4\ZZ\rtimes\ZZ/2\ZZ.$ This turns out to be the symmetries of the square, which we name $D_8.$ (Here, $a$ is a rotation by $90^\circ,$ and $b$ is a reflection.)
	\item In the case of $b^2=a,$ we have a problem because the order of $b$ looks like $8.$ In particular, we supposed that we have no element of order $8,$ so $a^2=b^4=e,$ which violates the order of $a.$
	\item The last case is the most interesting: it gives us the quaternion group. Renaming our elements to $i,j,$ we have the group presentation
	\[Q_8:=\left\langle i,j:i^4=j^4=iji^{-1}j=e\right\rangle.\]
	So does this group actually exist? Well, let's realize $Q_8$ as an action on a vector space. It turns out we can write
	\[i=\begin{bmatrix} i & 0 \\ 0 & -i\end{bmatrix},\qquad j=\begin{bmatrix}0 & 1 \\ -1 & 0\end{bmatrix},\qquad k=\begin{bmatrix}0 & i \\ i & 0\end{bmatrix}.\]
	We can check that $i$ and $j$ satisfy the relations needed of them from $Q_8$ and that they generate a group of order $8.$
\end{itemize}
So we have the following table.
\[\begin{array}{c|cc}
			& bab^{-1}=a & bab^{-1}=a^{-1}\\\hline
	b^2=1   & \ZZ/2\ZZ\times\ZZ/4\ZZ & D_8\\
	b^2=a   & \ZZ/8\ZZ & \text{impossible}\\
	b^2=a^2 & \ZZ/2\ZZ\times\ZZ/4\ZZ & Q_8
\end{array}\]
In total, we have the following proposition.
\begin{prop}
	We have the following classification of groups of order $8.$
	\begin{itemize}
		\item Abelian: $\ZZ/8\ZZ,$ $\ZZ/4\ZZ\times\ZZ/2\ZZ,$ and $(\ZZ/2\ZZ)^3.$
		\item Nonabelian: $D_8,Q_8.$
	\end{itemize}
\end{prop}
\begin{proof}
	Given above.
\end{proof}

\subsection{Quaternion Talk}
Let's study $D_8$ and $Q_8$ a bit more closely by studying their subgroups. Before giving the subgroup lattice for $D_8,$ we name our elements more concisely. They are as follows.
\begin{center}
	\begin{asy}
		unitsize(1.3cm);
		// ugh fine I'll code it correctly
		void drawSquare(pair b, string name, int p1, int p2, int p3, int p4)
		{
			int[] e = {p1,p2,p3,p4};
			pair[] vs = {(0,0), (1,0), (1,1), (0,1)};
			pair[] dirs = {NE, NW, SW, SE};
			draw(b+vs[0] -- b+vs[1] -- b+vs[2] -- b+vs[3] -- cycle);
			for(int i = 0; i < 4; ++i)
				label("$"+string(i+1)+"$", b+vs[i], dirs[i]);
			label("$\stackrel{"+name+"}\longrightarrow$", b+(1.5,0.5));
			b += (2,0);
			draw(b+vs[0] -- b+vs[1] -- b+vs[2] -- b+vs[3] -- cycle);
			for(int i = 0; i < 4; ++i)
				label("$"+string(e[i])+"$", b+vs[i], dirs[i]);
		}
		drawSquare((0,0), "e", 1,2,3,4);
		drawSquare((4,0), "90^\circ", 4,1,2,3);
		drawSquare((0,-1.5), "180^\circ", 3,4,1,2);
		drawSquare((4,-1.5), "270^\circ", 2,3,4,1);
		drawSquare((0,-3), "h", 2,1,4,3);
		drawSquare((4,-3), "v", 4,3,2,1);
		drawSquare((0,-4.5), "d_1", 3,2,1,4);
		drawSquare((4,-4.5), "d_2", 1,4,3,2);
	\end{asy}
\end{center}
And so here is the subgroup lattice for $D_8.$
% https://q.uiver.app/?q=WzAsMTEsWzAsMiwiXFxsYW5nbGUgaFxccmFuZ2xlIl0sWzEsMiwiXFxsYW5nbGUgdlxccmFuZ2xlIl0sWzIsMiwiXFxsYW5nbGUgMTgwXlxcY2lyY1xccmFuZ2xlIl0sWzMsMiwiXFxsYW5nbGUgZF8xXFxyYW5nbGUiXSxbNCwyLCJcXGxhbmdsZSBkXzJcXHJhbmdsZSJdLFsyLDMsIlxcbGFuZ2xlIGVcXHJhbmdsZSJdLFsyLDAsIkRfOCJdLFsxLDFdLFsyLDEsIlxcbGFuZ2xlIDkwXlxcY2lyY1xccmFuZ2xlIl0sWzAsMSwiXFxsYW5nbGUgaCx2XFxyYW5nbGUiXSxbNCwxLCJcXGxhbmdsZSBkXzEsZF8yXFxyYW5nbGUiXSxbNSwwLCIiLDIseyJzdHlsZSI6eyJoZWFkIjp7Im5hbWUiOiJub25lIn19fV0sWzUsMSwiIiwwLHsic3R5bGUiOnsiaGVhZCI6eyJuYW1lIjoibm9uZSJ9fX1dLFs1LDIsIiIsMCx7InN0eWxlIjp7ImhlYWQiOnsibmFtZSI6Im5vbmUifX19XSxbNSwzLCIiLDAseyJzdHlsZSI6eyJoZWFkIjp7Im5hbWUiOiJub25lIn19fV0sWzUsNCwiIiwwLHsic3R5bGUiOnsiaGVhZCI6eyJuYW1lIjoibm9uZSJ9fX1dLFsyLDgsIiIsMCx7InN0eWxlIjp7ImhlYWQiOnsibmFtZSI6Im5vbmUifX19XSxbOCw2LCIiLDAseyJzdHlsZSI6eyJoZWFkIjp7Im5hbWUiOiJub25lIn19fV0sWzAsOSwiIiwyLHsic3R5bGUiOnsiaGVhZCI6eyJuYW1lIjoibm9uZSJ9fX1dLFsxLDksIiIsMCx7InN0eWxlIjp7ImhlYWQiOnsibmFtZSI6Im5vbmUifX19XSxbMywxMCwiIiwwLHsic3R5bGUiOnsiaGVhZCI6eyJuYW1lIjoibm9uZSJ9fX1dLFs0LDEwLCIiLDEseyJzdHlsZSI6eyJoZWFkIjp7Im5hbWUiOiJub25lIn19fV0sWzksNiwiIiwxLHsic3R5bGUiOnsiaGVhZCI6eyJuYW1lIjoibm9uZSJ9fX1dLFsxMCw2LCIiLDEseyJzdHlsZSI6eyJoZWFkIjp7Im5hbWUiOiJub25lIn19fV1d
\[\begin{tikzcd}
	&& {D_8} \\
	{\langle h,v\rangle} & {} & {\langle 90^\circ\rangle} && {\langle d_1,d_2\rangle} \\
	{\langle h\rangle} & {\langle v\rangle} & {\langle 180^\circ\rangle} & {\langle d_1\rangle} & {\langle d_2\rangle} \\
	&& {\langle e\rangle}
	\arrow[no head, from=4-3, to=3-1]
	\arrow[no head, from=4-3, to=3-2]
	\arrow[no head, from=4-3, to=3-3]
	\arrow[no head, from=4-3, to=3-4]
	\arrow[no head, from=4-3, to=3-5]
	\arrow[no head, from=3-3, to=2-3]
	\arrow[no head, from=2-3, to=1-3]
	\arrow[no head, from=3-1, to=2-1]
	\arrow[no head, from=3-2, to=2-1]
	\arrow[no head, from=3-4, to=2-5]
	\arrow[no head, from=3-5, to=2-5]
	\arrow[no head, from=2-1, to=1-3]
	\arrow[no head, from=2-5, to=1-3]
\end{tikzcd}\]
Note in particular that all of our order-$4$ subgroups ($\langle h,v\rangle,$ $\langle 90^\circ\rangle,$ and $\langle d_1,d_2\rangle$) are normal because they are index-$2,$ but not all of the order-$2$ subgroups are normal. For example, conjugating $\langle h\rangle$ by $90^\circ$ gives $\langle v\rangle.$ (However, $\langle180^\circ\rangle$ is our center and hence normal.)

And here is the lattice for $Q_8.$
% https://q.uiver.app/?q=WzAsNixbMSwwLCJRXzgiXSxbMCwxLCJcXGxhbmdsZSBpXFxyYW5nbGUiXSxbMSwxLCJcXGxhbmdsZSBqXFxyYW5nbGUiXSxbMiwxLCJcXGxhbmdsZSBrXFxyYW5nbGUiXSxbMSwyLCJcXGxhbmdsZSAtMVxccmFuZ2xlIl0sWzEsMywiXFxsYW5nbGUgZVxccmFuZ2xlIl0sWzEsMCwiIiwwLHsic3R5bGUiOnsiaGVhZCI6eyJuYW1lIjoibm9uZSJ9fX1dLFsyLDAsIiIsMix7InN0eWxlIjp7ImhlYWQiOnsibmFtZSI6Im5vbmUifX19XSxbMywwLCIiLDIseyJzdHlsZSI6eyJoZWFkIjp7Im5hbWUiOiJub25lIn19fV0sWzQsMSwiIiwwLHsic3R5bGUiOnsiaGVhZCI6eyJuYW1lIjoibm9uZSJ9fX1dLFs0LDIsIiIsMix7InN0eWxlIjp7ImhlYWQiOnsibmFtZSI6Im5vbmUifX19XSxbNCwzLCIiLDIseyJzdHlsZSI6eyJoZWFkIjp7Im5hbWUiOiJub25lIn19fV0sWzUsNCwiIiwwLHsic3R5bGUiOnsiaGVhZCI6eyJuYW1lIjoibm9uZSJ9fX1dXQ==
\[\begin{tikzcd}
	& {Q_8} \\
	{\langle i\rangle} & {\langle j\rangle} & {\langle k\rangle} \\
	& {\langle -1\rangle} \\
	& {\langle e\rangle}
	\arrow[no head, from=2-1, to=1-2]
	\arrow[no head, from=2-2, to=1-2]
	\arrow[no head, from=2-3, to=1-2]
	\arrow[no head, from=3-2, to=2-1]
	\arrow[no head, from=3-2, to=2-2]
	\arrow[no head, from=3-2, to=2-3]
	\arrow[no head, from=4-2, to=3-2]
\end{tikzcd}\]
Again, our subgroups $\langle i\rangle,\langle j\rangle,$ and $\langle k\rangle$ are all normal because they are index-$2,$ but in fact all of our subgroups are normal! Indeed, we only have one element of order $2$ (which can be checked by hand), which is $-1,$ and $\langle-1\rangle$ is our center and hence normal.

Also, the $Q_8$ group also creates a group ring, which is called $\mathbb H,$ the \textit{Hamiltonians}.
\begin{defi}[Hamiltonians]
	The \textit{Hamiltonians} $\mathbb H:=\ZZ[Q_8]$ is a noncommutative ring satisfying the relations $i^2=j^2=k^2=ijk=-1$ and $ij=-ji=k^2$ and $jk=-kj=i$ and $ki=-ik=j.$
\end{defi}
Can we go further? There are octonians, but their multiplication isn't even associative, so we don't care much about them.
\begin{remark}
	For some reason, crackpots spend a long time trying to invent new $\RR$-algebras like the above.
\end{remark}

One reason that the quaternions are not too terrible to work with is that we were able to represent them inside of $\CC^{2\times2}$ as given above, so we have a pretty physical realization of these numbers. Also, quaternions are very good at describing rotations. The idea is to embed $\RR^3$ into $\mathbb H$ by
\[\langle x,y,z\rangle\longmapsto xi+yj+zk.\]
Then a quaternion $g\in\mathbb H$ acts on $\langle x,y,z\rangle$ by conjugation: $v\mapsto ghg^{-1}.$ We can check that this is a rotation of $\RR^3,$ which can be done by hand. And we can see that we can achieve all rotations by restricting our view to the elements with norm $1.$ In fact, the norm has the nice properties that $g=a+bi+cj+dk$ has
\[g\overline g=(a+bi+cj+dk)(a-bi-cj+dk)=a^2+b^2+c^2+d^2,\]
so in fact our norm is nicely multiplicative. In other words, we get a surjective homomorphism from $S^3=\{(a,b,c,d)\in\RR^4:a^2+b^2+c^2+d^2=1\}$ to rotations $\op{SO}_3(\RR).$ It turns out that there is nontrivial kernel here, and in fact we have the short exact sequence
\[1\to\{\pm1\}\to S^3\to\op{SO}_3(\RR)\to1,\]
and this sequence turns out to be non-split!
\begin{remark}
	Quaternions only require $4$ numbers to represent a rotation, which is much nicer than representing these as $3\times3$ matrices, which requires more than twice as many numbers. As far as making money is concerned, this is probably the most useful fact you'll learn in this course.
\end{remark}
Our non-split short exact sequence gives us ideas.
\begin{defi}[Binary rotation groups]
	Given a rotation group $G\subseteq\op{SO}_3(\RR),$ we can check what happens when we pull it back into $\op{SO}_3(\RR).$ For example, we can make $G$ the rotations of a cube or the pentagon. The pullback will have twice that size because of the kernel $S^3\to\op{SO}_3(\RR),$ which are called the \textit{binary rotation groups}.
\end{defi}

\subsection{Philosophy}
Our work above more or less classifies all extension problems
\[1\to\ZZ/4\ZZ\to G\to\ZZ/2\ZZ\to1.\]
Doing this in general is hard, but there are tools. For example, the following theorem exists.
\begin{thm}[Schur--Zassenhaus]
	Fix
	\[1\to A\to B\to C\to 1\]
	a short exact sequence such that $\#A$ and $\#C$ are coprime. Then the short exact sequence splits, so $B\cong A\rtimes C.$
\end{thm}
This isn't that terrible to prove, but the following theorem is very hard.
\begin{thm}
	Fix as above. Then all liftings of $C$ into $B$ are conjugate.
\end{thm}
This turns out to be very hard, which requires maybe 300 pages to prove. This happens in group theory, where simple statements turn out to have very long and difficult proofs; roughly speaking, this is because it requires a proof of the Feit--Thompson theorem, which is also notoriously hard (and has been computer-verified!).
\begin{remark}[Nir]
	In fact, this is a general property of math: simple statements can have complicated proofs, and in fact, some simple statements must have complicated proofs. Roughly speaking, this is because determining if a given statement is true is uncomputable.
\end{remark}

\subsection{Rooks on a Chessboard}
We have the following classical problem, which we'll talk about.
\begin{ques} \label{ques:rooks1}
	How many ways can we place $8$ rooks on a chessboard?
\end{ques}
In other words, we are placing $8$ objects on an $8\times8$ grid, none of which are in the same row or column. For example, the following is valid arrangement of rooks in a $4\times4$ grid.
\begin{center}
	\begin{asy}
		unitsize(0.5cm);
		usepackage("chessfss"); // thank you https://tex.stackexchange.com/a/543024
		for(int i = 0; i < 5; ++i)
		{
			draw((i,0)--(i,4));
			draw((0,i)--(4,i));
		}
		label("\rook", (0.5,1.5));
		label("\rook", (1.5,3.5));
		label("\rook", (2.5,2.5));
		label("\rook", (3.5,0.5));
	\end{asy}
\end{center}
The answer to \autoref{ques:rooks1} turns out to be not that hard: it's just $8\times7\times\cdots\times1$ because we can just move from each column, going left to right, choosing a row that hasn't been chosen before to place our new rook. The first column has $8$ options for row, then $7$ options, then $6$ options, and so on, totaling to $8!.$ Here is an example of the process for the $4\times4$ case.
\begin{center}
	\begin{asy}
		unitsize(0.5cm);
		usepackage("chessfss"); // thank you https://tex.stackexchange.com/a/543024
		fill((0,0)--(1,0)--(1,4)--(0,4)--cycle, lightgray);
		for(int i = 0; i < 5; ++i)
		{
			draw((i,0)--(i,4));
			draw((0,i)--(4,i));
		}
		label("\color{red}\rook", (0.5,1.5));
		label("\rook", (0.5,3.5));
		label("\rook", (0.5,2.5));
		label("\rook", (0.5,0.5));
		label("$\to$", (5,2));
	\end{asy}
	\hspace{0.13cm}
	\begin{asy}
		unitsize(0.5cm);
		usepackage("chessfss"); // thank you https://tex.stackexchange.com/a/543024
		fill((1,0)--(2,0)--(2,4)--(1,4)--cycle, lightgray);
		for(int i = 0; i < 5; ++i)
		{
			draw((i,0)--(i,4));
			draw((0,i)--(4,i));
		}
		label("\rook", (0.5,1.5));
		label("\color{red}\rook", (1.5,3.5));
		label("\rook", (1.5,2.5));
		label("\rook", (1.5,0.5));
		label("$\to$", (5,2));
	\end{asy}
	\hspace{0.13cm}
	\begin{asy}
		unitsize(0.5cm);
		usepackage("chessfss"); // thank you https://tex.stackexchange.com/a/543024
		fill((2,0)--(3,0)--(3,4)--(2,4)--cycle, lightgray);
		for(int i = 0; i < 5; ++i)
		{
			draw((i,0)--(i,4));
			draw((0,i)--(4,i));
		}
		label("\rook", (0.5,1.5));
		label("\rook", (1.5,3.5));
		label("\color{red}\rook", (2.5,2.5));
		label("\rook", (2.5,0.5));
		label("$\to$", (5,2));
	\end{asy}
	\hspace{0.13cm}
	\begin{asy}
		unitsize(0.5cm);
		usepackage("chessfss"); // thank you https://tex.stackexchange.com/a/543024
		fill((3,0)--(4,0)--(4,4)--(3,4)--cycle, lightgray);
		for(int i = 0; i < 5; ++i)
		{
			draw((i,0)--(i,4));
			draw((0,i)--(4,i));
		}
		label("\rook", (0.5,1.5));
		label("\rook", (1.5,3.5));
		label("\rook", (2.5,2.5));
		label("\color{red}\rook", (3.5,0.5));
	\end{asy}
\end{center}

Let's make \autoref{ques:rooks1} more difficult.
\begin{ques}
	How many ways, up to symmetry, can we place $8$ rooks on a chessboard?
\end{ques}
As an example of what we mean, here are two ways to place rooks on a $4\times4$ board, which are the same ``up to symmetry,'' the symmetry here being the horizontal flip $h.$
\begin{center}
	\begin{asy}
		unitsize(0.5cm);
		usepackage("chessfss"); // thank you https://tex.stackexchange.com/a/543024
		for(int i = 0; i < 5; ++i)
		{
			draw((i,0)--(i,4));
			draw((0,i)--(4,i));
		}
		label("\rook", (0.5,1.5));
		label("\rook", (1.5,3.5));
		label("\rook", (2.5,2.5));
		label("\rook", (3.5,0.5));
	\end{asy}
	\qquad
	\begin{asy}
		unitsize(0.5cm);
		usepackage("chessfss"); // thank you https://tex.stackexchange.com/a/543024
		for(int i = 0; i < 5; ++i)
		{
			draw((i,0)--(i,4));
			draw((0,i)--(4,i));
		}
		label("\rook", (4-0.5,1.5));
		label("\rook", (4-1.5,3.5));
		label("\rook", (4-2.5,2.5));
		label("\rook", (4-3.5,0.5));
	\end{asy}
\end{center}
For this, we need to understand the symmetries of chessboard, which is simply $D_8,$ which acts on the set of all $8!$ arrangements of rooks on a chessboard. We want to know how many orbits of this $D_8$-action there are.

A first approximation is that any given arrangement gives rise to $8$ different arrangements in its orbit, yielding $8!/8=5040$ total arrangements, but this is not the case. For example, the following two arrangements are a single orbit.
\begin{center}
	\begin{asy}
		unitsize(0.5cm);
		usepackage("chessfss"); // thank you https://tex.stackexchange.com/a/543024
		for(int i = 0; i < 5; ++i)
		{
			draw((i,0)--(i,4));
			draw((0,i)--(4,i));
		}
		label("\rook", (0.5,0.5));
		label("\rook", (1.5,1.5));
		label("\rook", (2.5,2.5));
		label("\rook", (3.5,3.5));
	\end{asy}
	\qquad
	\begin{asy}
		unitsize(0.5cm);
		usepackage("chessfss"); // thank you https://tex.stackexchange.com/a/543024
		for(int i = 0; i < 5; ++i)
		{
			draw((i,0)--(i,4));
			draw((0,i)--(4,i));
		}
		label("\rook", (0.5,4-0.5));
		label("\rook", (1.5,4-1.5));
		label("\rook", (2.5,4-2.5));
		label("\rook", (3.5,4-3.5));
	\end{asy}
\end{center}
Namely, the problem is that some arrangements are more symmetric than others: the above arrangement only has an orbit of size $2$ because it is fixed by $4$ symmetries. So this appears very hard because we would have to check each individual arrangement of rooks and then check their symmetries. This seems very hard.

\subsection{(Not) Burnside's Lemma}
To solve this problem, there is Burnside's lemma.
\begin{remark}
	Burnside's lemma is the Lemma which is not Burnside's. It was called Burnside's lemma by pure incompetence, and the name has stuck.
\end{remark}
\begin{thm}[Not Burnside's] \label{thm:burn}
	The number of orbits of $G$ on a set $S$ is the average number of fixed points of elements of $g.$ Namely,
	\[\#(S/G)=\frac1{\#G}\sum_{g\in G}\#\{x\in S:gx=x\}.\]
\end{thm}
This is much better because summing over the number of elements of $G$ is much more tractable than summing over all possible arrangements of the rooks.
\begin{proof}[Proof of \autoref{thm:burn}]
	The idea is to look at pairs $(g,x)\in G\times S$ such that $gx=x.$ We count these pairs in two ways. In one direction, we can write
	\[\{(g,x)\in G\times S:gx=x\}=\sum_{g\in G}\#\{x\in S:gx=x\}.\]
	Alternatively, we can sum over $S,$ which looks like
	\[\{(g,x)\in G\times S:gx=x\}=\sum_{x\in S}\{g\in G\times S:gx=x\}=\sum_{x\in S}\#\op{Stab}(x).\]
	However, because we have a $G$-action, we may group the sum by orbits $Gx_0\in S/G.$ Indeed, for each orbit $Gx_0\in S/G,$ we see that the size of the stabilizer $\{g\in G:gx=x\}$ is the same for any $x\in Gx_0.$ (Namely, $g$ fixes $x_0$ if and only if $hgh^{-1}$ fixes $hx_0\in Gx_0,$ so $\op{Stab}(hx_0)=h\op{Stab}(x_0)h^{-1}.$) Thus, we see
	\[\{(g,x)\in G\times S:gx=x\}=\sum_{Gx_0\in S/G}\#(Gx_0)\cdot\#\op{Stab}(x_0).\]
	However, by the Orbit-stabilizer theorem, we see that $\#(Gx_0)=[G:\op{Stab}(x_0),$ so
	\[\{(g,x)\in G\times S:gx=x\}=\#G\sum_{Gx_0\in S/G}1=\#G\cdot\#(S/G).\]
	It follows that
	\[\#G\times\#(S/G)=\sum_{g\in G}\{x\in S:gx=x\},\]
	which is what we wanted.
\end{proof}
\begin{remark}
	If we look at the group element $g=e,$ then we see that $\{x\in S:ex=x\}=S,$ so we get $\#(S/G)\approx\#S/\#G,$ which was our first-order approximation.
\end{remark}

\subsection{Back to the Rooks}
Let's use this for our rooks. Take $G=D_8.$ Our elements, as before, are $\id,$ $h,$ $v,$ $d_1,$ and $d_2.$ Then we also have $90^\circ,180^\circ,$ and $270^\circ.$ However, there is some repetition here because $h$ and $v$ have the same number of fixed points; similarly, $d_1$ and $d_2$ or $90^\circ$ and $270^\circ$ also have the same number of fixed points.
\begin{remark}
	Note that these are the conjugacy classes of $D_8.$ More generally in a group $G,$ if two elements of our group $g_1$ and $g_2$ are conjugate, then they have the same number of fixed points. Namely, if we have $g\in G$ such that $g_1=gg_2g^{-1},$ then
	\[\{x\in S:g_1x=x\}\stackrel{\times g}\longrightarrow\{x\in S:g_2x=x\}.\]
	In words, $x\in S$ is a fixed point of $g_1$ if and only if $gx$ is a fixed point of $g_2.$
\end{remark}
\begin{warn}
	Note that the $90^\circ$ and $270^\circ$ rotations, though they ``look the same,'' are conjugate only in $D_8$ but \textit{not} in the group of rotations $\langle90^\circ\rangle.$ In Rhea's words, we need a reflection to make this work.
\end{warn}
We now go down the list of $D_8.$
\begin{itemize}
	\item For $e,$ everything is a fixed point, so there are $8!$ arrangements here.
	\item For $h$ and $v,$ nothing is a fixed point because a rook in a particular row (respectively, column) would get moved somewhere else in the same row (respectively, column), which violates the conditions of placing rooks.
	
	Here is the image for the $4\times4$ case.
	\begin{center}
		\begin{asy}
			unitsize(0.5cm);
			usepackage("chessfss"); // thank you https://tex.stackexchange.com/a/543024
			for(int i = 0; i < 5; ++i)
			{
				draw((i,0)--(i,4));
				draw((0,i)--(4,i));
			}
			label("\color{red}\rook", (0.5,2.5));
			label("$\stackrel h\to$", (5,2));
		\end{asy}
		\hspace{0.13cm}
		\begin{asy}
			unitsize(0.5cm);
			usepackage("chessfss"); // thank you https://tex.stackexchange.com/a/543024
			fill((0,2)--(4,2)--(4,3)--(0,3)--cycle, rgb(1,0.9,0.9));
			for(int i = 0; i < 5; ++i)
			{
				draw((i,0)--(i,4));
				draw((0,i)--(4,i));
			}
			label("\rook", (0.5,2.5));
			label("\rook", (4-0.5,2.5));
		\end{asy}
	\end{center}
	\item For $180^\circ,$ there are $8\times6\times4\times2$ arrangements because placing one rook forces its inverse as well. So we place one rook and lose two options simultatenously. 
	
	Here is the image for the $4\times4$ case, where we only make two choices.
	\begin{center}
		\begin{asy}
			unitsize(0.5cm);
			usepackage("chessfss"); // thank you https://tex.stackexchange.com/a/543024
			fill((0,0)--(1,0)--(1,4)--(0,4)--cycle, lightgray);
			fill((4-0,0)--(4-1,0)--(4-1,4)--(4-0,4)--cycle, lightgray);
			for(int i = 0; i < 5; ++i)
			{
				draw((i,0)--(i,4));
				draw((0,i)--(4,i));
			}
			label("\color{red}\rook", (0.5,2.5));
			label("\rook", (0.5,1.5));
			label("\rook", (0.5,3.5));
			label("\rook", (0.5,0.5));
			label("$\to$", (5,2));
		\end{asy}
		\hspace{0.13cm}
		\begin{asy}
			unitsize(0.5cm);
			usepackage("chessfss"); // thank you https://tex.stackexchange.com/a/543024
			fill((1,0)--(2,0)--(2,4)--(1,4)--cycle, lightgray);
			fill((4-1,0)--(4-2,0)--(4-2,4)--(4-1,4)--cycle, lightgray);
			for(int i = 0; i < 5; ++i)
			{
				draw((i,0)--(i,4));
				draw((0,i)--(4,i));
			}
			label("\rook", (0.5,2.5)); label("\rook", (3.5,1.5));
			label("\color{red}\rook", (1.5,3.5));
			label("\rook", (1.5,0.5));
			label("$\to$", (5,2));
		\end{asy}
		\hspace{0.13cm}
		\begin{asy}
			unitsize(0.5cm);
			usepackage("chessfss"); // thank you https://tex.stackexchange.com/a/543024
			for(int i = 0; i < 5; ++i)
			{
				draw((i,0)--(i,4));
				draw((0,i)--(4,i));
			}
			label("\rook", (0.5,2.5)); label("\rook", (3.5,1.5));
			label("\rook", (1.5,3.5)); label("\rook", (4-1.5,4-3.5));
		\end{asy}
	\end{center}
	\item For $90^\circ$ and $270^\circ,$ we get $6\times2.$ This is because, placing a rook in the first row, we can only place rooks outside the corners (or else they run into each other), which gives $6$ options. Here is the image of this in the $4\times4$ case.
	\begin{center}
		\begin{asy}
			unitsize(0.5cm);
			usepackage("chessfss"); // thank you https://tex.stackexchange.com/a/543024
			for(int i = 0; i < 5; ++i)
			{
				draw((i,0)--(i,4));
				draw((0,i)--(4,i));
			}
			label("\color{red}\rook", (0.5,3.5));
			label("$\stackrel{90^\circ}\to$", (5,2));
		\end{asy}
		\hspace{0.05cm}
		\begin{asy}
			unitsize(0.5cm);
			usepackage("chessfss"); // thank you https://tex.stackexchange.com/a/543024
			fill((0,3)--(4,3)--(4,4)--(0,4)--cycle, rgb(1,0.9,0.9));
			for(int i = 0; i < 5; ++i)
			{
				draw((i,0)--(i,4));
				draw((0,i)--(4,i));
			}
			label("\rook", (0.5,3.5));
			label("\rook", (0.5,0.5));
			label("\rook", (3.5,0.5));
			label("\rook", (3.5,3.5));
		\end{asy}
	\end{center}
	After placing a rook in the first row, we lose four options because we need to place four rooks from the first one, which gives $2$ options afterwards because we still cannot place in corners.
	
	Here is the image for the $4\times4$ case, where we only make one choice.
	\begin{center}
		\begin{asy}
			unitsize(0.5cm);
			usepackage("chessfss"); // thank you https://tex.stackexchange.com/a/543024
			fill((0,0)--(1,0)--(1,4)--(0,4)--cycle, lightgray);
			fill((4-0,0)--(4-1,0)--(4-1,4)--(4-0,4)--cycle, lightgray);
			for(int i = 0; i < 5; ++i)
			{
				draw((i,0)--(i,4));
				draw((0,i)--(4,i));
			}
			label("\color{red}\rook", (0.5,1.5));
			label("\rook", (0.5,2.5));
			label("$\to$", (5,2));
		\end{asy}
		\hspace{0.13cm}
		\begin{asy}
			unitsize(0.5cm);
			usepackage("chessfss"); // thank you https://tex.stackexchange.com/a/543024
			for(int i = 0; i < 5; ++i)
			{
				draw((i,0)--(i,4));
				draw((0,i)--(4,i));
			}
			label("\rook", (0.5,1.5)); label("\rook", (1.5,3.5));
			label("\rook", (3.5,2.5)); label("\rook", (2.5,0.5));
		\end{asy}
	\end{center}
	\item We have $d_1$ and $d_2$ are the hardest. For concreteness, we count for $d_2.$ We do this by a recursion: let $c_n$ be the number of arrangements fixed by $d_1$ in a $n\times n$ board. We claim that $c_n=c_{n-1}+(n-1)c_{n-2}.$ We have two cases.
	
	If we place the first rook in the top left corner, then we reduce this problem to the $(n-1)\times(n-1)$ board. Here is the image for that in the $4\times4$ case.
	\begin{center}
		\begin{asy}
			unitsize(0.5cm);
			usepackage("chessfss"); // thank you https://tex.stackexchange.com/a/543024
			fill((0,0)--(1,0)--(1,4)--(0,4)--cycle, lightgray);
			for(int i = 0; i < 5; ++i)
			{
				draw((i,0)--(i,4));
				draw((0,i)--(4,i));
			}
			label("\color{red}\rook", (0.5,3.5));
			label("\rook", (0.5,0.5));
			label("\rook", (0.5,1.5));
			label("\rook", (0.5,2.5));
			label("$\to$", (5,2));
		\end{asy}
		\hspace{0.13cm}
		\begin{asy}
			unitsize(0.5cm);
			usepackage("chessfss"); // thank you https://tex.stackexchange.com/a/543024
			fill((0,0)--(1,0)--(1,3)--(4,3)--(4,4)--(0,4)--cycle, gray);
			for(int i = 0; i < 5; ++i)
			{
				draw((i,0)--(i,4));
				draw((0,i)--(4,i));
			}
			label("\rook", (0.5,3.5));
			label("$=c_{n-1}$", (4,2), E);
		\end{asy}
	\end{center}
	Otherwise, if we place our rook somewhere else in the first row, then we lose both a row and a column from the $d_1$ symmetry, reducing to the $(n-2)\times(n-2)$ case. Here is the image for that in the $4\times4$ case.
	\begin{center}
		\begin{asy}
			unitsize(0.5cm);
			usepackage("chessfss"); // thank you https://tex.stackexchange.com/a/543024
			fill((0,0)--(1,0)--(1,4)--(0,4)--cycle, lightgray);
			for(int i = 0; i < 5; ++i)
			{
				draw((i,0)--(i,4));
				draw((0,i)--(4,i));
			}
			label("\color{red}\rook", (0.5,1.5));
			label("\rook", (0.5,0.5));
			label("\rook", (0.5,2.5));
			label("\rook", (0.5,3.5));
			label("$\stackrel{d_2}\to$", (5,2));
		\end{asy}
		\hspace{0.13cm}
		\begin{asy}
			unitsize(0.5cm);
			usepackage("chessfss"); // thank you https://tex.stackexchange.com/a/543024
			fill((0,0)--(1,0)--(1,3)--(4,3)--(4,4)--(0,4)--cycle, gray);
			fill((0,1)--(4,1)--(4,2)--(0,2)--cycle, gray);
			fill((2,0)--(3,0)--(3,4)--(2,4)--cycle, gray);
			for(int i = 0; i < 5; ++i)
			{
				draw((i,0)--(i,4));
				draw((0,i)--(4,i));
			}
			label("\rook", (0.5,1.5));
			label("\rook", (2.5,3.5));
			label("$=c_{n-2}$", (4,2), E);
		\end{asy}
	\end{center}
	So indeed, $c_n=c_{n-1}+(n-1)c_{n-2},$ and we can compute that $c_8=764.$
\end{itemize}
In total, we get
\[\frac{40320+2\cdot0+2\cdot764+2\cdot12+8\cdot6\cdot4\cdot2}{8}=\boxed{5282}\]
This is a bit bigger than our first guess, which was $8!/8=5040.$
\begin{remark}
	We can bypass Burnside's lemma by cheating a bit. The idea is to weight each orbit in $S/G$ we are counting so that we don't have to look directly at the group: we weight an orbit by the reciprocal of its symmetry group. (This is contrast to weighting the orbits equally to count them.) For example, the following arrangement is weighted $1/4$ because of its four symmetries.
	\begin{center}
		\begin{asy}
			unitsize(0.5cm);
			usepackage("chessfss"); // thank you https://tex.stackexchange.com/a/543024
			for(int i = 0; i < 5; ++i)
			{
				draw((i,0)--(i,4));
				draw((0,i)--(4,i));
			}
			label("\rook", (0.5,0.5));
			label("\rook", (1.5,1.5));
			label("\rook", (2.5,2.5));
			label("\rook", (3.5,3.5));
		\end{asy}
	\end{center}
	Why do we do this? Well, it turns out that the number of weighted orbits is $\#S/\#G$ exactly: write
	\[\sum_{Gx\in S/G}\frac1{\#\{g\in G:gx=x\}}=\sum_{Gx\in S/G}\frac1{\#\op{Stab}(x)}=\sum_{Gx\in S/G}\frac{\#Gx}{\#G}=\frac{\#S}{\#G}.\]
\end{remark}

\subsection{Groups of Order Nine}
For groups of order $9,$ the obvious groups are $\ZZ/9\ZZ$ and $(\ZZ/3\ZZ)^2.$ These are the only abelian ones: if there's an element of order $9,$ then we are cyclic; otherwise, all elements have order $3,$ then we are an $\FF_3$-vector space, forcing us to be $(\ZZ/3\ZZ)^2$ for the usual size reasons.

What about non-abelian groups? We claim there are no nonabelian groups.
\begin{prop}
	All groups of order $p^2$ are abelian, for $p$ prime.
\end{prop}
\begin{proof}
	Fix $G$ of order $p^2.$ Note that all proper subgroups have index divisible by $p$ (the index is either $p$ or $p^2$). In particular, the center has order divisible by $p,$ borrowing the class equation logic from the proof of Cauchy's theorem.
	
	We claim that $Z(G)=G.$ Well, suppose for the sake of contradiction that there is $b\in G\setminus Z(G)$ so that $\#Z(G)=p.$ The problem is that the set of elements $C(b)$ of $G$ which commute with $b$ is a subgroup of $G,$ which contains $\{b\}\sqcup Z(G)$ and hence has order exceeding $p.$ Thus, $C(b)$ has order $p^2,$ implying that $b$ commutes with all elements, violating $b\notin Z(G).$
\end{proof}

\end{document}