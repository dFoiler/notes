\documentclass[../notes.tex]{subfiles}

\begin{document}



% !TEX root = ../notes.tex















You feel your sins crawling on your back.

\subsection{Categories}
We interrupt our regularly scheduled programming to talk about category theory. We're skipping most of the proofs.
\begin{remark}
	All proofs in category theory are trivial and not very interesting.
\end{remark}
We start some examples of categories.
\begin{example}
	The basic examples of categories are as follows.
	\begin{itemize}
		\item The category of sets is $\op{Set}.$ There are objects which are sets and morphisms which are functions.
		\item The category of groups is $\op{Grp}.$ There are objects which are groups and morphisms which are homomorphisms.
		\item The category of topological spaces is $\op{Top}.$ There are objects which are spaces and morphisms which are continuous maps.
	\end{itemize}
\end{example}
So we have the following definition, abstracting the common features.
\begin{definition}[Category]
	A \textit{category} $\mathcal C$ consists of the data of a set (or class) of objects with some morphisms. Namely, given objects $A$ and $B$ in the category $\mathcal C,$ there is a set (or class) $\op{Mor}(A,B)$ of morphisms from $A$ to $B$ satisfying the following.
	\begin{enumerate}[label=(\alph*)]
		\item We can compose morphisms: given $f:\op{Mor}(A,B)$ and $g:\op{Mor}(B,C),$ there is a morphism $g\circ f:\op{Nor}(A,C).$
		\item There are identity morphisms: given any object $A,$ there is a morphism $\id_A\in\op{Mor}(A,A).$
		\item Composition should associate: given objects $A,B,C,D$ with morphisms $h\in\op{Mor}(A,B)$ and $g\in\op{Mor}(B,C)$ and $f\in\op{Mor}(C,D),$ then $(f\circ g)\circ h=f\circ(g\circ h).$
		\item Identity should be an identity: given a morphism $f\in\op{Mor}(A,B),$ we have $f\circ\id_A=\id_B\circ f.$
	\end{enumerate}
\end{definition}
\begin{warn}
	I may occasionally abbreviate $f\in\op{Mor}(A,B)$ to $f:A\to B.$ For example, I might write composition as $\circ:\op{Mor}(B,C)\times\op{Mor}(A,B)\to\op{Mor}(A,C)$ as $\circ:(B\to C)\times(A\to B)\to(A\to C).$
\end{warn}
\begin{remark}
	We don't want to force all objects to live in a set because we would like $\op{Set}$ to be a category, and there is no set containing all sets.
\end{remark}
Let's have more examples.
\begin{example}
	Groups are categories, where the object is a single point, where the morphisms are the elements of group, and we define the composition of two morphisms $g,h$ to be $g\circ h:=gh.$
\end{example}
\begin{example}
	Preordered sets $X$ are categories. A preordered set is an ordered set where the order has reflexivity and transitivity. Our objects are elements of $X,$ and we place exactly one morphism $A\to B$ if and only if $A\le B$ for $A,B\in X.$ Namely, the identity morphism comes from $A\le A,$ and associativity of composition comes from transitivity.
\end{example}
Of course, there are lots more things which are categories.

\subsection{Fun with Morphisms}
Here is the fundamental idea of category theory.
\begin{idea}
	Ignore any internal structure of an object and instead look at its morphisms.
\end{idea}
To be more explicit, given a category $\mathcal C$ with an object $A\in\mathcal C,$ we study the morphisms of $A$ instead of $A$ directly.

For example, let's redefine injective. For sets, we usually say that $f:A\to B$ is injective if and only if $f(a_1)=f(a_2)$ for $a_1,a_2\in A,$ then $a_1=a_2.$ However, we would like to avoid talking about elements of $A.$ Here is our new definition.
\begin{definition}[Monic]
	Fix a category $\mathcal C$ with a morphism $f\in\op{Mor}(A,B).$ Then we say that $f$ is \textit{monic} (or ``is a \textit{monomorphism}'') if and only if, for each other object $X$ and morphisms $g_1,g_2:\op{Mor}(X,A),$ then $f\circ g_1=f\circ g_2$ implies $g_1=g_2.$ Here is the diagram.
	% https://q.uiver.app/?q=WzAsMyxbMSwwLCJBIl0sWzIsMCwiQiJdLFswLDAsIlgiXSxbMCwxLCJmIl0sWzIsMCwiZ18xIiwwLHsib2Zmc2V0IjotMX1dLFsyLDAsImdfMiIsMix7Im9mZnNldCI6MX1dXQ==
	\[\begin{tikzcd}
		X & A & B
		\arrow["f", from=1-2, to=1-3]
		\arrow["{g_1}", shift left=1, from=1-1, to=1-2]
		\arrow["{g_2}"', shift right=1, from=1-1, to=1-2]
	\end{tikzcd}\]
\end{definition}
Think ``monic means left-cancellative,'' which makes more sense after the example in $\op{Set}.$
\begin{example}
	We check that monic is equivalent to injective in $\op{Set}.$
	\begin{itemize}
		\item Suppose that $f:A\into B$ is injective. Then, suppose we have an object $X$ with maps $g_1,g_2:X\to A$ such that $f\circ g_1=g\circ g_2.$ Then, for any $x\in X,$ we have $f(g_1(x))=f(g_2(x)),$ from which $g_1(x)=g_2(x)$ follows by injectivity of $f.$ So indeed, $g_1=g_2.$
		\item Conversely, suppose that $f:A\into B$ is monic. Fix elements $a_1,a_2\in A$ such that we want to show $f(a_1)=f(a_2)$ implies $a_1=a_2.$ Then consider the object $X:=\{*\}$ for any $a.$ There is a map $g_1:X\to A$ by $*\mapsto a_1$ and a map $g_2:X_2\to A$ by $g_2:*\mapsto a_2.$ Then we have that $f\circ g_1=f\circ g_2$ because
		\[f(g_1(*))=f(a_1)=f(a_2)=f(g_2(*)).\]
		So because $f$ is monic, it follows $g_1=g_2,$ so $a_1=g_1(*)=g_2(*)=a_2.$
	\end{itemize}
\end{example}
So in $\op{Set},$ monic means injective, but now monic works for all categories.

What about surjectivity? Well, in $\op{Set},$ we are saying that $f:A\to B$ has, for each $b\in B,$ some $a\in A$ with $f(a)\in b.$ How do we do this without talking about elements? The answer turns out to be dualize the definition of monic.
\begin{definition}[Epic]
	Fix a category $\mathcal C$ with a morphism $f\in\op{Mor}(A,B).$ Then we say that $f$ is \textit{epic} (or ``is an \textit{epimorphism}'') if and only if, for each other object $X$ and morphisms $g_1,g_2:\op{Mor}(B,X),$ then $g_1\circ f=g_2\circ f$ implies $g_1=g_2.$ Here is the diagram.
	% https://q.uiver.app/?q=WzAsMyxbMCwwLCJBIl0sWzEsMCwiQiJdLFsyLDAsIlgiXSxbMCwxLCJmIl0sWzEsMiwiZ18xIiwwLHsib2Zmc2V0IjotMX1dLFsxLDIsImdfMiIsMix7Im9mZnNldCI6MX1dXQ==
	\[\begin{tikzcd}
		A & B & X
		\arrow["f", from=1-1, to=1-2]
		\arrow["{g_1}", shift left=1, from=1-2, to=1-3]
		\arrow["{g_2}"', shift right=1, from=1-2, to=1-3]
	\end{tikzcd}\]
\end{definition}
Think ``epic means right-cancellative.''
\begin{example}
	We check that epic is equivalent to surjective in $\op{Set}.$
	\begin{itemize}
		\item Suppose that $f:A\onto B$ is surjective. Then, suppose that we have an object $X$ with maps $g_1,g_2:B\to X$ such that $g_1\circ f=g_2\circ f$ so that we want $g_1=g_2.$ Well, for any $b\in B,$ there exists an $a\in A$ such that $f(a)=b$ because $f$ is surjective, so
		\[g_1(b)=g_1(f(a))=(g_1\circ f)(a)=(g_2\circ f)(a)=g_2(b),\]
		indeed implying that $g_1=g_2.$
		\item Conversely, we show the contrapositive: if $f:A\to B$ is not surjective, then $f$ is not epic. Indeed, consider the object $X=\{0,1\}$ with maps $g_1,g_2:B\to X$ defined by
		\[g_1(b):=\begin{cases}
			1 & b\in\im f, \\
			0 & b\notin\im f,
		\end{cases}\qquad\text{and}\qquad g_2(b)=1.\]
		We note that, for any $a\in A,$ we have $g_1(f(a))=1=g_2(f(a)),$ so $g_1\circ f=g_2\circ g.$ But $f$ not being surjective implies that there exists $b\in B\setminus\im f$ so that $g_1(b)\ne g_2(b).$ It follows $f$ is not epic.
	\end{itemize}
\end{example}
So again, in $\op{Set},$ this is equivalent to surjective, but now we can talk about epic in all categories.
\begin{warn}
	Monic and epic do not always turn out to mean injective and surjective. For example, in the category $\op{Ring},$ the map canonical map $\ZZ\into\QQ$ is epic but not surjective.
\end{warn}
To be more explicit, of course $\iota:\ZZ\into\QQ$ is not surjective, but showing it is epic needs some work. Essentially this is because $\QQ$ is the fraction field of $\ZZ,$ so what $\QQ$ does is determined by what $\ZZ$ does. More rigorously, any good ring map $\ZZ\to R$ can be uniquely lifted to a map $\QQ\to R.$

Indeed, suppose that we have some ring $R$ with ring homomorphisms $g_1,g_2:\QQ\to R$ such that $g_1\circ\iota=g_2\circ\iota.$ Then, for any rational $m/n\in\QQ$ with $m,n\in\ZZ,$ we have\footnote{Showing that $g(m/n)=g(m)/g(n)$ requires some care. Because certainly $g(n)g(m/n)=g(m),$ it suffices to show that $n\ne0$ implies $g(n)\in R^\times.$ But $g(n)g(1/n)=g(n/n)=g(1)=n1_R$ because ring homomorphisms send $g:1\mapsto1_R.$}
\[g_1\left(\frac mn\right)=\frac{g_1(m)}{g_1(n)}=\frac{(g_1\circ\iota)(m)}{(g_1\circ\iota)(n)}=\frac{(g_2\circ\iota)(m)}{(g_2\circ\iota)(n)}=\frac{g_2(m)}{g_2(n)}=g_2\left(\frac mn\right).\]
So indeed, $g_1=g_2$ on $\QQ.$
\begin{remark}
	Being epic can be subtle. In the category of planar graphs, it turns out that ``every epimorphism is surjective'' is equivalent to the Four color theorem.
\end{remark}

\subsection{Functors}
Let's start with some examples.
\begin{example}
	Here are some examples.
	\begin{itemize}
		\item We have a functor from $\op{Grp}$ to $\op{Set}$ by taking a group to its underlying set of objects.
		\item We have another functor from $\op{Set}$ to $\op{Grp}$ by taking any set $S$ to the free group on $S.$
		\item Homology is a functor from topological spaces $X$ to abelian groups $H_\bullet(X).$
	\end{itemize}
\end{example}
Again, let's extract our common information. For example, a topological map $X\to Y$ turns into a corresponding map of homology groups $H_\bullet(X)\to H_\bullet(Y).$ Well in fact this is true for the other examples as well: a map of groups is of course a map of sets, and a map of sets can be lifted to a map of free groups.

So here is our definition.
\begin{definition}[Covariant functor]
	Fix $\mathcal A$ and $\mathcal B$ categories. Then a \textit{(covariant) functor} $F:\mathcal A\to\mathcal B$ has the following data.
	\begin{itemize}
		\item The functor $F$ takes objects of $\mathcal A$ to objects of $\mathcal B.$
		\item Given a morphism $f\in\op{Mor}(A_1,A_2)$ with $A_1,A_2\in\mathcal A.$ Then there is a morphism $F(f):F(A_1)\to F(A_2).$
		\item Identity: given any object $A,$ we have that $F(\id_A)=\id_{F(A)}.$
		\item Composition: given objects $A_1,A_2,A_3$ with morphisms $f\in\op{Mor}(A_1,A_2)$ and $g\in\op{Mor}(A_2,A_3),$ then $F(g\circ f)=F(g)\circ F(f).$ 
	\end{itemize}
\end{definition}
\begin{remark}
	We may skip the boring parts of the definitions because we don't care. Definitions in category theory are rather unmemorable.
\end{remark}

Let's have another example.
\begin{example}
	Fix a group $G$ acting on a set $S.$ This is actually a functor from the category the one-object category representing $G$ to $\op{Set}.$ Namely, we take our one object to the set $S,$ and each morphism in the one-object category correspond to some specified function on $S.$ It turns out that the axioms of a group action correspond to the checks for being a functor.
\end{example}
To motivate the next definition, consider taking a vector space $V$ over $k$ in the category $\op{Vec}_k$ to its dual. Namely, we take $V\in\op{Vec}_k$ to $V^*:=\op{Hom}_k(V,k)$ its dual. Then given a map $f:V\to W,$ we might want a morphism $f^*:V^*\to W^*.$ Here is the diagram: given $\varphi\in V^*,$ how do we induce $\gamma\in W^*$?
% https://q.uiver.app/?q=WzAsMyxbMCwwLCJWIl0sWzIsMCwiVyJdLFsxLDIsImsiXSxbMCwxLCJmIl0sWzAsMiwiXFx2YXJwaGkiLDJdLFsxLDIsIlxcZ2FtbWEiLDAseyJzdHlsZSI6eyJib2R5Ijp7Im5hbWUiOiJkYXNoZWQifX19XV0=
\[\begin{tikzcd}
	V && W \\
	\\
	& k
	\arrow["f", from=1-1, to=1-3]
	\arrow["\varphi"', from=1-1, to=3-2]
	\arrow["\gamma", dashed, from=1-3, to=3-2]
\end{tikzcd}\]
However, there is no useful way to take $\varphi:V\to k$ to a map $W\to k,$ so we appear stuck. But conversely, it appears that given $\gamma:W\to k,$ we can induce $\varphi:V\to k$ by precomposing as $\varphi:=\gamma\circ f$! Here is the diagram.
% https://q.uiver.app/?q=WzAsMyxbMCwwLCJWIl0sWzIsMCwiVyJdLFsxLDIsImsiXSxbMCwxLCJmIl0sWzAsMiwiXFxnYW1tYVxcY2lyYyBmIiwyLHsic3R5bGUiOnsiYm9keSI6eyJuYW1lIjoiZGFzaGVkIn19fV0sWzEsMiwiXFxnYW1tYSJdXQ==
\[\begin{tikzcd}
	V && W \\
	\\
	& k
	\arrow["f", from=1-1, to=1-3]
	\arrow["{\gamma\circ f}"', dashed, from=1-1, to=3-2]
	\arrow["\gamma", from=1-3, to=3-2]
\end{tikzcd}\]
So we can define the map $f^*:W^*\to V^*$ by taking $\gamma\in W^*$ to $\gamma\circ f\in V^*.$ But now the direction of our morphisms is reversed, so we pick up the following definition.
\begin{definition}[Contravariant functor]
	A functor $F:\mathcal A\to\mathcal B$ is called \textit{contravariant} if we ``reverse'' morphisms $f:A_1\to A_2$ by $F(f):F(A_2)\to F(A_1).$ The same definition still works, except we need to write the composition law as follows.
	\begin{itemize}
		\item Composition: given objects $A_1,A_2,A_3$ with morphisms $f\in\op{Mor}(A_1,A_2)$ and $g\in\op{Mor}(A_2,A_3),$ then $F(g\circ f)=F(f)\circ F(g).$
	\end{itemize}
\end{definition}
So we can check that $V\mapsto V^*$ and $f\mapsto f^*$ is a contravariant functor $\op{Vec}_k\to\op{Vec}_k.$ The composition law boils down to checking that
\[f^*(g^*\varphi)=f^*(v\mapsto\varphi gv)=(v\mapsto(\varphi g)fv)=v\mapsto\varphi(gf)v=(fg)^*(\varphi),\]
which is more annoying than interesting.
\begin{remark}
	It turns out homology is a covariant functor, but cohomology is contravariant. So it goes.
\end{remark}

\subsection{Size Problems}
We can also compose functors! For example, given $F:\mathcal A\to\mathcal B$ and $G:\mathcal B\to\mathcal C,$ there is a functor $G\circ F:\mathcal A\to\mathcal C.$ This lets us make the category of all categories where the objects are categories and the morphisms are functors! Except this doesn't work for size reasons, for the same reason there is no set of all sets.

There are a few ways to fix this.
\begin{itemize}
	\item We can merely consider the category of all ``small'' categories, where we only consider the categories with at most some cardinal number of objects and morphisms.\footnote{\textit{The category of small categories would like to know your location.}}
	\item We can use ``classes'' if we are careful.
	\item We can also use Grothendieck universes, which is common algebraic geometry.
	\item Ignore the problem completely.
\end{itemize}
We will take the last approach for our introduction here. We will have to encounter it later in life but not now.

\subsection{Natural Transformations}
Of course, we start with an example.
\begin{example}
	Fix $V$ a finite-dimensional vector space and $V^*$ its dual. We know that $V\cong V^*,$ but there is no ``natural'' isomorphism in general because we have to pick a basis first.
	
	However, there is a ``natural'' isomorphism $V\cong V^{**}.$ Namely, given an element of $V,$ we can conically exhibit a map $(V\to k)\to k$ in $V^{**}.$ To be explicit, we have a function $V\to V^{**}$ by
	\[v\longmapsto(\varphi\mapsto\varphi v).\]
	In $\lambda$-calculus, this reads $\lambda(v:V).\lambda(\varphi:V^*).\varphi v.$ And we can take linear transformations $f:V\to W$ to $f^{**}:V^{**}\to W^{**},$ which looks like $f^{**}:\varphi\mapsto\varphi\circ f.$
\end{example}
We would like to rigorize what ``natural'' means. To start, we note that we have two functors $F,G:\op{Vec}_k\to\op{Vec}_k$ by $F:V\to V$ and $G:V\to V^{**},$ and we would like there to be a natural transformation between them. What does this mean? Well, what do we have?
\begin{itemize}
	\item For any $V,$ there is a map $\eta_V:F(V)\to G(V).$ Indeed, this was the map $V\mapsto V^{**}$ given in the example.
	\item For any morphism $f:V\to W,$ the following diagram commutes.
	% https://q.uiver.app/?q=WzAsNixbMCwwLCJWIl0sWzAsMSwiVyJdLFsyLDAsIkYoVikiXSxbMywwLCJHKFYpIl0sWzIsMSwiRihXKSJdLFszLDEsIkcoVykiXSxbMCwxLCJmIl0sWzIsNCwiRihmKSIsMl0sWzQsNSwiXFxldGFfVyIsMl0sWzIsMywiXFxldGFfViJdLFszLDUsIkcoZikiXV0=
	\[\begin{tikzcd}
		V && {F(V)} & {G(V)} \\
		W && {F(W)} & {G(W)}
		\arrow["f", from=1-1, to=2-1]
		\arrow["{F(f)}"', from=1-3, to=2-3]
		\arrow["{\eta_V}"', from=2-3, to=2-4]
		\arrow["{\eta_W}", from=1-3, to=1-4]
		\arrow["{G(f)}", from=1-4, to=2-4]
	\end{tikzcd}\]
	Namely, given $v\in V,$ we can track some $v\in V$ along both directions of the square, writing
	\[v\stackrel{F(f)}\longmapsto fv\stackrel{\eta_W}\longmapsto(\varphi\mapsto\varphi fv)\]
	while
	\[v\stackrel{\eta_V}\longmapsto(\varphi\mapsto\varphi v)\stackrel{G(f)}\longmapsto(\varphi\mapsto\varphi fv).\]
\end{itemize}
This is the general definition.
\begin{definition}[Natural transformations]
	Given two functors $F,G:\mathcal A\to\mathcal B,$ we say that there is a \textit{natural transformation} $\eta_\bullet:F\to G$ if we have the following data.
	\begin{itemize}
		\item For any $A\in\mathcal A,$ there is a map $\eta_A:F(A)\to G(A).$
		\item For any morphism $A_1\to A_2,$ the following diagram commutes.
		% https://q.uiver.app/?q=WzAsNixbMCwwLCJWIl0sWzAsMSwiVyJdLFsyLDAsIkYoVikiXSxbMywwLCJHKFYpIl0sWzIsMSwiRihXKSJdLFszLDEsIkcoVykiXSxbMCwxLCJmIl0sWzIsNCwiRihmKSIsMl0sWzQsNSwiXFxldGFfVyIsMl0sWzIsMywiXFxldGFfViJdLFszLDUsIkcoZikiXV0=
		\[\begin{tikzcd}
			A_1 && {F(A_1)} & {G(A_1)} \\
			A_2 && {F(A_2)} & {G(A_2)}
			\arrow["f", from=1-1, to=2-1]
			\arrow["{F(f)}"', from=1-3, to=2-3]
			\arrow["{\eta_{A_1}}"', from=2-3, to=2-4]
			\arrow["{\eta_{A_2}}", from=1-3, to=1-4]
			\arrow["{G(f)}", from=1-4, to=2-4]
		\end{tikzcd}\]
	\end{itemize}
\end{definition}
So the statement that $V$ is naturally isomorphic with $V^{**}$ turns into the fact that there is a natural transformation between $V\mapsto V$ and $V\mapsto V^{**}.$
\begin{definition}[Natural isomrphism]
	Fix everything as in the definition of a natural transformation. If $\eta_A:F(A)\to G(A)$ is an isomorphism, we call $\eta$ a \textit{natural isomorphism} and the functors $F$ and $G$ are naturally isomorphic.
\end{definition}
So we see that $V\mapsto V$ and $V\mapsto V^{**}$ are naturally isomorphic in the category of finite-dimensional vector spaces over $k.$
\begin{example}
	Fix $\mathcal A$ and $\mathcal B$ categories. Then the functors from $\mathcal A$ to $\mathcal B$ make a category, where objects are functors and morphisms are natural transformations. This is called a $2$-category; we can even go further to $3$-categories, $4$-categories, and so on upwards to $\infty$-categories in the limit.
\end{example}

\subsection{Adjoint Functors}
Here is our basic example.
\begin{example}
	Consider the categories of $\op{Set}$ and $\op{Grp}.$ We have two functors.
	\begin{itemize}
		\item The forgetful functor from $G:\op{Grp}\to\op{Set}$ by simply forgetting the group structure.
		\item The free functor from $F:\op{Set}\to\op{Grp}$ by sending sets to their free groups.
	\end{itemize}
\end{example}
These are not inverses, but they are ``adjoint.'' Namely, suppose we take a group $X$ and set $S$ and consider $G(X)$ and $F(S).$ Last class we noticed that we have the universal property that any function of sets $f:S\to G(X)$ induces (arguably, ``lifts to'') a unique morphism $g:F(S)\to X.$ Here is the diagram.
% https://q.uiver.app/?q=WzAsNCxbMCwxLCJTIl0sWzAsMCwiRyhYKSJdLFsxLDEsIkYoUykiXSxbMSwwLCJYIl0sWzAsMSwiZiJdLFszLDFdLFswLDJdLFsyLDMsImciLDIseyJzdHlsZSI6eyJib2R5Ijp7Im5hbWUiOiJkYXNoZWQifX19XV0=
\[\begin{tikzcd}
	{G(X)} & X \\
	S & {F(S)}
	\arrow["f", from=2-1, to=1-1]
	\arrow[from=1-2, to=1-1]
	\arrow[from=2-1, to=2-2]
	\arrow["g"', dashed, from=2-2, to=1-2]
\end{tikzcd}\]
In particular, we have a bijection between $\op{Mor}(S,GX)$ and $\op{Mor}(FS,X),$ and this bijection is ``natural'' in some sense. Rigorizing what we mean vey ``natural'' gives the following definition.
\begin{definition}[Adjoints]
	Fix functors $F:\mathcal A\to\mathcal B$ and $G:\mathcal B\to\mathcal A.$ We say that $(F,G)$ form an \textit{adjoint pair} if and only if we have bijections
	\[\tau_{A,B}:(A\to GB)\to(FA\to B)\]
	causing the following two diagrams to commute: for any $\mathcal A$-morphism $f:A_1\to A_2$ and $\mathcal B$-object $B,$ the following must commute.
	% https://q.uiver.app/?q=WzAsNixbMCwwLCJBXzEiXSxbMCwxLCJBXzIiXSxbMSwwLCJBXzJcXHRvIEdCIl0sWzIsMCwiRkFfMlxcdG8gQiJdLFsxLDEsIkFfMVxcdG8gR0IiXSxbMiwxLCJGQV8xXFx0byBHQiJdLFswLDEsImYiXSxbMiwzLCJcXHRhdV97QV8yLEJ9Il0sWzIsNCwiXFxjaXJjIGYiLDJdLFszLDUsIlxcY2lyYyBGZiJdLFs0LDUsIlxcdGF1X3tBXzEsQn0iLDJdXQ==
	\[\begin{tikzcd}
		{A_1} & {A_2\to GB} & {FA_2\to B} \\
		{A_2} & {A_1\to GB} & {FA_1\to GB}
		\arrow["f", from=1-1, to=2-1]
		\arrow["{\tau_{A_2,B}}", from=1-2, to=1-3]
		\arrow["{\circ f}"', from=1-2, to=2-2]
		\arrow["{\circ Ff}", from=1-3, to=2-3]
		\arrow["{\tau_{A_1,B}}"', from=2-2, to=2-3]
	\end{tikzcd}\]
	Additionally, there is an inverse diagram for $\tau_{A,B}^{-1}:(FA\to B)\to(A\to GB),$ which we won't write down. In this case, $F$ is called \textit{left adjoint} and $G$ is called \textit{right adjoint}.
\end{definition}
It turns out that free functor is left adjoint with the forgetful functor; we won't check this here because it is a bit arduous.
\begin{remark}
	We call these adjoint functors because they are related to adjoint linear transformations. Namely, given two linear transformations $F,G:V\to V$ quipped with an inner product $\langle\cdot,\cdot\rangle,$ they are \textit{adjoint} if and only if
	\[\langle s,Gx\rangle=\langle Fs,x\rangle\]
	for $s,x\in V.$
\end{remark}
In general, right adjoints tend to be ``forgetful'' while left adjoints are ``free.'' Here are some more examples of this.
\begin{example}
	The following functors are adjoint.
	\begin{itemize}
		\item The forgetful functor from commutative rings $\op{CRing}$ to sets $\op{Set}.$
		\item The free functor from a set $S\in\op{Set}$ to the polynomial ring $\ZZ[S]\in\op{CRing}.$
	\end{itemize}
\end{example}
\begin{example}
	The following functors are adjoint.
	\begin{itemize}
		\item The forgetful functor taking a complete metric space to a metric space.
		\item The completion functor taking a metric space $X$ to its completion $\overline X$ by Cauchy sequences modded by some equivalence relation.
	\end{itemize}
	Here completion is left adjoint to the forgetful functor.
\end{example}

\subsection{Products and Coproducts}
Let's talk about products. The universal definition, say in $\op{Set}$ is the set of pairs. This won't do in category theory, so here is our definition.
\begin{definition}[Prodcuts]
	Given objects $X$ and $Y,$ the \textit{product} object $X\times Y$ is universal in the set of morphisms to $X$ and $Y.$ Namely, we are given maps $\pi_X:X\times Y\to X$ and $\pi_Y:X\times Y\to Y$ such that, given any object $Z$ with maps $f_X:Z\to X$ and $f_Y:Z\to Y,$ there is a unique map $f:Z\to X\times Y$ making the following diagram commute.
	% https://q.uiver.app/?q=WzAsNCxbMCwwLCJaIl0sWzEsMSwiWFxcdGltZXMgWSJdLFsxLDIsIlgiXSxbMiwxLCJZIl0sWzEsMywiXFxwaV9ZIl0sWzEsMiwiXFxwaV9YIiwyXSxbMCwyLCJmX1giLDAseyJjdXJ2ZSI6Mn1dLFswLDMsImZfWSIsMix7ImN1cnZlIjotMn1dLFswLDEsImYiLDEseyJzdHlsZSI6eyJib2R5Ijp7Im5hbWUiOiJkYXNoZWQifX19XV0=
	\[\begin{tikzcd}
		Z \\
		& {X\times Y} & Y \\
		& X
		\arrow["{\pi_Y}"', from=2-2, to=2-3]
		\arrow["{\pi_X}", from=2-2, to=3-2]
		\arrow["{f_X}"', curve={height=12pt}, from=1-1, to=3-2]
		\arrow["{f_Y}", curve={height=-12pt}, from=1-1, to=2-3]
		\arrow["f"{description}, dashed, from=1-1, to=2-2]
	\end{tikzcd}\]
\end{definition}
\begin{example}
	We show that the product of sets $S$ and $T$ is indeed the product of sets; the maps $\pi_S:S\times T\onto S$ and $\pi_T:S\times T\onto T$ are the projections onto the corresponding coordinate. Now, suppose that we have a set $R$ and maps $f_S:R\to S$ and $f_R:R\to T$ making the following diagram commute.
	\[\begin{tikzcd}
		R \\
		& {S\times T} & T \\
		& S
		\arrow["{\pi_S}"', from=2-2, to=2-3]
		\arrow["{\pi_T}", from=2-2, to=3-2]
		\arrow["{f_S}"', curve={height=12pt}, from=1-1, to=3-2]
		\arrow["{f_R}", curve={height=-12pt}, from=1-1, to=2-3]
		\arrow["f"{description}, dashed, from=1-1, to=2-2]
	\end{tikzcd}\]
	Then we define $f:R\to S\times T$ by $f(r):=(f_Sr,f_Tr).$ This works because $\pi_S(f_Sr,f_Tr)=f_Sr$ and $\pi_T(f_Sr,f_Tr)=f_Tr,$ and this is forced because we must have $\pi_S(fr)=f_Sr$ and $\pi_T(fr)=f_Tr.$
\end{example}
\begin{warn} \label{warn:upiscaniso}
	Note that the product  $X\times Y,$ is not unique, but any two products are canonically isomorphic.
\end{warn}
Indeed, suppose we have two products $X\times Y$ and $X\times'Y$ with projections $\pi_X:X\times Y\to X$ and $\pi_Y:X\times Y\to Y$ and $\pi_X':X\times'Y\to X$ and $\pi_Y':X\times'Y\to Y.$ Then, we see that we get a $\pi':X\times' Y\to X\times Y$ making the following diagram commute.
% https://q.uiver.app/?q=WzAsNCxbMCwwLCJYXFx0aW1lcydZIl0sWzEsMSwiWFxcdGltZXMgWSJdLFsxLDIsIlgiXSxbMiwxLCJZIl0sWzEsMywiXFxwaV9ZIiwyXSxbMSwyLCJcXHBpX1giXSxbMCwyLCJcXHBpX1gnIiwyLHsiY3VydmUiOjJ9XSxbMCwzLCJcXHBpX1knIiwwLHsiY3VydmUiOi0yfV0sWzAsMSwiXFxwaSciLDEseyJzdHlsZSI6eyJib2R5Ijp7Im5hbWUiOiJkYXNoZWQifX19XV0=
\[\begin{tikzcd}
	{X\times'Y} \\
	& {X\times Y} & Y \\
	& X
	\arrow["{\pi_Y}"', from=2-2, to=2-3]
	\arrow["{\pi_X}", from=2-2, to=3-2]
	\arrow["{\pi_X'}"', curve={height=12pt}, from=1-1, to=3-2]
	\arrow["{\pi_Y'}", curve={height=-12pt}, from=1-1, to=2-3]
	\arrow["{\pi'}"{description}, dashed, from=1-1, to=2-2]
\end{tikzcd}\]
Similarly, we get a $\pi:X\times Y\to X\times'Y$ making the following diagram commute.
% https://q.uiver.app/?q=WzAsNCxbMCwwLCJYXFx0aW1lcyBZIl0sWzEsMSwiWFxcdGltZXMnIFkiXSxbMSwyLCJYIl0sWzIsMSwiWSJdLFsxLDMsIlxccGlfWSciLDJdLFsxLDIsIlxccGlfWCciXSxbMCwyLCJcXHBpX1giLDIseyJjdXJ2ZSI6Mn1dLFswLDMsIlxccGlfWSIsMCx7ImN1cnZlIjotMn1dLFswLDEsIlxccGkiLDEseyJzdHlsZSI6eyJib2R5Ijp7Im5hbWUiOiJkYXNoZWQifX19XV0=
\[\begin{tikzcd}
	{X\times Y} \\
	& {X\times' Y} & Y \\
	& X
	\arrow["{\pi_Y'}"', from=2-2, to=2-3]
	\arrow["{\pi_X'}", from=2-2, to=3-2]
	\arrow["{\pi_X}"', curve={height=12pt}, from=1-1, to=3-2]
	\arrow["{\pi_Y}", curve={height=-12pt}, from=1-1, to=2-3]
	\arrow["\pi"{description}, dashed, from=1-1, to=2-2]
\end{tikzcd}\]
But now we claim that $\pi$ and $\pi'$ are isomorphisms, for which we show that $\pi\circ\pi'=\id_{X\times'Y}$ and $\pi'\circ\pi=\id_{X\times Y}.$ We show that $\pi\circ\pi'=\id_{X\times'Y},$ and the other follows by symmetry. Well, for any $p\in X\times'Y,$ we see
\[\pi_\bullet'(p)=(\pi_\bullet\circ\pi')(p)=\pi_\bullet(\pi'p)=\pi'_\bullet\big((\pi\circ\pi')(p)\big).\]
for either projection $\pi'_\bullet.$ It follows that the following diagram commutes.
% https://q.uiver.app/?q=WzAsNCxbMCwwLCJYXFx0aW1lcycgWSJdLFsxLDEsIlhcXHRpbWVzJyBZIl0sWzEsMiwiWCJdLFsyLDEsIlkiXSxbMSwzLCJcXHBpX1knIiwyXSxbMSwyLCJcXHBpX1gnIl0sWzAsMiwiXFxwaV9YIiwyLHsiY3VydmUiOjJ9XSxbMCwzLCJcXHBpX1kiLDAseyJjdXJ2ZSI6LTJ9XSxbMCwxLCJcXHBpXFxjaXJjXFxwaSciLDEseyJzdHlsZSI6eyJib2R5Ijp7Im5hbWUiOiJkYXNoZWQifX19XV0=
\[\begin{tikzcd}
	{X\times' Y} \\
	& {X\times' Y} & Y \\
	& X
	\arrow["{\pi_Y'}"', from=2-2, to=2-3]
	\arrow["{\pi_X'}", from=2-2, to=3-2]
	\arrow["{\pi_X}"', curve={height=12pt}, from=1-1, to=3-2]
	\arrow["{\pi_Y}", curve={height=-12pt}, from=1-1, to=2-3]
	\arrow["{\pi\circ\pi'}"{description}, dashed, from=1-1, to=2-2]
\end{tikzcd}\]
However, making $\id_{X\times'Y}$ the induced arrow also makes the above diagram commute, so we must have $\pi\circ\pi'=\id_{X\times'Y}$ because the induced arrow is unique! (Here we used the uniqueness of the induced arrow.)

And now let's talk about coproducts, which is defined by reversing the arrows of products.
\begin{definition}[Coproducts]
	Given objects $X$ and $Y,$ the \textit{coproduct} object $X\oplus Y$ is universal in the set of morphisms to $X$ and $Y.$ Namely, we are given maps $\iota_X:X\to X\oplus Y$ and $\iota_Y:Y\to X\oplus Y$ such that, given any object $Z$ with maps $f_X:X\to Z$ and $f_Y:Y\to Z,$ there is a unique map $f:X\oplus Y\to Z$ making the following diagram commute.
	% https://q.uiver.app/?q=WzAsNCxbMSwwLCJYIl0sWzAsMSwiWSJdLFsxLDEsIlhcXG9wbHVzIFkiXSxbMiwyLCJaIl0sWzAsMywiZl9YIiwwLHsiY3VydmUiOi0yfV0sWzEsMywiZl9ZIiwyLHsiY3VydmUiOjJ9XSxbMCwyLCJcXGlvdGFfWCIsMl0sWzEsMiwiXFxpb3RhX1kiXSxbMiwzLCJmIiwxLHsic3R5bGUiOnsiYm9keSI6eyJuYW1lIjoiZGFzaGVkIn19fV1d
	\[\begin{tikzcd}
		& X \\
		Y & {X\oplus Y} \\
		&& Z
		\arrow["{f_X}", curve={height=-12pt}, from=1-2, to=3-3]
		\arrow["{f_Y}"', curve={height=12pt}, from=2-1, to=3-3]
		\arrow["{\iota_X}"', from=1-2, to=2-2]
		\arrow["{\iota_Y}", from=2-1, to=2-2]
		\arrow["f"{description}, dashed, from=2-2, to=3-3]
	\end{tikzcd}\]
\end{definition}
A similar warning as \autoref{warn:upiscaniso} applies here, but a similar proof gives our canonical isomorphisms; we will not write it down here.

Intuitively, the coproduct is the smallest object containing $X$ and $Y.$ Here are some objects.
\begin{example}
	The coproduct of two sets $S$ and $T$ is the disjoint union $S\sqcup T.$ The inclusions $\iota_S:S\into S\sqcup T$ and $\iota_T:T\into S\sqcup T$ are standard. Now, suppose that we have a set $R$ and maps $f_S:S\to R$ and $f_R:T\to R$ making the following diagram commute.
	\[\begin{tikzcd}
		& S \\
		T & {S\oplus T} \\
		&& R
		\arrow["{f_S}", curve={height=-12pt}, from=1-2, to=3-3]
		\arrow["{f_T}"', curve={height=12pt}, from=2-1, to=3-3]
		\arrow["{\iota_S}"', from=1-2, to=2-2]
		\arrow["{\iota_T}", from=2-1, to=2-2]
		\arrow["f"{description}, dashed, from=2-2, to=3-3]
	\end{tikzcd}\]
	Then we define $f:S\sqcup T\to R$ by $f((s,0))=f_Ss$ and $f((1,t))=f_Tt.$ This works because $f(\iota_Ss)=f_Ss$ and $f(\iota_Tt)=f_Ts,$ and this is forced by the same constraints.
\end{example}
\begin{example}
	Given two abelian groups $G_1$ and $G_2,$ the coproduct is $G_1\times G_2,$ where the inclusions $\iota_1:G_1\into G_1\times G_2$ and $\iota_2:G_2\into G_1\times G_2$ are defined by
	\[\iota_1:g_1\mapsto(g_1,0)\qquad\text{and}\qquad\iota_2:g_2\mapsto(0,g_2).\]
	In general, the (finite) product of two abelian groups is the coproduct, which is quite remarkable and rare. For example, the product and coproduct are different in $\op{Set}.$
\end{example}
\begin{example}
	More generally in groups, the coproduct of the two groups $G$ and $H,$ we want a group $G*H$ which is ``as big as possible'' given $G$ and $H$ as generators. So we take ``reduced'' words that look like
	\[g_1h_1h_2h_2\cdots,\]
	where $g_\bullet$ and $h_\bullet$ are nontrivial elements of $G$ and $H$ respectively. To show that all these ``reduced'' words are nontrivial, we make our words act on some giant graph. The idea is to build a Cayley graph by hand.
\end{example}
The point of this last example is that categorical coproducts, which look simple, are potentially very annoying.

\end{document}