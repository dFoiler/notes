\documentclass[../notes.tex]{subfiles}

\begin{document}

% !TEX root = ../notes.tex














Are you feeling nervous? Are you having fun?

\subsection{Modules}
Today we talk about modules. Here is the definition.
\begin{definition}[Module]
	A (left) module $M$ over a ring $R$ is an abelian group with a ``(left) ring action.'' In other words, we have an operation $\cdot:R\times M\to M$ satisfying some linearity axioms, as follows; fix $r,s\in R$ and $m,n\in M.$
	\begin{itemize}
		\item Distributive: $r(m+n)=rm+rn.$
		\item Distributive: $(r+s)m=rm+sm.$
		\item Associative: $(rs)m=r(sm).$
		\item Identity: $1_Rm=m.$
	\end{itemize}
\end{definition}
As usual, there is also a notion of right modules and two-sided modules, and this distinction matters for non-commutative rings.
\begin{example}
	Vector spaces are modules over fields. The field action is the scalar multiplication.
\end{example}
\begin{example}
	Abelian groups are modules over $\ZZ.$ The $\ZZ$-action on an abelian group $G$ is exponentiation by $n\cdot g\mapsto g^n.$
\end{example}
\begin{example}
	Ideals are equivalent to $R$-submodules of $R.$ Indeed, we have that left/right/two-sided ideals are left/right/two-sided $R$-submodules. I will not do this check because I am lazy; the main point is that closure of $I$ under multiplication by $R$ is the same thing as closure of $I$ under the $R$-action.
\end{example}
We also have maps between modules.
\begin{definition}[Module homomorphism]
	Fix $M$ and $N$ left modules over $R.$ Then $\varphi\in\op{Hom}_R(M,N)$ is a group homomorphism $\varphi:M\to N$ such that
	\[(r_1m_1+r_2m_2)\varphi=r_1(m_1\varphi)+r_2(m_2\varphi)\]
	where $r_1,r_2\in R$ and $m_1,m_2\in M.$
\end{definition}
Explicitly, if $M$ and $N$ are left modules, then $\varphi\in\op{Hom}_R(M,N)$ should be written on the right because the linearity condition requires
\[(rm)\varphi=r(m\varphi)\]
for $r\in R,m\in M,\varphi\in\op{Hom}_R(M,N).$ What is bad here is that writing on the other side gives $r\varphi(m)=\varphi(rm),$ which requires a switching of variables. This distinction matters for non-commutative rings, but I will largely ignore this and continue to write functions on the left.

We note the following.
\begin{proposition}
	If $M$ and $N$ are (left) modules over a commutative ring $R,$ then $\op{Hom}_R(M,N)$ is an $R$-module, where the action is
	\[(r\varphi)(m):=r\cdot\varphi(m).\]
\end{proposition}
\begin{proof}
	We have that $\op{Hom}_R(M,N)$ is an abelian group where addition is done pointwise; indeed, it is a subgroup of $\op{Hom}(M,N)$ closed under the subgroup test because $\varphi,\gamma\in\op{Hom}_R(M,N)$ still has $(\varphi-\gamma)$ an $R$-module homomorphism.

	Lastly we have to check that the $R$-action makes $\op{Hom}_R(M,N)$ into an $R$-module. This is relatively unenlightening. For example, we can check that
	\[\big((r_1+r_2)\varphi\big)(m)=(r_1+r_2)\varphi(m)=r_1\varphi(m)+r_2\varphi(m)=(r_1\varphi+r_2\varphi)(m)\]
	for any $r_1,r_2\in R,\varphi\in\op{Hom}_R(M,N),m\in M.$ I won't do the other checks out of lazy.
\end{proof}
We remark that if $R$ is not commutative, then $\op{Hom}_R(M,N)$ is merely an abelian group, not an $R$-module.

We continue with our examples.
\begin{definition}[Opposite ring]
	Fix $R$ a ring. Then we define the \textit{opposite ring} $R^{\op{op}}$ to have elements $r^{\op{op}}$ for $r\in R$ where our operations are defined by
	\[r^{\op{op}}+s^{\op{op}}:=(r+s)^{\op{op}}\qquad\text{and}\qquad r^{\op{op}}\cdot s^{\op{op}}:=(sr)^{\op{op}}\]
	This forms a ring, which can be checked by hand. In other words, the underlying abelian group is the same for $R$ and $R^{\op{op}},$ but the ring multiplication is flipped.
\end{definition}
The point of the above definition is the following example.
\begin{example}
	If $M$ is a left $R$-module, then $M$ is a right $R^{\op{op}}$-module by $m\cdot r^{\op{op}}:=r\cdot m.$ All of the distributivity axioms come for free from $M$ being a left $R$-module, and the associativity axiom holds because
	\[\left(mr^{\op{op}}\right)s^{\op{op}}=s(r(m))=(sr)m=m(sr)^{\op{op}}=m\left(r^{\op{op}}s^{\op{op}}\right).\]
\end{example}
\begin{warn}
	Left and right modules can be very different for a particular ring, namely when non-commutative.
\end{warn}
Explicitly, some rings have $R\cong R^{\op{op}},$ but not all. Of course this is true when $R$ is commutative; here is a less trivial example.
\begin{proposition}
	Fix $R$ a ring and $G$ a group so that $R[G]$ is the group ring. Then $R[G]\cong R[G]^{\op{op}}.$
\end{proposition}
\begin{proof}
	The idea is to consider the map $G\to R[G]^{\op{op}}$ by
	\[g\mapsto1_Rg^{-1}\]
	and use the universal property to lift this to a map $\varphi:R[G]\to R[G]^{\op{op}}.$ Explicitly,
	\[\varphi\left(\sum_{g\in G}r_gg\right):=\sum_{g\in G}r_gg^{-1}.\]
	We now check that $\varphi$ is an isomorphism of rings. This is not terribly interesting, but we will do it anyways.
	\begin{itemize}
		\item We see that $\varphi$ preserves addition because
		\[\varphi\left(\sum_{g\in G}r_gg+\sum_{g\in G}s_gg\right)=\varphi\left(\sum_{g\in G}(r_g+s_g)g\right)=\sum_{g\in G}(r_g+s_g)g^{-1}=\sum_{g\in G}r_gg^{-1}+\sum_{g\in G}s_gg^{-1},\]
		which is what we need.
		\item We see that $\varphi$ preserves multiplication because
		\[\varphi\left(\sum_{x\in G}r_xx\times\sum_{y\in G}s_yy\right)=\varphi\left(\sum_{g\in G}\left(\sum_{xy=g}r_xs_y\right)g\right)=\sum_{g\in G}\left(\sum_{xy=g}r_xs_y\right)g^{-1},\]
		but in the opposite ring, we have
		\[\varphi\left(\sum_{y\in G}s_yy\right)\varphi\left(\sum_{x\in G}r_xx\right)=\left(\sum_{y\in G}s_yy^{-1}\right)\left(\sum_{x\in G}r_xx^{-1}\right)=\sum_{g\in G}\left(\sum_{xy=g}r_xs_y\right)\left(y^{-1}x^{-1}\right),\]
		which is indeed $\sum_{g\in G}\left(\sum_{xy=g}r_xs_y\right)g^{-1}.$
		\item We see that $\varphi$ preserves identity because $\varphi(1e)=1e^{-1}=1e.$
		\item We see that $\varphi$ is surjective because, for any $\sum r_gg\in R[G]^{\op{op}},$ we have
		\[\varphi\left(\sum_{g\in G}r_{g^{-1}}g\right)=\sum_{g\in G}r_{g^{-1}}g^{-1}=\sum_{g\in G}r_gg.\]
		\item We see that $\varphi$ is injective because, if $\sum r_gg^{-1}=\sum s_gg^{-1},$ then $r_g=s_g$ for each $g.$
		\qedhere
	\end{itemize}
\end{proof}

\subsection{Hom Is Left Exact}
Suppose that we have an exact sequence of $R$-modules as follows.
% https://q.uiver.app/?q=WzAsNSxbMCwwLCIwIl0sWzEsMCwiQSJdLFsyLDAsIkIiXSxbMywwLCJDIl0sWzQsMCwiMCJdLFswLDFdLFsxLDIsIlxcdmFycGhpIl0sWzIsMywiXFxnYW1tYSJdLFszLDRdXQ==
\[\begin{tikzcd}
	0 & A & B & C & 0
	\arrow[from=1-1, to=1-2]
	\arrow["\varphi", from=1-2, to=1-3]
	\arrow["\gamma", from=1-3, to=1-4]
	\arrow[from=1-4, to=1-5]
\end{tikzcd}\]
Given a a fixed $R$-module $M,$ we can look at the following sequence.
% https://q.uiver.app/?q=WzAsNSxbMCwwLCIwIl0sWzEsMCwiXFxvcGVyYXRvcm5hbWV7SG9tfShNLEEpIl0sWzIsMCwiXFxvcGVyYXRvcm5hbWV7SG9tfShNLEIpIl0sWzMsMCwiXFxvcGVyYXRvcm5hbWV7SG9tfShNLEMpIl0sWzQsMCwiMCJdLFswLDFdLFsxLDIsIlxcdmFycGhpXFxjaXJjLSJdLFsyLDMsIlxcZ2FtbWFcXGNpcmMtIl0sWzMsNF1d
\[\begin{tikzcd}
	0 & {\operatorname{Hom}(M,A)} & {\operatorname{Hom}(M,B)} & {\operatorname{Hom}(M,C)} & 0
	\arrow[from=1-1, to=1-2]
	\arrow["{\varphi\circ-}", from=1-2, to=1-3]
	\arrow["{\gamma\circ-}", from=1-3, to=1-4]
	\arrow[from=1-4, to=1-5]
\end{tikzcd}\]
Most of this sequence is exact but not all.
\begin{proposition} \label{prop:homleft}
	Suppose that we have an exact sequence of $R$-modules
	\[\begin{tikzcd}
		0 & A & B & C & 0
		\arrow[from=1-1, to=1-2]
		\arrow["\varphi", from=1-2, to=1-3]
		\arrow["\gamma", from=1-3, to=1-4]
		\arrow[from=1-4, to=1-5]
	\end{tikzcd}\]
	Then, for any $R$-module $M,$
	% https://q.uiver.app/?q=WzAsNSxbMCwwLCIwIl0sWzEsMCwiXFxvcGVyYXRvcm5hbWV7SG9tfShNLEEpIl0sWzIsMCwiXFxvcGVyYXRvcm5hbWV7SG9tfShNLEIpIl0sWzMsMCwiXFxvcGVyYXRvcm5hbWV7SG9tfShNLEMpIl0sWzQsMCwiMCJdLFswLDFdLFsxLDIsIlxcdmFycGhpXFxjaXJjLSJdLFsyLDMsIlxcZ2FtbWFcXGNpcmMtIl0sWzMsNF1d
	\[\begin{tikzcd}
		0 & {\operatorname{Hom}(M,A)} & {\operatorname{Hom}(M,B)} & {\operatorname{Hom}(M,C)}
		\arrow[from=1-1, to=1-2]
		\arrow["{\varphi\circ-}", from=1-2, to=1-3]
		\arrow["{\gamma\circ-}", from=1-3, to=1-4]
	\end{tikzcd}\]
	is exact.
\end{proposition}
\begin{proof}
	We have two things to check
	\begin{itemize}
		\item Exact at $\op{Hom}(M,A)$: we have to show that $\op{Hom}(M,A)\to\op{Hom}(M,B)$ by $f\mapsto\varphi\circ f$ is injective. Well, suppose that $\varphi\circ f_1=\varphi\circ f_2$ for $f_1,f_2\in\op{Hom}(M,A).$ Then for any $m\in M,$ we have
		\[\varphi(f_1m)=\varphi(f_2m),\]
		so $f_1(m)=f_2(m)$ because $\varphi$ is injective. So indeed, $f_1=f_2.$
		\item Exact at $\op{Hom}(M,B)$: we have to show that the kernel of $\op{Hom}(M,B)\to\op{Hom}(M,C)$ by $g\mapsto\gamma\circ g$ is exactly the image of $f\mapsto\varphi\circ f.$

		In one direction, if $\varphi\circ f$ is in the image of $\op{Hom}(M,A)\to\op{Hom}(M,B),$ then for any $m\in M$ we have
		\[(\gamma\circ\varphi\circ f)(m)=(\gamma\circ\varphi)(fm)=0(fm)=0,\]
		so indeed, $\varphi\circ f$ is in the kernel of $\gamma\circ-.$

		In the other direction, fix any $g\in\op{Hom}(M,B)$ in the kernel of $\gamma\circ-$ so that $\gamma\circ g=0.$ This is equivalent to, for any $m\in M,$ having
		\[\gamma(g(m))=0,\]
		which is equivalent to $g(m)\in\ker\gamma,$ which is equivalent to $g(m)\in\im\varphi$ by exactness. Now, $\varphi$ is injective, so each $g(m)$ has a unique lift into $A,$ letting us define
		\[f(m):=\varphi^{-1}(g(m)).\]
		There is some check here to make sure $f\in\op{Hom}(M,A),$ which is not very interesting.\footnote{Note $f(r_1m_1+r_2m_2)$ is the unique element such that $\varphi(f(r_1m_1+r_2m_2))=g(r_1m_1+r_2m_2),$ but because $g(r_1m_1+r_2m_2)=r_1g(m_1)+r_2g(m_2),$ we see that $r_1f(m_1)+r_2f(m_2)$ goes to the same place under $\varphi.$} The point is that $g=\varphi\circ f,$ so $g$ is in the image of $\varphi\circ-.$
		\qedhere
	\end{itemize}
\end{proof}
However, $\op{Hom}(M,-)$ does not always produce sequences always exact at the end.
\begin{example}
	Consider the short exact sequence of $\ZZ$-modules
	\[0\to\ZZ\stackrel{\times2}\to\ZZ\to\ZZ/2\ZZ\to0\]
	and take $M:=\ZZ/2\ZZ.$ Then applying $\op{Hom}(\ZZ/2\ZZ,-),$ we note there are no nontrivial maps $\ZZ/2\ZZ\to\ZZ$ (the image of $1$ must double to $0,$ but the only element of additive order dividing $2$ is $0$ itself).
	
	On the other hand, $\op{Hom}(\ZZ/2\ZZ,\ZZ/2\ZZ)\cong\ZZ/2\ZZ$ by sending $1\mapsto0$ or $1\mapsto1,$ so the sequence
	\[0\to\op{Hom}(\ZZ/2\ZZ,\ZZ)\to\op{Hom}(\ZZ/2\ZZ,\ZZ)\to\op{Hom}(\ZZ/2\ZZ,\ZZ/2\ZZ)\to0\]
	becomes the sequences
	\[0\to0\to0\to\ZZ/2\ZZ\to0,\]
	which is not exact at the end, sadly.
\end{example}
\begin{remark}
	The high-level way to see \autoref{prop:homleft} is that $\op{Hom}(M,-)$ is right adjoint (to tensor), so $\op{Hom}$ preserves limits, so $\op{Hom}$ is left exact.
\end{remark}
Similarly, we can continue fixing an $R$-module $N$ and apply $\op{Hom}(-,N).$ This turns the sequence
\[\begin{tikzcd}
	0 & A & B & C & 0
	\arrow[from=1-1, to=1-2]
	\arrow["\varphi", from=1-2, to=1-3]
	\arrow["\gamma", from=1-3, to=1-4]
	\arrow[from=1-4, to=1-5]
\end{tikzcd}\]
into the sequence
% https://q.uiver.app/?q=WzAsNSxbMCwwLCIwIl0sWzEsMCwiXFxvcGVyYXRvcm5hbWV7SG9tfShBLE4pIl0sWzIsMCwiXFxvcGVyYXRvcm5hbWV7SG9tfShCLE4pIl0sWzMsMCwiXFxvcGVyYXRvcm5hbWV7SG9tfShDLE4pIl0sWzQsMCwiMCJdLFsxLDBdLFsyLDEsIi1cXGNpcmNcXHZhcnBoaSIsMl0sWzMsMiwiLVxcY2lyY1xcZ2FtbWEiLDJdLFs0LDNdXQ==
\[\begin{tikzcd}
	0 & {\operatorname{Hom}(A,N)} & {\operatorname{Hom}(B,N)} & {\operatorname{Hom}(C,N)} & 0
	\arrow[from=1-2, to=1-1]
	\arrow["{-\circ\varphi}"', from=1-3, to=1-2]
	\arrow["{-\circ\gamma}"', from=1-4, to=1-3]
	\arrow[from=1-5, to=1-4]
\end{tikzcd}\]
where the arrows are still composition as labeled.
\begin{warn}
	The above sequence of maps has the arrows reversed.
\end{warn}
In this case, we have the following.
\begin{proposition}
	Suppose that we have an exact sequence of $R$-modules
	\[\begin{tikzcd}
		0 & A & B & C & 0
		\arrow[from=1-1, to=1-2]
		\arrow["\varphi", from=1-2, to=1-3]
		\arrow["\gamma", from=1-3, to=1-4]
		\arrow[from=1-4, to=1-5]
	\end{tikzcd}\]
	Then, for any $R$-module $N,$ the sequence
	% https://q.uiver.app/?q=WzAsNCxbMCwwLCJcXG9wZXJhdG9ybmFtZXtIb219KEEsTikiXSxbMSwwLCJcXG9wZXJhdG9ybmFtZXtIb219KEIsTikiXSxbMiwwLCJcXG9wZXJhdG9ybmFtZXtIb219KE0sQykiXSxbMywwLCIwIl0sWzEsMCwiLVxcY2lyY1xcdmFycGhpIiwyXSxbMiwxLCItXFxjaXJjXFxnYW1tYSIsMl0sWzMsMl1d
	\[\begin{tikzcd}
		{\operatorname{Hom}(A,N)} & {\operatorname{Hom}(B,N)} & {\operatorname{Hom}(C,N)} & 0
		\arrow["{-\circ\varphi}"', from=1-2, to=1-1]
		\arrow["{-\circ\gamma}"', from=1-3, to=1-2]
		\arrow[from=1-4, to=1-3]
	\end{tikzcd}\]
	is exact.
\end{proposition}
\begin{proof}
	This is essentially the same as \autoref{prop:homleft}. We have two things to check.
	\begin{itemize}
		\item Exact at $\op{Hom}(C,N)$: essentially, we have to show that the kernel of $-\circ\gamma$ is trivial. So suppose that we have $f\in\op{Hom}(C,N)$ such that $f\circ\gamma=0.$ The, for any $c\in C,$ we note that the surjectivity of $\gamma$ promises $b\in B$ such that $\gamma b=c,$ implying
		\[f(c)=(f\circ\gamma)(b)=0,\]
		so $f$ is the zero map. So indeed, $\ker(f\mapsto f\circ\gamma)=\{0\}.$

		\item Exact at $\op{Hom}(B,N)$: we have to show that a map $g\in\op{Hom}(B,N)$ has $g\circ\varphi=0$ if and only if $g=f\circ\gamma$ for some $f\in\op{Hom}(C,N).$

		In one direction, suppose that $g=f\circ\gamma$ for some $f\in\op{Hom}(M,C).$ Then we have that
		\[g\circ\varphi=(f\circ\gamma)\circ\varphi=f\circ(\gamma\circ\varphi)=f\circ0=0,\]
		where $\gamma\circ\varphi=0$ because $\im\varphi\subseteq\ker\gamma$ by exactness.

		In the other direction, suppose that $g\circ\varphi=0.$ Then, $g(\im\varphi)=\{0\},$ so $\im\varphi\subseteq\ker g.$ In particular, $\ker\gamma\subseteq\im\varphi$ by exactness, so $\ker\gamma\subseteq\ker g.$ It follows that $g:B\to N$ can be made into a well-defined map
		\[\overline g:B/\ker\gamma\to N\]
		such that $B\to N$ is the same as $B\onto B/\ker\gamma\to N.$ Now, $\im\gamma\cong B/\ker\gamma,$ so we have the sequence of maps
		\[B\stackrel\gamma\onto\im\gamma\cong B/\ker\gamma\stackrel{\overline g}\to N\]
		whose composite is equal to $g$ by pushing through elements. Letting $f:C\to N$ be the composite $C=\im\gamma\cong B/\ker\gamma\to N,$ we find that $g=f\circ\gamma,$ which is exactly what we wanted.
		\qedhere
	\end{itemize}
\end{proof}
And again, we don't have to be fully exact.
\begin{example}
	Consider the short exact sequence of $\ZZ$-modules
	\[0\to\ZZ\stackrel{\times2}\to\ZZ\to\ZZ/2\ZZ\to0\]
	and take $N:=\ZZ/2\ZZ.$ Then applying $\op{Hom}(-,N),$ we see that $\op{Hom}(\ZZ,\ZZ/2\ZZ)\cong\ZZ/2\ZZ$ (we send $1\mapsto0$ or $1\mapsto1$), but also $\op{Hom}(\ZZ/2\ZZ)\cong\ZZ/2\ZZ$ as discussed last time.
	
	So the sequence
	\[0\from\op{Hom}(\ZZ,\ZZ/2\ZZ)\from\op{Hom}(\ZZ,\ZZ/2\ZZ)\from\op{Hom}(\ZZ/2\ZZ,\ZZ/2\ZZ)\from0\]
	is
	\[0\from\ZZ/2\ZZ\from\ZZ/2\ZZ\from\ZZ/2\ZZ\from0,\]
	which cannot be exact for size reasons: the left end would have to have size $2/2=1.$ And indeed, we can verify that the $\times2$ mapping is losing surjectivity at the end.
\end{example}
The lack of these exactness turns out to be a huge problem in algebra. The entire field of homological algebra is dedicated to fixing this problem.
\begin{remark}
	The short exact sequence
	\[0\to\ZZ\to\ZZ\to\ZZ/2\ZZ\to0\]
	is a good universal counterexample to various statements.
\end{remark}

\subsection{Free Modules}
We have the following definition.
\begin{defi}[Free]
	An $R$-module $M$ is \textit{free} if it is the direct sum of some number of copies of $R.$
\end{defi}
We have the following sequence of propositions.
\begin{proposition}
	Suppose that we have a split short exact sequence of $R$-modules
	\[0\to A\to B\to C\to 0.\]
	Then $B\cong A\oplus C,$ canonical up to the choice of lift $C\to B.$
\end{proposition}
\begin{proof}
	This requires some care. Label $\iota:A\to B,$ $\pi:B\to C,$ and $\rho:C\to B,$ where $\rho\circ\pi=\id_C$ is our lift of $\pi.$ Here is the diagram.
	% https://q.uiver.app/?q=WzAsNSxbMCwwLCIwIl0sWzEsMCwiQSJdLFsyLDAsIkIiXSxbMywwLCJDIl0sWzQsMCwiMCJdLFswLDFdLFsxLDIsIlxcaW90YSJdLFsyLDMsIlxccGkiLDAseyJvZmZzZXQiOi0xfV0sWzMsNF0sWzMsMiwiXFxyaG8iLDAseyJvZmZzZXQiOi0xfV1d
	\[\begin{tikzcd}
		0 & A & B & C & 0
		\arrow[from=1-1, to=1-2]
		\arrow["\iota", from=1-2, to=1-3]
		\arrow["\pi", shift left=1, from=1-3, to=1-4]
		\arrow[from=1-4, to=1-5]
		\arrow["\rho", shift left=1, from=1-4, to=1-3]
	\end{tikzcd}\]

	Now we note that we have a map $\varphi:A\oplus C\to B$ by
	\[\varphi:(a,c)\mapsto\iota a+\rho c.\]
	We show that this map is an $R$-module isomorphism. Note that it is an $R$-module homomorphism by using the universal property of $A\oplus C$ on the morphisms $\iota:A\to B$ and $\rho:C\to B,$ so the main obstruction is showing the isomorphism. We have two things to check.
	\begin{itemize}
		\item We show that $\varphi$ is injective. Indeed, suppose that $\varphi((a,c))=0$ so that $\iota a+\rho c=0$ and
		\[\iota(a)=\rho(-c).\]
		Applying $\pi$ to both sides implies that $0=-c$ because $\im\iota\subseteq\ker\pi$ by exactness (!). Thus, $c=0,$ implying $\iota(a)=0,$ so $a=0$ because $\ker\iota=\{0\}.$ Thus, $\ker\varphi=\{(0,0)\}.$
		\item We show that $\varphi$ is surjective. Indeed, fix any $b\in B.$ We start by taking $c:=\pi b$ and observe that
		\[\pi(b-\rho c)=\pi(b)-(\pi\circ\rho)(c)=c-c=0,\]
		so $b-\rho c\in\ker\pi.$ But $\ker\pi\subseteq\im\iota$ by exactness (!), so $b-\rho c=\iota a$ for some $a\in A.$ Thus,
		\[\varphi:(a,c)\mapsto\iota a+\rho c=b,\]
		which is what we needed.
		\qedhere
	\end{itemize}
\end{proof}
The reason we bring this up is to talk about free modules.
\begin{proposition} \label{prop:freeimpliessplit}
	If $C$ is a free $R$-module in the short exact sequence
	\[0\to A\to B\to C\to 0,\]
	then this short exact sequence splits so that $B\cong A\oplus C.$
\end{proposition}
\begin{proof}
	Label our short exact sequence as follows.
	% https://q.uiver.app/?q=WzAsNSxbMCwwLCIwIl0sWzEsMCwiQSJdLFsyLDAsIkIiXSxbMywwLCJDIl0sWzQsMCwiMCJdLFswLDFdLFsxLDIsIlxcaW90YSJdLFsyLDMsIlxccGkiXSxbMyw0XV0=
	\[\begin{tikzcd}
		0 & A & B & C & 0
		\arrow[from=1-1, to=1-2]
		\arrow["\iota", from=1-2, to=1-3]
		\arrow["\pi", from=1-3, to=1-4]
		\arrow[from=1-4, to=1-5]
	\end{tikzcd}\]
	Now, find a basis  $\{c_\alpha\}_{\alpha\in\lambda}$ for $C,$ and lift each element to $\{b_\alpha\}_{\alpha\in\lambda}$ in $B$ along $\pi$ so that $\pi:b_\alpha\mapsto c_\alpha.$ This induces a map $\rho:C\to B$ defined by
	\[\rho\left(\sum_{\alpha\in\lambda}r_\alpha c_\alpha\right):=\sum_{\alpha\in\lambda}r_\alpha b_\alpha,\]
	where we might have to mumble something about the universal property of free objects. (Here, all but finitely many of the $r_\alpha$ vanish.) Then we note that, for any element $\sum_\alpha r_\alpha c_\alpha\in C,$ we have
	\[(\pi\circ\rho)\left(\sum_{\alpha\in\lambda}r_\alpha c_\alpha\right)=\pi\left(\sum_{\alpha\in\lambda}r_\alpha b_\alpha\right)=\sum_{\alpha\in\lambda}r_\alpha\pi(b_\alpha)=\sum_{\alpha\in\lambda}r_\alpha c_\alpha,\]
	so $\pi\circ\rho=\id_C,$ so the short exact sequence splits due to this map. So indeed, $B\cong A\oplus C.$
\end{proof}
In particular, free modules make $\op{Hom}(M,-)$ into an exact functor.
\begin{proposition} \label{prop:hompreservessplit}
	Fix a split short exact sequence of $R$-modules
	\[0\to A\to B\to C\to 0.\]
	Then we have the short exact sequence
	\[0\to\op{Hom}(M,A)\to\op{Hom}(M,B)\to\op{Hom}(M,C)\to0.\]
	In particular, we get this whenever $C$ is free.
\end{proposition}
\begin{proof}
	Set $B\cong A\oplus C$ as induced by the short exact sequence and label our sequence by
	% https://q.uiver.app/?q=WzAsNSxbMCwwLCIwIl0sWzEsMCwiQSJdLFsyLDAsIkFcXG9wbHVzIEMiXSxbMywwLCJDIl0sWzQsMCwiMCJdLFswLDFdLFsxLDIsIlxcaW90YSJdLFsyLDMsIlxccGkiXSxbMyw0XV0=
	\[\begin{tikzcd}
		0 & A & {A\oplus C} & C & 0
		\arrow[from=1-1, to=1-2]
		\arrow["\iota", from=1-2, to=1-3]
		\arrow["\pi", from=1-3, to=1-4]
		\arrow[from=1-4, to=1-5]
	\end{tikzcd}\]
	and
	% https://q.uiver.app/?q=WzAsNSxbMCwwLCIwIl0sWzEsMCwiXFxvcGVyYXRvcm5hbWV7SG9tfShNLEEpIl0sWzIsMCwiXFxvcGVyYXRvcm5hbWV7SG9tfShNLEIpIl0sWzMsMCwiXFxvcGVyYXRvcm5hbWV7SG9tfShNLEMpIl0sWzQsMCwiMCJdLFswLDFdLFsxLDIsIlxcaW90YVxcY2lyYy0iXSxbMiwzLCJcXHBpXFxjaXJjLSJdLFszLDRdXQ==
	\[\begin{tikzcd}
		0 & {\operatorname{Hom}(M,A)} & {\operatorname{Hom}(M,A\oplus C)} & {\operatorname{Hom}(M,C)} & 0
		\arrow[from=1-1, to=1-2]
		\arrow["{\iota\circ-}", from=1-2, to=1-3]
		\arrow["{\pi\circ-}", from=1-3, to=1-4]
		\arrow[from=1-4, to=1-5]
	\end{tikzcd}\]
	At this point we only have to check that $\op{Hom}(M,A\oplus C)\to\op{Hom}(M,C)$ by $f\mapsto\pi\circ f$ is surjective. Indeed, fix any $g\in\op{Hom}(M,C),$ and we can lift it to $m\mapsto gm\mapsto(0,gm),$ which is what we needed.
\end{proof}
We remark that, given a free module $M$ generated by $\{m_\alpha\}_{\alpha\in\lambda},$ then we can describe $\op{Hom}_R(M,M)$ essentially just using matrices: if $f\in\op{Hom}_R(M,M),$ then we can describe
\[f(m_\alpha)=\sum_{\beta\in\lambda}a_{\alpha\beta}m_\beta,\]
which extends to just matrix multiplication. For example, if $M$ is finitely generated by $\{m_k\}_{k=1}^n,$ we can write
\[f(m_\ell)=\sum_{k=1}^na_{k\ell}m_k\]
so that $f$ corresponds to the matrix
\[\begin{bmatrix}
	a_{11} & \cdots & a_{1n} \\
	\vdots & \ddots & \vdots \\
	a_{n1} & \cdots & a_{nn}
\end{bmatrix}.\]
Of course, we have to be careful about the direction here.
\begin{warn}
	If $M$ is a left module, then matrix multiplication should (in a moral sense) happen on the right, as discussed earlier.
\end{warn}
Namely,
\[f\left(\sum_{\ell=1}^nr_\ell m_\ell\right)=\sum_{k=1}^n\sum_{\ell=1}^nr_\ell a_{k\ell}m_k.\]
corresponds to the multiplication
\[\begin{bmatrix}
	r_1 \\ \vdots \\ r_n
\end{bmatrix}\begin{bmatrix}
	a_{11} & \cdots & a_{1n} \\
	\vdots & \ddots & \vdots \\
	a_{n1} & \cdots & a_{nn}
\end{bmatrix}=\begin{bmatrix}
	r_1a_{11} + \cdots + r_na_{1n} \\
	\vdots \\
	r_1a_{n1} + \cdots + r_na_{nn}
\end{bmatrix}.\]

\subsection{Ranks of Free Modules}
We would like to define the rank of an $R$-module. The goal is for $\op{rank}R^n=n$ at the end, but this needs to depend only on the module itself. This is harder than it looks.
\begin{nex}
	Take $R=\{0\}.$ Then the rank is not well-defined because $R^n\cong\{0\}$ always.
\end{nex}
So there is something to do when trying to define the rank.
\begin{example}
	If $R$ is a field, then we can use dimension, which we are assured is well-defined.
\end{example}
\begin{example}
	If $R$ is a nonzero commutative ring, then given a module $M,$ we can take a maximal ideal $\mf m$ (which exists because $R\ne\{0\}$) and measure
	\[\op{rank}M:=\dim_{R/\mf m}M/\mf mM.\]
\end{example}
\begin{proposition}
	The above rank is well-defined. Explicitly, if $R$ is a nonzero commutative ring with $\mf m$ any maximal ideal, then $n=\dim_{R/\mf m}R^n/\mf mR^n.$
\end{proposition}
\begin{proof}
	Note that it suffices to show that $n=\dim_{R/\mf m}R^n/\mf mR^n,$ which will tell us that the rank does not depend on $\mf m.$ The main point is to show that
	\[R^n/\mf mR^n\cong(R/\mf m)^n\]
	as $R/\mf m$-vector spaces, which will finish because $(R/\mf m)^n$ is n $n$-dimensional $R/\mf m$-vector space.

	Indeed, we note that we have an $R$-module homomorphism
	\[\varphi:R^n\to(R/\mf m)^n\]
	by taking $(r_1,\ldots,r_n)\mapsto(r_1+\mf m,\ldots,r_n+\mf m).$ (That $\varphi$ is actually an $R$-module homomorphism comes down to checking that $R\onto R/\mf m$ is a ring map, which is true.) Further, $\varphi$ is surjective because we can lift any $(r_1+\mf m,\ldots,r_n+\mf m)$ back up to some $(r_1,\ldots,r_n).$

	Lastly, we note that $(r_1,\ldots,r_n)\in\ker\varphi$ if and only if $r_\bullet+\mf m=0+\mf m$ for each $r_\bullet,$ which is equivalent to $r_\bullet\in\mf m$ for each $r_\bullet.$ So the kernel is exactly $\mf m^n=\mf mR^n.$ So we have an $R$-module isomorphism
	\[\overline\varphi:R^n/\mf mR^n\to(R/\mf m)^n\]
	induced by $\varphi.$ To make this is an $R/\mf m$-module isomorphism, we first note that $R^n/\mf mR^n$ is indeed an $R/\mf m$-module where the $R/\mf m$-action is induced by $R$: the thing to check here is that the action is well-defined, for which it suffices to note $r_1+\mf m=r_2+\mf m$ implies
	\[r_1\left(v+\mf mR^n\right)=r_1v+\mf mR^n=r_2v+\underbrace{(r_1-r_2)}_{\in\mf m}v+\mf mR^n=r_2v+\mf mR^n,\]
	so the action is well-defined up to coset of $R/\mf m.$ So we still have that
	\[\overline\varphi:R^n/\mf mR^n\to(R/\mf m)^n\]
	is an isomorphism of abelian groups, and the $R/\mf m$-action is preserved because the $R/\mf m$-action is induced by the $R$-action, which is preserved. This finishes.
\end{proof}
\begin{example}
	Take $R:=k^{n\times n}.$ Then $R$ is a finite-dimensional vector space over $k,$ so may define $\op{rank}R^n:=\dim_kR^n/\dim_kR.$ The point here is that we have managed to define rank for a special non-commutative ring.
\end{example}
So it looks like maybe the rank is always well-defined for nonzero rings? This turns out to be true for ``small'' rings in some sense, but of course not for general rings because algebra is terrible.
\begin{proposition}
	We can construct a nonzero ring $R$ so that $R\cong R\oplus R.$
\end{proposition}
\begin{proof}
	Fix $S$ a ring with a free, right module $M$ so that $M\cong M\oplus M$ with $M\ne\{0\}.$ For example, $M=\ZZ^{\oplus\NN}$ as a $\ZZ$-module will do the trick.

	Now we define $R:=\op{End}_S(M).$ Checking that this is a ring is annoying, so we will not do it in detail. The point is that we can add endomorphisms, and multiplication is composition. (In particular, $R$ is not commutative.) Closure under addition and composition are a matter of writing out what we need to check.
	
	Visually, if $M$ is free over $S,$ then $R$ is the ring of matrices with infinite rows and columns with only finitely many nonzero elements. Here, $R$ acts on the left of $M$ in order to preserve the $S$-action on the right, and we can check that $M$ is a left $R$-module.
	\begin{itemize}
		\item $\varphi(m_1+m_2)=\varphi m_1+\varphi m_2$ because these are endomorphisms.
		\item $(\varphi_1+\varphi_2)m=\varphi_1m+\varphi_2m$ by definition of addition in $R.$
		\item $1_Rm=\id_Mm=m.$
	\end{itemize}
	The idea is to study how $R$ behaves with the isomorphism $\varphi:M\to M\oplus M.$ Because this is an isomorphism, it must be a homomorphism in each of the coordinates, so $a:=\pi_1\circ\varphi$ and $c:=\pi_2\circ\varphi$ (where $\pi_\bullet$ are the projections) must be homomorphisms. So we have $a,c\in\op{End}_S(M)=:R$ with
	\[\varphi:m\mapsto(am,cm).\]
	Conversely, because $\varphi$ is an isomorphism, we have a map $M\oplus M\to M$ which again must be a homomorphism in both coordinate by using the inclusions defining $M\oplus M.$ So by the universal property of $\oplus,$ we have $b,d\in\op{End}_S(M)=:R$ with
	\[\varphi^{-1}(m,n)\mapsto bm+dn.\]
	Composing as $\varphi\circ\varphi^{-1}=\id_{M\oplus M}$ and $\varphi^{-1}\circ\varphi=\id_M,$ we find
	\[(m,n)=\left(\varphi\circ\varphi^{-1}\right)(m,n)=\varphi(bm+dn)=(abm+adn,cbm+cdn),\]
	and
	\[m=\left(\varphi^{-1}\circ\varphi\right)(m)=\varphi^{-1}(am,cm)=bam+dcm.\]
	Comparing componentwise, we see that
	\[ab=1,\qquad ad=0,\qquad cb=0,\qquad cd=1,\qquad ba+dc=1.\]
	We remark that if $R$ were commutative, $ab=1$ and $cd=1$ and $ba+dc=1$ would imply that $1=2$ and $0=1,$ forcing $R$ to be the zero ring.
	
	Anyways, the point is that we have $R\cong R\oplus R$ by
	\[\gamma:r\mapsto(ar,cr)\qquad\text{and}\gamma^{-1}:(r,s)\mapsto br+ds.\]
	Essentially directly from the above computations we can check that
	\[\left(\gamma\circ\gamma^{-1}\right)(r,s)=\gamma(br+ds)=(abr+ads,cbr+cds)=(r,s),\]
	and
	\[\left(\gamma^{-1}\circ\gamma\right)(r)=\gamma^{-1}(ar,cr)=bar+dcr=(ba+dc)r=r.\]
	So we have a group isomorphism $R\cong R\oplus R.$ To make this an $R$-module homomorphism, we have $R$ act on itself on the right by multiplication, which is safe because $\gamma$ and $\gamma^{-1}$ only ever multiple on the left. For example, $\gamma$ is an $R$-module homomorphism because, for $x,r\in R,$ we have
	\[\gamma(x)\cdot r=(ax,cx)\cdot r=(axr,cxr)=\gamma(xr),\]
	and similar works for $\gamma^{-1}.$ Thus, $R\cong R\oplus R$ as (right) $R$-modules. If we wanted left $R$-modules, we could switch the directions of everything above.
\end{proof}
\begin{remark}
	It is also true that $R\cong R^{2\times2},$ but we've seen enough weird properties of this ring for today.
\end{remark}
So our rank is not always well-defined for the above $R.$ This is why people (or at least I) don't like modules over non-commutative rings.

\subsection{Projective Modules}
Recall from \autoref{prop:hompreservessplit} that free $R$-modules $C$ had the nice property of preserving the exactness of
\[B\to C\to0\]
upon applying $\op{Hom}(M,-).$ In other words, for any map $M\to C,$ we can lift it to a map $M\to B$ so that $\op{Hom}(M,B)\to\op{Hom}(M,C)$ is surjective. Here is the diagram.
% https://q.uiver.app/?q=WzAsNCxbMCwwLCJNIl0sWzEsMSwiQyJdLFsyLDEsIjAiXSxbMCwxLCJCIl0sWzAsMywiIiwwLHsic3R5bGUiOnsiYm9keSI6eyJuYW1lIjoiZGFzaGVkIn19fV0sWzAsMV0sWzMsMV0sWzEsMl1d
\[\begin{tikzcd}
	M \\
	B & C & 0
	\arrow[dashed, from=1-1, to=2-1]
	\arrow[from=1-1, to=2-2]
	\arrow[from=2-1, to=2-2]
	\arrow[from=2-2, to=2-3]
\end{tikzcd}\]
This property of ``lifting surjections'' is so nice that it has a name.
\begin{definition}[Projective]
	A module $M$ is \textit{projective} if it has the above property.
\end{definition}
\begin{example}
	Any free module is projective. Roughly speaking, this is by \autoref{prop:hompreservessplit}.
\end{example}
\begin{remark}[Nir]
	Here is a quick way to be convinced that ``projective'' is a good idea to care about: projective is what makes \autoref{prop:freeimpliessplit} work. Indeed, if
	% https://q.uiver.app/?q=WzAsNSxbMCwwLCIwIl0sWzEsMCwiQSJdLFsyLDAsIkIiXSxbMywwLCJDIl0sWzQsMCwiMCJdLFswLDFdLFsxLDIsIlxcaW90YSJdLFsyLDMsIlxccGkiXSxbMyw0XV0=
	\[\begin{tikzcd}
		0 & A & B & C & 0
		\arrow[from=1-1, to=1-2]
		\arrow["\iota", from=1-2, to=1-3]
		\arrow["\pi", from=1-3, to=1-4]
		\arrow[from=1-4, to=1-5]
	\end{tikzcd}\]
	is a short exact sequence with $C$ projective, then the surjection $\pi:B\onto C$ induces $\rho:C\to B$ so that $\pi\circ\rho=\id_C$ by projectivity, as in the following diagram.
	% https://q.uiver.app/?q=WzAsNCxbMCwxLCJCIl0sWzEsMSwiQyJdLFsyLDEsIjAiXSxbMCwwLCJDIl0sWzAsMSwiXFxwaSJdLFsxLDJdLFszLDAsIlxccmhvIiwyLHsic3R5bGUiOnsiYm9keSI6eyJuYW1lIjoiZGFzaGVkIn19fV0sWzMsMSwiXFxpZF9DIl1d
	\[\begin{tikzcd}
		C \\
		B & C & 0
		\arrow["\pi", from=2-1, to=2-2]
		\arrow[from=2-2, to=2-3]
		\arrow["\rho"', dashed, from=1-1, to=2-1]
		\arrow["{\id_C}", from=1-1, to=2-2]
	\end{tikzcd}\]
	So indeed, our original short exact sequence splits.
\end{remark}
Here is one way for us to generate lots of projective modules.
\begin{proposition} \label{prop:splitprojectivesum}
	Fix $M$ a projective $R$-module. Then if $M\cong P\oplus Q$ (as $R$-modules), then both $P$ and $Q$ are projective. We may call $P$ and $Q$ the ``split/direct summands.''
\end{proposition}
\begin{proof}
	We show that $P$ is projective, and $Q$ projective will follow by symmetry (because $M=Q\oplus P$). Fix any surjection $\varphi:B\onto C$ with a map $f:P\to C$ so that we lift $f$ to $\overline f$ making the following diagram commute.
	% https://q.uiver.app/?q=WzAsNCxbMCwxLCJCIl0sWzEsMSwiQyJdLFsyLDEsIjAiXSxbMSwwLCJQIl0sWzAsMSwiXFx2YXJwaGkiLDJdLFsxLDJdLFszLDEsImYiXSxbMywwLCJcXG92ZXJsaW5lIGYiLDEseyJzdHlsZSI6eyJib2R5Ijp7Im5hbWUiOiJkYXNoZWQifX19XV0=
	\[\begin{tikzcd}
		& P \\
		B & C & 0
		\arrow["\varphi"', from=2-1, to=2-2]
		\arrow[from=2-2, to=2-3]
		\arrow["f", from=1-2, to=2-2]
		\arrow["{\overline f}?"{description}, dashed, from=1-2, to=2-1]
	\end{tikzcd}\]
	Now, $f:P\to C$ induces a composite map $g$ by $M\onto P\to C$ using the canonical projection $M\onto P$ by $(p,q)\mapsto p,$ so because we have a map $g:M\to C,$ this lifts to a map $\overline g$ making the following diagram commute.
	% https://q.uiver.app/?q=WzAsNSxbMCwxLCJCIl0sWzEsMSwiQyJdLFsyLDEsIjAiXSxbMSwwLCJQIl0sWzAsMCwiTSJdLFswLDEsIlxcdmFycGhpIiwyXSxbMSwyXSxbMywxLCJmIl0sWzQsMywiIiwwLHsic3R5bGUiOnsiaGVhZCI6eyJuYW1lIjoiZXBpIn19fV0sWzQsMCwiXFxvdmVybGluZSBnIiwyXV0=
	\[\begin{tikzcd}
		M & P \\
		B & C & 0
		\arrow["\varphi"', from=2-1, to=2-2]
		\arrow[from=2-2, to=2-3]
		\arrow["f", from=1-2, to=2-2]
		\arrow[two heads, from=1-1, to=1-2]
		\arrow["{\overline g}"', from=1-1, to=2-1]
	\end{tikzcd}\]
	However, we also have a canonical inclusion $P\into M$ by $p\mapsto(p,0),$ so we have induced a map $\overline f$ by $P\into M\to B.$ We claim that this is the map we want. Indeed, we know that $(\varphi\circ\overline g)(p,q)=f(p)$ by construction of $\overline g,$ so
	\[(\varphi\circ\overline f)(p)=(\varphi\circ\overline g)(p,0)=f(p),\]
	which is exactly what we need.
\end{proof}
Note that we are not claiming that general submodules $P$ of free modules $M$ are projective: we need the short exact sequence
\[0\to P\to M\to M/P\to 0\]
split.

Here is another nice property of projective modules: this is the converse of \autoref{prop:splitprojectivesum}.
\begin{proposition} \label{prop:projextendfree}
	Fix $M$ a projective $R$-module. Then $M\oplus N$ is free for some $R$-module $N.$ If $M$ is finitely generated, we may let $M\oplus N$ be finitely generated.
\end{proposition}
\begin{proof}
	Fix $F$ any free module which can surject onto $M,$ and let $\pi:F\onto M$ be our surjection. (For example, the free module $\bigoplus_{m\in M}Rm$ generated by the letters of $M$ would do the trick. If $M$ is finitely generated, use the corresponding $F.$) Then the idea is to lift $\id_M:M\to M$ along the surjection $\pi:F\to M$ to some $\rho:M\to F.$ Here is the diagram.
	% https://q.uiver.app/?q=WzAsNCxbMCwxLCJGIl0sWzEsMSwiTSJdLFsyLDEsIjAiXSxbMCwwLCJNIl0sWzAsMSwiXFxwaSIsMix7InN0eWxlIjp7ImhlYWQiOnsibmFtZSI6ImVwaSJ9fX1dLFsxLDJdLFszLDEsIlxcaWRfTSJdLFszLDAsIlxcaW90YSIsMix7InN0eWxlIjp7ImJvZHkiOnsibmFtZSI6ImRhc2hlZCJ9fX1dXQ==
	\[\begin{tikzcd}
		M \\
		F & M & 0
		\arrow["\pi"', two heads, from=2-1, to=2-2]
		\arrow[from=2-2, to=2-3]
		\arrow["{\id_M}", from=1-1, to=2-2]
		\arrow["\rho"', dashed, from=1-1, to=2-1]
	\end{tikzcd}\]
	The point here is that the short exact sequence
	\[0\to\ker\pi\into F\onto M\to0\]
	will split due to $\rho$: by construction of $\rho,$ we have that $\pi\circ\rho=\id_M,$ which is exactly the condition to make this short exact sequence split. Thus, $M\oplus\ker\pi\cong F$ is free, which is what we wanted.
\end{proof}
\begin{remark}[Nir]
	Collecting our facts about projective modules, we have the following criteria for an $R$-module $C,$ which we claim are equivalent.
	\begin{enumerate}[label=(\roman*)]
		\item For each surjection $\pi:B\onto C$ and map $\varphi:M\to C,$ there exists a map $\overline\varphi:M\to B$ so that $\varphi=\pi\circ\overline\varphi.$ (I.e., the induced map $\op{Hom}(M,B)\stackrel{\pi\circ-}\to\op{Hom}(M,C)$ is surjective.)
		\item Every short exact sequence $0\to A\to B\to C\to 0$ splits.
		\item There exists a module $N$ such that $C\oplus N$ is free.
	\end{enumerate}
	From the above discussion, we already know (i) implies (ii) as well as (iii) implies (i). We can also see that (ii) implies (iii) by considering the short exact sequence
	\[0\to\ker\pi\into\bigoplus_{c\in C}Rc\stackrel\pi\onto C\to0,\]
	which must split and gives $C\oplus\ker\pi$ free.
\end{remark}

\subsection{Examples of Projective Modules}
We've been providing some theory on projective modules, but most of what we've done would only produce free modules as our examples. So it looks like projective might mean free, but here is an example saying no.
\begin{example}
	Note that $R$ is a free $R$-module, so if we can decompose $R=A\oplus B$ into $R$-modules, then $A$ and $B$ will be projective. For example, fix $R=\ZZ/6\ZZ=\ZZ/2\ZZ\oplus\ZZ/3\ZZ$ implies that $\ZZ/2\ZZ$ and $\ZZ/3\ZZ$ must then be projective $\ZZ/6\ZZ$-modules. (We technically have to check that $\ZZ/2\ZZ$ and $\ZZ/3\ZZ$ are $\ZZ/6\ZZ$-modules, and they are, induced by the $\ZZ$-action.)
	
	However, $\ZZ/2\ZZ$ is not free over $\ZZ/6\ZZ$ because it would have to have dimension strictly between $0$ and $1,$ which is impossible.
\end{example}
One potential complaint is that the above example more or less has zero-divisors built into $R$: if we can decompose $R\cong A\oplus B$ into two nonzero $R$-modules, then\footnote{Technically we are forcing some multiplication structure here.}
\[(a,0)\cdot(0,b)=(0,0)\]
for $a\in A\setminus\{0\}$ and $b\in B\setminus\{0\}$ forces $R$ to have zero-divisors.

So here is an example where $R$ is an integral domain.
\begin{exercise}
	Fix $R:=\ZZ[\sqrt{-5}]$ and $\mf p:=\left(2,1+\sqrt{-5}\right)$ a non-principal $R$-ideal. Then $\mf p$ is a projective but not free module.
\end{exercise}
\begin{proof}
	We start by showing that $\mf p$ is not free. Roughly speaking, this comes from the fact $\mf p$ is not a principal ideal. Quickly, we see that $\mf p$ is not freely generated by zero elements because $\mf p\ne\{0\},$ and $\mf p$ is not freely generated by one element because $\mf p$ is not principal.
	
	Now, supposing that $\mf p$ is generated by some set $\{z_\alpha\}_{\alpha\in\lambda}\subseteq\mf p$ with $\#\lambda\ge2,$ and we show the $z_\bullet$ do not freely generate. Indeed, the trick is that
	\[\big(\op N(z_\beta)\overline{z_\alpha}\big)z_\alpha+\big(-\op N(z_\alpha)\overline{z_\beta}\big)z_\beta=0\tag{$*$}\]
	for any $\alpha,\beta\in\lambda$ distinct elements. If $z_\alpha=0$ or $z_\beta=0,$ then of course the $z_\bullet$ do not freely generate. Otherwise, $(*)$ tells us that the map
	\[\bigoplus_{\alpha\in\lambda}Rz_\alpha\onto\mf p\]
	has kernel, making the $z_\bullet$ still not freely generate.

	We now check that $\mf p$ is projective. To start, we note that we have a surjection $\pi:R\oplus R\onto\mf p$ by
	\[\pi:(r,s)\mapsto2r+\left(1+\sqrt{-5}\right)s.\]
	But in fact we can split $\pi$ with $\rho:\mf p\to R\oplus R$ by
	\[\rho:x\mapsto\left(-x,\frac{1-\sqrt{-5}}2x\right).\]
	This is well-defined because $\rho(2)=\left(-2,1-\sqrt{-5}\right)\in R\oplus R,$ and $\rho\left(1+\sqrt{-5}\right)=\left(-1-\sqrt{-5},3\right)\in R\oplus R.$ Further, we can compute
	\[(\pi\circ\rho)(x)=\pi\left(-x,\frac{1-\sqrt{-5}}2x\right)=-2x+\left(1+\sqrt{-5}\right)\left(\frac{1-\sqrt{-5}}2\right)x=-2x+3x=x,\]
	so indeed, $\pi\circ\rho=\id_\mf p.$ The point is that the short exact sequence
	\[0\to\ker\pi\into R\oplus R\onto\mf p\to 0\]
	splits, so $\mf p\oplus\ker\pi\cong R\oplus R.$ It follows that $\mf p$ is projective by \autoref{prop:splitprojectivesum}.
\end{proof}

We continue with the examples.
\begin{exercise}
	Consider the M\"obius strip $X$ as a line bundle over $S^1$; let $\pi$ be our standard projection $X\onto S^1.$ I will not TeX a diagram of this, out of laziness. To get our module, we define
	\[R:=\left\{\text{continuous functions }r:S^1\to\RR\right\}\]
	and
	\[M:=\left\{\text{continuous functions }m:S^1\to X\text{ such that }\pi\circ\rho=\id_{S^1}\right\}.\]
	Then $M$ is a projective but not free $R$-module.
\end{exercise}
\begin{proof}
	Here is the image of the M\"obius strip $X$ projecting on $S^1$ by $\pi:X\to S^1.$ We have highlighted the fiber of a particular point $x\in S^1$ and explicitly note that it is a (one-dimensional) vector space.
	\begin{center}
		\begin{asy}
			import graph3;
			// thank you https://asymptote.sourceforge.io/asymptote_tutorial.pdf
			unitsize(2cm);
			currentprojection=orthographic(0,-2,1);
			
			real offset = pi;
			triple F(pair uv) {
				real t = uv.x;
				real r = uv.y;
				return (
					cos(t+offset) + r*cos(t)*sin(t/2),
					sin(t+offset) + r*sin(t)*sin(t/2),
					r*cos(t/2)
				);
			}
			real r = 0.2;
			surface moeb = surface(F, (0,-r), (2pi,r), Spline);
			draw(moeb, surfacepen=material(gray(0.7)+opacity(0.7), emissivepen=0.2 white));
			
			real x(real t) {return cos(2pi*t);}
			real y(real t) {return sin(2pi*t);}
			real z1(real t) {return 0;}
			real z2(real t) {return -2;}
			draw(graph(x,y,z1,0,1,operator ..));
			draw(graph(x,y,z2,0,1,operator ..));
		
			label("$X$", (-1.5,0,0));
			label("$S^1$", (-1.5,0,-2));
			draw((-1.5,0,-0.3) -- (-1.5,0,-2+0.3), EndArrow3);
			label("$\pi$", (-1.63,0,-1));
			
			real theta = -1;
			dot((cos(theta), sin(theta), 0), red);
			dot((cos(theta), sin(theta), -2), red);
			draw((cos(theta), sin(theta), -2+0.3) -- (cos(theta), sin(theta), -0.3), arrow=Arrow3(DefaultHead2(),Fill,emissive(red)), p=red+dashed);
			
			real t0 = theta-offset;
			real r0 = -0.4;
			draw(F((t0,-r0)) -- F((t0,r0)), arrow=Arrow3(DefaultHead2(),Fill,emissive(red)), p=red);
			draw(F((t0,r0)) -- F((t0,-r0)), arrow=Arrow3(DefaultHead2(),Fill,emissive(red)), p=red);
			
			label("$x$", (1.3*cos(theta), 1.3*sin(theta), -2), p=red);
			label("$\pi^{-1}(x)$", (1.3*cos(theta-0.1)+0.5, 1.3*sin(theta-0.1), 0), p=red);
		\end{asy}
	\end{center}
	Now, an element of $M$ is a ``global section'' of $X,$ which means it is a continuous function $m:S^1\to X$ such that $S^1\stackrel m\to X\stackrel\pi\to S^1$ is the identity. For example, here is such a section.
	\begin{center}
		\begin{asy}
			import graph3;
			// thank you https://asymptote.sourceforge.io/asymptote_tutorial.pdf
			unitsize(2cm);
			currentprojection=orthographic(0,-2,1);
			
			real offset = pi;
			triple F(pair uv) {
				real t = uv.x;
				real r = uv.y;
				return (
					cos(t+offset) + r*cos(t)*sin(t/2),
					sin(t+offset) + r*sin(t)*sin(t/2),
					r*cos(t/2)
				);
			}
			real r = 0.2;
			surface moeb = surface(F, (0,-r), (2pi,r), Spline);
			draw(moeb, surfacepen=material(gray(0.7)+opacity(0.7), emissivepen=0.2 white));
			
			real x(real t) {return cos(2pi*t);}
			real y(real t) {return sin(2pi*t);}
			real z1(real t) {return 0;}
			real z2(real t) {return -2;}
			draw(graph(x,y,z1,0,1,operator ..));
			draw(graph(x,y,z2,0,1,operator ..));
	
			label("$X$", (-1.5,0,0));
			label("$S^1$", (-1.5,0,-2));
			draw((-1.55,0,-0.3) -- (-1.55,0,-2+0.3), EndArrow3);
			label("$\pi$", (-1.68,0,-1));
			draw((-1.45,0,-2+0.3) -- (-1.45,0,-0.3), Arrow3(DefaultHead2(),Fill,emissive(green)), p=green);
			label("$m$", (-1.31,0,-1), p=green);
			
			real rs(real t)
			{
				return 0.15*cos(t-0.5) + 0.03*sin(7*(t-0.5)) + 0.08*sin(2*(t-0.5));
			}
			real fx(real t) {return cos(t+offset) + rs(t)*cos(t)*sin(t/2);}
			real fy(real t) {return sin(t+offset) + rs(t)*sin(t)*sin(t/2);}
			real fz(real t) {return rs(t)*cos(t/2);}
			draw(graph(fx,fy,fz, 0, 2pi, operator ..), green+linewidth(1.5));
			
			void drawProjection(real theta, pen thisPen)
			{
				triple top = (fx(theta),fy(theta),fz(theta));
				triple bot = (cos(theta+pi),sin(theta+pi),-2);
				draw(0.07*top+0.93*bot -- 0.93*top+0.07*bot, arrow=Arrow3(DefaultHead2(),Fill,emissive(thisPen)), p=thisPen);
				dot(top, p=thisPen);
				dot(bot, p=thisPen);
			}
			drawProjection(-2pi/3-offset+2pi, orange);
			drawProjection(-pi/6-offset+2pi, red);
		\end{asy}
	\end{center}
	Now, for any $r\in R$ and $m\in M,$ we define $rm$ to act pointwise. For example, if $r\equiv2,$ then $rm$ essentially stretches $r$ by $2.$

	To start, we claim that
	\[M\oplus M\cong R\oplus R,\]
	which will give the needed projectivity. Essentially, $M\oplus M$ allows us two copies of the M\"obius strip, which we can lay orthogonally as in the following diagram.
	\begin{center}
		\begin{asy}
			import graph3;
			// thank you https://asymptote.sourceforge.io/asymptote_tutorial.pdf
			unitsize(2cm);
			currentprojection=orthographic(0,-2,1);
			
			real offset = pi;
			triple F(pair uv) {
				real t = uv.x;
				real r = uv.y;
				return (
					cos(t+offset) + r*cos(t)*sin(t/2),
					sin(t+offset) + r*sin(t)*sin(t/2),
					r*cos(t/2)
				);
			}
			real r = 0.2;
			surface moeb = surface(F, (0,-r), (2pi,r), Spline);
			draw(moeb, surfacepen=material(rgb(0.5,0.9,0.5)+opacity(0.4), emissivepen=0.2 white));
			triple Fperp(pair uv) {
				real t = uv.x;
				real r = uv.y;
				return (
					cos(t+offset) + r*cos(t+pi)*sin((t+pi)/2),
					sin(t+offset) + r*sin(t+pi)*sin((t+pi)/2),
					r*cos((t+pi)/2)
				);
			}
			r = 0.1;
			surface moeb = surface(Fperp, (0,-r), (2pi,r), Spline);
			draw(moeb, surfacepen=material(rgb(0.5,0.5,0.9)+opacity(0.4), emissivepen=0.2 white));
			
			real x(real t) {return cos(2pi*t);}
			real y(real t) {return sin(2pi*t);}
			real z1(real t) {return 0;}
			real z2(real t) {return -2;}
			draw(graph(x,y,z1,0,1,operator ..));
			draw(graph(x,y,z2,0,1,operator ..));
		
			label("$X$", (-1.5,0,0));
			label("$S^1$", (-1.5,0,-2));
			draw((-1.5,0,-0.3) -- (-1.5,0,-2+0.3), EndArrow3);
			label("$\pi$", (-1.63,0,-1));
			
			real theta = -0.4;
			dot((cos(theta), sin(theta), 0), red);
			dot((cos(theta), sin(theta), -2), red);
			draw((cos(theta), sin(theta), -2+0.3) -- (cos(theta), sin(theta), -0.3), arrow=Arrow3(DefaultHead2(),Fill,emissive(red)), p=red+dashed);
			
			real t0 = theta-offset;
			real r0 = -0.4;
			draw(F((t0,-r0)) -- F((t0,r0)), arrow=Arrow3(DefaultHead2(),Fill,emissive(rgb(0.5,0.9,0.5))), p=rgb(0.5,0.9,0.5));
			draw(F((t0,r0)) -- F((t0,-r0)), arrow=Arrow3(DefaultHead2(),Fill,emissive(rgb(0.5,0.9,0.5))), p=rgb(0.5,0.9,0.5));
			draw(Fperp((t0,-r0)) -- Fperp((t0,r0)), arrow=Arrow3(DefaultHead2(),Fill,emissive(rgb(0.5,0.5,0.9))), p=rgb(0.5,0.5,0.9));
			draw(Fperp((t0,r0)) -- Fperp((t0,-r0)), arrow=Arrow3(DefaultHead2(),Fill,emissive(rgb(0.5,0.5,0.9))), p=rgb(0.5,0.5,0.9));
			
			label("$x$", (1.3*cos(theta), 1.3*sin(theta), -2), p=red);
			label("$\pi^{-1}(x)$", (1.3*cos(theta-0.1)+0.5, 1.3*sin(theta-0.1), 0), p=rgb(0.5,0.9,0.5));
			label("$\pi^{-1}(x)$", (1.3*cos(theta+0.1)+0.1, 1.3*sin(theta+0.1), -0.4), p=rgb(0.5,0.5,0.9));
		\end{asy}
	\end{center}
	But having two axes at each point is merely assigning an $\RR^2$ to each point of $S^1,$ so this is the vector bundle $S^1\times\RR^2.$ Now, a section $S^1\to S^1\times\RR^2$ is pretty much just a pair of functions $S^1\to\RR,$ which precisely describes $R\oplus R.$ So indeed, $M\oplus M\cong R\oplus R.$

	We now check that $M$ is not free. The main point is that, due to the twisting, going around the edge of the M\"obius returns us to the opposite side, so any continuous section $m:S^1\to X$ must intersect the central $S^1$ somewhere.\footnote{Thinking about $m$ as the composite $\left[0,2\pi\right)\to S^1\to X,$ we are saying $m(0)=-m(2\pi)$ because walking around the loop flips the sign.} In terms of the line bundle, we are saying that all global sections vanish somewhere.
	
	Now suppose for the sake of contradiction $M$ is free. Because we already know that
	\[M\oplus M\cong R\oplus R,\]
	and because $R$ is a commutative ring, we see that ranks are well-defined, so $M$ free implies that $M$ is generated by a single element, say $m.$ But $m$ vanishes at some $x_0,$ so anything in $Rm$ will also vanish at $x_0,$ so $Rm\ne M,$ which is a contradiction.
\end{proof}
\begin{remark}
	Nobody seems to be able to understand the above example in lecture. I am no exception.
\end{remark}
In general, the above examples have been taking the theme of finding a module $M$ such that $M\oplus M\cong R\oplus R.$ What this means is that our projective modules look ``locally'' like a free module but not globally, in the sense that we can add enough copies to get the free module.

% We remark that we can also turn $\mf p:=\left(2,1+\sqrt{-5}\right)$ will also show up similar to this global twist. For example, we have the map $\mf p\to\op{Spec}R$ which essentially turns the example into some kind of line bundle, analogous to the M\"obius strip.

\subsection{Stably Free Modules}
Recall from \autoref{prop:projextendfree} that, if $P$ is a projective $R$-module, then $P\oplus Q$ is free for some $Q.$ We might hope to be able to constrain $Q$ in some way; of course it must be projective by \autoref{prop:splitprojectivesum}, but perhaps we can do more.
\begin{proposition}
	Fix $P$ a projective $R$-module. Then $P\oplus Q$ is free for some free $R$-module $Q.$
\end{proposition}
\begin{proof}
	We start by using \autoref{prop:projextendfree} to get some $N$ such that $P\oplus N$ is free. Then the trick is to study
	\[Q:=\bigoplus_{k\in\ZZ}(N\oplus P).\]
	On one hand, $N\oplus P\cong P\oplus N$ is free, so $Q$ is free because it is the direct sum of free modules. On the other hand, we can write
	\[P\oplus Q=P\oplus\bigoplus_{k\in\ZZ}(N\oplus P)=\bigoplus_{k\in\ZZ}(P\oplus N).\]
	In other words,
	\[P\oplus Q=P\oplus(N\oplus P)\oplus(N\oplus P)\oplus\cdots=(P\oplus N)\oplus(P\oplus N)\oplus\cdots.\]
	Anyways, the point is that $P\oplus Q$ is free because it is the direct sum of the free modules $P\oplus N.$ So $Q$ is a free module, and $P\oplus Q$ is a free module.
\end{proof}
\begin{remark}
	This is similar to the ``proof'' that $0=1$ by
	\[0=(1+-1)+(1+-1)+\cdots=1+(-1+1)+(-1+1)+\cdots=1.\]
\end{remark}
We might hope to make $Q$ of finite rank, but this is not always possible. Such modules have a name.
\begin{definition}[Stably free]
	Fix an $R$-module $M.$ Then we say that $M$ is \textit{stably free} if and only if there exists $n\in\NN$ such that $M\oplus R^n$ is free and finitely generated.
\end{definition}
As promised, not all projective modules are stably free.
\begin{proposition}
	There exist projective modules which are not stably free.
\end{proposition}
\begin{proof}
	As above, take $M$ the global sections M\"obius strip form a $R$-module as defined above. Intuitively, $M$ is not stably free because adding any finite number of $R$ to the M\"obius band cannot ``untwist'' the M\"obius band, implying that it will never be free.
	
	Explicitly, if $M\oplus R^n$ were free for some $n\ge0,$ then, comparing ranks of
	\[\left(M\oplus R^n\right)\oplus \left(M\oplus R^n\right)\cong(M\oplus M)\oplus R^{2n}\cong R^{2n+2},\]
	we still would need $M\oplus R^n$ generated by $n+1$ element. But we need $n$ generators for $R^n,$ so we only have one degree of freedom for $M,$ which fails for the same reasons as before. (Please don't ask me to rigorize this.)
\end{proof}
But nontrivial stably free modules do exist.
\begin{proposition}
	There exist stably free modules which are not free modules.
\end{proposition}
\begin{proof}
	Take the tangent bundle of $S^2,$ where the total space consists of the tangent space of each point on $S^2.$ Now our ring $R$ consists of continuous functions $S^2\to\RR$ and our module $M$ is vector fields on $S^2,$ where the action is again just scalar multiplication.
	
	We have the following checks.
	\begin{itemize}
		\item We see that $M$ is stably free (and hence projective) because
		\[M\oplus R\cong R^3.\]
		Indeed, what is happening is that the extra $\oplus R$ will encode the orthogonal bundle to the tangent bundle, so now are essentially associating a full $\RR^3$ to each point on $S^2,$ which is exactly $R^3.$
		\item We check that $M$ is not free. This follows from the Hairy ball theorem, which tells us that every vector field in $M$ must vanish somewhere.
		
		Indeed, suppose for the sake of contradiction $M$ was free. Then, comparing ranks of $M\oplus R\cong R^3,$ we see that $M$ must have rank $2,$ with generators we name $m_1,m_2,$ which vanish at (say) $x_1,x_2\in S^2.$ But then, any $R$-linear combination
		\[r_1m_1+r_2m_2\]
		will have $(r_1m_1+r_2m_2)(x_1)=r_2m_2(x_1)$ parallel to $m_2(x_1)$ at $x_1.$ But there are vector fields which are perpendicular to $m_2(x_1)$ at $x_1,$ so we have not covered all of $M,$ which is our contradiction.
		\qedhere
	\end{itemize}
\end{proof}
We remark that if we use a torus $S^1\times S^1$ instead of the sphere, then the module is free because we no longer have access to the Hairy ball theorem.

\end{document}