% !TEX root = ../notes.tex













I am a small boat, and these are big waves.

\subsection{Duality for Vector Spaces}
Today is module miscellany. Recall duality for vector spaces.
\begin{definition}[Vector space duality]
	Given a $k$-vector space $V,$ we define
	\[V^*:=\op{Hom}_k(V,k).\]
\end{definition}
Then we know there is a natural map $V\to V^{**},$ given by
\[v\mapsto(\varphi\mapsto\varphi v),\]
which is notably canonical. This is an isomorphism if $V$ is finite-dimensional because we can check $V\to V^{**}$ is injective\footnote{If $\varphi\mapsto\varphi v$ is the same as $\varphi\mapsto\varphi w$ for each $\varphi\in V^*,$ then fixing a basis $\{\beta_\alpha\}_{\alpha\in\lambda},$ $\varphi$ projecting onto a basis element detects $v=w$} and $\dim V=\dim V^*=\dim V^{**}$ shows that we are bijective for size reasons.

However, these size reasons are no longer valid for infinite-dimensional vector spaces, so we might have $V^*$ larger than $V.$
\begin{nex}
	Fix a $k$-vector space $V$ with basis given by $\{v_n\}_{n\in\NN}$ such that
	\[V=\bigoplus_{n\in\NN}kv_n.\]
	Now we claim that $V^*\cong k^\NN,$ which is of strictly larger cardinality that $V\cong k^{\oplus\NN}.$ Indeed, we associate $\{a_n\}_{n\in\NN}$ with the linear map
	\[\sum_{n=1}^\infty k_nv_n\mapsto\sum_{n=1}^\infty a_nk_n.\]
	The sum converges because all but finitely many terms are nonzero. Now, certainly each $\{a_n\}_{n\in\NN}$ is a linear map, and all linear maps take this form by tracking where each individual $v_\bullet$ goes.
\end{nex}
In analysis, we usually put a topology on $V,$ which makes things better to name.

Before going into the next example, we take the following definition.
\begin{definition}[\texorpdfstring{$L^p$}{} spaces]
	Fix $X$ an integrable space. Then, for a real number $p>0,$ we define the space $L^p(X)$ to consist of integrable functions $f:X\to\RR$ such that
	\[\left(\int_X|f(x)|^p\,dx\right)^{1/p}\in\RR.\]
	In line with this definition, we define $L^\infty(X)$ to consist of bounded integrable functions.
\end{definition}
Our example will take $X=\NN,$ where $\int_Xdx$ turns into $\sum_{n\in\NN}.$
\begin{example}
	Fix $V:=C_0(\NN),$ which consists all (continuous) sequences $\NN\to\RR$ which tend to $0,$ and we note we have a topology induced by
	\[\sup_n|c_n|\]
	where $\{c_n\}_{n\in\NN}\in V.$
\end{example}
Let's talk through some of the duals of $V=C_0(\NN).$
\begin{itemize}
	\item We see $V^*\cong L^1(\NN).$ Indeed, for each $\{d_n\}_{n\in\NN}\in L^1(\NN),$ we have the linear map
	\[\{c_n\}_{n\in\NN}\mapsto\sum_{n\in\NN}c_nd_n,\]
	which converges because the $c_k$ are bounded. In one direction, certainly all of these are linear maps. In the other direction, suppose $T:V\to\RR$ is a linear transformation. Not by caring convergence too much, we note that
	\[T\left(\sum_{k=1}^\infty c_k\right)=\sum_{k=1}^\infty c_k\underbrace{T\left(\{1_{\ell=k}\}_{\ell\in\NN}\right)}_{=:d_k},\]
	where this works over finite sums and extends to infinite sums in the limit. In particular, because the left-hand side must converge, the right-hand side needs to converge as well, so $d_k\to0$ as $k\to\infty.$

	It remains to show that $\{d_k\}_{k\in\NN}\in L^1(\NN).$ Suppose for the sake of contradiction that
	\[\sum_{k\in\NN}|d_k|=\infty.\]
	Then,  we set $m=0$ and say that for each $m\ge1,$ there exists $n_m\ge n_{m-1}$ such that
	\[\sum_{k=n_{m-1}+1}^{n_m}|d_k|>1.\]
	Now we set $c_k=\op{sgn}(d_k)\frac1m,$ where $n_{m-1}<k\le n_m.$ The point is that
	\[\sum_{k=1}^\infty d_kc_k=\sum_{m=1}^\infty\sum_{k=n_{m-1}+1}^{n_m}d_kc_k=\sum_{m=1}^\infty\frac1m\sum_{k=n_{m-1}+1}^{n_m}|d_k|<\sum_{m=1}^\infty\frac1m=\infty,\]
	which contradicts the $\{d_k\}_{k\in\NN}$ defining an element of $V^*.$
	\item We see $V^{**}\cong L^\infty(\NN)$ using a similar argument as above. I am too lazy to work this out.
	% Indeed, for each $\{d_n\}_{n\in\NN}\in L^1(\NN),$ we have the linear map
	% \[\{c_n\}_{n\in\NN}\mapsto\sum_{n\in\NN}c_nd_n,\]
	% which converges because the $d_k$ are bounded this time. In one direction, certainly all of these are linear maps. In the other direction, suppose $T:V\to\RR$ is a linear transformation so that
	% \[T\left(\sum_{k=1}^\infty c_k\right)=\sum_{k=1}^\infty c_k\underbrace{T\left(\{1_{\ell=k}\}_{\ell\in\NN}\right)}_{=:d_k},\]
	% where this works over finite sums and extends to infinite sums in the limit.

	% It remains to show that $\{d_k\}_{k\in\NN}\in L^\infty(\NN).$ Well, using the sequence $c_n:=\op{sgn}(d_n),$ we see that
	% \[\sum_{k=1}^\infty c_kd_k=\sum_{k=1}^\infty|d_k|\]
	% must converge, which is what we wanted.
	\item Continuing $V^{***}$ has to do with ``contents'' of $\NN,$ but at this point, we need the axiom of choice to find an example which isn't from $V^*=L^1(\NN).$ Here we'll stop.
\end{itemize}
So the above examples show a strictly ascending chain of double duals even though, say, $V$ and $V^{**}$ do have the same cardinality.

We also have the following general result for $L^p$ spaces.
\begin{exercise}
	Fixing $p\in(0,\infty),$ we have that $L^p(\NN)^*=L^q(\NN),$ where $q\in\RR$ is chosen with $\frac1p+\frac1q=1.$
\end{exercise}
\begin{proof}
	Again, the point is that we can associate $\{d_n\}_{n\in\NN}\in L^q(\NN)$ with the linear map
	\[\{c_n\}_{n\in\NN}\mapsto\sum_{n\in\NN}c_nd_n.\]
	Indeed, this converges by H\"older's inequality: we have
	\[\sum_{n\in\NN}|c_nd_n|\le\left(\sum_{n\in\NN}|c_n|^p\right)^{1/p}\left(\sum_{n\in\NN}|c_n|^q\right)^{1/q}<\infty.\]
	And of course these functions are linear. We will omit the proof that this mapping is bijective because I fear it is nontrivial, and I am lazy.
\end{proof}

\subsection{Duality for General Rings}
Now we turn to generalizing duality from vector spaces because we're algebraists.
\begin{definition}[Duality for free and projective modules]
	Fix $M$ an $R$-module over a commutative ring $R,$ and we can define the \textit{dual module} $F^*:=\op{Hom}_R(F,R).$
\end{definition}
\begin{remark}
	We use commutative rings because noncommutative rings make Professor Borcherds nervous.
\end{remark}
Note that this definition makes sense because $\op{Hom}_R(F,R)$ is an $R$-module by
\[(r\varphi)(x)=r\cdot\varphi(x)\]
for $r\in R$ and $x\in F.$

Some of the theory for vector spaces carries over nicely.
\begin{proposition}
	Given a free $R$-module $F$ over a commutative ring, we have a canonical injection $F\into F^{**}.$ If $F$ is of finite rank, then $F\cong F^{**}.$
\end{proposition}
\begin{proof}
	This is the same as for vector spaces. For $m\in F$ and $\varphi\in F*,$ the mapping is
	\[\psi_\bullet:m\mapsto(\varphi\mapsto\varphi m).\]
	We see $\psi_\bullet$ is linear because, given $r_1,r_2\in R$ and $m_1,m_2\in F,$ we have
	\[\psi_{r_1m_1+r_2m_2}(\varphi)=\varphi(r_1m_1+r_2m_2)=r_1\varphi(m_1)+r_2\varphi(m_2)=r_1\psi_{m_1}(\varphi)+r_2\psi_{m_2}(\varphi)\]
	by plugging into the various module actions.
	
	We can also check that this is injective: suppose that $m\in\ker\psi_\bullet$ so that $\psi_m:F^*\to R$ is the zero mapping. The key point is that $F$ being free promises it is freely generated by some set $\{m_\alpha\}_{\alpha\in\lambda},$ and we note that
	\[\pi_\alpha:\sum_{\alpha\in\lambda}r_\alpha m_\alpha\mapsto r_\alpha\]
	is a linear transformation, well-defined because the $\{m_\alpha\}_{\alpha\in\lambda}$ are a basis. Then we note that $\psi_m(\pi_\alpha)=\pi_\alpha=m,$ so each component of $m$ under the basis will vanish. Thus, $m=0.$

	To show that $F\cong F^{**}$ when $F$ is of finite rank, we actually show that $F\cong F^*,$ non-canonically. Indeed, letting our basis be $\{m_k\}_{k=1}^n$ where $d=\op{rank}F,$ we see we have an isomorphism $R^n\cong F^*$ by
	\[(a_1,\ldots,a_n)\mapsto\left(\sum_{k=1}^na_km_k\mapsto\sum_{k=1}^na_kr_k\right),\]
	where $(a_1,\ldots,a_n)\in R^n$ and $\sum_{k=1}^na_km_k$ is an arbitrary element of $F.$ We won't bother showing that this is an isomorphism.
\end{proof}
The following also holds more generally.
\begin{lemma} \label{lem:cohompreservecoprod}
	Given $R$-modules $A,B,X$ we have that
	\[\op{Hom}_R(A\oplus B,X)\cong\op{Hom}_R(A,X)\oplus\op{Hom}_R(B,X).\]
\end{lemma}
\begin{proof}
	This is essentially the universal property of $A\oplus B$: maps $A\oplus B\to X$ are in bijection with maps $A\to X$ and $B\to X.$ We won't check that this is an $R$-module homomorphism and so on.
\end{proof}
The point of \autoref{lem:cohompreservecoprod} is to give the following.
\begin{proposition}
	Given a projective $R$-module $P$ over a commutative ring, we have a canonical, injective morphism $P\into P^{**}.$ If $P$ is finitely generated, this is an isomorphism.
\end{proposition}
\begin{proof}
	The point is that there is an $R$-module $Q$ such that $F:=P\oplus Q$ is free, and if $P$ is finitely generated, we can force $F$ to be finitely generated. To start, our canonical injective homomorphism is $\psi^P_\bullet:P\to P^{**}$ defined by
	\[\psi^P_\bullet:p\mapsto(\varphi\mapsto\varphi p),\]
	and we define $\psi^Q_\bullet$; as usual, we won't check that this is an $R$-module homomorphism and so on.
	
	We start by showing $\psi^P_\bullet$ is injective and that $\psi^Q_\bullet$ is injective will be similar. Indeed, if $p\in\ker\psi^P_\bullet,$ then $\psi^P_p$ is the zero map so that $\varphi(p)=0$ for each $\varphi:P\to R.$ But now, for each $\overline{\varphi}:F\to R,$ we see $\overline\varphi=\varphi^P+\varphi^Q$ for $\varphi^P:P\to R$ and $\varphi^Q\to R$ by universal property, so
	\[\overline\varphi(p,0)=\varphi^P(p)+\varphi^Q(0)=0+0=0\]
	for each $\overline\varphi:F\to R.$ But we know that the only element of $F$ which vanishes under all morphisms $F\to R$ is $(0,0),$ so we must have $p=0.$ This finishes.
	
	To show that $\psi^P_\bullet:P\into P^{**}$ is an isomorphism when $P$ is finitely generated, we bound the size of $P^{**}.$ Indeed, we note that we have the isomorphisms
	\begin{align*}
		\op{Hom}_R(\op{Hom}_R(P\oplus Q,R),R) &\cong \op{Hom}_R(\op{Hom}_R(P,R)\oplus\op{Hom}_R(Q,R),R) \\
		&\cong\op{Hom}_R(\op{Hom}_R(P,R),R)\oplus\op{Hom}_R(\op{Hom}_R(Q,R),R)
	\end{align*}
	by repeatedly applying \autoref{lem:cohompreservecoprod}. But now we have maps
	\[F=P\oplus Q\into P^{**}\oplus Q^{**}\cong F^{**}\cong F,\]
	where all maps are injective, and in fact the composition is $\id_F$ if we track everything through.\footnote{Please don't ask me to actually track this through.} It follows that our map $P\into P^{**}$ and $Q\into Q^{**}$ are actually isomorphisms.
\end{proof}
\begin{remark}
	It is not necessarily true that $P\cong P^*$ when $P$ is a finitely generated projective module; there is an example \href{https://math.stackexchange.com/a/1343962/869257}{here}.
\end{remark}
Anyways, we should see some examples.
\begin{example}
	For $R=\ZZ$ and $M=\ZZ/2\ZZ,$ we have $M^*=\op{Hom}(\ZZ/2\ZZ,\ZZ)=0,$ which is not very interesting.
\end{example}
This is not interesting because we've immediately killed our module. So here is another definition which works better for abelian groups.
\begin{definition}[Duality for abelian groups]
	For $M$ an abelian group, define
	\[M^*:=\op{Hom}_\ZZ(M,\QQ/\ZZ).\]
\end{definition}
\begin{remark}
	We have chosen $\QQ/\ZZ$ for our ``dualizing object'' instead of $\ZZ$ because we can do what we want. In specific cases, there might be good reasons to choose a different dualizing object than our original ring.
\end{remark}
We can check the following, continuing the idea that double duals should behave well.
\begin{proposition}
	For $M$ a finitely generated abelian group. Then $M^{**}\cong M.$
\end{proposition}
\begin{proof}
	The proof proceeds in steps.
	\begin{enumerate}
		\item We start by checking cyclic groups. We see
		\[\op{Hom}_\ZZ(\ZZ/n\ZZ,\QQ/\ZZ)\cong\ZZ/n\ZZ\]
		because an element of $\ZZ/n\ZZ$ must map to an element of (additive) order $n,$ so $1\mapsto\frac kn$ for some $k\in\ZZ/n\ZZ.$ So our maps are in bijection to $\ZZ/n\ZZ,$ and it is not too hard to check that the map $\ZZ/n\ZZ\to\op{Hom}(\ZZ/n\ZZ,\QQ/\ZZ)$ is in fact homomorphic.

		Thus, $(\ZZ/n\ZZ)^*\cong\ZZ/n\ZZ.$
		\item We note that \autoref{lem:cohompreservecoprod} implies that $(M\oplus N)^{**}\cong M^{**}\oplus N^{**}$ as before. In particular, if we know $M^{**}\cong M$ and $N^{**}\cong N$ already, then we get $(M\oplus N)^{**}\cong M^{**}\oplus N^{**}.$
		\item To finish, any finitely generated abelian group $M$ is the direct sum of some cyclic group, so we finish by applying 2 and then 1.
		\qedhere
	\end{enumerate}
\end{proof}
Even though the above example worked so nicely, we lost things being canonical. Namely, the isomorphism $\op{Hom}_\ZZ(\ZZ/n\ZZ,\QQ/\ZZ)\cong\ZZ/n\ZZ$ was not canonical because we had to choose the generator $1\in\ZZ/n\ZZ.$ Regardless, we do still have the canonical isomorphism $M\cong M^{**},$ where the homomorphism is canonical and bijective for size reasons.

\subsection{Fourier Analysis}
Let's do some more examples. In what follows, we use $S^1$ as the dualizing object for our abelian groups instead of $\QQ/\ZZ$; to make the distinction clear, we have the following definition.
\begin{defi}[Character]
	Given an abelian group $M,$ $\chi$ is a \textit{character} if and only if $\chi\in\op{Hom}(M,S^1).$
\end{defi}
If the abelian group $G$ was finite to begin with, then this is the same as $G^*$ from earlier because each $g\in G$ must map into
\[e^{2\pi ik/\#G}\text{ for some }k\in\ZZ,\]
which is in the image of $\QQ/\ZZ$ in $S^1.$ If our abelian group $G$ was infinite to begin with, then it likely has some topology going on, so it still makes sense to use $S^1$ instead of $\QQ/\ZZ.$
% \begin{remark}
% 	I am not sure we ever worked with $\QQ/\ZZ$ at all. The only good reason to bring this abelian group up is that it matters later for the story on injective modules.
% \end{remark}
\begin{example}
	For $\left(\ZZ/8\ZZ\right)^\times,$ we see that $(\ZZ/8\ZZ)^\times=\langle3,5\rangle\cong(\ZZ/2\ZZ)^2.$ So we can write out our characters explicitly by tracking where $3$ and $5$ go.
	\[\begin{array}{c|cccc}
		 & 1 & 3 & 5 & 7 \\\hline
		\chi_0 & 1 & 1 & 1 & 1 \\
		\chi_1 & 1 & -1 & 1 & -1 \\
		\chi_2 & 1 & -1 & -1 & 1 \\
		\chi_3 & 1 & 1 & -1 & -1 \\
	\end{array}\]
\end{example}
\begin{remark}
	Dirichlet's original use of Dirichlet characters $\chi:(\ZZ/m\ZZ)^\times\to S^1$ was to work with $L$-series of the form
	\[L(\chi,s)=\sum_{n=1}^\infty\frac{\chi(n)}{n^2},\]
	where $\chi(n)=0$ when $\gcd(m,n)>1.$ If you are interested in why he cared about such things, read up on analytic number theory.
\end{remark}

These ideas give us some notion of a Fourier transform. The following is our main result here.
\begin{theorem}
	Fix $M$ some finite abelian group. Then we can define the (Hermitian) inner product on functions $f,g:M\to\CC$ by
	\[\langle f,g\rangle:=\sum_{x\in M}f(x)\overline{g(x)}.\]
	Then the characters in $\op{Hom}_\ZZ(M,S^1)$ form an orthogonal basis of these functions in $\op{Mor}(M,\CC).$
\end{theorem}
The idea here is to generalize Fourier series, where we have as our specific case the abelian group $\RR/\ZZ,$ and we have
\[f(x)=\sum_{k\in\ZZ}c_ke^{2\pi ikx}\]
where
\[c_k:=\int f(x)e^{-2\pi inx}\,dx.\]
\begin{proof}
	We show the claims in reverse order.
	\begin{itemize}
		\item We start by showing that distinct characters are orthogonal. Indeed, pick up $\chi_1,\chi_2:\op{Hom}(M,S^1).$ Then we have
		\[\sum_{x\in M}\chi_1(x)\overline{\chi_2}(x)=\sum_{x\in M}(\chi_1\chi_2^{-1})(x).\]
		If $\chi_1=\chi_2,$ then all entries are the trivial character $\chi_0\equiv1,$ so we get out $\#M$ from the sum. On the other hand, we claim that
		\[\sum_{x\in M}\chi(x)=0\]
		where $\chi$ is not the trivial character. Explicitly, take some $y\in M$ such that $\chi(y)\ne1.$ Then
		\[\chi(y)\sum_{x\in M}\chi(x)=\sum_{x\in M}\chi(xy)=\sum_{xy\in M}\chi(xy).\]
		So we are forced to conclude that
		\[(1-\chi(y))\sum_{x\in M}\chi(x)=0,\]
		so $\sum_{x\in M}\chi(x)=0.$
	
		So in total, we find that
		\[\langle\chi_1,\chi_2\rangle=\sum_{x\in M}\chi_1(x)\overline{\chi_2(x)}=\begin{cases}
			\#M & \chi_1=\chi_2, \\
			0 & \chi_1\ne\chi_2.
		\end{cases}\]
		In particular, distinct characters are indeed orthogonal.
		\item To show that characters span $\op{Mor}(M,\CC),$ fix any $f\in\op{Mor}(M,\CC),$ and we claim
		\[f\stackrel?=\frac1{\#M}\sum_{\chi\in\op{Hom}(M,S^1)}\langle f,\chi\rangle\chi.\]
		Expand this out, we are interested in evaluating, for some $x\in M,$
		\[\sum_{\chi\in\op{Hom}(M,S^1)}\langle f,\chi\rangle\chi(x)=\sum_\chi\left(\sum_{y\in M}f(y)\overline\chi(y)\right)\chi(x)=\sum_{y\in M}f(y)\sum_\chi\chi(xy^{-1}).\]
		We claim that all terms except $x=y$ vanish over the first sum, where the sum reads $\sum_\chi\chi(e)=\#M.$ Well, if $x\ne y,$ fix $z:=xy^{-1}\ne e$ so that we want to evaluate
		\[\sum_\chi\chi(z).\]
		The main point is that there is some character $\chi_1$ such that $\chi(z)\ne1$ because $z\ne e.$\footnote{This is surprisingly technical. One way to do this is to decompose $M\cong\bigoplus_{k=1}^N\ZZ/n_k\ZZ,$ find some coordinate $\ZZ/n_\bullet\ZZ$ where $z$ is nonzero, and then send $1\in\ZZ/n_\bullet\ZZ\mapsto e^{2\pi i/n_\bullet}$ while the other coordinates are sent to $1.$} Then we see that
		\[\chi_1(x)\sum_\chi\chi(z)=\sum_\chi(\chi_1\chi)(z)=\sum_\chi\chi(z),\]
		so $\chi_1(x)\ne1$ forces $\sum_\chi\chi(z)=0.$
		
		To finish, we see that
		\[\sum_{\chi\in\op{Hom}(M,S^1)}\langle f,\chi\rangle\chi(x)=\sum_{y\in M}f(y)\sum_\chi\chi(xy^{-1})=f(x)\#M,\]
		which is what we wanted.
		\qedhere
	\end{itemize}
\end{proof}
The above also holds in some form in more generality for locally compact abelian groups.
\begin{example}
	Consider the following two character duals. (We won't prove these in detail because they would take us too far afield.)
	\begin{itemize}
		\item For $G=\ZZ,$ we have $\op{Hom}(\ZZ,S^1)\cong S^1$ by tracking where $1$ goes.
		\item For $G=S^1,$ we have $\op{Hom}(S^1,S^1)\cong\ZZ$ because all such homomorphism take the form $z\mapsto z^n$ for $n\in\NN.$
	\end{itemize}
	These together give us the theory of Fourier series.
\end{example}
\begin{example}
	For $G=\RR,$ we have $\op{Hom}(\RR,S^1)\cong\RR$ by $y\mapsto(x\mapsto e^{2\pi ixy}).$ The above is the theory for the Fourier transform.
\end{example}
\begin{example}
	For $G=\QQ_p,$ we still have $\op{Hom}(\QQ_p,S^1)\cong\QQ_p.$
\end{example}
\begin{remark}[Nir]
	It is a remarkable fact that $\op{Hom}(\QQ_\nu,S^1)\cong\QQ_\nu,$ even though non-canonically. The correct theory here turns out to be the fact that the $\QQ_\nu$ are ``local fields.''
\end{remark}
\begin{remark}
	A lot of number theory has to do with Fourier analysis on things like $\QQ_p$ or $\AA_\QQ.$
\end{remark}

\subsection{Injective Modules for Abelian Groups}
Injective modules are roughly dual to projective modules, where duality does not mean what we have been talking about so far. So we take the definition of projective and reverse the arrows. Here is the definition of projective.
\begin{definition}[Projective]
	A module $M$ is \textit{projective} if and only if each surjection $B\onto C$ with a map $\varphi:M\to C,$ then we have a lifting map $\overline\varphi:M\to B$ making the diagram commute.
	% https://q.uiver.app/?q=WzAsNCxbMCwxLCJCIl0sWzEsMSwiQyJdLFsyLDEsIjAiXSxbMCwwLCJNIl0sWzAsMSwiIiwwLHsic3R5bGUiOnsiaGVhZCI6eyJuYW1lIjoiZXBpIn19fV0sWzEsMl0sWzMsMSwiXFx2YXJwaGkiXSxbMywwLCJcXG92ZXJsaW5lXFx2YXJwaGkiLDIseyJzdHlsZSI6eyJib2R5Ijp7Im5hbWUiOiJkYXNoZWQifX19XV0=
	\[\begin{tikzcd}
		M \\
		B & C & 0
		\arrow[two heads, from=2-1, to=2-2]
		\arrow[from=2-2, to=2-3]
		\arrow["\varphi", from=1-1, to=2-2]
		\arrow["\overline\varphi"', dashed, from=1-1, to=2-1]
	\end{tikzcd}\]
\end{definition}
And now we reverse the arrows.
\begin{definition}[Injective]
	A module $M$ is \textit{injective} if and only if each injection $B\into C$ with a map $M\to B\to M,$ then we have a lifting map $C\to M$ making the diagram commute.
	% https://q.uiver.app/?q=WzAsNCxbMCwwLCIwIl0sWzEsMCwiQiJdLFsyLDAsIkMiXSxbMiwxLCJNIl0sWzEsMiwiIiwwLHsic3R5bGUiOnsidGFpbCI6eyJuYW1lIjoiaG9vayIsInNpZGUiOiJ0b3AifX19XSxbMiwzLCJcXG92ZXJsaW5lXFx2YXJwaGkiLDAseyJzdHlsZSI6eyJib2R5Ijp7Im5hbWUiOiJkYXNoZWQifX19XSxbMSwzLCJcXHZhcnBoaSIsMl0sWzAsMV1d
	\[\begin{tikzcd}
		0 & B & C \\
		&& M
		\arrow[hook, from=1-2, to=1-3]
		\arrow["\overline\varphi", dashed, from=1-3, to=2-3]
		\arrow["\varphi"', from=1-2, to=2-3]
		\arrow[from=1-1, to=1-2]
	\end{tikzcd}\]
\end{definition}
\begin{remark}[Nir]
	Here is one reason why injective modules are nice; fix $I$ an injective module. Dual to projective modules, any short exact sequence
	\[0\to I\stackrel\iota\to B\stackrel\pi\to C\to 0\]
	will split. The way to see this is that the diagram
	\[\begin{tikzcd}
		0 & I & B \\
		&& I
		\arrow["\iota",hook, from=1-2, to=1-3]
		\arrow["\rho", dashed, from=1-3, to=2-3]
		\arrow["\id_I"', from=1-2, to=2-3]
		\arrow[from=1-1, to=1-2]
	\end{tikzcd}\]
	gives us some $\rho:B\to I$ such that $\rho\circ\iota=\id_I.$ This $\rho$ can be used to induce an isomorphism $B\cong I\oplus C$ by $b\mapsto(\rho b,\pi b)$; we won't actually check that this is an isomorphism here.
\end{remark}
At a high level, being injective means that each homomorphism from a submodule to an injective module extends to a homomorphism from the full module. Let's try to find some injective modules.
\begin{nex}
	We have that $\ZZ$ is not injective because, for $\ZZ\subseteq\frac12\ZZ,$ we cannot extend $\id_\ZZ:\ZZ\to\ZZ$ to a full map $\frac12\ZZ\to\ZZ$ because $\ZZ$ has no element which squares to $1.$ Here is the diagram.
	% https://q.uiver.app/?q=WzAsNCxbMCwwLCIwIl0sWzEsMCwiXFxtYXRoYmIgWiJdLFsyLDAsIlxcZnJhYzEyXFxtYXRoYmIgWiJdLFsyLDEsIlxcbWF0aGJiIFoiXSxbMSwyLCIiLDAseyJzdHlsZSI6eyJ0YWlsIjp7Im5hbWUiOiJob29rIiwic2lkZSI6InRvcCJ9fX1dLFsyLDMsIlxcb3ZlcmxpbmVcXHZhcnBoaSIsMCx7InN0eWxlIjp7ImJvZHkiOnsibmFtZSI6ImRhc2hlZCJ9fX1dLFsxLDMsIlxcaWRfXFxaWiIsMl0sWzAsMV1d
	\[\begin{tikzcd}
		0 & {\mathbb Z} & {\frac12\mathbb Z} \\
		&& {\mathbb Z}
		\arrow[hook, from=1-2, to=1-3]
		\arrow[dashed, from=1-3, to=2-3]
		\arrow["{\id_\ZZ}"', from=1-2, to=2-3]
		\arrow[from=1-1, to=1-2]
	\end{tikzcd}\]
\end{nex}
\begin{nex}
	We have that $\ZZ/2\ZZ$ is not injective because, for $\ZZ/2\ZZ\into\ZZ/4\ZZ,$ we cannot extend $\id_{\ZZ/2\ZZ}:\ZZ/2\ZZ\to\ZZ/2\ZZ.$ The problem is that no map $\ZZ/4\ZZ\to\ZZ/2\ZZ$ sends $2\to1.$ Here is the diagram.
	% https://q.uiver.app/?q=WzAsNCxbMCwwLCIwIl0sWzEsMCwiXFxtYXRoYmIgWi8yXFxaWiJdLFsyLDAsIlxcWlovNFxcWloiXSxbMiwxLCJcXFpaLzJcXFpaIl0sWzEsMiwiIiwwLHsic3R5bGUiOnsidGFpbCI6eyJuYW1lIjoiaG9vayIsInNpZGUiOiJ0b3AifX19XSxbMiwzLCJcXG92ZXJsaW5lXFx2YXJwaGkiLDAseyJzdHlsZSI6eyJib2R5Ijp7Im5hbWUiOiJkYXNoZWQifX19XSxbMSwzLCJcXGlkX1xcWloiLDJdLFswLDFdXQ==
	\[\begin{tikzcd}
		0 & {\mathbb Z/2\ZZ} & {\ZZ/4\ZZ} \\
		&& {\ZZ/2\ZZ}
		\arrow[hook, from=1-2, to=1-3]
		\arrow[dashed, from=1-3, to=2-3]
		\arrow["{\id_{\ZZ/2\ZZ}}"', from=1-2, to=2-3]
		\arrow[from=1-1, to=1-2]
	\end{tikzcd}\]
\end{nex}
\begin{nex}
	More generally, no nonzero finite abelian group $G$ is injective. For example, we can use any $g\in G\setminus\{0\}$ to fix a map $\ZZ\to G$ by $1\mapsto g,$ but then this cannot be extended to $\frac1{\#G}\ZZ$ because every element $h\in G$ has $\#G\cdot h=e\ne g.$
	% https://q.uiver.app/?q=WzAsNCxbMCwwLCIwIl0sWzEsMCwiXFxtYXRoYmIgWi8yXFxaWiJdLFsyLDAsIlxcWlovNFxcWloiXSxbMiwxLCJcXFpaLzJcXFpaIl0sWzEsMiwiIiwwLHsic3R5bGUiOnsidGFpbCI6eyJuYW1lIjoiaG9vayIsInNpZGUiOiJ0b3AifX19XSxbMiwzLCJcXG92ZXJsaW5lXFx2YXJwaGkiLDAseyJzdHlsZSI6eyJib2R5Ijp7Im5hbWUiOiJkYXNoZWQifX19XSxbMSwzLCJcXGlkX1xcWloiLDJdLFswLDFdXQ==
	\[\begin{tikzcd}
		0 & {\ZZ} & {\frac1{\#G}\ZZ} \\
		&& {G}
		\arrow[hook, from=1-2, to=1-3]
		\arrow[dashed, from=1-3, to=2-3]
		\arrow["(1\mapsto g)"', from=1-2, to=2-3]
		\arrow[from=1-1, to=1-2]
	\end{tikzcd}\]
\end{nex}
\begin{example}
	The group $\QQ$ is injective. This will be true because $\QQ$ is ``divisible.''
\end{example}
Roughly speaking, the problem with $\ZZ$ being injective is that we could not ``divide by $2,$'' but $\QQ$ has no such problems. So we have the following property.
\begin{definition}
	Fix $G$ an abelian group. Then $G$ is \textit{divisible} if and only if the map $x\mapsto nx$ for any $n\in\ZZ^+$ is surjective.
\end{definition}
\begin{proposition}
	We have that $M$ an injective abelian group if and only if $M$ is divisible.
\end{proposition}
\begin{proof}
	We show this in two parts.
	\begin{itemize}
		\item Take $M$ injective and $n\in\ZZ^+.$ Fixing any $m\in M,$ we need to show that there is $x\in M$ with $m=nx.$ The key point is the following diagram, well-defined because $n>0.$ Set $\varphi:\ZZ\to M$ by $1\mapsto m.$
		% https://q.uiver.app/?q=WzAsNCxbMCwwLCIwIl0sWzEsMCwiXFxaWiJdLFsyLDAsIlxcZnJhYzFuXFxaWiJdLFsyLDEsIk0iXSxbMSwyLCIiLDAseyJzdHlsZSI6eyJ0YWlsIjp7Im5hbWUiOiJob29rIiwic2lkZSI6InRvcCJ9fX1dLFsyLDMsIlxcb3ZlcmxpbmVcXHZhcnBoaSIsMCx7InN0eWxlIjp7ImJvZHkiOnsibmFtZSI6ImRhc2hlZCJ9fX1dLFsxLDMsIlxcdmFycGhpIiwyXSxbMCwxXV0=
		\[\begin{tikzcd}
			0 & \ZZ & \frac1n\ZZ \\
			&& M
			\arrow[hook, from=1-2, to=1-3]
			\arrow["\overline\varphi", dashed, from=1-3, to=2-3]
			\arrow["\varphi"', from=1-2, to=2-3]
			\arrow[from=1-1, to=1-2]
		\end{tikzcd}\]
		Now injectivity induces $\overline\varphi:\frac1n\ZZ\to M$ such that $1\mapsto m.$ But then $n\cdot\overline\varphi\left(\frac1n\right)=\overline\varphi(1)=m,$ so $\overline\varphi\left(\frac1n\right)$ is the desired element of $M.$
		\item Take $M$ divisible, and we want to show $M$ is injective. We are given an injection $C\into B$ with a map $\varphi:C\to M$ which we want to extend to  a map $\overline\varphi:B\to M.$ Well, take any $b\in B.$ We have two cases.
		\begin{enumerate}[label=(\alph*)]
			\item If $nb\notin\CC$ for all $n\in\ZZ\setminus\{0\},$ then $\langle C,b\rangle\cong C\oplus\ZZ,$ so we can send $f:b\mapsto0.$
			\item Otherwise, suppose $nb\in C$ where $n$ is the least such positive integer. But now $M$ is divisible, so $f(nb)=ny$ for some $y\in M,$ so we can send $f:b\mapsto y.$
		\end{enumerate}
		Now we invoke the Axiom of choice (specifically Zorn's lemma) to extend this all the way up to $B.$
		\qedhere
	\end{itemize}
\end{proof}
It turns out that divisible implies injective even for principal ideal domains, but we won't show this.
\begin{remark}
	Roughly speaking, it is harder to find injective modules than projective modules. Namely, we needed the axiom of choice for the above proof.
\end{remark}
Anyways, let's see some more examples.
\begin{ex}
	We have that $\QQ/\ZZ$ is injective because it is divisible: for any $q\in\QQ$ and $n\in\ZZ^+,$ we have $\frac qn\stackrel{\times n}q.$ In fact, we can split up
	\[\QQ/\ZZ\cong\bigoplus_p\underbrace{\left\{x\in\QQ/\ZZ:p^kx=0\text{ for some }k\in\ZZ\right\}}_{M_p:=},\]
	which is roughly the Chinese remainder theorem. (We noted this many lectures ago.) It happens that each $M_p$ is also injective, again because they are divisible. Namely, for any $M_p\in\QQ$ we want to hit and $n\in\ZZ^+,$ we can decompose $n=p^\nu m$ where $p\nmid m.$ Then there is $m'$ so that $mm'\equiv1\pmod{p^\nu}$ so that $n\cdot\frac{m'}{p^\nu}=1$ and $n\cdot\left(\frac{m'}{p^\nu}\cdot q\right)=q.$
\end{ex}

\subsection{Injective Modules for General Rings}
So let's find injective modules for more general rings. We claim the following.
\begin{proposition} \label{prop:injecexist}
	Given a ring $R,$ we have that $\ZZ$-module $R^*:=\op{Hom}_\ZZ(R,I)$ is an injective $R$-module, where $I$ is a divisible abelian group.
\end{proposition}
\begin{proof}
	The point is that being injective requires control of $\op{Hom}_R(M,R^*).$ This is somewhat confusing because $R^*$ is itself a $\op{Hom}$-set.

	Regardless, $R^*$ is in fact an $R$-module with $R$-action defined by
	\[(r\varphi)(q):=\varphi(qr).\]
	Here multiplication is on the right because we want $\big((r_1r_2)\varphi\big)(q)=\varphi(qr_1r_2)$ to be equal to $\big(r_1(r_2\varphi)\big)(q)=(r_2\varphi)(qr_1)=\varphi(qr_1r_2).$ We will repeat this because it is confusing.
	\begin{warn}
		If $M$ is a right $R$-module and $G$ is an abelian group, then $\op{Hom}_\ZZ(M,G)$ is a left $R$-module, and conversely.
	\end{warn}
	
	The main claim is that, for any fixed $R$-module $M,$
	\[\op{Hom}_R(M,\op{Hom}_\ZZ(R,I))\cong\op{Hom}_\ZZ(M,I).\]
	Indeed, we take $\psi_\bullet\in\op{Hom}_R(M,\op{Hom}_\ZZ(R,I))$ to $\psi_\bullet(1)\in\op{Hom}_\ZZ(M,I).$ We have the following checks.
	\begin{itemize}
		\item Well-defined: note that $\psi_{m_1+m_2}(1)=\psi_{m_1}(1)+\psi_{m_2}(1),$ so $\psi_\bullet\in\op{Hom}_\ZZ(M,I).$
		\item Homomorphic: given $\psi^1_\bullet,\psi^2_\bullet\in\op{Hom}_R(M,\op{Hom}_\ZZ(I)),$ we see that each $m\in M$ has
		\[(\psi^1_\bullet+\psi^2_\bullet)(m)(1)=\psi^1_m(1)+\psi^2_m(1)=(\psi^1_\bullet(1)+\psi^2_\bullet(1))(m),\]
		where the left-hand side has addition in $\op{Hom}_R(M,\op{Hom}_\ZZ(R,I)),$ and the right-hand side has addition in $\op{Hom}_\ZZ(M,I).$
		\item Injective: we show trivial kernel. Suppose that $\psi_\bullet\in\op{Hom}_R(M,\op{Hom}_\ZZ(R,I))$ has $\psi_m(1)=0$ for each $m\in M.$ Then, for each $r\in R,$ the fact that $\psi_m$ is an $R$-module homomorphism forces
		\[\psi_m(r)=r\cdot\psi_m(1)=r\cdot0=0,\]
		so in fact $\psi_\bullet$ always return the zero map, so it is the zero object in $\op{Hom}_R(M,\op{Hom}_\ZZ(R,I)).$
		\item Surjective: fix $\varphi\in\op{Hom}_\ZZ(M,I),$ and we define
		\[\psi_m(r):=\varphi(rm).\]
		Note that this does have $\psi_m(1)=\varphi(m)$ for each $m\in M,$ so $\psi_\bullet$ will go to $\varphi,$ upon checking that $\psi_\bullet\in\op{Hom}_R(M,\op{Hom}_\ZZ(R,I)).$ Indeed, fixing any $m\in M$ and $r_1,r_2\in R,$ we see
		\[\psi_m(r_1+r_2)=\varphi((r_1+r_2)m)=\varphi(r_1m+r_2m)+\varphi(r_1m)+\varphi(r_2m)=\psi_m(r_1)+\psi_m(r_2),\]
		so indeed, $\psi_m\in\op{Hom}_\ZZ(R,I).$ Then for $r,r_1,r_2\in R$ and $m_1,m_2\in M,$ we see
		\[\psi_{r_1m_1+r_2m_2}(r)=\varphi(rr_1m_1+rr_2m_2)=\varphi(rr_1m_1)+\varphi(rr_2m_2)=\psi_{m_1}(rr_1)+\psi_{m_2}(rr_2).\]
		Now the key point is that the definition of the $R$-action on $\op{Hom}_\ZZ(R,I)$ makes this equal to $(r_1\psi_{m_1}+r_2\psi_{m_2})(r),$ which is what we wanted.
	\end{itemize}
	We now show that $R^*=\op{Hom}_\ZZ(R,I)$ is actually injective. Fix some injection $B\into C$ and map $\varphi_\bullet:B\to R^*$ so that we want to get a map $\overline\varphi_\bullet:C\to R^*.$ Well, we can take $\varphi\in\op{Hom}_R(B,\op{Hom}_\ZZ(R,I))$ so that $\varphi_\bullet(1)\in\op{Hom}(B,I)$ as above.

	Now, the injectivity of $I$ promises some map $\overline\varphi:C\to I$ such that $\overline\varphi(b)=\varphi_b(1)$ for each $b\in B.$ Then we can take $\overline\varphi$ up to the map
	\[\overline\varphi_\bullet\in\op{Hom}(C,R^*)\]
	satisfying $\overline\varphi_c(1)=\overline\varphi(c).$ We claim this $\overline\varphi_\bullet$ is the map we want. Indeed, for each $b\in B$ and $r\in R,$ we see that
	\[\overline\varphi_b(r)=(r\overline\varphi_b)(1)=\overline\varphi_{rb}(1)=\varphi_{rb}(1)=(r\varphi_b)(1)=\varphi_b(r),\]
	so indeed, $\overline\varphi$ does extend $\varphi.$
\end{proof}
Here is another source of injective modules, if we already have some.
\begin{proposition}
	Suppose $\{I_\alpha\}_{\alpha\in\lambda}$ are injective $R$-modules. Then $\prod_{\alpha\in\lambda}I_\alpha$ is an injective $R$-module.
\end{proposition}
\begin{proof}
	Fix an inclusion of $R$-modules $B\into C$ with a map $\varphi:B\to\prod_{\alpha} I_\alpha.$ Then we have the composite maps
	\[\varphi_\beta:B\to\prod_{\alpha\in\lambda}I_\alpha\stackrel{\pi_\beta}\onto I_\beta\]
	for some fixed $\beta.$ But because $I_\beta$ is injective, we have the extension $\overline{\varphi_\beta}:C\to I_\beta.$ So to finish, we define
	\[\overline\varphi(c):=(\overline{\varphi_\alpha}c)_{\alpha\in\lambda}\in\prod_{\alpha\in\lambda}I_\alpha,\]
	for each $c\in C.$ To see that this works, we note that any $b\in B$ has, for any $\beta\in\lambda,$
	\[\pi_\beta(\overline\varphi(b))=\pi_\beta\left((\overline{\varphi_\alpha}b)_{\alpha\in\lambda}\right)=\overline{\varphi_\beta}(b)=\varphi_\beta(b),\]
	so we conclude that $\overline\varphi(b)=\varphi(b).$ This is what we wanted.
\end{proof}

So we have a reasonable supply of injective modules. Namely, we can show the following.
\begin{proposition}
	Fix $R$ a commutative ring. Then we have ``enough injectives'' in the category of $R$-modules. Explicitly, for any module $M,$ we can find an injective $R$-module $N$ for which there is an injection $M\into N.$
\end{proposition}
\begin{remark}
	This is dual to saying that all $R$-modules $M$ are projected onto from some projective module, which is much easier because all $R$-modules $M$ are projected onto by some free module (e.g., $\bigoplus_{m\in M}Rm$).
\end{remark}
\begin{proof}
	We show this for abelian groups, so fix $M$ an abelian group. The key step is that, given an $m\in M\setminus\{0\},$ we want an injective module $N$ so that $f:M\to N$ has $f(m)\ne0.$ Take $N:=\QQ/\ZZ.$ We have two cases.
	\begin{itemize}
		\item If $m$ has infinite order, then take $f(m)$ to be anything in $N\setminus\{0\},$ and extend this map to $M$ by injectivity. This uses that $\QQ/\ZZ$ is injective.
		\item If $m$ has finite order $n,$ then take $f(m)=\frac1n,$ and extend this map to $M$ by injectivity. This also uses that $\QQ/\ZZ$ has elements of all finite orders.
	\end{itemize}
	Now, for each $m\in M,$ we let the above map be $f_m:M\to\QQ/\ZZ.$ So we can glue these maps together to get
	\[f:M\to\prod_{m\in M}\QQ/\ZZ,\]
	where the map $f$ is injective because it has trivial kernel,\footnote{For each $m\in M\setminus\{0\},$ we see that $f_m(m)\ne0$ by construction, so $m\notin\ker f.$} and the product is injective because products of injective modules is injective. So we have embedded $M$ into an injective module.
\end{proof}
The proof for general rings is similar, so we will not show it here. We will say that the main obstacle is again, for each $m\in M,$ finding some $\psi_\bullet\in\op{Hom}_R(M,\op{Hom}_\ZZ(R,\QQ/\ZZ))$ such that $\psi_m$ is nonzero.

To overcome this obstacle, the discussion above tells us how to find some $\psi\in\op{Hom}_\ZZ(M,\QQ/\ZZ)$ with $\psi(m)\ne0,$ and the discussion in the proof of \autoref{prop:injecexist} shows us how to lift $\psi$ into some $\psi_\bullet\in\op{Hom}(M,\op{Hom}_\ZZ(R,\QQ/\ZZ))$ such that $\psi_m(1)=\psi(m)\ne0.$

The injective module we made is frankly huge, but usually we can find a smaller one, and it turns out there is a ``best'' such injective module. Observe that the same is not true for projective modules.
\begin{example}
	For $M=\ZZ/5\ZZ,$ the free module $\ZZ$ can map $1\mapsto1$ or $1\mapsto2,$ and neither of these appears to be ``best projective module.''
\end{example}
But here is what we have for injective modules.
\begin{definition}[Injective evenlope]
	The smallest injective module containing some $R$-module $M$ is called the \textit{injective envelope}.
\end{definition}
We'll give examples in abelian groups.
\begin{example}
	For $\ZZ,$ the injective envelope is $\QQ.$ Note this is not finitely generated, sadly.
\end{example}
\begin{example}
	For $\ZZ/p^n\ZZ,$ the injective envelope is
	\[M_p=\left\{x\in\QQ/\ZZ:p^kx=0\text{ for some }k\in\ZZ\right\}\]
	from earlier.
\end{example}

\subsection{Modules over Euclidean Domains}
We're running out of time, so let's do something else. The main thing we have to say here is the following.
\begin{theorem} \label{thm:fgmodoverpid}
	Any finitely generated module $M$ over a principal ideal domain $R$ is a direct sum of cyclic modules $R/aR$ for $a\in R.$
\end{theorem}
We are not going to show this because it is somewhat technical. Instead, we provide a quick proof of the following corollary to \autoref{thm:fgmodoverpid}.
\begin{corollary} \label{cor:fgmodovered}
	Any finitely generated module $M$ over a Euclidean domain $R$ is a direct sum of cyclic modules $R/aR$ for $a\in R.$
\end{corollary}
\begin{proof}
	The exact same proof as for $R=\ZZ$ will work here. Namely, all that proof required was the ability to use the division algorithm in $\ZZ,$ so the proof extends to Euclidean domains $R.$
\end{proof}
To see that we don't need the power of all principal ideal domains for applications, we have the following application of \autoref{cor:fgmodovered}.
\begin{theorem}[Jordan normal form]
	Fix $V$ a finite-dimensional $k$-vector space, where $k$ is algebraically closed. Then all linear transformations $T\in\op{End}(V)$ are a direct sum of linear transformations which under a suitable basis look like
	\[\begin{bmatrix}
		\lambda & 1 & 0 & \cdots & 0 & 0 \\
		0 & \lambda & 1 & \cdots & 0 & 0 \\
		0 & 0 & \lambda & \cdots & 0 & 0 \\
		\vdots & \vdots & \vdots & \ddots & \vdots & \vdots \\
		0 & 0 & 0 & \cdots & \lambda & 1 \\
		0 & 0 & 0 & \cdots & 0 & \lambda
	\end{bmatrix}\]
	where $\alpha\in k.$
\end{theorem}
\begin{proof}
	The key trick is to throw \autoref{cor:fgmodovered} at $R:=k[x].$ We see that $R$ is Euclidean by using $\deg.$ Now, by \autoref{cor:fgmodovered}, we have that any $k[x]$-module named $M$ will take the form
	\[\bigoplus_{n=1}^N\frac{k[x]}{(p_n)},\]
	where $p_\bullet\in k[x]$ for each $p_\bullet.$ Because $k$ is algebraically closed, we may take $p_\bullet=(x_\bullet-\lambda_\bullet)^{d_\bullet}.$

	Now fix some $T\in\op{End}(V).$ To get the desired statement, the idea is to view $V$ itself as a $k[x]$-module, where our action is given by
	\[p(x)\cdot v=p(T)v,\]
	where $p(x)\in k[x]$ and $v\in V.$ Essentially, we are taking the typical $k$-action on $V$ and adding in a ``transcendental linear operator $T$'' to get out a $k[x]$-module. Anyways, the point is that we can write
	\[V\cong\bigoplus_{n=1}^N\frac{k[x]}{(x-\lambda_n)^{d_n}}\]
	for some $\lambda_\bullet,d_\bullet\in k.$ For concreteness, we note that we can pull back each $k[x]/(x_n-\lambda_n)^{d_n}$ so that we can decompose
	\[V=\bigoplus_{k=1}^NV_n\qquad\text{such that}\qquad V_n\cong\frac{k[x]}{(x-\lambda_n)^{d_n}}.\]
	So now we see that the action of $T$ on $V$ will decompose nicely into the direct sum of the action of $T$ on the $V_n.$

	In particular, we claim that we can find a basis for which $T$ restricted to $V_n$ looks like
	\[\begin{bmatrix}
		\lambda_n & 1 & 0 & \cdots & 0 & 0 \\
		0 & \lambda_n & 1 & \cdots & 0 & 0 \\
		0 & 0 & \lambda_n & \cdots & 0 & 0 \\
		\vdots & \vdots & \vdots & \ddots & \vdots & \vdots \\
		0 & 0 & 0 & \cdots & \lambda_n & 1 \\
		0 & 0 & 0 & \cdots & 0 & \lambda_n \\
	\end{bmatrix}\]
	Indeed, using $V_n\cong k[x]/(x-\lambda_n)^{d_n},$ we may pull the basis $\left\{(x-\lambda_n)^{d_n-1-e}\right\}_{e=0}^{d_n-1}$ back to $\{b_e\}_{e=0}^{d_n-1}\subseteq V_n$ so that
	\[x\cdot(x-\lambda_n)^e=(x-\lambda_n)^{e+1}+\lambda_n(x-\lambda_n)^e\]
	corresponds to
	\[T\cdot b_e=\lambda_nb_e+b_{e-1}\]
	for $0\le e\le d_n-1,$ where $b_{-1}$ is the pull-back of $(x-\lambda_n)^{d_n}=0,$ which is $0.$ These equations exactly describe the matrix we need, so we are done here.
\end{proof}
We leave with the exercise to describe the Jordan normal form over $\RR$ by using the above proof, where we have to add in the possible irreducible quadratics.