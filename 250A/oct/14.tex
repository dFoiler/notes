% !TEX root = ../notes.tex














There's always more show.

\subsection{Limits and Colimits}
We're talking limits and colimits today. We have the following definitions. Here is the limit.
\begin{definition}[Limit]
	Fix $\mathcal I$ an index category and $F:\mathcal I\to\mathcal C$ a functor. Then the \textit{limit} is an object $L:=\limit_{\mathcal I}F(I)$ with maps $\pi_I:L\to F(I)$ for each $I\in\mathcal I,$ which commute such that, for any $f:I_1\to I_2$ in $\mathcal I,$ we have $\pi_{I_2}=F(f)\circ\pi_{I_1}.$
	
	Further, $L$ is universal with respect to this property: for any object $X$ with maps $\varphi_I:X\to F(I)$ for each $I\in\mathcal I$ (which commute in the same way), then there is a unique induced map $\varphi:X\to L$ making the following diagram commute.
	% https://q.uiver.app/?q=WzAsNCxbMSwxLCJMIl0sWzAsMiwiRihpXzEpIl0sWzIsMiwiRihpXzIpIl0sWzEsMCwiWCJdLFsxLDIsIkYoZikiLDJdLFszLDAsIlxcdmFycGhpIiwxLHsic3R5bGUiOnsiYm9keSI6eyJuYW1lIjoiZGFzaGVkIn19fV0sWzAsMSwiXFxwaV97aV8xfSJdLFswLDIsIlxccGlfe2lfMn0iLDJdLFszLDEsIlxcdmFycGhpX3tpXzF9IiwyLHsiY3VydmUiOjJ9XSxbMywyLCJcXHZhcnBoaV97aV8yfSIsMCx7ImN1cnZlIjotMn1dXQ==
	\[\begin{tikzcd}
		& X \\
		& L \\
		{F(I_1)} && {F(I_2)}
		\arrow["{F(f)}"', from=3-1, to=3-3]
		\arrow["\varphi"{description}, dashed, from=1-2, to=2-2]
		\arrow["{\pi_{I_1}}", from=2-2, to=3-1]
		\arrow["{\pi_{I_2}}"', from=2-2, to=3-3]
		\arrow["{\varphi_{I_1}}"', curve={height=12pt}, from=1-2, to=3-1]
		\arrow["{\varphi_{I_2}}", curve={height=-12pt}, from=1-2, to=3-3]
	\end{tikzcd}\]
\end{definition}
The dual notion of a limit is the colimit.
\begin{definition}[Colimit]
	Fix $\mathcal I$ an index category and $F:\mathcal I\to\mathcal C$ a functor. Then the \textit{colimit} is an object $L:=\colimit_{\mathcal I}F(I)$ with maps $\iota_I:F(I)\to L$ for each $i\in\mathcal I,$ which commute such that, for any $f:I_1\to I_2$ in $\mathcal I,$ we have $\iota_{I_2}\circ F(f)=\iota_{I_1}.$
	
	Further, $L$ is universal with respect to this property: for any object $X$ with maps $\varphi_I:F(I)\to X$ for each $i\in\mathcal I$ (which commute in the same way), then there is a unique induced map $\varphi:L\to X$ making the following diagram commute.
	% https://q.uiver.app/?q=WzAsNCxbMCwwLCJGKGlfMSkiXSxbMiwwLCJGKGlfMikiXSxbMSwxLCJMIl0sWzEsMiwiWCJdLFswLDEsIkYoZikiXSxbMCwyLCJcXGlvdGFfe2lfMX0iXSxbMSwyLCJcXGlvdGFfe2lfMn0iLDJdLFswLDMsIlxcdmFycGhpX3tpXzF9IiwyLHsiY3VydmUiOjJ9XSxbMSwzLCJcXHZhcnBoaV97aV8yfSIsMCx7ImN1cnZlIjotMn1dLFsyLDMsIlxcdmFycGhpIiwxLHsic3R5bGUiOnsiYm9keSI6eyJuYW1lIjoiZGFzaGVkIn19fV1d
	\[\begin{tikzcd}
		{F(I_1)} && {F(I_2)} \\
		& L \\
		& X
		\arrow["{F(f)}", from=1-1, to=1-3]
		\arrow["{\iota_{I_1}}", from=1-1, to=2-2]
		\arrow["{\iota_{I_2}}"', from=1-3, to=2-2]
		\arrow["{\varphi_{I_1}}"', curve={height=12pt}, from=1-1, to=3-2]
		\arrow["{\varphi_{I_2}}", curve={height=-12pt}, from=1-3, to=3-2]
		\arrow["\varphi"{description}, dashed, from=2-2, to=3-2]
	\end{tikzcd}\]
\end{definition}
And here are the standard examples.
\begin{example}
	Fix the discrete category $\mathcal I$ and functor $F:\mathcal I\to\mathcal A$ as follows.
	% https://q.uiver.app/?q=WzAsOCxbMCwwLCJcXGJ1bGxldCJdLFsxLDAsIlxcYnVsbGV0Il0sWzIsMCwiXFxidWxsZXQiXSxbMywwLCJcXGNkb3RzIl0sWzAsMSwiQV8xIl0sWzEsMSwiQV8yIl0sWzIsMSwiQV8zIl0sWzMsMSwiXFxjZG90cyJdLFsxLDUsIkYiLDAseyJsZXZlbCI6Mn1dXQ==
	\[\begin{tikzcd}
		\bullet & \bullet & \bullet & \cdots \\
		{A_1} & {A_2} & {A_3} & \cdots
		\arrow["F", Rightarrow, from=1-2, to=2-2]
	\end{tikzcd}\]
	Then the limit is the direct product (the universal object projecting down into each individual), and the colimit is the direct sum (the universal objecting including each individual).
\end{example}
\begin{example}
	A kernel of a morphism $f:A\to B$ is the limit of the following index category and functor.
	% https://q.uiver.app/?q=WzAsNCxbMCwwLCJcXGJ1bGxldCJdLFsxLDAsIlxcYnVsbGV0Il0sWzAsMSwiQSJdLFsxLDEsIkIiXSxbMCwxLCIiLDAseyJjdXJ2ZSI6LTF9XSxbMCwxLCIiLDIseyJjdXJ2ZSI6MX1dLFsyLDMsImYiLDEseyJjdXJ2ZSI6LTF9XSxbMiwzLCIwIiwxLHsiY3VydmUiOjF9XSxbNSw2LCJGIiwwLHsic2hvcnRlbiI6eyJzb3VyY2UiOjIwLCJ0YXJnZXQiOjIwfX1dXQ==
	\[\begin{tikzcd}
		\bullet & \bullet \\
		A & B
		\arrow[curve={height=-6pt}, from=1-1, to=1-2]
		\arrow[""{name=0, anchor=center, inner sep=0}, curve={height=6pt}, from=1-1, to=1-2]
		\arrow[""{name=1, anchor=center, inner sep=0}, "f"{description}, curve={height=-6pt}, from=2-1, to=2-2]
		\arrow["0"{description}, curve={height=6pt}, from=2-1, to=2-2]
		\arrow["F", shorten <=3pt, shorten >=3pt, Rightarrow, from=0, to=1]
	\end{tikzcd}\]
	Indeed, the kernel is the universal object $\op{Ker}f$ with a map $\iota:\op{Ker}f\to A$ such that $f\circ\iota=0\circ\iota=0.$ The cokernel/quotient object is the colimit of this diagram: it is the universal object $\op{Coker}f$ with a map $\pi:B\to\op{Coker}f$ such that $\pi\circ f=\pi\circ0=0.$
\end{example}
\begin{example}
	Pull-backs/fiber products are the limit of the following index category and functor.
	% https://q.uiver.app/?q=WzAsNixbMCwxLCJcXGJ1bGxldCJdLFsxLDAsIlxcYnVsbGV0Il0sWzEsMSwiXFxidWxsZXQiXSxbMiwxLCJYIl0sWzMsMCwiWSJdLFszLDEsIloiXSxbMCwyXSxbMSwyXSxbMyw1LCJcXHZhcnBoaV9YIl0sWzQsNSwiXFx2YXJwaGlfWSIsMl0sWzcsOSwiRiIsMCx7ImxhYmVsX3Bvc2l0aW9uIjozMCwic2hvcnRlbiI6eyJzb3VyY2UiOjEwLCJ0YXJnZXQiOjUwfX1dXQ==
	\[\begin{tikzcd}
		& \bullet && Y \\
		\bullet & \bullet & X & Z
		\arrow[from=2-1, to=2-2]
		\arrow[""{name=0, anchor=center, inner sep=0}, from=1-2, to=2-2]
		\arrow["{\varphi_X}", from=2-3, to=2-4]
		\arrow[""{name=1, anchor=center, inner sep=0}, "{\varphi_Y}"', from=1-4, to=2-4]
		\arrow["F"{pos=0.35}, shorten <=6pt, shorten >=32pt, Rightarrow, from=0, to=1]
	\end{tikzcd}\]
	We were asked in the homework to show that, in the category of abelian groups,
	\[X\times_ZY=\{(x,y)\in X\times Y:\varphi_Xx=\varphi_Yy\}.\]
	This also holds in $R$-modules, but we will not show it explicitly; roughly speaking, $X\times_ZY$ should consist of pairs of $X\times Y$ which are the ``same'' under $\varphi_X$ and $\varphi_Y.$
\end{example}
\begin{example}
	Push-outs/fiber coproducts are the colimit of the following index category and functor.
	% https://q.uiver.app/?q=WzAsNixbMCwwLCJcXGJ1bGxldCJdLFswLDEsIlxcYnVsbGV0Il0sWzEsMCwiXFxidWxsZXQiXSxbMiwwLCJaIl0sWzIsMSwiWCJdLFszLDAsIlkiXSxbMCwxXSxbMCwyXSxbMyw0LCJcXHZhcnBoaV9YIl0sWzMsNSwiXFx2YXJwaGlfWSIsMl0sWzYsOCwiRiIsMix7ImxhYmVsX3Bvc2l0aW9uIjo2MCwic2hvcnRlbiI6eyJzb3VyY2UiOjUwLCJ0YXJnZXQiOjEwfX1dXQ==
	\[\begin{tikzcd}
		\bullet & \bullet & Z & Y \\
		\bullet && X
		\arrow[""{name=0, anchor=center, inner sep=0}, from=1-1, to=2-1]
		\arrow[from=1-1, to=1-2]
		\arrow[""{name=1, anchor=center, inner sep=0}, "{\varphi_X}", from=1-3, to=2-3]
		\arrow["{\varphi_Y}"', from=1-3, to=1-4]
		\arrow["F"'{pos=0.65}, shorten <=32pt, shorten >=6pt, Rightarrow, from=0, to=1]
	\end{tikzcd}\]
	In commutative rings, this is the tensor product $X\otimes_ZY,$ where $X$ and $Y$ have $Z$-action given by $z\cdot x:=\varphi_X(z)x$ and $z\cdot y:=\varphi_Y(z)y.$ We showed this a few days ago in the case where $Z=\ZZ$ (so that we are looking at the coproduct), and I am too lazy to do it again. (If someone wants me to, yell at me.)
\end{example}

\subsection{Direct Limits}
\begin{warn}
	For this section, I work in slightly more generality than Borcherds did in lecture. Namely, all directed systems should be thought of as $\NN$ under the usual ordering, and all inverse systems should be thought of as $\NN$ under the reverse ordering.
\end{warn}
A special example of a colimit is the ``direct limit.'' We have the following definitions.
\begin{definition}[Directed system]
	Fix $\mathcal I$ a partially ordered set/category where every finite set has an upper bound. Then an \textit{directed system} is a covariant functor $F:\mathcal I\to\mathcal A$ satisfying the commutativity requirements of a functor. Explicitly,
	\begin{itemize}
		\item $I\stackrel f\preceq I$ goes to $\id_{F(I)}:F(I)\to F(I),$
		\item $I\stackrel f\preceq J\stackrel g\preceq K$ implies $F(g\circ f)=F(g)\circ F(f).$
	\end{itemize}
\end{definition}
\begin{definition}[Direct limit]
	A \textit{direct limit} is the colimit of a directed system.
\end{definition}
\begin{warn}
	The condition that $\mathcal I$ gives every finite set an upper bound will help us much later.
\end{warn}
The most common example of such a directed system we might run into is $\NN$ under the usual ordering, which is the category presented as follows.
\[1\to2\to3\to\cdots.\]
Intuitively, if these maps are injective after applying the functor, then we are doing a kind of union along a chain of objects. If they aren't injective, we have to be more careful.

We have already seen an example before.
\begin{exercise}
	We have that $M_p$ is the direct limit of
	% https://q.uiver.app/?q=WzAsNCxbMCwwLCJcXFpaL3BcXFpaIl0sWzEsMCwiXFxaWi9wXjJcXFpaIl0sWzIsMCwiXFxaWi9wXjNcXFpaIl0sWzMsMCwiXFxjZG90cyJdLFswLDEsIiIsMCx7InN0eWxlIjp7InRhaWwiOnsibmFtZSI6Imhvb2siLCJzaWRlIjoidG9wIn19fV0sWzEsMiwiIiwwLHsic3R5bGUiOnsidGFpbCI6eyJuYW1lIjoiaG9vayIsInNpZGUiOiJ0b3AifX19XSxbMiwzLCIiLDAseyJzdHlsZSI6eyJ0YWlsIjp7Im5hbWUiOiJob29rIiwic2lkZSI6InRvcCJ9fX1dXQ==
	\[\begin{tikzcd}
		{\ZZ/p\ZZ} & {\ZZ/p^2\ZZ} & {\ZZ/p^3\ZZ} & \cdots
		\arrow[hook, from=1-1, to=1-2]
		\arrow[hook, from=1-2, to=1-3]
		\arrow[hook, from=1-3, to=1-4]
	\end{tikzcd}\]
\end{exercise}
\begin{proof}
	Indeed, our inclusions are given by $\ZZ/p^k\ZZ\cong\frac1{p^k}\ZZ/\ZZ\into M_p,$ or in other words, $1\mapsto\frac1{p^k}.$ We can check these commute: we need to check the map $f_{k\ell}:\ZZ/p^k\ZZ\to\ZZ/p^\ell\ZZ$ (for $k\le\ell$) does indeed satisfy $\iota_k=\iota_\ell\circ f_{k\ell},$ which is simply
	\[\iota_k(n)=\frac n{p^k}=\frac{np^{\ell-k}}{p^\ell}=\iota_\ell\left(np^{\ell-k}\right)=\iota_\ell(f_{k\ell}n),\]
	as needed.

	We now show the universal property. Fix an object $X$ with maps $\varphi_k:\ZZ/p^k\ZZ\to X$ such that $\varphi_k=\varphi_\ell\circ f_{k\ell}.$ We need to exhibit a unique induced map $\varphi:M_p\to X$ making the following diagram commute.
	% https://q.uiver.app/?q=WzAsNixbMCwwLCJcXFpaL3BcXFpaIl0sWzEsMCwiXFxaWi9wXjJcXFpaIl0sWzIsMCwiXFxaWi9wXjNcXFpaIl0sWzMsMCwiXFxjZG90cyJdLFsxLDEsIk1fcCJdLFsxLDIsIlgiXSxbMCwxXSxbMSwyXSxbMiwzXSxbMCw0LCJcXGlvdGFfMSJdLFsxLDQsIlxcaW90YV8yIl0sWzIsNCwiXFxpb3RhXzMiXSxbNCw1LCJcXHZhcnBoaSIsMSx7InN0eWxlIjp7ImJvZHkiOnsibmFtZSI6ImRhc2hlZCJ9fX1dLFswLDUsIlxcdmFycGhpXzEiLDIseyJjdXJ2ZSI6Mn1dLFsxLDUsIlxcdmFycGhpXzIiLDIseyJjdXJ2ZSI6Mn1dLFsyLDUsIlxcdmFycGhpXzMiLDIseyJjdXJ2ZSI6LTJ9XV0=
	\[\begin{tikzcd}
		{\ZZ/p\ZZ} & {\ZZ/p^2\ZZ} & {\ZZ/p^3\ZZ} & \cdots \\
		& {M_p} \\
		& X
		\arrow[from=1-1, to=1-2]
		\arrow[from=1-2, to=1-3]
		\arrow[from=1-3, to=1-4]
		\arrow["{\iota_1}", from=1-1, to=2-2]
		\arrow["{\iota_2}", from=1-2, to=2-2]
		\arrow["{\iota_3}", from=1-3, to=2-2]
		\arrow["\varphi"{description}, dashed, from=2-2, to=3-2]
		\arrow["{\varphi_1}"', curve={height=12pt}, from=1-1, to=3-2]
		\arrow["{\varphi_2}"', curve={height=12pt}, from=1-2, to=3-2]
		\arrow["{\varphi_3}"', curve={height=-12pt}, from=1-3, to=3-2]
	\end{tikzcd}\]
	We show uniqueness and existence one at a time.
	\begin{itemize}
		\item We show that $\varphi$ is unique. Indeed, fix any $\frac n{p^k}\in M_p.$ Then we have $\frac n{p^k}=\iota_k(n),$ so if the diagram is to commute, we must have
		\[\varphi\left(\frac n{p^k}\right)=(\varphi\circ\iota_k)(n)=\varphi_k(n),\]
		so $\varphi$ is indeed forced.
		\item We now show that $\varphi$ exists. As we worked out, we need to define $\varphi$ by
		\[\varphi\left(\frac n{p^k}\right):=\varphi_k(n).\]
		We see $\varphi$ is well-defined as a function because, even though we might $\frac n{p^k}=\frac{np^{\ell-k}}{p^\ell}$ (where $k\le\ell$ without loss of generality), it is still true that
		\[\varphi_k(n)=\varphi_\ell\left(np^{\ell-k}\right)=\left(\varphi_\ell\circ f_{k\ell}\right)(n),\]
		where $\varphi_k=\varphi_\ell\circ f_{k\ell}$ by hypothesis on the $\varphi_\bullet.$

		We technically have to check that $\varphi$ is a group homomorphism. Well,
		\[\varphi\left(\frac n{p^k}+\frac m{p^\ell}\right)=\varphi\left(\frac{np^\ell+mp^k}{p^{k+\ell}}\right)=\varphi_{k+\ell}\left(np^\ell+mp^k\right),\]
		but now $\varphi_{k+\ell}$ is a group homomorphism, so this reads
		\[\varphi_{k+\ell}\left(np^\ell\right)+\varphi_{k+\ell}\left(mp^k\right)=\varphi_k(n)+\varphi_\ell(m)=\varphi\left(\frac n{p^k}\right)+\varphi\left(\frac m{p^\ell}\right),\]
		which is what we wanted.
		\qedhere
	\end{itemize}
\end{proof}

\subsection{Inverse Limits}
If we wanted to compute the dual as $\op{Hom}_\ZZ(M_p,\QQ/\ZZ),$ we see that $\op{Hom}_\ZZ\left(\ZZ/p^n\ZZ,\QQ/\ZZ\right)\cong\ZZ/p^n\ZZ$ (tracking where $1$ goes shows $\op{Hom}\left(\ZZ/p^n\ZZ,\QQ/\ZZ\right)\cong\frac1{p^n}\ZZ$), but now the arrows are reversed, so we end up with the following system.
% https://q.uiver.app/?q=WzAsNCxbMCwwLCJcXFpaL3BcXFpaIl0sWzEsMCwiXFxaWi9wXjJcXFpaIl0sWzIsMCwiXFxaWi9wXjNcXFpaIl0sWzMsMCwiXFxjZG90cyJdLFszLDJdLFsyLDFdLFsxLDBdXQ==
\[\begin{tikzcd}
	{\ZZ/p\ZZ} & {\ZZ/p^2\ZZ} & {\ZZ/p^3\ZZ} & \cdots
	\arrow[from=1-4, to=1-3]
	\arrow[from=1-3, to=1-2]
	\arrow[from=1-2, to=1-1]
\end{tikzcd}\]
Explicitly, our map $g_{\ell k}:\ZZ/p^\ell\ZZ\to\ZZ/p^k\ZZ$ for $k\le\ell$ is really referring to the map $\op{Hom}_\ZZ\left(\ZZ/p^\ell\ZZ,\QQ/\ZZ\right)\to\op{Hom}_\ZZ\left(\ZZ/p^k\ZZ,\QQ/\ZZ\right)$ induced by $-\circ f_{k\ell}.$ So we see
\[\left(1\mapsto\frac{n}{p^\ell}\right)\longmapsto\left(1\stackrel{f_{k\ell}}\mapsto p^{\ell-k}\mapsto\frac{n}{p^k}\right).\]
Thus, our map $g_{\ell k}:\ZZ/p^\ell\ZZ\to\ZZ/p^k\ZZ$ is just the projection $n\mapsto n.$ We can check the commutativity laws $g_{\ell k}\circ g_{m\ell}=g_{mk}$ for $k\le\ell\le m$ because both sides are $n\mapsto n$

Now, when taking the limit (!), we are roughly asking for a ``compatible'' system of elements from each of the $\ZZ/p^\bullet\ZZ.$ This sort of limit is called an ``inverse limit.'' We have the following definitions.
\begin{definition}[Invese system]
	Fix $\mathcal I$ a partially ordered set/category where every finite set has an upper bound. Then an \textit{inverse system} is a contravariant functor $F:\mathcal I\to\mathcal A$ satisfying the commutativity requirements of a functor. Explicitly,
	\begin{itemize}
		\item $I\stackrel f\preceq I$ goes to $\id_{F(I)}:F(I)\to F(I),$
		\item $I\stackrel f\preceq J\stackrel g\preceq K$ implies $F(g\circ f)=F(f)\circ F(g).$
	\end{itemize}
\end{definition}
\begin{definition}[Inverse limit]
	An \textit{inverse limit} is the limit of an inverse system.
\end{definition}
And now let's work out our example because I should do this at least once in my life.
\begin{exercise}
	The $p$-adic integers $\ZZ_p$ is the inverse limit of the following diagram.
	\[\begin{tikzcd}
		{\ZZ/p\ZZ} & {\ZZ/p^2\ZZ} & {\ZZ/p^3\ZZ} & \cdots
		\arrow[from=1-4, to=1-3]
		\arrow[from=1-3, to=1-2]
		\arrow[from=1-2, to=1-1]
	\end{tikzcd}\]
\end{exercise}
\begin{proof}
	This is essentially the definition of $\ZZ_p.$ Because I should say something here, I will show the following to describe $\ZZ_p.$
	\begin{lemma} \label{lem:explicitlimit}
		Fix $\mathcal I$ an index category, and fix $\mathcal C$ any of the category of sets, groups, rings, or modules. Then, for any functor $F:\mathcal I\to\mathcal C,$ we can write
		\[\limit_\mathcal IF(I)\cong\left\{(a_I)_{I\in\mathcal I}\in\prod_{I\in\mathcal I}F(I):F(f)(a_I)=a_J\text{ for each }f:I\to J\right\}.\]
	\end{lemma}
	\begin{proof}
		We will in all of those categories at once as much as possible.\footnote{Regardless, I will be somewhat vague in the checks that functions are morphisms because I don't want to each check four times.} For brevity, let $L$ be the given construction. If we are in the category of sets, $L$ is allowed to empty; in the other categories, $L$ is nonempty because it contains the identity. We can also check that the condition
		\[F(f)(a_I)=a_J\]
		for each $f:I\to J$ in $\mathcal I$ preserves group operation, ring multiplication, and linear combination, so $L$ is closed under each of these under the respective categories. Then because
		\[L\subseteq\prod_{I\in\mathcal I}F(I)\]
		as constructed, we really only needed to test $L$ as a subobject.
		
		Continuing, we have projection maps $\pi_J:L\to F(J)$ for each $J\in\mathcal I$ by taking $(a_I)_{I\in\mathcal I}$ to $a_J.$ This map preserves the (pointwise) operations on $L,$ so it is a morphism in any of the given categories. We see that $\pi_J$ commute as needed because, for any $f:J\to K,$ we have
		\[(F(f)\circ\pi_J)\left((a_I)_{I\in\mathcal I}\right)=F(f)(a_J)=a_K=\pi_K\left((a_I)_{I\in\mathcal I}\right),\]
		for any $(a_I)_{I\in\mathcal I}\in L,$ by hypothesis on the $\left(a_I\right)_{I\in\mathcal I}.$

		It remains to show the universal property. Fix $X$ any object with maps $\varphi_I:X\to F(I)$ for each $I\in\mathcal I$ such that $F(f)\circ\varphi_I=\varphi_J$ for each $f:I\to J.$ Then we need to induce a unique map $\varphi:X\to L$ making the following diagram commute.
		% https://q.uiver.app/?q=WzAsNCxbMCwyLCJGKEkpIl0sWzIsMiwiRihKKSJdLFsxLDEsIkwiXSxbMSwwLCJYIl0sWzAsMSwiRihmKSIsMl0sWzIsMSwiXFxwaV9KIiwyXSxbMiwwLCJcXHBpX0kiXSxbMywxLCJcXHZhcnBoaV9KIiwwLHsiY3VydmUiOi0yfV0sWzMsMCwiXFx2YXJwaGlfSSIsMix7ImN1cnZlIjoyfV0sWzMsMiwiXFx2YXJwaGkiLDEseyJzdHlsZSI6eyJib2R5Ijp7Im5hbWUiOiJkYXNoZWQifX19XV0=
		\[\begin{tikzcd}
			& X \\
			& L \\
			{F(I)} && {F(J)}
			\arrow["{F(f)}"', from=3-1, to=3-3]
			\arrow["{\pi_J}"', from=2-2, to=3-3]
			\arrow["{\pi_I}", from=2-2, to=3-1]
			\arrow["{\varphi_J}", curve={height=-12pt}, from=1-2, to=3-3]
			\arrow["{\varphi_I}"', curve={height=12pt}, from=1-2, to=3-1]
			\arrow["\varphi"{description}, dashed, from=1-2, to=2-2]
		\end{tikzcd}\]
		As usual, we show uniqueness and existence of $\varphi$ one at a time.
		\begin{itemize}
			\item We show that $\varphi$ is unique. Indeed, for any $x\in X,$ if $\varphi(x)=(a_I)_{I\in\mathcal I},$ then the commutativity of the diagram forces
			\[a_I=\pi_I(\varphi(x))=\varphi_I(x)\]
			for each $I\in\mathcal I,$ so we are forced to have $\varphi(x)=(\varphi_Ix)_{i\in\mathcal I}.$
			\item We show that $\varphi$ exists. As above, we are forced to define
			\[\varphi(x):=(\varphi_Ix)_{I\in\mathcal I}\]
			for each $x\in X.$ This is indeed an element of $L$ because, for each $f:I\to J,$ we see $F(f)(\varphi_Ix)=\varphi_Jx$ by assumption on the $\varphi_\bullet.$

			Technically, we do have to show that $\varphi$ is also a morphism. Well, we note that $\varphi$ is actually the induced map
			\[X\to\prod_{I\in\mathcal I}F(I)\]
			where we have restricted the output to live in $L.$ So because the product exists as constructed in each of the given categories, we see $X\to L$ is a morphism.
			\qedhere
		\end{itemize}
	\end{proof}
	The point is that we can realize $\ZZ_p$ as
	\[\limit\ZZ/p^\bullet\ZZ\cong\left\{(a_k)_{k\ge1}\in\prod_{k=1}^\infty\ZZ/p^k\ZZ:a_\ell\equiv a_k\pmod{p^k}\text{ for each }\ell\ge k\right\}\]
	using the above construction. In words, $\ZZ_p$ consists of infinite sequences of elements of $\ZZ?p^\bullet\ZZ$ where the elements are ``compatible'' with each other.
\end{proof}

\subsection{Duals of Direct Limits}
Our story of $\ZZ_p$ was about dualizing the diagram for $M_p,$ so it is a reasonable to hope that the dual of $M_p$ is $\ZZ_p.$ This is indeed true.
\begin{exercise}
	We have that $\op{Hom}(M_p,\QQ/\ZZ)\cong\ZZ_p.$
\end{exercise}
\begin{proof}
	We use the explicit construction of $\ZZ_p$ given by \autoref{lem:explicitlimit}. We map $\varphi:\ZZ_p\to\op{Hom}(M_p,\QQ/\ZZ)$ explicitly by taking $(a_k)_{k\ge1}\in\ZZ_p$ to the map
	\[\varphi\left((a_k)_{k\ge1}\right):\frac n{p^k}\mapsto\frac{na_k}{p^k}.\]
	The rest of the proof is book-keeping; we check that $\varphi$ is indeed an isomorphism.
	\begin{itemize}
		\item This map is well-defined because, even though we might have $\frac n{p^k}=\frac m{p^\ell}$ where $k\le\ell$ without loss of generality, we see that $np^{\ell-k}\equiv m\pmod{p^\ell},$ which implies that
		\[a_knp^{\ell-k}\equiv a_\ell m\pmod{p^\ell}\]
		because $a_k\equiv a_\ell\pmod{p^k}.$ So indeed, $\frac{na_k}{p^k}\equiv\frac{a_\ell m}{p^\ell}\pmod1.$
		\item We see $\varphi\left((a_k)_{k\ge1}\right)$ is indeed a homomorphism. Indeed, given $\frac n{p^k},\frac m{p^\ell}\in M_p,$ we have
		\[\varphi(a)\left(\frac n{p^k}+\frac m{p^\ell}\right)=\varphi(a)\left(\frac{np^\ell+mp^k}{p^{k+\ell}}\right)=\frac{a_{k+\ell}np^\ell}{p^{k+\ell}}+\frac{a_{k+\ell}mp^k}{p^{k+\ell}}=\varphi\left(\frac{np^\ell}{p^{k+\ell}}\right)+\varphi\left(\frac{mp^k}{p^{k+\ell}}\right),\]
		which collapses to what we want.
		\item We see $\varphi$ is itself a homomorphism. Fix $a,b\in\ZZ_p.$ Then for any $\frac n{p^k}\in M_p,$ we have
		\[\varphi(a+b)\left(\frac n{p^k}\right)=\frac{(a_k+b_k)n}{p^k}=\frac{a_kn}{p^k}+\frac{b_kn}{p^k}=\big(\varphi(a)+\varphi(b)\big)\left(\frac n{p^k}\right).\]
		\item We see $\varphi$ is injective. Indeed, it suffices to show that $\varphi$ has trivial kernel. So suppose $a\in\ZZ_p$ has $\varphi(a)$ the zero map. Well, for any $k\ge1,$ we see
		\[\frac{a_k}{p^k}=\varphi(a)\left(\frac1{p^k}\right)=0,\]
		so $a_k\equiv0\pmod{p^k}.$ So indeed, $a$ is the zero element.
		\item We show $\varphi$ is surjective. Fix $f\in\op{Hom}(M_p,\QQ/\ZZ)$ some homomorphism. Then, for any $k\ge1,$ we note $f\left(1/p^k\right)=a_k/p^k$ for some $a_k\in\ZZ$ because $p^k\cdot f\left(1/p^k\right)=f(1)=f(0)=0.$ We claim that
		\[a:=(a_k)_{k\ge1}\in\ZZ_p.\]
		Indeed, for any $k\le\ell,$ we see that
		\[\frac{a_k}{p^k}=f\left(\frac1{p^k}\right)=f\left(\frac{p^{\ell-k}}{p^\ell}\right)=\frac{a_\ell p^{\ell-k}}{p^\ell}=\frac{a_\ell}{p^k},\]
		so $a_k\equiv a_\ell\pmod{p^k}.$

		So we claim that $f=\varphi(a).$ Indeed, for any $\frac n{p^k}\in M_p,$ we see that
		\[f\left(\frac n{p^k}\right)=n\cdot f\left(\frac1{p^k}\right)=n\cdot\frac{a_k}{p^k}=\frac{a_kn}{p^k}=\varphi(a)\left(\frac n{p^k}\right),\]
		which is what we needed.
		\qedhere
	\end{itemize}
\end{proof}
Putting everything together, we saw that
\[\op{Hom}\left(\colimit\ZZ/p^\bullet\ZZ,\QQ/\ZZ\right)\cong\limit\op{Hom}\left(\ZZ/p^\bullet\ZZ,\QQ/\ZZ\right).\]
In fact, this holds more generally.
\begin{proposition} \label{prop:homlimits}
	Fix $\mathcal I$ an index category and $F:\mathcal I\to\mathcal C$ a functor, where $\mathcal C$ is the category of sets, groups, rings, or modules. Then, for any object $X\in\mathcal C,$
	\[\op{Hom}_R\left(\colimit_{\mathcal I}F(I),X\right)\cong\limit_{\mathcal I}\op{Hom}_R\left(F(I),X\right).\]
\end{proposition}
\begin{proof}
	We essentially imitate the example. We use \autoref{lem:explicitlimit}, which tells us that
	\[\limit_{\mathcal I}\op{Hom}_R\left(F(I),X\right)\cong\left\{(\varphi_I)\in\prod_{I\in\mathcal I}\op{Hom}_R(F(I),X):\varphi_I=\varphi_J\circ F(f)\text{ for each }f:I\to J\right\}=:L,\]
	where the commutativity laws come from the fact that $f:I\to J$ induces the morphism $\op{Hom}(F(J),X)\to\op{Hom}(F(I),X)$ by $-\circ F(f),$ so the condition $F(f)(\varphi_J)=\varphi_I$ (where $F(f)$ here is heavy abuse of notation) reads as $\varphi_J\circ F(f)=\varphi_I.$

	Staring harder, we see $L$ is tuples of maps $\varphi_I:F(I)\to X$ which make the following diagram commute. (Here, we are naming the maps of $\colimit_{\mathcal I}F(I)$ by $\iota_I:F(I)\to\colimit_{\mathcal I}F(I).$)
	% https://q.uiver.app/?q=WzAsNCxbMCwwLCJGKEkpIl0sWzIsMCwiRihKKSJdLFsxLDIsIlgiXSxbMSwxLCJcXGNvbGltaXRfe1xcbWF0aGNhbCBJfUYoSSkiXSxbMCwxLCJGKGYpIl0sWzAsMiwiXFx2YXJwaGlfSSIsMix7ImN1cnZlIjoyfV0sWzEsMiwiXFx2YXJwaGlfSiIsMCx7ImN1cnZlIjotMn1dLFswLDMsIlxcaW90YV9JIl0sWzEsMywiXFxpb3RhX0oiLDJdXQ==
	\[\begin{tikzcd}
		{F(I)} && {F(J)} \\
		& {\colimit_{\mathcal I}F(I)} \\
		& X
		\arrow["{F(f)}", from=1-1, to=1-3]
		\arrow["{\varphi_I}"', curve={height=12pt}, from=1-1, to=3-2]
		\arrow["{\varphi_J}", curve={height=-12pt}, from=1-3, to=3-2]
		\arrow["{\iota_I}", from=1-1, to=2-2]
		\arrow["{\iota_J}"', from=1-3, to=2-2]
	\end{tikzcd}\]
	But by the universal property of $\colimit_{\mathcal I}F(I),$ we can take tuples $(\varphi_I)$ which commute with the $F(f)$ to unique morphisms $\varphi:\colimit_{\mathcal I}F(I)\to X$ such that $\varphi_I=\varphi\circ\iota_Ipi.$ Call this map
	\[\psi:L\to\op{Hom}_R\left(\colimit_{\mathcal I}F(I),X\right).\]
	We can check by hand that $\psi$ is an $R$-module homomorphism. This is more or less book-keeping.
	\begin{itemize}
		\item We see that $\psi$ is well-defined because the morphism $\colimit_{\mathcal I}F(I)\to X$ induced by the universal property is unique.
		\item We show that $\psi$ is an $R$-module homomorphism. Indeed, fix $r_1,r_2\in R$ and $(\varphi_I^1),(\varphi_I^2)\in L$ so that $\varphi^1:=\psi\left((\varphi_I^1)\right)$ and $\varphi^2:=\psi\left((\varphi_I^2)\right).$ Then we see that $\varphi:=r_1\varphi^1+r_2\varphi^2$ satisfies
		\[(\varphi\circ\iota_I)(a_I)=r_1\varphi^1(\iota_Ia_I)+r_2\varphi^2(\iota_Ia_I)=(r_1\varphi_I^1+r_2\varphi_I^2)(a_I),\]
		for any $a_I\in F(I).$ The point of this computation is that $r_1\varphi^1+r_2\varphi^2$ commutes with the same diagram that $\psi\left((r_1\varphi_I^1+r_2\varphi_I^2)_I\right)$ commutes with, so they are equal by uniqueness.
		\item We show that $\psi$ is injective, for which it suffices to check that $\psi$ has trivial kernel. Well, suppose $\psi\left((\varphi_I)\right)=0.$ Then, by the commuting in the universal property, we see that $\varphi_I=0\circ\iota_I=0$ for each $I\in\mathcal I,$ so indeed, $(\varphi_I)$ is the zero element.
		\item  We show that $\psi$ is surjective. Indeed, given a morphism $\varphi:\colimit_{\mathcal I}F(I)\to X,$ we set $\varphi_I:=\varphi\circ\iota_I,$ which has $\varphi_I:F(I)\to X.$ We can check that $(\varphi_I)\in L$ because, for each $f:I\to J,$
		\[\varphi_J\circ F(f)=\varphi\circ\iota_J\circ F(f)=\varphi\circ\iota_I=\varphi_I\]
		by hypothesis on the $\iota_\bullet.$

		And to finish, we see that $\psi\left((\varphi_I)\right)=\varphi$ because $\varphi\circ\iota_I=\varphi_I$ by construction.
		\qedhere
	\end{itemize}
\end{proof}
As a remark, the dual of \autoref{prop:homlimits} is not generally true for size reasons: limits are big, and hom-sets tend to be bigger, so
\[\op{Hom}_R\left(X,\limit_{\mathcal I}F(I)\right)\]
is frankly huge. However, the direct limit tends to have a topology, and when this is taken to account, things tend to be better behaved. But this is more analysis than algebra.

\subsection{Profinite Groups}
Let's have another example.
\begin{example}
	We have the ``profinite'' completion $\widehat\ZZ$ of $\ZZ$ by the inverse limit of the system $\ZZ/n\ZZ,$ where we have maps $\ZZ/n\ZZ\to\ZZ/m\ZZ$ by projection whenever $m\mid n.$ Of course, we do have
	\[\ZZ\into\limit\ZZ/n\ZZ,\]
	and in fact this is a compact ring because it is the product of lots of compact $\ZZ_p$ rings.
\end{example}
Explicitly, we can show the following.
\begin{exercise} \label{lem:profinitez}
	We have that
	\[\widehat\ZZ\cong\prod_p\ZZ_p,\]
	where the product is taken over primes $p.$
\end{exercise}
\begin{proof}
	There is some Yoneda stuff that we can do because both are inverse limits\footnote{Maps into $\prod_p\ZZ_p$ are essentially maps into each of the $\ZZ/p^k\ZZ$ for each prime power $p^k$ which commute at each prime. These maps can be uniquely assembled into a map into each $\ZZ/n\ZZ$ for each $n\in\NN,$ which are in bijection with maps into $\widehat\ZZ.$}, but we can just exhibit the isomorphism by hand.

	Indeed, we use \autoref{lem:explicitlimit} to write
	\[\widehat\ZZ\cong\left\{(a_n)\in\prod_{n\in\NN}\ZZ/n\ZZ:a_n\equiv a_m\pmod m\text{ for }m\mid n\right\}.\]
	In particular, we note that we have a map $\varphi_p:\widehat\ZZ\to\ZZ_p$ for each prime $p$ by taking
	\[\varphi_p\left((a_n)\right):=\left(a_{p^k}\right)_{k\ge1}\in\ZZ_p.\]
	We do indeed get out an element of $\ZZ_p$ because, for any $k\ge\ell,$ we need $a_{p^\ell}\equiv a_{p^k}\pmod{p^k},$ which is true because $p^k\mid p^\ell.$ We can also check that this is a group homomorphism: given $(a_n)\in\widehat\ZZ$ and $b_n\in\widehat\ZZ,$ we see that
	\[\varphi_p((a_n)+(b_n))=\varphi_p((a_n+b_n))=\left(a_{p^k}+b_{p^k}\right)_{k\ge1}=\left(a_{p^k}\right)_{k\ge1}+\left(b_{p^k}\right)_{k\ge1}=\varphi_p((a_n))+\varphi_p(n).\]
	Anyways, the morphisms $\varphi_p:\widehat{\ZZ}\to\ZZ_p$ for each $p$ can be used to assemble a morphism
	\[\varphi:\widehat{\ZZ}\to\prod_p\ZZ_p\]
	by the universal property of the product. We claim that $\varphi$ is an isomorphism.
	\begin{itemize}
		\item We show that $\varphi$ is injective. We already know that $\varphi$ is a morphism, so it suffices to show that $\varphi$ has trivial kernel. Well, fix some $a=(a_n)_{n\in\NN}$ which goes to $0$ under $\varphi.$ Then, for any $n\in\NN,$ we show $a_n=0,$ which will be enough to conclude $(a_n)=0.$ Indeed, fix the prime factorization
		\[n=\prod_{p\mid n}p^{\alpha_p}.\]
		Then $p^{\alpha_p}\mid n$ for each $p\mid n,$ so
		\[a_n\equiv a_{p^{\alpha_p}}\pmod{p^{\alpha_p}},\]
		but $a_{p^\alpha_p}=\varphi_p(a)=0$ because $a\in\ker\varphi,$ so we see
		\[a_n\equiv0\pmod{p^{\alpha_p}}\]
		for each $p\mid n.$ Using the Chinese remainder theorem to assemble this (finite) system of congruences, we see that
		\[a_n\equiv0\pmod n,\]
		which is exactly what we wanted.
		\item We show that $\varphi$ is surjective. Fix some tuple
		\[a=(a_p)_p=\left((a_k)_{k\ge1}\right)_p\in\prod_p\ZZ_p\]
		that we want to hit by $\varphi.$ Well, for each $n\in\NN,$ we note that we have the prime factorization
		\[n=\prod_pp^{\nu_p(n)},\]
		so we conjure $a_n$ by using the Chinese remainder theorem to solve the (finite) system of congruences
		\[a_n:=(a_{\nu_p(n)})_p\]
		for each prime $p.$ (This system is finite because we can ignore all the primes $p$ where $\nu_p(n)=0,$ and only finitely many primes divide $p.$) We check that $(a_n)_{n\in\NN}$ is a well-defined element of $\widehat{\ZZ}$: if $m\mid n,$ then $\nu_p(m)\le\nu_p(n)$ for each prime $p,$ so
		\[a_n=(a_{\nu_p(n)})_p\equiv (a_{\nu_p(n)})_p=a_m\pmod{p^{\nu_p(m)}}\]
		because $(a_k)_p\in\ZZ_p.$ So the Chinese remainder theorem now promises that $a_n\equiv a_m\pmod m,$ as needed.

		It remains to check that $\varphi\left((a_n)_{n\in\NN}\right)=a.$ Well, by construction, we see that
		\[\varphi\left((a_n)_{n\in\NN}\right)=\left(\left(a_{p^k}\right)_{k\ge1}\right)_p=\left((a_k)_{k\ge1}\right)_p=a,\]
		as needed. This finishes.
		\qedhere
	\end{itemize}
\end{proof}
Anyways, here is the definition of profinite.
\begin{definition}[Profinite]
	A \textit{profinite group} is a group which is the inverse limit of some finite groups.
\end{definition}
\begin{remark}
	``Profinite'' is short of ``projective limit of finite things.'' These are compact, topologically, which is nice. (In fact, we can define profinite groups topologically as Hausdorff, compact, totally disconnected topological groups.)
\end{remark}
As a warning, sometimes taking the profinite completion just gives $0,$ which is indeed compact, though not very useful.

Number theorists tend to like $\widehat{\ZZ}$ because it almost looks like the ring of adeles.
\begin{example}
	The ring of adeles is $(\RR\times\widehat{\ZZ})\otimes_\ZZ\QQ,$ so we do have some reason to care about the profinite completion. Roughly speaking, this is because we are really looking at
	\[\left(\RR\times\prod_p\ZZ_p\right)\otimes_\ZZ\QQ,\]
	and a specific tensor with $\QQ$ can only ``introduce'' finitely primes into the denominator. At a high level, the finite places are coming from the $\ZZ_p,$ and the infinite places are coming from $\RR.$
\end{example}

\subsection{Colimits and Exactness}
Let's see if limits and colimits preserve exactness. Namely, fix our index category $\mathcal I$ with functors $A_\bullet,B_\bullet,C_\bullet:\mathcal I\to\mathcal C,$ with prescribed exact sequences
\[0\to A_I\stackrel{\iota_I}\to B_I\stackrel{\pi_I}\to C_I\to 0\]
for each $I\in\mathcal I.$ We will also require each square induced by $f:I\to J$
% https://q.uiver.app/?q=WzAsOCxbMCwwLCJJIl0sWzAsMSwiSiJdLFsxLDAsIkFfSSJdLFsxLDEsIkFfSiJdLFsyLDAsIkJfSSJdLFsyLDEsIkJfSiJdLFszLDAsIkNfSSJdLFszLDEsIkNfSiJdLFswLDEsImYiLDJdLFsyLDMsIkFfZiIsMl0sWzIsNCwiXFxpb3RhX0kiXSxbMyw1LCJcXGlvdGFfSiJdLFs0LDUsIkJfZiIsMl0sWzYsNywiQ19mIiwyXSxbNSw3LCJcXGlvdGFfSiJdLFs0LDYsIlxcaW90YV9JIl1d
\[\begin{tikzcd}
	I & {A_I} & {B_I} & {C_I} \\
	J & {A_J} & {B_J} & {C_J}
	\arrow["f"', from=1-1, to=2-1]
	\arrow["{A_f}"', from=1-2, to=2-2]
	\arrow["{\iota_I}", from=1-2, to=1-3]
	\arrow["{\iota_J}", from=2-2, to=2-3]
	\arrow["{B_f}"', from=1-3, to=2-3]
	\arrow["{C_f}"', from=1-4, to=2-4]
	\arrow["{\pi_J}", from=2-3, to=2-4]
	\arrow["{\pi_I}", from=1-3, to=1-4]
\end{tikzcd}\]
to commute. Then we can ask if
\[0\to\limit_{\mathcal I} A_I\to\limit_{\mathcal I} B_I\to \limit_{\mathcal I} C_I\to 0\]
is exact, as well as the same question for $\colimit.$ To be explicit, the composite maps
\[\limit_{\mathcal I}A_I\to A_I\stackrel{\iota_I}\to B_I\]
induce a (unique commuting) map $\limit_{\mathcal I}A_I\to\limit_{\mathcal I}B_I$ because we can see that $\limit_{\mathcal I}\to B_I\to B_J$ and $\limit_{\mathcal I}A_I\to B_J$ are the same by the commutativity hypothesis on the $\iota_\bullet.$ In other words, the following diagram commutes.
% https://q.uiver.app/?q=WzAsNyxbMiwwLCJBX0kiXSxbMiwxLCJBX0oiXSxbMywwLCJCX0kiXSxbMywxLCJCX0oiXSxbMSwwLCJcXGxpbWl0X3tcXG1hdGhjYWwgSX1BX0kiXSxbMCwwLCJJIl0sWzAsMSwiSiJdLFswLDEsIkFfZiIsMl0sWzAsMiwiXFxpb3RhX0kiXSxbMSwzLCJcXGlvdGFfSiJdLFsyLDMsIkJfZiIsMl0sWzQsMF0sWzQsMV0sWzUsNiwiZiIsMl1d
\[\begin{tikzcd}
	I & {\limit_{\mathcal I}A_I} & {A_I} & {B_I} \\
	J && {A_J} & {B_J}
	\arrow["{A_f}"', from=1-3, to=2-3]
	\arrow["{\iota_I}", from=1-3, to=1-4]
	\arrow["{\iota_J}", from=2-3, to=2-4]
	\arrow["{B_f}"', from=1-4, to=2-4]
	\arrow[from=1-2, to=1-3]
	\arrow[from=1-2, to=2-3]
	\arrow["f"', from=1-1, to=2-1]
\end{tikzcd}\]
Then we can induce $\limit_{\mathcal I}B_I\to\limit_{\mathcal I}C_I$ in the same way.

Similarly, the composite maps
\[A_I\to B_I\to\colimit_{\mathcal I}B_I\]
induce a (unique commuting) map $\colimit_{\mathcal I}A_I\to\colimit_{\mathcal I}B_I,$ and we can induce $\colimit_{\mathcal I}B_I\to\colimit_{\mathcal I}C_I$ in the same way.

As a more concrete example, taking $\mathcal I$ to be a category with no morphisms, $\limit$ is asking if products preserve exactness, and $\colimit$ is asking if coproducts preserve exactness.
\begin{example}
	Taking direct sums if two short exact sequences
	\[0\to A_1\to B_1\to C_1\to 0\]
	and
	\[0\to A_2\to B_2\to C_2\to 0\]
	we get
	\[0\to A_1\oplus A_2\to B_1\oplus B_2\to C_1\oplus C_2\to 0,\]
	which is still exact.
\end{example}
In general, however, the best we can say is that limits will preserve left exactness, giving the exact sequence
\[0\to\limit A_\alpha\to\limit B_\alpha\to \limit C_\alpha,\]
and colimits will preserve right exactness, giving the short exactness
\[\colimit A_\alpha\to\colimit B_\alpha\to \colimit C_\alpha\to 0.\]
Essentially this is because limits preserve kernels, which is equivalent to the left exactness; more generally, limits preserve limits. (And dually, on the other hand, colimits will preserve colimits.)
\begin{proposition}
	Limits preserve limits. Namely, if we have index categories $\mathcal I$ and $\mathcal J$ with the system $\{F(I,J)\}_{i\in I,j\in J},$ then
	\[\limit_I\limit_JF(I,J)\cong\limit_J\limit_IF(I,J).\]
\end{proposition}
\begin{proof}
	The idea is to show that both sides are
	\[\limit_{I\times J}F(I,J).\]
	We leave the details as an exercise. % \todo{}
\end{proof}
By considering duals, we have the following.
\begin{cor}
	Colimits preserve colimits.
\end{cor}
\begin{proof}
	Push everything into the opposite category, where the statement is that limits preserve limits.
\end{proof}
% \begin{example}
% 	Given short exact sequences of groups going downwards, we can show that we get a left exact sequence of the limits.
% \end{example}
In particular, because cokernels are colimits, we find that colimits preserve cokernels, so colimits are right exact.

And here is the corresponding counterexample: colimits do not always preserve left exactness.
\begin{exercise}
	Colimits do not always preserve kernels.
\end{exercise}
\begin{proof}
	% Consider the fiber coproduct of the following diagram.
	% % https://q.uiver.app/?q=WzAsMyxbMCwwLCJcXFpaIl0sWzAsMSwiXFxaWiJdLFsxLDAsIlxcWloiXSxbMCwxLCJcXHRpbWVzMiIsMl0sWzAsMiwiXFx0aW1lczIiXV0=
	% \[\begin{tikzcd}
	% 	\ZZ & \ZZ \\
	% 	\ZZ
	% 	\arrow["\times2"', from=1-1, to=2-1]
	% 	\arrow["\times2", from=1-1, to=1-2]
	% \end{tikzcd}\]
	% From the homework, we know that our fiber coproduct is
	% \[G:=\frac{\ZZ\oplus\ZZ}{\{(2k,-2k):k\in\ZZ\}}.\]
	% (Intuitively, this is $X\oplus Y,$ where we do as little damage to make the diagram commute: we mod out by making the two $\ZZ\stackrel{\times2}\ZZ$ equal each other.) However, we have the map $\varphi:\ZZ\oplus\ZZ\to\ZZ\oplus\ZZ/2\ZZ$ by
	% \[\varphi:(x,y)\mapsto\left(x+y,y\pmod2\right).\]
	% We claim that $\varphi$ exhibits $G\cong\ZZ\oplus\ZZ/2\ZZ.$ We'll run through the checks quickly.
	% \begin{itemize}
	% 	\item We see $\varphi$ is homomorphic because $(x,y)\mapsto x+y$ is associated to the matrix
	% 	\[\begin{bmatrix}
	% 		1 & 1
	% 	\end{bmatrix}\begin{bmatrix}
	% 		x \\ y
	% 	\end{bmatrix}=\begin{bmatrix}
	% 		x+y
	% 	\end{bmatrix}.\]
	% 	Then the map $(x,y)\mapsto y\mapsto y\pmod 2$ is the composite of $\ZZ$-linear maps and hence $\ZZ$-linear, so $\varphi$ is $\ZZ$-linear componentwise and hence componentwise.
	% 	\item We see $\varphi$ is surjective because, for any $(a,b)\in\ZZ\oplus\ZZ/2\ZZ,$ we see that $\varphi:(a-b,b)\mapsto(a-b+b,b)=(a,b).$
	% 	\item We see $\varphi$ has kernel $\{(2k,-2k):k\in\ZZ\}.$ Indeed, if $(x,y)$ goes to $0,$ then the first coordinate forces $x+y=0,$ so $x=-x.$ Then $y$ must also be even, so $x$ must be even, so set $x=2k,$ and we see $(x,y)=(2k,-2k).$

	% 	Conversely, we can check that any point of the form $(2k,-2k)$ goes to $(2k+-2k,-2k\pmod2)=(0,0).$
	% \end{itemize}
	Pick up our favorite counterexample
	\[0\to\ZZ\stackrel{\times2}\to\ZZ\to\ZZ/2\ZZ\to0,\]
	and because Professor Borcherds is in an incredibly unimaginative mood this morning, we simply take the fiber products of this sequence three times.
	% https://q.uiver.app/?q=WzAsMTUsWzAsMCwiMCJdLFswLDEsIjAiXSxbMCwyLCIwIl0sWzEsMCwiXFxaWiJdLFsxLDEsIlxcWloiXSxbMSwyLCJcXFpaIl0sWzIsMCwiXFxaWiJdLFsyLDEsIlxcWloiXSxbMiwyLCJcXFpaIl0sWzMsMCwiXFxaWi8yXFxaWiJdLFszLDEsIlxcWlovMlxcWloiXSxbMywyLCJcXFpaLzJcXFpaIl0sWzQsMCwiMCJdLFs0LDEsIjAiXSxbNCwyLCIwIl0sWzAsM10sWzMsNiwiXFx0aW1lczIiXSxbNiw5XSxbOSwxMl0sWzEsNF0sWzQsNywiXFx0aW1lczIiXSxbNywxMF0sWzEwLDEzXSxbMiw1XSxbNSw4LCJcXHRpbWVzMiJdLFs4LDExXSxbMTEsMTRdLFs0LDMsIlxcdGltZXMyIl0sWzQsNSwiXFx0aW1lczIiLDJdLFs3LDYsIlxcdGltZXMyIl0sWzcsOCwiXFx0aW1lczIiLDJdLFsxMCw5LCJcXHRpbWVzMiJdLFsxMCwxMSwiXFx0aW1lczIiLDJdXQ==
	\[\begin{tikzcd}
		0 & \ZZ & \ZZ & {\ZZ/2\ZZ} & 0 \\
		0 & \ZZ & \ZZ & {\ZZ/2\ZZ} & 0 \\
		0 & \ZZ & \ZZ & {\ZZ/2\ZZ} & 0
		\arrow[from=1-1, to=1-2]
		\arrow["\times2", from=1-2, to=1-3]
		\arrow[from=1-3, to=1-4]
		\arrow[from=1-4, to=1-5]
		\arrow[from=2-1, to=2-2]
		\arrow["\times2", from=2-2, to=2-3]
		\arrow[from=2-3, to=2-4]
		\arrow[from=2-4, to=2-5]
		\arrow[from=3-1, to=3-2]
		\arrow["\times2", from=3-2, to=3-3]
		\arrow[from=3-3, to=3-4]
		\arrow[from=3-4, to=3-5]
		\arrow["\times2", from=2-2, to=1-2]
		\arrow["\times2"', from=2-2, to=3-2]
		\arrow["\times2", from=2-3, to=1-3]
		\arrow["\times2"', from=2-3, to=3-3]
		\arrow["\times2", from=2-4, to=1-4]
		\arrow["\times2"', from=2-4, to=3-4]
	\end{tikzcd}\]
	We really only have to pay attention to the map on the left. Indeed, we stare at the following diagram.
	% https://q.uiver.app/?q=WzAsNixbMCwwLCJcXG1hdGhiYiBaIl0sWzAsMiwiXFxtYXRoYmIgWiJdLFsxLDEsIlxcZnJhY3tcXG1hdGhiYiBaXFxvcGx1c1xcbWF0aGJiIFp9e1xceygyaywtMmspXFx9fSJdLFsyLDAsIlxcWloiXSxbMywxLCJcXGZyYWN7XFxtYXRoYmIgWlxcb3BsdXNcXG1hdGhiYiBafXtcXHsoMmssLTJrKVxcfX0iXSxbMiwyLCJcXFpaIl0sWzAsMl0sWzEsMl0sWzMsNF0sWzUsNF0sWzAsMywiXFx0aW1lczIiXSxbMSw1LCJcXHRpbWVzMiJdLFsyLDQsIiIsMSx7InN0eWxlIjp7ImJvZHkiOnsibmFtZSI6ImRhc2hlZCJ9fX1dXQ==
	\[\begin{tikzcd}
		{\mathbb Z} && \ZZ \\
		& {\frac{\mathbb Z\oplus\mathbb Z}{\{(2k,-2k)\}}} && {\frac{\mathbb Z\oplus\mathbb Z}{\{(2k,-2k)\}}} \\
		{\mathbb Z} && \ZZ
		\arrow[from=1-1, to=2-2]
		\arrow[from=3-1, to=2-2]
		\arrow[from=1-3, to=2-4]
		\arrow[from=3-3, to=2-4]
		\arrow["\times2", from=1-1, to=1-3]
		\arrow["\times2", from=3-1, to=3-3]
		\arrow[dashed, from=2-2, to=2-4]
	\end{tikzcd}\]
	The induced map here will follow the $\times2$ through, so our map is
	\[\frac{\mathbb Z\oplus\mathbb Z}{\{(2k,-2k)\}}\stackrel{(\times2,\times2)}\to\frac{\mathbb Z\oplus\mathbb Z}{\{(2k,-2k)\}}.\]
	We can check that this is not surjective because any representative in the output will vanish under projecting into $(\ZZ/2\ZZ)^2,$ but the projection $\frac{\mathbb Z\oplus\mathbb Z}{\{(2k,-2k)\}}\to\frac{\ZZ\oplus\ZZ}{2\ZZ\oplus2\ZZ}$ still sends $(1,1)$ somewhere nontrivial.
\end{proof}
\begin{remark}
	In fact, we have that
	\[\frac{\mathbb Z\oplus\mathbb Z}{\{(2k,-2k)\}}\cong\ZZ\oplus\ZZ/2\ZZ,\]
	say by taking $(x,y)\mapsto(x+y,y\pmod2).$ We won't explicitly check that this is an isomorphism, but it can be checked.
\end{remark}
Anyways, it turns out that many cases do have colimits preserving kernels. Namely, in the case of direct limits they do because we added the condition that every finite set has an upper bound (!).
\begin{example}
	The integers with the usual ordering is a ``directed set'': any finite set does indeed have an upper bound by taking the maximum. More generally, any totally ordered set is directed.
\end{example}
Anyways, we have the following.
\begin{proposition}
	Colimits over directed sets do preserve exactness.
\end{proposition}
\begin{proof}
	We show this for index category given by the partially ordered set $\NN,$ for ease of notation. Namely, we have a list of short exact sequences as follows.
	% https://q.uiver.app/?q=WzAsMTgsWzEsMCwiXFx2ZG90cyJdLFsxLDMsIkFfMCJdLFsyLDMsIkJfMCJdLFszLDMsIkNfMCJdLFs0LDMsIjAiXSxbMCwzLCIwIl0sWzEsMiwiQV8xIl0sWzIsMiwiQl8xIl0sWzMsMiwiQ18xIl0sWzQsMiwiMCJdLFswLDIsIjAiXSxbMSwxLCJBXzIiXSxbMiwxLCJCXzIiXSxbMywxLCJDXzIiXSxbNCwxLCIwIl0sWzAsMSwiMCJdLFsyLDAsIlxcdmRvdHMiXSxbMywwLCJcXHZkb3RzIl0sWzUsMV0sWzEsMl0sWzIsM10sWzMsNF0sWzE1LDExXSxbMTEsMTJdLFsxMiwxM10sWzEzLDE0XSxbMTAsNl0sWzYsN10sWzcsOF0sWzgsOV0sWzYsMTFdLFsxLDZdLFs3LDEyXSxbMiw3XSxbOCwxM10sWzMsOF0sWzExLDBdLFsxMiwxNl0sWzEzLDE3XV0=
	\[\begin{tikzcd}
		& \vdots & \vdots & \vdots \\
		0 & {A_2} & {B_2} & {C_2} & 0 \\
		0 & {A_1} & {B_1} & {C_1} & 0 \\
		0 & {A_0} & {B_0} & {C_0} & 0
		\arrow[from=4-1, to=4-2]
		\arrow[from=4-2, to=4-3]
		\arrow[from=4-3, to=4-4]
		\arrow[from=4-4, to=4-5]
		\arrow[from=2-1, to=2-2]
		\arrow[from=2-2, to=2-3]
		\arrow[from=2-3, to=2-4]
		\arrow[from=2-4, to=2-5]
		\arrow[from=3-1, to=3-2]
		\arrow[from=3-2, to=3-3]
		\arrow[from=3-3, to=3-4]
		\arrow[from=3-4, to=3-5]
		\arrow[from=3-2, to=2-2]
		\arrow[from=4-2, to=3-2]
		\arrow[from=3-3, to=2-3]
		\arrow[from=4-3, to=3-3]
		\arrow[from=3-4, to=2-4]
		\arrow[from=4-4, to=3-4]
		\arrow[from=2-2, to=1-2]
		\arrow[from=2-3, to=1-3]
		\arrow[from=2-4, to=1-4]
	\end{tikzcd}\]
	Recall that because colimits preserve colimits and hence cokernels (i.e., quotients), we know that
	\[\colimit_{k\in \NN}A_k\to\colimit_{k\in\NN}B_k\to\colimit_{k\in\NN}C_k\to0\]
	is exact. We want to get injectivity of the map $\colimit A_k\to\colimit B_k.$ Well, pick $a\in\limit A_k$ which is in the kernel of this map, and because the set is directed, we may choose a particular $a_k\in A_k$ representing $A.$ % \todo{}
	Essentially, what is happening here is that
	\[\colimit A_k=\bigcup_{k\in\NN}A_k\bigg/\text{some equivalence relation}.\]
	But now $a_k\to 0$ for some $B_\ell$ where $\ell\ge k,$ so $a=0$ by exactness of the original sequence.
\end{proof}
Importantly, the above argument fails for the diagram
% https://q.uiver.app/?q=WzAsMyxbMCwwLCJcXGJ1bGxldCJdLFswLDEsIlxcYnVsbGV0Il0sWzEsMCwiXFxidWxsZXQiXSxbMCwxXSxbMCwyXV0=
\[\begin{tikzcd}
	\bullet & \bullet \\
	\bullet
	\arrow[from=1-1, to=2-1]
	\arrow[from=1-1, to=1-2]
\end{tikzcd}\]
because elements of the fiber product do not need to come from either $A$ or $B,$ for they might come from a pair of both.
\begin{remark}
	Colimits do still preserve exactness over ``filtered'' categories, which are categories $\mathcal C$ for which any objects $A$ and $B$ have a third object $C$ with $A\to C$ and $B\to C,$ as well as the condition that any time we have two maps $A\to B,$ there is a map $B\to C$ for which the composites are equal.
\end{remark}

We also have the following.
\begin{proposition}
	Colimits preserve exactness over discrete categories.
\end{proposition}
\begin{proof}
	Colimits over discrete categories are the direct sum, so we are saying that a set of short exact sequences
	\[0\to A_\alpha\stackrel{\iota_\alpha}\to B_\alpha\stackrel{\pi_\alpha}\to C_\alpha\to 0\]
	for $\alpha\in\lambda$ induces a short exact sequence
	\[0\to\bigoplus_{\alpha\in\lambda} A_\alpha\stackrel{\iota}\to\bigoplus_{\alpha\in\lambda} B_\alpha\stackrel\pi\to\bigoplus_{\alpha\in\lambda} C_\alpha\to 0.\]
	To show this, we again note that we are only interested in the injectivity of left-hand map. Well, suppose that an element $(a_\alpha)_{\alpha\in\lambda}\in\bigoplus_\alpha A_\alpha$ is in the kernel; tracking through the inclusions, we see that we must have
	\[\pi\left((a_\alpha)_{\alpha\in\lambda})\right)=\left(\pi_\alpha a_\alpha\right)_{\alpha\in\lambda},\]
	so being in the kernel forces that $a_\alpha\in\ker\pi_\alpha$ for each $\alpha\in\lambda.$ But by exactness, we must have $a_\alpha=0$ for each $\alpha\in\lambda,$ so indeed, $(a_\alpha)_{\alpha\in\lambda}$ is the zero element.
\end{proof}
It is very sad that colimits preserve exactness of discrete categories and directed sets, but not in general. These are perhaps our ``trial-run'' categories, so something very weird is happening to cause this to fail in general.

\subsection{The Mittag-Leffler Condition}
As in our general set-up with limits, we have $\mathcal I$ our index category and a given short exact sequence
\[0\to A_I\stackrel{\iota_I}\to B_I\stackrel{\pi_I}\to C_I\to0\]
for each $I\in\mathcal I,$ such that, for each $f:I\to J$ in $\mathcal I,$ we have $B_f\circ\iota_I=\iota_J\circ A_f$ and $C_f\circ\pi_I=\pi_J\circ B_f.$
Because limits commute with limits and hence with kernels, we at least know that
\[0\to\limit A_k\to\limit B_k\to\limit C_k\]
is exact, so we are worried about the surjectivity of the last map.

For example, do limits preserve exactness over ``co-directed'' categories? The answer is no.
% \begin{exercise}
% 	Limits do not necessarily preserve exactness over 
% \end{exercise}
\begin{example}
	Of course, we start with our favorite sequence
	\[0\to\ZZ\stackrel{\times2}\to\ZZ\to\ZZ/2\ZZ\to0\]
	and place all these sequences into a vertical sequence, where the vertical maps are multiplication by 3.
	% https://q.uiver.app/?q=WzAsMTgsWzAsMSwiMCJdLFsxLDEsIlxcWloiXSxbMCwyLCIwIl0sWzAsMywiMCJdLFsxLDIsIlxcWloiXSxbMSwzLCJcXFpaIl0sWzIsMSwiXFxaWiJdLFsyLDIsIlxcWloiXSxbMiwzLCJcXFpaIl0sWzMsMSwiXFxaWi8yXFxaWiJdLFszLDIsIlxcWlovMlxcWloiXSxbMywzLCJcXFpaLzJcXFpaIl0sWzQsMSwiMCJdLFs0LDIsIjAiXSxbNCwzLCIwIl0sWzEsMCwiXFx2ZG90cyJdLFsyLDAsIlxcdmRvdHMiXSxbMywwLCJcXHZkb3RzIl0sWzAsMV0sWzIsNF0sWzMsNV0sWzEsNiwiXFx0aW1lczIiXSxbNCw3LCJcXHRpbWVzMiJdLFs1LDgsIlxcdGltZXMyIl0sWzYsOV0sWzcsMTBdLFs4LDExXSxbOSwxMl0sWzEwLDEzXSxbMTEsMTRdLFsxNyw5LCJcXHRpbWVzMyJdLFs5LDEwLCJcXHRpbWVzMyJdLFsxMCwxMSwiXFx0aW1lczMiXSxbMTYsNiwiXFx0aW1lczMiXSxbNiw3LCJcXHRpbWVzMyJdLFs3LDgsIlxcdGltZXMzIl0sWzE1LDEsIlxcdGltZXMzIl0sWzEsNCwiXFx0aW1lczMiXSxbNCw1LCJcXHRpbWVzMyJdXQ==
	\[\begin{tikzcd}
		& \vdots & \vdots & \vdots \\
		0 & \ZZ & \ZZ & {\ZZ/2\ZZ} & 0 \\
		0 & \ZZ & \ZZ & {\ZZ/2\ZZ} & 0 \\
		0 & \ZZ & \ZZ & {\ZZ/2\ZZ} & 0
		\arrow[from=2-1, to=2-2]
		\arrow[from=3-1, to=3-2]
		\arrow[from=4-1, to=4-2]
		\arrow["\times2", from=2-2, to=2-3]
		\arrow["\times2", from=3-2, to=3-3]
		\arrow["\times2", from=4-2, to=4-3]
		\arrow[from=2-3, to=2-4]
		\arrow[from=3-3, to=3-4]
		\arrow[from=4-3, to=4-4]
		\arrow[from=2-4, to=2-5]
		\arrow[from=3-4, to=3-5]
		\arrow[from=4-4, to=4-5]
		\arrow["\times3", from=1-4, to=2-4]
		\arrow["\times3", from=2-4, to=3-4]
		\arrow["\times3", from=3-4, to=4-4]
		\arrow["\times3", from=1-3, to=2-3]
		\arrow["\times3", from=2-3, to=3-3]
		\arrow["\times3", from=3-3, to=4-3]
		\arrow["\times3", from=1-2, to=2-2]
		\arrow["\times3", from=2-2, to=3-2]
		\arrow["\times3", from=3-2, to=4-2]
	\end{tikzcd}\]
	The limit of the $\ZZ$s must be $0$ because no element can be tripled to itself indefinitely, but the limit of the $\ZZ/2\ZZ$s will be nontrivial because multiplication by three is an isomorphism. So the resulting sequence is
	\[0\to 0\to 0\to\text{nontrivial}\to0,\]
	which is sadly not exact.
\end{example}
So when do limits preserve exactness? The answer is the Mittag-Leffler condition. It turns out that here we only care about the $A_\bullet$ instead of trying to care about the $B_\bullet$ or the map $B_\bullet\to C_\bullet,$ which is a testament to short exact sequences caring about all the terms.

We slowly build towards the Mittag-Leffler condition. The set-up is that $\mathcal I=\NN$ and $A_k,B_k,$ and $C_k$ functors giving a commuting sequence of short exact sequences in that, for $k\le\ell,$ the following diagram commutes with exact rows.
% https://q.uiver.app/?q=WzAsMTAsWzAsMCwiMCJdLFsxLDAsIkFfXFxlbGwiXSxbMiwwLCJCX1xcZWxsIl0sWzMsMCwiQ19cXGVsbCJdLFsxLDEsIkFfayJdLFsyLDEsIkJfayJdLFszLDEsIkNfayJdLFs0LDAsIjAiXSxbNCwxLCIwIl0sWzAsMSwiMCJdLFswLDFdLFsxLDIsIlxcaW90YV9cXGVsbCJdLFsyLDMsIlxccGlfXFxlbGwiXSxbMyw3XSxbOSw0XSxbNCw1LCJcXGlvdGFfayJdLFs1LDYsIlxccGlfayJdLFs2LDhdLFsxLDQsImZfa15cXGVsbCIsMl0sWzIsNSwiZ19rXlxcZWxsIiwyXSxbMyw2LCJoX2teXFxlbGwiLDJdXQ==
\[\begin{tikzcd}
	0 & {A_\ell} & {B_\ell} & {C_\ell} & 0 \\
	0 & {A_k} & {B_k} & {C_k} & 0
	\arrow[from=1-1, to=1-2]
	\arrow["{\iota_\ell}", from=1-2, to=1-3]
	\arrow["{\pi_\ell}", from=1-3, to=1-4]
	\arrow[from=1-4, to=1-5]
	\arrow[from=2-1, to=2-2]
	\arrow["{\iota_k}", from=2-2, to=2-3]
	\arrow["{\pi_k}", from=2-3, to=2-4]
	\arrow[from=2-4, to=2-5]
	\arrow["{f_k^\ell}"', from=1-2, to=2-2]
	\arrow["{g_k^\ell}"', from=1-3, to=2-3]
	\arrow["{h_k^\ell}"', from=1-4, to=2-4]
\end{tikzcd}\]
Now here is our starting case.
\begin{exercise} \label{exe:mlsurject}
	Fix everything as above. If the maps $f_k^{k+1}:A_{k+1}\to A_k$ are surjective, then $\limit B_k\to\limit C_k$ is surjective.
\end{exercise}
\begin{proof}
	The idea is to diagram-chase with \autoref{lem:explicitlimit}. Fix any $c:=\{c_k\}_{k\in\NN}\limit C_k$ which we want to hit. We recursively\footnote{It is possible to rigorize the following argument with Zorn's lemma. In short, the partially ordered set is over countable sequences $\{b_k\}_{k=0}^N$ with $N\in\NN\cup\{\infty\}$ such that $\pi_kb_k=c_k$ for each $0\le k\le N.$ The ordering is given by restriction: $\{b_k\}_{k=0}^{N}\preceq\{b'_k\}_{k=0}^{N'}$ if and only if $N\le N'$ and $b_k=b'_k$ for each $0\le k\le N.$ The inductive step shows that maximal elements have $N=\infty.$} construct a sequence $b=\{b_k\}_{k\in\NN}\in\limit B_k$ which hits $c.$

	Our base case is to pick up any $b_0$ which maps to $c_0,$ which exists by exactness of
	\[0\to A_0\to B_0\to C_0\to 0.\]
	For the inductive step, we start with $\{b_k\}_{k=1}^n$ and want to find $b_{n+1}$ which maps down to $b_n$ and across to $c_{n+1}.$ Well, we start by simply picking up any $b'_{n+1}$ which goes to $c_{n+1}.$ Here is our diagram so far.
	% https://q.uiver.app/?q=WzAsNixbMCwxLCJiX24iXSxbMSwxLCJjX24iXSxbMCwwLCJiX3tuKzF9JyJdLFsxLDAsImNfe24rMX0iXSxbMCwyLCJcXHZkb3RzIl0sWzEsMiwiXFx2ZG90cyJdLFswLDEsIlxccGlfbiIsMCx7InN0eWxlIjp7InRhaWwiOnsibmFtZSI6Im1hcHMgdG8ifX19XSxbMywxLCJoX25ee24rMX0iLDAseyJzdHlsZSI6eyJ0YWlsIjp7Im5hbWUiOiJtYXBzIHRvIn19fV0sWzIsMywiXFxwaV97bisxfSIsMCx7InN0eWxlIjp7InRhaWwiOnsibmFtZSI6Im1hcHMgdG8ifX19XSxbMCw0LCIiLDIseyJzdHlsZSI6eyJ0YWlsIjp7Im5hbWUiOiJtYXBzIHRvIn19fV0sWzEsNSwiIiwwLHsic3R5bGUiOnsidGFpbCI6eyJuYW1lIjoibWFwcyB0byJ9fX1dXQ==
	\[\begin{tikzcd}
		{b_{n+1}'} & {c_{n+1}} \\
		{b_n} & {c_n} \\
		\vdots & \vdots
		\arrow["{\pi_n}", maps to, from=2-1, to=2-2]
		\arrow["{h_n^{n+1}}", maps to, from=1-2, to=2-2]
		\arrow["{\pi_{n+1}}", maps to, from=1-1, to=1-2]
		\arrow[maps to, from=2-1, to=3-1]
		\arrow[maps to, from=2-2, to=3-2]
	\end{tikzcd}\]
	Now, the point is that $b_{n+1}\stackrel{\pi_{n+1]}}\mapsto c_{n+1}\stackrel{h_n^{n+1}}\mapsto c_n$ along one side of the diagram, so along the other side of the diagram,
	\[\pi_n(g_n^{n+1}b_{n+1}-b_n)=c_n-c_n=0.\]
	So $g_n^{n+1}b_{n+1}-b_n\in\ker\pi_n=\im\iota_n,$ so there is some $a_n$ such that $\iota_na_n=g_n^{n+1}b_{n+1}-b_n.$

	But now the surjectivity of $A_{n+1}\to A_n$ lets us lift $a_n$ to some $a_{n+1}\in A_{n+1}$ with $f_n^{n+1}a_{n+1}=a_n.$ And lastly, we push $a_{n+1}$ forwards to define
	\[b_{n+1}:=b_{n+1}'-\iota_{n+1}a_{n+1}.\]
	The point is that we still have $\pi_{n+1}b_{n+1}=\pi_{n+1}b'_{n+1}-(\pi_{n+1}\circ\iota_{n+1})a_{n+1}=c_{n+1},$ but now
	\[g_n^{n+1}b_{n+1}=g_n^{n+1}b_{n+1}'(g_n^{n+1}\circ\iota_{n+1})a_{n+1}=g_n^{n+1}b_{n+1}'0(\iota_n\circ f_{n+1})a_{n+1}=g_n^{n+1}b_{n+1}'-\iota_na_n,\]
	which collapses into what we want after plugging in for $\iota_na_n.$ This finishes the inductive step.
\end{proof}
And here is our next case.
\begin{exercise} \label{exe:mlzero}
	Fix everything as in our set-up from earlier. Further suppose that $A_{k+1}\to A_k$ is the zero map for each $k\in\NN.$ Then $\limit B_k\to\limit C_k$ is surjective.
\end{exercise}
\begin{proof}
	Fix any $\{c_n\}_{n\in\NN}\in\limit C_k$ which we want to hit. For each $n\in\NN,$ find any $b_{n+2}\in B_{n+2}$ such that $\pi_{n+2}b_{n+2}=c_{n+2}.$ Similarly, find any $b_{n+1}\in B_{n+1}$ such that $\pi_{n+1}b_{n+1}=c_{n+1}.$ Then the main claim is that
	\[g^{n+2}_nb_{n+2}\stackrel?=g^{n+1}_nb_{n+1}.\]
	Here is the diagram.
	% https://q.uiver.app/?q=WzAsNSxbMCwwLCJiX3tuKzJ9Il0sWzAsMSwiYl97bisxfSJdLFsxLDAsImNfe24rMn0iXSxbMSwxLCJjX3tuKzF9Il0sWzAsMiwiXFxidWxsZXQiXSxbMiwzLCJoX3tuKzF9XntuKzJ9IiwwLHsic3R5bGUiOnsidGFpbCI6eyJuYW1lIjoibWFwcyB0byJ9fX1dLFsxLDQsImdfbl57bisxfSIsMCx7InN0eWxlIjp7InRhaWwiOnsibmFtZSI6Im1hcHMgdG8ifX19XSxbMCw0LCJnX25ee24rMn0iLDIseyJjdXJ2ZSI6Miwic3R5bGUiOnsidGFpbCI6eyJuYW1lIjoibWFwcyB0byJ9fX1dLFsxLDMsIlxccGlfe24rMX0iXSxbMCwyLCJcXHBpX3tuKzJ9Il1d
	\[\begin{tikzcd}
		{b_{n+2}} & {c_{n+2}} \\
		{b_{n+1}} & {c_{n+1}} \\
		\bullet
		\arrow["{h_{n+1}^{n+2}}", maps to, from=1-2, to=2-2]
		\arrow["{g_n^{n+1}}", maps to, from=2-1, to=3-1]
		\arrow["{g_n^{n+2}}"', curve={height=18pt}, maps to, from=1-1, to=3-1]
		\arrow["{\pi_{n+1}}", maps to, from=2-1, to=2-2]
		\arrow["{\pi_{n+2}}", maps to, from=1-1, to=1-2]
	\end{tikzcd}\]
	The claim is equivalent to evaluating $g_n^{n+1}\left(g^{n+2}_{n+1}b_{n+2}-b_{n+1}\right).$ Well, we see that
	\[\pi_{n+1}\left(g^{n+2}_{n+1}b_{n+2}-b_{n+1}\right)=\left(\pi_{n+2}\circ h_{n+1}^{n+2}\right)(b_{n+2})-\pi_{n+1}b_{n+1}=c_{n+1}-c_{n+1}=0.\]
	It follows that $g^{n+2}_{n+1}b_{n+2}-b_{n+1}\in\im\iota_{n+1}$ by exactness. But now we see that
	\[g_n^{n+1}(\im\iota_{n+1})=\im g_{n}^{n+1}\circ\iota_{n+1}=\im\iota_n\circ f^{n+1}_n=\iota_n(\im f^{n+1}_n)=\iota_n(\{0\})=\{0\},\]
	so indeed, we have that
	\[g_n^{n+1}\left(g^{n+2}_{n+1}b_{n+2}-b_{n+1}\right)=0,\]
	which is what we needed.

	With the claim finished, we are primed to give the proof. the point is that, we can pick any $\{b'_k\}_{k\in\NN}$ such that $\pi_kb'_k=c_k$ for each $k\in\NN.$ Now the final trick is to set
	\[b_n:=g_n^{n+1}b'_{n+1},\]
	for each $n\in\NN.$ Even though the original $b'_\bullet$ sequence need not be compatible to live in $\limit B_k,$ we now see that each $n\in\NN$ has
	\[g_n^{n+1}b_{n+1}=g_n^{n+1}\left(g_{n+1}^{n+2}b`_{n+2}\right)=g_n^{n+2}b'_{n+2}=g_n^{n+1}b'_{n+1}=b_n\]
	by the claim from earlier. So indeed, $\{b_k\}_{k\in\NN}\in B$ while
	\[\pi_nb_n=\pi_ng_n^{n+1}b'_{n+1}=h_n^{n+1}\pi_{n+1}b'_{n+1}=\pi_{n+1}c_{n+1}=c_n,\]
	which is exactly what we needed.
\end{proof}
\begin{remark}[Nir]
	We can in fact extend this to merely require $A_\ell\to A_k$ to be the zero map for sufficiently large $\ell,$ given $k.$ This was the way it was presented in lecture, but I did not do this for psychological reasons.
\end{remark}
% \begin{example}
% 	Suppose that, for each $k,$ there is $\ell$ so that $\ell>k$ has $A_\ell\to A_k$ the zero map, which is about as close to the opposite of surjective we could have.
	
% 	Replacing the $\{A_k\}_{k\in\NN}$ with a subsequence, we can just take $A_{k+1}\to A_k$ actually the zero map, simply by skipping some $A_\bullet.$ This does not change the inverse limit. Now again, thinking inductively, suppose we have the following diagram and want to lift up to some $b_3.$
% 	% https://q.uiver.app/?q=WzAsNixbMCwwLCI/Il0sWzEsMCwiY18yIl0sWzEsMSwiY18xIl0sWzEsMiwiY18wIl0sWzAsMiwiYl8wIl0sWzAsMSwiYl8xIl0sWzUsNF0sWzQsM10sWzUsMl0sWzIsM10sWzAsNV0sWzAsMV0sWzEsMl1d
% 	\[\begin{tikzcd}
% 		{?} & {c_2} \\
% 		{b_1} & {c_1} \\
% 		{b_0} & {c_0}
% 		\arrow[from=2-1, to=3-1]
% 		\arrow[from=3-1, to=3-2]
% 		\arrow[from=2-1, to=2-2]
% 		\arrow[from=2-2, to=3-2]
% 		\arrow[from=1-1, to=2-1]
% 		\arrow[from=1-1, to=1-2]
% 		\arrow[from=1-2, to=2-2]
% 	\end{tikzcd}\]
% 	Now pick some $b_2$ which goes into $c_2,$ but we might not have $b_2\to b_1.$ To fix this, we just shift the entire diagram to get the following
% 	\[\begin{tikzcd}
% 		{b_2} & {c_2} \\
% 		{\im b_2} & {c_1} \\
% 		{\im b_1} & {c_0}
% 		\arrow[from=2-1, to=3-1]
% 		\arrow[from=3-1, to=3-2]
% 		\arrow[from=2-1, to=2-2]
% 		\arrow[from=2-2, to=3-2]
% 		\arrow[from=1-1, to=2-1]
% 		\arrow[from=1-1, to=1-2]
% 		\arrow[from=1-2, to=2-2]
% 	\end{tikzcd}\]
% 	This becomes compatible because $A_{k+1}\to A_k$ is the zero map.
% \end{example}
We would like to unify the above two examples. Even though the examples are essentially opposite (trivial cokernel vs. trivial kernel). Regardless, the way to do this is the following somewhat odd condition.
\begin{defi}[Mittag-Leffler condition] \label{defi:ml}
	Suppose we have a sequence of (say) modules $\{A_k\}_{k\in\NN}$ with morphisms $f^\ell_k:A_\ell\to A_k$ for each $\ell\ge k,$ which commute in that $f^m_\ell\circ f^\ell_k=f^m_k$ for each $k\le\ell\le m.$
	
	Now, for each $k,$ we check the sequence
	\[\im(A_k\to A_k),\qquad\im(A_{k+1}\to A_k),\qquad\im(A_{k+2}\to A_k),\qquad\cdots\]
	If the images here stabilize, then we satisfy the \textit{Mittag-Leffler condition}.
\end{defi}
Briefly, we can check \autoref{defi:ml} is satisfied in the given examples: when the $A_{k+1}\to A_k$ are surjective, then the images stabilize to $A_k$; and when the maps are equal, then the images stabilize to $0.$

We remark that because $A_{k+n}\to A_k$ is equal to the composite
\[A_{k+n}\to A_{k+n-1}\to\cdots\to A_{k+1}\to A_k,\]
the sequence
\[\im(A_k\to A_k),\qquad\im(A_{k+1}\to A_k),\qquad\im(A_{k+2}\to A_k),\qquad\cdots\]
is in fact decreasing.

Before doing anything formal, we outline ``where'' this condition is coming from. For each $k,$ set
\[\overline{A_k}=\bigcap_{\ell\in\NN}\im(A_{k+\ell}\to A_k)\]
to be the stable image of our sequence. Now, the point is that we have a system of short exact sequences as follows.
% https://q.uiver.app/?q=WzAsMTgsWzEsMywiXFxvdmVybGluZXtBXzB9Il0sWzIsMywiQV8wIl0sWzMsMywiQV8wL1xcb3ZlcmxpbmV7QV8wfSJdLFsxLDIsIlxcb3ZlcmxpbmV7QV8xfSJdLFsyLDIsIkFfMSJdLFszLDIsIkFfMS9cXG92ZXJsaW5le0FfMX0iXSxbMSwxLCJcXG92ZXJsaW5le0FfMn0iXSxbMiwxLCJBXzIiXSxbMywxLCJBXzIvXFxvdmVybGluZXtBXzJ9Il0sWzAsMSwiMCJdLFswLDIsIjAiXSxbMCwzLCIwIl0sWzQsMSwiMCJdLFs0LDIsIjAiXSxbNCwzLCIwIl0sWzEsMCwiXFx2ZG90cyJdLFsyLDAsIlxcdmRvdHMiXSxbMywwLCJcXHZkb3RzIl0sWzE1LDZdLFsxNiw3XSxbMTcsOF0sWzYsM10sWzcsNF0sWzgsNV0sWzMsMF0sWzQsMV0sWzUsMl0sWzksNl0sWzEwLDNdLFsxMSwwXSxbNiw3XSxbMyw0XSxbMCwxXSxbNyw4XSxbNCw1XSxbMSwyXSxbOCwxMl0sWzUsMTNdLFsyLDE0XV0=
\[\begin{tikzcd}
	& \vdots & \vdots & \vdots \\
	0 & {\overline{A_2}} & {A_2} & {A_2/\overline{A_2}} & 0 \\
	0 & {\overline{A_1}} & {A_1} & {A_1/\overline{A_1}} & 0 \\
	0 & {\overline{A_0}} & {A_0} & {A_0/\overline{A_0}} & 0
	\arrow[from=1-2, to=2-2]
	\arrow[from=1-3, to=2-3]
	\arrow[from=1-4, to=2-4]
	\arrow[from=2-2, to=3-2]
	\arrow[from=2-3, to=3-3]
	\arrow[from=2-4, to=3-4]
	\arrow[from=3-2, to=4-2]
	\arrow[from=3-3, to=4-3]
	\arrow[from=3-4, to=4-4]
	\arrow[from=2-1, to=2-2]
	\arrow[from=3-1, to=3-2]
	\arrow[from=4-1, to=4-2]
	\arrow[from=2-2, to=2-3]
	\arrow[from=3-2, to=3-3]
	\arrow[from=4-2, to=4-3]
	\arrow[from=2-3, to=2-4]
	\arrow[from=3-3, to=3-4]
	\arrow[from=4-3, to=4-4]
	\arrow[from=2-4, to=2-5]
	\arrow[from=3-4, to=3-5]
	\arrow[from=4-4, to=4-5]
\end{tikzcd}\]
Indeed, the maps $A_\ell\to A_k$ for $\ell\ge k$ induce the maps on the left and right, where $\overline{A_\ell}$ maps into $\overline{A_k}$ by the stability.

In fact, the image of $\overline{A_\ell}\to A_k$ is the stabilized image of $A_m\to A_\ell\to A_k$ for $m\ge k,$ which is the stabilized image of $A_m\to A_k,$ which is $\overline{A_k}.$ Thus, the maps on the left of our diagram are all surjective! So our work from \autoref{exe:mlsurject} emerges.

On the other hand, given some $k,$ there is some $\ell\ge k$ so that the image of $A_\ell\to A_k$ is $\overline{A_k}$ by the Mittag-Leffler condition (!), in which case $A_\ell/\overline{A_k}\to A_k/\overline{A_k}$ is the zero map. So the maps on the right of our diagram are all (eventually) zero! Again, again our work from \autoref{exe:mlzero} will come into play.

It turns out that there is a way to meld the given arguments for the left column and right column together to get ths surjectivity of $\limit B_k\to\limit C_k$ from the Mittag-Leffler condition on the middle. Here is the main result, as promised.
\begin{theorem}
	Fix everything as in the set-up from earlier, and suppose that the $\{A_k\}_{k\in\NN}$ satisfy the Mittag-Leffler condition. Then $\limit B_k\to\limit C_k$ is surjective.
\end{theorem}
\begin{proof}
	Omitted; see Lang. %\todo{}
\end{proof}

To finish off, here's a useful case of the Mittag-Leffler condition at work.
\begin{example}
	If all the $A_k$ are finite, then we satisfy the Mittag-Leffler condition, and here we do indeed need the full Mittag-Leffler condition. Essentially this is because
	\[A_k\supseteq\im (A_{k+1}\to A_k)\supseteq\im (A_{k+1}\to A_k)\supseteq\im (A_{k+3}\to A_k)\supseteq\cdots\]
	is a decreasing sequence of finite groups and hence must stabilize.
\end{example}

\subsection{Combining Limits and Colimits}
Let's do some more abstract category theory.
\begin{remark}
	Professor Borcherds is quite aware how much everyone loves category theory.
\end{remark}
Recall that limits preserve limits and colimits preserve colimits. However, limits do not necessarily preserve colimits. For example, limits did not preserve right exactness.

However, there is something present.
\begin{proposition}
	Given index categories $\mathcal I$ and $\mathcal J$ with a functor $F:\mathcal I\times \mathcal J\to\mathcal C,$ then there is a natural map
	\[\colimit_{\mathcal I}\limit_{\mathcal J} F(I,J)\to\limit_{\mathcal J}\colimit_{\mathcal I}F(I,J).\]
	The direction of the arrows here matters significantly.
\end{proposition}
\begin{proof}
	Fix some objects $i\in\mathcal I$ and $j\in\mathcal J.$ We start with our maps promised by the colimit, which are
	\[F(i,j)\to\colimit_{I\in \mathcal I}F(I,j).\]
	Taking the limit over $\mathcal J,$ we see that these maps induce\footnote{Technically we have to show that maps we provided commute with the internal maps of the system $\colimit_{I\in\mathcal I}F(I,j).$ I am going to ignore these sorts of checks for this proof.} a map
	\[F(i,j)\to\limit_{J\in\mathcal J}\colimit_{I\in\mathcal I}F(I,J).\]
	Additionally, we have a map $\limit_{J\in\mathcal J}F(i,J)\to F(i,j)$ promised by the limit, so we have the composites
	\[\limit_{J\in\mathcal J}F(i,J)\to F(i,j)\to\limit_{J\in\mathcal J}\colimit_{I\in\mathcal I}F(I,J).\]
	So now we have these maps into an object for each $i\in\mathcal I,$ so we may assemble these into a map
	\[\colimit_{I\in\mathcal I}\limit_{J\in \mathcal J}F(I,J)\to\limit_{J\in\mathcal J}\colimit_{I\in\mathcal I}F(I,J),\]
	which is what we wanted.
\end{proof}
Let's have an explicit example.
\begin{example}
	We work in the category of sets, where $\mathcal I=\{0,1\}$ is a category with no maps, where colimits are the coproduct $\sqcup.$ We take $\mathcal J=\{a,b\}$ as the same category so that limits are the product is $\times.$ Now given four sets $S_{0a},S_{0b},S_{1a},S_{1b}.$ Now
	\[\limit_{\mathcal J}\colimit_{\mathcal I}S_{ij}\cong\limit_{\mathcal J}(S_{0j}\sqcup S_{1j})\cong(S_{0a}\sqcup S_{1a})\times(S_{0b}\sqcup S_{1b})\]
	while
	\[\colimit_{\mathcal I}\limit_{\mathcal J}S_{ij}\cong\colimit_{\mathcal I}(S_{ia}\times S_{ib})\cong(S_{0a}\times S_{0b})\sqcup(S_{1a}\times S_{1b}).\]
	These are not equal most of the time for size reasons (e.g., make all sets have size $4,$ and then $4\cdot4+4\cdot4<(4+4)(4+4)$), though there is an inclusion map upwards by assembling $S_{0a}\times S_{0b}\into(S_{0a}\sqcup S_{1a})\times(S_{0b}\sqcup S_{1b})$ and $S_{1a}\times S_{1b}\into(S_{0a}\sqcup S_{1a})\times(S_{0b}\sqcup S_{1b}).$
\end{example}