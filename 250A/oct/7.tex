% !TEX root = ../notes.tex
















There must be some way out of here.

\subsection{Tensor Products over Abelian Groups}
Today we're going to define tensor products. For now, we work in the category of abelian groups.

To begin with we have the following warning.
\begin{warn}
	% Given two abelian groups $A,B,$ the tensor product is not $A\oplus B,$ which is universal (as a colimit) with respect to the following diagram.
	% % https://q.uiver.app/?q=WzAsMyxbMCwxLCJBIl0sWzEsMCwiQiJdLFsxLDEsIkFcXG9wbHVzIEIiXSxbMSwyXSxbMCwyXV0=
	% \[\begin{tikzcd}
	% 	& B \\
	% 	A & {A\oplus B}
	% 	\arrow[from=1-2, to=2-2]
	% 	\arrow[from=2-1, to=2-2]
	% \end{tikzcd}\]
	The tensor product is not $A\times B,$ and in fact has little to do with this product. Instead, the tensor product is arguably closer to $A\oplus B.$
	% which is universal (as a limit) with respect to the following diagram.
	% % https://q.uiver.app/?q=WzAsMyxbMCwxLCJBIl0sWzEsMCwiQiJdLFswLDAsIkFcXHRpbWVzIEIiXSxbMiwwXSxbMiwxXV0=
	% \[\begin{tikzcd}
	% 	{A\times B} & B \\
	% 	A
	% 	\arrow[from=1-1, to=2-1]
	% 	\arrow[from=1-1, to=1-2]
	% \end{tikzcd}\]
\end{warn}
Anyways, the main idea behind tensor products is the following.
\begin{idea}
	Tensor products turn bilinear maps $A\times B\to C$ into linear maps from $A\otimes B\to C.$
\end{idea}
In particular, we should start by defining ``bilinear.''
\begin{definition}[Bilinear]
	Fix abelian groups $A,B,C.$ A map $\varphi:A\times B\to C$ is \textit{bilinear} if and only if
	\[f(a_1+a_2,b)=f(a_1,b)+f(a_2,b)\qquad\text{and}\qquad f(a,b_1+b_2)=f(a,b_1)+f(a,b_2)\]
	for each $a_1,a_2,a\in A$ and $b_1,b_2,b\in B.$
\end{definition}
Equivalently, we see that
\[f(a_1+a_2,b)=f(a_1,b)+f(a_2,b)\]
for any $a_1,a_2\in A$ and $b\in B$ is merely asserting that $b\mapsto f(-,b)$ is a function $B\to\op{Hom}(A,C).$ Similarly,
\[f(a,b_1+b_2)=f(a,b_1)+f(a,b_2)\]
for $a\in A$ and $b_1,b_2\in B$ is asserting that $a\mapsto f(a,-)$ is a function $A\to\op{Hom}(B,C).$

We defined ``bilinear'' so that we could define tensor products.
\begin{defi}[Tensor products]
	The \textit{tensor product} $A\otimes B$ of two abelian groups $A$ and $B$ to be ``universal'' as an abelian group equipped with a bilinear map $\iota:A\times B\to A\otimes B.$ Explicitly, for any bilinear map $\varphi:A\times B\to C,$ there exists a unique induced homomorphism (!) $A\otimes B\to C$ making the following diagram commute.
	% https://q.uiver.app/?q=WzAsMyxbMCwwLCJBXFx0aW1lcyBCIl0sWzEsMSwiQyJdLFsxLDAsIkFcXG90aW1lcyBCIl0sWzAsMSwiXFx0ZXh0e2JpbGluZWFyfSIsMV0sWzIsMSwiIiwxLHsic3R5bGUiOnsiYm9keSI6eyJuYW1lIjoiZGFzaGVkIn19fV0sWzAsMiwiXFxpb3RhIl1d
	\[\begin{tikzcd}
		{A\times B} & {A\otimes B} \\
		& C
		\arrow["{\varphi\text{ bilinear}}"{description}, from=1-1, to=2-2]
		\arrow[dashed, from=1-2, to=2-2]
		\arrow["\iota", from=1-1, to=1-2]
	\end{tikzcd}\]
\end{defi}
We remark that tensor products are unique up to isomorphism, using a fairly typical argument.

As usual, we start by showing that tensor products actually exist.
\begin{proposition}
	Given two abelian groups $A$ and $B,$ their tensor product $A\otimes B$ exists.
\end{proposition}
\begin{proof}
	Essentially, we want the ``largest'' abelian group with a bilinear map from $A\times B.$ To start off, we'll say that we send $\iota:(a,b)\mapsto a\otimes b\in A\otimes B,$ and then we mod out by the minimal relations which will make $\iota$ bilinear.
	
	Explicitly, $A\times B$ has a subgroup generated by
	\[G:=\big\langle(a_1,b)+(a_2,b)-(a_1+a_2,b)\qquad\text{and}\qquad(a,b_1)+(a,b_2)-(a,b_1+b_2)\big\rangle.\]
	(For example, take the image from the free abelian group on $(A\times B)^6.$) Now we define
	\[A\otimes B:=\frac{A\times B}G.\]
	It remains to show the universal property. Well, suppose we have a bilinear map $\varphi:A\times B\to C.$ Then by hypothesis on $\varphi,$ we know that
	\[\varphi\big((a_1,b)+(a_2,b)-(a_1+a_2,b)\big)=0\qquad\text{and}\qquad\varphi\big((b_1,a)+(b_2,a)-(b_1+b_2,a)\big)=0.\]
	As these elements generate $G,$ we see $G\subseteq\ker\varphi,$ so we have a unique induced homomorphism from $A\otimes B=(A\times B)/G$ to $C,$ as requested.
\end{proof}
The above proof does establish existence, but, as seems to be the case a lot, we have just taken a huge thing modulo a huge thing, so it is not even obvious if the tensor product is nonzero zero. And (unlike with localization) there is actually danger here! For example,
\[\ZZ/91\ZZ\otimes\ZZ/119\ZZ\ne0\qquad\text{but}\qquad\ZZ/91\ZZ\otimes\ZZ/120\ZZ=0.\]

So in practice, to actually compute the tensor product, we use the universal property and notably not the explicit construction. Here are some examples of doing this by hand.
\begin{proposition}
	Fix $A$ any abelian group. Then $\ZZ\otimes A\cong A.$
\end{proposition}
By symmetry, we remark that $A\otimes\ZZ\cong A$ as well.
\begin{proof}
	The main point is that we can take elements of the form $k\otimes a$ and turn them into $1\otimes a^k,$ which projects nicely into $A.$ Formally, we show that $A$ satisfies the universal property of $\ZZ\otimes A.$ We define the needed inclusion $\iota:\ZZ\times A\to A$ by
	\[(k,a)\mapsto a^k\]
	for any $k\in\ZZ$ and $a\in A.$
	
	Now, for any bilinear map $\varphi:\ZZ\times A\to C,$ we have to show that there exists a unique $\overline\varphi$ making the following diagram commute.
	% https://q.uiver.app/?q=WzAsMyxbMCwwLCJcXG1hdGhiYiBaXFx0aW1lcyBBIl0sWzEsMSwiQyJdLFsxLDAsIkEiXSxbMCwxLCJcXHZhcnBoaSIsMl0sWzIsMSwiXFxvdmVybGluZVxcdmFycGhpIiwwLHsic3R5bGUiOnsiYm9keSI6eyJuYW1lIjoiZGFzaGVkIn19fV0sWzAsMiwiXFxpb3RhIl1d
	\[\begin{tikzcd}
		{\mathbb Z\times A} & A \\
		& C
		\arrow["\varphi"', from=1-1, to=2-2]
		\arrow["\overline\varphi", dashed, from=1-2, to=2-2]
		\arrow["\pi", from=1-1, to=1-2]
	\end{tikzcd}\]
	To start, fix any bilinear map $\varphi:\ZZ\times A\to C$ for some abelian group $C.$ Then, for $a_1,a_2\in A,$ we see
	\[\varphi(1,a_1+a_2)=\varphi(1,a_1)+\varphi(1,a_2)\]
	because $\varphi$ is bilinear, so $\varphi(1,-)\in\op{Hom}(A,C).$ And we can check that the needed diagram commutes because, for any $(k,a)\in\ZZ\times A,$
	\[\varphi(k,a)=\varphi(\underbrace{1+\cdots+1}_k,a)=\underbrace{\varphi(1,a)+\cdots+\varphi(1,a)}_k=k\varphi(1,a)=k\overline\varphi(a)=\overline\varphi\left(a^k\right),\]
	which is $(\overline\varphi\circ\iota)((k,a)),$ as needed.

	We now show uniqueness. Suppose that some $\overline\varphi:A\to C$ makes the given diagram commute. Then we find that
	\[\overline\varphi(a)=(\overline\varphi\circ\iota)(1,a)=\varphi(1,a)\]
	uniquely determines $\overline\varphi.$
\end{proof}
For completeness, we observe that the induced isomorphism $A\cong\ZZ\otimes A$ is by $a\otimes k\mapsto a^k,$ which we see by applying the universal property to the canonical bilinear map $\ZZ\times A\to\ZZ\otimes A.$ The inverse map is $a\mapsto a\otimes1.$
\begin{remark}[Nir] \label{rem:ringtensoristrivial}
	In fact, we remark that a homomorphism of abelian groups $\varphi:A\to B$ will remain unchanged after taking $\ZZ\otimes-$ and applying the above isomorphism. Indeed, the induced morphism $\varphi:\ZZ\otimes A\to\ZZ\otimes B$ is by
	\[\varphi_1(k\otimes a)=k\otimes\varphi(a),\]
	which can be checked to be homomorphic. But applying the isomorphism $x\otimes k\mapsto x^k,$ we get $\varphi_2:A\to B$ which satisfies $\varphi_2\left(a^k\right)=\varphi(a)^k$ for each $a,k,$ which is true and the exact same $\varphi$ morphism we had before.
\end{remark}
\begin{proposition} \label{prop:tensordist}
	Fix abelian groups $A,B,C.$ Then we have the ``distributive'' law
	\[(A\oplus B)\otimes C\cong(A\otimes C)\oplus(B\otimes C).\]
\end{proposition}
\begin{proof}
	The point is that bilinear maps from $(A\oplus B)\times C\to X$ are the same as a pair of bilinear maps $A\times C\to X$ and $B\times C\to X.$ Formally we show that $(A\otimes C)\oplus(B\otimes C)$ satisfies the universal property of $(A\oplus B)\otimes C.$

	To start off, we note that we have the map $\iota:(A\oplus B)\times C\to(A\otimes C)\oplus(B\otimes C)$ by
	\[\iota:((a,b),c)\mapsto(a\otimes c,b\otimes c).\]
	This map is bilinear, roughly by construction. For example,
	\[\iota((a_1,b_1)+(a_2,b_2),c)=(a_1\otimes c+a_2\otimes c,b_1\otimes c+b_2\otimes c)=(a_1\otimes c,b_1\otimes c)+(a_2\otimes c,b_2\otimes c),\]
	and the other side is similar.

	It remains to show the universal property. Suppose that we have any bilinear map $\varphi:(A\oplus B)\times C\to X$ so that we want to exhibit a unique linear map $\overline\varphi:(A\otimes C)\oplus(B\otimes C)\to X$ making the following diagram commute.
	% https://q.uiver.app/?q=WzAsMyxbMCwwLCIoQVxcb3BsdXMgQilcXHRpbWVzIEMiXSxbMSwwLCIoQVxcb3RpbWVzIEMpXFxvcGx1cyhCXFxvdGltZXMgQykiXSxbMSwxLCJYIl0sWzAsMSwiXFxpb3RhIl0sWzAsMiwiXFx2YXJwaGkiLDFdLFsxLDIsIlxcb3ZlcmxpbmVcXHZhcnBoaSIsMSx7InN0eWxlIjp7ImJvZHkiOnsibmFtZSI6ImRhc2hlZCJ9fX1dXQ==
	\[\begin{tikzcd}
		{(A\oplus B)\times C} & {(A\otimes C)\oplus(B\otimes C)} \\
		& X
		\arrow["\iota", from=1-1, to=1-2]
		\arrow["\varphi"{description}, from=1-1, to=2-2]
		\arrow["\overline\varphi"{description}, dashed, from=1-2, to=2-2]
	\end{tikzcd}\]
	We start by showing the uniqueness of $\overline\varphi.$ Indeed, for any $a\in A,b\in B,c\in C,$ we can push $((a,b),c)\in(A\oplus B)\times C$ through the diagram to see the following.
	% https://q.uiver.app/?q=WzAsMyxbMCwwLCJcXGJpZygoYSxiKSxjXFxiaWcpIl0sWzEsMCwiKGFcXG90aW1lcyBjLGJcXG90aW1lcyBjKSJdLFsxLDEsIlxcdmFycGhpXFxiaWcoKGEsYiksY1xcYmlnKSJdLFswLDEsIlxcaW90YSIsMCx7InN0eWxlIjp7InRhaWwiOnsibmFtZSI6Im1hcHMgdG8ifX19XSxbMCwyLCJcXHZhcnBoaSIsMSx7InN0eWxlIjp7InRhaWwiOnsibmFtZSI6Im1hcHMgdG8ifX19XSxbMSwyLCJcXG92ZXJsaW5lXFx2YXJwaGkiLDEseyJzdHlsZSI6eyJ0YWlsIjp7Im5hbWUiOiJtYXBzIHRvIn0sImJvZHkiOnsibmFtZSI6ImRhc2hlZCJ9fX1dXQ==
	\[\begin{tikzcd}
		{\big((a,b),c\big)} & {(a\otimes c,b\otimes c)} \\
		& {\varphi\big((a,b),c\big)}
		\arrow["\iota", maps to, from=1-1, to=1-2]
		\arrow["\varphi"{description}, maps to, from=1-1, to=2-2]
		\arrow["\overline\varphi"{description}, dashed, maps to, from=1-2, to=2-2]
	\end{tikzcd}\]
	Namely, we must have
	\[\overline\varphi(a\otimes c,b\otimes c)=\varphi((a,b),c)\]
	for each $a\in A,b\in B,c\in C.$ It follows that
	\[\overline\varphi(a\otimes c_1,b\otimes c_2)=\overline\varphi(a\otimes c_1,b\otimes c_1)+\overline\varphi(a\otimes0,b\otimes(c_2-c_1))=\varphi((a,b),c_1)+\varphi((0,b),c_2-c_1),\]
	so indeed, $\overline\varphi$ is uniquely determined. More simply this is $\varphi((a,0),c_1)+\varphi((0,b),c_2)$ after some rearranging.
	
	It remains to show that $\overline\varphi$ is actually well-defined. Well, by projecting on the $a$ coordinate, we see that $\varphi$ induces a bilinear map $\varphi_A:A\times C\to X$ by
	\[\varphi_A(a,c)=\varphi((a,0),c).\]
	Similarly, we get a bilinear map $\varphi_B:B\times C\to X$ by $\varphi_B(b,c)=\varphi((0,b),c).$ We will not check that these are bilinear explicitly.

	The point is that our bilinear maps $\varphi_A$ and $\varphi_B$ induces linear maps $\overline{\varphi_A}:A\otimes C\to X$ (by $a\otimes c\mapsto\varphi((a,0),c)$) and $\overline{\varphi_B}:B\otimes C\to X$ (by $b\otimes c\mapsto\varphi((0,b),c)$), so we have the following diagram.
	% https://q.uiver.app/?q=WzAsNCxbMSwwLCJBXFxvdGltZXMgQyJdLFswLDEsIkJcXG90aW1lcyBDIl0sWzEsMSwiKEFcXG90aW1lcyBDKVxcb3BsdXMoQlxcb3RpbWVzIEMpIl0sWzIsMiwiWCJdLFswLDMsIlxcb3ZlcmxpbmV7XFx2YXJwaGlfQX0iLDAseyJjdXJ2ZSI6LTJ9XSxbMCwyLCJcXGlvdGFfQSIsMl0sWzEsMiwiXFxpb3RhX0IiXSxbMSwzLCJcXG92ZXJsaW5le1xcdmFycGhpX0J9IiwyLHsiY3VydmUiOjJ9XSxbMiwzLCJcXG92ZXJsaW5lXFx2YXJwaGkiLDEseyJzdHlsZSI6eyJib2R5Ijp7Im5hbWUiOiJkYXNoZWQifX19XV0=
	\[\begin{tikzcd}
		& {A\otimes C} \\
		{B\otimes C} & {(A\otimes C)\oplus(B\otimes C)} \\
		&& X
		\arrow["{\overline{\varphi_A}}", curve={height=-12pt}, from=1-2, to=3-3]
		\arrow["{\iota_A}"', from=1-2, to=2-2]
		\arrow["{\iota_B}", from=2-1, to=2-2]
		\arrow["{\overline{\varphi_B}}"', curve={height=12pt}, from=2-1, to=3-3]
		\arrow["\overline\varphi"{description}, dashed, from=2-2, to=3-3]
	\end{tikzcd}\]
	Namely, we have an induced $\overline\varphi$ defined by
	\[\overline\varphi(a\otimes c_1,b\otimes c_2)=\overline{\varphi_A}(a\otimes c_1)+\overline{\varphi_B}(b\otimes c_2)=\varphi_A(a,c_1)+\varphi_B(b,c_2),\]
	which is indeed $\varphi_A((a,0),c_1)+\varphi_B((0,b),c_2).$ So this map does exist.
\end{proof}
\begin{remark}[Nir]
	The second part of the proof can be stated in terms of bijections between $\op{Hom}$ sets and show the uniqueness and existence simultaneously. However, the above proof feels more concrete to me.
\end{remark}
\begin{example}
	We have that
	\[\ZZ^m\otimes\ZZ^n=(\underbrace{\ZZ\oplus\cdots\oplus\ZZ}_m)\otimes\ZZ^n\cong\underbrace{\left(\ZZ\otimes\ZZ^n\right)\oplus\cdots\oplus\left(\ZZ\otimes\ZZ^n\right)}_m\cong\left(\ZZ^n\right)^m\cong\ZZ^{mn}.\]
\end{example}

\subsection{Tensor Is Right Exact}
In general, we might want to compute tensor products with quotients. This would involve taking the short exact sequence
\[0\to A\to B\to C\to 0\]
to a sequence
\[0\to A\otimes M\to B\otimes M\to C\otimes M\to0.\]
The best possible world would make this sequence short exact. Well, at least part of the sequence is exact.
\begin{theorem} \label{prop:tensorrightexact}
	If
	\[0\to A\to B\to C\to0\]
	is a short exact sequence, then
	\[A\otimes M\to B\otimes M\to C\otimes M\to0\]
	is exact.
\end{theorem}
\begin{proof}
	This is difficult to do with the specific construction we provided for the tensor product. So we use category theory, which makes this result trivial but not obvious. The main point is the following lemma.
	\begin{lemma} \label{lem:tensorisleft}
		Fix $B$ an abelian group. Then the tensor functor $-\otimes B$ is left adjoint to the hom functor $\op{Hom}(B,-).$
	\end{lemma}
	\begin{proof}
		We note that $-\otimes B$ is actually a functor because a map $f:A_1\to A_2$ will induce a map $\varphi:A_1\otimes B\to A_2\otimes B$ by
		\[a_1\otimes b\mapsto f(a_1)\otimes b.\]
		Less explicitly, we have a bilinear map defined as the composite $A_1\times B\stackrel f\to A_2\times B\to A_2\otimes B,$ which will induce a map $A_1\otimes B\to A_2\otimes B,$ defined as above.

		Anyways, the main idea for the adjunction is that, for any abelian groups $A,C,$
		\[\op{Hom}(A\otimes B,C)\cong\op{Bilinear}(A\times B,C)=\op{Hom}(A,\op{Hom}(B,C)),\]
		where the last step is by currying, and these isomorphisms are exactly we need for the lemma. We will establish these isomorphisms, but we will not actually show the coherence laws for the adjunction because I'm lazy.
		
		Namely, linear maps $A\otimes B\to C$ are in canonical bijection with bilinear maps $A\times B\to C$ by definition of $\otimes.$ In fact this is a group isomorphism, where the operation on $\op{Bilinear}(A\times B,C)$ is pointwise addition. Indeed, we are homomorphic because $\varphi_1,\varphi_2\in\op{Bilinear}(A\times B,C)$ have
		\[(\overline{\varphi_1}+\overline{\varphi_2})(a\otimes b)=\overline{\varphi_1}(a\otimes b)+\overline{\varphi_2}(a\otimes b)=(\varphi_1+\varphi_2)(a,b).\]
		This establishes the isomorphism $\op{Hom}(A\otimes B,C)\cong\op{Bilinear}(A\times B,C).$

		Now, currying says that bilinear maps $\varphi:A\times B\to C$ are really curried homomorphisms: given $a\in A,$ define $\varphi_a\in\op{Hom}(B,C)$ by $\varphi_a(b):=\varphi(a,b).$ Then $\varphi_a$ is indeed in $\op{Hom}(B,C)$ because
		\[\varphi_a(b_1+b_2)=\varphi(a,b_1+b_2)=\varphi(a,b_1)+\varphi(a,b_2)=\varphi_a(b_1)+\varphi_a(b_2).\]
		But further, the map $a\mapsto\varphi_a$ is itself a group homomorphism in $\op{Hom}(A,\op{Hom}(B,C))$; indeed, we have
		\[\varphi_{a_1+a_2}(b)=\varphi(a_1+a_2,b)=\varphi(a_1,b)+\varphi(a_2,b)+\varphi_{a_1}(b)+\varphi_{a_2}(b).\]
		So we have a map of sets $\op{Bilinear}(A\times B,C)\to\op{Hom}(A,\op{Hom}(B,C)).$ In fact, this is homomorphic because the sum $\varphi_1+\varphi_2$ will have
		\[\big((\varphi_1)_a+(\varphi_2)_a\big)(b)+(\varphi_2)_a(b)=\varphi_1(a,b)+\varphi_2(a,b)=(\varphi_1+\varphi_2)(a,b).\]
		And our map is injective because $\varphi\in\op{Bilinear}(A\times B,C)$ going to the zero map in $\op[Hom](A,\op{Hom}(B,C))$ would mean that $\varphi(a,b)=\varphi_a(b)=0(b)=0$ for each $(a,b)\in A\times B.$ So the set map has trivial kernel.

		Lastly, our maps is surjective because $\varphi_\bullet\in\op{Hom}(A,\op{Hom}(B,C))$ can be induced by a $\varphi\in\op{Bilinear}(A\times B,C)$ by $\varphi(a,b):=\varphi_a(b).$ We see that $\varphi$ is indeed bilinear because
		\[\varphi(a_1+a_2,b)=\varphi_{a_1+a_2}(b)=\varphi_{a_1}(b)+\varphi_{a_2}(b)=\varphi(a_1,b)+\varphi(a_2,b)\]
		and
		\[\varphi(a,b_1+b_2)=\varphi_a(b_1+b_2)=\varphi_a(b_1)+\varphi_a(b_2)=\varphi(a,b_1)+\varphi(a,b_2).\]
		This finishes the isomorphism $\op{Bilinear}(A\times B,C)\cong\op{Hom}(A,\op{Hom}(B,C)).$
	\end{proof}
	Now that we know $-\otimes B$ is a left adjoint, we pick up the following fact about left adjoints.
	\begin{lemma} \label{lem:leftcolimit}
		Left adjoints preserve colimits. In other words, fix categories $\mathcal A,\mathcal B$ and an adjoint pair $F:\mathcal A\to\mathcal B$ and $G:\mathcal B\to\mathcal A.$ Then suppose that we objects $\{A_\alpha\}_{\alpha\in\lambda}$ with commuting maps $\varphi_{\beta\alpha}:A_\alpha\to A_\beta.$ (Given $\alpha,\beta,$ there might be no $\varphi+{\beta\alpha},$ or there might even be multiple.) Then
		\[F\left(\colimit A_\alpha\right)\cong\colimit F(A_\alpha),\]
		supposing that the colimit on the left exists.
	\end{lemma}
	\begin{proof}
		We outline the proof that $F\left(\colimit A_\alpha\right)$ satisfies the universal property of $\colimit F(A_\alpha).$ For concreteness, set $A:=\colimit A_\alpha,$ and let $\iota_\alpha$ be the promised map $A_\alpha\to A.$
		
		Now, fix any object $X\in\mathcal B$ with maps $x_\alpha:FA_\alpha\to X$ which commute with the $\varphi_{\beta\alpha}$ (i.e., $x_\beta\circ F\varphi_{\beta\alpha}=x_\alpha$). Here is our diagram, where we need to show that there is a unique induced arrow.
		% https://q.uiver.app/?q=WzAsNCxbMCwwLCJGQV9cXGFscGhhIl0sWzIsMCwiRkFfXFxiZXRhIl0sWzEsMiwiWCJdLFsxLDEsIkZBIl0sWzAsMSwiRlxcdmFycGhpX3tcXGJldGFcXGFscGhhfSJdLFswLDIsIlxcZ2FtbWFfXFxhbHBoYSIsMix7ImN1cnZlIjoyfV0sWzEsMiwiXFxnYW1tYV9cXGJldGEiLDAseyJjdXJ2ZSI6LTJ9XSxbMCwzLCJGXFxpb3RhX1xcYWxwaGEiXSxbMSwzLCJGXFxpb3RhX1xcYmV0YSIsMl0sWzMsMiwiIiwxLHsic3R5bGUiOnsiYm9keSI6eyJuYW1lIjoiZGFzaGVkIn19fV1d
		\[\begin{tikzcd}
			{FA_\alpha} && {FA_\beta} \\
			& FA \\
			& X
			\arrow["{F\varphi_{\beta\alpha}}", from=1-1, to=1-3]
			\arrow["{x_\alpha}"', curve={height=12pt}, from=1-1, to=3-2]
			\arrow["{x_\beta}", curve={height=-12pt}, from=1-3, to=3-2]
			\arrow["{F\iota_\alpha}", from=1-1, to=2-2]
			\arrow["{F\iota_\beta}"', from=1-3, to=2-2]
			\arrow[dashed, from=2-2, to=3-2]
		\end{tikzcd}\]
		Well, $\op{Hom}(FA,X)\cong\op{Hom}(A,GX)$ (naturally) by the adjunction. But by definition of $A$ as a colimit, we see that $\op{Hom}(A,GX)$ is in natural isomorphism with commuting tuples of morphisms as in
		\[\op{Hom}(A,GX)\cong\left\{\{a_\alpha\}_{\alpha\in\lambda}\in\prod_{\alpha\in\lambda}\op{Hom}(A_\alpha,GX):a_\beta\circ\varphi_{\beta\alpha}=\varphi_\beta\right\}.\]
		But commuting tuples of morphisms in $\op{Hom}(A_\alpha,GX)$ can be pushed back to $\op{Hom}(FA_\alpha,X)$ by the adjunction again, and the fact that the adjunction natural means that the morphisms will commute afterwards as needed. So we have
		\[\left\{\{a_\alpha\}_{\alpha\in\lambda}\in\prod_{\alpha\in\lambda}\op{Hom}(A_\alpha,GX):a_\beta\circ\varphi_{\beta\alpha}=\varphi_\beta\right\}\cong\left\{\{b_\alpha\}_{\alpha\in\lambda}\in\prod_{\alpha\in\lambda}\op{Hom}(FA_\alpha,X):b_\beta\circ F\varphi_{\beta\alpha}=F\varphi_\beta\right\}.\]
		So in total,
		\[\op{Hom}(FA,X)\cong\left\{\{b_\alpha\}_{\alpha\in\lambda}\in\prod_{\alpha\in\lambda}\op{Hom}(FA_\alpha,X):b_\beta\circ F\varphi_{\beta\alpha}=F\varphi_\beta\right\},\]
		which is exactly what we need for $FA$ to be the colimit of the $FA_\alpha.$ This finishes.
	\end{proof}
	And now we can realize right-exactness as a special kind of colimit.
	\begin{lemma}
		Suppose that the functor of abelian categories $F:\mathcal A\to\mathcal B$ preserves colimits. (For example, $F$ might be a left adjoint.) Then $F$ is right exact.
	\end{lemma}
	\begin{proof}
		The point is the right short exact sequence
		% https://q.uiver.app/?q=WzAsNCxbMCwwLCJBJyJdLFsxLDAsIkEiXSxbMiwwLCJBJyciXSxbMywwLCIwIl0sWzAsMSwiXFxpb3RhIl0sWzEsMiwiXFxwaSJdLFsyLDNdXQ==
		\[\begin{tikzcd}
			{A'} & A & {A''} & 0
			\arrow["\iota", from=1-1, to=1-2]
			\arrow["\pi", from=1-2, to=1-3]
			\arrow[from=1-3, to=1-4]
		\end{tikzcd}\]
		is equivalent to saying that $A''$ is the colimit of the following diagram.
		% https://q.uiver.app/?q=WzAsMixbMCwwLCJBJyJdLFsxLDAsIkEiXSxbMCwxLCJcXGlvdGEiLDAseyJvZmZzZXQiOi0xfV0sWzAsMSwiMCIsMix7Im9mZnNldCI6MX1dXQ==
		\[\begin{tikzcd}
			{A'} & A
			\arrow["\iota", shift left=1, from=1-1, to=1-2]
			\arrow["0"', shift right=1, from=1-1, to=1-2]
		\end{tikzcd}\]
		Indeed, the right short exact sequence is equivalent to $A''\cong A/\im\iota$ by using the Homomorphism theorem, and $A/\im\iota$ is the colimit of the above: for any $X$ with maps $A'\to X$ and $A\to X$ making the above commute, surely there is at most one map $A/\im\iota\to X,$ and this map exists because $A\to X=A'\to A\to X$ implies that $A'$ vanishes under $A\to X.$

		Thus, because $F$ preserves colimits, it will preserves quotients in the above way. Explicitly, if
		% https://q.uiver.app/?q=WzAsNCxbMCwwLCJBJyJdLFsxLDAsIkEiXSxbMiwwLCJBJyciXSxbMywwLCIwIl0sWzAsMSwiXFxpb3RhIl0sWzEsMiwiXFxwaSJdLFsyLDNdXQ==
		\[\begin{tikzcd}
			{A'} & A & {A''} & 0
			\arrow["\iota", from=1-1, to=1-2]
			\arrow["\pi", from=1-2, to=1-3]
			\arrow[from=1-3, to=1-4]
		\end{tikzcd}\]
		is right exact, then
		% https://q.uiver.app/?q=WzAsNCxbMCwwLCJBJyJdLFsxLDAsIkEiXSxbMiwwLCJBJyciXSxbMywwLCIwIl0sWzAsMSwiXFxpb3RhIl0sWzEsMiwiXFxwaSJdLFsyLDNdXQ==
		\[\begin{tikzcd}
			{FA'} & FA & {FA''} & 0
			\arrow["\iota", from=1-1, to=1-2]
			\arrow["\pi", from=1-2, to=1-3]
			\arrow[from=1-3, to=1-4]
		\end{tikzcd}\]
		will be right exact.
	\end{proof}
	\begin{remark}
		We also have the dual statement that right adjoints preserves limits, which implies right adjoints preserve left exactness. For example, we could just move everything into an opposite category and repeat the proofs above.
	\end{remark}
	Now \autoref{prop:tensorrightexact} follows by stringing the above lemmas together.
\end{proof}
\begin{remark}
	As promised, category theory is a nice tool for making trivial results trivial. However, it is not obvious that the result is trivial.
\end{remark}
Category theory also gives us some other nice properties. For example, we have the following, practically for free.
\begin{proposition} \label{prop:tensorsums}
	Given abelian groups $\{A_\alpha\}_{\alpha\in\lambda},$ we have
	\[\left(\bigoplus_{\alpha\in\lambda}A_\alpha\right)\otimes B\cong\bigoplus_{\alpha\in\lambda}(A_\alpha\otimes B)\]
\end{proposition}
\begin{proof}
	Direct sums are colimits (where there are no commuting morphisms to worry about), so this follows directly from \autoref{lem:tensorisleft} and \autoref{lem:leftcolimit}.
\end{proof}
\begin{proposition}
	We can show that $\op{Hom}(B,-)$ is left exact.
\end{proposition}
\begin{proof}
	The point is that right adjoints preserve limits, so $\op{Hom}(B,-)$ preserves limits. Then, as before, we show that
	% https://q.uiver.app/?q=WzAsNCxbMSwwLCJBJyJdLFswLDAsIjAiXSxbMiwwLCJBIl0sWzMsMCwiQScnIl0sWzEsMF0sWzIsMywiXFxwaSJdLFswLDIsIlxcaW90YSJdXQ==
	\[\begin{tikzcd}
		0 & {A'} & A & {A''}
		\arrow[from=1-1, to=1-2]
		\arrow["\pi", from=1-3, to=1-4]
		\arrow["\iota", from=1-2, to=1-3]
	\end{tikzcd}\]
	is left exact if and only if $A'$ is the limit of the following diagram.
	% https://q.uiver.app/?q=WzAsMixbMCwwLCJBIl0sWzEsMCwiQScnIl0sWzAsMSwiXFxwaSIsMCx7Im9mZnNldCI6LTF9XSxbMCwxLCIwIiwyLHsib2Zmc2V0IjoxfV1d
	\[\begin{tikzcd}
		A & {A''}
		\arrow["\pi", shift left=1, from=1-1, to=1-2]
		\arrow["0"', shift right=1, from=1-1, to=1-2]
	\end{tikzcd}\]
	Indeed, the left short exact sequence is equivalent to $A'\cong\ker\pi$ by using the Homomorphism theorem. And this is equivalent to being the limit of the above diagram because for any $X$ with maps $X\to A$ and $X\to A''$ causing everything to commute, we see that $X\to A$ must map into $\ker\pi\cong A',$ so the induced map exists and is unique by restricting the image.

	So we see that
	% https://q.uiver.app/?q=WzAsNCxbMSwwLCJBJyJdLFswLDAsIjAiXSxbMiwwLCJBIl0sWzMsMCwiQScnIl0sWzEsMF0sWzIsMywiXFxwaSJdLFswLDIsIlxcaW90YSJdXQ==
	\[\begin{tikzcd}
		0 & {A'} & A & {A''}
		\arrow[from=1-1, to=1-2]
		\arrow["\pi", from=1-3, to=1-4]
		\arrow["\iota", from=1-2, to=1-3]
	\end{tikzcd}\]
	is left exact if and only if $A'$ is the limit of some diagram if and only if $F(A')$ is the limit of a similar diagram if and only if
	% https://q.uiver.app/?q=WzAsNCxbMSwwLCJBJyJdLFswLDAsIjAiXSxbMiwwLCJBIl0sWzMsMCwiQScnIl0sWzEsMF0sWzIsMywiXFxwaSJdLFswLDIsIlxcaW90YSJdXQ==
	\[\begin{tikzcd}
		0 & {FA'} & FA & {FA''}
		\arrow[from=1-1, to=1-2]
		\arrow["F\pi", from=1-3, to=1-4]
		\arrow["F\iota", from=1-2, to=1-3]
	\end{tikzcd}\]
	is left exact.
\end{proof}
Similarly, $\op{Hom}(B,-)$ preserves products in the same way that $-\otimes B$ preserves direct sums.

\subsection{Back To Examples}
Let's go back to examples.
\begin{example}
	To compute $\ZZ/2\ZZ\otimes\ZZ/2\ZZ,$ we look at the short exact sequence
	\[0\to\ZZ\stackrel{\times2}\to\ZZ\to\ZZ\ZZ/2\ZZ\to0.\]
	Taking $-\otimes\ZZ/2\ZZ$ and using \autoref{rem:ringtensoristrivial} to keep track of the morphisms, we get right exact sequence
	\[\ZZ/2\ZZ\stackrel{\times2}\to\ZZ/2\ZZ\to\ZZ/2\ZZ\to0,\]
	so $\ZZ/2\ZZ\otimes\ZZ/2\ZZ\cong\ZZ/2\ZZ.$ (Namely, the $\ZZ/2\ZZ\stackrel{\times2}\to\ZZ/2\ZZ$ at the front is the zero map.)
\end{example}
From the above example, we notice that the full sequence
\[0\to\ZZ/2\ZZ\stackrel{\times2}\to\ZZ/2\ZZ\to\ZZ/2\ZZ\to0\]
is not short exact, at the very least for size reasons but more immediately because the first $\ZZ/2\ZZ\to\ZZ/2\ZZ$ is not injective (it is the zero map). So indeed, tensor products do not preserve left exactness.
\begin{example}
	Let's compute $\ZZ/2\ZZ\otimes\ZZ/3\ZZ.$ Again, take
	\[\ZZ\stackrel{\times2}\to\ZZ\to\ZZ/2\ZZ\to0,\]
	and we apply $-\otimes\ZZ/3\ZZ.$ This gives us
	\[\ZZ/3\ZZ\stackrel{\times2}\to\ZZ/3\ZZ\to\ZZ/2\ZZ\otimes\ZZ/3\ZZ\to0.\]
	However, the $\stackrel{\times2}\to$ is surjective, so $\ZZ/2\ZZ\otimes\ZZ/3\ZZ\cong0.$
\end{example}
So nonzero tensor products can give $0,$ sadly. Here is the general case.
\begin{exercise} \label{prop:tensorcyc}
	Fix $m,n$ positive integers. Then $\ZZ/m\ZZ\otimes\ZZ/n\ZZ\cong\ZZ/\gcd(m,n)\ZZ.$
\end{exercise}
\begin{proof}
	Again, consider the exact sequence
	\[\ZZ\stackrel{\times m}\to\ZZ\to\ZZ/m\ZZ\to0\]
	and apply $-\otimes\ZZ/n\ZZ$ to get
	\[\ZZ/n\ZZ\stackrel{\times m}\to\ZZ/n\ZZ\to\ZZ/m\ZZ\otimes\ZZ/n\ZZ\to0.\]
	Taking the quotient, our tensor product is
	\[\frac{\ZZ/n\ZZ}{\im\stackrel{\times m}\to}\cong\frac{\ZZ/n\ZZ}{m\ZZ/n\ZZ}\cong\frac\ZZ{m\ZZ+n\ZZ}\cong\frac\ZZ{\gcd(m,n)\ZZ},\]
	which is what we wanted.
\end{proof}
This gets us tensor products for finitely generated abelian groups by distributing \autoref{prop:tensordist} repeatedly while applying \autoref{prop:tensorcyc} to each distributed factor. The actual statement is somewhat obnoxious because a prime can appear multiple times, which is annoying to keep track of, so we will not write this out explicitly.

What about groups which are not finitely generated?
\begin{example}
	We compute $\ZZ/n\ZZ\otimes\QQ$ for $n$ a positive integer. Well, take
	\[\ZZ\stackrel{\times n}\to\ZZ\to\ZZ/n\ZZ\to0.\]
	Applying $-\otimes\QQ,$ we get
	\[\QQ\stackrel{\times n}\to\QQ\to\ZZ/n\ZZ\otimes\QQ\to0.\]
	But now $\QQ\stackrel{\times n}\to\QQ$ is surjective, so $\ZZ/n\ZZ\otimes\QQ\cong0.$ In general, $A\otimes\QQ$ for $A$ a finite group will vanish.
\end{example}
However, if we want to work more closely with $\QQ,$ we should realize it as a colimit. We claim that $\QQ$ behaves as the colimit of the system
\[\ZZ\stackrel{\times1}\to\ZZ\stackrel{\times2}\to\ZZ\stackrel{\times3}\to\ZZ\stackrel{\times4}\to\cdots.\]
To see this, observe that this is the same system as
\[\ZZ\to\ZZ\to\frac12\ZZ\to\frac16\ZZ\to\cdots.\]
Indeed, $\QQ$ is the colimit of this system because, for any $A$ with maps $\frac1{n!}\ZZ\to A$ which commute nicely, we can induce the unique map $\QQ\to A$ by taking any $\frac pq$ and running it through $\frac1{q!}\ZZ\to A.$ This map is well-defined because the map $\frac1n\ZZ\to A$ commute nicely.

Namely, if we have an abelian group $A$ and want to compute $A\otimes\QQ,$ then it is the colimit of the diagram
\[A\stackrel{\times1}\to A\stackrel{\times2}\to A\stackrel{\times3}\to A\stackrel{\times4}\to\cdots.\]
Now let's do some computations.
\begin{example}
	We compute $\QQ\otimes\QQ.$ From our work above, this will be the colimit of the diagram
	\[\QQ\stackrel{\times1}\to \QQ\stackrel{\times2}\to \QQ\stackrel{\times3}\to \QQ\stackrel{\times4}\to\cdots.\]
	However, each of the $\stackrel{\times n}\to$ maps are isomorphisms, so we can just embed all these groups into $\QQ.$ Explicitly, for any abelian group $G$ with maps from the above system, we have a unique map $\QQ\to G$ commuting with the above maps by using the leftmost $\QQ\to G,$ and this commutes because we had isomorphisms. So $\QQ\otimes\QQ\cong\QQ.$
\end{example}
The above example was nice because applying $\otimes$ didn't lose injectivity, but we are not always so lucky.
\begin{example}
	If we wanted to compute $\ZZ/2\ZZ\otimes\QQ,$ then we are computing the colimit of the diagram
	\[\ZZ/2\ZZ\stackrel{\times1}\to \ZZ/2\ZZ\stackrel{\times2}\to \ZZ/2\ZZ\stackrel{\times3}\to \ZZ/2\ZZ\stackrel{\times4}\to\cdots,\]
	but every other map is the zero map (and notably not injective!), so we just get $\ZZ/2\ZZ\otimes\QQ\cong0.$
	
	Explicitly, for any abelian group $G$ with maps from the above system, we see that commuting with the zero maps forces each $\ZZ/2\ZZ\to G$ to be the zero map: $\ZZ/2\ZZ\stackrel{\times(2n)}\to\ZZ/2\ZZ\to G$ is the zero map, and then $\ZZ/2\ZZ\stackrel{\times(2n+1)}\to\ZZ/2\ZZ\to G$ is the zero map because $\ZZ/2\ZZ\to G$ is zero by the previous case.
\end{example}
So that covers abelian groups pretty well. Here are some last exercises.
\begin{itemize}
	\item Compute $\QQ\otimes\QQ/\ZZ.$
	\item Compute $\QQ/\ZZ\otimes\ZZ/n\ZZ.$
	\item Compute $\QQ/\ZZ\otimes\QQ/\ZZ.$
\end{itemize}

\subsection{Tensor Products over Commutative Rings}
Our definition in general commutative rings is roughly the same as for abelian groups.
\begin{defi}[Bilinear]
	Fix $R$ a commutative ring and (left) $R$-modules $A,B,C.$ Then $f:A\times B\to C$ is \textit{bilinear} if and only if, for each $a,a_1,a_2\in A$ and $b,b_1,b_2\in B,$ we have
	\[f(a,b_1+b_2)=f(a,b_1)+f(a,b_2)\qquad\text{and}\qquad f(a_1+a_2,b)=f(a_1,b)+f(a_2,b).\]
	Additionally, we require, for each $r\in R,$
	\[f(ra,b)=f(a,rb)=rf(a,b).\]
\end{defi}
Observe that the second condition was automatic for $\ZZ$-modules by inducting off of the first condition. But general rings do not have access to such an induction, so we want to say this explicitly to more closely emulate an $R$-module homomorphism.

Anyways, we define tensor products by universal property again.
\begin{definition}[Tensor products]
	Fix $R$ a commutative ring. Then for $R$-modules $A$ and $B,$ then we take the \textit{tensor product} $A\otimes_RB$ to be ``universal'' as an $R$-module equipped with a bilinear map $\iota:A\times B\to A\otimes B.$ Explicitly, for any bilinear map $\varphi:A\times B\to C,$ there exists a unique induced homomorphism (!) $A\otimes B\to C$ making the following diagram commute.
	% https://q.uiver.app/?q=WzAsMyxbMCwwLCJBXFx0aW1lcyBCIl0sWzEsMSwiQyJdLFsxLDAsIkFcXG90aW1lcyBCIl0sWzAsMSwiXFx0ZXh0e2JpbGluZWFyfSIsMV0sWzIsMSwiIiwxLHsic3R5bGUiOnsiYm9keSI6eyJuYW1lIjoiZGFzaGVkIn19fV0sWzAsMiwiXFxpb3RhIl1d
	\[\begin{tikzcd}
		{A\times B} & {A\otimes B} \\
		& C
		\arrow["{\varphi\text{ bilinear}}"{description}, from=1-1, to=2-2]
		\arrow[dashed, from=1-2, to=2-2]
		\arrow["\iota", from=1-1, to=1-2]
	\end{tikzcd}\]
\end{definition}
We can quickly show that tensor products exist.
\begin{prop}
	Fix $R$ a commutative ring. Then for $R$-modules $A$ and $B,$ $A\otimes_RB$ exists.
\end{prop}
\begin{proof}
	This construction is essentially the same as with abelian groups. Define $N$ as the submodule of $A\oplus B$ generated by the elements
	\[\begin{cases}
		(a,b_1)+(a,b_2)-(a,b_1+b_2), \\
		(b,a_1)+(b,a_2)-(b,a_1+a_2), \\
		(ra,b) - (a,rb),
	\end{cases}\]
	for any $a,a_1,a_2\in A$ and $b,b_1,b_2\in B$ and $r\in R.$ Then we define $A\otimes_RB:=(A\oplus B)/R,$ with an $R$-action defined by $r(a\otimes b):=(ra)\otimes b.$ We will omit the checks here because they are essentially the same as in $\ZZ$-modules, though we do note that we add the condition $(ra,b)=(a,rb)$ because of the added condition to being bilinear.
\end{proof}

We have many of the same properties. We will outline the properties bu tno more; they are pretty much the same as for $\ZZ$-modules.
\begin{proposition}
	Fix $R$ a commutative ring and $A$ an $R$-module. Then we have $A\otimes_R R\cong A.$
\end{proposition}
\begin{proof}[Outline]
	We can show $A$ satisfies the universal property $A\otimes_RR$ in essentially the same way as in $\ZZ.$ At a high level, for any $R$-module $X,$ we see from the adjunction (written out below) that
	\[\op{Hom}_R(A\otimes_RR,X)\cong\op{Hom}_R(A,\op{Hom}_R(R,X)),\]
	but $\op{Hom}_R(R,X)\cong X$ by tracking where $1$ goes. So $\op{Hom}_R(A\otimes_RR,X)\cong\op{Hom}_R(A,X)$ for all $R$-modules $X,$ so we are done by the Yoneda lemma.
\end{proof}
If we actually track everything through, then again, the isomorphism $A\to A\otimes_RR$ is $a\mapsto a\otimes1,$ and the inverse mapping is $a\otimes r\mapsto ra.$
\begin{proposition}
	We have that $-\otimes A$ is left adjoint to $\op{Hom}_R(A,-),$ so $-\otimes_RA$ is right exact.
\end{proposition}
\begin{proof}
	This is essentially the same proof as for abelian groups, so we won't say much here. We will remark that the extra bilinear condition on $B\otimes_RA$ corresponds to needing
	\[\varphi(ra)=r\varphi(a)\]
	for an $R$-module homomorphism $\varphi:A\to B.$
\end{proof}
\begin{example}
	Fix $M$ a module and $I$ an ideal of a commutative ring $R,$ and we compute $M\otimes(R/I).$ For this we have the exact sequence
	\[I\to R\to R/I\to 0\]
	which becomes
	\[M\otimes I\to M\otimes R\to M\otimes(R/I)\to0\]
	after applying $M\otimes-.$ Tracking our quotient through, we see $M\otimes I\to M$ by $m\otimes i\mapsto im,$ which surjects onto $IM,$ so $M\otimes I\cong IM.$ So we have $M\otimes(R/I)\cong M/IM$ here.
\end{example}

\subsection{Tensor Products Over General Rings}
In commutative rings $R,$ we had the very nice property that $M\otimes_RN$ was an $R$-module for $R$-modules $M$ and $N$ by the linearity in the bottom. However, in general rings, the relations
\[rm\otimes n=m\otimes rn=r(m\otimes n)\]
are a bit fuzzy because it moves $r$ from the outside left to the inside left, which are different! So in general rings, we should take
\[m\otimes rn=mr\otimes n,\]
where $M$ is a right $R$-module and $N$ is a left $R$-module, but now there is no good eay to make $M\otimes_RN$ is not an $R$-module, so $M\otimes_RN$ is merely an abelian group. We say this again.
\begin{warn}
	For general rings, the functor $M\otimes_R-$ for general rings takes right $R$-modules to abelian groups, not $R$-modules to $R$-modules.
\end{warn}
But we still have our definition as follows.
\begin{definition}[Tensor product, I]
	Fix $M$ a right $R$-module and $N$ a left $R$-module. The \textit{tensor product} $M\otimes_RN$ takes funny bilinear maps $f:A\times B\to C$ (satisfying $f(ar,b)=f(a,rb)$) to linear maps $f:A\otimes B\to C.$
\end{definition}
If we want to make this good again, we should take bimodules.
\begin{definition}[Tensor products, II]
	Fix $M$ and $N$ $R$-bimodules. Then the \textit{tensor product} $M\otimes_RN$ imposes the conditions
	\[ar\otimes b=a\otimes rb,\qquad r(a\otimes b)=(ra)\otimes b,\qquad a\otimes(br)=(a\otimes b)r,\]
	where the last two laws turn $M\otimes_RN$ into an $R$-bimodule.
\end{definition}
Note that the above roughly just includes the commutative case because right $R$-modules can be turned into right $R$-modules (by $r\cdot m:=mr$) when $R$ is commutative.

\subsection{More Applications and Examples}
Let's have some fun.
\begin{example}
	Fix $k$-vector spaces $V$ and $W,$ and we study $V\otimes W.$ We claim that $W\cong k^{(\dim V)(\dim W)}.$ Tangibly, we can fix bases $\{v_\alpha\}_{\alpha\in I}$ and $\{w_\beta\}_{\beta\in J}$ for $V$ and $W$ respectively, and then $V\otimes W$ will have basis given by
	\[\{v_\alpha\otimes w_\beta\}_{(\alpha,\beta)\in I\times J}.\]
	Checking linear independence is nontrivial, but we can see this because tensor products preserve direct sums, which implies
	\[V\otimes W=\left(\bigoplus_{\alpha\in I}kv_\alpha\right)\otimes\left(\bigoplus_{\beta\in J}kw_\beta\right)\cong\bigoplus_{\alpha\in I}\left(kv_\alpha\otimes\bigoplus_{\beta\in J}kw_\beta\right)\cong\bigoplus_{(\alpha,\beta)\in I\times J)}k(v_\alpha\otimes w_\beta).\]
	There is some work to track through the isomorphisms, but we have more or less done this in the notation above.
\end{example}
\begin{example}
	Fix a (finite-dimensional) $k$-vector space $V,$ and we study $W:=V\otimes V\otimes V\otimes V^*,$ where $V^*$ is the dual space. Then if $V$ has a basis $\{v_k\}_{k=1}^{\dim V},$ then $W$ has a basis
	\[v_a\otimes v_b\otimes v_c\otimes v_*^d,\]
	totaling to a dimension of $(\dim V)^4.$ Again, these elements span $W$ by looking component-wise, and these elements are linearly independent, roughly speaking, because there isn't a way to combine them meaningfully. Alternatively, we could just inductively apply the previous example.
\end{example}
In differential geometry, we might omit everything except the coefficients of the basis in the above example because they are a mess.

Tensor products also help out category theory (which is perhaps unsurprising).
\begin{proposition}
	The coproduct of two commutative rings $R$ and $S$ is $R\otimes S,$ where $R\otimes S$ is a ring with multiplication defined by extending
	\[(r_1\otimes s_1)(r_2\otimes s_2):=(r_1r_2)\otimes(s_1s_2)\]
	linearly.
\end{proposition}
\begin{proof}
	We already have that $R\otimes S$ is an abelian group (because we took the tensor product in $\ZZ$-modules), so checking that it is a ring only needs to worry about the multiplication law. Showing that multiplication is well-defined is surprisingly annoying; we do this in steps.
	\begin{enumerate}[label=(\roman*)]
		\item We know that we have a bilinear map $R\times S\to R\otimes S$ by $(r,s)\mapsto r\otimes s$ is bilinear, and the distributive law in $R$ and $S$ promise that, for given $(r_0,s_0)\in R\times S,$ the map $\mu_{(r_0,s_0)}:(r,s)\mapsto(r_0r)\otimes(s_0s)$ is still bilinear:
		\[\mu_{(r_0,s_0)}(r_1+r_2,s)=(r_0(r_1+r_2))\otimes(s_0s)=(r_0r_1)\otimes(s_0s)+(r_0r_2)\otimes(s_0s)=\mu_{(r_0,s_0)}(r_1,s)+\mu_{(r_0,s_0)}(r_2,s),\]
		and similarly,
		\[\mu_{(r_0,s_0)}(r,s_1+s_2)=(r_0r)\otimes(s_0(s_1+s_2))=(r_0r)\otimes(s_0s_1)+(r_0r)\otimes(s_0s_2)=\mu_{(r_0,s_0)}(r,s_1)+\mu_{(r_0,s_0)}(r,s_2).\]
		\item Because $\mu_{(r_0,s_0)}:R\times S\to R\otimes S$ is bilinear, it induces a linear map $R\otimes S\to R\otimes S$ by $r\otimes s\mapsto(r_0r)\otimes(s_0s).$
		\item In fact, we claim that $(r_0,s_0)\mapsto\mu_{(r_0,s_0)}$ is itself a bilinear map $R\times S\to\op{Hom}(R\otimes S,R\otimes S).$ Indeed, we have to check that
		\[\mu_{(r_1+r_2,s)}(r_0\otimes s_0)=((r_1+r_2)r_0)\otimes (ss_0)=(r_1r_0)\otimes(ss_0)+(r_2r_0)\otimes(ss_0),\]
		and
		\[\mu_{(r,s_1+s_2)}(r_0\otimes s_0)=(rr_0)\otimes ((s_1+s_2)s_0)=(rr_0)\otimes(s_1s_0)+(rr_0)\otimes(s_2s_0)\]
		and then these extend out to all of $R\otimes S.$
		\item So because $\mu_\bullet:R\times S\to\op{Hom}(R\otimes S,R\otimes S)$ is a bilinear map, we have a linear map $R\otimes S\to\op{Hom}(R\otimes S,R\otimes S)$ by
		\[(r_0\otimes s_0)\mapsto((r\otimes s)\mapsto(r_0r\otimes s_0s)),\]
		which is exactly what we wanted.
	\end{enumerate}
	To finish checking that $R\otimes S$ is a ring, associativity is inherited from $R$ and $S.$ Our identity is $1\otimes 1.$ The right distributive law holds because $\mu_\bullet$ is a group homomorphism, and then we can get the left distributive law because multiplication is commutative.
	
	Now we have to actually check that $R\otimes S$ is the coproduct. To start, we see that we have inclusions $\iota_R:R\to R\otimes S$ and $\iota_S:S\to R\otimes S$ by $r\mapsto r\otimes1$ and $s\mapsto s\otimes 1$ respectively. To show the universal property, fix $X$ a ring with maps $\varphi_R:R\to X$ and $\varphi_S:S\to X.$
	% https://q.uiver.app/?q=WzAsNCxbMSwwLCJSIl0sWzAsMSwiUyJdLFsxLDEsIlJcXG90aW1lcyBTIl0sWzIsMiwiWCJdLFswLDIsIlxcaW90YV9SIiwyXSxbMSwyLCJcXGlvdGFfUyJdLFswLDMsIlxcdmFycGhpX1IiLDAseyJjdXJ2ZSI6LTJ9XSxbMSwzLCJcXHZhcnBoaV9TIiwyLHsiY3VydmUiOjJ9XSxbMiwzLCJcXHZhcnBoaSIsMSx7InN0eWxlIjp7ImJvZHkiOnsibmFtZSI6ImRhc2hlZCJ9fX1dXQ==
	\[\begin{tikzcd}
		& R \\
		S & {R\otimes S} \\
		&& X
		\arrow["{\iota_R}"', from=1-2, to=2-2]
		\arrow["{\iota_S}", from=2-1, to=2-2]
		\arrow["{\varphi_R}", curve={height=-12pt}, from=1-2, to=3-3]
		\arrow["{\varphi_S}"', curve={height=12pt}, from=2-1, to=3-3]
		\arrow["\varphi"{description}, dashed, from=2-2, to=3-3]
	\end{tikzcd}\]
	For the universal property, we need to induce $\varphi$ uniquely.
	\begin{itemize}
		\item We start by showing it is unique; it suffices to show that $\varphi(r\otimes s)$ is forced for $r\in R$ and $s\in S.$ Well, we see
		\[\varphi(r\otimes s)=\varphi\big((r\otimes1)\cdot(1\otimes s)\big)=\varphi(r\otimes1)\cdot\varphi(1\otimes s)=(\varphi\circ\iota_R)(r)(\varphi\circ\iota_S)(s)=\varphi_R(r)\varphi_S(s),\]
		which is now indeed forced.
		\item We now show that $\varphi$ exists. Indeed, we note that the maps $\varphi_R$ and $\varphi_S$ induce a bilinear map $(r,s)\mapsto\varphi_R(r)\varphi_S(s)$; we won't write out the check that this is bilinear this time, but it comes from the distributive laws in $X.$
		
		The point is that the bilinear map $R\times S\to X$ induces a linear map $\varphi:R\otimes S\to X$ by
		\[\varphi(r\otimes s)=\varphi_R(r)\varphi_S(s).\]
		We have to actually show that $\varphi$ is a ring map; we are already given that it is a group homomorphism. Then
		\[\varphi((r_1r_2)\otimes(s_1s_2))=\varphi_R(r_1r_2)\varphi_S(s_1s_2)=\big(\varphi_R(r_1)\varphi_S(s_1)\big)\big(\varphi_R(r_2)\varphi_S(s_2)\big)\]
		shows $\varphi$ respects multiplication, and we can see $\varphi(1\otimes1)=\varphi_R(1)\varphi_S(1)=1\cdot1=1,$ so $\varphi$ also preserves the identity. This finishes.
		\qedhere
	\end{itemize}
\end{proof}
Asn an aside, we note that we can do something similar for $R$-algebras.
\begin{defi}[Algebra]
	Given a ring $R,$ an \textit{$R$-algebra} is a commutative ring with an $R$-action.
\end{defi}
\begin{proposition}
	The tensor product is the coproduct in the category of $R$-algebras.
\end{proposition}
\begin{proof}
	We omit this proof because I don't want to think about algebras.
\end{proof}
In algebraic geometry, algebras are roughly schemes, and then the their tensor product is the ``fiber product'' of the schemes.
\begin{remark}
	Rigorizing the above sentence takes about five hours of book-keeping.
\end{remark}
\begin{example}
	Fix $R=\CC$ and $A=\CC[x]$ with $B=\CC[y]$ which are $R$-algebras. Well, these are really $R$-vector spaces, where $A$ has a basis $\left\{x^k\right\}_{k\in\NN}$ and $B$ has a basis $\left\{y^\ell\right\}_{\ell\in\NN},$ so $A\otimes_RB$ has a basis (as an $R$-module) $x^k\otimes y^\ell,$ which is $\CC[x,y].$
\end{example}
Let's keep working with the above example. Taking spectrums, we see
\[\op{Spec}\CC[x]\approx\CC\cup\{\infty\},\]
where $\infty$ corresponds to the zero ideal. What about $\op{Spec}\CC[x,y]$? Well, certainly some of its primes look like $(x-\alpha)$ or $(y-\beta)$ or $(x-\alpha,y-\beta),$ which correspond to $(\alpha,\infty)$ or $(\beta,\infty)$ or $(\alpha,\beta)$ respectively.

But there are lots of other primes in $\op{Spec}\CC[x,y]$ to keep track of. For example,
\[\left(x^2+y^2-1\right)\]
is not from anyone in $\op{Spec}A$ or $\op{Spec}B.$ So in general, we do not always have
\[\op{Spec}A\times\op{Spec}B\stackrel?=\op{Spec}(A\otimes_RB),\]
which is sad. Making these actually equal requires some care to redefine the product on the left.

\subsection{Group Actions}
We continue. Let's talk about representations.
\begin{ex}
	We compute $V:=\CC\otimes_\RR\CC.$ Viewing as an $\RR$-vector space, $V$ has a basis $1\otimes 1$ and $1\otimes i$ and $i\otimes1$ and $i\otimes i.$ Viewing $V$ a ring, the elements
	\[\frac{1\otimes1-i\otimes i}2\qquad\text{and}\qquad\frac{1\otimes i+1\otimes i}2\]
	are orthogonal idempotents (I won't check this explicitly), so we have a decomposition of rings (!)
	\[\CC\otimes_\RR\CC\cong\CC\left[\frac{1\otimes1-i\otimes i}2\right]\oplus\CC\left[\frac{1\otimes i+1\otimes i}2\right],\]
	and we can check that the $\RR$-dimension on both sides is $2\cdot2=2+2.$
\end{ex}
\begin{remark}
	Tensor products of fields like this come up in algebraic number theory quite a bit.
\end{remark}
For our story here, fix $G$ a group which acts on the vector spaces $V$ and $W.$ Then $G$ acts on $V\oplus W$ pointwise, and in fact $G$ acts on $V\otimes W$ by
\[g(v\otimes w)=(gv)\otimes(gw)\]
for $g\in G$ and $v\in V$ and $w\in W.$ Indeed, the map $\mu_g:V\times W\to V\otimes W$ by $(v,w)\mapsto(gv\otimes gw)$ is bilinear because $(v,w)\mapsto(v\otimes w)$ is, and $g\mapsto\mu_g$ is a group homomorphism because $G$ acts on $V$ and $W.$ (We won't write these out.)
\begin{example}
	Take $G=\ZZ/n\ZZ,$ we can take $V=W=\CC$ as $\CC$-vector spaces, where the $G$-action on $V$ is given by $g\cdot z:=ze^{2\pi iag/n}$ for some fixed $a\in\ZZ,$ and the $G$-action on $W$ is given by $g\cdot z:=ze^{2\pi ibg/n}$ for some fixed $b\in\ZZ.$ Then
	\[g(v\otimes w)=(gv)\otimes(gw)=e^{2\pi i(a+b)g/n}(v\otimes w)\]
	is our $G$-action on $V\otimes W.$
\end{example}
Recall from earlier that we had the Burnside ring of (equivalence classes of) sets with a $G$-action. The above ideas let us maybe define an arithmetic on (equivalence classes of) linear representations of $G.$ Here we define
\[V+W:=V\oplus W\qquad\text{and}\qquad V\times W:=V\otimes W.\]
From earlier we had a distributive law
\[(A\oplus B)\otimes C\cong(A\otimes C)\oplus(B\otimes C).\]
We even have nice association
\[(A\otimes B)\otimes C\cong A\otimes(B\otimes C),\]
which is simply by $(a\otimes b)\otimes c\mapsto a\otimes (b\otimes c).$ So we have most of what we need for a ring!

But again, we have no subtraction, but we can do a similar construction as with the Burnside ring, where we just forced a subtraction to exist. This requires some care because it is possible for $A\oplus R\cong B\oplus R$ while $A\not\cong B$ for general rings $R,$ as we saw last class. Regardless, this is still possible, and gives us the representation ring.
\begin{definition}
	Fix a field $k.$ The \textit{representation ring} of a group $G$ is the ring more or less generated by the $k$-linear representations of $G$ with addition given by $\oplus$ and multiplication given by $\otimes.$
\end{definition}