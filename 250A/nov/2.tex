% !TEX root = ../notes.tex


















A few hours grace before the madness begins again.

\subsection{Algebraic Extensions}
So we're talking about fields and Galois theory for the last third of the class. Today we're mostly doing a field review.
\begin{definition}[Field extension]
	A \textit{field extension} $L/K$ is a field $L$ containing a field $K.$
\end{definition}
We are interested in field extensions rather than the field itself because oftentimes we can decompose some complicated field $M$ into its subfields and be able to study $M$ in this more controlled way.

Here is an important invariant.
\begin{definition}[Degree]
	The \textit{degree} of a field extension $L/K,$ denoted $[L:K],$ is the dimension of $L$ as a $K$-vector spaces.
\end{definition}
\begin{remark}
	Yes, field containments induce vector spaces. This is a good thing to check once.
\end{remark}
\begin{example}
	The degree of $\CC/\RR$ is $[\CC:\RR]=2,$ where our basis is (say) $\{1,i\}.$
\end{example}
We have the following definition from this.
\begin{definition}[Algebraic]
	An element $\alpha\in L$ is \textit{algebraic} over $K$ if and only if $\alpha$ is the root of a polynomial in $K[x].$ We say that $L$ is an \textit{algebraic extension over $K$} if all of its elements are algebraic over $K.$
\end{definition}
\begin{example}
	The number $i\in\CC$ is algebraic over $\RR$ and $\QQ.$
\end{example}
\begin{nex}
	The number $\pi$ is not algebraic, and the proof is hard.
\end{nex}
\begin{nex}
	Look at the extension $\QQ\subseteq\QQ(x),$ where $\QQ(x)$ is the field of rational functions. Then, by construction essentially, $x$ is not algebraic over $\QQ.$
\end{nex}
``Not algebraic'' elements have a name.
\begin{definition}[Transcendental]
	An element $\alpha\in L$ which is not algebraic over $K$ is called \textit{transcendental}.
\end{definition}
In Galois theory, we mostly care about finite, algebraic extensions.

\subsection{Constructing Algebraic Extensions}
To construct an algebraic extension, we have the following proposition.
\begin{proposition} \label{prop:constructalg}
	Start with a field $K$ and some polynomial $\pi\in K[x].$ Then it happens that
	\[L:=\frac{K[x]}{(p)}\]
	is a ring, and it is a field if and only if $p$ is irreducible. 
\end{proposition}
\begin{proof}
	Checking that this is a ring is somewhat annoying, so we won't do it here.
	
	The key point to getting a field is that $p$ is irreducible if and only if all nonzero elements have inverses. Indeed, fix $q$ nonzero in $K[x]/p(x).$ Then because $p$ is irreducible and $p\nmid q,$ we have that $p$ and $q$ are coprime, so $(p)+(q)=(1),$ where we are using the fact that $K[x]$ is a principal ideal domain. It follows that there are polynomials $a$ and $b$ such that
	\[ap+bq=1,\]
	so $bq\equiv1\pmod p,$ finishing.

	In the reverse direction, We can ask what happens if $p$ is not irreducible. Well, if we can write $p=fg$ where $f$ and $g$ are coprime, irreducible polynomials, then the Chinese remainder theorem lets us write
	\[\frac{K[x]}{(p)}\cong\frac{K[x]}{(f)}\times\frac{K[x]}{(g)}.\]
	In particular, it follows that this has zero-divisors and hence is not a field.
\end{proof}
\begin{remark}
	The above proof is actually effective for finding inverses: we can use the (extended) Euclidean algorithm to find the $a$ and $b$ such that $ap+bq=1,$ and then we can extract our inverse like that.
\end{remark}
\begin{example}
	We have that $\CC\cong\RR[x]/\left(x^2+1\right).$
\end{example}
We give some remarks in the case where $p$ is not irreducible in \autoref{prop:constructalg}. In general, if
\[p=\prod_{\pi\mid p}\pi^{\alpha_\pi}\]
so that
\[\frac{K[x]}{(p)}\cong\prod_{\pi\mid k}\frac{K[x]}{(\pi^{\alpha_\pi})}.\]
If $\alpha_\pi=1,$ then we get a field, which is nice, and when everything has $\alpha=1,$ then we just have a product of fields. But when $\alpha_\pi>1,$ then we get nilpotent elements, which is very not good.

We also have the following statement.
\begin{proposition}
	Fix $L/K$ an extension. Then $\alpha\in L$ is algebraic over $K$ if and only if $\alpha$ is contained in a finite sub-extension.
\end{proposition}
\begin{proof}
	In one direction, if $\alpha$ is algebraic, then $\alpha$ is the root of some (without loss of generality) irreducible $p\in K[x].$ Then we can place
	\[K[\alpha]\cong\frac{K[x]}{(p)},\]
	which we can place inside of $L,$ and this is our finite extension.

	In the reverse direction, if $\alpha\in M,$ for $[M:K]$ finite, then the infinitely many elements
	\[1,\alpha,\alpha^2,\ldots\]
	cannot all be linearly independent, so there is some linear relation present here.
\end{proof}
To count degrees, we have the following.
\begin{proposition}
	Suppose we have a tower of fields $M/L/K.$ Then
	\[[M:K]=[M:L][L:K].\]
	In other words, the degree is multiplicative.
\end{proposition}
\begin{proof}
	In brief, the idea is to pick a basis $\{a_k\}_{k=1}^m$ for $L/K$ and a basis $\{b_\ell\}_{\ell=1}^n$ for $M/L,$ and we can check that the $\{a_kb_\ell\}$ are a basis for $M/K.$ This gives the result because there are $[M:L]\cdot[L:K]$ basis elements.
\end{proof}
This gives us the following.
\begin{proposition}
	If $\alpha,\beta\in L$ are algebraic over a base field $K,$ then $\alpha+\beta,\alpha\beta,\alpha-\beta,\alpha/\beta$ are also all algebraic, where the last case requires $\beta\ne0.$
\end{proposition}
\begin{proof}
	The point is that we have the following tower of fields.
	% https://q.uiver.app/?q=WzAsMyxbMCwyLCJLIl0sWzAsMSwiS1tcXGFscGhhXSJdLFswLDAsIktbXFxhbHBoYSxcXGJldGFdIl0sWzIsMSwiIiwwLHsic3R5bGUiOnsiaGVhZCI6eyJuYW1lIjoibm9uZSJ9fX1dLFsxLDAsIiIsMCx7InN0eWxlIjp7ImhlYWQiOnsibmFtZSI6Im5vbmUifX19XV0=
	\[\begin{tikzcd}
		{K[\alpha,\beta]} \\
		{K[\alpha]} \\
		K
		\arrow[no head, from=1-1, to=2-1]
		\arrow[no head, from=2-1, to=3-1]
	\end{tikzcd}\]
	The degree of $[K[\alpha]:K]$ is finite by hypothesis, and the degree of $[K[\alpha,\beta]:K]$ is less than $[K[\beta]:K]$ by checking the polynomial, so the entire extension is going to be finite. This gives the result.
\end{proof}
It is actually quite hard to find these polynomials, which is why we are giving these abstract degree arguments.
\begin{example}
	We could try to find the irreducible polynomial
	\[\sqrt2+\sqrt[3]2+\sqrt[5]2,\]
	but it is of degree $30.$
\end{example}
Here are some open problems.
\begin{example}
	We don't know if either $e+\pi$ or $e\pi$ is algebraic, and you'll be very famous if you can solve either of them. Let's solve one of them, but we won't know which. Indeed, $e$ and $\pi$ are roots of the polynomial
	\[x^2-(e+\pi)x+e\pi.\]
	So if $e+\pi$ and $e\pi$ were both algebraic, then we could use the following statement to conclude that both $e$ and $\pi$ would have to be algebraic, which is false.
\end{example}
\begin{remark}
	This argument really annoys intuitionist/constructivist mathematicians because technically we haven't actually showed either $e+\pi$ or $e\pi$ is algebraic.
\end{remark}
\begin{proposition}
	Fix $p(x)\in L[x].$ If the coefficients of $p$ are algebraic over $K,$ then the roots of polynomial are also algebraic.
\end{proposition}
\begin{proof}
	Fix
	\[p(x)=\sum_{k=0}^na_kx^k.\]
	Then let $\alpha$ be a root off $p,$ and we see that the chain
	\[K\subseteq K[a_0]\subseteq K[a_0,a_1]\subseteq\cdots\subseteq K[a_0,\ldots,a_n]\subseteq K[a_0,\ldots,a_n,\alpha]\]
	is a finite chain of finite extensions (the last extension is finite because $p\in K[a_0,\ldots,a_n],$ so we are only adjoining the root $\alpha$ here), so the entire extension is finite, so the final root $\alpha$ in the last field is algebraic.
\end{proof}
\begin{remark}
	Again, it is difficult to find the polynomial in the above argument. For example, we won't try to find the explicit polynomial for a root for $x^3-\sqrt3 x+\sqrt 5.$
\end{remark}

\subsection{Splitting Fields}
Here, suppose that $\pi\in K[x]$ is an irreducible polynomial. Then, looking in
\[L=\frac{K[x]}{(\pi)},\]
we note that $\pi$ has a root in $L,$ but does it fully factor?
\begin{example}
	Take $K=\QQ$ and $p(x)=x^4+1.$ We can check that $p$ is irreducible because $p(x+1)=x^4+4x^3+6x^2+4x+2$ satisfies Eisenstein's criterion. Now, the roots of $p$ are the primitive $8$th roots of $1,$ roughly graphed as follows.
	\begin{center}
		\begin{asy}
			unitsize(1.3cm);
			draw(circle((0,0),1));
			for(int i = 0; i < 4; ++i)
			{
				dot(dir(45+90*i));
			}
			draw((-1.5,0)--(1.5,0));
			draw((0,-1.5)--(0,1.5));
		\end{asy}
	\end{center}
	So if we let $L=\QQ(\zeta)\cong K[x]/(p),$ then we see that the roots of $p$ are simply $\zeta,\zeta^3,\zeta^5,\zeta^7,$ so indeed we get all of our roots.
\end{example}
\begin{example}
	Take $K=\QQ$ and $p(x)=x^3-2,$ which is irreducible because it has no linear factor. Taking $\sqrt[3]2$ to be one of its roots, we have that
	\[L:=\QQ(\sqrt[3]2)\cong\frac{K[x]}{(p)}\]
	does not contain the other roots of $x^3-2.$ Explicitly, the other roots are $\omega\sqrt[3]2$ and $\omega^2\sqrt[3]2$ where $\omega$ is a primitive third root of unity, but $\QQ(\sqrt[3]2)\subseteq\RR$ cannot contain those complex roots.
\end{example}
In the above example, we can manifest the problem by writing
\[x^3-2=\left(x-\sqrt[3]2\right)\left(x^3+\sqrt[3]2-\sqrt[3]4\right)\]
as our irreducible factorization in $x^3-2$ in $\QQ(\sqrt[3]2).$

We would like our polynomials to fully factor, so we have the following definition.
\begin{definition}[Splittin field]
	A \textit{splitting field} of a polynomial $p\in K[x]$ is an extension $L/K$ such that $p$ splits into linear factors in $L,$ and $L$ is actually generated by these roots. I.e., $L$ is the smallest possible field 
\end{definition}
\begin{remark}
	We should probably call this a splitting extension, but so it goes.
\end{remark}
The main theorem here is as follows.
\begin{theorem} \label{thm:splittingfield}
	Splitting fields exist and are isomorphic as $K$-extensions. In other words, given two splitting fields $L_1$ and $L_2,$ there is a field isomorphism $L_1\cong L_2,$ and this field isomorphism is also a $K$-linear map.
\end{theorem}
\begin{example}
	Given $x^4+1\in\QQ[x],$ we have that $\QQ(\zeta_8),$ where $\zeta_8$ is our primitive eight root of unity, is our splitting field.
\end{example}
\begin{example}
	Given $x^3-2\in\QQ[x],$ we have that $\QQ(\sqrt[3]2)$ is not out splitting field because we are still missing the roots $\omega\sqrt[3]2$ and $\omega^2\sqrt[3]2$ in this extension. Namely, we still have to factor the quadratic polynomial in the factorization
	\[x^3-2=\left(x-\sqrt[3]2\right)\left(x^2+\sqrt[3]2x-\sqrt[3]4\right),\]
	which we do by looking at
	\[\frac{\QQ[\sqrt[3]2][x]}{\left(x^2+\sqrt[3]2x-\sqrt[3]4\right)},\]
	which is now a perfectly fine splitting field.
\end{example}
This last example gives the idea behind the proof of \autoref{thm:splittingfield}.
\begin{proof}[Proof of existence in \autoref{thm:splittingfield}]
	We proceed inductively; set $K_0:=K$ and $p:=p_0.$ We start with some polynomial $p.$ If it has no irreducible factors of degree larger than $1,$ then we are done. Otherwise, fix $\pi_1\in K_0[x]$ an irreducible factor of $p$ of degree larger than $1.$ Now we look at
	\[K_1:=\frac{K[\alpha_1]}{\pi_1(\alpha)}.\]
	Now we can factor $p$ with at least one root $\alpha$ from $\pi,$ so $p$ will at least partially factor in $K_1.$ So we can factor
	\[p(x)=(x-\alpha_1)p_1(x),\]
	and because $\deg p_1<\deg p_0.$ Repeating this process (find an irreducible factor $\pi_2$ of $p_1,$ and then look at $K_2:=K_1[\alpha_2]/(\pi_2),$ and continue) makes the degree continue to decrease, so we finish by induction. At each point we are only adding roots to $K_0,$ so we have that this field we made is also only generated by the roots of $p.$
\end{proof}
\begin{remark}
	The above proof in fact gives us a bound of $n!$ for the degree of the splitting field. To be explicit, the extension $[K_{m+1}:K_m]$ will have degree at most $n-m$ because the polynomial $p_m$ at this step has degree $n-m.$
\end{remark}
And here is our uniqueness.
\begin{proof}[Proof of uniqueness in \autoref{thm:splittingfield}]
	Suppose that we have a splitting field $L',$ and we build our own splitting field using the above algorithm with the chain
	\[K\subseteq K_1\subseteq K_2\subseteq\cdots\subseteq L.\]
	Now the point is that $K_1\cong K_0[\alpha_1]/(\pi_1)$ has a root in $L',$ and we can send this $\alpha_1$ to $L'$ to find a subfield of $L$ isomorphic to $K_1.$ Continuing this process will eventually give us a map from our splitting field
\end{proof}
\begin{remark}
	The above proof of uniqueness is somewhat problematic because the isomorphism between splitting fields is not unique, which turns out to cause problems. For example, the math department denotes $\CC$ by $\RR[i],$ and the engineering department denotes this by $\RR[j],$ and the chemistry department denotes this by $\RR[k].$ The issue is that it is very possible for $i=-j,$ and $j=-k,$ but then we need to have $k=i,$ even though there is another isomorphism present.
\end{remark}
\begin{remark}
	The point here is that having a unique isomorphisms are very nice.
\end{remark}

\subsection{Finite Fields}
Now let's do number theory because why else would we study algebra? We start with the following small step.
\begin{proposition}
	Any finite field $F$ contains some finite field $\FF_p,$ for a unique prime $p.$
\end{proposition}
\begin{proof}
	Look at the image of the map $\ZZ\to F.$ The kernel here must be a prime ideal because it quotients into an integral domain, so it is either $(0)$ or $(p),$ but $(0)$ would force $F$ to be infinite. So we have an embedding of $\ZZ/(p)\into F.$
\end{proof}
So $F$ contains $\FF_p$ of finite degree say $n,$ so $F$ will be some $n$-dimensional vector space, so it will have $q:=p^n$ elements.

Now, the main statement is as follows.
\begin{theorem}
	For each prime-power $q,$ there is one finite field of order $q,$ up to (non-unique) isomorphism.
\end{theorem}
\begin{proof}
	The point is that $F,$ being an finite field of order $p^n,$ is equivalent to being the splitting field of the polynomial $x^{p^n}-x.$ This will make the given statement follow from existence and uniqueness of splitting fields.

	In one direction, fix $F$ a splitting field of $f(x):=x^{p^n}-x.$ We would like to show that $F$ has order $p^n.$ For this, we show that
	\[F=\left\{\alpha:\alpha\text{ is a root of }x^{p^n}-x\right\}.\]
	Surely $f(x)=x^{p^n}-x$ has $p^n$ roots because $f'(x)=-1,$ so $f$ has no multiple roots in the algebraic closure. So the above works at least set-theoretically. However, we do need to show that these roots form a subfield structure to show the other inclusion, getting that the field $F$ generated by these roots.
	\begin{itemize}
		\item Given roots $\alpha$ and $\beta,$ we see that $\alpha\beta$ is a root because $(\alpha\beta)^{p^n}=\alpha^{p^n}\beta^{p^n}=\alpha\beta.$
		\item For closure under addition, we fix $\alpha$ and $\beta$ roots, and the point is that
		\[(\alpha+\beta)^p=\sum_{k=0}^p\binom pk\alpha^k\beta^{p-k}=\alpha^p+\beta^p,\]
		where the point is that the middle binomial coefficients vanish$\pmod p.$ Repeating this map enough times, we see that
		\[(\alpha+\beta)^{p^n}=\alpha^{p^n}+\beta^{p^n}=\alpha+\beta,\]
		so we have closure under addition.
		\item $1$ and $0$ are roots by simply plugging them in.
	\end{itemize}
	So indeed, the roots for a subfield of $F$ with $p^n$ elements, so the roots must make up all of $F.$

	Now, in the reverse direction, we need to show that any field $F$ of $p^n$ elements is a splitting field for this polynomial. Well, the point is that each $x\in F$ either has $x=0$ or $x\in F^\times$ so that
	\[x^{p^n-1}-1=0\]
	by Lagrange's theorem. So in all cases, the elements of $F$ are roots of $x^{p^n}-x,$ so indeed $F$ will be generated by these roots and is apparently a field. This finishes.
\end{proof}
\begin{ex}
	We can find a field of order $2^4$ by finding the splitting field of $x^{16}-x$ in $\FF_2.$ How do we factor this polynomial? Well, it factors as
	\[\left(x^4+x+1\right)\left(x^4+x^3+1\right)\left(x^4+x^3+x^2+x+1\right)\left(x^2+x+1\right)\left(x+1\right)x.\]
	We now see that there are three irreducible factors of degree $4,$ and we notice that the finite field
	\[\frac{\FF_2[x]}{\left(x^4+x+1\right)}\]
	will have the required dimension.
\end{ex}
In general, we see from the above that we are really searching for irreducible polynomials of prescribed degree$\pmod p.$ However, proving the existence of such polynomials is somewhat hard.

As an aside, there does not appear to be a ``canonical'' choice for the irreducible polynomial to construct our finite fields. We could just choose according to lexicographic order, but there is no good reason to do this.

More manifestly, we can see this as the fact that there is no good choice for a square root of $-1$ in $\FF_5$; do we choose $2$ or $3$? So essentially this is made worse by the fact that even if we were to choose an irreducible polynomial for $\FF_{16},$ this might not communicate well with the polynomial generating its $\FF_4$ subfield.

We remark that we also have the following statement.
\begin{proposition} \label{prop:factorsplitter}
	In $\FF_p[x],$ we have the irreducible factorization
	\[x^{p^n}-x=\prod_{\substack{\pi\text{ irred.}\\\deg\pi\mid n}}\pi(x).\]
\end{proposition}
\begin{proof}
	We have a few things to show here.
	\begin{itemize}
		\item We show that
		\[\prod_{\substack{\pi\text{ irred.}\\\deg\pi\mid n}}\pi(x)\]
		divides into $x^{p^n}-x.$ Each of these factors are distinct irreducibles and hence coprime, so it suffices to show that, if $\pi\in\FF_p[x]$ is an irreducible polynomial of degree $d\mid n,$ then $\pi(x)\mid x^{p^n}-x.$ Indeed, we see that
		\[\frac{\FF_p[x]}{(\pi)}\]
		is a field with $\FF_{p^d}$ elements and hence a subfield of our field $\FF_{p^n}.$ More explicitly, elements which are roots of $\pi$ will be roots of $x^{p^d}-x,$ which turn into roots of $x^{p^n}-x$ by taking higher powers.

		It follows that all roots of $\pi$ in the algebraic closure are roots of $x^{p^n}-x.$ Thus, $\gcd\left(\pi(x),x^{p^n}-x\right)\in\FF_p[x]$ will be a polynomial with nonzero degree dividing $\pi,$ so it must be equal to $\pi,$ so it follows $\pi\mid x^{p^n}-x.$
		
		\item We can compute the exponent of each irreducible $\pi$ with $\deg\pi\mid n$ dividing into $x^{p^n}-x.$ Indeed, we recall $x^{p^n}-x$ has all of its roots of multiplicity $1,$ so $\pi$ cannot have multiplicity greater than $1$ dividing into $x^{p^n}-x.$

		\item Lastly, we classify the irreducibles dividing into $x^{p^n}-x.$ Namely, if $\pi$ is some irreducible dividing $x^{p^n}-x,$ then we show that the $d:=\deg\pi$ must divide into $n.$ Indeed, fixing any root $\alpha$ of $\pi,$ we see that $\alpha$ is a root of $x^{p^n}-x,$ so $\alpha\in\FF_{p^n}.$ But also
		\[\FF_p[\alpha]\cong\frac{\FF_p[x]}{(\pi)}\]
		is a field of size $p^d,$ so it follows from degree arguments that $d\mid n.$
		\qedhere
	\end{itemize}
\end{proof}
This lets us answer fun questions.
\begin{example}
	We can compute the number of irreducible polynomials $N_d$ of degree $d$ in $\FF_2[x].$ For example, summing over the degrees given in the factorization of \autoref{prop:factorsplitter}, we have
	\[2^6=6N_6+3N_3+2N_2+1N_1,\]
	which gives $N_6=9.$ One could imagine doing this recursively to solve for the number of irreducibles of degree $6.$
\end{example}
\begin{remark}[Nir]
	More generally, in $\FF_p[x],$ we can let $N_d$ be the umber of irreducible polynomials of degree $d$ so that the factorization in \autoref{prop:factorsplitter} implies
	\[p^n=\sum_{d\mid n}dN_d.\]
	Applying M\"obius inversion to this implies the ``prime number theorem in $\FF_p[x]$'' by
	\[N_n=\frac1n\sum_{d\mid n}\mu(d)p^{n/d}=\frac{p^n}n+O\left(\frac{p^{n/2}}n\right).\]
	We also remark that, if $d$ is the largest squarefree divisor of $n$ (so that all squarefree divisors of $n$ divide into $d$), then
	\[nN_n\equiv\mu(d)p^{n/d}\not\equiv0\pmod{p^{n/d+1}}\]
	because all other terms of the sum will vanish. It follows there is indeed an irreducible of degree $n$ in $\FF_p[x].$ (One could also see this by directly bounding the sum for $N_n$ by $\frac1n\left(p^n-\sum_{d=1}^{n-1}p^{n/d}\right)>0.$)
\end{remark}
We close with a remark.
\begin{remark}
	We are able to construct a splitting field of any finite set of polynomials, simply by iterating. We can extend to a countable set of polynomials using a transfinite induction (read: Zorn's lemma). For example, if we take the splitting field of all polynomials, we get the algebraic closure of our field.
\end{remark}
