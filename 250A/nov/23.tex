% !TEX root = ../notes.tex














The billboard said ``The End is Near.''

\subsection{Kummer Theory in Characteristic Not Dividing \texorpdfstring{$n$}{n}}
Last lecture we were trying to describe Galois extensions $L/K$ where $\op{Gal}(L/K)\cong\ZZ/n\ZZ.$ Quickly, we recall that for $n=2,$ we found the following.
\begin{proposition}
	Fix $L/K$ a Galois extension with $\op{Gal}(L/K)\cong\ZZ/2\ZZ.$
	\begin{itemize}
		\item If $\op{char}K\ne2,$ then $L=K(\sqrt a)$ for some $a\in K.$
		\item If $\op{char}K=2,$ then $L\cong K[x]/\left(x^2-x-a\right)$ for some $a\in K.$
	\end{itemize}
\end{proposition}
\begin{remark}
	In particular, in the former case, $\sqrt a$ was an eigenvector of the generator of $\op{Gal}(L/K).$ ``Eigenvector'' will be today's magic word.
\end{remark}
We would like to extend this to all positive integers $n.$
\begin{warn}
	In class, Professor Borcherds focused on the case where $n$ is prime, for psychological reasons. For my personal benefit, I have generalized to the general case below.
\end{warn}
To start, work with $\op{char}K\nmid n.$ Well, fix $\sigma$ a generator of $\op{Gal}(L/K)$ so that $\op{Gal}(L/K)=\langle\sigma\rangle,$ and our end goal will (roughly speaking) be to diagonalize $\sigma.$ Namely, we view $L$ as a $K$-vector space upon which $\op{Gal}(L/K)$ acts.

Well, the eigenvalues of $\sigma$ are going to be the $n$th roots of unity because $\sigma^n=\sigma^{\#\op{Gal}(L/K)}=\id$: indeed, if $v$ is an eigenvector with eigenvalue $\lambda,$ then
\[v=\sigma^n(v)=\lambda^nv,\]
so $\lambda$ is indeed an $n$th root of unity. So we will just add the assumption that $K$ contains the $p$th roots of unity.
\begin{warn}
	The problem becomes substantially harder in the case where $K$ does not contain all the $p$th roots of unity. Namely, over a global field, it is roughly class field theory.
\end{warn}
Note that, by the derivative trick, $X^n-1$ is separable over $K$ (the only root of its derivative is $0$), so there are in fact $n$ roots of unity. Because the roots of $X^n-1$ form a finite multiplicative subgroup of $K^\times,$ it is cyclic,\footnote{I don't see an easy way to avoid invoking this machinery, so I have invoked it.} so (for convenience) fix $\zeta$ some primitive (generating) $n$th root of unity.

In order to claim an eigenbasis, let's go find our eigenvectors for each eigenvalue. Namely, we fix some eigenvalue $\zeta^k,$ and we want to find vectors $v\in L$ such that
\[\sigma v=\zeta^kv.\]
In other words, we want to find vectors fixed by $\sigma\zeta^{-k}.$ (Here $\sigma\zeta^{-k}=\zeta^{-k}\sigma$ because $\sigma$ fixes $K$; namely, we are using the assumption that $\zeta\in K.$) For this, we do something clever: we apply $G$-averages to get the eigenvectors, taking some fixed $v\in L$ and setting
\[v_k:=\sum_{\ell=0}^{n-1}\left(\sigma\zeta^{-k}\right)^\ell v.\]
Indeed, we see that applying $\sigma\zeta^{-k}$ will simply cycle the terms in the sum, so $v_k$ does indeed have $\sigma v_k=\zeta^kv_k.$ So $(*)$ gives us access to lots of eigenvectors (one from each $v\in L$), but it is technically possible that any choice $v\in L$ will give $v_k=0$ so that we are not actually generating a dimension-$1$ eigenspace.

There are a few ways to finish from here. Here is one finish in the case where $n$ is prime.
\begin{theorem}
	Fix $n$ a prime, and fix a field $K$ with $\op{char}K\nmid n$ such that $K$ contains all $n$th roots of unity. Now, if $L/K$ is a Galois extension such that $\op{Gal}(L/K)\cong\ZZ/n\ZZ,$ then $L=K(\sqrt[n]a)$ for some $a\in K$ such that $(\sqrt[n]a)^m\notin K$ for any $0<m<n.$
\end{theorem}
\begin{proof}
	We continue from the above discussion, trying to show that the claimed eigenspaces do promise an eigenbasis. The trick for this is to add our eigenvectors $v_k$ together. We see that
	\[\sum_{k=0}^{n-1}v_k=\sum_{k=0}^{n-1}\left(\sum_{\ell=0}^{n-1}\left(\sigma\zeta^{-k}\right)^\ell v\right)=\left(\sum_{\ell=0}^{n-1}\sigma^\ell\left(\sum_{k=0}^{n-1}\zeta^{-k\ell}\right)\right)v.\]
	Now, for $\ell\ne0,$ the inner sum will vanish as $\frac{\zeta^{-n\ell}-1}{\zeta^{-\ell}-1}=0,$ so we only have to worry about the $\ell=0$ term, which tells us
	\[\sum_{k=0}^{n-1}v_k=nv.\]
	So because $\op{char}K\nmid n$ (!), we see that $v$ is a sum of elements in each individual eigenspace. To be explicit, if we let $L_k\subseteq L$ be the eigenspace for the eigenvalue $\zeta^k,$ we have found that
	\[L\subseteq\bigoplus_{k=0}^{n-1}L_k,\]
	so the equality follows.
	
	In particular, at least one of the eigenspaces is nonzero, and we cannot just have the eigenspace $L_0=K$ be nonempty because this would imply $L=L_0=K.$ So there is some nonzero eigenvector $v$ with eigenvalue $\zeta^k\ne1,$ but because $n$ is prime (here we use that $n$ is prime), $\zeta^k$ is still a primitive $n$th root of unity. Now, for any $m\in\ZZ,$
	\[\sigma\cdot v^m=(\sigma v)^m=(\zeta^k v)^m=\zeta^{km}v^m,\]
	so $v^m\in K$ if and only if $v^m$ is fixed by $\op{Gal}(L/K)$ if and only if $\zeta^{km}=1$ if and only if $n\mid km$ if and only if $n\mid m.$ So $v^n\in K,$ and $n$ is the least such positive integer. So $K(\sqrt[n]{v^n})$ is indeed a degree-$n$ extension and will be equal to $L,$ finishing.
\end{proof}
And here is one way to finish in the general case.
\begin{theorem} \label{thm:kummerchar0}
	Fix $n$ a positive integer, and fix a field $K$ with $\op{char}K\nmid n$ such that $K$ contains all $n$th roots of unity. Now, if $L/K$ is a Galois extension such that $\op{Gal}(L/K)\cong\ZZ/n\ZZ,$ then $L=K(\sqrt[n]a)$ for some $a\in K$ such that $(\sqrt[n]a)^m\notin K$ for any $0<m<n.$
\end{theorem}
\begin{proof}
	We continue where we left off before the previous theorem, attempting to show that the claimed eigenspaces do promise an eigenbasis. That is, we would like to show that, for fixed $k,$
	\[\sum_{\ell=0}^{n-1}\left(\sigma\zeta^{-k}\right)^\ell v\]
	is not identically $0$ for each $v\in L.$ Squinting a bit harder at this, we see that we are basically trying to prove that
	\[\sum_{\ell=0}^{n-1}\zeta^{-k\ell}\sigma^\ell\ne0.\]
	For this, we pick up the following somewhat technical lemma.
	\begin{lemma} \label{lem:linindauto}
		Fix $L$ a field. Then a finite set of automorphisms of $L$ are $L$-linearly independent. In other words, given a finite set of distinct automorphisms $\{\sigma_k\}_{k=1}^n\subseteq\op{Aut}(L),$ we have that
		\[\sum_{k=1}^na_k\sigma_k=0\]
		for $\{a_k\}_{k=1}^n\subseteq L$ implies that $a_k=0$ for each $k.$
	\end{lemma}
	\begin{proof}
		We proceed by contradiction. Suppose for the sake of contradiction that such a linearly dependent set $\{\sigma_k\}_{k=1}^n$ exists, and find a set with the smallest such $n,$ and we will find a smaller counterexample.
		
		Well, $\sigma_n\ne\sigma_1,$ so there exists some $y\in L$ such that $\sigma_n(y)\ne\sigma_1(y).$ Then, for any $x\in L,$ we see
		\[a_1\sigma_1(xy)+\cdots+a_n\sigma_n(xy)=0\qquad\text{and}\qquad\sigma_n(y)\big(a_1\sigma_1(x)+\cdots+a_n\sigma_n(x)\big)=0.\]
		Subtracting these two equations, we find that the $a_n\sigma_n(x)\sigma_n(y)$ term will cancel out, leaving us with
		\[a_1(\sigma_1(y)-\sigma_n(y))\sigma_1(x)+\cdots+a_{n-1}(\sigma_{n-1}(y)-\sigma_n(y))\sigma_n(x)=0\]
		for each $x\in L.$ Namely, this a nontrivial relation between the $\{\sigma_k\}_{k=1}^{n-1}$ and so is a smaller counterexample. This finishes.
	\end{proof}
	In particular, we find that
	\[\sum_{\ell=0}^{n-1}\zeta^{-k\ell}\sigma^\ell\ne0\]
	is forced, so the given eigenspace must be nonempty.

	We now finish as before. We see that there exists a vector $v\in L$ with eigenvalue $\zeta,$ which satisfies, for any $m\in\ZZ,$
	\[\sigma\cdot v^m=(\sigma v)^m=(\zeta^k v)^m=\zeta^{km}v^m,\]
	so $v^m\in K$ if and only if $v^m$ is fixed by $\op{Gal}(L/K)$ if and only if $\zeta^{km}=1$ if and only if $n\mid km$ if and only if $n\mid m.$ So $v^n\in K,$ and $n$ is the least such positive integer. So $K(\sqrt[n]{v^n})$ is indeed a degree-$n$ extension and will be equal to $L,$ finishing.
\end{proof}
\begin{remark}[Nir]
	This second finish for the general case was not done in class. Some reader is likely to complain that I am essentially proving Hilbert's Theorem 90 without ever saying that I am proving Hilbert's Theorem 90. My response to such readers is to try to salvage the general case without doing something like this and please tell me how so that I can adjust the above accordingly.
\end{remark}
While we're here, we note that the converse of \autoref{thm:kummerchar0} is also true; the argument is in Theorem VI.6.2(ii) of Lang.
\begin{proposition}
	Fix $n$ a positive integer, and fix a field $K$ with $\op{char}K\nmid n$ such that $K$ contains all $n$th roots of unity. Now, if $L=K(\sqrt[n]a)$ for some $a\in K$ such that $(\sqrt[n]a)^m\notin K$ for any $0<m<n,$ then $L/K$ is a Galois extension such that $\op{Gal}(L/K)\cong\ZZ/n\ZZ.$
\end{proposition}
\begin{proof}
	Note $L$ is the splitting field of the separable polynomial $X^n-a,$ so $L/K$ is Galois. In particular, fixing $\sigma\in\op{Gal}(L/K),$
	\[\sigma\sqrt[n]a=\zeta^\bullet\sqrt[n]a\]
	for some root $\zeta^\bullet$ of $X^n-1.$ The map $\sigma\mapsto\zeta^\bullet$ is an injective group homomorphism $\op{Gal}(L/K)\to\ZZ/n\ZZ$; we would like this map to be an isomorphism.
	
	Well, fixing $\sigma$ a generator of $\op{Gal}(L/K)$ (which is cyclic because it is a subgroup of $\ZZ/n\ZZ$), we see that
	\[\sigma(\sqrt[n]a^{\#\op{Gal}(L/K)})=\sigma(\sqrt[n]a)^{\#\op{Gal}(L/K)}=\sqrt[n]a^{\#\op{Gal}(L/K)},\]
	so $\sqrt[n]a^{\#\op{Gal}(L/K)}\in K,$ so $\#\op{Gal}(L/K)\ge n.$ It follows $\op{Gal}(L/K)\cong\ZZ/n\ZZ,$ as needed.
\end{proof}

\subsection{Kummer Theory in Characteristic \texorpdfstring{$p$}{p}}
Here we are still interested in Galois extensions $L/K$ wth $\op{Gal}(L/K)\cong\ZZ/n\ZZ,$ but now we discuss $\op{char}K\mid n.$ However, because it matters this time, we will focus on the case where $n$ is prime so that $n=p.$

Again, we fix $\sigma$ a generator of $\op{Gal}(L/K).$ We would still like to find eigenvectors, but we find that
\[X^p-1=(X-1)^p,\]
so our only eigenvalue is $1.$ Explicitly, if $v$ is an eigenvector of $\sigma$ with eigenvalue $\lambda,$ then
\[v=\sigma^pv=\lambda^pv,\]
so $\lambda$ is a root of $X^p-1,$ so $\lambda=1.$ So our only eigenvectors have $\sigma v=v,$ which is equivalent to $v\in K.$ Thus, the entire process we did for characteristic not equal to $p$ (namely, trying to diagonalize $\sigma$) is impossible here.

Well, if we cannot get eigenvectors, we for generalized eigenvectors. Explicitly, we see that the ring $\op{End}_K(L)$ is a $K$-module, so $p\varphi=0\varphi=0$ for any $\varphi\in\op{End}_K(L).$ It follows $\sigma-1$ is nilpotent, for
\[(\sigma-1)^p=\sigma^p-1=0,\]
so we can be interested in generalized eigenvectors $v$ such that
\[(\sigma-1)^nv=0\]
for some fixed $n.$ Namely, we see that we have the increasing sequence of spaces
\[K=\ker(\sigma-1)\subseteq\ker(\sigma-1)^2\subseteq\ker(\sigma-1)^3\subseteq\cdots\subseteq\ker(\sigma-1)^p=L.\]
Anyways, $\ker(\sigma-1)$ is boring, so let's look at $\ker(\sigma-1)^2.$ We want $(g-1)^2v_2=0,$ and in fact, we claim we can find such a vector $v_2\in L$ with $(g-1)v_2\ne0$ as well. Indeed, fix $n$ the smallest positive integer such that $(\sigma-1)^n=0,$ which means that we can find $w\in L$ such that $(\sigma-1)^{n-1}w\ne0.$ So we see that
\[v_2:=(\sigma-1)^{n-1}v\]
is the vector we want.
\begin{remark}[Nir]
	In fact, $p$ is the smallest positive integer $n$ such that $(\sigma-1)^n=0.$ One way to see this is by direct expansion: if we are to have
	\[0=(\sigma-1)^n=\sum_{k=0}^n\binom nk(-1)^{n-k}\sigma^k,\]
	then we see that $n<p$ would imply that the above equation is a nontrivial relation of automorphisms, which cannot exist. But of course $(\sigma-1)^p=0$ works.
	
	The main point of saying this is that, choosing $v$ such that $(\sigma-1)^{p-1}v\ne0,$ we can put $\sigma$ in Jordan canonical form by using the basis $\left\{(\sigma-1)^\bullet v\right\}.$ So these generalized eigenvectors are almost diagonalizing. Regardless, we will not need this much power for the argument.
\end{remark}
So why does this generalized eigenvector help us? Well, we fix $a:=(\sigma-1)v_2$ so that $(\sigma-1)a=0,$ so $a\in K.$ So we may take the equation
\[\sigma v_2=v_2+a\]
and divide through by $a$ (note $a\ne0$ because $v_2\notin\ker(\sigma-1)$), giving an element $v:=v_2/a\in L$ such that
\[\sigma v=v+1.\]
This is our analogue to finding an element $v\in L$ such that $\sigma v=\zeta v,$ as we had in the characteristic not equal to $p$ case. Namely, $v$ seems to have the simplest possible behavior with respect to the Galois action.

Continuing with the analogy, we hope that $v$ generates $L/K,$ so we would like to find the minimum polynomial for our $v.$ Well, we find that
\[\sigma\left(v^p\right)=(\sigma v)^p=(v+1)^p=v^p+1.\]
In particular, $\sigma\left(v^p-v\right)=v^p+1-(v+1)=v^p-v,$ so $v^p-v$ is fixed by $\sigma$ and hence in $K,$ so we find some $b:=v^p-v\in K$ such that $v$ satisfies
\[X^p-X-b\in K[X].\]
This equation has a name.
\begin{definition}[Artin--Shreier]
	Equations of the form $X^p-X-b$ are called Artin--Schreier polynomials.
\end{definition}
Anyways, we get the following.
\begin{theorem}
	Fix $p$ a prime, and fix a field $K$ with $\op{char}K=p.$ Now, if $L/K$ a Galois extension with $\op{Gal}(L/K)\cong\ZZ/p\ZZ,$ then there exists $b\in K$ such that
	\[L\cong\frac{K[X]}{\left(X^p-X-b\right)}.\]
\end{theorem}
\begin{proof}
	This mostly follows from the above discussion. Namely, we have been promised an element $v\in K$ such that $v$ is the root of some $X^p-X-b.$ Further, we see that
	\[K(v)\subseteq L\]
	is strictly larger than $K$ because $v\notin K$ (else $v\in\ker(\sigma-1)$ which was hypothesized false), so $K(v)=L$ is forced because $[L:K]$ is prime, and $[K(v):K]$ is a nontrivial factor. Here is where we used that $\#\op{Gal}(L/K)$ is prime.
\end{proof}
\begin{remark}[Nir]
	Note that the above argument did not require $K$ to have the $p$th roots of unity. This is because $1$ is the only $p$th root of unity in characteristic $p,$ so $K$ already had them.
\end{remark}
The converse is almost true.
\begin{proposition}
	Fix $p$ a prime, and fix a field $K$ with $\op{char}K=p.$ Then, given $b\in K,$ either $X^p-X-b$ will fully split in $K$ or, fixing any root $\alpha$ of $X^p-X-b,$ we have that $L:=K(\alpha)$ makes $L/K$ a Galois extension such that $\op{Gal}(L/K)\cong\ZZ/p\ZZ.$
\end{proposition}
\begin{proof}
	Note that, if $\alpha$ is a root of $X^p-X-b$ (say, in $\overline K$), then $\alpha+1$ is also a root by the Frobenius automorphism, so continuing this process gives $p$ roots
	\[\alpha,\quad\alpha+1,\quad\alpha+2,\quad\ldots,\quad\alpha+(p-1),\]
	and all these roots must be distinct, so they must all of them by Lagrange's theorem on polynomials. In particular, if an extension $L/K$ contains any roots of $X^p-X-b,$ then $L$ will contain all of them.

	We now look closer at $L:=K(\alpha),$ which forcibly contains one root and hence all of them. Now, $L$ contains all of the roots of $X^p-X-1$ while being generated by such a root, so $L/K$ is normal. Further, $X^p-X-1$ has all distinct roots, so $L/K$ is also separable and hence Galois.

	So now we may study $\op{Gal}(L/K)$ more closely. Any automorphism $\sigma\in\op{Gal}(L/K)$ must send $\alpha$ to one of the other roots $\alpha+k_\sigma,$ for some $k_\sigma\in\ZZ/p\ZZ.$ Because the action of an automorphism is fully defined by the action on $\alpha,$ we see that
	\[\sigma\mapsto k_\sigma\]
	gives an injective group homomorphism $\op{Gal}(L/K)\to\ZZ/p\ZZ.$ We have two cases.
	\begin{itemize}
		\item If $L=K$ so that $\op{Gal}(L/K)$ is trivial, then $X^p-X-b$ fully splits because $L=K$ contains all the roots of $X^p-X-b.$
		\item Otherwise, $\op{Gal}(L/K)$ is nontrivial, so because $\ZZ/p\ZZ$ has no nontrivial proper subgroups, we must have $\op{Gal}(L/K)\cong\ZZ/p\ZZ.$
		\qedhere
	\end{itemize}
\end{proof}

\subsection{Applications of Kummer Theory}
Let's do an application, for fun.
\begin{exe}
	We construct $\FF_{p^p}$ as an extension of $\FF_p.$
\end{exe}
\begin{proof}
	By previous work with finite fields, we know that $\op{Gal}(\FF_{p^p}/\FF_q)$ is cyclic of order $p$ (generated by $x\mapsto x^p$). So by our work with $\ZZ/p\ZZ$-extensions in characteristic $p,$ it suffices to note that
	\[X^p-X-1\]
	is irreducible over $\FF_q$: namely, $X^p-X-1$ has no roots over $\FF_p$ because $X^p-X$ fully vanishes on $\FF_p,$ so instead we must have $X^p-X-1$ irreducible of degree $p.$
\end{proof}
\begin{remark}
	Professor Borcherds said that this can be extended to explicitly construct $\FF_{p^\bullet}/\FF_p,$ though I am not sure how to do this. Lang roughly asserts that the correct machinery comes from Witt vectors.
\end{remark}
Our work above also gives the following statement.
\begin{theorem}
	A polynomial $f\in K[X]$ can be solved via radicals or Artin--Schreier equations (of degree $\op{char}K$) if and only if the Galois group over $K$ is solvable.
\end{theorem}
\begin{proof}
	We show the directions one at a time. Set $L$ the splitting field of $f$ over $K.$
	% We work in characteristic $0$ so that Professor Borcherds does not have to say the word ``Artin--Schreier'' repeatedly.
	\begin{itemize}
		\item Suppose that the polynomial is solvable, and we want to show that the Galois group is solvable. We start by taking
		\[K_1:=K(\zeta_{(\deg f)!})\]
		and work over $K_1$ so that it suffices to show that the Galois group of $f$ over $K_1$ is solvable because $K_1/K$ is an abelian extension. The main point of introducing $K_1$ is to get the $m$th roots of unity for any intermediate extension between $K_1$ and the splitting field $L'$ of $f$ (over $K_1$) because this intermediate extension will have degree less than or equal to $\deg f.$
		
		Now, if our equation is solvable by radicals and Artin--Schreier equations of degree $\op{char}K,$ then we can build it up by one such extension at a time, so there is a chain of fields
		\[K\subseteq K_1\subseteq \underbrace{K_1(\alpha_1)}_{K_2:=}\subseteq \underbrace{K_2(\alpha_2)}_{K_3:=}\subseteq\cdots\subseteq K_n(\alpha_n)= L',\]
		where each $K_\bullet(\alpha_\bullet)/K_\bullet$ is defined so that $\alpha_\bullet$ is either a radical or the root of an Artin--Schreier equation of degree $\op{char}K.$ For convenience, set $n_\bullet:=[K_\bullet(\alpha_\bullet):K_\bullet]\le\deg f.$
		
		Because $K_\bullet(\alpha_\bullet)/K_\bullet$ has $K_\bullet\supseteq K_1$ containing all the $n_\bullet$th roots of unity because $n_\bullet\le\deg f.$ So by the converse to our Kummer theory work above, we see
		\[\op{Gal}(K_\bullet(\alpha_\bullet)/K_\bullet)\cong\ZZ/n_\bullet\ZZ,\]
		so taking $G_\bullet:=\op{Gal}(L'/K_\bullet),$ we get the sequence of subgroups
		\[\op{Gal}(L'/K)\supseteq G_1\supseteq G_2\supseteq\cdots\supseteq\langle e\rangle\]
		where $\op{Gal}(L/K')/G_1\cong\op{Gal}(K_1/K)$ is abelian, and $G_{k+1}/G_k\cong\op{Gal}(K_{k+1}/K_k)\cong\ZZ/n_k\ZZ$ is cyclic. So indeed this sequence witnesses that $\op{Gal}(L/K)$ is solvable.

		Technically, we are actually interested in $\op{Gal}(L/K)$ and not $\op{Gal}(L'/K),$ but we do know
		\[\op{Gal}(L'/K)\onto\op{Gal}(L/K)\]
		by restriction, so the solvability of $\op{Gal}(L'/K)$ implies the solvability of $\op{Gal}(L/K).$\footnote{I should probably say something about solvability in short exact sequences, but I can't be bothered.}
		\item Conversely, suppose that $\op{Gal}(L/K)$ is a solvable group. Lifting up to $L'/K'$ where $K':=K(\zeta_{(\deg f)!})$ and $L':=LK_1.$ It is still true that $L'/K$ is a solvable extension because $\op{Gal}(L'/K)/\op{Gal}(L/K)\cong\op{Gal}(L'/K)$ is a cyclotomic extension and hence abelian. So taking subgroups, we get that $\op{GaL}(L'/K')$ is solvable as well.

		So it suffices to show that $f$ is solvable by radicals and Artin--Schreier equations over $K',$ where we assume $\op{Gal}(L'/K')$ is solvable. (In particular, the roots of unity we added to $K_1$ are legal because they are ``radicals'' of a sort.) Well, because $\op{Gal}(L'/K_1)$ is solvable, we may build a chain of subgroups
		\[G_0:=\op{Gal}(L'/K')\supseteq G_1\supseteq G_2\supseteq\cdots\supseteq\langle e\rangle=:G_n\]
		so that $G_{k+1}/G_k\cong\ZZ/p_k\ZZ$ for some prime $p_k.$ Looking at the corresponding fields, we set $K_\bullet:=(L')^{G_\bullet}$ so that
		\[K'=K_0\subseteq K_1\subseteq K_2\subseteq\cdots\subseteq K_n=L'\]
		has $\op{Gal}(K_{k+1}/K_k)\cong G_{k+1}/G_k\cong\ZZ/p_k\ZZ.$ But each of these field extensions contains the $p_k$th roots of unity ($p_k\le[L':K']\le\deg f$), so our classification of these extensions promises that
		\[K_{k+1}=K_k(\alpha_k),\]
		where $\alpha_k$ is the root of some element of $K_k$ or the root of an Artin--Schreier equation of degree $\op{char}K.$ In particular, any element of $L'$---namely, the roots of $f$---can be built from radicals and solutions to Artin--Schreier equations.
		\qedhere
	\end{itemize}
\end{proof}
\begin{remark}
	This is where the term ``solvable'' comes from.
\end{remark}
\begin{ex}
	Any polynomial of degree at most $4$ can be solved by radicals or Artin--Schreier equations. Indeed, the groups $S_1,S_2,S_3,S_4$ are all solvable, and the Galois group of any polynomial of degree at most $4$ is a subgroup of $S_4.$
\end{ex}
\begin{example}
	The polynomial $x^5-4x+2$ has Galois group $S_5$ as we showed earlier, so it is not solvable by radicals.
\end{example}
\begin{example}
	Of course, $x^5-2$ is solvable by radicals.
\end{example}
In general, it is difficult to tell if an equation is solvable by radicals because finding the Galois group is difficult.

\subsection{Cyclotomic Polynomials: Examples}
We saw that roots of unity were somewhat important for our discussion, so let's study them.
\begin{remark}
	``Cyclotomic'' means cutting up the circle, which comes from their picture in the complex plane. For example, here are the $7$th roots of unity cutting up the circle.
	\begin{center}
		\begin{asy}
			unitsize(1.5cm);
			draw((-1.25,0)--(1.25,0));
			draw((0,-1.25)--(0,1.25));
			draw(circle((0,0),1));
			for(int i = 0; i < 7; ++i)
				dot(dir(360*i/7));
		\end{asy}
	\end{center}
\end{remark}
Cyclotomic extension are essentially the only higher-degree extensions we can control. The next easiest are Artin--Schreier extensions or adjoining $n$th roots, but aside from these, it is difficult to control other extensions.

In particular, we study cyclotomic extensions of $\QQ.$ To start, we need to find the minimal polynomial for some primitive $n$th root of unity, which we name $\zeta_n.$ We don't know it yet, but the following definition will become the minimal polynomial of $\zeta_n.$
\begin{definition}[Cyclotomic polynomial]
	Given a positive integer $n,$ we define \textit{$n$th cyclotomic polynomial} as
	\[\Phi_n(X):=\prod_{\substack{1\le k\le n\\\gcd(k,n)=1}}\left(X-\zeta_n^k\right).\]
	In other words, $\Phi_n(X)$ is constructed to have roots which are the primitive $n$th roots of unity.
\end{definition}
Note that, because all roots of $\Phi_n(X)$ are $n$th roots of unity, we will have
\[\Phi_n(X)=\prod_{\substack{1\le k\le n\\\gcd(k,n)=1}}\left(X-\zeta_n^k\right)~\bigg|~\prod_{\substack{1\le k\le n}}\left(X-\zeta_n^k\right)=X^n-1.\]
However, $X^n-1$ will typically reduce (e.g., it is divisible by $X-1$), and it will turn out that our minimal polynomial for $\zeta_n$ will be $\Phi_n(X).$
\begin{example}
	If $n=p$ is prime, then $f(X)=\frac{X^p-1}{X-1}$ is irreducible because
	\[\frac{(X+1)^p-1}{(X+1)-1}=\frac1X\sum_{k=1}^p\binom pkX^k=\sum_{k=1}^p\binom pkX^{k-1}\]
	is Eisenstein at the prime $p.$ So $f$ is the minimal polynomial for $\zeta_p.$
\end{example}
\begin{example} \label{ex:primepowercyclo}
	If $n=p^k$ is a prime power, then
	\[f(X):=\frac{X^{p^n}-1}{X^{p^{n-1}}-1}=\sum_{k=0}^{p-1}X^{np^k}\]
	is irreducible again by taking $X\mapsto X+1$ and applying Eisenstein's criterion at $p.$ To be explicit, reducing to $\FF_p(X),$ we have
	\[f(X+1)=\frac{(X+1)^{p^n}-1}{(X+1)^{p^{n-1}}-1}=\frac{X^{p^n}+1-1}{X^{p^{n-1}}+1-1}=X^{p^n-p^{n-1}}.\]
	Then to evaluate the constant term, we evaluate $f(0+1)=f(1)=p,$ which is indeed not divisible by $p^2.$ So $f$ is the minimal polynomial for $\zeta_{p^k}.$
\end{example}
However, for numbers which are not prime powers, this becomes harder. In fact, we see that if $m\mid n,$ then $m$th roots of unity are $n$th roots of unity, so these need to be thrown out by hand if we want to focus on primitive $n$th roots of unity. For prime-powers, this is not so bad because we have relative control over divisors of prime-powers.
\begin{exe}
	We compute lots of cyclotomic polynomials.
\end{exe}
\begin{proof}
	We have the following list. We remark some properties as we go down the list, which we will rigorize in the next subsection.
	\begin{itemize}
		\item For $n=1,$ our minimal polynomial is $\Phi_1(X)=X-1.$
		\item For $n=2,$ our minimal polynomial is $\Phi_2(X)=X+1$ by dividing $X^2-1$ by $X-1.$
		\item For $n=3,$ our minimal polynomial is $\Phi_3(X)=X^2+X+1$ by dividing $X^3-1$ by $X-1.$
		\item For $n=4,$ our minimal polynomial is $\Phi_4(X)=X^2+1$ by dividing $X^4-1$ by $X^2-1.$
		\item For $n=5,$ our minimal polynomial is $\Phi_5(X)=X^4+X^3+X^2+X+1$ by dividing $X^5-1$ by $X-1.$ 
		\item For $n=6,$ we are looking at the roots of $X^6-1,$ but we need to kill the third roots of unity as well as the square roots roots of unity. So we get
		\[X^6-1=(X-1)\left(X^2+X+1\right)\left(X+1\right)\left(X^2-X-1\right),\]
		so the one that we want is $\Phi_6(X)=X^2-X-1.$
		\item For $n=7,$ we get $X^6+X^5+X^4+X^3+X^2+X+1$ by primality.
		\item For $n=8,$ we need to divide $X^8-1$ by the first, second, and fourth roots of unity, so we get $X^4+1.$
		\item For $n=9,$ we need to divide $X^9-1$ by the cube roots of unity, so we get $X^6+X^3+1.$ (Visually, we can write down the ninth roots of unity and kill the third roots of unity by hand.)
		\item For $n=10,$ we need to divide $X^{10}-1$ out by the square roots of unity and the fifth roots of unity, so we get
		\[X^{10}-1=(X-1)(X+1)\left(X^4+X^3+X^2+X+1\right)\left(X^4-X^3+X^2-X+1\right).\]
		In particular, $\Phi_{10}(X)=\Phi_5(-X),$ which makes sense because any negative fifth root of unity will artificially gain a factor of $2$ in its order, so the roots are negatives.
		\item For $n=15,$ we need to divide $X^{15}-1$ out by the third roots of unity and fifth roots of unity, so we find that we want
		\[\Phi_{15}(X)=\frac{\left(X^{15}-1\right)\left(X-1\right)}{\left(X^3-1\right)\left(X^5-1\right)}=X^8-X^7+X^5-X^4+X^3-X+1.\]
	\end{itemize}
	We remark that all of the above coefficients were in $\{-1,0,1\}.$ This is not true in general, and the smallest counterexample is $105.$ We will discuss this more shortly.
\end{proof}

\subsection{Cyclotomic Polynomials: Theory}
Let's list some basic properties of $\Phi_n.$
\begin{proposition}
	We have the following.
	\begin{listalph}
		\item We have that
		\[X^n-1=\prod_{d\mid n}\Phi_d(X).\]
		\item We have that
		\[\Phi_n(X)=\prod_{d\mid n}\left(X^d-1\right)^{\mu(n/d)},\]
		where $\mu$ is the M\"obius function.
	\end{listalph}
\end{proposition}
\begin{proof}
	We take these one at a time.
	\begin{listalph}
		\item This is saying that all $n$th roots of unity are a primitive $d$th root of unity for some $d\mid n.$ Rigorously, we write
		\begin{align*}
			\prod_{d\mid n}\Phi_d(X) &= \prod_{d\mid n}\bigg(\prod_{\substack{1\le k\le d\\\gcd(k,d)=1}}\left(X-e^{2\pi ik/d}\right)\bigg) \\
			&= \prod_{d\mid n}\bigg(\prod_{\substack{1\le kn/d\le n\\\gcd(kn/d,n)=n/d}}\left(X-e^{2\pi i(k/(n/d))/d}\right)\bigg) \\
			&= \prod_{d\mid n}\bigg(\prod_{\substack{1\le k\le n\\\gcd(k,n)=n/d}}\left(X-e^{2\pi ik/n}\right)\bigg).
		\end{align*}
		Now, every $k$ between $1$ and $n$ will have exactly one greatest common divisor with $n,$ and this greatest common divisor will be a divisor $n/d$ of $n$ for some $d\mid n.$ So in fact the above product is over all the $k$ with $1\le k\le n,$ so we find
		\[\prod_{d\mid n}\Phi_d(X)=\prod_{1\le k\le n}\left(X-e^{2\pi ik/n}\right)=X^n-1,\]
		which is what we wanted.
		\item This comes from applying M\"obius inversion to (a), in a multiplicative form. Doing this formally is somewhat annoying (we essentially have to reprove M\"obius inversion), but one can see what we are supposed to do by noting we want to prove something like
		\[\log\Phi_n(X)=\sum_{d\mid n}\log\left(X^d-1\right)\mu\left(\frac nd\right)\]
		given that
		\[\log\left(X^n-1\right)=\sum_{d\mid n}\log\Phi_n(X),\]
		which looks more immediately like M\"obius inversion. (Formalizing this would require a rigorously defined $\log$ function, but it is easier to just show the inversion by hand.)
		\qedhere
	\end{listalph}
\end{proof}
The above two formulae give us a recursive way to compute cyclotomic polynomials, which will focus more on in the next subsection.
\begin{remark}[Nir]
	The recursion is probably the most direct way to show that $\Phi_n(X)\in\ZZ[X].$ For example, we see that
	\[\Phi_n(X)=\prod_{d\mid n}\left(X^d-1\right)^{\mu(n/d)}=\frac{\prod_{d\mid n,\mu(n/d)=1}\left(X^d-1\right)}{\prod_{d\mid n,\mu(n/d)=-1}\left(X^d-1\right)}=:\frac{f(X)}{g(X)}\in\QQ(X).\]
	But by definition, $\Phi_n(X)\in\CC[X],$ so $\Phi_n(X)\in\QQ(X)\cap\CC[X]=\QQ[X].$
	
	To get $\Phi_n(X)\in\ZZ[X],$ finer study is required. We see that $f(X),g(X)\in\ZZ[X]$ with $c(f)=c(g)=1,$ and $f(X)=\Phi_n(X)g(X),$ for some $\Phi_n(X)\in\QQ[X].$ It follows from Gauss's lemma that $c(\Phi_n)=1,$ so $\Phi_n(X)=\Phi_n(X)/c(\Phi_n)\in\ZZ[X].$
\end{remark}

While we're here, we should probably show that $\Phi_n$ is actually an irreducible polynomial, completing the proof that $\Phi_n$ is the monic irreducible polynomial for $\zeta_n$ over $\QQ.$ Namely, we have just remarked that $\Phi_n\in\QQ[X].$
\begin{proposition}
	We have that $\Phi_n(x)$ is irreducible (in characteristic $0$).
\end{proposition}
\begin{proof}
	We have done this in the case where $n$ is prime or a prime-power using Eisenstein's criterion (see \autoref{ex:primepowercyclo}). Technically we did not know that those polynomials were $\Phi_{p^r}(X)$ at the time, but we an see it via the recursion now, for
	\[\Phi_1(X)=X-1\qquad\text{and}\qquad\Phi_{p^r}(X)\stackrel?=\frac{X^{p^r}-1}{X^{p^{r-1}}-1}\text{  for  }r\ge1\]
	satisfies
	\[\prod_{d\mid p^r}\Phi_d(X)=\prod_{k=0}^r\Phi_{p^k}(X)=(X-1)\prod_{k=1}^r\frac{X^{p^r}-1}{X^{p^{r-1}}-1}=X^{p^r}-1,\]
	so get the equality by an induction.

	To get that $\Phi_n$ is irreducible in general, we lift from the prime case. We have the following technical lemma.
	\begin{lemma} \label{lem:embedroots}
		Fix $n$ a positive integer and $p$ a prime with $p\nmid n.$ Further, suppose that $f(X)\in\ZZ[X]$ divides $\Phi_n(X)$ and has $\zeta_n$ as a root. Then the roots $\zeta$ of $f$ are preserved under the mapping $\zeta\mapsto\zeta^p.$
	\end{lemma}
	\begin{proof}
		The trick is to reduce$\pmod p$ and carry the roots of $f$ with us. We start by fixing the root $\zeta_n$ of $f.$ Now, we set $f_n\mid f$ to be the minimal polynomial for $\zeta_n$ in $\ZZ[X],$
		% \footnote{We see $f_0\in\QQ[X]$ and $f_0\mid\Phi_n(X),$ so we may force $f_0$ to be in $\ZZ[X]$ by Gauss's lemma.}
		and we reduce $f_n$ to $\overline{f_n}\in\FF_p[X].$ Picking up an irreducible factor $\overline g$ of $\overline{f_n},$ we see that we can induce a map
		\[\ZZ[\zeta]\cong\frac{\ZZ[X]}{(f_n)}\to\frac{\FF_p[X]}{(\overline g)}\cong\FF_p[\overline{\zeta_n}]\]
		lifting $\ZZ\to\FF_p[\zeta_n]$ by sending $\zeta_n\mapsto\overline{\zeta_n}.$ In particular, all the work we did above was to guarantee that $f_n$ is in the kernel of this induced map so that we may safely mod it out as we did above.
		
		We now attempt to map the roots of $f$ from $\ZZ[\zeta_n]$ to $\FF_p[\overline{\zeta_p}].$ We understand the roots of $f$ in $\ZZ[\zeta_n]$ will be a subset of the roots of $X^n-1,$ which is $\langle\zeta_n\rangle,$ so we need to understand the roots of $X^n-1$ in $\FF_p[\overline{\zeta_n}].$
		
		Well, each $\overline{\zeta_n}^\bullet$ will be a root because $\overline{\zeta_n}$ is a root of $\overline g\mid\overline f\mid\Phi_n(X)\mid X^n-1.$ We claim that these are all of the roots of $X^n-1,$ for which it suffices to show that there are $n$ distinct powers of $\overline{\zeta_n}.$ This is surprisingly technical because we need to use the condition that $p\nmid n$ here.

		Let $m$ be the least positive integer such that $\overline{\zeta_n}^m=1,$ and we need to show that $m=n$; because $\overline{\zeta_n}^n=1,$ we know $m\mid n.$ Observe that $\overline{\zeta_n}$ will also be a root of
		\[\Phi_m(X)=\prod_{d\mid m}\left(X^d-1\right)^{\mu(m/d)}\]
		because the $X^m-1$ factor will vanish while none of the smaller factors will. Thus, supposing for the sake of contradiction that $m<n,$ we see
		\[X^n-1=\prod_{d\mid n}\Phi_d(X)\]
		has at least a double root at $\overline{\zeta_n}$---one root coming from $\Phi_m$ and one root coming from $\overline g(X)\mid\Phi_n(X).$ But this is impossible because $X^n-1$ has no double roots by the derivative trick (here we use the fact $p\nmid n$)!

		From all of our hard work, we see that the map
		\[\zeta_n^\bullet\mapsto\overline{\zeta_n}^\bullet\]
		is injective and in fact a group isomorphism $\langle\zeta_n\rangle\to\langle\overline{\zeta_n}\rangle.$ In particular, we may restrict this to an injective map
		\[\{\zeta\in\ZZ[\zeta_n]:f(\zeta)=0\}\to\left\{\overline\zeta\in\FF_p[\overline{\zeta_n}]:\overline f(\overline\zeta)=0\right\},\]
		which is well-defined because $f(\zeta)=0$ implies $\overline f(\overline\zeta)=0.$ In fact, the set on the left has $\deg f$ elements, and the set on the right has at most $\deg\overline f=\deg f$ elements, so they both have $\deg f$ elements, so this injection is a bijection.

		To finish, we see that, by the Frobenius automorphism, the roots on the right-hand side are fixed by the map $\overline\zeta\mapsto\overline\zeta^p,$ so the roots on the left-hand side are fixed by $\zeta\mapsto\zeta^p$ as well. To be explicit, if $\zeta_n^k$ is a root on the left-hand side, then $\overline{\zeta_n}^k$ is a root on the right-hand side, then $\overline{\zeta_n}^{pk}$ is a root on the right-hand side, so $\zeta_n^{pk}$ is a root on the left-hand side. This finishes.
	\end{proof}
	The point of the lemma is to show that the Galois group of $\Phi_n(X)$ (which is $\QQ(\zeta_n)$) is equal to $\left(\ZZ/n\ZZ\right)^\times.$ Certainly it is a subgroup because any $\sigma\in\op{Gal}(\QQ(\zeta_n)/\QQ)$ must map $\zeta_n$ to some $\zeta_n^{k_\sigma}$ for $k_\sigma\in\left(\ZZ/n\ZZ\right)^\times$ and is uniquely determined by this action. So
	\[\sigma\mapsto k_\sigma\]
	gives an injective homomorphism $\op{Gal}(\QQ(\zeta_n)/\QQ)\to(\ZZ/n\ZZ)^\times.$

	But now, by \autoref{lem:embedroots}, the Galois group of $\Phi_n(X)$ contains
	\[\zeta_n\mapsto\zeta_n^p\]
	for each $p$ coprime to $n,$ but these elements actually generate $\left(\ZZ/n\ZZ\right)^\times$ by simply prime-factoring a number in each of the various equivalence classes of $\left(\ZZ/n\ZZ\right)^\times.$

	So to finish, we see that
	\[\deg\Phi_n=\varphi(n)=\#(\ZZ/n\ZZ)^\times=[\QQ(\zeta_n):\QQ],\]
	so in fact $\Phi_n$ must be the minimal irreducible polynomial of $\zeta_n.$
\end{proof}

\subsection{Cyclotomic Polynomials: Computation}
\begin{warn}
	This subsection covers more explicit computation of cyclotomic polynomials, which was not covered in class. The main point here is to find the smallest $n$ for which $\Phi_n(X)$ has a coefficient outside of $\{-1,0,1\}.$
\end{warn}
As a quick example before doing any theory, we evaluate cyclotomic polynomials for semiprimes.
\begin{exe}
	Fix $p$ and $q$ distinct primes. Then $\Phi_{pq}(X)$ only has coefficients in $\{-1,0,1\}.$
\end{exe}
\begin{proof}
	Because $p$ and $q$ are distinct, we see that
	\[\Phi_{pq}(X)=\prod_{d\mid n}\left(X^d-1\right)^{\mu(pq/d)}=\frac{\left(X^{pq}-1\right)(X-1)}{\left(X^p-1\right)\left(X^q-1\right)}=\frac{X-1}{X^q-1}\sum_{n=0}^{q-1}X^{pn}.\]
	Expanding this out, we see that
	\[\Phi_{pq}(X)=\sum_{n=0}^{q-1}\frac{X^{pn+1}-X^{pn}}{X^q-1}=\sum_{n=0}^{q-1}\frac{X^{pn+1}}{X^q-1}-\sum_{n=0}^{q-1}\frac{X^{pn}}{X^q-1}.\]
	Now, we see that, by polynomial division, we get
	\[\frac{X^N}{X^q-1}=\sum_{k=1}^{\floor{n/q}}X^{N-qk}+\frac{X^{N\pmod q}}{X^q-1},\]
	where $N\pmod q$ is referring specifically to the smallest nonnegative integer in the residue class. Summing over all of our $N,$ we get
	\[\Phi_{pq}(X)=\left(\sum_{n=0}^{q-1}\sum_{k=1}^{\floor{(pn+1)/q}}X^{pn+1-qk}-\sum_{n=0}^{q-1}\sum_{k=1}^{\floor{pn/q}}X^{pn-qk}\right)+\left(\sum_{n=0}^{q-1}\frac{X^{pn\pmod q}}{X^q-1}-\sum_{n=0}^{q-1}\frac{X^{pn+1\pmod q}}{X^q-1}\right).\]
	We note that the last two sums on the right-hand side will cancel out because $pn\pmod q$ and $pn+1\pmod q$ will both loop over all possible residue classes$\pmod q$ because $p$ and $q$ are coprime. Thus,
	\[\Phi_{pq}(X)=\sum_{n=0}^{q-1}\sum_{k=1}^{\floor{(pn+1)/q}}X^{pn+1-qk}-\sum_{n=0}^{q-1}\sum_{k=1}^{\floor{pn/q}}X^{pn-qk}.\tag{$*$}\]
	By the Chinese remainder theorem, we see that $(n,k)\mapsto pn-qk$ from $\ZZ/q\ZZ\times\ZZ/p\ZZ\to\ZZ/pq\ZZ$ is a bijection, so $(n,k)\mapsto pn-qk+1$ is also bijection. So because the inner sums have $k$ range over at most $[1,q],$ we see that each outer sum above will have no repeated $X^\bullet$ terms.

	So when we collect $\Phi_{pq}(X),$ we see that any coefficient $X^\bullet$ gets at most $+1$ from the left sum of $(*)$ and gets at most $-1$ from the right sum. So each $X^\bullet$ will have coefficient contained in $\{-1,0,1\}.$
\end{proof}
Now let's see some small things we can do with our recursion.
\begin{lemma} \label{lem:cyclocompute}
	Fix $n$ a positive integer and $p$ a prime.
	\begin{listalph}
		\item If $p\nmid n,$ then $\Phi_{np}(X)=\Phi_n\left(X^p\right)/\Phi_n(X).$
		\item If $p\mid n,$ then $\Phi_{np}(X)=\Phi_n\left(X^p\right).$
	\end{listalph}
\end{lemma}
\begin{proof}
	The main idea is to use the M\"obius inversion formula, from which we find
	\[\Phi_{np}(X)=\prod_{d\mid np}\left(X^d-1\right)^{\mu(np/d)}.\]
	We now split into cases.
	\begin{listalph}
		\item Take $p\nmid n.$ Here we see that $d\mid np$ has two cases: either $p\mid d$ or $p\nmid d,$ but either way, $d/p^{\nu_p(d)}$ divides $n.$ In particular, divisors $d\mid np$ such that $p\mid d$ will only have one power of $p,$ so we can parameterize these by the divisor $d/p$ of $n.$ Namely,
		\[\Phi_{np}(X)=\underbrace{\prod_{d\mid n}\left(X^d-1\right)^{\mu(np/d)}}_{p\nmid d}\cdot\underbrace{\prod_{d\mid n}\left(X^{dp}-1\right)^{\mu(np/(dp))}}_{p\mid dp}.\]
		The first factor is
		\[\prod_{d\mid n}\left(X^d-1\right)^{\mu(np/d)}=\prod_{d\mid n}\left(X^d-1\right)^{-\mu(n/d)}=\frac1{\Phi_n(X)}\]
		where the M\"obius function got a sign because of the extra prime $p\nmid n.$ The second factor is
		\[\prod_{d\mid n}\left(X^{dp}-1\right)^{\mu(np/(dp))}=\prod_{d\mid n}\left(\left(X^p\right)^d-1\right)^{\mu(n/d)}=\Phi_n\left(X^p\right).\]
		So indeed, $\Phi_{np}(X)=\Phi_n\left(X^p\right)/\Phi_n(X).$
		\item Take $p\mid n.$ Then we see that each $d\mid np$ giving $\mu(np/d)\ne0$ had better have $p\mid d,$ for otherwise $np$ is divisible by $p^2$ so that $\mu(np/d)=0.$ So we only care about divisors $d\mid np$ such that $p\mid np,$ which again we can parameterize by the underlying divisor $d/p\mid n.$ So we see that
		\[\Phi_{np}(X)=\prod_{d\mid np}\left(X^d-1\right)^{\mu(np/d)}=\prod_{d\mid n}\left(X^{pd}-1\right)^{\mu(np/(dp))}=\prod_{d\mid n}\left(\left(X^p\right)^d-1\right)^{\mu(n/d)}=\Phi_n\left(X^p\right),\]
		which is what we wanted.
		\qedhere
	\end{listalph}
\end{proof}
\begin{example}
	If $n$ is odd, then we claim $\Phi_{2n}(X)=\Phi_n(-X).$ We could see this directly by studying the primitive $2n$th roots of unity and finding they are all $-\zeta_n^\bullet.$ Alternatively, we see that any divisor $d\mid n$ will be odd, so
	\[\Phi_n(X)\Phi_n(-X)=\prod_{d\mid n}\left(X^d-1\right)^{\mu(n/d)}\left((-X)^d-1\right)^{\mu(n/d)}=\prod_{d\mid n}(-1)^{\mu(n/d)}\left(X^{2d}-1\right)^{\mu(n/d)}.\]
	Because $\deg\Phi_n(X)=\varphi(n)$ is even, we know in advance that $\Phi_n(X)\Phi_(-X)$ is the product of monic polynomials, so the above becomes an equality. So $\Phi_{2n}(X)=\Phi_n\left(X^2\right)/\Phi_n(X)=\Phi_n(-X),$ finishing.
\end{example}
The point of these results above is that it implies that many $\Phi_n(X)$ will have coefficients in $\{-1,0,1\}.$
\begin{exe}
	Fix $p$ and $q$ distinct odd primes and suppose $n:=2^ap^bq^c$ for nonnegative integers $a,b,c$; in other words, $n$ has at most two distinct odd prime divisors. Then $\Phi_n(X)$ only has coefficients in $\{-1,0,1\}.$
\end{exe}
\begin{proof}
	We have the following cases.
	\begin{itemize}
		\item If $\{a,b,c\}=\{0\},$ then we are looking at $\Phi_1(X)=X-1.$
		\item If two of $\{a,b,c\}$ are zero while the third is nonzero, then we are evaluating
		\[\Phi_{r^d}(X)=\sum_{k=0}^{d-1}X^{r^k}\]
		for some prime-power $r^k,$ so indeed, our coefficients are $\{-1,0,1\}.$
		\item If one of $\{a,b,c\}$ is zero while the other two are nonzero, then we are rename $p$ and $q$ so that $n=p^aq^b$ for primes $p,q$ where $a,b>0.$ Now, by inductively applying \autoref{lem:cyclocompute} part (b), we see that
		\[\Phi_{p^aq^b}(X)=\Phi_{pq}\left(X^{p^{a-1}q^{b-1}}\right).\]
		So because $\Phi_{pq}(X)$ has coefficients in $\{-1,0,1\},$ we see that $\Phi_{p^aq^b}(X)$ will also have coefficients in $\{-1,0,1\}.$
		\item Lastly, take all of $\{a,b,c\}$ nonzero. Again inductively applying \autoref{lem:cyclocompute}, we see that
		\[\Phi_{2^ap^bq^c}(X)=\Phi_{2pq}\left(X^{2^{a-1}p^{b-1}q^{c-1}}\right).\]
		Now, $pq$ is odd, so $\Phi_{2^ap^bq^c}(X)=\Phi_{pq}\left(-X^{2^{a-1}p^{b-1}q^{c-1}}\right)$ will still have all coefficients in $\{-1,0,1\}$ again from $\Phi_{pq}(X).$
		\qedhere
	\end{itemize}
\end{proof}
In particular, if we are to have coefficients outside of $\{-1,0,1\},$ we must have at least three distinct odd prime divisors, for which $\Phi_{3\cdot5\cdot7}(X)=\Phi_{105}(X)$ is the first candidate. This polynomial has degree $48,$ so actually computing it by hand would be quite annoying, but indeed it does have a
\[-2X^{41}\]
term. So $\boxed{\Phi_{105}(X)}$ is the first cyclotomic polynomial with a coefficient outside of $\{-1,0,1\}.$

\subsection{Cyclotomic Polynomials: Application}
As an application, we show a special case of Dirichlet's theorem on arithmetic progressions.
\begin{exe}
	We show that there are infinitely many primes $1\pmod n$ for each positive integer $n.$
\end{exe}
\begin{proof}
	The main point is to take primes $p\mid\Phi_n(b)$ for some $b.$ We have the following lemma.
	\begin{lemma}
		Fix $n$ a positive integer and $p\nmid n$ a prime factor of $\Phi_n(b)$ for some integer $b.$ Then the order of $b\pmod p$ is $n.$
	\end{lemma}
	\begin{proof}
		Now, we see that
		\[b^n\equiv1\pmod p\]
		because $\Phi_n(b)\mid b^n-1,$ so the order of $b\pmod p$ divides into $n.$ Because the roots of $\Phi_n(X)\mid X^n-1$ are distinct$\pmod p$ when $p\nmid b,$ we can be sure that $b\pmod p$ has exactly the order $n.$ To be explicit, we proceed as in \autoref{lem:embedroots}: if the order is $m\mid n,$ with $m<n,$ then $b$ will be a root of
		\[\Phi_m(X)=\prod_{d\mid m}\left(X^d-1\right)^{\mu(m/d)},\]
		which forces
		\[X^n-1=\prod_{d\mid n}\Phi_d(X)\]
		to have a double root, which is a contradiction.
	\end{proof}
	In particular, given $p\mid\Phi_n(b)$ where $p\nmid n,$ we see that applying Lagrange's theorem to $\left(\ZZ/p\ZZ\right)^\times,$ we find that our element of order $n$ witnesses $p\equiv1\pmod n.$

	So to finish, we suppose for the sake of contraction that there are finitely many primes which are $1\pmod n.$ Then let their (finite) product be $P,$ and we look at the polynomial
	\[\Phi_n(PnX).\]
	Any prime $p$ dividing into this will be coprime to $n$ and $P$ (because $\Phi_n(PnX)\equiv\Phi_n(0)\equiv\pm1\pmod{nP}$), forcing $p\equiv1\pmod n$ by the argument above.
	
	But now, sending $X\to\infty$ will make $\Phi_n(PnX)$ number large enough to be at least $1$ and hence have a prime factor, so we have a prime $p\equiv1\pmod n$ not dividing $P,$ which is our contradiction.
\end{proof}
\begin{remark}
	This is not a good way to find $1\pmod n$ primes because $\Phi_n(PnX)$ will get quite large quite quickly. For example, to find $1\pmod{10}$ primes, we want divisors of
	\[\Phi_{10}(X)=X^4-X^3+X^2-X+1,\]
	but as soon as we have one prime $11,$ we want to compute $\Phi_{10}(110x),$ which is huge.
\end{remark}
\begin{example}
	In the case of $\Phi_4(X)=X^2+1,$ we see that all prime factors of $X^2+1$ for $X$ even will be $1\pmod4.$
\end{example}
Next lecture will be on the coming Tuesday because there is some sort of holiday or something.