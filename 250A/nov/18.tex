% !TEX root = ../notes.tex

















This is a roadkill song about a kid who followed the bouncing ball in a singalong.

\subsection{Normal Loose Ends}
Last time we were cut off discussing normal extensions. We will give an alternate proof of the fact that normal extensions correspond to normal subgroups. Here is the key lemma.
\begin{lemma} \label{lem:conjugategalois}
	Fix $M/K$ a Galois extension with Galois group $G:=\op{Gal}(M/K)$ and $L$ an intermediate extension with corresponding subgroup $H:=\op{Gal}(M/L)\subseteq G$ fixing $L.$ Then, for any $\sigma\in G,$ the subgroup fixing $\sigma L$ is $\op{Gal}(M/\sigma L)=\sigma H\sigma^{-1}.$
\end{lemma}
\begin{proof}
	Fix $\sigma\in G.$ We start by noting that $\sigma L$ is a field because $\sigma$ is an automorphism, so the field structure of $L$ will carry over to $\sigma L.$\footnote{For example, $\sigma L$ has inverses: for any $\sigma\alpha\in(\sigma L)\setminus\{0\},$ we have $\alpha\ne0,$ so $(\sigma\alpha)^{-1}=\sigma\left(\alpha^{-1}\right)$ provides an inverse.}

	We are interested in computing $\op{Gal}(M/\sigma L).$ Well, certainly $\op{Gal}(M/\sigma L)\subseteq\op{Gal}(M/K)$ because $K\subseteq\sigma L,$ so it suffices to check which $\tau\in G$ fix $\sigma L.$ But now any element of $\sigma L$ takes the form $\sigma\alpha$ for some $\alpha\in L,$ so $\tau\in G$ fixes each $\sigma\alpha\in L$ if and only if
	\[\tau\sigma\alpha=\sigma\alpha\]
	if and only if $\left(\sigma^{-1}\tau\sigma\right)(\alpha)=\alpha$ if and only if $\sigma^{-1}\tau\sigma$ fixes $L.$ But the subgroup fixing $L$ is $H,$ so this is equivalent to $\tau\in\sigma H\sigma^{-1},$ so indeed the subgroup of $G$ fixing $L$ is $\sigma H\sigma^{-1}.$
\end{proof}
\begin{remark}[Nir]
	Mnemonically, we have
	\[\left(\sigma H\sigma^{-1}\right)(\sigma L)=\sigma\left(H\cdot\sigma^{-1}\sigma L\right)=\sigma(H\cdot L)=\sigma L,\]
	where we have commited heavy abuse of notation mutliplying the subgroup $H$ by a field $L.$
\end{remark}
Another way to phrase the above lemma is that the set of intermediate extensions of $M/K$ and subgroups of $G$ are isomorphic as $G$-sets, where the bijection is by the Galois correspondence by
\[\varphi:L\mapsto\op{Gal}(M/L).\]
To be explicit, the $G$-action, for some $\sigma\in G,$ on the intermediate extensions is by $\sigma\cdot L=\sigma L$ and on the subgroups by conjugation.

Indeed, we only need to show that this mapping is a $G$-set homomorphism because we already know it is bijective. Well, fixing some $\sigma\in G,$ we see that
\[\varphi(\sigma\cdot L)=\varphi(\sigma L)=\op{Gal}(M/\sigma L)=\sigma\op{Gal}(M/L)\sigma^{-1}=\sigma\cdot\op{Gal}(M/L)=\sigma\cdot\varphi(L),\]
which is what we wanted.

In particular, to show that normal subgroups correspond to normal extensions, we note that normal subgroups are exactly the subgroups fixed by the $G$-action (by conjugation), so by the above isomorphism as $G$-sets, it suffices to talk about the intermediate extensions fixed by the $G$-action.
\begin{lemma}
	Fix $M/K$ a (finite) Galois extension with Galois group $G:=\op{Gal}(M/K).$ Then an intermediate extension $L$ is normal if and only if $\sigma L=L$ for each $\sigma\in G.$
\end{lemma}
\begin{proof}
	This is somewhat techincal; we more or less showed this last time. Fix an algebraic closure $\overline K.$ In one direction, if $L$ is normal, then we note that each $\sigma\in G$ restricts to a map
	\[L\stackrel\sigma\to\sigma L\subseteq M\subseteq\overline K.\]
	But any embedding $L\into\overline K$ must output into $L$ because $L$ is normal, so the composite of the above maps into $L,$ so $\sigma L\subseteq L.$ A similar argument shows $\sigma^{-1}L\subseteq L,$ so $L\subseteq\sigma L$ as well, so $L=\sigma L.$

	For the other direction, suppose $\sigma L=L$ for each $\sigma\in G.$ Fix any embedding $\sigma:L\into\overline K$ so that we want to show $\sigma L\subseteq L.$ Well, by using some chain argument, we can extend $\sigma:L\into\overline K$ to
	\[\sigma:M\into\overline K,\]
	but now we know that $M$ is normal, so $\sigma$ must output into $M.$ In particular, $\sigma:M\into M$ fixing $K,$ so we claim $\sigma\in\op{Gal}(M/K).$ The only concern is for surjectivity, but we notice that $\sigma M\subseteq M$ while $[M:K]=[\sigma M:K]$ by tracking a basis, so $M=\sigma M.$\footnote{We have used the fact that $M/K$ is finite (or at lease profinite) here.}	So $\sigma\in\op{Gal}(M/K),$ so $\sigma L=L,$ so $L$ is indeed normal.
\end{proof}
So we get the following result.
\begin{proposition}
	Fix $M/K$ a (finite) Galois extension with Galois group $G:=\op{Gal}(M/K).$ Then normal extensions correspond to normal subgroups. Explicitly, we have the following.
	\begin{itemize}
		\item Fix $L/K$ a normal intermediate extension. Then $\op{Gal}(M/L)\subseteq G$ is normal.
		\item FIx $H\subseteq G$ a normal subgroup. Then $M^H/K$ is a normal extension.
	\end{itemize}
\end{proposition}
\begin{proof}
	These more or less follow directly from the above discussion.
	\begin{itemize}
		\item The extension $L/K$ is normal if and only if $L$ is a fixed point of the $G$-action on intermediate extensions if and only if $\op{Gal}(M/L)\subseteq G$ is a fixed point of the $G$-action on subgroups if and only if $\op{Gal}(M/L)\subseteq G$ is normal.
		\item The subgroup $H\subseteq G$ is the subgroup fixing $M^H$ with $\op{Gal}(M/M^H)=H$ by the Galois correspondence, so the previous part promises that $M^H/K$ is normal because $\op{Gal}(M/M^H)\subseteq G$ is a normal subgroup.
		\qedhere
	\end{itemize}
\end{proof}
To close off, we note that
\[\op{Gal}(L/K)\cong\frac{\op{Gal}(M/K)}{\op{Gal}(M/L)}\]
as we claimed last time. Indeed, as we said last time, we have a map $\op{Gal}(M/K)\to\op{Gal}(L/K)$ by restriction, and it has kernel $\op{Gal}(M/L).$

Anyways, here is an example.
\begin{exe}
	We find the intermediate normal extensions of $\QQ(\sqrt[3]2,\omega)/\QQ.$
\end{exe}
\begin{proof}
	Here is our lattice of subgroups.
	% https://q.uiver.app/?q=WzAsNixbMiwwLCJcXGxhbmdsZSBlXFxyYW5nbGUiXSxbMSwxLCJcXGxhbmdsZSgxMilcXHJhbmdsZSJdLFsyLDEsIlxcbGFuZ2xlKDIzKVxccmFuZ2xlIl0sWzQsMiwiXFxsYW5nbGUoMTIzKVxccmFuZ2xlIl0sWzAsMSwiXFxsYW5nbGUoMzIpXFxyYW5nbGUiXSxbMiwzLCJTXzMiXSxbMCw0LCIiLDAseyJzdHlsZSI6eyJoZWFkIjp7Im5hbWUiOiJub25lIn19fV0sWzAsMSwiIiwyLHsic3R5bGUiOnsiaGVhZCI6eyJuYW1lIjoibm9uZSJ9fX1dLFswLDIsIiIsMix7InN0eWxlIjp7ImhlYWQiOnsibmFtZSI6Im5vbmUifX19XSxbMCwzLCIiLDIseyJzdHlsZSI6eyJoZWFkIjp7Im5hbWUiOiJub25lIn19fV0sWzQsNSwiIiwwLHsic3R5bGUiOnsiaGVhZCI6eyJuYW1lIjoibm9uZSJ9fX1dLFsxLDUsIiIsMix7InN0eWxlIjp7ImhlYWQiOnsibmFtZSI6Im5vbmUifX19XSxbMiw1LCIiLDIseyJzdHlsZSI6eyJoZWFkIjp7Im5hbWUiOiJub25lIn19fV0sWzMsNSwiIiwyLHsic3R5bGUiOnsiaGVhZCI6eyJuYW1lIjoibm9uZSJ9fX1dXQ==
	\[\begin{tikzcd}
		&& {\langle e\rangle} \\
		{\langle(12)\rangle} & {\langle(23)\rangle} & {\langle(31)\rangle} \\
		&&&& {\langle(123)\rangle} \\
		&& {S_3}
		\arrow[no head, from=1-3, to=2-1]
		\arrow[no head, from=1-3, to=2-2]
		\arrow[no head, from=1-3, to=2-3]
		\arrow[no head, from=1-3, to=3-5]
		\arrow[no head, from=2-1, to=4-3]
		\arrow[no head, from=2-2, to=4-3]
		\arrow[no head, from=2-3, to=4-3]
		\arrow[no head, from=3-5, to=4-3]
	\end{tikzcd}\]
	And here is our lattice of fields, where we have numbered the roots $\{\sqrt[3]2,\omega\sqrt[3]2,\omega^2\sqrt[3]2$ off by $\{1,2,3\}$ respectively. We provided the details to this last time.
	% https://q.uiver.app/?q=WzAsNixbMiwwLCJcXFFRKFxcc3FydFszXTIsXFxvbWVnYSkiXSxbMSwxLCJcXFFRKFxcc3FydFszXTIpIl0sWzIsMSwiXFxRUShcXG9tZWdhXFxzcXJ0WzNdMikiXSxbNCwyLCJcXFFRKFxcb21lZ2EpIl0sWzAsMSwiXFxRUShcXG9tZWdhXjJcXHNxcnRbM10yKSJdLFsyLDMsIlxcUVEiXSxbMCw0LCIiLDAseyJzdHlsZSI6eyJoZWFkIjp7Im5hbWUiOiJub25lIn19fV0sWzAsMSwiIiwyLHsic3R5bGUiOnsiaGVhZCI6eyJuYW1lIjoibm9uZSJ9fX1dLFswLDIsIiIsMix7InN0eWxlIjp7ImhlYWQiOnsibmFtZSI6Im5vbmUifX19XSxbMCwzLCIiLDIseyJzdHlsZSI6eyJoZWFkIjp7Im5hbWUiOiJub25lIn19fV0sWzQsNSwiIiwwLHsic3R5bGUiOnsiaGVhZCI6eyJuYW1lIjoibm9uZSJ9fX1dLFsxLDUsIiIsMix7InN0eWxlIjp7ImhlYWQiOnsibmFtZSI6Im5vbmUifX19XSxbMiw1LCIiLDIseyJzdHlsZSI6eyJoZWFkIjp7Im5hbWUiOiJub25lIn19fV0sWzMsNSwiIiwyLHsic3R5bGUiOnsiaGVhZCI6eyJuYW1lIjoibm9uZSJ9fX1dXQ==&macro_url=https%3A%2F%2Fgist.githubusercontent.com%2FdFoiler%2F1e12fec404cad7e185260f0c9b68977d%2Fraw%2F909cc7837a29133fb63fb0e9300d15bfe7417fc5%2Fnir.sty
	\[\begin{tikzcd}
		&& {\QQ(\sqrt[3]2,\omega)} \\
		{\QQ(\omega^2\sqrt[3]2)} & {\QQ(\sqrt[3]2)} & {\QQ(\omega\sqrt[3]2)} \\
		&&&& {\QQ(\omega)} \\
		&& \QQ
		\arrow[no head, from=1-3, to=2-1]
		\arrow[no head, from=1-3, to=2-2]
		\arrow[no head, from=1-3, to=2-3]
		\arrow[no head, from=1-3, to=3-5]
		\arrow[no head, from=2-1, to=4-3]
		\arrow[no head, from=2-2, to=4-3]
		\arrow[no head, from=2-3, to=4-3]
		\arrow[no head, from=3-5, to=4-3]
	\end{tikzcd}\]
	So, for example, we see that conjugation by $(12)$ takes $(23)$ to $(12)(23)(12)=(13)$ and so the subgroup $\langle(12)\rangle$ to $\langle(13)\rangle.$ Isomorphcially, the action by $(12)$ will permute the corresponding fields $\QQ(\sqrt[3]2)$ and $\QQ(\omega\sqrt[3]2).$ In particular, the extensions $\QQ(\omega^\bullet\sqrt[3]2)/\QQ$ are not normal.

	However, the subgroup $\langle(123)\rangle$ is normal (it is index $2$ in $S_3$), so the extension $\QQ(\omega)/\QQ$ is a normal extension (it is quadratic), and the Galois group here is
	\[\op{Gal}(\QQ(\omega)/\QQ)\cong\frac{\op{Gal}(\QQ(\sqrt[3]2,\omega)/\QQ)}{\op{Gal}(\QQ(\sqrt[3]2,\omega)\rangle)}\cong\frac{S_3}{\langle(123)\rangle}\cong\ZZ/2\ZZ.\]
	This is what we wanted.
\end{proof}

\subsection{Inverse Galois Problem: Cyclic Extensions}
Before jumping into the proof, we pick up the following technical lemma.
\begin{lemma}[Nir] \label{lem:normalcyclotomic}
	Fix $p$ a prime, and fix $K:=\QQ(\zeta_p)$ so that $G:=\op{Gal}(K/\QQ)\cong(\ZZ/p\ZZ)^\times.$ Then, for any subgroup $H\subseteq G,$ we have
	\[K^H=\QQ\left(\sum_{\sigma\in H}\sigma\zeta_p\right).\]
\end{lemma}
\begin{proof}
	This is surprisingly technical. Fix $\alpha:=\sum_{\sigma\in H}\sigma\zeta_p.$ We show that $H=\op{Gal}(\QQ(\zeta_p)/\QQ(\alpha)),$ which will be enough by the Galois correspondence. Well, $\tau\in G$ fixes $\QQ(\alpha)$ if and only if $\tau$ fixes $\alpha$ ($\tau$ already fixes $\QQ$) if and only if
	\[\sum_{\sigma\in H}(\tau\sigma)\zeta_p=\tau\cdot\sum_{\sigma\in H}\sigma\zeta_p=\tau\alpha=\alpha=\sum_{\sigma\in H}\sigma\zeta_p.\]
	Now, if $\tau\in H,$ then the map $\sigma\mapsto\tau\sigma$ is a bijection $H\to H,$ so $\tau$ certainly fixes $H.$

	The converse requires a little more care. The main point is that, because $p$ is prime, we see $\{\sigma\zeta_p\}_{\sigma\in G}$ is a basis for $\QQ(\zeta_p)/\QQ.$ Indeed, by our classification of $\op{Gal}(\QQ(\zeta_p)/\QQ),$ we have
	\[\{\sigma\zeta_p\}_{\sigma\in G}=\left\{\zeta_p^k\right\}_{k=1}^{p-1}.\]
	There are $p-1$ of these elements, which is indeed the degree $[\QQ(\zeta_p):\QQ],$ and these elements are $\QQ$-linearly independent because the minimal polynomial of $\zeta_p$ has degree $p-1.$

	Thus, we see that both sides of the equality
	\[\sum_{\sigma\in H}(\tau\sigma)\zeta_p=\sum_{\sigma\in H}\sigma\zeta_p.\]
	feature decompositions of the same element under a basis, so they must be permutations of each other. In particular, $\tau\zeta_p$ appears somewhere on the right, so $\tau\in H.$ This finishes.
\end{proof}
\begin{remark}
	In office hours, Professor Borcherds pointed out that this need not be true if we remove the prime condition: for $\zeta_8,$ the elements $\{\sigma\zeta_8\}_{\sigma\in\op{Gal}(\QQ(\zeta_8)/\QQ)}=\{\zeta_8,\zeta_8^3,\zeta_8^5,\zeta_8^7\}$ are not linearly independent (e.g., they sum to $0$). We can manifest this into a problem as
	\[\zeta_8+\zeta_8^5=0=\zeta_8^3+\zeta_8^7,\]
	and, in particular, the fixed field of $H=\{1,5\}$ is not $\QQ\left(\zeta_8+\zeta_8^5\right)=\QQ.$
\end{remark}
With that annoyance out of the way, let's move into examples.
\begin{exe}
	We find a $\QQ$-extension with Galois group $\ZZ/5\ZZ.$
\end{exe}
\begin{proof}
	By our discussion of normal subgroups and quotients it suffices to find some extension $L/\QQ$ such that $\op{Gal}(L/\QQ)$ has $\ZZ/5\ZZ$ as a quotient. Well, we have that
	\[\op{Gal}(\QQ(\zeta_{11})/\QQ)\cong(\ZZ/11\ZZ)^\times\cong\ZZ/10\ZZ,\]
	where the first isomorphism is by associating the automorphism $\sigma_k:\zeta_{11}\mapsto\zeta_{11}^k$ to $k\in(\ZZ/11\ZZ)^\times.$

	Now, $\ZZ/10\ZZ$ does surject onto $\ZZ/5\ZZ,$ so tracking things backwards, we are looking at the quotient
	\[\ZZ/5\ZZ\cong\frac{\ZZ/10\ZZ}{5\ZZ/10\ZZ}\cong\frac{\op{Gal}(\QQ(\zeta_{11})/\QQ)}{\langle\zeta_{11}\mapsto\zeta_{11}^{-1}\rangle}.\]
	In other words, we want the elements of $\QQ(\zeta_{11})$ which are fixed by the action $\zeta_{11}\mapsto\zeta_{11}^{-1},$ so \autoref{lem:normalcyclotomic} tells us that this field is
	\[\QQ\left(\zeta_{11}+\zeta_{11}^{-1}\right)\cong\QQ\left(\cos\left(\frac{2\pi}{11}\right)\right).\]
	Namely,
	\[\boxed{\op{Gal}\left(\QQ\left(\cos\left(\frac{2\pi}{11}\right)\right)/\QQ\right)\cong\ZZ/5\ZZ}.\]
	This is what we wanted.
\end{proof}
Here is the general statement.
\begin{prop}
	We find a $\QQ$-extension with Galois group $\ZZ/n\ZZ.$
\end{prop}
\begin{proof}
	We again start with $\QQ(\zeta_p)$ for some prime $p$ to be chosen later. Then we find that
	\[\op{Gal}(\QQ(\zeta_p)/\QQ)\cong(\ZZ/p\ZZ)^\times\cong\ZZ/(p-1)\ZZ.\]
	To get this to surject onto $\ZZ/n\ZZ,$ so that means we want $p\equiv1\pmod n,$ of which there are infinitely many by Dirichlet's theorem on arithmetic progressions.
	
	To finish, we can do the algorithm suggested above. Fix $g$ a generator of $(\ZZ/p\ZZ)^\times$ so that $g^{n}$ has order $(p-1)/n$ and so generates a subgroup of order $\frac{p-1}n.$ This subgroup will be normal because the group $(\ZZ/p\ZZ)^\times$ is abelian, so accordingly we set
	\[\alpha:=\sum_{k\in\left\langle g^n\right\rangle}\zeta_p^k\]
	so that
	\[\op{Gal}(\QQ(\alpha)/\QQ)\cong\frac{\op{Gal}(\QQ(\zeta_p)/\QQ)}{\op{Gal}(\QQ(\alpha)/\QQ)}\cong\frac{\langle g\rangle}{\langle g^n\rangle}\cong\ZZ/n\ZZ.\]
	This is what we wanted.
\end{proof}
Let's see this in practice again.
\begin{example}
	We find a $\QQ$-extension with Galois group $\ZZ/7\ZZ.$ We fix $p:=29\equiv1\pmod7.$ So we would need to find the subgroup of $(\ZZ/29\ZZ)^\times$ fixed by $\langle\sqrt{-1}\rangle=\langle12\rangle,$ so we can find that our subfield is
	\[\QQ\left(\zeta_{29}+\zeta_{29}^{-1}+\zeta_{29}^{12}+\zeta_{29}^{-12}\right).\]
	This is what we wanted.
\end{example}
As an aside, we note that our application of Dirichlet's theorem was a bit unnecessary because there are easier ways to go about this.
\begin{lemma}
	Fix $n$ a prime. Then there are infinietly many primes $p\equiv1\pmod n.$
\end{lemma}
\begin{proof}
	The main idea is to  look at the prime factors of
	\[\Phi_n(m):=\frac{m^n-1}{m-1}\]
	as $m$ varies. Indeed, if $p$ divides $\frac{m^n-1}{m-1},$ then $p\mid m^n-1,$ so $m\pmod p$ will have multiplicative order dividing $n.$ Because $n$ is prime, the multiplicative order will thus either be $1$ or $n.$ We deal with these cases one at a time.
	\begin{itemize}
		\item If the multiplicative order is $1,$ then $m\equiv1\pmod p,$ so
		\[\frac{m^n-1}{m-1}=1+m+\cdots+m^{n-1}\equiv\underbrace{1+1+\cdots+1}_n=n\pmod p.\]
		So because $p$ divides the left-hand side, $p\mid n$ as well.
		\item If the multiplicative order is $n,$ then $n\mid\#(\ZZ/p\ZZ)^\times=p-1$ by Lagrange's theorem on groups, so $n\mid p-1.$
	\end{itemize}
	So $p\mid\Phi_n(m)$ implies that $p\mid n$ or $p\equiv1\pmod n.$ We would like to focus on the second kind of prime, so we note that $p\mid\Phi_n(nm)$ has $\Phi_n(nm)\equiv1\pmod n,$ so $p\nmid n,$ and $p\equiv1\pmod n$ is forced.

	Now we finish the proof in a Euclidean way. We show that no finite set $S$ contains all $1\pmod n$ primes. Indeed, let the product of the primes in $S$ be $P,$ and we study
	\[\Phi_n(knP)\]
	as $k\to\infty.$ In particular, for sufficiently large $k,$ we can promise\footnote{The only reason to introduce this $k$ variable is this end behavior argument. It is surprisingly annoying.} $\Phi_n(knP)>1$ so that it must have a prime factor $p.$ By the argument above, $p\equiv1\pmod p,$ but we can also see that $p\nmid P,$ so $p\notin S.$ This finishes.
\end{proof}
\begin{remark}[Nir]
	Something like his can be done for more general $n,$ using cyclotomic polynomials in a similar way.
\end{remark}

\subsection{Inverse Galois Problem: Symmetric Groups}
We have the following exercise.
\begin{exe}
	We find an extension of $\QQ$ with Galois group $S_5.$
\end{exe}
\begin{proof}
	We take $L$ to be the splitting field of $f(x):=x^5-4x+2,$ which is irreducible by Eisenstein's criterion at the prime $2.$ Now we have the following observations.
	\begin{itemize}
		\item Surely $\op{Gal}(L/\QQ)\subseteq S_5$ because $\op{Gal}(L/\QQ)$ acts on the roots of $f,$ and this action determines the rest of the automorphism because $L$ is generated by the roots of $f.$
		\item The fact that $\deg f=5$ is quintic implies that there is a subfield of degree $5,$ so $5\mid[L:\QQ]=\#\op{Gal}(L/\QQ).$ Thus, $\op{Gal}(L/\QQ)$ contains a $5$-cycle by Cauchy's theorem.
		\item The polynomial $f(x)$ has exactly thee real roots, which we can check graphically (we won't do this here). In particular, the action of complex conjugation restricted to $L$ induces an automorphism of $L/\QQ,$ and this automorphism must permute two roots. So $\op{Gal}(L/\QQ)$ has a transposition.
	\end{itemize}
	But now the point is that any $5$-cycle and $2$-cycle in $S_5$ will generate all of $S_5.$ Indeed, we have the following lemma.
	\begin{lemma}
		Fix $p$ a prime. Then the $p$-cycle $(0,1,\ldots,p-1)\in S_p$ and any transposition $(a,b)\in S_p$ will fully generate $S_p.$
	\end{lemma}
	\begin{proof}
		Without loss of generality, take $a<b.$ Set $\sigma:=(0,1,\ldots,p-1).$ We see that
		\[\sigma^{p-a}(a,b)\sigma^{-(p-a)}=(0,b-a),\]
		so we have some transposition of the form $(0,c)$ where $c\ne0.$ Then we see that
		\[\sigma^{kc}(0,c)\sigma^{-kc}=(kc,(k+1)c),\]
		so we may chain
		\[(c,2c)(0,c)(c,2c)=(0,2c),\quad(2c,3c)(0,2c)(2c,3c)=(0,3c),\quad\cdots.\]
		The point is that we can get $(0,ck)$ for any nonnegative integer $k,$ so taking $k\equiv c^{-1}\pmod p$ (here we use the fact that $p$ is prime!), we see that we can get $(0,1).$ Repeating the above chain, we see that we can get $(0,1)$ and $(0,2)$ and $(0,3)$ and so on. Further, we then see we can get
		\[(0,k)(0,\ell)(0,k)=(k,\ell)\]
		for any $k$ and $\ell,$ so we can get any transposition. So it follows that we can indeed get all of $S_p.$
	\end{proof}
	Finishing up, we number the roots so that the $5$-cycle in $\op{Gal}(L/\QQ)$ is $(12345),$ and noting that conjugation gives us our transposition as above, we see that these fully generate an $S_5.$ So we do find
	\[\op{Gal}(L/\QQ)\cong S_5,\]
	which is what we wanted.
\end{proof}
A similar approach will work for any prime $p,$ not just $5.$ We are restricted to primes to make the final argument about the transposition and $5$-cycle to work.
\begin{proposition} \label{prop:gettingsymmetric}
	For any prime $p,$ there exists a Galois extension $K/\QQ$ with Galois group $S_p.$
\end{proposition}
\begin{proof} 
	Rigorizing this is a bit annoying, but here is one sketch: by the argument above provided in the example, we need to find an irreducible polynomial $f\in\QQ[x]$ with degree $p$ and exactly $2$ complex roots. Well, we can start with
	\[g(x)=\left(x^2+1\right)\cdot\prod_{k=1}^{p-2}(x-k),\]
	which does indeed have exactly $2$ complex roots (in particular intersecting the $x$-axis $p-2$ times). Because $g$ has no repeated roots and is locally linear, it follows there is an interval $(-\alpha,\alpha)$ such that any $\varepsilon\in(-\alpha,\alpha)$ will still have $g(x)+\varepsilon$ with exactly $p-2$ real roots.
	
	Because the generic polynomial is irreducible, such an $\varepsilon\in\QQ$ should exist to make $g(x)+\varepsilon$ irreducible; for example, for sufficiently large primes $q,$ we have $\varepsilon=\frac1{q^{\deg g-1}}<\alpha$ will make
	\[f(x):=q^{\deg g}\left(g\left(\frac xq\right)+\frac1{q^{\deg g-1}}\right)=q^{\deg g}g\left(\frac xq\right)+q\in\ZZ[x]\]
	Eisenstein at the prime $q$ while still having exactly $p-2$ real roots.
\end{proof}
Here is a nice consequence.
\begin{proposition}
	Fix $G$ any finite group. Then we can find an extension $M/L$ of $\QQ$ such that $\op{Gal}(M/L)\cong G.$
\end{proposition}
\begin{proof}
	The point is that, by \autoref{prop:gettingsymmetric}, we may take $M/\QQ$ to have Galois group $S_p$ such that $G\subseteq S_p$; for example, if $p>\#G,$ then we can embed $G\into S_{\#G}$ (by having $G$ act on itself by left multiplication) and then embed $S_{\#G}\into S_p$ (by fixing the last $p-\#G$ coordinates). Then we take $L=M^G$ so that
	\[\op{Gal}(M/L)=\op{Gal}\left(M/M^G\right)\cong G\]
	by the Galois correspondence.
\end{proof}

\subsection{Cubic Polynomials}
Finding the Galois group of specific polynomials is somewhat hard; let's see what we can do with cubic polynomials.

Fix $f\in K[x]$ a cubic irreducible polynomial with $L/K$ its splitting field. As with last time, we start with the following two facts.
\begin{itemize}
	\item The Galois group $\op{Gal}(L/K)$ is contained in $S_3$ because $S_3$ will act on the roots of $f,$ and the action on the roots will uniquely determine an automorphism because $L$ is genreated by these roots.
	\item Because $f$ is of degree $3,$ adjoining one root creates a cubic subextension of the splitting field, so $3\mid[L:K]=\#\op{Gal}(L/K).$
\end{itemize}
So we find that must be a subgroup of $S_3$ with at least $3$ elements, of which our options are
\[\op{Gal}(L/K)\in\{A_3,S_3\}.\]
We would like to determine between the two. Quickly we verify that both are possible.
\begin{example}
	The splitting field of $x^3-2$ over $\QQ$ has Galois group $\op{Gal}(\QQ(\sqrt[3]2,\omega)/\QQ)\cong S_3,$ as we discussed in an earlier example.
\end{example}
\begin{example}
	In the case of $x^3+x+1$ over $\FF_2,$ we are looking at $\op{Gal}(\FF_8/\FF_2)\cong\ZZ/3\ZZ,$ where the Galois group is generated by the Frobenius automorphism.
\end{example}
To describe our algorithm, we fix $\alpha,\beta,\gamma$ the roots of $f$ in $L.$ The main idea is to fix
\[\Delta:=(\alpha-\beta)(\beta-\gamma)(\gamma-\alpha).\]
Indeed, we see that $\Delta^2$ is fully fixed by any permutation of the roots given by $\op{Gal}(L/K),$ so $\Delta^2\in K.$ So the question is if $\Delta\in K$ or $\Delta\notin K.$ We have the following cases.
\begin{itemize}
	\item Take $\Delta\in K.$ Then any $\sigma\in\op{Gal}(L/K)$ will fix $\Delta,$ so in particular, $\op{Gal}(L/K)$ does not contain the transposition $(\alpha,\beta).$ So $\op{Gal}(L/K)$ is striclty contained in $S_3,$ so $\op{Gal}(L/K)\cong A_3.$
	\item Take $\Delta\notin K.$ But we do know $\Delta^2\in K$ and $\Delta\in L,$ so the chain
	\[K\subseteq K(\Delta)\subseteq L\]
	provides a quadratic subextension of $L.$ In particular, $2\mid[L:K]=\#\op[Gal](L/K),$ so $\#\op{Gal}(L/K)\ge6,$ so $\op{Gal}(L/K)\cong S_3.$\footnote{We could also argue as we did before: the fact that $\Delta\notin K$ implies that there must be an element in $\op{Gal}(L/K)\setminus A_3,$ so $\op{Gal}(L/K)\cong S_3.$}
\end{itemize}
So we see that $\Delta\in K$ can detect the Galois group.

But now we notice that $\Delta^2$ was just the discriminant all along, so we know how to compute this. In particular, if our cubic polynomial is $x^3+bx+c,$ then we are asking if $-4b^3-27c^2$ is a square in $K.$ This gives the following result.
\begin{prop}
	Fix $K$ a field and $f\in K[x]$ a cubic polynomial in the form $f(x)=x^3+bx+c.$ Set $\Delta^2:=-4b^3-27c^2,$ and we have two cases.
	\begin{itemize}
		\item If $\Delta^2$ is a square in $K,$ then the Galois group of $f$ is isomorphic to $A_3\cong\ZZ/3\ZZ.$
		\item If $\Delta^2$ is not a square in $K,$ then the Galois group of $f$ is isomorphic so $S_3.$
	\end{itemize}
\end{prop}
\begin{proof}
	This essentially follows from the above discussion. The point is that $\Delta^2$ is a square in $K$ if and only if $\pm\Delta\in K,$ so we get to reduce to the casework from earlier.
\end{proof}
Let's finish with some examples.
\begin{example}
	Fix $x^3-3x-1\in\QQ[x].$ We can see this is irreducible because it has no roots in $\QQ.$ Now, we can compute the discriminant as
	\[\Delta=-4(-3)^3-27(-1)^2=81,\]
	which is a square in $\QQ,$ so the Galois group of $x^3-3x-1$ is $A_3.$
\end{example}
\begin{example}
	Fix $x^3-x-1\in\QQ[x].$ We can see this is irreducible because it has no roots in $\QQ.$ Now, we can compute the discriminant as
	\[\Delta=-4(-1)^3-27(-1)^2=-23,\]
	which is not a square in $\QQ,$ so the Galois group of $x^3-x-1$ is $S_3.$
\end{example}
We close this subsection with a remark.
\begin{remark}
	There is an analogous process for degree-$4$ polynomial, but it gets very annoying. It can be done by hand, but it requires a lot of invariant computations. In general, degree $2$ is easy, $3$ is fine, $4$ is really annoying, and $5$ and up need a computer.
\end{remark}

\subsection{Fundamental Theorem of Algebra}
Let's give some proofs of the Fundamental theorem of algebra.
\begin{theorem}[Fundamental theorem of algebra]
	We have that $\CC$ is algebraically closed.
\end{theorem}
\begin{proof}[Proof by complex analysis]
	We sketch a proof using Louiville's theorem. Given any nonconstant polynomial $p\in\CC[z],$ we show that it has a root somewhere.
	
	Suppose that $p$ has no roots in $\CC,$ and we show that $p$ is constant. Well, because $|p(z)|\to\infty$ as $|z|\to\infty$ (e.g., by the triangle inequality), so it follows that $|p(z)|$ has a well-defined and achieved minimum on $\CC$ (by compactness). Set the minimum to be $m$ so that
	\[\left|\frac1{p(z)}\right|\le m\]
	for each $z\in\CC.$ Note that the left-hand side is always well-defined because $p$ has no roots in $\CC.$ But this makes $z\mapsto1/p(z)$ a bounded holomorphic function on $\CC,$ so Louiville's theorem implies that $1/p(z)$ is constant, so $p(z)$ is constant.
\end{proof}
\begin{proof}[Proof by Galois theory]
	We pick up the following facts.
	\begin{listalph}
		\item Any polynomial of odd degree has a root somewhere. This is by the Intermediate value theorem because the end behavior of any odd-degree polynomial will be different going to $+\infty$ and $-\infty.$
		
		Note that this where we are using topology in our proof; for example, this step does not work for $\QQ,$ say.
		\item All elements of $\CC$ have a square root in $\CC.$ For example, if we write our complex number as $re^{i\theta},$ then $\sqrt re^{i\theta/2}$ is a square root.

		In fact, this can be extended by the quadratic formula (here we are using that the characteristic of $\RR$ is not $2$) to show that any quadratic has roots.
	\end{listalph}
	Now, to show that $\CC$ is algebraically closed, we pick any element element $\alpha$ a root of a polynomial in $\CC[z],$ and it will generate a splitting field $L/\CC.$ We would like to show that $\alpha\in\CC,$ for which we show $L=\CC.$
	
	Less specifically, we show that any finite Galois extension $L/\CC$ will collapse to $L=\CC.$ We note that we have the tower
	\[\RR\subseteq\CC\subseteq L,\]
	so $L/\RR$ is also Galois. Now, fix $G:=\op{Gal}(L/\RR),$ and we use Galois theory to turn our theorem into a group theory problem. Then we note the following.
	\begin{listalph}
		\item We claim that that $G$ has no proper subgroups of odd index; this follows from (a) earlier.
		
		Indeed, a subgroup $H\subseteq G$ of odd index would induce a field $L^H$ with
		\[[L^H:\RR]=\frac{[L:\RR]}{[L:L^H]}=\frac{\#G}{\#\op{Gal}(L/L^H)}=[G:H].\]
		But there are no nontrivial extensions of $\RR$ of odd degree because all polynomials of odd degree over $\RR$ have a root and are not irreducible. So we must have $[L^H:\RR]=1$ so that $[G:H]=1,$ making $H$ not a proper subgroup.

		In particular, fixing $S$ to be a Sylow $2$-subgroup, we find that $S$ has odd index by construction, so $S=G.$ So $G$ is a $2$-group, so $\op{Gal}(L/\CC)\subseteq G$ is a $2$-group.

		\item It remains to show that $\op{Gal}(L/\CC)$ must be trivial. This follows from (b) earlier.

		Indeed, supposing for contradiction that $\#\op{Gal}(L/\CC)>1,$ we see that, being a $2$-group and hence nilpotent, $\op{Gal}(L/\CC)$ will contain an index-$2$ subgroup. But this corresponds to a nontrivial quadratic extension of $\CC,$ which does not exist by (b) above.
	\end{listalph}
	So from the above reasoning we have that $\op{Gal}(L/\CC)$ must trivial, forcing $L/\CC$ to collapse into $L=\CC.$
\end{proof}
\begin{remark}
	Essentially what happened in the above proof is that we turned a result into fields into some logic about groups. Namely, the Intermediate value theorem turned into no subgroups of odd index larger than $1$; and every element having a square root turned into no quadratic subextensions. The Galois theory bridges these.
\end{remark}
\begin{remark}
	We used a lot of group theory in the above proof: we used the Sylow theorems and some theory of $p$-groups/nilpotent groups.
\end{remark}
Anyways let's see another proof.
\begin{proof}[Proof by winding numbers]
	Let's make the Fundamental theorem of algebra intuitively obvious. Fix $f\in\CC[z]$ some polynomial, and we would like to give it a root. Imagine we have our parameter $z$ run around a large circle of radius $R.$
	\begin{center}
		\begin{asy}
			unitsize(2cm);
			draw((-1.5,0)--(1.5,0));
			draw((0,-1.5)--(0,1.5));
			draw(circle((0,0),1));
			draw((0,0)--dir(30));
			label("$R$", dir(30)/2, dir(30+90));
			dot("$z$", dir(30), dir(30));
		\end{asy}
	\end{center}
	Now, we watch what happens with $f.$ For $R$ large enough, then $f(z)\approx z^{\deg f},$ so $f(z)$ will loop around the origin $\deg f$ times. But if we contract $R$ to $0,$ then we will go around the origin $0$ times. So by the ``continuity'' the contraction as $R\to0$ must send the image of $f$ of this circle to intersect the origin, which is the root we were looking for.
\end{proof}
\begin{remark}
	The hard part of this proof is to verify that the number of times we go around the origin is a well-defined integer, namely $\deg f.$ To rigorize this, take an algebraic topology class.
\end{remark}
\begin{remark}
	This proof actually works for $f(z)=z^n+g(z),$ where $g(z)$ is any continuous function for which $|g(z)|<R^n$ when $|z|=R,$ for some given $R.$ So perhaps this proof has little to do with polynomials.
\end{remark}
\begin{remark}
	One reason this proof is subtle and undiscovered for a while is that thinking topologically is hard.
\end{remark}

\subsection{Separable Extensions}
And we continue with our applications.
\begin{prop}
	Fix $L/K$ a finite separable extension. Then there are only finitely many extensions between $L/K.$
\end{prop}
\begin{proof}
	We extend $L$ to some finite Galois extension $M/K,$ say by taking a splitting field of the polynomials for some finite generating set for $L/K.$
	
	But now the number of extensions between $M/K$ is finite because they correspond to subgroups of $\op{Gal}(M/K),$ which is finite because its size is $[M:K]<\infty.$ Because each intermediate extension between $L/K$ will also be between $M/K,$ we conclude that there are finitely many intermediate extensions between $L/K.$
\end{proof}
Importantly, inseparable extensions cannot be embedded into Galois extensions, so this proof does not work for free. Explicitly, if $L/K$ is inseparable, then any extension $M/K$ with $L$ as an intermediate field will still be inseparable. Here is the standard example.
\begin{example}
	Fix $k$ an infinite field of characteristic $p,$ and we consider the extension
	\[k\left(t^p,u^p\right)\subseteq k(t,u).\]
	This is an extension of degree $p^2,$ which we can check by hand. But if $x$ is any element of $k(t,u),$ then $x^p\in k\left(t^p,u^p\right)$ by the Frobenius automorphism, so $[k(x):k]=p.$ This gives us an infinite number of extensions of degree $p$; in particular no finite number of them can cover all of $L$ because no finite number of proper $K$-subspaces can fully cover $L.$
\end{example}
Technically in the above example, we do need to check that proper subspaces cannot fully cover a space (over an infinite field), which requires the following lemma.
\begin{lemma} \label{lem:unionpropersubspace}
	If $L$ is a vector space over an infinite field $K.$ Then $L$ is not the union of a finite number of proper subspaces.
\end{lemma}
\begin{proof}
	This is surprisingly technical; we take our proof from \href{https://mathoverflow.net/a/14241}{here}. Suppose for the sake of contradiction we can write
	\[L=\bigcup_{k=1}^nL_k\]
	for some proper subspaces $L_k\subsetneq L.$ We show that $V_1$ is contained in $\bigcup_{k=2}^nL_k,$ from which an induction will collapse the entire union to $L=L_n,$ which will be a contradiction.

	Well, fix any $\ell\in L_1$ so that we want to show $\ell\in\bigcup_{k=2}^nL_k.$ We should use the fact that $L_1$ is proper, so we note that we can also find $\ell'\in L\setminus L_1.$ Now, because $K$ is infinite (!), we can look at the family of vectors in the form
	\[\ell+\ell'k\]
	as $k\in K\setminus\{0\}$ varies. None of these vectors can go into $L_1,$ for this would imply $\ell'\in L_1,$ but as they must go into one of the $L_\bullet$s of our union. In particular, because there are infinitely many of these vectors, two of them must fit into some particular $L_k.$ But then
	\[\ell+\ell'k_1,\ell+\ell'k_2\in L_k\]
	implies that $\ell\in L_k.$ This finishes.
\end{proof}
In particular, this tells us that there cannot be finitely many fields $k(x),$ for these fields must generate the full $L.$

We remark that this gives us another proof of the Primitive element theorem, more or less by ``set theory.''
\begin{theorem}[Primitive element]
	Fix $L/K$ a finite extension with only finitely many intermediate subfields; for example, we can take $L/K$ finite and separable. Then there exists $\alpha\in L$ such that $L=K(\alpha).$
\end{theorem}
\begin{proof}
	For finite fields, proceed as we did before: fix a generator $g\in L^\times,$ and we see that $L=K(g).$ (This is technically the only place that we use the fact that $L/K$ is a finite extension.)
	
	For $K$ infinite, we let $\{L_k\}_{k=1}^n$ a list of the proper intermediate extensions between $L/K.$ Then we see that
	\[\bigcup_{k=1}^nL_k\]
	is a finite union of proper $K$-subspaces of $L,$ so because $K$ is infinite, this cannot cover all of $L$ by \autoref{lem:unionpropersubspace}. In particular, fix $\alpha$ in $L$ but not in the above union so that $K(\alpha)$ contains $\alpha$ and hence cannot be a proper intermediate extension. So $L=K(\alpha),$ finishing.
\end{proof}
\begin{remark}[Nir]
	Technically we may remove the condition that $L/K$ is a finite extension, for this follows from only having finitely many intermediate subfields. Fix $L/K$ an infinite extension, and we show that there are infinitely many intermediate subfields.
	\begin{itemize}
		\item If $L/K$ can be generated by a single element $L=K(\alpha),$ then $\alpha$ must be transcendental, so the various $K(\alpha^\bullet)$ provide our intermediate subfields.
		\item If $L/K$ cannot be generated by a single element, then fix $K(\alpha,\beta)$ a subextension which cannot be generated by a single element. If $K$ is finite, then $\alpha$ and $\beta$ cannot be algeraic, for then $K(\alpha,\beta)$ would be finite and generated by a single element; so one of $\alpha$ or $\beta$ is transcendental, reducing the previous case.

		Otherwise $K$ is infinite. Then we can show that $K(\alpha+k\beta)$ for various $k\in K$ will each give distinct subspaces, for $K(\alpha+k_1\beta)=K(\alpha+k_2\beta)$ for $k_1\ne k_2$ would imply that $K(\alpha,\beta)=K(\alpha+k_1\beta)=K(\alpha+k_2\beta)$ is generated by a single element.
	\end{itemize}
\end{remark}

\subsection{Kummer Theory Advertisement}
We will focus on the following question.
\begin{ques}
	Suppose a Galois extension $L/K$ has Galois group $G.$ Then what can we say about the extension?
\end{ques}
Here are some examples.
\begin{exe}
	Fix $G=\ZZ/2\ZZ,$ and we discuss the Galois extensions $L/K$ with $\op{Gal}(L/K)\cong G.$
\end{exe}
\begin{proof}
	Our extension is Galois and in particular separable, so $L/K$ will have $L=K(\alpha),$ and $\alpha$ must now be a degree-$2$ element because $[L:K]=2.$ Namely, we will have
	\[\alpha^2+b\alpha+c=0\]
	for some $b,c\in K.$ Assuming our characterisitc is not $2,$ we can solve for $\alpha$ as in $\frac{-b\pm\sqrt{b^2-4c}}2,$ so $L=K\left(\sqrt\beta\right)$ for some $\beta\in K.$ Then here, setting $g\in G$ to be the nontrivial element, we see that we must have
	\[g\sqrt\beta=-\sqrt\beta\]
	because $g\sqrt\beta\in\{\pm\sqrt\beta\}$ as these are the roots of $x^2-\beta=0,$ but $g$ cannot fix $\sqrt\beta$ because this would make $G$ fix all of $K(\beta).$
	
	In particular, we remark that $G$ is acting as not just field automorphisms, but viewing $L$ as a $K$-vector space, we are getting a lienar representation of $G$ as $G\to\op{Aut}(L)$ by viewing the automorphisms as linear transformation. Under this view, $\sqrt\beta$ is an eigenvector with eigenvalue $-1.$ Of course, we have another eigenvector as $1\in L$ with eigenvalue $1.$ The point is that
	\[G\cong\left\langle\begin{bmatrix}
		1 & 0 \\
		0 & -1
	\end{bmatrix}\right\rangle\]
	by diagonalizing with our eigenbasis $\{1,\sqrt\beta\}.$

	However, if the characteristic of $K$ is $2,$ we need to worry a bit more. Here we cannot even hope to get $L=K(\sqrt\beta)$ because $K(\sqrt\beta)/K$ is not separable because the minimal polynomial $x^2-\beta=(x-\sqrt\beta)^2$ is not a separable polynomial. So we return to our polynomial
	\[x^2+bx+c=0.\]
	Now here we should have $b\ne0$ to make $L/K$ separable, as just described, so we scale $x\mapsto bx$ and divide out by $b^2$ to get
	\[x^2+x+c'=0,\]
	for $c':=c/b^2.$ The point is that we have somewhat controlled our linear term, which gives us an ``Artin-Shreier polynomial.'' Because $1=-1,$ we may rewrite this as
	\[x^2-x-c'=0.\]
	Now, $\alpha$ is a root implies that $\alpha+1$ is a root because $(\alpha+1)^2-(\alpha+1)=\alpha^2-\alpha,$ so do indeed have distinct roots. So $\ZZ/2\ZZ$-extensions in characteristic $2$ are controlled by polynomials in the form $x^2-x-c=0.$
	
	Here we have that our nontrivial element $g\in G$ must send $\alpha\mapsto\alpha+1$ to the other root, so expanding $g\in G$ with the basis $\{1,\alpha\},$ we find that
	\[G\cong\left\langle\begin{bmatrix}
		1 & 1 \\
		0 & 1
	\end{bmatrix}\right\rangle.\]
	This is not diagonalizable, but it does at least have all eigenvalues equal to $1,$ which is called ``unipotent.''
\end{proof}
\begin{example}
	The extension $\FF_4/\FF_2$ will have $\FF_4=\FF_2(\omega),$ where $\omega$ is a root of $x^2+x+1.$ Here the Galois group takes $1\mapsto1$ and $\omega\mapsto\omega^2=\omega+1,$ so it behaves as described.
\end{example}
Let's push harder.
\begin{exe}
	Fix $G=\ZZ/p\ZZ,$ and we discuss the extensions $L/K$ with $\op{Gal}(L/K)\cong G.$
\end{exe}
\begin{proof}
	As before, we have to quarantine out characteristic $p,$ so let's start with the case where $K$ has characteristic not equal to $p.$ To make our lives easier, we will assume that $K$ contains all $p$th roots of $1$; we did not have to do this for $p=2$ because $\pm1$ are always in our field.

	Now, the point is that
	\[K(\sqrt[p]a)\]
	is the splitting field of $x^p-a$ because the roots are $\zeta_p^\bullet\sqrt[p]a\in K(\sqrt[p]a).$ We will continue this discussion next lecture.
\end{proof}