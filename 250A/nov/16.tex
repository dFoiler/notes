\documentclass[../notes.tex]{subfiles}

\begin{document}

% !TEX root = ../notes.tex





















I'm gonna die to the sound of that noise.

\subsection{Galois Loose Ends}
Last lecture we were in the middle of proving the following statement.
\galgrabbag*
\begin{proof}
	To finish off, we have to show that (d) implies (a). For this, pick some element $\alpha\in L$ and look at the $G$-conjugates of $\alpha,$ namely $G\alpha=\{g\alpha:g\in G\}.$ To find a polynomial with $\alpha$ as a root, we take
	\[f_\alpha(x)=\prod_{\beta\in G\alpha}(x-\beta).\]
	This polynomial is separable because we took the product over the set of roots in $G\alpha,$ so there will be no repetition. Further, notice that the coefficients are fixed by $G$ because we can induce a $G$-action on $L[x]$ by fixing $x,$ upon which we see
	\[g\cdot f_\alpha(x)=\prod_{\beta\in G\alpha}(gx-g\beta)=\prod_{\beta\in G\alpha}(x-g\beta)=\prod_{\beta\in G\alpha}(x-\beta)\]
	because the $G$-action induces a bijection on $G\alpha.$ Thus, $f_\alpha(x)\in L^G[x]=K[x].$
	
	Generating the polynomial $f_\alpha$ for each element $\alpha\in L$ gives the set
	\[\{f_\alpha(x)\in K[x]:\alpha\in L\}\]
	of separable polynoials, whose splitting field is $K(\{\alpha\}_{\alpha\in L})=L.$
\end{proof}
\begin{remark}
	In the original statement, it is not at all obvious that any of the above are equivalent, but they are, which is nice.
\end{remark}
Anyways, \autoref{prop:galgrabbag} gives us lots of examples of Galois extensions.
\begin{example} \label{ex:galsep}
	The splitting field of any separable polynomial, as in (a), will form a Galois extension. For example, the splitting field of $x^7-3x^4+2$ over $\QQ$ makes a Galois extension, but actually finding what this splitting field is not easy; e.g., what is the degree? This probably has Galois group $S_7,$ but this is hard to prove.
\end{example}
\begin{example}
	We use (d): for example, set $L=\QQ(x_1,\ldots,x_n)$ to be rational functions in $n$ variables. This has an $G:=S_n$-action of permuting the coordinates, so we find that $L/L^G$ is a Galois extension. Recall that, by the Fundamental theorem of symmetric polynomials, we have
	\[L^G=\QQ(e_1,\ldots,e_n),\]
	where the $e_\bullet$ are the symmetric polynomials. (Explicitly, we see that any element $p/q\in L^G$ can have $p,q\in\QQ[e_1,\ldots,e_n],$ so $p/q\in\QQ(e_1,\ldots,e_n).$)
\end{example}
\begin{example} \label{ex:invgal}
	In general, take $G$ a finite group, and we see that we can embed $G\subseteq S_n$ with $n=\#G$ (say). So now if we take $L=\QQ(x_1,\ldots,x_n),$ we find that $L/L^G$ will have Galois group $G.$ Indeed, $G\subseteq\op{Gal}(L/L^G)$ because each element $\sigma\in G$ does act on $L$ in a way fixing $L.$ And conversely, any element $\tau\in\op{Gal}(L/L^G)$ will have to fix
	\[\alpha:=\sum_{\sigma\in G}x_{\sigma1}\in L^G,\]
	which we can see implies that $\tau\in G.$ To be explicit, $\alpha$ is fixed by $G$ because the $G$-action merely permutes the $x_{\sigma1}.$ And $\tau\alpha=\sum_{\sigma\in\tau G}x_{\sigma1}$ is the sum over a coset, so $\tau\alpha=\alpha$ implies that $\tau\in G.$ The point is that any finite group comes from some finite field extension.
\end{example}
As an aside, actually writing down what $L^G$ in \autoref{ex:invgal} is somewhat difficult (say) in terms of generators. This is more or less the same difficulty we were feeling in \autoref{ex:galsep}: actually describing our extension is hard, though we know it exists.
\begin{remark}
	It is an open problem in Galois theory that, given a finite group $G,$ if there exists an extension $K/\QQ$ with the prescribed Galois group. If we let the base field vary, the answer yes; if we fix the base field to be $\CC(t)$ or $\QQ_p(t),$ the answer is still yes.
\end{remark}

\subsection{Advertisement for the Galois Correspondence}
Here is the main theorem in Galois theory.
\begin{restatable}{theorem}{galcor} \label{thm:galcor}
	Fix $M/K$ a finite Galois extension of fields with Galois group $G:=\op{Gal}(M/K).$ Then we get a one-to-one correspondence between intermediate extensions $K\subseteq L\subseteq M$ and subgroups $H\subseteq G.$ To be explicit, our maps are as follows.
	\[\begin{array}{ccc}
		M^H & \reflectbox{$\longmapsto$} & H \\
		L & \longmapsto & \op{Gal}(M/L)
	\end{array}\]
	Additionally, we have $[M:M^H]=\#H$ and $[M:L]=\#\op{Gal}(M/L).$ In fact, this mapping is inclusion reversing: if we have subgroups $H_1\subseteq H_2\subseteq G,$ then $M^{H_2}\subseteq M^{H_1}$; and $K\subseteq L_1\subseteq L_2\subseteq M$ implies that $\op{Gal}(M/L_2)\subseteq\op{Gal}(M/L_1)\subseteq G.$
\end{restatable}
\begin{remark}
	Intuitively, inclusion-reversing means that small extensions become big subgroups, and big extensions becomes smalls subgroups. This is quite confusing.
\end{remark}
\begin{remark}
	The extension $L$ corresponds to $\op{Gal}(M/L),$ not $\op{Gal}(L/K).$ As an example of a reason this is bad, $L/K$ might not be a Galois extension, so this automorphism group need not be ``good.''
\end{remark}
Let's give some examples before proving \autoref{thm:galcor}.
\begin{exe}
	We work out the Galois correspondence for the splitting field of $x^3-2$ over $\QQ.$
\end{exe}
\begin{proof}
	The roots of $x^3-2$ are $\{\sqrt[3]2,\omega\sqrt[3]2,\omega^2\sqrt[3]2\},$ where $\omega$ is a primitive third root of unity. Thus, our splitting field is $K:=\QQ(\sqrt[3]2,\omega\sqrt[3]2,\omega^2\sqrt[3]2),$ and it is not hard to see that this is $K=\QQ(\sqrt[3]2,\omega).$
	
	Now, we see that $[K:\QQ]=6$ because we have the chain
	\[\QQ\subseteq\QQ(\sqrt[3]2)\subseteq\QQ(\sqrt[3]2,\omega).\]
	Namely, $[\QQ(\sqrt[3]2):\QQ]=3$ because the irreducible polynomial for $\sqrt[3]2$ is $x^3-2,$ and $[\QQ(\sqrt[3]2,\omega):\QQ(\sqrt[3]2)]=2$ because the irreducible polynomial for $\omega$ is $x^2+x+1,$ which is irreducible over $\QQ(\sqrt[3]2)$ because it is quadratic and has no roots in $\QQ(\sqrt[3]2)\subseteq\RR.$

	Thus, $\#\op{Gal}(K/\QQ)=[K:\QQ]=6$ and so must be $S_3,$ acting on the three roots $\{\sqrt[3]2,\omega\sqrt[3]2,\omega^2\sqrt[3]2\}$ of $x^3-2.$ (Amusingly, these fit on an equilateral triangle in the complex plane, though this is not a necessary picture.) Let's write out the lattice diagram of subgroups for $S_3.$
	% https://q.uiver.app/?q=WzAsNixbMiwwLCJcXGxhbmdsZSBlXFxyYW5nbGUiXSxbMSwxLCJcXGxhbmdsZSgxMilcXHJhbmdsZSJdLFsyLDEsIlxcbGFuZ2xlKDIzKVxccmFuZ2xlIl0sWzQsMiwiXFxsYW5nbGUoMTIzKVxccmFuZ2xlIl0sWzAsMSwiXFxsYW5nbGUoMzIpXFxyYW5nbGUiXSxbMiwzLCJTXzMiXSxbMCw0LCIiLDAseyJzdHlsZSI6eyJoZWFkIjp7Im5hbWUiOiJub25lIn19fV0sWzAsMSwiIiwyLHsic3R5bGUiOnsiaGVhZCI6eyJuYW1lIjoibm9uZSJ9fX1dLFswLDIsIiIsMix7InN0eWxlIjp7ImhlYWQiOnsibmFtZSI6Im5vbmUifX19XSxbMCwzLCIiLDIseyJzdHlsZSI6eyJoZWFkIjp7Im5hbWUiOiJub25lIn19fV0sWzQsNSwiIiwwLHsic3R5bGUiOnsiaGVhZCI6eyJuYW1lIjoibm9uZSJ9fX1dLFsxLDUsIiIsMix7InN0eWxlIjp7ImhlYWQiOnsibmFtZSI6Im5vbmUifX19XSxbMiw1LCIiLDIseyJzdHlsZSI6eyJoZWFkIjp7Im5hbWUiOiJub25lIn19fV0sWzMsNSwiIiwyLHsic3R5bGUiOnsiaGVhZCI6eyJuYW1lIjoibm9uZSJ9fX1dXQ==
	\[\begin{tikzcd}
		&& {\langle e\rangle} \\
		{\langle(12)\rangle} & {\langle(23)\rangle} & {\langle(31)\rangle} \\
		&&&& {\langle(123)\rangle} \\
		&& {S_3}
		\arrow[no head, from=1-3, to=2-1]
		\arrow[no head, from=1-3, to=2-2]
		\arrow[no head, from=1-3, to=2-3]
		\arrow[no head, from=1-3, to=3-5]
		\arrow[no head, from=2-1, to=4-3]
		\arrow[no head, from=2-2, to=4-3]
		\arrow[no head, from=2-3, to=4-3]
		\arrow[no head, from=3-5, to=4-3]
	\end{tikzcd}\]
	And we can write down the tower of fields. To be explicit, we number off $\{\sqrt[3]2,\omega\sqrt[3]2,\omega^2\sqrt[3]2\}$ by $\{1,2,3\}.$ 
	
	The point is that (for example) $\langle(23)\rangle$ will only fix $\sqrt[3]2$ because it swaps the other two roots, so $\QQ(\sqrt[3]2)$ is an example of such a field fixed by these automorphisms, and it is not fixed by the other automorphisms, so this fixed field must be $\QQ(\sqrt[3]2).$

	Similarly, $\langle(12)\rangle$ corresponds to $\QQ(\omega^2\sqrt[3]2),$ and $\langle(31)\rangle$ corresponds to $\QQ(\omega\sqrt[3]2).$ Lastly, we need to find the field corresponding to $\langle(123)\rangle.$ Well, this subgroup has index $2,$ so we need to find a quadratic subfield of $\QQ(\sqrt[3]2,\omega),$ which we see must be $\QQ(\omega).$ This gives us the following lattice.
	% https://q.uiver.app/?q=WzAsNixbMiwwLCJcXFFRKFxcc3FydFszXTIsXFxvbWVnYSkiXSxbMSwxLCJcXFFRKFxcc3FydFszXTIpIl0sWzIsMSwiXFxRUShcXG9tZWdhXFxzcXJ0WzNdMikiXSxbNCwyLCJcXFFRKFxcb21lZ2EpIl0sWzAsMSwiXFxRUShcXG9tZWdhXjJcXHNxcnRbM10yKSJdLFsyLDMsIlxcUVEiXSxbMCw0LCIiLDAseyJzdHlsZSI6eyJoZWFkIjp7Im5hbWUiOiJub25lIn19fV0sWzAsMSwiIiwyLHsic3R5bGUiOnsiaGVhZCI6eyJuYW1lIjoibm9uZSJ9fX1dLFswLDIsIiIsMix7InN0eWxlIjp7ImhlYWQiOnsibmFtZSI6Im5vbmUifX19XSxbMCwzLCIiLDIseyJzdHlsZSI6eyJoZWFkIjp7Im5hbWUiOiJub25lIn19fV0sWzQsNSwiIiwwLHsic3R5bGUiOnsiaGVhZCI6eyJuYW1lIjoibm9uZSJ9fX1dLFsxLDUsIiIsMix7InN0eWxlIjp7ImhlYWQiOnsibmFtZSI6Im5vbmUifX19XSxbMiw1LCIiLDIseyJzdHlsZSI6eyJoZWFkIjp7Im5hbWUiOiJub25lIn19fV0sWzMsNSwiIiwyLHsic3R5bGUiOnsiaGVhZCI6eyJuYW1lIjoibm9uZSJ9fX1dXQ==&macro_url=https%3A%2F%2Fgist.githubusercontent.com%2FdFoiler%2F1e12fec404cad7e185260f0c9b68977d%2Fraw%2F909cc7837a29133fb63fb0e9300d15bfe7417fc5%2Fnir.sty
	\[\begin{tikzcd}
		&& {\QQ(\sqrt[3]2,\omega)} \\
		{\QQ(\omega^2\sqrt[3]2)} & {\QQ(\sqrt[3]2)} & {\QQ(\omega\sqrt[3]2)} \\
		&&&& {\QQ(\omega)} \\
		&& \QQ
		\arrow[no head, from=1-3, to=2-1]
		\arrow[no head, from=1-3, to=2-2]
		\arrow[no head, from=1-3, to=2-3]
		\arrow[no head, from=1-3, to=3-5]
		\arrow[no head, from=2-1, to=4-3]
		\arrow[no head, from=2-2, to=4-3]
		\arrow[no head, from=2-3, to=4-3]
		\arrow[no head, from=3-5, to=4-3]
	\end{tikzcd}\]
	We remark that the Galois correspondence now tell us, automatically, that these are all of the intermediate fields.
\end{proof}
\begin{exe}
	We work out the Galois correspondence for $\FF_{64}/\FF_2.$
\end{exe}
\begin{proof}
	We see that $\sigma:x\mapsto x^2$ is our Frobenius automorphism, and this automorphism $\sigma$ has order $6$: the order $k$ of this automorphism is the smallest $k$ such that
	\[x^{2^k}=\sigma^k(x)=x.\]
	Because we are working in $\FF_{64}=\FF_{2^6},$ certainly $k=6$ suffices, and $k\ge6$ because all of $\FF_{64}$ must be the root of $x^{2^k}-x.$

	Additionally, we can see that $[\FF_{64}:\FF_2]=6,$ so in fact the Galois group must be generated by $\sigma,$ giving $\op{Gal}(\FF_{64}/\FF_2)\cong\langle\sigma\rangle\cong\ZZ/6\ZZ.$
	\begin{remark}[Nir]
		For any prime-power $q$ and positive integer $r,$ the above argument can be used to show that $\op{Gal}(\FF_{q^r}/\FF_q)$ is cyclic of order $r$ generated by the Frobenius automorphism $x\mapsto x^q.$
	\end{remark}
	Now, because $\ZZ/6\ZZ$ is cyclic, all of its subgroups are cyclic genreated by $\sigma^d$ for various $d\mid6.$ Then we can see that the fixed field of $\langle\sigma^d\rangle\cong d\ZZ/6\ZZ$ consists of the elements such that
	\[x^{2^d}=\sigma^d(x)=x,\]
	which is exactly $\FF_{2^d}.$ Running this correspondence through gives the following lattices.
	% https://q.uiver.app/?q=WzAsOCxbNSwzLCJcXGxhbmdsZVxcc2lnbWFcXHJhbmdsZSJdLFs2LDIsIlxcbGFuZ2xlXFxzaWdtYV4yXFxyYW5nbGUiXSxbNCwxLCJcXGxhbmdsZVxcc2lnbWFeM1xccmFuZ2xlIl0sWzUsMCwiXFxsYW5nbGUgZVxccmFuZ2xlIl0sWzEsMywiXFxGRl8yIl0sWzIsMiwiXFxGRl97Ml4yfSJdLFswLDEsIlxcRkZfezJeM30iXSxbMSwwLCJcXEZGX3syXjZ9Il0sWzcsNiwiIiwwLHsic3R5bGUiOnsiaGVhZCI6eyJuYW1lIjoibm9uZSJ9fX1dLFs2LDQsIiIsMCx7InN0eWxlIjp7ImhlYWQiOnsibmFtZSI6Im5vbmUifX19XSxbNyw1LCIiLDIseyJzdHlsZSI6eyJoZWFkIjp7Im5hbWUiOiJub25lIn19fV0sWzUsNCwiIiwyLHsic3R5bGUiOnsiaGVhZCI6eyJuYW1lIjoibm9uZSJ9fX1dLFszLDIsIiIsMix7InN0eWxlIjp7ImhlYWQiOnsibmFtZSI6Im5vbmUifX19XSxbMiwwLCIiLDIseyJzdHlsZSI6eyJoZWFkIjp7Im5hbWUiOiJub25lIn19fV0sWzMsMSwiIiwwLHsic3R5bGUiOnsiaGVhZCI6eyJuYW1lIjoibm9uZSJ9fX1dLFsxLDAsIiIsMCx7InN0eWxlIjp7ImhlYWQiOnsibmFtZSI6Im5vbmUifX19XV0=&macro_url=https%3A%2F%2Fgist.githubusercontent.com%2FdFoiler%2F1e12fec404cad7e185260f0c9b68977d%2Fraw%2F909cc7837a29133fb63fb0e9300d15bfe7417fc5%2Fnir.sty
	\[\begin{tikzcd}
		& {\FF_{2^6}} &&&& {\langle e\rangle} \\
		{\FF_{2^3}} &&&& {\langle\sigma^3\rangle} \\
		&& {\FF_{2^2}} &&&& {\langle\sigma^2\rangle} \\
		& {\FF_2} &&&& \langle\sigma\rangle
		\arrow[no head, from=1-2, to=2-1]
		\arrow[no head, from=2-1, to=4-2]
		\arrow[no head, from=1-2, to=3-3]
		\arrow[no head, from=3-3, to=4-2]
		\arrow[no head, from=1-6, to=2-5]
		\arrow[no head, from=2-5, to=4-6]
		\arrow[no head, from=1-6, to=3-7]
		\arrow[no head, from=3-7, to=4-6]
	\end{tikzcd}\]
	This finishes.
\end{proof}
In the above examples, we more or less knew what the subfields and the subgroups were in advance, and it was nice to see the lattice diagrams correspond. In the next example, the subfields are less obvious.
\begin{exe}
	We work out the Galois correspondence for $\QQ(\zeta_7)/\QQ$ where $\zeta_7$ is a primitive seventh root of unity.
\end{exe}
\begin{proof}
	The minimal polynomial for $\zeta_7$ is $\Phi_7(x)=1+x+\cdots+x^6=\frac{x^7-1}{x-1},$ which has degree $6$ and is irreducible because $\Phi_7(x+1)$ satisfies Eisenstein's criterion with the prime $7.$\footnote{More generally, we showed that $\Phi_p(x)$ is irreducible whenwe first introduced Eisenstein's criterion.} Visually, we can see all of the roots of $\Phi_7$ as follows.
	\begin{center}
		\begin{asy}
			unitsize(1.5cm);
			draw((-1.25,0)--(1.25,0));
			draw((0,-1.25)--(0,1.25));
			draw(circle((0,0),1));
			for(int i = 1; i < 7; ++i)
				dot(dir(360*i/7));
		\end{asy}
	\end{center}
	In particular, any root of $\Phi_7(x)=\frac{x^7-1}{x-1}$ must be a seventh root of unity which is not $1,$ so it must be a a primitive seventh root of unity. Conversely, we can see that all the primitive seventh roots of unity are indeed roots because they satisfy $x^7-1=0$ but $x-1\ne0.$
			
	Now, any automorphism in the Galois group must send $\zeta_7$ to some other root of $\Phi_7,$ say $\zeta_7^k$ for $k\in(\ZZ/7\ZZ)^\times.$ We can check that each of these maps does indeed induce a unique automorphism
	\[\QQ(\zeta_7)\cong\frac{\QQ[x]}{(\Phi_7(x))}\cong\QQ(\zeta_7^k)\]
	because the minimal polynomial of $\zeta_7^k$ is still $\Phi_7.$ So each of the constraints $\zeta_7\mapsto\zeta_7^k$ induces a unique automorphism, so in fact we get that
	\[\op{Gal}(\QQ(\zeta_7)/\QQ)\cong(\ZZ/7\ZZ)^\times\]
	by taking the automorphism $\sigma_k:\zeta_7\mapsto\zeta_7^k$ to $k.$ Technically, we shuold check that this is well-defined (it is because $\zeta_7$ exponents only matter$\pmod7$) and that $\sigma_k\circ\sigma_\ell=\sigma_{k\ell}$ to be a homomorphism (it is because ). So we get our isomorphism.
	\begin{remark}[Nir]
		The above argument can be carried out essentially verbatim by replacing $7$ with any prime. It is fact that, for any positive integer $n,$ we have
		\[\op{Gal}(\QQ(\zeta_n)/\QQ)\cong(\ZZ/n\ZZ)^\times,\]
		which we can see using the last half of the argument above, but some amount of care is required to show that $\Phi_n$ is in fact irreducible.
	\end{remark}
	Noting that $3$ is a generator of $(\ZZ/7\ZZ)^\times\cong\ZZ/6\ZZ,$ we can write down our subgroup lattice as follows.
	% https://q.uiver.app/?q=WzAsNCxbMSwwLCIxIl0sWzAsMSwiXFx7MSw2XFx9Il0sWzIsMiwiXFx7MSwyLDRcXH0iXSxbMSwzLCIoXFxaWi82XFxaWileXFx0aW1lcyJdLFswLDEsIiIsMCx7InN0eWxlIjp7ImhlYWQiOnsibmFtZSI6Im5vbmUifX19XSxbMSwzLCIiLDAseyJzdHlsZSI6eyJoZWFkIjp7Im5hbWUiOiJub25lIn19fV0sWzAsMiwiIiwyLHsic3R5bGUiOnsiaGVhZCI6eyJuYW1lIjoibm9uZSJ9fX1dLFsyLDMsIiIsMix7InN0eWxlIjp7ImhlYWQiOnsibmFtZSI6Im5vbmUifX19XV0=
	\[\begin{tikzcd}
		& \{1\} \\
		{\{1,6\}} \\
		&& {\{1,2,4\}} \\
		& {(\ZZ/7\ZZ)^\times}
		\arrow[no head, from=1-2, to=2-1]
		\arrow[no head, from=2-1, to=4-2]
		\arrow[no head, from=1-2, to=3-3]
		\arrow[no head, from=3-3, to=4-2]
	\end{tikzcd}\]
	So now let's try and find our fixed fields. Of course $\{1\}$ corresponds to $\QQ(\zeta_7),$ and $(\ZZ/7\ZZ)^\times$ corresponds to $\QQ.$ We now do the harder ones.
	\begin{itemize}
		\item For $\{1,6\},$ we note that the automorphism $\sigma_6:\zeta_7\mapsto\zeta_7^6=\overline{\zeta_7}$ is simply the conjugation automorphism: for any element $\alpha=\sum_{k=1}^6a_k\zeta_7^k,$ we see that $\sigma\alpha=\overline\alpha$ by direct expansion.

		Anyways, the point is that the fixed field of conjugation is $\RR,$ so restricting our view to $\QQ(\zeta_7),$ we are interested in $\QQ(\zeta_7)\cap\RR.$ It is not too hard to see that this field is $\QQ(\zeta_7+\zeta_7^{-1})$ (e.g., $[\QQ(\zeta_7):\QQ(\zeta_7+\zeta_7^{-1})]=2$ and then use the Galois correspondence), but this observation does not matter very much.

		\item For $\{1,2,4\},$ this subgroup has index $2,$ so we are looking for a field of degree $2$ over $\QQ.$ Namely, we want a quadratic subextension of $\QQ(\zeta_7)$; with all the $7$s floating around, it is reasonable to hope that we get $\QQ(\zeta7)$ or $\QQ(\sqrt{-7}).$
		
		Anyways, to find a generator, we pick up some random element fixed by $\{\sigma_1,\sigma_2,\sigma_4\},$ say
		\[\alpha:=\sigma_1(\zeta_7)+\sigma_2(\zeta_7)+\sigma_4(\zeta_7)=\zeta_7+\zeta_7^2+\zeta_7^4.\]
		We hope that this ``generic'' element will generate our subextension. Well, we can square $\alpha$ to get
		\begin{align*}
			\alpha^2 &= \zeta_7^2+\zeta_7^4+\zeta_7^8+2\left(\zeta_7^3+\zeta_7^6+\zeta_7^5\right) \\
			&= \zeta_7+\zeta_7^2+2\zeta_7^3+\zeta_7^4+2\zeta_7^5+2\zeta_7^6 \\
			\alpha^2+\alpha &= 2\zeta_7+2\zeta_7^2+2\zeta_7^3+2\zeta_7^4+2\zeta_7^5+2\zeta_7^6 \\
			&= 2\cdot-1,
		\end{align*}
		so we find that $\alpha^2+\alpha+2=0.$ Thus,
		\[\alpha=\frac{-1\pm\sqrt{-7}}2,\]
		so our corresponding field here is $\QQ(\sqrt{-7}).$
	\end{itemize}
	In total, we see that we have the following lattices.
	% https://q.uiver.app/?q=WzAsOCxbNSwwLCJcXHsxXFx9Il0sWzQsMSwiXFx7MSw2XFx9Il0sWzYsMiwiXFx7MSwyLDRcXH0iXSxbNSwzLCIoXFxaWi83XFxaWileXFx0aW1lcyJdLFsxLDAsIlxcUVEoXFx6ZXRhXzcpIl0sWzAsMSwiXFxRUShcXHpldGFfNylcXGNhcFxcUlIiXSxbMiwyLCJcXFFRKFxcc3FydHstN30pIl0sWzEsMywiXFxRUSJdLFswLDEsIiIsMCx7InN0eWxlIjp7ImhlYWQiOnsibmFtZSI6Im5vbmUifX19XSxbMSwzLCIiLDAseyJzdHlsZSI6eyJoZWFkIjp7Im5hbWUiOiJub25lIn19fV0sWzAsMiwiIiwyLHsic3R5bGUiOnsiaGVhZCI6eyJuYW1lIjoibm9uZSJ9fX1dLFsyLDMsIiIsMix7InN0eWxlIjp7ImhlYWQiOnsibmFtZSI6Im5vbmUifX19XSxbNCw1LCIiLDAseyJzdHlsZSI6eyJoZWFkIjp7Im5hbWUiOiJub25lIn19fV0sWzUsNywiIiwwLHsic3R5bGUiOnsiaGVhZCI6eyJuYW1lIjoibm9uZSJ9fX1dLFs0LDYsIiIsMix7InN0eWxlIjp7ImhlYWQiOnsibmFtZSI6Im5vbmUifX19XSxbNiw3LCIiLDIseyJzdHlsZSI6eyJoZWFkIjp7Im5hbWUiOiJub25lIn19fV1d&macro_url=https%3A%2F%2Fgist.githubusercontent.com%2FdFoiler%2F1e12fec404cad7e185260f0c9b68977d%2Fraw%2F909cc7837a29133fb63fb0e9300d15bfe7417fc5%2Fnir.sty
	\[\begin{tikzcd}
		& {\QQ(\zeta_7)} &&&& {\{1\}} \\
		{\QQ(\zeta_7)\cap\RR} &&&& {\{1,6\}} \\
		&& {\QQ(\sqrt{-7})} &&&& {\{1,2,4\}} \\
		& \QQ &&&& {(\ZZ/7\ZZ)^\times}
		\arrow[no head, from=1-6, to=2-5]
		\arrow[no head, from=2-5, to=4-6]
		\arrow[no head, from=1-6, to=3-7]
		\arrow[no head, from=3-7, to=4-6]
		\arrow[no head, from=1-2, to=2-1]
		\arrow[no head, from=2-1, to=4-2]
		\arrow[no head, from=1-2, to=3-3]
		\arrow[no head, from=3-3, to=4-2]
	\end{tikzcd}\]
	This finishes.
\end{proof}
\begin{remark}
	In the above, one might object that $\alpha$ is set to a specific number, but we only showed that $\alpha\in\left\{\frac{-1\pm\sqrt{-7}}2\right\}.$ However, this was good enough for our purposes, so we don't need to figure out which one is $\alpha,$ and thankfully so---actually figuring out which one is $\alpha$ requires much more effort.
\end{remark}
\begin{remark}
	If we used $5$ instead of $7$ in the above exampe, our quadratic subextension would have been $\QQ(\sqrt{+5}).$ Whether or not the minus sign is added has to do with quadratic reciprocity.
\end{remark}

\subsection{Proof of the Galois Correspondence}
Anyways, let's prove our theorem.
\galcor*
\begin{proof}
	For concreteness, we label our maps by $f:L\mapsto\op{Gal}(M/L)$ and $g:H\mapsto M^H.$ To show that $f$ and $g$ are inverses and bijective, it suffices to just show that they are inverses.

	We show that $g\circ f=\id.$ Indeed, we start with some subgroup $H\subseteq G$ and take
	\[H\stackrel g\longmapsto M^H\stackrel f\longmapsto\op{Gal}(M/M^H).\]
	Certainly $H\subseteq\op{Gal}(M/M^H)$ because any $\sigma\in H$ will fix $M^H$ by definition of $M^H.$ We would like to get the equality.
	\begin{remark}
		This is trivial, but it is easy to get it wrong.
	\end{remark}
	It suffices to show that these have the same size, so we claim that
	\[\#H\stackrel?=\#\op{Gal}(M/M^H).\]
	But we've done this: $M/M^H$ is a Galois extension by \autoref{prop:galgrabbag} part (d), so $\#H=\#\op{Gal}(M/M^H)$ by \autoref{prop:galgrabbag} part (c).

	We now show that $g\circ f=\id.$ Indeed, we start with some intermediate field $L$ and take
	\[L\stackrel f\longmapsto\op{Gal}(M/L)\stackrel g\longmapsto M^{\op{Gal}(M/L)}.\]
	We hope to show that $L=M^{\op{Gal}(M/L)}.$ Certainly $L\subseteq M^{\op{Gal}(M/L)}$ because each $\sigma\in\op{Gal}(M/L)$ will fix $L$ by definition of $\op{Gal}(M/L).$ We would like to show the equality.
	
	Well, again by size arguments, it suffices to show that both of these fields have the same ``size.'' Explicitly, we claim that
	\[[L:K]\stackrel?=[M^{\op{Gal}(M/L)}:K].\]
	Indeed, dividing both sides from $[M:K],$ it suffices to show that $[M:L]\stackrel?=[M:M^{\op{Gal}(M/L)}].$ But here we see that $M/M^{\op{Gal}(M/L)}$ is Galois by \autoref{prop:galgrabbag}, so $[M:M^{\op{Gal}(M/L)}]=\#\op{Gal}(M/L).$ So it suffices to show that $[M:L]\stackrel?=\#\op{Gal}(M/L).$
	\begin{warn}
		It is not true that the size of $\#\op{Gal}(M/L)$ equals $[L:K]$ directly. However, we do have $[M:L]=\#\op{Gal}(M/L).$
	\end{warn}
	At this point, the claim that $[M:L]\stackrel?=\#\op{Gal}(M/L)$ is more internal to just the extension $M/L,$ so we hope that it is more tractable. Observe that, by \autoref{prop:galgrabbag}, we are essentially showing that $M/L$ is a Galois extension.

	Of course, the only reason we have to believe that $M/L$ is Galois is that $M/K$ is Galois, so we will have to use this fact. Certainly we do get $[M:L]\ge\#\op{Gal}(M/L).$ To get the other inequality, we see that we can count elements $\sigma\in\op{Gal}(M/K)$ by counting embeddings $L\into M$ fixing $K$ and multiplying by the number of ways to extend these maps $L\into M$ up to $M\into M.$ Bounding both of these quantities,\footnote{We are using the fact, given an extension $L/K,$ the number of embeddings $L\into X$ fixing $K$ is bounded above by $[L:K].$ One way to see this is to use a chain argument as we did in the case of $X=\overline K.$} we see that
	\[\#\op{Gal}(M/K)\le[L:K]\cdot[M:L],\]
	but equality must hold because both sides here are $[M:K]$ because $M/K$ is Galois. In particular, there are $[M:L]$ maps extending the embedding $L\into L\subseteq M,$ which is the same as saying there are $[M:L]$ elements in $\op{Gal}(M/L).$
\end{proof}
\begin{remark}
	If $M/K$ is not Galois, we do get something: there is a correspondence between subgroups $G\subseteq\op{Gal}(M/K)$ and subextensions containing $M^G\supseteq K.$ One way to see this is to throw out $K$ and just work with the Galois extension $M/M^G$ instead.
\end{remark}
\begin{remark}[Nir]
	Technically we have not shown the inclusion-reversing in the above argument. We do this quickly.
	\begin{itemize}
		\item If $H_1\subseteq H_2\subseteq G,$ then we know that each $\alpha\in M^{H_2}$ will be fixed by each element of $H_1\subseteq H_2,$ so $M^{H_2}\subseteq M^{H_1}.$
		\item If $K\subseteq L_1\subseteq L_2\subseteq M,$ then each $\sigma\in\op{Gal}(M/L_2)$ will fix each element of $L_1\subseteq L_2,$ so $\op{Gal}(M/L_2)\subseteq\op{Gal}(M/L_1).$
	\end{itemize}
\end{remark}

\subsection{Applications of the Galois Correspondence}
Let's do some more applications.
\begin{exe} \label{exe:17gon}
	We construct the heptadecagon, the regular $17$-gon.
\end{exe}
\begin{remark}
	Gauss did this when he was a teenager. The Greeks had known about $2^n$-gons and $2^n\cdot3$-gons and $2^n\cdot5$-gons and $2^n\cdot3\cdot5$-gons. But the $17$-gon made Gauss somewhat famous.
\end{remark}
\begin{proof}
	Fix $\zeta_{17}$ a primitive seventeenth root of unity in $\CC$; here is the picture, to establish that this will in fact give us a $17$-gon on the unit circle.
	\begin{center}
		\begin{asy}
			unitsize(1.5cm);
			draw((-1.25,0)--(1.25,0));
			draw((0,-1.25)--(0,1.25));
			draw(circle((0,0),1));
			for(int i = 0; i < 17; ++i)
				dot(dir(360*i/17));
		\end{asy}
	\end{center}
	By taking powers, we essentially have to construct one of these because the ruler-and-compass constructions correspond to algebraic constructions with $+,-,\times,\div,\sqrt{\cdot}.$ Well, if we can take square roots, then we essentially need to find a sequence of quadratic extensions
	\[\QQ\subseteq F_1\subseteq F_2\subseteq F_3\subseteq\QQ(\zeta_{17}).\]
	Well, this is not that bad: we know that
	\[\op{Gal}(\QQ(\zeta_{17})/\QQ)\cong(\ZZ/17\ZZ)^\times\cong\ZZ/16\ZZ\]
	as we discussed earlier with $\QQ(\zeta_7).$ So in the Galois correspondence, we can use the sequence of index-$2$ subgroups
	\[\ZZ/16\ZZ\supseteq2\ZZ/16\ZZ\supseteq4\ZZ/16\ZZ\supseteq8\ZZ/16\ZZ\supseteq16\ZZ/16\ZZ.\]
	Thus, at least abstractly, we see that it is possible to construct $\zeta_{17}$ and hence the $17$-gon.
	
	Let's actually find some of these fields explicitly. Well, note that $(\ZZ/17\ZZ)^\times\cong\ZZ/16\ZZ$ is cyclic generated by $3$ (again), but I won't write out the log table here. Now let's find our subgroups and so the subextensions.
	\begin{itemize}
		\item The entire set $(\ZZ/17\ZZ)^\times$ correponds to $\QQ.$
		\item The squares of index $2$ becomes $\{1,9,\ldots\}$ of order $8.$
		\item The fourth-powers are the next index-$2$ subgroup, which are $\{1,13,16,4\}.$
		\item The next index-$2$ subgroup is $\{1,16\}.$
		\item Lastly, we are left with $\{1\}.$
	\end{itemize}
	Building our tower of fields as follows by taking the powrs of $\zeta_{17},$ as we did earlier. For example, set
	\[\alpha=\zeta_{17}^1+\zeta_{17}^9+\cdots\]
	and
	\[\beta=\zeta_{17}^3+\zeta_{17}^{10}+\cdots\]
	to be the the two cosets of the subgroup of order $8,$ namely the squares. Well, we see that $(x-\alpha)(x-\beta)$ will be fixed by the $(\ZZ/17\ZZ)^\times$-action, so it will be in $\QQ[x].$ We can check by hand that $\alpha+\beta=-1$ and $\alpha\beta=-4,$ which we leave as an exercise; then we find that $\alpha$ and $\beta$ are the roots of
	\[x^2+x-4=0,\]
	which gives%\todo{which one is which?}
	\[\alpha,\beta=\frac{-1\pm\sqrt{17}}2.\]
	So we have that $F_1=\QQ(\sqrt{17}).$

	So next let's look at the cosets of the subgroup of order $4$ in the subgroup of order $8.$ Namely, we set
	\[\gamma=\zeta_{17}+\zeta_{17}^3+\zeta_{17}^{16}+\zeta_{17}^4,\]
	and
	\[\delta=\zeta_{17}^4+\zeta_{17}^{15}+\zeta_{17}^8+\zeta_{17}^2.\]
	Again, we can find that $(x-\gamma)(z-\delta)$ is fixed by the right Galois group to get that this will live in $F_1=\QQ(\sqrt{17}).$ We can check that by hand that $\gamma+\delta=\frac{-1+\sqrt{17}}2$ and $\gamma\delta=-1,$ which means that we can write down $\gamma$ and $\delta$ using the quadratic formula. In theory, we could do everything explicitly, but it is somewhat tedious.
\end{proof}
\begin{remark}
	Gauss wanted the $17$-gon on his tombstone. This did not occur.
\end{remark}
\begin{remark}
	In general, any prime of the form $1+2^n$ will work using the above construction. For example, we can do $257$ and $65537.$ Folklore says that somewhat worked out an explicit construction of $65537,$ but this is a somewhat useless exercise.
\end{remark}
Let's do some more examples.
\begin{exe}
	We work out the Galois correspondence for the splitting field of $x^4+1$ over $\QQ.$
\end{exe}
\begin{proof}
	We find that $x^4+1=\frac{x^8-1}{x^4-1},$ so graphically its roots look as follows.
	\begin{center}
		\begin{asy}
			unitsize(1.5cm);
			draw((-1.25,0)--(1.25,0));
			draw((0,-1.25)--(0,1.25));
			draw(circle((0,0),1));
			for(int i = 1; i < 8; i += 2)
				dot(dir(360*i/8));
		\end{asy}
	\end{center}
	So we find that the roots are primitive $8$th roots of unity, so our splitting field is $\QQ(\zeta_8),$ which will have Galois group $\op{Gal}(\QQ(\zeta_8)/\QQ)\cong(\ZZ/8\ZZ)^\times.$ As a warning, this is not cyclic because all of the elements of $(\ZZ/8\ZZ)^\times$ have expoent $2.$ In particular, here is our lattice of subgroups.
	% https://q.uiver.app/?q=WzAsNSxbMSwyLCJcXHsxXFx9Il0sWzEsMCwiXFx7MSwzLDUsN1xcfSJdLFswLDEsIlxcezEsM1xcfSJdLFsxLDEsIlxcezEsNVxcfSJdLFsyLDEsIlxcezEsN1xcfSJdLFsxLDIsIiIsMCx7InN0eWxlIjp7ImhlYWQiOnsibmFtZSI6Im5vbmUifX19XSxbMiwwLCIiLDAseyJzdHlsZSI6eyJoZWFkIjp7Im5hbWUiOiJub25lIn19fV0sWzEsMywiIiwyLHsic3R5bGUiOnsiaGVhZCI6eyJuYW1lIjoibm9uZSJ9fX1dLFszLDAsIiIsMix7InN0eWxlIjp7ImhlYWQiOnsibmFtZSI6Im5vbmUifX19XSxbMSw0LCIiLDIseyJzdHlsZSI6eyJoZWFkIjp7Im5hbWUiOiJub25lIn19fV0sWzQsMCwiIiwyLHsic3R5bGUiOnsiaGVhZCI6eyJuYW1lIjoibm9uZSJ9fX1dXQ==
	\[\begin{tikzcd}
		& {\{1,3,5,7\}} \\
		{\{1,3\}} & {\{1,5\}} & {\{1,7\}} \\
		& {\{1\}}
		\arrow[no head, from=1-2, to=2-1]
		\arrow[no head, from=2-1, to=3-2]
		\arrow[no head, from=1-2, to=2-2]
		\arrow[no head, from=2-2, to=3-2]
		\arrow[no head, from=1-2, to=2-3]
		\arrow[no head, from=2-3, to=3-2]
	\end{tikzcd}\]
	We could work out the corresponding lattice of subfields, but it is difficult to do live, so we leave it as an exercise. %\todo{}
\end{proof}
\begin{exe}
	We work out the Galois correspondnece for the splitting field of $x^4-2$ over $\QQ.$
\end{exe}
\begin{proof}
	We note that $\sqrt[4]2$ is certainly a root, but $\QQ(\sqrt[4]2)$ is not the splitting field because we are missing the roots $i\sqrt[4]2$ and $i^3\sqrt[3]2.$ So to get the full splitting field, we want $\QQ(\sqrt[3]2,i).$ Here is our picture.
	\begin{center}
		\begin{asy}
			unitsize(2cm);
			draw((-1.5,0)--(1.5,0));
			draw((0,-1.5)--(0,1.5));
			draw(circle((0,0),1), gray);
			draw(circle((0,0),1.18920712));
			for(int i = 0; i < 4; ++i)
				dot(1.18920712*dir(360*i/4));
			label("$\sqrt[4]2$", (1.18920712, 0), NE);
		\end{asy}
	\end{center}
	It turns out that the Galois group must preserve the above square, which we can check algebraically, so our Galois group is $D_8.$ If we take $D_8$ generated by $90^\circ$ rotation $\sigma$ and a reflection $\tau$ we get the following lattice, which I won't write out because it is complicated.

	However, we can see that we get three quadratic extensions of $\QQ(\sqrt[4]2,i),$ which correspond to $\QQ(\sqrt2)$ and $\QQ(i)$ and $\QQ(\sqrt{-2}).$ then our extensions of degree $4$ are $\QQ(i,\sqrt2)$ and $\QQ(\sqrt[4]2)$ and $\QQ(i\sqrt[4]2)$ and $\QQ((1+i)\sqrt[4]2)$ and $\QQ((1-i)\sqrt[4]2).$
\end{proof}
\begin{remark}
	Again, some of these subfields are not obvious: namely, $\QQ((1+i)\sqrt[4]2)$ is somewhat subtle. But we can find it from the Galois correspondence.
\end{remark}

\subsection{Intermediate Normal Extensions}
As an aside, we can see from the above lattices that it appears normal subgroups correspond to normal extensions.
\begin{example}
	In the above examples, we see that any of our index-$2$ subgroups correspond to quadratic subextensions, and both of these objects are normal for the corresponding definitions of normal.
\end{example}
To be explicit, we have the following statement.
\begin{proposition}
	Fix $M/K$ a Galois extension with Galois group $G:=\op{Gal}(M/K).$ Then we have the following.
	\begin{itemize}
		\item If $K\subseteq L\subseteq M$ is an intermediate extension such that $L/K$ is normal, $\op{Gal}(M/L)\subseteq G$ is a normal subgroup, and$L/K$ is a Galois extension such that
		\[\op{Gal}(L/K)\cong\frac{\op{Gal}(M/K)}{\op{Gal}(M/L)}.\]
		\item Take $M/K$ finite. If $H\subseteq G$ is a normal subgroup, then the corresponding fixed field $M^H$ has $M^H/K$ a normal extension. In fact, $M^H/K$ is a Galois extension with Galois group $G/H.$
	\end{itemize}
\end{proposition}
\begin{proof}
	We show the claims one at a time.
	\begin{itemize}
		\item We are given that $L/K$ is normal, and we see that $L/K$ is separable because $L\subseteq M,$ so all elements of $L$ are separable over $K$ because all elements of $M$ are separable over $K.$

		So $L/K$ is a Galois extension. To compute its Galois group, we construct $\varphi:\op{Gal}(M/K)\to\op{Gal}(L/K)$ by restriction, taking
		\[\varphi:\sigma\mapsto\sigma|_L.\]
		Indeed, it is not hard to see that $\sigma|_L$ is in fact an automorphism: it is at least an embedding $L\into M,$ and because $L$ is normal, this embedding $L\into M\into\overline K$ must be an automorphism. Note that here is the only place in this argument where we use the fact that $L$ is normal: it makes $\varphi$ well-defined.

		Now, $\varphi$ is surjective because any $L\to L$ fixing $K$ becomes an embedding $L\to L\into M$ fixing $K,$ which can then be lifted up to an automoprhism $M\into M$ fixing $K$ by using a chain argument. And lastly, we see that the kernel of $\varphi$ is
		\[\{\sigma\in\op{Gal}(M/K):\sigma|_L=\id_L\},\]
		which is simply the automorphisms of $M$ fixing $K$ which also fix $L.$ But $K\subseteq L,$ so $\ker\varphi=\op{Gal}(M/L).$
		
		Thus, $\op{Gal}(M/L)$ is indeed a normal subgroup of $G$ because it is the kernel of $\varphi,$ and we find that
		\[\op{Gal}(L/K)\cong\frac{\im\varphi}{\ker\varphi}=\frac{\op{Gal}(M/K)}{\op{Gal}(M/L)}.\]
		This is what we wanted.

		\item Now take $H\subseteq G$ a normal subgroup, and we want to show that $M^H/K$ is a Galois extension. Surely this extension is separable because element of $M\supseteq M^H$ is separable over $K.$

		So the hard part is showing that $M^H/K$ is normal. Well, assign an embedding $M^H\subseteq\overline K,$ and suppose that we have some other embedding $\sigma:M^H\into\overline K$ so that it suffices to show $\sigma(M^H)\subseteq M^H,$ which will imply that $\sigma$ is an automorphism.

		Well, to show $\sigma(M^H)\subseteq M^H,$ we need to show that $\sigma(M^H)$ is fixed by $H.$ So pick up some $m\in M^H,$ and then we note, for any $h\in H,$ we have $\sigma^{-1}h\sigma\in H$ (here we use the fact that $H$ is normal), so $\left(\sigma^{-1}h\sigma\right)(m)=m,$ so
		\[h(\sigma m)=\sigma m.\]
		Thus, each $h\in H$ fixes each $\sigma m\in\sigma\sigma(M^H).$ So indeed, $\sigma(M^H)\subseteq M^H,$ so each embedding $M^H\into\overline K$ is an automorphism, so $M^H/K$ is normal.

		To finish, we see that
		\[\op{Gal}\left(M^H/K\right)\cong\frac{\op{Gal}\left(M/K\right)}{\op{Gal}\left(M/M^H\right)}=\frac GH,\]
		where we have used the previous part for $\cong$ and the Galois correspondence for $=.$ This finishes.
		\qedhere
	\end{itemize}
\end{proof}
\begin{remark}
	This is where the term ``normal subgroup'' came from: first normal was used for field extensions, and then second it was pushed into group theory from this correspondence.
\end{remark}

\end{document}