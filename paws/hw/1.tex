% !TEX root = ../notes.tex

\documentclass[../notes.tex]{subfiles}

\begin{document}

\section{Homework 1}

\begin{prob}
	Go to \url{https://www.lmfdb.org/Variety/Abelian/Fq/} and familiarize yourself with the database. Most of the words are probably unfamiliar right now, but by the end of PAWS you should have a pretty good idea of what most of them mean. Here are some questions to make this interesting:
    \begin{listalph}
        \item How many isogeny classes of elliptic curves defined over finite fields does the LMFDB currently contain?
        \item What percentage of these classes of curves are \href{https://en.wikipedia.org/wiki/Supersingular_elliptic_curve}{supersingular}?\footnote{See \cite[Chapter 5]{Silverman2009} for the definition of ordinary/supersingular elliptic curves.}
        \item \label{item:1point} How many isogeny classes of elliptic curves defined over finite fields in the LMFDB have exactly $1$ rational point?
        \item Write down all elliptic curves defined over $\FF_2$.
        \begin{listroman}
            \item How many of these are supersingular?
            \item How many rational points do they have?
            \item One of these curves should have exactly one rational point. What is the characteristic polynomial of its Frobenius endomorphism? 
            \item Compare it to the $L$-polynomial of the isogeny class found in \autoref{item:1point}.
        \end{listroman}
    \end{listalph}
\end{prob}

\begin{prob}
	Let $E$ be the elliptic curve over $\mathbb{Q}$ defined by the Weierstrass equation $y^2z = x^3 + 17z^3$.
    Note that the following points are on $E(\mathbb{Q})$: 
    \[P = [-2:3:1], \quad Q = [4:9:1].\]
    \begin{listalph}
        \item \label{item:2a} Find at least five points on $E(\mathbb{Q})$ that are integer linear combinations\footnote{In fact, we can obtain every point in $E(\QQ)$ in this way!} of $P$ and $Q$. 
        \item  If you did \autoref{item:2a} by hand, check your calculations using your favorite computer algebra system.\footnote{For the relevant commands in SAGE, see \href{https://doc.sagemath.org/html/en/reference/arithmetic_curves/sage/schemes/elliptic_curves/ell_point.html}{this link}.}
        \item Look up this curve in the LMFDB.
    \end{listalph}
\end{prob}

\begin{prob}
	Let $E$ be the elliptic curve over $\mathbb{Q}$ defined by
    \begin{equation*}
        \label{eq:y^2=x^3-x}
        y^2z = x^3 - xz^2.
    \end{equation*}
    \begin{listalph}
        \item Using the group law defined in the lecture notes, compute the set of 2-torsion points $E[2](\overline{\QQ})$.
        \item Compute the 3-torsion points $E[3](\overline{\QQ})$.\footnote{Hint: $3P = 0$ implies that $2P = -P$.}
        \item Verify that this is a minimal Weierstrass equation over $\QQ$, in the sense of \cite[Chapter VII]{Silverman2009}.
        Show that in characteristic 2, the same equation above defines a singular curve. In particular, conclude that $E$ has bad reduction at $2$. 
        
        \item Verify that the same equation defines an elliptic curve $\bar{E}$ over $\FF_3$. Compute the set of 3-torsion points $\bar{E}[3](\overline{\FF}_3)$, and determine whether $\bar{E}$ is ordinary or supersingular.
        \item ($\star \star \star$) Show that $(x,y)\mapsto (-x, iy)$ defines an endomorphism of $\bar{E}_{\FF_{3^2}}$. Here $i$ is a root of $x^2+1 \in \FF_3[x]$. Can you use this to determine the endomorphism ring of $\bar{E}_{\FF_{3^2}}$?\footnote{Hint: consider the $p$-Frobenius, c.f. below.}
    \end{listalph}
\end{prob}

\begin{prob}
	Let $q=p^r$ be a power of $p$, and assume $p > 3$. Let $E/\FF_q$ be an elliptic curve with Weierstrass equation $E: y^2z=x^3+Axz^2+Bz^3$, with $A,B \in \FF_q$. Define the \textit{$p$-Frobenius twist} $E^{(p)}$ of $E$ to be the curve defined by the Weierstrass equation $E^{(p)}: y^2z=x^3+A^px^2z+B^pxz^2$. We define the \textit{$p$-Frobenius morphism} $\phi_p: E \to E^{(p)}$ to be the morphism given by $\phi_p: [x_0: y_0: z_0] \mapsto [x_0^p: y_0^p: z_0^p]$ on $\overline{\FF_q}$-points.
    \begin{listalph}
        \item Show that $\Delta(E^{(p)})=\Delta(E)^p$ and $j(E^{(p)})=j(E)^p$. Conclude that $E^{(p)}$ is an elliptic curve.\footnote{This is to show that $E^{(p)}$ is a nonsingular plane cubic with a rational point $O$. You can use the fact that a plane cubic is nonsingular if and only if its discriminant is non-zero. For formulas of $\Delta(E)$ and $j(E)$, see \cite[Section III.1]{Silverman2009}.}
        \item Verify that $\phi_p$ is an isogeny. That is, verify that it is a morphism of abelian varieties which is surjective on $\bar{\FF}_q$-points and has finite kernel.
    \end{listalph}
    Now, define the \textit{$q$-Frobenius endomorphism} by $\phi_q \coloneqq \phi_p^{r}$. Note that $\phi_q([x_0: y_0: z_0])= [x_0^q: y_0^q: z_0^q]$.
    \begin{listalph}
        \item Show that $\phi_q$ is an endomorphism of $E$ that commutes with any other endomorphism of $E$. 
        \item Show that the $\FF_q$-rational points of $E$ are exactly the $\overline{\FF_q}$-points of $E$ fixed by $\phi_q$. More generally, we have $E(\FF_qn)$ is the set of fixed points of $\phi_{q^n}\colon E(\FF_qb) \to E(\FF_qb)$.
    \end{listalph}
\end{prob}

\begin{prob}
	Let $n$ be a square-free positive integer and let $E$ be the elliptic curve $y^2 = x^3-n^2x$. Let $q$ be a power of a prime $p$, such that $p$ does not divide $2n$, and $q \equiv 3 \pmod 4$. Show that 
    \[\#E(\FF_q) = q+1.\]
\end{prob}

\begin{prob}
	Let $X_1$ and $X_2$ be varieties over a field $k$.
    \begin{listalph}
        \item If $X_1$ and $X_2$ are given the structure of a group variety, show that their product $X_1 \times X_2$ naturally inherits the structure of a group variety.
        \item Suppose $Y := X_1 \times X_2$ carries the structure of an abelian variety. Show that $X_1$ and $X_2$ each have a unique structure of an abelian variety such that $Y = X_1 \times X_2$ as abelian varieties.
    \end{listalph}
\end{prob}
\begin{proof}
	Here we go. Let $(X_1,e_1,i_1,\mu_1)$ and $(X_2,e_2,\mu_2)$ be our needed abelian varieties throughout.
	\begin{listalph}
		\item This is direct. Note that the object $X_1\times X_2$ continues to be a geometrically integral projective variety because all these adjectives are preserved by base change and composition, allowing us to run around the following square.
		% https://q.uiver.app/#q=WzAsNCxbMCwwLCJYXzFcXHRpbWVzIFhfMiJdLFsxLDAsIlhfMSJdLFswLDEsIlhfMiJdLFsxLDEsIlxcU3BlYyBrIl0sWzAsMl0sWzIsM10sWzAsMV0sWzEsM11d&macro_url=https%3A%2F%2Fraw.githubusercontent.com%2FdFoiler%2Fnotes%2Fmaster%2Fnir.tex
		\[\begin{tikzcd}
			{X_1\times X_2} & {X_1} \\
			{X_2} & {\Spec k}
			\arrow[from=1-1, to=2-1]
			\arrow[from=2-1, to=2-2]
			\arrow[from=1-1, to=1-2]
			\arrow[from=1-2, to=2-2]
		\end{tikzcd}\]
		Now, we define $e\colon\Spec k\to X_1\times X_2$ as $e\coloneqq(e_1,e_2)$. We define $\mu\colon(X_1\times X_2)^2\to(X_1\times X_2)$ via the composite
		\[(X_1\times X_2)^2\cong X_1^2\times X_2^2\xrightarrow{(\mu_1,\mu_2)}X_1\times X_2.\]
		Lastly, we define $i\coloneqq(i_1,i_2)$ to be the inversion map $X_1\times X_2\to X_2\times X_2$.

		We would like to check that all of our diagrams commute. This is essentially immediate. For example, to check that
		% https://q.uiver.app/#q=WzAsNCxbMCwwLCIoWF8xXFx0aW1lcyBYXzIpXjMiXSxbMSwwLCIoWF8xXFx0aW1lcyBYXzIpXjIiXSxbMCwxLCIoWF8xXFx0aW1lcyBYXzIpXjIiXSxbMSwxLCJYXzFcXHRpbWVzIFhfMiJdLFswLDEsIihcXG11LDEpIl0sWzAsMiwiKDEsXFxtdSkiLDJdLFsyLDMsIlxcbXUiXSxbMSwzLCJcXG11Il1d&macro_url=https%3A%2F%2Fraw.githubusercontent.com%2FdFoiler%2Fnotes%2Fmaster%2Fnir.tex
		\[\begin{tikzcd}
			{(X_1\times X_2)^3} & {(X_1\times X_2)^2} \\
			{(X_1\times X_2)^2} & {X_1\times X_2}
			\arrow["{(\mu,1)}", from=1-1, to=1-2]
			\arrow["{(1,\mu)}"', from=1-1, to=2-1]
			\arrow["\mu", from=2-1, to=2-2]
			\arrow["\mu", from=1-2, to=2-2]
		\end{tikzcd}\]
		commutes, it is enough to check that it commutes on $S$-points for any scheme $S$ (in particular, one can take $S=(X_1\times X_2)^3$ and then plug in the point $\id_S$), but then this will directly follow from the construction and some group theory. I won't write this out.

		\item Note that there is an inclusion map $\iota_\bullet\colon X_\bullet\to Y$ by $\iota_\bullet(x)\coloneqq(x,e_2)$; rigorously, this is the map $({\id_{X_\bullet}},e_2)$; similarly, there are projection maps $\pi_\bullet\colon Y\to X_\bullet$ given by $\pi_\bullet(x_1,x_2)\coloneqq x_\bullet$.
		
		Now, for uniqueness, note that having $Y=X_1\times X_2$ implies that $\mu_Y=(\mu_1,\mu_2)$ and $i_Y=(i_1,i_2)$ and $e_Y=(e_1,e_2)$. Thus, we can recover $e_\circ$ as $\pi_\bullet\circ e_Y$, we can recover $i_\bullet$ as $\pi_\bullet\circ i_Y\circ\iota_\bullet$, and we can recover $\mu_\bullet$ as $\pi_\bullet\circ\mu_Y\circ(\iota_\bullet,\iota_\bullet)$.

		Now, for existence
		\[\]
	\end{listalph}
\end{proof}

\begin{prob}
	Let $A_1, A_2, B_1, B_2$ be abelian varieties over a field $k$. Show that 
    \begin{equation*}
        \label{eq:hom-prods}
        \op{Hom}(A_1\times A_2, B_1\times B_2) \cong \op{Hom}(A_1, B_1) \times \op{Hom}(A_1, B_2) \times \op{Hom}(A_2, B_1) \times \op{Hom}(A_2, B_2).
    \end{equation*}
\end{prob}
\begin{proof}
	We will actually show that $A\times B$ is a biproduct in the category of abelian varieties, using the maps $i_A$, $i_B$, $\pi_A$, and $\pi_B$ all defined in the previous problem. The result will then follow from general category theory.
\end{proof}

\begin{prob}
	A \textit{ring variety} over a field $k$ is a commutative group variety $(X, +, 0)$ over $k$, together with a ring multiplication morphism $X \times X \to X$ written as $(x,y) \mapsto x \cdot y$, and a $k$-rational point $1\in X(k)$, such that the ring multiplication is associative, distributive with respect to addition, and $1$ is a $2$-sided identity element. Show that the only connected complete ring variety is a point.
\end{prob}

\begin{prob}
	Let $G$ be a group variety over a field $k$.
    \begin{listalph}
        \item Show that there exists a unique irreducible component $N$ containing the identity element $e$. 
        \item Show that $N$ is a normal subgroup of finite index in $G$.
        \item Show that irreducible components of $G$ are exactly connected components of $G$. Conclude that if $G$ is connected, then $G$ is irreducible.
        \item Show that each open subgroup of $G$ contains $N$.
        \item Show that each closed subgroup of finite index in $G$ contains $N$.
        \item Conclude that if $G$ is connected, then $G$ is the only open subgroup and is the only closed subgroup of finite index.
    \end{listalph}
\end{prob}

\begin{prob}
	Let $X$ be a variety over a field $k$. Write $k[\epsilon] \coloneqq k[t]/(t^2)$ for the ring of dual numbers over $k$, and let $S := \Spec(k[\epsilon])$. Write $\text{Aut}^{1} (X_S/S)$ for the group of automorphisms of $X_S$ over $S$ which reduce to the identity on the special fiber $X \hookrightarrow X_S$.
    \begin{listalph}
        \item Let $x$ be a $k$-valued point of $X$. Show that the tangent space $(T_X)_x := (\mathfrak{m}_x/\mathfrak{m}_x^2)^\vee
        $ is in natural bijection with the space of $k[\epsilon]$-valued points of $X$ which reduce to $x$ modulo $\epsilon$. (cf. \cite[Chapter II, Exercise 2.8]{Hartshorne1977}.)
        \item Suppose $X=\Spec(A)$ is affine. Then we have:
        \begin{equation*}
            H^0(X, \mathcal{T}_{X/k}) \cong \op{Hom}(\Omega_{A/k}^1, A) \cong \op{Der}_k(A,A)
        \end{equation*}
        Show that $H^0(X, \mathcal{T}_{X/k}) \cong \text{Aut}^{1} (X_S/S)$. We denote this isomorphism as $h: H^0(X, \mathcal{T}_{X/k}) \to \text{Aut}^{1} (X_S/S)$. Then for a group variety $X$ that is not affine, we can take an affine cover of $X$ and get the isomorphism $h:H^0(X, \mathcal{T}_{X/k}) \to \text{Aut}^{1} (X_S/S)$.
        \item Suppose $X$ is a group variety over $k$. Let $\tau: S \to X$ be a tangent vector at $e$, the identity section. Let $t_\tau$ be the right translation by $\tau$ morphism, so it is an element in $\text{Aut}^{1}(X_S/S)$. Show that the associated global vector field $\zeta := h^{-1}(t_\tau)$ is invariant under the right-translation map. That is, $t_y^{*}\zeta=\zeta$ for all $y \in X(k)$. Here, $t_y(x)=m(x,y)$ is the right translation by $y$ morphism. \footnote{You can check \cite{EVdGM2012}[Proposition 15, pg. 8] for a more explicit description of the associated vector field $\zeta$. It turns out that the vector field is not preserved under the left translation. Can you see why?}
    \end{listalph}
\end{prob}

\begin{prob}
	Show that every morphism from the projective line to an abelian variety is constant.\footnote{Hint: The canonical bundle of an abelian variety is trivial.}
\end{prob}

\begin{prob}
	Show that $1$-dimensional abelian varieties have genus one. In particular, we can define an elliptic curve to be a $1$-dimensional abelian variety.
\end{prob}

\end{document}