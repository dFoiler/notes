% !TEX root = ../notes.tex

\documentclass[../notes.tex]{subfiles}

\begin{document}

\section{February 11}

Today we continue.

\subsection{Torsion Sheaves}
Recall that the affine scheme $\mu_{n,S}=\Spec\ZZ[T]/\left(T^n-1\right)\times S$ is a group scheme over $S$. This allows to build an \'etale sheaf, sending finite \'etale covers $X\to S$ to the group
\[\underline{\mu_n}_{S}(X)=\op{Hom}_S(\mu_{n,S},X).\]
Now, we can use this sheaf to say something about the equivalence of categories of finite \'etale covers of $S$ and finite sets with action by $\pi_1^{\mathrm{\acute et}}(S,\ov s)$ (for $S$ connected).
\begin{corollary}
	Fix a connected scheme $S$ and geometric point $\ov s\in S$. Then there is an equivalence of categories between locally constant constructible sheaves on $S$ with finite continuous $\pi_1^{\mathrm{\acute et}}(S,\ov s)$-sets.
\end{corollary}
\begin{proof}
	This equivalence of categories takes such a sheaf $\mc F$ to the stalk $\mc F_{\ov s}$. The fact that \'etale covers are producing the required sheaves is formal.
\end{proof}
\begin{remark}
	This equivalence is known as the finite monodromy correspondence.
\end{remark}
\begin{remark}
	If one works with the abelian category of locally constant constructible sheaves valued in an abelian category (frequently valued in finite abelian groups), then our stalks and $\pi_1^{\mathrm{\acute et}}(S,\ov s)$ inherit this extra structure.
\end{remark}
Let's be explicit about the sheaves we are interested in.
\begin{definition}
	An \'etale sheaf $\mc F$ of abelian groups on a scheme $S$ is \textit{torsion} if and only if either of the following conditions hold.
	\begin{listroman}
		\item All stalks of $\mc F$ are torsion abelian groups.
		\item One has that $\mc F(U)$ is a torsion abelian group for all \'etale open subsets $U\to S$.
	\end{listroman}
\end{definition}
\begin{remark}
	The embedding of $\mc F(U)$ into a stalk explains why (i) implies (ii). The fact that any element in a stalk comes from some \'etale open neighborhood explains why (i) implies (ii).
\end{remark}
Let's make a few more remarks about these sheaves.
\begin{lemma}
	Fix a scheme $S$.
	\begin{listalph}
		\item A torsion \'etale sheaf $\mc F$ on a scheme $S$ is Noetherian if and only if it is locally constant constructible.
		\item Every torsion sheaf is the (filtered) direct limit of its constructible subsheaves.
	\end{listalph}	
\end{lemma}
\begin{proof}[Sketch]
	For (a), the idea is to consider descending chains of the required filtrations. For (b), the idea is to show that any element in any stalk of the torsion sheaf can be found in some constructible subsheaf.
\end{proof}
\begin{remark}
	It follows that locally constant constructible sheaves form an abelian category!
\end{remark}
Eventually, we will have control over torsion sheaves. However, one would still like coefficients in characteristic $0$, which we do by taking a limit.
\begin{definition}
	Fix a scheme $S$ and a prime $\ell$. An \textit{$\ell$-adic sheaf} $\mc F$ on $S$ is a projective system $\{\mc F_n\}_{n\ge0}$ of constructible sheaves of abelian groups such that $\mc F_0=0$ and $\ell^{1}\mc F_n=0$ for $n\ge0$, and there are isomorphisms
	\[\mc F_{n+1}/\ell^{n+1}\mc F_{n+1}\to\mc F_n\]
	for each $n\ge0$. We say that $\mc F$ is \textit{smooth} if and only if each $\mc F_n$ is locally constant constructible. There is a similar definition to have coefficients in a finite extension $E$ of $\QQ_\ell$; then an \'etale $\ov\QQ_\ell$-sheaf is one with coefficients in some finite extension $E$ of $\QQ_\ell$.
\end{definition}
\begin{remark}
	As before, there is an equivalence of categories (given by taking stalks) between \'etale $\ov\QQ_\ell$-sheaves and continuous representations of $\pi_1(S,\ov s)$ on $\ov\QQ_\ell$-vector spaces. The sheaf is smooth if and only if the representation is finite-dimensional.
\end{remark}

\subsection{Back to Frobenius Morphisms}
Let's begin to do something with these sheaves. As usual, we take $k=\FF_q$. Here are some Frobenius elements.
\begin{itemize}
	\item There is an arithmetic Frobenius automorphism $\sigma\in\op{Gal}(\ov k/k)$ given by $\sigma\colon x\mapsto x^q$, and we recall that it is a topological generator of $\op{Gal}(\ov k/k)\cong\widehat\ZZ$. It generates the cyclic ``Weil'' group $W_{\ov k/k}\cong\ZZ$.
	\item The inverse $F\coloneqq\sigma^{-1}$ is the geometric Frobenius element.
	\item Then for any $\FF_q$-scheme $Y$, there is an absolute Frobenius $\sigma_{Y/k}\colon Y\to Y$. It is the identity on the topological sheaves, and it acts by $q$th powers on the level of structure sheaves. Technically, this is not a morphism of schemes over $\ov k$, so we may have occasion to write the target as $Y^{(q)}$.
\end{itemize}
Now, let $X$ be a scheme over $k$ of finite type, and let $\mc G_0$ be an \'etale sheaf on $X$; then we let $\mc G$ denote the pullback sheaf on $X_{\ov k}$. By functoriality of the absolute Frobenius, one gets a diagram as follows.
% https://q.uiver.app/#q=WzAsOCxbMSwxLCJYX3tcXG92IGt9Il0sWzEsMiwiXFxTcGVjXFxvdiBrIl0sWzIsMiwiXFxTcGVjIGsiXSxbMiwxLCJYIl0sWzAsMiwiXFxTcGVjXFxvdiBrIl0sWzIsMCwiWCJdLFsxLDAsIlhfe1xcb3Yga30iXSxbMCwwLCJYX3tcXG92IGt9Il0sWzMsMl0sWzEsMl0sWzAsMV0sWzAsM10sWzQsMSwiXFxzaWdtYSJdLFs1LDMsIiIsMix7ImxldmVsIjoyLCJzdHlsZSI6eyJoZWFkIjp7Im5hbWUiOiJub25lIn19fV0sWzYsNV0sWzYsMCwie1xcaWRfe1hfe1xcb3Yga319fVxcdGltZXNcXHNpZ21hIl0sWzcsNF0sWzcsMCwiXFxzaWdtYV97WCxcXG92IGt9IiwxXSxbNyw2LCIiLDEseyJzdHlsZSI6eyJib2R5Ijp7Im5hbWUiOiJkYXNoZWQifX19XV0=&macro_url=https%3A%2F%2Fraw.githubusercontent.com%2FdFoiler%2Fnotes%2Fmaster%2Fnir.tex
\[\begin{tikzcd}
	{X_{\ov k}} & {X_{\ov k}} & X \\
	& {X_{\ov k}} & X \\
	{\Spec\ov k} & {\Spec\ov k} & {\Spec k}
	\arrow[dashed, from=1-1, to=1-2]
	\arrow["{\sigma_{X,\ov k}}"{description}, from=1-1, to=2-2]
	\arrow[from=1-1, to=3-1]
	\arrow[from=1-2, to=1-3]
	\arrow["{{\id_{X_{\ov k}}}\times\sigma}", from=1-2, to=2-2]
	\arrow[equals, from=1-3, to=2-3]
	\arrow[from=2-2, to=2-3]
	\arrow[from=2-2, to=3-2]
	\arrow[from=2-3, to=3-3]
	\arrow["\sigma", from=3-1, to=3-2]
	\arrow[from=3-2, to=3-3]
\end{tikzcd}\]
We claim that the dashed arrow exists and is unique. Indeed, it is enough to note that the top-right square is Cartesian and then checking some commutativity using the relevant functoriality.

We are now able to call the dashed map $\mathrm{Fr}_{X_{\ov k}}$, which we note satisfies
\[({\id_{X_{\ov k}}}\times\sigma)\circ\mathrm{Fr}_{X_{\ov k}}=\sigma_{X_{\ov k}}\]
by construction. Let's see an example.
\begin{example}
	Take $X=\AA^1$. Then one has the following diagram.
	% https://q.uiver.app/#q=WzAsOSxbMCwwLCJcXEFBXjEiXSxbMSwxLCJcXEFBXjEiXSxbMSwwLCJcXEFBXjEiXSxbMiwwLCJcXG92IGtbdF0iXSxbMywxLCJcXG92IGtbdF0iXSxbMywwLCJcXG92IGtbdF0iXSxbNCwwLCJhdCJdLFs1LDEsImFecXRecSJdLFs1LDAsImF0XnEiXSxbMCwxLCJcXHNpZ21hX3tcXEFBXjF9IiwyXSxbMiwxLCJ7XFxpZH1cXHRpbWVzXFxzaWdtYSJdLFswLDIsIlxcbWF0aHJte0ZyfSIsMCx7InN0eWxlIjp7ImJvZHkiOnsibmFtZSI6ImRhc2hlZCJ9fX1dLFszLDRdLFszLDVdLFs1LDRdLFs4LDcsIiIsMCx7InN0eWxlIjp7InRhaWwiOnsibmFtZSI6Im1hcHMgdG8ifX19XSxbNiw3LCIiLDIseyJzdHlsZSI6eyJ0YWlsIjp7Im5hbWUiOiJtYXBzIHRvIn19fV0sWzYsOCwiIiwwLHsic3R5bGUiOnsidGFpbCI6eyJuYW1lIjoibWFwcyB0byJ9LCJib2R5Ijp7Im5hbWUiOiJkYXNoZWQifX19XV0=&macro_url=https%3A%2F%2Fraw.githubusercontent.com%2FdFoiler%2Fnotes%2Fmaster%2Fnir.tex
	\[\begin{tikzcd}
		{\AA^1} & {\AA^1} & {\ov k[t]} & {\ov k[t]} & at & {at^q} \\
		& {\AA^1} && {\ov k[t]} && {a^qt^q}
		\arrow["{\mathrm{Fr}}", dashed, from=1-1, to=1-2]
		\arrow["{\sigma_{\AA^1}}"', from=1-1, to=2-2]
		\arrow["{{\id}\times\sigma}", from=1-2, to=2-2]
		\arrow[from=1-3, to=1-4]
		\arrow[from=1-3, to=2-4]
		\arrow[from=1-4, to=2-4]
		\arrow[dashed, maps to, from=1-5, to=1-6]
		\arrow[maps to, from=1-5, to=2-6]
		\arrow[maps to, from=1-6, to=2-6]
	\end{tikzcd}\]
	Thus, we see on the structure sheaf that we have managed to construct $\mathrm{Fr}$ as a morphism of schemes over $\ov k$, but it is still more or less applying a $q$-power to points.
\end{example}
Now, let $\ov x\in X$ be a geometric point, where its image $x\in X$ is closed. Recall that sheaves on $x$ are essentially abelian groups, which one can see by taking the stalk at $\ov x$; of course, we do see that these abelian groups must come with an action by $\pi_1(x,\ov x)=\op{Gal}(\ov k/\kappa(x))$, which we will call $G$, so taking this stalk produces a discrete abelian $G$-module.

Let $\mc G$ be an \'etale $\ov\QQ_\ell$-sheaf on $X$, and we see that the stalk $\mc G_{\ov x}$ will admit an action by the geometric Frobenius element $F$, which we will call $F_x\colon\mc G_{\ov x}\to\mc G_{F\ov x}$. (This is some formality of scheme morphisms, and we are implicitly using that $F$ is an isomorphism of the topological spaces.) Now, $\mc G$ is really some \'etale $E$-sheaf for an extension $E$ of $\QQ_\ell$. Further, $\mc G$ is really some inverse system $\{\mc G_i\}_{i\ge1}$ of finite \'etale $E$-sheaves. Now, each $\mc G_i$ comes from a bona fide \'etale scheme $G_i$ over $X_{\ov k}$. Then the diagram
% https://q.uiver.app/#q=WzAsNSxbMSwyLCJYX3tcXG92IGt9Il0sWzIsMiwiWF97XFxvdiBrfSJdLFsyLDEsIkdfaSJdLFsxLDEsIkZfe1hfe1xcb3Yga319XipHX2kiXSxbMCwwLCJHX2kiXSxbMCwxLCJGX3tYX3tcXG92IGt9fSIsMl0sWzMsMF0sWzMsMl0sWzIsMV0sWzQsMywiIiwwLHsic3R5bGUiOnsiYm9keSI6eyJuYW1lIjoiZGFzaGVkIn19fV0sWzQsMiwiRl97R19pfSIsMCx7ImN1cnZlIjotMn1dLFs0LDAsIiIsMCx7ImN1cnZlIjoyfV1d&macro_url=https%3A%2F%2Fraw.githubusercontent.com%2FdFoiler%2Fnotes%2Fmaster%2Fnir.tex
\[\begin{tikzcd}
	{G_i} \\
	& {F_{X_{\ov k}}^*G_i} & {G_i} \\
	& {X_{\ov k}} & {X_{\ov k}}
	\arrow[dashed, from=1-1, to=2-2]
	\arrow["{F_{G_i}}", curve={height=-12pt}, from=1-1, to=2-3]
	\arrow[curve={height=12pt}, from=1-1, to=3-2]
	\arrow[from=2-2, to=2-3]
	\arrow[from=2-2, to=3-2]
	\arrow[from=2-3, to=3-3]
	\arrow["{F_{X_{\ov k}}}"', from=3-2, to=3-3]
\end{tikzcd}\]
is able to produce a relative Frobenius morphism $G_i\to F_{X_{\ov k}}\to G_i$, so we produce a genuine map of sheaves
\[\mc G\to F_{X_{\ov k}}^*\mc G\]
upon taking the limit (in the category of \'etale sheaves) again.

We claim there is a canonical isomorphism $F^*_{\mc G}\colon F^*_{X_{\ov k}}\mc G\to\mc G$, which describes the \'etale $\ov\QQ_\ell$-sheaf $\mc G$ as being $\op{Gal}(\ov k/k)$-equivariant sheaf. (Intuitively, one looks at the top horizontal map in the last diagram and takes a limit.) This claim is rather intricate, so we won't prove it: it comes down to explicating what the Galois action should be.

But now the consequence is that we receive a composite
\[\mc G\to F_{X_{\ov k}}^*\mc G\to\mc G.\]
By construction of the last isomorphism, one finds that this is the Frobenius morphism $\mc G_{\ov x}\to\mc G_{\ov x}$ at a geometric point $\ov x\into X$.
This allows to define the $L$-function for $\mc G$.
\begin{definition}
	Fix an \'etale $\ov\QQ_\ell$-sheaf $\mc G$, as above. Then we define
	\[L(X,\mc G,T)\coloneqq\prod_{\text{closed }x\in X}\det\left(1-T^{\deg x}F_x;\mc G_{\ov x}\right)^{-1}.\]
\end{definition}
By using the trace formula, one can prove that this $L$-series has a cohomological interpretation.
\begin{theorem}
	Fix an \'etale $\ov\QQ_\ell$-sheaf $\mc G$ on a scheme $X$ over $\FF_q$. Then
	\[L(X,\mc G;T)=\prod_{i=0}^{2\dim X}\det\left(1-tF^*;\mathrm H^i_c(X_{\ov k},\mc G)\right)^{(-1)^{i+1}}.\]
\end{theorem}
Now, as before, we note that one can construct a map $\mc G\to\mathrm{Fr}_{X_{\ov k}}^*\mc G$, and one can show that there is a canonical isomorphism $\mathrm{Fr}_{\mc G}\colon\mathrm{Fr}_{X_{\ov k}}^*\mc G\to\mc G$. Thus, we gain a diagram
% https://q.uiver.app/#q=WzAsMyxbMCwwLCJcXG1hdGhybSBIXmlfYyhYX3tcXG92IGt9LFxcbWMgRykiXSxbMSwwLCJcXG1hdGhybSBIXmlfYyhYX3tcXG92IGt9LFxcbWF0aHJte0ZyfV97WF97XFxvdiBrfX1eKlxcbWMgRykiXSxbMSwxLCJcXG1hdGhybSBIXmlfYyhYX3tcXG92IGt9LFxcbWMgRykiXSxbMCwxLCJcXG1hdGhybXtGcn1fe1hfe1xcb3Yga319XioiXSxbMSwyLCJcXG1hdGhybXtGcn1fe1xcbWMgR31eKiJdLFswLDIsIkZeKiIsMl1d&macro_url=https%3A%2F%2Fraw.githubusercontent.com%2FdFoiler%2Fnotes%2Fmaster%2Fnir.tex
\[\begin{tikzcd}
	{\mathrm H^i_c(X_{\ov k},\mc G)} & {\mathrm H^i_c(X_{\ov k},\mathrm{Fr}_{X_{\ov k}}^*\mc G)} \\
	& {\mathrm H^i_c(X_{\ov k},\mc G)}
	\arrow["{\mathrm{Fr}_{X_{\ov k}}^*}", from=1-1, to=1-2]
	\arrow["{F^*}"', from=1-1, to=2-2]
	\arrow["{\mathrm{Fr}_{\mc G}^*}", from=1-2, to=2-2]
\end{tikzcd}\]
for which we can use to prove the above theorem.

\end{document}