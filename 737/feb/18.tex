% !TEX root = ../notes.tex

\documentclass[../notes.tex]{subfiles}

\begin{document}

\section{February 18}
Today we continue discussing Weil sheaves.

\subsection{Weil Sheaves from \'Etale Sheaves}
Fix a scheme $X$ over a finite field $\FF_q$, and choose a geometric point $\ov x\into X$. Last time, we explained that Weil sheaves on $X$ correspond to continuous representations of the Weil group $W(X,\ov x)$. Thus, we are interested in the representation theory of $W(X,\ov x)$. The easiest representations (and indeed, the easiest sheaves) have rank $1$, so let's write these down.
\begin{proposition}
	Weil sheaves on $\Spec\FF_q$ of rank $1$ are in bijection with $\ov\QQ_\ell^\times$.
\end{proposition}
\begin{proof}
	A Weil sheaf of rank $1$ is known to correspond to a representation
	\[W(\ov\FF_q/\FF_q)\to\op{GL}_1(\ov\QQ_\ell).\]
	The left-hand side is $\ZZ$, so the result follows.
\end{proof}
\begin{notation}
	For $b\in\ov\QQ_\ell^\times$, we let $\mc L_b$ denote the corresponding Weil sheaf on $\Spec\FF_q$.
\end{notation}
These sheaves help explain when Weil sheaves come from \'etale sheaves.
\begin{theorem} \label{thm:weil-to-et-sheaf}
	Let $X$ be a scheme over $\FF_q$, and let $\mc G_0$ be a Weil sheaf on $X$. Suppose further that $X$ is normal and geometrically connected and that $\mc G_0$ is irreducible (as a representation) and smooth of rank $r$. Then $\mc G_0$ is an \'etale $\ov\QQ_\ell$-sheaf if and only if $\land^r\mc G_0$ is an \'etale sheaf on $X$.
\end{theorem}
Here are some applications of \Cref{thm:weil-to-et-sheaf}.
\begin{corollary} \label{cor:get-weil-is-twist}
	Fix a smooth irreducible Weil sheaf $\mc G_0$ on a normal and geometrically connected scheme $X$ over $\FF_q$. Then there is $b\in\ov\QQ_\ell^\times$ and an \'etale sheaf $\mc F_0$ on $X$ such that
	\[\mc G_0\cong\mc F_0\otimes\mc L_b.\]
\end{corollary}
\begin{proof}
	We will prove this later, along with the theorem.
\end{proof}
\begin{corollary}
	Fix a smooth Weil sheaf $\mc G_0$ on a normal and geometrically connected scheme $X$ over $\FF_q$. Then there is a filtration of subsheaves
	\[0=\mc G_{0}^{(0)}\subseteq\mc G_{0}^{(1)}\subseteq\cdots\subseteq\mc G_{0}^{(r)}=\mc G_0\]
	such that $\mc G_{0}^{(j)}/\mc G_{0}^{(j-1)}$ is isomorphic to some tensor $\mc F_0^{(j)}\otimes\mc L_{b_j}$ with $\mc F_0^{(j)}$ a smooth \'etale $\ov\QQ_\ell$-sheaf and $b_j\in\ov\QQ_\ell^\times$.
\end{corollary}
\begin{proof}
	This follows by using filtrations of sheaves of finite rank and applying the example.
\end{proof}
\begin{remark}
	One can think of the twisting $-\otimes\mc L_b$ as some kind of ampleness condition.
\end{remark}
As another application, we note that this machinery lets us write down an $L$-series for Weil sheaves. The idea is that filtrations decompose a Weil sheaf $\mc G_0$ into some extensions of irreducible sheaves by irreducible sheaves, allowing us to reduce to irreducible Weil sheaves. Then irreducible Weil sheaves can be controlled because there are merely \'etale sheaves twisted by an explicit character. Let's write this out.
\begin{notation}
	Fix a smooth Weil sheaf $\mc G_0$ on a scheme $X$ over $\FF_q$. Then we define its $L$-function by
	\[L(X,\mc G,t)\coloneqq\prod_{\text{closed }x\in X}\det\left(1-t^{\deg x}F_x^*;\mc G_{\ov x}\right)^{-1}.\]
\end{notation}
\begin{corollary} \label{cor:}
	Fix a smooth Weil sheaf $\mc G_0$ on a normal and geometrically connected scheme $X$ over $\FF_q$. Then
	\[L(X,\mc G,t)=\prod_{i=0}^{2\dim X}\det\left(1-tF^*;\mathrm H^i_c(X_{\ov k},\mc G)\right)^{(-1)^{i+1}}.\]
\end{corollary}
Again, we will write this out in more detail next week.

\subsection{Weights}
Let's say something about weights. The motivation is that one can control the location of poles and zeroes of the $L$-function. Throughout, we will fix an isomorphism $\iota\colon\ov\QQ_\ell\to\CC$.
\begin{definition}[pure]
	Fix a scheme $X$ over $\FF_q$, and choose an isomorphism $\iota\colon\ov\QQ_\ell\to\CC$. We say that a smooth Weil sheaf $\mc G_0$ is \textit{$\iota$-pure of weight $\beta$} for some $\beta>0$ if and only if all eigenvalues of $\alpha\in\ov\QQ_\ell$ of $F_x\colon\mc G_{0,\ov x}\to\mc G_{0,\ov x}$ (for any geometric point $\ov x\into X$) has
	\[\left|\iota(\alpha)\right|^2=\#\kappa(x)^\beta.\]
	We say that $\mc G_0$ is \textit{pure of weight $\beta$} if and only if it is $\iota$-pure of weight $\beta$ for all chosen $\iota$.
\end{definition}
The sheaves one comes across in practice may not be pure on the nose but instead have some pure part that we can remove and then handle via some induction.
\begin{definition}[mixed]
	Fix a scheme $X$ over $\FF_q$, and choose an isomorphism $\iota\colon\ov\QQ_\ell\to\CC$. We say that a smooth Weil sheaf $\mc G_0$ is \textit{$\iota$-mixed} if and only if there is a finite filtration
	\[0=\mc G_0^{(0)}\subseteq\mc G_0^{(1)}\subseteq\cdots\subseteq\mc G_0^{(r)}=\mc G_0\]
	such that the quotients $\mc G_0^{(i)}/\mc G_{0}^{(i-1)}$ are $\iota$-pure of some weight. We say that $\mc G_0$ is \textit{mixed} if and only if it is $\iota$-mixed for all chosen $\iota$.
\end{definition}
A motivation of Weil II is that we would like these weights to be found as some geometric invariant. For example, we expect that the weights of sheaves to be preserved under some controlled morphisms.
\begin{proposition}
	Suppose that $\pi\colon X\to Y$ is a morphism of schemes over $\FF_q$. Given a Weil sheaf $\mc G_0$ on $Y$.
	\begin{listalph}
		\item If $\mc G_0$ is $\iota$-pure of weight $\beta$, then $f^*\mc G_0$ is $\iota$-pure of weight $\beta$.
		\item If $f$ is surjective and $f^*\mc G_0$ is $\iota$-pure of weight $\beta$, then $\mc G_0$ is $\iota$-pure of weight $\beta$.
		\item If $f$ is finite and $\mc G_0$ is $\iota$-pure of weight $\beta$, then $f_*\mc G_0$ is $\iota$-pure of weight $\beta$.
		\item If $\mc G_0$ is a Weil sheaf on $X$, then $\mc G_0$ on $X$ is $\iota$-pure of weight $\beta$ if and only if $\mc G_0$ on $X_{\FF_{q^r}}$ is $\iota$-pure of weight $\beta$.
	\end{listalph}
\end{proposition}
\begin{proof}[Sketch]
	Here, (a) and (b) are essentially formal because the fibers of $f^*\mc G_0$ are the same as the fibers of $\mc G_0$. For (d), this is again purely formal from computing how the Frobenius and the degree simultaneously adjust on extension of the base field. Alternatively, (d) can be derived from (a)--(c) because the canonical morphism $X_{\FF_{q^r}}\to X_{\FF_q}$ is surjective and finite. We won't say anything about (c), but it should also follow from some fiber-wise computations.
\end{proof}
For inductive applications, it may be useful to have some largest weight.
\begin{definition}
	Fix a scheme $X$ over $\FF_q$, and let $\mc G_0$ be a Weil sheaf on $X$, and choose some isomorphism $\iota\colon\ov\QQ_\ell\to\CC$. Then we define
	\[w(\mc G_0)\coloneqq\sup_{\text{closed }x\in X}\Bigg(\sup_{\text{eigenvalue }\alpha}\frac{\log\left|\iota(\alpha)\right|^2}{\log\#\kappa(x)}\Bigg)\]
	if $\mc G_0$ is nontrivial, and $w(0)=-\infty$.
\end{definition}

\end{document}