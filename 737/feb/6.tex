% !TEX root = ../notes.tex

\documentclass[../notes.tex]{subfiles}

\begin{document}

\section{February 6}

We didn't have class on Tuesday. We will continue doing a little review of \'etale cohomology. In particular, we will talk about the \'etale fundamental group.

\subsection{Galois Theory of Schemes}
We would like to bring the notion of the topological fundamental group to the \'etale fundamental group. Roughly speaking, the topological fundamental group can be realized as the group of automorphisms of a universal cover, and this universal cover can be seen as some limit of finite covers. In algebraic geometry, we do not have access to this limit of finite covers, but we do have access to finite covers: they are \'etale coverings.

Before saying anything of substance, we recall some properties of \'etale morphisms.
\begin{lemma}
	Fix scheme morphisms $f\colon X\to Y$ and $g\colon Y\to X$.
	\begin{listalph}
		\item Composition: if $f$ and $g$ are \'etale, then $g\circ f$ is \'etale.
		\item Cancellation: if $g\circ f$ and $g$ are \'etale, then $f$ is \'etale.
	\end{listalph}
\end{lemma}
\begin{lemma}
	\'Etale morphisms are preserved by base-change. More precisely, fix a pullback square
	% https://q.uiver.app/#q=WzAsNCxbMSwxLCJTIl0sWzAsMSwiUyciXSxbMSwwLCJYIl0sWzAsMCwiWCciXSxbMSwwXSxbMiwwLCJmIl0sWzMsMSwiZiciLDJdLFszLDJdXQ==&macro_url=https%3A%2F%2Fraw.githubusercontent.com%2FdFoiler%2Fnotes%2Fmaster%2Fnir.tex
	\[\begin{tikzcd}
		{X'} & X \\
		{S'} & S
		\arrow[from=1-1, to=1-2]
		\arrow["{f'}"', from=1-1, to=2-1]
		\arrow["f", from=1-2, to=2-2]
		\arrow[from=2-1, to=2-2]
	\end{tikzcd}\]
	of schemes. If $f$ is \'etale, then so is $f'$.
\end{lemma}
\begin{lemma}
	Fix a scheme morphism $f\colon X\to Y$. Then $f$ is unramified if and only if the diagonal $\Delta_f\colon X\to X\times_YX$ is an open embedding.
\end{lemma}
The following result should be understood as a version of ``unique path-lifting.''
\begin{proposition}[Rigidity] \label{prop:rigidity-of-etale}
	Fix finite \'etale schemes $X$ and $Y$ over $S$. If $X$ is connected, and $Y$ is separated, then any geometric point $\ov x\into X$ (where $\ov x=\Spec\ov k$) induces an injection
	\[\op{Mor}_S(X,Y)\to Y(\ov k)\]
	given by $f\mapsto f(\ov x)$.
\end{proposition}
\begin{proof}[Sketch]
	Suppose we have two morphisms $f,g\colon X\to Y$ over $S$ with $f(\ov x)=g(\ov x)$. Because $X$ and $Y$ are both \'etale over $S$, we see that $f$ and $g$ are \'etale. Thus, $f$ and $g$ are local isomorphisms, so the connectivity of $X$ tells us that they are determined by the image of $\ov x$.
\end{proof}
\begin{corollary}
	Suppose $S$ is a connected scheme and that $X\to S$ is a finite \'etale morphism. Choose a geometric point $\ov s\into S$, and choose some $\ov x\into X$ in the fiber. Then the map
	\[\op{Aut}_S(X)\to X_{\ov s}\]
	given by $f\mapsto f(\ov x)$ is injective.
\end{corollary}
\begin{proof}
	Set $X=Y$ in \Cref{prop:rigidity-of-etale}.
\end{proof}
\begin{remark}
	If $X$ and $S$ are points, then we are looking at some finite separable field extension $L/K$, and the corollary amounts to asserting that the following map is an injective: define
	\[\op{Aut}_K(L)\into\op{Hom}_K(L,\ov K)\]
	by sending $\sigma\colon L\to L$ to the embedding $\iota\circ\sigma$, where $\iota\colon L\into\ov K$ is a chosen embedding. Note that $L/K$ is a Galois extension if and only if this map is an isomorphism!
\end{remark}
The above remark motivates the following definition.
\begin{definition}[Galois]
	Fix a connected scheme $S$. A finite \'etale morphism $f\colon X\to S$ is \textit{Galois} if and only if $\#\op{Aut}_S(X)=\#X_{\ov s}$ for some (or equivalently, any) geometric point $\ov s\into S$.
\end{definition}
We are now motivated to build a ``Galois theory'' of schemes. Here are some statements, which we will not prove.
\begin{proposition}
	Fix a connected scheme $S$, and choose a Galois covering $X\to S$.
	\begin{listalph}
		\item There is a map sending finite \'etale subcovers $Y\to S$ of $X\to S$ to the subgroup of $\op{Aut}_Y(X)\subseteq\op{Aut}_S(X)$.
		\item There is a map sending subgroups $H\subseteq\op{Aut}_S(X)$ to finite \'etale subcovers $X^H\to S$ of $X\to Y$.
		\item The two maps of (a) and (b) are order-preserving mutually inverse isomorphisms.
	\end{listalph}
\end{proposition}
\begin{remark}
	There is some difficulty in constructing the quotient $X^H$. Roughly speaking, one can take fixed points (as expected) on the level of affine schemes, and then one uses the fact that our morphisms are finite and \'etale to show that this is good enough for the general case.
\end{remark}
\begin{proposition}
	Fix a connected scheme $S$. For any finite \'etale covering $X\to S$, there is a unique (up to isomorphism) connected finite \'etale covering $X'\to X$ satisfying the following conditions.
	\begin{listroman}
		\item The composite covering $X'\to S$ is Galois.
		\item Any other Galois covering $X''\to S$ which factors through $X\to S$ will also factor uniquely through $X''\to X'$.
	\end{listroman}
\end{proposition}
So we have a Galois correspondence and a construction of normal closures.

\subsection{The \'Etale Fundamental Group}
Let's now turn to the absolute Galois group, extending our discussion of Galois theory of schemes.
\begin{defihelper}[\'etale fundamental group] \nirindex{etale fundamental group@\'etale fundamental group}
	Fix a connected scheme $S$ and a geometric point $\ov s\into S$. Then we define the \textit{\'etale fundamental group} $\pi_1^{\mathrm{\acute et}}(S,\ov s)$ as the automorphism group of the fiber functor $\mathrm{Fib}_{\ov s}$ sending finite \'etale covers $X\to S$ to the set $\op{Hom}_S(\ov s,X)$.
\end{defihelper}
\begin{remark}
	It is a theorem that $\mathrm{Fib}$ upgrades to an equivalence of categories between the category of finite \'etale covers $X\to S$ and the category of finite $\pi_1^{\mathrm{\acute et}}(S,\ov s)$-sets.
\end{remark}
\begin{example}
	Let $k$ be an algebraically closed field, and take $S=\Spec k$ with geometric points $S=S$. Then finite \'etale covers of $S$ all look like $S\sqcup\cdots\sqcup S$, which is a unique map down to $S$, so the fiber functor has no nontrivial automorphisms. We conclude that $\pi_1^{\mathrm{\acute et}}(S,\ov s)=1$.
\end{example}
\begin{remark}
	Intuitively, the automorphism group of the fiber functor should grow in size if there are fewer finite \'etale covers.
\end{remark}
One is even able to recover some kind of universal cover, but it is not exactly a scheme: it is a limit of schemes.
\begin{proposition}
	Fix a connected scheme $S$ and a geometric point $\ov s\into S$, and let $\{(X_\alpha,\ov x_\alpha)\}$ be the inverse system of all finite Galois covers of $(S,\ov s)$ preserving the (geometric) basepoint.
	\begin{listalph}
		\item For any finite \'etale cover $Y\to S$, we have
		\[\mathrm{Fib}_{\ov s}(Y)=\colimit\op{Hom}_S(X_\bullet,Y).\]
		\item We have
		\[\pi_1^{\mathrm{\acute et}}(S,\ov s)=\limit\op{Aut}_S(X_\bullet)\opp.\]
	\end{listalph}
\end{proposition}
\begin{example}
	Let $k$ be any field, and take $S=\Spec k$ with geometric point given by a chosen embedding $k\into\ov k$. Then the proposition tells us
	\[\pi_1^{\mathrm{\acute et}}(\op{Spec}k,\op{Spec}\ov k)=\limit_{\text{Galois }L/k}\op{Gal}(L/k)=\op{Gal}(k^{\mathrm{sep}}/k).\]
\end{example}
\begin{remark}
	For two basepoints $\ov s_1,\ov s_2\into S$, one can produce a natural isomorphism of the relevant fiber functors, so the \'etale fundamental groups become isomorphic by some isomorphism coming from conjugation.
\end{remark}
\begin{remark}
	Choose a connected scheme $W$ of finite type over $\ZZ$, and choose a closed point $w\in W$, which comes from a morphism $\op{Spec}k(w)\into W$. A choice of algebraic closure of $k(w)$ then gives a geometric point $\ov w\into W$. There is thus a map
	\[\pi_1^{\mathrm{\acute et}}(\op{Spec}k(w),\ov w)\to\pi_1^{\mathrm{\acute et}}(W,\ov w)\]
	by base-changing automorphisms of the fiber functor for $W$ to merely the closed point $w$. Now, $k(w)$ should be a finite field, so the left-hand group is topologically generated by a Frobenius element; note that one typically takes the geometric Frobenius to topologically generate $\pi_1^{\mathrm{\acute et}}(\op{Spec}k(w),\ov w)$.
	
	Changing the embedding $k(w)\into\ov {k(w)}$ tells us that this Frobenius element in $\pi_1^{\mathrm{\acute et}}(W,\ov w)$ should really be only defined up to conjugacy. In this way, we produce a canonical conjugacy class in $\pi_1^{\mathrm{\acute et}}(W,\ov w)$ for each closed point in $W$.
\end{remark}
As a last statement, we recall the homotopy exact sequence.
\begin{theorem}
	Fix a geometrically connected and quasicompact scheme $S$ over a field $k$, and choose a basepoint $\ov x\into S_{\mathrm{k^{\mathrm{sep}}}}$. Then there is a short exact sequence
	\[1\to\pi_1^{\mathrm{\acute et}}(S_{k^{\mathrm{sep}}},\ov s)\to\pi_1^{\mathrm{\acute et}}(S,\ov s)\to\pi_1^{\mathrm{\acute et}}(\op{Spec}k)\to1.\]
\end{theorem}
\begin{remark}
	We won't prove the theorem, but we will remark that the sequence of morphisms is induced by functoriality from the morphisms
	\[\ov s\into S\to\op{Spec}k.\]
\end{remark}

\end{document}