% !TEX root = ../notes.tex

\documentclass[../notes.tex]{subfiles}

\begin{document}

\section{February 25}
Today we will start moving towards a proof of \Cref{thm:weil-to-et-sheaf}, but this will require some preparation.

\subsection{Finite Geometric Monodromy}
Our first piece of preparation is the following result.
\begin{theorem}[Grothendieck] \label{thm:weil-char-torsion-finite-geo-mono}
	Fix a normal geometrically irreducible scheme $X$ over $\FF_q$, and choose a geometric point $\ov x\into X$. Further, fix a continuous character $\chi\colon W(X,\ov x)\to\ov\QQ_\ell^\times$. Then $\chi\left(\pi_1^{\mathrm{\acute et}}(X_{\ov\FF_q},\ov x)\right)$ is finite.
\end{theorem}
Roughly speaking, the above result will follow from geometric class field theory because we are looking at some degree-$0$ Picard group. We remark that this will enable us to prove \Cref{cor:get-weil-is-twist} because one can use the character given above to define $\mc F_0$ and then twist by $\mc L_b$ in order to fix the action of the Frobenius (which is not seen in the geometric \'etale fundamental group).
\begin{remark} \label{rem:power-of-weil-char}
	Equivalently, one can view \Cref{thm:weil-char-torsion-finite-geo-mono} as saying that $\chi^m$ is trivial on $\pi_1^{\mathrm{\acute et}}(X_{\ov\FF_q},\ov x)$ for sufficiently divisible $m$. As such, one may (without loss of generality) pass from $X$ to some base-change and pass from $\chi$ to some finite power. Also, this point of view allows us to use \Cref{thm:weil-char-torsion-finite-geo-mono} to factor $\chi$ into two characters $\chi_1\chi_2$ where $\chi_1$ has finite order and $\chi_2$ factors through the quotient $W(X,\ov x)\onto W(\ov\FF_q/\FF_q)$.
\end{remark}
\begin{remark}
	Morally, the above remark is decomposing the character $\chi$ into the unramified piece $\chi_2$ (vanishing on inertia) and the ramified piece $\chi_1$, and we are being given control $\chi_1$ by forcing it to be finite-order.
\end{remark}
To visualize the quotient in the remark, we draw the morphism
% https://q.uiver.app/#q=WzAsMTAsWzAsMCwiMSJdLFsxLDAsIlxccGlfMV57XFxtYXRocm17XFxhY3V0ZSBldH19KFhfe1xcb3ZcXEZGX3F9LFxcb3YgeCkiXSxbMSwxLCJcXHBpXzFee1xcbWF0aHJte1xcYWN1dGUgZXR9fShYX3tcXG92XFxGRl9xfSxcXG92IHgpIl0sWzAsMSwiMSJdLFsyLDEsIlcoWCxcXG92IHgpIl0sWzIsMCwiXFxwaV8xXntcXG1hdGhybXtcXGFjdXRlIGV0fX0oWCxcXG92IHgpIl0sWzMsMCwiXFxvcHtHYWx9KFxcb3ZcXEZGX3EvXFxGRl9xKSJdLFszLDEsIlcoXFxvdlxcRkZfcS9cXEZGX3EpIl0sWzQsMCwiMSJdLFs0LDEsIjEiXSxbMCwxXSxbMSw1XSxbNSw2XSxbNiw4XSxbMywyXSxbMiw0XSxbNCw3XSxbNyw5XSxbNCw1LCJcXHN1YnNldGVxIiwzLHsic3R5bGUiOnsiYm9keSI6eyJuYW1lIjoibm9uZSJ9LCJoZWFkIjp7Im5hbWUiOiJub25lIn19fV0sWzcsNiwiXFxzdWJzZXRlcSIsMyx7InN0eWxlIjp7ImJvZHkiOnsibmFtZSI6Im5vbmUifSwiaGVhZCI6eyJuYW1lIjoibm9uZSJ9fX1dLFsyLDEsIiIsMSx7ImxldmVsIjoyLCJzdHlsZSI6eyJoZWFkIjp7Im5hbWUiOiJub25lIn19fV1d&macro_url=https%3A%2F%2Fraw.githubusercontent.com%2FdFoiler%2Fnotes%2Fmaster%2Fnir.tex
\[\begin{tikzcd}
	1 & {\pi_1^{\mathrm{\acute et}}(X_{\ov\FF_q},\ov x)} & {\pi_1^{\mathrm{\acute et}}(X,\ov x)} & {\op{Gal}(\ov\FF_q/\FF_q)} & 1 \\
	1 & {\pi_1^{\mathrm{\acute et}}(X_{\ov\FF_q},\ov x)} & {W(X,\ov x)} & {W(\ov\FF_q/\FF_q)} & 1
	\arrow[from=1-1, to=1-2]
	\arrow[from=1-2, to=1-3]
	\arrow[from=1-3, to=1-4]
	\arrow[from=1-4, to=1-5]
	\arrow[from=2-1, to=2-2]
	\arrow[equals, from=2-2, to=1-2]
	\arrow[from=2-2, to=2-3]
	\arrow["\subseteq"{marking, allow upside down}, draw=none, from=2-3, to=1-3]
	\arrow[from=2-3, to=2-4]
	\arrow["\subseteq"{marking, allow upside down}, draw=none, from=2-4, to=1-4]
	\arrow[from=2-4, to=2-5]
\end{tikzcd}\]
of short exact sequences, which allows us to see what it means for the aforementioned second character $\chi_2$ to factor through $W(\ov\FF_q/\FF_q)\cong\ZZ$.
\begin{remark}
	Intuitively, the geometric \'etale fundamental group $\pi_1^{\mathrm{\acute et}}(X_{\ov\FF_q},\ov x)$ remembers topological information, notably forgetting the ``arithmetic'' information coming from having Frobenius elements in $\pi_1^{\mathrm{\acute et}}(X,\ov x)$.
\end{remark}
We now proceed with the argument.
\begin{proof}[Proof of \Cref{thm:weil-char-torsion-finite-geo-mono}]
	We have the following steps.
	\begin{enumerate}
		\item Let's describe a few reductions, which we will use freely.
		\begin{itemize}
			\item As discussed in \Cref{rem:power-of-weil-char}, we may base-change $X$, for example to ensure that it has a rational point. Additionally, we may pass from $\chi$ to a power.
	
			\item Because $\chi$ is continuous, we know that it should factor through $E^\times\subseteq\ov\QQ_\ell^\times$ for some finite extension $E$ of $\QQ_\ell$. By factoring $E^\times$ as a group, we find that by (passing to a power of $\chi$), we may automatically assume that the image $\chi\left(\pi_1^{\mathrm{\acute et}}(X_{\ov\FF_q},\ov x)\right)$ is a pro-$\ell$ group.
	
			\item Because $X$ is already normal, any open nonempty subscheme $U\subseteq X$ containing $\ov x$ yields a surjection $\pi_1^{\mathrm{\acute et}}(U,\ov x)\onto\pi_1^{\mathrm{\acute et}}(X,\ov x)$; this allows us to prove the theorem by passing to open subsets of $X$.
	
			For example, by passing from $X$ to the smooth locus (which is nonempty because $X$ is already normal), we may assume that $X$ is smooth. Certainly we may assume that $X$ is quasiprojective by passing to small affine open subsets.
		\end{itemize}
	
		\item To get our engines running, we treat the case of a curve. We use the theory of valuations. Because $X$ is normal and geometrically irreducible, one gets a surjection
		\[\op{Gal}(K^{\mathrm{sep}}/K)\onto\pi_1^{\mathrm{\acute et}}(X,\ov x),\]
		where $K$ is the function field of $X$; intuitively, the idea is that one can take function fields of any \'etale covers of $X$ to produce separable extensions of $K$.
		
		Now that we are in the curve case, we would like to do something with (geometric) class field theory. Define $I_K$ to fit into the short exact sequence
		\[0\to I_K\to W(X,\ov x)^{\mathrm{ab}}\stackrel\deg\to W(\ov\FF_q/\FF_q)\to0.\]
		In particular, $I_K$ is some kind of inertia subgroup.
	
		We now have a large diagram as follows.
		% https://q.uiver.app/#q=WzAsMTksWzAsMiwiMCJdLFsxLDIsIklfSyJdLFsyLDIsIlcoWCxcXG92IHgpXntcXG1hdGhybXthYn19Il0sWzMsMiwiVyhcXG92XFxGRl9xL1xcRkZfcSkiXSxbNCwyLCIwIl0sWzEsMywiXFxwaV8xXntcXG1hdGhybXtcXGFjdXRlIGV0fX0oXFxvdiBYLFxcb3YgeClee1xcbWF0aHJte2FifX0iXSxbMiwzLCJcXHBpXzFee1xcbWF0aHJte1xcYWN1dGUgZXR9fShYLFxcb3YgeClee1xcbWF0aHJte2FifX0iXSxbMiwxLCJLXlxcdGltZXNcXGJhY2tzbGFzaFxcQUFfS15cXHRpbWVzL1xcd2lkZWhhdFxcT09fS15cXHRpbWVzIl0sWzMsMSwicV5cXFpaIl0sWzQsMSwiMCJdLFsxLDEsIkteXFx0aW1lc1xcYmFja3NsYXNoXFxBQV9LXntcXHRpbWVzLDF9L1xcd2lkZWhhdFxcT09fS15cXHRpbWVzIl0sWzAsMSwiMCJdLFszLDMsIlxcb3B7R2FsfShcXG92XFxGRl9xL1xcRkZfcSkiXSxbNCwzLCIwIl0sWzIsMCwiXFxQaWMoWCkoXFxGRl9xKSJdLFsxLDAsIlxcUGljKFgpXlxcY2lyYyhcXEZGX3EpIl0sWzMsMCwiXFxaWiJdLFs0LDAsIjAiXSxbMCwwLCIwIl0sWzIsMywiXFxkZWciXSxbNyw4LCJcXG9wIE4iXSxbNywyLCJcXHNpbSJdLFsxMSwxMF0sWzEwLDddLFs4LDldLFsxMCwxLCJcXHNpbSJdLFs4LDMsIlxcc2ltIl0sWzAsMV0sWzEsMl0sWzMsNF0sWzUsNl0sWzYsMTJdLFsxMiwxM10sWzMsMTIsIlxcc3Vic2V0ZXEiLDMseyJzdHlsZSI6eyJib2R5Ijp7Im5hbWUiOiJub25lIn0sImhlYWQiOnsibmFtZSI6Im5vbmUifX19XSxbMiw2LCJcXHN1YnNldGVxIiwzLHsic3R5bGUiOnsiYm9keSI6eyJuYW1lIjoibm9uZSJ9LCJoZWFkIjp7Im5hbWUiOiJub25lIn19fV0sWzUsMSwiIiwzLHsic3R5bGUiOnsiaGVhZCI6eyJuYW1lIjoiZXBpIn19fV0sWzE1LDE0LCIiLDMseyJzdHlsZSI6eyJ0YWlsIjp7Im5hbWUiOiJob29rIiwic2lkZSI6InRvcCJ9fX1dLFsxNCwxNiwiXFxkZWciXSxbMTAsMTUsIiIsMCx7ImxldmVsIjoyLCJzdHlsZSI6eyJoZWFkIjp7Im5hbWUiOiJub25lIn19fV0sWzcsMTQsIiIsMCx7ImxldmVsIjoyLCJzdHlsZSI6eyJoZWFkIjp7Im5hbWUiOiJub25lIn19fV0sWzE4LDE1XSxbMTYsMTddLFsxNiw4LCJcXHNpbSJdXQ==&macro_url=https%3A%2F%2Fraw.githubusercontent.com%2FdFoiler%2Fnotes%2Fmaster%2Fnir.tex
		\[\begin{tikzcd}
			0 & {\Pic(X)^\circ(\FF_q)} & {\Pic(X)(\FF_q)} & \ZZ & 0 \\
			0 & {K^\times\backslash\AA_K^{\times,1}/\widehat\OO_K^\times} & {K^\times\backslash\AA_K^\times/\widehat\OO_K^\times} & {q^\ZZ} & 0 \\
			0 & {I_K} & {W(X,\ov x)^{\mathrm{ab}}} & {W(\ov\FF_q/\FF_q)} & 0 \\
			& {\pi_1^{\mathrm{\acute et}}(\ov X,\ov x)^{\mathrm{ab}}} & {\pi_1^{\mathrm{\acute et}}(X,\ov x)^{\mathrm{ab}}} & {\op{Gal}(\ov\FF_q/\FF_q)} & 0
			\arrow[from=1-1, to=1-2]
			\arrow[from=1-2, to=1-3]
			\arrow["\deg", from=1-3, to=1-4]
			\arrow[from=1-4, to=1-5]
			\arrow["\sim", from=1-4, to=2-4]
			\arrow[from=2-1, to=2-2]
			\arrow[equals, from=2-2, to=1-2]
			\arrow[from=2-2, to=2-3]
			\arrow["\sim", from=2-2, to=3-2]
			\arrow[equals, from=2-3, to=1-3]
			\arrow["{\op N}", from=2-3, to=2-4]
			\arrow["\sim", from=2-3, to=3-3]
			\arrow[from=2-4, to=2-5]
			\arrow["\sim", from=2-4, to=3-4]
			\arrow[from=3-1, to=3-2]
			\arrow[from=3-2, to=3-3]
			\arrow["\deg", from=3-3, to=3-4]
			\arrow["\subseteq"{marking, allow upside down}, draw=none, from=3-3, to=4-3]
			\arrow[from=3-4, to=3-5]
			\arrow["\subseteq"{marking, allow upside down}, draw=none, from=3-4, to=4-4]
			\arrow[two heads, from=4-2, to=3-2]
			\arrow[from=4-2, to=4-3]
			\arrow[from=4-3, to=4-4]
			\arrow[from=4-4, to=4-5]
		\end{tikzcd}\]
		The isomorphism of the top two rows is formal algebraic geometry. The isomorphism of the second and third row is essentially geometric class field theory. The maps between the third and fourth rows are more or less formal coming from the various constructions of the various Weil groups.
		
		Now, after extending $X$ so that it admits a rational point, we see that $\Pic(X)^\circ$ is the Jacobian of $X$ and hence projective (it's an abelian variety), so it will only have finitely many $\FF_q$-rational points. We now note that $\chi$ factors through $W(X,\ov x)^{\mathrm{ab}}$ (because it's a character), so the commutative diagram
		% https://q.uiver.app/#q=WzAsNSxbMCwwLCJcXHBpXzFee1xcbWF0aHJte1xcYWN1dGUgZXR9fShYX3tcXG92XFxGRl9xfSxcXG92IHgpIl0sWzEsMCwiVyhYLFxcb3YgeCkiXSxbMSwxLCJXKFgsXFxvdiB4KV57XFxtYXRocm17YWJ9fSJdLFswLDEsIlxcUGljKFgpXlxcY2lyYyhcXEZGX3EpIl0sWzIsMCwiXFxvdlxcUVFfXFxlbGxeXFx0aW1lcyJdLFswLDMsIiIsMCx7InN0eWxlIjp7ImhlYWQiOnsibmFtZSI6ImVwaSJ9fX1dLFswLDFdLFszLDJdLFsxLDJdLFsyLDQsIiIsMix7InN0eWxlIjp7ImJvZHkiOnsibmFtZSI6ImRhc2hlZCJ9fX1dLFsxLDQsIlxcY2hpIl1d&macro_url=https%3A%2F%2Fraw.githubusercontent.com%2FdFoiler%2Fnotes%2Fmaster%2Fnir.tex
		\[\begin{tikzcd}
			{\pi_1^{\mathrm{\acute et}}(X_{\ov\FF_q},\ov x)} & {W(X,\ov x)} & {\ov\QQ_\ell^\times} \\
			{\Pic(X)^\circ(\FF_q)} & {W(X,\ov x)^{\mathrm{ab}}}
			\arrow[from=1-1, to=1-2]
			\arrow[two heads, from=1-1, to=2-1]
			\arrow["\chi", from=1-2, to=1-3]
			\arrow[from=1-2, to=2-2]
			\arrow[from=2-1, to=2-2]
			\arrow[dashed, from=2-2, to=1-3]
		\end{tikzcd}\]
		allows us to use the finiteness of $\Pic(X)^\circ$ to conclude the proof.
	
		\item We now complete the proof, assuming that $X$ is smooth and quasiprojective. This requires a compactification. It is possible to produce a compactification $X'$ of $X$ which is normal and projective; say $X'\subseteq\PP^N$ for some $N\ge0$. Taking the regular locus of $X'$ grants an open smooth subscheme $Y\subseteq X'$ containing $X$; say $Z\coloneqq X'\setminus Y$, which has codimension at least $2$ by normality of $X'$.
	
		We quickly handle the case where $X=Y$. If $X=Y$ so that $D=\emp$, then we choose a generic linear subspace $L\subseteq\PP^N_{\ov\FF_q}$ (via Bertini's theorem) of codimension $\dim X_{\ov\FF_q}-1$ such that $Z\cap L=\emp$ and $C\coloneqq X_{\ov\FF_q}\cap L$ is a smooth irreducible curve. By moving the geometric point $\ov x$ to $C$, it turns out that we admit a surjection $\pi_1(C,\ov x)\onto\pi_1(X_{\ov\FF_q},\ov x)$. So the result follows from the case of curves.
		
		It remains to handle the case where $X\subsetneq Y$. Now, Chow's lemma tells us that $D\coloneqq Y\setminus X$ is a locally closed subscheme which is pure of codimension $1$ with smooth components; say $D=D_1\cup\cdots\cup D_r$ is a decomposition into connected components, and extending the base field actually allows us to assume that these components are geometrically irreducible and admitting a rational point. (Namely, such connected components are irreducible already by smoothness.)

		For each $D_i$, let $x_i\in D_i$ be an $\mathbb F_q$-rational point. Further, let $Y_i\coloneqq\op{Spec}\OO_{Y,x_i}^{\mathrm h}$, where $(\cdot)^{\mathrm h}$ denotes Henselization; and let $X_{i,\ov\FF_q}$ be the inverse image of $X_{\ov\FF_q}$ in $Y_{i,\ov\FF_q}$, where $Y_{i,\ov\FF_q}$ should technically denote a strict Henselization. With this notation, each $i$ produces a morphism of short exact sequences as follows.
		% https://q.uiver.app/#q=WzAsMTAsWzAsMCwiMSJdLFsxLDAsIlxccGlfMV57XFxtYXRocm17XFxhY3V0ZSBldH19KFhfe2ksXFxvdlxcRkZfcX0sXFxvdiB4X2kpIl0sWzIsMCwiVyhYX2ksXFxvdiB4KSJdLFszLDAsIlcoXFxvdlxcRkZfcS9cXEZGX3EpIl0sWzQsMCwiMSJdLFswLDEsIjEiXSxbMSwxLCJcXHBpXzFee1xcbWF0aHJte1xcYWN1dGUgZXR9fShYX3tcXG92XFxGRl9xfSxcXG92IHhfaSkiXSxbMiwxLCJXKFgsXFxvdiB4KSJdLFszLDEsIlcoXFxvdlxcRkZfcS9cXEZGX3EpIl0sWzQsMSwiMSJdLFswLDFdLFsxLDJdLFsyLDNdLFszLDRdLFs1LDZdLFs2LDddLFs3LDhdLFsxLDZdLFsyLDddLFszLDhdLFs4LDldXQ==&macro_url=https%3A%2F%2Fraw.githubusercontent.com%2FdFoiler%2Fnotes%2Fmaster%2Fnir.tex
		\[\begin{tikzcd}[cramped]
			1 & {\pi_1^{\mathrm{\acute et}}(X_{i,\ov\FF_q},\ov x)} & {W(X_i,\ov x)} & {W(\ov\FF_q/\FF_q)} & 1 \\
			1 & {\pi_1^{\mathrm{\acute et}}(X_{\ov\FF_q},\ov x)} & {W(X,\ov x)} & {W(\ov\FF_q/\FF_q)} & 1
			\arrow[from=1-1, to=1-2]
			\arrow[from=1-2, to=1-3]
			\arrow[from=1-2, to=2-2]
			\arrow[from=1-3, to=1-4]
			\arrow[from=1-3, to=2-3]
			\arrow[from=1-4, to=1-5]
			\arrow[from=1-4, to=2-4]
			\arrow[from=2-1, to=2-2]
			\arrow[from=2-2, to=2-3]
			\arrow[from=2-3, to=2-4]
			\arrow[from=2-4, to=2-5]
		\end{tikzcd}\]
		(We are ignoring complications which come from basepoints.) Now, $\chi\left(\pi_1^{\mathrm{\acute et}}(X_{\ov\FF_q},\ov x)\right)$ is already understood to be a pro-$\ell$ group, which we achieved because the image lives in $\ov\QQ_\ell^\times$, so any subgroup can be made into a pro-$\ell$ group by passing to some power. Thus, the character $\chi$ is merely tamely ramified (note $\ell\ne p$!) along $D_{i,\ov\FF_q}$ when restricted to $W(X_{i,\ov\FF_q},\ov x)$. In fact, the same reasoning explains that $\chi$ actually factors through the abelianization of the $\ell$-part of the tame ramification group, which turns out to be isomorphic to $\ZZ_\ell$. We denote this last group by $J$.

		Now, let $\sigma\in W(\ov\FF_q/\FF_q)$ be the Frobenius, and we choose an inverse image $\widetilde\sigma$ in $W(X_i,\ov x)$. Then the action of $W(\ov\FF_q/\FF_q)$ on $W(X,\ov x)$ is given by $\widetilde\sigma\gamma\widetilde\sigma^{-1}=\gamma^q$ for $\gamma\in J$. However, $\chi$ (still restricted to $W(X_{i,\ov\FF_q},\ov x)$) needs to behave the same on $\gamma$ and $\widetilde\sigma\gamma\widetilde\sigma^{-1}$, so we are forced to have $\chi$ factor through $J/J^q$, which is a finite group!

		Looping over all $i$, we know that a power $\chi^m$ of $\chi$ will be trivial on each $W(X_{i,\ov\FF_q},\ov x)$. We conclude that $\chi^m$ factors through $W(Y,\ov x)$, and we are allowed to reduce to the case where $Y=X$.
		\qedhere
	\end{enumerate}
\end{proof}
We now achieve the following corollary.
\begin{corollary}
	Fix a smooth Weil sheaf $\mc G_0$ of rank $1$ on a normal geometrically irreducible scheme $X$ over $\FF_q$.
	\begin{listalph}
		\item There is an \'etale sheaf $\mc F_0$ (attached to a finite character) and $b\in\ov\QQ_\ell^\times$ such that $\mc G_0=\mc F_0\otimes\mc L_b$.
		\item The sheaf $\mc G_0$ is $\iota$-pure for $\iota=\log\left|\iota(b)\right|^2/\log q$.
	\end{listalph}
\end{corollary}
\begin{proof}
	The sheaf $\mc G_0$ arises from a representation $\chi\colon W(X,\ov x)\to\ov\QQ_\ell^\times$ for some $\chi$. Then one factors $\chi$ as $\chi_1\chi_2$, where $\chi_1$ has finite order and $\chi_2$ factors through $W(\ov\FF_q/\FF_q)\cong\ZZ$. Then $\chi_1$ produces $\mc F_0$, and $\chi_2$ produces $\mc L_b$, which proves (a). To prove (a), we note that $\mc F_0$ has weight $0$ (because its eigenvalues of Frobenius are roots of unity) and $\mc L_b$ has the described weight.
\end{proof}

\end{document}