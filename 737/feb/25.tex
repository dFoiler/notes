% !TEX root = ../notes.tex

\documentclass[../notes.tex]{subfiles}

\begin{document}

\section{February 25}
Today we will start moving towards a proof of \Cref{thm:weil-to-et-sheaf}, but this will require some preparation.

\subsection{}
Our first piece of preparation is the following result.
\begin{theorem}[Grothendieck] \label{thm:weil-char-torsion-finite-geo-mono}
	Fix a normal geometrically irreducible scheme $X$ over $\FF_q$, and choose a geometric point $\ov x\into X$. Further, fix a continuous character $\chi\colon W(X,\ov x)\to\ov\QQ_\ell^\times$. Then $\chi\left(\pi_1^{\mathrm{\acute et}}(X_{\ov\FF_q},\ov x)\right)$ is finite.
\end{theorem}
Roughly speaking, the above result will follow from geometric class field theory because we are looking at some degree-$0$ Picard group. We remark that this will enable us to prove \Cref{cor:get-weil-is-twist} because one can use the character given above to define $\mc F_0$ and then twist by $\mc L_b$ in order to fix the action of the Frobenius (which is not seen in the geometric \'etale fundamental group).
\begin{remark} \label{rem:power-of-weil-char}
	Equivalently, one can view \Cref{thm:weil-char-torsion-finite-geo-mono} as saying that $\chi^m$ is trivial on $\pi_1^{\mathrm{\acute et}}(X_{\ov\FF_q},\ov x)$ for sufficiently divisible $m$. As such, one may (without loss of generality) pass from $X$ to some base-change and pass from $\chi$ to some finite power. Also, this point of view allows us to use \Cref{thm:weil-char-torsion-finite-geo-mono} to factor $\chi$ into two characters $\chi_1\chi_2$ where $\chi_1$ has finite order and $\chi$ factors through the quotient $W(X,\ov x)\onto W(\ov\FF_q/\FF_q)$.
\end{remark}
To visualize the quotient in the remark, we draw the morphism
% https://q.uiver.app/#q=WzAsMTAsWzAsMCwiMSJdLFsxLDAsIlxccGlfMV57XFxtYXRocm17XFxhY3V0ZSBldH19KFhfe1xcb3ZcXEZGX3F9LFxcb3YgeCkiXSxbMSwxLCJcXHBpXzFee1xcbWF0aHJte1xcYWN1dGUgZXR9fShYX3tcXG92XFxGRl9xfSxcXG92IHgpIl0sWzAsMSwiMSJdLFsyLDEsIlcoWCxcXG92IHgpIl0sWzIsMCwiXFxwaV8xXntcXG1hdGhybXtcXGFjdXRlIGV0fX0oWCxcXG92IHgpIl0sWzMsMCwiXFxvcHtHYWx9KFxcb3ZcXEZGX3EvXFxGRl9xKSJdLFszLDEsIlcoXFxvdlxcRkZfcS9cXEZGX3EpIl0sWzQsMCwiMSJdLFs0LDEsIjEiXSxbMCwxXSxbMSw1XSxbNSw2XSxbNiw4XSxbMywyXSxbMiw0XSxbNCw3XSxbNyw5XSxbNCw1LCJcXHN1YnNldGVxIiwzLHsic3R5bGUiOnsiYm9keSI6eyJuYW1lIjoibm9uZSJ9LCJoZWFkIjp7Im5hbWUiOiJub25lIn19fV0sWzcsNiwiXFxzdWJzZXRlcSIsMyx7InN0eWxlIjp7ImJvZHkiOnsibmFtZSI6Im5vbmUifSwiaGVhZCI6eyJuYW1lIjoibm9uZSJ9fX1dLFsyLDEsIiIsMSx7ImxldmVsIjoyLCJzdHlsZSI6eyJoZWFkIjp7Im5hbWUiOiJub25lIn19fV1d&macro_url=https%3A%2F%2Fraw.githubusercontent.com%2FdFoiler%2Fnotes%2Fmaster%2Fnir.tex
\[\begin{tikzcd}
	1 & {\pi_1^{\mathrm{\acute et}}(X_{\ov\FF_q},\ov x)} & {\pi_1^{\mathrm{\acute et}}(X,\ov x)} & {\op{Gal}(\ov\FF_q/\FF_q)} & 1 \\
	1 & {\pi_1^{\mathrm{\acute et}}(X_{\ov\FF_q},\ov x)} & {W(X,\ov x)} & {W(\ov\FF_q/\FF_q)} & 1
	\arrow[from=1-1, to=1-2]
	\arrow[from=1-2, to=1-3]
	\arrow[from=1-3, to=1-4]
	\arrow[from=1-4, to=1-5]
	\arrow[from=2-1, to=2-2]
	\arrow[equals, from=2-2, to=1-2]
	\arrow[from=2-2, to=2-3]
	\arrow["\subseteq"{marking, allow upside down}, draw=none, from=2-3, to=1-3]
	\arrow[from=2-3, to=2-4]
	\arrow["\subseteq"{marking, allow upside down}, draw=none, from=2-4, to=1-4]
	\arrow[from=2-4, to=2-5]
\end{tikzcd}\]
of short exact sequences, which allows us to see what it means for the aforementioned second character $\chi_2$ to factor through $W(\ov\FF_q/\FF_q)\cong\ZZ$.
\begin{remark}
	Intuitively, the geometric \'etale fundamental group $\pi_1^{\mathrm{\acute et}}(X_{\ov\FF_q},\ov x)$ remembers topological information, notably forgetting the ``arithmetic'' information coming from having Frobenius elements in $\pi_1^{\mathrm{\acute et}}(X,\ov x)$.
\end{remark}
Let's spend a moment to give a few reductions.
\begin{enumerate}
	\item As discussed in \Cref{rem:power-of-weil-char}, we may base-change $X$, for example to ensure that it has a rational point. Additionally, we may pass from $\chi$ to a power.

	\item Because $\chi$ is continuous, we know that it should factor through $E^\times\subseteq\ov\QQ_\ell^\times$ for some finite extension $E$ of $\QQ_\ell$. By factoring $E^\times$ as a group, we find that by (passing to a power of $\chi$), we may automatically assume that the image $\chi\left(\pi_1^{\mathrm{\acute et}}(X_{\ov\FF_q},\ov x)\right)$ is a pro-$\ell$ group.

	\item One can reduce to the case where $X$ is smooth and quasi-projective. Because $X$ is already normal, any open nonempty subscheme $U\subseteq X$ containing $\ov x$ yields a surjection $\pi_1^{\mathrm{\acute et}}(U,\ov x)\onto\pi_1^{\mathrm{\acute et}}(X,\ov x)$; this allows us to prove the theorem by passing to open subsets of $X$. For example, we now achieve smoothness by replacing $X$ with the (open) smooth locus. By passing to small open subsets, we may even assume that $X$ is quasi-projective (because we are covered by behaved affine open subsets).

	\item We now reduce to the case where $X$ is smooth and projective. This requires a compactification. It is possible to produce a compactification $X'$ of $X$ which is normal and projective; say $X'\subseteq\PP^N$ for some $N\ge0$. Taking the smooth locus of $X\subseteq X'$ grants an open smooth subscheme $Y\subseteq X'$ containing $X$; say $Z\coloneqq X'\setminus Y$, which has codimension at least $2$ by normality of $X'$. Now, Chow's lemma tells us that $D\coloneqq Y\setminus X$ is a smooth locally closed subscheme which is pure of codimension $1$; say $D=D_1\cup\cdots\cup D_r$ is a decomposition into irreducible components, and extending the base field actually allows us to assume that these components are geometrically irreducible and admitting a rational point.

	There are now two cases. If $X=Y$, then we go ahead and assume $\dim X\ge2$ (we will handle curves on their own later), and if $X\subsetneq Y$, then we will do something else.

	\item Once $X$ is known to be smooth and projective, one can slice the ambient projective space by generic hyperplanes. In this way, we will be able to reduce to curves because the generic intersection with well-chosen hyperplanes produce the required curves.
\end{enumerate}
So we take $X$ to be a curve. We use the theory of valuations. Because $X$ is normal and geometrically irreducible, one gets a surjection
\[\op{Gal}(K^{\mathrm{sep}}/K)\onto\pi_1^{\mathrm{\acute et}}(X,\ov x),\]
where $K$ is the function field of $X$; intuitively, the idea is that one can take function fields of any \'etale covers of $X$ to produce separable extensions of $K$.

Now that we are in the curve case, we would like to do something with (geometric) class field theory. Define $I_K$ to fit into the short exact sequence
\[0\to I_K\to W(X,\ov x)^{\mathrm{ab}}\stackrel\deg\to W(\ov\FF_q/\FF_q)\to0.\]
We now have a large diagram as follows.
% https://q.uiver.app/#q=WzAsMTksWzAsMiwiMCJdLFsxLDIsIklfSyJdLFsyLDIsIlcoWCxcXG92IHgpXntcXG1hdGhybXthYn19Il0sWzMsMiwiVyhcXG92XFxGRl9xL1xcRkZfcSkiXSxbNCwyLCIwIl0sWzEsMywiXFxwaV8xXntcXG1hdGhybXtcXGFjdXRlIGV0fX0oXFxvdiBYLFxcb3YgeClee1xcbWF0aHJte2FifX0iXSxbMiwzLCJcXHBpXzFee1xcbWF0aHJte1xcYWN1dGUgZXR9fShYLFxcb3YgeClee1xcbWF0aHJte2FifX0iXSxbMiwxLCJLXlxcdGltZXNcXGJhY2tzbGFzaFxcQUFfS15cXHRpbWVzL1xcd2lkZWhhdFxcT09fS15cXHRpbWVzIl0sWzMsMSwicV5cXFpaIl0sWzQsMSwiMCJdLFsxLDEsIkteXFx0aW1lc1xcYmFja3NsYXNoXFxBQV9LXntcXHRpbWVzLDF9L1xcd2lkZWhhdFxcT09fS15cXHRpbWVzIl0sWzAsMSwiMCJdLFszLDMsIlxcb3B7R2FsfShcXG92XFxGRl9xL1xcRkZfcSkiXSxbNCwzLCIwIl0sWzIsMCwiXFxQaWMoWCkoXFxGRl9xKSJdLFsxLDAsIlxcUGljKFgpXlxcY2lyYyhcXEZGX3EpIl0sWzMsMCwiXFxaWiJdLFs0LDAsIjAiXSxbMCwwLCIwIl0sWzIsMywiXFxkZWciXSxbNyw4LCJcXG9wIE4iXSxbNywyLCJcXHNpbSJdLFsxMSwxMF0sWzEwLDddLFs4LDldLFsxMCwxLCJcXHNpbSJdLFs4LDMsIlxcc2ltIl0sWzAsMV0sWzEsMl0sWzMsNF0sWzUsNl0sWzYsMTJdLFsxMiwxM10sWzMsMTIsIlxcc3Vic2V0ZXEiLDMseyJzdHlsZSI6eyJib2R5Ijp7Im5hbWUiOiJub25lIn0sImhlYWQiOnsibmFtZSI6Im5vbmUifX19XSxbMiw2LCJcXHN1YnNldGVxIiwzLHsic3R5bGUiOnsiYm9keSI6eyJuYW1lIjoibm9uZSJ9LCJoZWFkIjp7Im5hbWUiOiJub25lIn19fV0sWzUsMSwiIiwzLHsic3R5bGUiOnsiaGVhZCI6eyJuYW1lIjoiZXBpIn19fV0sWzE1LDE0LCIiLDMseyJzdHlsZSI6eyJ0YWlsIjp7Im5hbWUiOiJob29rIiwic2lkZSI6InRvcCJ9fX1dLFsxNCwxNiwiXFxkZWciXSxbMTAsMTUsIiIsMCx7ImxldmVsIjoyLCJzdHlsZSI6eyJoZWFkIjp7Im5hbWUiOiJub25lIn19fV0sWzcsMTQsIiIsMCx7ImxldmVsIjoyLCJzdHlsZSI6eyJoZWFkIjp7Im5hbWUiOiJub25lIn19fV0sWzE4LDE1XSxbMTYsMTddLFsxNiw4LCJcXHNpbSJdXQ==&macro_url=https%3A%2F%2Fraw.githubusercontent.com%2FdFoiler%2Fnotes%2Fmaster%2Fnir.tex
\[\begin{tikzcd}
	0 & {\Pic(X)^\circ(\FF_q)} & {\Pic(X)(\FF_q)} & \ZZ & 0 \\
	0 & {K^\times\backslash\AA_K^{\times,1}/\widehat\OO_K^\times} & {K^\times\backslash\AA_K^\times/\widehat\OO_K^\times} & {q^\ZZ} & 0 \\
	0 & {I_K} & {W(X,\ov x)^{\mathrm{ab}}} & {W(\ov\FF_q/\FF_q)} & 0 \\
	& {\pi_1^{\mathrm{\acute et}}(\ov X,\ov x)^{\mathrm{ab}}} & {\pi_1^{\mathrm{\acute et}}(X,\ov x)^{\mathrm{ab}}} & {\op{Gal}(\ov\FF_q/\FF_q)} & 0
	\arrow[from=1-1, to=1-2]
	\arrow[from=1-2, to=1-3]
	\arrow["\deg", from=1-3, to=1-4]
	\arrow[from=1-4, to=1-5]
	\arrow["\sim", from=1-4, to=2-4]
	\arrow[from=2-1, to=2-2]
	\arrow[equals, from=2-2, to=1-2]
	\arrow[from=2-2, to=2-3]
	\arrow["\sim", from=2-2, to=3-2]
	\arrow[equals, from=2-3, to=1-3]
	\arrow["{\op N}", from=2-3, to=2-4]
	\arrow["\sim", from=2-3, to=3-3]
	\arrow[from=2-4, to=2-5]
	\arrow["\sim", from=2-4, to=3-4]
	\arrow[from=3-1, to=3-2]
	\arrow[from=3-2, to=3-3]
	\arrow["\deg", from=3-3, to=3-4]
	\arrow["\subseteq"{marking, allow upside down}, draw=none, from=3-3, to=4-3]
	\arrow[from=3-4, to=3-5]
	\arrow["\subseteq"{marking, allow upside down}, draw=none, from=3-4, to=4-4]
	\arrow[two heads, from=4-2, to=3-2]
	\arrow[from=4-2, to=4-3]
	\arrow[from=4-3, to=4-4]
	\arrow[from=4-4, to=4-5]
\end{tikzcd}\]
The isomorphism of the top two rows is formal algebraic geometry. The isomorphism of the second and third row is essentially geometric class field theory. The maps between the third and fourth rows are more or less formal coming from the various constructions of the various Weil groups.

Now, after extending $X$ so that it admits a rational point, we see that $\Pic(X)^\circ$ is the Jacobian of $X$ and hence projective (it's an abelian variety), so it will only have finitely many $\FF_q$-rational points. We now note that $\chi$ factors through $W(X,\ov x)^{\mathrm{ab}}$ (because it's a character), so the commutative diagram
% https://q.uiver.app/#q=WzAsNSxbMCwwLCJcXHBpXzFee1xcbWF0aHJte1xcYWN1dGUgZXR9fShYX3tcXG92XFxGRl9xfSxcXG92IHgpIl0sWzEsMCwiVyhYLFxcb3YgeCkiXSxbMSwxLCJXKFgsXFxvdiB4KV57XFxtYXRocm17YWJ9fSJdLFswLDEsIlxcUGljKFgpXlxcY2lyYyhcXEZGX3EpIl0sWzIsMCwiXFxvdlxcUVFfXFxlbGxeXFx0aW1lcyJdLFswLDMsIiIsMCx7InN0eWxlIjp7ImhlYWQiOnsibmFtZSI6ImVwaSJ9fX1dLFswLDFdLFszLDJdLFsxLDJdLFsyLDQsIiIsMix7InN0eWxlIjp7ImJvZHkiOnsibmFtZSI6ImRhc2hlZCJ9fX1dLFsxLDQsIlxcY2hpIl1d&macro_url=https%3A%2F%2Fraw.githubusercontent.com%2FdFoiler%2Fnotes%2Fmaster%2Fnir.tex
\[\begin{tikzcd}
	{\pi_1^{\mathrm{\acute et}}(X_{\ov\FF_q},\ov x)} & {W(X,\ov x)} & {\ov\QQ_\ell^\times} \\
	{\Pic(X)^\circ(\FF_q)} & {W(X,\ov x)^{\mathrm{ab}}}
	\arrow[from=1-1, to=1-2]
	\arrow[two heads, from=1-1, to=2-1]
	\arrow["\chi", from=1-2, to=1-3]
	\arrow[from=1-2, to=2-2]
	\arrow[from=2-1, to=2-2]
	\arrow[dashed, from=2-2, to=1-3]
\end{tikzcd}\]
allows us to use the finiteness of $\Pic(X)^\circ$ to conclude the proof.

\end{document}