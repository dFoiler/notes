% !TEX root = ../notes.tex

\documentclass[../notes.tex]{subfiles}

\begin{document}

\section{February 13}

Here we go.

\subsection{Weil Sheaves}
We would like to generalize our sheaves somewhat. Note that we had an action of the geometric Frobenius element on $X_{\ov k}$ from the start. In particular, our \'etale sheaves have a full action by $\pi_1(X,\ov x)$, but we really only need to know about Frobenius action.
\begin{definition}[Weil group]
	Fix a finite field $k\coloneqq\FF_q$. The \textit{Weil group} $W(\ov k/k)$ is the cyclic group generated by the Frobenius.
\end{definition}
Now, note $W(\ov k/k)$ acts on $X_{\ov k}=X\times_k\ov k$, so any \'etale sheaf $\mc G$ on $X$ gains an action of $W(\ov k/k)$ after base-changing to $X_{\ov k}$. Because this sheaf comes from $X$, it should be $W(\ov k/k)$-equivariant, which amounts to providing the data of an isomorphism $F_{X_{\ov k}}^*\mc G\to\mc G$. We are now ready to make the following definition, which generalizes our \'etale sheaves.
\begin{definition}[Weil sheaf]
	A \textit{Weil sheaf} $\mc G_0$ on $X$ consists is a $\ov\QQ_\ell$-sheaf $\mc G$ on $X_{\ov k}$ together with an isomorphism $F^*\colon F^*_{X_{\ov k}}\mc G\to\mc G$. This Weil sheaf $\mc G_0$ is \textit{smooth of rank $r$} if and only if the same is true of $\mc G$ on $X_{\ov k}$.
\end{definition}
Notably, this Weil sheaf is not required to actually come from a sheaf on $X$! (This is the whole point!) This extra flexibility of not having to actually come from a scheme over $\FF_q$ will be helpful for us later.

Let's give a few remarks about the category of Weil sheaves.
\begin{remark}
	The category of Weil sheaves on $X$ form an abelian category. This category contains the category of \'etale sheaves on $X$, embedded as described above.
\end{remark}
\begin{remark}
	If $k'\subseteq\ov k$ is a finite extension of $k$, then the category of Weil sheaves on $X_{k'}$ is the same as the category of Weil sheaves on $X$. Namely, restriction of scalars from $X_{k'}$ to $X$ defines the required equivalence.
\end{remark}
\begin{remark}
	The category of Weil sheaves has the six operations (e.g., pullbacks, derived direct images, and direct image with compact support). The point is that one can do these operations on $X_{\ov k}$, and then one just needs to carry around the extra data of the isomorphism $F^*$.
\end{remark}
We now remark that a Weil sheaf $\mc G_0$ on $X$ is still going to come with an isomorphism $F\colon\mathrm H^i_c(X_{\ov k},\mc G)\to\mathrm H^i_c(X_{\ov k},\mc G)$, so one can define $L(X,\mc G;T)$ as before. There is still a cohomological interpretation, but we will omit its proof.

These Weil sheaves appear to form a Tannakian category, so we are allowed to ask for the corresponding reductive group. To see this, we recall that the short exact sequence
\[1\to\pi_1^{\mathrm{\acute et}}(X_{\ov k},\ov x)\to\pi_1^{\mathrm{\acute et}}(X,\ov x)\to\op{Gal}(\ov k/k)\to1.\]
The inverse image of $W(\ov k/k)\subseteq\op{Gal}(\ov k/k)$ is some Weil group $W(X,\ov x)\subseteq\pi_1^{\mathrm{\acute et}}(X,\ov x)$. It turns out that the induced quotient $W(X,\ov x)/\pi_1(X_{\ov k},\ov x)$ is $W(\ov k/k)\cong\ZZ$. We are now ready to compare these Tannakian categories.
\begin{itemize}
	\item Recall that the category of \'etale sheaves is equivalent to the category of continuous representations of $\pi_1^{\mathrm{\acute et}}(X,\ov x)$ on vector spaces over $\ov\QQ_\ell$.
	\item This then restricts to an equivalence between the category of smooth \'etale sheaves and the category of finite-dimensional continuous representations of $\pi_1^{\mathrm{\acute et}}(X,\ov x)$.
	\item However, the category of Weil sheaves can be seen as the category of continuous representations of $W(X,\ov x)$ of $\ov\QQ_\ell$-vector spaces. To see this, we note that we are essentially removing some data from being a sheaf over $\pi_1^{\mathrm{\acute et}}(X,\ov x)$.

	The adjective of smoothness adds a finite-dimensional requirement.
\end{itemize}
\begin{remark}
	This Tannakian point of view explains one point of working with Weil sheaves instead of \'etale sheaves. \'Etale sheaves, viewed as representations of the (compact!) profinite fundamental group, are forced to have bounded eigenvalues when Frobenius acts continuously on a vector space. Weil sheaves do not have this requirement, so we are granted more flexibility.
\end{remark}

\end{document}