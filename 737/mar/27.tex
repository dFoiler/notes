% !TEX root = ../notes.tex

\documentclass[../notes.tex]{subfiles}

\begin{document}

\section{March 27}
I missed last class because I was sick. We completed the proof of \Cref{thm:weil-to-et-sheaf}.

\subsection{\texorpdfstring{$L$}{ L}-Functions of Weil Sheaves}
Today we apply \Cref{thm:weil-to-et-sheaf}. Throughout, $X$ continues to be a normal and geometrically connected scheme over $\FF_q$, and $\mc G_0$ is a Weil sheaf on $X$. If $\mc G_0$ is further irreducible, we now know that there is a decomposition $\mc G_0\cong\mc F_0\otimes\mc L_b$, where $\mc F_0$ is \'etale and $\mc L_b$ is a line bundle. The moral is that we can adjust Weil sheaves (which are representations of the merely locally compact Weil group) to \'etale sheaves (which are representations of the compact \'etale fundamental group) at little cost. The ability to work with compact groups is beneficial from the point of view of representation theory, for example.

As another application, we get a trace formula.
\begin{corollary} \label{cor:weil-l-func-rational}
	Fix a smooth Weil sheaf $\mc G_0$ on a normal and geometrically scheme $X$ over $\FF_q$. Then
	\[L(X,\mc G_0,t)\coloneqq\prod_{\text{closed }x\in X}\det\left(1-t^{\deg x}F_x^*;\mc G_{\ov x}\right)^{-1}\]
	is a rational function in $\QQ(t)$, admitting a decomposition
	\[L(X,\mc G_0,t)=\prod_{i=0}^{2\dim X}\det\left(1-tF^*;\mathrm H^i_c(X_{\ov\FF_q},\mc G)\right)^{(-1)^{i+1}}.\]
\end{corollary}
\begin{proof}
	By using filtrations (for which we note that the determinants and cohomology are suitable multiplicative), we may reduce to the case where $\mc G_0$ is irreducible. We use the decomposition $\mc G_0\cong\mc F_0\otimes\mc L_b$ granted by \Cref{cor:get-weil-is-twist}. Then
	\[\det\left(1-t^{\deg x}F_x^*;\mc G_{\ov x}\right)=\det\left(1-(tb)^{\deg x}F_x^*;\mc F_{\ov x}\right),\]
	and
	\[\det\left(1-tF^*;\mathrm H^i_c(X_{\ov\FF_q},\mc G)\right)=\det\left(1-(tb)F^*;\mathrm H^i_c(X_{\ov\FF_q},\mc F)\right).\]
	The result now follows by plugging in $(tb)$ into the corresponding trace formula decomposition for $\mc F$.
\end{proof}
For our next application, we use the notion of weight.
\begin{lemma} \label{lem:l-func-converge-by-weight}
	Fix a normal scheme $X$ of finite type over $\FF_q$, and let $\mc G_0$ be a Weil sheaf on $X$, and choose some isomorphism $\iota\colon\ov\QQ_\ell\to\CC$. Suppose $w(\mc G_0)\le\beta$ for some $\beta>0$. Then the $L$-series
	\[\iota L(X,\mc G,t)\coloneqq\prod_{\text{closed }x\in X}\iota\det\left(1-t^{\deg x}F_x^*;\mc G_{\ov x}\right)^{-1}\]
	converges for all $t\in\CC$ with $\left|t\right|<q^{-\beta/2-\dim X}$.
\end{lemma}
\begin{proof}
	By induction on dimension, we may reduce to the case where $X$ is affine and irreducible (by covering $X$ with irreducible or affine open subschemes), and we may assume that $X$ is reduced because reduction does not affect our Frobenius actions. Now, we recall that
	\[\exp\Bigg(\sum_{n=1}^\infty\tr\left(\varphi^n;V\right)\frac{t^n}n\Bigg)=\det\left(1-\varphi t;V\right)^{-1}\]
	for any endomorphism $\varphi$ on a finite-dimensional vector space $V$, so we can compute that the logarithmic derivative of $\iota L$ is given by
	\[\frac{\iota L'(X,\mc G,t)}{\iota L(X,\mc G,t)}=\sum_{\text{closed }x\in X}\Bigg(\sum_{n=1}^\infty\deg(x)\tr\left( F_x^n;\mc G_{\ov x}\right)t^{\deg(x)n}\Bigg).\]
	All sums in sight are countable, and we will find that they absolutely converge with the given $t$, which will notably complete the proof upon undoing the logarithmic differentiation: checking that the infinite product converges is equivalent to checking that its logarithm converges, but the logarithm is some power series in $t$ which must have the same radius of convergence as its derivative. So to check our absolute convergence, we may as well switch the order of the summation, writing
	\[\frac{\iota L'(X,\mc G,t)}{\iota L(X,\mc G,t)}=\sum_{n=1}^\infty\Bigg(\sum_{\substack{\text{closed }x\in X\\\deg(x)\mid n}}\deg(x)\tr\left(F_x^{n/\deg x};\mc G_{\ov x}\right)\Bigg)t^{n-1}.\]
	By the weight hypothesis, the trace is bounded in magnitude by $rq^{n\beta/2}$, where $r$ is the maximum of the dimensions of the fibers $\mc G_{0,\ov x}$. We are left with wanting to show the absolute convergence of
	\[\sum_{n=1}^\infty\#X(\FF_q^n)rq^{n\beta/2}t^{n-1}.\]
	Now, because $X$ is affine and normal of finite type over a field, we may use Noether normalization to bound the point-counts $\#X(\FF_{q^n})$ with $\#\AA^{\dim X}(\FF_{q^n})$ up to some explicit constant. Thus, we are left with the checking the absolute convergence of
	\[\sum_{n=1}^\infty q^{n(\beta/2+\dim X)}t^{n-1},\]
	which is true by hypothesis on $t$.
\end{proof}
We will continue this discussion next class.

\end{document}