% !TEX root = ../notes.tex

\documentclass[../notes.tex]{subfiles}

\begin{document}

\section{March 11}
We began class by finishing the proof of \Cref{thm:semisimple-monodromy}.

\subsection{More on Semisimple Monodromy}
It remains to prove (b) of \Cref{thm:semisimple-monodromy}. We proceed in steps. For brevity, set $Z\coloneqq Z(G(\ov\QQ_\ell))$.
\begin{enumerate}
	\item Note that an appropriate choice of the Frobenius (in degree $1$) as in our proof of (a) tells us that $G(\ov\QQ_\ell)$ splits into the direct product $G_{\mathrm{geom}}(\ov\QQ_\ell)\times\ZZ$. Thus, we have a morphism
	% https://q.uiver.app/#q=WzAsMTAsWzAsMCwiMSJdLFsxLDAsIlxccGlfMV57XFxtYXRocm17XFxhY3V0ZSBldH19KFhfe1xcb3ZcXEZGX3F9LFxcb3YgeCkiXSxbMiwwLCJXKFgsXFxvdiB4KSJdLFszLDAsIlcoXFxvdlxcRkZfcS9cXEZGX3EpIl0sWzQsMCwiMSJdLFswLDEsIjEiXSxbMSwxLCJHX3tcXG1hdGhybXtnZW9tfX0oXFxvdlxcUVFfXFxlbGwpIl0sWzIsMSwiR197XFxtYXRocm17Z2VvbX19KFxcb3ZcXFFRX1xcZWxsKVxcdGltZXNcXFpaIl0sWzMsMSwiVyhcXG92XFxGRl9xL1xcRkZfcSkiXSxbNCwxLCIxIl0sWzAsMV0sWzEsMl0sWzIsM10sWzMsNF0sWzUsNl0sWzYsN10sWzcsOF0sWzgsOV0sWzIsN10sWzEsNl0sWzMsOCwiIiwxLHsibGV2ZWwiOjIsInN0eWxlIjp7ImhlYWQiOnsibmFtZSI6Im5vbmUifX19XV0=&macro_url=https%3A%2F%2Fraw.githubusercontent.com%2FdFoiler%2Fnotes%2Fmaster%2Fnir.tex
	\[\begin{tikzcd}[cramped]
		1 & {\pi_1^{\mathrm{\acute et}}(X_{\ov\FF_q},\ov x)} & {W(X,\ov x)} & {W(\ov\FF_q/\FF_q)} & 1 \\
		1 & {G_{\mathrm{geom}}(\ov\QQ_\ell)} & {G_{\mathrm{geom}}(\ov\QQ_\ell)\times\ZZ} & {W(\ov\FF_q/\FF_q)} & 1
		\arrow[from=1-1, to=1-2]
		\arrow[from=1-2, to=1-3]
		\arrow[from=1-2, to=2-2]
		\arrow[from=1-3, to=1-4]
		\arrow[from=1-3, to=2-3]
		\arrow[from=1-4, to=1-5]
		\arrow[from=2-1, to=2-2]
		\arrow[from=2-2, to=2-3]
		\arrow[from=2-3, to=2-4]
		\arrow[from=2-4, to=2-5]
		\arrow[equals, from=1-4, to=2-4]
	\end{tikzcd}\]
	of short exact sequences, as soon as we pass to a finite extension of $\FF_q$ in the bottom.

	% Let's describe how to construct the required map $\psi\colon Z\to W(\ov\FF_q/\FF_q)$. Note that $W(\ov\FF_q/\FF_q)$ acts on $\pi_1^{\mathrm{\acute et}}(\pi_1(X_{\ov\FF_q}/\ov k))$ by conjugation, as in the above exact sequence. In fact, we may as well take the quotient down to a morphism $W(\ov\FF_q/\FF_q)\to\op{Out}\pi_1^{\mathrm{\acute et}}(X_{\ov\FF_q},\ov x)$.

	The diagram now defines our map $\psi\colon Z\to W(\ov\FF_q/\FF_q)$ (without using the splitting). Note $\ker\psi$ lives in $Z\cap G_{\mathrm{geom}}(\ov\QQ_\ell)$ by exactness, which is finite because $G_{\mathrm{geom}}$ is semisimple.

	\item It remains to show that the cokernel is finite. Note that if we were allowed to pass to a finite extension of $\FF_q$, then the short exact sequence defining $\psi$ would split, so the result would have no content. Thus, the difficulty lives in dealing with the component group, which potentially makes the center smaller. Nonetheless, there is $\zeta\in G(\ov\QQ_\ell)$ of positive degree by using the short exact sequence. In fact, we may even assume that $\zeta$ commutes with $G_{\mathrm{geom}}(\ov\QQ_\ell)$ by using the splitting of the (second) short exact sequence in the previous step. We would like to show that a power of $\zeta$ lives in $Z$, which will complete the proof.

	To detect the center, we define the $1$-cocycle $\varphi_g\colon\ZZ\to G_{\mathrm{geom}}(\ov\QQ_\ell)$
	\[\varphi_g(n)\coloneqq g\zeta g^{-1}\zeta^{-n}.\]
	Because $\zeta$ commutes with $G_{\mathrm{geom}}(\ov\QQ_\ell)$ already, we know that $\varphi_g$ is in fact a homomorphism. Then one can compute that $\im\varphi_g$ lives in $Z(G_{\mathrm{geom}}(\ov\QQ_\ell))$ for all $g$, but this codomain is a finite group, so we can find $n$ such that $\varphi_g(n)=1$ for all $g$, which is the same as asserting that $\zeta^n$ commutes with $G_{\mathrm{geom}}(\ov\QQ_\ell)$. This upgrades to $\zeta^n\in Z$ because $G(\ov\QQ_\ell)$ is generated by some root of $\zeta$ and $G_{\mathrm{geom}}(\ov\QQ_\ell)$, so we are done.
\end{enumerate}
This completes the proof of \Cref{thm:semisimple-monodromy}.
\begin{remark}
	Let's now sketch the idea of \Cref{thm:weil-to-et-sheaf}. The point is that \'etale sheaves should have controlled absolute values because we are looking at representations of the compact group $\pi_1^{\mathrm{\acute et}}(X,\ov x)$. Thus, our goal is to control eigenvalues when we are told that the determinant has controlled eigenvalues.
\end{remark}

\end{document}