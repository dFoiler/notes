% !TEX root = ../notes.tex

\documentclass[../notes.tex]{subfiles}

\begin{document}

\section{March 4}
As an aside, let's examine how the fundamental group changes when one passes to open subsets. Let $E$ be an elliptic curve over $\FF_q$. Any \'etale cover of $E$ continues to be an elliptic curve, which allows one to compute the $\ell$-primary component $\pi_1^{\mathrm{\acute et}}(E_{\ov\FF_q},\ov\eta)_\ell=T_\ell E$. Thus, we have a short exact sequence
\[1\to\pi_1^{\mathrm{\acute et}}(E_{\ov\FF_q},\ov\eta)\to\pi_1^{\mathrm{\acute et}}(E,\ov\eta)\to\op{Gal}(\ov\FF_q/\FF_q)\to1\]
whose $\ell$-primary component reads as
\[0\to\ZZ_\ell^2\to\pi_1^{\mathrm{\acute et}}(E,\ov\eta)\to\ZZ_\ell\to0.\]
But now if we subtract out the point at infinity, we find
\[\pi_1^{\mathrm{\acute et}}(E_{\ov\FF_q}\setminus0,\ov\eta)\]
is basically $\widehat\ZZ\oplus\widehat\ZZ\rtimes\widehat\ZZ$ (again, with maybe some problems at $p$). The moral of the story is that removing a point allows some ramification at the point we removed.
\begin{example}
	Philosophically, one can tell the same story for $\Spec\ZZ$, which we view as a quasiprojective subvariety of some Arakelov projectification. For example, if we remove a ``second'' point to work with $\Spec\ZZ\setminus\{p\}$ as \'etale fundamental group given by
	\[\op{Gal}(\QQ(\zeta_{p^\infty})/\QQ)\cong\ZZ_p^\times.\]
\end{example}

\subsection{Semisimple Monodromy}
The previous theorem is all about image of our monodromy, so we are motivated to make the following definitions.
\begin{definition}[monodromy group]
	Fix a smooth Weil $\ov\QQ_\ell$-sheaf $\mc G_0$ on a normal geometrically irreducible scheme $X$ over $\FF_q$, and let $\rho\colon W(X,\ov x)\to\op{GL}(V)$ be the corresponding representation where $V\coloneqq\mc G_{0,\ov x}$.
	\begin{itemize}
		\item Then the \textit{geometric monodromy group} is the image $\rho\left(\pi_1^{\mathrm{\acute et}}(X_{\ov\FF_q},\ov x)\right)$, and the \textit{arithmetic monodromy group} is $\rho\left(W(X,\ov x)\right)$.
		\item We let $G_{\mathrm{geom}}$ be the Zariski closure of the geometric monodromy group in $\op{GL}(V)_E$, where $E/\QQ_\ell$ is the finite extension coming from $\mc G_0$, and we define
		\[G\coloneqq G_{\mathrm{geom}}\rtimes_\sigma W(\ov\FF_q/\FF_q),\]
		where $\sigma$ is a chosen generator of the Weil group.
	\end{itemize}
\end{definition}
\begin{remark}
	We note that any $g\in W(X,\ov x)$ normalizes $G_{\mathrm{geom}}$ because $\rho(g)$ normalizing a subgroup amounts to an algebraic equation which must cut out some subvariety containing $G_{\mathrm{geom}}$.
\end{remark}
\begin{remark}
	Note $G$ is Zariski-locally a finitely generated group scheme. We also note that $\rho$ induces the following morphism of short exact sequences.
	% https://q.uiver.app/#q=WzAsMTAsWzAsMCwiMSJdLFsxLDAsIlxccGlfMV57XFxtYXRocm17XFxhY3V0ZSBldH19KFhfe1xcb3ZcXEZGX3F9LFxcb3YgeCkiXSxbMiwwLCJXKFgsXFxvdiB4KSJdLFszLDAsIlcoXFxvdlxcRkZfcS9cXEZGX3EpIl0sWzQsMCwiMSJdLFswLDEsIjEiXSxbMSwxLCJHX3tcXG1hdGhybXtnZW9tfX0iXSxbMiwxLCJHIl0sWzMsMSwiVyhcXG92XFxGRl9xL1xcRkZfcSkiXSxbNCwxLCIxIl0sWzAsMV0sWzUsNl0sWzEsNiwiXFxyaG8iXSxbMSwyXSxbNiw3XSxbMiw3LCJcXHJobyJdLFs3LDhdLFsyLDNdLFszLDhdLFs4LDldLFszLDRdXQ==&macro_url=https%3A%2F%2Fraw.githubusercontent.com%2FdFoiler%2Fnotes%2Fmaster%2Fnir.tex
	\[\begin{tikzcd}[cramped]
		1 & {\pi_1^{\mathrm{\acute et}}(X_{\ov\FF_q},\ov x)} & {W(X,\ov x)} & {W(\ov\FF_q/\FF_q)} & 1 \\
		1 & {G_{\mathrm{geom}}} & G & {W(\ov\FF_q/\FF_q)} & 1
		\arrow[from=1-1, to=1-2]
		\arrow[from=1-2, to=1-3]
		\arrow["\rho", from=1-2, to=2-2]
		\arrow[from=1-3, to=1-4]
		\arrow["\rho", from=1-3, to=2-3]
		\arrow[from=1-4, to=1-5]
		\arrow[from=1-4, to=2-4]
		\arrow[from=2-1, to=2-2]
		\arrow[from=2-2, to=2-3]
		\arrow[from=2-3, to=2-4]
		\arrow[from=2-4, to=2-5]
	\end{tikzcd}\]
\end{remark}
It will be helpful to know how the Weil group behaves in extensions.
\begin{lemma} \label{lem:monodromy-extend-scalar}
	One has a morphism of
	% https://q.uiver.app/#q=WzAsMTAsWzAsMCwiMSJdLFsxLDAsIlxccGlfMV57XFxtYXRocm17XFxhY3V0ZSBldH19KFhfe1xcb3ZcXEZGX3txXnJ9fSxcXG92IHgpIl0sWzAsMSwiMSJdLFsxLDEsIlxccGlfMV57XFxtYXRocm17XFxhY3V0ZSBldH19KFhfe1xcb3ZcXEZGX3F9LFxcb3YgeCkiXSxbMiwwLCJXKFhfe1xcRkZfe3Fecn19LFxcb3YgeCkiXSxbMiwxLCJXKFgsXFxvdiB4KSJdLFszLDAsIlcoXFxvdlxcRkZfe3Fecn0vXFxGRl97cV5yfSkiXSxbMywxLCJXKFxcb3ZcXEZGX3txfS9cXEZGX3txfSkiXSxbNCwwLCIxIl0sWzQsMSwiMSJdLFswLDFdLFsxLDRdLFs0LDZdLFs2LDhdLFsyLDNdLFszLDVdLFs1LDddLFs3LDldLFsxLDNdLFs0LDVdLFs2LDddXQ==&macro_url=https%3A%2F%2Fraw.githubusercontent.com%2FdFoiler%2Fnotes%2Fmaster%2Fnir.tex
	\[\begin{tikzcd}[cramped]
		1 & {\pi_1^{\mathrm{\acute et}}(X_{\ov\FF_{q^r}},\ov x)} & {W(X_{\FF_{q^r}},\ov x)} & {W(\ov\FF_{q^r}/\FF_{q^r})} & 1 \\
		1 & {\pi_1^{\mathrm{\acute et}}(X_{\ov\FF_q},\ov x)} & {W(X,\ov x)} & {W(\ov\FF_{q}/\FF_{q})} & 1
		\arrow[from=1-1, to=1-2]
		\arrow[from=1-2, to=1-3]
		\arrow[from=1-2, to=2-2]
		\arrow[from=1-3, to=1-4]
		\arrow[from=1-3, to=2-3]
		\arrow[from=1-4, to=1-5]
		\arrow[from=1-4, to=2-4]
		\arrow[from=2-1, to=2-2]
		\arrow[from=2-2, to=2-3]
		\arrow[from=2-3, to=2-4]
		\arrow[from=2-4, to=2-5]
	\end{tikzcd}\]
	short exact sequences in which the right square is a pullback square.
\end{lemma}
\begin{remark}
	Identifying $W(\ov\FF_q/\FF_q)\cong\ZZ$ in the usual way (with the Frobenius as the generator $1\in\ZZ$), we find that the inclusion $W(\ov\FF_q/\FF_{q^r})\subseteq W(\ov\FF_q/\FF_q)$ is a subgroup of index $m$ and must be identified with $m\ZZ$.
\end{remark}
\begin{lemma} \label{lem:monodromy-extend-cover}
	Fix a connected \'etale cover $\pi\colon X'\to X$ of degree $n$ with geometric basepoints $\pi(\ov x')=\ov x$. Then $W(X',\ov x')$ is canonically isomorphic to a subgroup of index $n$ in $\pi_1(X_{\ov\FF_q},\ov x)\cap W(X',\ov x')$.
\end{lemma}
We omit the proofs of the previous two lemmas.

We are now ready to state our theorem.
\begin{theorem} \label{thm:semisimple-monodromy}
	Fix a semisimple smooth Weil $\ov\QQ_\ell$-sheaf $\mc G_0$ on a normal geometrically irreducible sch\-eme $X$ over $\FF_q$.
	\begin{listalph}
		\item The group $G_{\mathrm{geom}}$ and $G_{\mathrm{geom}}^\circ$ are semisimple groups.
		\item The induced map $Z(G(\ov\QQ_\ell))\to W(\ov\FF_q/\FF_q)$ has finite kernel and cokernel.
	\end{listalph}
\end{theorem}
\begin{remark}
	Let's explain why we might want $\mc G_0$ to be semisimple. Let $\mc G_0$ be a semisimple smooth Weil $\ov\QQ_\ell$-sheaf. Then the representation $\rho$ of $W(X,\ov x)$ and its restriction to $\pi_1(X_{\ov\FF_q},\ov x)$ is semisimple, so the representation of $G_{\mathrm{geom}}$ on $W$ is semisimple. Thus, if this representation is faithful (which is the case in our context), it follows that $G_{\mathrm{geom}}$ is reductive.
\end{remark}
\begin{remark}
	More concretely, (b) is asserting that the map $Z(G(\ov\QQ_\ell))\to W(\ov\FF_q/\FF_q)$ is surjective with finite kernel after applying a finite extension of $\FF_q$. The moral is that the center already knows everything about the Frobenius action.
\end{remark}

\end{document}