% !TEX root = ../notes.tex

\documentclass[../notes.tex]{subfiles}

\begin{document}

\section{March 6}
Today, we prove \Cref{thm:semisimple-monodromy}.

\subsection{Proving Semisimple Monodromy}
We sketch the steps of the proof of (a). The idea is to reduce the proof to tori, whose representation theory is understood.
\begin{enumerate}
	\item We begin by reducing to the case where $G_{\mathrm{geom}}$ is connected. This is done by replacing $\FF_q$ with a finite extension and $X$ with a finite \'etale cover. The moral is that the pre-image of $G_{\mathrm{geom}}^\circ$ through the continuous Galois representation must give some finite-index subgroup of $W(X,\ov x)$, and such a finite-index subgroup can be alternatively achieved by the aforementioned operations by \Cref{lem:monodromy-extend-scalar,lem:monodromy-extend-cover}.

	\item By passing to another finite extension, we may assume that the representation $\rho\colon W(X,\ov x)\to\op{GL}(V)$ attached to $\mc G_0$ is faithful. Thus, $G_{\mathrm{geom}}$ is a connected reductive group; let $T\subseteq G_{\mathrm{geom}}$ be a maximal central torus. We will study the restriction of our representation to $T$.

	Choose $\sigma\in W(X,\ov x)$ of degree $1$, which acts by conjugation on $G_{\mathrm{geom}}$ and in particular fixing $G_{\mathrm{geom}}$: indeed, this is true for elements of $\pi_1^{\mathrm{\acute et}}(X_{\ov\FF_q},\ov x)$. Thus, the same is true for the $T\subseteq G_{\mathrm{geom}}$.

	Eventually we are going to need to understand automorphisms of $G_{\mathrm{geom}}$, so we recall that $G_{\mathrm{geom},\ov\QQ_\ell}$ has only finitely many outer automorphisms. Indeed, $G_{\mathrm{geom}}$ is reductive such things are parameterized by automorphisms of the finite Dynkin diagram. This finiteness will be eventually be dealt with by the same sort of operations done in \Cref{lem:monodromy-extend-scalar,lem:monodromy-extend-cover}.

	\item We apply produce some finite permutation of characters of $T$. Note that the action of $\rho(\sigma)$ permutes the finitely many characters which make a basis of $X^*(T)$. In particular, it permutes the finitely many characters which appear in the action of $T$ on $V$, which we write as
	\[V=\bigoplus_{\chi\in X^*(T)}V^\chi.\]
	(Because $V$ is finite-dimensional, only finitely many $V^\chi$ are allowed to be nonzero.) So by replacing $\FF_q$ with a finite extension we may assume that $\rho(\sigma)$ acts trivially.

	\item We now get rid of the ambient outer automorphism. Because $\rho(\sigma)$ is fixing $T$, we use the finiteness of the outer automorphism group to know that some power of $\rho(\sigma)$ acts by inner automorphism on $T$. Namely, we are granted some $g\in G_{\mathrm{geom}}(\ov\QQ_\ell)$ such that
	\[\rho(\sigma)h\rho(\sigma)^{-1}=g^{-1}hg\]
	for any $h$; here we have replaced $\sigma$ with a power implicitly.

	We now pass to the Weil group of a larger field to force $g$ to behave. Note now that $g\rho(\sigma)$ commutes with $G_{\mathrm{geom}}(\ov\QQ_\ell)$ by construction, so $g\rho(\sigma)\in Z(G_{\mathrm{geom}}(\ov\QQ_\ell))$.

	\item We now write
	\[G(\ov\QQ_\ell)=\bigcup_{j\in\ZZ}G_{\mathrm{geom}}(\ov\QQ_\ell)(g\rho(\sigma))^j\]
	to permit a representation $\varphi\colon W(X,\ov x)\to G_{\mathrm{geom}}(\ov\QQ_\ell)$ given by the inclusion $W(X,\ov x)\to G(\ov\QQ_\ell)$ followed by the projection down to $G_{\mathrm{geom}}(\ov\QQ_\ell)$.

	\item We are now ready to argue that $G_{\mathrm{geom}}$ is semisimple. This will require some geometric input. Failure to be semisimple produces a nontrivial central character $\chi\colon G_{\mathrm{geom}}\to\mathbb G_m$ which in particular must have dense image. However, \Cref{thm:weil-char-torsion-finite-geo-mono} tells us that the restriction to $\pi_1(X_{\ov\FF_q},\ov x)$ has finite image, so no such algebraic character $\chi$ with dense image may exist. Semisimplicity follows.
\end{enumerate}

\end{document}