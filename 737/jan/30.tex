% !TEX root = ../notes.tex

\documentclass[../notes.tex]{subfiles}

\begin{document}

\section{January 30}

Today we review some \'etale cohomology.

\subsection{The \'Etale Site}
For completeness, here is the definition of \'etale.
\begin{defihelper}[\'etale] \nirindex{etale@\'etale}
	Fix a scheme morphism $\varphi\colon X\to Y$.
	\begin{listalph}
		\item $\varphi$ is \textit{locally of finite presentation} if and only if $\OO_X$ is finitely presented as a $\varphi^{-1}\OO_Y$-module (say, on Zariski open neighborhoods or on stalks).
		\item $\varphi$ is \textit{flat} if and only if the pushforward $\varphi_*\OO_X$ is flat over $\OO_Y$ (say, on Zariski open neighborhoods or on stalks).
		\item $\varphi$ is \textit{unramified} if and only if $\Omega_{X/Y}=0$.
		\item $\varphi$ is \textit{\'etale} if and only if it is locally of finite presentation, flat, and unramified.
	\end{listalph}
\end{defihelper}
\begin{example}
	One can check that open embeddings are \'etale.
\end{example}
\begin{remark}
	Note that the unramified condition adds some separability, which is a rough explanation for where Galois representations enter the story.
\end{remark}
And here is the relevant site.
\begin{defihelper}[\'etale site] \nirindex{etale site@\'etale site}
	Given a scheme $S$, the \textit{\'etale site} $\mathrm{\acute Et}_S$ is the category of \'etale morphisms to $S$. This site comes with a notion of covering: a collection of morphisms $\{U_i\to U\}$ in $\mathrm{\acute Et}_S$ is a \textit{covering} if and only if the whole covering is surjective on the underlying topological spaces.
\end{defihelper}
\begin{remark}
	Technically, we have defined the ``small'' \'etale topos.
\end{remark}
An advantage of working with \'etale cohomology is that our points gain automorphism groups arising from Galois information.
\begin{definition}[geometric point]
	Fix a scheme $S$. A \textit{geometric point} $\ov x\into S$ is a morphism of schemes from an algebraically closed field; abusing notation, we may write $\ov x$ as $\op{Spec}K$ or as the morphism $\ov x\colon\op{Spec}K\to S$.
\end{definition}
\begin{remark}
	We do not require that our geometric points have closed image in $S$.
\end{remark}
\begin{remark}
	Requiring that we have a morphism of schemes amounts to requiring that the algebraic\-ally closed field $K$ contains the residue field of the image $x\in S$ of $\ov x$. In other words, the data of the morphism $\ov x\colon\op{Spec}K\into S$ amounts to the choice of a point $x\in S$ and an embedding $\kappa(x)\into K$.
\end{remark}
\begin{defihelper}[\'etale neighborhood] \nirindex{etale neighborhood@\'etale neighborhood}
	Fix a scheme $S$. Then an \textit{\'etale neighborhood} $(U,\ov u)$ of a geometric point $\ov x\into S$ is an \'etale morphism $\pi\colon U\to S$ equipped with a geometric point $\ov u\into U$ together with an embedding $\ov x\into U$ over $S$. A morphism of \'etale neighborhoods is a morphism of \'etale covers of $S$ preserving the basepoint.
\end{defihelper}

\subsection{Sheaves on the \'Etale Site}
With a site, one wants sheaves.
\begin{definition}[sheaf]
	Fix a small category $\mc C$ and a scheme $S$. An \textit{\'etale presheaf} $\mc F$ of $\mc C$ on $S$ is a contravariant functor $\mathrm{\acute Et}_S\opp\to\mc C$. An \textit{\'etale sheaf} is a presheaf $\mc F$ such that any $U\in\mathrm{\acute Et}_S$ equipped with a covering $\{U_i\to U\}$ makes $\mc F(U)$ equal the equalizer
	\[\mc F(U)=\op{eq}\Bigg(\prod_i\mc F(U_i)\rightrightarrows\prod_{i,j}\mc F(U_i\times_U U_j)\Bigg).\]
\end{definition}
\begin{remark}
	As in the Zariski topology, there is a sheafification functor $(-)^{\mathrm{sh}}\colon\mathrm{PSh}(S)\to\mathrm{Sh}(S)$ sending the category of presheaves to sheaves. It is a left adjoint to the forgetful functor. The construction of this functor is rather technical, so we will only mention the key property that the sheafification functor is an isomorphism on stalks. For example, one can use sheafification to show that the category $\mathrm{Sh}(S)$ is abelian.
\end{remark}
With sheaves, one has stalks.
\begin{definition}
	Fix a scheme $S$ and a geometric point $\ov x\into S$. For a presheaf $\mc F$ on $S$, we define the \textit{stalk} $\mc F_{\ov x}$ as
	\[\mc F_{\ov x}\coloneqq\colimit_{(U,\ov u)}\mc F(U),\]
	where the direct limit is taken over \'etale neighborhoods of $\ov x$.
\end{definition}
% \begin{example}
% 	For $\mc F=\OO_S$, one has
% 	\[\OO_{S,\ov x}=\colimit_{(U,\ov u}\Gamma(U,\OO_U).\]
% 	It turns out that this is a strictly Henselian ring.
% \end{example}
Let's give some examples.
\begin{proposition}
	Fix a scheme $S$. For any \'etale scheme $V$ over $S$, the presheaf $\underline V$ given by $\underline V(U)\coloneqq\op{Hom}_S(U,V)$ is an \'etale sheaf.
\end{proposition}
\begin{proof}
	This follows from some descent argument.
\end{proof}
\begin{example}
	Using the identity map $S\to S$ reveals that $\OO_S$ is an \'etale sheaf. The stalk is
	\[\OO_{S,\ov x}=\colimit_{(U,\ov u}\Gamma(U,\OO_U).\]
	It turns out that this is a strictly Henselian ring.
\end{example}
\begin{example}
	Fix a positive integer $n\ge1$. Define the $S$-scheme $\mu_n$ as $\Spec\ZZ[T]/\left(T^n-1\right)\times_{\Spec\ZZ}S$. If the multiplication map $n\colon\OO_S\to\OO_S$ is an isomorphism, then $\mu_n$ is \'etale over $S$, so this is an \'etale sheaf.
\end{example}
\begin{example}
	For a finite set $\Sigma$, we may define the $S$-scheme $\Sigma$ given by $\Sigma\times S$ (namely, a disjoint union of $\Sigma$-many copies of $S$). This then produces an \'etale sheaf $\underline\Sigma$.
\end{example}
It is too hard to work with all sheaves. Roughly speaking, we will be interested in ``local systems.'' Here is the version of this notion in algebraic geometry.
\begin{defihelper}[locally constant, constructible] \nirindex{locally constant} \nirindex{constructible}
	Fix an \'etale sheaf $\mc F$ on a scheme $S$ and valued in a category $\mc C$.
	\begin{listalph}
		\item $\mc F$ is \textit{locally constant} if and only if there is a finite \'etale covering $\{U_i\to S\}$ such that $\mc F|_{U_i}$ is isomorphic to a constant sheaf (still valued in $\mc C$).
		\item $\mc F$ is \textit{constructible} if and only if there is a finite stratification $\{S_i\}$ of $S$ into locally closed subsets such that $\mc F|_{S_i}$ is a locally constant sheaf of finite type.
	\end{listalph}
\end{defihelper}
\begin{remark}
	The notion of ``finite type'' changes depending on $\mc C$. For example, if $\mc C$ is the category of abelian groups, then one wants to consider finite abelian groups. If $\mc C$ is a category of vector spaces, then one wants to consider finite-dimensional vector spaces.
\end{remark}
\begin{example}
	The constant sheaf $\underline A$ of an abelian group $A$ is a locally constant constructible sheaf.
\end{example}
\begin{example}
	If $S$ is a variety over $\CC$, and $\pi\colon X\to S$ is an \'etale covering, then the pushforward $\pi_*\underline{\ZZ}$ is locally constant and constructible.
\end{example}
\begin{example}
	If $\pi\colon X\to S$ is an \'etale covering of schemes, then the sheaves $R^i\pi_*\ZZ_\ell$ (suitably interpreted) is locally constant and constructible.
\end{example}
\begin{remark}
	As in topology, it turns out that one can think about locally constant constructible \'etale sheaves are representations of a fundamental group $\pi_1^{\mathrm{\acute et}}(S,\ov x)$.
\end{remark}

\end{document}