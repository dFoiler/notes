% !TEX root = ../notes.tex

\documentclass[../notes.tex]{subfiles}

\begin{document}

\section{April 8}
Today we continue talking about weights of Weil sheaves.

\subsection{Weights of Weil Sheaves on Curves}
For curves, one can say quite a bit about weights.
\begin{definition}
	Fix an $\iota$-mixed Weil sheaf $\mc G_0$ on a one-dimensional scheme $X$ of finite type over $\FF_q$. Then there is some open dense subset $j\colon U\subseteq X$ such that $j^*\mc G_0$ is smooth on $U$. Then we define
	\[w_{\mathrm{gen}}(\mc G_0)\coloneqq w(j^*\mc G_0).\]
\end{definition}
\begin{theorem}
	Fix an $\iota$-mixed Weil sheaf $\mc G_0$ on a scheme $X$ of finite type over $\FF_q$. Assume $\dim X\le1$, and let $j\colon U\subseteq X$ be an open subscheme consisting of all irreducible components of $X$ of dimension equal to $\dim X$.
	\begin{listalph}
		\item We have $\norm{\mc G_0}=\max\{w_{\mathrm{gen}}(j^*\mc G_0),w(\mc G_0)-\dim X\}$. Here, $w_{\mathrm{gen}}(j^*\mc G_0)$ is defined as follows.
		\item If $X$ is a smooth curve with $\mathrm H^0_E(X,\mc G)=0$ for all closed subsets $E\subseteq X$, then $\norm{\mc G_0}=w(\mc G_0)$.
	\end{listalph}
\end{theorem}
\begin{remark}
	The $-1$ in (a) has to do with the fact that $X$ may have disconnected points.
\end{remark}
\begin{proof}
	Note quickly that (a) implies (b). Indeed, in (b), $X$ is smooth and in particular pure of dimension $1$, so $X=U$. Then (a) allows us to compute
	\[\norm{\mc G_0}=\max(w_{\mathrm{gen}}(j^*\mc G_0),w(\mc G_0)-1),\]
	and now $w_{\mathrm{gen}}(j^*\mc G_0)=w(\mc G_0)$ by \Cref{lem:semicontinuity-curves} (and some semicontinuity), so the result follows.

	We now focus on (a). We proceed in steps.
	\begin{enumerate}
		\item We make some easy reductions. We may pass to the reduction of $X$ with no harm, so we may assume that $X$ is (geometrically) reduced. Further, we note that we may assume that $X$ is connected. Indeed, viewing $X$ as the disjoint union of some connected components, we see that
		\[\varphi^{\mc G_0}(t)=\sum_{m=1}^\infty\norm{f^{\mc G_0}}_m^2t^{m-1}\]
		can be decomposed into a finite sum over the connected components, so its radius of convergence is simply the maximum of the radii of convergence of the various sums given by each individual connected component. So having (a) for each connected component produces the result for the union $X$.

		\item We quickly handle the case where $X$ is a point. This is not so hard, but it does require us to unravel all the definitions involved. Because $X$ is reduced and connected, it has a unique closed point $s$. The embedding $\iota\colon\ov\QQ_\ell\to\CC$ allows us to view $V\coloneqq\mc G_{0,\ov s}$ as a vector space over $\CC$, so as soon as we provide $V$ with a basis, we may view the Frobenius $F_s\colon V\to V$ as given by a matrix $A$. We now recall that the logarithmic derivative of $\det\left(1-A\otimes\ov A\cdot t^{\deg s}\right)^{-1}$ is
		\[\sum_{n=1}^\infty\deg(s)\tr\left((A\otimes\ov A)^n\right)t^{n\deg s-1}=\sum_{n=1}^\infty\deg(s)\left|\tr A^n\right|^2t^{n\deg(s)-1}.\]
		This is now seen to be $\varphi^{\mc G_0}(t)$, so we would like to compute the radius of convergence. We already have an interpretation in terms of $\det$, so we are basically looking for the smallest pole, so our radius of convergence comes out to be
		\[\min_{\alpha,\beta}\left|\iota(\alpha)\iota(\beta)\right|^{-1/\deg s}=\min_{\alpha}\left|\iota(\alpha)\right|^{-2/\deg(s)},\]
		where $\alpha$ and $\beta$ vary over eigenvalues of $F_s$. This can be computed to be $q^{-w(\mc G_0)}$, which unravels into showing $\norm{\mc G_0}=w(\mc G_0)$, as required.

		\item Next, we handle the case where $X$ is smooth and $\iota$-pure on a smooth affine curve $X$. By a similar argument to the one in the first step, we may assume that $A$ is geometrically irreducible (by passing to a large extension and dividing up the sum). Note that being able to write down $A\otimes\ov A$ was important to the previous computation, so we aim to construct such an object here as well. Set $\beta\coloneqq w(\mc G_0)$ to be the weight of $\mc G_0$, and then we define
		\[\ov{\mc G}_0\coloneqq\mc G_0^\lor\otimes\mc L_{\iota^{-1}q^\beta}\]
		so that $w(\ov{\mc G}_0)=\beta$ as well. Now, because $X$ is affine, the Grothendieck--Lefschetz trace formula yields
		\[\iota L(X,\mc G_0\otimes\ov{\mc G_0}_0;t)=\frac{\iota\det\left(1-Ft;\mathrm H^1_c(X_{\ov{\FF}_q}(\mc G\otimes\ov{\mc G}_0))\right)}{\iota\det\left(1-Ft;\mathrm H^2_c(X_{\ov{\FF}_q}(\mc G\otimes\ov{\mc G}_0))\right)}.\]
		We now make a few remarks.
		\begin{itemize}
			\item To begin, we note that any zero of $\iota\det\left(1-Ft;\mathrm H^2_c(X_{\ov{\FF}_q}(\mc G\otimes\ov{\mc G}_0))\right)$ satisfies $\left|\alpha\right|=q^{-\beta-1}$ by using Poincar\'e duality (and a little on weight considerations), so a pole of the $L$-function can only occur in this circle.
			\item Being affine allows us to apply \Cref{lem:l-func-converge-by-weight}, which tells us that our $L$-function converges for $\left|t\right|<q^{-\beta-1}$. On the other hand, we can see that the local factors in the product
			\[\prod_{\text{closed }x\in X}\iota\det\left(1-t^{\deg x}F_x^*;\mc G_{0,\ov x},\ov{\mc G}_{0,\ov x}\right)^{-1}\]
			are some power series with real coefficients and leading coefficient $1$.
		\end{itemize}
		In total, we note that $\varphi^{\mc G_0}(t)$ is the logarithmic derivative of the above $L$-function, so the above inputs produce the result.

		\item Then we can handle the case where $\mc G_0$ is merely $\iota$-mixed while $X$ remains smooth and affine.

		\item We are now ready to complete the proof.
	\end{enumerate}
	
\end{proof}

\end{document}