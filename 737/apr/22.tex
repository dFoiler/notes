% !TEX root = ../notes.tex

\documentclass[../notes.tex]{subfiles}

\begin{document}

\section{April 17}
Today, we finish our discussion of the Fourier transform.

\subsection{Fourier Theory for Sheaves}
We would like a version of the Plancherel formula and Fourier inversion for our sheaves. Because it is easier, we begin with the Plancherel formula.
\begin{theorem}[Plancherel]
	Choose $m\ge1$ and a complex $K_0\in D^b_c(\AA^1,\ov\QQ_\ell)$. Then
	\[\norm{f^{T_\psi K_0}}_m=q^{m/2}\norm{f^{K_0}}_m.\]
\end{theorem}
\begin{proof}
	This is a direct computation. By definition,
	\[\norm{f^{T_\psi K_0}}_m^2=\sum_{x\in\FF_{q^m}}f^{T_\psi K_0}(x)\overline{f^{T_\psi K_0}(x)}.\]
	By \Cref{lem:trace-fourier-sheaf}, this expands to
	\[\norm{f^{T_\psi K_0}}_m^2=\sum_{x,y,z\in\FF_{q^m}}f^{K_0}(y)\overline{f^{K_0}(z)}\psi(-xy)\psi(xz).\]
	We now isolate the sum over $x$, noting this internal sum is
	\[\sum_{x\in\FF_{q^m}}\psi(x(y-z))=\begin{cases}
		q^m & \text{if }y=z, \\
		0 & \text{else}.
	\end{cases}\]
	We conclude that
	\[\norm{f^{T_\psi K_0}}_m^2=q^m\sum_{y\in\FF_{q^m}}f^{K_0}(y)\overline{f^{K_0}(y)},\]
	completing the proof.
\end{proof}
\begin{remark}
	Here is a more sheaf-theoretic implication: \Cref{thm:weight-on-curves} more or less explains that $w(\mc G_0)=w(\mc G_0)$, so we are finding that the Fourier transform more or less preserves the weight. More precisely, suppose that $K_0$ has all the sheaves $\mc H^i(K_0)$ being $\iota$-mixed. Then one may define
	\[w(K_0)\coloneqq\max_{i\in\ZZ}\{w(\mc H^i(K_0))-i\}.\]
\end{remark}
The character sum at the heart of the previous proof can be isolated into the following computation: Let's see a special case of this computation: the constant function $1$ has Fourier transform
\[T_\psi1(y)=\sum_{x\in\FF_{q^m}}\psi(xy)=\begin{cases}
	q^m & \text{if }y=0, \\
	0 & \text{else}.
\end{cases}\]
This motivates the following lemma.
\begin{lemma}
	Let $\delta_0$ be the skyscraper sheaf $i_{0*}\ov\QQ_\ell$, where $i_0\colon\{0\}\into\AA^1$ is the embedding. Then
	\[T_\psi\ov\QQ_\ell=\delta_0(-1)[-1].\]
\end{lemma}
\begin{proof}
	We use the Leray spectral sequence on the Artin--Schreier cover $\pi\colon\AA^1\to\AA^1$ given by $\pi(x)\coloneqq x^q-x$. Recall from \Cref{rem:diagonalize-regular} that
	\[p_*\ov\QQ_\ell=\bigoplus_{x\in\FF_q}\mc L_0(\psi_x).\]
	Now, the Leray spectral sequence produces the fact
	\[\mathrm H^i_c(\AA^1_{\ov\FF_q},\mc L_0(\psi_x))=\begin{cases}
		\ov\QQ_\ell(-1) & \text{if }(x,i)=(0,2), \\
		0 & \text{else}.
	\end{cases}\]
	We now compute from the definition that $T_\psi\ov\QQ_\ell$ is $\mathrm R\pi_{1!}(m^*\mc L_0(\psi))[1]$, whose stalk at $x$ is
	\[\mathrm R\Gamma_c(\mc L(\psi_x))[1]=\delta_0(-1)[-1]\]
	by using base-change. The result follows because having the same functions forces the same sheaves for the sheaf-function correspondence.
\end{proof}
\begin{remark}
	Let's compare weights. Note $\delta_0$ has weight $0$, so $\delta_0(-1)$ has weight $2$, so $\delta_0(-1)[-1]$ has weight $1$.
\end{remark}
We now turn to Fourier inversion.
\begin{theorem}[Fourier inversion]
	For any complex $K_0\in D^b_c(\AA^1,\ov\QQ_\ell)$, we have
	\[T_{\psi^{-1}}T_\psi K_0=K_0(-1).\]
\end{theorem}
\begin{proof}
	One more or less copies the proof of the usual Fourier inversion formula for $\FF_q$, but every time one wants to exchange sums, one has to use some base-change formulae.
\end{proof}

\subsection{Reductions to \texorpdfstring{$\AA^1$}{ A1}}
We are now ready to enter the proof of the main theorem.
\begin{theorem}
	Let $U$ be a smooth affine curve over $\FF_q$, and let $\mc F_0$ be a smooth, $\iota$-mixed Weil sheaf of weight at most $w$. Then the space $\mathrm H^i(X,j_*\mc F_0)$ is $\iota$-mixed of weight at most $w+i$.
\end{theorem}
\begin{remark}
	By dual considerations, we achieve the following: if $\mc F_0$ is $\iota$-pure of weight $w$, then the space $\mathrm H^i(X,j_*\mc F_0)$ is $\iota$-mixed of weight at most $w+i$.
\end{remark}
\begin{proof}[Proof of reductions]
	There is a long series of reductions to reduce from the general case $U\subseteq X$ to the case of $\AA^1\subseteq\PP^1$.
	\begin{enumerate}
		\item We may focus on the case $i=1$.
		\begin{itemize}
			\item For $i=0$, we have two cases. If $\mc F_0$ is irreducible and non-constant, then $\mathrm H^0(X,j_*\mc F_0)=0$. Namely, $\mc F_0$ has no global sections. Otherwise, one finds that $\mathrm H^0_c(U,\mc F)$ is pure of weight $w$.
			\item For $i=2$, we see that Poincar\'e duality has
			\[\mathrm H^2_c(U,\mc F)=\mathrm H^0_c(U,\mc F^\lor)^\lor,\]
			so we reduce to the case $i=0$ again.
		\end{itemize}
		
		\item Next, we note that one may shrink $U$. Let $V\subseteq U$ be some dense open subscheme. Because $V\subseteq U$ is open, one has $\mathrm H^1(U,j_!\mc F)=\mathrm H^1_c(V,\mc F|_V)$. Further, because $U\subseteq V$ is open, the natural map $\mathrm H^1_c(V,\mc F|_V)\to\mathrm H^1_c(U,\mc F)$ is surjective: by Mayer--Vietoris, the cokernel is measured by some $\mathrm H^1$ on $U\setminus V$, which is a zero-dimensional set!

		\item Note that weight is essentially independent of extending the base field $\FF_q$, so we may do so freely.

		\item We may assume that $U\subseteq\AA^1$. We use Noether normalization, which provides a finite map $\pi\colon U\to\AA^1$. By restricting to the smooth locus of $\pi$ (passing to some open subset of $U$), we may assume that $\pi$ is smooth and hence \'etale. Now, let $U'$ be the image of $\pi$, which is some open subset of $\AA^1$, and we see that
		\[\mathrm H^1_c(U',\pi_*(\mc F|_{U'}))=\mathrm H^1_c(U,\mc F|_U),\]
		so we may as well replace $U$ with $U'$ and $\mc F|_U$ with $\pi_*(\mc F|_{U'})$. Do note that this process may make $\pi_*(\mc F|_{U'})$ larger because $\pi$ may upset some monodromy, and in fact $\pi_*(\mc F|_{U'})$ is no longer required to be smooth on all $\PP^1$!

		\item We may assume that $\mc F_0$ is geometrically irreducible. Indeed, the functor $\mathrm H^1_c(U,-)$ is now left exact, so we may split $\mc F_0$ into its irreducible components (and extend the base field until we can see geometrically irreducible components).

		\item We may assume that $\mc F_0$ is unramified at $\infty$. Indeed, choose any point $u\in U$, shrink $U$ to avoid $u$, and then move $u$ to $\infty$ be a M\"obius transformation.

		\item We handle the case where $\mc F_0$ is constant. Let $j\colon U\to\PP^1$ be the embedding, and let $i\colon Z\to\PP^1$ be the complement. Then one has an exact sequence
		\[0\to j_!\QQ_\ell\to j_*\QQ_\ell\to Q\to0,\]
		where $Q=i_*\QQ_\ell$ by construction (taking stalks). In fact, note that $j_*\ov\QQ_\ell$ is simply $\ov\QQ_\ell$ because $(\PP^1)^{\mathrm{sh}}_x\setminus\{x\}$ is connected.\footnote{I am not sure what this sentence means.} Note this short exact sequence is Galois-invariant, so we can take cohomology
		\[\underbrace{\mathrm H^0_c(U,j_*\QQ_\ell)}_{\QQ_\ell}\to\underbrace{\mathrm H^0_c(U,i_*\QQ_\ell)}_{\QQ_\ell^{\#Z_0}}\to\underbrace{\mathrm H^1_c(U,j_!\QQ_\ell)}_{\mathrm H^1_c(U,\QQ_\ell)}\to\underbrace{\mathrm H^1_c(U,j_*\QQ_\ell)}_{\mathrm H^1(\PP^1,\QQ_\ell)},\]
		which upon making these substitutions implies that the map $\QQ_\ell^{\#Z_0-1}\to\mathrm H^1_c(U,\QQ_\ell)$ is an isomorphism; notably, the last term above is $\mathrm H^1(\PP^1,\QQ_\ell)=0$.
	\end{enumerate}
	We will complete the proof next class.
\end{proof}

\end{document}