% !TEX root = ../notes.tex

\documentclass[../notes.tex]{subfiles}

\begin{document}

\section{April 1}
Today is a serious class.

\subsection{Semicontinuity of Weights}
\Cref{lem:l-func-converge-by-weight} merely requires an upper bound on the weight, so it is desirable to be able to produce such bounds. The following lemma achieves a ``spreading out'' result of this type.
\begin{lemma} \label{lem:semicontinuity-curves}
	Fix a smooth irreducible curve $X$ over $\FF_q$, and let $j\colon U\subseteq X$ be a nonempty open subset with complement $S\coloneqq X\setminus U$. Fix a Weil sheaf $\mc G_0$ on $X$, and suppose that $j^*\mc G_0$ is smooth and $\mathrm H^0_S(X,\mc G_0)=0$ and $w(j^*\mc G_0)\le\beta$ for given $\beta$. Then $w(\mc G_0)\le\beta$.
\end{lemma}
\begin{proof}
	Intuitively, we are trying to pass a bound on the weight from a dense open subset to the entire curve. We proceed in steps.
	\begin{enumerate}
		\item We do some reductions. By passing to a finite extension of $\FF_q$, we may assume that $X$ is geometrically irreducible. For brevity, we define $\mc F_0\coloneqq j^*\mc G_0$. Now, because $\mathrm H^0_S(X_{\ov\FF_q},\mc G_0)=0$, it follows that there is an embedding $\mc G_0\into j_*\mc F_0$, allowing us to basically replace $\mc G_0$ with $j_*\mc F_0$ so that $j_*j^*\mc G_0=\mc G_0$. Continuing, by shrinking $U$, we may as well assume that $X$ is affine.

		Now, the trace formula from \Cref{cor:weil-l-func-rational} expands as
		\begin{align*}
			\iota L(X,\mc G_0,t) &= \iota L(U,j^*\mc G_0,t)\prod_{\text{closed }s\in S}\iota\det\left(1-F_st^{\deg s};\mc G_{0,\ov s}\right)^{-1} \\
			&= \frac{\iota\det\left(1-Ft;\mathrm H^1_c(X_{\ov\FF_q};\mc G)\right)}{\iota\det\left(1-Ft;\mathrm H^2_c(X_{\ov\FF_q};\mc G)\right)}.
		\end{align*}
		We are interested in controlling eigenvalues of in the product over closed points $s\in S$, which more or less amounts to controlling the location of our poles.

		\item Thus, we would like to compute a $\mathrm H^2_c$, for which we write
		\begin{align*}
			\mathrm H^2_c(X_{\ov\FF_q},\mc G) &= \mathrm H^2_c(U_{\ov\FF_q},\mc F) \\
			&= \mathrm H^0(U_{\ov\FF_q},\mc F^\lor)(-1) \\
			&= V_{\pi_1^{\mathrm{\acute et}}(U_{\ov\FF_q},\ov x)}(-1),
		\end{align*}
		where $V=\mc G_{0,\ov x}$ is some fiber, and $V_{\pi_1^{\mathrm{\acute et}}(U_{\ov\FF_q},\ov x)}$ refers to the co-invariants. Now, the poles of $\iota L(X,\mc G_0,t)$ may only occur at numbers of the form $\iota\left(\alpha^{-1}q^{-1}\right)$ for eigenvalues $\alpha$ of $F$ acting on $V_{\pi_1^{\mathrm{\acute et}}(U_{\ov\FF_q},\ov x)}$. Certainly \Cref{lem:l-func-converge-by-weight} tells us that we will find no singularities in the region $\left|t\right|<q^{-\beta/2-1}$.

		\item Note $\dim S=0$, so the product
		\[\prod_{\text{closed }s\in S}\iota\det\left(1-F_st^{\deg s};\mc G_{0,\ov s}\right)^{-1}\]
		is finite and cannot produce poles for $\left|t\right|<q^{-\beta/2-1}$. Comparing with the previous step, it follows that $\left|\iota\alpha\right|^2\le q^{\deg(s)(\beta+2)}$ for all eigenvalues. If not for this $+2$, we would be done.

		\item We now use the ``tensor power trick,'' considering eigenvalues of the sheaves $j_*\mc F_0^{\otimes m}$. Namely, if $\alpha$ is an eigenvalue of $F_s$ acting on $(j_*\mc F_0)_{\ov s}$, then $\alpha^m$ is an eigenvalue of $F_s$ acting on the tensor power $(j_*\mc F_0^{\otimes m})_{\ov s}$. Note that this $F_s$ further succeeds at being injective: we can rewrite this map as $j_*(\mc F_0)_{\ov s}^{\otimes m}\to (j_*\mc F^{\otimes m}_0)_{\ov s}$. The point is that we may rerun the entire argument to now get an estimate $\left|\iota\alpha^m\right|\le q^{\deg(s)(m\beta+2)}$. Taking $m\to\infty$ completes the proof.
		\qedhere
	\end{enumerate}
\end{proof}
Continuing with our spreading out discussion, it may be desirable to control the representations outside a dense open subset.
\begin{lemma}
	Fix a smooth irreducible curve $X$ over $\FF_q$, and let $\mc F_0$ be a smooth Weil sheaf given by a representation $\rho\colon W(X,\ov x)\to\op{GL}(V)$, where $V=\mc G_{0,\ov x}$ for some $x$. As before, let $j\colon U\subseteq X$ be a nonempty open subset with complement $S\coloneqq X\setminus U$. For $s\in S$, let $I_s\subseteq\pi_1^{\mathrm{\acute et}}(U_{\ov\FF_q},\ov x)$ be the ramification subgroup. Then
	\[(j_*\mc F)_{\ov s}=V^{I_s}.\]
\end{lemma}
\begin{proof}
	Omitted.
\end{proof}
\begin{remark}
	In order to use the tensor power trick again later, we note that one thus has an inclusion
	\[(j_*\mc F)_{\ov s}^{\otimes m}=\left(V^{I_s}\right)^{\otimes m}\subseteq\left(V^{\otimes m}\right)^{I_s}=(j_*\mc F^{\otimes m})_{\ov s}.\]
\end{remark}
\begin{lemma} \label{lem:shrinking-stays-irred}
	Let $X$ be a normal irreducible scheme over $\FF_q$, and choose an irreducible and smooth Weil sheaf $\mc G_0$. For any open dense subset $j\colon U\subseteq X$, the sheaf $j^*\mc G_0$ is still irreducible.
\end{lemma}
\begin{proof}
	The normality of $X$ implies that we have a surjection $\pi_1(U_{\ov\FF_q},\ov x)\onto\pi_1(X_{\ov\FF_q},\ov x)$ for any choice of geometric basepoint $\ov x\into U$. Thus, we similarly get a surjection $W(U,\ov x)\onto W(X,\ov x)$. The result now follows because any subrepresentation for $W(U,\ov x)$ would also be a subrepresentation for $W(X,\ov x)$.
\end{proof}
We are now ready to state a main theorem on weights.
\begin{theorem}[Semicontinuity of weights, I] \label{thm:semicont-weight-i}
	Fix a smooth Weil sheaf $\mc G_0$ on a scheme $X$, and let $j\colon U\subseteq X$ be an open dense subscheme.
	\begin{listalph}
		\item $w(\mc G_0)=w(j_*\mc G_0)$.
		\item If $j^*\mc G_0$ is $\iota$-pure of weight $\beta$, then $\mc G_0$ is $\iota$-pure of weight of $\beta$.
	\end{listalph}
\end{theorem}
\begin{proof}
	Here we go.
	\begin{listalph}
		\item We may as well assume that $X$ is irreducible by taking subsets, and we may as well replace $X$ with the normalization of its reduction. Now, for $\dim X=0$, there is nothing to say, and for $\dim X=1$, we get the result from \Cref{lem:semicontinuity-curves}, where we note that smoothness for $X$ is equivalent to being normal.

		It remains to handle $\dim X>1$. Because $X$ is irreducible, any $s\in X\setminus U$ can be found in a curve $Y\subseteq X$ which intersects $U$ nontrivially. The result now follows from the curve case.

		\item By (a), we see that $w(\mc G_0)=w(j_*\mc G_0)$, which gets an equality of the supremum of the possible eigenvalues. To get an equality of the infima, one simply reruns the argument with $\mc G_0^\lor$, which has inverse eigenvalues.
		\qedhere
	\end{listalph}
\end{proof}

\end{document}