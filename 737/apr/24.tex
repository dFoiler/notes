% !TEX root = ../notes.tex

\documentclass[../notes.tex]{subfiles}

\begin{document}

\section{April 24}
Today, we finish the proof!

\subsection{Completion of the Proof}
Today, we complete the proof of \Cref{thm:weil-ii-curve}. Last time, we reduced the proof to the case of $U\subseteq\AA^1$. Here is what we will show today.
\begin{theorem} \label{thm:reduced-weil-ii}
	Let $\mc F_0$ be a smooth, nonconstant, geometrically irreducible $\ov\QQ_\ell$-sheaf on a smooth affine curve $U\subseteq\AA^1$. Suppose that $\mc F_0$ is $\iota$-mixed of weight at most $w$. Then $\mathrm H^1_c(U,\mc F)$ has weight at most $w+1$.
\end{theorem}
Throughout, we work in the context of the theorem. There are three key points, which we state now.
\begin{lemma} \label{lem:key-weil-ii}
	Let $\mc G_0\coloneqq j_!\mc F_0$ be a sheaf on $\AA^1$, and fix an additive character $\psi\colon\FF_q\to\ov\QQ_\ell^\times$.
	\begin{listalph}
		\item Sheaf: $T_\psi\mc G_0$ is a sheaf. Namely, the complex $T_\psi\mc G_0$ is concentrated in cohomological degree $0$.
		\item Vanishing: $\mathrm H^0_c(\AA^1,T_\psi\mc G_0)=0$.
		\item Mixed: $T_\psi\mc G_0$ is $\iota$-mixed.
	\end{listalph}
\end{lemma}
\begin{remark}
	Note that (a) fails if $\mc G_0$ is geometrically constant, which is why we dealt with this case separately in our reductions last class.
\end{remark}
Let's go ahead and prove the theorem from the lemma.
\begin{proof}[Proof of \Cref{thm:reduced-weil-ii} from \Cref{lem:key-weil-ii}]
	Let's explain how the Fourier transform enters the picture. Set $\mc G_0\coloneqq j_!\mc F_0$. Recall from \Cref{thm:calculate-fourier} that
	\[(T_\psi\mc G_0)_0=\mathrm R\Gamma_c(\AA^1,\mc G)[1].\]
	However, this is just $\mathrm R\Gamma_c(U,\mc F)[1]=\mathrm H^1_c(U,\mc F)$ by definition of $j_!$. Thus, we see that it is enough to show $w(T_\psi\mc G_0)\le w+1$.

	We now study $T_\psi\mc G_0$. Well, $w(T_\psi\mc G_0)$ is known to be $\norm{T_\psi\mc G_0}$ by \Cref{thm:weight-on-curves}, which we note needs (b) and (c) of \Cref{lem:key-weil-ii} to make sense of.\footnote{This step is remarkable because it has moved our notion of weight from something ``local'' about Frobenius eigenvalues at stalks to a more ``global'' invariant in the sheaf-function correspondence.} On the other hand, we similarly know $w(\mc G_0)=\norm{\mc G_0}$ (which is $w(\mc F_0)$ by \Cref{lem:semicontinuity-curves}), so it remains to check
	\[\norm{T_\psi\mc G_0}\stackrel?\le\norm{\mc G_0}+1,\]
	which is exactly \Cref{thm:plancherel}!
\end{proof}
It remains to prove \Cref{lem:key-weil-ii}. We will show each part individually. The proofs of (a) and (b) are geometric; (c) will require some more work.
\begin{proof}[Proof of \Cref{lem:key-weil-ii}(a)]
	By definition of the Fourier transform, We must show that $\mathrm H^0_c(\AA^1,j_!\mc F\otimes\mc L(\psi_x))$ and $\mathrm H^2_c(\AA^1,j_!\mc F\otimes\mc L(\psi_x))$ both vanish. (Note $\mc L(\psi_x)$ is smooth, so smoothness is being carried around.) Quickly, the $\mathrm H^0_c$ piece vanishes because $U$ is affine: Poincar\'e duality relates this to some $\mathrm H^2$, which vanishes for affine $U$.

	For the $\mathrm H^2_c$ piece, we do a little representation theory. Let $\rho$ be the underlying representation for $j_!\mc F$ so that we are interested in the representation $\rho\otimes\psi_x$ acting on some finite-dimensional vector space $(\mc F_0)_{\ov a}\otimes\mc L(\psi_x)_{\ov a}$. By Poincar\'e duality, we must show that
	\[\mathrm H^2_c(U,\mc F\otimes\mc L(\psi_x))=V(-1),\]
	where $V$ is the largest quotient of our underlying representation on which the representation acts trivially. To show that this vanishes, we have two cases.
	\begin{itemize}
		\item If $x=0$, then we are simply looking at the irreducible representation $\rho\colon\pi_1^{\mathrm{\acute et}}(U,\ov a)\to\op{GL}(\mc F_{\ov a})$, which is nontrivial and irreducible. Thus, $V=0$.
		\item If $x\ne0$, we note that $\rho$ and hence $\rho\otimes\psi_x$ is unramified at $\infty$, so $\rho\otimes\psi_x$ factors through $\pi_1^{\mathrm{\acute et}}(\PP^1,\ov a)=\op{Gal}(\ov\FF_q/\FF_q)$, which contradicts geometric irreducibility if $V\ne0$.
		\qedhere
	\end{itemize}
\end{proof}
We omit the proof of (b), which is less interesting than (c).
\begin{proof}[Proof of \Cref{lem:key-weil-ii}(c)]
	We would like to show that $T_\psi\mc G_0$ is $\iota$-mixed. For this, we use the fact (not included in these notes) that any subsheaf of a real sheaf is $\iota$-mixed. In particular, consider the sheaf
	\[\mc H\coloneqq\big(\underbrace{(\mathrm{pr}_2^*(j_!\mc F_0\otimes m^*\mc L(\psi))}_{\mc A\coloneqq})\oplus(\underbrace{\mathrm{pr}_2^*(j_!\mc F_0\otimes m^*\mc L(\ov\psi))}_{\mc B\coloneqq})\big)\otimes\mc L_b\]
	is a real sheaf for some $b$ chosen to have $\iota(b)=q^{w(\mc F_0)}$. Note $\mathrm R^i\mathrm{pr}_{1!}\mc A$ and $\mathrm R^i\mathrm{pr}_{2!}\mc B$ are both concentrated in degree $1$ by (a). As such, the Grothendieck trace formula yields
	\[\prod_{y\in\mathrm{pr}_2^{-1}(\{x\})}\det\left(1-tF_y^{\deg y};\mc H_{\ov y}\right)=\det\left(1-tF_x^{\deg x};\mathrm R^1\mathrm{pr}_{1!}\mc H_{\ov x}\right).\]
	We conclude that these factors have real coefficients, so $\mathrm R^1\mathrm{pr}_{1!}\mc H[-1]$ is a real sheaf. Summing appropriately, we see that $T_\psi\mc G_0$ is a factor of $\mathrm R^1\mathrm{pr}_{1!}\mc H[-1]$, so we are done.
\end{proof}
\begin{remark}
	We only achieved the main theorem for curves, but we remark that techniques of algebraic geometry (fibering by hyperplanes) are able to achieve the result for arbitrary dimension.
\end{remark}

\end{document}