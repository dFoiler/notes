% !TEX root = ../notes.tex

So class is in-person today. The lecture hall is very large and quite empty.

\subsection{Formula Induction}
Big-picture, we are focusing on symbolic logic. The main point of introducing our formal language right now is that it will help us formally define what rigorous reasoning is and how it works.

Today we're going to discuss induction on formulae, which will be the main technique to show that some property holds for all formulae. Namely, we will be using \autoref{def:indformula} in order to do out proofs.
\begin{remark}
	We never actually showed that \autoref{def:formulaconstruct} and \autoref{def:indformula} are equivalent, but they are. The annoying thing to do is to check that any set satisfying \autoref{def:formulaconstruct} will automatically satisfy \autoref{def:indformula}.
\end{remark}
Let's do an example proof.
\begin{proposition}
	All expressions in $\mathcal L(P)$ have the same number of left and right parentheses.
\end{proposition}
\begin{proof}
	We proceed by induction. Let $S$ be the set of expressions that have the same number of left and right parentheses, and we show that $\mathcal L(P)\subseteq S$. We have the following two checks.
	\begin{listalph}
		\item Each atomic expressions $p\in P$ have no parentheses at all, so $p\in S$.
		\item Suppose that $\varphi,\psi\in S$; suppose $\varphi$ has $x$ left parentheses (and therefore $x$ right parentheses) and $\psi$ has $y$ left parentheses (and therefore $y$ right parentheses). Then we check what happens with our connectives.
		\begin{itemize}
			\item We see that $\lnot\varphi$ has the same number left/right parentheses as $\varphi$, so these quantities are equal, so $\lnot\varphi\in S$.
			\item We see that $(\varphi\lor\psi)$ and $(\varphi\land\psi)$ and $(\varphi\to\psi)$ and $(\varphi\leftrightarrow\psi)$ have $x+y+1$ left and right parentheses, so we are done.
		\end{itemize}
	\end{listalph}
	Because $\mathcal L(P)$ is the minimal set satisfying (a) and (b), it follows $\mathcal L(P)\subseteq S$, so we are done.
\end{proof}
An alternate way to see that satisfying (a) and (b) is enough is because we can build any formula in $\mathcal L(P)$ by using the steps (a) and (b), by definition of $\mathcal L(P)$. The minimality of $\mathcal L(P)$ is a way to formalize this intuition.
\begin{remark}
	The part where we assume $\varphi,\psi\in S$ before checking $\lnot\varphi$ and $(\varphi\lor\psi)$ and its friends is called the ``inductive hypothesis.'' It is very important: if $\varphi=\left(p\right.\notin S$, then in fact $\lnot\varphi=\lnot\left(p\right.\notin S$ as well.
\end{remark}
Here is another example: we show that $\lnot\to p$ is not a formula, via the following lemma.
\begin{lemma}
	Every formula neither starts with $\lnot\to$ nor with $\to$.
\end{lemma}
\begin{proof}
	Let $S$ be the set of expressions that neither start with $\lnot\to$ nor with $\to$. We now induct.
	\begin{listalph}
		\item No atomic formula starts with $\lnot\to$ nor with $\to$.
		\item If we have a formula $\varphi$ not starting with $\lnot\to$ nor with $\to$, then $\lnot\varphi$ will not start with $\lnot\to$ because $\varphi$ did not start with $\to$.

		On the other hand, if $\varphi$ and $\psi$ neither start with $\lnot\to$ nor with $\to$, then (for any binary connector $\#$) $(\varphi\#\psi)$ will start with a parenthesis and not with $\lnot\to$ nor with $\to$.
	\end{listalph}
	Thus, because $\mathcal L(P)$ is minimal satisfying (a) and (b), it follows $\mathcal L(P)\subseteq S$, so we are done.
\end{proof}
Here is another result, which we can show without much effort by induction.
\begin{lemma}
	If we have letters $P\subseteq Q$, then $\mathcal L(P)\subseteq\mathcal L(Q)$. We will not prove this here, but it is just another induction.
\end{lemma}
The point of this is to say that a single formula can belong to multiple languages. For example,
\[(p_1\land p_2)\in\mathcal L(\{p_1,p_2\})\cap\mathcal L(\{p_1,p_2,p_3\}).\]
The proof of the lemma is by induction and not hard, so we will omit it.%\todo{}

\subsection{Subformula}
Here is a motivating example.
\begin{example}
	The formula $\varphi=(\lnot p_1\to\lnot(p_2\lor p_3))$ has the following subformulae.
	\[p_1,\quad\lnot p_1,\quad p_2,\quad p_3,\quad(p_2\lor p_3),\quad\lnot(p_2\lor p_3),\quad(\lnot p_1\to(\lnot(p_2\lor p_3))).\]
	The set of proper formulae is the same set except for $\varphi$.
\end{example}
\begin{nex}
	The formula $(p_1\lor p_2)$ is not a subformula of $(p_1\land p_2)$.
\end{nex}
We would like to define a function $\op{sub}:\mathcal L(P)\to\mathcal P(\mathcal L(P))$. For this, we will create a recursive definition for $\op{sub}$.
\begin{definition}[Subformula]
	Given a formula $\varphi\in\mathcal L(P)$, we define the \textit{subformulae} $\op{sub}(\varphi)$ to be defined as follows.
	\begin{itemize}
		\item $\op{sub}(p):=\{p\}$ for each atomic formula $p\in P$.
		\item $\op{sub}(\lnot\varphi)=\op{sub}(\varphi)\cup\{\lnot\varphi\}$.
		\item $\op{sub}((\varphi\#\psi))=\op{sub}(\varphi)\cup\op{sub}(\psi)\cup\{(\varphi\#\psi)\}$ for any two formulae $\varphi,\psi\in\mathcal L(P)$ and binary connective $\#\in\{\land,\lor,\to,\leftrightarrow\}$.
	\end{itemize}
\end{definition}
The above definition is called ``recursive'' because we did not explicitly define it but only how to do this by breaking down connectives. We can show that $\op{sub}$ is defined on $\mathcal L(P)$ by yet another induction which we will omit.\footnote{By definition, $\op{sub}$ is defined on atomic formulae, and the domain is closed under connectives, so $\op{sub}$ is defined on $\mathcal L(P)$.}

Let's see some examples.
\begin{example}
	We have the following computation.
	\begin{align*}
		\op{sub}(\lnot(p\land q)) &= \op{sub}((p\land q)) \cup \{\lnot(p\land q)\} \\
		&= \op{sub}(p)\cup\op{sub}(q)\cup\{(p\land q)\} \cup \{\lnot(p\land q)\} \\
		&= \{p\}\cup\{q\}\cup\{(p\land q)\} \cup \{\lnot(p\land q)\} \\
		&= \{p,q,(p\land q),\lnot(p\land q)\}.
	\end{align*}
\end{example}
\begin{example}
	We have the following computation.
	\begin{align*}
		\op{sub}((p\land p)) &= \op{sub}(p)\cup\op{sub}(p)\cup\{(p\land p)\} \\
		&= \{p\}\cup\{p\}\cup\{(p\land p)\} \\
		&= \{p,p\land p)\}.
	\end{align*}
	Note that the union of the two sets has killed the duplicate $p$.
\end{example}
\begin{example}
	We have the following computation.
	\begin{align*}
		\op{sub}((\lnot p_1\to p_2)) &= \op{sub}(\lnot p_1)\cup\op{sub}(p_2)\cup\{(\lnot p_1\to p_2)\} \\
		&= \op{sub}(p_1)\cup\{\lnot p_1\}\cup\op{sub}(p_2)\cup\{(\lnot p_1\to p_2)\} \\
		&= \{p_1\}\cup\{\lnot p_1\}\cup\{p_2\}\cup\{(\lnot p_1\to p_2)\} \\
		&= \{p_1,\lnot p_1,p_2,(\lnot p_1\to p_2)\}.
	\end{align*}
\end{example}
It is true that the definition of ``subformula'' feels intuitively obvious, but we have given the above rigorous definition to introduce the idea of a recursive definition.

Let's discuss the idea of recursion a little more closely because it will come up again.
\begin{itemize}
	\item Define a set $S$ inductively as the smallest set containing some base objects and closed under some operations. We will also require the following coherence conditions.\footnote{The coherence conditions ensure that the function we create is well-defined.}
	\begin{itemize}
		\item We will require that the operations never output the same formula. For example, for two formulae $\varphi$ and $\psi$, we have
		\[(\varphi\land\psi)\ne(\varphi\lor\psi).\]
		\item No operation can output a base object. For example, no operation can output an atomic formula because operations will always start with $\lnot$ or a parenthesis.
	\end{itemize}
	\item Then to define a function $f$ inductively, we need to define $f(b)$ for each base object $b\in S$ and provide some function $g$ such that
	\[f(o(a_1,\ldots,a_n))=g_o(f(a_1),\ldots,f(a_n),a_1,\ldots,a_n)\]
	for each $n$-ary operation $o$. For example, we had
	\[\op{sub}(o_\lnot(\varphi))=g_\lnot(\op{sub}(\varphi),\varphi),\]
	where $g_\lnot(S,\varphi):=S\cup\varphi$.
\end{itemize}
Let's do another example recursive definition.
\begin{definition}[Depth]
	We define the \textit{depth} of a formula $\varphi\in\mathcal L(P)$ to be given be a function $\op{depth}:\mathcal L(P)\to\NN$ defined as followed.
	\begin{itemize}
		\item $\op{depth}(p)=0$ for each atomic formula $p\in P$.
		\item $\op{depth}(\lnot\varphi)=\op{depth}(\varphi)+1$ for each $\varphi\in\mathcal L(P)$. In other words, adding $\lnot$ adds one level of depth.
		\item $\op{depth}((\varphi\#\psi))=\max\{\op{depth}(\varphi),\op{depth}(\psi)\}+1$ for each $\varphi,\psi\in\mathcal L(P)$ and binary connective $\#$. In other words, the depth of $(\varphi\#\psi)$ is one more than the larger of the two depths of $\varphi$ and $\psi$.
	\end{itemize}
\end{definition}
\begin{example}
	We have the following computation.
	\begin{align*}
		\op{depth}((\lnot p_1\to p_2)) &= \max\{\op{depth}(\lnot p_1),\op{depth}(p_2)\}+1 \\
		&= \max\{\op{depth}(p_1)+1,\op{depth}(p_2)\}+1 \\
		&= \max\{0+1,0\}+1 \\
		&= 2.
	\end{align*}
\end{example}
We can see that this definition is indeed recursive: we computed it as $0$ for the basic objects, and then we had
\[f(o_\lnot(\varphi))=g_\lnot(f(\varphi),\varphi),\]
where $g_\lnot(n,\varphi)=n+1$, and
\[f(o_\#(\varphi,\psi))=g_\#(f(\varphi),f(\psi),\varphi,\psi),\]
where $g_\#(n,m,\varphi,\psi)=\max\{n,m\}+1$.