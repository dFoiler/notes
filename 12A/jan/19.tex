% !TEX root = ../notes.tex






Let's go ahead and get going. Today we are hyping the course. 

\subsection{Symbolic Logic}
In this course, we are more interested in symbolic logic. More broadly, we are interested in what kind of reasoning is ``logical,'' and we will do this by abstracting what a good argument is.

In symbolic logic, we have lots of symbols for our reasoning words. Here is a table of such symbols.
\begin{center}
	\begin{tabular}{c|c}
		word & symbol \\\hline
		not & $\lnot$ \\
		and & $\land$ \\
		or & $\lor$ \\
		if, then & $\to$ \\
		for all & $\forall$ \\
		there exists & $\exists$
	\end{tabular}
\end{center}
In this course we will be able to give a rigorous definition for what a valid formula is in the language of these symbols. The truth value of a statement will have no dispute.
\begin{example}
	The formula
	\[(\forall x\text{Red}(x)\lor\exists y\text{Square}(y))\to\exists z(\text{Red}(z)\land\text{Square}(z))\]
	asserts that ``if everything is red and something is square, then there is something which is both red and square.'' This is a good, true assertion: that something which is square must also be red, finishing.
\end{example}

\subsection{Advertisements}
This sort of reasoning has applications in lots of fields.
\begin{itemize}
	\item Logic is a main branch of philosophy. For example, we will study the syllogistic reasoning of Aristotle.\footnote{Here is an example of a syllogism: Suppose that all men are mortal, and that Socrates is a man. Then it follows Socrates is mortal.}
	\item Logic is at the base of mathematics, and careful logical reasoning informs foundational mathematics (e.g., G\"odel's incompleteness theorems or the independence of the Continuum hypothesis). For example, we will have to understand (basic) mathematical proofs in this course. We will talk about foundational mathematics a bit at the end of the course.
	\item Logic and its methods (e.g., $\lambda$-calculus) impacts how one does computer programming. For a concrete example, logic is used in SQL to give statements for database queries. As another example, formal hardware and software verification comes down to very careful logical analysis.
	\item One approach to artificial intelligence is by trying to create a machine which spits out true facts from old ones, for which the formal language of first-order logic is quite important.
	\item The kind of epistemic logic of trying to reason about what people know and do not know is important in game theory and hence has applications to economics. This can quickly get complicated: for example, we might want to keep track of the fact that (e.g., in poker) Player 1 knows that Player 2 knows that Player 3 has an ace card, for this fact might affect Player 2's behavior.
	\item Linguistics is interested in what sentences mean, for which one had to keep track of formal semantics to determine truth values.
	\item In cognitive science, how hard it is to understand/learn something turns out to be directly proportional to the length of the shortest logically equivalent propositional formula. In other words, longer formulae are harder to get in one's head.
\end{itemize}

\subsection{Logistics}
Let's talk about logistics.
\begin{itemize}
	\item There is a class \href{https://piazza.com/class/ky91rs7rzr02bc}{Piazza}, which hopefully will get some use.
	\item The course outline in the syllabus is more of a guess than a promise; we may get ahead or behind it, but the syllabus will be updated frequently to match.
	\item There is a textbook. It is more like a math textbook: one is expected to read things multiple times instead of in an English class where one tries to skim as much as possible. While the material is dense, the course has been designed to try to make the course accessible to everyone.
	\item All reading (including the textbook) will be freely available online.
	\item It is better to skim the reading before lecture to not completely be lost during lecture. It is also good to do another pass on the reading after lecture.
	\item There are weekly problem sets, released on Monday (starting next Monday) and due on Sunday midnight. They will be graded via GradeScope, and there are regrade requests (as usual).
	\item In theory, the problem sets will not depend on a great deal on the lecture Friday before the deadline.
	\item The class will be curved upwards depending on its difficulty, at the very end.
	\item Please come to office hours instead of struggling needlessly on one's own.
\end{itemize}
Next class we will talk about propositional logic.