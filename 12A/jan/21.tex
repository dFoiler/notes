\documentclass[../notes.tex]{subfiles}
\graphicspath{{\subfix{../figs/}}}

\begin{document}

% !TEX root = ../notes.tex


Today we are talking about propositional logic.

\subsection{Proportions and Connectives}
Roughly, propositional logic is about reasoning with propositions with propositional connectives.
\begin{remark}
	Propositional logic might also be called sentential logic or boolean logic.
\end{remark}
Here are propositions.
\begin{definition}[Proposition]
	In this class, a \textit{proposition} will simply be any declarative sentence.
\end{definition}
\begin{example}
	The sentence ``Paris the capital of France'' is a proposition.
\end{example}
\begin{nex}
	The question ``Is Paris the capital of France?'' is not a proposition.
\end{nex}
\begin{remark}
	Logic can handle questions, but we will not discuss it. It is called inquisitive logic.
\end{remark}
Here are propositional connectives.
\begin{definition}[Connective]
	A \textit{propositional connective} is some word or phrase that we can use to build new propositions from old ones.
\end{definition}
There are lots of propositional connectives. Have some examples.
\begin{example}
	Suppose $p$ and $q$ are propositions. For example, $p$ can be ``the thief entered through the back door,'' and $q$ can be ``the thief left through the side door.'' Then we have the following.
	\begin{itemize}
		\item ``$p$ or $q$'' is a proposition.
		\item ``$p$ and $q$'' is a proposition.
		\item ``If $p$ then $q$'' is a proposition.
		\item ``Not $p$'' is a proposition. (Grammatically, ``it is not the case that $p$.'')
	\end{itemize}
	These are read by replacing $p$ and $q$ with the propositions they represent.
\end{example}
\begin{example}
	Propositional connectives do not have to be absolute. For example, ``it is more likely that $p$ than it is that $q$'' is a proposition.
\end{example}

\subsection{Reasoning}
So thus far we have propositions and their connectives. How might we reason with them? Here's an example.
\begin{example}
	If we happen to know that ``$p$ or $q$,'' and we know that ``it is not the case that $p$,'' then $q$ must be true. This is essentially process of elimination; e.g., we might imagine running this argument with $p$ that ``the infection is viral'' and $q$ that ``the infection is bacterial.''
\end{example}
The reasoning in the above argument feels quite true, even without making $p$ or $q$ concrete propositions. The reasoning makes us feel good.
\begin{warn}
	Suppose we have the following premises.
	\begin{itemize}
		\item $p$ or $q$.
		\item $p$.
	\end{itemize}
	It does not follow that $q$ is false. In short, $p$ or $q$ permits both $p$ and $q$ to be true.
\end{warn}
Here is another example of reasoning.
\begin{example}
	We have the following premises.
	\begin{itemize}
		\item If $p$, then $q$.
		\item Not $q$.
	\end{itemize}
	Then it follows that not $p$. Concretely, we can set $p$ to be ``the reactor is cooling down'' and $q$ to be ``the blue light is on.'' Then the fact that the blue light is not on would imply that the reactor is not cooling down.
\end{example}
We do have to be careful with conditionals, however.
\begin{warn}
	The premise ``if $p$ then $q$'' does not imply that ``if $q$ then $p$.''
\end{warn}
And here is some reasoning with less concrete connectives.
\begin{example}
	Suppose we are given that $p$ is more likely than $q$. Then it follows that $p$ is more likely than $q$ and $r$, for any other event $r$. In essence, trying to make more events happen is harder. For concreteness, think through this argument with the following concrete premises:
	\begin{itemize}
		\item Set $p$ to be ``the US will sign the treaty.''
		\item Set $q$ to be ``Russia will sign the treaty.''
		\item Set $r$ to be ``China will sign the treaty.''
	\end{itemize}
	Having more people sign the treaty is harder.
\end{example}
Reasoning can be hard, and sometimes our intuition might be wrong. Here is some bad reasoning.
\begin{nex}
	Suppose we have the following premises.
	\begin{itemize}
		\item If $p$, then $q$.
		\item Not $p$.
	\end{itemize}
	Then it does not follow that $q$. Concretely, set $p$ to be ``the patient is taking her medicine'' and $q$ to be ``the patient will get better.'' Then the fact that patient is not taking her medicine does not imply that the patient will not get better: perhaps the patient will get better for some other reason.
\end{nex}
The above reasoning is bad because we went from true premises to false conclusions. This is what we try to avoid.

Here is more bad reasoning.
\begin{nex}
	Suppose we have the following premises.
	\begin{itemize}
		\item It is more likely that $p$ than $q$.
		\item It is more likely that $p$ than $r$.
	\end{itemize}
	Then it does not follow that $p$ is more likely than $q$ or $r$. For example, take $p$ to be the event that a die roll is odd, $q$ to be the event that a die rolls is $\{1,2\}$, and $r$ to be the vent that a die rolls is $\{3,4\}$. The probability that $q$ or $r$ exceeds the probability that $p$ in this case.
\end{nex}
As an aside, our bad reasoning might still give good premises at the end. The reason that we like good reasoning better is that every single time we will get good premises from good ones; with bad reasoning, we run the risk of getting bad results at the end.
\begin{example}
	The argument that the proposition ``grass is green'' directly implies ``the sky is blue'' is not valid reasoning because these two propositions have effectively nothing to do with each other. Nevertheless, ``the sky is blue'' is a true conclusion.
\end{example}

\subsection{Truth-Functional Connectives}
Earlier we gave examples of lots of different propositional connectives. It turns out that we only care about very few of these: the truth-functional propositional connectives.

We need to have a reasonable notion of truth. We adopt the following conventions.
\begin{convention}
	In this course, we take the following.
	\begin{itemize}
		\item All propositions are either true or false.
		\item No proposition is both true and false.
	\end{itemize}
\end{convention}
These probably seem obvious, but we need to be careful.
\begin{example}
	The following propositions are bad in that they have unclear truth value.
	\begin{itemize}
		\item The proposition ``ice cream is delicious'' is a proposition of taste (this depends on the person), so we will ignore it.
		\item ``Bob is bald'' is a bit vague because ``bald'' is not well-defined, so we will ignore it.
		\item ``This proposition is false'' is self-referential and more or less breaks down truth (if the proposition is true, then the proposition declares its own falsehood), so we will ignore it. Similar is ``this proposition is true.'' (This is known as the liar paradox.)
	\end{itemize}
	One way to escape these problems is to simply declare that they are not propositions. We choose to ignore the altogether.
\end{example}
Nevertheless, we go forwards with our notion of truth value.
\begin{definition}[Truth value]
	The \textit{truth value} of a proposition is ``true'' if the proposition is true and ``false'' if it is false.
\end{definition}

Very quickly, we note that the connectives we've talked about come in two classes.
\begin{definition}[Unary, binary]
	A propositional connective is \textit{unary} (respectively, \textit{binary}) if and only if it acts on one (respectively, two) proposition.
\end{definition}
\begin{example}
	The connective ``not'' is a unary connective. The connective ``We know that'' is a unary connective.
\end{example}
\begin{definition}
	More generally, the \textit{arity} of a connective is the number of propositions the connective acts on.
\end{definition}
\begin{example}
	The arity of ``not'' is $1$.
\end{example}
Natural language tends to focus on unary and binary connectives.

We are now ready to define what a truth-functional connective is. Here is the definition for unary connectives.
\begin{definition}[Truth-functional]
	A unary connective $\#$ is \textit{truth-functional} means that $\#p$ has truth value which is a function of (i.e., is completely determined by) the truth value of $p$.
\end{definition}
\begin{example}
	The connective ``not'' is a truth-functional connective: the truth value of $p$ tells us what the truth value of ``not $p$'' is.
\end{example}
\begin{nex}
	The connective ``the police know that'' is not a truth-functional connective: a statement being true or false does not immediately tell us whether the police know it. Hopefully if $p$ is false, then the police do not know $p$; but if $p$ is true, perhaps the police simply do not it yet. The point is that the truth value of ``the police know that $p$'' is simply not a function of $p$.
\end{nex}
Here is truth-functionality for binary connectives.
\begin{definition}[Truth-functional]
	A binary connective $\#$ is \textit{truth-functional} means that $p\#q$ has truth value which is a function of (i.e., is completely determined by) the truth values of $p$ and $q$.
\end{definition}
\begin{example}
	The connective taking the propositions $p,q$ to ``$p$ and $q$'' is truth-functional.
\end{example}

\end{document}