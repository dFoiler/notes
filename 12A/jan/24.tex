\documentclass[../notes.tex]{subfiles}
\graphicspath{{\subfix{../figs/}}}

\begin{document}

% !TEX root = ../notes.tex

Okay, welcome back everybody.

\subsection{Truth-Functionality}
Recall the definition.
\begin{definition}[Truth-functional]
	A binary connective $\#$ is \textit{truth-functional} means that $p\#q$ has truth value which is a function of (i.e., is completely determined by) the truth values of $p$ and $q$.
\end{definition}
\begin{example}
	The connectives ``\ldots and \ldots'' and ``\ldots or \ldots'' are both truth-functional.
\end{example}
\begin{nex}
	The \textit{counterfactual} connective ``if it had been the case that $p$, then it would have been the case that $q$'' is not truth-functional. To see this, note that $p$ being false permits $q$ to be whatever it wants without assigning a truth value depending on $p$ and $q$. Here are some examples.
	\begin{itemize}
		\item Consider ``if I had overslept, then the speaker gave her lecture.'' In fact, I did not oversleep, and the speaker would give her lecture anyways, so this is $p$ being false and $q$ being true. Here, the counterfactual is in total true.
		\item Consider ``if I had overslept, then I would have arrived late.'' In fact, I did not oversleep, and in fact I would have arrived on time in this case, so this is $p$ being false and $q$ being true. Here the counterfactual is in total false.
	\end{itemize}
\end{nex}

\subsection{The Material Conditional}
Let's talk about conditionals; they are potentially confusing. The motivation here is that ``if \ldots then \ldots'' is used mathematically quite often, so we are going to formalize this. The following is our truth table.
\begin{definition}[Material conditional \texorpdfstring{($\to$)}{}]
	Given two propositions $p$ and $q$, we define the \textit{material conditional} read as ``if $p$ then $q$'' is defined by the following truth table.
	\begin{center}
		\begin{tabular}{cc|c}
			$p$ & $q$ & if $p$, then $q$ \\\hline
			T & T & T \\
			T & F & F \\
			F & T & T \\
			F & F & T
		\end{tabular}
	\end{center}
\end{definition}
Mnemonically, the only way to make an if-then statement false is for the premise to be true and the conclusion to be false.
\begin{warn}
	The statement ``if I were 90, then I would be king'' is a true statement because the premise is false. However, natural language would dictate this would probably not be accepted as true.
\end{warn}
\begin{example}
	The only way for Goldbach's conjecture (all even numbers greater than $2$ are the sum of two prime numbers) to be false is for there to exist an even number which is not the sum of two prime numbers.
	
	Notably, we do not falsify by showing the existence of odd numbers (such as $11$) which are not the sum of two prime numbers. Explicitly, the statement
	\begin{center}
		If $11$ is an even number greater than $2$, then $11$ is the sum of two primes.
	\end{center}
	is true because the premise is false.
\end{example}
\begin{remark}
	We can think of the material conditional $p\to q$ as $(\lnot p)\lor q$. Namely, as long as $p$ is false or $q$ is true, then $p\to q$ will be true (and conversely).
\end{remark}
In this course, we will focus on truth-functional connectives.

\subsection{Validity and Soundness}
For the previous classes, we have been talking about what ``good'' arguments feel like. The technical version of this is ``valid.''
\begin{definition}[Valid]
	A form of argument is \textit{valid} if, no matter the truth values of the propositions, whenever the premises are true, the conclusion will also be true.
\end{definition}
\begin{example}
	The following argument form is valid.
	\begin{enumerate}
		\item $p$ or $q$.
		\item Not $p$.
		\item Therefore, $q$.
	\end{enumerate}
	Explicitly, all cases where the first two premises hold require $q$ to be true. (In fact, the only way for this to be true is for $q$ to be true and $p$ to be false.)
\end{example}
We can explicitly plug in to the above ``argument form'' to generate a real, physical argument.
\begin{example}
	The following argument is of the above form.
	\begin{enumerate}
		\item Paris is the capital of France or Berlin is the capital of Belgium.
		\item It is not the case that Paris is the capital of France.
		\item Therefore, Berlin is the capital of Belgium.
	\end{enumerate}
	Note that this argument is valid, even though the second premise is simply false. Giving false premises provides no guarantees that our conclusion is true, and indeed, the conclusion is false.
\end{example}
Take note that many arguments take the given form.
\begin{warn}
	We apply the word ``valid'' to forms of arguments, not actual arguments. This prevents confusion about the truth-values of the actual premises.
\end{warn}
To deal with the above confusion, we have the following definition.
\begin{definition}[Sound]
	If an argument is of valid form, and its premises are true (!), then the argument is \textit{sound}.
\end{definition}
Importantly, note that soundness is applies to arguments while validity applies to argument forms.
\begin{example}
	The following argument is sound.
	\begin{enumerate}
		\item The number $527$ is prime, or the number $527$ is composite.
		\item It is not the case that the number $527$ is prime.
		\item Therefore, the number $527$ is composite.
	\end{enumerate}
\end{example}
\begin{nex}
	The following argument from earlier is not sound.
	\begin{enumerate}
		\item Paris is the capital of France or Berlin is the capital of Belgium.
		\item It is not the case that Paris is the capital of France.
		\item Therefore, Berlin is the capital of Belgium.
	\end{enumerate}
	Notably, the second premise here is false, so even though the argument has valid form, the entire argument is no longer sound.
\end{nex}
In this case, we will mostly not care too much about soundness because we don't want to consider the truth-values of premises. Our job is to describe how to reason, which means we care more about valid argument forms than actually sound arguments.
\begin{remark}
	We do not say that arguments are ``true'' or ``false'' because those words belong to propositions in this class. We will only say ``sound'' or ``unsound.''
\end{remark}
Let's see some more examples.
\begin{nex}
	The following is an invalid form of argument.
	\begin{enumerate}
		\item It is not the case that ($p$ and $q$).
		\item Therefore, it is not the case that $p$.
	\end{enumerate}
	To see that this is invalid, we note that it's possible for $p$ to be true while $q$ is false. Explicitly, take $p$ to be ``$5$ is prime'' and $q$ to be ``$4$ is prime.'' Here, the premise is true (not both $4$ and $5$ are prime), but the conclusion is false ($5$ is actually prime.)
\end{nex}
However, we could get lucky. For example, in the above argument form, if we swap the roles of $p$ and $q$, then the conclusion ($4$ is not prime) will be true. This is an invalid argument still producing true conclusions; the issue is that we are not guaranteed true conclusions from true premises and invalid arguments.

\subsection{Truth Tables}
We would like to have a more mechanical procedure to check the validity of arguments. For this class, we are restricting our attention to the truth-functional case, which means that we directly compute everything depending on the truth values of the propositions. This computation is typically organized into a truth table.

Here is an example.
\begin{exe}
	The following argument form is valid.
	\begin{enumerate}
		\item $p$ or $q$.
		\item It is not the case that $p$.
		\item Therefore, $q$.
	\end{enumerate}
\end{exe}
\begin{proof}
	The only possibilities for $p$ and $q$ are to be true or false in various combinations. We run through all possible cases in the following table.
	\begin{center}
		\begin{tabular}{c|c||c|c||c}
			$p$ & $q$ & $p$ or $q$ & It is not the case that $p$ & $q$ \\\hline
			T & T & T & F & T \\
			T & F & T & F & F \\
			\color{red}F & \color{red}T & \color{red}T & \color{red}T & \color{red}T \\
			F & F & F & T & F
		\end{tabular}
	\end{center}
	From this table we see that the only possibility where all the premises (i.e., both ``$p$ and $q$'' and ``It is not the case that $p$'') are true is when $p$ is false and $q$ is the highlighted row, and in this case we do see that $q$ is true. Thus, the argument is valid: all cases where the premises are true also happen to have that the conclusion is true.
\end{proof}
\begin{remark}
	An alternate way to do this computation would be to note that the truth table computation comes down to showing the single formula
	\[((p\lor q)\land\lnot p)\to q\]
	is always true, independent of the truth values of $p$ and $q$.
\end{remark}
The previous example is a bit misleading because the simple example had only one row in which all premises are true: in general we must check all rows which have the conclusion true. Here is another example.
\begin{exe}
	The following argument form is valid.
	\begin{enumerate}
		\item $p$.
		\item $q$ or $r$.
		\item Therefore, ($p$ and $q$) or ($q$ and $r$).
	\end{enumerate}
\end{exe}
\begin{proof}
	Here is our truth table.
	\begin{center}
		\begin{tabular}{c|c|c||c|c||c|c|c}
			$p$ & $q$ & $r$ & $p$ & $q$ or $r$ & $p$ and $q$ & $p$ and $r$ & ($p$ and $q$) or ($p$ and $r$) \\\hline
			\color{red}T & \color{red}T & \color{red}T & \color{red}T & \color{red}T & \color{red}T & \color{red}T & \color{red}T\\
			\color{red}T & \color{red}T & \color{red}F & \color{red}T & \color{red}T & \color{red}T & \color{red}F & \color{red}T\\
			\color{red}T & \color{red}F & \color{red}T & \color{red}T & \color{red}T & \color{red}F & \color{red}T & \color{red}T\\
			T & F & F & T & F & F & F & F\\
			F & T & T & F & T & F & F & F\\
			F & T & F & F & T & F & F & F\\
			F & F & T & F & T & F & F & F\\
			F & F & F & F & F & F & F & F
		\end{tabular}
	\end{center}
	The first three rows are the only ones where all the premises are true, and from there we can see that in all these rows the conclusion is true. (In fact, those are exactly the rows where the conclusion is true.)
\end{proof}

\end{document}