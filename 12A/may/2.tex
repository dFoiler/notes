% !TEX root = ../notes.tex

Today we are reviewing.

\subsection{Review}
Here are some points from review in class.
\begin{itemize}
	\item Given proposition letters $P$, a valuation is a function $V\colon P\to\{0,1\}$, which extends (by induction) to a function $\hat V\colon\mathcal L(P)\to\{0,1\}$.
	\item As an example of an induction, we can define the function $\bb\varphi$ to be the number of valuations $V$ for which $\hat V(\varphi)$. Then we can see that
	\[\bb{\lnot\varphi}=2^{\#P}-\bb{\varphi}\]
	and
	\begin{align*}
		\bb{\varphi\oplus\psi} &= \bb{\varphi\land\lnot\psi}+\bb{\lnot\varphi\land\psi} \\
		&= \bb{\varphi}-\bb{\varphi\land\psi}+\bb{\psi}-\bb{\varphi\land\psi} \\
		&= \bb{\varphi}+\bb{\psi}-2\bb{\varphi\land\psi}.
	\end{align*}
	In particular, $\mathcal L_{\lnot,\oplus}(P)$ makes $\bb{\varphi}$ always even for $\varphi\in\mathcal L_{\lnot,\oplus}(P)$.
	\item To be able to say something like ``Every dog likes a cat,'' we really mean ``everything which is a dog likes something which is a cat.'' As such, this translates as
	\[\forall x(\op{Dog}(x)\to\exists y(\op{Cat}(y)\land\op{Likes}(x,y))).\]
	\item Set theory might be in scope for extra credit on the final.
	\item Given a set of proposition letters $P$, recall that we defined $\mathcal L(P)$ as being the intersection of all sets containing $P$ and closed under the operation functions
	\[o_\lnot(\varphi)\coloneqq\lnot\varphi\qquad\text{and}\qquad o_\#(\varphi,\psi)\coloneqq(\varphi\op{\#}\psi)\]
	for our various connectives $\#$. As such, to define a function $\mathcal L(P)\to S$, we merely need to lift a function $P\to S$ and argue how it should operate (inductively) on $o_\lnot$ and $o_\#$.
	\item Showing that $\{\varphi_1,\ldots,\varphi_n\}\models\psi$ is a little hard in predicate logic because we need to be able to argue with general models. Namely, truth-table bashing will not work. As an example,
	\[\{\forall xP(x)\}\models P(c).\]
	To see this, we want to argue about all models, so let $\mc M=(D,I)$ be an arbitrary model and $g$ a variable assignment. Then suppose $\mc M\models_g\forall xP(x)$. In particular, it follows that
	\[\mc M\models_{g[x\coloneqq I(c)]}P(x),\]
	so $\bb x^{g[x\coloneqq I(c)]}_I\in I(P)$, which means $I(c)\in I(P)$, giving $\bb c^g_I\in I(P)$, so $\mc M\models_gP(c)$.
	\item We get one two-sided sheet of paper for the final.
	\item The final will emphasize predicate logic, but there will be small bits of propositional logic.
\end{itemize}

\subsection{Overview}
We close class with a high-level organization of the class. Each part of the class was interested in a formal language's syntax and semantics.
\begin{itemize}
	\item The first half of the course was focused on propositional logic, which was the language $\mc L(P)$. Models were valuations, which were rows of a truth table; as we should expect by now, models gave our semantics.

	Lastly, there was a proof theory, using Fitch's natural deduction. Here we made introduction and elimination rules for all of our connectives, adding in reiteration in RAA for completeness.

	\item The second half of the course was focused on predicate logic. Essentially we added terms (variables, constants, functions) to our language and also new quantifiers (and identity) to help build formulae. Models (relative to a variable assignment) still gave rise to our semantics by properly defining truth.

	And of course, we still have natural deduction proofs, where we add rules for identity and the quantifiers.
\end{itemize}