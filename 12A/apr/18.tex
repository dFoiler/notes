% !TEX root = ../notes.tex

We take a pause from lecturing to discuss concepts from the problem set.

\subsection{Concept Review}
Here are some notes from the class's discussion.
\begin{itemize}
	\item To say ``all cats are cute,'' write $\forall x(\op{Cat}(x)\to\op{Cute}(x))$. Namely, if $x$ is a cat, then $x$ is cute; if $x$ is not a cat, we don't assert anything.
	\item To say ``there is a cute cat,'' write $\exists x(\op{Cat}(x)\land\op{Cute}(x))$.
	\item To show that a formula is not valid, we merely have to give a single model. For example,
	\[\models\forall x\forall y\forall z\big((R(x,y)\land R(y,z))\to R(x,z)\big)\]
	is not valid. To see this, take the model $\mc M=(D,I)$ with $D=\{0,1,2\}$. To make the above conditional false, we would like to make the antecedent $R(x,y)\land R(y,z)$ true and $R(x,z)$ false. As such, we set
	\[I(R):=\{\langle0,1\rangle,\langle1,2\rangle\}\]
	which has
	\[\mc M\nmodels_{g}\forall x\forall y\forall z\big((R(x,y)\land R(y,z))\to R(x,z)\big).\]
	Smaller counterexamples do exist: for example, take $\mc M=(\{0,1\},I)$ where $I(R)=\{\langle0,1\rangle,\langle1,0\rangle\}$.
	\item In general, to translate a sentence with a quantified object, we need a quantified variable. For example, to translate ``Some dogs are not liked by all cats,'' we need a quantifier for the existence of dogs and a quantifier for all cats. Concretely, we have
	\[\exists x(\op{Dog}(x)\land\lnot\forall y(\op{Cat}(y)\to\op{Loves}(y,x))).\]
	As a warning, many of the problems on the problem set are ambiguous, so do not be too worried about choosing the exact correct interpretation as long as it is reasonable.
	\item For complicated sentences, it is often helpful to ``chunk'' parts of the sentence and work on them individually. For example, in ``Every cat that is loved by all cats doesn't love any dog,'' we might start by thinking
	\[\forall x\big((\op{Cat}(x)\land\text{$x$ is loved by all cats}))\to\text{$x$ loves no dog}\big).\]
	Then we can break down each piece as follows
	\[\forall x\big((\op{Cat}(x)\land\forall y(\op{Cat}(y)\to\op{Loves}(y,x)))\to\lnot\exists z(D(z)\land\op{Loves}(x,z))\big).\]
	\item In general, it is very awkward to give implications inside an existential quantifier. If we must, the way to think about this is (for example) as
	\[\exists x(P(x)\to Q(x))\equiv\exists x(\lnot P(x)\lor Q(x))\equiv\exists x\lnot P(x)\lor\exists xQ(x).\]
	Namely, to say $\exists x(P(x)\to Q(x))$ can be cleanly split.
	\item Another trick to translate is to translate first into English as an intermediate step (to some easier statement) and then finish. For example, to translate ``Nothing has $A$ except those that have $B$,'' we can be convinced that this means ``Everything has either not $A$ or has $B$.'' So this translates as
	\[\forall x(\lnot A(x)\lor B(x)).\]
	Equivalently, we can write this as $\forall x(A(x)\to B(x))$ or even $\lnot\exists x(A(x)\land\lnot B(x))$.

	So for the last one,
	\begin{center}
		No cat loves a cat who is loved by a dog, except for the cats who love cats
	\end{center}
	can be turned into
	\begin{center}
		All cats $x$ either do not love a cat who is loved by a dog or $x$ loves cats.
	\end{center}
	It might be hard to formalize ``$x$ loves cats,'' but we do our best.

	\item We might want to add quantifiers like ``most'' to our first-order logic so that we can say things like ``most cats are cute.''
\end{itemize}