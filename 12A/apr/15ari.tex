% !TEX root = ../notes.tex

We now transition into discussing arithmetic, using first-order logic. The idea is to add a few axioms to be able to discuss actual mathematics.

\subsection{Peano Arithmetic}
Let's see an example of an arithmetic proof, to start off.
\begin{lemma}
	Let $n$ be a nonnegative integer. Then $n$ is even or odd.
\end{lemma}
\begin{proof}
	We proceed by induction on $n$. As our base case, we note that $n=0$ has $0=2\cdot0$, so $0$ is even, so $0$ is even or odd.

	We now show the inductive step. Suppose that $n$ is even or odd, and we will show $n+1$ is even or odd. We have two cases.
	\begin{itemize}
		\item If $n$ is even, then write $n=2k$. Thus, $n+1=2k+1$, so $n+1$ is odd, so $n+1$ is even or odd.
		\item If $n$ is odd, then write $n=2k+1$. Thus, $n+1=2(k+1)$, so $n+1$ is even, so $n+1$ is even or odd.
	\end{itemize}
	In all cases we were able to conclude that $n+1$ is even or odd, so we are done.
\end{proof}
The main point is that we are able to formalize all parts of this proof except for the induction. So we will build an induction.

For our arithmetic, we will build a language with the following tools.
\begin{definition}[Peano arithmetic language]
	The language of Peano arithmetic has the following data.
	\begin{itemize}
		\item No predicate symbols.
		\item One constant symbol $\dot0$, sometimes written $0$.
		\item One unary function symbol $S$.
		\item Two binary function symbols $\dot+$ and $\dot\times$, sometimes written $+$ and $\times$.
	\end{itemize}
\end{definition}
And here are some conventions.
\begin{itemize}
	\item $S(0)$ will be denoted $1$, $S(S(0))$ will be denoted $2$, and so on.
	\item We might also omit $\times$ and write (for example, $2k$) for $2\times k$.
	\item We'll write $=$ for $\dot=$ for brevity.
\end{itemize}
For our arithmetic (which will be called Peano arithmetic) is that we will start with some basic axioms to avoid an infinite regress. Here they are.
\begin{axiom}[Peano arithmetic]
	Peano arithmetic takes the following axioms. We begin by pinning down $S$.
	\begin{enumerate}[label=(S\arabic*)]
		\item $\forall x(\lnot S(x)=0)$. In other words, $0$ is not the successor (of any natural number).
		\item $\forall x\forall y((S(x)=S(y))\to (x=y))$. In other words, taking successors is one-to-one.
	\end{enumerate}
	Next we talk about how $+$ behaves, more or less recursively.
	\begin{enumerate}[label=(A\arabic*)]
		\item $\forall x(x+0=x)$. In other words, adding $0$ doesn't do anything.
		\item $\forall x\forall y(x+S(y)=S(x+y))$. This codifies the associativity $x+(y+1)=(x+y)+1$ and defines $+$.
	\end{enumerate}
	Here is multiplication, again recursively.
	\begin{enumerate}[label=(M\arabic*)]
		\item $\forall x(x\times0=0)$. In other words, multiplying by $0$ gets $0$.
		\item $\forall x\forall y(x\times S(y))=(x\times y)+x$. Expanding out, we are roughly saying that $x\times(y+1)=x\times y+x$, codifying the notion that multiplication is repeated addition.
	\end{enumerate}
	Lastly, we need induction.
	\begin{enumerate}[label=(IND)]
		\item For any formula $\varphi$, we have $(\varphi^x_0\land\forall x(\varphi\to\varphi^x_{S(x)})\to\forall x\varphi$.

		In other words, if we can show $\varphi$ ``holds'' for $0$ and that $\varphi$ ``holding'' for $x$ implies that $\varphi$ ``holding'' for $x+1$, then we may deduce $\forall x\varphi$.
	\end{enumerate}
\end{axiom}
Notably, we needed to talk about function symbols, constant symbols, and quantifiers in our discussion of first-order logic to be able to codify the above.
\begin{remark}
	Even though our axiom list is infinite (namely, induction gives an axiom for each formula $\varphi$), our axiom list is at least computable: all the axioms provided by induction have a very specific form.
\end{remark}
Let's close with an example.
\begin{exe} \label{exe:bettersucc}
	We show $\forall x(x+1=S(x))$.
\end{exe}
\begin{proof}
	We have the following.
	\begin{align*}
		\begin{nd}
			\hypo{addident}{\forall x(x+0=x)}
			\hypo{succ}{\forall x(x+S(y)=S(x+y))}
			\open[]
				\hypo{beg}{\boxed c}
				\have{addidc}{c+0=c} \by{L$\forall$Elim}{addident}
				\have{succc}{c+1=S(c+0)} \by{L$\forall$Elim}{succ}
				\have{end}{c+1=S(c)} \by{$=$Elim}{addidc,succc}
			\close
			\have{}{\forall x(x+1=S(x))} \by{$\forall$Intro}{beg-end}
		\end{nd}
	\end{align*}
	This finishes. Notably, we didn't even need induction.
\end{proof}
\begin{remark}
	Notably, we can now proof $\forall$ statements by either $\forall$ introduction or by induction.
\end{remark}
\begin{remark}
	Without any axioms, the formula ``$\forall x(x+1=S(x))$'' need not be true. Our axioms are now giving our semantics.
\end{remark}