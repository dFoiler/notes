\documentclass[../notes.tex]{subfiles}
\graphicspath{{\subfix{../figs/}}}

\begin{document}

% !TEX root = ../notes.tex

Welcome back, everyone.

\subsection{Limits of Peano Arithmetic}
It turns out that not everything is possible in Peano arithmetic. Nonetheless, here are some things that we can define.
\begin{itemize}
	\item Evenness and oddness can be defined.
	\item The less than or equal to relation $\le$ can be defined as $x\le y$ as $\exists z(x+z=y)$.
	\item Primality can be defined as $\op{Prime}(x)$, meaning
	\[x\ne1\land\forall y\forall z(x=yz\to(y=1\lor z=1)).\]
	\item It also turns out that we can define exponentiation in the following sense: we can make a formula $\op{Exp}(x,y,z)$ which is true if and only if $x^y=z$. However, this is hard. More frequently, we just add the axioms
	\[\forall x\left(x^0=S(1)\right)\quad\text{and}\qquad\forall x\forall y\left(x^{S(y)}=x^y\cdot x\right)\]
	to our arithmetic.
\end{itemize}
One can also formalize many results into Peano arithmetic, such as the following.
\begin{itemize}
	\item One can prove that there are infinitely many prime numbers.
	\item Some believe that the proof of Fermat's last theorem
	\[\forall x\forall y\forall z\left(2<n\to x^n+y^n=z^n\right)\]
	can be formalized in Peano arithmetic. However, this would take a while.
\end{itemize}
To see our limits, we need some terminology.
\begin{definition}[Negation complete]
	A set of axioms $\Sigma$ is \textit{negation complete} if and only if, for every sentence $\varphi$, we either have $\Sigma\vdash\varphi$ or $\Sigma\vdash\lnot\varphi$. In other words, $\Sigma$ can prove $\varphi$ or can prove $\lnot\varphi$. Otherwise, $\Sigma$ is \textit{negation incomplete}.
\end{definition}
And here is our corresponding result.
\begin{theorem}[G\"odel] \label{thm:godel}
	If Peano arithmetic is consistent (i.e., one cannot derive contradiction), then Peano arithmetic is negation incomplete.
\end{theorem}
\begin{remark}
	Note that there is no notion of truth in this explanation: we are not claiming that Peano arithmetic is unable to prove true statements. What we are saying is that there is a formula $\varphi$ such that Peano arithmetic cannot prove $\varphi$ nor $\lnot\varphi$.
\end{remark}
The statement of \autoref{thm:godel} might appear weak: perhaps we need more axioms. However, there is a stronger notion.
\begin{remark}
	Most people believe that Peano arithmetic is consistent (e.g., there are proofs of its consistency from other principles which we believe), so we will just assume this.
\end{remark}

\subsection{Incompleteness of Peano Arithmetic}
Let's give an example to exhibit \autoref{thm:godel}. We will need to discuss the hydra game.
\begin{definition}[Hydra]
	A \textit{hydra} is a finite rooted tree.
\end{definition}
\begin{example}
	The following is a hydra.
	% https://q.uiver.app/?q=WzAsNyxbMiwyLCJcXGJ1bGxldCJdLFsxLDEsIlxcYnVsbGV0Il0sWzIsMSwiXFxidWxsZXQiXSxbMCwwLCJcXGNpcmMiXSxbMSwwLCJcXGNpcmMiXSxbMywwLCJcXGNpcmMiXSxbMiwwLCJcXGNpcmMiXSxbMywxLCIiLDAseyJzdHlsZSI6eyJoZWFkIjp7Im5hbWUiOiJub25lIn19fV0sWzQsMiwiIiwyLHsic3R5bGUiOnsiaGVhZCI6eyJuYW1lIjoibm9uZSJ9fX1dLFs2LDIsIiIsMCx7InN0eWxlIjp7ImhlYWQiOnsibmFtZSI6Im5vbmUifX19XSxbNSwyLCIiLDAseyJzdHlsZSI6eyJoZWFkIjp7Im5hbWUiOiJub25lIn19fV0sWzEsMCwiIiwwLHsic3R5bGUiOnsiaGVhZCI6eyJuYW1lIjoibm9uZSJ9fX1dLFsyLDAsIiIsMCx7InN0eWxlIjp7ImhlYWQiOnsibmFtZSI6Im5vbmUifX19XV0=
	\[\begin{tikzcd}
		\circ & \circ & \circ & \circ \\
		& \bullet & \bullet \\
		&& \bullet
		\arrow[no head, from=1-1, to=2-2]
		\arrow[no head, from=1-2, to=2-3]
		\arrow[no head, from=1-3, to=2-3]
		\arrow[no head, from=1-4, to=2-3]
		\arrow[no head, from=2-2, to=3-3]
		\arrow[no head, from=2-3, to=3-3]
	\end{tikzcd}\]
\end{example}
Now, here is the game we will play with the hydra.
\begin{enumerate}
	\item Every round, we choose some leaf $x$ other than the root, and we chop off its connection to its parent $y$ below.
	\item Then the hydra responds.
	\begin{itemize}
		\item If $y$ is the root, nothing happens.
		\item If there is a node $z$ below $y$, then the hydra regrows $n$ copies of the subtree from $y$ (with $x$ chopped off), all connected to $z$.
	\end{itemize}
\end{enumerate}
For example, we might chop off $x$ as follows, on round $n=2$.
% https://q.uiver.app/?q=WzAsNyxbMiwyLCJ6XFwsXFxidWxsZXRcXHBoYW50b217XFwsen0iXSxbMSwxLCJcXGJ1bGxldCJdLFsyLDEsInlcXCxcXGJ1bGxldFxccGhhbnRvbXt5XFwsfSJdLFswLDAsIlxcY2lyYyJdLFsxLDAsInhcXCxcXGNpcmNcXHBoYW50b217XFwseH0iXSxbMywwLCJcXGNpcmMiXSxbMiwwLCJcXGNpcmMiXSxbMywxLCIiLDAseyJzdHlsZSI6eyJoZWFkIjp7Im5hbWUiOiJub25lIn19fV0sWzQsMiwiIiwyLHsic3R5bGUiOnsiaGVhZCI6eyJuYW1lIjoibm9uZSJ9fX1dLFs2LDIsIiIsMCx7InN0eWxlIjp7ImhlYWQiOnsibmFtZSI6Im5vbmUifX19XSxbNSwyLCIiLDAseyJzdHlsZSI6eyJoZWFkIjp7Im5hbWUiOiJub25lIn19fV0sWzEsMCwiIiwwLHsic3R5bGUiOnsiaGVhZCI6eyJuYW1lIjoibm9uZSJ9fX1dLFsyLDAsIiIsMCx7InN0eWxlIjp7ImhlYWQiOnsibmFtZSI6Im5vbmUifX19XV0=
\[\begin{tikzcd}
	\circ & {x\,\circ\phantom{\,x}} & \circ & \circ \\
	& \bullet & {y\,\bullet\phantom{y\,}} \\
	&& {z\,\bullet\phantom{\,z}}
	\arrow[no head, from=1-1, to=2-2]
	\arrow[no head, from=1-2, to=2-3]
	\arrow[no head, from=1-3, to=2-3]
	\arrow[no head, from=1-4, to=2-3]
	\arrow[no head, from=2-2, to=3-3]
	\arrow[no head, from=2-3, to=3-3]
\end{tikzcd}\]
This gives the following.
% https://q.uiver.app/?q=WzAsMTIsWzIsMiwielxcLFxcYnVsbGV0XFxwaGFudG9te1xcLHp9Il0sWzEsMSwiXFxidWxsZXQiXSxbMiwxLCJ5XFwsXFxidWxsZXRcXHBoYW50b217eVxcLH0iXSxbMCwwLCJcXGNpcmMiXSxbMywwLCJcXGNpcmMiXSxbMiwwLCJcXGNpcmMiXSxbNCwxLCJcXGJ1bGxldCJdLFs2LDEsIlxcYnVsbGV0Il0sWzQsMCwiXFxjaXJjIl0sWzUsMCwiXFxjaXJjIl0sWzYsMCwiXFxjaXJjIl0sWzcsMCwiXFxjaXJjIl0sWzMsMSwiIiwwLHsic3R5bGUiOnsiaGVhZCI6eyJuYW1lIjoibm9uZSJ9fX1dLFs1LDIsIiIsMCx7InN0eWxlIjp7ImhlYWQiOnsibmFtZSI6Im5vbmUifX19XSxbNCwyLCIiLDAseyJzdHlsZSI6eyJoZWFkIjp7Im5hbWUiOiJub25lIn19fV0sWzEsMCwiIiwwLHsic3R5bGUiOnsiaGVhZCI6eyJuYW1lIjoibm9uZSJ9fX1dLFsyLDAsIiIsMCx7InN0eWxlIjp7ImhlYWQiOnsibmFtZSI6Im5vbmUifX19XSxbMCw2LCIiLDAseyJzdHlsZSI6eyJoZWFkIjp7Im5hbWUiOiJub25lIn19fV0sWzYsOCwiIiwwLHsic3R5bGUiOnsiaGVhZCI6eyJuYW1lIjoibm9uZSJ9fX1dLFs2LDksIiIsMCx7InN0eWxlIjp7ImhlYWQiOnsibmFtZSI6Im5vbmUifX19XSxbNywxMCwiIiwwLHsic3R5bGUiOnsiaGVhZCI6eyJuYW1lIjoibm9uZSJ9fX1dLFs3LDExLCIiLDIseyJzdHlsZSI6eyJoZWFkIjp7Im5hbWUiOiJub25lIn19fV0sWzAsNywiIiwyLHsic3R5bGUiOnsiaGVhZCI6eyJuYW1lIjoibm9uZSJ9fX1dXQ==
\[\begin{tikzcd}
	\circ && \circ & \circ & \circ & \circ & \circ & \circ \\
	& \bullet & {y\,\bullet\phantom{y\,}} && \bullet && \bullet \\
	&& {z\,\bullet\phantom{\,z}}
	\arrow[no head, from=1-1, to=2-2]
	\arrow[no head, from=1-3, to=2-3]
	\arrow[no head, from=1-4, to=2-3]
	\arrow[no head, from=2-2, to=3-3]
	\arrow[no head, from=2-3, to=3-3]
	\arrow[no head, from=3-3, to=2-5]
	\arrow[no head, from=2-5, to=1-5]
	\arrow[no head, from=2-5, to=1-6]
	\arrow[no head, from=2-7, to=1-7]
	\arrow[no head, from=2-7, to=1-8]
	\arrow[no head, from=3-3, to=2-7]
\end{tikzcd}\]
Now, we win this game if and only if we eventually chop off the root. The issue, of course, is that winning the game will take a while because the hydra gets very big as the step counter $n$ increases.

So the question is if we can create an algorithm to slay the hydra.
\begin{example}
	We might try to simply chop off the leftmost head of the hydra.
\end{example}
\begin{example}
	We might try to chop off a random node furthest from the root.
\end{example}
However, it is actually not that hard to beat the hydra.
\begin{theorem}[Kirby and Paris] \label{thm:hydra}
	Any algorithm to play the hydra game will always win against any hydra.
\end{theorem}
The point is that \autoref{thm:hydra} can be formalized in Peano arithmetic; roughly speaking, the point is that we can encode algorithms, finite trees, and games all with nonnegative integers. To believe this, computers essentially do this already with binary.

However, we have the following concrete incompleteness.
\begin{theorem}[Kirby and Paris]
	Peano arithmetic, if consistent, cannot prove \autoref{thm:hydra}.
\end{theorem}
\begin{remark}
	Even though we cannot prove \autoref{thm:hydra}, it is still true (say, in ZFC).
\end{remark}
\begin{remark}
	It also turns out that consistency of Peano arithmetic is not provable from inside Peano arithmetic. However, this consistency can be encoded into Peano arithmetic (via integers, as usual), so this is another witness to incompleteness.
\end{remark}
Again, one might complain that Peano arithmetic is simply too weak, and we just need to add some axioms to fix it. However, G\"odel disagrees.
\begin{theorem}[G\"odel]
	Let $T$ be any set of sentence containing all sentences that Peano arithmetic can prove. Further, suppose that there is an algorithm to determine if a sentence is in $T$. Then $T$ is negation incomplete.
\end{theorem}
\begin{remark}
	Do note that the axioms for Peano arithmetic are algorithmically recognizable---simply check the fixed sentences other than induction by hand and then check the ``format'' for the induction schema.
\end{remark}
\begin{remark}
	The algorithmic requirement might seem awkward, but it is necessary, for otherwise we could just set $T$ to be the set of sentences which are modeled by $\NN$. It follows that there is no algorithm to determine if a sentence is modeled by $\NN$.
\end{remark}
Thus, Peano arithmetic cannot be fixed by merely adding axioms.

\subsection{Set Theory Appetizer}
To be able to talk about ``real'' mathematics, we will end the semester by discussing set theory, which is the actual foundation for mathematics. Roughly speaking, we are claiming the following.
\begin{itemize}
	\item All mathematical statements can be translated into statements about sets.
	\item Proving statements in mathematics is equivalent to proving them in our set theory.
\end{itemize}

\end{document}