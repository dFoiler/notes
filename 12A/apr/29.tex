\documentclass[../notes.tex]{subfiles}
\graphicspath{{\subfix{../figs/}}}

\begin{document}

% !TEX root = ../notes.tex

Welcome to the last day of class.

\subsection{Zermelo--Fraenkel Speed-Run}
We're going to try to cover a lot of ground, praying for a pedagogical miracle. Last class we brought in our language of set theory; this class we are going to take it further. The only object above propositional logic in our language of set theory is the relation $\dot\in$, which we think of as membership.

To reason about out sets, we are going to require some axioms for our set theory. So let's go through some axioms.
\begin{axiom}[Existentionality]
	Two sets are the same if and only if they have the same elements:
	\[\forall x\forall y(x\doteq y\liff\forall w(w\dotin x\liff w\dotin y)).\]
\end{axiom}
\begin{example}
	We will think that $\{\emp,\{\emp\}\}=\{\{\emp\},\emp\}=\{\{\emp\},\emp,\emp\}$.
\end{example}
So we can talk about equality of sets. Next we need sets to talk about. One way we might want to do this is by asserting any property $\varphi$ can dictate a set by filtering through all elements and check $\varphi$.
\begin{example}
	We can build the set of all prime numbers by forcing $\varphi$ to assert that our elements are prime.
\end{example}
Thus, we might want formulae $\varphi(z)$ with a free variable $z$ to dictate sets by writing
\[\exists s\forall z(z\dotin s\liff\varphi(z)).\tag{$*$}\label{eq:comprehension}\]
Namely, there is a set $s$ made up of the $z$ which have $\varphi(z)$.
\begin{example}
	Let's make ordered pairs. We have
	\[\exists s\forall z(z\dotin s\liff(z\doteq a\lor z\doteq b)).\]
	This makes the ordered pair $\{a,b\}$.
\end{example}
Similarly, we can build power sets and so on.

However, \autoref{eq:comprehension} is a problem because it derives contradiction. The idea is that it is impossible to have a town where a barber shaves exactly the people who do not shave themselves: the barber must either shave themself or not shave themself. So we have the following.
\begin{proposition}
	The statement \autoref{eq:comprehension} derives contradiction.
\end{proposition}
\begin{proof}
	Fix $\varphi(z)\coloneqq\lnot(z\dotin z)$, so comprehension gives
	\[\exists s\forall z(z\dotin s\liff\lnot(z\dotin z)).\]
	But then $s\dotin s$ is equivalent to $s\dot{{}\notin{}}s$, so we have our contradiction. Formally, we have the following.
	\begin{align*}
		\begin{nd}
			\open[]
				\hypo{hypo}{\exists s\forall z(z\dotin s\liff\lnot(z\dotin z))}
				\open[]
					\hypo{bad}{\forall z(z\dotin r\liff\lnot(z\dotin z))\qquad\boxed r}
					\have{iff}{r\dotin r\liff\lnot(r\dotin r)} \by{$\forall$Elim}{bad}
					\open[]
						\hypo{in}{r\dotin r}
						\have{iffagain}{r\dotin r\liff\lnot(r\dotin r)} \by{Reit}{iff}
						\have{innotin}{\lnot(r\dotin r)} \by{$\liff$Elim}{in,iffagain}
						\have{incontra}{(r\dotin r)\land\lnot(r\dotin r)} \by{$\land$Intro}{innotin}
					\close
					\have{notin}{\lnot(r\dotin r)} \by{$\lnot$Intro}{in-incontra}
					\have{butin}{r\dotin r} \by{$\liff$Elim}{iff,notin}
					\have{contra}{(r\dotin r)\land\lnot(r\notin r)} \by{$\land$Intro}{notin,butin}
				\close
			\close
			\have{}{\lnot\exists s\forall z(z\dotin s\liff\lnot(z\dotin z))}
		\end{nd}
	\end{align*}
	Thus, we have derived the negation of \autoref{eq:comprehension} in this case.
\end{proof}
The point is that we need to be more careful about how we make our sets: not all properties are okay. Namely, we want to restrict our $\varphi$ in \autoref{eq:comprehension}. So here are some sets we'll make.
\begin{axiom}[Empty set]
	There is an empty set: $\exists s\forall x(\lnot(x\dotin s))$.
\end{axiom}
\begin{axiom}[Unordered pair]
	There is an ordered pair: $\forall x\forall y\exists s\forall z(z\dotin s\liff(z\doteq x\land z\doteq y)$.
\end{axiom}
\begin{axiom}[Union]
	We can take the union of some set of sets:
	\[\forall x\exists s\forall z(z\dotin s\liff\exists y(y\dotin x\land z\dotin y)).\]
	Here $y$ is to be thought of as ``looping'' through the various sets within $x$.
\end{axiom}
\begin{axiom}[Power set]
	We can build power sets by writing
	\[\forall x\exists s\forall z(z\dot{{}\in{}} s\liff z\dot{{}\subseteq{}} s).\]
	Namely, $s$ contains all the various subsets $z$ of $x$.
\end{axiom}
\begin{axiom}[Separation]
	Given a set $x$, we can filter out a subset from a property: given a formula $\varphi$ with a free variable $z$ (but $s$ is not free in $z$),
	\[\forall x\exists s\forall z(z\dotin s\liff(z\dotin s\land\varphi(z)).\]
\end{axiom}
We will also want to make an infinite set.
\begin{axiom}[Infinity]
	There is an infinite set, closed under successor: $\exists s(\emp\dotin s\land\forall x(x\dotin s\to x\cup\{x\}\dotin s)$.
\end{axiom}
These are most of the axioms in set theory, with a few others. They make up Zermelo--Fraenkel set theory, and they can create a foundation for all current mathematics.

\subsection{Sizes of Infinity}
Zermelo--Fraenkel set theory is going to permit infinities of different sizes in a non-paradoxical way. The key point is that the definition of ``size'' is fairly technical.
\begin{definition}[Bijection]
	Given two sets $X$ and $Y$, a bijection $f\colon X\to Y$ is an onto, one-to-one function between $X$ and $Y$.
\end{definition}
The point is that bijections should assert that sets have the same size without having to enumerate all the elements.
\begin{definition}[Equinumerous]
	Two sets $X$ and $Y$ are \textit{equinumerous} if and only if there is a bijection between them. We might also say that they have the ``same size.''
\end{definition}
\begin{remark}
	If $X\subsetneq Y$ and $X$ and $Y$ are finite, then $X$ and $Y$ have different sizes. However, $\NN$ and $2\NN$ have the same size because we can biject $\NN\to2\NN$ by sending $n\mapsto2n$.
\end{remark}
One can even put $\QQ$ and $\NN$ into bijection, by moving diagonally and across through the following table.
% https://q.uiver.app/?q=WzAsMTAsWzAsMCwiMS8xIl0sWzEsMCwiMS8yIl0sWzAsMSwiMi8xIl0sWzIsMCwiMS8zIl0sWzMsMCwiXFxjZG90cyJdLFsxLDEsIjIvMiJdLFswLDIsIjMvMSJdLFsyLDEsIlxcZGRvdHMiXSxbMSwyLCJcXGRkb3RzIl0sWzAsMywiXFx2ZG90cyJdLFswLDFdLFsxLDJdLFsyLDZdLFs2LDVdLFs1LDNdLFszLDRdXQ==&macro_url=https%3A%2F%2Fraw.githubusercontent.com%2FdFoiler%2Fnotes%2Fmaster%2Fnir.tex
\[\begin{tikzcd}
	{1/1} & {1/2} & {1/3} & \cdots \\
	{2/1} & {2/2} & \ddots \\
	{3/1} & \ddots \\
	\vdots
	\arrow[from=1-1, to=1-2]
	\arrow[from=1-2, to=2-1]
	\arrow[from=2-1, to=3-1]
	\arrow[from=3-1, to=2-2]
	\arrow[from=2-2, to=1-3]
	\arrow[from=1-3, to=1-4]
\end{tikzcd}\]
However, it turns out that there are more real numbers than natural numbers.
\begin{proposition}
	There are strictly more real numbers than natural numbers.
\end{proposition}
\begin{proof}
	We show that any injective function $f\colon\NN\to\RR$ is not surjective, thus implying there is no bijection. Now, write out the decimal expansion of any $d(n)$ for $n\in\NN$ by
	\[f(n)\coloneqq a_{n}.d_{n,0}d_{n,1}d_{n,2}\ldots.\]
	As such, we construct a real number as having decimal expansion given by
	\[d_n\coloneqq\begin{cases}
		4 & d_{n,n}=5, \\
		5 & d_{n,n}\ne 4.
	\end{cases}\]
	Then the number $\alpha\coloneqq0.d_0d_1d_2\ldots$ is not in the image of $f$ because $\alpha$ and $f(n)$ differ in the $n$th decimal digit, implying $f(n)\ne\alpha$ for each $n\in\NN$. This finishes.
\end{proof}
\begin{remark}
	From here, one can build larger and larger infinities. However, we cannot tell the difference between these larger infinities. For example, the Continuum hypothesis (which asks if there is a set $X$ of size strictly between $\NN$ and $\RR$) cannot be settled by Zermelo--Fraenkel set theory. Some questions must be left unanswered.
\end{remark}

\end{document}