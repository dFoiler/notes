% !TEX root = ../notes.tex

Welcome back, everyone.

\subsection{Existential Elimination}
Last class we were talking about existential elimination. The idea is that a statement
\[\exists x\varphi\]
allows us to conjure a ``witness'' constant $n$ for $x$ satisfying $\varphi$ and then proceed with the proof. Here is the formal rule.
\begin{defihelper}[$\exists$ Elimination] \nirindex{Existential Elimination@$\exists$Elim}
	Suppose we have a statement $\exists x\varphi$. If granted a constant $c$ (not in the original language $\mc L$) with a subproof starting with $\varphi^x_c$ and ending with $\psi$ (which does not contain $c$!), then we can add $\psi$ to the main line outside the subproof and cite $\exists$ elimination.
\end{defihelper}
Diagrammatically, this looks like the following.
\begin{align*}
	\begin{nd}
		\have[\vdots]{}{\vdots}
		\have[j]{exists}{\exists x\varphi}
		\open[]
			\hypo{beg}{\varphi^x_c\quad\boxed c}
			\have[\vdots]{}{\vdots}
			\have[k]{end}{\psi}
		\close
		\have{}{\psi} \by{$\exists$Elim}{j,beg-end}
	\end{nd}
\end{align*}
And let's see an example.
\begin{exe}
	We derive $\exists xP(x)\lor\exists xQ(x)$ from $\exists x(P(x)\lor Q(x))$.
\end{exe}
\begin{proof}
	We have the following.
	\begin{align*}
		\begin{nd}
			\hypo{start}{\exists x(P(x)\lor Q(x))}
			\open[]
				\hypo{beg}{P(c)\lor Q(c)\quad\boxed c}
				\open[]
					\hypo{p}{P(c)}
					\have{exp}{\exists xP(x)} \by{$\exists$Intro}{p}
					\have{end1}{\exists xP(x)\lor\exists xQ(x)} \by{$\lor$Intro}{exp}
				\close
				\open[]
					\hypo{q}{Q(c)}
					\have{exq}{\exists xP(x)} \by{$\exists$Intro}{q}
					\have{end2}{\exists xP(x)\lor\exists xQ(x)} \by{$\lor$Intro}{exq}
				\close
				\have{end}{\exists xP(x)\lor\exists xQ(x)} \by{$\lor$Elim}{beg,p-end1,q-end2}
			\close
			\have{}{\exists xP(x)\lor\exists xQ(x)} \by{$\exists$Elim}{start,beg-end}
		\end{nd}
	\end{align*}
	This finishes.
\end{proof}
\begin{remark}
	Again, this is like $\lor$ elimination as a proof-by-cases for an arbitrary number of cases. To see this, we note we can derive $\psi$ from
	\[P(c_1)\lor\cdots\lor P(c_n)\]
	by doing a proof that $P(c_i)$ would imply $\psi$ for each case, and from here we can use $\lor$ elimination. The ability to do all cases at once for some ``arbitrary'' $c$ (to show $\psi$ from $P(c)$) is what we've done above. In some sense, we are doing ``all'' of the cases at the same time, via some ``template'' proof.
\end{remark}
\begin{remark}
	As such, the above example is more or less like $\lor$ distributing over $\lor$, using the intuition that $\exists$ is a very large $\lor$.
\end{remark}
\begin{remark}
	We also have $(\forall xP(x)\land\forall xQ(x))\liff(\forall x(P(x)\land Q(x))$.
\end{remark}
This finishes our discussion of natural deduction.