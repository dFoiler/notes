\documentclass[../notes.tex]{subfiles}
\graphicspath{{\subfix{../figs/}}}

\begin{document}

% !TEX root = ../notes.tex

We are still talking about semantics. Today we are focusing on satisfiability.

\subsection{Satisfiability}
Satisfiability is dual to validity: if there is any way for the formula to be true, then the formula is satisfiable.
\begin{nex}
	The formula $p\land\lnot p$ is unsatisfiable: it is always false.
\end{nex}
To help build some intuition, we note that we have some feeling for when multiple propositions may be true or false.
\begin{exe}
	Suppose we have the following statements of agencies responsible for an action.
	\begin{enumerate}
		\item Agency A or Agency B is responsible.
		\item It's not the case that both Agencies A and B are responsible.
		\item If Agency A is responsible, then the money is coming from fun X.
		\item If Agency A is not responsible, then the money is coming from fun Y.
		\item If Agency B is responsible, then the money is not coming from fun Y.
		\item If Agency B is responsible, then the money is not coming from fun X.
	\end{enumerate}
	We show not all of these can be true simultaneously; namely, this set of premises is ``inconsistent.''
\end{exe}
\begin{proof}
	We can translate as follows into our formal language. Here $a$ (resp., $b$) is ``Agency A (resp., B) is responsible'' and $x$ (resp., $y$) is ``the money is coming from X (resp., Y).''
	\begin{enumerate}
		\item $a\lor b$.
		\item $\lnot(a\land b)$.
		\item $a\to x$.
		\item $\lnot a\to y$.
		\item $b\to\lnot y$.
		\item $\lnot b\to\lnot x$.
	\end{enumerate}
	There is no valuation will make all of these true. If $a$ were true, then $x$; but also $a$ means $\lnot b$, which means $\lnot x$. Conversely, if $a$ were not true, then $y$; but also $\lnot a$ means $b$, which means $\lnot y$. So all cases don't make sense.
\end{proof}
Here are our formal definitions.
\begin{definition}[Inconsistent]
	A set of formulae is \textit{inconsistent} if and only if one can prove contradiction from the formulae, using some proof system.
\end{definition}
\begin{definition}[Satisfiable]
	A set of formulae $\{\varphi_1,\ldots,\varphi_n\}\subseteq\mathcal L(P)$ is \textit{satisfiable} if and only if there is a valuation $V:P\to\{0,1\}$ which satisfies all the formulae. A set of formulae is \textit{unsatisfiable} if and only if it is not satisfiable.

	We say that a single formula $\varphi$ is \textit{(un)satisfiable} if and only if $\{\varphi\}$ is (un)satisfiable. In other words, there is a valuation $V$ such that $V\models\varphi$.
\end{definition}
\begin{example}
	The formula $p\land\lnot q$ is satisfiable: $p=1$ and $q=0$.
\end{example}
\begin{example}
	The formula $p\land\lnot p$ is unsatisfiable: $p=0$ and $p=1$ both give $p\land\lnot p$ false.
\end{example}
We could also say that $\{\varphi_1,\ldots,\varphi_n\}$ is satisfiable if and only if $\varphi_1\land\cdots\land\varphi_n$ is satisfiable.
\begin{remark}
	In fact, it will be true that any unsatisfiable set of formulae will be able to derive contradiction from our proof system. So unsatisfiability will be equivalent to inconsistent, though this is not immediately obvious.
\end{remark}

Satisfiability in fact is more directly dual to validity.
\begin{proposition} \label{prop:dualvalid}
	Fix $\varphi\in\mathcal L(P)$. Then $\varphi$ is satisfiable if and only if $\lnot\varphi$ is not valid.
\end{proposition}
\begin{proof}
	We have $\varphi$ is satisfiable if and only if there exists some $V:P\to\{0,1\}$ such that $V\models\varphi$.

	Conversely, $\lnot\varphi$ is not valid if and only if there exists a valuation $V:P\to\{0,1\}$ such that $\hat V(\lnot\varphi)=0$ if and only if $\hat V(\varphi)=1$.
\end{proof}
\begin{example}
	The expression $p\land\lnot p$ is unsatisfiable, so $\lnot(p\land\lnot p)$ is valid.
\end{example}
\begin{corollary}
	Fix $\varphi\in\mathcal L(P)$. Then $\varphi$ is valid if and only if $\lnot\varphi$ is not satisfiable.
\end{corollary}
\begin{proof}
	This is the contraposition of \autoref{prop:dualvalid}.
\end{proof}
To review, we have defined the following semantic notions.
\begin{itemize}
	\item Valid argument forms.
	\item Valid formulae.
	\item Satisfiable sets of formulae.
	\item Satisfiable formulae.
\end{itemize}
In fact, we know that studying any one of these can study any other, assuming finiteness.

\subsection{Infinite Arguments}
We quickly remark that we can generalize some of our notions to the infinite case.
\begin{definition}[Satisfiable]
	Any set $S\subseteq\mathcal L(P)$ is \textit{satisfiable} if and only if there is a valuation $V:P\to\{0,1\}$ satisfying all of them.
\end{definition}
However, this is no longer the same as some formula
\[\bigwedge_{\varphi\in S}\varphi\]
where we and everything together: all of our formulae have finite length. One could generalize formulae to allow infinite formulae, but we won't do so here.

We remark while we are here that we have the following notion of compactness.
\begin{theorem}[Compactness]
	A set $S$ of formulae is satisfiable if and only if every finite subset of $S$ is satisfiable.
\end{theorem}
\begin{proof}
	The forwards direction is clear: if a valuation satisfies $S$, then the valuation will satisfy any subset.

	The backwards direction is significantly harder. In essence, one shows that a set of formula is unsatisfiable if and only if one is able to derive contradiction. However, deriving contradiction is a finite process which only takes finitely many propositions, so this finite subset would also be unsatisfiable.
\end{proof}
\begin{example}
	Any finite graph can be $4$-colored, so by compactness, any graph (possibly infinite) can be $4$-colored.
\end{example}
Next class we will start talking about economy of language: exactly what connectives do we need to construct all possible truth functions? It will turn out that we do not need many.

\end{document}