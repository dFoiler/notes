% !TEX root = ../notes.tex

Here we go.

\subsection{Validity for Argument Forms}
Last time we left off talking about the validity of an argument form. We had the following definition.
\begin{definition}[Valid]
	An argument form $(\{\varphi_1,\ldots,\varphi_n\},\psi)$ is \textit{valid} if and only if each valuation $V:P\to\{0,1\}$ has
	\[\hat V(\varphi_1)=\cdots=\hat V(\varphi_n)=1\implies\hat V(\psi)=1.\]
	We notate this by $\varphi_1,\ldots,\varphi_n\models\psi$ and say ``$\psi$ is a \textit{consequence} of $\{\varphi_1,\ldots,\varphi_n\}$.''
\end{definition}
\begin{remark}
	Importantly, a computer could always decide validity in finite time, essentially by making some huge truth table. In other words, determining validity is ``decidable.''
\end{remark}
Let's see an example.
\begin{exe}
	We show that $\{\lnot(p\land q),p\}\models\lnot q$.
\end{exe}
\begin{proof}
	We reason using the equations for the connectives. Namely, suppose $V:\{p,q\}\to\{0,1\}$ is a valuation such that $\hat V(\lnot(p\land q))=\hat V(p)=1$. Now, we see
	\[1=\hat V(\lnot(p\land q))=1-\hat V((p\land q))=1-\min\{\hat V(p),\hat V(q)\}.\]
	Thus, $\min\{\hat V(p),\hat V(q)\}=0$, so one of $\hat V(p)=0$ or $\hat V(q)=0$. But $\hat V(p)=1$, so $\hat V(q)=0$ is forced instead. Thus,
	\[\hat V(\lnot q)=1-\hat V(q)=1-0=1,\]
	which is what we wanted.
\end{proof}
In fact, the above exercise shows the following.
\begin{proposition} \label{prop:easysub}
	Fix $\alpha,\beta\in\mathcal L(P)$. Then $\{\lnot(\alpha\land\beta),\beta\}\models\beta$.
\end{proposition}
\begin{proof}
	Reuse the proof above.
\end{proof}
More generally, we have the following notion of substitution.
\begin{proposition}[Substitution]
	Suppose that $\{\varphi_1,\ldots,\varphi_n\}\models\psi$ in $\mathcal L(P)$, then any substitution of the atomic formulae $P$ by formulae in $\mathcal L(Q)$ (which induces a map $f:\mathcal L(P)\to\mathcal L(Q)$), then $\{f\varphi_1,\ldots,f\varphi_n\}\models\psi$.
\end{proposition}
\begin{proof}
	Induction to make $f$ and then show the statement.
\end{proof}
\begin{example}
	We know that
	\[\{\lnot({\color{red}(r\lor s)}\land{\color{blue}(s\to p)}),{\color{red}(r\lor s)}\}\models\lnot{\color{blue}(s\to p)}\]
	is valid by \autoref{prop:easysub}.
\end{example}
Here are some more examples.
\begin{exe}
	We show that $\{(p\to q),\lnot p\}\nvDash\lnot q$.
\end{exe}
\begin{proof}
	We set a valuation $V$ by $V(p)=0$ and $V(q)=1$. Then
	\[\hat V((p\to q))=1_{\hat V(p)\le\hat V(q)}=1_{0\le1}=1,\]
	and $\hat V(\lnot p)=1-\hat V(p)=1-0=1$. Thus, our premises are true. But $\hat V(\lnot q)=1-\hat V(q)=1-1=0$ means the conclusion is true, so we are done.
\end{proof}
\begin{remark}
	Propositional logic cannot see the validity of all arguments. For example, the following argument cannot be broken down into propositions in the ways that $\mathcal L(P)$ provides.
	\begin{enumerate}
		\item Every integer greater than $1$ is a product of prime numbers.
		\item $999$ is an integer greater than $1$.
		\item Therefore, $999$ is a product of prime numbers.
	\end{enumerate}
	The issue is that we need quantifiers: we need a way to plug in $999$ into ``every integer,'' which propositional logic does not provide. We will fix this when we talk about predicate logic later in the course.
\end{remark}
Let's run down some valid forms of argument. They can be proven from a similar logic to the above.
\begin{itemize}
	\item Modus ponens: $\{\varphi\to\psi,\varphi\}\models\psi$.
	\item Modus teollens: $\{\varphi\to\psi,\lnot\psi\}\models\lnot\varphi$.
	\item Contraposition: $\{\varphi\to\psi\}\models\lnot\psi\to\lnot\varphi$.
	\item Disjunctive syllogism: $\{\varphi\lor\psi,\lnot\varphi\}\models\psi$ and $\{\varphi\lor\psi,\lnot\psi\}\models\varphi$.
	\item Hypothetical syllgism: $\{\varphi\to\psi,\psi\to\chi\}\models\varphi\to\chi$.
	\item Proof by cases: $\{\varphi\lor\psi,\varphi\to\chi,\psi\to\chi\}\models\chi$. The point is that either antecedant $\varphi$ or $\psi$ lead to the same conclusion $\chi$.
\end{itemize}

\subsection{Validity for Formulae}
Here is our definition.
\begin{definition}[Valid formula, tautology]
	A formula $\varphi\in\mathcal L(P)$ is \textit{valid} or a \textit{tautology} if and only if each valuation $V:P\to\{0,1\}$ has $\hat V(\varphi)=1$.
\end{definition}
\begin{example}
	Given any formula $\varphi\in\mathcal L(P)$, then any valuation $V$ has $\hat V(\varphi)=1$ or $\hat V(\varphi)=0$, so $\hat V(\varphi\lor\lnot\varphi)=1$ in all cases.
\end{example}
Here are some examples. The main point is that valid forms of arguments can turn $\models$ into $\to$ to make a tautology.
\begin{itemize}
	\item $\varphi\to\varphi$.
	\item $\varphi\to(\varphi\lor\psi)$ and $\psi\to(\varphi\land\psi)$.
	\item $\varphi\to(\varphi\lor\psi)$ and $\psi\to(\varphi\lor\psi)$.
	\item ``Modus pollens'': $((\varphi\to\psi)\land\varphi)\to\psi$.
	\item $((\varphi\to\psi)\land\lnot\psi)\to\lnot\varphi$.
	\item $((\varphi\to\psi)\land(\psi\to\chi))\to(\varphi\to\chi)$.
	\item $\varphi\to(\psi\to\varphi)$.
	\item Peirce's Law: $((\varphi\to\psi)\to\varphi)\to\varphi$.
\end{itemize}
The last one is not actually a tautology, but it is. The only care we have to worry about is when $\varphi$ is true, but when $\varphi$ is true, $(\varphi\to\psi)\to\varphi$ can only be false when $\varphi\to\psi$ can only be false when $\varphi$ is false.
\begin{remark}
	Another way to say that $\varphi$ is a valid formula is that $\emp\models\varphi$. So we will notationally write $\models\varphi$.
\end{remark}
Validity of arguments can be turned into validity of formulae, as follows.
\begin{theorem}[Deduction]
	An argument form $\{\varphi_1,\ldots,\varphi_n\}$ with conclusion $\psi$ is valid if and only if $(\varphi_1\land\ldots\land\varphi_n)\to\psi$ is valid. In other words, $\{\varphi_1,\ldots,\varphi_n\}\models\psi$ if and only if $\psi(\varphi_1\land\cdots\land\varphi_n)\to\psi$.
\end{theorem}
\begin{proof}
	By definition, $(\varphi_1\land\cdots\land\varphi_n)\to\psi$ will be valid if and only if every valuation has $V$
	\[1=\hat V((\varphi_1\land\cdots\land\varphi_n)\to\psi),\]
	which is equivalent to $\hat V(\varphi_1\land\dots\land\varphi_n)\le\hat V(\psi)$ which is equivalent to
	\[\min\{\hat V(\varphi_1),\ldots,\hat V(\varphi_n)\}\le\hat V(\psi).\]
	The only case above that we actually care about is that $\hat V(\varphi_1)=\cdots=\hat V(\varphi_n)=1$ implies $\hat V(\psi)=1$, which is exactly what $\{\varphi_1,\ldots,\varphi_n\}\models\psi$.
\end{proof}

\subsection{Equivalence}
We have the following definition.
\begin{definition}[Equivalence]
	Two formulae $\varphi,\psi\in\mathcal L(P)$ are \textit{equivalent} if and only if
	\[\hat V(\varphi)=\hat V(\psi)\]
	for each valuation $V:P\to\{0,1\}$.
\end{definition}
\begin{example}
	We have that $\lnot\lnot p$ is equivalent to $p$.
\end{example}
\begin{remark}
	Saying that $\varphi$ and $\psi$ are equivalent is the same as asserting $\models\varphi\leftrightarrow\psi$. In fact, we can define $\varphi$ being equivalent if and only if $\varphi$ is equivalent to $p\lor\lnot p$.
\end{remark}
Here are some more examples.
\begin{itemize}
	\item Idempotence: $\varphi\leftrightarrow(\varphi\land\varphi)$.
	\item Commutativity: $(\varphi\land\psi)\leftrightarrow(\psi\land\varphi)$ and $(\varphi\lor\psi)\leftrightarrow(\psi\lor\varphi)$.
	\item Associativity: $(\varphi\land\psi)\land\chi\leftrightarrow\varphi\land(\psi\land\chi)$ and $(\varphi\lor\psi)\lor\chi\leftrightarrow\varphi\lor(\psi\lor\chi)$.
	\item Absorption: $\varphi\leftrightarrow((\varphi\land(\varphi\lor\psi)))$, which we can see by a direct computation.
	\item Distributivity: $(\varphi\land(\psi\lor\chi))\leftarrow((\varphi\land\psi)\lor(\varphi\land\chi))$.
	\item Distributivity: $(\varphi\lor(\psi\land\chi))\leftarrow((\varphi\lor\psi)\land(\varphi\lor\chi))$.
	\item De Morgan Laws: $\lnot(\psi\land\varphi)\leftarrow(\lnot\psi\lor\lnot\varphi)$.
	\item De Morgan Laws: $\lnot(\psi\lor\varphi)\leftarrow(\lnot\psi\land\lnot\varphi)$.
\end{itemize}