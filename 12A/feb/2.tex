% !TEX root = ../notes.tex

Here we go again.

\subsection{Semantics for Connectives}
Last time we finished our discussion of creating the language of propositional logic. Today we will actually give formulae meaning as a ``truth'' function of the propositions.
\begin{example}
	We hinted for $\lnot p$ to mean ``not $p$.''
\end{example}
Our goal is to give precise definitions to all of our symbols, in a mathematically precise way.

The way we do this is by ``truth-conditional semantics.'' Namely, we will give meaning to certain formulae by describing when a formula is true or false.
\begin{example}
	We can compute the meaning of $\lnot p$ by the following truth table.
	\begin{center}
		\begin{tabular}{c|c}
			$p$ & $\lnot p$ \\\hline
			T & F \\
			F & T
		\end{tabular}
	\end{center}
	This truth table fully captures what $\lnot$ should mean.
\end{example}
\begin{convention}
	For the remainder of the class, we will replace ``T'' with $1$ and ``F'' with $0$.
\end{convention}
\begin{example}
	We have that the truth value of $\lnot p$ is $1-$ the truth value of $p$.
\end{example}
So using numbers will have some utility in this class.
\begin{example}
	We have that
	\[\begin{array}{c|c||c}
		p & q & p\land q \\\hline
		1 & 1 & 1 \\
		1 & 0 & 0 \\
		0 & 1 & 0 \\
		0 & 0 & 0
	\end{array}\]
	In short, we can verify that $p\land q$ has truth value equal to the minimum of the truth values of $p$ and $q$.
\end{example}
We could also say that $p\land q$ has truth value equal to the product of the truth values of $p$ and $q$. However, we do not do this to make better analogy with $\lor$, as follows.
\begin{example}
	We have that
	\[\begin{array}{c|c||c}
		p & q & p\lor q \\\hline
		1 & 1 & 1 \\
		1 & 0 & 1 \\
		0 & 1 & 1 \\
		0 & 0 & 0
	\end{array}\]
	In short, we can verify that $p\land q$ has truth value equal to the maximum of the truth values of $p$ and $q$.
\end{example}
Let's wrap up with the last two connectives.
\begin{example}
	We have that
	\[\begin{array}{c|c||c}
		p & q & p\to q \\\hline
		1 & 1 & 1 \\
		1 & 0 & 0 \\
		0 & 1 & 1 \\
		0 & 0 & 1
	\end{array}\]
	We can say this as the truth value of $p\to q$ is the indicator for when the truth value of $p$ is less than or equal to the truth value of $q$.
\end{example}
\begin{example}
	We have that
	\[\begin{array}{c|c||c}
		p & q & p\leftrightarrow q \\\hline
		1 & 1 & 1 \\
		1 & 0 & 0 \\
		0 & 1 & 0 \\
		0 & 0 & 1
	\end{array}\]
	Now, we can say this as the truth value of $p\to q$ is the indicator for when the truth value of $p$ is equal to the truth value of $q$.
\end{example}

\subsection{Semantics for Formulae}
The key idea to extending semantics to all formulae is to use a recursive definition, using the connectives to build arbitrary semantics for formulae. We will need to be somewhat technical for this.
\begin{definition}[Valuation]
	A \textit{valuation} $V$ for $P$ is a function $V:P\to\{0,1\}$ which assigns each proposition $p\in P$ a value of $0$ or $1$.
\end{definition}
\begin{example}
	There is a valuation $V:P\to\{0,1\}$ which gives $V(p)=1$ for each $p\in P$. There is also a valuation $V:P\to\{0,1\}$ which gives $V(p)=0$ for each $p\in P$.
\end{example}
\begin{example}
	The following describes a valuation for $P=\{p,q\}$.
	\[\begin{array}{c|c}
		p & 1 \\
		q & 0
	\end{array}\]
\end{example}
\begin{remark}
	The ``valuation'' $\hat V$ is meant to be an abbreviation of ``the truth value of.''
\end{remark}
Intuitively, a valuation is a description of the world of $P$: maybe $p_1\in P$ should be true with $p_2$ false, maybe something else.
\begin{remark}
	If $P$ is finite, then there are only finitely many valuations. Precisely, there are $2^\#P$ total valuations: each proposition $p\in P$ has two options (namely, $0$ or $1$), and we have to make $\#P$ total choices (simultaneously) to determine a unique valuation $V$.
\end{remark}
Now here is our promised idea.
\begin{idea}
	Recursion can extend any valuation $V:P\to\{0,1\}$ uniquely to all of $\hat V:\mathcal L(P)\to\{0,1\}$.
\end{idea}
\begin{example}
	Once we know $V(p)$ and $V(q)$, we get $V((p\land q))$ for free.
\end{example}
Let's give the recursive definition for $\hat V$; it works as follows.
\begin{itemize}
	\item For each atomic formula $p\in P\subseteq\mathcal L(P)$, we have $\hat V(p):=V(p)$. (This is our base case.)
	\item We have $\hat V(\lnot\varphi):=1-\hat V(\varphi)$.
	\item We have $\hat V((\varphi\land\psi)):=\min\{\hat V(\varphi),\hat V(\psi)\}$.
	\item We have $\hat V((\varphi\lor\psi)):=\max\{\hat V(\varphi),\hat V(\psi)\}$.
	\item We have $\hat V((\varphi\to\psi)):=1_{\hat V(\varphi)\le\hat V(\psi)}$.
	\item We have $\hat V((\varphi\leftrightarrow\psi)):=1_{\hat V(\varphi)=\hat V(\psi)}$.
\end{itemize}
A standard induction can show that the domain of $\hat V$ is indeed $\mathcal L(P)$. In particular, the domain of $\hat V$ contains all the letters $p\in P$ and is closed under our connectives.
\begin{remark}
	We could write the above discussion by writing out the full truth tables for each connective, but the above is arguably a cleaner connective.
\end{remark}
Now that we've extended $V$, we can make terminology for when formulae might be true or false.
\begin{definition}[Satisfies]
	A valuation $V:P\to\{0,1\}$ \textit{satisfies} $\varphi\in\mathcal L(P)$ if and only if $\hat V(\varphi)=1$. We will notate this by $V\models\varphi$.
\end{definition}
Intuitively, the world that $V$ describes makes the formula $\varphi$ true.

Let's do an example.
\begin{exe}
	Fix $P=\{p,q,r\}$ and $V$ a valuation by $V(p)=V(q)=1$ and $V(r)=0$. We determine if $V$ satisfies $(p\land(r\to q))$.
\end{exe}
\begin{proof}
	We do the computation by hand.
	\begin{align*}
		\hat V(((p\land(r\to q)))) &= \min\{\hat V(p),\hat V((r\to q))\} \\
		&= \begin{cases}
			\min\{\hat V(p),1\} & \hat V(r)\le\hat V(q), \\
			\min\{\hat V(p),0\} & \hat V(r)\le\hat V(q), \\
		\end{cases} \\
		&= \min\{\hat V(p),1\} \\
		&= \min\{1,1\} \\
		&= \boxed1.\qedhere
	\end{align*}
\end{proof}
The above approach felt more top-down as an evaluation, in contrast to the ore bottom-up a typical truth-table computation, as follows.
\[\begin{array}{c|c|c||c|c}
	p & q & r & r\to q & p\land(r\to q) \\\hline
	1 & 1 & 0 & 1 & 1
\end{array}\]

\subsection{Validity, Again}
We would like to use our precise definitions for logic to recreate our definitions for ``valid.''

We start with a small remark.
\begin{remark}
	Suppose $\varphi\in\mathcal L(P_1)\cap\mathcal L(P_2)$ for some proposition letters $P_1$ and $P_2$. (Namely, the letters of $\varphi$ live in both $P_1$ and $P_2$.) Then if $V_1:P_1\to\{0,1\}$ and $V_2:P_2\to\{0,1\}$ are valuations such that each $p$ present in $\varphi$ has $V_1(p)=V_2(p)$, then
	\[\hat V_1(\varphi)=\hat V_2(\varphi).\]
\end{remark}
To formalize this, we would need a notion of which propositions are present in $\varphi$, which we could define as a function recursively. We will not bother to do this here.

Here is our definition of an argument form.
\begin{definition}[Argument form]
	An \textit{argument form} in $\mathcal L(P)$ is a pair of \textit{premises} $\{\varphi_1,\ldots,\varphi_n\}\subseteq\mathcal L(P)$ and a \textit{conclusion} $\psi\in\mathcal L(P)$.
\end{definition}
And here is our definition of validity.
\begin{definition}[Valid]
	An argument form $(\{\varphi_1,\ldots,\varphi_n\},\psi)$ is \textit{valid} if and only if each valuation $V:P\to\{0,1\}$ has
	\[\hat V(\varphi_1)=\cdots=\hat V(\varphi_n)=1\implies\hat V(\psi)=1.\]
	We notate this by $\varphi_1,\ldots,\varphi_n\models\psi$ and say ``$\psi$ is a \textit{consequence} of $\{\varphi_1,\ldots,\varphi_n\}$.''
\end{definition}
\begin{example}
	We have that $(p\lor q),\lnot q\models p$.
\end{example}
Intuitively, an argument form is valid if, when the premises are true, the conclusion is also true. This is essentially the definition that we wanted.
\begin{remark}
	One can extend this definition to work with infinitely many premises, but we will not do this here.
\end{remark}

Observe that checking validity is a finite matter because, to check $\varphi_1,\ldots,\varphi_n\models\psi$, we only need to consider the finitely many propositions that take place in any of $\varphi_1,\ldots,\varphi_n,\psi$. And then each proposition has only two options, so in total we are safely in a finite universe.