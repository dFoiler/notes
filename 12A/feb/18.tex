% !TEX root = ../notes.tex

Here we go.

\subsection{Algorithms}
Today we start talking about algorithms.
\begin{definition}[Algorithm]
	An \textit{algorithm} is a procedure or set of rules or instructions that will tell us how to preform successive steps, which will complete a task after a finite number of steps.
\end{definition}
This definition can be made more rigorous, but we will not bother for this class.

We have already seen some algorithms so far.
\begin{example}
	To determine if a formula is satisfiable, we employ the following algorithm.
	\begin{enumerate}
		\item Fix a formula $\varphi\in\mathcal L(\{p_1,\ldots,p_n\})$.
		\item Fix some order of the valuations.
		\item Choose the first valuation $V$.
		\item Compute $\hat V(\varphi)$ by a recursive procedure.
		\item If $\hat V(\varphi)=1$, then $\varphi$ is satisfiable.
		\item Otherwise, choose the next valuation $V$ and go back to step 4.
		\item If we finish the truth table and never return satisfiable, then $\varphi$ is unsatisfiable.
	\end{enumerate}
\end{example}
\begin{remark}
	The above algorithm requires something like $2^n$ total valuations to check, making the worst-case (if unsatisfiable) requiring $2^n$ time. This makes this run in ``exponential'' time.
\end{remark}
\begin{remark}
	There are other algorithms for satisfiability. The textbook uses the semantic tableaux. We will shortly talk about resolution.
\end{remark}

\subsection{Another CNF Algorithm}
For resolution, we will need to start in conjunctive normal form. Recall the definition, as follows.
\begin{definition}[Conjunctive normal form]
	A formula $\varphi$ is in \textit{conjunctive normal form} if and only if it is a conjunction of disjunctions.
\end{definition}
We do already have an algorithm for this: find the disjunctive normal form $\lnot\varphi$ and then negate and apply De Morgan's laws.

However, the disjunctive normal form requires us to in advance have a truth table, which needs a notion of semantics. Here is an algorithm which is more syntactic.
\begin{prop}
	Fix a formula $\varphi$. We put $\varphi$ into conjunctive normal form.
	\begin{enumerate}
		\item Remove $\leftrightarrow$s: for each $\leftrightarrow$ starting from the left, replace each subformula of the form $(\alpha\leftrightarrow\beta)$ with $((\alpha\to\beta)\land(\beta\to\alpha))$.
		\item Remove $\to$s: for each $\to$ starting from the left, replace each subformula $(\alpha\to\beta)$ with $(\lnot\alpha\lor\beta)$.
	\end{enumerate}
	At this point, the only legal connectives are $\land$, $\lor$, and $\lnot$. In particular, if $\lnot\psi$ is a subformula, then $\psi$ is either of the form $(\alpha\land\beta)$ or $(\alpha\lor\beta)$ or $\lnot\alpha$.
	\begin{enumerate}[resume]
		\item Drive $\lnot$ inside: for each subformula of the form $\lnot(\alpha\lor\beta)$ starting from the left, replace it with $(\lnot\alpha\land\lnot\beta)$.
		\item Drive $\lnot$ inside: for each subformula of the form $\lnot(\alpha\land\beta)$ starting from the left, replace it with $(\lnot\alpha\lor\lnot\beta)$.
		\item Drive $\lnot$ inside: for each subformula of the form $\lnot\lnot\alpha$ starting from the left, replace it with $\alpha$.
	\end{enumerate}
	So now $\lnot$ only applies to propositions, so we are only applying $\land$ and $\lor$ to literals. We need to normalize to get conjunctives of disjunctions; the only way we violate being in conjunctive of normal form is if we have a subformula which is a $\lor$ of non-literals, and one of these non-literals had better have $\land$ lest we just break apart into separate disjunctive terms.
	\begin{enumerate}[resume]
		\item Distribute: for each subformula of the form $(\alpha\lor(\beta\land\gamma))$ starting from the left, replace it with $((\alpha\lor\beta)\land(\alpha\lor\gamma))$.
		\item Distribute: for each subformula of the form $((\alpha\land\beta)\lor\gamma)$ starting from the left, replace it with $(((\alpha\gamma)\land(\beta\lor\gamma))$.
		\item Go back to step 6 until done.
	\end{enumerate}
\end{prop}
We will not prove that this works, but there are remarks throughout the description. More formally, we are defining a recursive function $\op{CNF}:\mathcal L(P)\to\mathcal L(P)$ with many cases. For completeness, here is the definition.
\begin{enumerate}
	\item $\op{CNF}(p)=p$ for each $p\in P$.
	\item $\op{CNF}(\varphi\leftrightarrow\psi)=(\lnot\op{CNF}(\alpha)\lor\op{CNF}(\beta))\land(\lnot\op{CNF}(\beta)\lor\op{CNF}(\alpha))$.
	\item $\op{CNF}(\varphi\to\psi)=\lnot\op{CNF}(\alpha)\lor\op{CNF}(\beta)$.
	\item $\op{CNF}(\varphi\land\psi)=\op{CNF}(\varphi)\land\op{CNF}(\psi)$.
	\item $\op{CNF}(\lnot(\alpha\land\lnot\beta))=\lnot\op{CNF}(\alpha)\lor\lnot\op{CNF}(\beta)$.
	\item $\op{CNF}(\lnot(\alpha\lor\lnot\beta))=\lnot\op{CNF}(\alpha)\land\lnot\op{CNF}(\beta)$.
	\item $\op{CNF}(\lnot\lnot\varphi)=\op{CNF}(\varphi)$.
	\item $\op{CNF}(\alpha\lor(\beta\land\gamma))=\op{CNF}(\alpha\lor\beta)\land\op{CNF}(\alpha\lor\gamma)$.
	\item $\op{CNF}((\alpha\land\beta)\lor\gamma)=\op{CNF}(\alpha\lor\gamma)\land\op{CNF}(\beta\lor\gamma)$.
\end{enumerate}
It is not obvious that the above conditions fully determine $\op{CNF}$ recursively (in fact, the last two steps make $\op{CNF}$ not well-defined because we might have a subformula of the form $(\alpha\land\beta)\lor(\gamma\land\delta)$), but there is a small justification present in the definition of the algorithm, and it is not too hard to imagine how to fix this. We will not fix this formally out of laziness.
\begin{remark}
	The above algorithm need not give the same conjunctive normal form as with the truth table approach. The above algorithm is purely syntactic.
\end{remark}
Let's see some examples.
\begin{exe}
	We put $\varphi:=\lnot(p_1\land\lnot(\lnot p_2\land(p_3\to p_4)))$ in conjunctive normal form.
\end{exe}
\begin{proof}
	We have the following computation. To start, we get rid of $\to$.
	\begin{align*}
		\varphi &\equiv\lnot(p_1\land\lnot(\lnot p_2\land(p_3\to p_4)))
	\end{align*}
	Next we push $\lnot$s inside and cancel out double negations.
	\begin{align*}
		\varphi &\equiv\lnot(p_1\land\lnot(\lnot p_2\land(\lnot p_3\lor p_4))) \\
		&\equiv\lnot p_1\lor\lnot\lnot(\lnot p_2\land(\lnot p_3\lor p_4)) \\
		&\equiv\lnot p_1\lor\lnot(\lnot\lnot p_2\lor\lnot(\lnot p_3\lor p_4)) \\
		&\equiv\lnot p_1\lor(\lnot\lnot\lnot p_2\land\lnot\lnot(\lnot p_3\lor p_4)) \\
		&\equiv\lnot p_1\lor(\lnot\lnot\lnot p_2\land\lnot(\lnot\lnot p_3\land \lnot p_4)) \\
		&\equiv\lnot p_1\lor(\lnot\lnot\lnot p_2\land(\lnot\lnot\lnot p_3\lor \lnot\lnot p_4)) \\
		&\equiv\lnot p_1\lor(\lnot p_2\land(\lnot\lnot\lnot p_3\lor \lnot\lnot p_4)) \\
		&\equiv\lnot p_1\lor(\lnot p_2\land(\lnot p_3\lor \lnot\lnot p_4)) \\
		&\equiv\lnot p_1\lor(\lnot p_2\land(\lnot p_3\lor p_4)).
	\end{align*}
	Now we distribute. We see
	\begin{align*}
		\varphi & \equiv\lnot p_1\lor(\lnot p_2\land(\lnot p_3\lor p_4)) \\
		& \equiv(\lnot p_1\lor\lnot p_2)\land(\lnot p_1\lor(\lnot p_3\lor p_4))) \\
		& \equiv\boxed{(\lnot p_1\lor\lnot p_2)\land(\lnot p_1\lor\lnot p_3\lor p_4)},
	\end{align*}
	which is what we wanted.
\end{proof}
\begin{exe}
	We convert $\varphi:=(p_1\land p_2)\lor(\lnot p_1\land(\lnot p_2\land p_3))$ to conjunctive normal form using the algorithm.
\end{exe}
\begin{proof}
	We only have to distribute right now, so we compute
	\begin{align*}
		\varphi &= (p_1\land p_2)\lor(\lnot p_1\land(\lnot p_2\land p_3)) \\
		&\equiv ((p_1\land p_2)\lor\lnot p_1)\land((p_1\land p_2)\lor(\lnot p_2\land p_3)) \\
		&\equiv {\color{red}(}(p_1\lor\lnot p_1)\land(p_2\lor\lnot p_1){\color{red})}\land((p_1\land p_2)\lor(\lnot p_2\land p_3)) \\
		&\equiv ((p_1\land p_2)\lor\lnot p_1)\land((p_1\lor(\lnot p_2\land p_3)\land( p_2\lor(\lnot p_2\land p_3)))) \\
		&\equiv ((p_1\land p_2)\lor\lnot p_1)\land((((p_1\lor\lnot p_2)\land (p_1\lor p_3))\land( (p_2\lor\lnot p_2)\land(p_2\lor p_3)))),
	\end{align*}
	which finishes.
\end{proof}