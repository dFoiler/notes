\documentclass[../notes.tex]{subfiles}
\graphicspath{{\subfix{../figs/}}}

\begin{document}

% !TEX root = ../notes.tex

We continue our journey in natural deduction.

\subsection{Disjunction Introduction}
The story so far as that we have given introduction and elimination rules for $\lif$ and $\land$ and $\liff$ and $\lnot$. It remains to talk about $\lor$. We have also introduced reductio ad absurdum, which made our proof system complete.
\begin{remark}
	A computer could theoretically use this system to prove statements. One silly way is to just list out all possible sequences of steps in a Fitch-style proof and just check until we find a well-formed proof of the conclusion.
\end{remark}
Here is our way to introduce disjunctions.
\begin{defihelper}[$\lor$Introduction] \nirindex{Disjunction Introduction@$\lor$Introduction}
	Give $\varphi$, we can introduce $\varphi\lor\psi$ or introduce $\psi\lor\varphi$.
\end{defihelper}
More formally, either of the following are valid uses of $\lor$ introduction.
\begin{align*}
	\begin{nd}
		\have[\vdots]{}{\vdots}
		\have[i]{zeroeven}{\varphi}
		\have[\vdots]{}{\vdots}
		\have[j]{}{\varphi\lor\psi} \by{$\lor$Intro}{zeroeven}
	\end{nd}
	\qquad
	\begin{nd}
		\have[\vdots]{}{\vdots}
		\have[i]{zeroeven}{\psi}
		\have[\vdots]{}{\vdots}
		\have[j]{}{\varphi\lor\psi} \by{$\lor$Intro}{zeroeven}
	\end{nd}
\end{align*}
A priori, it might look like we are only even weakening in this step, so let's see an example where this is useful.
\evenorodd*
\begin{proof}
	Given a nonnegative integer $n$, we need to show that $n=2k$ for some integer $k$ or that $n=2k+1$ for some integer $k$. We prove this by induction on $n$.
	\begin{itemize}
		\item For our base case, we note that $n=0=2\cdot0$, so $n$ is even and in particular even or odd.

		Note here is where we applied weakening! After all, we need to go back to our inductive step somehow.
		\item We will do the inductive step later when we talk about disjunctive elimination.
		\qedhere
	\end{itemize}
\end{proof}
Diagrammatically, the base case of the proof looks like the following.
\begin{align*}
	\begin{nd}
		\have[\vdots]{}{\vdots}
		\have[i]{zeroeven}{0\text{ is even}}
		\have{}{(0\text{ is even})\lor(0\text{ is odd})} \by{$\lor$Intro}{zeroeven}
	\end{nd}
\end{align*}
So yes, we are weakening, but we had to.

There are other ways---non-constructive ways!---to prove disjunctions without $\lor$ introduction by reductio ad absurdum. Here's an example.
\begin{exe} \label{exe:reductoex}
	We deduce $p\lor\lnot p$.
\end{exe}
\begin{proof}
	We will never be able to prove $p$ or $\lnot p$ on its own, but we will still be able to derive $p\lor\lnot p$. Instead, we use reductio ad absurdum, proceeding as follows; the key trick is that our contradiction will be $(p\lor\lnot p)\land\lnot(p\lor\lnot p)$, from which we can work backwards, trying to prove $(p\lor\lnot p)$ and derive contradiction.
	\begin{align*}
		\begin{nd}
			\open[]
				\hypo{npornp}{\lnot(p\lor\lnot p)}
				\open[]
					\hypo{p}{p}
					\have{pornp0}{p\lor\lnot p} \by{$\lor$Intro}{p}
					\have{npornp0}{\lnot(p\lor\lnot p)} \by{Reit}{npornp0}
					\have{contra0}{(p\lor\lnot p)\land\lnot(p\lor\lnot p)} \by{$\land$Intro}{pornp0,npornp0}
				\close
				\have{np}{\lnot p} \by{$\lnot$Intro}{p,contra0}
				\have{pornp}{p\lor\lnot p} \by{$\lor$Intro}{np}
				\have{contra}{(p\lor\lnot p)\land\lnot(p\lor\lnot p)} \by{$\land$Intro}{pornp,npornp}
			\close
			\have{}{p\lor\lnot p} \by{RAA}{npornp-contra}
		\end{nd}
	\end{align*}
	Observe that we could show $\lnot\lnot(p\lor\lnot p)$ by simply using $\lnot$ introduction at the end instead of reductio ad absurdum.
\end{proof}
\begin{remark}
	Intuitionist logic bans reductio ad absurdum, essentially because it is annoying that proving $p\lor\lnot p$ doesn't actually tell you which one which is true; in fact, it is impossible to prove $p$ or $\lnot p$ from no assumptions because these formulae need not be valid, and we know that our proof system is sound.
	
	However, removing reductio ad absurdum is nice in some sense: for example, without reductio ad absurdum, any proof of $\varphi\lor\psi$ must actually prove $\varphi$ or prove $\psi$.
\end{remark}
\begin{remark}
	Any theorem $\varphi$ provable with reductio ad absurdum, one can prove $\lnot\lnot\varphi$ without reductio ad absurdum.
\end{remark}

\subsection{Proof by Cases}
So how do we use $\alpha\lor\beta$? After all, we don't know which one is true. But never fear---if we have some $\varphi$ with both $\alpha\to\varphi$ and $\beta\to\varphi$, then surely we can conclude $\varphi$. This is called ``proof by cases,'' where we try $\alpha$ and then try $\beta$, and we see that both of them go the conclusion.

Let's see an example.
\begin{lemma}
	Let $n$ be a nonnegative integer. Then the remainder when $n^2$ is divided by $4$ is either $0$ or $1$.
\end{lemma}
\begin{proof}
	By \autoref{lem:evenorodd}, we know that $n$ is even or odd. So we proceed by cases.
	\begin{itemize}
		\item Suppose that $n$ is even so that $n=2k$. Then $n^2=(2k)^2=4k^2$, so the remainder when $n$ is divided by $4$ is $0$, so we can introduce a disjunction and assert that the remainder is $0$ or $1$.
		\item Suppose that $n$ is odd so that $n=2k+1$. Then
		\[n^2=(2k+1)^2=4k^2+4k+1=4\left(k^2+k\right)+1,\]
		so the remainder when $n$ is divided by $4$ is $0$, so we can introduce a disjunction and assert that the remainder is $0$ or $1$.
	\end{itemize}
	Thus, both cases were able to derive the conclusion, so we are done.
\end{proof}
In a Fitch style proof, this looks like the following.
\begin{align*}
	\begin{nd}
		\have{hypo}{(n\text{ is even})\lor(n\text{ is odd})}
		\open[]
			\hypo{hypo0}{n\text{ is even}}
			\have[\vdots]{}{\vdots}
			\have[i]{}{n^2\text{ has remainder }0}
			\have{}{(n^2\text{ has remainder }0)\lor(n^2\text{ has remainder }1)}
		\close
		\open[]
			\hypo{hypo0}{n\text{ is odd}}
			\have[\vdots]{}{\vdots}
			\have[j]{}{n^2\text{ has remainder }1}
			\have{}{(n^2\text{ has remainder }0)\lor(n^2\text{ has remainder }1)}
		\close
		\have{}{(n^2\text{ has remainder }0)\lor(n^2\text{ has remainder }1)}
	\end{nd}
\end{align*}
Formally, we have the following.
\begin{restatable}[\texorpdfstring{$\lor$}{}Elimination]{defihelper}{orelim} \nirindex{Disjunction Elimination@$\lor$Elimination}
	Given $\alpha\lor\beta$ and two subproofs deducing $\varphi$ from $\alpha$ and $\varphi$ from $\beta$, then we can deduce $\varphi$.
\end{restatable}
Diagrammatically, this looks like the following.
\begin{align*}
	\begin{nd}
		\have[a]{a}{\alpha\lor\beta}
		\open[]
			\hypo[b]{b}{\alpha}
			\have[\vdots]{}{\vdots}
			\have[c]{c}{\varphi}
		\close
		\have[\vdots]{}{\vdots}
		\open[]
			\hypo[d]{d}{\beta}
			\have[\vdots]{}{\vdots}
			\have[e]{e}{\varphi}
		\close
		\have[\vdots]{}{\vdots}
		\have[n]{}{\varphi} \by{$\lor$Elim}{a,b-c,d-e}
	\end{nd}
\end{align*}
We now close by discussing the inductive step for \autoref{lem:evenorodd}.
\begin{proof}[Proof of the inductive step in \autoref{lem:evenorodd}]
	We know that $n$ is even or odd, so we have the following two cases.
	\begin{itemize}
		\item If $n$ is even, then set $n=2k$, so $n+1=2k+1$, so $n$ is odd, so $n$ is even or odd.
		\item If $n$ is odd, then set $n=2k+1$, so $n+1=(2k+1)$, so $n$ is even, so $n$ is even or odd.
	\end{itemize}
	It follows that $n+1$ is even or odd, finishing our inductive step.
\end{proof}

\end{document}