\documentclass[../notes.tex]{subfiles}
\graphicspath{{\subfix{../figs/}}}

\begin{document}

% !TEX root = ../notes.tex

We roll. As a note on the midterm, we will be given a take-home, $48$-hour midterm. It is not intended to take that long; it should be approximately a shorter problem set.

\subsection{Deduction Theorem}
We are still talking about natural deduction proofs. So far we have talked about how to introduce and eliminate the connectives $\to$ and $\land$. For example, the following proof uses all of our rules.
\begin{align*}
	\begin{nd}
		\hypo{ptoq0}{p\to q}
		\open[]
			\hypo{pandr}{p\land r}
			\hypo{ptoq}{p\to q} \by{Reit}{ptoq0}
			\have{p}{p} \by{$\land$Elim}{pandr}
			\have{q}{q} \by{L$\to$Elim}{ptoq,p}
			\have{r}{r} \by{$\land$Elim}{pandr}
			\have{qandr}{q\land r} \by{$\land$-Intro}{q,r}
		\close
		\have{}{(p\land r)\to(q\land r)} \by{$\to$Intro}{pandr-qandr}
	\end{nd}
\end{align*}
In our story so far, we saw that $\{\varphi_1,\ldots,\varphi_n\}\models\psi$ if and only if the formula $(\varphi_1\land\cdots\land\varphi_n)\to\psi$ is a tautology.

There is also a proof-theoretic analog of this statement.
\begin{theorem}[Deduction]
	There is a proof with assumptions $\{\varphi_1,\ldots,\varphi_n\}$ and conclusion $\psi$ if and only if there is a proof with no assumption $(\varphi_1\land\cdots\land\varphi_n)\to\psi$.
\end{theorem}
\begin{proof}
	In one direction, suppose that we can prove $\{\varphi_1,\ldots,\varphi_n\}$ gives $\psi$. Well, simply open a subproof to place $\varphi_1\land\cdots\land\varphi_n$ into the hypotheses and use $\land$ elimination to deduce $\varphi_1,\ldots,\varphi_n$. Then the old proof to conclude $\psi$ still works. Visually, we are applying the following process.
	\begin{align*}
		\underbrace{\begin{nd}
			\hypo{}{\varphi_1}
			\hypo[\vdots]{}{\vdots}
			\hypo[n]{}{\varphi_n}
			\have[\vdots]{}{\vdots}
			\have[m]{}{\psi}
		\end{nd}}_P
		\implies\qquad
		\begin{nd}
			\open[]
				\hypo{1}{\varphi_1\land\cdots\land\varphi_n}
				\have{}{\varphi_1} \by{$\land$Intro}{1}
				\have[\vdots]{}{\vdots}
				\have[n]{}{\varphi_n} \by{$\land$Intro}{1}
				\have[\vdots]{}{P}
				\have[m]{n}{\psi}
			\close
			\have{}{(\varphi_1\land\cdots\land\varphi_n)\to\psi}
		\end{nd}
	\end{align*}
	In the other direction, $\{\varphi_1,\ldots,\varphi_n\}$ as assumptions lets us conclude $(\varphi_1\land\cdots\land\varphi_n)$, from which $\to$ elimination lets us conclude $\psi$.
\end{proof}
% \begin{example}
% 	We ca
% \end{example}

\subsection{Biconditional}
The biconditional $\varphi\leftrightarrow\psi$ simply means $(\varphi\to\psi)\land(\psi\to\varphi)$. As such, we introduce $\varphi\leftrightarrow\psi$ by having $\varphi\to\psi$ and $\psi\to\varphi$ and using $\land$ introduction. Formally, we have the following.
\begin{defihelper}[\texorpdfstring{$\leftrightarrow$}{}Introduction] \nirindex{Biconditional Introduction@$\leftrightarrow$Introduction}
	If we have a subproof hypothesizing $\varphi$ and proving $\psi$ and another subproof hypothesizing $\psi$ and proving $\varphi$, then we may conclude that $\varphi\leftrightarrow\psi$.
\end{defihelper}
For example, we might write this as follows.
\begin{align*}
	\begin{nd}
		\open[]
			\hypo[i]{i}{\varphi}
			\have[\vdots]{}{\vdots}
			\have[j]{j}{\psi}
		\close
		\have[\vdots]{}{\vdots}
		\open[]
			\hypo[k]{k}{\psi}
			\have[\vdots]{}{\vdots}
			\have[\ell]{l}{\varphi}
		\close
		\have[\vdots]{}{\vdots}
		\have[n]{}{\varphi\leftrightarrow\psi} \by{$\leftrightarrow$Intro}{i-j,k-l}
	\end{nd}
\end{align*}
Here are some examples.
\begin{exe}
	We show $q\leftrightarrow q$.
\end{exe}
\begin{proof}
	The proof just uses the reiteration rule.
	\begin{align*}
		\begin{nd}
			\open[]
				\hypo{i}{q}
				\have{j}{q} \by{Reit}{i}
			\close
			\open[]
				\hypo{k}{q}
				\have{l}{q} \by{Reit}{k}
			\close
			\have{}{q\leftrightarrow q} \by{$\leftrightarrow$Intro}{i-j,k-l}
		\end{nd}
	\end{align*}
	This works.
\end{proof}
\begin{exe}
	We show $(\varphi\to\psi)\land(\psi\to\varphi)$ lets us conclude $\varphi\leftrightarrow\psi$.
\end{exe}
\begin{proof}
	\begin{align*}
		\begin{nd}
			\hypo{1}{(\varphi\to\psi)\land(\psi\to\varphi)}
			\open[]
				\hypo{2}{\varphi}
				\have{3}{\varphi\to\psi} \by{$\land$Elim}{1}
				\have{4}{\psi} \by{$\to$Elim}{2,3}
			\close
			\open[]
				\hypo{5}{\psi}
				\have{6}{\psi\to\varphi} \by{$\land$Elim}{1}
				\have{7}{\varphi} \by{$\to$Elim}{5,6}
			\close
			\have{}{\varphi\leftrightarrow\psi} \by{$\leftrightarrow$Intro}{2-4,5-7}
		\end{nd}
	\end{align*}
\end{proof}
Now, to use $\leftrightarrow$, we have the following.
\begin{align*}
	\begin{nd}
		\have[\vdots]{}{\vdots}
		\have[i]{i}{\varphi\leftrightarrow\psi}
		\have[\vdots]{}{\vdots}
		\have[j]{j}{\varphi}
		\have[\vdots]{}{\vdots}
		\have[n]{}{\psi} \by{$\leftrightarrow$Elim}{i,j}
	\end{nd}
\end{align*}
Essentially, having one of them lets you conclude the other. As usual, this can be reduced to $(\varphi\to\psi)\land(\psi\to\varphi)$. We codify this as follows.
\begin{defihelper}[$\leftrightarrow$Elimination] \nirindex{Biconditional Elimination@$\leftrightarrow$Elimination}
	If we have $\varphi\leftrightarrow\psi$ and $\varphi$, then we can conclude $\psi$.
\end{defihelper}

\subsection{Negation Introduction}
The idea to introduce $\lnot$, we have to use proof by contradiction: to prove $\lnot\varphi$, we assume $\varphi$ and derive a contradiction.

Let's see the most famous example of a proof by contradiction.
\begin{proposition} \label{prop:sqrttwo}
	We show $\sqrt2$ is irrational.
\end{proposition}
\begin{proof}
	Suppose for the sake of contradiction that $\sqrt2$ is rational. Well, then we can write $\sqrt2=\frac ab$ for integers $a$ and $b$ such that $b\ne0$ and $a$ and $b$ cannot be reduced. Squaring, we see that $2=\frac{a^2}{b^2}$, so
	\[a^2=2b^2.\]
	Now, $a^2$ is even, so $a$ is even (which we will more formally justify later in the course), so we can write $a=2k$. Plugging in, we see that
	\[4k^2=(2k)^2=a^2=2b^2,\]
	so $b^2=2k^2$. But now $b^2$ is even, so $b$ is even! Thus, the fraction $\frac ab$ has even numerator and denominator, so it could have been reduced, which is our contradiction.
\end{proof}
At a high level, the proof looked like the following.
\begin{align*}
	\begin{nd}
		\open[]
			\hypo{1}{\sqrt2\in\QQ}
			\have[\vdots]{}{\vdots}
			\have[n]{n}{\frac ab\text{ is reducible}\land\frac ab\text{ is not reducible}}
		\close
		\have{}{\lnot(\sqrt2\in\QQ)} \by{$\lnot$Intro}{1-n}
	\end{nd}
\end{align*}
This is our introduction rule.
\begin{defihelper}[\texorpdfstring{$\lnot$}{}Intro] \nirindex{Not Introduction@$\lnot$Intro}
	If we have a subproof starting with $\varphi$ and concluding $\beta\land\lnot\beta$ for some formula $\beta$, then we can conclude $\lnot\varphi$.
\end{defihelper}
\begin{remark}
	Many texts write $\beta\land\lnot\beta$ as $\perp$ to mean ``false.''
\end{remark}
In a diagram, such a proof would look as follows.
\begin{align*}
	\begin{nd}
		\have[\vdots]{}{\vdots}
		\open[]
			\hypo[i]{i}{\varphi}
			\have[\vdots]{}{\vdots}
			\have[j]{j}{\psi\land\lnot\psi}
		\close
		\have[\vdots]{}{\vdots}
		\have[n]{}{\lnot\varphi} \by{$\lnot$Intro}{i-j}
	\end{nd}
\end{align*}
Here is an example.
\begin{exe}
	From $p\to q$ we may conclude $\lnot q\to\lnot p$.
\end{exe}
\begin{proof}
	We simply use $\to$ introduction and $\lnot$ introduction for fun and profit.
	\begin{align*}
		\begin{nd}
			\hypo{ptoq}{p\to q}
			\open[]
				\hypo{nq0}{\lnot q}
				\open[]
					\hypo{p}{p}
					\have{q}{q} \by{$\to$Elim}{ptoq,p}
					\have{nq}{\lnot q} \by{Reit}{nq0}
					\have{false}{q\land\lnot q} \by{$\land$Intro}{q,nq}
				\close
				\have{np}{\lnot p} \by{$\lnot$Intro}{p-false}
			\close
			\have{}{\lnot q\to\lnot p} \by{$\to$Intro}{nq-np}
		\end{nd}
	\end{align*}
	This works.
\end{proof}
\begin{remark}
	By the deduction theorem, this also proves $(p\to q)\to(\lnot q\to\lnot p)$.
\end{remark}
In fact, one can try to prove that $\lnot q\to\lnot p$ can prove $p\to q$, but for this we need to know that $\lnot\lnot p\leftrightarrow p$, which we have to be careful about.

Now, there are two ways to talk about ``negation elimination.'' Here is one way.
\begin{restatable}[\texorpdfstring{$\lnot$}{}Elimination]{defihelper}{notelim} \nirindex{Not Elimination@$\lnot$Elimination}
	If we are given $\lnot\varphi$, then we may infer $\varphi\to\psi$ for any $\psi$.
\end{restatable}
Essentially, the idea is that the antecedent is false, so the conditional is true.
\begin{restatable}{proposition}{sqrttwoirrational}
	We show that the empty set $\emp$ is a subset of a set $A$.
\end{restatable}
\begin{proof}
	Fix $A$ any set. Now, for any $x$, we have $x\notin\emp$. So (vacuously) $x\in\emp$ implies $x\in A$ by our elimination rule. This finishes.
\end{proof}
We can represent this proof as follows.
\begin{align*}
	\begin{nd}
		\have{hypo}{\lnot(x\in\emp)}
		\have{}{(x\in\emp)\to(x\in A)} \by{$\lnot$Elim}{hypo}
	\end{nd}
\end{align*}

\end{document}