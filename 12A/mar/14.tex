% !TEX root = ../notes.tex

Here we go.

\subsection{Finishing Disjunction Elimination}
Today we finish natural deduction by talking about syllogistic logic. Recall the following definition.
\orelim*
\noindent Here is an example from propositional logic.
\begin{exe}
	We derive $p\lor q$ from $\lnot p\to q$.
\end{exe}
\begin{proof}
	We want to do a proof by cases. Recall that we have shown $p\lor\lnot p$ (from reductio ad absurdum in \autoref{exe:reductoex}), so we will just start from there; formally, we should copy and paste the whole proof in, but we won't bother. The point is that the conclusion has $p\lor q$, so we need to get a $p$ there somehow.
	\begin{align*}
		\begin{nd}
			\hypo{hypo}{\lnot p\to q}
			\have{pornp}{p\lor\lnot p}
			\open[]
				\hypo{p}{p}
				\have{porq}{p\lor q} \by{$\lor$Intro}{p}
			\close
			\open[]
				\hypo{np}{\lnot p}
				\have{nptoq}{\lnot p\to q} \by{Reit}{hypo}
				\have{q1}{q} \by{$\to$Elim}{np,nptoq}
				\have{porq2}{p\lor q} \by{$\lor$Intro}{q}
			\close
			\have{end}{p\lor q} \by{$\lor$Elim}{pornp,p-porq,np-porq2}
		\end{nd}
	\end{align*}
	Again, we see that $\lor$ introduction is crucial.
\end{proof}
To review, we see that we have now built a full proof system for our propositional logic. With reductio ad absurdum, this proof system is sound (it only proves true statements) and complete (it can prove any true statement).
\begin{remark}
	More precisely, a formula $\varphi$ can be called \textit{provable} if and only if there is a proof of $\varphi$ via our proof system. Then it is a powerful theorem that provability is equivalent to validity. For example, a computer can (somewhat) quickly check a proof, which is nice.
\end{remark}

\subsection{Contradiction}
We are now essentially done with propositional logic, but we'll add in a small bonus. The point here is that many of our rules (namely, $\lnot$ introduction, ex falso quodlibet, and reductio ad absurdum) have contradictions of the form $\varphi\lor\lnot\varphi$ play a crucial role.

It will be psychologically easier to introduce the following notion.
\begin{definition}[Falsum, bottom]
	We introduce the symbol $\perp$ to our propositional language $\mathcal L(P)$ by deciding to live in $P\cup\{\perp\}$ as our set of atomic formulae.
	
	We will have $\perp$ ``stand in for'' any contradiction. Explicitly, in order to extend our valuations to $\perp$, we will require that
	\[V(\perp)=0\]
	for any valuation $V:P\to\{0,1\}$. In other words, $\perp$ is always false.
\end{definition}
To use our $\perp$ symbol, we now need introduction and elimination rules.
\begin{defihelper}[\texorpdfstring{$\perp$}{}Introducion] \nirindex{Falsum introduction@$\perp$Introduction}
	Given $\alpha$ and $\lnot\alpha$ in a partial proof, we may introduce $\perp$.	
\end{defihelper}
Diagrammatically, this looks like the following.
\begin{align*}
	\begin{nd}
		\have[\vdots]{}{\vdots}
		\have[i]{a}{\alpha}
		\have[\vdots]{}{\vdots}
		\have[j]{na}{\lnot\alpha}
		\have[\vdots]{}{\vdots}
		\have[k]{}{\perp} \by{$\perp$Intro}{a,na}
	\end{nd}
\end{align*}
Namely, we can introduce $\perp$ from any explicit contradiction.

Here is our elimination rule.
\begin{defihelper}[\texorpdfstring{$\perp$}{}Elimination] \nirindex{Falsum elimination@$\perp$Elimination}
	Given $\perp$, we may introduce any formula $\psi$.
\end{defihelper}
Diagrammatically, this looks like the following.
\begin{align*}
	\begin{nd}
		\have[\vdots]{}{\vdots}
		\have[i]{bad}{\perp}
		\have[\vdots]{}{\vdots}
		\have[j]{}{\psi} \by{$\perp$Elim}{i}
	\end{nd}
\end{align*}
This is essentially ex falso quodlibet, where we took a contradiction $\varphi\land\lnot\varphi$ and derived any formula.

Now, the symbol $\perp$ will let us remove the explicit contradictions from our proof system. Our construction of $\perp$ elimination was essentially ex falso quodlibet, which means that we have to talk about $\lnot$ introduction and reductio ad absurdum. The point is to take our definitions and replace the explicit contradictions with $\perp$. Here are our rules.
\begin{defihelper}[\texorpdfstring{$\lnot$}{}Introduction] \nirindex{Not Introduction@$\lnot$Introduction}
	If a subproof with hypothesis $\varphi$ derives $\perp$, we can deduce $\lnot\varphi$.
\end{defihelper}
\begin{definition}[Reductio ad absurdum]
	If a subproof with hypothesis $\lnot\varphi$ derives $\perp$, we can deduce $\varphi$.
\end{definition}
This looks like the following.
\begin{align*}
	\begin{nd}
		\have[\vdots]{}{\vdots}
		\open[]
			\hypo[i]{np}{\lnot\varphi}
			\have[\vdots]{}{\vdots}
			\have[j]{contra}{\perp}
		\close
		\have[\vdots]{}{\vdots}
		\have[k]{}{\varphi} \by{RAA}{np-contra}
	\end{nd}
\end{align*}
Let's see an example.
\begin{exe}
	If we have $p\lor q$ and $\lnot p\to\lnot q$, then we can deduce $p$.
\end{exe}
\begin{proof}
	Here is the first version of the proof, without $\perp$.
	\begin{align*}
		\begin{nd}
			\hypo{porq}{p\lor q}
			\hypo{nptonq}{\lnot p\to\lnot q}
			\open[]
				\hypo{np}{\lnot p}
				\have{nq}{\lnot q} \by{L$\to$Elim}{np,nptonq}
				\open[]
					\hypo{p}{p}
					\have{contra0}{p\land\lnot p} \by{L$\land$Intro}{p,np}
				\close
				\open[]
					\hypo{q}{q}
					\have{contra1}{q\land\lnot q} \by{L$\land$Intro}{q,nq}
					\have{contra2}{p\land\lnot p} \by{EFQ}{contra1}
				\close
				\have{contra}{p\land\lnot p} \by{$\lor$Elim}{porq,p-contra0,q-contra2}
			\close
			\have{}{p} \by{RAA}{np-contra}
		\end{nd}
	\end{align*}
	What is inelegant here is that we have to move between contradictions $q\land\lnot q$ to $p\land\lnot p$. To remove this inelegance, we can use $\perp$.
	\begin{align*}
		\begin{nd}
			\hypo{porq}{p\lor q}
			\hypo{nptonq}{\lnot p\to\lnot q}
			\open[]
				\hypo{np}{\lnot p}
				\have{nq}{\lnot q} \by{L$\to$Elim}{np,nptonq}
				\open[]
					\hypo{p}{p}
					\have{contra0}{\perp} \by{L$\perp$Intro}{p,np}
				\close
				\open[]
					\hypo{q}{q}
					\have{contra2}{\perp} \by{L$\perp$Intro}{q,nq}
				\close
				\have{contra}{\perp} \by{$\lor$Elim}{porq,p-contra0,q-contra2}
			\close
			\have{}{p} \by{RAA}{np-contra}
		\end{nd}
	\end{align*}
	So we got to save one line, but also our proof looks a little cleaner.
\end{proof}
This finishes contradiction. The midterm will only use content from our discussion on propositional logic.