% !TEX root = ../notes.tex

\documentclass[../notes.tex]{subfiles}

\begin{document}

\section{November 7}

Today we prove \autoref{thm:rw}.

\subsection{Rapidly Cauchy Intermission}
As an intermission, we introduce the following definition.
\begin{definition}[Rapidly Cauchy]
	Fix a metric space $(X,d)$. Then a sequence $\{x_n\}_{n\in\NN}$ in $X$ is \textit{rapidly Cauchy} if and only if all $\varepsilon>0$ have some $N$ for which
	\[\sum_{k=1}^\infty d(x_k,x_{k+1})<\infty.\]
\end{definition}
We won't use this definition in any meaningful way, but it will be enlightening to note that the main idea to the proof of \autoref{thm:rw} is similar to the proof that a Cauchy sequence has a rapidly Cauchy subsequence.

As such, let's see our checks on being rapidly Cauchy.
\begin{lemma}
	Fix a metric space $(X,d)$. Then any rapidly Cauchy sequence $\{x_n\}_{n\in\NN}$ in $X$ is also a Cauchy sequence.
\end{lemma}
\begin{proof}
	Fix any $\varepsilon>0$. We want $N$ for which $n,m\ge N$ give $d(x_n,x_m)<\varepsilon$. Well, set $S\coloneqq\sum_{k=1}^\infty d(x_k,x_{k+1})$, so we note that there is some $N$ for which
	\[\Bigg|S-\sum_{k=1}^nd(x_k,x_{k+1})\Bigg|<\varepsilon\]
	for each $n\ge N$. It follows that
	\[S-\sum_{k=1}^nd(x_k,x_{k+1})<\varepsilon\]
	for any $n\ge N$. Thus, for any $m\ge n\ge N+1$, the triangle inequality yields
	\[d(x_n,x_m)\le\sum_{k=n}^{m-1}d(x_k,x_{k+1})=\sum_{k=1}^{m-1}d(x_k,x_{k+1})-\sum_{k=1}^{n-1}d(x_k,x_{k+1})\stackrel*\le S-\sum_{k=1}^{n-1}d(x_k,x_{k+1})<\varepsilon.\]
	Notably, $\stackrel*\le$ holds because all terms in the series of $S$ are nonnegative, so the sequence of partial sums is increasing, so $S$ is greater than or equal to any individual partial sum.
\end{proof}
\begin{proposition} \label{prop:cauchy-has-rapidly-cauchy}
	Fix a metric space $(X,d)$. Then any Cauchy sequence $\{x_n\}_{n\in\NN}$ has a rapidly Cauchy subsequence.
\end{proposition}
\begin{proof}
	We proceed inductively. Set $n_0=1$. Next, suppose we already have some $n_k$. Because $\{x_n\}_{n\in\NN}$, we can find a constant $n_{k+1}\ge n_g$ such that $m,n\ge n_{k+1}$ implies $d(x_m,x_n)<2^{-k}$. In particular, we see that $n_{k+1}\ge n_k$ in this construction tells us that
	\[d(x_{n_k},x_{n_{k+1}})\le 2^{-k}\]
	for $j\ge1$. Summing, we see
	\[\sum_{k=1}^\infty d(x_k,x_{k+1})\le\sum_{k=1}^\infty2^{-k}=1<\infty,\]
	so $\{x_{n_k}\}_{k\in\NN}$ is the desired rapidly Cauchy subsequence.
\end{proof}

\subsection{The Riesz--Weyl Theorem}
And now, our feature presentation.
\rwthm*
\begin{proof}
	We proceed as in \autoref{prop:cauchy-has-rapidly-cauchy}. Set $n_0=1$. Then we proceed inductively: suppose we already know our $n_k$ for some $k$, and we construct $n_{k+1}$. Note that
	\[\lim_{m,n\to\infty}\mu\left(\{x\in X:\norm{f_m(x)-f_n(x)}\ge2^{-k}\}\right)=0,\]
	so we can find a constant $n_{k+1}>n_k$ such that $m,n\ge n_{k+1}$ gives
	\[\mu\left(\{x\in X:\norm{f_m(x)-f_n(x)}>2^{-k}\}\right)<2^{-k}.\]
	We now claim that the sequence $\{f_{n_k}\}_{k\in\NN}$ is almost uniformly Cauchy. This has two steps. Fix any $\varepsilon>0$.
	\begin{enumerate}
		\item We select our small $E\in\mc S$ to avoid. Choose $N$ for which
		\[\sum_{k=N}^\infty 2^{-k}=2^{-N+1}<\varepsilon.\]
		As such, we set
		\[E_k\coloneqq\{x\in X:\norm{f_{n_k}(x)-f_{n_{k+1}}(x)}\ge2^{-k}\}\]
		so that $\mu(E_k)<2^{-k}$ by construction of the sequence $\{n_k\}_{k\in\NN}$ because $n_k,n_{k+1}\ge n_k$. Thus, we define our $E$ as
		\[E\coloneqq\bigcup_{k=N}^\infty E_k.\]
		Indeed, $E\in\mc S$ because $E_k\in\mc S$ for each $k\ge N$, so by \autoref{lem:premeas-is-countable-subadd}, we may say
		\[\mu(E)\le\sum_{k=N}^\infty\mu(E_k)=\sum_{k=N}^\infty2^{-k}<\varepsilon,\]
		where the last inequality is by construction of $N$.
		\item It remains to check that the subsequence $\{f_{n_k}|_{X\setminus E}\}_{k\in\NN}$ is uniformly Cauchy. Well, given $\delta>0$, we need $M$ so that $i,j>M$ and $x\notin E$ gives
		\[\norm{f_{n_i}(x)-f_{n_j}(x)}\stackrel?<\delta\]
		for all $x\notin E$. Well, find some $M>N$ such that
		\[\sum_{j\ge M}2^{-j}=2^{-M+1}<\delta.\]
		As such, it follows from the triangle inequality that any $j>i>M$ will have
		\[\norm{f_{n_i}(x)-f_{n_j}(x)}\le\sum_{k=i}^{j-1}\norm{f_{n_k}(x)-f_{n_{k+1}}(x)}\stackrel*\le\sum_{k=i}^{j-1}2^{-k}<\sum_{k=M}^\infty2^{-k}<\varepsilon,\]
		which is what we wanted. Notably, $\stackrel*\le$ holds by construction of $E$ as a subset of $E_k$.
		\qedhere
	\end{enumerate}
\end{proof}
\begin{example}
	Even in \autoref{ex:the-bad-news}, there is a subsequence which is almost uniformly converging to $0$. Indeed, consider the subsequence $\{1_{E_{2^n}}\}_{n\in\NN}$. Then for any $\varepsilon>0$, we find some $N$ for which $2^{-N}<\varepsilon$ and set $E\coloneqq\left[0,1/2^N\right)$ to have measure less than $\varepsilon$. But now, for $n\ge N$, we see that $1_{E_{2^n}}|_{X\setminus E}=0$ because $E_{2^n}\subseteq E$. Thus, $1_{E_{2^n}}|_{X\setminus E}\to 0$ uniformly as $n\to\infty$.
\end{example}
We are now ready to use the condition that we are integrating into a Banach space!
\begin{lemma} \label{lem:almost-uniform-cauchy-converges}
	Fix a measure space $(X,\mc S,\mu)$ and a Banach space $B$. Further, fix an almost uniformly Cauchy sequence $\{f_n\}_{n\in\NN}$ of $\mc S$-measurable functions. Then there is an $\mathcal S$-measurable function $f\colon X\to B$ such that $f_n\to f$ almost uniformly as $n\to\infty$.
\end{lemma}
\begin{proof}
	The main idea is that the almost uniformity condition allows us to define $f$ outside a null set, which is good enough.

	For each $n\in\NN$, we get some $E_n$ such that $\mu(E_n)<1/n$ and such that $\{f_i|_{X\setminus E_n}\}_{i\in\NN}$ is uniformly Cauchy. We now set
	\[E\coloneqq\bigcap_{n=1}^\infty E_n.\]
	Note $E\in\mc S$ by \autoref{rem:sring-has-intersections}, and \autoref{lem:fin-additive-is-monotone} tells us that $\mu(E)\le\mu(E_n)<1/n$ for each $n$, so it follows $\mu(E)=0$.

	Now, for any $x\in X\setminus E$, we can find $k$ for which $x\notin E_k$. Thus, because $\{f_n|_{X\setminus E_k}\}_{n\in\NN}$ is uniformly Cauchy, we see that $\{f_n(x)|_{X\setminus E_k}\}_{n\in\NN}$ is Cauchy; we define $f(x)$ as its limit. Note we have used the fact that $B$ is a Banach space here! This defines $f$ outside the null set $E$.
	
	It doesn't really matter what $f$ does on $E$, so we just define $f(x)=0$ for $x\in E$. We will quickly run checks to show that $f$ is $\mc S$-measurable, but they are not terribly important.
	\begin{itemize}
		\item We show $f_n1_{X\setminus E}$ is $\mc S$-measurable for each $n$. This proof is similar to \autoref{lem:restrict-meas-functions}, so we will use the ideas of that proof. For example, $\im f_n$ is separable, and as remarked in \autoref{lem:restrict-meas-functions}, we have
		\[\im f_n1_{X\setminus E}\subseteq\{0\}\cup\im f_n.\]
		As such, $\{0\}\cup\im f_n$ is separable by \autoref{ex:union-of-seps-is-sep}, so $\im f_n1_{X\setminus E}$ is separable by \autoref{rem:subspace-of-sep-is-sep}.

		Further, for any open $U\subseteq B\setminus\{0\}$, we again note from the proof of \autoref{lem:restrict-meas-functions} that
		\[(f_n1_E)^{-1}(U)=(X\setminus E)\cap f_n^{-1}(U)=f_n^{-1}(U)\setminus E.\]
		In particular, $f_n^{-1}(U)\in\mc S$ because $f_n$ is $\mc S$-measurable, so $f_n^{-1}(U)\setminus E\in\mc S$ follows.

		\item We show $f_n1_{X\setminus E}\to f$ pointwise as $n\to\infty$. If $x\notin E$, then we recall that we defined $f(x)$ as the limit of $\{f_n(x)\}_{n\in\NN}$, so the result follows by real analysis because $(f_n1_{X\setminus E})(x)=f_n(x)$ in this case.
		
		Otherwise, $x\in E$, so $(f_n1_{X\setminus E})(x)=0$ for each $n$, and $f(x)=0$ by construction of $f$. So this case is just looking at a constant sequence.
	\end{itemize}
	It follows that $f$ is $\mc S$-measurable by \autoref{cor:limit-of-meas-is-meas}.

	We now check that $f_n\to f$ almost uniformly as $n\to\infty$. This is done in steps. Fix some $\varepsilon>0$. We begin by selecting our small subset to avoid. Choose some $N$ with $N>1/\varepsilon$. Note $\mu(E_N)<1/N<\varepsilon$, so we will show that $f_n|_{X\setminus E_N}\to f|_{X\setminus E_N}$ uniformly as $n\to\infty$.

	For this, we proceed as in \autoref{prop:completeboundedfuncs}. Fix any $\delta>0$. Because $\{f_n|_{X\setminus E_N}\}$ is uniformly, we are promised some $M$ such that $m,n\ge M$ and $x\notin E_N$ gives
	\[\norm{f_m(x)-f_n(x)}<\delta/2.\]
	Now, for any $x\in X$ and $n\ge M$, we see
	\[\norm{f(x)-f_n(x)}\le\norm{f(x)-f_m(x)}+\norm{f_m(x)-f_n(x)}<\norm{f(x)-f_m(x)}+\frac\delta2\]
	for any $m\ge N$. However, we see $x\notin E_N$ implies $x\notin E$, so $f(x)$ was constructed to be the limit of $\{f_m(x)\}_{m\in\NN}$, so all $m$ sufficiently large have $\norm{f(x)-f_m(x)}<\delta/2$. Ensuring that we choose an $m$ with $m\ge M$ as well allows us to conclude
	\[\norm{f(x)-f_n(x)}<\norm{f(x)-f_m(x)}+\frac\delta2<\delta\]
	for any $n\ge N$ and $x\notin E_N$.
\end{proof}
\begin{remark}
	Note that the limit $f$ is unique by \autoref{lem:uniq-almost-uniform-limit}.
\end{remark}

\end{document}