% !TEX root = ../notes.tex

\documentclass[../notes.tex]{subfiles}

\begin{document}

\section{November 7}

Today we prove \autoref{thm:rw}.

\subsection{Rapidly Cauchy Intermission}
The main idea to the proof of \autoref{thm:rw} is similar to what we have from the following definition.
\begin{definition}[Rapidly Cauchy]
	Fix a metric space $(X,d)$. Then a sequence $\{x_n\}_{n\in\NN}$ in $X$ is \textit{rapidly Cauchy} if and only if all $\varepsilon>0$ have some $N$ for which
	\[\sum_{i=1}^\infty d(x_i,x_{i+1})<\infty.\]
\end{definition}
\begin{remark}
	Of course, being rapidly Cauchy implies that any $\varepsilon>0$ has some $N$ for which
	\[\sum_{i=1}^\infty d(x_i,x_{i+1})-\sum_{i\le N}d(x_i,x_{i+1})<\varepsilon.\]
	Thus, $i>j>N$ implies
	\[d(x_i,x_j)\le\sum_{k=i}^{j-1}d(x_k,x_{k+1})\le\sum_{k>N}d(x_k,x_{k+1})<\varepsilon,\]
	so the sequence is Cauchy.
\end{remark}
However, we have the following fact.
\begin{proposition} \label{prop:cauchy-has-rapidly-cauchy}
	Fix a metric space $(X,d)$. Then any Cauchy sequence $\{x_n\}_{n\in\NN}$ is a rapidly Cauchy subsequence.
\end{proposition}
\begin{proof}
	We proceed inductively. Set $n_1=1$. Then once given $n_j$, we find $n_{j+1}>n_j$ such that $m,n\ge n_{j+1}$ such that $d(x_m,x_n)<2^{-j}$ by using the Cauchy condition. Then we can check directly that
	\[\sum_{j=1}^\infty d(x_j,x_{j+1})\le\sum_{j=1}^\infty2^{-j}=1<\infty,\]
	which is what we wanted.
\end{proof}

\subsection{The Riesz--Weyl Theorem}
And now for the main event.
\rwthm*
\begin{proof}
	We proceed as in \autoref{prop:cauchy-has-rapidly-cauchy}. Namely, set $n_1=1$. Then we proceed inductively: note that
	\[\mu\left(\{x\in X:\norm{f_m(x)-f_n(x)}>2^{-j}\}\right)\to0\]
	as $m,n\to\infty$, so we can find $n_{j+1}>n_j$ so that $m,n>n_{j+1}$ gives
	\[\mu\left(\{x\in X:\norm{f_m(x)-f_n(x)}>2^{-j}\}\right)<2^{-j}.\]
	We now claim that the sequence $\{f_{n_j}\}_{j\in\NN}$ is almost uniformly Cauchy. Well, fix any $\varepsilon>0$. Then choose $N$ for which
	\[\sum_{j\ge N}2^{-j}=2^{-N+1}<\varepsilon.\]
	As such, we set
	\[E_j\coloneqq\{x\in X:\norm{f_{n_j}(x)-f_{n_{j+1}}(x)}\ge2^{-j}\}\]
	so that $\mu(E_j)<2^{-j}$ by construction. Thus, we define our $E$ as
	\[E\coloneqq\bigcup_{j\ge N}E_j\]
	so that $\mu(E)<\varepsilon$ by construction of $N$.

	It remains to show that $\{f_{n_j}|_{X\setminus E}\}_{j\in\NN}$ is uniformly Cauchy. Well, given $\delta>0$, we need $M$ so that $i,j>M$ implies $\norm{f_{n_i}(x)-f_{n_j}(x)}<\delta$ for all $x\notin E$. Well, we set $M>N$ so that
	\[\sum_{j\ge M}2^{-j}=2^{-M+1}<\delta.\]
	In particular, it follows that any $j>i>M$ will have
	\[\norm{f_{n_i}(x)-f_{n_j}(x)}\le\sum_{k=i}^{j-1}\norm{f_{n_i}(x)-f_{n_{i+1}}(x)}\le\sum_{k=i}^{j-1}2^{-k}<\sum_{k\ge M}2^{-k}<\varepsilon,\]
	which is what we wanted.
\end{proof}
\begin{example}
	Even in \autoref{ex:the-bad-news}, there is a subsequence which is almost uniformly converging to $0$.
\end{example}
We are now ready to use the condition that we are integrating into a Banach space.
\begin{lemma}
	Fix a measure space $(X,\mc S,\mu)$ and a Banach space $B$. A sequence $\{f_n\}_{n\in\NN}$ of $\mc S$-measurable $B$-valued functions which is almost uniformly Cauchy. Then there is an $\mathcal S$-measurable function $f\colon X\to B$ such that $f_n\to f$ almost uniformly as $n\to\infty$. In fact, this $f$ is unique almost everywhere.
\end{lemma}
\begin{proof}
	For each $n\in\NN$, we get some $E_n$ such that $\mu(E_n)<1/n$ and such that $\{f_i|_{X\setminus E_n}\}_{i\in\NN}$ is uniformly Cauchy. We now set
	\[E\coloneqq\bigcap_{n=1}^\infty E_n\]
	so that $\mu(E)=0$ because $\mu(E)\le\mu(E_n)$ for each $n$ and $\mu(E_n)\to0$ as $N\to\infty$.

	Now, for any $x\in X\setminus E$, we can find $n$ for which $x\notin E_n$, so the sequence $\{f_i(x)\}_{i\in\NN}$ is Cauchy and allows us to define $f(x)$ to be the limit. Note we have used the fact that $B$ is a Banach space here! Thus far we have defined $f$ on $X\setminus E$; it doesn't matter what we do on $E$, so we will just define $f(x)=0$ for $x\in E$. Because $f$ is the limit of some $\mc S$-measurable functions, we conclude that $f$ is still $\mc S$-measurable.\footnote{Technically, we should redefine the $f_i$ to vanish on $E$.} We note that this construction is unique.

	We now check that $f_i\to f$ almost uniformly as $i\to\infty$. Well, for $\varepsilon>0$, we see that $\{f_i\}_{i\in\NN}$ is almost uniformly Cauchy on $X\setminus E_n$, where we choose $n$ with $n<2/\varepsilon$. Then for any $x\in E_n$ and $\delta>0$, we see
	\[\norm{f_n(x)-f(x)}\le\norm{f_n(x)-f_m(x)}+\norm{f_m(x)-f(x)}\]
	for any $m$. However, we can force $m$ to be large enough so that $\norm{f_m(x)-f(x)}<\delta/2$ and $\norm{f_n(x)-f_m(x)}<\delta/2$ by the uniformity. This finishes.
\end{proof}

\end{document}