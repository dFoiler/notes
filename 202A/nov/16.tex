% !TEX root = ../notes.tex

\documentclass[../notes.tex]{subfiles}

\begin{document}

\section{November 16}

Here we go.

\subsection{Integration Facts}
We continue our fact-collection.
\begin{lemma} \label{lem:almost-simple-int-are-dense}
	Fix a measure space $(X,\mc S,\mu)$ and a Banach space $B$. Further, fix a $\mu$-integrable function $f\colon X\to B$ with corresponding sequence mean Cauchy sequence $\{f_n\}_{n\in\NN}$ of simple $\mu$-integrable functions such that $f_n\to f$ in measure. Then $\norm{f-f_n}_1\to0$ as $n\to\infty$.
\end{lemma}
\begin{proof}
	This essentially follows directly from the definition of integration. Indeed, fix some $\varepsilon>0$. Our sequence is mean Cauchy, so choose some $N$ for which $m,n\ge N$ implies $\norm{f_m-f_n}_1<\varepsilon/2$.
	
	Now, for some fixed $m$, define $g_n\coloneqq f_m-f_n$ for each $n\in\NN$, which is a simple $\mu$-integrable function by \autoref{lem:simple-int-is-k-vec}, and we see $\{g_n\}_{n\in\NN}$ is mean Cauchy by \autoref{lem:linear-combo-mean-cauchy} with $g_n\to f_m-f$ in measure by \autoref{lem:linear-combo-in-measure}. Now, it follows from the proof that \autoref{lem:norm-int-function} that $\{\norm{g_n}\}_{n\in\NN}$ is still a mean Cauchy sequence of simple $\mu$-integrable functions such that $\norm{g_n}\to\norm{f_m-f}$ in measure, so
	\[\norm{f_m-f}_1=\int_X\norm{f_m-f}\,d\mu=\lim_{n\to\infty}\int_X\norm{g_n}\,d\mu=\lim_{n\to\infty}\norm{f_m-f_n}_1.\]
	(All the relevant functions are $\mu$-integrable by \autoref{lem:int-is-vec-space} and \autoref{lem:norm-int-function}.) Thus, taking $m\ge N$, we see $\norm{f_m-f_n}_1<\varepsilon/2$ for $n\ge N$, so
	\[\norm{f_m-f}_1=\lim_{n\to\infty}\norm{f_m-f_n}_1\le\varepsilon/2<\varepsilon.\]
	This completes the proof.
\end{proof}
The point of the above lemma is the following density result.
\begin{corollary} \label{cor:simple-int-are-dense}
	Fix a measure space and a Banach space $B$. For any $\mu$-integrable function $f\colon X\to B$ and error $\varepsilon>0$, there is a simple $\mu$-integrable function $g\colon X\to B$ such that $\norm{f-g}_1<\varepsilon$.
\end{corollary}
\begin{proof}
	Because $f\colon X\to B$ is integrable, there is a mean Cauchy sequence $\{f_n\}_{n\in\NN}$ of simple $\mu$-integrable functions such that $f_n\to f$ in measure. But then \autoref{lem:almost-simple-int-are-dense} tells us that
	\[\lim_{n\to\infty}\norm{f-f_n}_1=0,\]
	so there is some $N$ such that $n\ge N$ implies $\norm{f-f_n}_1<\varepsilon$. Choosing any $n\ge N$ and setting $g\coloneqq f_n$ thus finishes.
\end{proof}
\begin{lemma} \label{lem:almost-support-l1}
	Fix a measure space $(X,\mc S,\mu)$ and a normed vector space $B$. Given a $\mu$-integrable function $f\colon X\to B$ and bound $\varepsilon>0$, there is some $F\subseteq X$ with $F\in\mc S$ and $\mu(F)<\infty$ such that
	\[\int_X\norm f1_{X\setminus F}\,d\mu<\varepsilon.\]
\end{lemma}
\begin{proof}
	Because $f\colon X\to B$ is $\mu$-integrable, we may choose some simple $\mu$-integrable function $g\colon X\to B$ such that $\norm{f-g}_1<\varepsilon$, where we are using \autoref{cor:simple-int-are-dense}. Now, choose $F\coloneqq g^{-1}(B\setminus\{0\})$, which is in $\mc S$ again using \autoref{lem:simple-measurable-pre-image}, and we note \autoref{lem:fin-additive-is-monotone} implies
	\[\mu(F)\le\mu\left(g^{-1}(B\setminus\{0\})\right),\]
	where $\mu\left(g^{-1}(B\setminus\{0\})\right)$ is finite by \autoref{rem:better-simp-int}. Thus, $\mu(F)<\infty$.
	
	It remains to compute $\int_X\norm f1_{X\setminus F}\,d\mu$. Well, we see $g(x)=0$ for $x\notin F$, so
	\[(\norm f1_{X\setminus F})(x)=(\norm{f-g}1_{X\setminus F})(x)\le\norm{f-g}(x)\]
	for each $x\in X$, so \autoref{lem:bound-ints} tells us
	\[\int_X\norm f1_{X\setminus F}\,d\mu\le\int_X\norm{f-g}\,d\mu=\norm{f-g}_1<\varepsilon,\]
	which is what we wanted.
\end{proof}
\begin{remark}
	The above result basically says that $f$ is almost supported on a set of finite measure.
\end{remark}
\begin{lemma} \label{lem:bound-meas-set}
	Fix a measure space $(X,\mc S,\mu)$. Given a $\mu$-integrable function $f\colon X\to\RR$, given $E\in\mc S$ with $f(x)\ge1_E(x)$ almost everywhere, then
	\[\mu(E)\le\int_Xf\,d\mu.\]
\end{lemma}
\begin{proof}
	The main difficulty here is that we don't actually know if $1_E$ is an integrable function at the outset.
	
	For convenience, we set $F\coloneqq f^{-1}(B\setminus\{0\})$. We claim that $F$ is contained in the countable union of sets of finite measure; this is annoying, so we will brief. Well, because $f\colon X\to\RR$ is $\mu$-integrable, we can find a mean Cauchy sequence $\{f_n\}_{n\in\NN}$ of simple $\mu$-integrable functions such that $f_n\to f$ in measure. Now, $g_n\to f$ almost everywhere (because $g_n\to f$ in measure), so there is some $N\in\mc S$ such that $\mu(N)=0$ while $g_n1_{X\setminus N}\to f1_{X\setminus N}$. We now define
	\[G_n\coloneqq g_n^{-1}(B\setminus\{0\}),\]
	which is in $\mc S$ has finite measure by \autoref{rem:better-simp-int}. In particular, $f(x)\ne0$ implies that either $x\in N$ or $g_n(x)\to f(x)$ as $n\to\infty$, which requires $g_n(x)\ne0$ for some $n$ and thus $x\in G_n$ for some $n$. As such, we see
	\[F\subseteq N\cup\bigcup_{n=1}^\infty G_n,\]
	which competes the proof of the claim.

	Now, $f(x)\ge1_E(x)$ almost everywhere, so select some $N'\in\mc S$ such that $\mu(N')=0$ and $x\in X\setminus N'$ implies $f(x)\ge1_E(x)$. With this in mind, we define
	\[E_n\coloneqq (E\setminus N')\cap\left(N\cup\bigcup_{i=1}^n G_i\right).\]
	In particular, we see that \autoref{lem:fin-additive-is-monotone} and \autoref{lem:finitely-additive-is-subaddtive} imply
	\[\mu(E_n)\le\mu(N')+\mu(N)+\sum_{i=1}^n\mu(G_i)\]
	is a finite sum of finite real numbers and is therefore finite.

	As such, we note $x\in E\setminus N'$ implies $f(x)\ne0$ and thus $x\in F$, so $E\setminus N'\subseteq F$, so $E\setminus N'=\bigcup_{n=1}^\infty E_n$. Further, we see $E_n\subseteq E_n\cup G_{n+1}=E_{n+1}$ straight from the definition, so \autoref{prop:measure-union-up} tells us
	\[\mu(E\setminus N')=\lim_{n\to\infty}\mu(E_n).\]
	However, $E_n\subseteq E\setminus N'$ implies $1_{E_n}(x)\le1_{E\setminus N'}(x)$ for each $x\in X$, so $1_{E_n}(x)\le f(x)$ for $x\in X\setminus N'$, so $1_{E_n}(x)\le f(x)$ almost everywhere, so \autoref{lem:bound-ints} tells us
	\[\int_X1_{E_n}\,d\mu\le\int_Xf\,d\mu.\]
	Noting $\mu(E_n)=\int_X1_{E_n}\,d\mu$ by \autoref{ex:integrate-indicator}, we see $\mu(E_n)\le\int_Xf\,d\mu$ for each $n$. It follows that
	\[\mu(E\setminus N')\le\int_Xf\,d\mu.\]
	However, $\mu(N')=0$, so $\mu(E\cap N')=0$ by \autoref{lem:fin-additive-is-monotone}, so $\mu(E\setminus N')=\mu(E)-\mu(E\cap N')=\mu(E)$. This finishes.
\end{proof}
\begin{corollary}
	Fix a measure space $(X,\mc S,\mu)$ and a normed vector space $B$. Further, fix a simple $\mc S$-measurable function $f\colon X\to B$ and a $\mu$-integrable function $g\colon X\to\RR$. If $\norm{f(x)}\le g(x)$ almost everywhere, then $f$ is $\mu$-integrable.
\end{corollary}
\begin{proof}
	Fixing any $y\in(\im f)\setminus\{0\}$, we have to show that $f^{-1}(\{y\})$ has finite measure. Well, by \autoref{lem:fin-additive-is-monotone}, we can just show $E\coloneqq f^{-1}(B\setminus\{0\})$ has finite measure, where $E\in\mc S$ already.

	For this, we note that $\im f$ is finite, so $\{\norm{y}:y\in(\im f)\setminus\{0\}\}$ is finite and therefore has a minimum value $r$. Note $r>0$ because $\norm y=0$ implies $y=0$. As such, we note that
	\[r1_E(x)\le\norm{f}(x)\]
	for all $x\in X$ because either $x\notin E$ and thus $f(x)=0$ or $x\in E$ and thus $r\le\norm{f(x)}$. It follows $1_E(x)\le\frac1r\norm f(x)\le\frac1rg(x)$ almost everywhere, so \autoref{lem:bound-meas-set} tells us that $E$ has finite measure. In particular, $\frac 1rg$ is $\mu$-integrable by \autoref{lem:int-is-vec-space}. 
\end{proof}
\begin{lemma}
	Fix a measure space $(X,\mc S,\mu)$ and a normed vector space $B$. Given a $\mu$-integrable function $f\colon X\to B$, if $\norm f_1=0$, then $\mu\left(f^{-1}(B\setminus\{0\})\right)=0$.
\end{lemma}
\begin{proof}
	Note that the sequence $\{f_n\}_{n\in\NN}$ of functions defined by $f_n(x)\coloneqq0$ are all simple $\mu$-integrable functions (vacuously) and is also mean Cauchy because $\norm{f_m-f_n}_1=0$ for all $m$ and $n$. We also note that $f_n\to0$ in measure, and $f_n\to f$ in measure by tracking through our convergences, so $f=0$ almost everywhere by \autoref{lem:uniq-limit-in-measure}. The result follows.
\end{proof}

\subsection{Completeness of \texorpdfstring{$L^1$}{ L1}}
We now move towards showing that $\mc L^1$ is complete. To state the result, we need to (re)define converging in mean.
\begin{definition}[Converge in mean]
	Fix a measure space $(X,\mc S,\mu)$ and a normed vector space $B$. Then a sequence $\{f_n\}_{n\in\NN}$ of $\mu$-integrable functions \textit{converges in mean} to a $\mu$-integrable function $f\colon X\to B$ if and only if $\norm{f-f_n}_1\to0$ as $n\to\infty$.
\end{definition}
\begin{definition}[Mean Cauchy]
	Fix a measure space $(X,\mc S,\mu)$ and a normed vector space $B$. Then a sequence $\{f_n\}_{n\in\NN}$ of $\mu$-integrable functions is \textit{mean Cauchy} if and only if $\norm{f_m-f_n}_1\to0$ as $m,n\to\infty$.
\end{definition}
And now for our feature presentation.
\begin{proposition} \label{prop:l1-complete}
	Fix a measure space $(X,\mc S,\mu)$ and a Banach space $B$. Then a mean Cauchy sequence $\{f_n\}_{n\in\NN}$ of $\mu$-integrable functions converges in mean to some $\mu$-integrable function $f\colon X\to B$.
\end{proposition}
\begin{proof}
	For each $n$, we may choose some simple $\mu$-integrable function $g_n\colon X\to B$ such that $\norm{f_n-g_n}_1<1/n$, where we are using \autoref{cor:simple-int-are-dense}. We can check directly that $\{g_n\}_{n\in\NN}$ is mean Cauchy, so we may find some $f\colon X\to B$ such that $g_n\to f$ in measure. It follows $\norm{f-g_n}_1\to0$ as $n\to\infty$, so $\norm{f-f_n}_1\to0$ as $n\to\infty$ follows.
\end{proof}
In order to actually state this as a completeness result, we need to turn the semi-norm $\norm\cdot_1$ into an actual norm.
\begin{notation}
	Fix a measure space $(X,\mc S,\mu)$ and a normed vector space $B$. We set $\mc N(X,\mc S,\mu,B)\coloneqq\{f\in\mc L(X,\mc S,\mu,B):\norm f_1=0\}$ and
	\[L^1(X,\mc S,\mu,B)\coloneqq\mc L^1(X,\mc S,\mu,B)/\mc N(X,\mc S,\mu,B)\]
\end{notation}
\begin{lemma}
	Fix a measure space $(X,\mc S,\mu)$ and a normed vector space $B$. The function $\norm\cdot_1$ descends to a norm on $L^1(X,\mc S,\mu,B)$.
\end{lemma}
\begin{proof}
	Apply \autoref{prop:semi-norm-to-norm}.
\end{proof}
\begin{corollary} \label{cor:l1-complete}
	Fix a measure space $(X,\mc S,\mu)$ and a Banach space $B$. Then $L^1(X,\mc S,\mu,B)$ is complete.
\end{corollary}
\begin{proof}
	This follows directly from \autoref{prop:l1-complete}.
\end{proof}
\begin{remark}
	Using \autoref{cor:simple-int-are-dense}, one can now check that the simple $\mu$-integrable functions have dense image in $L^1(X,\mc S,\mu,B)$. So $L^1(X,\mc S,\mu,B)$ is in fact the completion of the space of simple $\mu$-integrable functions (modded out by the ``null functions'' as usual).
\end{remark}
Next class we will begin trying to compute integrals.

\end{document}