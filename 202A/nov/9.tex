% !TEX root = ../notes.tex

\documentclass[../notes.tex]{subfiles}

\begin{document}

\section{November 9}

Today we define integrable functions. We went through this discussion quickly last class but are now going through it in more detail, so I have just moved the exposition to today.

\subsection{Convergence in Mean}
We are going to want yet another notion of convergence, to align with our desire to integrate.
\begin{defi}[Converge in mean]
	Fix a measure space $(X,\mc S,\mu)$ and a normed vector space $B$. Then a sequence $\{f_n\}_{n\in\NN}$ of simple $\mu$-integrable functions \textit{converges in mean} to a simple $\mu$-integrable function $f$ if and only if $\norm{f-f_n}_1\to0$ as $n\to\infty$.
\end{defi}
\begin{defi}[Mean Cauchy]
	Fix a normed vector space $B$ and a measure space $(X,\mc S,\mu)$. A sequence of simple $\mu$-integrable functions $\{f_n\}_{n\in\NN}$ is \textit{mean Cauchy} if and only if it is Cauchy for the semi-norm $\norm\cdot_1$. In other words, we require
	\[\lim_{m,n\to\infty}\norm{f_m-f_n}_1=0.\]
\end{defi}
\begin{remark}
	Because simple $\mu$-integrable functions form a vector space by \autoref{lem:simple-int-is-k-vec}, we see that $\norm{f-f_n}_1$ and $\norm{f_m-f_n}_1$ are legal expressions.
\end{remark}
Here are the usual checks.
\begin{lemma}
	Fix a normed $k$-vector space $(B,\norm\cdot)$ and a measure space $(X,\mc S,\mu)$. Further, fix a sequence of simple $\mu$-integrable functions $f_n\colon X\to B$ and $g_n\colon X\to B$ with $f_n\to f$ and $g_n\to g$ in mean as $n\to\infty$.
	\begin{listalph}
		\item We have $f_n+g_n\to f+g$ in mean.
		\item Given some scalar $a\in k$, we have $af_n\to af$ in mean.
		\item We have $\norm{f_n}\to\norm f$ in mean.
	\end{listalph}
\end{lemma}
\begin{proof}
	For (a) and (b), note the relevant functions are simple $\mu$-integrable by \autoref{lem:simple-int-is-k-vec}; for (c), the relevant functions are simple $\mu$-integrable by \autoref{lem:norm-is-simple-int}. Now, (a) and (b) follow directly from \autoref{lem:converge-linear-combo}, where we are using the fact that $\norm\cdot_1$ is a semi-norm by \autoref{lem:simple-l1-seminorm}.
	
	It remains to show (c). For any $\varepsilon>0$, we are promised $N$ such that $n\ge N$ implies
	\[\norm{f-f_n}_1<\varepsilon.\]
	By the reverse triangle inequality, we see
	\[|\norm{f(x)}-\norm{f_n(x)}|\le\norm{f(x)-f_n(x)}\]
	for each $x\in X$, so \autoref{cor:simple-bound-ints} tells us
	\[|\norm f-\norm{f_n}|_1=\int_X|\norm f-\norm{f_n}|\,d\mu\le\int_X\norm{f-f_n}\,d\mu=\norm{f-f_n}_1<\varepsilon\]
	for each $n\ge N$. This finishes.
\end{proof}
\begin{lemma} \label{lem:linear-combo-mean-cauchy}
	Fix a normed $k$-vector space $(B,\norm\cdot)$ and a measure space $(X,\mc S,\mu)$. Further, fix a sequence of simple $\mu$-integrable functions $\{f_n\}_{n\in\NN}$ and $\{g_n\}_{n\in\NN}$ which are mean Cauchy.
	\begin{listalph}
		\item The sequence $\{f_n+g_n\}_{n\in\NN}$ is mean Cauchy.
		\item Given some scalar $a\in k$, the sequence $\{af_n\}_{n\in\NN}$ is mean Cauchy.
		\item The sequence $\{\norm{f_n}\}_{n\in\NN}$ is mean Cauchy.
	\end{listalph}
\end{lemma}
\begin{proof}
	These proofs are essentially identical. For (a) and (b), note the relevant functions are simple $\mu$-integrable by \autoref{lem:simple-int-is-k-vec}; for (c), the relevant functions are simple $\mu$-integrable by \autoref{lem:norm-is-simple-int}. As before, (a) and (b) follow from \autoref{lem:cauchy-linear-combo} upon noting $\norm\cdot_1$ is a semi-norm by \autoref{lem:simple-l1-seminorm}.

	It remains to show (c). For any $\varepsilon>0$, we are promised $N$ such that $m,n\ge N$ implies
	\[\norm{f_m-f_n}_1<\varepsilon.\]
	By the reverse triangle inequality, we see
	\[|\norm{f_m(x)}-\norm{f_n(x)}|\le\norm{f(x)-f_n(x)}\]
	for each $x\in X$, so \autoref{cor:simple-bound-ints} tells us
	\[|\norm {f_m}-\norm{f_n}|_1=\int_X|\norm {f_m}-\norm{f_n}|\,d\mu\le\int_X\norm{{f_m}-f_n}\,d\mu=\norm{f_m-f_n}_1<\varepsilon\]
	for each $m,n\ge N$. This finishes.
\end{proof}
\begin{lemma} \label{lem:restrict-mean-cauchy}
	Fix a normed $k$-vector space $(B,\norm\cdot)$ and a measure space $(X,\mc S,\mu)$. Given a sequence of simple $\mu$-integrable function $\{f_n\}_{n\in\NN}$ which is mean Cauchy and a set $E\in\mc S$, then the sequence $\{f_n1_E\}_{n\in\NN}$ is still mean Cauchy.
\end{lemma}
\begin{proof}
	Fix some $\varepsilon>0$. We are promised some $N$ such that $m,n\ge N$ implies $\norm{f_m-f_n}_1<\varepsilon$. Now, for any $x\in X$, we see
	\[\norm{f_m1_E-f_n1_E}(x)=\big(\norm{f_m-f_n}1_E\big)(x)\le\norm{f_m-f_n}(x),\]
	so \autoref{cor:simple-bound-ints} tells us
	\[\norm{f_m1_E-f_n1_E}_1=\int_X\norm{f_m1_E-f_n1_E}\,d\mu\le\int_X\norm{f_m-f_n}\,d\mu=\norm{f_m-f_n}_1,\]
	which is less than $\varepsilon$ for $m,n\ge N$. This is what we wanted.
\end{proof}
\begin{lemma} \label{lem:mean-cauchy-subsequence}
	Fix a normed vector space $(B,\norm\cdot)$ and a measure space $(X,\mc S,\mu)$. If $\{f_n\}_{n\in\NN}$ is a mean Cauchy sequence of simple $\mu$-integrable functions, then any subsequence $\{f_{n_i}\}_{i\in\NN}$ is a mean Cauchy sequence of simple $\mu$-integrable functions.
\end{lemma}
\begin{proof}
	For any $\varepsilon>0$, we are given $N$ such that $m,n\ge N$ implies $\norm{f_m-f_n}_1<\varepsilon$. Because $n_i\ge n$ for each $i$, we see $i,j\ge N$ has $\norm{f_{n_i}-f_{n_j}}_1<\varepsilon$ as well, which is what we wanted.
\end{proof}

\subsection{Comparing Convergences}
We are going to want to see the comparative strengths of different convergences. Here is a starting result, which was moved from an earlier lecture for thematic reasons. Note this generalizes \autoref{ex:fixing-bad-news}.
\begin{lemma} \label{lem:mean-cauchy-is-cauchy-in-measure}
	Fix a normed vector space $(B,\norm\cdot)$ and a measure space $(X,\mc S,\mu)$. Then a sequence of simple $\mu$-integrable functions $f_n\colon X\to B$ for $n\in\NN$ which is mean Cauchy is also Cauchy in measure.
\end{lemma}
\begin{proof}
	Fix $\varepsilon>0$ and set
	\[E^\varepsilon_{m,n}\coloneqq\{x\in X:\norm{f_m(x)-f_n(x)}\ge\varepsilon\},\]
	which has finite measure by \autoref{rem:sets-in-cauchy-in-measure-are-finite}. We need to show that
	\[\lim_{m,n\to\infty}\mu(E_{m,n}^\varepsilon)\stackrel?=0.\]
	Notably, for each $x\in X$, we must have
	\begin{equation}
		1_{E^\varepsilon_{m,n}}(x)\le\frac{\norm{f_m(x)-f_n(x)}}\varepsilon \label{eq:bound-indicator-e-m-n}
	\end{equation}
	by definition of $E^{\varepsilon}_{m,n}$. Now, both sides of this equation are simple $\mu$-integrable functions: $1_{E^\varepsilon_{m,n}}$ is by \autoref{ex:indicator-is-simple}; and $f_m-f_n$ is simple $\mu$-integrable by \autoref{lem:simple-int-is-k-vec}, as is $\norm{f_m-f_n}$ by \autoref{lem:norm-is-simple-int}, so $\frac1\varepsilon\norm{f_m-f_n}$ is simple $\mu$-integrable by \autoref{lem:simple-int-is-k-vec} again.
	
	Thus, we may integrate, for which \autoref{cor:simple-bound-ints} tells us
	\[\mu(E^\varepsilon_{m,n})=\int_X1_{E^{\varepsilon}_{m,n}}\,d\mu\le\int_X\frac{\norm{f_m-f_n}}\varepsilon\,d\mu=\frac{\norm{f_m-f_n}_1}\varepsilon,\]
	where the first integral was computed using \autoref{ex:integrate-indicator}. But as $m,n\to\infty$, the right-hand value goes to $0$ because $\{f_n\}_{n\in\NN}$ is Cauchy for $\norm\cdot_1$, so the left-hand value must also go to $0$.
\end{proof}
\begin{remark}
	A similar proof works for when we are Cauchy for $\norm\cdot_p$ for finite $p$ by taking $p$th powers of \autoref{eq:bound-indicator-e-m-n}. For example, in probability theory, the result for $\norm\cdot_2$ is essentially Chebyshev's inequality.
\end{remark}
We now note that converging almost uniformly is stronger than in measure.
\begin{lemma} \label{lem:almost-uniform-to-in-measure}
	Fix a normed vector space $(B,\norm\cdot)$ and a measure space $(X,\mc S,\mu)$. Further, fix a sequence $\{f_n\}_{n\in\NN}$ of $\mc S$-measurable functions and an $\mathcal S$-measurable function $f$.
	\begin{listalph}
		\item If $f_n\to f$ almost uniformly as $n\to\infty$, then $f_n\to f$ in measure.
		\item If $\{f_n\}_{n\in\NN}$ is almost uniformly Cauchy, then $\{f_n\}_{n\in\NN}$ is Cauchy in measure.
	\end{listalph}
\end{lemma}
\begin{proof}
	Here we go.
	\begin{listalph}
		\item For any $\varepsilon>0$, we need to show that
		\[\lim_{n\to\infty}\mu(\{x\in X:\norm{f(x)-f_n(x)}\})\stackrel?=0.\]
		Well, for any $\delta>0$, we need $N$ such that $n\ge N$ has
		\[\mu(\{x\in X:\norm{f(x)-f_n(x)})\stackrel?<\delta.\]
		Now, by the almost uniform convergence, we are promised $F\in\mc S$ such that $\mu(X\setminus F)<\delta$ and $f_n\to f$ uniformly as $n\to\infty$ on $F$. Now using our uniform convergence, we choose $N$ such that $n\ge N$ implies
		\[\norm{f(x)-f_n(x)}<\varepsilon\]
		for each $x\in F$. In particular, for $n\ge N$, we see
		\[\{x\in X:\norm{f(x)-f_n(x)}\ge\varepsilon\}\subseteq X\setminus F,\]
		so \autoref{lem:fin-additive-is-monotone} tells us
		\[\mu(\{x:\norm{f(x)-f_n(x)}\ge\varepsilon\})\le\mu(X\setminus F)<\delta,\]
		which finishes.
		\item This proof is essentially the same. For any $\varepsilon>0$, we need to show that
		\[\lim_{n\to\infty}\mu(\{x\in X:\norm{f_m(x)-f_n(x)}\})\stackrel?=0.\]
		Well, for any $\delta>0$, we need $N$ such that $m,n\ge N$ has
		\[\mu(\{x\in X:\norm{f_m(x)-f_n(x)})\stackrel?<\delta.\]
		Now, by the almost uniform convergence, we are promised $F\in\mc S$ such that $\mu(X\setminus F)<\delta$ and $\{f_n\}_{n\in\NN}$ is uniformly Cauchy on $F$. Now using the fact we're uniformly Cauchy, we choose $N$ such that $m,n\ge N$ implies
		\[\norm{f_m(x)-f_n(x)}<\varepsilon\]
		for each $x\in F$. In particular, for $m,n\ge N$, we see
		\[\{x\in X:\norm{f_m(x)-f_n(x)}\ge\varepsilon\}\subseteq X\setminus F,\]
		so \autoref{lem:fin-additive-is-monotone} tells us
		\[\mu(\{x:\norm{f_m(x)-f_n(x)}\ge\varepsilon\})\le\mu(X\setminus F)<\delta,\]
		which finishes.
		\qedhere
	\end{listalph}
\end{proof}
Further, convergence almost uniformly is stronger than convergence almost everywhere.
\begin{lemma} \label{lem:almost-uniform-to-almost-everywhere}
	Fix a normed vector space $(B,\norm\cdot)$ and a measure space $(X,\mc S,\mu)$. Further, fix a sequence $\{f_n\}_{n\in\NN}$ of functions which converge to $f$ almost uniformly as $n\to\infty$. Then $f_n\to f$ almost everywhere.
\end{lemma}
\begin{proof}
	Let $E$ be the set of points such that $\{f_n(x)\}_{n\in\NN}$ does not converge to $f(x)$ as $n\to\infty$. Note $x\in E$ is equivalent to having some $\varepsilon>0$ and $N$ such that $\norm{f_n(x)-f(x)}\ge\varepsilon$ for each $n\ge N$, so
	\[E=\bigcup_{\varepsilon>0}\bigcup_{N=1}^\infty\bigcap_{n\ge N}\{x\in X:\norm{f(x)-f_n(x)}\ge\varepsilon\}.\]
	Note that $\norm{f_n(x)-f(x)}\ge\varepsilon$ for some $\varepsilon>0$ and all $n\ge N$ is equivalent to $\norm{f(x)-f_n(x)}\ge1/m$ for some $m$ and all $n\ge N$ because $1/m\to0$ as $m\to\infty$. So in fact
	\[E=\bigcup_{m=1}^\infty\bigcup_{N=1}^\infty\bigcap_{n\ge N}\{x\in X:\norm{f(x)-f_n(x)}\ge1/m\}.\]
	By \autoref{lem:premeas-is-countable-subadd}, it suffices to show that
	\[\mu\Bigg(\bigcap_{n\ge N}\{x\in X:\norm{f(x)-f_n(x)}\ge1/m\}\Bigg)\stackrel?=0\]
	for each $m$ and $N$.
	
	Well, $f_n\to f$ almost uniformly, so any $\delta>0$ has $F\in\mc S$ such that $\mu(X\setminus F)<\delta$ and $f_n\to f$ uniformly as $n\to\infty$. Thus, there is $N'$ such that $n\ge N'$ implies $\norm{f(x)-f_n(x)}<1/m$ for all $x\in X$ so that
	\[\{x\in X:\norm{f(x)-f_n(x)}\ge1/m\}\subseteq X\setminus F.\]
	In particular, choosing any $n_0$ greater than both $N$ and $N'$, we see from \autoref{lem:fin-additive-is-monotone} that
	\[\mu\Bigg(\bigcap_{n\ge N}\{x\in X:\norm{f(x)-f_n(x)}\ge1/m\}\Bigg)\le\mu(\{x\in X:\norm{f(x)-f_{n_0}(x)}\ge1/m\})\le\mu(X\setminus F)<\delta.\]
	It follows $\mu\left(\bigcap_{n\ge N}\{x\in X:\norm{f(x)-f_n(x)}\ge1/m\}\right)=0$.
\end{proof}

\subsection{Integrable Functions}
Our payoff to our hard work is a definition of integrable functions. Here it is.
\begin{theorem} \label{thm:int-funcs}
	Fix a normed $k$-vector space $(B,\norm\cdot)$ and a measure space $(X,\mc S,\mu)$. Then given an $\mathcal S$-measurable function $f$, the following are equivalent.
	\begin{listalph}
		\item There is a mean Cauchy sequence of simple $\mu$-integrable functions that converges to $f$ in measure.
		\item There is a mean Cauchy sequence of simple $\mu$-integrable functions that converges to $f$ almost uniformly.
		\item There is a mean Cauchy sequence of simple $\mu$-integrable functions that converges to $f$ almost everywhere.
	\end{listalph}
\end{theorem}
\begin{proof}
	We show our implications in sequence. In all parts, let $\{f_n\}_{n\in\NN}$ be the requested mean Cauchy sequence of simple $\mu$-integrable functions.
	\begin{itemize}
		\item We show (a) implies (b). This holds from the Riesz--Weyl theorem. Namely, by \autoref{thm:rw}, $\{f_n\}_{n\in\NN}$ will have a subsequence $\{f_{n_i}\}_{i\in\NN}$ which is almost uniformly Cauchy; this subsequence remains mean Cauchy by \autoref{lem:mean-cauchy-subsequence}.

		It remains to show that $f_{n_i}\to f$ almost uniformly as $i\to\infty$. By \autoref{lem:almost-uniform-cauchy-converges}, we see that $f_{n_i}\to g$ almost uniformly for some $\mc S$-measurable function $g\colon X\to B$, but then \autoref{lem:almost-uniform-to-in-measure} tells us $f_{n_i}\to g$ in measure.

		However, $f_n\to f$ in measure implies that $f_{n_i}\to f$ in measure by \autoref{lem:in-measure-subsequence}, so $f=g$ almost everywhere by \autoref{lem:uniq-limit-in-measure}, so $f_{n_i}\to f$ almost uniformly by \autoref{lem:uniq-almost-uniform-limit}.
		% subsequences of mean cauchy is mean cauchy
		\item We show (b) implies (c) and (a). Well, converging almost uniformly automatically forces us to converge in measure by \autoref{lem:almost-uniform-to-in-measure} and almost everywhere by \autoref{lem:almost-uniform-to-almost-everywhere}.
		\item We show (c) implies (a). Well, if $\{f_n\}_{n\in\NN}$ is mean Cauchy, then the sequence is Cauchy in measure by \autoref{lem:almost-uniform-to-in-measure} and therefore has a subsequence $\{f_{n_i}\}_{i\in\NN}$ which is almost uniformly Cauchy by \autoref{thm:rw}. Notably, $f_{n_i}\to f$ almost everywhere because $f_n\to f$ almost everywhere, using the same null set.
		
		However, this subsequence $\{f_{n_i}\}_{i\in\NN}$ will then converge to some $\mc S$-measurable $g\colon X\to B$ almost uniformly by \autoref{lem:almost-uniform-cauchy-converges}, so $f_{n_i}\to g$ almost everywhere by \autoref{lem:almost-uniform-to-almost-everywhere}. It follows that $f=g$ almost everywhere,\footnote{We know $f_n\to f$ outside some null set $F$, and $f_n\to g$ outside some null set $G$, so $f(x)=g(x)$ outside the null set $F\cup G$.} so $f_{n_i}\to f$ almost uniformly by \autoref{lem:uniq-almost-uniform-limit}.
		% subsequences of mean cauchy is mean cauchy
		% subsequences of almost uniformly converges almost uniformly
		\qedhere
	\end{itemize}
\end{proof}
As such, we have the following definition.
\begin{definition}[Integrable]
	Fix a measure space $(X,\mc S,\mu)$ and a normed vector space $B$. Then an $\mathcal S$-measurable function $f\colon X\to B$ is \textit{$\mu$-integrable} if and only if one of the equivalent conditions from \autoref{thm:int-funcs} is satisfied. This set of integrable functions is often denoted $\mc L^1(X,\mc S,\mu,B)$, where some data might be omitted when we want to.
\end{definition}
\begin{remark}
	Later on, we will define
	\[\int_X f\,d\mu\coloneqq\lim_{n\to\infty}\int_Xf_n\,d\mu.\]
	However, we have not yet checked that this definition is well-defined.
\end{remark}
\begin{remark}
	Later on we will also define $\mc L^\infty(X,\mc S,\mu,B)$ as the bounded $\mc S$-measurable functions as well as more general $\mc L^p(X,\mc S,\mu,B)$ for finite $p$ where
	\[\int_X\norm{f}^p\,d\mu<\infty.\]
	As an example fact, we can see that $\mc L^1(X,\mc S,\mu,B)$ is a module over the ring $\mc L^\infty(X,\mc S,\mu,k)$, where $B$ is a normed $k$-vector space. We will not check this here.
\end{remark}
\begin{remark}
	Morally perhaps, one should define integrable functions to be merely $\mu$-measurable instead of $\mc S$-measurable. I have not done this for technical reasons because I find it exceedingly annoying to have to keep removing a null set. If this distinction is distressing, then replace $\mc S$ with the $\sigma$-algebra generated by $\mc S$ and the null sets of $\mu$.
\end{remark}
\begin{example} \label{ex:simple-int-is-int}
	If $f$ is a simple $\mu$-integrable function, then the sequence $\{f\}_{n\in\NN}$ is mean Cauchy and converges to $f$ almost everywhere, so $f$ is also a $\mu$-integrable function.
\end{example}
Here are the usual checks.
\begin{lemma} \label{lem:int-is-vec-space}
	Fix a measure space $(X,\mc S,\mu)$ and a normed $k$-vector space $B$. Then $\mc L^1(X,\mc S,\mu,B)$ forms a $k$-vector space.
\end{lemma}
\begin{proof}
	Here are our checks. For brevity, set $\mc L^1\coloneqq\mc L^1(X,\mc S,\mu,B)$.
	\begin{itemize}
		\item Zero: note that the zero function is $1_\emp$ and thus a simple $\mu$-integrable function (by \autoref{ex:integrate-indicator}) and thus a simple $\mu$-integrable function (by \autoref{ex:simple-int-is-int}).
		\item Addition: given $f,g\in\mc L^1$, we show $f+g\in\mc L^1$. Well, pick up mean Cauchy sequences $\{f_n\}_{n\in\NN}$ and $\{g_n\}_{n\in\NN}$ of simple $\mu$-integrable functions which converge in measure to $f$ and $g$ respectively.
		
		Now, note $\{f_n+g_n\}_{n\in\NN}$ is mean Cauchy by \autoref{lem:linear-combo-mean-cauchy}, and $f_n+g_n\to f+g$ in measure by \autoref{lem:linear-combo-cauchy-in-measure}, so $f+g\in\mc L^1$.
		\item Scalar multiplication: given a scalar $a\in k$ and $f\in\mc L^1$, we show $af\in\mc L^1$. Well, pick up our mean Cauchy sequence $\{f_n\}_{n\in\NN}$ of simple $\mu$-integrable functions which converges to $f$ in measure. Then $\{af_n\}_{n\in\NN}$ is mean Cauchy by \autoref{lem:linear-combo-mean-cauchy} and converges in measure to $af$ by \autoref{lem:linear-combo-in-measure}.
		\qedhere
	\end{itemize}
\end{proof}
\begin{lemma} \label{lem:restrict-int-function}
	Fix a measure space $(X,\mc S,\mu)$ and a normed vector space $B$. Given a $\mu$-integrable function $f\colon X\to B$ and measurable set $E\in\mc S$, the function $f1_E$ is still $\mu$-integrable.
\end{lemma}
\begin{proof}
	As usual, pick up our mean Cauchy sequence $\{f_n\}_{n\in\NN}$ of simple $\mu$-integrable functions converging to $f$ in measure. Then \autoref{lem:restrict-meas-functions} tells us that $f_n1_E$ is still simple $\mu$-integrable. Further, \autoref{lem:restrict-mean-cauchy} tells us $\{f_n1_E\}_{n\in\NN}$ is still mean Cauchy, and \autoref{lem:restrict-converge-in-measure} tells us $f_n1_E\to f1_E$ in measure. Thus, $f1_E$ is in fact $\mu$-integrable.
\end{proof}
\begin{remark} \label{rem:complement-restrict-int}
	As in \autoref{rem:complement-restrict-meas-functions}, we note that $E\in\mc S$ will have $1_X=1_E+1_{X\setminus E}$, so $f1_{X\setminus E}=f-f1_E$ is still $\mu$-integrable by \autoref{lem:restrict-int-function} and \autoref{lem:int-is-vec-space}.
\end{remark}
\begin{lemma} \label{lem:norm-int-function}
	Fix a measure space $(X,\mc S,\mu)$ and a normed vector space $B$. Given a $\mu$-integrable function $f\colon X\to B$, the function $\norm f$ is still $\mu$-integrable.
\end{lemma}
\begin{proof}
	As usual, pick up our mean Cauchy sequence $\{f_n\}_{n\in\NN}$ of simple $\mu$-integrable functions converging to $f$ in measure. Then \autoref{lem:norm-is-simple-int} tells us that each $\norm{f_n}$ is still a simple $\mu$-integrable function. As such, we see \autoref{lem:linear-combo-mean-cauchy} tells us $\{\norm{f_n}\}_{n\in\NN}$ is mean Cauchy, and $\norm{f_n}\to\norm f$ in measure by \autoref{lem:linear-combo-in-measure}. It follows $\norm f$ is $\mu$-integrable.
\end{proof}

\subsection{Towards Defining Integrals}
We now move towards defining integration.
\begin{lemma}
	Fix a measure space $(X,\mc S,\mu)$ and a normed $k$-vector space $B$. Further, fix mean Cauchy sequences of simple $\mu$-integrable functions $\{f_n\}_{n\in\NN}$ and $\{g_n\}_{n\in\NN}$ which converges to $f$ and $g$ in measure, respectively. If $\norm{f_n-g_n}_1\to0$ as $n\to\infty$, then $f=g$ almost everywhere.
\end{lemma}
\begin{proof}
	The key trick is to consider the sequence $f_1,g_1,f_2,g_2,\ldots$. To be explicit, define $\{h_n\}_{n\in\NN}$ by $h_{2n}=f_n$ and $h_{2n-1}=g_n$. Here are our checks on $\{h_n\}_{n\in\NN}$.
	\begin{itemize}
		\item Note that each $n\in\NN$ has $h_n$ is either an $f_i$ or $g_i$ and is therefore a simple $\mu$-integrable functions.
		\item We claim that $\{h_n\}_{n\in\NN}$ is Cauchy in measure; it suffices to show that $\{h_n\}_{n\in\NN}$ is mean Cauchy by \autoref{lem:mean-cauchy-is-cauchy-in-measure}.

		Well, fix any $\varepsilon>0$. Because $\{f_n\}_{n\in\NN}$ and $\{g_n\}_{n\in\NN}$ are mean Cauchy, we get $N_f$ and $N_g$ such that
		\[m,n\ge N_f\implies\norm{f_m-f_n}_1<\varepsilon\qquad\text{and}\qquad m,n\ge N_g\implies\norm{f_m-f_n}_1<\varepsilon.\]
		Further, $\norm{f_n-g_n}_1\to0$ as $n\to\infty$, so we get $N'$ such that $n\ge N'$ implies $\norm{f_n-g_n}_1<\varepsilon$.

		Combining, set $N\coloneqq\max\{2N_f,2N_g+1,2N'\}$. Then, for $m,n\ge N$, we have three cases according to parity.
		\begin{itemize}
			\item If $m=2k$ and $n=2\ell$, then $k,\ell\ge N_f$, so $\norm{h_m-h_n}_1=\norm{f_k-f_\ell}_1<\varepsilon$.
			\item If $m=2k+1$ and $n=2\ell+1$
		\end{itemize}
		The case with $m$ odd and $n$ even is analogous to the last one, by symmetry. This finishes our check.
		\item Note $h_{2n}=f_n$ for each $n$, so the subsequence $\{h_{2n}\}_{n\in\NN}$ converges to $f$ in measure. Thus, \autoref{lem:subsequence-has-same-limit-in-measure} tells us $h_n\to f$ in measure.
		\item Analogously, note $h_{2n-1}=g_n$ for each $n$, so the subsequence $\{h_{2n-1}\}_{n\in\NN}$ converges to $g$ in measure. Thus, \autoref{lem:subsequence-has-same-limit-in-measure} tells us $h_n\to g$ in measure.
	\end{itemize}
	From the above checks, we see from \autoref{lem:uniq-limit-in-measure} that $f=g$ almost everywhere.
\end{proof}
The point here is that we can take equivalence classes in $\mathcal L^1(X,\mc S,\mu,B)$ to get a bona fide norm from our semi-norm $\norm\cdot_1$.

To finish our discussion of completeness, we will need the following result, which we will state but not prove today.
\begin{restatable}{proposition}{intisdefinedprop} \label{prop:equivalent-mean-cauchy}
	Fix a measure space $(X,\mc S,\mu)$ and a normed vector space $(B,\norm\cdot)$. Suppose $\{f_n\}_{n\in\NN}$ and $\{g_n\}_{n\in\NN}$ are mean Cauchy sequences of simple $\mu$-integrable functions which both converge to some $\mc S$-measurable function $f$ in measure. Then $\norm{f_n-g_n}_1\to0$ as $n\to\infty$.
\end{restatable}
\noindent Roughly speaking, this will imply that the integral $\int_Xf\,d\mu$ is well-defined.

\end{document}