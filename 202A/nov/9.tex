% !TEX root = ../notes.tex

\documentclass[../notes.tex]{subfiles}

\begin{document}

\section{November 6}

Today we define integrable functions. We went through this discussion quickly last class but are now going through it in more detail, so I have just moved the exposition to today.

\subsection{Integrable Functions}
Here is the payoff to our hard work.
\begin{definition}[Mean Cauchy]
	Fix a measure space $(X,\mc S,\mu)$ and a normed vector space $B$. Then a sequence $\{f_n\}_{n\in\NN}$ of simple integrable functions is \textit{mean Cauchy} if and only if $\norm{f_n-f_m}_1\to0$ as $m,n\to\infty$.
\end{definition}
\begin{lemma}
	Fix a measure space $(X,\mc S,\mu)$ and a Banach space $B$. Further, fix a sequence $\{f_n\}_{n\in\NN}$ of $\mc S$-measurable functions.
	\begin{listalph}
		\item If $f_n\to f$ almost uniformly as $n\to\infty$, then $f_n\to f$ in measure.
		\item If $\{f_n\}_{n\in\NN}$ is almost uniformly Cauchy, then $\{f_n\}_{n\in\NN}$ is Cauchy in measure.
	\end{listalph}
\end{lemma}
\begin{proof}
	We show (a); the proof of (b) is similar. Fix $\varepsilon>0$, and we would like to show that
	\[\lim_{n\to\infty}\mu(\{x:\norm{f(x)-f_n(x)}\ge\varepsilon\})\stackrel?=0.\]
	Well, for any $\delta>0$, we need $N$ such that $n\ge N$ has
	\[\mu(\{x:\norm{f(x)-f_n(x)}\ge\varepsilon\})<\delta.\]
	Because the $f_\bullet$ are $\mc S$-measurable, we may find some $E\in\mc S$ containing each $f_n^{-1}(B\setminus\{0\})$.

	Now, by the almost uniform convergence, we are promised $F\in\mc S$ such that $\mu(F)<\delta$ and $f_n\to f$ uniformly as $n\to\infty$ on $X\setminus F$. As such, we choose $N$ such that $n\ge N$implies
	\[\norm{f(x)-f_n(x)}<\varepsilon.\]
	In particular, for $n\ge N$, we see
	\[\mu(\{x:\norm{f(x)-f_n(x)}\ge\varepsilon\})\le\mu(X\setminus F)<\delta,\]
	which finishes.
\end{proof}
\begin{theorem} \label{thm:int-funcs}
	Fix a measure space $(X,\mc S,\mu)$ and a Banach space $B$. Given an $\mathcal S$-measurable function $f$, the following are equivalent.
	\begin{listalph}
		\item There is a mean Cauchy sequence of simple integrable functions that converges to $f$ in measure.
		\item There is a mean Cauchy sequence of simple integrable functions that converges to $f$ almost uniformly.
		\item There is a mean Cauchy sequence of simple integrable functions that converges to $f$ almost everywhere.
	\end{listalph}
\end{theorem}
\begin{proof}
	We show our implications in sequence.
	\begin{itemize}
		\item We show (a) implies (b). This holds from \autoref{thm:rw}: namely, our mean Cauchy sequence converging in measure will have a subsequence which is almost uniformly Cauchy. Then we note that the limits are the same by checking almost everywhere.
		% \item We show (b) implies (a). Well, any sequence which is almost uniformly Cauchy is automatically Cauchy in measure, finishing.
		\item We show (b) implies (c). Well, converging almost uniformly automatically forces us to converge both in measure and almost everywhere.
		\item We show (c) implies (a). Well, if $\{f_n\}_{n\in\NN}$ is mean Cauchy, then the sequence is Cauchy in measure and therefore has a subsequence which is almost uniformly Cauchy. However, this subsequence will then converge to some $g$ almost everywhere, and we see that $f=g$ almost everywhere, so this subsequence actually converges to $f$ almost everywhere.
		\qedhere
	\end{itemize}
\end{proof}
As such, we have the following definition.
\begin{definition}[Integrable]
	Fix a measure space $(X,\mc S,\mu)$ and a normed vector space $B$. Then an $\mathcal S$-measurable function $f\colon X\to B$ is \textit{$\mu$-integrable} if and only if one of the equivalent conditions from \autoref{thm:int-funcs} is satisfied. This set of integrable functions is often denoted $\mc L^1(X,\mc S,\mu,B)$, where some data might be omitted when we want to.
\end{definition}
\begin{remark}
	Later on, we will define
	\[\int_X f\,d\mu\coloneqq\lim_{n\to\infty}\int_Xf_n\,d\mu.\]
	However, we have not yet checked that this definition is well-defined.
\end{remark}
\begin{remark}
	Later on we will also define $\mc L^\infty(X,\mc S,\mu,B)$ as the bounded $\mc S$-measurable functions as well as more general $\mc L^p(X,\mc S,\mu,B)$ for finite $p$ where
	\[\int_X\norm{f}^p\,d\mu.\]
	As an example fact, we can see that $\mc L^1(X,\mc S,\mu,B)$ is a module over $\mc L^\infty(X,\mc S,\mu,k)$, where $B$ is a normed $k$-vector space.
\end{remark}
Here are the usual checks.
\begin{lemma}
	Fix a measure space $(X,\mc S,\mu)$ and a normed $k$-vector space $B$. Then $\mc L^1(X,\mc S,\mu,B)$ forms a $k$-vector space.
\end{lemma}
\begin{proof}
	Just take linear combinations of the requested sequences in \autoref{thm:int-funcs}.
\end{proof}

\subsection{Completeness of \texorpdfstring{$\mathcal L^1$}{ L1}}
We would like for our $\mathcal L^1$ to actually be a completion for our simple integrable functions.
\begin{lemma} \label{lem:subsequence-has-same-limit}
	Fix a measure space $(X,\mc S,\mu)$ and a normed $k$-vector space $B$. Further, fix a sequence of simple $\mc S$-measurable functions $\{f_n\}_{n\in\NN}$ which is Cauchy in measure. If a subsequence $\{f_{n_k}\}_{k\in\NN}$ converges to a function $f$ in measure, then $\{f_n\}_{n\in\NN}$ fully converges to $f$ in measure.
\end{lemma}
\begin{proof}
	Fix some $\varepsilon>0$. Then we note that any $k$ will have
	\[\{x\in X:\norm{f_n(x)-f(x)}\ge\varepsilon\}\subseteq\{x\in X:\norm{f_{n_k}(x)-f(x)}\ge\varepsilon/2\}\cup\{x\in X:\norm{f_n(x)-f_{n_k}(x)}\ge\varepsilon/2\}.\]
	Now taking $k$ large enough recovers the result; notably, the left term is small because our subsequence converges in measure, and the right term is small because our sequence is Cauchy in measure.
\end{proof}
\begin{lemma}
	Fix a measure space $(X,\mc S,\mu)$ and a normed $k$-vector space $B$. Further, fix mean Cauchy sequences of simple $\mu$-integrable functions $\{f_n\}_{n\in\NN}$ and $\{g_n\}_{n\in\NN}$ which converges to $f$ and $g$ in measure, respectively. If $\norm{f_n-g_n}_1\to0$ as $n\to\infty$, then $f=g$ almost everywhere.
\end{lemma}
\begin{proof}
	The sequence $f_1,g_1,f_2,g_2,\ldots$ is mean Cauchy and hence Cauchy in measure. However, the subsequence $\{f_n\}_{n\in\NN}$ tells us that the sequence converges to $f$ in measure by \autoref{lem:subsequence-has-same-limit}; similarly, the subsequence $\{g_n\}_{n\in\NN}$ tells us that the sequence converges to $g$ in measure as well. Thus, our single sequence converges to both $f$ and $g$ in measure, so $f=g$ almost everywhere
\end{proof}
The point here is that we can take equivalence classes in $\mathcal L^1(X,\mc S,\mu,B)$ to get a bona fide norm from our semi-norm $\norm\cdot_1$.

To finish our discussion of completeness, we will need the following result, which we will state but not prove today.
\begin{proposition}
	Fix a measure space $(X,\mc S,\mu)$ and a normed $k$-vector space $B$. Suppose that $\{f_n\}_{n\in\NN}$ and $\{g_n\}_{n\in\NN}$ are mean Cauchy sequences of simple $\mu$-integrable functions which both converge to some function $f$ in measure. Then the sequences $\{f_n\}_{n\in\NN}$ and $\{g_n\}_{n\in\NN}$ are equivalent mean Cauchy sequences.
\end{proposition}
In particular, this will imply that the integral $\int_Xf\,d\mu$ is well-defined.

\end{document}