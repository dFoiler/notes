% !TEX root = ../notes.tex

\documentclass[../notes.tex]{subfiles}

\begin{document}

\section{November 2}

We begin class by finishing the proof of \autoref{thm:better-measurable}. I have simply edited that proof for continuity reasons.

\subsection{Some Measurable Facts}
We now use \autoref{thm:better-measurable} for fun and profit.
\begin{corollary} \label{cor:limit-of-meas-is-meas}
	Fix a normed vector space $B$ and a set $X$ with a $\sigma$-ring $\mc S$ on $X$. If a sequence of functions $\{f_n\}_{n\in\NN}$ are $\mc S$-measurable, and $f_n\to f$ pointwise as $n\to\infty$, then $f$ is also $\mc S$-measurable.
\end{corollary}
\begin{proof}
	By \autoref{thm:better-measurable}, we have two checks.
	\begin{listroman}
		\item We show that $\im f$ is separable. Well, each $\im f_n$ is separable by \autoref{thm:better-measurable}, so this follows from \autoref{lem:limit-of-sep-ims-has-sep-im}.
		\item We show that $f^{-1}(U\setminus\{0\})\in\mc S$ for each open $U\subseteq B$. Well, each $f_n$ has $f_n^{-1}(U\setminus\{0\})\in\mc S$ for each open $U\subseteq B$, so the same holds for $f$ by \autoref{lem:limit-of-sep-ims-has-sep-im}.
	\end{listroman}
	The above checks show that $f$ is $\mc S$-measurable by \autoref{thm:better-measurable}.
\end{proof}
\begin{corollary} \label{cor:take-norms-is-measurable}
	Fix a normed vector space $(B,\norm\cdot)$ and a set $X$ with a $\sigma$-ring $\mc S$ on $X$. If $f$ is an $\mathcal S$-measurable function, then $x\mapsto\norm{f(x)}$ is as well.
\end{corollary}
\begin{proof}
	For brevity, set $g\colon X\to\RR$ by $g(x)\coloneqq\norm{f(x)}$. By \autoref{thm:better-measurable}, there are two checks.
	\begin{listroman}
		\item Note that $\im g\subseteq\RR$ must be separable by \autoref{ex:all-sep-in-r}, so there is nothing more to say here.
		\item For any open $U\subseteq\RR$, we see that
		\[U'\coloneqq\{x\in B:\norm x\in U\}\]
		is open in $B$ because $x\mapsto\norm x$ is continuous by \autoref{ex:norm-is-cont}. Thus, $g^{-1}(U)=f^{-1}(U')\in\mc S$ because $f$ is $\mc S$-measurable.
		\qedhere
	\end{listroman}
\end{proof}

\end{document}