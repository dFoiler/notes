% !TEX root = ../notes.tex

\documentclass[../notes.tex]{subfiles}

\begin{document}

\section{November 2}

We begin class by finishing the proof of \autoref{thm:better-measurable}. I have simply edited that proof for continuity reasons.

\subsection{Some Measurable Facts}
We now use \autoref{thm:better-measurable} for fun and profit.
\begin{corollary}
	Fix a normed vector space $B$ and a set $X$ with a $\sigma$-ring $\mc S$ on $X$. If a sequence of functions $\{f_n\}_{n\in\NN}$ are $\mc S$-measurable, and $f_n\to f$ pointwise as $n\to\infty$, then $f$ is also $\mc S$-measurable.
\end{corollary}
\begin{proof}
	By \autoref{thm:better-measurable}, we have two checks.
	\begin{listroman}
		\item We show that $\im f$ is separable. Well, each $\im f_n$ is separable by \autoref{thm:better-measurable}, so this follows from \autoref{lem:limit-of-sep-ims-has-sep-im}.
		\item We show that $f^{-1}(U\setminus\{0\})\in\mc S$ for each open $U\subseteq B$. Well, each $f_n$ has $f_n^{-1}(U\setminus\{0\})\in\mc S$ for each open $U\subseteq B$, so the same holds for $f$ by \autoref{lem:limit-of-sep-ims-has-sep-im}.
	\end{listroman}
	The above checks show that $f$ is $\mc S$-measurable by \autoref{thm:better-measurable}.
\end{proof}
\begin{lemma} \label{lem:norm-is-cont}
	Fix a normed vector space $(B,\norm\cdot)$. Then the function $\norm\cdot\colon B\to\RR$ is continuous.
\end{lemma}
\begin{proof}
	We proceed directly; in fact, we claim that $\norm\cdot$ is Lipschitz continuous, which will be enough by \autoref{ex:lip-is-uni-cont} and \autoref{ex:uniform-cont-is-cont}.

	Well, observe that $\norm x\le\norm{x-y}+\norm y$, so by symmetry, it follows that
	\[|\norm{x}-\norm y|\le\norm{x-y}.\]
	This finishes.
\end{proof}
\begin{corollary} \label{cor:take-norms-is-measurable}
	Fix a normed vector space $(B,\norm\cdot)$ and a set $X$ with a $\sigma$-ring $\mc S$ on $X$. If $f$ is an $\mathcal S$-measurable function, then $x\mapsto\norm{f(x)}$ is as well.
\end{corollary}
\begin{proof}
	For brevity, set $g\colon X\to\RR$ by $g(x)\coloneqq\norm{f(x)}$. By \autoref{thm:better-measurable}, there are two checks.
	\begin{listroman}
		\item Note that $\im g\subseteq\RR$ must be separable by \autoref{ex:all-sep-in-r}, so there is nothing more to say here.
		\item For any open $U\subseteq\RR$, we see that
		\[U'\coloneqq\{x\in B:\norm x\in U\}\]
		is open in $B$ because $x\mapsto\norm x$ is continuous by \autoref{lem:norm-is-cont}. Thus, $g^{-1}(U)=f^{-1}(U')\in\mc S$ because $f$ is $\mc S$-measurable.
		\qedhere
	\end{listroman}
\end{proof}

\subsection{Integrating Measurable Functions}
We begin by picking up some facts about our integral.
\begin{lemma}
	Fix a ring $\mc S$ on a set $X$ equipped with a finitely additive measure $\mu$ and a normed vector space $B$. Then the mapping
	\[f\mapsto\int_Xf\,d\mu\]
	from simple $\mu$-integrable functions to $B$ is $k$-linear.
\end{lemma}
\begin{proof}
	Unsurprisingly, we use the ideas of \autoref{lem:simple-int-is-k-vec} to compute our integrals. We have two checks.
	\begin{itemize}
		\item Scalar multiplication: fix a simple $\mu$-integrable function $f$ and a scalar $r\in k$. If $r=0$, then $rf=0$, so $\int_X(rf)\,d\mu=0=r\int_Xf\,d\mu$ vacuously, so there is nothing more to say.

		Otherwise, we have $r\ne0$, and we remarked in \autoref{lem:simple-int-is-k-vec} that we have
		\[(rf)^{-1}(\{y\})=f^{-1}(\{1/r\cdot y\}).\]
		In other words, $\im rf=\{ry:y\in\im f\}$ with $(rf)^{-1}(\{ry\})=f^{-1}(\{y\})$ for each $y\in\im f$, so
		\begin{align*}
			\int_X(rf)\,d\mu &= \sum_{y\in(\im rf)\setminus\{0\}}\mu\left((rf)^{-1}(\{y\})\right)y \\
			&= \sum_{ry\in(\im rf)\setminus\{0\}}\mu\left((rf)^{-1}(\{ry\})\right)\cdot ry \\
			&= r\sum_{y\in(\im f)\setminus\{0\}}\mu\left(f^{-1}(\{y\})\right)y \\
			&= r\int_Xf\,d\mu.
		\end{align*}
		\item Addition: fix simple $\mu$-integrable functions $f$ and $g$. We remarked in \autoref{lem:simple-int-is-k-vec} that any $y\in B\setminus\{0\}$ will have
		\[(f+g)^{-1}(\{y\})=\bigcup_{c\in(\im g)}\left(f^{-1}(\{y-c\})\cap g^{-1}(\{y\})\right)=\bigcup_{\substack{b\in(\im f),c\in(\im g)\\b+c=y}}\left(f^{-1}(\{b\})\cap g^{-1}(\{c\})\right).\]
		Now, note that this union is in fact disjoint because the fibers $f^{-1}(\{b\})$ are disjoint. Thus, we may say that
		\[\mu\left((f+g)^{-1}(\{y\})\right)=\sum_{\substack{b\in(\im f),c\in(\im g)\\b+c=y}}\mu\left(f^{-1}(\{b\})\cap g^{-1}(\{c\})\right).\]
		Looping through all $y$, we see
		\begin{align*}
			\int_X(f+g)\,d\mu &= \sum_{y\in\im(f+g)\setminus\{0\}}\mu\left((f+g)^{-1}(\{y\})\right)y \\
			&= \sum_{y\in\im(f+g)\setminus\{0\}}\sum_{\substack{b\in(\im f),c\in(\im g)\\b+c=y}}\mu\left(f^{-1}(\{b\})\cap g^{-1}(\{c\})\right)(b+c) \\
			&= \sum_{b\in(\im f)}\sum_{c\in(\im g)}\mu\left(f^{-1}(\{b\})\cap g^{-1}(\{c\})\right)(b+c) \\
			&= \sum_{b\in(\im f)}\sum_{c\in(\im g)}\mu\left(f^{-1}(\{b\})\cap g^{-1}(\{c\})\right)b+\sum_{c\in(\im g)}\sum_{b\in(\im f)}\mu\left(f^{-1}(\{b\})\cap g^{-1}(\{c\})\right)c.
		\end{align*}
		Now, we note that
		\[\bigsqcup_{b\in(\im f)}f^{-1}(\{b\})=X\qquad\text{and}\qquad\bigsqcup_{c\in(\im g)}g^{-1}(\{c\})=X\]
		because the fibers should cover the domain and are disjoint. It follows that from the finite additivity of $\mu$ that
		\[\int_X(f+g)\,d\mu = \sum_{b\in(\im f)}\mu\left(f^{-1}(\{b\})\right)b+\sum_{c\in(\im g)}\mu\left(g^{-1}(\{c\})\right)c,\]
		which is $\int_Xf\,d\mu+\int_Xg\,d\mu$, which is what we wanted.
		\qedhere
	\end{itemize}
\end{proof}
\begin{lemma}
	If $f$ and $g$ are real-valued, then $f(x)\ge g(x)$ for each $x$ implies $\int f\,d\mu\ge\int f\,d\mu$.
\end{lemma}
\begin{proof}
	Computation.
\end{proof}
We now give a semi-norm on our functions.
\begin{definition}
	Given a simple integrable function $f$, we define
	\[\norm f_1\coloneqq\int_X\norm f\,d\mu.\]
\end{definition}
\begin{lemma}
	Given simple integrable functions $f$ and $g$, we have $\norm{f+g}_1\le\norm f_1+\norm g_1$.
\end{lemma}
To make this a norm, we need to remove the problematic functions.
\begin{notation}
	We define
	\[\mc N(X,\mc S,\mu)=\{f\text{ simple integrable}:\norm f_1=0\}.\]
\end{notation}
So we see that we're going to get a norm on the quotient of all simple integrable functions by $\mc N(X,\mc S,\mu)$. We are interested in elements of the completion here.
\begin{example}
	Give $[0,1]$ the usual Lebesgue measure $\mu$. Further, let $\{E_i\}_{i=1}^\infty$ be pairwise disjoint Borel subsets of $\RR$, where $\mu(E_i)\le 4^{-i}$ for each $i$. Then we see that
	\[\sum_{i=1}^n1_{E_i}\to\sum_{i=1}^\infty1_{E_i}\]
	as $n\to\infty$, but the function on the right may be potentially quite hard to handle. Namely, we want
	\[\int_Xf\,d\mu=\lim_{n\to\infty}\int_Xf\,d\mu=\sum_{i=1}^\infty\mu(E_i),\]
	where we have used countable additivity at the end here.
\end{example}

\subsection{Convergence in Measure}
In order to avoid the constant repetition of hypotheses, we pick up the following definition.
\begin{definition}[Measure space]
	A \textit{measure space} is a triple $(X,\mc S,\mu)$, where $\mc S$ is a $\sigma$-ring and $\mu$ is a measure on $\mc S$.
\end{definition}
Now, let me tell you the bad news.
\begin{warn}
	A sequence $\{f_n\}_{n\in\NN}$ of simple integrable functions which is Cauchy for $\norm\cdot_1$ need not converge pointwise, at any point!
\end{warn}
\begin{example} \label{ex:the-bad-news}
	Give $[0,1]$ the usual Lebesgue measure $\mu$. Then the sequence of functions
	\[1_{[0,1/2]},\quad1_{[1/2,1]},\quad1_{[0,1/4]},\quad1_{[1/4,2/4]},\quad1_{[2/4,3/4]},\quad\cdots\]
	should approach $0$, but it does not converge to $0$ pointwise.
\end{example}
We close class by discussing the following definition.
\begin{definition}[Converge in measure]
	Fix a normed vector space $(B,\norm\cdot)$ and a $\sigma$-ring $\mc S$ on a set $X$ equipped with a measure $\mu$. Then a sequence $\{f_n\}_{n\in\NN}$ of $\mc S$-measurable functions \textit{converges in measure} to a simple $\mc S$-measurable function $f$ if and only if all $\varepsilon>0$ have
	\[\lim_{n\to\infty}\mu(\{x:\norm{f(x)-f_n(x)}\ge\varepsilon\})=0.\]
\end{definition}
\begin{remark}
	Notably, $f$ and $f_n$ are both simple $\mc S$-measurable, so $f-f_n$ is $\mc S$-measurable, so $g\coloneqq\norm{f-f_n}$ is $\mc S$-measurable, so $\{x:\norm{f(x)-f_n(x)}\ge\varepsilon\}=g^{-1}([\varepsilon,\infty))$ is in fact in $\mc S$.
\end{remark}
\begin{definition}[Cauchy in measure]
	Fix a normed vector space $(B,\norm\cdot)$ and a $\sigma$-ring $\mc S$ on a set $X$ equipped with a measure $\mu$. Then a sequence $\{f_n\}_{n\in\NN}$ of $\mc S$-measurable functions is \textit{Cauchy in measure} if and only if all $\varepsilon>0$ have
	\[\lim_{m,n\to\infty}\mu(\{x:\norm{f_m(x)-f_n(x)}\ge\varepsilon\})=0.\]
\end{definition}
The point of these definitions is that the sequence from \autoref{ex:the-bad-news} converges in measure to the zero function.

\end{document}