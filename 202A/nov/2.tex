% !TEX root = ../notes.tex

\documentclass[../notes.tex]{subfiles}

\begin{document}

\section{November 2}

We begin class by finishing the proof of \autoref{thm:better-measurable}. I have simply edited that proof for continuity reasons.

\subsection{Some Measurable Facts}
We now use \autoref{thm:better-measurable} for fun and profit.
\begin{corollary}
	Fix a normed vector space $B$ and a set $X$ with a $\sigma$-ring $\mc S$ on $X$. If a sequence of functions $\{f_n\}_{n\in\NN}$ are $\mc S$-measurable, and $f_n\to f$ pointwise as $n\to\infty$, then $f$ is also $\mc S$-measurable.
\end{corollary}
\begin{proof}
	By \autoref{thm:better-measurable}, we have two checks.
	\begin{listroman}
		\item We show that $\im f$ is separable. Well, each $\im f_n$ is separable by \autoref{thm:better-measurable}, so this follows from \autoref{lem:limit-of-sep-ims-has-sep-im}.
		\item We show that $f^{-1}(U\setminus\{0\})\in\mc S$ for each open $U\subseteq B$. Well, each $f_n$ has $f_n^{-1}(U\setminus\{0\})\in\mc S$ for each open $U\subseteq B$, so the same holds for $f$ by \autoref{lem:limit-of-sep-ims-has-sep-im}.
	\end{listroman}
	The above checks show that $f$ is $\mc S$-measurable by \autoref{thm:better-measurable}.
\end{proof}
\begin{lemma} \label{lem:norm-is-cont}
	Fix a normed vector space $(B,\norm\cdot)$. Then the function $\norm\cdot\colon B\to\RR$ is continuous.
\end{lemma}
\begin{proof}
	We proceed directly; in fact, we claim that $\norm\cdot$ is Lipschitz continuous, which will be enough by \autoref{ex:lip-is-uni-cont} and \autoref{ex:uniform-cont-is-cont}.

	Well, observe that $\norm x\le\norm{x-y}+\norm y$, so by symmetry, it follows that
	\[|\norm{x}-\norm y|\le\norm{x-y}.\]
	This finishes.
\end{proof}
\begin{corollary}
	Fix a normed vector space $(B,\norm\cdot)$ and a set $X$ with a $\sigma$-ring $\mc S$ on $X$. If $f$ is an $\mathcal S$-measurable function, then $x\mapsto\norm{f(x)}$ is as well.
\end{corollary}
\begin{proof}
	For brevity, set $g\colon X\to\RR$ by $g(x)\coloneqq\norm{f(x)}$. By \autoref{thm:better-measurable}, there are two checks.
	\begin{listroman}
		\item Note that $\im g\subseteq\RR$ must be separable by \autoref{ex:all-sep-in-r}, so there is nothing more to say here.
		\item For any open $U\subseteq\RR$, we see that
		\[U'\coloneqq\{x\in B:\norm x\in U\}\]
		is open in $B$ because $x\mapsto\norm x$ is continuous by \autoref{lem:norm-is-cont}. Thus, $g^{-1}(U)=f^{-1}(U')\in\mc S$ because $f$ is $\mc S$-measurable.
		\qedhere
	\end{listroman}
\end{proof}

\subsection{Integrating Measurable Functions}
We begin by picking up some facts about simple integrable functions.
\begin{lemma}
	Fix a normed $k$-vector space $B$ and a set $X$ with a ring $\mc S$ on $X$ equipped with a finitely additive measure $\mu$. Then the set of all simple integrable functions forms a $k$-vector space under pointwise operations.
\end{lemma}
\begin{proof}
	Use ideas from \autoref{lem:simple-meas-is-k-vec}.
\end{proof}
\begin{lemma}
	The function $f\mapsto\int_Xf\,d\mu$ is $k$-linear.
\end{lemma}
\begin{proof}
	Add directly.
\end{proof}
\begin{lemma}
	The function $x\mapsto\norm{f(x)}$ is a simple integrable function.
\end{lemma}
\begin{proof}
	Computation.
\end{proof}
\begin{lemma}
	If $f$ and $g$ are real-valued, then $f(x)\ge g(x)$ for each $x$ implies $\int f\,d\mu\ge\int f\,d\mu$.
\end{lemma}
\begin{proof}
	Computation.
\end{proof}
We now give a semi-norm on our functions.
\begin{definition}
	Given a simple integrable function $f$, we define
	\[\norm f_1\coloneqq\int_X\norm f\,d\mu.\]
\end{definition}
\begin{lemma}
	Given simple integrable functions $f$ and $g$, we have $\norm{f+g}_1\le\norm f_1+\norm g_1$.
\end{lemma}
To make this a norm, we need to remove the problematic functions.
\begin{notation}
	We define
	\[\mc N(X,\mc S,\mu)=\{f\text{ simple integrable}:\norm f_1=0\}.\]
\end{notation}
So we see that we're going to get a norm on the quotient of all simple integrable functions by $\mc N(X,\mc S,\mu)$. We are interested in elements of the completion here.
\begin{example}
	Give $[0,1]$ the usual Lebesgue measure $\mu$. Further, let $\{E_i\}_{i=1}^\infty$ be pairwise disjoint Borel subsets of $\RR$, where $\mu(E_i)\le 4^{-i}$ for each $i$. Then we see that
	\[\sum_{i=1}^n1_{E_i}\to\sum_{i=1}^\infty1_{E_i}\]
	as $n\to\infty$, but the function on the right may be potentially quite hard to handle. Namely, we want
	\[\int_Xf\,d\mu=\lim_{n\to\infty}\int_Xf\,d\mu=\sum_{i=1}^\infty\mu(E_i),\]
	where we have used countable additivity at the end here.
\end{example}

\subsection{Convergence in Measure}
Now, let me tell you the bad news.
\begin{warn}
	A sequence $\{f_n\}_{n\in\NN}$ of simple integrable functions which is Cauchy for $\norm\cdot_1$ need not converge pointwise, at any point!
\end{warn}
\begin{example} \label{ex:the-bad-news}
	Give $[0,1]$ the usual Lebesgue measure $\mu$. Then the sequence of functions
	\[1_{[0,1/2]},\quad1_{[1/2,1]},\quad1_{[0,1/4]},\quad1_{[1/4,2/4]},\quad1_{[2/4,3/4]},\quad\cdots\]
	should approach $0$, but it does not converge to $0$ pointwise.
\end{example}
We close class by discussing the following definition.
\begin{definition}[Converge in measure]
	Fix a normed vector space $(B,\norm\cdot)$ and a $\sigma$-ring $\mc S$ on a set $X$ equipped with a measure $\mu$. Then a sequence $\{f_n\}_{n\in\NN}$ of $\mc S$-measurable functions \textit{converges in measure} to a simple $\mc S$-measurable function $f$ if and only if all $\varepsilon>0$ have
	\[\lim_{n\to\infty}\mu(\{x:\norm{f(x)-f_n(x)}\ge\varepsilon\})=0.\]
\end{definition}
\begin{remark}
	Notably, $f$ and $f_n$ are both simple $\mc S$-measurable, so $f-f_n$ is $\mc S$-measurable, so $g\coloneqq\norm{f-f_n}$ is $\mc S$-measurable, so $\{x:\norm{f(x)-f_n(x)}\ge\varepsilon\}=g^{-1}([\varepsilon,\infty))$ is in fact in $\mc S$.
\end{remark}
\begin{definition}[Cauchy in measure]
	Fix a normed vector space $(B,\norm\cdot)$ and a $\sigma$-ring $\mc S$ on a set $X$ equipped with a measure $\mu$. Then a sequence $\{f_n\}_{n\in\NN}$ of $\mc S$-measurable functions is \textit{Cauchy in measure} if and only if all $\varepsilon>0$ have
	\[\lim_{m,n\to\infty}\mu(\{x:\norm{f_m(x)-f_n(x)}\ge\varepsilon\})=0.\]
\end{definition}
The point of these definitions is that the sequence from \autoref{ex:the-bad-news} converges in measure to the zero function.

\end{document}