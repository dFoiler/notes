% !TEX root = ../notes.tex

\documentclass[../notes.tex]{subfiles}

\begin{document}

\section{November 2}

We begin class by finishing the proof of \autoref{thm:better-measurable}. I have simply edited that proof for continuity reasons.

\subsection{Some Measurable Facts}
We now use \autoref{thm:better-measurable} for fun and profit.
\begin{corollary} \label{cor:limit-of-meas-is-meas}
	Fix a normed vector space $B$ and a set $X$ with a $\sigma$-ring $\mc S$ on $X$. If a sequence of functions $\{f_n\}_{n\in\NN}$ are $\mc S$-measurable, and $f_n\to f$ pointwise as $n\to\infty$, then $f$ is also $\mc S$-measurable.
\end{corollary}
\begin{proof}
	By \autoref{thm:better-measurable}, we have two checks.
	\begin{listroman}
		\item We show that $\im f$ is separable. Well, each $\im f_n$ is separable by \autoref{thm:better-measurable}, so this follows from \autoref{lem:limit-of-sep-ims-has-sep-im}.
		\item We show that $f^{-1}(U\setminus\{0\})\in\mc S$ for each open $U\subseteq B$. Well, each $f_n$ has $f_n^{-1}(U\setminus\{0\})\in\mc S$ for each open $U\subseteq B$, so the same holds for $f$ by \autoref{lem:limit-of-sep-ims-has-sep-im}.
	\end{listroman}
	The above checks show that $f$ is $\mc S$-measurable by \autoref{thm:better-measurable}.
\end{proof}
\begin{corollary} \label{cor:take-norms-is-measurable}
	Fix a normed vector space $(B,\norm\cdot)$ and a set $X$ with a $\sigma$-ring $\mc S$ on $X$. If $f$ is an $\mathcal S$-measurable function, then $x\mapsto\norm{f(x)}$ is as well.
\end{corollary}
\begin{proof}
	For brevity, set $g\colon X\to\RR$ by $g(x)\coloneqq\norm{f(x)}$. By \autoref{thm:better-measurable}, there are two checks.
	\begin{listroman}
		\item Note that $\im g\subseteq\RR$ must be separable by \autoref{ex:all-sep-in-r}, so there is nothing more to say here.
		\item For any open $U\subseteq\RR$, we see that
		\[U'\coloneqq\{x\in B:\norm x\in U\}\]
		is open in $B$ because $x\mapsto\norm x$ is continuous by \autoref{ex:norm-is-cont}. Thus, $g^{-1}(U)=f^{-1}(U')\in\mc S$ because $f$ is $\mc S$-measurable.
		\qedhere
	\end{listroman}
\end{proof}
\begin{example} \label{ex:min-max-meas}
	If $f\colon X\to\RR$ is $\mc S$-measurable, then \autoref{cor:take-norms-is-measurable} tells us that $|f|$ is also $\mc S$-measurable. As such, if $f,g\colon X\to\RR$ are $\mc S$-measurable, then $(f+g)$ and $(f-g)$ are $\mc S$-measurable by \autoref{lem:meas-is-vec-space}, so $|f-g|$ is $\mc S$-measurable, so
	\[\min\{f,g\}=\frac{(f+g)+|f-g|}2\qquad\text{and}\qquad\max\{f,g\}=\frac{(f+g)-|f-g|}2\]
	are $\mc S$-measurable by \autoref{lem:meas-is-vec-space} again. Inducting, for any $\mc S$-measurable functions $\{f_i\}_{i=1}^n$, the minimum function $\min\{f_1,\ldots,f_n\}$ and maximum function $\max\{f_1,\ldots,f_n\}$ are both $\mc S$-measurable.
\end{example}
We next talk a little about restriction.
\begin{lemma} \label{lem:restrict-meas-functions}
	Fix a normed vector space $B$ and a measure space $(X,\mc S,\mu)$ and a set $E\in\mc S$. If $f\colon X\to B$ is simple $\mc S$-measurable or $\mc S$-measurable or simple $\mu$-integrable, then $f1_E$ is as well.
\end{lemma}
\begin{proof}
	Before doing anything, we pick up a few facts. Note that
	\[\im f1_E=\{f(x)1_E(x):x\in X\}\subseteq\{0\}\cup\{f(x):x\in E\}\subseteq\{0\}\cup\im f.\]
	Also, if $S\subseteq B\setminus\{0\}$, then we claim
	\[(f1_E)^{-1}(S)=E\cap f^{-1}(S).\]
	In one direction, note $x\in E\cap f^{-1}(S)$ implies that $(f1_E)(x)=f(x)\in S$. In the other direction, if $x\in(f1_E)^{-1}(S)$, then note $x\in E$ is forced because otherwise $f(x)=0\notin S$. Thus, with $x\in E$, we have $(f1_E)(x)=f(x)$, so $(f1_E)(x)\in S$ forces $x\in f^{-1}(S)$ as well.

	We now note that we actually have three claims to show, which we show in sequence.
	\begin{itemize}
		\item Suppose that $f$ is a simple $\mc S$-measurable function. As such, $\im f$ is finite, so $\im f1_E\subseteq\{0\}\cup\im f$ is also finite.
		
		Further, for each $y\in(\im f1_E)\setminus\{0\}$, we see that $(f1_E)^{-1}(\{y\})=E\cap f^{-1}(\{y\})$ as discussed above, which lives in $\mc S$ because $E\in\mc S$ and $f^{-1}(\{y\})\in\mc S$.
		\item Suppose that $f$ is an $\mathcal S$-measurable function. Then $\im f$ is separable, so it follows $\{0\}\cup\im f$ is separable (by \autoref{ex:union-of-seps-is-sep}), so $\im f1_E\subseteq\{0\}\cup\im f$ is separable (by \autoref{rem:subspace-of-sep-is-sep}).

		Now, for any open subset $U\subseteq B\setminus\{0\}$, we see $(f1_E)^{-1}(U)=E\cap f^{-1}(U)$ as discussed above, which lives in $\mc S$ because $E\in\mc S$ and $f^{-1}(\{y\})\in\mc S$.
		\item Suppose that $f$ is a simple $\mu$-integrable function. As before, $\im f$ is finite implies that $\im f1_E\subseteq\{0\}\cup\im f$ is still finite.

		Further, for each $y\in(\im f1_E)\setminus\{0\}$, we see $(f1_E)^{-1}(\{y\})=E\cap f^{-1}(\{y\})$, which saw in our first point to live in $\mc S$, but now we note that \autoref{lem:fin-additive-is-monotone} tells us
		\[\mu\left((f1_E)^{-1}(\{y\})\right)\le\mu\left(f^{-1}(\{y\})\right)<\infty\]
		is finite.
		\qedhere
	\end{itemize}
\end{proof}
\begin{remark} \label{rem:complement-restrict-meas-functions}
	On the other hand, if $X\setminus E\in\mc S$, then we see that $f1_E$ still gets the relevant adjectives. Indeed, each of the classes is a vector space (by \autoref{lem:simple-meas-is-k-vec} and \autoref{lem:meas-is-vec-space} and \autoref{lem:simple-int-is-k-vec}), so it's enough to see $f1_E=f-f1_{X\setminus E}$ and apply \autoref{lem:restrict-meas-functions}.
\end{remark}
\begin{corollary} \label{cor:mu-meas-to-s-meas}
	Fix a measure space $(X,\mc S,\mu)$ and a normed vector space $B$. Further, fix a $\mu$-measurable function $f\colon X\to B$. Then there is some $N\in\mc S$ such that $\mu(N)=0$ while $f1_N$ is $\mc S$-measurable.
\end{corollary}
\begin{proof}
	Because $f$ is $\mu$-measurable, there is a sequence of simple $\mc S$-measurable functions $\{f_n\}_{n\in\NN}$ such that $f_n\to f$ almost everywhere. Thus, there is some $N\in\mc S$ such that $\mu(N)=0$ while $f_n(x)\to f(x)$ as $n\to\infty$ for each $x\in X\setminus N$.

	We now show that $f1_N$ is $\mc S$-measurable. Indeed, we claim that $f_n1_N\to f1_N$ as $n\to\infty$ pointwise, which will finish because each $f_n1_N$ is simple $\mc S$-measurable by \autoref{lem:restrict-meas-functions}. If $x\notin N$, then we're just asking for $f_n(x)\to f(x)$ as $n\to\infty$, which we know. On the other hand, if $x\notin N$, then we're asking for $0\to0$ as $n\to\infty$, for which there's nothing to say.
\end{proof}

\end{document}