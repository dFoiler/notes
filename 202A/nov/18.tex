% !TEX root = ../notes.tex

\documentclass[../notes.tex]{subfiles}

\begin{document}

\section{November 18}

There will be at most two more homework assignments.

\subsection{Measures from Integrals}
Now that we have a reasonable notion of what functions to integrate, given a measure, we would like to take these integrable functions to build measures. It will be convenient to have the following notation.
\begin{notation}
	Fix a measure space $(X,\mc S,\mu)$ and a Banach space $B$. Given a $\mu$-integrable function $f$ and some $E\subseteq X$ such that either $E\in\mc S$ or $X\setminus E\in\mc S$, we define
	\[\mu_f(E)\coloneqq\int_Ef\,d\mu\coloneqq\int_Xf1_E\,d\mu.\]
	Note that $f1_E$ is $\mu$-integrable by \autoref{lem:restrict-int-function} when $E\in\mc S$ and by \autoref{rem:complement-restrict-int} when $X\setminus E\in\mc S$.
\end{notation}
\begin{remark} \label{rem:linear-combo-mu-f}
	We note that $\mu_f$ has good additivity properties. Namely, given scalars $a,b\in k$, where $k$ is the base field of $B$, and two $\mu$-integrable functions $f,g\colon X\to B$, we have
	\[\int_X(af+bg)1_E\,d\mu=\int_X\big(a(f1_E)+b(g1_E)\big)\,d\mu\stackrel*=a\int_Xf1_E\,d\mu+\int_Xb1_E\,d\mu,\]
	where $\stackrel*=$ is by \autoref{prop:int-is-linear}. Thus, $\mu_{af+bg}(E)=a\mu_f(E)+b\mu_g(E)$.
\end{remark}
Here are a few quick inequalities.
\begin{lemma} \label{lem:bound-mu-f}
	Fix a measure space $(X,\mc S,\mu)$ and a Banach space $(B,\norm\cdot)$. Further, fix a $\mu$-integrable function $f\colon X\to B$ and some $E\in\mc S$.
	\begin{listalph}
		\item We have $\norm{\mu_f(E)}\le\norm f_1$.
		\item Given a bound $M\ge0$ such that $\norm{f(x)}\le M$ almost everywhere for $x\in E$, then
		\[\norm{\mu_f(E)}\le M\mu(E).\]
	\end{listalph}
\end{lemma}
\begin{proof}
	By \autoref{lem:bound-norm-int}, we see
	\[\norm{\mu_f(E)}=\norm{\int_Xf1_E\,d\mu}\le\int_X\norm{f1_E}\,d\mu.\]
	We now approach the two parts separately.
	\begin{listalph}
		\item For each $x\in X$, we note that $\norm{f1_E}(x)$ is either $0$ or $\norm{f(x)}$, so $\norm{f1_E}(x)\le\norm f(x)$ for each $x\in X$. Thus, \autoref{lem:bound-ints} tells us
		\[\int_X\norm{f1_E}\,d\mu\le\int_X\norm f\,d\mu=\norm f_1,\]
		which finishes.
		\item We claim that $\norm{f1_E}(x)\le M1_E(x)$ almost everywhere for $x\in X$. Indeed, $\norm{f(x)}\le M$ almost everywhere for $x\in E$, so there is some $N\in\mc S$ such that $\mu(N)=0$ and $x\in E\setminus N$ implies $\norm{f(x)}\le M$. Thus, $x\in X\setminus N$ either has $x\in X\setminus E$ so that $\norm{f1_E}(x)=0\le0=M1_E(x)$ or $x\in E\setminus N$ so that
		\[\norm{f1_E}(x)=\norm{f(x)}\le M=M1_E(x).\]
		Finishing up, \autoref{lem:bound-ints} kicks in to tell us that
		\[\int_X\norm{f1_E}\,d\mu\le\int_XM1_E\,d\mu.\]
		The right-hand side is $M\int_X1_E\,d\mu$ by \autoref{lem:simple-int-is-linear}, which is $M\mu(E)$ by \autoref{ex:integrate-indicator}. This finishes.
		\qedhere
	\end{listalph}
\end{proof}
Now, the notation $\mu_f$ is intended to be suggestive that we're going to have a measure. Finite additivity is relatively quick.
\begin{remark} \label{rem:mu-f-fin-additive}
	Suppose $f\colon X\to B$ is $\mu$-integrable. It's pretty fast to see that $\mu_f\colon\mc S\to B$ is finitely additive: if $E,F\in\mc S$ are disjoint, we need to show that $\mu_f(E\sqcup F)=\mu_f(E)+\mu_f(F)$. (By induction, this extends to any finite collection.) Well, $1_E+1_F=1_{E\sqcup F}$ because $x\in E\sqcup F$ if and only if $x\in E$ or $x\in F$, but only one of $x\in E$ or $x\in F$ is possible. Thus, \autoref{prop:int-is-linear} tells us
	\[\mu_f(E\sqcup F)=\int_Xf1_{E\sqcup F}\,d\mu=\int_Xf(1_E+1_F)\,d\mu=\int_Xf1_E\,d\mu+\int_Xf1_F\,d\mu=\mu_f(E)+\mu_f(F).\]
\end{remark}
In fact, we can extend \autoref{rem:mu-f-fin-additive} to make $\mu_f$ countably additive.
\begin{proposition} \label{prop:int-gives-measure}
	Fix a measure space $(X,\mc S,\mu)$ and a Banach space $(B,\norm\cdot)$. Given some $\mu$-integrable function $f\colon X\to B$, the function $\mu_f\colon\mc S\to B$ is countably additive.
\end{proposition}
\begin{proof}
	Suppose we have a pairwise disjoint collection $\{E_i\}_{i\in\NN}\subseteq\mc S$. Set $E\coloneqq\bigsqcup_{i=1}^\infty E_i$ (which is in $\mc S$) so that we want to show
	\begin{equation}
		\mu_f(E)\stackrel?=\sum_{i=1}^\infty\mu_f(E_i). \label{eq:desired-mu-f-measure}
	\end{equation}
	We have two steps.
	\begin{enumerate}
		\item Suppose $f\colon X\to B$ is a simple $\mu$-integrable function; we show \autoref{eq:desired-mu-f-measure}. Well, by \autoref{lem:simple-form}, we can write
		\[f=\sum_{j=1}^ny_j1_{F_j}\]
		for some nonzero distinct points $y_j\in B$ and pairwise disjoint $F_j\coloneqq f^{-1}(\{y_j\})\in\mc S$.

		Now, we note $1_{F\cap F_j}=1_F1_{F_j}$ for any $F\in\mc S$ because $(1_F1_{F_j})(x)=1$ if and only if $1_F(x)=1$ and $1_{F_j}(x)=1$, which is equivalent to $x\in F$ and $x\in F_j$. Applying this multiple times, we compute
		\begin{align*}
			\mu_f(F) &= \int_Xf1_F\,d\mu \\
			&= \int_X\Bigg(\sum_{j=1}^ny_j1_{F_j}1_F\Bigg)\,d\mu \\
			&= \int_X\Bigg(\sum_{j=1}^ny_j1_{F_j\cap F}\Bigg)\,d\mu \\
			&= \sum_{j=1}^n\Bigg(y_j\int_X1_{F_j\cap F}\,d\mu\Bigg) \\
			&= \sum_{j=1}^ny_j\mu(F_j\cap F),
		\end{align*}
		where the last two equalities follow from \autoref{prop:int-is-linear} and then \autoref{ex:integrate-indicator}. As such, we can use the fact that $\mu$ is countably additive already: for each $j$, note that $\{F_j\cap E_i\}_{i\in\NN}$ is a pairwise disjoint collection because $x\in (F_j\cap E_i)\cap(F_j\cap E_{i'})$ implies $x\in E_i\cap E_{i'}$ implies $i=i'$. Thus,
		\[F_j\cap E=F_j\cap\bigsqcup_{i=1}^\infty E_i=\bigsqcup_{i=1}^\infty(F_j\cap E_i)\]
		implies
		\[\mu(F_j\cap E)=\sum_{i=1}^\infty\mu(F_j\cap E_i).\]
		Summing over all $j$, we can write
		\begin{align*}
			\mu_f(E) &= \sum_{j=1}^ny_j\mu(F_j\cap E) \\
			&= \sum_{j=1}^n\Bigg(y_j\sum_{i=1}^\infty\mu(F_j\cap E_i)\Bigg) \\
			&\stackrel*= \sum_{i=1}^\infty\Bigg(\sum_{j=1}^ny_j\mu(F_j\cap E_i)\Bigg) \\
			&= \sum_{i=1}^\infty\mu_f(F_j\cap E_i),
		\end{align*}
		which is what we wanted. Note that we are allowed to switch the order of summation in $\stackrel*=$ because the outer sum is finite. (This is effectively just the linearity of limits.)

		\item We now let $f\colon X\to B$ be an arbitrary $\mu$-integrable function. Fix any $\varepsilon>0$, and we need some $N$ such that $n\ge N$ implies
		\[\norm{\mu_f(E)-\sum_{i=1}^n\mu_f(E_i)}<\varepsilon.\]
		The idea is to relate $f$ to a simple $\mu$-integrable function: \autoref{cor:simple-int-are-dense} grants us some simple $\mu$-integrable function $g\colon X\to B$ such that $\norm{f-g}_1<\varepsilon/3$. Now, for any finite $n$, we can compute
		\begin{align*}
			\norm{\mu_f(E)-\sum_{i=1}^n\mu_f(E_i)} &\le \norm{\mu_f(E)-\mu_g(E)} + \norm{\mu_g(E)-\sum_{i=1}^n\mu_g(E_i)}+\norm{\sum_{i=1}^n\big(\mu_g(E_i)-\mu_f(E_i)\big)} \\
			&= \norm{\mu_f(E)-\mu_g(E)}+\norm{\mu_g(E)-\sum_{i=1}^n\mu_g(E_i)}+\norm{\mu_g\Bigg(\bigsqcup_{i=1}^nE_i\Bigg)-\mu_f\Bigg(\bigsqcup_{i=1}^nE_i\Bigg)},
		\end{align*}
		where the last equality is because $\mu_f$ and $\mu_g$ are already finitely additive by \autoref{rem:mu-f-fin-additive}. Now, for any $F\in\mc S$, we note \autoref{rem:linear-combo-mu-f} tells us
		\[\norm{\mu_f(F)-\mu_g(F)}=\norm{\mu_{f-g}(F)},\]
		which is upper-bounded by $\norm{f-g}_1$ by \autoref{lem:bound-mu-f}. Thus,
		\[\norm{\mu_f(E)-\sum_{i=1}^n\mu_f(E_i)}\le2\norm{f-g}_1+\norm{\mu_g(E)-\sum_{i=1}^n\mu_g(E_i)}.\]
		To finish, we use the previous step to note that there is some $N$ such that $n\ge N$ implies
		\[\norm{\mu_g(E)-\sum_{i=1}^n\mu_g(E_i)}<\frac\varepsilon3.\]
		In total,
		\[\norm{\mu_f(E)-\sum_{i=1}^n\mu_f(E_i)}<2\cdot\frac\varepsilon3+\frac\varepsilon3=\varepsilon,\]
		which is what we wanted.
		\qedhere
	\end{enumerate}
\end{proof}
Having a notion of countably additive functions encourages us to extend our definition of measure.
\begin{definition}[Measure]
	Fix a measure space $(X,\mc S,\mu)$ and a Banach space $(B,\norm\cdot)$. Then a \textit{$B$-valued measure $\mu$ on $\mc S$} is a countably additive function $\mu\colon\mc S\to B$.
\end{definition}
\begin{example}
	By \autoref{prop:int-gives-measure}, we see that a $\mu$-integrable function $f\colon X\to B$ gives a $B$-valued measure $\mu_f$.
\end{example}
We now note that $\mu_f$ cannot be terribly large.
\begin{lemma} \label{lem:int-measure-is-continuous}
	Fix a measure space $(X,\mc S,\mu)$ and a normed vector space $B$. Further, fix a $\mu$-integrable function $f\colon X\to B$. For any $\varepsilon>0$, there is some $\delta>0$ such that $E\in\mc S$ with $\mu(E)<\delta$ implies $\norm{\mu_f(E)}<\varepsilon$.
\end{lemma}
\begin{proof}
	As usual, use \autoref{cor:simple-int-are-dense} to find some simple $\mu$-integrable function $g\colon X\to B$ with $\norm{f-g}_1<\varepsilon/2$. Then any $E\in\mc S$ grants
	\[\norm{\mu_f(E)}\le\norm{\mu_f(E)-\mu_g(E)}+\norm{\mu_g(E)}.\]
	We now bound the terms individually.
	\begin{itemize}
		\item By \autoref{rem:linear-combo-mu-f}, we see $\mu_f(E)-\mu_g(E)=\mu_{f-g}(E)$, so
		\[\norm{\mu_f(E)-\mu_g(E)}=\norm{\mu_{f-g}(E)}\le\norm{f-g}_1,\]
		where the inequality is by \autoref{lem:bound-mu-f}.
		\item Note that $g$ has finite image, so we may set $M\coloneqq\max\{\norm{g(x)}:x\in X\}$ so that $\norm{g(x)}\le M$ for each $x\in X$. Thus, \autoref{lem:bound-mu-f} tells us
		\[\norm{\mu_g(E)}\le M\mu(E).\]
	\end{itemize}
	In total, we see
	\[\norm{\mu_f(E)}\le\frac\varepsilon2+M\mu(E)\]
	for any $E\in\mc S$. Thus, we set $\delta\coloneqq\frac\varepsilon{2(M+1)}$ so that $\mu(E)<\delta$ implies
	\[\norm{\mu_f(E)}<\frac\varepsilon2+M\cdot\frac\varepsilon{2(M+1)}<\varepsilon,\]
	which is what we wanted.
\end{proof}
The above lemma motivates the following definition.
\begin{definition}[Strongly absolutely continuous]
	Fix a measure space $(X,\mc S,\mu)$ and some Banach space $(B,\norm\cdot)$. Then a $B$-valued measure $\nu\colon\mc S\to B$ is \textit{strongly absolutely continuous} if and only if each $\varepsilon>0$ have some $\delta>0$ such that $\mu(E)<\delta$ implies $\norm{\nu(E)}<\varepsilon$.
\end{definition}
\begin{example}
	By \autoref{lem:int-measure-is-continuous}, each $\mu_f$ coming from a $\mu$-integrable function $f$ is strongly absolutely continuous.
\end{example}
\begin{remark}
	If $\nu$ is strongly absolutely continuous, then note that any $E\in\mc S$ with $\mu(E)=0$ will have $\nu(E)=0$. Indeed, for any $\varepsilon>0$, there is $\delta>0$ such that $\mu(E')<\delta$ implies $\norm{\nu(E')}<\varepsilon$. But we will always have $\mu(E)=0<\delta$, so $\norm{\nu(E)}<\varepsilon$ for all $\varepsilon>0$, so $\norm{\nu(E)}=0$, so $\nu(E)=0$. (This condition is that $\nu$ is ``absolutely continuous.'' We will not need it later.)
\end{remark}
\begin{remark}
	The Radon--Nikodym theorem says that sufficiently nice $B$-valued measures $\nu$ which are absolutely continuous will have $\nu=\mu_f$ for some $\mu$-integrable function $f$.
\end{remark}

\subsection{Egorov's Theorem}
To help us later, we pick up the following result on $\mc S$-measurable functions.
\begin{theorem}[Egorov's] \label{thm:ego}
	Fix a measure space $(X,\mc S,\mu)$ and a Banach space $B$. Further, fix some sequence $\{f_n\}_{n\in\NN}$ of $\mc S$-measurable functions. Suppose $E\in\mc S$ has $\mu(E)<\infty$ such that the $\{f_n\}_{n\in\NN}$ converge almost everywhere on $E$ to a function $f\colon X\to B$. Then $f_n|_E\to f$ almost uniformly on $E$.
\end{theorem}
\begin{proof}
	This is a little tricky. We'll take this in steps.
	\begin{enumerate}
		\item We begin by removing a few null sets, for psychological reasons. Note we are given some $N\in\mc S$ such that $\mu(N)=0$ while $f_k(x)\to f(x)$ as $n\to\infty$ for each $x\in E\setminus N$. As such,
		\[f_n1_{E\setminus N}\to f1_{E\setminus N}\]
		on $E$ because if $x\notin N$, then $f_n1_{E\setminus N}(x)=0=f1_{E\setminus N}(x)$ for each $n$.
		
		We thus claim that $f_n1_{E\setminus N}\to f1_{E\setminus N}$ almost uniformly on $E$. To see that this is enough, note that any $\varepsilon>0$ has some $F\subseteq E$ with $\mu(E\setminus F)<\varepsilon$ while $f_n1_{E\setminus N}|_F\to f1_{E\setminus N}|_F$ uniformly. But then we set $F'\coloneqq F\setminus N$ so that \autoref{lem:fin-additive-is-monotone} and \autoref{lem:finitely-additive-is-subaddtive} tells us
		\[\mu(E\setminus F')\le\mu((E\setminus F)\cup N)\le\mu(E\setminus F)+\mu(N)<\delta+0=\delta.\]
		But now $f_n1_{E\setminus N}|_{F'}=f_n|_{F'}$ and $f1_{E\setminus N}|_{F'}=f|_{F'}$ because each $x\in F'$ has $x\notin N$ already. Thus, $f_n|_{F'}\to f|_{F'}$ uniformly, which is what we needed.
	
		In total, we are given $f_n1_{E\setminus N}\to f1_{E\setminus N}$ everywhere on $E$ and would like to show this convergence is almost uniform. As such, we replace each $f_n1_{E\setminus N}$ with $f_n$ and $f1_{E\setminus N}$ with $f$ to no detriment, except now we know $f_n\to f$ everywhere. In particular, $f$ is $\mc S$-measuralbe by \autoref{cor:limit-of-meas-is-meas}.

		\item Now, for each $m$ and $n$, set
		\[E^m_n\coloneqq\bigcup_{k\ge n}\{x\in E:\norm{(f-f_k)(x)}\ge1/m\}=\{x\in E:\norm{(f-f_k)(x)}\ge1/m\text{ for some }k\ge n\}.\]
		Note each $\norm{f-f_k}$ is $\mc S$-measurable by \autoref{lem:meas-is-vec-space} and \autoref{cor:take-norms-is-measurable}, so the union $E^m_n$ is in $\mc S$ by \autoref{cor:meas-has-meas-pre-image}. Now, for fixed $m$, we note that $f_k(x)\to f(x)$ for $x\in E$ forces $\bigcap_{n=1}^\infty E^m_n=\emp$. However, $\mu(E_1)\le\mu(E)<\infty$ by \autoref{lem:fin-additive-is-monotone}, so \autoref{cor:measure-inter-down} tell us that
		\[\lim_{n\to\infty}\mu\left(E^m_n\right)=\mu\Bigg(\bigcap_{n=1}^\infty E^m_n\Bigg)=\mu(\emp)=0.\]

		\item We now attack the proof directly. Set $\varepsilon>0$. For each $m$, we may choose $n_m$ so that $\mu\left(E^m_{n}\right)<\varepsilon/2^m$ for $n\ge n_m$. As such, we set
		\[F\coloneqq E{\mathbin\bigg\backslash}\bigcup_{m=1}^\infty E^m_{n_m}\]
		so that \autoref{lem:premeas-is-countable-subadd} tells us
		\[\mu(E\setminus F)=\mu\Bigg(\bigcup_{m=1}^\infty E^m_{n_m}\Bigg)\le\sum_{m=1}^\infty\mu\left(E^m_{n_m}\right)<\sum_{m=1}^\infty\frac\varepsilon{2^m}=\varepsilon.\]
		It remains to show $f_n|_F\to f|_F$ uniformly. Fix any $\delta>0$. To set $N$, find $m$ with $m>1/\delta$, and we set $N\coloneqq n_{m}$.
		
		To see that this construction works, fix some $n\ge N$ and $x\in F$. Well, $x\in F$ implies that $x\notin E^m_{n_m}$ for our $m$, so
		\[\norm{f(x)-f_k(x)}<1/m<\delta\]
		for each $k\ge n_m$. In particular, $n\ge n_m$, so $\norm{f(x)-f_n(x)}<\delta$ follows.
		\qedhere
	\end{enumerate}
\end{proof}
The point of picking up \autoref{thm:ego} is so that we can prove the Dominated convergence theorem.
\begin{restatable}[Dominated convergence]{theorem}{domconvthm} \label{thm:dom-conv}
	Fix a measure space $(X,\mc S,\mu)$ and a Banach space $(B,\norm\cdot)$. Further, fix some sequence $\{f_n\}_{n\in\NN}$ of $\mu$-integrable functions converging almost everywhere to a function $f$. If there is a $\mu$-integrable function $g\colon X\to\RR$ such that $\norm{f_n(x)}\le g(x)$ almost everywhere for each $n$, then $\{f_n\}_{n\in\NN}$ is in fact mean Cauchy.
\end{restatable}
\noindent We will prove this next class.
\begin{remark} \label{rem:apply-dom-conv}
	It will follow from the conclusion that $f_n\to f$ in mean and so
	\[\int_Xf\,d\mu=\lim_{n\to\infty}\int_Xf_n\,d\mu.\]
\end{remark}

\end{document}