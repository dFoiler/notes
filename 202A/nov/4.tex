% !TEX root = ../notes.tex

\documentclass[../notes.tex]{subfiles}

\begin{document}

\section{November 4}

We continue our journey towards integrating functions.

\subsection{Sequences Converging in Measure}
Here is our convergence result, which convinces us that convergence in measure will be one that we can work with; note this generalizes \autoref{ex:fixing-bad-news}.
\begin{defi}[Mean Cauchy]
	Fix a normed vector space $B$ and a measure space $(X,\mc S,\mu)$. A sequence of simple $\mu$-integrable functions $\{f_n\}_{n\in\NN}$ is \textit{mean Cauchy} if and only if it is Cauchy for the semi-norm $\norm\cdot_1$. In other words, we require
	\[\lim_{m,n\to\infty}\norm{f_m-f_n}_1=0.\]
\end{defi}
\begin{lemma}
	Fix a normed vector space $(B,\norm\cdot)$ and a measure space $(X,\mc S,\mu)$. Then a sequence of simple $\mu$-integrable functions $f_n\colon X\to B$ for $n\in\NN$ which is mean Cauchy is also Cauchy in measure.
\end{lemma}
\begin{proof}
	Fix $\varepsilon>0$ and set
	\[E^\varepsilon_{m,n}\coloneqq\{x\in X:\norm{f_m(x)-f_n(x)}\ge\varepsilon\},\]
	which has finite measure by \autoref{rem:sets-in-cauchy-in-measure-are-finite}. We need to show that
	\[\lim_{m,n\to\infty}\mu(E_{m,n}^\varepsilon)\stackrel?=0.\]
	Notably, for each $x\in X$, we must have
	\begin{equation}
		1_{E^\varepsilon_{m,n}}(x)\le\frac{\norm{f_m(x)-f_n(x)}}\varepsilon \label{eq:bound-indicator-e-m-n}
	\end{equation}
	by definition of $E^{\varepsilon}_{m,n}$. Now, both sides of this equation are simple $\mu$-integrable functions: $1_{E^\varepsilon_{m,n}}$ is by \autoref{ex:indicator-is-simple}; and $f_m-f_n$ is simple $\mu$-integrable by \autoref{lem:simple-int-is-k-vec}, as is $\norm{f_m-f_n}$ by \autoref{lem:norm-is-simple-int}, so $\frac1\varepsilon\norm{f_m-f_n}$ is simple $\mu$-integrable by \autoref{lem:simple-int-is-k-vec} again.
	
	Thus, we may integrate, for which \autoref{cor:bound-ints} tells us
	\[\mu(E^\varepsilon_{m,n})=\int_X1_{E^{\varepsilon}_{m,n}}\,d\mu\le\int_X\frac{\norm{f_m-f_n}}\varepsilon\,d\mu=\frac{\norm{f_m-f_n}_1}\varepsilon,\]
	where the first integral was computed using \autoref{ex:integrate-indicator}. But as $m,n\to\infty$, the right-hand value goes to $0$ because $\{f_n\}_{n\in\NN}$ is Cauchy for $\norm\cdot_1$, so the left-hand value must also go to $0$.
\end{proof}
\begin{remark}
	A similar proof works for when we are Cauchy for $\norm\cdot_p$ for finite $p$ by taking $p$th powers of \autoref{eq:bound-indicator-e-m-n}. For example, in probability theory, the result for $\norm\cdot_2$ is essentially Chebyshev's inequality.
\end{remark}
Here are some quick sanity checks our convergence.
\begin{lemma} \label{lem:converge-in-measure-uniq-ae}
	Fix a normed vector space $(B,\norm\cdot)$ and a measure space $(X,\mc S,\mu)$. Now, suppose a sequence $\{f_n\}_{n\in\NN}$ of $\mc S$-measurable functions converges to both $f$ and $g$ in measure, where $f$ and $g$ are both $\mc S$-measurable. Then $f=g$ almost everywhere; i.e., $\{x\in X:f(x)\ne g(x)\}$ is a null set.
\end{lemma}
\begin{proof}
	Before we do anything at all, we note that $f-g$ is $\mc S$-measurable by \autoref{lem:meas-is-vec-space}, so
	\[N\coloneqq\{x\in X:f(x)\ne g(x)\}=(f-g)^{-1}(B\setminus\{0\})\]
	is $\mc S$-measurable by \autoref{cor:meas-has-meas-pre-image}.
	
	Now, fix any $\varepsilon>0$; we show $\mu(N)<\varepsilon$. The key observation is that
	\[\norm{f(x)-g(x)}\le\norm{f(x)-f_n(x)}+\norm{f_n(x)-g(x)},\]
	so it follows that $\norm{f(x)-g(x)}\ge\varepsilon$ forces $\norm{f(x)-f_n(x)}\ge\varepsilon/2$ or $\norm{g(x)-g_n(x)}\ge\varepsilon/2$. Thus,
	\[\{x:\norm{f(x)-g(x)}\ge\varepsilon\}\subseteq\{x:\norm{f(x)-f_n(x)}\ge\varepsilon\}\cup\{x:\norm{g(x)-f_n(x)}\ge\varepsilon\},\]
	so \autoref{lem:finitely-additive-is-subaddtive} tells us
	\[\mu(\{x\in X:\norm{f(x)-g(x)}\ge\varepsilon\})\le\mu(\{x\in X:\norm{f(x)-f_n(x)}\ge\varepsilon/2\})+\mu(\{x\in X:\norm{g(x)-f_n(x)}\ge\varepsilon/2\}).\]
	But now, as $n\to\infty$, we see that the right-hand side goes to $0+0=0$ because $f_n\to f$ and $f_n\to g$ in measure, so it follows that
	\begin{equation}
		\mu(\{x\in X:\norm{f(x)-g(x)}\ge\varepsilon\})=0. \label{eq:norm-big-vanishes-equal-limits}
	\end{equation}
	We now send $\varepsilon\to0^+$. Namely, we see $f(x)\ne g(x)$ is equivalent to $\norm{f(x)-g(x)}>0$ is equivalent to $\norm{f(x)-g(x)}\ge1/n$ for some $n\in\NN$, so
	\[N\coloneqq\{x\in X:f(x)\ne g(x)\}=\{x\in X:\norm{f(x)-g(x)}>0\}=\bigcup_{n\in\NN}\{x\in X:\norm{f(x)-g(x)}\ge1/n\}.\]
	Thus,
	\[\mu(N)\le\sum_{n\in\NN}\mu(\{x\in X:\norm{f(x)-g(x)}\ge1/n\})\stackrel*=\sum_{x\in X}0=0,\]
	so $N$ is in fact a null set. Notably, $\stackrel*=$ has used \autoref{eq:norm-big-vanishes-equal-limits}.
\end{proof}
\begin{lemma} \label{lem:linear-combo-in-measure}
	Fix a normed $k$-vector space $(B,\norm\cdot)$ and a measure space $(X,\mc S,\mu)$. Fix sequences of $\mc S$-measurable functions $\{f_n\}_{n\in\NN}$ and $\{g_n\}_{n\in\NN}$ with $f_n\to f$ and $g_n\to g$ in measure as $n\to\infty$.
	\begin{listalph}
		\item We have $f_n+g_n\to f+g$ in measure.
		\item Given some scalar $a\in k$, we have $af_n\to af$ in measure.
		\item The functions $x\mapsto\norm{f_n(x)}$ converge to $x\mapsto\norm{f(x)}$ in measure.
	\end{listalph}
\end{lemma}
\begin{proof}
	We go ahead and let $|\cdot|$ denote the norm on base field $k$ of $B$.
	\begin{listalph}
		\item Note that the $f_n+g_n$ and $f+g$ are all $\mc S$-measurable by \autoref{lem:meas-is-vec-space}.
		
		Now, by the triangle inequality, we see
		\[\norm{(f(x)+g(x))-(f_n(x)+g_n(x))}\le\norm{f(x)-f_n(x)}+\norm{g(x)-g_n(x)}.\]
		We now proceed as in \autoref{lem:converge-in-measure-uniq-ae}. Fix $\varepsilon>0$. If the left-hand side exceeds $\varepsilon$, then one of the terms on the right-hand side must exceed $\varepsilon/2$, so
		\begin{align*}
			\{x:\norm{(f(x)+g(x))-(f_n(x)+g_n(x))}\ge\varepsilon\}&\subseteq\{x:\norm{f(x)-f_n(x)}\ge\varepsilon/2\} \\
			&\qquad\cup\{x:\norm{g(x)-g_n(x)}\ge\varepsilon/2\}.
		\end{align*}
		Thus, \autoref{lem:finitely-additive-is-subaddtive} tells us
		\begin{align*}
			\mu(\{x:\norm{(f(x)+g(x))-(f_n(x)+g_n(x))}\ge\varepsilon\}) &\le \mu(\{x:\norm{f(x)-f_n(x)}\ge\varepsilon/2\}) \\
			&\qquad+\mu(\{x:\norm{g(x)-g_n(x)}\ge\varepsilon/2\}).
		\end{align*}
		However, $\varepsilon/2>0$, so taking $n\to\infty$ and using our convergence in measure tells us that
		\[\lim_{n\to\infty}\mu(\{x:\norm{(f(x)+g(x))-(f_n(x)+g_n(x))}\ge\varepsilon\})\le0+0=0,\]
		so we are done after noting that $\mu$ will only output nonnegative values, so the limit is at least nonnegative.

		\item Note that the $af_n$ and $af$ are all $\mc S$-measurable by \autoref{lem:meas-is-vec-space}.
		
		Now, fix some $\varepsilon>0$ so that we want to show that
		\[L\coloneqq\lim_{n\to\infty}\mu(\{x\in X:\norm{af_n(x)-af(x)}\ge\varepsilon\})\stackrel?=0.\]
		If $a=0$, then $af_n(x)=af(x)=0$ for all $x\in X$, so $\{x\in X:\norm{af_n(x)-af(x)}\ge\varepsilon\}$ is empty, so the result follows.

		Otherwise, take $a\ne0$ so that $|a|>0$. Now, note $\norm{af_n(x)-af(x)}=|a|\cdot\norm{f_n(x)-f(x)}$, so it follows $\norm{af_n(x)-af(x)}\ge\varepsilon$ if and only if $\norm{f_n(x)-f(x)}\ge\varepsilon/|a|$. Thus,
		\[L=\lim_{n\to\infty}\mu(\{x\in X:\norm{f_n(x)-f(x)}\ge\varepsilon/|a|\}).\]
		However, $\varepsilon/|a|>0$ because $\varepsilon>0$, so the above limit vanishes because $f_n\to f$ in measure as $n\to\infty$.

		\item Observe that the $g_n$ and $g$ are all $\mc S$-measurable by \autoref{cor:take-norms-is-measurable}.

		Now, fix some $\varepsilon>0$. By the (reverse) triangle inequality,
		\[|\norm{f(x)}-\norm{f_n(x)}|\le\norm{f(x)-f_n(x)},\]
		so any $\varepsilon>0$ has
		\[\{x:|\norm{f(x)}-\norm{f_n(x)}|\ge\varepsilon\}\subseteq\{x:\norm{f(x)-f_n(x)}\ge\varepsilon\}.\]
		Thus, \autoref{lem:fin-additive-is-monotone} tells us
		\[\lim_{n\to\infty}\mu(\{x:|\norm{f(x)}-\norm{f_n(x)}|\ge\varepsilon\})\le\lim_{n\to\infty}\mu(\{x:\norm{f(x)-f_n(x)}\ge\varepsilon\}).\]
		The right-hand limit vanishes because $f_n\to f$ in measure, so the left-hand limit must vanish as well because the limit's terms are nonnegative.
		\qedhere
	\end{listalph}
\end{proof}
Here is the analogous result for sequences Cauchy in measure.
\begin{lemma} \label{lem:linear-combo-cauchy-in-measure}
	Fix a normed $k$-vector space $(B,\norm\cdot)$ and a measure space $(X,\mc S,\mu)$. Fix sequences of $\mc S$-measurable functions $\{f_n\}_{n\in\NN}$ and $\{g_n\}_{n\in\NN}$ which are Cauchy in measure.
	\begin{listalph}
		\item The sequence $\{f_n+g_n\}_{n\in\NN}$ is Cauchy in measure.
		\item Given some scalar $a\in k$, the sequence $\{af_n\}_{n\in\NN}$ is Cauchy in measure.
		\item The sequence of functions $\{x\mapsto\norm{f_n(x)}\}_{n\in\NN}$ are Cauchy in measure.
	\end{listalph}
\end{lemma}
\begin{proof}
	These proofs are essentially the same as \autoref{lem:linear-combo-in-measure} with the appropriate names changed. Again, we let $|\cdot|$ denote the norm on base field $k$ of $B$.
	\begin{listalph}
		\item Note that the $f_n+g_n$ are all $\mc S$-measurable by \autoref{lem:meas-is-vec-space}.
		
		Now, by the triangle inequality, we see
		\[\norm{(f_m(x)+g_m(x))-(f_n(x)+g_n(x))}\le\norm{f_m(x)-f_n(x)}+\norm{g_m(x)-g_n(x)}.\]
		Fix $\varepsilon>0$. As usual
		\begin{align*}
			\{x:\norm{(f_m(x)+g_m(x))-(f_n(x)+g_n(x))}\ge\varepsilon\}&\subseteq\{x:\norm{f_m(x)-f_n(x)}\ge\varepsilon/2\} \\
			&\qquad\cup\{x:\norm{g_m(x)-g_n(x)}\ge\varepsilon/2\},
		\end{align*}
		so \autoref{lem:finitely-additive-is-subaddtive} tells us
		\begin{align*}
			\mu(\{x:\norm{(f_m(x)+g_m(x))-(f_n(x)+g_n(x))}\ge\varepsilon\}) &\le \mu(\{x:\norm{f_m(x)-f_n(x)}\ge\varepsilon/2\}) \\
			&\qquad+\mu(\{x:\norm{g_m(x)-g_n(x)}\ge\varepsilon/2\}).
		\end{align*}
		However, $\varepsilon/2>0$, so taking $m,n\to\infty$ and using our Cauchy in measure conditions tells us that
		\[\lim_{m,n\to\infty}\mu(\{x:\norm{(f_m(x)+g_m(x))-(f_n(x)+g_n(x))}\ge\varepsilon\})\le0+0=0,\]
		so we are done after noting that $\mu$ will only output nonnegative values, so the limit is at least nonnegative.

		\item Note that the $af_n$ are all $\mc S$-measurable by \autoref{lem:meas-is-vec-space}.
		
		Now, fix some $\varepsilon>0$ so that we want to show that
		\[L\coloneqq\lim_{m,n\to\infty}\mu(\{x\in X:\norm{af_m(x)-af_n(x)}\ge\varepsilon\})\stackrel?=0.\]
		If $a=0$, then $af_n(x)=af(x)=0$ for all $x\in X$, so $\{x\in X:\norm{af_m(x)-af_n(x)}\ge\varepsilon\}$ is empty, so the result follows.

		Otherwise, take $a\ne0$ so that $|a|>0$. Now, note $\norm{af_m(x)-af_n(x)}=|a|\cdot\norm{f_m(x)-f_n(x)}$, so it follows $\norm{af_m(x)-af_n(x)}\ge\varepsilon$ if and only if $\norm{f_m(x)-f_n(x)}\ge\varepsilon/|a|$. Thus,
		\[L=\lim_{m,n\to\infty}\mu(\{x\in X:\norm{f_m(x)-f_n(x)}\ge\varepsilon/|a|\}).\]
		However, $\varepsilon/|a|>0$ because $\varepsilon>0$, so the above limit vanishes because $\{f_n\}_{n\in\NN}$ is Cauchy in measure.

		\item Observe that the $g_n$ are all $\mc S$-measurable by \autoref{cor:take-norms-is-measurable}.

		Now, fix some $\varepsilon>0$. By the (reverse) triangle inequality,
		\[|\norm{f_m(x)}-\norm{f_n(x)}|\le\norm{f_m(x)-f_n(x)},\]
		so any $\varepsilon>0$ has
		\[\{x:|\norm{f_m(x)}-\norm{f_n(x)}|\ge\varepsilon\}\subseteq\{x:\norm{f_m(x)-f_n(x)}\ge\varepsilon\}.\]
		Thus, \autoref{lem:fin-additive-is-monotone} tells us
		\[\lim_{m,n\to\infty}\mu(\{x:|\norm{f_m(x)}-\norm{f_n(x)}|\ge\varepsilon\})\le\lim_{m,n\to\infty}\mu(\{x:\norm{f_m(x)-f_n(x)}\ge\varepsilon\}).\]
		The right-hand limit vanishes because $\{f_n\}_{n\in\NN}$ is Cauchy in measure, so the left-hand limit must vanish as well because the limit's terms are also nonnegative.
		\qedhere
	\end{listalph}
\end{proof}

\subsection{Restricting Measurable Functions}
We begin by introducing the construction we will be interested in.
\begin{lemma} \label{lem:restrict-meas-functions}
	Fix a normed vector space $B$ and a measure space $(X,\mc S,\mu)$ and a set $E\in\mc S$. If $f\colon X\to B$ is simple $\mc S$-measurable or $\mc S$-measurable or simple $\mu$-integrable, then $f1_E$ is as well.
\end{lemma}
\begin{proof}
	Omitted.
	Before doing anything, we pick up a few facts. Note that
	\[\im f1_E=\{f(x)1_E(x):x\in X\}\subseteq\{0\}\cup\{f(x):x\in E\}\subseteq\{0\}\cup\im f.\]
	Also, if $S\subseteq B\setminus\{0\}$, then we claim
	\[(f1_E)^{-1}(S)=E\cap f^{-1}(S).\]
	In one direction, note $x\in E\cap f^{-1}(S)$ implies that $(f1_E)(x)=f(x)\in S$. In the other direction, if $x\in(f1_E)^{-1}(S)$, then note $x\in E$ is forced because otherwise $f(x)=0\notin S$. Thus, with $x\in E$, we have $(f1_E)(x)=f(x)$, so $(f1_E)(x)\in S$ forces $x\in f^{-1}(S)$ as well.

	We now note that we actually have three claims to show, which we show in sequence.
	\begin{itemize}
		\item Suppose that $f$ is a simple $\mc S$-measurable function. As such, $\im f$ is finite, so $\im f1_E\subseteq\{0\}\cup\im f$ is also finite.
		
		Further, for each $y\in(\im f1_E)\setminus\{0\}$, we see that $(f1_E)^{-1}(\{y\})=E\cap f^{-1}(\{y\})$ as discussed above, which lives in $\mc S$ because $E\in\mc S$ and $f^{-1}(\{y\})\in\mc S$.
		\item Suppose that $f$ is an $\mathcal S$-measurable function. Then $\im f$ is separable, so it follows $\{0\}\cup\im f$ is separable (by \autoref{ex:union-of-seps-is-sep}), so $\im f1_E\subseteq\{0\}\cup\im f$ is separable (by \autoref{rem:subspace-of-sep-is-sep}).

		Now, for any open subset $U\subseteq B\setminus\{0\}$, we see $(f1_E)^{-1}(U)=E\cap f^{-1}(U)$ as discussed above, which lives in $\mc S$ because $E\in\mc S$ and $f^{-1}(\{y\})\in\mc S$.
		\item Suppose that $f$ is a simple $\mu$-integrable function. As before, $\im f$ is finite implies that $\im f1_E\subseteq\{0\}\cup\im f$ is still finite.

		Further, for each $y\in(\im f1_E)\setminus\{0\}$, we see $(f1_E)^{-1}(\{y\})=E\cap f^{-1}(\{y\})$, which saw in our first point to live in $\mc S$, but now we note that \autoref{lem:fin-additive-is-monotone} tells us
		\[\mu\left((f1_E)^{-1}(\{y\})\right)\le\mu\left(f^{-1}(\{y\})\right)<\infty\]
		is finite.
		\qedhere
	\end{itemize}
\end{proof}
Analogously to \autoref{lem:linear-combo-in-measure} and \autoref{lem:linear-combo-cauchy-in-measure}, we have the following.
\begin{lemma}
	Fix a normed vector space $(B,\norm\cdot)$ and a measure space $(X,\mc S,\mu)$, and fix some $E\in\mc S$. Given a sequence $\{f_n\}_{n\in\NN}$ of $\mc S$-measurable functions with $f_n\to f$ in measure as $n\to\infty$, then $f_n1_E\to f1_E$ in measure as $n\to\infty$.
\end{lemma}
\begin{proof}
	Note that the $f_n1_E$ and $f1_E$ are all $\mc S$-measurable by \autoref{lem:restrict-meas-functions}, so the claim at least makes sense.

	For brevity, we set $g_n\coloneqq\norm{f1_E-f_n1_E}$ for each $n$. We would like to show
	\[\lim_{n\to\infty}\mu\left(g_n^{-1}([\varepsilon,\infty))\right)\stackrel?=0.\]
	If $x\notin E$, then note $g_n(x)=0$; otherwise, $g_n(x)=\norm{f(x)-f_n(x)}$ because $1_E(x)=1$. As such, for $\varepsilon>0$, we see $x\in g_n^{-1}([\varepsilon,\infty))$ requires $x\in E$ and then $\norm{f(x)-f_n(x)}\ge\varepsilon$; conversely, $x\in E$ with $\norm{f(x)-f_n(x)}\ge\varepsilon$ does give $g_n(x)\ge\varepsilon$.

	Thus, we note that
	\[g_n^{-1}([\varepsilon,\infty))\subseteq\{x:\norm{f(x)-f_n(x)}\ge\varepsilon\},\]
	so \autoref{lem:fin-additive-is-monotone} tells us
	\[\lim_{n\to\infty}\mu\left(g_n^{-1}([\varepsilon,\infty))\right)\le\lim_{n\to\infty}\mu\left(\{x:\norm{f(x)-f_n(x)}\ge\varepsilon\}\right),\]
	where the right-hand limit vanishes because $f_n\to f$ in measure as $n\to\infty$. Thus, the left-hand limit also vanishes because the terms of the limit are nonnegative.
\end{proof}
\begin{lemma}
	Fix a normed vector space $(B,\norm\cdot)$ and a measure space $(X,\mc S,\mu)$, and fix some $E\in\mc S$. Given a sequence $\{f_n\}_{n\in\NN}$ of $\mc S$-measurable functions which is Cauchy in measure, then $\{f_n1_E\}_{n\in\NN}$ is still Cauchy in measure.
\end{lemma}
\begin{proof}
	As usual, the proof is exactly the same as before. Note that the $f_n1_E$ and $f1_E$ are all $\mc S$-measurable by \autoref{lem:restrict-meas-functions}, so the claim at least makes sense.

	For brevity, we set $g_{m,n}\coloneqq\norm{f_m1_E-f_n1_E}$ for each $m$ and $n$. We would like to show
	\[\lim_{m,n\to\infty}\mu\left(g_{m,n}^{-1}([\varepsilon,\infty))\right)\stackrel?=0.\]
	If $x\notin E$, then note $g_{m,n}(x)=0$; otherwise, $g_{m,n}(x)=\norm{f_m(x)-f_n(x)}$ because $1_E(x)=1$. As such, for $\varepsilon>0$, we see $x\in g_{m,n}^{-1}([\varepsilon,\infty))$ requires $x\in E$ and then $\norm{f_m(x)-f_n(x)}\ge\varepsilon$; conversely, $x\in E$ with $\norm{f_m(x)-f_n(x)}\ge\varepsilon$ does give $g_{m,n}(x)\ge\varepsilon$.

	Thus, we note that
	\[g_{m,n}^{-1}([\varepsilon,\infty))\subseteq\{x:\norm{f_m(x)-f_n(x)}\ge\varepsilon\},\]
	so \autoref{lem:fin-additive-is-monotone} tells us
	\[\lim_{m,n\to\infty}\mu\left(g_{m,n}^{-1}([\varepsilon,\infty))\right)\le\lim_{m,n\to\infty}\mu\left(\{x:\norm{f_m(x)-f_n(x)}\ge\varepsilon\}\right),\]
	where the right-hand limit vanishes because $\{f_{n}\}_{n\in\NN}$ is Cauchy in measure. Thus, the left-hand limit also vanishes because the terms of the limit are nonnegative.
\end{proof}
The above corollary promises the following notation.
\begin{notation}
	Fix a normed vector space $(B,\norm\cdot)$ and a measure space $(X,\mc S,\mu)$. Then a simple integrable function $f$ on $X$ and $E\in\mc S$ will have
	\[\int_Ef\,d\mu\coloneqq\int_Xf1_E\,d\mu.\]
\end{notation}
\begin{remark}
	One can define
	\[\mu_f(E)\coloneqq\int_Ef\,d\mu,\]
	and it is not too hard to check that this defines a measure on $\mc S$ which is valued in $B$. This $\mu_f$ will later be called the ``indefinite integral for $f$.'' We will postpone writing this out until we are ready to talk about what this looks like when $f$ is a general $\mu$-integrable function instead of a simple $\mu$-integrable function.
\end{remark}

\subsection{Almost Uniform Convergence}
As we tend to do, we now return to a context which is perhaps too general.
\begin{definition}[Almost uniformly]
	Fix a measure space $(X,\mc S,\mu)$ and a normed vector space $B$. Then a sequence of functions $f_n\colon X\to B$ for $n\in\NN$ converges \textit{almost uniformly} to $f$ if and only if every $\varepsilon>0$ has some $E^\varepsilon\in\mc S$ such that $\mu(E^\varepsilon)<\varepsilon$ and $f_n|_{X\setminus E}\to f|_{X\setminus E}$ uniformly.
\end{definition}
\begin{remark}
	The term ``almost'' above is different from the ``almost everywhere'' that we've been seeing.
\end{remark}
As usual, with a convergence definition, we have a Cauchy definition.
\begin{definition}[Almost uniformly Cauchy]
	Fix a measure space $(X,\mc S,\mu)$ and a normed vector space $B$. Then a sequence of functions $f_n\colon X\to B$ for $n\in\NN$ is \textit{almost uniformly Cauchy} if and only if every $\varepsilon>0$ has some $E^\varepsilon\in\mc S$ such that $\mu(E^\varepsilon)<\varepsilon$ and $\{f_n|_{X\setminus E}\}_{n\in\NN}$ is uniformly Cauchy.
\end{definition}
Now, here is the main result, which we will not prove today.
\begin{restatable}[Riesz--Weyl]{thm}{rwthm} \label{thm:rw}
	Fix a measure space $(X,\mc S,\mu)$ and a normed vector space $B$. Let $\{f_n\}_{n\in\NN}$ be a sequence of $\mc S$-measurable $B$-valued functions which are Cauchy in measure. Then there is a subsequence $\{f_n\}_{n\in\NN}$ which is almost uniformly Cauchy.
\end{restatable}
\noindent In particular, we will be able to define a limit function for the sequence $\{f_n\}_{n\in\NN}$ outside some null set, which will finally allow us to take limits of simple integrable functions in a way that makes sense.

\end{document}