% !TEX root = ../notes.tex

\documentclass[../notes.tex]{subfiles}

\begin{document}

\section{November 14}

Today we show that the space $L^1$ is complete. Here is a challenge problem.
\begin{remark}
	Here is a challenge problem. Fix a sequence of continuous functions $f_n\colon[0,1]\to[0,1]$. Show that if $f_n\to f$ pointwise, then $\norm{f_n}_1\to0$ as $n\to\infty$, where $\norm{f_n}_1$ is defined using the Riemann integral. There are proofs which do not use any measure theory!
\end{remark}

\subsection{Defining Integrals}
Last class we were about to prove the following result.
\intisdefinedprop*
\begin{proof}
	Unsurprisingly, set $h_n\coloneqq f_n-g_n$, which is a simple $\mu$-integrable function by \autoref{lem:simple-int-is-k-vec}. Note $\{h_n\}_{n\in\NN}$ is a mean Cauchy sequence by \autoref{lem:linear-combo-mean-cauchy} and converges to $f-f=0$ in measure by \autoref{lem:linear-combo-in-measure}. We want to show that $\norm{h_n}_1\to0$ as $n\to\infty$, so we define $j_n(x)\coloneqq\norm{h_n(x)}$, which is mean Cauchy by \autoref{lem:linear-combo-mean-cauchy}.

	This now reduces to the following lemma.
	\begin{lemma}
		Fix a measure space $(X,\mc S,\mu)$ and a mean Cauchy sequence of nonnegative simple $\mu$-integrable functions $f_n\colon X\to\RR$. If $f_n\to0$ in measure, then $\norm{f_n}_1\to0$ in measure.
	\end{lemma}
	\begin{proof}
		We use \autoref{thm:rw} to extract an almost uniformly Cauchy subsequence $\{f_{n_k}\}_{k\in\NN}$, which we then see almost uniformly converges to $0$. Because our sequence is mean Cauchy, it suffices to show $\norm{f_{n_k}}_1\to0$ as $k\to\infty$.

		As such, to simplify notation, set $g_n\coloneqq f_{n_k}$ so that $g_n\to0$ almost uniformly, and we want to show $\norm{g_n}_1\to0$ as $n\to\infty$. Fix $\varepsilon>0$, so we want $N$ such that $n\ge N$ has
		\[0\le\norm{g_n}_1\stackrel?<\varepsilon.\]
		This means we have to bound an integral, which we do in many pieces. To begin, our sequence $\{g_n\}_{n\in\NN}$ is mean Cauchy, so we start with some $N$ such that $m,n\ge N$ implies $\norm{g_m-g_n}_1<\varepsilon/4$.
		
		Continuing, we choose $E\in\mc S$ equal to $\bigcup_{n\in\NN}g_n^{-1}(B\setminus\{0\})$, and in fact, we set $F=g_N^{-1}(\{0\})$, which we note has finite measure because $f$ is a simple $\mu$-integrable function. Now, for $n\ge N$, we see
		\[\int_{E\setminus F}g_n\,d\mu=\int_{E\setminus F}|g_n(x)-g_N(x)|\,d\mu=\norm{g_n-g_N}_1<\frac\varepsilon4.\]
		Now, we see $\{g_n\}_{n\in\NN}$ converges almost uniformly on $F$ (because it converges almost uniformly on $X$), so we set $\delta\coloneqq\frac\varepsilon{4(1+\norm{g_N}_\infty)}$. This grants us $G\in\mc S$ with $\mu(F\setminus G)<\delta$ and $\{g_n\}_{n\in\NN}$ converging uniformly on $F\setminus G$.

		As such, we can choose $N'$ for which $n>N'$ has
		\[g_n(x)\le\frac\varepsilon{4(1+\mu(G))}.\]
		As such, $n\ge N'$ grants
		\[\int_{G}g_n\,d\mu\le\int\frac\varepsilon{4\mu(G)}\,d\mu=\frac\varepsilon4.\]
		It remains to handle what's happening on $F\setminus G$. Well, $\mu(F\setminus G)<\delta$, so whatever happens here is pretty small: note $n\ge N'$ has
		\begin{align*}
			\int_{F\setminus G}g_n\,d\mu &= \int_G\norm{g_n-g_N}\,d\mu+\int_Gg_N(x)\,d\mu \\
			&\le \norm{g_n-g_N}_1+\int_G\norm{g_N}_\infty\,d\mu \\
			&= \norm{g_n-g_N}_1+\mu(G)\cdot\norm{g_N}_\infty \\
			&\le \frac\varepsilon4+\frac\varepsilon{4(1+\norm{g_N}_\infty)}\cdot\norm{g_N}_\infty \\
			&\le \frac\varepsilon2.
		\end{align*}
		Adding our integrals together completes the proof.
	\end{proof}
	The above lemma finishes the proof.
\end{proof}
The above result grants us a natural bijection between equivalence classes of mean Cauchy sequences of simple $\mu$-integrable functions and ``almost everywhere'' equivalence classes of $\mu$-integrable functions. So we have constructed our completion of simple $\mu$-integrable functions.
\begin{remark}
	As an aside, we note that the $\norm\cdot_1$ norm is pretty poorly behaved at points. For example, the function $C([0,1])\to[0,1]$ by $f\mapsto f(1)$ is not continuous for $\norm\cdot_1$. Namely, define $f_n(x)=x^n$ so that $f_n\to0$ in mean as $n\to\infty$, but $f_n(1)\to1$ as $n\to\infty$.
\end{remark}
\begin{notation}
	We define $\mc L^1(X,\mc S,\mu,B)$ to be the set of all $\mu$-integrable functions.
\end{notation}
Now, we have just shown that $\mc L^1(X,\mc S,\mu,B)$ is the completion of the space of simple $\mu$-integrable functions and is therefore complete. One can show that this forms a $k$-vector space (where $k$ is the base field for $B$), and one can show some positivity and bounding results.

\end{document}