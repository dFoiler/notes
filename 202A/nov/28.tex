% !TEX root = ../notes.tex

\documentclass[../notes.tex]{subfiles}

\begin{document}

\section{November 28}

It's the last week of class, so it's time to go off the rails.

\subsection{Infinite Integrals}
As an application of \autoref{thm:mono-conv}, we get the following convention.
\begin{defi}
	Fix a measure space $(X,\mc S,\mu)$. Given an $\mathcal S$-measurable function $f\colon X\to\RR$ such that $f(x)\ge0$ always, we say
	\[\int_Xf\,d\mu\coloneqq+\infty\]
	if and only if there is a sequence of $\mu$-integrable functions $f_n\colon X\to\RR$ such that $f_{n+1}(x)\ge f_n(x)\ge0$ for each $x$ and $n$ such that $f_n\to f$ pointwise and $\int_Xf_n\,d\mu\to+\infty$.
\end{defi}
For this definition to make sense, we need a few lemmas.
\begin{lemma} \label{lem:get-monotone-seq-below}
	Fix a measure space $(X,\mc S,\mu)$. Given an $\mathcal S$-measurable function $f\colon X\to\RR$ such that $f(x)\ge0$, there exists some sequence of $\mu$-integrable functions $f_n\colon X\to\RR$ such that $f_{n+1}(x)\ge f_n(x)\ge0$ for each $x$ and $n$ such that $f_n\to f$ pointwise.
\end{lemma}
\begin{proof}
	We have two steps.
	\begin{enumerate}
		\item Because $f$ is $\mc S$-measurable, we may find a sequence of simple $\mc S$-measurable functions $g_n\colon X\to\RR$ such that $g_n\to f$ pointwise. Thus, by \autoref{lem:restrict-meas-functions}, the functions $g_n1_{[-n,n]}$ are still simple $\mc S$-measurable, but now we claim they are simple $\mu$-integrable. Indeed, for any $y\in\RR\setminus\{0\}$, we see that
		\[\left(g_n1_{[-n,n]}\right)^{-1}(\{y\})\subseteq[-n,n+1)\]
		because $x\notin[-n,n+1)$ gives $1_{[-n,n]}(x)=0$; thus, $\mu\left((g_n1_{[-n,n]})^{-1}(\{y\})\right)\le(n+1)--n<\infty$ by \autoref{lem:fin-additive-is-monotone}, which is what we wanted.
	
		Further, we claim that $g_n1_{[-n,n]}\to f$ pointwise as $n\to\infty$. Indeed, for any $x\in X$, fix some $\varepsilon>0$. We may find some $N_1$ such that $n\ge N$ implies
		\[|f(x)-g_n(x)|<\varepsilon.\]
		As such, we set $N\coloneqq\max\{N_1,|x|\}$ so that $n\ge N\ge|x|$ gives $g_n1_{[-n,n]}(x)=g_n(x)$, and so $n\ge N\ge N_1$ gives $|f(x)-g_n(x)|<\varepsilon$.
	
		\item Relabeling, the previous step constructed a sequence of simple $\mu$-integrable functions $g_n\colon X\to\RR$ converging to $f$ pointwise. It remains to deal with our bounding. For this, we delete our sequence of functions $f_n$ recursively. Define $f_1=0$, which is $\mu$-integrable by \autoref{lem:int-is-vec-space}.

		Now, given $f_n$, we define $f_{n+1}$ by
		\[f_{n+1}\coloneqq\max\{f_n,\min\{g_{n+1},f\}\}.\]
		Note $\min\{g_{n+1},f\}$ is $\mc S$-measurable by \autoref{ex:min-max-meas} and thus $\mu$-integrable by \autoref{cor:bounded-meas-is-int} because $\min\{g_{n+1}(x),f(x)\}\le g_{n+1}(x)$ for each $x\in X$. Thus, we see $f_{n+1}$ is $\mu$-integrable (inductively) by \autoref{ex:min-max-int}. We also note $f_n(x)\le f_{n+1}(x)$ for any $n$ and $x$ by construction, so we get $f_n(x)\ge f_1(x)=0$.

		It remains to check $f_n\to f$ pointwise; fix any $x\in X$. To begin, note $f_n(x)\le f(x)$ for each $n$. For $n=1$, this is by hypothesis on $f$, and in general we note that $f_n(x)\le f(x)$ and $\min\{g_{n+1}(x),f(x)\}\le f(x)$ forces $f_{n+1}(x)\le f(x)$.

		However, for all $\varepsilon>0$, we can find some $N>0$ such that $n\ge N$ implies $|g_n(x)-f(x)|<\varepsilon$. We claim that $|f_n(x)-f(x)|<\varepsilon$ for each $n\ge N>0$ as well. There are two cases.
		\begin{itemize}
			\item If $g_n(x)\le f(x)$, then we note
			\[g_n(x)=\min\{g_n(x),f(x)\}\le f_n(x)\le f(x),\]
			so $|f(x)-f_n(x)|=f(x)-f_n(x)\le f(x)-g_n(x)<\varepsilon$.
			\item If $g_n(x)\ge f(x)$, then we note $f_n(x)\le f(x)$ while $f_n(x)\ge\min\{g_n(x),f(x)\}=f(x)$, so $f_n(x)=f(x)$.
		\end{itemize}
		The above checks complete the proof.
		\qedhere
	\end{enumerate}
\end{proof}
\begin{proposition} \label{prop:extended-int-makes-sense}
	Fix a measure space $(X,\mc S,\mu)$ and an $\mathcal S$-measurable function $f\colon X\to\RR$ with $f(x)\ge0$ for all $x\in X$. Then exactly one of the following is true.
	\begin{itemize}
		\item $f$ is $\mu$-integrable.
		\item $\int_Xf\,d\mu=+\infty$.
	\end{itemize}
\end{proposition}
\begin{proof}
	By \autoref{lem:get-monotone-seq-below}, there certainly exists some sequence of $\mu$-integrable functions $f_n\colon X\to\RR$ such that $f_{n+1}(x)\ge f_n(x)\ge0$ for each $x$ and $n$ such that $f_n\to f$ pointwise. As such, note that the sequence of integrals
	\[I_n\coloneqq\int_Xf_n\,d\mu\]
	are increasing by \autoref{lem:bound-ints}. Thus, if the sequence is bounded above, we note $f$ is $\mu$-integrable by \autoref{thm:mono-conv}. Otherwise, $I_n\to+\infty$ as $n\to\infty$, so $\int_Xf\,d\mu=+\infty$.

	Thus, we have so far shown that at least one of the conclusions is true. It remains to show that they cannot both be true. Well, suppose $f\colon X\to\RR$ is $\mu$-integrable, and we show $\int_Xf\,d\mu\ne+\infty$. If we have any increasing sequence $\{g_n\}_{n\in\NN}$ of $\mu$-integrable functions such that $g_n\to f$ pointwise, then we see $g_n(x)\le f(x)$ for each $x$, so we may use \autoref{lem:bound-ints} to upper-bound
	\[\int_Xf_n\,d\mu\le\int_Xf\,d\mu.\]
	Thus, the sequence $\int_Xf_n\,d\mu$ does not go to $+\infty$.
\end{proof}
While we're here, we pick up a few of our standard bounds.
\begin{lemma} \label{lem:extended-equal-ae-gives-equal-ints}
	Fix a measure space $(X,\mc S,\mu)$ and some $\mc S$-measurable functions $f,g\colon X\to\RR$ such that $f(x),g(x)\ge0$ for all $x\in X$. If $f(x)=g(x)$ almost everywhere, then
	\[\int_Xf\,d\mu=\int_Xg\,d\mu.\]
\end{lemma}
\begin{proof}
	We are given $E\in\mc S$ such that $\mu(E)=0$ while $f(x)=g(x)$ for $x\in X\setminus E$. Now, we have the following cases.
	\begin{itemize}
		\item If $f$ is $\mu$-integrable or $g$ is $\mu$-integrable, then we note $f(x)\le g(x)$ and $g(x)\le f(x)$ almost everywhere (namely, on $X\setminus E$), so \autoref{cor:bounded-meas-is-int} implies that both $f$ and $g$ are both $\mu$-integrable. To finish, note $f(x)=g(x)$ almost everywhere implies that $\int_Xf\,d\mu=\int_Xg\,d\mu$ by \autoref{rem:equal-ae-gives-equal-ints}.
		\item If neither $f$ nor $h$ are $\mu$-integrable, then \autoref{prop:extended-int-makes-sense} tells us that $\int_Xf\,d\mu=\int_Xg\,d\mu=\infty$.
		\qedhere
	\end{itemize}
\end{proof}
\begin{lemma} \label{lem:add-extended-int}
	Fix a measure space $(X,\mc S,\mu)$. Given $\mc S$-measurable functions $f,g\colon X\to\RR$ such that $f(x),g(x)\ge0$ for all $x\in X$, we have
	\[\int_X(f+g)\,d\mu=\int_Xf\,d\mu+\int_Xg\,d\mu,\]
	where we permit values to be $+\infty$.
\end{lemma}
\begin{proof}
	Note that $f+g$ is $\mc S$-measurable by \autoref{lem:meas-is-vec-space}. We have two cases.
	\begin{itemize}
		\item If $\int_X(f+g)\,d\mu\ne\infty$, then $(f+g)$ is $\mu$-integrable by \autoref{prop:extended-int-makes-sense}. However, we note $f(x),g(x)\le(f+g)(x)$ for each $x\in X$ because $f(x),g(x)\ge0$, so \autoref{lem:bound-ints} tells us that $f$ and $g$ are both $\mu$-integrable. Thus, the result follows from \autoref{prop:int-is-linear}.
		\item Suppose $\int_X(f+g)\,d\mu=+\infty$. If $f$ and $g$ are both $\mu$-integrable, $(f+g)$ is $\mu$-integrable by \autoref{lem:int-is-vec-space}, which violates the hypothesis of this case by \autoref{prop:extended-int-makes-sense}. Thus, one of $f$ or $g$ is not $\mu$-integrable. Without loss of generality, we say $f$ is not $\mu$-integrable, so
		\[\int_Xf\,d\mu=+\infty\]
		follows from \autoref{prop:extended-int-makes-sense}. Because $\int_Xg\,d\mu\ge0$ either when $g$ is $\mu$-integrable (by \autoref{lem:bound-ints} and \autoref{prop:int-is-linear}) or when $g$ is not $\mu$-integrable (by \autoref{prop:extended-int-makes-sense}), the result follows.
		\qedhere
	\end{itemize}
\end{proof}
\begin{lemma} \label{lem:bound-extended-int}
	Fix a measure space $(X,\mc S,\mu)$. Given $\mc S$-measurable functions $f,g\colon X\to\RR$ such that $f(x)\ge g(x)\ge0$ for all $x\in X$, we have
	\[\int_Xf(x)\,d\mu\ge\int_Xg(x)\,d\mu,\]
	where we permit values to be $+\infty$.
\end{lemma}
\begin{proof}
	We have two cases.
	\begin{itemize}
		\item If $f$ is $\mu$-integrable, then $g$ is $\mu$-integrable by \autoref{cor:bounded-meas-is-int}, so the result follows from \autoref{lem:bound-ints}.
		\item If $f$ is not $\mu$-integrable, then $\int_Xf\,d\mu=+\infty$ by \autoref{prop:extended-int-makes-sense}, so the result follows.
		\qedhere
	\end{itemize}
\end{proof}
\begin{lemma} \label{lem:extended-mono-conv}
	Fix a measure space $(X,\mc S,\mu)$. Given some $\mc S$-measurable $f\colon X\to\RR$ with $f(x)\ge0$ for each $x$, suppose there is a sequence of $\mc S$-measurable functions $f_n\colon X\to\RR$ such that $f_{n+1}(x)\ge f_n(x)\ge0$ for each $x$ and $n$ and $f_n\to f$ pointwise. Then
	\[\int_Xf\,d\mu=\lim_{n\to\infty}\int_Xf_n\,d\mu.\]
\end{lemma}
\begin{proof}
	The difficulty here is that the $f_n$ are merely $\mc S$-measurable. We quickly check that the limit makes sense: for any $n$, note that $0\le f_n(x)\le f_{n+1}(x)$ for each $x\in X$ tells us $0\le\int_Xf_n\,d\mu\le\int_Xf_{n+1}\,d\mu$ by \autoref{lem:bound-extended-int}, so the right-hand limit is either finite or $+\infty$.
	
	Thus, we have two cases to appropriately deal with infinity.
	\begin{itemize}
		\item If each $f_n$ is $\mu$-integrable, then this follows from either \autoref{rem:apply-mono-conv} or the definition of an integral evaluating to $+\infty$. (Technically, we are using the fact that the limit function of $\{f_n\}_{n\in\NN}$ is unique almost everywhere by \autoref{lem:norm-diff-zero-iff-equal-ae}, so $f_n\to f$ in mean.)
		\item If any $f_n$ is not $\mu$-integrable, then \autoref{prop:extended-int-makes-sense} tells us $\int_Xf_n\,d\mu=+\infty$.
		
		Now, for any $m\ge n$ has $f_m(x)\le f_n(x)$ for all $x$, so
		\[+\infty=\int_Xf_n\,d\mu=\int_Xf_m\,d\mu\]
		by \autoref{lem:bound-extended-int}, so $\int_Xf_m\,d\mu=+\infty$. It follows that
		\[\lim_{m\to\infty}\int_Xf_m\,d\mu=+\infty.\]
		On the other hand, for any $x$, we note that $f_m(x)\to f(x)$ as $m\to\infty$, but $f_m(x)\ge f_n(x)$ for $m\ge n$, so $f_n(x)\le f(x)$ follows. Applying \autoref{lem:bound-extended-int} again tells us $\int_Xf\,d\mu=+\infty$.
		\qedhere
	\end{itemize}
\end{proof}

\subsection{Defining \texorpdfstring{$\mc L^p$}{ Lp}}
Here is our definition.
\begin{defihelper}[{$\mathcal L^p$-space}] \nirindex{Lp-space@$\mathcal L^p$-space}
	Fix a measure space $(X,\mc S,\mu)$ and a Banach space $(B,\norm\cdot)$. Given some $p\in(0,\infty)$, we define
	\[\mc L^p(X,\mc S,\mu,B)\coloneqq\left\{\mc S\text{-measurable}f:\norm f^p\text{ is }\mu\text{-integrable}\right\}.\]
\end{defihelper}
\begin{lemma} \label{lem:lp-vec-space}
	Fix a measure space $(X,\mc S,\mu)$ and a $k$-Banach space $(B,\norm\cdot)$. Then $\mc L^p(X,\mc S,\mu,B)$ is a $k$-vector space.
\end{lemma}
\begin{proof}
	We have the following checks. As usual, let $|\cdot|$ denote the norm on $k$.
	\begin{itemize}
		\item Zero: note the zero function $z\colon X\to B$ has $\norm z^p(x)=0$ for each $x\in X$, which is $\mu$-integrable by \autoref{lem:int-is-vec-space}.
		\item Scalar multiplication: if $f\in\mc L^p(X,\mc S,\mu,B)$, then $\norm f^p$ is $\mu$-integrable. However, for $r\in k$, we note that $rf$ is $\mc S$-maesurable by \autoref{lem:meas-is-vec-space}, and
		\[\norm{rf}^p(x)=\left(|r|^p\cdot\norm f^p\right)(x)\]
		for each $x\in X$, so the fact that $\norm f^p$ is $\mu$-integrable implies that $\norm{rf}^p$ is also $\mu$-integrable. This finishes.
		\item Additive: suppose $f,g\in\mc L^p(X,\mc S,\mu,B)$ so that we want to show $f+g\in\mc L^p(X,\mc S,\mu,B)$, which means that we want $\norm{f+g}^p\in\mc L^1(X,\mc S,\mu,B)$. Well, $\norm{f+g}^p$ is $\mc S$-measurable by applying \autoref{lem:meas-is-vec-space} and \autoref{cor:take-norms-is-measurable} and \autoref{cor:compose-cont-is-meas} (with the continuous function $x\mapsto|x|^p$), so we merely need to upper-bound $\norm{f+g}^p$ and use \autoref{cor:bounded-meas-is-int}.
		
		Indeed, the triangle inequality implies that
		\[\norm{f(x)+g(x)}\le\norm{f(x)}+\norm{g(x)}\le2\max\{\norm{f(x)},\norm{g(x)}\}\]
		for each $x\in X$, so
		\begin{align*}
			\norm{f(x)+g(x)}^p &\le 2^p\big(\max\{\norm{f(x)},\norm{g(x)}\}\big)^p \\
			&= 2^p\big(\max\{\norm{f(x)}^p,\norm{g(x)}^p\}\big) \\
			&\le 2^p\left(\norm{f(x)}^p+\norm{g(x)}^p\right).
		\end{align*}
		However, each $\norm f^p$ and $\norm g^p$ are $\mu$-integrable by hypothesis, so $2^p\norm f^p+2^p\norm g^p$ is $\mu$-integrable by \autoref{lem:int-is-vec-space}. Thus, \autoref{cor:bounded-meas-is-int} finishes.
		\qedhere
	\end{itemize}
\end{proof}
Here is a reason to care about $\mc L^p$: just like $\mc L^1$, they have a well-behaved semi-norm (in good cases).
\begin{notation}
	Fix a measure space $(X,\mc S,\mu)$ and a Banach space $(B,\norm\cdot)$. For $p\in(0,\infty)$ and $f\in\mc L^p(X,\mc S,\mu,B)$, we define
	\[\norm f_p\coloneqq\left(\int_X\norm f^p\,d\mu\right)^{1/p}.\]
	Note that this integral is well-defined by definition of $L^p(X,\mc S,\mu,B)$.
\end{notation}
Here is the analogue for \autoref{lem:norm-diff-zero-iff-equal-ae}.
\begin{lemma} \label{lem:lp-zero-iff-zero-ae}
	Fix a measure space $(X,\mc S,\mu)$ and a Banach space $(B,\norm\cdot)$, and fix $p\in(0,\infty)$. Given some $f\in\mc L^p(X,\mc S,\mu,B)$, we see $\norm f_p=0$ if and only if $f(x)=0$ almost everywhere.
\end{lemma}
\begin{proof}
	By \autoref{lem:norm-diff-zero-iff-equal-ae}, we see that
	\[\norm f^p_p=\int_X\norm f^p\,d\mu\]
	equals zero if and only if $\norm f^p(x)=0$ almost everywhere. However, $\norm f^p(x)=0$ is equivalent to saying $\norm{f(x)}^p=0$, which is equivalent to $\norm{f(x)}=0$, which is equivalent to $f(x)=0$. Thus, one side of
	\[\left\{x\in X:\norm f^p(x)\ne0\right\}=\{x\in X:f(x)\ne0\}\]
	is a null set if and only if the other is, which finishes.
\end{proof}
\begin{corollary} \label{cor:lp-diff-zero-iff-equal-ae}
	Fix a measure space $(X,\mc S,\mu)$ and a Banach space $(B,\norm\cdot)$, and fix $p\in(0,\infty)$. Given some $f,g\in\mc L^p(X,\mc S,\mu,B)$, we see $\norm{f-g}_p=0$ if and only if $f(x)=g(x)$ almost everywhere.
\end{corollary}
\begin{proof}
	Note that $\norm{f-g}_p=0$ is equivalent to $(f-g)(x)=0$ almost everywhere by \autoref{lem:lp-zero-iff-zero-ae}, which is equivalent to $\{x\in X:f(x)\ne g(x)\}$ being a null set, which is what we wanted.
\end{proof}
Continuing, we now show that $\norm\cdot_2$ is a semi-norm.
\begin{proposition} \label{prop:l2-semi-norm}
	Fix a measure space $(X,\mc S,\mu)$ and a Banach space $(B,\norm\cdot)$. Then the function $\norm\cdot_2$ defines a semi-norm on $\mc L^2(X,\mc S,\mu,B)$.
\end{proposition}
\begin{proof}
	We quickly run our easier checks. Let $|\cdot|$ denote the norm on our base field $k$, and fix some $f\in\mc L^2(X,\mc S,\mu,B)$.
	\begin{itemize}
		\item Zero: the zero function $z\colon X\to B$ has $\norm z^2\colon X\to\RR$ equal to zero everywhere, so $\norm z^2=1_\emp$, so
		\[\int_X\norm z^2\,d\mu=\mu(\emp)=0\]
		by \autoref{ex:integrate-indicator}.
		\item Nonnegative: we note that $\norm f(x)\ge0$ for each $x\in X$, so $\norm f^2(x)\ge0$ for each $x\in X$, so
		\[\int_X\norm f^2\,d\mu\ge0\]
		by \autoref{lem:int-positivity}, so $\norm f_2\ge0$ follows.
		\item Homogeneous: if $r\in k$, then we note $\norm{rf}(x)=(|r|\cdot\norm f)(x)$ for each $x\in X$. Thus, by \autoref{prop:int-is-linear} tells us
		\[\norm{rf}_2=\left(\int_X\norm{rf}^2\,d\mu\right)^{1/2}=\left(\int_X|r|\cdot\norm f^2\,d\mu\right)^{1/2}=|r|\cdot\left(\int_X\norm f^2\,d\mu\right)^{1/2}=|r|\cdot\norm f_2.\]
	\end{itemize}
	It remains to check the triangle inequality, which is harder. We have the following lemma.
	\begin{lemma}[Cauchy--Schwarz] \label{lem:cs-ineq}
		Fix a measure space $(X,\mc S,\mu)$ and a Banach space $(B,\norm\cdot)$. Given $\mc S$-measurable functions $f,g\in\mc L^2(X,\mc S,\mu,B)$, then $\norm f\cdot\norm g$ is $\mu$-integrable, and
		\[\int_X\big(\norm f\cdot\norm g\big)\,d\mu\le\frac{\norm f_2^2+\norm g_2^2}2.\]
	\end{lemma}
	\begin{proof}
		Quickly, we see $\norm f\cdot\norm g$ is $\mc S$-measurable by applying \autoref{lem:prod-of-meas} to \autoref{cor:take-norms-is-measurable}. It remains to upper-bound $\norm f\cdot\norm g$.
		
		Now, the main point is the arithmetic mean-geometric mean inequality: for $r,s\in\RR_{\ge0}$, we see
		\[0\le(r-s)^2=r^2+s^2-2rs,\]
		so $r^2+s^2\ge2rs$. Applying this to our situation, we see
		\[2\norm {f(x)}\cdot\norm {g(x)}\le\norm {f(x)}^2+\norm {g(x)}^2\]
		for each $x\in X$, so
		\[\big(\norm f\cdot\norm g\big)(x)\le\left(\frac{\norm f^2+\norm g^2}2\right)(x)\]
		for each $x\in X$. However, $\frac{\norm f^2+\norm g^2}2$ is $\mu$-integrable by \autoref{lem:int-is-vec-space}, so we conclude that $\norm f\cdot\norm g$ is $\mu$-integrable by \autoref{cor:bounded-meas-is-int}.
		
		Continuing, \autoref{lem:bound-ints} tells us
		\[\int_X\big(\norm f\cdot\norm g\big)\,d\mu\le\int_X\frac{\norm f^2+\norm g^2}2\,d\mu,\]
		and now the right-hand side simplifies to $\frac12\left(\norm f_2^2+\norm g_2^2\right)$ by \autoref{prop:int-is-linear}. This finishes.
	\end{proof}
	We now proceed with the proof of the triangle inequality. Fix $f,g\in\mc L^2(X,\mc S,\mu,B)$. We quickly deal with the case of $\norm f_2=0$. Here, $\norm f_2=0$ implies that $f(x)=0$ almost everywhere by \autoref{lem:lp-zero-iff-zero-ae}, so $(f+g)(x)=g(x)$ almost everywhere, so $\norm{f+g}^2(x)\le\norm g^2(x)$ almost everywhere, so \autoref{lem:bound-ints} implies
	\[\int_X\norm{f+g}^2\,d\mu\le\int_X\norm{g}^2\,d\mu\le\int_X\norm{f+g}^2\,d\mu.\]
	Thus, $\norm{f+g}_2\le\norm g_2=\norm f_2+\norm g_2$ follows. Note that a similar argument works for $\norm g_2=0$.

	Thus, we may assume that $\norm f_2,\norm g_2\ne0$, which allows us to set $h\coloneqq\frac f{\norm f_2}$ and $k\coloneqq\frac g{\norm g_2}$. Notably, $\norm h_2=\norm k_2=1$ by the homogeneity check above. As such, \autoref{lem:cs-ineq} grants
	\[\int_X\big(\norm h\cdot\norm k\big)\,d\mu\le\frac{1+1}2=1.\]
	However, $\norm{h}=\norm{f}/\norm f_2$ and $\norm k=\norm g/\norm g_2$, so \autoref{prop:int-is-linear} implies
	\[\int_X\big(\norm f\cdot\norm g\big)\,d\mu\le2\norm f_2\cdot\norm g_2.\]
	This now rearranges to the desired inequality: given $f,g\in\mc L^2(X,\mc S,\mu,B)$, we see
	\begin{align*}
		\norm{f+g}_2^2 &= \int_X\big(\norm{f+g}\big)^2\,d\mu \\
		&\stackrel*\le \int_X\big(\norm{f}+\norm{g}\big)^2\,d\mu \\
		&= \int_X\norm f^2\,d\mu+\int_X\norm g^2\,d\mu+2\int_X\big(\norm f\cdot\norm g\big)\,d\mu \\
		&\le \norm f_2^2+\norm g_2^2+2\norm f_2\cdot\norm g_2 \\
		&= \big(\norm f_2+\norm g_2\big)^2,
	\end{align*}
	and taking the square root finishes. Notably, $\stackrel*\le$ has used the triangle inequality and \autoref{lem:bound-ints} (and the following equality used \autoref{prop:int-is-linear}).
\end{proof}
\begin{remark}
	In fact, $\norm f_p$ is a norm in general, but it is somewhat harder to show. Roughly speaking, the difficulty lies in establishing an analogue for \autoref{lem:cs-ineq}.
\end{remark}
\begin{remark}
	If $p\in(0,1)$, then $\norm f_p$ is not a norm. In particular, it does not satisfy the triangle inequality. For that matter, $p\in(0,1)$ do very strange things. For example, if we define
	\[U_r\coloneqq\left\{f\in\mc L^p(X,\mc S,\mu,B):\int_X\norm f^p\,d\mu<r\right\},\]
	then the convex hull of $U_r$ recovers all of $\mc L^p(X,\mc S,\mu,B)$. In particular, one can show that there are thus no nonzero continuous linear functionals on $\mc L^p(X,\mc S,\mu,B)$.
\end{remark}

\subsection{Defining \texorpdfstring{$L^p$}{ Lp}}
Now that we have a semi-norm, we can mod out to get our norm.
\begin{notation}
	Fix a measure space $(X,\mc S,\mu)$ and a Banach space $B$, and choose $p\in[1,\infty)$ such that $\norm\cdot_p$ defines a semi-norm on $\mc L^p(X,\mc S,\mu,B)$. We set $\mc N(X,\mc S,\mu,B)\coloneqq\{f\in\mc L^2(X,\mc S,\mu,B):\norm f_2=0\}$ and
	\[L^2(X,\mc S,\mu,B)\coloneqq\mc L^2(X,\mc S,\mu,B)/\mc N(X,\mc S,\mu,B).\]
\end{notation}
\begin{remark}
	Given $f,g\in\mc L^p(X,\mc S,\mu,B)$, note that $[f]=[g]$ in $L^p(X,\mc S,\mu,B)$ if and only if $\norm{f-g}_p=0$ by definition, which is equivalent to $f(x)=g(x)$ almost everywhere by \autoref{cor:lp-diff-zero-iff-equal-ae}.
\end{remark}
\begin{proposition} \label{prop:lp-norm}
	Fix a measure space $(X,\mc S,\mu)$ and a Banach space $B$, and choose $p\in[1,\infty)$ such that $\norm\cdot_p$ defines a semi-norm on $\mc L^p(X,\mc S,\mu,B)$. The function $\norm\cdot_p$ descends to a norm on $L^p(X,\mc S,\mu,B)$.
\end{proposition}
\begin{proof}
	This follows from \autoref{prop:semi-norm-to-norm}.
\end{proof}
We would like to show that $L^p$ is complete, but this requires some work. Namely, we will require Fatou's lemma, a result we will state and prove next class.

\end{document}