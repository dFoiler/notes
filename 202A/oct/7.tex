% !TEX root = ../notes.tex

\documentclass[../notes.tex]{subfiles}

\begin{document}

\section{October 7}

We continue our discussion into measure theory.

\subsection{\texorpdfstring{$\sigma$}{ Sigma}-Things}
Motivated by \autoref{ex:want-countable-union}, we see that we want to be able to measure countable unions. As such, we have the following definitions.
\begin{defihelper}[{$\sigma$-ring}] \nirindex{Sigma-ring@$\sigma$-ring}
	Fix a set $X$. Then a ring $\mc R\subseteq\mc P(X)$ is a \textit{$\sigma$-ring} if and only if $R$ is closed under countable unions.
\end{defihelper}
\begin{remark} \label{rem:sring-has-intersections}
	As in \autoref{rem:ring-has-intersections}, we note $\sigma$-rings $\mc S$ have countable intersections. Fix some $\{E_i\}_{i=1}^\infty\subseteq\mc S$. Then we note
	\[E_1{\mathbin\bigg\backslash}\bigcup_{i=1}^\infty(E_1\setminus E_i)=\bigcap_{i=1}^\infty(E_1\setminus(E_1\setminus E_i))=\bigcap_{i=1}^\infty(E_1\cap E_i)=E_1\cap\bigcap_{i=1}^\infty E_i=\bigcap_{i=1}^\infty E_i\]
	lives in $\mc S$, finishing.
\end{remark}
\begin{defihelper}[{$\sigma$-algebra}] \nirindex{Sigma-algebra@$\sigma$-algebra}
	Fix a set $X$. Then a ring $\mc R\subseteq\mc P(X)$ is a \textit{$\sigma$-algebra} if and only if $R$ is a $\sigma$-ring and contains $X$.
\end{defihelper}
\begin{ex}
	Given a set $X$, we see $\mc P(X)$ is a $\sigma$-ring because a countable union of subsets of $X$ is still a subset of $X$. Further, $\mc P(X)$ is a $\sigma$-algebra because $X\in\mc P(X)$.
\end{ex}
\begin{example} \label{ex:countable-sigma-ring}
	Fix a set $X$. Then the collection $\mc S\subseteq\mc P(X)$ of countable subsets of $X$ is a $\sigma$-ring; here are our checks.
	\begin{itemize}
		\item Countable union: suppose $\{E_i\}_{i=1}^\infty\subseteq\mc S$. Then
		\[\bigcup_{i=1}^\infty E_i\]
		is the countable union of countable subsets of $X$ and therefore countable. It follows that $\bigcup_{i=1}^\infty E_i\in\mc S$.
		\item Subtraction: if $E,F\in\mc S$, then $E$ and $F$ are both countable, so $E\setminus F\subseteq E$ is still countable, so $E\setminus F\in\mc S$.
	\end{itemize}
	Notably, if $X$ itself is not an uncountable set, then $X\notin\mc S$, so $\mc S$ is not a $\sigma$-algebra.
\end{example}
As usual, we may give the collection of all $\sigma$-rings (and $\sigma$-algebras) the subposet structure coming from inclusion on $\mc P(X)$. For example, $\mc P(X)$ is the largest collection in $\mc P(X)$ and is thus the largest $\sigma$-ring and also the largest $\sigma$-algebra.

Analogous to our discussion of topologies in \autoref{prop:intersecttops}, we pick up the following lemma to make our $\sigma$-rings smaller.
\begin{lemma} \label{lem:intersect-rings}
	Fix a set $X$, and fix a collection $\Sigma$ of rings, $\sigma$-rings, or $\sigma$-algebra. Then
	\[\mc S\coloneqq\bigcap_{\mc R\in\Sigma}\mc R\]
	is another ring, $\sigma$-ring, or $\sigma$-algebra, respectively.
\end{lemma}
\begin{proof}
	We show the axioms get inherited individually.
	\begin{listalph}
		\item Suppose that each $\mc R\in\Sigma$ is closed under finite unions. Then for any $E,F\in\mc S$, we see $E,F\in\mc R$ for each $\mc R\in\Sigma$, so $E\cup F\in\mc R$ for each $\mc R\in\Sigma$, so $E\cup F\in\mc S$.
		\item Suppose that each $\mc R\in\Sigma$ is closed under subtraction. Then for any $E,F\in\mc S$, we see $E,F\in\mc R$ for each $\mc R\in\Sigma$, so $E\setminus F\in\mc R$ for each $\mc R\in\Sigma$, so $E\setminus F\in\mc S$.
		\item Suppose that each $\mc R\in\Sigma$ is closed under countable union. Then for any countable collection $\{E_i\}_{i=1}^\infty\in\mc S$, we see $\{E_i\}_{i=1}^\infty\in\mc R$ for each $\mc R\in\Sigma$, so $\bigcup_{i=1}^\infty E_i\setminus F\in\mc R$ for each $\mc R\in\Sigma$, so $\bigcup_{i=1}^\infty E_i\in\mc S$.
		\item Suppose that each $\mc R\in\Sigma$ contains $X$. Then $X\in\mc S$.
	\end{listalph}
	The above checks complete the proof. For example, if $\Sigma$ contains $\sigma$-rings, then checks (a)--(c) show $\mc S$ is still a $\sigma$-ring.
\end{proof}
\begin{corollary} \label{cor:get-small-sigma-ring}
	Fix a set $X$ and a collection $\mc C\subseteq\mc P(X)$. Then there is a unique smallest ring, $\sigma$-ring, or $\sigma$-algebra containing $\mc C$.
\end{corollary}
\begin{proof}
	Let $\Sigma$ denote the collection of all rings, $\sigma$-rings, or $\sigma$-algebras containing $\mc C$. We want to show that $\Sigma$ contains a unique minimum element. Well, we set
	\[\mc S\coloneqq\bigcap_{\mc R\in\Sigma}\mc R.\]
	Notably, $\mc S\in\Sigma$ by \autoref{lem:intersect-rings}, and $\mc S$ is its minimum somewhat directly: for any $\mc R\in\Sigma$, we have $\mc S\subseteq\mc R$ by construction of $\mc S$.
\end{proof}
This gives us the following definition.
\begin{defihelper}[{$\sigma$-ring generated by}] \nirindex{Sigma-ring@$\sigma$-ring!Generated $\sigma$-ring}
	Fix a set $X$. Then give a collection $\mc C$, we let $\mc S(\mc C)$ denote the \textit{$\sigma$-ring generated by $\mc C$}, as conjured by \autoref{cor:get-small-sigma-ring}.
\end{defihelper}
There are analogous definitions for ring and $\sigma$-algebra, but we won't state them explicitly.
\begin{remark}
	As usual, we note that $\mc C\subseteq\mc C'$ implies $\mc S(\mc C)\subseteq\mc S(C')$ because $\mc S(\mc C')$ is a $\sigma$-ring containing $\mc C$.
\end{remark}
\begin{remark}
	Also as usual, if $\mc S$ is already a $\sigma$-ring, then $\mc S(\mc S)=\mc S$. Of course, $\mc S\subseteq\mc S(\mc S)$, but also $\mc S$ is a $\sigma$-ring containing $\mc S$, so $\mc S(\mc S)\subseteq\mc S$ follows.
\end{remark}
\begin{example}
	Fix a set $X$. We claim $\sigma$-ring generated by the collection $\mc F$ finite subsets of $X$ is the $\sigma$-ring $\mc S$ of countable subsets of $X$. Certainly $\mc S(\mf F)\subseteq\mc S$ because $\mc S$ is a $\sigma$-ring by \autoref{ex:countable-sigma-ring}. On the other hand, any countable subset $E\subseteq X$ has
	\[E=\bigcup_{x\in E}\{x\}\]
	while $\{x\}\in\mc F\subseteq\mc S(\mc F)$ and therefore $E\in\mc S(\mc F)$. Thus, $\mc S\subseteq\mc S(\mc F)$.
\end{example}

\subsection{Measures}
We are now ready to define measures.
\begin{definition}[Countably additive]
	Fix a set $X$ and a collection of subsets $\mc C\subseteq\mc P(X)$. A function $\mu\colon\mc C\to[0,\infty]$ is \textit{countably additive} if and only if any pairwise disjoint subcollection $\{E_i\}_{i=1}^\infty\subseteq\mc C$ with $\bigsqcup_{i=1}^\infty E_i\in\mc C$ has
	\[\mu\Bigg(\bigsqcup_{i=1}^\infty E_i\Bigg)=\sum_{i=1}^\infty\mu(E_i).\]
	Notably, we are allowed to have the right-hand side diverge to $\infty$ if the left-hand side is $\infty$.
\end{definition}
\begin{remark} \label{rem:avoid-checking-countably-additive}
	In general, it is pretty difficult to actually show that a function is countably additive, but one can take advantage of the fact that
	\[\bigsqcup_{i=1}^\infty E_i\]
	might not actually be in $\mc C$.
\end{remark}
And here is our definition.
\begin{definition}[Measure]
	Fix a set $X$ and $\sigma$-ring $\mc S$. Then a \textit{measure} on $\mc S$ is a function $\mu\colon\mc S\to[0,\infty]$ which is countably additive.
\end{definition}
\begin{remark}
	Note that the countable unions of sets in $\mc S$ to check the countably additive condition are always in $\mc S$ because $\mc S$ is a $\sigma$-ring. Namely, the trick suggested in \autoref{rem:avoid-checking-countably-additive} doesn't help us.
\end{remark}
\begin{remark}
	In general, it is not a good idea to ask for unions larger than countable. Approximately speaking, we really want to have countable unions, but we need to be careful adding any other infinities. The main problem is that those infinite sums don't have easy notions of convergence. Even if we don't want to work with something like nets to allow larger convergences, then allowing arbitrary unions for $E\subseteq X$ gives
	\[\mu(E)=\sum_{x\in X}\mu(\{x\}),\]
	which intuitively should vanish if we make our points have measure $0$.
\end{remark}
\begin{remark} \label{rem:measure-emp}
	Fix a set $X$ and measure $\mu\colon\mc S\to[0,\infty]$. If $\mu(\emp)<\infty$, then note $\emp=\bigsqcup_{i=1}^\infty\emp$ implies that the sum $\sum_{i=1}^\infty\mu(\emp)=\mu(\emp)$ converges, so $\mu(\emp)=0$ is forced. Otherwise, if $\mu(\emp)=\infty$, then any $E\in\mc S$ has $E=E\sqcup\emp$, so $\mu(E)=\mu(E)+\mu(\emp)=\infty$.
\end{remark}
\begin{remark} \label{rem:restrict-measure}
	If $\mu$ is a measure a $\sigma$-ring $\mc S$, then $\mu|_{\mc T}$ remains a measure on any $\sigma$-ring $\mc T\subseteq\mc S$. Indeed, any pairwise disjoint subcollection $\{T_i\}_{i=1}^\infty\subseteq\mc T$ also lives in $\mc S$, so we maintain having
	\[\mu|_{\mc T}\Bigg(\bigsqcup_{i=1}^\infty T_i\Bigg)=\mu\Bigg(\bigsqcup_{i=1}^\infty T_i\Bigg)=\sum_{i=1}^\infty\mu(T_i)=\sum_{i=1}^\infty\mu|_{\mc T}(T_i).\]
\end{remark}
Let's see some examples.
\begin{exe} \label{exe:discrete-integral}
	More generally, fix a set $X$ using the $\sigma$-ring $\mc S\coloneqq\mc P(X)$ of countable subsets of $X$. For a function $f\colon X\to[0,\infty)$, we define
	\[\mu_f(E)\coloneqq\sum_{x\in E}f(x)\]
	for each countable subset $E\subseteq X$. Then $\mu_f$ is a measure.
\end{exe}
\begin{proof}
	Note that the order of the sum over $x\in X$ doesn't matter because if the sum converges, then it absolutely converges because all the terms in the sum are positive. Now, to see that we have a measure, pick up some countably many pairwise disjoint countable subsets $\{E_i\}_{i=1}^\infty$ of $X$. Then
	\[\mu_f\Bigg(\bigsqcup_{i=1}^\infty E_i\Bigg)=\sum_{x\in\bigsqcup_{i=1}^\infty E_i}f(x)\stackrel*=\sum_{i=1}^\infty\sum_{x\in E_i}f(x)=\sum_{i=1}^\infty\mu_f(E_i),\]
	where $\stackrel*=$ holds because each $x\in\bigsqcup_{i=1}^\infty E_i$ lives in exactly one of the $E_i$.
\end{proof}
\begin{ex}
	Fix a set $X$ with $\sigma$-ring $\mc S\coloneqq\mc P(X)$. Then we set $\mu(E)\coloneqq\#E$ for each $E\subseteq X$; namely, if $\mu(E)=\infty$ if and only if $E$ is infinite. We claim that $\mu$ is a measure: if $\{E_i\}_{i=1}^\infty$ is a countable collection of pairwise disjoint subsets of $X$, then it's a property of cardinality that the cardinality of the (disjoint) union is the sum of the cardinalities.
\end{ex}

\subsection{Premeasures}
We are going to want to build measures, but this is somewhat difficult. So we begin with something a little weaker. We begin by weakening our rings.
\begin{definition}[Prering]
	Fix a set $X$. A \textit{prering} of a set $X$ is a nonempty collection $\mc P\subseteq\mc P(X)$ satisfying the following.
	\begin{itemize}
		\item Intersection: if $E,F\in\mc P$, then $E\cap F=\mc P$.
		\item Decomposition: if $E,F\in\mc P$, then we can write
		\[E\setminus F=\bigsqcup_{i=1}^nG_i\]
		for some finite disjoint union on the right-hand side with $G_i\in\mc P$ for each $i$.
	\end{itemize}
\end{definition}
\begin{remark}
	Fix a prering $\mc P$. Note any $E\in\mc P$ has $E\setminus E=\emp$, so $\emp\in\mc P$ always because $\mc P$ is required to be nonempty.
\end{remark}
And now here are our weaker measures.
\begin{definition}[Premeasure]
	Fix a set $X$ and a prering $\mc P\subseteq\mc P(X)$. A \textit{premeasure} on $\mc P$ is a countably additive function $\mu\colon\mc P\to[0,\infty]$.
\end{definition}
It will turn out that premeasures on prering will give measures on the generated $\sigma$-ring. This is nicer because the countably additive condition might be easier to check on a prering, using ideas of \autoref{rem:avoid-checking-countably-additive}.

Here is our main example.
\begin{exe} \label{exe:leb-premeasure}
	Fix our set $X\coloneqq\RR$, and let $\mc P$ be the collection of half-open intervals $[a,b)$ where $a,b\in\RR$. Then $\mc P$ is a prering.
\end{exe}
\begin{proof}
	We begin by checking that $\mc P$ is a prering.
	\begin{itemize}
		\item Intersection: suppose that $[a,b),[a',b')\in\mc P$; without loss of generality, take $a\le a'$ so that $x\in[a,b)\cap[a',b')$ requires $a'\le x$. Now, note
		\begin{align*}
			[a,b)\cap[a',b') &= \{x\in\RR:a\le x\text{ and }a'\le x\text{ and }x<b\text{ and }x<b'\} \\
			&= [\max\{a,a'\},\min\{b,b'\}).
		\end{align*}
		\item Decomposition: suppose that $[a,b),[a',b')\in\mc P$.  Now, note
		\begin{align*}
			[a,b)\setminus[a',b') &= \{x\in\RR:a\le x\text{ and }x<b\text{ and }(a'>x\text{ or }x\ge b')\} \\
			&= \{x\in\RR:a\le x\text{ and }x<b\text{ and }x<a'\}\cup\{x\in\RR:a\le x\text{ and }x<b\text{ and }b'\le x\} \\
			&= [a,\min\{b,a'\})\cup[\max\{a,b'\},b).
		\end{align*}
	\end{itemize}
	The above checks complete the proof.
\end{proof}
Continuing from \autoref{exe:leb-premeasure}, it will turn out that the function $\mu\colon\mc P\to\RR$ given by
\[\mu([a,b))\coloneqq b-a\]
will give a premeasure, but we will not show this today. (We will say that one should use ideas of \autoref{exe:leb-premeasure}.) This is surprisingly annoying to prove.
\begin{example}
	Give $\QQ\cap[0,1)$ an enumeration $\{q_k\}_{k\in\NN}$. Then define the interval $F_k\coloneqq[q_k,q_{k+1})\cup[q_{k+1},q_k)$ and ``disjoint-ize'' these intervals by taking
	\[E_k\coloneqq F_k{\mathbin\bigg\backslash}\bigcup_{\ell=1}^{k-1}F_\ell\]
	and then decompose $E_k$ into a finite disjoint union of $G_k$s so that the $G_k$s are now disjoint. Any proof that $\mu$ is a premeasure must account for pathologies like this.
\end{example}

\end{document}