% !TEX root = ../notes.tex

\documentclass[../notes.tex]{subfiles}

\begin{document}

\section{October 31}

We continue our discussion of measurable functions by giving an alternate definition.

\subsection{A Better Measurable}
Last class, we saw that measurable functions have some nice properties. Today we show that these properties actually characterize our measurable functions.
\begin{theorem} \label{thm:better-measurable}
	Fix a normed vector space $B$ and a set $X$ with a $\sigma$-ring $\mc S$ on $X$. Then a function $f\colon X\to B$ is $\mc S$-measurable if and only if
	\begin{listroman}
		\item $\im f$ is separable, and
		\item for any open $U\subseteq B$, we have $f^{-1}(U\setminus\{0\})\in\mc S$.
	\end{listroman}
\end{theorem}
\begin{remark}
	Using ideas of \autoref{prop:checkonsubbase}, it suffices to check (ii) on a sub-base for the topology on $B$. In particular, it suffices to check (ii) on open balls.
\end{remark}
\begin{proof}
	Last class we provided the forward direction; namely, (i) follows from \autoref{cor:meas-has-sep-limit}, and (ii) follows from \autoref{cor:meas-has-meas-pre-image}. Today we show that (i) and (ii) imply that $f$ is the limit of simple $\mc S$-measurable functions. There are two steps.
	\begin{enumerate}
		\item We construct our simple $\mc S$-measurable functions $\{f_n\}_{n\in\NN}$. Because $\im f$ is separable by (ii), we may find some countable subset $\{b_i\}_{i=1}^\infty\subseteq\im f$ dense in $\im f$. Now, for each $i,j\in\NN$, define
		\[C_{ji}\coloneqq f^{-1}(B(b_i,1/j)\setminus\{0\}),\]
		which is always in $\mc S$ by (ii). Our goal is to carefully make the $C_{ji}$ disjoint in order to define our sequence $\{f_n\}_{n\in\NN}$ of simple $\mc S$-measurable functions, and we prefer $C_{ji}$ with $j$ large because these will give a finer approximation of $f$. In particular, we order $C_{ji}$ lexicographically by $(j,i)$: namely, $(j,i)<(\ell,k)$ if and only if $j<\ell$ or $j=\ell$ and $i<k$.
	
		We now fix $n$ and define our $f_n$. To make our $C_{ji}$ appropriately disjoint, we will focus on the $(j,i)$ bounded above by $(n,n)$. Namely, for $(j,i)\in\{1,2,\ldots,n\}^2$, we set
		\[E_{ji}^n\coloneqq C_{ji}{\mathbin\bigg\backslash}\bigcup_{\substack{(j,i)<(\ell,k)\\1\le\ell,k\le n}}C_{\ell k}.\]
		For example, $E_{nn}^n=C_{nn}$ and $E_{n,n-1}=C_{n,n-1}\setminus C_{n,n}$ and $E_{n,n-2}=C_{n,n-2}\setminus(C_{n,n}\cup C_{n,n-2})$ and so on.
	
		Notably, $E_{ji}^n\subseteq C_{ji}$ always, which means that the $E_{\bullet}^n$ are all disjoint: note $(j,i)\ne(j',i')$ implies that $(j,i)<(j',i')$ or $(j',i')<(j,i)$. Taking $(j,i)<(j',i')$ without loss of generality, we see that $E_{ji}^n\subseteq C_{ji}\setminus C_{j'i'}$ is disjoint from $E_{j'i'}^n\subseteq C_{j'i'}$.
		
		With this in mind, we define
		\[f_n\coloneqq\sum_{j=1}^n\sum_{i=1}^nb_i1_{E_{ji}^n}.\]
		Note that $\im f_n=\{0,b_1,\ldots,b_n\}$ because the $E_{ji}^n$ are disjoint, which we see is finite. Further, for any $b_i$, we can compute
		\[f_n^{-1}(\{b_i\})=\bigcup_{j=1}^nE_{ji}^n,\]
		which is in $\mc S$ because $\mc S$ is a ring. Thus, $f_n$ is in fact a simple $\mc S$-measurable function.
		
		\item It remains to check that $f_n\to f$ pointwise as $n\to\infty$. If $x\notin f^{-1}(B\setminus\{0\})$, then $f(x)=0$ while $x\notin C_{ji}$ always and so $x\notin E^n_{ji}$ always and so $f_n(x)=0$ for all $n$; so $f_n(x)\to f(x)$ follows with nothing to say in this case. Thus, we may assume $f(x)\ne0$.

		Now, take $\varepsilon>0$, and we need to find $N$ such that $n>N$ implies $|f(x)-f_n(x)|<\varepsilon$ for $n>N$. This has two steps: first, take some $j$ with $\frac1j<\varepsilon$, and second, we choose $i_0$ by density such that $f(x)\in B(b_{i_0},1/j)$.\footnote{Note that $B(f(x),1/j)\cap\{b_i:i\in\NN\}$ is nonempty because $\im f\subseteq\overline{\{b_i:i\in\NN\}}$; we are choosing $i_0$ with $b_{i_0}\in B(f(x),1/j)$, which means $f(x)\in B(b_{i_0},1/j)$} As such, set $N\coloneqq\max\{j,i_0\}+1$ so that $\frac1N<\frac1j<\varepsilon$ and $i_0<N$. Notably, $f(x)\in B(b_{i_0},1/h)\setminus\{0\}$ implies that
		\[x\in C_{j_0i_0}.\]
		We now begin our check. If $n>N$, then $x\in E^{n}_{\ell k}$, where
		\[(\ell,k)\coloneqq\max\{(j,i):x\in C_{ji}\text{ and }1\le j,i\le n\}.\]
		Namely, there is certainly some $(j,i)$ with $x\in C_{ji}$ and $1\le j,i\le n$ because $x\in C_{j_0,i_0}$ while $j_0,i_0<N<n$, so the maximum certainly exists. And we see $x\in E^n_{\ell k}$ because having $(j,i)>(\ell,k)$ with $1\le j,i\le n$ will imply that $x\notin C_{ji}$ by maximality of $(\ell,k)$.

		Now, $f_n(x)=b_k$ by construction, and $(j_0,i_0)\le(\ell,k)$ by maximality implies that $j_0\le\ell$ and so
		\[f(x)\in B(b_k,1/\ell)\subseteq B(b_k,1/j_0)\subseteq B(b_k,\varepsilon),\]
		so $|f(x)-f_n(x)|<\varepsilon$ follows.
	\end{enumerate}
	The above steps complete the proof.
\end{proof}
\begin{corollary} \label{cor:better-measurable-for-r}
	Fix a set $X$ with $\sigma$-ring $\mc S$. A function $f\colon X\to\RR$ is $\mc S$-measurable if and only if $f^{-1}(U\setminus\{0\})\in\mc S$ for each open $U\subseteq\RR$.
\end{corollary}
\begin{proof}
	If $f$ is $\mc S$-measurable, then this follows from \autoref{cor:meas-has-meas-pre-image}. Conversely, if $f^{-1}(U\setminus\{0\})\in\mc S$ for each open $U\subseteq\RR$, then we note $\im f\subseteq\RR$ is separable by \autoref{ex:all-sep-in-r}, so $f$ is $\mc S$-measurable by \autoref{thm:better-measurable}.
\end{proof}
\begin{corollary} \label{cor:compose-cont-is-meas}
	Fix a set $X$ with $\sigma$-ring $\mc S$. If $f\colon X\to\RR$ is $\mc S$-measurable, and $g\colon\RR\to\RR$ is continuous such that $g(0)=0$, then $g\circ f$ is still $\mc S$-measurable.
\end{corollary}
\begin{proof}
	For any open $U\subseteq\RR$, we note
	\[(g\circ f)^{-1}(U\setminus\{0\})=f^{-1}\left(g^{-1}(U\setminus\{0\})\right)=f^{-1}\left(g^{-1}(U)\setminus g^{-1}(\{0\})\right).\]
	Now, $g^{-1}(U)\subseteq\RR$ is open because $g\colon\RR\to\RR$ is continuous, and $0\in g^{-1}(\{0\})$ because $g(0)=0$, so $f^{-1}\left(g^{-1}(U)\setminus g^{-1}(\{0\})\right)\in\mc S$ by \autoref{cor:meas-has-meas-pre-image}. Thus, $g\circ f$ is $\mc S$-measurable by \autoref{cor:better-measurable-for-r}.
\end{proof}

\end{document}