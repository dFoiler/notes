% !TEX root = ../notes.tex

\documentclass[../notes.tex]{subfiles}

\begin{document}

\section{October 31}

Here we go.

\subsection{A Better Measurable}
Last class, we saw that measurable functions have some nice properties. Today we show that these properties actually characterize our measurable functions.
\begin{theorem}
	Fix a normed vector space $B$ and a set $X$ with a $\sigma$-ring $\mc S$ on $X$. Then a function $f\colon X\to B$ is $\mc S$-measurable if and only if
	\begin{listroman}
		\item there is some $E\in\mc S$ such that $\{x\in X:f(x)\ne0\}\subseteq E$,
		\item $\im f$ is separable, and
		\item for any open $U\subseteq B$, we have $f^{-1}(U)\cap E\in\mc S$.
	\end{listroman}
\end{theorem}
\begin{remark}
	Using ideas of \autoref{prop:checkonsubbase}, it suffices to check (iii) on a sub-base for the topology on $B$. In particular, it suffices to check (iii) on open balls.
\end{remark}
\begin{proof}
	Last class we provided the forward direction. Today we show that (i)--(iii) implies that $f$ is the limit of simple $\mc S$-measurable functions. There are two steps.
	\begin{enumerate}
		\item We construct our simple $\mc S$-measurable functions $\{f_n\}_{n\in\NN}$. Because $\im f$ is separable by (ii), we may find some countable subset $\{b_i\}_{i=1}^\infty\subseteq\im f$ dense in $\im f$. Now, for each $i,j\in\NN$, define
		\[C_{ji}\coloneqq f^{-1}(B(b_i,1/j))\cap E,\]
		which is always in $\mc S$ by (iii). Our goal is to carefully make the $C_{ji}$ disjoint in order to define our sequence $\{f_n\}_{n\in\NN}$ of simple $\mc S$-measurable functions, and we prefer $C_{ji}$ with $j$ large because these will give a finer approximation of $f$. In particular, we order $C_{ji}$ lexicographically by $(j,i)$: namely, $(j,i)<(\ell,k)$ if and only if $j<\ell$ or $j=\ell$ and $i<k$.
	
		We now fix $n$ and define our $f_n$. To make our $C_{ji}$ appropriately disjoint, we will focus on the $(j,i)$ bounded above by $(n,n)$. Namely, for $(j,i)\in\{1,2,\ldots,n\}^2$, we set
		\[E_{ji}^n\coloneqq C_{ji}{\mathbin\bigg\backslash}\bigcup_{(j,i)<(\ell,k)\le(n,n)}C_{\ell k}.\]
		For example, $E_{nn}^n=C_{nn}$ and $E_{n,n-1}=C_{n,n-1}\setminus C_{n,n}$ and $E_{n,n-2}=C_{n,n-2}\setminus(C_{n,n}\cup C_{n,n-2})$ and so on.
	
		Notably, $E_{ji}^n\subseteq C_{ji}$ always, which means that the $E_{\bullet}^n$ are all disjoint. As such, we define
		\[f_n\coloneqq\sum_{j=1}^n\sum_{i=1}^nb_i1_{E_{ji}^n}.\]
		By checking the pre-images of each $b_i$ for $i\in\{1,2,\ldots,n\}$, we see that $f_n$ is in fact a simple $\mc S$-measurable function.
		
		\item It remains to check that $f_n\to f$ pointwise as $n\to\infty$. If $x\notin E$, then $f(x)=0$ while $f_n(x)=0$ for all $n$, so there is nothing to say. Thus, we may assume that $x\in E$.

		Now, take $\varepsilon>0$, and we need to find $N$ such that $n>N$ implies $|f(x)-f_n(x)|<\varepsilon$ for $n>N$. To begin, take some $j$ with $\frac1j<\varepsilon$, and we choose $i_0$ by density such that $f(x)\in B(b_{i_0},1/j)$; now set $N\coloneqq\max\{j,i_0\}+1$ so that $\frac1N<\varepsilon$ and $i_0<N$. In particular, we have $x\in C_{ji}$.

		We now begin our check. If $n>N$, then $x\in E^{n}_{\ell k}$, where
		\[(\ell,k)\coloneqq\max\{(j,i):x\in C_{ji}\text{ and }(j_0,i_0)<(j,i)\le(n,n)\}.\]
		In particular, $f(x)=b_k$ while $f(x)\in B(b_k,1/\ell)\subseteq B(b_k,1/j_0)\subseteq B(b_k,\varepsilon)$, which completes the check.
	\end{enumerate}
	The above steps complete the proof.
\end{proof}

\end{document}