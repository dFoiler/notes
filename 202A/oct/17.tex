% !TEX root = ../notes.tex

\documentclass[../notes.tex]{subfiles}

\begin{document}

\section{October 17}

Please write neatly on the exam.

\subsection{Restricting Outer Measures}
Last time, in \autoref{ex:hered-measure-is-outer}, we constructed an outer measure from a premeasure. We might hope that this outer measure is actually countably additive, thus giving us our measure, but most of the time it is not.

Instead, we are going to restrict our outer measure to some ``$\sigma$-subring'' which will then be a measure. The following definition is due to Carath\'eodory.
\begin{definition}
	Fix a set $X$ and a hereditary $\sigma$-ring $\mc H$ on $X$, and fix an outer measure $\nu\colon\mc H\to[0,\infty]$. Then a set $E\subseteq\mc H$ is \textit{$\nu$-measurable} if and only if
	\[\nu(A)=\nu(A\cap E)+\nu(A\setminus E)\]
	for any $A\in\mc H$. We will let $\mc M(\nu)$ denote the set of $\nu$-measurable sets.
\end{definition}
\begin{remark} \label{rem:measurable-by-ineq}
	Because $\nu$ is already an outer measure, it is countably subadditive, so $\nu(A)\le\nu(A\cap E)+\nu(A\setminus E)$. Thus, we really only need to focus on proving
	\[\nu(A)\ge\nu(A\cap E)+\nu(A\setminus E).\]
\end{remark}
Here are the main results.
\begin{theorem} \label{thm:from-outer}
	Fix a set $X$ and a hereditary $\sigma$-ring $\mc H$ on $X$, and fix an outer measure $\nu\colon\mc H\to[0,\infty]$. If nonempty, $\mc M(\nu)$ is a $\sigma$-ring, and $\nu|_{\mc M(\nu)}$ is a measure.
\end{theorem}
\begin{remark}
	Later, we will also show that, give a premeasure $\mu$ on a prering $\mc P$, we will see $\mc P\subseteq\mc M(\mu^*)$, so $\mu^*|_{\mc S(\mc P)}$ will be a measure on $\mc S(\mc P)$ extending $\mu$. We won't be precise about this until we need to, but we do want to see that we are close to the finish line.
\end{remark}
\begin{remark}
	It is indeed possible for $\mc M(\nu)$ to be empty. For example, the outer measure $\nu\colon\mc P(X)\to[0,\infty]$ by $\nu(E)\coloneqq1$ for any $E\subseteq X$ has no $\nu$-measurable sets.
\end{remark}
Let's begin our proof.
\begin{proof}[Proof of \autoref{thm:from-outer}]
	We proceed in steps.
	\begin{enumerate}
		\item Finite union: given $E,F\in\mc M(\nu)$, we show $E\cup F\in\mc M(\nu)$. Well, for any $A\in\mc H$, we compute
		\begin{align*}
			\nu(A\cap(E\cup F))+\nu(A\setminus(E\cup F)) &= \nu((A\cap E)\sqcup((A\setminus E)\cap F))+\nu((A\setminus E)\setminus F) \\
			&\le \nu(A\cap E)+\nu((A\setminus E)\cap F)+\nu((A\setminus E)\setminus F),
		\end{align*}
		where we have used subadditivity at the end.

		Because $F$ is $\nu$-measurable, the last two pieces become $\nu(A\setminus E)$, where we note $A\setminus E\subseteq E\in\mc H$ implies $A\setminus E\in\mc H$. Thus, because $E$ is $\nu$-measurable, this in total collapses down to $\nu(A)$, which is enough by \autoref{rem:measurable-by-ineq}.

		\item Subtraction: given $E,F\in\mc M(\nu)$, we show $E\setminus F\in\mc M(\nu)$. Well, for any $A\in\mc H$, we compute
		\begin{align*}
			\nu(A\cap(E\setminus F))+\nu(A\setminus(E\setminus F)) &= \nu((A\cap E)\setminus F)+\nu((A\setminus E)\sqcup(A\cap E\cap F)) \\
			&\le \nu((A\cap E)\setminus F)+\nu(A\setminus E)+\nu((A\cap E)\cap F),
		\end{align*}
		where we have used subadditivity at the end.
		
		Now, because $F$ is $\nu$-measurable, we see $\nu((A\cap E)\setminus F)+\nu((A\cap E)\cap F)=\nu(A\cap E)$, where $A\cap E\subseteq E$ is in $\mc H$ because $E\in\mc H$. Thus, because $E$ is $\nu$-measurable, this in total collapses down to $\nu(A)$, which is enough by \autoref{rem:measurable-by-ineq}.

		\item Strong finitely additive: for any $A\in\mc H$ and disjoint $E,F\in\mc M(\nu)$, we claim
		\[\nu(A\cap(E\sqcup F))\stackrel?=\nu(A\cap E)+\nu(A\cap F).\]
		Well, because $E$ is measurable, we note $A\cap(E\sqcup F)\subseteq A$ must live in $\mc H$ and so
		\[\nu(A\cap(E\sqcup F))=\nu(A\cap(E\sqcup F)\cap E)+\nu(A\cap(E\sqcup F)\setminus E)=\nu(A\cap E)+\nu(A\cap F),\]
		where the last equality has used the fact that $E\cap F=\emp$.

		By induction, for finitely many pairwise disjoint $\nu$-measurable subsets $\{E_i\}_{i=1}^n\subseteq\mc M(\nu)$, we see
		\[\nu\Bigg(A\cap\bigsqcup_{i=1}^nE_i\Bigg)=\sum_{i=1}^n\nu(A\cap E_i).\]

		\item Finitely additive: we show $\nu|_{\mc M(\nu)}$ is finitely additive: for finitely many pairwise disjoint $\nu$-measurable subsets $\{E_i\}_{i=1}^nE_i\in\mc M(\nu)$, we note $A\coloneqq\bigcup_{i=1}^nE_i\in\mc M(\nu)$ because $\mc M(\nu)$ is a preserved by finite unions, so we set $A\coloneqq\bigcup_{i=1}^nE_i$ to give
		\[\nu\Bigg(\bigsqcup_{i=1}^nE_i\Bigg)=\nu\Bigg(A\cap\bigsqcup_{i=1}^nE_i\Bigg)=\sum_{i=1}^n\nu(A\cap E_i)=\sum_{i=1}^n\nu(E_i).\]

		\item Countable union: given some countable subcollection $\{E_i\}_{i=1}^\infty\subseteq\mc M(\nu)$, and let $F$ be their union. Now, we set
		\[F_i\coloneqq E_i\setminus\bigcup_{j<i}E_j\]
		so that $F$ is the union of the $F_i$ (certainly $F_i\subseteq E_i\subseteq F$ for any $i$, and conversely any $x\in F$ is in some $E_i$, for $i$ as small as possible, so $x\in F_i$), and the $F_i$ are pairwise disjoint (if $i\ne j$, then $i<j$ without loss of generality, so $F_i\subseteq E_i\setminus F_j$ is disjoint from $F_j$). The point is that we are now dealing with pairwise disjoint subsets.

		We now need to show that $F$ is $\nu$-measurable. Well, fix any $A\in\mc H$. Then, for any $n$, we note that $\mc M(\nu)$ is already a ring, so $\bigsqcup_{i=1}^nF_i$ is in $\mc M(\nu)$ for any finite $n$, so
		\[\nu(A)=\nu\Bigg(A\cap\bigsqcup_{i=1}^nF_i\bigg)+\nu\Bigg(A\setminus\bigsqcup_{i=1}^nF_i\Bigg)=\sum_{i=1}^n\nu(A\cap F_i)+\nu\Bigg(A\setminus\bigsqcup_{i=1}^nF_i\Bigg),\]
		where we have used finite additivity. Because $\nu$ is monotone, we may lower-bound this by
		\[\nu(A)\ge\sum_{i=1}^n\nu(A\cap F_i)+\nu(A\setminus F)\]
		for any $n$. Sending $n\to\infty$ now, we see
		\[\nu(A)\ge\sum_{i=1}^\infty\nu(A\cap F_i)+\nu(A\setminus F),\]
		but then countable subadditivity of $\nu$ kicks in and tells us that
		\[\nu(A)\ge\nu(A\cap F)+\nu(A\setminus F),\]
		so we are done by \autoref{rem:measurable-by-ineq}.

		\item Countably additive: we show $\nu|_{\mc M(\nu)}$ is countably additive. Well, given some countable pairwise disjoint collection of $\nu$-measurable sets $\{E_i\}_{i=1}^n\subseteq\mc M(\nu)$, we see that the previous step has told us
		\[\nu(A)\ge\sum_{i=1}^\infty\nu(A\cap E_i)+\nu\Bigg(A\setminus\bigsqcup_{i=1}^nE_i\Bigg)\ge\nu\Bigg(A\cap\bigsqcup_{i=1}^nE_i\Bigg)+\nu\Bigg(A\setminus\bigsqcup_{i=1}^nE_i\Bigg)\ge\nu(A).\]
		But $\mc M(\nu)$ is a $\sigma$-ring, so may set $A\coloneqq\bigsqcup_{i=1}^\infty E_i$ so that the above equalities actually read
		\[\nu(A)=\sum_{i=1}^n\nu(E_i),\]
		which is what we wanted.
		\qedhere
	\end{enumerate}
\end{proof}

\subsection{Completeness}
We have a notion of completeness for our measures; here is the definition.
\begin{definition}[Compelete]
	Fix a set $X$ and a family $\mc F\subseteq\mc P(X)$. Then a function $\nu\colon\mc F\to[0,\infty]$ is \textit{complete} if and only if any $A,E\in\mc F$ with $\nu(E)=0$ and $A\subseteq E$ must have $\nu(A)=0$.
\end{definition}
\begin{example}
	If $\nu$ is monotone, then $\nu$ is complete. Indeed, $A\subseteq E$ forces $0\le\mu(A)\le\mu(E)=0$ and so $\mu(A)=0$.
\end{example}
We will continue this after the midterm.

\end{document}