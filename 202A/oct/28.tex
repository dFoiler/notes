% !TEX root = ../notes.tex

\documentclass[../notes.tex]{subfiles}

\begin{document}

\section{October 28}

We continue our story with integration, defining what we mean by a measurable function.

\subsection{Measurable Functions}
The following definition is non-standard but is how to think about our integrals in practice.
\begin{definition}[Measurable function]
	Fix a set $X$ and a $\sigma$-ring $\mc S$ on $X$. Given a normed vector space $B$, an \textit{$\mathcal S$-measurable function} is a function $f\colon X\to B$ such that there is a sequence of simple $\mc S$-measurable functions $\{f_n\}_{n\in\NN}$ which converge to $f$ pointwise.
\end{definition}
\begin{remark}
	Later in life, when we take $B=\RR$, we will allow the functions $f_n$ to output at $\infty$, but we will not do so while we allow $B$ to be a general metric space or even a Banach space.
\end{remark}
Sometimes we won't converge ``on the nose,'' so we will want a little freedom.
\begin{definition}[Null set]
	Fix a set $X$ and a $\sigma$-ring $\mc S$ on $X$ equipped with a measure $\mu$. A \textit{null set} is a subset $N\in\mc S$ such that $\mu(N)=0$.
\end{definition}
\begin{definition}[Measurable function]
	Fix a set $X$ and a $\sigma$-ring $\mc S$ on $X$ equipped with a measure $\mu$. Given a metric space $B$, a \textit{$\mu$-measurable function} is a function $f\colon X\to B$ such that there is a sequence of simple $\mc S$-measurable functions $\{f_n\}_{n\in\NN}$ which converge to $f$ pointwise outside some fixed null set $N\in\mc S$.
\end{definition}

\subsection{Properties of Measurable Functions}
Something annoying about our definition is that we can only work simple $\mc S$-measurable functions ``directly.'' For example, it is quite unclear if a sequence of $\mc S$-measurable functions (not necessarily simple!) which converge to a function $f$ will in fact be $\mc S$-measurable.

For this, we will want a better description of $\mc S$-measurable functions.
\begin{definition}
	Fix a metric space $M$. A subset $A\subseteq M$ is \textit{separable} if and only if there is a countable subset $B\subseteq M$ such that $A\subseteq\overline B$.
\end{definition}
\begin{remark}
	It is equivalent to require a countable subset $B\subseteq A$ such that $A\subseteq\overline B$.
\end{remark}
\begin{lemma}
	Fix a normed vector space $B$ and a set $X$ with $\sigma$-ring $\mc S$ on $X$. Given an $\mathcal S$-measurable function $f$, then $\im f$ is separable.
\end{lemma}
\begin{proof}
	By definition, we find our simple $\mc S$-measurable functions $\{f_n\}_{n\in\NN}$ with $f_n\to f$ pointwise. Now, set
	\[E\coloneqq\bigcup_{n\in\NN}\im f_n,\]
	which is a countable union of finite sets and hence countable. For any $y\in\im f$, find the $x\in X$ such that $y=f(x)$. Now, we are told that $f_n(x)\to f(x)$ as $n\to\infty$, so $f(x)$ is a limit point of elements of $E$. Thus, noting that $\overline E$ is closed, we conclude from \autoref{lem:metricclosed} that $f(x)\in\overline E$. In particular, $\im f\subseteq\overline E$, so $E$ is separable.
\end{proof}
\begin{remark}
	In fact, the above proof can show that, if a sequence of functions $\{f_n\}_{n\in\NN}$ with $\im f_n$ separable and $f_n\to f$ pointwise, then $f$ also $\im f$ separable.
\end{remark}
\begin{lemma}
	Fix a normed vector space $B$ and a set $X$ with $\sigma$-ring $\mc S$ on $X$. Given an $\mathcal S$-measurable function $f$, then
	\[\{x\in X:f(x)\ne0\}\]
	is contained in a subset in $\mc S$.
\end{lemma}
\begin{proof}
	Take the union of the individual simple $\mc S$-measurable functions.
\end{proof}
\begin{remark}
	In fact, the above proof can show that, if a sequence of functions $\{f_n\}_{n\in\NN}$ with $\im f_n$ separable, and $\{x\in X:f_n(x)\}$ is contained in a subset of $\mc S$ for each $n$, then the same is true for $f$.
\end{remark}
More generally, we have the following.
\begin{lemma}
	Fix a simple $\mc S$-measurable function $f$. Given an open subset $U\subseteq B$, we have $f^{-1}(U)\cap E\in\mc S$, where $E\in\mc S$ is any set containing the points where $f$ is nonzero.
\end{lemma}
\begin{proof}
	As usual, pick up the promised sequence $\{f_n\}_{n\in\NN}$ of simple $\mc S$-measurable functions. Now, notice that $x\in f^{-1}(U)$ if and only if $f(x)\in U$, so we want to understand $U$ a little better. Without loss of generality, we may assume that $0\notin U$ because we are intersecting with $E$.

	We would like to say that $f(x)$ is just a little far from $\overline U$. For this, we set
	\[U_m\coloneqq\{x\in U:d(x,B\setminus U)<1/m\}\]
	for any $m\ge1$. Notably, we have that $U$ is the union of all the $U_m$, and $\overline{U_m}\subseteq U_{m+1}$ always.

	Continuing, we now see that $f(x)\in U$ is equivalent to $f(x)\in U_n$ for some $n$. Next, this is equivalent to saying that $f_k(x)\in U_n$ for all sufficiently large $k$ and some $n$. Certainly if $f(x)\in U_n$, then we have $f_k(x)\in U_n$ for sufficiently large $k$, and conversely, if $f_k(x)\in U_n$ for sufficiently large $k$, then $f(x)\in\overline{U_n}\subseteq U_{n+1}$.

	Thus, in total we see that
	\[f^{-1}(U)=\bigcup_{n=1}^\infty\bigcup_{K=1}^\infty\bigcap_{k\ge K}f_k^{-1}(U_n).\]
	However, $f_k^{-1}(U_n)\in\mc S$ by definition, so this full union of unions of intersections is still in $\mc S$.
\end{proof}
\begin{remark}
	As usual, we have bought more with the above proof: fix a sequence of functions $\{f_n\}_{n\in\NN}$ such that $f_n^{-1}(U)\in\mc S$ for each open $U\subseteq B\setminus\{0\}$. If $f_n\to f$ pointwise, then $f^{-1}(U)$ as well.
\end{remark}
\begin{remark}
	It follows from our remarks that a sequence of $\mc S$-measurable functions converging to a function $f$ will satisfy the conclusions of the previous three lemmas.
\end{remark}
\begin{remark}
	Note that the case of $B=\RR$, we see that $f^{-1}(U)$ is measurable for any open $U\subseteq\RR$, where $f$ is an $\mathcal S$-measurable function. By taking unions and complements appropriately, we in fact see that $f^{-1}(U)$ is measurable for any Borel set $U\subseteq\RR$. This is the usual definition of a measurable function, and we will show it next class.
\end{remark}

\end{document}