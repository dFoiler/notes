% !TEX root = ../notes.tex

\documentclass[../notes.tex]{subfiles}

\begin{document}

\section{October 28}

We continue our story with integration by defining what we mean by a measurable function.

\subsection{Measurable Functions}
The following definition is non-standard but is how to think about our integrals in practice.
\begin{definition}[Measurable function]
	Fix a set $X$ and a $\sigma$-ring $\mc S$ on $X$. Given a normed vector space $B$, an \textit{$\mathcal S$-measurable function} is a function $f\colon X\to B$ such that there is a sequence of simple $\mc S$-measurable functions $\{f_n\}_{n\in\NN}$ which converge to $f$ pointwise.
\end{definition}
\begin{remark}
	Later in life, when we take $B=\RR$, we will allow the functions $f_n$ to output at $\infty$, but we will not do so while we allow $B$ to be a normed vector space.
\end{remark}
Sometimes we won't converge ``on the nose,'' so we will want a little freedom.
\begin{definition}[Null set]
	Fix a set $X$ and a $\sigma$-ring $\mc S$ on $X$ equipped with a measure $\mu$. A \textit{null set} is a subset $N\subseteq X$ such that there is some $E\in\mc S$ such that $N\subseteq E$ and $\mu(N)=0$.
\end{definition}
\begin{definition}[Almost everywhere]
	Fix a set $X$ and a $\sigma$-ring $\mc S$ on $X$ equipped with a measure $\mu$. A property $P(x)$ for points $x\in X$ holds \textit{almost everywhere} if and only if $\{x\in X:\lnot P(x)\}$ is a null set.
\end{definition}
\begin{definition}[Converges almost everywhere]
	Fix a set $X$ and a $\sigma$-ring $\mc S$ on $X$ equipped with a measure $\mu$. Given a metric space $B$, a sequence of functions $f_n\colon X\to B$ with $n\in\NN$ converges to a function $f\colon X\to B$ \textit{almost everywhere} if and only if $f_n(x)\to f(x)$ almost everywhere.
\end{definition}
\begin{definition}[Measurable function]
	Fix a set $X$ and a $\sigma$-ring $\mc S$ on $X$ equipped with a measure $\mu$. Given a metric space $B$, a \textit{$\mu$-measurable function} is a function $f\colon X\to B$ such that there is a sequence of simple $\mc S$-measurable functions $\{f_n\}_{n\in\NN}$ which converge to $f$ almost everywhere.
\end{definition}
Here is the usual sanity check.
\begin{lemma} \label{lem:meas-is-vec-space}
	Fix a normed $k$-vector space $B$ and a set $X$ with a $\sigma$-ring $\mc S$ on $X$. Then the set of all $\mc S$-measurable functions forms a $k$-vector space under pointwise operations.
\end{lemma}
\begin{proof}
	We already know that the set of all functions $X\to B$ will form a $k$-vector space under the pointwise operations, so we just need to show that we have a subspace. Well, pick up $\mc S$-measurable functions $f$ and $g$ and some scalars $a,b\in k$. We show that $h\coloneqq af+bg$ is still $\mc S$-measurable.

	Well, $f$ being $\mc S$-measurable promises simple $\mc S$-measurable functions $\{f_n\}_{n\in\NN}$ with $f_n\to f$ pointwise; similarly, we get simple $\mc S$-measurable functions $\{g_n\}_{n\in\NN}$ with $g_n\to g$ pointwise. Now, we define
	\[h_n\coloneqq af_n+bg_n,\]
	which is a simple $\mc S$-measurable function by \autoref{lem:simple-meas-is-k-vec}.

	It remains to check that $h_n\to h$ as $n\to\infty$. Let $|\cdot|$ be the norm on $k$, and let $\norm\cdot$ be the norm on $B$, and fix some $x\in X$. Now, for any $\varepsilon>0$, find $N_f>0$ such that
	\[n>N_f\implies\norm{f(x)-f_n(x)}<\frac\varepsilon{2(\norm a+1)},\]
	where we note that $|a|+1>0$ makes this division legal. Similarly, we find $N_g>0$ such that
	\[n>N_f\implies\norm{f(x)-f_n(x)}<\frac\varepsilon{2(\norm b+1)}.\]
	Thus, $n>\max\{N_f,N_g\}$ will have
	\[\norm{h(x)-h_n(x)}\le |a|\cdot\norm{f(x)-f_n(x)}+|b|\cdot\norm{g(x)-g_n(x)}<|a|\cdot\frac\varepsilon{2(\norm a+1)}+|b|\cdot\frac\varepsilon{2(\norm b+1)}<\varepsilon,\]
	which finishes.
\end{proof}
\begin{lemma} \label{lem:prod-of-meas}
	Fix a ring $\mc S$ on a set $X$. Given two $\mc S$-measurable functions $f,g\colon X\to\RR$, the function $fg$ is $\mc S$-measurable.
\end{lemma}
\begin{proof}
	We are given sequences of simple $\mc S$-measurable functions $\{f_n\}_{n\in\NN}$ and $\{g_n\}_{n\in\NN}$ such that $f_n\to f$ and $g_n\to g$ pointwise. Thus, for each $x\in X$, we see $(f_ng_n)(x)\to(fg)(x)$ by taking products of limits, so we conclude $f_ng_n\to fg$ pointwise. However, $f_ng_n$ is simple $\mc S$-measurable by \autoref{lem:prod-of-simp-meas}.
\end{proof}

\subsection{Properties of Simple Measurable Functions}
Something annoying about our definition is that we can only work simple $\mc S$-measurable functions ``directly.'' One might wonder, for example, if a looking at limits $f_n\to f$ as $n\to\infty$ where each $f_n$ is $\mc S$-measurable might give a function $f$ which is not $\mc S$-measurable. This turns out to not be the case, but it will take some work to prove.

In particular, we will want a better description of $\mc S$-measurable functions. For today, we will content ourselves with necessary conditions.
\begin{definition}[Separable]
	A topological space $M$ is \textit{separable} if and only if there is a countable dense subset of $M$. As such, a subset $A\subseteq M$ is \textit{separable} if and only if $A$ is separable with the restricted metric; in other words, $A\subseteq M$ is separable if and only if there is a countable subset $B\subseteq A$ such that $A\subseteq\overline B$.
\end{definition}
\begin{example} \label{ex:countable-is-sep}
	If $A\subseteq M$ is countable, then we can see that $A\subseteq\overline A$ by definition of the closure, so $A$ is a countable dense subset with $A\subseteq\overline A$.
\end{example}
Here is a quick sanity check.
\begin{lemma} \label{lem:better-sep}
	Fix a metric space $(M,d)$. A subset $A\subseteq M$ is separable if and only if there is a countable subset $B\subseteq M$ such that $A\subseteq\overline B$.
\end{lemma}
\begin{proof}
	In the forward direction, having a countable subset $B\subseteq A$ with $A\subseteq\overline B$ will certainly imply having a countable subset $B\subseteq M$ with $A\subseteq\overline B$.

	In the reverse direction, we begin with a countable subset $B\subseteq M$ with $A\subseteq\overline B$. For now, fix some $\varepsilon>0$. Then each $a\in A$ has $B(a,\varepsilon/2)\cap B\ne\emp$ by \autoref{lem:denseformetricspaces}, so choose some $b_a\in B$ with $d(a,b_{\varepsilon,a})<\varepsilon/2$. Now, the subset
	\[B_\varepsilon\coloneqq\{b_{\varepsilon,a}:a\in A\}\subseteq B\]
	must be countable, so enumerate its elements by $B_\varepsilon=\{b_{\varepsilon,1},b_{\varepsilon,2},\ldots\}$, and for each $b_{\varepsilon,k}$, we select some $a_{\varepsilon,k}\in A$ such that $d(b_{\varepsilon,k},a_{\varepsilon,k})<\varepsilon/2$, which exists by construction of $B_\varepsilon$.

	We now go back to letting $\varepsilon>0$ vary. As our countable subset, we now set
	\[B'\coloneqq\bigcup_{n=1}^\infty\{a_{1/n,k}:k\in\ZZ_{>0}\}.\]
	Indeed, we claim that $A\subseteq\overline {B'}$, which shows density by \autoref{lem:denseformetricspaces}. For this, we pick up any $a\in A$ and $\varepsilon>0$, and we show that $B(a,\varepsilon)\cap B'\ne\emp$. Well, find some $N$ with $N>1/\varepsilon$. By construction of $B_{1/N}$, we may find some $k$ with $b_{1/N,k}=b_{1/N,a}$, which means that
	\[d(a,a_{1/N,k})\le d_(a,b_{1/N,k})+d(b_{1/N,k},a_{1/N,k})<\frac1{2N}+\frac1{2N}<\varepsilon.\]
	Thus, $a_{1/N,k}\in B'$ is the element we are looking for.
\end{proof}
\begin{example} \label{ex:all-sep-in-r}
	We give $\RR$ the usual metric. Then any subset $A\subseteq\RR$ is separable: set $B\coloneqq\QQ$. Then $\overline B=\RR$ contains $A$, but $B$ is countable, so we are done by \autoref{lem:better-sep}.
\end{example}
\begin{remark} \label{rem:subspace-of-sep-is-sep}
	To help our intuition that this should be a smallness condition, we note that if $M$ is a separable space, then any subspace $A\subseteq M$ is still separable. Indeed, there is some countable subset $B\subseteq M$ with $\overline B=M$, so $A\subseteq\overline M$ follows.
\end{remark}
\begin{example} \label{ex:union-of-seps-is-sep}
	Given countably many separable subsets $\{A_n\}_{n\in\NN}$ of a metric space $B$, the union $A\coloneqq\bigcup_{n\in\NN}A_n$ is separable. Indeed, each $A_n$ has a countable $B_n\subseteq M$ with $A_n\subseteq\overline{B_n}$.

	Now, set $B\coloneqq\bigcup_{n\in\NN}B_n$, which is countable; we claim that $A\subseteq\overline B$, which will finish. Because $B_n\subseteq B\subseteq\overline B$, we see that $\overline B$ is a closed subset containing $B_n$, so $A_n\subseteq\overline{B_n}\subseteq\overline B$ follows. Thus, $A\subseteq\overline B$.
\end{example}
Here is why we just defined separable subsets.
\begin{lemma} \label{lem:simple-has-sep-image}
	Fix a normed vector space $B$ and a set $X$ with $\sigma$-ring $\mc S$ on $X$. Any simple $\mc S$-measurable function $f\colon X\to B$ has $\im f\subseteq B$ separable.
\end{lemma}
\begin{proof}
	By definition of simple measurable functions, $\im f$ is finite and hence separable by \autoref{ex:countable-is-sep}.
\end{proof}
Here is a last moderately silly checks.
\begin{lemma} \label{lem:simple-measurable-pre-image}
	Fix a normed vector space $B$ and a set $X$ with $\sigma$-ring $\mc S$ on $X$. Any simple $\mc S$-measurable function $f\colon X\to B$ has $f^{-1}(U\setminus\{0\})\in\mc S$ for any open $U\subseteq X$.
\end{lemma}
\begin{proof}
	Recall that $\im f$ is finite by definition, so enumerate $\im f$ by $(\im f)\cap(U\setminus\{0\})=\{y_1,\ldots,y_n\}$. Then we note that
	\[f^{-1}(U\setminus\{0\})=f^{-1}((\im f)\cap(U\setminus\{0\}))=\bigcup_{k=1}^nf^{-1}(\{y_k\}).\]
	However, $f^{-1}(\{y_k\})\in\mc S$ for each $k$, so the total union lives in $\mc S$ because $\mc S$ is a ring.
\end{proof}

\subsection{Properties Preserved by Limits}
Now, to upgrade from simple $\mc S$-measurable functions to $\mc S$-measurable functions, we take limits. Here is the separability check.
\begin{lemma} \label{lem:limit-of-sep-ims-has-sep-im}
	Fix a metric space $M$ and a set $X$. Suppose a sequence of functions $f_n\colon X\to B$ for $n\in\NN$ have a pointwise limit $f_n\to f$ as $n\to\infty$. If each $\im f_n\subseteq M$ is separable, then $\im f\subseteq M$ is separable as well.
\end{lemma}
\begin{proof}
	For each $n$, find the countable subset $C_n\subseteq\im f_n$ with $\im f_n\subseteq\overline{C_n}$. Then we set
	\[C\coloneqq\bigcup_{n\in\NN}C_n,\]
	and we note that $C$ is the countable union of countable subsets and hence countable. We thus claim that $\im f\subseteq\overline C$, which will finish by \autoref{lem:better-sep}.

	Well, fix any $y\in\im f$, and find some $x\in X$ with $y=f(x)$. For any $\varepsilon>0$, we need to show that $B(y,\varepsilon)\cap C\ne\emp$; this is enough by \autoref{lem:denseformetricspaces}. For this, we note that there is some $N$ such that $n\ge N$ implies
	\[d(f(x),f_n(x))<\varepsilon/2,\]
	where $d$ is the metric of $M$; set $n\coloneqq N$. Further, we recall $\im f_n\subseteq\overline{C_n}$, so \autoref{lem:denseformetricspaces} promises us some $c\in C_n$ such that $d(f_n(x),c)<\varepsilon/2$. In total,
	\[d(y,c)\le d(f(x),f_n(x))+d(f_n(x),c)<\frac\varepsilon2+\frac\varepsilon2=\varepsilon,\]
	which is what we wanted.
\end{proof}
\begin{corollary} \label{cor:meas-has-sep-limit}
	Fix a normed vector space $B$ and a set $X$ with $\sigma$-ring $\mc S$ on $X$. Any $\mc S$-measurable function $f\colon X\to B$ has $\im f\subseteq B$ separable.
\end{corollary}
\begin{proof}
	By definition of being $\mc S$-measurable, there is a sequence of simple $\mc S$-measurable functions $f_n\colon X\to B$ with $f_n\to f$ as $n\to\infty$ pointwise. Each $f_n$ has $\im f_n$ separable by \autoref{lem:simple-has-sep-image}, so $f$ does as well by \autoref{lem:limit-of-sep-ims-has-sep-im}.
\end{proof}
Making \autoref{lem:simple-measurable-pre-image} work in limits requires a little more care.
% \begin{lemma} \label{lem:limit-measurable-pre-image}
% 	Fix a normed vector space $B$ and a set $X$ with $\sigma$-ring $\mc S$ on $X$. Suppose that a sequence of functions $f_n\colon X\to B$ satisfy the following conditions for each $f_n$.
% 	\begin{listroman}
% 		\item There is a subset $E\in\mc S$ containing $f_n^{-1}(B\setminus\{0\})$.
% 		\item For any $E\in\mc S$ containing $f_n^{-1}(B\setminus\{0\})$ and open $U\subseteq B$, we have $f_n^{-1}(U)\cap E\in\mc S$.
% 	\end{listroman}
% 	If $f_n\to f$ pointwise as $n\to\infty$, then $f$ satisfies (i) and (ii) above as well.
% \end{lemma}
\begin{lemma} \label{lem:limit-measurable-pre-image}
	Fix a normed vector space $B$ and a set $X$ with $\sigma$-ring $\mc S$ on $X$. Suppose that a sequence of functions $f_n\colon X\to B$ have $f_n^{-1}(U\setminus\{0\})\in\mc S$ for each open $U\subseteq B$. Then satisfy the following conditions for each $f_n$. If $f_n\to f$ pointwise as $n\to\infty$, then $f^{-1}(U\setminus\{0\})\in\mc S$ for each open $U\subseteq B$ as well.
\end{lemma}
\begin{proof}
	This is a little tricky. We will replace $U$ with $U\setminus\{0\}$ and simply remember that $0\notin U$.
	
	The main point is that any $x\in f^{-1}(U)$ will have $f(x)\in U$, and elements of $U$ should have some small positive distance away from $B\setminus U$. Namely, we set
	\[U_m\coloneqq\{x\in U:d(x,B\setminus U)>1/m\}\]
	for any $m\ge1$; here, $d$ is the metric of $B$, and $d(x,B\setminus U)=\inf_{y\in B\setminus U}d(x,y)$. Here are a few checks.
	\begin{itemize}
		\item As an intermediate claim, we note that $d(x,y)+d(y,B\setminus U)\ge d(x,B\setminus U)$. Indeed, for any $a\in B\setminus U$, note that
		\[d(x,y)+d(y,a)\ge d(b,a)\ge d(x,B\setminus U),\]
		so
		\[d(y,a)\ge d(x,B\setminus U)-d(x,y).\]
		Letting $a\in B\setminus U$ vary in this last inequality tells us that $d(y,B\setminus U)\ge d(x,B\setminus U)-d(x,y)$.
		\item Note that each $U_m$ is open. Indeed, if $x\in U_m$, then set $\varepsilon\coloneqq 1/m-d(x,B\setminus U)>0$. This means that $d(x,y)<\varepsilon$ implies, from the previous check,
		\[d(y,B\setminus U)\le d(x,y)+d(x,B\setminus U)<\varepsilon+d(x,B\setminus U)=1/m,\]
		so $y\in U_m$. Thus, $B(x,\varepsilon)\subseteq U$.
		\item We claim
		\[U\stackrel?=\bigcup_{m=1}^\infty U_m.\]
		In one direction, $x\in U_m$ implies $d(x,B\setminus U)\ne0$, so $x\notin B\setminus U$, so $x\in U$.
		
		In the other direction, note $x\in U$ implies there is some $\varepsilon>0$ with $B(x,\varepsilon)\subseteq U$, so $B(x,\varepsilon)\cap(B\setminus U)=\emp$, so $d(x,B\setminus U)>\varepsilon$. Thus, there is some $m>1/\varepsilon$ with $d(x,B\setminus U)>1/m$ and so $x\in U_m$ for this $m$.
		\item From the above claim, we note that $0\notin U_m$ for each $m$ because $0\notin U$.
		\item We claim $\overline{U_m}\subseteq U_{m+1}$ for each $m$; set $\varepsilon\coloneqq\frac1m-\frac1{m+1}>0$. Now, if $x\in\overline{U_m}$, then we see $B(x,\varepsilon)\cap U_m\ne\emp$ by \autoref{lem:denseformetricspaces}, so we may find $y\in U_m$ with $d(x,y)<\varepsilon$. It follows from the first check that
		\[d(x,B\setminus U)\le d(x,y)+d(y,B\setminus U)<\varepsilon+\frac1m=\frac1{m+1},\]
		so $x\in U_{m+1}$ follows.
	\end{itemize}
	Now, we see from the above that
	\[f^{-1}(U)=\bigcup_{m=1}^\infty f^{-1}(U_m).\]
	Thus, $x\in f^{-1}(U_m)$ implies that there is some $\varepsilon>0$ with $B(x,\varepsilon)\subseteq U_m$; because $f_n\to f$ as $n\to\infty$ pointwise (!), there is some $N$ for which $f_n(x)\in B(x,\varepsilon)\subseteq U_m$ for each $n\ge N$, so
	\[f^{-1}(U)\subseteq\bigcup_{m=1}^\infty\bigcup_{N=1}^\infty\bigcap_{n\ge N}f_n^{-1}(U_m).\]
	Conversely, if $x$ lives in this right-hand set, we have some $m$ and $N$ with $f_n(x)\in U_m\subseteq\overline{U_m}$ for all $n\ge N$. So $f(x)\in\overline{U_m}$ by \autoref{lem:metricclosed}, so $f(x)\in U_{m+1}\subseteq U$ follows. Thus, equality in the above containment follows.

	In total, we see that
	\[f^{-1}(U)=\bigcup_{m=1}^\infty\bigcup_{N=1}^\infty\bigcap_{n\ge N}f_n^{-1}(U_m).\]
	Notably, $f_n^{-1}(U_m)\cap E\in\mc S$ for each $n$ and $m$, by construction of the $f_n$s, so this full union of unions of intersections is still in $\mc S$, using the fact that $\mc S$ is a $\sigma$-ring and \autoref{rem:sring-has-intersections}.
\end{proof}
% \begin{proof}
% 	We show (i) and (ii) separately. For each $n$, let $E_n\in\mc S$ be the subset containing $f^{-1}(B\setminus\{0\})$.
% 	\begin{listroman}
% 		\item We set
% 		\[E\coloneqq\bigcup_{n\in\NN}E_n,\]
% 		and we note that $E\in\mc S$ because $\mc S$ is a $\sigma$-ring. As such, we claim that $E$ contains $f^{-1}(B\setminus\{0\})$, which will finish.

% 		Well, if $x\in f^{-1}(B\setminus\{0\})$, then $f(x)\ne0$. Because $f_n(x)\to f(x)$ as $n\to\infty$, there is some $N$ for which $n\ge N$ implies $|f(x)-f_n(x)|<|f(x)|$ and so $f_n(x)\ne0$. Thus, $x\in f_n^{-1}(B\setminus\{0\})\subseteq E_n$, so the claim follows.

% 		\item This is a little tricky. The main point is that any $x\in f^{-1}(U)$ will have $f(x)\in U$, and elements of $U$ should have some small positive distance away from $B\setminus U$. Namely, we set
% 		\[U_m\coloneqq\{x\in U:d(x,B\setminus U)>1/m\}\]
% 		for any $m\ge1$; here, $d$ is the metric of $B$, and $d(x,B\setminus U)=\inf_{y\in B\setminus U}d(x,y)$. Here are a few checks.
% 		\begin{itemize}
% 			\item As an intermediate claim, we note that $d(x,y)+d(y,B\setminus U)\ge d(x,B\setminus U)$. Indeed, for any $a\in B\setminus U$, note that
% 			\[d(x,y)+d(y,a)\ge d(b,a)\ge d(x,B\setminus U),\]
% 			so
% 			\[d(y,a)\ge d(x,B\setminus U)-d(x,y).\]
% 			Letting $a\in B\setminus U$ vary in this last inequality tells us that $d(y,B\setminus U)\ge d(x,B\setminus U)-d(x,y)$.
% 			\item Note that each $U_m$ is open. Indeed, if $x\in U_m$, then set $\varepsilon\coloneqq 1/m-d(x,B\setminus U)>0$. This means that $d(x,y)<\varepsilon$ implies, from the previous check,
% 			\[d(y,B\setminus U)\le d(x,y)+d(x,B\setminus U)<\varepsilon+d(x,B\setminus U)=1/m,\]
% 			so $y\in U_m$. Thus, $B(x,\varepsilon)\subseteq U$.
% 			\item We claim
% 			\[U\stackrel?=\bigcup_{m=1}^\infty U_m.\]
% 			In one direction, $x\in U_m$ implies $d(x,B\setminus U)\ne0$, so $x\notin B\setminus U$, so $x\in U$.
			
% 			In the other direction, note $x\in U$ implies there is some $\varepsilon>0$ with $B(x,\varepsilon)\subseteq U$, so $B(x,\varepsilon)\cap(B\setminus U)=\emp$, so $d(x,B\setminus U)>\varepsilon$. Thus, there is some $m>1/\varepsilon$ with $d(x,B\setminus U)>1/m$ and so $x\in U_m$ for this $m$.
% 			\item We claim $\overline{U_m}\subseteq U_{m+1}$ for each $m$; set $\varepsilon\coloneqq\frac1m-\frac1{m+1}>0$. Now, if $x\in\overline{U_m}$, then we see $B(x,\varepsilon)\cap U_m\ne\emp$ by \autoref{lem:denseformetricspaces}, so we may find $y\in U_m$ with $d(x,y)<\varepsilon$. It follows from the first check that
% 			\[d(x,B\setminus U)\le d(x,y)+d(y,B\setminus U)<\varepsilon+\frac1m=\frac1{m+1},\]
% 			so $x\in U_{m+1}$ follows.
% 		\end{itemize}
% 		Now, we see from the above that
% 		\[f^{-1}(U)=\bigcup_{m=1}^\infty f^{-1}(U_m).\]
% 		Now, $x\in f^{-1}(U_m)$ implies that there is some $\varepsilon>0$ with $B(x,\varepsilon)\subseteq U_m$; because $f_n\to f$ as $n\to\infty$ pointwise (!), there is some $N$ for which $f_n(x)\in B(x,\varepsilon)\subseteq U_m$ for each $n\ge N$, so
% 		\[f^{-1}(U)\subseteq\bigcup_{m=1}^\infty\bigcup_{N=1}^\infty\bigcap_{n\ge N}f_n^{-1}(U_m).\]
% 		Conversely, if $x$ lives in this right-hand set, we have some $m$ and $N$ with $f_n(x)\in U_m\subseteq\overline{U_m}$ for all $n\ge N$. So $f(x)\in\overline{U_m}$ by \autoref{lem:metricclosed}, so $f(x)\in U_{m+1}\subseteq U$ follows. Thus, equality in the above containment follows.

% 		Thus, in total we see that
% 		\[f^{-1}(U)=\bigcup_{m=1}^\infty\bigcup_{N=1}^\infty\bigcap_{n\ge N}f_n^{-1}(U_m).\]
% 		Only now do we pick up the promised $E$ from our hypothesis and intersect as
% 		\[f^{-1}(U)\cap E=\bigcup_{m=1}^\infty\bigcup_{N=1}^\infty\bigcap_{n\ge N}\left(f_n^{-1}(U_m)\cap E\right).\]
% 		Notably, $f_n^{-1}(U_m)\cap E\in\mc S$ for each $n$ and $m$, by construction of the $f_n$s, so this full union of unions of intersections is still in $\mc S$, using the fact that $\mc S$ is a $\sigma$-ring and \autoref{rem:sring-has-intersections}.
% 		\qedhere
% 	\end{listroman}
% \end{proof}
% \begin{remark}
% 	The above conditions are forced to be so weird because it need not be true that $f^{-1}(B\setminus\{0\})\in\mc S$ even if $f_n^{-1}(B\setminus\{0\})\in\mc S$ for each $n$. The above proof gets around this by merely showing
% 	\[f^{-1}(B\setminus\{0\})\subseteq\bigcup_{n\in\NN}f_n^{-1}(B\setminus\{0\}).\]
% \end{remark}
% \begin{remark}
% 	For peace of mind, we can equivalently replace (ii) in \autoref{lem:limit-measurable-pre-image} with
% 	\begin{itemize}
% 		\item[(ii')] There is some $E\in\mc S$ containing $f_n^{-1}(B\setminus\{0\})$ such that any open $U\subseteq B$ has $f_n^{-1}(U)\cap E\in\mc S$.
% 	\end{itemize}
% 	By (i), certainly (ii) implies (ii') by choosing any available $E$. Conversely, given (ii'), let $E\in\mc S$ be the special subset. Then any $E'\in\mc S$ containing $f_n^{-1}(B\setminus\{0\})$ will have
% 	\[f_n^{-1}(U)\cap E'=\left(f_n^{-1}(U)\cap E\right)\cup\left(f_n^{-1}(\{0\})\cap (E'\setminus E)\right),\]
% 	so it remains to show $f_n^{-1}(\{0\})\cap E'\in\mc S$.
% \end{remark}
\begin{corollary} \label{cor:meas-has-meas-pre-image}
	Fix a normed vector space $B$ and a set $X$ with $\sigma$-ring $\mc S$ on $X$. Any $\mc S$-measurable function $f\colon X\to B$ has $f^{-1}(U\setminus\{0\})\in\mc S$ for each open $U\subseteq B$.
\end{corollary}
\begin{proof}
	Any simple $\mc S$-measurable function satisfies the conclusion by \autoref{lem:simple-measurable-pre-image}. However, because $\mc S$-measurable functions are limits of simple $\mc S$-measurable functions, $\mc S$-measurable functions satisfy the conclusion as well by \autoref{lem:limit-measurable-pre-image}.
\end{proof}
% \begin{proof}
% 	By definition, we find our simple $\mc S$-measurable functions $\{f_n\}_{n\in\NN}$ with $f_n\to f$ pointwise. Now, set
% 	\[E\coloneqq\bigcup_{n\in\NN}\im f_n,\]
% 	which is a countable union of finite sets and hence countable. For any $y\in\im f$, find the $x\in X$ such that $y=f(x)$. Now, we are told that $f_n(x)\to f(x)$ as $n\to\infty$, so $f(x)$ is a limit point of elements of $E$. Thus, noting that $\overline E$ is closed, we conclude from \autoref{lem:metricclosed} that $f(x)\in\overline E$. In particular, $\im f\subseteq\overline E$, so $E$ is separable.
% \end{proof}
% \begin{remark}
% 	In fact, the above proof can show that, if a sequence of functions $\{f_n\}_{n\in\NN}$ with $\im f_n$ separable and $f_n\to f$ pointwise, then $f$ also $\im f$ separable.
% \end{remark}
% \begin{lemma}
% 	Fix a normed vector space $B$ and a set $X$ with $\sigma$-ring $\mc S$ on $X$. Given an $\mathcal S$-measurable function $f$, then
% 	\[\{x\in X:f(x)\ne0\}\]
% 	is contained in a subset in $\mc S$.
% \end{lemma}
% \begin{proof}
% 	Take the union of the individual simple $\mc S$-measurable functions.
% \end{proof}
% \begin{remark}
% 	In fact, the above proof can show that, if a sequence of functions $\{f_n\}_{n\in\NN}$ with $\im f_n$ separable, and $\{x\in X:f_n(x)\}$ is contained in a subset of $\mc S$ for each $n$, then the same is true for $f$.
% \end{remark}
% More generally, we have the following.
% \begin{lemma}
% 	Fix a simple $\mc S$-measurable function $f$. Given an open subset $U\subseteq B$, we have $f^{-1}(U)\cap E\in\mc S$, where $E\in\mc S$ is any set containing the points where $f$ is nonzero.
% \end{lemma}
% \begin{proof}
% 	As usual, pick up the promised sequence $\{f_n\}_{n\in\NN}$ of simple $\mc S$-measurable functions. Now, notice that $x\in f^{-1}(U)$ if and only if $f(x)\in U$, so we want to understand $U$ a little better. Without loss of generality, we may assume that $0\notin U$ because we are intersecting with $E$.

% 	We would like to say that $f(x)$ is just a little far from $\overline U$. For this, we set
% 	\[U_m\coloneqq\{x\in U:d(x,B\setminus U)<1/m\}\]
% 	for any $m\ge1$. Notably, we have that $U$ is the union of all the $U_m$, and $\overline{U_m}\subseteq U_{m+1}$ always.

% 	Continuing, we now see that $f(x)\in U$ is equivalent to $f(x)\in U_n$ for some $n$. Next, this is equivalent to saying that $f_k(x)\in U_n$ for all sufficiently large $k$ and some $n$. Certainly if $f(x)\in U_n$, then we have $f_k(x)\in U_n$ for sufficiently large $k$, and conversely, if $f_k(x)\in U_n$ for sufficiently large $k$, then $f(x)\in\overline{U_n}\subseteq U_{n+1}$.

% 	Thus, in total we see that
% 	\[f^{-1}(U)=\bigcup_{n=1}^\infty\bigcup_{K=1}^\infty\bigcap_{k\ge K}f_k^{-1}(U_n).\]
% 	However, $f_k^{-1}(U_n)\in\mc S$ by definition, so this full union of unions of intersections is still in $\mc S$.
% \end{proof}
% \begin{remark}
% 	As usual, we have bought more with the above proof: fix a sequence of functions $\{f_n\}_{n\in\NN}$ such that $f_n^{-1}(U)\in\mc S$ for each open $U\subseteq B\setminus\{0\}$. If $f_n\to f$ pointwise, then $f^{-1}(U)$ as well.
% \end{remark}
% \begin{remark}
% 	It follows from our remarks that a sequence of $\mc S$-measurable functions converging to a function $f$ will satisfy the conclusions of the previous three lemmas.
% \end{remark}
\begin{remark}
	Note that the case of $B=\RR$, we see that $f^{-1}(U)$ is measurable for any open $U\subseteq\RR$, where $f$ is an $\mathcal S$-measurable function. By taking unions and complements appropriately, we in fact see that $f^{-1}(U)$ is measurable for any Borel set $U\subseteq\RR$. This is the usual definition of a (Borel) measurable function $X\to\RR$, and we will show it is equivalent to the one we gave next class.
\end{remark}

\end{document}