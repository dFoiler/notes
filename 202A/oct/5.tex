% !TEX root = ../notes.tex

\documentclass[../notes.tex]{subfiles}

\begin{document}

\section{October 5}

We begin today by making some motivating remarks on $C^*$-algebras and the like. I hope it's not important because I didn't understand it very well.

\subsection{Evaluation Maps}
For this subsection, we will want to work with the fields $\RR$ and $\CC$ at the same time, so we pick up the following definition.
\begin{definition}
	An \textit{archimedean field} is either $\RR$ or $\CC$.
\end{definition}
% The precise meaning of this definition is unimportant; we will essentially only use $k=\RR$ and $k=\CC$.
% \begin{example}
% 	Both $k=\RR$ and $k=\CC$ are local fields. However, given a prime $p$, the $p$-adics $\QQ_p$ is also a local field, where the topology is determined by the norm map $|\cdot|_p\colon\QQ_p\to\RR_{\ge0}$.
% \end{example}
% \begin{remark}
% 	It turns out that all local fields $k$ come equipped with a norm map $\norm\cdot\colon k\to\RR_{\ge0}$ which induces their topology. We will use this fact freely; for example, it makes the notion of ``vanish at infinity'' generalize immediately from $\RR$ or $\CC$ to any local field $k$.
% \end{remark}
We now recall the following piece of notation, which we will state in the case we now care about.
\begin{notation}
	Fix an archimedean field $k$ and a compact Hausdorff space $X$. Then we let $C(X)$ denote the continuous functions $X\to k$.
\end{notation}
\begin{remark}
	Note that $C(X)$ is a $k$-subalgebra of $k^X$ because the constantly one function is continuous and the sum and product of two continuous functions is still continuous.
\end{remark}
It will turn out that $C(X)$ can tell us a lot about $X$. For example, homomorphisms we can use $X$ to build homomorphisms $C(X)\to k$.
\begin{example}
	Given any $x\in X$, the function $\op{ev}_x\colon C(X)\to k$ by $f\mapsto f(x)$ is a homomorphism. To see that this is a homomorphism, note that $\op{ev}_x(1)=1$, and $\op{ev}_x(f+g)=(f+g)(x)=f(x)+g(x)$, and $\op{ev}_x(fg)(x)=(fg)(x)=f(x)g(x)$.
\end{example}
In fact, these are all the homomorphisms!
\begin{theorem} \label{thm:cx-homs}
	Fix an archimedean field $k$ and a compact Hausdorff space $X$. Then all homomorphisms $C(X)\to k$ take the form $\op{ev}_x$ for some $x\in X$.
\end{theorem}
\begin{proof}
	Fix some homomorphism $\varphi\colon C(X)\to k$, and suppose for the sake of contradiction that $\varphi\coloneqq\op{ev}_x$ for each $x\in X$. To relate our geometry and our algebra, we will use the fact that the ``algebraic'' set $k\setminus\{0\}$ is open.
	
	Now, we can find $f_x\in C(X)$ with $\varphi(f_x)\ne f_x(x)$ for each $x\in X$. However, $(f_x-\varphi(f_x)1_X)\colon X\to k$ is a continuous function, so
	\[U_x\coloneqq\{y\in X:(f_x-\varphi(f_x)1_X)(y)\ne0\}\]
	is the preimage of the open subset $k\setminus\{0\}$ through the continuous function $(f_x-\varphi(f_x)1_X)$.
	
	Further, $x\in U_x$ because $(\varphi(f_x)1_X)(x)=\varphi(f_x)\ne f_x(x)$, so the open sets $\{U_x\}_{x\in X}$ produce an open cover of $X$, so we can finitely many of these points in $\{x_1,\ldots,x_n\}$ so that the open sets $U_i\coloneqq U_{x_i}$ cover $X$. Thus, the function
	\[f\coloneqq\sum_{i=1}^n(f_x-\varphi(f_x)1_X)^2\]
	is nonzero everywhere and thus a unit in $C(X)$. On the other hand, $\varphi(f_x-\varphi(f_x)1_X)=\varphi(f_x)-\varphi(f_x)\varphi(1_X)=0$, so summing gives $\varphi(f)=0$, which is a contradiction because ring homomorphisms send units to units!
\end{proof}
This motivates us to work in a little more generality: fix a local field $k$. Given a $k$-algebra $A$, set $A^*\coloneqq\op{Hom}_k(A,k)$, and given $a\in A$, define the homomorphism $\op{ev}_a\colon A^*\to k$ by $f\mapsto f(a)$. Then, by convention, we will give $A^*$ the weakest topology making the $a\mapsto\op{ev}_a$ continuous.
\begin{example}
	Using the $k$-algebra $A=C(X)$, the map $X\to A^*$ by $x\mapsto\op{ev}_x$ is a homeomorphism. Namely, this is a bijection by \autoref{thm:cx-homs}, and it is a homeomorphism essentially by the definition of the topology on $A^*$.
\end{example}
The point of the above example is that the algebra $C(X)$ and its evaluation maps are able to fully recover the topological space $X$!

\subsection{The Gelfand--Naimark Theorem}
By adding in a little more data, we can read even more information off $C(X)$.
\begin{remark}
	With $k=\CC$, note that complex conjugation extends to a function $C(X)\to C(X)$ by $f\mapsto\overline f$. Then one can check that
	\[\norm{\overline f\cdot f}_\infty=\norm f_\infty^2.\]
\end{remark}
In fact, the converse is true!
\begin{theorem}[Gelfand--Naimark] \label{thm:gelfand-naimark}
	Suppose that $A$ is a commutative Banach $\RR$-algebra or $\CC$-algebra equipped with an involution $a\mapsto a^*$ such that $\norm{aa^*}=\norm a^2$. Then there is an isomorphism
	\[A\simeq C(A^*).\]
\end{theorem}
In particular, all of these Banach algebras come from a topological space!
\begin{example}
	When $X$ is locally compact, set $C_\infty(X)$ to be the set of continuous functions $X\to k$ which vanish at infinity. Even though $C_\infty(X)$ has no multiplicative unit, it is still the case that $C_\infty(A^*)\cong C_\infty(X)$, and in fact $A^*\cong X$. Not having a unit turns out to not be a problem because we can have a function be $1$ over a large interval, which is topologically close enough to a unit.
\end{example}
\begin{example}
	In contrast, the bounded continuous functions $A\coloneqq C_b(X)$ have $A\cong C(A^*)$ still, even though $A^*$ is compact. This is weird: the embedding $X\into A^*$ is going to have elements not live in the image, but the elements outside the image require the Axiom of Choice to see.
\end{example}
The above example is why we prefer to work with $C_\infty(X)$ when we talk about locally compact spaces $X$.

Before jumping into measure theory, we will want to pick up the following definition.
\begin{definition}
	A \textit{Hilbert space} is a complete inner product $\RR$- or $\CC$-vector space.
\end{definition}
\begin{example}
	Given a Hilbert space $H$, the set of linear operators $B(H)$ on $H$ has a conjugation again, giving us an involution $T\mapsto \overline T$. One still has $\norm{T\overline T}=\overline T^2$, so \autoref{thm:gelfand-naimark} applies, and we can think about these spaces as spaces of functions.
\end{example}
The above example will generalize to the study of $C^*$ algebras, but we won't discuss this further.

\subsection{Finitely Additive Measures}
We begin with a motivating example. Consider the set of functions $f_n\colon[0,2]\to\RR$, given by the following image.
\begin{center}
	\begin{asy}
		unitsize(2cm);
		int n = 5;
		draw((0,1) -- (0,0) -- (2,0));
		draw((0,1) -- (1,1) -- (1+1/n,0) -- (2,0), red);
		label("$f_n$", (1,1), N, red);
		label("$0$", (0,0), S);
		label("$0$", (0,0), W);
		label("$2$", (2,0), S);
		label("$1$", (0,1), W);
	\end{asy}
\end{center}
More precisely, we can write
\[f_n(x)\coloneqq\min\{1,\max\{0,1-n(x-1)\}\}.\]
These functions are all continuous by definition, but we can also give them a piecewise definition as
\[f_n(x)\coloneqq\begin{cases}
	1 & x\le1, \\
	1-n(x-1) & 1\le x\le1+1/n, \\
	0 & 1+1/n\le x\le2.
\end{cases}\]
In particular, we can see that $f_n\to1_{[0,1]}$ as $n\to\infty$ with respect to the $\norm\cdot_p$ norm for $p\in[1,\infty)$: the error is
\[\norm{1_{[0,1]}-f_n}_p^p=\int_0^2|1_{[0,1/2]}(t)-f_n(t)|^p\,dt=\int_1^{1+1/n}|f_n(t)|^p\,dt\le\int_1^{1+1/n}dt=1/n,\]
which goes to $0$ as $n\to\infty$. Namely, to complete the set of our continuous functions $C([0,1])$ equipped with $\norm\cdot_p$ for $p\in[1,\infty)$, we need to add in these indicator functions. Nonetheless, we just integrated over $1_{[0,1]}$ just fine above, so we will want to build a class of functions which includes $1_{[0,1]}$ both for completeness reasons but also for integration reasons.

It turns out that not all sets should be able to be integrated over; this leads to the notion of measurable sets. So we will have some collection of subsets $\mc R\subseteq\mc P(\RR)$ and then some measuring function $\mu\colon\mc R\to[0,\infty]$ (we must allow infinity!). Let's discuss what we want to be true of $\mu$.
\begin{itemize}
	\item Additivity: if $E,F\in\mc R$ are disjoint, then $E\sqcup F$ should be in $\mc R$, and we had better have $\mu(E\sqcup F)=\mu(E)+\mu(F)$. Namely, the sum of the sizes of two disjoint sets had better just be size of the disjoint union.
	\item Splitting: if $E,F\in\mc R$ with $F\subseteq E$, then we want (from the above)
	\[\mu(E)=\mu(E\cap F)+\mu(E\setminus F),\]
	where the idea is that we can look at just the size of $E\cap F$ and $E\setminus F$ individually to more locally compute our sizes.
\end{itemize}
We can view the above rules as first dictating what sets should be measured at all. As such, we have the following definition.
\begin{definition}[Ring]
	Fix a set $X$. A ring is a nonempty collection $\mc R\subseteq\mc P(X)$ with the following properties.
	\begin{itemize}
		\item Union: if $E,F\in\mc R$, then $E\cup F\in\mc R$.
		\item Subtraction: if $E,F\in R$, then $E\setminus F\in\mc R$.
	\end{itemize}
\end{definition}
\begin{example}
	Of course, the full collection $\mc P(X)$ is a ring. More generally, given a subset $S\subseteq X$, the collection of subsets of $S$ is a ring: if $E,F$ are subsets of $S$, we see $E\cup F$ and $E\setminus F$ are both subsets of $F$.
\end{example}
\begin{example}
	Of course, $\{\emp\}$ is a ring.
\end{example}
\begin{example}
	The set of all finite subsets of $X$ is a ring. Indeed, if $E,F\subseteq X$ are finite, then both $E\cup F$ and $E\setminus F$ are finite as well.
\end{example}
\begin{remark} \label{rem:ring-has-intersections}
	Fix a ring $\mc R$ and some $E,F\in\mc R$. Note that $E\cap F=E\setminus(E\setminus F)$, so $E\cap F\in\mc R$ as well.
\end{remark}
\begin{remark} \label{rem:emp-is-in-ring}
	Given a ring $\mc R$, we note that $\emp\in\mc R$: we know there is some $E\in\mc R$, so it follows $\emp=E\setminus E\in\mc R$.
\end{remark}
Adding in the desired properties for our $\mu$, we can now define ``small'' measures.
\begin{definition}[Finitely additive measure]
	Fix a set $X$ and ring $\mc R\subseteq\mc P(X)$. Then a \textit{finitely additive measure} is a function $\mu\colon\mc R\to[0,\infty]$ such that any disjoint $E,F\in\mc R$ have
	\[\mu(E\sqcup F)=\mu(E)+\mu(F)\]
\end{definition}
\begin{remark}
	Note that $\mu(\emp)=\mu(\emp\sqcup\emp)=2\mu(\emp)$, so it follows $\mu(\emp)=0$.
\end{remark}
\begin{remark} \label{rem:fin-additive-is-subtractive}
	Note that being finitely additive tells us that $E\subseteq F$ implies $E=F\sqcup(E\setminus F)$ because an element of $E$ is either in $F$ or not in $F$. Thus, we see $\mu(E)=\mu(F)+\mu(E\setminus F)$, so if $\mu(F)<\infty$, we may write $\mu(E\setminus F)=\mu(E)-\mu(F)$.
\end{remark}
It turns out that being finitely additive is not good enough.
\begin{example} \label{ex:want-countable-union}
	We use the usual measure $\mu$ on $\RR$. Fix a sequence of disjoint intervals $\{E_i\}_{i=1}^\infty$ in $[0,1]$, and we see that we should have
	\[\sum_{i=1}^\infty\mu(E_i)<\infty.\]
	Defining $F_n\coloneqq\bigsqcup_{i\le n}E_i$ and $F\coloneqq\bigsqcup_{i=1}^\infty E_i$, we see that the characteristic functions $1_{F_n}$ is a Cauchy sequence converging to $1_F$, but we don't immediately have access to $1_F$ because it's an infinite union!
\end{example}
So next class we will discuss how adding a countably additive condition will help us.

\end{document}