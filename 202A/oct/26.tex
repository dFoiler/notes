% !TEX root = ../notes.tex

\documentclass[../notes.tex]{subfiles}

\begin{document}

\section{October 26}

Today we explain why we keep marking our sets as being measurable.

\subsection{A Non-measurable Set}
Here is our result.
\begin{exe}[Vitali]
	Let $T=\RR/\ZZ$ be the circle group, and let $\mu$ be the translation-invariant measure on $\RR/\ZZ$ with $\mu(T)=1$. It turns out that $\mu$ is complete. We produce a subset of $T$ which is not $\mu$-measurable.
\end{exe}
\begin{proof}
	Let $T_{\op{tors}}$ be the torsion subgroup of $T$. Namely, $r\in T_{\op{tors}}$ if and only if there exists some $n\in\ZZ_{>0}$ for which $nr=0$ in $T$, which means that $nr=k$ in $\RR$ for some integer $k$ and so $r=k/n$. Thus, $T_{\op{tors}}=\QQ/\ZZ$; the important point is that $T_{\op{tors}}$ is countable.

	Now, for each coset in $T/T_{\op{tors}}$, let $V\subseteq T$ be a set of representatives of these cosets.\footnote{Note that we have used the Axiom of Choice here.} In particular, it follows that
	\[T=\bigsqcup_{q\in T_{\op{tors}}}(q+V).\]
	Indeed, there are two checks.
	\begin{itemize}
		\item To see the union, for any $r\in T$, we see that $r\in x+T_{\op{tors}}$ for some $x\in V$, so $r=xq$ for some $q\in T_{\op{tors}}$, so $r\in qV$.
		\item To see that the union is disjoint, suppose $q_1+V=q_2+V$. Then we can find $r_1,r_2\in V$ such that $q_1+r_1=q_2+r_2$. It follows that $r_1=r_2+(q_2-q_1)\in r_2+T_{\op{tors}}$, so $r_1+T_{\op{tors}}=r_2+T_{\op{tors}}$, so $r_1=r_2$ because $V$ is made of representatives of $T/T_{\op{tors}}$. Thus, $q_1+r_1=q_2+r_2$ has forced $q_1=q_2$.
	\end{itemize}
	We are now ready to complete the proof. Suppose for the sake of contradiction that $V$ is measurable. It follows that
	\[1=\mu(T)=\mu\Bigg(\bigsqcup_{q\in T_{\op{tors}}}(x+V)\Bigg)=\sum_{q\in T_{\op{tors}}}\mu(q+V)\stackrel*=\sum_{q\in T_{\op{tors}}}\mu(V).\]
	Note that we have used the translation-invariance of $\mu$ in $\stackrel*=$. However, this is impossible: if $\mu(V)>0$, then the rightmost sum does not converge, and if $\mu(V)=0$, then the rightmost sum vanishes, so it is impossible for the sum to actually equal $1$.
\end{proof}
\begin{remark}
	The above proof used the Axiom of Choice to construct $V$. It is a result of Solovay that there are models of the real numbers where all subsets are measurable. Of course, the model does not include the Axiom of Choice.
\end{remark}

\end{document}