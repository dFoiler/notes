% !TEX root = ../notes.tex

\documentclass[../notes.tex]{subfiles}

\begin{document}

\section{December 2}

It's the last lecture. The exam will be cumulative, weighted towards material after the midterm (namely, measure theory).

\subsection{Defining \texorpdfstring{$\mc L^\infty$}{ L inf}}
Let's talk about $L^\infty$.
\begin{notation}
	Fix a measure space $(X,\mc S,\mu)$ and a Banach space $(B,\norm\cdot)$. Then $\mc L^\infty(X,\mc S,\mu,B)$ consists of the $\mc S$-measurable functions $f\colon X\to B$ such that there exists some $M$ for which
	\[\mu(\{x\in X:\norm{f(x)}\ge R\})=0.\]
	Intuitively, these are functions bounded away from a null set.
\end{notation}
Note that $\{x\in X:\norm{f(x)}\ge R\}$ is in fact in $\mc S$ by \autoref{cor:meas-has-meas-pre-image} because $\norm f$ is $\mc S$-measurable by \autoref{cor:take-norms-is-measurable}.

Here is our semi-norm.
\begin{notation}
	Fix a measure space $(X,\mc S,\mu)$ and a Banach space $(B,\norm\cdot)$. Then we define $\norm\cdot_\infty$ on $\mc L^\infty(X,\mc S,\mu,B)$ by
	\[\norm f_\infty\coloneqq\inf\{M\in\RR:\mu(\{x\in X:\norm{f(x)}\ge R\})=0\}.\]
\end{notation}
\begin{remark}
	Given $f\in\mc L^\infty(X,\mc S,\mu,B)$, we can see that $f\in\mc L^p(X,\mc S,\mu,B)$ for each $p\in[1,\infty)$ as well, and
	\[\lim_{p\to\infty}\norm f_p=\norm f_\infty.\]
\end{remark}
\begin{proposition}
	Fix a measure space $(X,\mc S,\mu)$ and a Banach space $(B,\norm\cdot)$. Then $\norm\cdot_\infty$ defines a semi-norm on $\mc L^\infty$.
\end{proposition}
\begin{proof}
	Omitted.
\end{proof}
\begin{proposition}
	Fix a measure space $(X,\mc S,\mu)$ and a Banach space $(B,\norm\cdot)$. With $f,g\in\mc L^\infty(X,\mc S,\mu,B)$, we see $f(x)=g(x)$ almost everywhere if and only if $\norm{f-g}_\infty=0$.
\end{proposition}
\begin{proof}
	Omitted.
\end{proof}
\begin{proposition}
	Fix a measure space $(X,\mc S,\mu)$ and a Banach space $(B,\norm\cdot)$. If a sequence of functions $\{f_n\}_{n\in\NN}$ is Cauchy in $\infty$-mean, then $\{f_n\}_{n\in\NN}$ is uniformly Cauchy outside a null set.
\end{proposition}
\begin{proof}
	Omitted.
\end{proof}
\begin{cor}
	Fix a measure space $(X,\mc S,\mu)$ and a Banach space $(B,\norm\cdot)$. Then $L^\infty(X,\mc S,\mu,B)$ is complete.
\end{cor}
\begin{proof}
	This follows from the previous result.
\end{proof}

Here are some of our usual checks.
\begin{lemma}
	Fix a measure space $(X,\mc S,\mu)$ and a $k$-Banach space $(B,\norm\cdot)$. Then $\mc L^\infty(X,\mc S,\mu,B)$ is a $k$-vector space.
\end{lemma}
\begin{proof}
	Omitted.
\end{proof}
In fact, we have a notion of multiplication!
\begin{lemma}
	Fix a measure space $(X,\mc S,\mu)$, and let $k\in\{\RR,\CC\}$. If $f,g\in\mc L^\infty(X,\mc S,\mu,k)$, then $fg\in\mc L^\infty(X,\mc S,\mu,B)$. In fact, $\norm f_\infty\cdot\norm g_\infty\le\norm{fg}_\infty$.
\end{lemma}
\begin{proof}
	Omitted.
\end{proof}
\begin{lemma}
	Fix a measure space $(X,\mc S,\mu)$, and let $B$ be a $k$-Banach space where $k\in\{\RR,\CC\}$. Then, given some $p\in[1,\infty)$, for any $a\in\mc L^\infty(X,\mc S,\mu,k)$ and $f\in\mc L^p(X,\mc S,\mu,B)$, we see $af\in\mc L^p(X,\mc S,\mu,B)$.
\end{lemma}
\begin{proof}
	Of course $\norm{af}^p$ is nonnegative and $\mc S$-measurable, so we just have to show that its integral is finite. The point is that
	\[\norm{af}^p\le\norm f_\infty^p\cdot\norm f^p\]
	almost everywhere, so
	\[\norm{af}_p\le\norm f_\infty\cdot\norm f_p\]
	after integrating.
\end{proof}
In other words, $\mc L^p$ is an $\mathcal L^\infty$-module.
\begin{example} \label{ex:first-bounded-functional}
	In particular, we see that $a\in L^\infty(X,\mc S,\mu,B)$ defines a continuous linear functional $\mu_a\colon L^p(X,\mc S,\mu,B)\to L^p(X,\mc S,\mu,B)$ such that $\norm{\mu(a)}_p\le\norm a_\infty\cdot\norm f_p$.
\end{example}

\subsection{Bounded Linear Functionals}
Let's generalize \autoref{ex:first-bounded-functional}.
\begin{definition}[Bounded]
	Fix a normed $k$-vector spaces $(V,\norm\cdot_V)$ and $(W,\norm\cdot_W)$. Then a linear transformation $T\colon V\to W$ defines
	\[\norm T\coloneqq\inf_{v\in V\setminus\{0\}}\{\norm{Tv}_W/\norm v_V\}.\]
	Then $T$ is \textit{bounded} if and only if $\norm T<\infty$.
\end{definition}
\begin{remark}
	It turns out that $T$ is continuous if and only if $T$ is a bounded linear functional.
\end{remark}
\begin{example}
	One can check that $\norm{\mu_a}=\norm a_\infty$.
\end{example}
Let's look at all our bounded linear operators at once.
\begin{notation}
	Given a normed $k$-vector spaces $(V,\norm\cdot_V)$ and $(W,\norm\cdot_W)$, then we let $\mc B(V,W)$ denote the normed $k$-vector space of bounded linear transformations $T\colon V\to W$. If $V=W$, we set $\mc B(V)\coloneqq\mc B(V,V)$.
\end{notation}
\begin{remark}
	One can check that the pointwise operations on $\mc B(V)$ also give $\norm{ST}\le\norm S\cdot\norm T$, so $\mc B(V)$ is a normed algebra as well.
\end{remark}
\begin{remark}
	It's also true that $V$ being complete implies that $\mc B(V)$ is complete. More generally, given normed vector space $(V,\norm\cdot_V)$ and $(W,\norm\cdot_W)$, then $W$ is complete implies that $\mc B(V,W)$ is complete, where $\mc B(V,W)$ has been defined in the only way which makes sense.
\end{remark}
It turns out that $\mc B(V)$ is the correct object to discuss representations.
\begin{definition}
	Fix a normed algebra $A$ and a normed vector space $V$. Then a \textit{representation} of $A$ in $V$ is an algebra homomorphism $A\to\mc B(V)$.
\end{definition}
The point is that we would like to respect the topologies on both $A$ and $V$, so we want a representation to only output continuous actions on $V$.
\begin{example}
	The map $a\mapsto\mu_a$ from earlier is a representation of $\mc L^p(X,\mc S,\mu,B)$ for $p\in[1,\infty]$.
\end{example}

\subsection{A Little Duality}
Let's move towards a little duality.
\begin{notation}
	Fix a normed $k$-vector space $(V,\norm\cdot)$. Then we set $\widehat V=V^\lor\coloneqq\mc B(V,k)$.
\end{notation}
We start with $L^1$.
\begin{notation}
	Fix a measure space $(X,\mc S,\mu)$ and set $k\in\{\RR,\CC\}$. Then given $f\in\mc L^\infty(X,\mc S,\mu,k)$ and $g\in\mc L^1(X,\mc S,\mu,k)$, we define
	\[\varphi_f\coloneqq\int_kfg\,d\mu.\]
\end{notation}
\begin{remark} \label{rem:define-iso-to-bl1}
	One can see that $\norm{\varphi_f(\xi)}\le\norm f_\infty\cdot\norm g_1$. In fact, equality holds. Thus, $\varphi_f$ is a bounded linear functional on $\mc L^1(X,\mc S,\mu,k)$. As such, we have an isometry
	\[\varphi_\bullet\colon L^\infty(X,\mc S,\mu,k)\to L^1(X,\mc S,\mu,k)^\lor.\]
\end{remark}
We might want the isometry $\varphi_\bullet$ to be surjective. It turns out that we have to add a few conditions on our measure space.
\begin{theorem}
	Fix a measure space $(X,\mc S,\mu)$ and a Banach space $(B,\norm\cdot)$. If $\mu$ is $\sigma$-finite, and $\mc S$ is a $\sigma$-algebra, then $\varphi_\bullet$ defined in \autoref{rem:define-iso-to-bl1} is an isomorphism of normed vector spaces.
\end{theorem}
\begin{proof}
	Take Math~202B.
\end{proof}
\begin{remark}
	There is an injection $\mc L^1(X,\mc S,\mu,B)\into\mc L^\infty(X,\mc S,\mu,B)^\lor$ by something similar to $\varphi_\bullet$, but it often fails to be surjective. Namely, we have an injection from $\mc L^1$ to its double-dual, but when $\mc L^1$ is infinite-dimensional, then the double-dual of a vector space will generally be larger. (One does have to use the axiom of choice to explicitly show this, however.)
\end{remark}
Now let's talk a little about $L^2$. Suppose $(X,\mc S,\mu)$ is a measure space such that $\mc S$ is a $\sigma$-algebra and $\mu$ is $\sigma$-finite. We still have our isometry of normed algebras
\[\mu_f\colon L^\infty(X,\mc S,k)\to\mc B\left(L^2(X,\mc S,\mu,k)\right),\]
and one can check that the image of $\mu_f$ is a ``von Neumann algebra,'' where we are given an adjoint $(\cdot)^*$ given by conjugation: $\mu_f^*\coloneqq\mu_{\overline f}$. Further, the image is closed, provided the topology is defined correctly: we use the ``strong operator topology'' induced by the semi-norms $T\mapsto\norm{T\xi}$ for all $\xi$.

In fact, these properties are sharp in the following sense.
\begin{theorem}
	Every commutative con Neumann algebra is isomorphic (preserving all the data) to some $\mc L^\infty$.
\end{theorem}
This is perhaps unsatisfying to analysts, who are comfortable removing commutativity hypotheses. We close the class by saying that non-commutative von Neumann algebras are quite interesting.

\end{document}