% LTeX: enabled=false

\documentclass{article}
\usepackage[utf8]{inputenc}

\newcommand{\nirpdftitle}{Measure Theory Speedrun}
\usepackage{import}
\inputfrom{..}{nir}

\pagestyle{contentpage}

\title{Measure Theory for the Impatient}
\author{Nir Elber}
\date{Fall 2022}
\rhead{\textit{MEASURE THEORY SPEEDRUN}}
\lhead{}

\setcounter{tocdepth}{2}

\begin{document}

\maketitle

\begin{abstract}
	This document collects a variety of definitions and results from measure theory.
\end{abstract}

\tableofcontents

\newpage

\section{Definitions}

\subsection{Rings and Friends}
\begin{definition}[Prering]
	Fix a set $X$. A \textit{prering} of a set $X$ is a nonempty collection $\mc P\subseteq\mc P(X)$ satisfying the following.
	\begin{itemize}
		\item Intersection: if $E,F\in\mc P$, then $E\cap F=\mc P$.
		\item Decomposition: if $E,F\in\mc P$, then we can write
		\[E\setminus F=\bigsqcup_{i=1}^nG_i\]
		for some finite disjoint union on the right-hand side with $G_i\in\mc P$ for each $i$.
	\end{itemize}
\end{definition}
\begin{definition}[Ring]
	Fix a set $X$. A ring is a nonempty collection $\mc R\subseteq\mc P(X)$ with the following properties.
	\begin{itemize}
		\item Union: if $E,F\in\mc R$, then $E\cup F\in\mc R$.
		\item Subtraction: if $E,F\in R$, then $E\setminus F\in\mc R$.
	\end{itemize}
\end{definition}
\begin{defihelper}[{$\sigma$-ring}] \nirindex{Sigma-ring@$\sigma$-ring}
	Fix a set $X$. Then a ring $\mc R\subseteq\mc P(X)$ is a \textit{$\sigma$-ring} if and only if $R$ is closed under countable unions.
\end{defihelper}
\begin{defihelper}[{$\sigma$-algebra}] \nirindex{Sigma-algebra@$\sigma$-algebra}
	Fix a set $X$. Then a ring $\mc R\subseteq\mc P(X)$ is a \textit{$\sigma$-algebra} if and only if $R$ is a $\sigma$-ring and contains $X$.
\end{defihelper}
\begin{example}
	Given a topological space $(X,\mc T)$, the $\sigma$-algebra generated by $\mc T$ make the $\sigma$-algebra of \textit{Borel subsets} of $X$.
\end{example}
\begin{definition}[Hereditary]
	Fix a set $X$ and nonempty family $\mc G\subseteq\mc P(X)$. Then $\mc G$ is \textit{hereditary} if and only if $A\in\mc G$ and $A'\subseteq A$ implies $A'\in\mc G$.
\end{definition}
\begin{definition}[{Hereditary $\sigma$-ring}]
	Fix a set $X$ and nonempty family $\mc F\subseteq\mc P(X)$. Then the \textit{hereditary $\sigma$-ring} $\mc H(F)$ generated by $\mc F$ consists of all subsets $E\subseteq X$ such that there exists a countable subcollection $\{F_i\}_{i=1}^\infty\subseteq\mc F$ such that
	\[E\subseteq\bigcup_{i=1}^\infty E_i.\]
\end{definition}

\subsection{Measures and Friends}
\begin{definition}[Finitely additive measure]
	Fix a set $X$ and ring $\mc R\subseteq\mc P(X)$. Then a \textit{finitely additive measure} is a function $\mu\colon\mc R\to[0,\infty]$ such that any disjoint $E,F\in\mc R$ have
	\[\mu(E\sqcup F)=\mu(E)+\mu(F)\]
\end{definition}
\begin{definition}[Countably additive]
	Fix a set $X$ and a collection of subsets $\mc C\subseteq\mc P(X)$. A function $\mu\colon\mc C\to[0,\infty]$ is \textit{countably additive} if and only if any pairwise disjoint subcollection $\{E_i\}_{i=1}^\infty\subseteq\mc C$ with $\bigsqcup_{i=1}^\infty E_i\in\mc C$ has
	\[\mu\Bigg(\bigsqcup_{i=1}^\infty E_i\Bigg)=\sum_{i=1}^\infty\mu(E_i).\]
	Notably, we are allowed to have the right-hand side diverge to $\infty$ if the left-hand side is $\infty$.
\end{definition}
\begin{definition}[Premeasure]
	Fix a set $X$ and a prering $\mc P\subseteq\mc P(X)$. A \textit{premeasure} on $\mc P$ is a countably additive function $\mu\colon\mc P\to[0,\infty]$.
\end{definition}
\begin{definition}[Measure]
	Fix a set $X$ and $\sigma$-ring $\mc S$. Then a \textit{measure} on $\mc S$ is a function $\mu\colon\mc S\to[0,\infty]$ which is countably additive. We call the triple $(X,\mc S,\mu)$ a \textit{measure space}.
\end{definition}
\begin{example} \label{prop:leb-premeasure}
	Fix a left-continuous, increasing function $\alpha\colon\RR\to\RR$, and let $\mc P\subseteq\mc P(\RR)$ as the prering of half-open intervals $[a,b)$ for $a<b$. Then
	\[\mu_\alpha([a,b))\coloneqq\alpha(b)-\alpha(a)\]
	is a premeasure on $\mc P$.
\end{example}
\begin{notation}
	Fix a set $X$ and nonempty family $\mc F\subseteq\mc P(X)$. Then give $\mu\colon\mc F\to[0,\infty]$, we will define $\mu^*\colon\mc H(\mc F)\to[0,\infty]$ by
	\[\mu^*(E)\coloneqq\inf\Bigg\{\sum_{i=1}^\infty\mu(E_i):\{E_i\}_{i=1}^\infty\subseteq\mc F\text{ and }E\subseteq\bigcup_{i=1}^\infty E_i\Bigg\}.\]
\end{notation}
\begin{definition}[Lebesgue--Stieltjes measure]
	Let $\mc P$ be the prering of right-half-open intervals, and fix a left-continuous function $\alpha\colon\RR\to\RR$. Then the measure $\mu_\alpha^*|_{\mc M(\mu_\alpha^*)}$ from the premeasure of \autoref{prop:leb-premeasure} is the \textit{Lebesgue--Stieltjes measure}. The \textit{Lebesgue measure} is the measure coming from $\alpha(t)=t$.
\end{definition}

\subsection{Adjectives for Measures}
\begin{definition}[Monotone]
	Fix a collection $\mc F$ of subsets of a set $X$. A function $\mu\colon\mc F\to[0,\infty]$ is \textit{monotone} if and only if any $E,F\in\mc F$ with $E\subseteq F$ have $\mu(E)\le\mu(F)$.
\end{definition}
\begin{definition}[Countably subadditive]
	Fix a set $X$ and a collection $\mc F\subseteq\mc P(X)$. A function $\mu\colon\mc F\to[0,\infty]$ is \textit{countably subadditive} if and only if
	\[E\subseteq\bigcup_{i=1}^\infty E_i\implies\mu(E)\le\sum_{i=1}^\infty\mu(E_i)\]
	for any $E\in\mc F$ and $\{E_i\}_{i=1}^\infty\subseteq\mc F$.
\end{definition}
\begin{definition}
	Fix a set $X$ and a hereditary $\sigma$-ring $\mc H$ on $X$, and fix an outer measure $\nu\colon\mc H\to[0,\infty]$. Then a set $E\subseteq\mc H$ is \textit{$\nu$-measurable} if and only if
	\[\nu(A)=\nu(A\cap E)+\nu(A\setminus E)\]
	for any $A\in\mc H$. We will let $\mc M(\nu)$ denote the set of $\nu$-measurable sets.
\end{definition}
\begin{definition}[Compelete]
	Fix a set $X$ and a family $\mc F\subseteq\mc P(X)$. Then a function $\nu\colon\mc F\to[0,\infty]$ is \textit{complete} if and only if any $E\in\mc F$ with $F\subseteq E$ and $\nu(E)=0$ must have $F\in\mc F$ and $\nu(F)=0$.
\end{definition}
\begin{example}
	If $\nu$ is an outer measure on a hereditary $\sigma$-ring $\mc H$, then $\nu|_{\mc M(\nu)}$ is complete when $\mc M(\nu)$ is nonempty.
\end{example}

\subsection{Measurable Functions and Friends}
\begin{definition}[Simple measurable function]
	Fix a ring $\mc S$ on a set $X$ and a normed vector space $B$. Then a \textit{simple $\mc S$-measurable $B$-valued function} is a function $f\colon X\to B$ such that $\im f$ is finite and $f^{-1}(\{y\})\in\mc S$ for any $y\in B\setminus\{0\}$.
\end{definition}
\begin{remark}
	Any simple $\mc S$-measurable function $f\colon X\to S$ can be written as
	\[f=\sum_{y\in(\im f)\setminus\{0\}}y1_{f^{-1}(\{y\})}.\]
\end{remark}
\begin{definition}[Measurable function]
	Fix a set $X$ and a $\sigma$-ring $\mc S$ on $X$. Given a normed vector space $B$, an \textit{$\mathcal S$-measurable function} is a function $f\colon X\to B$ such that there is a sequence of simple $\mc S$-measurable functions $\{f_n\}_{n\in\NN}$ which converge to $f$ pointwise.
\end{definition}
\begin{definition}[Separable]
	A topological space $M$ is \textit{separable} if and only if there is a countable dense subset of $M$. As such, a subset $A\subseteq M$ is \textit{separable} if and only if $A$ is separable with the restricted metric; in other words, $A\subseteq M$ is separable if and only if there is a countable subset $B\subseteq A$ such that $A\subseteq\overline B$.
\end{definition}
\begin{theorem} \label{thm:better-measurable}
	Fix a normed vector space $B$ and a set $X$ with a $\sigma$-ring $\mc S$ on $X$. Then a function $f\colon X\to B$ is $\mc S$-measurable if and only if
	\begin{listroman}
		\item $\im f$ is separable, and
		\item for any open $U\subseteq B$, we have $f^{-1}(U\setminus\{0\})\in\mc S$.
	\end{listroman}
\end{theorem}

\subsection{All the Convergences}
\begin{definition}[Null set]
	Fix a set $X$ and a $\sigma$-ring $\mc S$ on $X$ equipped with a measure $\mu$. A \textit{null set} is a subset $N\subseteq X$ such that there is some $E\in\mc S$ such that $N\subseteq E$ and $\mu(N)=0$.
\end{definition}
\begin{definition}[Almost everywhere]
	Fix a set $X$ and a $\sigma$-ring $\mc S$ on $X$ equipped with a measure $\mu$. A property $P(x)$ for points $x\in X$ holds \textit{almost everywhere} if and only if $\{x\in X:\lnot P(x)\}$ is a null set.
\end{definition}
\begin{definition}[Converge in measure] \label{def:converge-in-measure}
	Fix a measure space $(X,\mc S,\mu)$ and normed vector space $(B,\norm\cdot)$. Then a sequence $\{f_n\}_{n\in\NN}$ of $\mc S$-measurable functions \textit{converges in measure} to an $\mathcal S$-measurable function $f$ if and only if all $\varepsilon>0$ have
	\[\lim_{n\to\infty}\mu(\{x\in X:\norm{f(x)-f_n(x)}\ge\varepsilon\})=0.\]
\end{definition}

\subsection{Integration}
\begin{definition}[Simple integrable function]
	Fix a ring $\mc S$ on a set $X$ and a metric space $B$. Further, let $\mu$ be a finitely additive measure $\mu$ on $\mc S$. Then a function $f\colon X\to B$ is a \textit{simple $\mc S$-integrable function} if and only if $\im f$ is finite, and $f^{-1}(\{y\})\in\mc S$ has finite measure for each $y\in(\im f)\setminus\{0\}$.
\end{definition}
\begin{definition}[Integral]
	Fix a ring $\mc S$ on a set $X$ and a metric space $B$. Further, let $\mu$ be a finitely additive measure $\mu$ on $\mc S$. Given a simple $\mu$-integrable function $f$, we define the \textit{integral}
	\[\int_Xf\,d\mu\coloneqq\sum_{y\in(\im f)\setminus\{0\}}\mu\left(f^{-1}(\{y\})\right)y.\]
	Note this is a finite sum with $\mu\left(f^{-1}(\{y\})\right)$ finite, so $\int_Xf\,d\mu$ is finite.
\end{definition}
\begin{notation}
	Fix a normed vector space $B$ and a ring $\mc S$ on a set $X$ equipped with a finitely additive measure $\mu$. Given a (simple) $\mu$-integrable function $f\colon X\to B$, we define
	\[\norm f_1\coloneqq\int_X\norm f\,d\mu.\]
	Note $\norm f$ is in fact simple $\mu$-integrable.
\end{notation}

\newpage
\section{Lemmas and Results}

\subsection{Checks on Measures}
\begin{lemma}[Monotone] \label{lem:fin-additive-is-monotone}
	Fix a prering $\mc P$ on $X$ and a finitely additive function $\mu\colon\mc P\to[0,\infty]$. Given $E,F\in\mc P$, then $\mu(E)\ge\mu(E\cap F)$. In particular, if $E\supseteq F$, then $\mu(E)\ge\mu(F)$.
\end{lemma}
\begin{lemma}[Countably subadditive]
	Fix a prering $\mc P$ on a set $X$, and let $\mu$ be a premeasure on $\mc P$. Then $\mu$ is countably subadditive.
\end{lemma}
\begin{lemma} \label{lem:hered-measure-facts}
	Fix a set $X$ and nonempty family $\mc F\subseteq\mc P(X)$. Further, fix some $\mu\colon\mc F\to[0,\infty]$. Then we have the following.
	\begin{listalph}
		\item $\mu^*(E)\le\mu(E)$ for any $E\in\mc F$.
		\item $\mu^*$ is monotone.
		\item $\mu^*$ is countably subadditive.
	\end{listalph}
\end{lemma}
\begin{lemma} \label{lem:hered-measure-extends}
	Fix a set $X$ and a prering $\mc P$ on $X$ equipped with a premeasure $\mu$ on $\mc P$. Then $\mu^*(E)=\mu(E)$ for any $E\in\mc P$.
\end{lemma}
\begin{theorem} \label{thm:from-outer}
	Fix a set $X$ and a hereditary $\sigma$-ring $\mc H$ on $X$, and fix an outer measure $\nu\colon\mc H\to[0,\infty]$. If nonempty, $\mc M(\nu)$ is a $\sigma$-ring, and $\nu|_{\mc M(\nu)}$ is a measure.
\end{theorem}
\begin{theorem} \label{thm:prering-is-measurable}
	Fix a set $X$ and a prering $\mc P$ on $X$ equipped with a premeasure $\mu$ on $\mc P$. Then $\mc P\subseteq\mc M(\mu^*)$.
\end{theorem}
\begin{theorem} \label{thm:measure-extension-unique}
	Fix a set $X$ and a prering $\mc P$ on $X$ equipped with a $\sigma$-finite premeasure $\mu$ on $X$. Then, for some $\sigma$-ring $\mc S\subseteq\mc M(\mu^*)$, our $\mu^*|_{\mc S}$ is the unique extension of $\mu$ to a measure on $\mc S$.
\end{theorem}
\begin{proposition} \label{prop:measure-union-up}
	Fix a $\sigma$-ring $\mc S$ on a set $X$ equipped with a measure $\mu$ on $\mc S$. A collection $\{E_i\}_{i=1}^\infty\subseteq\mc S$ such that $E_n\subseteq E_{n+1}$ for each $i$ will have
	\[\lim_{n\to\infty}\mu(E_n)=\mu\Bigg(\bigcup_{i=1}^\infty E_i\Bigg).\]
\end{proposition}
\begin{cor} \label{cor:measure-inter-down}
	Fix a $\sigma$-ring $\mc S$ on a set $X$ equipped with a measure $\mu$ on $\mc S$. Suppose we have a collection $\{E_i\}_{i=1}^\infty\subseteq\mc S$ such that $\mu(E_1)<\infty$ and $E_n\supseteq E_{n+1}$ for each $i$. Then we have
	\[\lim_{n\to\infty}\mu(E_n)=\mu\Bigg(\bigcap_{i=1}^\infty E_i\Bigg).\]
\end{cor}

\subsection{Checks on Measurable Functions}
\begin{lemma}
	Fix a measure space $(X,\mc S,\mu)$ and a Banach space $(B,\norm\cdot)$ over a normed field $k$. Given $a,b\in k$ and $E\in\mc S$, if $f$ and $g$ are (simple $\mc S$-measurable, $\mc S$-measurable, simple $\mu$-integrable, $\mu$-integrable), then $af+b$ and $\norm f$ and $f1_E$ and $f1_{X\setminus E}$ are as well.
\end{lemma}
\begin{lemma}
	Convergence statement.
\end{lemma}

\end{document}