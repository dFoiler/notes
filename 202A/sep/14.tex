% !TEX root = ../notes.tex

\documentclass[../notes.tex]{subfiles}

\begin{document}

\section{September 14}

The march continues.

\subsection{The Tietze Extension Theorem}
Here is the main result for today.
\begin{restatable}[Tietze extension]{theorem}{tiethm} \label{thm:tie}
	Fix a normal topological space $(X,\mc T)$, and give some closed subset $A\subseteq X$ the relative topology from $X$. Given a continuous function $f\colon A\to\RR$, there exists a continuous function $\widetilde f\colon X\to\RR$ such that $\widetilde f|_A=f$. In fact, if $\im f\subseteq[a,b]$, then we may enforce $\im\widetilde f\subseteq[a,b]$ as well.
\end{restatable}
This property is quite special to $\RR$ shared by a few other spaces.
\begin{example}
	Take $X\coloneqq\overline{B(0,1)}\subseteq\RR^2$ given the relative topology, and let $A=\del X$ be the boundary, which is the unit circle. Then the identity function $\id_A\colon A\to A$ does not extend continuously to a function $\widetilde{\id_A}\colon X\to A$. To see this rigorously, take a course in algebraic topology.
\end{example}
\begin{example}
	Of course, any set $Y$ given the indiscrete topology will be such that a continuous function $f\colon A\to Y$ can be extended to continuously to a function $\widetilde f\colon X\to Y$ because all functions to $Y$ are continuous for free.
\end{example}
\begin{remark}
	The condition of $\im f\subseteq[a,b]$ might as well be replaced by $\im f\subseteq[0,1]$ by using the homeomorphism $\RR\to\RR$ by $x\mapsto(x-a)/(b-b)$ which will send $[a,b]$ to $[0,1]$.
\end{remark}
Here is a lemma which will help the proof of \autoref{thm:tie}.
\begin{lemma} \label{lem:techtie}
	Fix a normal topological space $(X,\mc T)$, and give some closed subset $A\subseteq X$ the relative topology from $X$. Given a continuous function $f\colon A\to[0,r]$ (where $r>0$), there exists a continuous function $g\colon X\to[0,r/3]$ such that
	\[0\le f(a)-g(a)\le2r/3\]
	for each $a\in A$.
\end{lemma}
\begin{proof}
	Set $B \coloneqq\{x\in A:f(x)\le r/3\}=f^{-1}([0,r/3])$ and $C \coloneqq\{x\in A:f(x)\ge 2r/3\}=f^{-1}([(2r/3,r])$. Both $B,C\subseteq A$ and $C$ are closed because they are the pre-image of closed subsets under $f\colon A\to\RR$. In fact, by the relative topology, we can write $B=B'\cap A$ where $B'\subseteq X$ is closed. However, $B'$ and $A$ are both closed in $X$, so $B\subseteq X$ is closed. Similar holds for $C$.
	
	Thus, so Urysohn's lemma provides (\autoref{thm:ury}) a continuous function $g\colon X\to[0,1]$ such that $g|_B=0$ and $g|_C=1$. As such, we define $g\colon X\to[0,r/3]$ by
	\[g(x)\coloneqq(r/3)\cdot g(x),\]
	which is still continuous because the map $x\mapsto(r/3)x$ is a homeomorphism $[0,1]\to[0,r/3]$ by \autoref{ex:closedintervalhomeo}. We can now see that $g$ satisfies the needed properties. Fix some $a\in A$.
	\begin{itemize}
		\item If $a\in B$, then $g(a)=0$ while $f(a)\le r/3$, so $0\le f(a)-g(a)\le r/3$.
		\item If $a\in C$, then $g(a)=r/3$ while $f(a)\in[2r/3,r]$, so $0\le f(a)-g(a)\le 2r/3$.
		\item Lastly, $a\notin B$ and $a\notin C$ means that $r/3<f(a)<2r/3$ while $0\le g(a)\le r/3$, so it follows $0\le f(a)-g(a)\le 2r/3$ still.
	\end{itemize}
	The above checks finish.
\end{proof}
We now show the following special case of \autoref{thm:tie}.
\begin{prop} \label{prop:finitetie}
	Fix a normal topological space $(X,\mc T)$, and give some closed subset $A\subseteq X$ the relative topology from $X$. Given a continuous function $f\colon A\to[0,1]$, there exists a continuous function $\widetilde f\colon X\to[0,1]$ such that $\widetilde f|_A=f$.
\end{prop}
\begin{proof}
	For brevity, define $\sigma\coloneqq2/3$. Taking $r=1$ in \autoref{lem:techtie}, we get a function $g_1\colon X\to[0,1/3]$ with
	\[0\le f(a)-g_1(a)\le\sigma\]
	for all $a\in A$, so define $\widetilde f_1\coloneqq g_1$. Next applying \autoref{lem:techtie} to $(f-\widetilde f_1|_A)\colon A\to[0,\sigma]$ with $r=\sigma$, we get promised a function $g_2\colon X\to[0,\sigma/3]$ with
	\[0\le f(a)-\widetilde f_1(a)-g_2(a)\le\sigma^2\]
	for any $a\in A$, so define $\widetilde f_2\coloneqq\widetilde f_1+g_2$.
	
	In general, suppose given a function $\widetilde f_n\colon X\to[0,1]$ with
	\[0\le f(a)-\widetilde f_n(a)\le\sigma^n\]
	for $a\in A$, we can use \autoref{lem:techtie} to $(f-\widetilde f_n|_A)\colon A\to[0,\sigma^n]$ to get a function $g_{n+1}\colon X\to[0,\sigma^n/3]$ with
	\[0\le f(a)-\widetilde f_n(a)-g_{n+1}(a)\le\sigma^{n+1}\]
	for $a\in A$, allowing us to then set $\widetilde f_{n+1}\coloneqq\widetilde f_n+g_{n+1}$.

	Applying the above process inductively, we get a function
	\[\widetilde f_n=\sum_{k=1}^ng_k\]
	going to $[0,1]$ such that $\norm{g_k}_\infty\le\sigma^{k-1}/3$ and $0\le f(a)-\widetilde f_n(a)\le(2/3)^n$ for each $a\in A$ and $n\ge1$. Notably, using the uniform metric $d_u$, we see that any $n\ge m$ has
	\[d_u(\widetilde f_n,\widetilde f_m)=\sup_{x\in X}\Bigg(\sum_{k=m+1}^ng_k(x)\Bigg)\le\sum_{k=m+1}^n\frac13\sigma^{k-1}
	\le\frac{\sigma^m}3\sum_{k=0}^\infty\sigma^k=\frac{\sigma^m}3\cdot\frac1{1-\sigma}=\left(\frac23\right)^m,
	\]
	which gets arbitrarily small. Thus, $\{\widetilde f_n\}_{n\in\NN}$ is a Cauchy sequence: for any $\varepsilon>0$, we can find $N$ with $n>N$ having $(2/3)^n<\varepsilon$, meaning $n,m\ge N$ will have $d_u(\widetilde f_n,\widetilde f_m)<\varepsilon$. Now, because $[0,1]\subseteq\RR$ is a closed subset of a complete metric space and hence complete by \autoref{cor:closediscomplete}, the sequence $\{\widetilde f_n\}_{n\in\NN}$ converges to a continuous function $\widetilde f\colon X\to[0,1]$ by \autoref{prop:contiscomplete}.

	It remains to check that $\widetilde f|_A=f$. Well, any $a\in A$ and $n\in\NN$ have
	\[|f(a)-\widetilde f(a)|\le|f(a)-\widetilde f_n(a)|+|\widetilde f_n(a)-\widetilde f(a)|\le\left(\frac23\right)^n+|\widetilde f_n(a)-f(a)|.\]
	Because $\widetilde f_n\to f$ as $n\to\infty$ under the metric $d_u$, we see that $|\widetilde f_n(a)-f(a)|\to0$ as $n\to\infty$. Additionally, $(2/3)^n\to0$ as $n\to\infty$, so the entire right-hand side goes to $0$ as $n\to\infty$, meaning that $|f(a)-\widetilde f(a)|<\varepsilon$ for all $\varepsilon>0$. Thus, $f(a)=\widetilde f(a)$ for each $a\in A$.
\end{proof}

\end{document}