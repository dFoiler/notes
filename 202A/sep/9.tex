% !TEX root = ../notes.tex

\documentclass[../notes.tex]{subfiles}

\begin{document}

\section{September 9}

The fun continues. The next problem set is going to be long but only in words, not in what we actually have to prove. We are being told not to be intimidated.
\begin{remark} \label{rem:adjectivemotivation}
	We are about to transition from making topologies to coming up with adjectives which will give ``lots'' of continuous maps to, say, the real numbers. A rigorization of this shall be provided shortly.
\end{remark}

\subsection{Normal Spaces}
Last class we briefly mentioned the Hausdorff property.
\begin{definition}[Hausdroff]
	Fix a topological space $(X,\mc T)$. Then $(X,\mc T)$ is \textit{Hausdorff} if and only if, for any two distinct points $x,x'\in X$, there are disjoint open sets $U$ and $U'$ such that $x\in U$ and $x'\in U'$.
\end{definition}
\begin{example}
	A metric space $(X,d)$ is Hausdorff. Indeed, given distinct points $x,x'\in X$, we have $d(x,x')>0$, so we set $r\coloneqq\frac12d(x,x')$. Then $x\in B(x,r)$ and $x'\in B(x',r)$ (which are open sets by \autoref{ex:openballisopen}), we see $B(x,r)\cap B(x',r)=\emp$. Indeed, if we had $y\in B(x,r)\cap B(x',r)$, then we must have
	\[d(x,x')\le d(x,y)+d(x',y)<2r=d(x,x'),\]
	which is a contradiction.
\end{example}
Here is the image
\begin{center}
	\begin{asy}
		unitsize(1cm);
		dot("$x$", (0,0), S); dot("$x'$", (3,0), S);
		draw(circle((0,0),1), dashed); draw(circle((3,0),1),dashed);
		label("$U$", (0,-1), S);
		label("$U'$", (3,-1), S);
	\end{asy}
\end{center}
Here is another adjective.
\begin{definition}[Normal]
	Fix a topological space $(X,\mc T)$. Then $(X,\mc T)$ is \textit{Hausdorff} if and only if, for any two disjoint closed sets $V,V'\subseteq X$, there are disjoint open sets $U$ and $U'$ such that $V\in U$ and $V'\in U'$.
\end{definition}
\begin{remark}
	Intuitively, Hausdorff is approximately the normal property with singleton sets. In particular, some authors require ``Hausdorff'' in the definition of a normal space. We will not do this.
\end{remark}
\begin{example}
	Any set $X$ given the indiscrete topology is normal. The problem here is that the only closed sets $\{\emp,X\}$, so the only possible pair of disjoint closed sets are $V_1\coloneqq\emp$ and $V_2\coloneqq\emp$, for which the open sets $U_1\coloneqq\emp$ and $U_2\coloneqq\emp$ are disjoint and cover these.
\end{example}
\begin{example}
	A set $X$ with more than $2$ elements is normal, as shown in the previous example, but it is not Hausdorff. Namely, finding distinct points $x_1,x_2\in X$, the only open subset of $X$ containing $x_1$ or $x_2$ is $X$, so there are no disjoint open subsets $U_1$ containing $x_1$ and $U_2$ containing $x_2$.
\end{example}
Here is the image.
\begin{center}
	\begin{asy}
		unitsize(1cm);
		fill(circle((0,0),1),lightgray);
		draw(circle((0,0),1));
		fill(circle((4,0),1),lightgray);
		draw(circle((4,0),1));
		draw(circle((0,0),1.5), dashed);
		draw(circle((4,0),1.5), dashed);
		label("$V$", (0,0));
		label("$V'$", (4,0));
		label("$U$", (0,-1.5), S);
		label("$U'$", (4,-1.5), S);
	\end{asy}
\end{center}
It is not completely obvious that metric spaces are normal, but we will see that they are.

Here is the main result for today.
\begin{restatable}[Urysohn's lemma]{theorem}{urythm} \label{thm:ury}
	Fix a topological space $(X,\mc T)$. If $(X,\mc T)$ is normal, then for any disjoint closed subsets $V_0,V_1\subseteq X$, there is a continuous function $f\colon X\to[0,1]$ such that $f(V_0)=\{0\}$ and $f(V_1)=\{1\}$.
\end{restatable}
\noindent So the point here is to realize \autoref{rem:adjectivemotivation}, where being normal is implying that we have ``lots'' of continuous functions.
\begin{remark} \label{rem:uryconv}
	Certainly if a topological space $(X,\mc T)$ satisfies the conclusion of \autoref{thm:ury}, then $(X,\mc T)$ is normal. Indeed, for any disjoint closed subsets $V_0,V_1\subseteq X$, pick up the promised continuous function $f$. Then
	\[V_0\subseteq f^{-1}((-1/2,1/2))\qquad\text{and}\qquad V_1\subseteq f^{-1}((1/2,3/2))\]
	are disjoint open sets; namely, these are open because $f$ is continuous, and they are disjoint because $f^{-1}((-1/2,1/2))\cap f^{-1}((1/2,3/2))=f^{-1}\big((-1/2,1/2)\cap(1/2,3/2)\big)=f^{-1}(\emp)=\emp$.
\end{remark}

\subsection{Urysohn's Lemma: Metric Spaces}
Let's see \autoref{thm:ury} for metric spaces, which will prove that metric spaces are normal by \autoref{rem:uryconv}. We pick up the following definition.
\begin{definition}
	Fix a metric space $(X,d)$. Then we define, for any $x\in X$ and nonempty subset $A\subseteq X$,
	\[d_A(x)\coloneqq\inf_{a\in A}d(x,a).\]
\end{definition}
\begin{remark}
	The infimum here exists because $A$ is nonempty, so the set $\{d(x,a):a\in A\}$ is nonempty (and bounded below by $0$).
\end{remark}
The image is that $d_A(x)$ is the distance from $x$ to $A$.
\begin{center}
	\begin{asy}
		unitsize(1cm);
		fill(circle((0,0),1), lightgray);
		draw(circle((0,0),1), dashed);
		label("$A$", (0,0));
		dot("$x$", (-3,0), W);
		draw((-3,0) -- (-1,0), red);
		label("$d_A(x)$", (-2,0), N);
	\end{asy}
\end{center}
We have the following continuity check.
\begin{lemma} \label{lem:distiscont}
	Fix a metric space $(X,d)$. Then, for any nonempty subset $A\subseteq X$, the function $d_A\colon X\to\RR$ is Lipschitz continuous.
\end{lemma}
\begin{proof}
	Fix any $x,y\in X$. Then, for any given $a\in A$, we find that
	\[d_A(x)\le d(x,a)\le d(x,y)+d(y,a).\]
	Thus, $d_A(x)-d(x,y)\le d(y,a)$ for all $a\in A$, so we conclude that
	\[d_A(x)-d(x,y)\le\inf_{a\in A}d(y,a)=d_A(y),\]
	so $d_A(x)-d_A(y)\le d(x,y)$. By symmetry, we also have $d_A(y)-d_A(x)\le d(x,y)$, so it follows
	\[\big|d_A(x)-d_A(y)\big|\le d(x,y),\]
	which is what we need for our Lipschitz continuous.
\end{proof}
As a sanity-check that this function behaves like it should, we pick up the following.
\begin{lemma} \label{lem:zerolocusofda}
	Fix a metric space $(X,d)$. Then, for any nonempty subset $A\subseteq X$, we have
	\[d_A^{-1}(\{0\})=\overline A.\]
\end{lemma}
\begin{proof}
	Certainly $A\subseteq d_A^{-1}(\{0\})$ because $d_A(a)=0$ for all $a\in A$. (In particular, $d_A(x)\ge0$ everywhere, and $a\in A$ implies that $d_A(a)\le d(a,a)=0$.) Because $d_A$ is continuous by \autoref{lem:distiscont}, we see $d_A^{-1}(\{0\})$ is closed, so containing $A$ forces
	\[\overline A\subseteq d_A^{-1}(\{0\}).\]
	Conversely, suppose that $x\notin X\setminus\overline A$, and we show that $d_A(x)>0$. Indeed, $X\setminus\overline A$ is open, so there is some open ball $B(x,\varepsilon)$ with $\varepsilon>0$ such that $B(x,\varepsilon)\subseteq X\setminus\overline A$. It follows $B(x,\varepsilon)\cap\overline A=\emp$, so
	\[d(a,x)\ge\varepsilon\]
	for all $a\in A$. Thus, $d_A(x)\ge\varepsilon>0$, so $d_A(x)\ne0$.
\end{proof}
\begin{example}
	If $A\subseteq X$ is a dense subset, then $\overline A=X$, so $d_A\colon X\to\RR$ is the constantly zero function.
\end{example}
\begin{example}
	If $A\subseteq X$ is closed, then $\overline A=A$ by \autoref{ex:closureofclosed}, so $d_A^{-1}(\{0\})=A$. In other words, we have $x\in A$ if and only if $d_A(x)=0$.
\end{example}
Let's now show \autoref{thm:ury} for metric spaces.
\begin{proposition} \label{prop:metricury}
	Fix a metric space $(X,d)$. For any disjoint closed subsets $V_0,V_1\subseteq X$, there is a continuous function $f\colon X\to[0,1]$ such that $f(V_0)=\{0\}$ and $f(V_1)=\{1\}$.
\end{proposition}
\begin{proof}
	The point is to use the Lipschitz continuous functions $d_{V_0},d_{V_1}$. Then we define
	\[f(x)\coloneqq\frac{d_{V_0}(x)}{d_{V_0}(x)+d_{V_1}(x)}.\]
	Note that defining $f\colon X\to\RR$ does not have division-by-zero problems: because $d_{V_0}(x),d_{V_1}(x)\ge0$, the only way to get zero in the denominator is by $d_{V_0}(x)=d_{V_1}(x)=0$. However, this forces $x\in V_0\cap V_1$ by \autoref{lem:zerolocusofda} because $V_0$ and $V_1$ are closed, but in fact $V_0\cap V_1=\emp$.
	
	We now run our checks on $f$.
	\begin{itemize}
		\item Because the quotient of two continuous functions is still continuous, we see that $f$ is continuous.
		\item Using the fact that $d_A(x)\ge0$ for any nonempty $A\subseteq X$ and $x\in X$, we find
		\[f(x)=\frac{d_{V_0}(x)}{d_{V_0}(x)+d_{V_1}(x)}\ge0,\]
		and
		\[f(x)=1-\frac{d_{V_1}(x)}{d_{V_0}(x)+d_{V_1}(x)}\le1,\]
		so $\im f\subseteq[0,1]$.
		\item If $x\in V_0$, then $d_{V_0}(x)=0$, so $f(x)=0/(0+d_{V_1}(x))=0$. If $x\in V_1$, then $d_{V_1}(x)=0$, so $f(x)=d_{V_0}(x)/(d_{V_0}(x)+0)=1$.
		\qedhere
	\end{itemize}
\end{proof}
And here is our check.
\begin{corollary}
	Any metric space $(X,d)$ is normal.
\end{corollary}
\begin{proof}
	Plug \autoref{prop:metricury} into \autoref{rem:uryconv}.
\end{proof}

\subsection{Urysohn's Lemma: The General Case}
We will not prove the general case of \autoref{thm:ury} today, but we will make some progress. Here is a useful lemma.
\begin{lemma} \label{lem:usenormal}
	Fix a normal topological space $(X,\mc T)$. Given a closed subset $V\subseteq X$ and an open subset $U_0\subseteq X$ with $V\subseteq U_0$, there is an open set $U$ such that
	\[V\subseteq U\subseteq\overline U\subseteq U_0.\]
\end{lemma}
\begin{proof}
	Because $V\subseteq U_0$, we define $V'\coloneqq X\setminus U_0$, which is closed because $U_0$ is open. Further, $V'\subseteq X\setminus U_0\subseteq X\setminus V$ forces $V\cap V'=\emp$. Thus, using the normality of $(X,\mc T)$, we are promised disjoint open sets $U$ and $U'$ such that
	\[V\subseteq U\qquad\text{and}\qquad V'\subseteq U'.\]
	In particular, we see that
	\[U\subseteq X\setminus U'\]
	while $X\setminus U'$ is closed by definition. Thus, by definition of the closure, $\overline U\subseteq X\setminus U'\subseteq X\setminus V'=U_0$. This finishes the proof.
\end{proof}

\end{document}