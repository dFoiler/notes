% !TEX root = ../notes.tex

\documentclass[../notes.tex]{subfiles}

\begin{document}

\section{September 2}

There are no questions about anything.

\subsection{Closed Sets}
We begin, as always, with a definition.
\begin{definition}[Closed]
	Fix a topological space $(X,\mc T)$. A subset $V\subseteq X$ is \textit{closed} if and only if $(X\setminus V)\in\mc T$.
\end{definition}
Here are some basic properties.
\begin{lemma} \label{lem:closedtop}
	Fix a topological space $(X,\mc T)$.
	\begin{listalph}
		\item The set $\emp$ and $X$ are both closed.
		\item Arbitrary intersection: given a collection of closed sets $\mc V$, the intersection $\bigcap_{V\in\mc V}V$ is closed.
		\item Finite union: given a finite collection of closed sets $\{V_1,\ldots,V_n\}$, the union $\bigcup_{i=1}^nV_i$ is closed.
	\end{listalph}
\end{lemma}
\begin{proof}
	We proceed in sequence.
	\begin{listalph}
		\item Note that $X\setminus\emp=X$ and $X\setminus X=\emp$ are both open so $\emp$ and $X$ are closed.
		\item Arbitrary intersection: observe that
		\[X\mathbin{\bigg\backslash}\bigcap_{V\in\mc V}V=\bigcup_{V\in\mc V}(X\setminus V)\]
		is an arbitrary union of open sets and therefore open. Thus, $\bigcap_{V\in\mc V}V$ is closed.
		\item Finite union: observe that
		\[X\mathbin{\bigg\backslash}\bigcup_{i=1}^nV_i=\bigcap_{i=1}^n(X\setminus V_i)\]
		is the finite intersection of open sets and therefore open. Thus, $\bigcup_{i=1}^nV_i$ is closed.
		\qedhere
	\end{listalph}
\end{proof}
\begin{remark}
	Observe that both $X$ and $\emp$ are both open and closed. This is allowed.
\end{remark}
\begin{example}
	Fix a metric space $(X,d)$. Then any closed ball $\overline{B(x_0,r)}$ is closed: we need to show
	\[U\coloneqq X\setminus\overline{B(x_0,r)}=\{x\in X:d(x,x_0)>r\}\]
	is open. Well, for any $y\in U$, we see $d(y,x_0)>r$, so set $\varepsilon_y\coloneqq d(y,x_0)-r$, so $y'\in B(y,\varepsilon_y)$ has $d(x_0,y')\ge d(x_0,y)-d(y,y')>r$. Thus, any $y\in U$ has $B(y,\varepsilon_y)\subseteq U$, finishing.
\end{example}
\begin{remark}
	In $\RR^2$ with the Euclidean metric,
	\[\bigcup_{\varepsilon<1}^\infty\overline{B(0,\varepsilon)}=\left\{x\in\RR^2:d(0,x)<\varepsilon\text{ for some }\varepsilon<1\right\}=B(0,1)\]
	is not closed. Indeed, we need to show $U\coloneqq X\setminus B(0,1)=\left\{x\in\RR^2:d(0,x)\ge1\right\}$ is not open. Well, note $(1,0)\in U$, but any $\varepsilon>0$ has $(1-\varepsilon/2,0)\in B((1,0),\varepsilon)$ despite $(1-\varepsilon/2,0)\notin U$. Thus, $U$ is not open.
\end{remark}
\begin{remark}
	One can define a topology by defining its closed sets to satisfy the axioms of \autoref{lem:closedtop}. Then one defines the open sets as the complements of open sets.
\end{remark}
Given a general set, we can define the closure as follows.
\begin{definition}[Closure]
	Fix a topological space $(X,\mc T)$. Given a subset $S\subseteq X$, we define the \textit{closure} as
	\[\overline S\coloneqq\bigcap_{\substack{V\supseteq S\\V\text{ closed}}}V.\]
\end{definition}
\begin{lemma}
	Fix a topological space $(X,\mc T)$. Given a subset $S\subseteq X$, the closure $\overline S$ is the unique smallest closed set containing $S$.
\end{lemma}
\begin{proof}
	Note that
	\[\overline S\coloneqq\bigcap_{\substack{V\supseteq S\\V\text{ closed}}}V\]
	is closed as the arbitrary intersection of closed sets, by \autoref{lem:closedtop}. To see that $\overline S$ is a minimal such closed set, note that any closed $V$ containing $S$ must have $\overline S\subseteq V$ by definition of $\overline S$.
	
	Lastly, to see that $\overline S$ is unique, note that if we have two minimal closed sets $\overline S_1$ and $\overline S_2$ containing $S$, then note $\overline S_1\cap\overline S_2$ are both closed sets containing $S$ by \autoref{lem:closedtop}, so minimality forces $\overline S_1=\overline S_1\cap\overline S_2=\overline S_2$.
\end{proof}
We can move our notion of density from metric spaces to general topology.
\begin{defi}[Dense]
	Fix a topological space $(X,\mc T)$. Given subsets $A\subseteq B$, we say $A$ is \textit{dense} in $B$ if and only if $B\subseteq\overline A$.\todo{Check this is correct for metric spaces}
\end{defi}
\begin{remark}
	We are not requiring that $B$ be closed for the definition of density. For example, $\QQ\subseteq\RR$ is dense in $\QQ$.
\end{remark}

\subsection{The Product Topology}
Let's see more examples of induced topologies. We start with the easiest example of the product topology.
\begin{defihelper}[Product topology]
	Fix a finite collection of topological spaces $(X_1,\mc T_1)$ and $(X_2,\mc T_2)$. The \textit{product topology} on $X_1\times X_2$ is the topology induced by the canonical projection mappings
	\[\pi_1\colon X_1\times X_2\to X_1\qquad\text{and}\qquad\pi_2\colon X_1\times X_2\to X_2.\]
\end{defihelper}
We now give the following more concrete description of the product topology.
\begin{lemma} \label{lem:twoprodtop}
	Fix topological spaces $(X_1,\mc T_1)$ and $(X_2,\mc T_2)$. The product topology $\mc T$ on $X\coloneqq X_1\times X_2$ has a base given by
	\[\mc B\coloneqq\{U_1\times U_2:U_1\in\mc T_1,U_2\in\mc T_2\}.\]
\end{lemma}
\begin{proof}
	The product topology is the minimal topology making $\pi_1\colon X_1\times X_2\to X_1$ and $\pi_2\colon X_1\times X_2\to X_2$ continuous. Namely, the product topology has a sub-base given by the sets
	\[\pi_1^{-1}(U_1)=U_1\times X_2\qquad\text{and}\qquad\pi_2^{-1}(U_2)=X_1\times U_2\]
	for any $U_1\in\mc T_1$ and $U_2\in\mc T_2$. Using \autoref{ex:fininterisbase}, we let $\mc I$ denote the finite intersections of these open sets and note $\mc I$ is a base for our topology.
	
	Now, we finish by claiming $\mc B=\mc I$. On one hand, any $U_1\times U_2\in\mc B$ with $U_1\in\mc T_1$ and $U_2\in\mc T_2$ can be written as the finite intersection
	\[U_1\times U_2=(U_1\times X_2)\cap(X_1\times U_2)=\pi_1^{-1}(U_1)\cap\pi_2^{-1}(U_2)\in\mc I.\]
	On the other hand, pick finitely many sets of the form $\pi_1^{-1}(U_1)$ and $\pi_2^{-1}(U_2)$; dividing them into their classes, we can write our finite collection of sets as in $\{U_1^{(i)}\times X_2\}_{i=1}^m$ or $\{X_1\times U_2^{(j)}\}_{j=1}^n$. Their intersection is
	\[\Bigg(\bigcap_{i=1}^mU_1^{(i)}\times X_2\Bigg)\cap\Bigg(\bigcap_{j=1}^nX_1\times U_2^{(j)}\Bigg)=\underbrace{\Bigg(\bigcap_{i=1}^mU_1^{(i)}\Bigg)}_{U_1\coloneqq}\cap\underbrace{\Bigg(\bigcap_{j=1}^nU_2^{(j)}\Bigg)}_{U_2\coloneqq}.\]
	Now, $U_1\subseteq X_1$ and $U_2\subseteq X_2$ are finite intersection of open sets and therefore open, so our finite intersection takes the form $U_1\times U_2$ and thus lives in $\mc B$.
\end{proof}
\begin{remark}
	Later in life we will discuss measurable sets, which are not quite topologies but will have similar ideas in spirit. For example, they will also care deeply about ``rectangles.''
\end{remark}
We can define this more generally.
\begin{defi}[Product topology]
	Fix a collection of topological spaces $\{(X_\alpha,\mc T_\alpha)\}_{\alpha\in\lambda}$. The \textit{product topology} on $X\coloneqq\prod_{\alpha\in\lambda}X_\alpha$ is induced by the canonical projection maps
	\[\pi_\alpha\colon X\to X_\alpha.\]
\end{defi}
Here is our more concrete description.
\begin{lemma} \label{lem:prodtopbase}
	Fix a collection of topological spaces $\{(X_\alpha,\mc T_\alpha)\}_{\alpha\in\lambda}$. The product topology on $X\coloneqq\prod_{\alpha\in\lambda}$ has a base
	\[\mc B\coloneqq\Bigg\{\prod_{\alpha\in\lambda}U_\alpha:U_\alpha\in\mc T_\alpha,U_\alpha=X_\alpha\text{ for all but finitely many }\alpha\Bigg\}.\]
\end{lemma}
\begin{proof}
	We are immediately given the sub-base of $\mc S\coloneqq\{\pi_\alpha^{-1}(U_\alpha):U_\alpha\in\mc T_\alpha\}$. Using \autoref{ex:fininterisbase}, we let $\mc I$ denote the finite intersections of $\mc S$ so that $\mc I$ is a base for our product topology.
	
	As before, we finish by claiming $\mc I=\mc B$. To stay organized, we proceed in steps.
	\begin{itemize}
		\item We show $\mc B\subseteq\mc I$. Namely, for any $\prod_{\alpha\in\lambda}U_\alpha$ in $\mc B$, we set $\lambda'\coloneqq\{\alpha:U_\alpha\ne X_\alpha\}$, which we know must be finite. Then
		\[\prod_{\alpha\in\lambda}U_\alpha=\bigcap_{\alpha\in\lambda}\pi^{-1}(U_\alpha)=\bigcap_{\alpha\in\lambda'}\pi_\alpha^{-1}(U_\alpha)\]
		because $\pi^{-1}(X_\alpha)=X$. The right-hand side is indeed a finite intersection of elements of $\mc S$ and therefore in $\mc I$.
		\item We show $\mc S\subseteq\mc B$. For a given $\beta$ and $U_\beta\in\mc T_\beta$, set $U_\alpha\coloneqq X_\alpha$ for each $\alpha\ne\beta$. Then we see that
		\[\pi_\beta^{-1}(U_\beta)=\prod_{\alpha\in\lambda}U_\alpha\]
		is in $\mc B$ because $U_\alpha=X_\alpha$ for all but a single $\alpha\in\lambda$.
		\item We show $\mc B$ is closed under finite intersection. By induction, it suffices to pick up $U,U'\in\mc B$ and show that $U\cap U'\in\mc B$. Indeed, write
		\[U=\prod_{\alpha\in\lambda}U_\alpha\qquad\text{and}\qquad U'=\prod_{\alpha\in\lambda}U'_\alpha,\]
		where $\lambda_0=\{\alpha:U_\alpha\ne X_\alpha\}$ and $\lambda_0'=\{\alpha:U'_\alpha\ne X_\alpha\}$ are both finite. Then
		\[U\cap U'=\prod_{\alpha\in\lambda}(U_\alpha\cap U'_\alpha),\]
		and we have $U_\alpha\cap U'_\alpha=X_\alpha$ whenever $\alpha\notin(\lambda_0\cup\lambda_0')$, which is only finitely many exceptions because both $\lambda_0$ and $\lambda_0$ are finite.
		\item We show $\mc I\subseteq\mc B$. Indeed, $\mc I$ is made of the finite intersections of $\mc S$, and we see that $\mc B$ does indeed contain the finite intersections of $\mc S$ because $\mc B$ contains the finite intersections of itself, and $\mc S\subseteq\mc B$.
		\qedhere
	\end{itemize}
\end{proof}
\begin{remark}
	If $\lambda$ is finite, then the arguments of \autoref{lem:twoprodtop} generalize to give the cleaner base
	\[\Bigg\{\prod_{\alpha\in\lambda}U_\alpha:U_\alpha\in\mc T_\alpha\Bigg\}.\]
	This also follows directly from \autoref{lem:prodtopbase}, where we note that the ``finitely many exceptions'' actually permits all $\alpha\in\lambda$ to be an exception because $\lambda$ is finite.
\end{remark}
% \begin{example}
% 	The open set $(0,1)^\NN$ is not open in $\RR^\NN$, where we have given $\RR^\NN$ the product topology.
% \end{example}
\begin{example}
	Give $\{0,1\}$ the discrete topology. Then the space $X\coloneqq\{0,1\}^\NN$ given the product topology does not have
	\[U\coloneqq\prod_{n\in\NN}\{0\}\]
	open in $X$ even though $\{0\}\subseteq\{0,1\}$ is always open. To see this, we note $U$ has only a single element. On the other hand, for $U$ to be open, \autoref{lem:prodtopbase} tells us $U$ must contain a basis element $B$ of the form
	\[B\coloneqq\prod_{n\in\NN}U_n\]
	where $U_n=\{0,1\}$ for all but finitely many $n$. However, $B$ is infinite as the infinite product of sets containing more than $1$ element, so $B\not\subseteq U$.
\end{example}

\subsection{Comments on the Dual Space}
Given a vector space $V$ with a norm $\norm\cdot$, we might be interested in the linear functionals on $V$, but because $V$ is a metric space, we should actually be looking at the continuous linear functional. One can show (in Math 202B) that one has ``plenty'' of continuous linear functionals. Here is a lemma we will use a few times.
\begin{lemma} \label{lem:boundediffcont}
	Let $\norm\cdot$ be a norm on an $\RR$-vector space $V$. Then a linear functional $f\colon V\to\RR$ is continuous if and only if there exists a real number $c>0$ such that
	\begin{equation}
		|f(v)|\le c\norm v \label{eq:fisbounded}
	\end{equation}
	for all $v\in V$.
\end{lemma}
\begin{proof}
	In one direction, suppose that we can find a real number $c>0$ satisfying \autoref{eq:fisbounded} for all $v\in V$. To show $f$ is continuous, we use \autoref{lem:continuityconverse}: suppose that we have a sequence $\{v_n\}_{n\in\NN}$ such that $v_n\to v$ as $n\to\infty$. Then, for any $\varepsilon>0$, find $N$ such that $n>N$ implies
	\[\norm{v-v_n}<\varepsilon/c\]
	so that
	\[|f(v)-f(v_n)|\le c\norm{v-v_n}<\varepsilon.\]
	Conversely, suppose that $f$ is continuous. Note that we don't have to worry about $v=0$ because this gives equality. Now, we can find $\delta>0$ such that $\norm{v}<\delta$ implies $|f(v)|<1$. It follows that any nonzero $v\in V$ will have
	\[\norm{\frac\delta{2\norm v}v}<\delta,\]
	so we see
	\[|f(v)|=\frac{2\norm v}\delta\left|f\left(\frac\delta{2\norm v}v\right)\right|\le\frac2\delta\cdot\norm v,\]
	so $c\coloneqq2/\delta$ will do the trick.
\end{proof}
Here is an example.
\begin{exe}
	Give $V\coloneqq C([0,1])$ a $p$-norm $\norm{\cdot}_p$ for some $p\ge1$ or $p=\infty$. Then $g\in C([0,1])$ defines a continuous linear functional
	\[\varphi_g\colon f\mapsto\int_0^1f(t)g(t)\,dt.\]
\end{exe}
\begin{proof}
	To show $\varphi_g$ is linear, pick up any $r_1,r_2\in\RR$ and $f_1,f_2\in V$; then
	\[\varphi_g(r_1f_1+r_2f_2)=\int_0^1(r_1f_1+r_2f_2)(t)g(t)\,dt=r_1\int_0^1f_1(t)g(t)\,dt+r_2\int_0^1f_2(t)g(t)\,dt=r_1\varphi_g(f_1)+r_2\varphi_g(f_2).\]
	Checking continuity is a little more involved. Note $|g|$ is a continuous function on a compact set $[0,1]$ and therefore has a maximum $M$. We now use \autoref{lem:boundediffcont}; we have two cases.
	\begin{itemize}
		\item Suppose $p=\infty$. Then, for any $f\in V$, we see
		\[|\varphi_g(f)|=\left|\int_0^1f(t)g(t)\,dt\right|\le M\int_0^1|f(t)|\,dt\le M\norm f_\infty,\]
		which finishes by \autoref{lem:boundediffcont}.
		\item Suppose $p\ge1$ is finite. To begin, we note
		\[|\varphi_g(f)|=\left|\int_0^1f(t)g(t)\,dt\right|\le M\int_0^1|f(t)|\,dt.\]
		Now, because the function $x\mapsto x^p$ is convex, we see that
		\[\left(\int_0^1|f(t)|\,dt\right)^p\le\int_0^1|f(t)|^p\,dt=\norm f_p^p,\]
		so $|\varphi_g(f)|\le M\norm f_p$. \autoref{lem:boundediffcont} finishes.
		\qedhere
	\end{itemize}
\end{proof}
Even though the linear functionals we found were continuous for all $\norm\cdot_p$, it is possible to find linear functionals continuous for some of our norms but not others.
\begin{exe}
	Fix $V\coloneqq C([0,1])$, and select some $t_0\in[0,1]$. Then
	\[\varphi\colon f\mapsto f(t_0)\]
	defines a linear functional on $V$ which is continuous for $\norm\cdot_\infty$ but not for $\norm\cdot_p$ for any finite $p\ge1$.
\end{exe}
\begin{proof}
	To see continuity with $\norm\cdot_\infty$, we note that any $f\in V$ has
	\[|\varphi(f)|=|f(t_0)|\le\norm f_\infty,\]
	so \autoref{lem:boundediffcont} finishes.

	We now show that $\varphi$ is not continuous for a fixed $\norm\cdot_p$, where $p\ge1$ is finite. Using \autoref{lem:boundediffcont}, we just have to show that the ratio $|\varphi(v)|/\norm v_p$ is unbounded for $v\in V$. For this, we define $f_c\colon[0,1]\to\RR$ by
	\[f(t)\coloneqq\max\left\{0,c-c^{2p+1}(t-t_0)^2\right\}.\]
	The idea here is that $f$ has a sharp bump at $t_0$. Now, $f$ is a continuous function on $[0,1]$ because it is the composition of continuous functions, so $f\in V$. We can compute
	\[\norm f_p=\bigg(\int_0^1|f(t)|^p\,dt\bigg)^{1/p}.\]
	Now, $f(t)$ will only be nonzero when $c-c^{2p+1}(t-t_0)^2\ge0$, which is equivalent to $t-t_0\in\left(-c^{-p},c^{-p}\right)$, so we bound
	\[\norm f_p^p=\int_0^1|f(t)|^p\,dt\le\int_{-c^{-p}}^{c^{-p}}\left(c-c^{2p+1}z^2\right)\,dz\le2c^{1-p}.\]
	Notably, as $c\to\infty$, we have that $\norm f_p\le2^{1/p}\cdot c^{1/p-1}$ is bounded, but $|\varphi(f)|=c$ grows unbounded. Thus, $\varphi$ is discontinuous.
\end{proof}
\begin{remark}
	Now, we have exhibited many continuous functions
	\[\varphi_g\colon C([0,1])\to\RR,\]
	so we can ask for the topology on $C([0,1])$ induced by these. It turns out that this induced topology is much weaker than any individual norm topology; this topology is often called the weak topology determined by $C([0,1])$.
\end{remark}
\begin{remark}
	By the end of the class, we will have a reasonable notion of the dual space of $\norm{\cdot}_1$ and $\norm{\cdot}_2$. The dual space for $\norm{\cdot}_\infty$ will come up in Math 202B.
\end{remark}
\begin{remark}
	Still working with $C([0,1])$ given a specific norm $\norm\cdot_p$, one can show that any $g\in C([0,1])$ has some $r_g\in\RR$ with
	\[\varphi_g(B(0,1))\subseteq B(0,r_g).\]
	It turns out to be helpful to be able to consider the product topology on the (very large) product
	\[\prod_{g\in C([0,1])}B(0,r_g).\]
\end{remark}

\end{document}