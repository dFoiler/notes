% !TEX root = ../notes.tex

\documentclass[../notes.tex]{subfiles}

\begin{document}

\section{September 7}

It's another day of sun.

\subsection{Quotient Spaces}
Here is a different way to induce a topology, the reverse of the induced topology.
\begin{definition}[Final topology]
	Fix a set $Y$ and some topological spaces $\{(X_\alpha,\mc T_\alpha)\}_{\alpha\in\lambda}$. Given functions $f_\alpha\colon X_\alpha\to Y$, we define the \textit{final topology} on $Y$ to be the ``strongest'' (i.e., with the most open sets) making the $f_\alpha$ continuous.
\end{definition}
\begin{remark}
	Note that certainly some topology on $Y$ exists making the $f_\alpha$ continuous because we can give $Y$ the indiscrete topology, where $f_\alpha^{-1}(\emp)=\emp$ and $f_\alpha^{-1}(Y)=X_\alpha$ are open for each $\alpha\in\lambda$.
\end{remark}
Here is a more concrete description.
\begin{lemma}
	Fix a set $Y$ and some topological spaces $\{(X_\alpha,\mc T_\alpha)\}_{\alpha\in\lambda}$, with functions $f_\alpha\colon X_\alpha\to Y$. Then the final topology is
	\[\mc T\coloneqq\bigcap_{\alpha\in\lambda}\left\{S\subseteq Y:f^{-1}_\alpha(S)\in\mc T_\alpha\right\}.\]
\end{lemma}
\begin{proof}
	Certainly each $\left\{S\subseteq Y:f^{-1}_\alpha(S)\in\mc T_\alpha\right\}$ is a topology by \autoref{lem:topoindonefunc}, as is their intersection by \autoref{prop:intersecttops}. Thus, $\mc T$ is a topology.
	
	It remains to show that $\mc T$ is the strongest topology making each of the $f_\alpha$ continuous. Well, suppose $\mc T'$ is a topology making each of the $f_\alpha$ continuous. Then, for each $U\in\mc T'$, we have
	\[f_\alpha^{-1}(U)\in\mc T_\alpha\text{ for each }\alpha\in\lambda,\]
	so $U\in\mc T$ follows. Thus, $\mc T'\subseteq\mc T$.
\end{proof}
We will be primarily interested in the case with just one function.
\begin{remark}
	In the case of one function, which is \autoref{lem:topoindonefunc}, note that we might as well assume that $f\colon X\to Y$ is onto for otherwise we might as well just pass to the relative topology on $\im f$. To be explicit, we see $U\subseteq Y$ is open if and only if $f^{-1}(U)$ is open if and only if $f^{-1}(U\cap\im f)$ is open if and only if $U\cap\im f$ is open.
\end{remark}
We are now ready to define the quotient space.
\begin{lemma}
	Given sets $f\colon X\to Y$, there is an equivalence relation $\sim$ on $X$ with $x\sim x'$ if and only if $f(x)=f(x')$.
\end{lemma}
\begin{proof}
	We check the conditions one at a time. Find $x,x',x''\in X$.
	\begin{itemize}
		\item Reflexive: note $f(x)=f(x)$, so $x\sim x$.
		\item Symmetric: if $x\sim x'$, then $f(x)=f(x')$, so $f(x')=f(x)$, so $x'\sim x$.
		\item Transitive: if $x\sim x'$ and $x'\sim x''$, then $f(x)=f(x')=f(x'')$, so $f(x)=f(x'')$, so $x\sim x''$.
		\qedhere
	\end{itemize}
\end{proof}
With an equivalence relation, we may consider the set of equivalence classes $X/{\sim}$.
\begin{remark}
	Conversely, given some partition $P\subseteq\mc P(X)$ of $X$, we can define $f\colon X\to P$ by $f\colon x\mapsto[x]$, where $[x]\in P$ is the element of $P$ containing $x$. (Note $[x]\in P$ exists and is well-defined because $P$ is a partition.) The point is that surjective functions give rise to equivalence relations, and equivalence relations give rise to surjective functions.
\end{remark}
Anyway, here is our definition.
\begin{definition}[Quotient topology]
	Fix an equivalence relation $\sim$ on a set $X$ with a topology $\mc T$. Then the \textit{quotient topology} on $X/{\sim}$ is the final topology for the natural projection $X\onto X/{\sim}$.
\end{definition}
It turns out that we can talk about the quotient space by universal property as well.
\begin{proposition} \label{prop:quotientup}
	Fix an equivalence relation $\sim$ on a set $X$ with a topology $\mc T$; let $\pi\colon X\onto(X/{\sim})$ be the natural projection. Then, for any continuous map $f\colon X\to Z$ such that any $x\sim x'$ has $f(x)=f(x')$, there is a unique continuous map $\overline f\colon(X/{\sim})\to Z$ such that
	\[f=\overline f\circ\pi.\]
\end{proposition}
\begin{proof}
	We show uniqueness and existence separately.
	\begin{itemize}
		\item Uniqueness: for any $[x]\in(X/{\sim})$, we see that we must have
		\[\overline f([x])=\overline f(\pi(x))=f(x),\]
		so $\overline f([x])$ is forced by our other data.
		\item Existence: for each $[x]\in(X/{\sim})$, define $\overline f([x])\coloneqq f(x)$. Note that this is well-defined: if $[x]=[x']$, then $x\sim x'$, so $f(x)=f(x')$ by hypothesis.

		It remains to show that $\overline f$ is continuous. Well, for an open set $U\subseteq Z$, we note that
		\[\overline f^{-1}(U)=\{[x]:\overline f([x])\in U\}=\{[x]:f(x)\in U\}=\pi\left(f^{-1}(U)\right).\]
		Now, $\pi^{-1}\left(\pi\left(f^{-1}(U)\right)\right)=f^{-1}(U)$ because $x\in\pi^{-1}\left(\pi\left(f^{-1}(U)\right)\right)$ if and only if $\pi(x)\in\pi\left(f^{-1}(U)\right)$, which is equivalent to there being $x'\in f^{-1}(U)$ with $\pi(x)=\pi(x')$, which is equivalent to there being $x'$ with $x\sim x'$ while $f(x)=f(x')\in U$.
		
		Thus, $\pi^{-1}\left(\pi\left(f^{-1}(U)\right)\right)$ is open, so it follows $\pi\left(f^{-1}(U)\right)\subseteq(X/{\sim})$ is open.
		\qedhere
	\end{itemize}
\end{proof}

\subsection{Homeomorphism}
Homeomorphisms are isomorphisms in our category $\mathrm{Top}$. To be technical, here is our definition.
\begin{definition}[Homeomorphism]
	A function $f\colon X\to Y$ between topological spaces $(X,\mc T_X)$ and $(Y,\mc T_Y)$ is a \textit{homeomorphism} if and only if $f$ is continuous and has a continuous inverse. Formally, we require a continuous map $g\colon Y\to X$ such that
	\[f\circ g={\id_Y}\qquad\text{and}\qquad g\circ f={\id_X}.\]
\end{definition}
\begin{warn}
	It is not enough for $f$ to be continuous and bijective to be a homeomorphism. The hypothesis that the inverse function be continuous is necessary.
\end{warn}
\begin{remark}
	The definition above does not require that $f$ be bijective, but this follows from $f$ having an inverse.
\end{remark}
\begin{example}
	Give $\RR$ the Euclidean topology, and let $\RR_d$ be the real numbers with the discrete topology. Then the identity function $\iota\colon\RR_d\to\RR$ is continuous because all functions from the discrete topology are continuous. However, $\iota$ is its own inverse, and the inverse function
	\[\pi\colon\RR\to\RR_d\]
	(which is also the identity on $\RR$) is not continuous. For example, $\pi^{-1}(\{0\})=\{0\}$ is not open in $\RR$ (by \autoref{rem:singletonnotopen}) even though $\RR\setminus\{0\}\subseteq\RR_d$ is open.
\end{example}
Here are some actual examples.
\begin{exe}
	Give $X\coloneqq[0,1]$ the subspace topology, and define the equivalence relation $\sim$ as having equivalence classes $\{0,1\}$ and $\{r\}$ for each $r\in(0,1)$. Then the quotient topology $X/{\sim}$ is homeomorphic to $S^1\subseteq\CC$.
\end{exe}
\begin{proof}
	We note that $\sim$ is an equivalence relation because its equivalence classes are a partition. Now, we define the maps
	\[\arraycolsep=1.4pt\begin{array}{ccc}
		(X/{\sim}) &\cong& S^1 \\
		t &\mapsto& e^{2\pi it} \\
		\theta/2\pi &\mapsfrom& e^{i\theta}
	\end{array}\]
	which we can see to be well-defined inverse. Note that $\RR\to\CC$ by $t\mapsto e^{it}$ is continuous by complex analysis (it's in fact holomorphic). Restricting, we get the continuous map $[0,1]\to S^1$, and then we can see that we can mod out by $0\sim1$ because they both go to the same place (using \autoref{prop:quotientup}). One can check by hand that the inverse map is continuous, but we won't bother.
	% To see that $f$ is continuous, we use \autoref{prop:checkonsubbase}. Well, given an open set $U\subseteq S^1$, we need to show
	% \[f^{-1}(U)\subseteq X/{\sim}\]
	% is open for each $z\in\CC$ and $r>0$. Because $S^1$ has the subspace topology, we may write $U=V\cap S^1$ for some open $V\subseteq\CC$.
	% Now, if we have some $t\in(X/{\sim})$ such that $f(t)\in V$, then the fact that $V\subseteq\CC$ is open promises some $\varepsilon>0$ with $B(f(t),\varepsilon)\subseteq U$. 
\end{proof}
\begin{remark}[Nir]
	Here is a quick way to see that the inverse map is continuous: any continuous bijection $f\colon(X/{\sim})\to S^1$ with $(X/{\sim})$ compact---which is true because $X$ is compact---and $S^1$ Hausdorff will send closed subsets $V\subseteq(X/{\sim})$ (which are compact) to compact subsets of $S^1$ (which are closed). Thus, $f$ is a closed map, so its inverse is continuous because $f$ is bijective.
\end{remark}
For the next few examples, we won't be very rigorous because we haven't provided good definitions of the relevant spaces.
\begin{ex} \label{ex:cylinder}
	Give $X\coloneqq[0,2]\times[0,1]$ the subspace topology, and define the equivalence relation $\sim$ as requiring $(0,r)\sim(2,r)$ only. Then $X$ is homeomorphic to a circle by gluing its edges. One might draw $X$ as follows.
	\begin{center}
		\begin{asy}
			unitsize(1.0cm);
			draw((0,0)--(0,1)--(2,1)--(2,0)--cycle);
			draw((0,0)--(0,0.65), arrow=EndArrow);
			draw((2,0)--(2,0.65), arrow=EndArrow);
		\end{asy}
		% \begin{asy}
		% 	import three;
		% 	unitsize(1.0cm);
		% 	//create segment
		% 	path3 segment = (0,1,0) -- (0,1,1);
		% 	//create surface of revolution
		% 	surface lampshade = surface(segment, c=O, axis=Z);
		% 	//draw surface
		% 	draw(lampshade, gray);
		% 	draw(segment, black);
		% \end{asy}
	\end{center}
\end{ex}
\begin{example}
	Continuing with the drawing style of \autoref{ex:cylinder}, we have that
	\begin{center}
		\begin{asy}
			unitsize(1.0cm);
			draw((0,0)--(0,1)--(2,1)--(2,0)--cycle);
			draw((0,0)--(0,0.65), arrow=EndArrow);
			draw((2,1)--(2,0.35), arrow=EndArrow);
		\end{asy}
	\end{center}
	is the M\"obius strip.
\end{example}
\begin{remark}
	Note that these homeomorphisms do not care for the metric of our spaces. All that matters is the continuity.
\end{remark}
\begin{example}
	Let $X$ be the unit sphere in $\RR^3$ with the subspace topology, and define the equivalence relation on $X$ by equivalence classes $\{v,-v\}$ for each $v\in X$. Then $X/{\sim}$ turns out to be $\mathbb{RP}^2$, which is hard to visualize.
\end{example}

\subsection{Group Actions}
A space might even have interesting homeomorphisms to itself.
\begin{example}
	Fix a real number $\theta$. The circle $S^1$ in $\CC$ (given the subspace topology) has the rotation homeomorphism
	\[r_\theta\colon e^{it}\mapsto e^{i(t+\theta)}.\]
\end{example}
\begin{remark}
	In general, given a topological space $(X,\mc T)$, we can make the group of homeomorphisms $\op{Aut}(X)$ of homeomorphisms whose operation is composition.
\end{remark}
This gives the following definition.
\begin{definition}[Group action]
	A \textit{group action} by a group $G$ on a topological space $X$ is a group homomorphism
	\[\varphi_\bullet\colon G\to\op{Aut}(X).\]
\end{definition}
\begin{example}
	The group $\langle\sigma\rangle\simeq\ZZ/2\ZZ$ acts on a normed vector space $(V,\norm\cdot)$ by sending $\sigma^k$ to
	\[\varphi_{\sigma^k}\cdot v\coloneqq(-1)^kv.\]
	Notably, $\varphi_{\sigma^k}$ is continuous and its own inverse for any $k$, so it is a homeomorphism. In fact, we can see directly that $\varphi_{\sigma^k}\circ\varphi_{\sigma^\ell}=\varphi_{\sigma^{k+\ell}}$.
\end{example}
Notably, with a group action comes a partition.
\begin{definition}[Orbit]
	Let $G$ act on a topological space $X$ by $\varphi_\bullet\colon G\to\op{Aut}(X)$. Then the \textit{$G$-orbit} $Gx$ of a point $x\in x$ is the set
	\[Gx\coloneqq\{\varphi_g(x):g\in G\}.\]
	We denote the set of all orbits $\mc O_x$ be $X/G$.
\end{definition}
\begin{remark}
	Note that the map $x\mapsto\mc O_x$ is a well-defined (surjective) map $X\to X/G$. In particular, we need to know that $x\in\mc O_{x'}$ implies that $\mc O_x=\mc O_{x'}$ so that there is exactly one orbit containing $x$. Well, $x\in\mc O_{x'}$ means we can find $g_0\in G$ such that $x=\varphi_{g_0}(x')$, so
	\[\mc O_x=\{\varphi_g(x):g\in G\}=\{\varphi_g(\varphi_{g_0}(x')):g\in G\}=\{\varphi_{gg_0}(x'):g\in G\}\subseteq\mc O_{x'}.\]
	Conversely, we note that $x'=\varphi_{g_0^{-1}}(x)$, so $\mc O_{x'}\subseteq\mc O_x$ follows, giving equality.
\end{remark}
Thus, the $G$-orbits partition $X$, so we can give the set $X/G$ the quotient topology as the final topology of the natural projection $X\onto X/G$.

\end{document}