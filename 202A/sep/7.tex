% !TEX root = ../notes.tex

\documentclass[../notes.tex]{subfiles}

\begin{document}

\section{September 7}

It's another day of sun.

\subsection{Quotient Spaces}
Here is a different way to induce a topology, the reverse of the induced topology.
\begin{definition}[Final topology]
	Fix a set $Y$ and some topological spaces $\{(X_\alpha,\mc T_\alpha)\}_{\alpha\in\lambda}$. Given functions $f_\alpha\colon X_\alpha\to Y$, we define the \textit{final topology} on $Y$ to be the ``strongest'' (i.e., with the most open sets) making the $f_\alpha$ continuous.
\end{definition}
\begin{remark}
	Note that certainly some topology on $Y$ exists making the $f_\alpha$ continuous because we can give $Y$ the indiscrete topology, where $f_\alpha^{-1}(\emp)=\emp$ and $f_\alpha^{-1}(Y)=X_\alpha$ are open for each $\alpha\in\lambda$.
\end{remark}
Here is a more concrete description.
\begin{lemma}
	Fix a set $Y$ and some topological spaces $\{(X_\alpha,\mc T_\alpha)\}_{\alpha\in\lambda}$, with functions $f_\alpha\colon X_\alpha\to Y$. Then the final topology is
	\[\mc T\coloneqq\bigcap_{\alpha\in\lambda}\left\{S\subseteq Y:f^{-1}_\alpha(S)\in\mc T_\alpha\right\}.\]
\end{lemma}
\begin{proof}
	Certainly each $\left\{S\subseteq Y:f^{-1}_\alpha(S)\in\mc T_\alpha\right\}$ is a topology by \autoref{lem:topoindonefunc}, as is their intersection by \autoref{prop:intersecttops}. Thus, $\mc T$ is a topology.
	
	It remains to show that $\mc T$ is the strongest topology making each of the $f_\alpha$ continuous. Well, suppose $\mc T'$ is a topology making each of the $f_\alpha$ continuous. Then, for each $U\in\mc T'$, we have
	\[f_\alpha^{-1}(U)\in\mc T_\alpha\text{ for each }\alpha\in\lambda,\]
	so $U\in\mc T$ follows. Thus, $\mc T'\subseteq\mc T$.
\end{proof}
We will be primarily interested in the case with just one function.
\begin{remark}
	In the case of one function, which is \autoref{lem:topoindonefunc}, note that we might as well assume that $f\colon X\to Y$ is onto for otherwise we can just pass to the relative topology.
\end{remark}
We are now ready to define the quotient space.
\begin{lemma}
	Given sets $f\colon X\to Y$, there is an equivalence relation $\sim$ on $X$ with $x\sim x'$ if and only if $f(x)=f(x')$.
\end{lemma}
\begin{proof}
	We check the conditions one at a time. Find $x,x',x''\in X$.
	\begin{itemize}
		\item Reflexive: note $f(x)=f(x)$, so $x\sim x$.
		\item Symmetric: if $x\sim x'$, then $f(x)=f(x')$, so $f(x')=f(x)$, so $x'\sim x$.
		\item Transitive: if $x\sim x'$ and $x'\sim x''$, then $f(x)=f(x')=f(x'')$, so $f(x)=f(x'')$, so $x\sim x''$.
		\qedhere
	\end{itemize}
\end{proof}
With an equivalence relation, we may consider the set of equivalence classes $X/{\sim}$.
\begin{remark}
	Conversely, given some partition $P\subseteq\mc P(X)$ of $X$, we can define $f\colon X\to P$ by $f\colon x\mapsto[x]$, where $[x]\in P$ is the element of $P$ containing $x$. The point is that surjective functions give rise to equivalence relations, and equivalence relations give rise to surjective functions.
\end{remark}
Anyway, here is the point.
\begin{definition}[Quotient topology]
	Fix an equivalence relation $\sim$ on a set $X$ with a topology $\mc T$. Then the \textit{quotient topology} on $X/{\sim}$ is the final topology for the natural projection $X\onto X/{\sim}$.
\end{definition}

\subsection{Homeomorphism}
Homeomorphisms are isomorphisms in our category $\mathrm{Top}$. To be technical, here is our definition.
\begin{definition}[Homeomorphism]
	A function $f\colon X\to Y$ between topological spaces $(X,\mc T_X)$ and $(Y,\mc T_Y)$ is a \textit{homeomorphism} if and only if $f$ is continuous and has a continuous inverse. Formally, we require a continuous map $g\colon Y\to X$ such that
	\[f\circ g={\id_Y}\qquad\text{and}\qquad g\circ f={\id_X}.\]
\end{definition}
\begin{warn}
	It is not enough for $f$ to be continuous and bijective to be a homeomorphism. The hypothesis that the inverse function be continuous is necessary.
\end{warn}
\begin{remark}
	The definition above does not require that $f$ be bijective, but this follows from $f$ having an inverse.
\end{remark}
\begin{example}
	Give $\RR$ the Euclidean topology, and let $\RR_d$ be the real numbers with the discrete topology. Then the identity function $\iota\colon\RR_d\to\RR$ is continuous because all functions from the discrete topology are continuous. However, $\iota$ is its own inverse, and the inverse function
	\[\pi\colon\RR\to\RR_d\]
	(which is also the identity on $\RR$) is not continuous. For example, $\pi^{-1}(\{0\})=\{0\}$ is not open in $\RR$ (by \autoref{rem:singletonnotopen}) even though $\RR\setminus\{0\}\subseteq\RR_d$ is open.
\end{example}
Here are some actual examples.
\begin{exe}
	Give $X\coloneqq[0,1]$ the subspace topology, and define the equivalence relation $\sim$ as having equivalence classes $\{0,1\}$ and $\{r\}$ for each $r\in(0,1)$. Then the quotient topology $X/{\sim}$ is homeomorphic to $S^1\subseteq\CC$.
\end{exe}
\begin{proof}
	Our map $f\colon(X/{\sim})\to S^1$ is by $f\colon t\mapsto e^{2\pi it}$.
\end{proof}
For the next few examples, we won't be very rigorous because we haven't provided good definitions of the relevant spaces.
\begin{ex} \label{ex:cylinder}
	Give $X\coloneqq[0,2]\times[0,1]$ the subspace topology, and define the equivalence relation $\sim$ as requiring $(0,r)\sim(2,r)$ only. Then $X$ is homeomorphic to a circle by gluing its edges. One might draw $X$ as follows.
	\begin{center}
		\begin{asy}
			unitsize(1cm);
			draw((0,0)--(0,1)--(2,1)--(2,0)--cycle);
			draw((0,0)--(0,0.65), arrow=EndArrow);
			draw((2,0)--(2,0.65), arrow=EndArrow);
		\end{asy}
	\end{center}
\end{ex}
\begin{example}
	Continuing with the drawing style of \autoref{ex:cylinder}, we have that
	\begin{center}
		\begin{asy}
			unitsize(1cm);
			draw((0,0)--(0,1)--(2,1)--(2,0)--cycle);
			draw((0,0)--(0,0.65), arrow=EndArrow);
			draw((2,1)--(2,0.35), arrow=EndArrow);
		\end{asy}
	\end{center}
	is the M\"obius strip.
\end{example}
\begin{remark}
	Note that these homeomorphisms do not care for the metric of our spaces. All that matters is the continuity.
\end{remark}
\begin{example}
	Let $X$ be the unit sphere in $\RR^3$ with the subspace topology, and define the equivalence relation on $X$ by $v\sim(-v)$ for each $v\in\RR^3$. Then $X/{\sim}$ turns out to be $\mathbb{RP}^2$, which is hard to visualize.
\end{example}

\subsection{Group Actions}
A space might even have interesting homeomorphisms to itself.
\begin{example}
	Fix a real number $\theta$. The circle $S^1$ in $\CC$ (given the subspace topology) has the rotation homeomorphism
	\[r_\theta\colon e^{it}\mapsto e^{i(t+\theta)}.\]
\end{example}
\begin{remark}
	In general, given a topological space $(X,\mc T)$, we can make the group of homeomorphisms $\op{Aut}(X)$ of homeomorphisms whose operation is composition.
\end{remark}
This gives the following definition.
\begin{definition}[Group action]
	A \textit{group action} by a group $G$ on a topological space $X$ is a group homomorphism
	\[\varphi_\bullet\colon G\to\op{Aut}(X).\]
\end{definition}
\begin{example}
	The group $\langle\sigma\rangle\simeq\ZZ/2\ZZ$ acts on a normed vector space $V$ by
	\[\varphi_\sigma\cdot v\coloneqq-v.\]
\end{example}
Notably, with a group action comes a partition.
\begin{definition}[Orbit]
	Let $G$ act on a topological space $X$ by $\varphi_\bullet\colon G\to\op{Aut}(X)$. Then the \textit{orbit} $Gx$ of a point $x\in x$ is the set
	\[Gx\coloneqq\{\varphi_g(x):g\in G\}.\]
	We denote the set of all orbits $\mc O_x$ be $X/G$.
\end{definition}
It turns out that the orbits partition $X$, so we can give the set $X/G$ the quotient topology as the final topology of the natural projection $X\onto X/G$.
\begin{remark}
	We are about to transition from making topologies to coming up with adjectives which will give ``lots'' of continuous maps to, say, the real numbers. For example, a space $(X,\mc T)$ will be Hausdorff if and only if, for any distinct $x,x'\in X$, there are disjoint open sets $U$ and $U'$ with $x\in U$ and $x'\in U'$. Here is the image.
	\begin{center}
		\begin{asy}
			unitsize(1cm);
			dot((0,0)); dot((3,0));
			draw(circle((0,0),1), dashed); draw(circle((3,0),1),dashed);
		\end{asy}
	\end{center}
	It is easy to check that metric spaces are Hausdorff.
\end{remark}

\end{document}