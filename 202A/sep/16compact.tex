% !TEX root = ../notes.tex

\documentclass[../notes.tex]{subfiles}

\begin{document}

\section{September 16}

We continue the lecture, into compactness.

\subsection{Compactness}
The following is perhaps the most important definition in point-set topology.
\begin{definition}[Open cover]
	Fix a topological space $(X,\mc T)$. An \textit{open cover} of $X$ is a collection $\mc U\subseteq\mc T$ of open sets such that
	\[X=\bigcup_{U\in\mc U}U.\]
\end{definition}
\begin{definition}[Open subcover]
	Fix a topological space $(X,\mc T)$. An \textit{(open) subcover} $\mc U'$ of an open cover $\mc U$ is an open cover $\mc U'$ of $X$ such that $\mc U'\subseteq\mc U$.
\end{definition}
And here is the relevant definition.
\begin{defi}[Compact]
	Fix a topological space $(X,\mc T)$. We say that $(X,\mc T)$ is \textit{compact} if and only if every open cover of $X$ has a finite subcover.
\end{defi}
\begin{example}
	The subset $[0,1]\subseteq\RR$ given the relative topology is compact.
\end{example}
In light of the previous example, it is helpful to extend our definition to subsets of a topological space.
\begin{definition}[Compact]
	Fix a topological space $(X,\mc T)$. A subset $A\subseteq X$ is \textit{compact} if and only if $A$ is compact when given the relative topology from $X$.
\end{definition}
\begin{lemma} \label{lem:bettercompactsubset}
	Fix a topological space $(X,\mc T)$. Then $A$ is compact if and only if any $\mc U\subseteq\mc T$ covering $A$ has a finite subcover covering $A$.
\end{lemma}
\begin{proof}
	The point is to use \autoref{lem:betterrelative}. In one direction, suppose $A$ is compact. Then a cover $\{U_\alpha\}_{\alpha\in\lambda}\subseteq\mc T$ of $A$ provides the open cover by
	\[V_\alpha\coloneqq A\cap U_\alpha\]
	of $A$. Indeed, $A\cap U_\alpha\subseteq A$ is open, and $\bigcup_{\alpha\in\lambda}V_\alpha=\bigcup_{\alpha\in\lambda}(A\cap U_\alpha)=A$. Thus, compactness provides a finite subset $\lambda'\subseteq\lambda$ such that $\{V_\alpha\}_{\alpha\in\lambda'}$ still covers $A$, so
	\[A=\bigcup_{\alpha\in\lambda'}(A\cap U_\alpha)\subseteq\bigcup_{\alpha\in\lambda'}U_\alpha,\]
	meaning that the finite subcover $\{U_\alpha\}_{\alpha\in\lambda'}\subseteq\{U_\alpha\}_{\alpha\in\lambda}$ still covers $A$.
	
	In the other direction, suppose that each open cover of $A$ from $\mc T$ has a finite subcover. Now, give $A$ some open cover $\{V_\alpha\}_{\alpha\in\lambda}$ from the relative topology on $A$. Each open subset $V_\alpha$ can be written as $U_\alpha\cap A$ where $U_\alpha\subseteq X$ is open by \autoref{lem:betterrelative}, so we define
	\[\mc U\coloneqq\{U_\alpha\}_{\alpha\in\lambda}.\]
	Notably, $\bigcup_{\alpha\in\lambda}U_\alpha$ contains $\bigcup_{\alpha\in\lambda}V_\alpha$, which is $A$, so $\mc U$ covers $A$ and hence has a finite subset $\lambda'\subseteq\lambda$ such that $\{U_\alpha\}_{\alpha\in\lambda'}$ covers $A$. But then
	\[A=\bigcup_{\alpha\in\lambda'}(A\cap U_\alpha)=\bigcup_{\alpha\in\lambda'}V_\alpha,\]
	so $\{V_\alpha\}_{\alpha\in\lambda'}$ provides a finite subcover of $\{V_\alpha\}_{\alpha\in\lambda}$.
\end{proof}
In light of the above proof, it will be helpful to extend our notion of an open cover.
\begin{notation}
	Given a topological space $(X,\mc T)$, we will say that some open sets $\mc U\subseteq\mc T$ form an open cover for a subset $A\subseteq X$ if and only if
	\[A\subseteq\bigcup_{U\in\mc U}U.\]
\end{notation}
\begin{remark}
	We will freely use \autoref{lem:bettercompactsubset} as a ``definition'' of compactness without reference.
\end{remark}
\begin{example} \label{ex:finite-union-compacts}
	Given compact subsets $A_1,A_2\subseteq X$ of a topological space $(X,\mc T)$, we see that $A_1\cup A_2$ is also compact. Indeed, given an open cover $\mc U$ of $A_1\cup A_2$, we see that $\mc U$ is an open cover for both $A_1$ and $A_2$, so we can find our finite subcovers $\mc U_1\subseteq\mc U$ and $\mc U_2\subseteq\mc U$ by the compactness of $A_1$ and $A_2$, respectively. Thus, $\mc U_1\cup\mc U_2\subseteq\mc U$ is a finite collection covering $A_1$ and $A_2$ and therefore covering $A_1\cup A_2$.
\end{example}
Here is a quick fact about compactness.
\begin{lemma} \label{lem:closedincompactiscompact}
	Fix a compact topological space $(X,\mc T)$. Then any closed subset $A\subseteq X$ is compact.
\end{lemma}
\begin{proof}
	By \autoref{lem:bettercompactsubset}, pick up an open cover $\mc U$ of $A$, and we would like to find a finite subcover. Then we set
	\[\mc V\coloneqq\mc U\cup\{X\setminus A\}.\]
	Notably, $X\setminus A$ is open in $X$ because $A$ is closed, so we see
	\[\bigcup_{U\in\mc V}U=(X\setminus A)\cup\bigcup_{U\in\mc U}U\supseteq(X\setminus A)\cup A=X,\]
	so $\mc V$ is an open cover for $X$. As such, we can find a finite subcover $\mc V'$ for $X$, and we set $\mc U'\coloneqq\mc V\cap\mc U$.
	
	We claim that $\mc U'$ is a finite subcover of $\mc U$; indeed, $\mc U'\subseteq\mc V$ is finite, and $\mc U'\subseteq\mc U$ is a subset. It remains to check that $\mc U'$ covers $A$. Well, for any $a\in A$, we can find some $U'\in\mc V'$ containing $a$ because $\mc V'$ covers $X$. However, $a\notin X\setminus A$, so $U'\ne X\setminus A$, so actually $U'\in\mc U'$. Thus,
	\[A\subseteq\bigcup_{U\in\mc U'}U,\]
	which is what we wanted.
\end{proof}
\begin{example} \label{ex:compactnotclosed}
	Give $X=\RR$ the indiscrete topology. Then $X$ has only two open sets, so any nonempty subset $S\subseteq X$ can only be covered by $\{X\}$, which is its own finite subcover. For example, $\{0\}$ is compact in $X$, but it is not closed because $\RR\setminus\{0\}$ is not open.
\end{example}
% \begin{remark}
% 	Even though \autoref{ex:compactnotclosed} has showed that compact does not imply closed, this is true in Hausdorff spaces.
% \end{remark}

\end{document}