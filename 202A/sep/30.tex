% !TEX root = ../notes.tex

\documentclass[../notes.tex]{subfiles}

\begin{document}

\section{September 30}

There are no questions.

\subsection{Totally Bounded for Function Spaces}
We continue our discussion of compactness in metric spaces. Fix a topological space $(X,\mc T)$ and a metric space $(M,d)$ so that we can give the space of bounded continuous functions $B_c(X,M)$ the uniform metric $d_u$ by \autoref{prop:contiscomplete}. We would like to understand the compact subset of $B_c(X,M)$, so \autoref{cor:metric-compact} tells us that we are really interested in totally bounded subsets, and we'll take the closure afterward to get our compact sets.

Here are a few lemmas.
\begin{lemma} \label{lem:tot-bound-is-pointwise}
	Fix a topological space $(X,\mc T)$ and a metric space $(M,d)$ so that we can give the space of bounded continuous functions $B_c(X,M)$ the uniform metric $d_u$. Fixing a totally bounded subset $\mc F\subseteq B_c(X,M)$, the set
	\[\{s(x):s\in \mc F\}\]
	totally bounded for any fixed $x\in X$.
\end{lemma}
\begin{proof}
	For any $\varepsilon>0$ has a finite set $\{f_1,\ldots,f_n\}\subseteq \mc F$ so that $\mc F$ is covered by the $B(s_i,\varepsilon/2)$. This is equivalent to saying that any $f\in \mc F$ has some $f_i$ with
	\[d(f(x),f_i(x))\le\varepsilon/2\]
	for all $x\in X$, so
	\[\{f(x):s\in \mc F\}\subseteq\bigcup_{i=1}^nB(f_i(x),\varepsilon),\]
	so the claim follows.
\end{proof}
\autoref{lem:tot-bound-is-pointwise} motivates the following definition.
\begin{definition}[Pointwise totally bounded]
	Fix topological spaces $(X,\mc T_X)$ and a metric space $(M,d)$, and let $\mc F$ be a family of continuous functions $f\colon X\to M$. Then $\mc F$ is \textit{pointwise totally bounded} if and only if any $x\in \mc F$ makes the set
	\[\{f(x):f\in\mc F\}\]
	totally bounded.
\end{definition}
\begin{example}
	By \autoref{lem:tot-bound-is-equicont}, any totally bounded subset of $B_c(X,M)$ is pointwise totally bounded.
\end{example}
\begin{lemma} \label{lem:tot-bound-is-equicont}
	Fix a topological space $(X,\mc T)$ and a metric space $(M,d)$ so that we can give the space of bounded continuous functions $B_c(X,M)$ the uniform metric $d_u$. Fixing a totally bounded subset $\mc F\subseteq B_c(X,M)$ and a point $x\in X$, any $\varepsilon>0$ has some open subset $U\subseteq X$ containing $x$ such that
	\[d(f(x),f(y))<\varepsilon\]
	for any $y\in X$ and $f\in \mc F$.
\end{lemma}
\begin{proof}
	Fix any $\varepsilon>0$ and use our totally boundedness to extract $\{f_1,\ldots,f_n\}\subseteq\mc F$ such that the $B(f_i,\varepsilon/3)$ cover $\mc F$. Now, for any $f\in \mc F$, find some $f_i$ with $d(f_i,f)<\varepsilon/3$, we see that any $y\in \mc F$ can write
	\[d(f(x),f(y))\le d(f(x),f_i(x))+d(f_i(x),f_i(y))+d(f_i(y),f(y))\le2\varepsilon/3+d(f_i(x),f_i(y)).\]
	Now, by the continuity of $f_i$, we see that there is an open subset $U_i$ containing $x$ such that $y\in U_j$ implies $d(f_i(x),f_i(y))<\varepsilon/3$, so $d(f(x),f(y))<\varepsilon$ follows.

	We now let $f$ vary, which allows the $U_i$ to vary. Defining
	\[U\coloneqq\bigcap_{i=1}^nU_i,\]
	we see $U$ is an open subset of $X$ containing $x$, and each $y\in U$ has $d(f(x),f(y))<\varepsilon$ for any (!) $f\in \mc F$.
\end{proof}
\autoref{lem:tot-bound-is-equicont} motivates the following definition.
\begin{definition}[Equicontinuous]
	Fix topological spaces $(X,\mc T_X)$ and a metric space $(M,d)$, and let $\mc F$ be a family of continuous functions $f\colon X\to M$. We say that the family $\mc F$ is \textit{equicontinuous} at some $x\in X$ if and only if any $\varepsilon>0$ has some open subset $U\subseteq X$ such that $y\in U$ has
	\[d(f(y),f(x))<\varepsilon\]
	for all $f\in \mc F$. The entire family $\mc F$ is \textit{equicontinuous} if any only if it is equicontinuous at all $x\in X$.
\end{definition}
\begin{example}
	By \autoref{lem:tot-bound-is-equicont}, any totally bounded subset of $B_c(X,M)$ is equicontinuous.
\end{example}

\subsection{Arzel\'a--Ascoli's Theorem}
We might hope for a converse of our given lemmas. Here is the result.
\begin{theorem}[Arzel\'a--Ascoli] \label{thm:ascoli}
	Fix a compact
	 topological space $(X,\mc T)$ and a metric space $(M,d)$ so that we can give the space of bounded continuous functions $B_c(X,M)$ the uniform metric $d_u$. Then any equicontinuous and pointwise totally bounded family $\mc F\subseteq B_c(X,M)$ is totally bounded.
\end{theorem}
\begin{proof}
	Fix some $\varepsilon>0$ so that we want to cover $\mc F$ with finitely balls of radius $\varepsilon>0$.
	
	The point is to use compactness on the equicontinuous statement. Indeed, for any $x\in X$, we are promised an open subset $U_x\subseteq X$ such that any $y\in U_x$ and $f\in\mc F$ has $d(f(x),f(y))<\varepsilon/4$. However, this means
	\[X\subseteq\bigcup_{x\in X}U_x\]
	gives us an open cover of $X$, so compactness tells us that there is some finite sequence of points $\{x_i\}_{i=1}^n$ such that the $U_i\coloneqq U_{x_i}$ cover $X$.

	Now, fixing any particular $i$, we use the pointwise totally bounded condition to note
	\[\{f(x_i):f\in\mc F\}\]
	is totally bounded, so we get a finite subset $S_i\subseteq\mc F$ such that
	\[\{f(x_i):f\in\mc F\}\subseteq\bigcup_{f\in S_i}B(f(x_i),\varepsilon/4).\]
	We now define $S$ as the union of all the $S_i$, which is finite as the finite union of finite sets.

	To finish the proof, we will need to do a little bookkeeping. Let $\Psi$ denote the set of functions from $\{1,\ldots,n\}$ to $S$ so that we can set
	\[\mc F_\psi\coloneqq\{f\in\mc F:f(x_i)\in B(\psi(i),\varepsilon/4)\text{ for each }1\le i\le n\}\]
	for any $\psi\in\Psi$. By construction, the $\mc F_\psi$ cover $\mc F$: fix some $f\in\mc F$. Note that any $y\in U_i$ implies that $d(f(y),f(x_i))\le\varepsilon/4$ by construction of $U_i$, and for any given $f(x_i)$, there is an element $s_i\in S_i$ such that $d(f(x_i),s_i)<\varepsilon/4$ by construction of the $S_i\subseteq S$. Defining $\psi$ by $\psi(i)\coloneqq s_i$ for this particular $s_i$, we see that any $y\in U_i$ for any $i$ has
	\[d(f(y),f(x_i))\le d(f(y),s_i)+d(s_i,f(x_i))\le\varepsilon/2.\]
	Letting $i$ vary, we recall that the $U_i$ cover $X$, so we have found $\psi\in\Psi$ with $f\in\mc F_\psi$, which is what we wanted.
	
	We will finish upon showing that $\mc F_\psi$ has diameter less than $\varepsilon$. Well, for any $f,g\in\mc F_\psi$, we need to show that $d_u(f,g)<\varepsilon$. Well, fix any $x\in X$ and find some $j$ with $x\in U_j$. Then we see
	\[d(f(x),g(x))\le d(f(x),f(x_j))+d(f(x_j),g(x_j))+d(g(x_j),g(x))\le\varepsilon/2+d(f(x_j),g(x_j)).\]
	Now, by construction of $\psi$, we see
	\[d(f(x_j),g(x_j))\le d(f(x_j),\psi(j))+d(\psi(j),g(x_j))<\varepsilon/2,\]
	so we see that $d(f(x),g(x))<\varepsilon$ in total. It follows $\norm{f-g}_\infty\le\varepsilon$, so, say, dividing all $\varepsilon$s by two will give $\mc F_\psi$ all with radius less than $\varepsilon$.
\end{proof}

\end{document}