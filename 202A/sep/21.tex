% !TEX root = ../notes.tex

\documentclass[../notes.tex]{subfiles}

\begin{document}

\section{September 21}

Today we begin talking about Tychonoff's theorem.

\subsection{Comments on Choice}
Here is our main result for today.
\begin{theorem}[Tychonoff] \label{thm:tych}
	Fix a collection $\{(X_\alpha,\mc T_\alpha)\}_{\alpha\in\lambda}$ of compact topological spaces, and give the product space $X\coloneqq\prod_{\alpha\in\lambda}X_\alpha$ the product topology. Then $X$ is compact.
\end{theorem}
Notably, we are not requiring the spaces $X_\alpha$ to be Hausdorff.
\begin{warn} \label{warn:tychhard}
	The proof of \autoref{thm:tych} will be the hardest part of this course.
\end{warn}
\begin{remark}
	The reason for \autoref{warn:tychhard} is that we need to at least know that $X$ is nonempty to say anything about $X$ at all, and an arbitrary product being nonempty is equivalent to the Axiom of Choice. In fact, \autoref{thm:tych} (notably not assuming that the $X_\alpha$ are Hausdorff!) actually implies the Axiom of Choice, as shown by John Kelly.
\end{remark}
To prepare ourselves, we will point out a few of the main ingredients we will use. We will use the Axiom of Choice, which we will go ahead and state now.
\begin{ax}[Choice]
	Given a collection of nonempty sets $\{S_\alpha\}_{\alpha\in\lambda}$, the product $\prod_{\alpha\in\lambda}S_\alpha$ is nonempty.
\end{ax}
We will also use Zorn's lemma. To state Zorn's lemma, we begin by defining a partially ordered set and its chains.
\begin{definition}[Poset]
	A \textit{partially ordered set} or \textit{poset} is a set $P$ equipped with a reflexive, antisymmetric, and transitive relation ${\le}\subseteq P\times P$.
\end{definition}
\begin{example} \label{ex:subsetsposet}
	Given a set $X$, the power set $\mc P(X)$ is a partially ordered set under inclusion $\subseteq$. Here are the checks.
	\begin{itemize}
		\item Reflexive: for $A\in\mc P(X)$, we see $A\subseteq A$.
		\item Antisymmetric: for $A,B\in\mc P(X)$, we see $A\subseteq B$ and $B\subseteq A$ implies $A=B$.
		\item Transitive: for $A,B,C\in\mc P(X)$, we see $A\subseteq B$ and $B\subseteq C$ implies $A\subseteq C$.
	\end{itemize}
	Replacing all the $\subseteq$s with $\supseteq$s shows that $\mc P(X)$ is also a partially ordered set under containment $\supseteq$.
\end{example}
Posets have very natural subposets.
\begin{definition}[Subposet]
	Given a partially ordered $(P,\le)$, a \textit{subposet} is a subset $S\subseteq P$ equipped with the restricted partial order ${\le}\cap(S\times S)$.
\end{definition}
All the checks for $(S,{\le}\cap(S\times S))$ being a partially ordered set are inherited directly from $P$, so the proof amounts to just writing them down.
\begin{example}
	Given a topological space $(X,\mc T)$, we see that $\mc T$ is a subposet of $\mc P(X)$, where $\mc P(X)$ can be given the partial order $\subseteq$ or $\supseteq$ from \autoref{ex:subsetsposet}.
\end{example}
And here are our chains.
\begin{definition}[Chain]
	Fix a partially ordered set $(P,\le)$. Then a \textit{chain} is a subset $C\subseteq P$ such that the subposet $(C,\le)$ is totally ordered.
\end{definition}
Zorn's lemma is interested in special kinds of partially ordered sets.
\begin{definition}[Inductively ordered]
	A partially ordered set $(P,\le)$ is \textit{inductively ordered} if and only if every chain $C\subseteq P$ has an upper bound in $P$. In other words, there is an element $p\in P$ such that $c\le p$ for all $c\in C$.
\end{definition}
And here is Zorn's lemma.
\begin{ax}[Zorn's lemma]
	An inductively ordered partially ordered set $(P,\le)$ has a maximal element.
\end{ax}
\begin{remark}
	It turns out that the Axiom of Choice (in the form of Zorn's lemma) is also equivalent to every vector space having a basis. (In one direction, given a vector space $V$, one can build a basis by taking a maximal linearly independent set of vectors in $V$.) One can get a feeling for the other direction because the $\QQ$-vector space $\RR$ doesn't have any ``constructible'' basis.
\end{remark}
\begin{remark}
	The fact that every (commutative) ring has a maximal ideal containing any given proper ideal is also equivalent to the Axiom of Choice (in the form of Zorn's lemma). Here are two examples.
	\begin{itemize}
		\item Given any set $S$, finding a maximal ideal of the ring $R\coloneqq\FF_2^S$ (whose operations are pointwise from $\FF_2$) which contains the ideal $\FF_2^{\oplus S}$ requires knowing that $R$ is nonempty.
		\item The ring $R\coloneqq C([0,\infty))$ of continuous $\RR$-valued functions has the ideal
		\[I\coloneqq\left\{f\in R:\lim_{x\to\infty}f(x)=0\right\}\]
		doesn't have any constructible maximal ideals containing it.
	\end{itemize}
\end{remark}
For our next example, we define a filter.
\begin{definition}[Filter]
	Fix a set $X$. A \textit{filter} $\mc F$ on $X$ is a collection of nonempty subsets of $X$ satisfying the following conditions.
	\begin{listalph}
		\item $\mc F$ is closed under finite intersection.
		\item If $A\in\mc F$ and $A\subseteq B\subseteq X$, then $B\in\mc F$.
	\end{listalph}
\end{definition}
\begin{example}
	Given a topological space $(X,\mc T)$ and a subset $A\subseteq X$, the subposet $\mc T$ of $(\mc P(X),\subseteq)$ has a filter $\mc F$ of all those open subsets containing $A$.
\end{example}
\begin{example}
	Given a set $X$, the collection of subsets containing a given point $p\in X$ is a filter and in fact a ``maximal'' filter.
\end{example}
The point is that Zorn's lemma automatically promises us maximal filters, or ``ultrafilters.''
\begin{example}
	Fix $X\coloneqq[0,\infty)$. Then the collection $\mc F$ of the subsets of $A\subseteq X$ which contain $[n,\infty)$ for some integer $n$ is a filter. However, there is no obvious maximal filter.
\end{example}

\end{document}