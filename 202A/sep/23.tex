% !TEX root = ../notes.tex

\documentclass[../notes.tex]{subfiles}

\begin{document}

\section{September 23}

We continue discussing Tychonoff's theorem.

\subsection{Tychonoff's Theorem}
Here is our statement.
\tychthm*
\begin{proof}
	We will use \autoref{prop:compactviafip}. For each $\alpha$, let $\pi_\alpha\colon X\to X_\alpha$ denote the canonical projection. Let $\mc V$ be a collection of closed subsets of $X$ satisfying the finite intersection property, and we will show that $\bigcap_{V\in\mc V}V$ is nonempty. We proceed in steps.
	\begin{enumerate}
		\item The beginning of this proof does not use topology. Let $\Omega_\mc V$ be the collection of families of subsets $\mc F$ of $X$ which contain $\mc V$ and have the finite intersection property. We claim that $W_\mc V$ is inductively ordered under $\supseteq$.

		Well, let $\Omega\subseteq\Omega_\mc V$ be some chain, and we define the collection
		\[\mc U\coloneqq\bigcup_{\mc F\in\Omega}\mc F,\]
		which we claim is the required upper bound for $\Omega$. Of course, each $\mc F$ contains $\mc V$, and $\mc U\supseteq\mc F$ for each $\mc F$, so $\mc U$ both contains $\mc V$ and is an upper bound for $\Omega$. It remains to show $\mc U\in\Omega_\mc V$, for which we need to show that $\mc U$ has the finite intersection property.
		
		For this, find some finite subcollection of nonempty subsets $\{A_k\}_{k=1}^n\subseteq\mc U$ which we would like to show have nonempty intersection. Now, for each $k$, there is some $\mc F_k\in\Omega$ containing $A_k$, by construction of $\mc U$ as the union over $\Omega$. Because the number of subsets is finite, and because $\Omega$ is totally ordered, we may find the largest of the $\mc F_k$, which we call $\mc F$.

		Now, $\mc F\in\Omega\subseteq\Omega_\mc V$ must have the finite intersection property, so $\{A_k\}_{k=1}^n\subseteq\mc F$ forces
		\[\bigcap_{k=1}^nA_k\ne\emp,\]
		which is what we wanted. This completes the proof.

		\item \label{item:mfip} From the previous step, Zorn's lemma promises a maximal family $\mc M$. We claim that $\mc M$ is closed under taking finite intersections. Indeed, define $\mc M'$ as the set of all finite intersections of $\mc M$, and we will show that $\mc M'=\mc M$.

		Well, certainly $\mc M\subseteq\mc M'$ because intersections of exactly one set $F\in\mc M$ will just recover $F\in\mc M'$. Thus, if we can show $\mc M'\in\Omega_{\mc V}$, the desired equality $\mc M'=\mc M$ will follow by maximality.
		
		Certainly, we of course have $\mc M\supseteq\mc V$, so $\mc M'\supseteq\mc V$ as well. So to show $\mc M'\in\Omega_{\mc V}$, it remains to show the finite intersection property. Well, let $\{A_k\}_{k=1}^n\subseteq\mc M'$ be some finite subcollection of nonempty subsets, and we show their intersection is nonempty. By definition of $\mc M'$, each $k$ lets us write
		\[A_k=\bigcap_{\ell=1}^{n_k}B_{k,\ell}\]
		for some subsets $B_{k,\ell}\in\mc M$; because $A_k\in\mc M'$ is nonempty, we see that $B_{k,\ell}\in\mc M$ is nonempty, so the finite intersection property on $\mc M$ tells us that
		\[\bigcap_{k=1}^nA_k=\bigcap_{k=1}^n\bigcap_{\ell=1}^{n_k}B_{k,\ell}\]
		is nonempty, which is what we wanted.

		\item \label{item:mprop2} We claim that if a subset $B\subseteq X$ has $B\cap A\ne\emp$ for each $A\in\mc M$, then in fact $B\in\mc M$. Indeed, define $\mc M''\coloneqq\mc M\cup\{B\}$, and we show $\mc M''=\mc M$.

		Certainly $\mc M\subseteq\mc M''$, so it is enough by maximality of $\mc M$ to show $\mc M''\in\Omega_\mc V$. Certainly $\mc V\subseteq\mc M\subseteq\mc M''$, so it remains to show that $\mc M''$ satisfies the finite intersection property.

		For this, pick up some finite subcollection of nonempty subsets $\{A_k\}_{k=1}^n\subseteq\mc M''$, and we show their intersection is nonempty. If none of these subsets are $B$, then in fact $\{A_k\}_{k=1}^n\subseteq\mc M$, so the finite intersection property for $\mc M$ forces
		\[\bigcap_{k=1}^nA_k\ne\emp.\]
		Otherwise, say $B=A_1$ without loss of generality. Then we may assume $B\ne A_k$ for each $k>1$, so $A_k\in\mc M$ for each $k>1$, so we note
		\[\bigcap_{k=1}^nA_k=B\cap\bigcap_{k=2}^nA_k.\]
		However, $\mc M$ is closed under finite intersection, so in fact $\bigcap_{k=2}^nA_k\in\mc M$, and by the finite intersection property, we have that $\bigcap_{k=2}^nA_k$ is nonempty. Thus, by hypothesis on $B$, we see
		\[B\cap\bigcap_{k=2}^nA_k\ne\emp,\]
		which is what we wanted.

		\item We now begin touching our product. For given $\alpha\in\lambda$ and $\mc F\in\Omega_\mc V$, we claim that
		\[\pi_\alpha(\mc F)\coloneqq\{\pi_\alpha(A):A\in\mc F\}\]
		satisfies the finite intersection property. Fix a finite subcollection of nonempty subsets $\{\pi_\alpha(A_k)\}_{k=1}^n$ of $\pi_\alpha(\mc F)$, and we will show its intersection is nonempty. Then we must have $A_k$ being nonempty for each $k$, so the finite intersection property on $\mc F$ forces
		\[\bigcap_{k=1}^nA_k\ne\emp.\]
		Finding some $a$ in this intersection, we see $\pi_\alpha(a)\in\pi_\alpha(A_k)$ for each $k$, so $\pi_\alpha(a)$ belongs in
		\[\bigcap_{k=1}^n\pi_\alpha(A_k),\]
		thus making this intersection nonempty.

		\item And now the topology begins. For given $\alpha$, note that
		\[\overline{\mc M}_\alpha\coloneqq\{\overline{\pi_\alpha(A)}:A\in\mc M\}\]
		has the finite intersection property by the previous step. Namely, any finite subcollection of nonempty subsets $\{\pi_\alpha A_k\}_{k=1}^n$ has a nonempty intersection, so writing
		\[\emp\ne\bigcap_{k=1}^n\pi_\alpha(A_k)\subseteq\bigcap_{k=1}^n\pi_\alpha(A_k)\]
		gives what we want. However, $X_\alpha$ is compact (!), so \autoref{prop:compactviafip} tells us that
		\[\bigcap_{A\in\overline{\mc M}_\alpha}A\ne\emp.\]
		
		\item Directly invoking the Axiom of Choice, we may find some $x_\alpha\in\bigcap_{A\in\overline{\mc M}_\alpha}A$ for each $\alpha$. Set $x\coloneqq(x_\alpha)_{\alpha\in\lambda}$ to be the corresponding element of $X$.

		We claim that each nonempty $A\in\mc M$ has $x\in\overline A$. By \autoref{lem:betterclosure}, it suffices to show that every open subset $U$ containing $x$ has nonempty intersection with $A$. Because each open subset $U$ containing $x$ has a(n open) basis set $B\subseteq U$ containing $x$, it suffices to check $B\cap A\ne\emp$ for basis elements, and $B\cap A\subseteq U\cap A$ will give the result.
		
		There are three steps. Observe that we must invoke the definition of the product topology on $X$ to talk topologically about $X$, so we do so here.
		\begin{listalph}
			\item We begin by checking this on the sub-base. For each $\alpha\in\lambda$, fix some sub-base element $\pi_\alpha^{-1}(U_\alpha)$ (where $U_\alpha\subseteq X_\alpha$ is open) containing $x$, and we claim
			\[\pi^{-1}_\alpha(U_\alpha)\cap A\stackrel?\ne\emp.\]
			Well, $x\in\pi^{-1}_\alpha(U_\alpha)$ requires $x_\alpha\in U_\alpha$, but $A\in\mc M$ forces $x_\alpha\in\overline{\pi_\alpha(A)}$. Thus, $U_\alpha\cap\pi_\alpha(A)\ne\emp$ by \autoref{lem:betterclosure}, so there is some $a\in A$ with $\pi_\alpha(a)\in U_\alpha$, so $\pi^{-1}_\alpha(U_\alpha)\cap A$ is in fact nonempty.
			\item We show that each basis set containing $x$ lives in $\mc M$. Part (a) above added to \autoref{item:mprop2} directly shows that every sub-base open set containing $x$ lives in $\mc M$. Thus, \autoref{item:mfip} tells us that any finite intersection of sub-basic sets containing $x$ live in $\mc M$ as well, but these are exactly the basic sets containing $x$. (Namely, any basic set is the intersection of sub-basic sets, and $x$ living in the basic set forces $x$ to still live in those sub-basic sets.)
			\item It follows from the finite intersection property for $\mc M$ that any basic set $B$ containing $x$ has $B\in\mc M$ and therefore $A\cap B\ne\emp$ because $A$ is nonempty.
		\end{listalph}
		The above steps finish this part.

		\item We finish the proof. Any $V\in\mc V$ is closed and has $V\in\mc M$. By the above point, we see $x\in\overline V$, so $x\in V$ by \autoref{ex:closureofclosed}, so we have exhibited
		\[\bigcap_{V\in\mc V}V\ne\emp.\]
	\end{enumerate}
	The above steps have showed that $\bigcap_{V\in\mc V}\mc V\ne\emp$ from $\mc V$ having the finite intersection property, so we conclude that $X$ is compact by \autoref{prop:compactviafip}. 
\end{proof}

\end{document}