% !TEX root = ../notes.tex

\documentclass[../notes.tex]{subfiles}

\begin{document}

\section{September 26}

We begin class by finishing the proof of Tychonoff's theorem (\autoref{thm:tych}). I have gone ahead and just edited Friday's lecture for continuity.

\subsection{Remarks on Tychonoff's Theorem}
Here are some remarks.
\begin{remark}
	Intuitively, the maximal element $\mc M$ is constructed in order to become some filter focused around the single point $x$. Similar to maximal ideals corresponding to points, adding in all the ``maximality'' constraints for $\mc M$ hones in our focus to the single constructed point $x$.
\end{remark}
\begin{remark}
	Here is an application of \autoref{thm:tych}. One can show that any normed vector space $(V,\norm\cdot)$ has ``lots'' of continuous functionals by extending those found on a finite-dimensional subspace; let $V'$ be the complete normed vector space of continuous linear functionals. (The norm of some $v'\in V'$ is its Lipschitz constant, using \autoref{lem:boundediffcont}.) Fixing the unit ball $B$ of $V'$, one can give $V'$ the weakest topology making all the linear functionals ``from $V$'' continuous (this is the weak-$^*$ topology), which one can show is both Hausdorff and compact (!). This is the Banach--Alaoglu theorem, and it follows from \autoref{thm:tych} by showing the space we want is a closed subspace of the compact space
	\[\prod_{v\in V}[-\norm v,\norm v].\]
\end{remark}
\begin{remark}[{\texorpdfstring{$\beta$}{b}}-compactification]
	Let $A\coloneqq C([0,\infty))$ be the space of bounded continuous function $[0,\infty)\to\RR$, which we can see directly is an $\mathbb R$-algebra by taking $r$ to the constant function $r$. Let $A'$ be the set of continuous functions $A\to\RR$. Notably, any $x\in\RR$ gives a continuous ring homomorphism $A\to\RR$ by $f\mapsto f(x)$, so we let $Y$ be the set of all homomorphisms $A\to\RR$. Again, $A'$ is compact using the weak-$^*$ topology, and so $Y$ as a closed subset of $A'$ can be given a compact topology. Then one can show that $A$ is homeomorphic to $C(Y)$.
\end{remark}

\subsection{Tychonoff's Theorem and Choice}
We now show that Tychonoff's theorem implies the Axiom of Choice.
\begin{theorem}[Kelley]
	Tychonoff's theorem implies the Axiom of Choice.
\end{theorem}
\begin{proof}
	Assume \autoref{thm:tych} is true. Let $\{X_\alpha\}_{\alpha\in\lambda}$ be a collection of nonempty sets. We want to show that
	\[X\coloneqq\prod_{\alpha\in\lambda}X_\alpha\]
	is nonempty.
	
	The trick is to enlarge the $X_\alpha$ to be able to give them a suitable topology. Choose some (set) $\omega$ which does not live in $\bigcup_{\alpha}X_\alpha$; for example, setting $\omega$ to be equal to this set will do (using the Axiom of Foundation). Then we set
	\[Y_\alpha\coloneqq X_\alpha\cup\{\omega\},\]
	which we give the topology $\mc T_\alpha\coloneqq\{Y_\alpha,\emp,X_\alpha,\{\omega\}\}$. We quickly check that this is a topology.
	\begin{itemize}
		\item Note $\emp$ and $Y_\alpha$ are open.
		\item Arbitrary union: let $\mc U\subseteq\mc T_\alpha$ be a collection. Note that $\mc U$ is necessarily finite, so it suffices by induction to show that $U\cup U'\in\mc T_\alpha$ for any $U,U'\in\mc T_\alpha$. We have the following cases.
		\begin{itemize}
			\item If $U=\emp$ or $U'=\emp$, then we get $U\cup U'\in\{U,U'\}\subseteq\mc T_\alpha$.
			\item If $U=Y_\alpha$ or $U'=Y_\alpha$, then $U\cup U'=Y_\alpha\in\mc T_\alpha$.
			\item Note $U=U'$ gives $U\cup U'=U\in\mc T_\alpha$.
			\item We have left to deal with $\{U,U'\}\subseteq\{X_\alpha,\{\omega\}\}$ where $U$ and $U'$ are distinct, which means we just have to check $\mc T_\alpha\cup\{\omega\}=Y_\alpha$ is open.
		\end{itemize}
		\item Finite intersection: note that $U\in\mc T_\alpha$ implies $Y_\alpha\setminus U\in\mc T_\alpha$ because $Y_\alpha\setminus\{\omega\}=X_\alpha$ and $Y_\alpha\setminus\emp=Y_\alpha$, and the other checks follow. Thus, we note any finite collection $\mc U\subseteq\mc T_\alpha$ has
		\[Y_\alpha{\mathbin\bigg\backslash}\bigcap_{U\in\mc U}U=\bigcup_{\alpha\in\lambda}(Y_\alpha\setminus U)\]
		is a union of open sets and hence open. It follows that our intersection also lives in $\mc T_\alpha$.
	\end{itemize}
	Additionally, because $\mc T_\alpha$ has only finitely many sets, the space $(Y_\alpha,\mc T_\alpha)$ is compact: any subcollection of $\mc T_\alpha$ is finite, so all open covers of $Y_\alpha$ are automatically finite. It follows that the product
	\[Y\coloneqq\prod_{\alpha\in\lambda}Y_\alpha\]
	is compact by applying \autoref{thm:tych} (!).

	We will now extract out our element of $X$ using compactness of $Y$ via \autoref{prop:compactviafip}. Let $\pi_\alpha\colon Y\to Y_\alpha$ be the canonical projection. Note that $Y_\alpha\setminus X_\alpha=\{\omega\}$ is open in $Y_\alpha$, so $X_\alpha\subseteq Y_\alpha$ is closed, so $V_\alpha\coloneqq\pi_\alpha^{-1}(X_\alpha)$ is a closed subset of $Y$ by the continuity of $\pi_\alpha$.

	We now claim that the closed sets $\{V_\alpha\}_{\alpha\in\lambda}$ satisfy the finite intersection property: given a finite subcollection $\{V_{\alpha_i}\}_{i=1}^n$, one may finitely (!) choose a point $x_{\alpha_i}\in X_{\alpha_i}$. So we define
	\[y_\alpha\coloneqq\begin{cases}
		x_{\alpha_i} & \alpha\in\{\alpha_1,\ldots,\alpha_n\}, \\
		\omega & \alpha\notin\{\alpha_1,\ldots,\alpha_n\},
	\end{cases}\]
	so the point $(y_\alpha)_{\alpha\in\lambda}\in Y$ has $\pi_{\alpha_i}(y)\in X_{\alpha_i}$ for each $\alpha_i$, so $y_{\alpha_i}\in\pi_{\alpha_i}^{-1}(X_\alpha)=V_\alpha$, so
	\[y\in\bigcap_{i=1}^nV_{\alpha_i}.\]
	So we have verified the finite intersection property.

	It follows from \autoref{prop:compactviafip} that we can find
	\[y\in\bigcap_{\alpha\in\lambda}V_\alpha.\]
	However, this implies that each $\alpha\in\lambda$ has $y\in V_\alpha$ and so $\pi_\alpha(y)\in X_\alpha$. It follows that
	\[y\in\prod_{\alpha\in\lambda}X_\alpha,\]
	which finishes the proof.
\end{proof}
\begin{remark}
	The topology on $Y_\alpha$ need not be Hausdorff, so we needed \autoref{thm:tych} to allow non-Hausdorff spaces.
\end{remark}

\end{document}