% !TEX root = ../notes.tex

\documentclass[../notes.tex]{subfiles}

\begin{document}

Today we're talking about completeness of metric spaces.

\subsection{Many Kinds of Morphisms}
In mathematics, we are interested in objects not in isolation but as they relate to each other. Namely, we are interested also in the maps between our objects.

The philosophy here comes from category theory, where one is really most interested in the ``morphisms'' between ``objects'' instead of the objects themselves. For concreteness, here is a definition of a category.
\begin{definition}[Category]
	A \textit{category} $\mathcal C$ consists of a class of objects $\op{Ob}\mathcal C$ and morphisms $\op{Mor}\mathcal C$ such that any two objects $A,B\in\op{Ob}\mathcal C$ have a morphism set $\op{Mor}(A,B)$. This data satisfy the following properties.
	\begin{itemize}
		\item Given $f\in\op{Mor}(A,B)$ and $g\in\op{Mor}(B,C)$, there is a composition $(g\circ f)\in\op{Mor}(A,C)$.
		\item Given $A\in\op{Ob}\mathcal C$, there is an identity morphism ${\id_A}\in\op{Mor}(A,A)$.
		\item Identity: any $f\in\op{Mor}(A,B)$ has $f\circ{\id_A}=f={\id_B}\circ f$.
		\item Associativity: any $f\in\op{Mor}(A,B)$ and $g\in\op{Mor}(B,C)$ and $h\in\op{Mor}(C,D)$ has $(h\circ g)\circ f=h\circ (g\circ f)$.
	\end{itemize}
\end{definition}
\begin{example}
	There is a category of groups, where the morphisms are group homomorphisms.
\end{example}
\begin{example}
	There is a category of metric spaces, where the morphisms will be continuous maps.
\end{example}
One can actually specify more carefully what kinds of morphisms we're paying attention to. Here are some kinds of morphisms.
\begin{definition}[Isometry]
	Given metric spaces $(X,d_X)$ and $(Y,d_Y)$, an \textit{isometry} is a function $f\colon X\to Y$ preserving the metric as
	\[d_Y(f(x),f(x'))=d_X(x,x').\]
\end{definition}
One can check quickly that the composition of two isometries is an isometry.
\begin{remark}
	Of course, one can then ask for surjective isometries as well, whose compositions remain surjective. It turns out that being surjective promises an inverse isometry.
\end{remark}
Isometries are somewhat restrictive, so we might weaken this as follows.
\begin{definition}[Lipschitz continuous]
	Given metric spaces $(X,d_X)$ and $(Y,d_Y)$, a function $f\colon X\to Y$ is a \textit{Lipschitz continuous} if and only if there is a constant $c\in\RR$ such that
	\[d_Y(f(x),f(x'))\le c\cdot d_X(x,x').\]
	The smallest constant $c$ which works here is called the \textit{Lipschitz constant} for $f$. If $f$ has a Lipschitz continuous inverse, then $f$ is a \textit{Lipschitz isomorphism}.
\end{definition}
Again, one checks that the composition of Lipschitz continuous functions and Lipschitz isomorphisms are preserved under composition.
\begin{remark}
	A good reason to care about this notion of continuity is that all normed vector spaces of some finite dimension $n$ are Lipschitz isomorphic.
\end{remark}
Here is yet a weaker notion of morphism.
\begin{definition}[Uniformly continuous]
	Given metric spaces $(X,d_X)$ and $(Y,d_Y)$, a function $f\colon X\to Y$ is a \textit{uniformly continuous} if and only if every $\varepsilon>0$ has some $\delta>0$ such that
	\[d_X(x,x')<\delta\implies d_Y(f(x),f(x'))<\delta\]
	for all $x,x'\in X$.
\end{definition}
By rearranging quantifiers, we get another useful (but weaker) notion.
\begin{definition}[Continuous]
	Given metric spaces $(X,d_X)$ and $(Y,d_Y)$, a function $f\colon X\to Y$ is a \textit{continuous} if and only if, at any $x\in X$, all $\varepsilon>0$ have some $\delta_x>0$ such that
	\[d_X(x,x')<\delta_x\implies d_Y(f(x),f(x'))<\varepsilon.\]
\end{definition}
As usual, composition is preserved and so on.
\begin{remark}
	In some sense, isometries and Lipschitz continuous functions have their definition fundamentally interrelated with the metric. In contrast, the weaker notion of continuity will readily generalize to general topological spaces. Uniform continuity also generalizes to ``uniformities,'' which is a different notion.
\end{remark}

\subsection{Completeness}
To discuss completeness, we need to talk about convergence.
\begin{definition}[Converge]
	Fix a metric space $(X,d)$. A sequence of points $\{x_n\}_{n\in\NN}\subseteq X$ \textit{converges to} $x\in X$ if and only if, for any $\varepsilon>0$, we can find $N>0$ such that
	\[n>N\implies d(x_n,x)<\varepsilon.\]
	We might write this as $x_n\to x$ as $n\to\infty$.
\end{definition}
We would like a notion of convergence which only uses data internal to the sequence, and this leads to the following definition.
\begin{definition}[Cauchy]
	Fix a metric space $(X,d)$. A sequence of points $\{x_n\}_{n\in\NN}\subseteq X$ is a \textit{Cauchy sequence} if and only if, for any $\varepsilon>0$, we can find $N>0$ such that
	\[n,m>N\implies d(x_n,x_m)<\varepsilon.\]
\end{definition}
One can check that all convergent sequences are Cauchy, and we in general hope that our Cauchy sequences will converge. As such, we have the following definition.
\begin{definition}[Complete]
	A metric space $(X,d)$ is \textit{complete} if and only if every Cauchy sequence in $X$ converges to a point in $X$.	
\end{definition}
We are sad when a metric space is not complete, so we hope to have a way to make it complete. The most natural way to do this is by using the notion of density.
\begin{definition}[Density]
	Fix a metric space $(X,d)$. Then $S\subseteq X$ is \textit{dense} if and only if, given any $x\in X$ and $\varepsilon>0$, we may find $x'\in S$ with $d(x,x')<\varepsilon$.
\end{definition}
And here is our completion.
\begin{definition}[Completion]
	A \textit{completion} of the metric space $(X,d)$ is a metric space $(\overline X,\overline d)$ equipped with an isometry $\iota\colon X\to\overline X$ such that $(\overline X,\overline d)$ is complete and $\im\iota$ is dense in $\overline X$.
\end{definition}
One can show that any metric space has a completion and that they are all isometric and therefore in some sense the same. The uniqueness result will appear on the homework, so for now we will discuss existence.
\begin{theorem}
	Any metric space $(X,d)$ has a completion.
\end{theorem}
\begin{proof}[Sketch]
	Let $\widetilde X$ denote the set of all Cauchy sequences in $X$. We hope to make $\widetilde X$ into our completion, but this requires a little care. To begin, we have the following lemma.
	\begin{lemma}
		Given a metric space $(X,d)$ with two Cauchy sequences $\{x_n\}_{n\in\NN}$ and $\{y_n\}_{n\in\NN}$, then the sequence
		\[\{d(x_n,y_n)\}_{n\in\NN}\subseteq\RR\]
		converges.
	\end{lemma}
	\begin{proof}
		Omitted.
	\end{proof}
	Thus, we define $\widetilde d\colon\widetilde X\times\widetilde X\to\RR_{\ge0}$ be this well-defined function. One can show that $\widetilde d$ is a semi-norm, and so we use \autoref{prop:semitometric} to induce a metric $\overline d$ on $\overline X\coloneqq\widetilde X/{\sim}$.
	
	To finish the proof, one has to check that $\overline X$ is in fact a completion. It is somewhat annoying to check that $\overline X$ is complete (though it is not terribly tricky), and the required isometry $\iota\colon X\to\overline X$ is given by
	\[\iota(x)\coloneqq\{x\}_{n\in\NN},\]
	which is certainly Cauchy.
\end{proof}

\end{document}