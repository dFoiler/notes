% !TEX root = ../notes.tex

\documentclass[../notes.tex]{subfiles}

\begin{document}

Today we're talking about completeness of metric spaces.

\subsection{Isometries}
In mathematics, we are interested in objects not in isolation but as they relate to each other. Namely, we are interested also in the maps between our objects.

The philosophy here comes from category theory, where one is really most interested in the ``morphisms'' between ``objects'' instead of the objects themselves. For concreteness, here is a definition of a category.
\begin{definition}[Category]
	A \textit{category} $\mathcal C$ consists of a class of objects $\op{Ob}\mathcal C$ and class of morphisms $\op{Mor}\mathcal C$ such that any two objects $A,B\in\op{Ob}\mathcal C$ have a morphism class $\op{Mor}(A,B)$. This data satisfy the following properties.
	\begin{itemize}
		\item Composition: given objects $A,B,C\in\op{Ob}\mathcal C$, there is a binary composition operation
		\[\circ\colon\op{Mor}(B,C)\times\op{Mor}(A,B)\to\op{Mor}(A,C).\]
		Explicitly, given $f\in\op{Mor}(A,B)$ and $g\in\op{Mor}(B,C)$, there is a composition $(g\circ f)\in\op{Mor}(A,C)$.
		\item Given $A\in\op{Ob}\mathcal C$, there is an identity morphism ${\id_A}\in\op{Mor}(A,A)$.
		\item Identity: any $f\in\op{Mor}(A,B)$ has $f\circ{\id_A}=f={\id_B}\circ f$.
		\item Associativity: any $f\in\op{Mor}(A,B)$ and $g\in\op{Mor}(B,C)$ and $h\in\op{Mor}(C,D)$ has $(h\circ g)\circ f=h\circ (g\circ f)$.
	\end{itemize}
\end{definition}
\begin{example}
	There is a category of groups, where the morphisms are group homomorphisms. The identity function gives the identity morphism, and composition of functions gives the required composition.
	
	For completeness, we check that composition is well-defined: given homomorphisms $f\colon A\to B$ and $g\colon B\to C$, we need $(g\circ f)\colon A\to C$ to be a group homomorphism. Well,
	\[(g\circ f)(a\cdot a')=g(f(a\cdot a'))=g(f(a)\cdot f(a'))=g(f(a))\cdot g(f(a'))=(g\circ f)(a)\cdot (g\circ f)(a').\]
\end{example}
In our discussion of metric spaces, there are many possible kinds of morphisms for us to consider. Here is the strongest type.
\begin{definition}[Isometry]
	Given metric spaces $(X,d_X)$ and $(Y,d_Y)$, an \textit{isometry} is a function $f\colon X\to Y$ preserving the metric as
	\[d_Y(f(x),f(x'))=d_X(x,x').\]
\end{definition}
\begin{example}
	The $90^\circ$ rotation $r\colon\RR^2\to\RR^2$ given by $r(x,y)\mapsto(y,-x)$ is an isometry, where $\RR^2$ is given the Euclidean metric. Indeed, any $(x,y),(x',y')\in\RR^2$ have
	\begin{align*}
		d\big(r(x,y),r(x',y')\big) &= d\big((y,-x),(y',-x')\big) \\
		&= \sqrt{(y-y')^2+(-x--x')^2} \\
		&= \sqrt{(x-x')^2+(y-y')^2} \\
		&= d\big((x,y),(x',y')\big).
	\end{align*}
\end{example}
\begin{notation}
	Fix two metric spaces $(X,d_X)$ and $(Y,d_Y)$. Given a function $f\colon X\to Y$ with extra structure respecting some aspect of the metric, we might write $f\colon(X,d_X)\to(Y,d_Y)$ to emphasize this.
\end{notation}
To show that isometries are valid morphisms, we need to check that the identity function ${\id_X}\colon X\to X$ is an isometry (which of course it is) and that the composition of two isometries is an isometry. We check this last one in a quick lemma.
\begin{lemma}
	Given two isometries $f\colon(X,d_X)\to(Y,d_Y)$ and $g\colon(Y,d_Y)\to(Z,d_Z)$, the composition $g\circ f$ is an isometry.
\end{lemma}
\begin{proof}
	Well, any two points $x,x'\in X$ have
	\[d_Z(g(f(x)),g(f(x')))=d_Y(f(x),f(x'))=d_X(x,x'),\]
	which is what we wanted.
\end{proof}
One can restrict further to surjective isometries, where the main point is that (again) the composition of two surjective functions remains surjective. (Note that the identity is of course surjective.) The following is the reason why a surjective isometry is a good notion.
\begin{lemma}
	A surjective isometry $f\colon(X,d_X)\to(Y,d_Y)$ is bijective, and its inverse function is also an isometry.
\end{lemma}
\begin{proof}
	To see that $f$ is bijective, we only need to know that $f$ is injective. Well, given $x,x'\in X$, note that $f(x)=f(x')$ if and only if $d_Y(f(x),f(x'))=0$ if and only if $d(x,x')=0$ if and only if $x=x'$.\footnote{In fact, this argument shows that all isometries are injective. We will shortly see that all actually Lipschitz continuous functions are injective.}

	Thus, $f$ is indeed bijective; let $g\colon Y\to X$ be its inverse. We now need to show that $g$ is an isometry. Well, given $y,y'\in Y$, we may find $x,x'\in X$ such that $f(x)=y$ and $f(x')=y'$. Then
	\[d_X(g(y),g(y'))=d_X((g\circ f)(x),(g\circ f)(x'))=d_X(x,x')\stackrel*=d_Y(f(x),f(x'))=d_Y(y,y'),\]
	where in $\stackrel*=$ we have used the fact that $f$ is an isometry.
\end{proof}
\begin{remark} \label{rem:bijvsiso}
	The above result is somewhat subtle in its importance: the inverse function of a bijection is only an inverse in the category of sets. The above result is saying that this inverse morphism in the category of sets is lifting to an inverse morphism in the category of metric spaces with isometries as morphisms. In general, it is not always true that bijective morphisms are invertible, as we shall soon see.
\end{remark}

\subsection{Lipschitz Continuity}
Isometries are somewhat restrictive, so we might weaken this as follows.
\begin{defihelper}[Lipschitz continuous] \label{defi:lipcont} \nirindex{Continuous!Lipschitz continuous}
	Given metric spaces $(X,d_X)$ and $(Y,d_Y)$, a function $f\colon X\to Y$ is a \textit{Lipschitz continuous} if and only if there is a constant $c\in\RR$ such that
	\[d_Y(f(x),f(x'))\le c\cdot d_X(x,x').\]
\end{defihelper}
\begin{remark} \label{rem:lipasratio}
	Equivalently, we are asking for the ratio
	\[\frac{d_Y(f(x),f(x'))}{d_X(x,x')}\]
	to be uniformly bounded above for all $x\ne x'$. Notably, the inequality is trivially satisfied whenever $x=x'$, or equivalently whenever $d(x,x')=0$.
\end{remark}
\begin{example}
	Any isometry $f\colon(X,d_X)\to(Y,d_Y)$ is Lipschitz continuous: indeed, set $c\coloneqq1$ so that, for any $x,x'\in X$,
	\[d_Y(f(x),f(x'))=d_X(x,x')\le1\cdot d_X(x,x').\]
\end{example}
\begin{example} \label{ex:projectislip}
	Provide $\RR$ and $\RR^2$ their usual Euclidean metrics. Then the projection $\pi\colon\RR^2\to\RR$ by $\pi\colon(x,y)\mapsto x$ is Lipschitz continuous: indeed, set $c\coloneqq1$ so that, for any $(x,y),(x',y')\in\RR^2$, we have
	\[d_{\RR^2}\big((x,y),(x',y')\big)=\sqrt{(x-x')^2+(y-y')^2}\ge\sqrt{(x-x')^2}=d_\RR(x,x')=d_\RR\big(\pi((x,y)),\pi((x',y'))\big).\]
\end{example}
Again, one can see that the identity function ${\id_X}\colon(X,d_X)\to(X,d_X)$ is Lipschitz continuous (with $c\coloneqq1$), and here is our composition check.
\begin{lemma}
	If $f\colon(X,d_X)\to(Y,d_Y)$ and $g\colon(Y,d_Y)\to(Z,d_Z)$ are Lipschitz continuous, then the composition $(g\circ f)\colon(X,d_X)\to(Z,d_Z)$ is also Lipschitz continuous.
\end{lemma}
\begin{proof}
	We are given constants $c$ and $d$ such that any $x,x'\in X$ and $y,y'\in Y$ have
	\[d_Y(f(x),f(x'))\le c\cdot d_X(x,x')\qquad\text{and}\qquad d_Z(g(y),g(y'))\le d\cdot d_Y(y,y').\]
	As such, we use the constant $cd$ to witness our Lipschitz continuity: any $x,x'\in X$ have
	\[d_Z(g(f(x)),g(f(x')))\le d\cdot d_Y(f(x),f(x'))\le cd\cdot d_X(x,x'),\]
	which is what we wanted.
\end{proof}
It will be shortly worth our time to talk about the constant $c$ appearing in \autoref{defi:lipcont}.
\begin{lemma} \label{lem:getlipconst}
	Fix a Lipschitz continuous function $f\colon(X,d_X)\to(Y,d_Y)$. Then there exists a constant $c_f$ (possibly $-\infty$) such that any real number $c\ge c_f$ is equivalent to the following property: any $x,x'\in X$ have
	\[d_Y(f(x),f(x'))\le c\cdot d_X(x,x').\]
\end{lemma}
\begin{proof}
	Let $S$ denote the set of all constants $c$ such that any $x,x'\in X$ have
	\[d_Y(f(x),f(x'))\le c\cdot d_X(x,x').\]
	Equivalently, using \autoref{rem:lipasratio}, $S$ is the set of upper-bounds for
	\[R\coloneqq\left\{\frac{d_Y(f(x),f(x'))}{d_X(x,x')}:x,x'\in X,x\ne x'\right\}.\]
	Now, $S$ is nonempty because $f$ is Lipschitz continuity, so we set $c_f\coloneqq\sup R$ to be the least upper bound for $R$---observe that $c_f=-\infty$ is permissible when $X$ has one point. It is now pretty clear that $S=[c_f,\infty)$.
\end{proof}
Note that $c_f$ the property stated in the lemma automatically implies that $c_f$ is the least possible constant and is unique. Being least is immediate (by the backwards direction), and being unique follows from being least. So because we have some uniqueness, we get a definition.
\begin{defihelper}[Lipschitz constant] \nirindex{Continuous!Lipschitz continuous!Lipschitz constant}
	Given a Lipschitz continuous function $f\colon(X,d_X)\to(Y,d_Y)$, the \textit{Lipschitz constant $c_f$ for $f$} is the least real number $c$ such that
	\[d_Y(f(x),f(x'))\le c\cdot d_X(x,x').\]
\end{defihelper}
We could, as before, look at surjective Lipschitz continuous functions, but these need not be bijective anymore as shown by \autoref{ex:projectislip}. What's worse is that, as warned possible in \autoref{rem:bijvsiso}, bijective Lipschitz continuous functions need not even have a Lipschitz continuous inverse.
\begin{exe} \label{exe:sadbijlip}
	We exhibit a function between metric spaces which is bijective and Lipschitz continuous, but its inverse function is not Lipschitz continuous.
\end{exe}
\begin{proof}
	Set $X\coloneqq(0,1)$ and $Y\coloneqq(1,\infty)$, both metric spaces with the Euclidean (subspace) metric, and set $f\colon(0,\infty)\to(0,\infty)$ by $f\colon x\mapsto1/x$. Notably, $x\in X$ implies $f(x)\in Y$, and $y\in Y$ implies $f(y)\in X$.
	\begin{itemize}
		\item Note $f|_Y$ is bijective with inverse $f|_X$ because $f(f(x))=f(1/x)=x$ for all $x\in(0,\infty)$.
		\item Note $f|_Y$ is Lipschitz continuous: set $c\coloneqq1$ and note that any $y,y'\in Y$ have
		\[|f(y)-f(y')|=\left|\frac1y-\frac1{y'}\right|=\left|\frac{y-y'}{yy'}\right|\le|y-y'|.\]
		\item But $f|_X$ is not Lipschitz continuous: suppose for contradiction that $f_X$ is Lipschitz continuous, and use \autoref{lem:getlipconst} to recover the needed constant $c_0$. Then set $c\coloneqq\max\{c_0,4\}$, which must also work as a constant, and set $x\coloneqq1/c$ and $x'\coloneqq1/(3c)$ so that
		\[|f(x)-f(x')|=\left|c-3c\right|=2c>c\cdot|x-x'|.\]
		This is a contradiction, so we are done.
		\qedhere
	\end{itemize}
\end{proof}
\begin{remark}[Nir]
	In some sense, the problem here is that the definition of Lipschitz continuity allows $d_Y(f(x),f(x'))$ to be ``too small,'' which permits the inverse function to have distances which blow up.
\end{remark}
In light of \autoref{exe:sadbijlip}, we introduce a new definition.
\begin{defihelper}[Lipschitz isomorphism] \nirindex{Continuous!Lipschitz continuous!Lipschitz isomorphism}
	Give metric spaces $(X,d_X)$ and $(Y,d_Y)$, a function $f\colon X\to Y$ is a \textit{Lipschitz isomorphism} if and only if $f$ is Lipschitz continuous and has an inverse function which is also Lipschitz continuous.
\end{defihelper}
\begin{remark}
	A good reason to care about this notion of continuity (and isomorphism) is that all normed $\RR$-vector spaces of some finite dimension $n$ are Lipschitz isomorphic.
	% One cannot hope for them all to be isometric because the two normed $\RR$-vector spaces $(\RR^n,\norm{\cdot}_2)$ and $(\RR^n,\norm{\cdot}_1)$ are not isometric.
\end{remark}


\subsection{Fun with Continuity}
Here is yet a weaker notion of morphism.
\begin{defihelper}[Uniformly continuous] \nirindex{Continuous!Uniformly continuous}
	Given metric spaces $(X,d_X)$ and $(Y,d_Y)$, a function $f\colon X\to Y$ is \textit{uniformly continuous} if and only if every $\varepsilon>0$ has some $\delta>0$ such that
	\[d_X(x,x')<\delta\implies d_Y(f(x),f(x'))<\varepsilon\]
	for all $x,x'\in X$.
\end{defihelper}
\begin{example}
	Any Lipschitz continuous function $f\colon(X,d_X)\to(Y,d_Y)$ is also uniformly continuous: indeed, for any $\varepsilon>0$, set $\delta\coloneqq\max\{c_f,1\}\varepsilon>0$ (where $c_f$ is the Lipschitz constant) so that
	\[d_X(x,x')<\varepsilon\implies d_Y(f(x),f(x'))\le c_f\cdot d(x,x')<\delta.\]
\end{example}
\begin{example}
	Give $[0,1]$ the Euclidean (subspace) metric, and set $f\colon[0,1]\to[0,1]$ by $f(x)\coloneqq\sqrt x$.
	\begin{itemize}
		\item Note $f$ is uniformly continuous because it is continuous on a compact set.
		\item However, $f$ is not Lipschitz continuous: for any constant $c>0$, set $x=1/(c+1)^2$ and $x'=0$ so that
		\[\left|\frac{f(x)-f(x')}{x-x'}\right|=\left|\frac{1/(c+1)}{1/(c+1)^2}\right|=|c+1|>c,\]
		so \autoref{rem:lipasratio} tells us that we are not Lipschitz continuous.
	\end{itemize}
\end{example}
By rearranging quantifiers, we get another useful (but weaker) notion.
\begin{restatable}[Continuous]{definition}{contdef}
	Given metric spaces $(X,d_X)$ and $(Y,d_Y)$, a function $f\colon X\to Y$ is \textit{continuous at $x\in X$} if and only if all $\varepsilon>0$ have some $\delta_x>0$ such that
	\[d_X(x,x')<\delta_x\implies d_Y(f(x),f(x'))<\varepsilon.\]
	Then $f$ is \textit{continuous} if and only if it is continuous at all $x\in X$.
\end{restatable}
\begin{example}
	All uniformly continuous functions $f\colon(X,d_X)\to(Y,d_Y)$ are continuous. Indeed, at any $x_0\in X$ with $\varepsilon>0$, uniform continuity promises $\delta>0$ so that
	\[|x-x'|<\delta\implies|f(x)-f(x')|<\varepsilon\]
	for all $x,x'\in X$. Setting $x'$ to $x_0$ recovers continuity.
\end{example}
\begin{example}
	Give $\RR$ the usual Euclidean metric, and set $f\colon\RR\to\RR$ by $f(x)\coloneqq x^2$.
	\begin{itemize}
		\item Note $f(x)$ is continuous because it is a polynomial.
		\item However, $f(x)$ is not uniformly continuous: take $\varepsilon=1$. Now, for any $\delta>0$, set $x=1/\delta$ and $x'=1/\delta+\delta/2$ so that $|x-x'|<\delta$, but
		\[|f(x)-f(x')|=\left(\frac1\delta+\frac\delta2\right)^2-\frac1{\delta^2}=1+\frac{\delta^2}4>\varepsilon.\]
	\end{itemize}
\end{example}
As usual, the identity function is uniformly continuous and continuous (it's an isometry), and these continuities are preserved by composition. We will have a different way to see that continuous functions remain continuous under composition later, so for now we will focus on uniform continuity.
\begin{lemma}
	Fix uniformly continuous morphisms $f\colon(X,d_X)\to(Y,d_Y)$ and $g\colon(Y,d_Y)\to(Z,d_Z)$. Then the function $(g\circ f)$ is uniformly continuous.
\end{lemma}
\begin{proof}
	For any $\varepsilon>0$, the uniform continuity of $g$ promises $\delta_g>0$ such that
	\[d_Y(y,y')<\delta_g\implies d_Z(g(y),g(y'))<\varepsilon\]
	for any $y,y'\in Y$. Continuing, the uniform continuity of $f$ promises $\delta_f>0$ such that
	\[d_X(x,x')<\delta_X\implies d_Y(f(x),f(x'))<\delta_Y\implies d_Z(g(f(x)),g(f(x')))<\varepsilon\]
	for any $x,x'\in X$, which is what we wanted.
\end{proof}
\begin{remark}
	In some sense, isometries and Lipschitz continuous functions have their definition fundamentally interrelated with the metric. In contrast, the weaker notion of continuity will readily generalize to general topological spaces. Uniform continuity also generalizes to ``uniformities,'' which is a different notion.
\end{remark}

\subsection{Convergence and Completeness}
To discuss completeness, we need to talk about convergence.
\begin{definition}[Converge]
	Fix a metric space $(X,d)$. A sequence of points $\{x_n\}_{n\in\NN}\subseteq X$ \textit{converges to} $x\in X$ if and only if, for any $\varepsilon>0$, we can find $N>0$ such that
	\[n>N\implies d(x_n,x)<\varepsilon.\]
	We might write this as ``$x_n\to x$ as $n\to\infty$'' or ``$\lim_{n\to\infty}x_n=x$.'' In this event, we may say that the sequence $\{x_n\}_{n\in\NN}$ \textit{converges}, and its limit is $x$.
\end{definition}
\begin{remark}[Nir] \label{rem:uniqlim}
	As a sanity check, the limit of a sequence is unique: if $x_n\to x$ and $x_n\to x'$ as $n\to\infty$, then any $\varepsilon>0$ can find some large $n$ so that $d(x_n,x),d(x_n,x')<\varepsilon/2$. As such,
	\[d(x,x')<d(x_n,x)+d(x_n,x')=\varepsilon\]
	for any $\varepsilon>0$, so $d(x,x')=0$ and thus $x=x'$ is forced.
\end{remark}
We have no reason yet to be convinced that any of our morphisms described previously are good notions, so let's start with continuity.
\begin{lemma} \label{lem:contpreservesconverge}
	Fix a continuous function between metric spaces $f\colon(X,d_X)\to(Y,d_Y)$. Then, if the sequence $\{x_n\}_{n\in\NN}\subseteq X$ converges to $x\in X$, then the sequence $\{f(x_n)\}_{n\in\NN}\subseteq Y$ converges to $f(x)\in Y$.
\end{lemma}
\begin{proof}
	For any $\varepsilon>0$,the continuity of $f$ implies that we can find $\delta_x>0$ so that
	\[d_X(x_n,x)<\delta_x\implies d_Y(f(x_n),f(x))<\varepsilon\]
	for any $x_n$. But the fact that $x_n\to x$ as $n\to\infty$ means that there is $N>0$ so that
	\[n>N\implies d_X(x_n,x)<\delta_x\implies d_Y(f(x_n),f(x))<\varepsilon,\]
	so indeed, $f(x_n)\to f(x)$ as $n\to\infty$.
\end{proof}
We would like a notion of convergence which only uses data internal to the sequence, and this leads to the following definition.
\begin{definition}[Cauchy]
	Fix a metric space $(X,d)$. A sequence of points $\{x_n\}_{n\in\NN}\subseteq X$ is a \textit{Cauchy sequence} if and only if, for any $\varepsilon>0$, we can find $N>0$ such that
	\[n,m>N\implies d(x_n,x_m)<\varepsilon.\]
\end{definition}
It would be rude if continuity was always the best kind of morphism, so this time around preserving Cauchy-ness requires something stronger.
\begin{lemma} \label{lem:uniformcontpreservecauchy}
	Fix a uniformly continuous function between metric spaces $f\colon(X,d_X)\to(Y,d_Y)$. Then, if the sequence $\{x_n\}_{n\in\NN}\subseteq X$ is Cauchy, then the sequence $\{f(x_n)\}_{n\in\NN}\subseteq Y$ is also Cauchy.
\end{lemma}
\begin{proof}
	For any $\varepsilon>0$, the uniform continuity of $f$ promises $\delta>0$ so that
	\[d_X(x_n,x_m)<\delta\implies d(f(x_n),f(x_m))<\varepsilon\]
	for any $x_n,x_m$. However, the fact that $\{x_n\}_{n\in\NN}$ is Cauchy promises $N$ so that
	\[n,m>N\implies d_X(x_n,x_m)<\delta\implies d(f(x_n),f(x_m))<\varepsilon,\]
	which is what we wanted.
\end{proof}
\begin{example}
	Continuous functions do not need to preserve Cauchy sequences: $f\colon(0,\infty)\to(0,\infty)$ by $f(x)\coloneqq 1/x$ is continuous, and the sequence $\{1/n\}_{n\in\NN}\subseteq(0,\infty)$ is Cauchy (it converges to $0$ in $\RR$) even though $\{f(1/n)\}_{n\in\NN}=\{n\}_{n\in\NN}$ certainly does not converge.
\end{example}
Anyway, it is quick to check that convergent sequences are Cauchy.
\begin{lemma}
	Fix a metric space $(X,d)$. Then all convergent sequences are Cauchy.
\end{lemma}
\begin{proof}
	Suppose that the sequence $\{x_n\}_{n\in\NN}\subseteq X$ converges to $x\in X$. Then, for any $\varepsilon>0$, find $N$ so that
	\[d(x_n,x)<\varepsilon/2\]
	for all $n>N$. Then any $n,m>N$ has
	\[d(x_n,x_m)\le d(x_n,x)+d(x_m,x)<\varepsilon,\]
	so the sequence $\{x_n\}_{n\in\NN}$ is Cauchy.
\end{proof}
We in general hope that our Cauchy sequences converge. As such, we have the following definition.
\begin{definition}[Complete]
	A metric space $(X,d)$ is \textit{complete} if and only if every Cauchy sequence in $X$ converges to a point in $X$.
\end{definition}
We are sad when a metric space is not complete, so we hope to have a way to make it complete. The most natural way to do this is by using the notion of density.
\begin{definition}[Density]
	Fix a metric space $(X,d)$. Then $S\subseteq X$ is \textit{dense} if and only if, given any $x\in X$ and $\varepsilon>0$, we may find $x'\in S$ with $d(x,x')<\varepsilon$.
\end{definition}
And here is our completion.
\begin{definition}[Completion]
	A \textit{completion} of the metric space $(X,d)$ is a metric space $(\overline X,\overline d)$ equipped with an isometry $\iota\colon X\to\overline X$ such that $(\overline X,\overline d)$ is complete and $\im\iota$ is dense in $\overline X$.
\end{definition}
One can show that any metric space has a completion and that they are all isometric and therefore in some sense the same. We'll do these separately.

\subsection{Existence of Completions}
Let's start with existence.
\begin{theorem}
	Any metric space $(X,d)$ has a completion.
\end{theorem}
\begin{proof}
	Let $\widetilde X$ denote the set of all Cauchy sequences in $X$. We hope to make $\widetilde X$ into our completion, but this requires a little care. To begin, we have the following lemma.
	\begin{lemma}
		Given a metric space $(X,d)$ with two Cauchy sequences $\{x_n\}_{n\in\NN}$ and $\{y_n\}_{n\in\NN}$, then the sequence
		\[\{d(x_n,y_n)\}_{n\in\NN}\subseteq\RR\]
		converges.
	\end{lemma}
	\begin{proof}
		Because $\RR$ is a complete metric space, it suffices to show that the sequence $\{d(x_n,y_n)\}_{n\in\NN}$ is Cauchy. Well, for any $\varepsilon>0$, find a sufficiently large $N$ so that
		\[n,m>N\implies d(x_n,x_m),d(y_n,y_m)<\varepsilon/2.\]
		Then any $n,m>N$ has
		\[d(x_n,y_n)\le d(x_n,x_m)+d(x_m,y_m)+d(y_m,y_n)<\varepsilon+d(y_m,y_n),\]
		and $d(x_m,y_m)<d(x_n,y_n)+\varepsilon$ as well by symmetry. It follows that any $n,m>N$ has
		\[\big|d(x_n,y_n)-d(x_m,y_m)\big|<\varepsilon,\]
		verifying that our sequence is Cauchy.
	\end{proof}
	\begin{remark} \label{rem:metriciscont}
		Here is a quick motivational remark for the definition of our metric below: if $(X,d)$ is a metric space with $x_n\to x$ and $y_n\to y$ as $n\to\infty$, then we claim $d(x_n,y_n)\to d(x,y)$ as $n\to\infty$. Indeed, for any $\varepsilon>0$, we can find $N$ large enough so that $d(x_n,x),d(y_n,y)<\varepsilon/2$ for any $n>N$. As such,
		\[d(x_n,y_n)\le d(x_n,x)+d(x,y)+d(y,y_n)<d(x,y)+\varepsilon.\]
		By symmetry, we get $d(x,y)\le d(x_n,y_n)+\varepsilon$ as well, finishing.
	\end{remark}
	Thus, we define $\widetilde d\colon\widetilde X\times\widetilde X\to\RR_{\ge0}$ by
	\[\widetilde d\left(\{x_n\},\{y_n\}\right)\coloneqq\lim_{n\to\infty}d(x_n,y_n).\]
	We claim that $\widetilde d$ is a semi-metric on $\widetilde X$. We have the following checks; fix Cauchy sequences $\{x_n\},\{y_n\},\{z_n\}$.
	\begin{itemize}
		\item Zero: note
		\[\widetilde d(\{x_n\},\{x_n\})=\lim_{n\to\infty}d(x_n,x_n)=0.\]
		\item Symmetry: note
		\[\widetilde d(\{x_n\},\{y_n\})=\lim_{n\to\infty}d(x_n,y_n)=\lim_{n\to\infty}d(y_n,x_n)=\widetilde d(\{y_n\},\{x_n\}).\]
		\item Triangle inequality: note
		\begin{align*}
			\widetilde d(\{x_n\},\{y_n\})+\widetilde d(\{y_n\},\{z_n\}) &= \lim_{n\to\infty}d(x_n,y_n)+\lim_{n\to\infty}d(y_n,z_n) \\
			&= \lim_{n\to\infty}(d(x_n,y_n)+d(y_n,z_n)) \\
			&\ge \lim_{n\to\infty}d(x_n,z_n) \\
			&= \widetilde d(x_n,z_n),
		\end{align*}
		where we have implicitly used a number of limit laws.
	\end{itemize}
	So because $\widetilde d$ is a semi-metric, \autoref{prop:semitometric} tells us that $\widetilde d$ will descend naturally to a metric $\overline d$ on $\overline X\coloneqq\widetilde X/{\sim}$, where $\{x_n\}\sim\{y_n\}$ if and only if $\widetilde d(\{x_n\},\{y_n\})=0$. We will let $[\{x_n\}]$ denote the equivalence class of the Cauchy sequence $\{x_n\}\in\widetilde X$ in $\overline X$.

	We now show that $(\overline X,\overline d)$ can be made into a completion for $X$.
	\begin{itemize}
		\item Given $x\in X$, note that the constant sequence $\{x\}$ is Cauchy (for any $\varepsilon>0$, set $N=0$), so we define $\iota\colon X\to\overline X$ by
		\[\iota(x)\coloneqq[\{x\}].\]
		To see that $\iota$ is an isometry, note any $x,x'\in X$ have
		\[\overline d(\iota(x),\iota(x'))=\widetilde d(\{x\},\{y\})=\lim_{n\to\infty}d(x,y)=d(x,y).\]
		\item We show that $\im\iota$ is dense in $\overline X$. Indeed, fix some $[\{x_n\}]\in\overline X$ and $\varepsilon>0$. Then there is some $N$ so that $n,m>N$ has
		\[d(x_n,x_m)<\varepsilon/2.\]
		Fixing a particular $n_0$ with $n_0>N$, we set $x\coloneqq x_{n_0}$ so that
		\[\overline d([\{x_n\}],\iota(x))=\widetilde d(\{x_n\},x_{n_0})=\lim_{n\to\infty}d(x_n,x_{n_0}).\]
		Now, for $n>N$, we have $d(x_n,x_{n_0})<\varepsilon/2$, so we conclude that this limit must be less than $\varepsilon$.
		\item We show that $(\overline X,\overline d)$ is a complete metric space. Fix a Cauchy sequence $\{\overline x_k\}$ in $\overline X$. To find the Cauchy sequence we are supposed to converge to, we use our density result: for each $k\in\NN$, we can find $y_k\in X$ such that $\overline d(\overline x_k,\iota(y_k))<1/k$.

		We claim that $\{y_k\}$ is Cauchy. Indeed, for any $\varepsilon>0$, we can find $N$ such that $k,\ell>N_0$ has
		\[\overline d(\overline x_k,\overline x_\ell)<\varepsilon/3.\]
		Then, setting $N\coloneqq\max\{3/\varepsilon,N_0\}$, we note that $k,\ell>N$ has
		\[d(y_k,y_\ell)=\overline d(\iota(y_k),\iota(d_\ell))\le\overline d(\overline x_k,\iota(y_k))+\overline d(\overline x_\ell,\iota(y_\ell))+\overline d(\overline x_k,\overline x_\ell)<\varepsilon.\]
		Lastly, we claim that $\overline x_k\to[\{y_n\}]$ in $\overline X$. Indeed, for any $\varepsilon>0$, find some sufficiently large $N$ so that
		\[k,\ell>N\implies d(y_k,y_\ell)<\varepsilon/2.\]
		Then $k>\max\{N,2/\varepsilon\}$ has
		\[\overline d(\overline x_k,[\{y_n\}])\le\overline d(\overline x_k,\iota(y_k))+\overline d([\{y_n\}],\iota(y_k))<\frac\varepsilon2+\lim_{n\to\infty}d(y_n,y_k).\]
		Because $k>N$, we have $d(y_n,y_k)<\varepsilon/2$ for any $n>N$, so the entire right-hand side must be upper-bounded by $\varepsilon$. This finishes.
	\end{itemize}
	The above checks complete the proof.
\end{proof}
\begin{remark}[Nir]
	One might complain that we used the completeness of $\RR$ in this proof because one common way to construct the real numbers is as the completion of $\QQ$ under the Euclidean metric. To remedy this, one ought to define the equivalence relation on Cauchy sequences more directly, saying that two Cauchy sequences $\{x_n\}_{n\in\NN}$ and $\{y_n\}_{n\in\NN}$ of real numbers are equivalent under $\sim$ if and only if
	\[\lim_{n\to\infty}d_\RR(x_n,y_n)=0.\]
\end{remark}

\subsection{Uniqueness of Completions}
We now show that any two completions of a metric space $(X,d)$ are isometric, which is our uniqueness result. Here is the main intermediate result.
\begin{lemma} \label{lem:completionup}
	Fix a metric space $(X,d)$ and a completion $(\overline X,\overline d)$ with its isometry $\iota\colon(X,d)\to(\overline X,\overline d)$. Then, for any complete metric space $(Y,d')$ and isometry $\varphi\colon(X,d)\to(Y,d')$, there is a unique isometry $\psi\colon(\overline X,\overline d)\to(Y,d')$ making the following diagram commute.
	% https://q.uiver.app/?q=WzAsMyxbMCwwLCJYIl0sWzEsMCwiXFxvdmVybGluZSBYIl0sWzEsMSwiWSJdLFswLDEsIlxcaW90YSJdLFswLDIsIlxcdmFycGhpIiwyXSxbMSwyLCJcXHBzaSIsMCx7InN0eWxlIjp7ImJvZHkiOnsibmFtZSI6ImRhc2hlZCJ9fX1dXQ==&macro_url=https%3A%2F%2Fraw.githubusercontent.com%2FdFoiler%2Fnotes%2Fmaster%2Fnir.tex
	\[\begin{tikzcd}
		X & {\overline X} \\
		& Y
		\arrow["\iota", from=1-1, to=1-2]
		\arrow["\varphi"', from=1-1, to=2-2]
		\arrow["\psi", dashed, from=1-2, to=2-2]
	\end{tikzcd}\]
\end{lemma}
\begin{proof}
	We start by showing the uniqueness of $\psi$. Well, for any $\overline x\in\overline X$, note that any $n\in\NN$ allows us to find $x_n\in X$ with
	\[\overline d(\overline x,\iota(x_n))<1/n\]
	because $\im\iota$ is dense in $\overline X$. Now, we notice that $\iota(x_n)\to\overline x$ as $n\to\infty$ because any $\varepsilon>0$ can set $N=1/\varepsilon$. As such, we see that \autoref{lem:contpreservesconverge} applied to any possible $\psi\colon\overline X\to Y$ forces
	\[\psi(\overline x)=\psi\Big(\lim_{n\to\infty}\iota(x_n)\Big)=\lim_{n\to\infty}\psi(\iota(x_n))=\lim_{n\to\infty}\varphi(x_n).\]
	Note that, a priori, we do not know if the sequence $\{\varphi(x_n)\}_{n\in\NN}$ converges, but this argument tells us that it must; the limit is unique by \autoref{rem:uniqlim}, so $\psi(\overline x)$ is unique as well.

	We now show that $\psi$ exists. As before, any $\overline x\in\overline X$ can find a sequence $\{x_n\}\subseteq X$ such that $\iota(x_n)\to\overline x$ as $n\to\infty$. Thus, we note that $\{\varphi(x_n)\}$ is Cauchy by \autoref{lem:uniformcontpreservecauchy}, so the completeness of $Y$ gives it a limit; we set
	\[\psi(\overline x)\coloneqq\lim_{n\to\infty}\varphi(x_n).\]
	We have the following checks on $\psi$.
	\begin{itemize}
		\item Well-defined: if we have two sequences $\{x_n\}$ and $\{x'_n\}$ such that $\iota(x_n)\to x$ and $\iota(x'_n)\to x$ as $n\to\infty$, we need to show that
		\[\lim_{n\to\infty}\varphi(x_n)=\lim_{n\to\infty}\varphi(x'_n).\]
		For brevity, set $y$ and $y'$ to be the limits of $\{\varphi(x_n)\}$ and $\{\varphi(x'_n)\}$, respectively. Then, for any $\varepsilon>0$, we note that there is a sufficiently large $N$ such that
		\[n>N\implies d_Y(y,\varphi(x_n)),d_Y(y',\varphi(x'_n))<\varepsilon/4.\]
		Further, we can make $N$ even larger so that
		\[n>N\implies\overline d(\overline x,\iota(x_n)),\overline d(\overline x,\iota(x'_n))<\varepsilon/4.\]
		As such, any $n>N$ has
		\begin{align*}
			d_Y(y,y') &\le d_Y(y,\varphi(x_n))+d_Y(\varphi(x_n),\varphi(x'_n))+d_Y(y',\varphi(x'_n)) \\
			&< \varepsilon/4+d_X(x_n,x'_n)+\varepsilon/4 \\
			&= \varepsilon/2+\overline d(\iota(x_n),\iota(x'_n)) \\
			&\le \varepsilon/2+\overline d(\overline x,\iota(x_n))+\overline d(\overline x,\iota(x'_n)) \\
			&< \varepsilon.
		\end{align*}
		It follows $d_Y(y,y')=0$, so $y=y'$.
		\item Isometry: given $\overline x,\overline x'\in\overline X$, find sequences $\{x_n\}$ and $\{x'_n\}$ in $X$ so that $\iota(x_n)\to\overline x$ and $\iota(x'_n)\to\overline x'$ as $n\to\infty$. Thus,
		\begin{align*}
			d_Y(\psi(\overline x),\psi(\overline x')) &= d_Y\Big(\lim_{n\to\infty}\varphi(x_n),\lim_{n\to\infty}\varphi(x'_n)\Big) \\
			&\stackrel*= \lim_{n\to\infty}d_Y(\varphi(x_n),\varphi(x'_n)) \\
			&= \lim_{n\to\infty}d(x_n,x'_n) \\
			&= \lim_{n\to\infty}\overline d(\iota(x_n),\iota(x'_n)) \\
			&= \overline d\Big(\lim_{n\to\infty}\iota(x_n),\lim_{n\to\infty}\iota(x'_n)\Big) \\
			&\stackrel*= \overline d(\overline x,\overline x'),
		\end{align*}
		where we have used \autoref{rem:metriciscont} at the $\stackrel*=$.
		\item For any $x\in X$, we see that the (constant) Cauchy sequence $\{\iota(x)\}$ converges to $\iota(x)$, so
		\[\psi(\iota(x))=\lim_{n\to\infty}\varphi(x)=\varphi(x).\]
		It follows $\psi\circ\iota=\varphi$.
	\end{itemize}
	Thus, we have finished establishing the existence of an isometry $\psi\colon\overline X\to Y$ such that $\varphi=\psi\circ\iota$.
\end{proof}
\begin{remark}
	One can also replace all isometries with uniformly continuous functions in the statement.
\end{remark}
And here is our uniqueness result.
\begin{theorem}
	Fix a metric space $(X,d)$ and two completions $\iota\colon(X,d)\to(\overline X,\overline d)$ and $\iota'\colon(X,d)\to(\overline X',\overline d')$. Then there is a surjective isometry $\psi\colon(\overline X,\overline d)\to(\overline X',\overline d')$.
\end{theorem}
\begin{proof}
	Applying \autoref{lem:completionup} twice, we get isometries $\psi\colon(\overline X,\overline d)\to(\overline X',\overline d')$ and $\psi'\colon(\overline X',\overline d')\to(\overline X,\overline d)$ making the following diagrams commute.
	% https://q.uiver.app/?q=WzAsNixbMCwwLCJYIl0sWzEsMCwiXFxvdmVybGluZSBYIl0sWzEsMSwiXFxvdmVybGluZSBYJyJdLFsyLDAsIlgiXSxbMywwLCJcXG92ZXJsaW5lIFgnIl0sWzMsMSwiXFxvdmVybGluZSBYIl0sWzAsMSwiXFxpb3RhIl0sWzAsMiwiXFxpb3RhJyIsMl0sWzEsMiwiXFxwc2kiXSxbMyw0LCJcXGlvdGEnIl0sWzQsNSwiXFxwc2knIl0sWzMsNSwiXFxpb3RhIiwyXV0=&macro_url=https%3A%2F%2Fraw.githubusercontent.com%2FdFoiler%2Fnotes%2Fmaster%2Fnir.tex
	\[\begin{tikzcd}
		X & {\overline X} & X & {\overline X'} \\
		& {\overline X'} && {\overline X}
		\arrow["\iota", from=1-1, to=1-2]
		\arrow["{\iota'}"', from=1-1, to=2-2]
		\arrow["\psi", from=1-2, to=2-2]
		\arrow["{\iota'}", from=1-3, to=1-4]
		\arrow["{\psi'}", from=1-4, to=2-4]
		\arrow["\iota"', from=1-3, to=2-4]
	\end{tikzcd}\]
	In particular, we see that $\psi'\circ\psi$ makes the following diagram commute.
	% https://q.uiver.app/?q=WzAsMyxbMCwwLCJYIl0sWzEsMCwiXFxvdmVybGluZSBYIl0sWzEsMSwiXFxvdmVybGluZSBYIl0sWzAsMSwiXFxpb3RhIl0sWzAsMiwiXFxpb3RhIiwyXSxbMSwyLCJcXHBzaSdcXGNpcmNcXHBzaSJdXQ==&macro_url=https%3A%2F%2Fraw.githubusercontent.com%2FdFoiler%2Fnotes%2Fmaster%2Fnir.tex
	\[\begin{tikzcd}
		X & {\overline X} \\
		& {\overline X}
		\arrow["\iota", from=1-1, to=1-2]
		\arrow["\iota"', from=1-1, to=2-2]
		\arrow["{\psi'\circ\psi}", from=1-2, to=2-2]
	\end{tikzcd}\]
	However, using \autoref{lem:completionup} again, this isometry $\psi'\circ\psi$ is unique to make the diagram commute, and we could of course put the isometry $\id_{\overline X}$ here if we wanted to. Thus,
	\[\psi'\circ\psi={\id_{\overline X}}.\]
	By symmetry, $\psi\circ\psi'=\id_{\overline X'}$, so we do see that $\psi$ and $\psi'$ are inverse isometries. This finishes the proof.
\end{proof}

\end{document}