% !TEX root = ../notes.tex

\documentclass[../notes.tex]{subfiles}

\begin{document}

Good morning everyone.

\subsection{Some Examples}
Let's give some more examples of metric spaces. Let's start with spaces of continuous functions.
\begin{definition}
	We denote the $\RR$-vector space of $\CC$-valued continuous function from a topological space $X$ as $C(X)$.
\end{definition}
And here are our two examples.
\begin{ex}
	Give $V\coloneqq C([0,1])$ the uniform norm
	\[\norm f_\infty\coloneqq\sup\{|f(t)|:t\in[0,1]\}.\]
	Then $V$ is complete. Indeed, this is merely the statement that a sequence of continuous functions which are uniformly Cauchy will converge uniformly to a continuous function.
\end{ex}
\begin{example}
	Give $V\coloneqq C([0,1])$ the $L^1$ norm as
	\[\norm f_1\coloneqq\int_0^1|f(t)|\,dt.\]
	Then $V$ is not complete: for each $n\in\NN$, define $f_n$ as the continuous function connecting $(0,1)$ to $(1/2,1)$ to $(1/2+1/n,0)$ to $(1,0)$. Then $f_n$ is trying to converge to a discontinuous function.%\todo{Add image.}
	% from the following graph.
	% \begin{center}
	% 	\begin{asy}
	% 		draw((1.5,0) -- (0,0) -- (1.5,0));
	% 		draw((0,1) -- (1/2,1) -- (1/2+1/5,0) -- (1,0), red);
	% 		label("$f_n$", (1/2,1/2), W, red);
	% 	\end{asy}
	% \end{center}
\end{example}
\begin{remark}
	In fact, all $p$-norms $\norm{\cdot}_p$ for finite $p\ge1$ are never complete. This motivates the theory of measure and integration, which will let us specify which functions we actually want to look at.
\end{remark}
\begin{remark}
	The $\norm{\cdot}_2$ norm on $C(X)$ for some (say) subset $X\subseteq\RR$ with finite measure as coming from an inner product
	\[\langle f,g\rangle\coloneqq\int_Xf(t)\overline{g(t)}\,dt.\]
	When $\norm{\cdot}_2$ is complete, we would then get a Hilbert space, which are very nice normed vector spaces, and we'll see more of them in Math 202B.
\end{remark}

\end{document}