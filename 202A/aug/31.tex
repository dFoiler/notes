% !TEX root = ../notes.tex

\documentclass[../notes.tex]{subfiles}

\begin{document}

\section{August 31}

It is once again the morning.

\subsection{Intersections of Topologies}
We will want to have lots of topologies to work with. Here is a basic way to build them.
\begin{proposition} \label{prop:intersecttops}
	Let $X$ be a set, and pick up some collection of topologies $\{\mc T_\alpha\}_{\alpha\in\lambda}$. Then the intersection
	\[\mc T\coloneqq\bigcap_{\alpha\in\lambda}\mc T_\alpha\]
	is also a topology on $X$.
\end{proposition}
\begin{proof}
	This is mostly a matter of writing out the axioms.
	\begin{listalph}
		\item Note that $\emp,X\in\mc T_\alpha$ for each $\alpha$, so $\emp,X\in\mc T$. 
		\item Arbitrary union: given a collection $\mc U\subseteq\mc T$, we have $\mc U\subseteq\mc T_\alpha$ for each $\alpha$, so $\bigcup_{U\in\mc U}U\in\mc T_\alpha$ for each $\alpha$, so
		\[\bigcup_{U\in\mc U}U\in\mc T\]
		as well.
		\item Finite intersection: given a finite collection $\{U_1,\ldots,U_n\}\subseteq\mc T$, we have $\{U_1,\ldots,U_n\}\subseteq\mc T_\alpha$ for each $\alpha$, so $\bigcap_{i=1}^nU_i\in\mc T_\alpha$ for each $\alpha$, so
		\[\bigcap_{i=1}^nU_i\in\mc T\]
		follows.
		\qedhere
	\end{listalph}
\end{proof}
\begin{corollary}
	Fix a set $X$. Given a collection $\mc S\subseteq\mc P(X)$, there is a smallest topology $\mc T$ containing $\mc S$.
\end{corollary}
\begin{proof}
	Certainly there is some topology containing $\mc S$, namely the discrete topology $\mc P(X)$. Thus, we can set our topology to be
	\[\mc T(\mc S)\coloneqq\bigcap_{\substack{\mc T\supseteq\mc S\\\mc T\text{ a topology}}}\mc T,\]
	which is a topology (by \autoref{prop:intersecttops}) which contains $\mc S$ (because each topology in the intersection contains $\mc S$), and of course any topology $\mc T$ containing $\mc S$ will have $\mc T(\mc S)\subseteq\mc T$.
\end{proof}
To codify this idea, we have the following idea.
\begin{defihelper}[Generated topology] \nirindex{Topology!Generated topology}
	Fix a set $X$. We say that a collection $\mc S\subseteq\mc P(X)$ \textit{generates} its smallest topology $\mc T$. We will write $\mc T(\mc S)$ for this topology.
\end{defihelper}
\begin{remark}[Nir]
	The topology $\mc T(\mc S)$ is unique. Indeed, suppose two topologies $\mc T$ and $\mc T'$ are minimal topologies containing $\mc S$. Then $\mc T\cap\mc T'$ is also a topology containing $\mc S$ by \autoref{prop:intersecttops}, but $\mc T\cap\mc T'\subseteq\mc T,\mc T'$ forces $\mc T=\mc T\cap\mc T'=\mc T'$.
\end{remark}
\begin{remark}[Nir] \label{rem:largersubbase}
	Given collections $\mc S\subseteq\mc S'$, then $\mc T(\mc S)\subseteq\mc T(\mc S')$. Indeed, we have
	\[\mc T(\mc S)=\bigcap_{\substack{\mc T\supseteq\mc S\\\mc T\text{ a topology}}}\mc T\subseteq\bigcap_{\substack{\mc T\supseteq\mc S'\\\mc T_0\text{ a topology}}}\mc T=\mc T(\mc S').\]
\end{remark}
\begin{remark}[Nir] \label{rem:topgenitself}
	If $\mc T$ is already a topology on $X$, then $\mc T(\mc T)=\mc T$. Indeed, of course $\mc T\subseteq\mc T(\mc T)$, but then also
	\[\mc T(\mc T)=\bigcap_{\substack{\mc T'\supseteq\mc T\\\mc T'\text{ a topology}}}\mc T'\subseteq\mc T\]
	because $\mc T$ is a topology containing $\mc T$.
\end{remark}

\subsection{Sub-bases}
On the other side of things, we pick up the following definition.
\begin{definition}[Sub-base]
	Let $(X,\mc T)$ be a topological space. A collection $\mc S\subseteq\mc T$ is a \textit{sub-base} for $\mc T$ if and only if the following hold.
	\begin{listalph}
		\item $\mc S$ covers $X$, in that $X=\bigcup_{U\in\mc S}U$.
		\item $\mc T$ is generated by $\mc S$.
	\end{listalph}
\end{definition}
The point is that collections $\mc S$ are easy to find, so we have therefore found many topologies.

It will be useful to give a more concrete description of the topology generated by a collection $\mc S$. We start by taking finite intersections.
\begin{lemma} \label{lem:finintersect}
	Fix a set $X$ and a collection $\mc S\subseteq\mc P(X)$ with $X=\bigcup_{U\in\mc S}U$. Then set
	\[\mc I^\mc S\coloneqq\Bigg\{\bigcap_{i=1}^nU_i:\{U_i\}_{i=1}^n\subseteq\mc S\Bigg\}.\]
	Then $\mc S\subseteq\mc I^\mc S$ and $\mc I^\mc S$ is closed under finite intersection. Further, the topology generated by $\mc I^\mc S$ is also the topology generated by $\mc S$.
\end{lemma}
\begin{proof}
	We show the claims in sequence
	\begin{itemize}
		\item That $\{U\}\subseteq\mc S$ for any $U\in\mc S$ implies that $U\in\mc I^\mc S$ for any $U\in\mc S$, so $\mc S\subseteq\mc I^\mc S$ follows.
		\item To show $\mc I^\mc S$ is closed under finite intersection, pick up some finite collection $\{U_1,\ldots,U_n\}\subseteq\mc I^\mc S$. Then, for each $i$, we can find some finite collection $\mc U_i\subseteq\mc S$ such that
		\[U_i=\bigcap_{V\in\mc U_i}V.\]
		Setting $\mc U\coloneqq\bigcup_{i=1}^n\mc U_i$, we see that $\mc U$ is finite and that
		\[\bigcap_{i=1}^nU_i=\bigcap_{i=1}^n\bigcap_{V\in\mc U_i}V=\bigcap_{V\in\mc U}V\]
		must live in $\mc I^\mc S$.
		\item Because $\mc S\subseteq\mc I^\mc S$, \autoref{rem:largersubbase} tells us $\mc T(\mc S)\subseteq\mc T\left(\mc I^\mc S\right)$. In the other direction, note that any finite collection $\{U_1,\ldots,U_n\}\subseteq\mc S$ also lives in $\mc T(\mc S)$, so
		\[\bigcap_{i=1}^nU_i\in\mc T(\mc S).\]
		It follows $\mc I^\mc S\subseteq\mc T(\mc S)$, so $\mc T\left(\mc I^\mc S\right)\subseteq\mc T(\mc T(\mc S))=\mc T(\mc S)$ by \autoref{rem:topgenitself}.
		\qedhere
	\end{itemize}
\end{proof}
After taking finite intersections, we take arbitrary unions.
\begin{lemma} \label{lem:takearbunion}
	Fix a set $X$ and a collection $\mc I\subseteq\mc P(X)$ closed under finite intersection with $\bigcup_{U\in\mc I}U=X$. Then the collection of (arbitrary) unions of elements in $\mc I$, denoted
	\[\mc T\coloneqq\Bigg\{\bigcup_{U\in\mc U}U:\mc U\subseteq\mc I\Bigg\},\]
	is $\mc T(\mc I)$.
\end{lemma}
\begin{proof}
	If $\mc T'$ is a topology containing $\mc I$, then note any collection $\mc U\subseteq\mc I$ lives in $\mc T'$, so the arbitrary union
	\[\bigcup_{U\in\mc U}U\]
	lives in $\mc T'$. It follows that $\mc T\subseteq\mc T'$, so
	\[\mc T\subseteq\bigcap_{\substack{\mc T'\supseteq\mc T\\\mc T'\text{ a topology}}}\mc T'=\mc T(\mc I).\]
	Thus, it remains to show that $\mc T$ is in fact a topology, which will imply from $\mc I\subseteq\mc T$ that $\mc T(\mc I)\subseteq\mc T(\mc T)=\mc T$ by \autoref{rem:largersubbase}. Here are our checks.
	\begin{itemize}
		\item Setting $\mc U=\emp\subseteq\mc I$, we see that $\bigcup_{U\in\mc U}U=\emp$, so $\emp\in\mc T$. Also, by hypothesis, we have
		\[X=\bigcup_{U\in\mc I}U\in\mc T.\]
		\item Arbitrary union: let $\mc U\subseteq\mc T$ be a subcollection. For any $U\in\mc U$, we can find a collection $\mc V_U\subseteq\mc I$ such that
		\[U=\bigcup_{V\in\mc V_U}V.\]
		Now, we set $\mc V$ to be the union of all the collections of $\mc V_U$ for each $U\in\mc U$, which is still contained in $\mc I$, so that
		\[\bigcup_{U\in\mc U}U=\bigcup_{U\in\mc U}\bigcup_{V\in\mc V_U}V=\bigcup_{V\in\mc V}V\in\mc T.\]
		\item Finite intersection: by induction, it suffices to pick up two sets $U,V\in\mc T$ and show $U\cap V\in\mc T$. Well, we can find collections $\mc U,\mc V\subseteq\mc I$ such that
		\[U=\bigcup_{U'\in\mc U}U'\qquad\text{and}\qquad V=\bigcup_{V'\in\mc V}V',\]
		from which it follows (by distribution) that
		\[U\cap V=\Bigg(\bigcup_{U'\in\mc U}U'\Bigg)\cap\Bigg(\bigcup_{V'\in\mc V}V'\Bigg)=\bigcup_{U'\in\mc U}\Bigg(U'\cap\bigcup_{V'\in\mc V}V'\Bigg)=\bigcup_{\substack{U'\in\mc U\\V'\in\mc V}}(U'\cap V').\]
		Now, $\mc I$ is closed under finite intersection, so $U'\cap V'\in\mc I$, so we have witnessed $U\cap V$ as an arbitrary union of elements of $\mc I$, so $U\cap V\in\mc T$ follows.
		\qedhere
	\end{itemize}
\end{proof}
\begin{corollary} \label{cor:concretesubbase}
	Fix a set $X$ and a collection $\mc S\subseteq\mc P(X)$ with $X=\bigcup_{U\in\mc S}U$. Letting $\mc I^\mc S$ be the collection of finite intersections of $\mc S$ and then $\mc T$ be the collection of arbitrary unions of $\mc I^\mc S$, we have that $\mc T=\mc T(\mc S)$.
\end{corollary}
\begin{proof}
	By \autoref{lem:finintersect}, we have $\mc T(\mc S)=\mc T\left(\mc I^\mc S\right)$. Plugging $\mc I^\mc S$ into \autoref{lem:takearbunion} (which applies because $\mc I^\mc S$ is closed under finite intersection and covers $X$ because $\mc S\subseteq\mc I^\mc S$), we see that $\mc T\left(\mc I^\mc S\right)=\mc T$, finishing.
\end{proof}
We quickly point out that the point of discussing sub-bases is that we will be allowed to check continuity on only a sub-base.
\begin{lemma} \label{lem:topoindonefunc}
	Fix a topological space $(X,\mc T_X)$ and a set $Y$. Given a function $f\colon X\to Y$, the collection
	\[\mc T(f)\coloneqq\left\{U\subseteq Y:f^{-1}(U)\in\mc T_X\right\}\]
	forms a topology on $Y$.
\end{lemma}
\begin{proof}
	Here are our checks.
	\begin{itemize}
		\item Note $f^{-1}(\emp)=\emp\in\mc T_X$, so $\emp\in\mc T(f)$. Also, $f^{-1}(Y)=X\in\mc T_X$, so $Y\in\mc T(f)$.
		\item Arbitrary union: given a collection $\mc U\subseteq\mc T(f)$, we see that
		\[f^{-1}\Bigg(\bigcup_{U\in\mc U}U\Bigg)=\bigcup_{U\in\mc U}f^{-1}(U)\]
		is a union of elements of $\mc T_X$ and therefore in $\mc T_X$. Thus, $\bigcup_{U\in\mc U}U\in\mc T(f)$.
		\item Finite intersection: this is identical to the previous check. Given a finite collection $\{U_1,\ldots,U_n\}\in\mc T(f)$, we see that
		\[f^{-1}\Bigg(\bigcap_{i=1}^nU_i\Bigg)=\bigcap_{i=1}^nf^{-1}(U_i)\]
		is a finite intersection of elements of $\mc T_X$ and therefore in $\mc T_X$. Thus, $\bigcap_{i=1}^nU_i\in\mc T(f)$.
		\qedhere
	\end{itemize}
\end{proof}
\begin{proposition} \label{prop:checkonsubbase}
	Fix topological spaces $(X,\mc T_X)$ and $(Y,\mc T_Y)$, and let $\mc S$ be a sub-base for $\mc T_Y$. Then a function $f\colon X\to Y$ is continuous if and only if
	\[f^{-1}(U)\in\mc T_X\]
	for all $U\in\mc S$.
\end{proposition}
\begin{proof}
	Certainly if $f$ is continuous then the pre-image of any open set $U\in\mc S\subseteq\mc T_Y$ must be open. On the other hand, let $\mc T(f)\subseteq\mc P(Y)$ be the collection of subsets $U$ for which $f^{-1}(U)\in\mc T_X$. This is a topology by \autoref{lem:topoindonefunc}, and it contains $\mc S$ by hypothesis, so it follows
	\[\mc T_Y=\mc T(\mc S)\subseteq\mc T(f).\]
	Thus, $f^{-1}(U)\in\mc T_X$ for any $U\in\mc T_Y$, so $f$ is continuous.
\end{proof}

\subsection{Bases}
Having defined a sub-base, we should be rightly upset that we have not defined a base.
\begin{definition}[Base]
	Fix a set $X$. A collection $\mc B\subseteq\mc P(X)$ is a \textit{base} (for a topology on $X$) if and only if the collection of arbitrary unions of $\mc B$ form a topology on $X$.
\end{definition}
This definition is a little hard to access because we still don't have a good notion of what a topology is.
\begin{example} \label{ex:fininterisbase}
	Fix a set $X$. Given any collection $\mc S\subseteq\mc P(X)$, the collection of finite intersections $\mc I^\mc S$ is a base by \autoref{lem:takearbunion}.
\end{example}
However, in general we do not require a base to be closed under finite intersection.
\begin{example}
	Fix a metric space $(X,d)$. Then the collection of open balls $\mc B$ forms a topology by \autoref{ex:metrictopo}. Notably, the intersection of two open balls need not be an open ball, as follows.
	\begin{center}
		\begin{asy}
			unitsize(0.7cm);
			fill(arc((0,0),2,-60,60) -- cycle, lightgray);
			fill(arc((2,0),2,180-60,180+60) -- cycle, lightgray);
			draw(circle((0,0),2), dashed);
			draw(circle((2,0),2), dashed);
		\end{asy}
	\end{center}
\end{example}
Even though bases are not closed under finite intersection, we do have the following.
\begin{prop}
	Fix a set $X$ and a collection $\mc B\subseteq\mc P(X)$. Then $\mc B$ is a base if and only if
	\begin{listalph}
		\item $X=\bigcup_{B\in\mc B}B$, and
		\item any $B_1,B_2\in\mc B$ has some collection $\mc U\subseteq\mc B$ such that
		\[B_1\cap B_2=\bigcup_{B\in\mc U}B.\]
	\end{listalph}
\end{prop}
\begin{proof}
	In one direction, suppose that $\mc B$ is a base generating the topology $\mc T$.
	\begin{listalph}
		\item Because $X\in\mc T$, we see that $X$ is the union of some subcollection $\mc U\subseteq\mc B$, so it follows
		\[X=\bigcup_{U\in\mc U}U\subseteq\bigcup_{B\in\mc B}B\subseteq X.\]
		\item Given $B_1,B_2\in\mc B\subseteq\mc T$, we see that $B_1\cap B_2\in\mc T$, so because $\mc T$ is made of arbitrary unions of $\mc B$, there is a collection $\mc U\subseteq\mc B$ such that
		\[B_1\cap B_2=\bigcup_{B\in\mc U}B.\]
	\end{listalph}
	We now go in the other direction. Suppose $\mc B$ satisfies (a) and (b), and define
	\[\mc T\coloneqq\Bigg\{\bigcup_{U\in\mc U}U:\mc U\subseteq\mc B\Bigg\}.\]
	We now check that $\mc T$ is a topology.
	\begin{itemize}
		\item Using $\mc U=\emp\subseteq\mc B$, so we see that $\bigcup_{U\in\mc U}U=\emp$ is in $\mc T$. Also, by (a), we have
		\[X=\bigcup_{B\in\mc B}B\in\mc T.\]
		\item Arbitrary union: this is the same as the check in \autoref{lem:takearbunion}. Given a collection $\mc U\subseteq\mc T$, each $U\in\mc U$ has some collection $\mc V_U\subseteq\mc B$ such that $\bigcup_{V\in\mc V_U}V=U$. Letting $\mc V\subseteq\mc B$ be the union of all the $\mc V_U$, we see
		\[\bigcup_{U\in\mc U}U=\bigcup_{U\in\mc U}\bigcup_{V\in\mc V_U}V=\bigcup_{V\in\mc V}V\]
		lives in $\mc T$.
		\item Finite intersection: by induction, it suffices to pick up $U_1,U_2\in\mc T$ and show $U_1\cap U_2\in\mc T$. Well, find $\mc B_1,\mc B_2\subseteq\mc B$ such that
		\[U_1=\bigcup_{B_1\in\mc B_1}B_1\qquad\text{and}\qquad U_2=\bigcup_{B_2\in\mc B_2}B_2,\]
		which implies
		\[U_1\cap U_2=\bigcup_{\substack{B_1\in\mc B_1\\B_2\in\mc B_2}}(B_1\cap B_2).\]
		Now, (b) implies that $B_1\cap B_2$ for any $B_1,B_2\in\mc B$ is a union of elements in $\mc B$, so $B_1\cap B_2\in\mc T$. Thus, $U_1\cap U_2$ is the arbitrary union of elements in $\mc T$, so $U_1\cap U_2\in\mc T$ by the previous check.
		\qedhere
	\end{itemize}
\end{proof}
\begin{remark}[Nir]
	Careful readers might realize that we could rearrange the given exposition to show that, given a sub-base $\mc S$, the collection of finite intersections $\mc I^\mc S$ is a base instead of going through \autoref{lem:takearbunion}.
\end{remark}
\begin{remark}
	Of course, any base is also a sub-base. Notably, sub-bases only require that $X=\bigcup_{U\in\mc S}U$, which must be satisfied for bases.
\end{remark}
\begin{example}
	Set $X=\RR$ with the usual topology $\mc T$. Then the collection $\mc B$ of open intervals $(a,b)$ form a base for the usual topology (these are our open balls). In contrast, the collection
	\[\mc S=\{(-\infty,a):a\in\RR\}\cup\{(a,\infty):a\in\RR\}\]
	forms a sub-base for the usual topology. Namely, certainly $\mc S\subseteq\mc T$, and $\mc B\subseteq\mc T(\mc S)$ because of the finite intersection $(-\infty,b)\cap(a,\infty)=(a,b)$ for any $a,b\in\RR$. Namely, $\mc T=\mc T(\mc B)\subseteq\mc T(\mc T(\mc S))=\mc T(\mc S)$ follows.
\end{example}

\subsection{Induced Topologies}
We start with the following motivating example.
\begin{example}
	Fix a set $X$, and give it the discrete topology. Then, for any topological space $(Y,\mc T_Y)$, any function $f\colon X\to Y$ is continuous because the pre-image of any open subset $U_Y\subseteq Y$ is open in $X$.
\end{example}
In general, we might have some smallish collection of functions which we want to force to be continuous, so we might ask what topology is forced by their continuity.
\begin{defihelper}[Induced topology] \nirindex{Topology!Induced topology}
	Fix a set $X$ and a collection of topologies $\{(Y_\alpha,\mc T_\alpha)\}_{\alpha\in\lambda}$ with some functions $f_\alpha\colon X\to Y_\alpha$ for each $\alpha\in\lambda$. Then
	\[\bigcup_{\alpha\in\lambda}\left\{f_\alpha^{-1}(U_\alpha):U_\alpha\in\mc T_\alpha\right\}\]
	is a sub-base for an \textit{induced topology}.
\end{defihelper}
The one thing to check is that $X$ belongs to the arbitrary unions of our collection, which is clear because $X=f^{-1}_\alpha(Y_\alpha)$.
\begin{defihelper}[Relative topology] \nirindex{Topology!Relative topology}
	Fix $(Y,\mc T)$ a topological space. Then the \textit{relative topology} for a subset $X\subseteq Y$ is the topology induced by the natural embedding $\iota\colon X\into Y$.
\end{defihelper}
We have the following more concrete description.
\begin{lemma} \label{lem:betterrelative}
	Fix $(Y,\mc T_Y)$ a topological space. Then the relative topology for a subset $X\subseteq Y$ consists of the subsets
	\[\left\{X\cap U:U\in\mc T_Y\right\}.\]
\end{lemma}
\begin{proof}
	Let $\iota\colon X\into Y$ be the natural embedding. Then we are given the sub-base
	\[\mc S\coloneqq\left\{\iota^{-1}(U):U\in\mc T_Y\right\}.\]
	Now, $\iota^{-1}(U)=X\cap U$, and then we can check directly that this collection $\mc S$ gives a topology and finish by \autoref{rem:topgenitself}. Here are the checks, which should be completely routine by now.
	\begin{itemize}
		\item Note $\emp\in\mc T_Y$ implies $\emp=X\cap\emp\in\mc S$. Also, $Y\in\mc T_Y$ implies $X=X\cap Y\in\mc S$.
		\item Arbitrary union: given a collection $\mc U\subseteq\mc S$, for each $U\in\mc U$ find $U_V\in\mc T_Y$ such that $U=X\cap U_V$. Then
		\[\bigcup_{U\in\mc U}=U=\bigcup_{U\in\mc U}X\cap U_V=X\cap\underbrace{\bigcup_{U\in\mc U}U_V}_{\in\mc T_Y}\]
		lives in $\mc S$.
		\item Finite intersection: given a finite collection $\{U_1,\ldots,U_n\}\subseteq\mc S$, find $V_i\in\mc T_Y$ such that $U_i=X\cap V_i$. Then
		\[\bigcap_{i=1}^nU_i=\bigcap_{i=1}^n(X\cap V_i)=X\cap\underbrace{\bigcap_{i=1}^nV_i}_{\in\mc T_Y}\]
		lives in $\mc S$.
		\qedhere
	\end{itemize}
\end{proof}
% It turns out that we can talk about this induced topology by universal property.
% \begin{lemma}
% 	Fix a set $X$ and a collection of topologies $\{(Y_\alpha,\mc T_\alpha)\}_{\alpha\in\lambda}$ with some functions $f_\alpha\colon X\to Y_\alpha$ for each $\alpha\in\lambda$. Now, equip $X$ with the induced topology by $\mc T$. Then, given a topological space $(Z,\mc T_Z)$ with continuous maps $g_\alpha\colon Z\to Y_\alpha$, there is a unique continuous map $g\colon Z\to X$ such that
% 	\[g_\alpha=f_\alpha\circ g.\]
% \end{lemma}
% \begin{proof}
% 	We show existence and uniqueness separately.
% 	\begin{itemize}
% 		\item Uniqueness: 
% 	\end{itemize}
% \end{proof}

\end{document}