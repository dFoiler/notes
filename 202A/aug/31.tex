% !TEX root = ../notes.tex

\documentclass[../notes.tex]{subfiles}

\begin{document}

It is once again the morning.

\subsection{Intersections of Topologies}
We will want to have lots of topologies to work with. Here is a basic way to build them.
\begin{proposition} \label{prop:intersecttops}
	Let $X$ be a set, and pick up some collection of topologies $\{\mc T_\alpha\}_{\alpha\in\lambda}$. Then the intersection
	\[\mc T\coloneqq\bigcap_{\alpha\in\lambda}\mc T_\alpha.\]
\end{proposition}
\begin{proof}
	This is mostly a matter of writing out the axioms.
	\begin{listalph}
		\item Note that $\emp,X\in\mc T_\alpha$ for each $\alpha$, so $\emp,X\in\mc T$. 
		\item Arbitrary union: given a collection $\mc U\subseteq\mc T$, we have $\mc U\subseteq\mc T_\alpha$ for each $\alpha$, so $\bigcup_{U\in\mc U}U\in\mc T_\alpha$ for each $\alpha$, so
		\[\bigcup_{U\in\mc U}U\in\mc T\]
		as well.
		\item Finite intersection: given a finite collection $\{U_1,\ldots,U_n\}\subseteq\mc T$, we have $\bigcap_{i=1}^nU_i\in\mc T_\alpha$ for each $\alpha$, so
		\[\bigcap_{i=1}^nU_i\in\mc T\]
		follows.
		\qedhere
	\end{listalph}
\end{proof}
\begin{corollary}
	Fix a set $X$. Given a collection $\mc S\subseteq\mc P(X)$, there is a smallest topology $\mc T$ containing $\mc S$.
\end{corollary}
\begin{proof}
	Certainly there is some topology containing $\mc S$, namely the discrete topology. Thus, we can set our topology to be
	\[\bigcap_{\substack{\mc T\supseteq\mc S\\\mc T\text{ a topology}}}\mc T,\]
	which is a topology (by \autoref{prop:intersecttops}) which contains $\mc S$.
\end{proof}
To codify this idea, we have the following idea.
\begin{notation}[Generated topology]
	Fix a set $X$. We say that a collection $\mc S\subseteq\mc P(X)$ \textit{generates} its smallest topology $\mc T$.
\end{notation}
\begin{remark}
	It is fairly clear that this smallest topology $\mc T$ generated by $\mc S$ is unique.
\end{remark}

\subsection{Sub-bases and Bases}
On the other side of things, we pick up the following definition.
\begin{definition}[Sub-base]
	Let $(X,\mc T)$ be a topological space. A collection $\mc S\subseteq\mc T$ is a \textit{sub-base} for $\mc T$ if and only if the following hold.
	\begin{listalph}
		\item $\mc S$ covers $X$, in that $X=\bigcup_{U\in\mc S}U$.
		\item $\mc T$ is generated by $\mc S$.
	\end{listalph}
\end{definition}
The point is that collections $\mc S$ are easy to find, so we have therefore found many topologies.

It will be useful to give a more concrete description of the topology generated by a collection $\mc S$.
\begin{lemma}
	Fix a set $X$ and a collection $\mc S\subseteq\mc P(X)$ with $X=\bigcup_{U\in\mc S}U$. Then set
	\[\mc I^\mc S\coloneqq\Bigg\{\bigcap_{i=1}^nU_i:\{U_i\}_{i=1}^n\subseteq\mc S\Bigg\}.\]
	Then $\mc S\subseteq\mc I^\mc S$ and $\mc I^\mc S$ is closed under finite intersection. Further, the topology generated by $\mc I$ is also the topology generated by $\mc S$.
\end{lemma}
\begin{proof}
	Omitted.
\end{proof}
Continuing, we have the following.
\begin{lemma} \label{cor:takearbunion}
	Fix a set $X$ and a collection $\mc I\subseteq\mc P(X)$ closed under finite intersection with $\bigcup_{U\in\mc I}U=X$. Then the collection of (arbitrary) unions of elements in $\mc I$, denoted
	\[\mc T\coloneqq\Bigg\{\bigcap_{U\in\mc U}U:\mc U\subseteq\mc I\Bigg\},\]
	is the smallest topology containing $\mc I$.
\end{lemma}
\begin{proof}
	Certainly any topology containing $\mc I$ must contain $\mc T$. Note $\emp\in\mc T$ because it is an empty union. Continuing, $\mc T$ is closed under arbitrary union because the union of unions is a union.

	Lastly, to show that $\mc T$ is closed under finite intersection, it suffices by induction to show that sets $U,V\in\mc T$ have $U\cap V\in\mc T$. However, we can find $\mc U\mc V\subseteq\mc T$ so that
	\[U=\bigcup_{U'\in\mc U}U'\qquad\text{and}\qquad\bigcup_{V'\in\mc V}V',\]
	from which it follows
	\[U\cap V=\bigcup_{\substack{U'\in\mc U\\V'\in\mc V}}\underbrace{(U'\cap V')}_{\in\mc I},\]
	which shows $U\cap V\in\mc T$.
\end{proof}
\begin{corollary}
	Fix a set $X$ and a collection $\mc S\subseteq\mc P(X)$ with $X=\bigcup_{U\in\mc S}U$. Letting $\mc I^\mc S$ be the collection of finite intersections of $\mc S$ and then $\mc T$ be the collection of arbitrary unions of $\mc I^\mc S$, we have that $\mc T$ is the topology generated by $\mc S$.
\end{corollary}
\begin{proof}
	Combine the previous two lemmas.
\end{proof}
\begin{remark} \label{rem:checkcontinuity}
	The point of discussing sub-bases is that we will be allowed to check continuity on only a sub-base.
\end{remark}
Having defined a sub-base, we should be rightly upset that we have not defined a base.
\begin{definition}[Base]
	Fix a set $X$. A collection $\mc B\subseteq\mc P(X)$ is a \textit{base} (for a topology on $X$) if and only if the collection of arbitrary unions of $\mc B$ form a topology on $X$.
\end{definition}
This definition is a little hard to access because we still don't have a good notion of what a topology is.
\begin{example}
	Fix a set $X$. Given any collection $\mc S\subseteq\mc P(X)$, the collection of finite intersections $\mc I^\mc S$ is a base by \autoref{cor:takearbunion}.
\end{example}
However, in general we do not require a base to be closed under finite intersection.
\begin{example}
	Fix a metric space $(X,d)$. Then the collection of open balls $\mc B$ forms a topology by \autoref{ex:metrictopo}. Notably, the intersection of two open balls need not be an open ball, as follows.
	\begin{center}
		\begin{asy}
			unitsize(1cm);
			fill(arc((0,0),2,-60,60) -- cycle, lightgray);
			fill(arc((2,0),2,180-60,180+60) -- cycle, lightgray);
			draw(circle((0,0),2), dashed);
			draw(circle((2,0),2), dashed);
		\end{asy}
	\end{center}
\end{example}
Even though we are not closed under finite intersection, we do have the following.
\begin{lemma}
	Fix a set $X$ with a base $\mc B\subseteq\mc P(X)$ for a topology $\mc T$. Then any finite intersection of elements of $\mc B$ is an arbitrary union of elements of $\mc B$.
\end{lemma}
\begin{proof}
	The finite intersection must live in $\mc T$ given by $\mc B$, but any element of $\mc T$ is an arbitrary union of elements of $\mc B$.
\end{proof}
\begin{remark}
	Of course, any base is also a sub-base. Notably, sub-bases only require that $X=\bigcup_{U\in\mc S}U$, which must be satisfied for bases.
\end{remark}
\begin{example}
	Set $X=\RR$ with the usual topology. Then the open intervals $(a,b)$ form a base for the usual topology (these are our open balls), and the collection
	\[\mc S=\{(-\infty,a):a\in\RR\}\cup\{(a,\infty):a\in\RR\}\]
	forms a sub-base for the usual topology because $(-\infty,b)\cap(a,\infty)=(a,b)$.
\end{example}
Let's go ahead and \autoref{rem:checkcontinuity}.
\begin{proposition}
	Fix topological spaces $(X,\mc T_X)$ and $(Y,\mc T_Y)$, and let $\mc S$ be a sub-base for $\mc T_Y$. Then a function $f\colon X\to Y$ is continuous if and only if
	\[f^{-1}(U)\in\mc T_X\]
	for all $U\in\mc S$.
\end{proposition}
\begin{proof}
	Certainly if $f$ is continuous then the pre-image of the open set $U\in\mc S$ must be open. On the other hand, let $\mc T_Y'\subseteq\mc P(Y)$ be the collection of subsets $U$ for which $f^{-1}(U)\in\mc T_X$. We claim that $\mc T_Y'$ is a topology.
	
	Certainly $f^{-1}(\emp)=\emp$, so $\emp\in\mc T'_Y$. Also, note $\mc T_Y'$ is closed under finite intersection: given $U,V\in\mc T_Y'$, we have
	\[f^{-1}(U\cap V)=f^{-1}(U)\cap f^{-1}(V)\in\mc T_X.\]
	Additionally, $\mc T_Y'$ is closed under arbitrary union: given $\mc U\subseteq\mc T'_Y$, we have
	\[f^{-1}\Bigg(\bigcup_{U\in\mc U}U\Bigg)=\bigcup_{U\in\mc U}f^{-1}(U)\in\mc T_X.\]
	Thus, $\mc T'_Y$ is indeed a topology, and it contains $\mc S$, so it follows that $\mc T_Y\subseteq\mc T'_Y$, so $f^{-1}(U)\in\mc T_X$ for all $U\in\mc T_Y$. This finishes the check that $f$ is continuous.
\end{proof}

\subsection{Induced Topologies}
We start with the following motivating example.
\begin{example}
	Fix a set $X$, and give it the discrete topology. Then, for any topological space $(Y,\mc T_Y)$, any function $f\colon X\to Y$ is continuous because the pre-image of any open subset $U_Y\subseteq Y$ is open in $X$.
\end{example}
In general, we might have some smallish collection of functions which we want to force to be continuous, so we might ask what topology is forced by their continuity.
\begin{defi}[Induced topology]
	Fix a set $X$ and a collection of topologies $\{(Y_\alpha,\mc T_\alpha)\}_{\alpha\in\lambda}$ with some functions $f_\alpha\colon X\to Y_\alpha$ for each $\alpha\in\lambda$. Then
	\[\bigcup_{\alpha\in\lambda}\left\{f_\alpha^{-1}(U_\alpha):U_\alpha\in\mc T_\alpha\right\}\]
	is a sub-base for an \textit{induced topology}.
\end{defi}
The one thing to check is that $X$ belongs to the arbitrary unions of our collection, which is clear because $X=f^{-1}_\alpha(Y_\alpha)$.
\begin{definition}[Relative topology]
	Fix $(Y,\mc T)$ a topological space. Then the \textit{relative topology} for a subset $X\subseteq Y$ is the topology induced by the natural embedding $\iota\colon X\into Y$.
\end{definition}
We have the following more concrete description.
\begin{lemma}
	Fix $(Y,\mc T_Y)$ a topological space. Then the relative topology for a subset $X\subseteq Y$ consists of the subsets
	\[\left\{X\cap U:U\in\mc T_Y\right\}.\]
\end{lemma}
\begin{proof}
	Let $\iota\colon X\into Y$ be the natural embedding. Then we are given the sub-base
	\[\mc S\coloneqq\left\{\iota^{-1}(U):U\in\mc T_Y\right\}.\]
	Now, $\iota^{-1}(U)=X\cap U$, and then we can check directly that this collection gives a topology.
\end{proof}

\end{document}