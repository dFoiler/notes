% !TEX root = ../notes.tex

\documentclass[../notes.tex]{subfiles}

\begin{document}

\section{August 24}

Good morning everyone. This is my first class of the semester.

\subsection{Administrative Notes}
Here are some housekeeping remarks.
\begin{itemize}
	\item The webpage for this class is \texttt{\href{https://math.berkeley.edu/~rieffel/202AannF22.html}{math.berkeley.edu/~rieffel/202AannF22.html}}.
	\item The midterm date is negotiable. We will have a vote on Friday. The possible dates are Friday 14 October, Monday 17 October, or Wednesday 19 October.
	\item There will be no vote on the final exam. It is on 15 December at 7PM.
	\item Homework will be due Fridays by midnight, approximately every week.
	\item There is no particular text for this course, and any given text covers more than we have time for. That said, we will (very) loosely follow \cite{lang-analysis}, but it is helpful to have a number of different expositions around.
	\item Please wear a mask during lectures and office hours.
\end{itemize}
Here is a summary of the course.
\begin{itemize}
	\item We will spend the next couple of lectures talking about metric spaces.
	\item We will then spend the first half of the course on general topology. The second half of the course will be on measure and integration.
	\item Throughout we will see a little on functional analysis.
\end{itemize}

\subsection{Metric Spaces}
Hopefully we remember something about metric spaces. Here's the definition.
\begin{definition}[Metric] \label{defi:metric}
	A \textit{metric} $d$ on a set $X$ is a function $d\colon X\times X\to\RR_{\ge0}$ satisfying the following rules for any $x,y,z\in X$.
	\begin{listalph}
		\item Zero: $d(x,x)=0$.
		\item Zero: $d(x,y)=0$ implies $x=y$.
		\item Symmetry: $d(x,y)=d(y,x)$.
		\item Triangle inequality: $d(x,y)+d(y,z)\ge d(x,z)$.
	\end{listalph}
	We call $(X,d)$ a \textit{metric space}.
\end{definition}
We will want some ``almost'' metrics as well. Here are their names.
\begin{defihelper}[Semi-metric] \nirindex{Metric!Semi-metric}
	A \textit{semi-metric} $d$ on a set $X$ satisfies (a), (c), and (d) of \autoref{defi:metric}. We call $(X,d)$ a \textit{semi-metric space}.
\end{defihelper}
\begin{defihelper}[Extended metric] \nirindex{Metric!Extended metric}
	An \textit{extended metric} $d$ on a set $X$ is a function $d\colon X\times X\to\RR_{\ge0}^\infty$ satisfying (a)--(d) of \autoref{defi:metric}. We call $(X,d)$ an \textit{extended metric space}.
\end{defihelper}
Intuitively, we might want extended metrics if we have points that we never want to be able to get to from other ones.

We can turn spaces with a semi-metric into a space with a metric.
\begin{lemma} \label{lem:semitometric}
	Fix a semi-metric space $(X,d)$, and define the relation $\sim$ on $X$ by $x\sim y$ if and only if $d(x,y)=0$. Then $\sim$ is an equivalence relation.
\end{lemma}
\begin{proof}
	We run these checks by hand. Fix any $x,y,z\in X$.
	\begin{itemize}
		\item Reflexive: $d(x,x)=0$ means that $x\sim x$.
		\item Symmetry: if $x\sim y$, then $d(x,y)=0$, so $d(y,x)=0$, so $y\sim x$.
		\item Transitive: if $x\sim y$ and $y\sim z$, then
		\[0\le d(x,z)\le d(x,y)+d(y,z)=0,\]
		so $d(x,z)=0$, so $x\sim z$.
		\qedhere
	\end{itemize}
\end{proof}
As such, given a semi-metric space $(X,d)$, we may look at the set of equivalence classes under $\sim$, which we will denote $X/{\sim}$.\footnote{The notation of $/{\sim}$ is intended to make us think of quotients.}
\begin{proposition} \label{prop:semitometric}
	Fix a semi-metric space $(X,d)$ and define $\sim$ as in \autoref{lem:semitometric}. Then $d$ naturally descends to a metric $\widetilde d$ on $X/{\sim}$.
\end{proposition}
\begin{proof}
	Let $[x]$ denote the equivalence class of $x\in X$ under $\sim$. We claim that the function
	\[\widetilde d([x],[y])\coloneqq d(x,y)\]
	is a well-defined metric. We have the following checks; fix any $x,y,z\in X$.
	\begin{itemize}
		\item Well-defined: if $x\sim x'$ and $y\sim y'$, then note that
		\[d(x,y)\le d(x,x')+d(x',y)=d(x',y)\le d(x',y')+d(y',y)=d(x',y').\]
		By symmetry, we also have $d(x',y')\le d(x,y)$, so equality follows. So $d$ does descent properly to the quotient $X/{\sim}$.
		\item Zero: note that $\widetilde d([x],[y])=0$ if and only if $d(x,y)=0$ if and only if $x\sim y$ if and only if $[x]=[y]$.
		\item Symmetry: note that
		\[\widetilde d([x],[y])=d(x,y)=d(y,x)=\widetilde d([y],[x]).\]
		\item Triangle inequality: note that
		\[\widetilde d([x],[z])=d(x,z)\le d(x,y)+d(y,z)=\widetilde d([x],[y])+\widetilde d([y],[z]),\]
		which finishes.
		\qedhere
	\end{itemize}
\end{proof}
Here are some examples of metric spaces.
\begin{example}
	Given a connected graph $G=(V,E)$ with a weighting function $w\colon E\to\RR_{\ge0}$, we can build a metric as follows: define the ``shortest-path'' function $d\colon V\times V\to\RR_{\ge0}$ sending two vertices $v,w\in V$ to the length of the shortest path. If the graph $G$ is not connected, we merely have an extended metric.
\end{example}
\begin{example}[Euclidean metric] \label{ex:euclidean}
	The function $d\colon\RR^n\times\RR^n\to\RR_{\ge0}$
	\[d\big((x_1,\ldots,x_n),(y_1,\ldots,y_n)\big)\coloneqq\sqrt{\sum_{i=1}^n(x_i-y_i)^2}\]
	is a metric.
\end{example}
Observe that it is not completely obvious that \autoref{ex:euclidean} satisfies the triangle inequality, but this will follow from the theory of the next subsections.

\subsection{Norms on Vector Spaces}
Norms provide convenient ways to build metrics.
\begin{definition}[Norm]
	Fix a vector space $V$ over $\RR$ or $\CC$. A norm $\lVert\cdot\rVert\colon V\to\RR_{\ge0}$ is a function satisfying the following, for any $r\in\RR$ and $v,w\in V$.
	\begin{listalph}
		\item Zero: $\lVert v\rVert=0$ if and only if $v=0$.
		\item Scaling: $\lVert rv\rVert=|r|\cdot\lVert v\rVert$.
		\item Triangle inequality: $\lVert v+w\rVert\le\lVert v\rVert+\lVert w\rVert$.
	\end{listalph}
\end{definition}
\begin{remark}
	We can probably work with a more general normed field instead of ``merely'' $\RR$ or $\CC$.
\end{remark}
And here is our result.
\begin{proposition}
	Given a metric space $V$ with a norm $\norm{\cdot}\colon V\to\RR_{\ge0}$, then the function
	\[d(v,w)\coloneqq\norm{v-w}\]
	defines a metric on $V$.
\end{proposition}
\begin{proof}
	We run the checks directly. Let $x,y,z\in V$ be points.
	\begin{itemize}
		\item Zero: note that $d(x,y)=0$ if and only if $\norm{x-y}=0$ if and only if $x-y=0$ if and only if $x=y$.
		\item Symmetry: note that
		\[d(x,y)=\norm{x-y}=|-1|\cdot\norm{y-x}=1\cdot\norm{y-x}=d(y,x).\]
		\item Triangle inequality: note that
		\[d(x,z)=\norm{x-z}=\norm{(x-y)+(y-z)}\le\norm{x-y}+\norm{y-z}=d(x,y)+d(y,z),\]
		which finishes the check.
		\qedhere
	\end{itemize}
\end{proof}
Here are the usual examples.
\begin{example}
	Set $V\coloneqq\RR^n$ or $V\coloneqq\CC^n$. Then the following are norms on $V$.
	\begin{itemize}
		\item $\norm{(x_1,\ldots,x_n)}_2\coloneqq\left(\sum_{i=1}^n|x_i|^2\right)^{1/2}$.
		\item $\norm{(x_1,\ldots,x_n)}_1\coloneqq\sum_{i=1}^n|x_i|$.
	\end{itemize}
\end{example}
Here are some more esotetric examples.
\begin{example}
	Set $V\coloneqq\RR^n$ or $V\coloneqq\CC^n$. Then
	\[\norm{(x_1,\ldots,x_n)}_\infty\coloneqq\sup\{|x_1|,\ldots,|x_n|\}\]
	provides a norm on $V$.
\end{example}
\begin{example} \label{ex:pnormonr}
	Set $V\coloneqq\RR^n$ or $V\coloneqq\CC^n$. Then, given $p\ge1$,
	\[\norm{(x_1,\ldots,x_n)}_p\coloneqq\Bigg(\sum_{i=1}^n|x_i|^p\Bigg)^{1/p}\]
	provides a norm on $V$.
\end{example}
\begin{remark}
	Taking the limit as $p\to\infty$ of $\norm f_p$ gives $\norm f_\infty$. This justifies the notation.
\end{remark}
\begin{remark}
	Despite having lots of examples, all of these norms are equivalent in a topological sense.
\end{remark}
These normed vector spaces actually allow us to define a metric on any subset.
\begin{proposition}
	Given a metric space $(X,d)$ and a subset $Y\subseteq X$, the restriction of $d$ to $Y\times Y$ is a metric.
\end{proposition}
\begin{proof}
	All the requirements for $d$ on $Y\times Y$ are satisfied for any points in $X$, so we are done by doing no work.
\end{proof}
\begin{example}
	Any subset $X\subseteq\RR^n$ has an induced metric by restricting the (say) Euclidean metric.
\end{example}

\subsection{A Hint of \texorpdfstring{$L^p$}{Lp} Spaces}
Here is a more complicated example of a metric.
\begin{example} \label{ex:contfuncnorm}
	Define $V\coloneqq C([0,1])$ to be the $\RR$-vector space of $\RR$-valued (or $\CC$-valued) continuous functions on $[0,1]$. The following are norms.
	\begin{itemize}
		\item $\norm f_\infty\coloneqq\sup\{|f(x)|\colon x\in[0,1]\}$.
		\item $\norm f_1\coloneqq\int_0^1|f(t)|\,dt$.
		\item $\norm f_2\coloneqq\left(\int_0^1|f(t)|^2\,dt\right)^{1/2}$.
		\item More generally, given $p\ge1$
		\[\norm f_p\coloneqq\left(\int_0^1|f(t)|^p\,dt\right)^{1/p}.\]
	\end{itemize}
	These integrals are finite because $[0,1]$ is compact, forcing $f$ to achieve a finite maximum on $[0,1]$.
\end{example}
\begin{remark}
	We can tell the same story for $C(X)$, for any measurable compact space $X$.
\end{remark}
\begin{remark} \label{rem:hintoflp}
	Note the analogy of \autoref{ex:contfuncnorm} with \autoref{ex:pnormonr}. To see this more rigorously, set $X$ to be the finite set $\{1,\ldots,n\}$ so that $C(X)=\RR^n$.
\end{remark}
We should probably justify the claims of this subsection, so here is our result.
\begin{proposition}
	Define $V\coloneqq C([0,1])$ to be the vector space of $\RR$-valued (or $\CC$-valued) continuous functions on $[0,1]$. Then, given $p\ge1$, the function $\norm{\cdot}_p\colon C\to\RR_{\ge0}$ by
	\[\norm f\coloneqq\left(\int_0^1|f(t)|^p\,dt\right)^{1/p}\]
	is a norm.
\end{proposition}
\begin{proof}
	We run the checks directly.
	\begin{itemize}
		\item Zero: if $f=0$, then of course $\int_0^1|f(t)|^p\,dt=0$.
		\item Zero: suppose that $f\in C([0,1])$ has $f(t_0)\ne0$ for any $t_0\in[0,1]$; set $y\coloneqq f(t_0)$. Then $f^{-1}((y/2,3y/2))$ is a nonempty open subset of $X$ and hence contains a nonempty open interval $(a,b)$ with $a<b$. As such,
		\[\int_X|f(t)|^p\,dt\ge\int_{a}^b|f(t)|^p\,dt\ge\int_{a}^b|y/2|^p\,dt>0,\]
		so we are done.
		\item Scaling: given $f\in C([0,1])$ and a scalar $r$, we have
		\[\norm{rf}=\left(\int_0^1|rf(t)|^p\,dt\right)^{1/p}=\left(|r|^p\int_0^1|f(t)|^p\,dt\right)^{1/p}=|r|\cdot\norm f.\]
		\item Triangle inequality: we borrow from \cite{lpspacestao}. Given $f,g\in C([0,1])$, for psychological reasons we will assume that $f$ and $g$ are nonzero (else this is clear); then $\norm f,\norm g\ne0$, so we may scale everything so that $\norm f+\norm g=1$. In fact, we may again use scaling to find $a,b\in V$ such that
		\[f=(1-\theta)a\qquad\text{and}\qquad g=\theta b\]
		where $\theta\in(0,1)$ and $\norm a=\norm b=1$. Now, the triangle inequality translates into showing
		\[\int_0^1|(1-\theta)a(t)+\theta b(t)|^p\,dt=\norm{(1-\theta)a+\theta b}^p_p\stackrel?\le\left(\norm{(1-\theta)a}_p+\norm{\theta b}_p\right)^p=1.\]
		Well, because $p\ge1$, the function $t\mapsto t^p$ is convex, so we get to write
		\[\int_0^1|(1-\theta)a(t)+\theta b(t)|^p\,dt\le(1-\theta)\int_0^1|a(t)|^p\,dt+\theta\int_0^1|b(t)|^p\,dt,\]
		which is what we wanted.
	\end{itemize}
	The above checks complete the proof; note that the proof of the triangle inequality was nontrivial.
\end{proof}
\begin{remark}
	Now, to show \autoref{rem:hintoflp}, replace all $\int_0^1$ with $\sum_{i=1}^n$ and adjust all the language accordingly. The point is that ``integrating over $[0,1]$'' is analogous to ``integrating over $\{1,\ldots,n\}$.'' A more thorough understanding of measure theory will allow us to rigorize this.
\end{remark}
Next class we will talk about completeness.

\end{document}