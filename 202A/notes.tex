% LTeX: enabled=false

\documentclass[openany]{book}
\usepackage[utf8]{inputenc}

\newcommand{\nirpdftitle}{202A Notes}
\usepackage{import}
\inputfrom{..}{nir}

\pagestyle{contentpage}

\title{202A: Introduction to Topology and Analysis}
\author{Nir Elber}
\date{Fall 2022}
\rhead{\textit{202A: TOPOLOGY AND ANALYSIS}}

\begin{document}

\maketitle

\toctrue
\tableofcontents
\tocfalse

\newpage

\chapter{Metric Spaces}

\epigraph{My personal view on spaces is that every space I ever work with is either metrizable or is the Zariski topology.}
{---Evan Chen, \cite{napkin}}

\foreach \n in {24,26}
{
	\subfile{aug/\n}
}

\subfile{aug/29metric}

\part{Topology}

\chapter{Building Topologies}

\epigraph{Sets are not doors.}
{---Munkres}

\subfile{aug/29topo}

\foreach \n in {31}
{
	\subfile{aug/\n}
}

\foreach \n in {2,7}
{
	\subfile{sep/\n}
}

\chapter{Building Functions}

\epigraph{I can assure you, at any rate, that my intentions are honourable and my results invariant, probably canonical, perhaps even functorial.}
{---Andre Weil, \cite{weil-functorial}}

\foreach \n in {9,12,14,16}
{
	\subfile{sep/\n}
}

\chapter{Compactness}

\epigraph{That something so small could be so beautiful.}
{---Anthony Doerr, \cite{light-we-cannot-see}}

\subfile{sep/16compact}

\foreach \n in {19,21,23,26,28,30}
{
	\subfile{sep/\n}
}

% \chapter{Compactness for Metric Spaces}

% \epigraph{Rarely is a picture a proof, but I hope a good picture will cement your understanding of why something is true. Seeing is believing.}
% {---Charles C. Pugh, \cite{pugh-analysis}}

% \foreach \n in {28,30}
% {
% 	\subfile{sep/\n}
% }

\foreach \n in {3}
{
	\subfile{oct/\n}
}

\part{Measure Theory}

\chapter{Defining Measures}

\epigraph{One fish, two fish, red fish, blue fish.}
{---Dr. Suess, \cite{one-fish-two-fish}}

\foreach \n in {5,7,10,12}
{
	\subfile{oct/\n}
}

\chapter{Building Measures}

\epigraph{So the man gave him the bricks, and he built his house with them.}
{---Joseph Jacobs, ``The Story of the Three Little Pigs'' \cite{english-fairy-tales}}

\foreach \n in {14,17,21,24,26}
{
	\subfile{oct/\n}
}

\chapter{Measurable Functions}

\subfile{oct/26int}

\foreach \n in {28,31}
{
	\subfile{oct/\n}
}

\foreach \n in {2}
{
	\subfile{nov/\n}
}

\chapter{Integration}

\epigraph{Having thus refreshed ourselves in the oasis of a proof, we now turn again into the desert of definitions}
{---Theodor Br{\"o}cker and Klaus J{\"a}nich, \cite{brocker-janich-diff-top}}

\subfile{nov/2int}

\foreach \n in {4,7,9,14,16}
{
	\subfile{nov/\n}
}

\nirprintbib
\nirprintindex

\end{document}