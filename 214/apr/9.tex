% !TEX root = ../notes.tex

\documentclass[../notes.tex]{subfiles}

\begin{document}

The sub is back. Today we continue discussing vector bundles.

\subsection{Bundle Homomorphisms}
Any reasonable object has a notion of morphisms between them. Here are the morphisms of vector bundles.
\begin{definition}[bundle homomorphism]
	Fix two smooth vector bundles $\pi\colon E\to M$ and $\pi'\colon E'\to M'$. Then a \textit{bundle homomorphism} $(F,f)\colon\pi\to\pi'$ is the data of a smooth map $f\colon M\to M'$ and smooth map $F\colon E\to E'$ such that
	% https://q.uiver.app/#q=WzAsNCxbMCwwLCJFIl0sWzEsMCwiRSciXSxbMCwxLCJNIl0sWzEsMSwiTSciXSxbMiwzLCJmIl0sWzAsMSwiRiJdLFswLDIsIlxccGkiLDJdLFsxLDMsIlxccGknIl1d&macro_url=https%3A%2F%2Fraw.githubusercontent.com%2FdFoiler%2Fnotes%2Fmaster%2Fnir.tex
	\[\begin{tikzcd}
		E & {E'} \\
		M & {M'}
		\arrow["f", from=2-1, to=2-2]
		\arrow["F", from=1-1, to=1-2]
		\arrow["\pi"', from=1-1, to=2-1]
		\arrow["{\pi'}", from=1-2, to=2-2]
	\end{tikzcd}\]
	commutes, and the restricted maps $F\colon E_p\to E'_{f(p)}$ are linear. If $f=\id_M$, we say that $F$ is a \textit{bundle homomorphism over $M$}; we denote the set of all bundle homomorphisms $E\to E'$ over $M$ as $\op{Hom}_M(E,E')$.
\end{definition}
\begin{remark}
	Note the commutativity of the diagram implies that $F$ does in fact map $E_p=\pi^{-1}(\{p\})$ to $E'_{f(p)}=(\pi')^{-1}(\{f(p)\})$.
\end{remark}
\begin{remark}
	We note that the function $f$ is uniquely determined by $F$. Indeed, suppose we have two functions $f_1$ and $f_2$ such that $\pi'\circ F=f_i\circ\pi$ for each $i$. Well, $\pi$ is surjective, so $f_1\circ\pi=f_2\circ\pi$ implies that $f_1=f_2$.
\end{remark}
\begin{remark}
	Suppose that we are given some smooth $F$ for which some function $f$ exists with $\pi'\circ F=f\circ\pi$. Then note that $\pi$ is a smooth surjective submersions, so to check that $f_0$ is smooth, it is enough to check that $f\circ\pi$ is smooth. But this is $\pi'\circ F$, so we get our smoothness from the smoothness of $F$.
\end{remark}
The point is that the bundle homomorphism is uniquely given by the data of $F$.
\begin{remark}
	The set $\op{Hom}_M(E,E')$ is in bijection with global sections $\Gamma(E^*\otimes E')$. The main point is that, on fibers, we see
	\[(E^*\otimes E')_p=E_p^*\otimes E_p'=\op{Hom}(E_p,E_p'),\]
	where the last map is by $\varphi\otimes v'\mapsto(v\mapsto\varphi(v)v')$, which is checked to be an isomorphism by hand. Thus, a global section $M\to E^*\otimes E'$ provides a family of linear maps $E_p\to E'_p$, which can be checked to be smooth. Conversely, a bundle homomorphism $F\colon E\to E'$ provides maps $F_p\colon E_p\to E'_p$ on fibers, which then provides an element of $(E^*\otimes E')_p$ as above, which will in total assemble into a smooth section by some examination on charts (where the question is about trivial vector bundles on Euclidean spaces).
\end{remark}
Here is a cute application.
\begin{lemma}
	Fix a smooth manifold $M$. Then the following are equivalent.
	\begin{listalph}
		\item $M$ is parallelizable.
		\item $M$ has a global frame for $TM$.
		\item The vector bundle $TM$ is trivial.
		\item There is a bundle isomorphism $TM\to M\times\RR^k$.
	\end{listalph}
\end{lemma}
\begin{proof}
	We already know that (a) and (b) are equivalent. Further, (c) and (d) are equivalent by definition of ``trivial.'' Lastly, (b) and (c) are equivalent because the global frame provides equivalent data to the isomorphism in (d).
\end{proof}
\begin{notation}
	Given a bundle homomorphism $F\colon E\to E'$ of vector bundles over $M$, we induce a map $\Gamma(F)\colon\Gamma(E)\to\Gamma(E')$ by sending a global section $\sigma\colon M\to E$ to the global section $(F\circ\sigma)\colon M\to E'$.
\end{notation}
\begin{remark}
	The fact that $F\circ\sigma$ is actually a vector bundle follows because $F$ is a homomorphism of vector bundles over $M$. Anyway, the above notation turns global sections $\Gamma$ into a functor, which we won't bother to check. For example, we note that $\Gamma(F)$ is a $C^\infty(M)$-linear map, essentially because composition is linear.
\end{remark}
The point of introducing $\Gamma(F)$ is that we are able to detect bundle homomorphisms purely on the level of our functions.
\begin{lemma}
	Fix smooth vector bundles $\pi\colon E\to M$ and $\pi'\colon E'\to M$ on the smooth manifold $M$. Then $\Gamma$ provides a bijection between $\op{Hom}_M(E,E')$ and $\op{Hom}_{C^\infty(M)}(\Gamma(E),\Gamma(E'))$.
\end{lemma}
\begin{proof}[Sketch]
	We begin with the injectivity check. Suppose $F_1,F_2\in\op{Hom}_M(E,E')$ satisfy $\Gamma(F_1)=\Gamma(F_2)$. Then we want to check that $F_1(v)=F_2(v)$ for any $v\in E$. Say $p\coloneqq\pi(v)$, and we at least know that $F_1(v),F_2(v)\in E'_p$ because our homomorphisms are over $M$. Now, choose some section $\sigma\colon M\to E$ such that $\sigma(p)=v$, which exists by some sort of partition of unity argument. Then $F_1\circ\sigma=F_2\circ\sigma$ implies $F_1(v)=F_2(v)$.

	We now turn to the surjectivity check. Suppose we are given some $\mc F\colon\Gamma(E)\to\Gamma(E')$ which is $C^\infty(M)$-linear. For any $v\in E$, set $p\coloneqq\pi(v)$, and we can find some smooth section $\sigma\colon M\to E$ such that $\sigma(p)=v$. Then $\mc F(\sigma)(p)\in E'$, and we can check using the linearity that we must have $\mc F(\sigma)(p)\in E'_p$. So we define $F(v)\coloneqq\mc F(\sigma)(p)$. It then remains to check that $F$ does not depend on the choice of $\sigma$ (this is true essentially by the $C^\infty(M)$-linearity requiring that the output of $\mc F$ cannot really adjust too much) and that $F$ is smooth (which can be checked locally, where the discussion becomes explicit as trivial bundles on Euclidean spaces, so by providing local frames to everything, $F$ basically becomes a matrix whose coefficients are smooth functions by hypothesis on $F$!).
\end{proof}
\begin{example}
	There is a canonical isomorphism $E\to E^{**}$. Well, on fibers, there is a natural isomorphism $E_p\to E_p^{**}$ given by $v\mapsto(\varphi\mapsto\varphi(v))$. Checking on charts (where the discussion becomes checking some linear map of trivial vector bundles on Euclidean spaces), we see that we are basically sending a global frame to its double dual global frame identically, which is certainly smooth.
\end{example}

\subsection{Subbundles}
A special kind of bundle homomorphism is given by a subbundle. Here is our definition.
\begin{definition}[subbundle]
	Fix a smooth vector bundle $\pi\colon E\to M$ on a smooth manifold $M$. Then a submanifold $D\subseteq E$ is a \textit{subbundle} if and only if $D_p=E_p\circ D$ is a linear subspace for all $p\in M$, and $\pi|_D\colon D\to M$ is a vector bundle with these vector subspaces.
\end{definition}
The main point is that we have some injection $D\to E$ of vector bundles.
\begin{example}
	Given some linearly independent global sections $\sigma_1,\ldots,\sigma_r\in\Gamma(E)$ for a vector bundle $\pi\colon E\to M$, we see that
	\[D\coloneqq\bigcup_{p\in M}\op{span}\{\sigma_1(p),\ldots,\sigma_r(p)\}\]
	is a vector subbundle of $E$ of rank $r$. Indeed, by construction, all fibers have the correct dimension, and by working locally on charts, we can extend our linearly independent sections to a local frame, whereupon we can build a defining function by asking for the newly added local sections to vanish.
\end{example}
\begin{example}
	Fix a bundle homomorphism $F\colon E\to E'$ over $M$. Assume that $F$ has constant rank. Then
	\[\ker F\coloneq\bigcup_{p\in M}\ker F|_{E_p}\qquad\text{and}\qquad\im F\coloneqq\bigcup_{p\in M}\im F|_{E_p}\]
	are subbundles. For example, $\ker F$ is a submanifold because it is the pre-image of $F$ of the image of the zero global section $M\subseteq E$. For the image, one fixes a local frame and then uses continuity of $F$ to show that this local frame will preserve its rank in a neighborhood, from which we are able to use the previous example.
\end{example}
\begin{example}
	The natural trace homomorphism $\tr\colon V\otimes V^*\to\RR$ for an $\mathbb R$-vector space $V$ (given by choosing a basis, identifying $V\otimes V^*\cong\op{Hom}_\RR(V,V)$, and then computing the trace in the usual matrix way) extends to a natural bundle homomorphism
	\[\tr\colon E\otimes E^*\to(\RR\times M)\]
	for any vector bundle $\pi\colon E\to M$. One can show that $E\otimes E^*$ now decomposes into $\ker\tr$ and the span of the global section associated to ${\id_E}\in\op{Hom}_M(E,E)$.
\end{example}

\subsection{The Cotangent Bundle}
We begin with a general discussion of $\RR$-vector spaces $V$. A basis $\{e_1,\ldots,e_n\}$ of $V$ gives rise to a dual basis $\{e_1^*,\ldots,e_n^*\}$ of $V^*$, where $e_i^*(e_j)\coloneqq1_{i=j}$ by definition. Notably, if we expand $v\in V$ and $w^*\in V^*$ in our basis as $v=\sum_{i=1}^nv_ie_i$ and $w^*=\sum_{i=1}^nw_ie_i^*$, and we find that
\[w^*(v)=\sum_{i=1}^nw_iv_i\]
by definition of $e_i^*$ and linearity.
\begin{remark}
	The vector $e_1^*$ is not determined by $e_1$ alone, as can be seen from its construction. Instead, the full dual basis of $V^*$ is determined by the full basis of $V$.
\end{remark}
Anyway, here is our definition.
\begin{definition}[cotangent bundle]
	Fix a smooth manifold $M$. The \textit{cotangent bundle} $T^*M$ is the dual of the tangent bundle $TM$.
\end{definition}
\begin{remark}
	Given $\sigma\in\Gamma(TM)$ and $\tau^*\in\Gamma(T^*M)$, then we produce a smooth function $\tau^*(\sigma)\in C^\infty(M)$. Explicitly, one has
	\[\tau^*(\sigma)(p)\coloneqq\tau^*_p(\sigma_p),\]
	which can be checked to be smooth locally on charts in the usual way.
\end{remark}
\begin{definition}[covector field]
	Fix a smooth manifold $M$. A \textit{smooth covector field} is a smooth section of the canonical projection $T*M\to M$. We let $\mf X^*(M)$ denote the collection of covector fields.
\end{definition}
As usual, one can check smoothness of such a section locally on charts.

\end{document}