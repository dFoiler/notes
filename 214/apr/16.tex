% !TEX root = ../notes.tex

\documentclass[../notes.tex]{subfiles}

\begin{document}

Today we discuss tensor products.

\subsection{Closed Covector Fields}
Here is our definition.
\begin{definition}[closed]
	Fix a covector field $\omega$ on a smooth manifold $M$. Then $\omega$ is \textit{closed} if and only if any smooth chart $(U,\varphi)$ where $\varphi=(x_1,\ldots,x_n)$, our expansion $\omega=\sum_i\omega_i\,dx_i$ has
	\[\frac{\del\omega_j}{\del x_i}=\frac{\del\omega_i}{x_j}\]
	for any $i$ and $j$.
\end{definition}
\begin{example}
	Exact forms are closed. Indeed, say $\omega=df$ where $f\in C^\infty(M)$. Then for any smooth chart $(U,\varphi)$ where $\varphi=(x_1,\ldots,x_n)$ makes
	\[\omega=df=\sum_{i=1}^n\frac{\del f}{\del x_i}\,dx_i,\]
	so we compute
	\[\frac{\del\omega_j}{\del x_i}=\frac{\del^2f}{\del x_i\del x_j}=\frac{\del^2f}{\del x_j\del x_i}=\frac{\del\omega_i}{\del x_j}.\]
\end{example}
\begin{remark}
	There are closed forms which fail to be exact. For example, set $\omega\coloneqq x\,dy-y\,dx$ on $\RR^2$, and pull it back to $i^*\omega$ where $i\colon S^1\to\RR^2$ is the inclusion. Because this is a $1$-manifold, this is vacuously closed by the definition. However, 
	\[\int_{\gamma}i^*\omega=2\pi\]
	for a path $\gamma\colon[0,1]\to S^1$ going around $S^1$ by a direct computation (see \Cref{ex:line-integrals}), so $i^*\omega$ fails to be conservative and hence fails to be exact.
\end{remark}
We quickly remove the ``any smooth chart'' part of the definition.
\begin{lemma}
	Fix a covector field $\omega$ on a smooth manifold $M$. Then the following are equivalent.
	\begin{listroman}
		\item $\omega$ is closed.
		\item Each $p\in M$ has some smooth chart $(U,\varphi)$ such that $\varphi=(x_1,\ldots,x_n)$ provides coordinates where $\omega=\sum_i\omega_i\,dx_i$ and
		\[\frac{\del\omega_j}{\del x_i}=\frac{\del\omega_i}{x_j}\]
		for any $i$ and $j$.
		\item For any local vector fields $X,Y\in\mf X(U)$ on an open subset $U$, we have
		\[X(\omega(Y))-Y(\omega(X))=\omega([X,Y]).\]
	\end{listroman}
\end{lemma}
\begin{proof}
	Note (i) implies (ii) with no content. To see (iii) implies (i), we work locally on charts. Choose a smooth chart $(U,\varphi)$ where $\varphi=(x_1,\ldots,x_n)$. Then for distinct indices $i$ and $j$, we take $X\coloneqq\del/\del x_i$ and $Y\coloneqq\del/\del x_j$ and compute
	\[X(\omega(Y))-Y(\omega(X)) = \frac\del{\del x_i}\omega\left(\frac\del{\del x_j}\right)-\frac{\del}{\del x_j}\omega\left(\frac\del{\del x_i}\right)=\omega\left(\left[\frac\del{\del x_i},\frac\del{\del x_j}\right]\right)=\omega(0)=0.\]
	Now, writing $\omega=\sum_i\omega_i\,dx_i$, we see that the left-hand side now reads
	\[\frac\del{\del x_i}\omega_j-\frac{\del}{\del x_j}\omega_i=0,\]
	as needed.

	It remains to show that (ii) implies (iii). Well, we can verify the last equality at points, so choose local vector fields $X,Y\in\mf X(U)$ and some point $p\in U$ to verify the equality around, and then we shrink $U$ around $p$ so that we have a smooth chart $(U,\varphi)$ where $\varphi=(x_1,\ldots,x_n)$. Then we can expand $X=\sum_iX_i\frac{\del}{\del x_i}$ and $Y=\sum_jY_j\frac{\del}{\del x_j}$ and $\omega=\sum_i\omega_i\,dx_i$, so we are now able to compute
	\begin{align*}
		X(\omega(Y)) &= \sum_{i,j}X_i\frac\del{\del x_i}(\omega_jY_j) \\
		&= \sum_{i,j}X_iY_j\frac{\del\omega_j}{\del x_i}+X_i\omega_j\frac{\del Y_j}{\del x_i} \\
		Y(\omega(X)) &= \sum_{i,j}Y_i\frac\del{\del x_i}(\omega_jX_j) \\
		&= \sum_{i,j}Y_iX_j\frac{\del\omega_j}{\del x_i}+Y_i\omega_j\frac{\del X_j}{\del x_i}.
	\end{align*}
	We now see that subtracting makes the left terms of the sum cancel, so we are left with $\omega([X,Y])$, as required.
\end{proof}
To close up our discussion of closed covector fields, we do note that simple spaces will be able to show that closed implies exact.
\begin{proposition}
	Fix an open star-like open subset $U\subseteq\RR^n$. Then any closed local covector field $\omega\in\mf X^*(U)$ is exact.
\end{proposition}
Here, ``star-like'' means that there is a point $p\in U$ such that the line segment connecting $p$ to any other $p'\in U$ is contained in $U$.
\begin{proof}
	By translating, we may as well assume that $U$ is star-like with the ``center point'' just the origin $0\in\RR^n$. Now, one has a global frame $\omega=\sum_i\omega_i\,dx_i$. Imitating the proof of \Cref{prop:conservative-is-exact}, we define our function $f\colon U\to\RR$ by
	\[f(x)\coloneqq\int_{\gamma_x}\omega\]
	where $\gamma_x$ is the straight-line line segment $\gamma_x(t)\coloneqq tx$ from $0$ to $x$. As such, we see that
	\[f(x)=\sum_{i=1}^n\int_0^1\omega_i(tx)x_i\,dt\]
	We omit the check that $f$ is smooth, which can be seen basically because each $\omega_i$ is smooth, and integration preserves smoothness (because one can integrate under the integral sign). We would like to check that $\omega=df$, so for any $x_j$, we compute
	\begin{align*}
		\frac{\del f}{\del x_j}(x) &= \int_0^1\sum_{i=1}^n\frac{\del\omega_i}{\del x_j}(tx)\cdot tx_i+\omega_j(tx)\,dt \\
		&= \int_0^1\sum_{i=1}^n\frac{\del\omega_j}{\del x_i}(tx)\cdot tx_i+\omega_j(tx)\,dt \\
		&= \int_0^1\frac{\del\omega_j}{\del t}(tx)\cdot t+\omega_j(tx)\,dt \\
		&= \int_0^1\frac{\del}{\del t}(t\omega_j(tx))\,dt \\
		&= t\omega_j(tx)\bigg|_{t=0}^{t=1} \\
		&= \omega_j(x),
	\end{align*}
	as required.
\end{proof}
\begin{remark}
	A more direct modification of the proof of \Cref{prop:conservative-is-exact} shows that any closed covector field on a simply connected manifold is exact. We won't bother to write this out; the main point is to check that the definition is well-defined up to the choice of path from a given basepoint to a given point in the (simply connected!) manifold.
\end{remark}

\subsection{Tensors}
We begin by discussing tensor products on vector spaces, which we will upgrade to vector bundles.
\begin{definition}[tensor]
	Fix a finite-dimensional $\RR$-vector space $V$ and nonnegative integers $k$ and $\ell$. Then we define the space of \textit{tensors} as
	\[T^{(k,\ell)}V\coloneqq V^{\otimes k}\otimes (V^*)^{\otimes\ell}.\]
	By way of convention, a \textit{covariant tensor} is an element of $T^{(0,\ell)}V$ for some $\ell$, and a \textit{contravariant tensor} is an element of $T^{(k,0)}V$ for some $k$.
\end{definition}
By the universal property of the tensor product (and the identification $V\simeq V^{**}$), we can think about an element of $T^{(k,\ell)}$ as a multilinear map $(V^*)^{k}\times V^\ell\to\RR$.
\begin{example}
	Here are some special cases.
	\begin{itemize}
		\item By convention, $T^{(0,0)}V=\RR$.
		\item $T^{(1,0)}V=V$.
		\item $T^{(0,1)}V=V^*$.
		\item $T^{(1,1)}V=\op{End}V$. One simply sends $v\otimes v^*$ to the linear map $V\to V$ given by $w\mapsto v^*(w)v$. Certainly this map is linear, and one can show that it is an isomorphism by working on a basis.
	\end{itemize}
\end{example}
\begin{remark}
	A basis $\{e_1,\ldots,e_n\}$ of $V$ produces a dual basis $\{\varepsilon_1,\ldots,\varepsilon_n\}$ of $V^*$. This produces a basis of $T^{(k,\ell)}$ of tensors of the form
	\[e_{i_1}\otimes\cdots\otimes e_{i_k}\otimes\varepsilon_{j_1}\otimes\cdots\otimes\varepsilon_{j_\ell}.\]
\end{remark}
\begin{remark}
	As usual, there are permutation morphisms $T^{(k,\ell)}V\to T^{(k,\ell)}V$ given by permuting the factors in $V\otimes\cdots\otimes V$ or in $V^*\otimes\cdots\otimes V^*$.
\end{remark}
\begin{remark}
	There is a canonical isomorphism $T^{(k,\ell)}V\otimes T^{(k',\ell')}V\to T^{(k+k',\ell+\ell')}V$, which then comes from a bilinear map
	\[T^{(k,\ell)}V\times T^{(k',\ell')}V\to T^{(k+k',\ell+\ell')}V\]
	given by ``attaching'' tensors.
\end{remark}
\begin{remark}
	There is a trace map $\tr\colon T^{(1,1)}V\to\RR$; for example, on the basis $\{e_1,\ldots,e_n\}$ of $V$, this is given by
	\[\sum_{i,j=1}^nt_{ij}e_i\otimes\varepsilon_j\mapsto\sum_{i=1}^nt_{ii}.\]
	This is basis-free, which can be checked directly or seen because trace is the map $v\otimes v^*\mapsto v^*(v)$ on pure tensors of $V\otimes V^*$.
\end{remark}
\begin{remark}
	More generally, there is a ``contraction'' map $C_{ij}\colon T^{(k,\ell)}V$ where $i\in\{1,\ldots,k\}$ and $j\in\{1,\ldots,\ell\}$ given by
	\[(v_1\otimes\cdots\otimes v_k)\otimes(v_1^*\otimes\cdots\otimes v_\ell^*)\mapsto v_j^*(v_i)(v_1\otimes\cdots\otimes\widehat v_i\otimes\cdots\otimes v_k)\otimes(v_1^*\otimes\cdots\otimes\widehat v_j\otimes\cdots\otimes v_\ell^*)\]
	on pure tensors. One can expand this out on a basis as before, in particular finding that there is a ``diagonal sum'' hiding.
\end{remark}
\begin{example}
	Given $A\in T^{(0,k)}V$, we can produce the multilinear map $V^k\to\RR$ as the repeated contraction
	\[(v_1,\ldots,v_k)\mapsto C_{11}\cdots C_{kk}(v_1\cdots v_kA).\]
	By expanding everything out on the basis, this is the usual identification of $A$ with a multilinear map $V^k\to\RR$.
\end{example}
We now upgrade to tensor products for vector bundles.
\begin{definition}[tensor bundle]
	Fix a smooth manifold $M$ and nonnegative integers $k$ and $\ell$. Then given a vector bundle $V$ on $M$, we define the \textit{tensor bundle} as
	\[T^{(k,\ell)}V\coloneqq V^{\otimes k}\otimes(V^*)^{\otimes\ell}.\]
	A global section of $T^{(k,\ell)}TM$ is called a \textit{tensor field}; a global section of $T^{(0,\ell)}TM$ is called a \textit{covariant tensor field}.
\end{definition}
\begin{example}
	As before, we find that $T^{(0,0)}V=M\times\RR$ and $T^{(1,0)}V=V$ and $T^{(0,1)}V=V^*$.
\end{example}
\begin{remark}
	Given a smooth chart $(U,\varphi)$ with $\varphi=(x_1,\ldots,x_n)$, one can provide a local frame for $TM$ and $T^*M$ as $\{\del/\del x_1,\ldots,\del/\del x_n\}$ and $\{dx_1,\ldots,dx_n\}$, respectively. Thus, we get a local frame of $T^{(k,\ell)}TM$ by
	\[\frac\del{\del x_{i_1}}\otimes\cdots\otimes\frac\del{\del x_{i_k}}\otimes dx_{j_1}\otimes\cdots\otimes dx_{j_\ell}.\]
	Something similar works using trivializing open subsets of a more general vector bundle.
\end{remark}
\begin{remark}
	A covariant tensor field $A\in\Gamma\left(T^{(0,\ell)}TM\right)$ can be viewed as a multilinear form $\mf X(M)^\ell\to C^\infty(M)$ given by
	\[(X_1,\ldots,X_\ell)\mapsto C_{11}\cdots C_{kk}(X_1\cdots X_\ell A).\]
	One can expand this out on coordinates in the typical way.
\end{remark}

\end{document}