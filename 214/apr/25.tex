% !TEX root = ../notes.tex

\documentclass[../notes.tex]{subfiles}

\begin{document}

Today we introduce Stokes's theorem.
\begin{remark}
	The final exam will cover chapters 1 through 16, though there will be basically nothing on chapter 15 other than the statement that some manifolds have orientations.
\end{remark}
Here is basically everything we will need to know about orientations.
\begin{definition}
	Fix a smooth manifold $M$. An \textit{orientation} on $M$ is a minimal smooth atlas $\mc A$ such that the determinants of the transition maps are positive.
\end{definition}
It turns out that a smooth manifold has an orientation if and only if it has a nowhere-vanishing volume form.

\subsection{Differential Forms}
Here is our definition.
\begin{definition}[differential form]
	Fix a smooth manifold $M$, possibly with boundary. Then a \textit{differential form} $\omega$ is a global section of the vector bundle $\Omega^k(M)\coloneqq\land^kT^*M$.
\end{definition}
\begin{remark}
	Here, $\land^kE$ for a vector bundle $E$ on $M$ is defined using the usual construction. For example, we can construct it as an ``alternating'' subbundle of $E^{\otimes k}$ cut out by the requirements of being alternating. Notably, a smooth chart $(U,\varphi)$ with $\varphi=(x_1,\ldots,x_n)$ gives rise to a local frame given by the sections of the form
	\[dx_I\coloneqq dx_{i_1}\land\cdots\land dx_{i_k}\]
	where $I=\{i_1,\ldots,i_k\}$ is an increasing sequence, so $\land^kE$ will have rank $\binom nk$ where $n=\op{rank}E$. This allows us to write any $\omega\in\Omega^k(M)$ as
	\[\omega=\sum_I\omega_Idx_I,\]
	where the sum varies over increasing sequences $I\subseteq\{1,\ldots,n\}$.
\end{remark}
\begin{example}
	Note $\Omega^0(M)=C^\infty(M)$ because we are asking for global sections of the trivial line bundle $M\times\RR$.
\end{example}
\begin{example}
	Note $\Omega^1(M)=\Gamma(T^*M)=\mf X^*(M)$.
\end{example}
\begin{example}
	As usual, let's do the usual computation on change of coordinates. Let $(x_1,\ldots,x_n)$ and $(y_1,\ldots,y_n)$ be two systems of coordinates about some $p\in M$. Then we can compute
	\[dy_1\land\cdots\land dy_n=\left(\sum_{i_1}\frac{\del y_1}{\del x_{i_1}}dx_{i_1}\right)\land\cdots\land\left(\sum_{i_n}\frac{\del y_n}{\del x_{i_n}}dx_{i_n}\right)=\sum_{i_1,\ldots,i_n}\frac{\del y_1}{\del x_{i_1}}\cdots\frac{\del y_n}{\del x_{i_n}}dx_{i_1}\land\cdots\land dx_{i_n}.\]
	Notably, after rearranging the coordinates to get back to $dx_1\land\cdots\land dx_n$, we get
	\[\det\left[\frac{\del y_j}{\del x_i}\right]_{i,j}dx_1\land\cdots\land dx_n\]
	by recalling the definition of $\det$ as some large sum over signed permutations.
\end{example}
\begin{remark}
	There is also a notion of pullback: given a smooth map $F\colon M\to N$ and some $\omega\in\Omega^k(N)$, we can define $F^*\omega$ as a covariant $k$-tensor field at least by
	\[(F^*\omega)_p(v_1,\ldots,v_k)\coloneqq\omega_{F(p)}(dF_p(v_1),\ldots,dF_p(v_k))\]
	for $p\in M$ and $v_1,\ldots,v_k\in T_pM$. (This is the usual pullback for covariant $k$-tensor fields.) But we now see from this definition that $F^*\omega$ is alternating, so we get to define our pullback as going to $\Omega^k(M)$.
\end{remark}
\begin{remark}
	There are some basic properties of the pullback that one should read about. For example, one can show by hand that $F^*(\omega\land\eta)=F^*\omega\land F^*\eta$.
\end{remark}

\subsection{The Exterior Derivative}
We like exact covectors, but exactness is not a local property: only being closed is exact. So perhaps we would like to understand obstructions to exactness.

Namely, for some $\omega\in\mf X^*(M)$, we write $\omega=\sum_i\omega_idx_i$ where $(x_1,\ldots,x_n)$ are some local coordinates. Then we can define
\[d\omega\coloneqq\sum_{i<j}\left(\frac{\del\omega_j}{\del x_i}-\frac{\del\omega_i}{\del x_j}\right)dx_i\land dx_j.\]
It is not totally clear that this is independent of the choice of coordinates, but one can in fact check this by hand, and then we see $d\omega$ actually glues together into a smooth covariant $2$-tensor field, and we can see by the above construction that $d\omega\in\Omega^2(M)$. The point is that $\omega$ is closed if and only if $d\omega=0$; for example, for $f\in\Omega^0(M)$, we have $d(df)=0$.

Next, we would like to define a similar map $d\colon\Omega^2(M)\to\Omega^3(M)$ and maybe even $d\colon\Omega^k(M)\to\Omega^{k+1}(M)$ in general.
\begin{definition}[exterior derivative]
	Fix an open subset of Euclidean space $U\subseteq\RR^n$. Given $\omega\in\Omega^k(U)$, we write $\omega=\sum_{I}\omega_Idx_I$, and we define the \textit{exterior derivative} $d\omega\in\Omega^{k+1}(U)$ by
	\[d\omega\coloneqq\sum_{j_1<\cdots<j_k}\sum_{i=1}^n\frac{\del\omega_{j_1\cdots j_k}}{\del x_i}\,dx_i\land dx_{j_1}\land\cdots\land dx_{j_k}.\]
	One can check that this definition glues to a map $d\colon\Omega^k(M)\to\Omega^{k+1}(M)$ for an arbitrary smooth manifold $M$, possibly with boundary.
\end{definition}
\begin{example}
	For $\omega\in\Omega^1(M)$ given by $\omega=\sum_i\omega_i\,dx_i$ locally, we can compute
	\[d\omega=\sum_{i,j}\frac{\del\omega_j}{\del x_i}\,dx_i\land dx_j,\]
	which agrees with our earlier definition.
\end{example}
\begin{remark}
	One can check that $d$ is $\RR$-linear by hand, and we can see $d\circ d=0$ by a length computation. Another by-hand computation shows that
	\[d(\omega\land\eta)=d\omega\land\eta+(-1)^k\omega\land d\eta.\]
	Everything is natural, so we also get $F^*(d\omega)=d(F^*\omega)$.
\end{remark}
\begin{example}
	Work in $\RR^3$, and let's compute $d(u\,dx_1\land dx_3)$ where $u$ is some smooth function. Then
	\begin{align*}
		d(u\,dx_1\land dx_3) &= du\land(dx_1\land dx_3)+(-1)^0u\,d(dx_1\land dx_3) \\
		&= du\land dx_1\land dx_3+u(d(dx_1)\land dx_3-dx_1\land(d(dx_3))) \\
		&= du\land dx_1\land dx_3.
	\end{align*}
	Writing $du=\sum_i\frac{du_i}{dx_i}\,dx_i$, we then see that only the $i=2$ term may contribute, so we are left with $-\frac{\del u_2}{\del x_2}\,dx_1\land dx_2\land dx_3$.
\end{example}
One may be interested in a more coordinate-free definition of the exterior derivative. At the very least, we will be able to note that it is unique from some of our listed properties.
\begin{theorem}
	Fix a smooth manifold $M$, possibly with boundary. Then there is a unique family of maps $d\colon\Omega^\bullet(M)\to\Omega^{\bullet+1}(M)$ satisfying the following conditions.
	\begin{listroman}
		\item Linear: $d$ is $\RR$-linear.
		\item Product rule: for $\omega\in\Omega^k(M)$ and $\eta\in\Omega^\ell(M)$, we have
		\[d(\omega\land\eta)=d\omega\land\eta+(-1)^k\omega\land d\eta.\]
		\item Complex: $d\circ d=0$.
		\item Degree $0$: for $f\in C^\infty(M)$, the $1$-form $df\in\Omega^1(M)$ is the usual differential of a function.
	\end{listroman}
\end{theorem}
\begin{proof}
	We already know existence. We won't show uniqueness.
\end{proof}
Let's do some more examples. We work with $\RR^3$.
\begin{itemize}
	\item We know $\Omega^0\left(\RR^3\right)=C^\infty\left(\RR^3\right)$.
	\item We may identify $\Omega^1\left(\RR^3\right)=\mf X^*\left(\RR^3\right)$ with $\mf X\left(\RR^3\right)$ by using the standard Riemannian metric (explicitly, we send $dx_i\in\mf X^*\left(\RR^3\right)$ to $\frac\del{\del x_i}\in\mf X\left(\RR^3\right)$).
	\item Continuing, we may identify $\Omega^2\left(\RR^3\right)$ with $\mf X\left(\RR^3\right)$ again by using the global frame of $\land^2T^*\RR^3$: we send $dx_i\land dx_{i+1}$ with $\frac\del{\del x_{i+2}}$, where indices are taken$\pmod3$. More canonically, we can take $X\in\mf X\left(\RR^3\right)$ to $\iota_X(dx_1\land dx_2\land dx_3)\in\Omega^2\left(\RR^3\right)$.
	\item Lastly, $\Omega^3\left(\RR^3\right)$ has global frame given by $dx_1\land dx_2\land dx_3$, so this space is isomorphic to $C^\infty\left(\RR^3\right)$ by sending a smooth function $u$ to $u\,dx_1\land dx_2\land dx_3$.
\end{itemize}
The point of doing all this is that it turns out that the following diagram commutes.
% https://q.uiver.app/#q=WzAsOCxbMCwwLCJDXlxcaW5mdHkoTSkiXSxbMSwwLCJcXG1mIFhcXGxlZnQoXFxSUl4zXFxyaWdodCkiXSxbMiwwLCJcXG1mIFhcXGxlZnQoXFxSUl4zXFxyaWdodCkiXSxbMywwLCJDXlxcaW5mdHkoTSkiXSxbMCwxLCJcXE9tZWdhXjBcXGxlZnQoXFxSUl4zXFxyaWdodCkiXSxbMSwxLCJcXE9tZWdhXjFcXGxlZnQoXFxSUl4zXFxyaWdodCkiXSxbMiwxLCJcXE9tZWdhXjJcXGxlZnQoXFxSUl4zXFxyaWdodCkiXSxbMywxLCJcXE9tZWdhXjNcXGxlZnQoXFxSUl4zXFxyaWdodCkiXSxbMCwxLCJcXG9we2dyYWR9Il0sWzEsMiwiXFxvcHtjdXJsfSJdLFsyLDMsIlxcb3B7ZGl2fSJdLFswLDQsIiIsMix7ImxldmVsIjoyLCJzdHlsZSI6eyJoZWFkIjp7Im5hbWUiOiJub25lIn19fV0sWzEsNSwiIiwyLHsibGV2ZWwiOjIsInN0eWxlIjp7ImhlYWQiOnsibmFtZSI6Im5vbmUifX19XSxbMiw2LCIiLDIseyJsZXZlbCI6Miwic3R5bGUiOnsiaGVhZCI6eyJuYW1lIjoibm9uZSJ9fX1dLFszLDcsIiIsMCx7ImxldmVsIjoyLCJzdHlsZSI6eyJoZWFkIjp7Im5hbWUiOiJub25lIn19fV0sWzQsNSwiZCJdLFs1LDYsImQiXSxbNiw3LCJkIl1d&macro_url=https%3A%2F%2Fraw.githubusercontent.com%2FdFoiler%2Fnotes%2Fmaster%2Fnir.tex
\[\begin{tikzcd}
	{C^\infty(M)} & {\mf X\left(\RR^3\right)} & {\mf X\left(\RR^3\right)} & {C^\infty(M)} \\
	{\Omega^0\left(\RR^3\right)} & {\Omega^1\left(\RR^3\right)} & {\Omega^2\left(\RR^3\right)} & {\Omega^3\left(\RR^3\right)}
	\arrow["{\op{grad}}", from=1-1, to=1-2]
	\arrow[Rightarrow, no head, from=1-1, to=2-1]
	\arrow["{\op{curl}}", from=1-2, to=1-3]
	\arrow[Rightarrow, no head, from=1-2, to=2-2]
	\arrow["{\op{div}}", from=1-3, to=1-4]
	\arrow[Rightarrow, no head, from=1-3, to=2-3]
	\arrow[Rightarrow, no head, from=1-4, to=2-4]
	\arrow["d", from=2-1, to=2-2]
	\arrow["d", from=2-2, to=2-3]
	\arrow["d", from=2-3, to=2-4]
\end{tikzcd}\]
Here, the vertical maps are the identifications we just described. For example, we discover that ${\op{curl}}\circ{\op{grad}}=0$ and ${\op{div}}\circ{\op{curl}}=0$.

We also note that we can see some pairing: given a Riemannian manifold $(M,g)$, one has a global ``volume'' form given by
\[dV_g\coloneqq\sqrt{\deg(g_{ij})}dx_1\land\cdots\land dx_n\]
for any local choice of coordinates $(x_1,\ldots,x_n)$. Then there is a unique map $*\colon\Omega^k(M)\to\Omega^{n-k}(M)$ such that
\[\omega\land *\eta=\langle\omega,\eta\rangle_gdV-g\]
In particular, we are seeing that $\lor$ somehow produces a perfect pairing.
\begin{remark}
	It turns out that our Laplacian operator $\Delta f$ for $f\in C^\infty\left(\RR^3\right)$ given by $*d*df$. One can compute this operator as ${\op{div}}\circ\op{grad}$ where the content becomes that our $*$ operator also commutes with the vertical isomorphisms.
\end{remark}
\begin{remark}
	Our discussion of the exterior derivative also has applications for $\RR^4$: an element of $\Omega^2\left(\RR^4\right)$ can be viewed as a smooth map from $\RR^4$ to the space of antisymmetric $4\times4$ matrices (by using the standard global frame, as usual). Professor Chen gave some discussion of Maxwell's equations; basically, it turns out that one can compress everything into two short equations on a single element $\omega\in\Omega^2\left(\RR^4\right)$.
\end{remark}

\subsection{Stokes's Theorem}
Here is the statement of Stokes's theorem, which may be helpful for the exam.
\begin{theorem}
	Fix some $(n-1)$-form $\omega$ on a smooth manifold $M$ with boundary $\del M$. Then
	\[\int_Md\omega=\int_{\del M}\omega.\]
\end{theorem}

\end{document}