% !TEX root = ../notes.tex

\documentclass[../notes.tex]{subfiles}

\begin{document}

Today we discuss Riemannian metrics.

\subsection{More on Tensors}
Let's discuss covariant tensor fields more explicitly. Quickly, we note that tensors have some notion of $C^\infty(M)$-multilinearity. Explicitly, a smooth covariant tensor field $A\in\Gamma\left(T^{(0,\ell)}TM\right)$ amounts to $C^\infty(M)$-multilinear map
\[\mf X(M)\times\cdots\times\mf X(M)\to C^\infty(M)\]
by sending $(X_1,\ldots,X_\ell)\mapsto A(X_1,\ldots,X_\ell)$ via the usual identification $(V\otimes\cdots\otimes V)^*=V^*\otimes\cdots\otimes V^*$. This turns out to characterize our tensors.
\begin{proposition}
	Fix a smooth manifold $M$. Given a $C^\infty(M)$-multilinear map $\mc A\colon\mf X(M)^\ell\to C^\infty(M)$, then there is a unique covariant tensor field $A\in\Gamma\left(T^{(0,\ell)}TM\right)$ such that $\mc A$ comes from $A$.
\end{proposition}
\begin{proof}[Sketch]
	Let's be brief.
	\begin{enumerate}
		\item We begin by using the $C^\infty(M)$-multilinearity to show that $\mc A$ is ``local.'' Explicitly, if $(X_1,\ldots,X_\ell)$ and $(X_1',\ldots,X_\ell')$ agree locally in a neighborhood of some point $p\in U$, then
		\[\mc A(X_1,\ldots,X_\ell)=\mc A(X_1',\ldots,X_\ell').\]
		The point is to use smooth cutoff functions to compare these two values.
		\item Next up, we can show that $\mc A(X_1,\ldots,X_\ell)$ only depends on the data of $(X_1(p),\ldots,X_\ell(p))$ by expanding out locally; here, one uses the $C^\infty(M)$-multilinearity more crucially to compare $X_\bullet$ and $X'_\bullet$. This constructs the (rough) section $A\colon M\to T^{(0,\ell)}TM$.
		\item Then one shows that $A$ is in fact smooth by working in local coordinates.
		\qedhere
	\end{enumerate}
\end{proof}
Next up, we discuss how change of coordinates happens to a tensor field $A\in\Gamma\left(T^{(k,\ell)}TM\right)$. Well, suppose we have some point $p$ contained in the two smooth charts $(U,\varphi)$ and $(V,\psi)$ with $\varphi=(x_1,\ldots,x_n)$ and $\psi=(y_1,\ldots,y_n)$. In these local coordinates, one can write
\[A=\sum A^{i_1,\ldots,i_k}_{j_1,\ldots,j_\ell}\frac\del{\del x_{i_1}}\otimes\cdots\otimes dx_{j_\ell},\]
and then the change of coordinates formulae for the individual differentials and covectors extends via the tensor product. Explicitly, 

As another remark, we discuss pullbacks of covariant tensor fields.
\begin{definition}[pullback]
	Fix a smooth manifold $M$ and a covariant tensor field $A\in\Gamma\left(T^{(0,\ell)}TM\right)$. For a smooth map $F\colon N\to M$, we define the \textit{pullback covariant tensor field} $F^*A$ by
	\[(F^*A)(v_1,\ldots,v_\ell)\coloneqq A_{F(p)}(dF_p(v_1),\ldots,dF_p(v_\ell)).\]
	We won't bother to check that this map is smooth, but it is; roughly speaking, we are taking the composite of the smooth functions $A$ and $dF$.
\end{definition}
One can also take Lie derivatives.
\begin{definition}[Lie derivative]
	Fix a smooth manifold $M$ and a smooth covariant tensor field $A\in\Gamma\left(T^{(0,\ell)}TM\right)$. For a vector field $V\in\mf X(M)$, let $\theta_\bullet$ be the flow of $V$, and we define the \textit{Lie derivative}
	\[\mc L_vA\coloneqq\frac d{dt}\theta_t^*A\bigg|_{t=0}.\]
	Explicitly, we see
	\[(\mc L_VA)_p(v_1,\ldots,v_\ell)=\frac d{dt}A_{\theta_t(p)}((d\theta_t)_p(v_1),\ldots,(d\theta_t)_p(v_\ell))\bigg|_{t=0}.\]
\end{definition}
The intuition here is exactly the same as what was done for just covector fields.
\begin{remark}
	Suppose that we have managed to get $V=\del/\del x_1$ locally (which is always doable), where $A=\sum A_{i_1,\ldots,i_\ell}\,dx_{i_1}\otimes\cdots\otimes dx_{i_\ell}$. Then the flow is constantly moving in the $x_1$ direction, so we see that
	\[\mc L_VA=\sum\frac{A_{i_1,\ldots,i_\ell}}{\del x_1}\,dx_{i_1}\otimes\cdots\otimes dx_{i_\ell}.\]
	Note $\mc L_VA\in\Gamma\left(T^{(0,\ell)}TM\right)$ by some $C^\infty(M)$-multilinearity check.
\end{remark}
\begin{remark}
	Viewing $f\in C^\infty(M)$ as a covariant tensor field in $\Gamma\left(T^{(0,0)}TM\right)$, we see that $(\mc L_Vf)_p=V_p(f)$. For example, one can track through the previous remark to see this.
\end{remark}
\begin{remark}
	One has a ``Leibniz rule'': for $X_1,\ldots,X_\ell\in\mf X(M)$, we have
	\[\mc L_V(A(X_1,\ldots,X_\ell))=\mc L_VA(X_1,\ldots,X_\ell)+A(\mc L_VX_1,X_2,\ldots,X_\ell)+\cdots+A(X_1,\ldots,\mc L_VX_\ell).\]
\end{remark}

\subsection{Riemannian Metrics}
As motivation, we note that the length of a curve $\gamma\colon[a,b]\to\RR^n$ is computed as
\[\ell(\gamma)=\int_a^b\left|\gamma'(t)\right|^2\,dt.\]
Here, $\left|\gamma'(t)\right|^2$ is a norm of the derivative, so if we want to generalize this notion to a manifold, we need a notion of a norm on our tangent spaces. It turns out that norms (with enough structure, namely a parallelogram law) must come from bilinear forms, so we may as well ask for our tangent spaces to have a bilinear form. In Euclidean space, this is easy because $T\RR^n$ has a global frame, so we may just identify all $T_p\RR^n$s with $\RR^n$ and then use the standard inner product on $\RR^n$, but in general it may not be so easy to produce a good inner product everywhere.

A good choice of inner product everywhere is essentially the data of a Riemannian metric.
\begin{definition}[Riemannian metric]
	Fix a smooth manifold $M$. A \textit{Riemannian metric} on $M$ is a smooth covariant tensor field $g\in\Gamma\left(T^{(0,2)}TM\right)$ such that each $p\in M$ makes $g_p\colon T_pM\times T_pM\to\RR$ induce a symmetric positive-definite inner product on $T_pM$. We will write $\langle\cdot,\cdot\rangle_g\coloneqq g(\cdot,\cdot)$ and $\left|\cdot\right|_g\coloneqq\sqrt{\langle\cdot,\cdot\rangle}$. (We will suppress the $g$ from our notation as much as possible.) A \textit{Riemannian manifold} is a pair $(M,g)$ of a smooth manifold $M$ equipped with a Riemannian metric $g$.
\end{definition}
\begin{remark}
	Let's explain how this looks on a smooth chart $(U,\varphi)$ where $\varphi=(x_1,\ldots,x_n)$. Then $g$ can be expanded out as on coordinates as
	\[g=\sum_{i,j}g_{ij}\,dx_i\otimes dx_j,\]
	so being symmetric and positive-definite corresponds to the same adjectives on the matrix $G\coloneqq\{g_{ij}\}$. Now, if we have some differentials $u=\sum_iu_i\frac{\del}{\del x_i}\big|_p$ and $v=\sum_jv_j\frac\del{\del x_j}\big|_p$, we see
	\[g_p(u,v)=\sum_{i,j}u_iv_jg_{ij}(p).\]
\end{remark}
\begin{example}
	On $\RR^n$, our standard Riemannian metric is given by
	\[\sum_{i=1}^ndx_i\otimes dx_i.\]
\end{example}
We would like to show that Riemannian metrics exist in general. It will be helpful to have a little freedom in our construction, as follows.
\begin{lemma}
	Fix an immersion $F\colon N\to M$ of smooth manifolds. If $g$ is a Riemannian metric on $M$, then $F^*g$ is a Riemannian metric on $N$.
\end{lemma}
\begin{proof}
	Certainly $F^*g$ is smooth because it is the pullback of something smooth, so we only need to do the other checks. At each $p\in N$, we compute
	\[(F^*g)_p(u,v)=g_{F(p)}(dF_p(u),dF_p(v))\]
	for any $u,v\in T_pM$. This is certainly symmetric and bilinear because $g_{F(p)}$ is. Additionally, this is certainly nonnegative because $g_{F(p)}$, and it is positive-definite because $(F^*g)_p(u,v)$ implies $dF_p(u)=dF_p(v)=0$ (because $g$ is positive-definite), which implies $u=v=0$ because $F$ is an immersion!
\end{proof}
\begin{example}
	For any embedded submanifold $S\subseteq M$, let $i\colon S\to M$ be the embedding. Then if $g$ is a Riemannian metric on $M$, we see that $i^*g$ is a Riemannian metric on $S$. For example, any smooth manifold can be embedded into Euclidean space by the Whitney embedding \cref{thm:whitney-embed}, so any smooth manifold
\end{example}
\begin{remark}
	One can avoid using \Cref{thm:whitney-embed} to show that Riemannian metrics exist. Indeed, one can more directly note that any smooth chart produces a ``local'' Riemannian metric, which extends a smooth covariant tensor field which is symmetric and bilinear but perhaps only positive-definite in a neighborhood. Then one can glue these local almost Riemannian metrics together via partition of unity. The main check here is that we can take convex linear combinations of positive-definite forms to get another positive-definite form, which can be checked directly (and indeed, is just linear algebra on the tangent spaces).
\end{remark}
\begin{remark}
	We have in fact produced some new structure: it is possible to prove two Riemannian metrics $g_1$ and $g_2$ on a smooth manifold $M$ which are distinct, and worse, it is possible for there to be no diffeomorphism $F\colon M\to M$ such that $g_1=F^*g_2$!
\end{remark}

\subsection{Metrics from Riemannian Metrics}
We now provide a notion of distance for our manifolds.
\begin{definition}[length]
	Fix a Riemannian manifold $(M,g)$. Given a (piecewise) $C^1$ curve $\gamma\colon[a,b]\to M$, we define the \textit{length} of $\gamma$ to be
	\[\ell_g(\gamma)\coloneqq\int_a^b\left|\gamma'(t)\right|_g^2\,dt.\]
	We will suppress the $g$ from our notation as much as possible.
\end{definition}
\begin{remark}
	One can check that $\ell_g(\gamma)$ is independent of reparameterization by the usual arguments. Namely, if $\varphi\colon[c,d]\to[a,b]$ is an increasing piecewise $C^1$ path, then one can show that $\ell_g(\gamma)=\ell_g(\gamma\circ\varphi)$. Indeed, by breaking up into intervals, it suffices to handle the smooth case, and then an argument with some $u$-substitution grants the equality.
\end{remark}
\begin{definition}
	Fix a Riemannian manifold $(M,g)$. Then we define
	\[d_g(p,q)\coloneqq\inf\left\{\ell_g(\gamma):\gamma\text{ is a piecewise $C^1$ path }[a,b]\to M,\gamma(a)=p,\gamma(b)=q\right\}.\]
\end{definition}
\begin{remark}
	The infimum in this definition need not be achieved. For example, in $M\coloneqq\RR^2\setminus\{(0,0)\}$, there is no path achieving the smallest possible distance between $(-1,0)$ and $(1,0)$. It turns out that this minimum is in fact always achieved as long as $M$ is complete as a metric space; these minimal curves are called ``geodesics.''
\end{remark}
And here are our checks.
\begin{theorem}
	Fix a Riemannian manifold $(M,g)$. Then $d_g$ is a metric on $M$, and it induces the topology on $M$.
\end{theorem}
\begin{proof}
	Let's quickly discuss some of these checks.
	\begin{itemize}
		\item $d_g(x,x)=0$ holds by using the constant path.
		\item $d_g(x,y)>0$ for $x\ne y$ holds because $x\ne y$ requires any piecewise $C^1$ path $\gamma\colon[a,b]\to M$ with $\gamma(a)=p$ and $\gamma(b)=q$ to have $\left|\gamma'(t)\right|^2$ positive on a set of positive measure.
		\item Symmetry holds by taking any path in one direction and reversing it to get a path in the opposite direction.
		\item The triangle inequality $d(x,y)+d(y,z)\ge d(x,z)$ is achieved by taking any path from $x$ to $y$ and path from $y$ to $z$ and attaching them (piecewise!) to produce a path from $x$ to $z$.
	\end{itemize}
	So we have a metric. It remains to show that $d_g$ induces the topology on $M$. This is a local question on $M$, so it suffices to work locally in a chart, meaning that we may assume that $M=\RR^n$. But now it turns out that any two Riemannian metrics on $\RR^n$ induces the same topology by some bounding argument, so we are done because the standard Riemannian metric on $\RR^n$ does induce the correct topology.
\end{proof}
\begin{remark}
	One can ``recover'' the Riemannian metric from the length of curves. Morally, one can take the derivative of length in a direction at a point $p\in M$ to recover the norm on $T_pM$ induced by $g$, which recovers the inner product on $T_pM$.
\end{remark}

\subsection{More on Riemannian Metrics}
Quickly, recall that a choice of inner product $\langle\cdot,\cdot\rangle$ on a finite-dimensional vector space $V$ defines an isomorphism $(\cdot)^\flat\colon V\to V^*$ via $v\mapsto\langle v,\cdot\rangle$; we let the inverse isomorphism be $(\cdot)^\sharp$. This extends smoothly to provide a $C^\infty(M)$-linear isomorphism between $\mf X(M)\to\mf X^*(M)$ for any smooth manifold $M$, which is basically the following result.
\begin{proposition}
	Fix a Riemannian manifold $(M,g)$. There is a vector bundle isomorphism $TM$ and $T^*M$.
\end{proposition}
\begin{proof}
	We define our vector bundle isomorphism $(\cdot)^\flat\colon TM\to T^*M$ by
	\[v^\flat(w)\coloneqq g_p(v,w)\]
	for any $p\in M$ and $v,w\in T_pM$. One can check that this is smooth by expanding out everything in terms of coordinates. Further, one can directly check that we have defined a homomorphism of vector bundles, so it remains to check that we have a diffeomorphism, which again can be checked locally on coordinates (because we are just doing some linear maps everywhere).
\end{proof}
% \subsection{Curvature}
As an aside, we note that it is not always possible to choose coordinates on a Riemannian manifold $(M,g)$ so that $g$ is locally $\sum_idx_i\otimes dx_i$. I didn't really follow the discussion on curvature in class.

\end{document}