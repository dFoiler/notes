% !TEX root = ../notes.tex

\documentclass[../notes.tex]{subfiles}

\begin{document}

Welcome back from spring break.

\subsection{The Flowout Theorem}
We begin with a remark.
\begin{remark}
	Fix a vector field $V$ on $M$. When $V$ fails to be complete, there is not much we can say. We do know that each $p\in M$ has some maximal open neighborhood $D_p\subseteq\RR$ such that we have a trajectory $\theta_t(p)$ defined on $D^p$, so we can glue these together into
	\[\mc D_V\coloneqq\bigcup_{p\in M}D_p\times\{p\}\]
	as a subset of $\RR\times M$. The point is that we can glue these together into a big flow-like trajectory.
\end{remark}
Flowouts essentially allow us to work in a small neighborhood of a submanifold. Here is our result.
\begin{theorem}[Flowout] \label{thm:flowout}
	Let $S$ be a smooth $k$-submanifold of the smooth $n$-manifold $M$, and let $V\in\mf X(M)$ be a vector field such that $V_p\notin T_pS$ for all $p\in S$. Set $\Gamma\coloneqq(\RR\times S)\cap\mc D_V$, and define $\Phi\coloneqq\theta|_\Omega$ to be a flow of $V$.
	\begin{listalph}
		\item $\Phi$ is an immersion.
		\item $\Phi$ relates $\frac\del/\del t$ and $V$.
		\item For some smooth $\delta\colon S\to\RR_{>0}$, set
		\[\Omega_\delta\coloneqq\{(t,p)\in\Omega:\left|t\right|<\delta(p)\}.\]
		Then $\Phi|_{\Omega_\delta}$ is injective.
		\item In the context of (c), further suppose that $k=n-1$. Then $\Phi|_{\Omega_\delta}$ is a diffeomorphism onto an open submanifold of $M$.
	\end{listalph}
\end{theorem}
\begin{proof}[Sketch]
	Note (b) is basically by the definition of being a trajectory. For (a), we note that it follows for points $p\in S$ because $V_p\notin T_pS$ by hypothesis and then use the $\RR$-action from the flow to extend the immersive property to other points. For (c), one can use $k$-slice charts to check it on locally; a clever argument with partition of unity makes the injectivity global. Lastly, (d) follows from (a) and (c) because we have established that we have an injective local diffeomorphism on $\Omega_\delta$.
\end{proof}
Computations with vector fields are aided by the following nice form.
\begin{lemma} \label{lem:get-vector-field-form}
	Fix a vector field $V$ on the smooth $n$-manifold $M$. Then each $p\in M$ with $V_p\\ne0$ has a smooth chart $(U,(x_1,\ldots,x_n))$ such that
	\[V|_U=\frac\del{\del x_1}.\]
	In fact, given an $(n-1)$-submanifold $S\subseteq M$ and $p\in S$ with $V_p\notin T_pS$, we may assume that the smooth chart is a local slice chart for $S$ cut out by $x_1=0$.
\end{lemma}
\begin{proof}
	The first claim follows from the second claim by just taking any chart $(U,(x_1,\ldots,x_n))$ and defining the needed $S$ by $x_1=0$. For the second claim, use \Cref{thm:flowout}.
\end{proof}

\subsection{The Lie Derivative}
Given two vector fields $V$ and $W$, we would like to compute something like $\del_VW$. It is not so obvious, however, how to interpret these derivatives when not working in Euclidean space where we have an obvious identification between points and differentials.

Flows will allow us to take derivatives by moving points (and thus their tangent spaces) around.
\begin{definition}[Lie derivative]
	Fix vector fields $V$ and $W$ of a smooth manifold $M$. Given $p\in M$, we define the \textit{Lie derivative} of $W$ with respect to $V$ at $p$ as
	\[(\mc L_VW)_p\coloneqq\frac d{dt}(d\theta_{t})_p^{-1}(W_{\theta_t(p)})\bigg|_{t=0}\in T_pM,\]
	where $\theta\colon\mc D\to M$ is the flow of $V$.
\end{definition}
The point of this definition is that we should want to take the derivative with respect to the ``direction'' of $V$ just by looking at how $W$ changes along the flow $\theta$ of $V$. This is essentially what the definition above is doing; note that the limit in the derivative makes sense because we only ever have vectors in $T_pM$.
\begin{remark}
	On smooth functions $f\in C^\infty(M)$, the analogous computation is
	\[(\mc L_Vf)_p=\frac d{dt}(d\theta_t)_p^{-1}=\frac d{dt}(d\theta_t)_p^{-1}f\bigg|_{t=0}=\frac d{dt}f(\theta_t(p))\bigg|_{t=0}=df_p(V_p)=V_p(f).\]
	So our notion of derivative makes some sense.
\end{remark}
\begin{remark}
	One can see that $\mc L_VW$ assembles into a vector field on $\mf X$. This amounts to checking that everything is sufficiently smooth on charts, which can be done by writing everything out. For example, one can use \Cref{lem:get-vector-field-form} to make $V$ particularly nice, allowing for easy computation of the flow, whereupon we see that we are basically differentiating $W$ along a direction in Euclidean space, which is legal because $V$ and $W$ started out as smooth anyway.
\end{remark}
In fact, the Lie derivative agrees with our bracket!
\begin{proposition}
	Fix vector fields $V$ and $W$ of a smooth manifold $M$. Then $\mc L_VW=[V,W]$.
\end{proposition}
\begin{proof}
	We check this at individual points $p\in M$. Note that the question is local, so we will repeatedly shrink $M$ without much comment. The hardest case is when $V_p\ne0$. In this case, we use \Cref{lem:get-vector-field-form} in order to replace $M$ with $\RR^n$ so that $V=\frac\del/\del x_1$. Then the flow $\theta_t(x)$ is just
	\[\theta_t(x)\coloneqq(t+x_1,x_2,\ldots,x_n).\]
	In this case, we expand out our Lie derivative to find
	\[(\mc L_VW)_p=\sum_j\frac{\del W_j}{\del x_1}\frac{\del}{\del x_1},\]
	and a direct computation shows
	\[[V,W]=\sum_{i,j}\left(V_i\frac{\del W_j}{\del x_i}-W_i\frac{\del V_j}{x_i}\right)\frac{\del}{\del x_j},\]
	which indeed agree because $V_i=1_{i=0}$.
	
	We now handle the other cases. If $V$ vanishes in a neighborhood of $p$, then everything in sight vanishes (namely, the flow is constant), so there is nothing to do; the previous two cases are dense in $M$, so we are done.
\end{proof}
\begin{example}
	Consider the vector field $V=r\del r$ on Euclidean space $\RR^2$, where we defined $V$ using polar coordinates $(r,\alpha)$. Then this vector field is complete with flow given by $\theta_t(p)\coloneqq e^tp$. With $W\coloneqq\del/\del\alpha$, we expect to have $\mc L_VW$ because the two vector fields are essentially perpendicular, and indeed we can compute
	\[[V,W]=[r\del r,\del\alpha]=r[\del r,\del\alpha]-\del_\alpha(r)\del r,\]
	which vanishes.
\end{example}
\begin{remark}
	Here are a few more properties of the Lie derivative.
	\begin{itemize}
		\item By the antisymmetry of the Lie bracket, we see $\mc L_VW=-\mc L_WV$.
		\item The Jacobi identity grants the ``product rule''
		\[\mc L_V[W,X]=[\mc L_VW,X]+[W,\mc L_VX].\]
		Alternatively, one can write $\mc L_{[V,W]}X=\mc L_V\mc L_WX-\mc L_W\mc L_VX$.
	\end{itemize}
\end{remark}

\subsection{Commuting Vector Fields}
We are now able to provide a better understanding of commuting vector fields.
\begin{proposition}
	Fix vector fields $V$ and $W$ on a smooth manifold $M$. Then the following are equivalent.
	\begin{listalph}
		\item $V$ and $W$ commute: $[V,W]=0$.
		\item $W$ is invariant under the flow of $V$: $(\theta_s^V)_*W=W$ for all $s$, where $\theta_s^V$ is the flow of $V$.
		\item $V$ is invariant under the flow of $W$: $(\theta_t^W)_*V=V$ for all $t$, where $\theta_t^W$ is the flow of $W$.
		\item The flows of $V$ and $W$ commute: one has $\theta_s^V\circ\theta_t^W=\theta_t^W\circ\theta_s^V$.
	\end{listalph}
\end{proposition}
\begin{proof}
	Note (b) implies that $\mc L_VW=0$ because it tells us that $(d\theta_s)_p^{-1}(W_{\theta_s(p)})$ is constant in $s$ by the invariance. So (b) implies (a); an analogous argument shows (c) implies (a).

	Next up, to use (d), we see that
	\[\frac d{dt}(\theta_s^V\circ\theta_t^W)(p)\bigg|_{t=0}=(d\theta_s^V)_p(W_p)\]
	by definition of the flow $\theta_t^W$, and
	\[\frac d{dt}(\theta_t^W\circ\theta_s^V)(p)\bigg|_{t=0}=W(\theta_s^V(p))\]
	again by using the definition of the flow $\theta_t^W$. Comparing our two equalities, we see that we have achieved (b); a symmetric argument is able to show that (d) implies (c).

	Lastly, we have to show (a) implies (b)--(d); we focus on (b) and (d) and get (c) by symmetry. Fix a smooth chart $(U,(x_1,\ldots,x_n))$ of $M$. Assuming $V_p\ne0$ for now, we may assume that $V|_U=\del/\del x_1$. Now writing $W=\sum_jW_j\frac\del{\del x_j}$, we note that commuting $[V,W]=0$ implies $\mc L_VW=0$, which by a direct computation expanding the Lie derivative forces
	\[\sum_j\frac\del{\del x_1}W_j=0.\]
	This implies (b) and (d) locally at $p$. We should also consider the case where $p$ is outside the support of $V$, but then everything in sight vanishes; a continuity argument now achieves the result for all $p\in M$.
\end{proof}
We can now expand to many commuting vector fields.
\begin{theorem}
	Fix an open submanifold $W\subseteq M$, and let $V_1,\ldots,V_k$ be linearly independent commuting vector fields of $W$. Then any $p\in W$ has a smooth chart $(U,(x_1,\ldots,x_n))$ with $V_i=\del/\del x_i$ for each $i$.
\end{theorem}
\begin{proof}[Sketch]
	Everything is local, so we may replace $M$ with $W$. By shrinking $M$ to a regular coordinate ball, we may assume that the $V_\bullet$s are complete. (Namely, being local means that we can replace them with compactly supported counterparts which agree in an open neighborhood $p$, but then we can shrink $M$ to this open neighborhood.) For simplicity, we will take $k=n$, but it has no impact on the actual logic of the argument.

	Now, the point is that we should be able to read off the needed coordinate functions $x_\bullet$ by following the flows of $V$. Indeed, one just checks that the map
	\[(t_1,\ldots,t_n)\mapsto\left(\theta^{V_1}_{s_1}\circ\cdots\circ\theta^{V_n}_{s_n}\right)\]
	is a local diffeomorphism, which is enough for our purposes.
\end{proof}
\begin{remark}
	If $k=n$, then we are noting that commuting local frames are actually local coordinate frames on (perhaps) a smaller open neighborhood.
\end{remark}

\end{document}