% LTeX: enabled=false

\documentclass{article}
\usepackage[utf8]{inputenc}

\newcommand{\nirpdftitle}{Manifolds Speedrun}
\usepackage{import}
\inputfrom{..}{nir}

\pagestyle{contentpage}

\title{Smooth Manifolds for the Impatient}
\author{Nir Elber}
\date{Spring 2024}
\rhead{\textit{MANIFOLDS SPEEDRUN}}
\lhead{}

\setcounter{tocdepth}{2}

\begin{document}

\maketitle

\begin{abstract}
	This document collects a variety of definitions and results from the starting theory of smooth manifolds.
\end{abstract}

\tableofcontents

\newpage

\section{Definitions}

\subsection{Point-Set Topology}
\begin{definition}[Hausdorff]
	A topological space $X$ is \textit{Hausdorff} if and only if any two distinct points $p,q\in X$ have disjoint open neighborhoods $U,V\subseteq M$; i.e., $p\in U$ and $q\in V$ but $U\cap V=\emp$.
\end{definition}
\begin{definition}[second-countable]
	A topological space $X$ is \textit{second-countable} if and only if the topology on $X$ has a countable base.
\end{definition}
\begin{definition}[locally Euclidean]
	Fix a nonnegative integer $n$. A topological space $X$ is \textit{locally Euclidean of dimension $n$} if and only if each $p\in X$ has some open neighborhood $U\subseteq X$ and open subset $\widehat U\subseteq\RR^n$ such that there is a homeomorphism $\varphi\colon U\to\widehat U$. The piar $(U,\varphi)$ is called a \textit{coordinate chart}.
\end{definition}
\begin{remark}
	It is a result of cohomology that if one has homeomorphic nonempty open subsets $U\subseteq\RR^m$ and $V\subseteq\RR^n$, then $m=n$.
\end{remark}
\begin{remark}
	In the above definition, one may assume that $\widehat U$ is an open ball, essentially by replacing $\widehat U$ with an open ball containing $\varphi(p)$ and replacing $U$ with the preimage of this open ball. One can even assume that $\widehat U$ is all of $\RR^n$ because $\RR^n$ is homeomorphic to the open ball.
\end{remark}
\begin{remark}
	The above remark allows us to give any topological $n$-manifold a countable base of precompact open subsets.
\end{remark}
\begin{definition}[topological manifold]
	A topological space $X$ is a \textit{topological $n$-manifold} if and only if it is Hausdorff, second-countable, and locally Euclidean of dimension $n$. If one weakens open subsets of $\RR^n$ in locally Euclidean to open subsets of $\HH^n$, then $X$ is a \textit{topological $n$-manifold with boundary}.
\end{definition}
\begin{definition}[connected]
	A topological space $X$ is \textit{connected} if and only if the only subsets of $X$ which are both open and closed are $\emp$ and $X$.
\end{definition}
\begin{definition}[path-connected]
	A topological space $X$ is \textit{path-connected} if and only if any two points $p,q\in X$ have some path $\gamma\colon I\to X$ such that $\gamma(0)=p$ and $\gamma(1)=q$.
\end{definition}
\begin{definition}[locally path-connected]
	A topological space $X$ is \textit{locally path-connected} if and only if $X$ has a base of path-connected open subsets.
\end{definition}
\begin{remark}
	Topological manifolds $M$ are locally path-connected because $\RR^n$ is path-connected. Thus, $M$ is connected if and only if path-connected.
\end{remark}
\begin{remark}
	Having a countable base of precompact coordinate balls implies by an inductive argument that $\pi_1(M)$ is actually countable.
\end{remark}
\begin{definition}[locally compact]
	A Hausdorff topological space $X$ is \textit{locally compact} if and only if each $x\in X$ has some open neighborhood $U$ contained in a compact subset $K$.
\end{definition}
\begin{remark}
	Any topological manifold is locally compact because a basis of smooth charts can be refined into one where each basic open subset is precompact.
\end{remark}
\begin{definition}[paracompact]
	A topological space $X$ is \textit{paracompact} if and only if any open cover $\mc U$ of $M$ admits an open, locally finite refinement. Here, \textit{locally finite} means that any $p\in X$ has some open neighborhood $V$ such that $\#\{U\in\mc U:U\cap V\ne\emp\}<\infty$.
\end{definition}

\subsection{Smooth Structures}
\begin{definition}[diffeomorphism]
	A \textit{diffeomorphism} $F\colon U\to V$ between two open subsets of Euclidean space is a bijective smooth map with smooth inverse.
\end{definition}
\begin{definition}[smooth structure]
	An \textit{atlas} $\mc A$ on a topological $n$-manifold $M$ is a collection of charts which cover $M$. The atlas $\mc A$ is \textit{smooth} if and only if any two charts $(U,\varphi),(V,\psi)\in\mc A$ makes the transition map
	\[\varphi|_{U\cap V}\circ\psi|_{U\cap V}^{-1}\]
	into a diffeomorphism. A \textit{smooth structure} is a maximal smooth atlas.
\end{definition}
\begin{remark}
	Any smooth atlas $\mc A$ on $M$ is contained in a unique smooth structure $\ov{\mc A}$. Explicitly, one can construct $\ov{\mc A}$ as the collection of smooth charts which make the relevant transition maps smooth.
\end{remark}
\begin{definition}[smooth manifold]
	A \textit{smooth $n$-manifold} is a pair $(M,\mc A)$ of a topological $n$-manifold $M$ and smooth structure $\mc A$ on $M$. If $M$ is merely a topological $n$-manifold with boundary, then, then $M$ is a \textit{smooth $n$-manifold with boundary}.
\end{definition}
\begin{definition}[interior, boundary]
	Fix a smooth $n$-manifold $M$ with boundary. A point $p\in M$ is an \textit{interior point} if and only if any smooth chart $(U,\varphi)$ on $M$ makes $\varphi(p)$ in the interior of $\HH^n$. A point $p\in M$ is a \textit{boundary point} if and only if any smooth chart $(U,\varphi)$ on $M$ makes $\varphi(p)$ in the boundary of $\HH^n$.
\end{definition}
\begin{remark}
	Any point in $M$ is either an interior point or boundary point.
\end{remark}

\subsection{Smooth Maps}
\begin{definition}[smooth]
	A map $F\colon M\to N$ is \textit{smooth} if and only if any $p\in M$ has smooth charts $(U,\varphi)$ on $M$ and $(V,\psi)$ on $N$ such that $p\in U$ and $F(U)\subseteq V$ and the composite $\psi\circ F\circ\varphi^{-1}$ is smooth.
\end{definition}
\begin{remark}
	By compatbility of smooth charts, if $F$ is smooth, then actually any smooth charts $(U,\varphi)$ on $M$ and $(V,\psi)$ with $F(U)\subseteq V$ will make $\psi\circ F\circ\varphi^{-1}$ smooth.
\end{remark}
\begin{definition}[diffeomorphism]
	A bijective smooth map $F$ is a \textit{diffeomorphism} if and only if $F^{-1}$ is smooth.
\end{definition}
\begin{definition}[tangent vector]
	Fix a smooth $n$-manifold $M$. Then a \textit{tangent vector} or \textit{derivation} at some $p\in M$ is an $\RR$-linear map $v\colon C^\infty(M)\to\RR$ satisfying the Leibniz rule
	\[v(fg)=f(p)v(g)+g(p)v(f).\]
	We let $T_pM$ denote the vector space of derivations at $p$.
\end{definition}
\begin{definition}[differential]
	Fix a smooth map $F\colon M\to N$ of smooth manifolds. Then the \textit{differential of $F$ at $p\in M$} is the linear map $dF_p\colon T_pM\to T_{F(p)}N$ defined by
	\[dF_p(v)(f)\coloneqq v(f\circ F).\]
\end{definition}
\begin{definition}[tangent bundle]
	Fix a smooth $n$-manifold $M$. Then the \textit{tangent bundle $TM$} is the smooth $n$-manifold
	\[TM\coloneqq\bigsqcup_{p\in M}T_pM\]
	equipped with the smooth charts given by $TU\cong T\widehat U\cong\widehat U\times\RR^n$ for any smooth chart $(U,\varphi)$ where $\widehat U\coloneqq\im\varphi$.
\end{definition}
\begin{definition}[constant rank]
	Fix a smooth map $F\colon M\to N$. If $\op{rank}dF_p$ is constant for all $p\in M$, then $F$ has \textit{constant rank $r$}. For example, if $\op{rank}dF_p$ is always surjective, then $F$ is a \textit{submersion}; if $\op{rank}dF_p$ is always injective, then $F$ is an \textit{immersion}.
\end{definition}
\begin{remark}
	These are local properties: if $\op{rank}dF_p$ is injective (resp., surjective, invertible), then there is an open neighborhood $U\subseteq M$ of $p$ such that $F|_U$ is an immersion (resp., submersion, diffeomorphism).
\end{remark}
\begin{definition}[embedding]
	A smooth map $F\colon M\to N$ of smooth manifolds is a \textit{smooth embedding} if and only if $F$ is an immersion and a topological embedding.
\end{definition}
\begin{remark}
	Injective smooth immersions $F\colon M\to N$ become embeddings as soon as one can show that they are topological embeddings. For example, it is enough for $F$ to be open, to be closed, to be proper (which implies closed), for $M$ to be compact (which implies proper), or for $\del M=0$ and $\dim M=\dim N$.
\end{remark}
\begin{definition}[smooth covering]
	A smooth map $\pi\colon E\to M$ of smooth manifolds is a \textit{smooth covering map} if and only if each $p\in M$ has an \textit{evenly covered open} neighborhood $U\subseteq M$ such that the restriction $\pi\colon\pi^{-1}(U)\to U$ maps its connected components diffeomorphically to $U$.
\end{definition}
\begin{remark}
	Suppose $M$ is a connected smooth $n$-manifold. Then topological covering maps $\pi\colon E\to M$ gives $E$ a unique smooth $n$-manifold structure making $\pi$ a smooth covering map.
\end{remark}
\begin{remark}
	A local diffeomorphism $\pi$ upgrades to a smooth covering map if it is proper. Proper is equivalent to finite fibers for smooth covering maps, so this condition is not necessary.
\end{remark}

\section{Coherences}
The following result is used to build the Grassmannian and the tangent bundle.
\begin{proposition}
	Fix a set $M$ and a collection of subsets $\{U_\alpha\}_{\alpha\in\kappa}$ of subsets together with maps $\varphi_\alpha\colon U_\alpha\to\RR^n$ satisfying the following.
	\begin{listroman}
		\item Charts: $\varphi_\alpha$ is a bijection to an open subset of $\RR^n$.
		\item Smooth charts: the sets $\varphi_\alpha(U_\alpha\cap U_\beta)$ and $\varphi_\beta(U_\alpha\cap U_\beta)$ are open, and the transition map $\varphi_\beta\circ\varphi_\alpha^{-1}$ is smooth.
		\item Second-countable: $\{U_\alpha\}_{\alpha\in\kappa}$ has a countable subcover.
		\item Hausdorff: for any distinct $p,q\in M$, either $p,q\in U_\alpha$ for some $\alpha$, or $p\in U_\alpha$ and $q\in U_\beta$ for disjoint $U_\alpha$ and $U_\beta$.
	\end{listroman}
	Then $M$ has a unique smooth $n$-manifold structure with smooth atlas $\{(U_\alpha,\varphi_\alpha)\}_{\alpha\in\kappa}$.
\end{proposition}
The following result explains how to use curves for differentials.
\begin{proposition}
	Fix a smooth $n$-manifold without boundary. For any $p\in M$ and $v\in T_pM$, there is a smooth injective map $\gamma\colon(-\varepsilon,\varepsilon)\to M$ with $\gamma(0)=p$ and $d\gamma_0\left(\frac d{dt}\big|_p\right)=v$.
\end{proposition}
The following result explains how to think about smooth submersions.
\begin{proposition}
	Fix a smooth map $\pi\colon M\to N$ of smooth manifolds. Then $\pi$ is a smooth submersion if and only if any point of $M$ is in the image of a local setion $\sigma\colon N\to M$ of $\pi$.
\end{proposition}

\section{Examples}

\subsection{Smooth Manifolds}
\begin{example}
	Note $\RR^n$ is a smooth manifold: it is certainly a topological manifold, and it has smooth atlas given by $(\RR^n,{\id_{\RR^N}})$.
\end{example}
\begin{example}
	Note $S^n\coloneqq\{x\in\RR^{n+1}:\left|x\right|=1\}$ is a smooth $n$-manifold. Indeed, it is the level set of the smooth function $\left|\cdot\right|^2\colon\RR^{n+1}\to\RR$ at the regular value $1$. For our smooth atlas, define the projection map $\pi_i^\pm\colon U_i^\pm\to B(0,1)$ where
	\[U_i^\pm\coloneqq\left\{x\in S^n:\pm x_i>0\right\}.\]
	Then one can check that $\pi_i$ is a diffeomorphism, providing our smooth atlas.

	Alternatively, one can define the stereographic projection $\sigma\colon\left(S^n\setminus\{(0,\ldots,0,1)\}\right)\to\RR^n$ by
	\[\sigma(x_1,\ldots,x_{n+1})\coloneqq\frac{(x_1,\ldots,x_n)}{1-x_{n+1}}\qquad\text{and}\qquad\sigma^{-1}(u_1,\ldots,u_n)\coloneqq\frac{\left(2u_1,\ldots,2u_n,\left|u\right|^2-1\right)}{\left|u\right|^2+1},\]
	which are both smooth because they are smooth as functions between Euclidean spaces.
\end{example}
\begin{example}
	Given a real vector space $\RR^{n+1}$, one can define the smooth $n$-manifold $\RP^n$ as equivalence classes of lines. The standard smooth atlas is given by the projection $\pi_i\colon U_i\to\RR^n$ where we set $U_i\coloneqq\left\{x\in\RP^n:x_i\ne0\right\}$; the inverse of the projection is given by
	\[(x_0,\ldots,\widehat x_i,\ldots,x_n)\mapsto[x_0:\cdots:x_{i-1}:1:x_{i+1}:\cdots:x_n].\]
	One can define $\PP(V)$ for a general real or complex vector space $V$ in an analogous way.
\end{example}
\begin{example}
	Given smooth manifolds $M_1,\ldots,M_k$ without boundary, one can define the product $M_1\times\cdots\times M_k$. Smooth charts are given by taking products of smooth charts on the individual $M_\bullet$s.
\end{example}
\begin{example}
	Given a smooth $n$-manifold $M$, any open subset $U\subseteq M$ is also a smooth $n$-manifold. The smooth structure on $U$ is given by restricting any smooth chart on $M$ to $U$.
\end{example}

\subsection{Smooth Maps}
\begin{example}
	Constant maps are smooth.
\end{example}
\begin{example}
	The identity map is smooth. More generally, if $U\subseteq M$ is an open subset, then the inclusion $i\colon U\to M$ is a smooth embedding.
\end{example}
\begin{example}
	A smooth chart $(U,\varphi)$ on a smooth $n$-manifold $M$ induces a diffeomorphism $\varphi\colon U\to\widehat U$ where $\widehat U\coloneqq\im\varphi\subseteq\RR^n$.
\end{example}
\begin{example}
	For any $p\in\RR^n$, the vector space $T_p\RR^n$ is $n$-dimensional spanned by the derivations $\frac{\del}{\del x_i}\big|_p$.
\end{example}
\begin{example}
	Fix a smooth $n$-manifold $M$. For any smooth chart $(U,\varphi)$ on $M$, set $\widetilde U\coloneqq\im\varphi$, and one finds that any $p\in U$ makes
	\[T_pM\cong T_pU\cong T_{\varphi(p)}\widehat U\cong T_{\varphi(p)}\RR^n.\]
	As such, $T_pM$ is $n$-dimensional spanned by the derivations
	\[\frac{\del}{\del x_i}\bigg|_p\coloneqq(d\varphi_p)^{-1}\left(\frac{\del}{\del x_i}\bigg|_{\varphi(p)}\right).\]
\end{example}
\begin{example}
	Fix a smooth map $F\colon M\to N$ of smooth $n$-manifolds. Given $p\in M$, fix smooth charts $(U,\varphi)$ on $M$ and $(V,\psi)$ such that $p\in U$ and $F(p)\in V$. (By restricting, we may assue that $F(U)\subseteq V$.) Let $\widetilde F\coloneqq\psi\circ F\circ\varphi^{-1}$ and $\widehat p\coloneqq\varphi(p)$ be the coordinate representations. Then
	\[dF_p\left(\frac{\del}{\del x_i}\bigg|_p\right)=d\left(F\circ\varphi^{-1}\right)_{\widehat p}\frac{\del}{\del x_i}\bigg|_{\widehat p}=d\left(\psi^{-1}\circ\widehat F\right)_{\widehat p}\frac{\del}{\del x_i}\bigg|_{\widehat p}=\left(d\psi_{\widehat F(\widehat p)}\right)^{-1}d\widehat F_{\widehat p}\left(\frac{\del}{\del x_i}\bigg|_{\widehat p}\right).\]
	By the multivariate chain rule (and an explicit computation on $f\in C^\infty(N)$), this is
	\[\left(d\psi_{\widehat F(\widehat p)}\right)^{-1}\left(\sum_{j=1}^{\dim N}\frac{\del\widehat F_j}{\del x_i}(\widehat p)\frac{\del}{\del y_j}\bigg|_{\widehat F(\widehat p)}\right)=\sum_{j=1}^{\dim N}\frac{\del\widehat F_j}{\del x_i}(\widehat p)\frac{\del}{\del y_j}\bigg|_{\widehat F(\widehat p)}.\]
	So $dF_p$ is given by the Jacobian matrix.
\end{example}
\begin{example}
	Taking $F=\id_M$ in the above example, we see
	\[d\left(\psi\circ\varphi^{-1}\right)_{\varphi(p)}\left(\frac\del{\del x_i}\bigg|_{\varphi(p)}\right)=\sum_{j=1}^{\dim M}\frac{\del y_j}{\del x_i}(\varphi(p))\frac{\del}{\del y_j}\bigg|_{\psi(p)}.\]
	Rearranging, we see\todo{}
	\[\frac\del{\del x_i}\bigg|_{p}=d\left(\psi^{-1}\right)_{\psi(p)}\circ d\left(\psi\circ\varphi^{-1}\right)_{\varphi(p)}\left(\frac{\del}{\del x_i}\bigg|_{\varphi(p)}\right)=\sum_{j=1}^{\dim M}\frac{\del y_j}{\del x_i}(\varphi(p))\frac{\del}{\del y_j}\bigg|_{\psi(p)}.\]
\end{example}
\begin{example}
	The projection map $\pi\colon\RR^{n+1}\setminus\{0\}\to\RP^n$ is a smooth surjective submersion. Surjectivity has little content, and smoothness follows by checking on charts. Being a submersion is also checked on charts: for $p\in\RR^{n+1}\setminus\{0\}$ given by $p\coloneqq(z_0,\ldots,z_n)$ such that $z_i\ne0$, it is enough to check that the composite
	\[T_p\RR^{n+1}\cong T_p\left(\RR^{n+1}\setminus\{0\}\right)\stackrel\pi\to T_{\pi(p)}\RP^n\stackrel{\varphi_i}\to T_{\varphi_i(\pi(p))}U_i\]
	given by $(x_0,\ldots,x_n)\mapsto(x_0/x_i,\ldots,\widehat1,\ldots,x_n/x_i)$ is a smooth submersion at $p$. But we see that $\frac{\del}{\del x_j}$ goes to $\frac1{z_i}\frac{\del}{\del x_j}$ for $j\ne i$, which is enough.
\end{example}
\begin{example}
	The projection map $\pi\colon\RR^{n+1}\setminus\{0\}\to S^n$ is a smooth surjective submersion. Surjectivity has little content, and smoothness follows because we have local sections: any $p\in\RR^{n+1}\setminus\{0\}$ is in the local section $S^n\to\RR^{n+1}\setminus\{0\}$ given by $x\mapsto\left|p\right|x$. Namely, this section is smooth because it is smooth as a map $\RR^{n+1}\to\RR^{n+1}$.
\end{example}
\begin{example}
	The projection map $\pi\colon S^n\to\RP^n$ is a smooth covering map: the projection $\RR^{n+1}\to S^n$ is smooth, so it is enough to check that $\RR^{n+1}\to\RP^n$ is smooth, which we know. Also, $\RR^{n+1}\to\RP^n$ is a submersion, so it is surjective on differentials, so the induced map $S^n\to\RP^n$ must also be surjective on differentials. Lastly, $S^n$ is compact, and this map is surjective, so we see that we have a proper local diffeomorphism, which is a smooth covering map.
\end{example}

\section{Theorems}
\begin{theorem}[Partition of unity]
	Fix a smooth $n$-manifold, possibly with boundary. Given any open cover $\mc U$, there is a partition of unity subordinate to $\mc U$. In other words, there are smooth functions $\psi_U\colon M\to\RR$ satisfying the following.
	\begin{listalph}
		\item $\im\psi_U\subseteq[0,1]$.
		\item $\op{supp}\psi_U\subseteq U$.
		\item The collection $\{\op{supp}\psi_U\}_{U\in\mc U}$ is locally finite.
		\item $\sum_{U\in\mc U}\psi_U=1$.
	\end{listalph}
\end{theorem}
\begin{corollary}[Extension lemma]
	Fix a smooth $n$-manifold, possibly with boundary. For any closed subset $A\subseteq M$ and smooth function $f\colon A\to\RR^k$ (i.e., $f$ has a smooth extension in a neighborhood of each point) and open neighborhood $U$ of $A$, there is a smooth function $\widetilde f\colon M\to\RR^k$ extending $f$ such that $\op{supp}\widetilde f\subseteq U$.
\end{corollary}
\begin{theorem}[Rank]
	Fix a smooth $m$-manifold $M$ and $n$-manifold $N$, and fix a smooth map $F\colon M\to N$ of constant rank $r$. For any $p\in M$, there is a chart $(U,\varphi)$ on $M$ and a chart $(V,\psi)$ on $N$ such that $p\in U$ and $F(U)\subseteq V$ and
	\[\left(\psi\circ F\circ\varphi^{-1}\right)(x_1,\ldots,x_r,x_{r+1},\ldots,x_m)=(x_1,\ldots,x_r,0,\ldots,0).\]
\end{theorem}
\begin{remark}
	Fix a smooth $n$-manifold. For any closed subset $A\subseteq M$, there is a smooth nonnegative function $f\colon M\to\RR$ such that $f^{-1}(\{0\})=K$.
\end{remark}
\begin{theorem}
	Fix a smooth surjective submersion $\pi\colon M\to N$ of smooth manifolds. Suppose $P$ is a smooth manifold.
	\begin{listalph}
		\item Check smoothness: a map $\ov F\colon N\to P$ is smooth if and only if $(\ov F\circ\pi)\colon M\to P$ is smooth.
		\item Produce smootheness: given a smooth map $F\colon M\to P$ constant on the fibers of $\pi$, there is a unique smooth map $\ov F\colon N\to P$ such that $F=\ov F\circ\pi$. 
	\end{listalph}
\end{theorem}

\end{document}