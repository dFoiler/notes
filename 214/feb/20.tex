% !TEX root = ../notes.tex

\documentclass[../notes.tex]{subfiles}

\begin{document}

The fun never ends.

\subsection{More on Level Sets}
Last class we were stated the following result.
\levelsubthm*
\begin{proof}
	We apply \Cref{thm:slice} to $S\coloneqq F^{-1}(\{q\})$, for which we will use \Cref{thm:rank}. For each $p_0\in S$, we receive smooth charts $(U,\varphi)$ on $M$ (around $p_0$) and $(V,\psi)$ on $N$ (with $F(U)\subseteq V$) such that
	\[\left(\psi\circ F\circ\varphi^{-1}\right)(x_1,\ldots,x_m)=(x_1,\ldots,x_r,0,\ldots,0).\]
	In particular, write $\psi(q)=(c_1,\ldots,c_r,0,\ldots,0)$, and we see that
	\[S\cap U=\{p\in U:\varphi_\ell(p)=c_\ell\text{ for }\ell\le r\},\]
	which is in fact an $(m-r)$-slice. Thus, \Cref{thm:slice} applies, and to finish up, we note that $S\subseteq M$ is certainly closed and hence proper by \Cref{prop:proper-is-closed}.
\end{proof}
\begin{example} \label{ex:submersion-level-set}
	If $F\colon M\to N$ is a submersion, then $F^{-1}(\{q\})$ is a proper embedded submanifold of dimension $(\dim M)-(\dim N)$ for any $q\in N$.
\end{example}

\subsection{Regularity}
We will want to understand the differential of a smooth map pointwise, for which we provide some language.
\begin{defihelper}[regular, critical] \nirindex{regular} \nirindex{critical}
	Fix a smooth map $F\colon M\to N$.
	\begin{itemize}
		\item A point $p\in M$ is \textit{regular} if and only if $dF_p$ is surjective; otherwise, $p\in M$ is \textit{critical}.
		\item A value $q\in N$ is \textit{regular} if and only if all points in $F^{-1}(\{q\})$ are regular; otherwise, $q\in N$ is \textit{critical}.
	\end{itemize}
\end{defihelper}
\begin{example}
	Given a function $f\colon\RR\to\RR$, we see that the point $x_0\in\RR$ is regular if and only if $f'(x_0)=0$, based on some Jacobian computation reducing $T_{x_0}f$ to $\frac d{dx}f\big|_{x_0}$. Accordingly, the critical values are exactly when some point in the fiber is critical.
\end{example}
\begin{example}
	Continuing from \Cref{ex:almost-submersion}, we see that the regular points of $\RR^2$ are just $\RR^2\setminus\{0\}$, so the collection of regular values is $\RR\setminus\{0\}$, which has pre-image $\RR^2\setminus\{(x,y):xy=0\}$.
\end{example}
It will turn out that the set of critical values will always be small (namely, measure zero).
\begin{remark} \label{rem:reg-locus-is-open}
	Note that the set of regular points $M'$ of $M$ is open: the map sending $p\in M$ to the ordered list of determinants of the largest square minors of $M$ is continuous by checking on charts (where this function is a polynomial), and being regular means that we are interested in the pre-image where at least one coordinate is nonzero. Thus, so $F|_{M'}\colon M'\to N$ will be a submersion provided that there is some regular input to $F$.
\end{remark}
Anyway, we get the following result.
\begin{proposition} \label{prop:level-set-regular}
	Fix a smooth map $F\colon M\to N$ from the $m$-manifold $M$ to the $n$-manifold $N$, and let $q\in N$ be a regular value. Then $F^{-1}(\{q\})$ is a proper embedded submanifold of dimension $m-n$.
\end{proposition}
\begin{proof}
	Let $U\subseteq M$ be the set of regular points in $M$, which is nonempty because $N$ has a regular value; in particular, $F^{-1}(\{q\})\subseteq U$. Now, $F|_U\colon U\to N$ is a submersion by the regularity of each $p\in U$, so \Cref{ex:submersion-level-set} tells us that $F^{-1}(\{q\})\subseteq M$ is an embedded submanifold of dimension $m-n$. Lastly, $F^{-1}(\{q\})$ is still proper by \Cref{prop:proper-is-closed} because it is closed.
\end{proof}
\begin{example} \label{ex:sphere-level-set}
	Define $F\colon\RR^{n+1}\to\RR$ by $F(p)\coloneqq\left|x\right|^2$. Then $S^n=F^{-1}(\{1\})$ will be a proper embedded submanifold of dimension $n$ by \Cref{prop:level-set-regular}. Indeed, it is enough to check that $1\in\RR$ is a regular value of $F$. Well, for $p=(x_0,\ldots,x_n)\in F$, we can compute $dF_p$ as the Jacobian matrix
	\[\begin{bmatrix}
		2x_0 & \cdots & 2x_n
	\end{bmatrix}.\]
	Notably, this has full rank $1$ unless $p=0$, and $F^{-1}(\{1\})\cap\{0\}=\emp$, so we are safe.
\end{example}
\begin{example}
	Define $F\colon\RR^2\to\RR$ by $F(x,y)\coloneqq\left(x^2+y^2-1\right)^2$. Then for $p=(x_0,y_0)$, we can compute $dF_p$ as the Jacobian matrix
	\[\begin{bmatrix}
		4x_0\left(x_0^2+x_1^2-1\right) & 4x_0\left(x_0^2+x_1^2-1\right)
	\end{bmatrix},\]
	so $S^1\subseteq\RR^2$ now contains entirely critical points even though $S^1=F^{-1}(\{1\})$ is a perfectly fine smooth embedded submanifold of dimension $1$.
\end{example}
\begin{example}
	Consider the torus $T^2\coloneqq S^1\times S^1$, and define $F\colon T^2\to\RR$ by some kind of height function, achieved by embedding $T^2\subseteq\RR^3$. Then the pre-image of the critical values of this height make figure-$8$s, which are not smooth embedded submanifolds.
\end{example}
These regular values also allow us to sensibly discuss defining functions.
\begin{definition}[defining function]
	Fix a smooth embedded submanifold $S\subseteq M$, where $k\coloneqq\dim S$ and $m\coloneqq\dim M$. Then a smooth function $F\colon M\to N$ is a \textit{defining function for $S$} if and only if $S=F^{-1}(\{q\})$ for some regular value $q\in N$. Locally over some open subset $U\subseteq M$, we say that a smooth map $F\colon M\to N$ is a \textit{local defining function for $F$ at some $p\in S$} if and only if $S\cap U=F^{-1}(\{q\})$ for some regular value $q\in N$.
\end{definition}
The local notion is useful because it is universal.
\begin{proposition}
	Fix a subset $S$ of a smooth $m$-manifold $M$. Then $S$ is a $k$-dimensional embedded submanifold of $M$ if and only if any $p\in S$ has some open neighborhood $U\subseteq M$ of $p$ such that there is a local defining function $F\colon U\to\RR^{m-k}$ for any $p\in S$.
\end{proposition}
\begin{proof}[Sketch]
	Use \Cref{thm:slice} to realize $F$ as a projection onto the relevant coordinates.
\end{proof}

\subsection{Tangent Vectors}
Embedded submanifolds will produce a natural embedding on tangent spaces, which we now use.
\begin{definition}[tangent vector]
	Fix an embedded $k$-submanifold $S$ of the smooth $m$-manifold $M$. For any $p\in S$, we define $T_p^{\mathrm{extrinsic}}S\coloneqq\im d\iota_p$, where $\iota\colon S\to M$ is the inclusion. Namely, we are viewing $T_pS$ as a $k$-dimensional subspace of $T_pM$.
\end{definition}
\begin{example}
	Let $(U,\varphi)$ be a local $k$-slice chart for $S$ so that
	\[S\cap U=\{p\in U:\varphi_\ell(p)=c_\ell\text{ for }\ell>k\}.\]
	Then we see $T_p^{\mathrm{extrinsic}}$ is just the span of $\left\{\frac{\del}{\del x_1}\big|_p,\ldots,\frac{\del}{\del x_k}\big|_p\right\}$.
\end{example}
\begin{example}
	Let $F\colon U\to N$ be a local defining function for $S$ so that $U\cap S=F^{-1}(\{q\})$ for some regular value $q\in N$. Then $T_p^{\mathrm{extrinsic}}S=\ker dF_p$ by tracking through what being a defining function means.
\end{example}
\begin{example}
	Consider the subset
	\[\op O(n)\coloneqq\left\{A\in\RR^{n\times n}:A^\intercal A=I_m\right\}.\]
	We have a natural defining map $F\colon\RR^{n\times n}\to\op{Sym}(n)$ by $A\mapsto A^\intercal A$, and $F$ is certainly smooth because it is a polynomial in the coordinates. We claim that $I_m\in\op{Sym}(n)$ is a regular value for $F$, which implies $\op O(n)\subseteq\RR^{n\times n}$ is a smooth embedded submanifold of codimension $1$ by \Cref{prop:level-set-regular}.
	
	Well, we compute $dF_A$ for $A\in\RR^{n\times n}$ via curves. A curve producing the differential $B\in T_A\RR^{n\times n}$ is simply given by $t\mapsto A+tB$, so
	\[dF_A(B)=\frac d{dt}F(A+tB)\bigg|_{t=0}=\frac d{dt}(A^\intercal+tB^\intercal)(A+tB)\bigg|_{t=0}=\frac d{dt}\left(A^\intercal A+t(B^\intercal A+A^\intercal B)+t^2B^\intercal B\right)\bigg|_{t=0},\]
	which is $B^\intercal A+A^\intercal B$. So we need the map $B\mapsto B^\intercal A+A^\intercal B$ to be surjective, so we will just check that it has kernel of dimension $n^2-\frac12n(n+1)=\frac12n(n-1)$. Well, $B$ lives in the kernel if and only if $B^\intercal A=-A^\intercal B$, or equivalently $A^\intercal B$ is alternating. Taking $A$ to be invertible, we are looking at $A$ times the space of alternating matrices, which is in fact of dimension $\frac12n(n-1)$.
\end{example}
\begin{remark}
	While we're here, we note that we have already computed $T_{I_m}\op O(n)$ extrinsically as
	\[\ker dF_A=\left\{B\in\RR^{n\times n}:B^\intercal+B=0\right\},\]
	which we will later understand as the Lie alegbra.
\end{remark}

\end{document}