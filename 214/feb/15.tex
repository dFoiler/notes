% !TEX root = ../notes.tex

\documentclass[../notes.tex]{subfiles}

\begin{document}

The midterm is in two weeks.

\subsection{Proper Embeddings}
The following notion will be useful.
\begin{definition}
	An embedded smooth submanifold $S\subseteq M$ is \textit{properly embedded} if and only if the inclusion $S\into M$ is proper; i.e., the inverse image of a compact subset of $M$ is still compact in $S$.
\end{definition}
\begin{nex}
	There is an embedding $\RR^2\to S^2$ by inverting the stereographic projection map $\left(S^2\setminus\{(0,0,1)\}\right)\to\RR^2$. However, this is not proper: all of $S^2$ is compact, but its pre-image in $\RR^2$ is all of $\RR^2$, which is not compact.
\end{nex}
Here is a nice way to check properness.
\begin{proposition}
	Fix an embedded smooth submanifold $S\subseteq M$. Then $S$ is properly embedded if and only if $S\subseteq M$ is closed.
\end{proposition}
\begin{proof}
	We have two directions to show.
	\begin{itemize}
		\item Suppose $S\subseteq M$ is closed. Well, for any compact subset $K\subseteq M$, we see that $S\cap K$ is closed in $M$ (it is the intersection of two closed subsets of $M$), so $S\cap K$ is a closed subset of the compact set $K$, so $S\cap K$ continues to be compact.

		\item Suppose $S\subseteq M$ is properly embedded. Then we want to show that $S\subseteq M$ is closed. Well, it suffices to check that $S$ contains all of its limit points, so suppose that $\{x_n\}_{n\in\NN}$ is a sequence of points in $S$ which converges to some point $x\in M$; then we want to show that $x\in S$.

		Well, we note that the subset $\{x_n:n\in\NN\}\cup\{x\}$ is compact (any open cover has an open neighborhood of $x$, and this open neighborhood has all but finitely many of the $x_n$s), so $(\{x_n:n\in\NN\}\cup\{x\})\cap S$ continues to be compact by the proper embedding. But if $x\notin S$, then $\{x_n:n\in\NN\}$ fails to be compact, so instead we must have $x\in S$.
		\qedhere
	\end{itemize}
\end{proof}

\subsection{Slice Charts}
Here is our definition.
\begin{definition}[slice]
	Fix a smooth $n$-manifold and a smooth chart $(U,\varphi)$, where we give $\varphi$ the coordinates $\varphi=(\varphi_1,\ldots,\varphi_n)$. Then a \textit{$k$-slice of $(U,\varphi)$} is the slice
	\[S(c_{k+1},\ldots,c_n)\coloneqq\{p\in U:\varphi_\ell(p)=c_\ell\text{ for }\ell>k\}.\]
	Conversely, a chart $(U,\varphi)$ is a \textit{$k$-slice chart} for a given subset $S\subseteq U$ if and only if $S=S(c_{k+1},\ldots,c_n)$ for some real numbers $(c_{k+1},\ldots,c_n)$. Then a subset $S\subseteq M$ satisfies the \textit{local $k$-slice condition} if and only if any $p\in S$ has a smooth chart $(U,\varphi)$ around $p$ such that $(U,\varphi)$ is a $k$-slice chart for $S\cap U$.
\end{definition}
\begin{example}
	Fix a smooth function $f\colon\RR^m\to\RR^n$, and define the graph
	\[\Gamma(f)\coloneqq\left\{(x,f(x)\in\RR^{m+n}:x\in\RR^m\right\}.\]
	Then $\Gamma(f)\subseteq\RR^{m+n}$ is a (global) $m$-slice chart for the chart $\left(\RR^{m+n},\varphi\right)$, where $\varphi$ is the map $\varphi(x,y)\coloneqq(x,y-f(x))$. (Note that $\varphi$ is of course smooth, and it has smooth inverse given by $(x,y)\mapsto(x,y+f(x))$.) Namely,
	\[\Gamma(f)=\left\{(x,y)\in\RR^m\times\RR^n:\varphi(x,y)=(x,0)\right\},\]
	so we are indeed a slice chart.
\end{example}
Here is our theorem. Approximately, we are saying embedded submanifolds locally look like slices.
\begin{theorem}[Slice] \label{thm:slice}
	Fix a smooth $n$-manifold $M$. A subset $S\subseteq M$ is an embedded $k$-dimensional submanifold if and only if $S$ satisfies the local $k$-slice condition.
\end{theorem}
\begin{proof}
	We have two implications to show, which we do separately.
	\begin{itemize}
		\item Suppose that $S$ is an embedded $k$-dimensional submanifold of $M$, and let $F\colon S\to M$ to be the embedding. We need to show that $S$ satisfies the local $k$-slice condition. Well, fix some $p\in S$, and we need a $k$-slice chart $(U,\varphi)$ around $p\in U$. For this, we use \Cref{thm:rank}, which provides us with smooth charts $(U,\varphi)$ and $(V,\psi)$ around $p\in S$ and $F(p)\in M$, respectively, such $F$ has a coordinate representation given by
		\[\widehat F(x_1,\ldots,x_k)=(x_1,\ldots,x_k,0,\ldots,0),\]
		where $\widehat F\coloneqq\psi\circ F\circ\varphi^{-1}$.
		
		We are almost done, except for a technicality that $V$ might contain other parts of $S$. For brevity, let $\widehat U\coloneqq\varphi(U)$ and $\widehat V\coloneqq\psi(V)$ to be subsets of Euclidean space; notably, $\widehat F(\widehat U)=\widehat U\times\{0\}$. To begin our restriction, set $\widehat V'\coloneqq\widehat V\cap\left(\widehat U\times\RR^{n-k}\right)$ and $V'\coloneqq\psi^{-1}(\widehat V')$, so we are excluding points of $S$ not in $U$ which are near $p$. To exclude points not near $p$, note we can write $U=U'\cap S$ where $U'\subseteq M$ is open, so we define
		\[V''\coloneqq V'\cap U'.\]
		We set $\psi''\coloneqq\psi|_{V''}$.

		We now claim that $(V'',\psi'')$ is the needed local $k$-slice chart of $S$ around $p$. Indeed, we claim that
		\[S\cap V''\stackrel?=\left\{q\in V'':\psi''_\ell(q)=0\text{ for }\ell>k\right\}.\]
		In one direction, $q\in V''\cap S$ implies $q\in U$ by construction, but then $\psi''(q)=\psi(q)\in\RR^{n-k}\times\{0\}$ by definition of $\psi$. In the other direction, if $q\in V''$ has $\psi''_\ell(q)=0$ for $\ell>k$, then (for example) $\psi(q)\in\widehat U\times\RR^{n-k}$ because that is where $V'$ goes to, so actually $\psi(p)\in\widehat U\times\{0\}=\widehat F(\widehat U)$, so $p\in\varphi^{-1}(\widehat U)$ by undoing $\widehat F$, so $p\in S$ by definition.

		\item Suppose that $S$ satisfies the local $k$-slice condition. Then we want to give a smooth structure to $S$ so that the inclusion makes $S$ into a smooth embedded submanifold. Well, give $S\subseteq M$ the subspace topology; then this makes $S$ a homeomorphism onto its image automatically, so notably $S$ is Hausdorff and second countable.

		It remains to give $S$ some smooth charts. Well, fix some $p\in S$, and satisfying the $k$-slice chart condition promises us a chart $(U,\varphi)$ around $p$ so that
		\[S\cap U=\{q\in U:\varphi_\ell(p)=c_\ell\text{ for }\ell>k\}\]
		for some given real numbers $c_{k+1},\ldots,c_n$. These last $(n-k)$ coordinates shouldn't matter, so we let $\pi\colon\RR^n\to\RR^k$ denote the projection onto the first $k$ coordinates. As such, we set $V\coloneqq U\cap S$ and $\widehat V\coloneqq(\pi\circ\varphi)(V)$, which is an open subset of
		\[\varphi(U)\cap\left\{x\in\RR^n:x_\ell=c_\ell\text{ for }\ell>k\right\}.\]
		The above is open in the subspace defined by the plane at the right, so it is open when projected down to $\pi$, which can be checked because $\pi$ is a quotient map.
		
		So we will let $(V,\pi\circ\varphi)$ become the relevant chart. For example, we can check that $\pi\circ\varphi$ is a homeomorphism: indeed, its inverse map is given by $\varphi^{-1}\circ j$, where $j(x_1,\ldots,x_k)\coloneqq(x_1,\ldots,x_k,c_{k+1},\ldots,c_n)$, and $\varphi^{-1}$ and $j$ are both smooth. This concludes the proof that $S$ is a topological $k$-manifold.

		We now check smooth compatibility of the given charts to show that we have actually given $S$ a smooth structure. Well, choose two charts $(V,\psi)$ and $(V',\psi')$ of $S$ which are constructed as above from charts $(U,\varphi)$ and $(U',\varphi')$ of $M$. Well, the transition map $\psi'\circ\psi^{-1}$ is given by
		\[\pi'\circ\varphi'\circ\varphi^{-1}\circ j,\]
		where $j$ and $\pi$ and $j'$ and $\pi'$ are given as above. This transition map is smooth because it is the composition of smooth maps.

		Lastly, we must check that the embedding $S\to M$ is smooth. Well, for any $p\in S$, choose a smooth chart $(V,\psi)$ arising from the smooth chart $(U,\varphi)$ on $M$, as constructed above. Then the inclusion $F\colon S\subseteq M$ sends $V\subseteq U$, and the composite $\varphi\circ F\circ\psi^{-1}$ is just the identity, so it is smooth.
		\qedhere
	\end{itemize}
\end{proof}
Here is a consequence of the above proof.
\begin{corollary}
	Fix a smooth $n$-manifold $M$, and let $S\subseteq M$ be a smooth embedded submanifold. Then for any $k$-slice chart $(U,\varphi)$ of $S$, one finds that $(U\cap S,(\varphi_1,\ldots,\varphi_k))$ where $\varphi=(\varphi_1,\ldots,\varphi_n)$ is a coordinate expansion.
\end{corollary}
\begin{proof}
	The second part of the proof of \Cref{thm:slice} establishes this.
\end{proof}

\subsection{Level Sets}
A common way to build embedded submanifolds is via level sets. Let's begin with a couple examples.
\begin{example}
	Consider the smooth function $f\colon\RR^2\to\RR$ given by $f(x,y)\coloneqq x^2+y^2$. For example, $f^{-1}(\{1\})=S^1$ and $f^{-1}(\{0\})=\{(0,0)\}$ and $f^{-1}(\{-1\})=\emp$.
\end{example}
\begin{example}
	Consider the smooth function $f\colon\RR^2\to\RR$ given by $f(x,y)\coloneqq x^2-y^2$. Then $f^{-1}(\{1\})$ is a hyperbola with two connected components, but $f^{-1}(\{0\})$ looks like two crossing lines.
\end{example}
\begin{remark}
	Given any closed set $A\subseteq M$, we remarked earlier that there is a smooth function $f\colon M\to\RR$ such that $f(\{0\})=A$. So it cannot be the case that level sets always give nice submanifolds.
\end{remark}
We do expect that we should get a submanifold ``generically.'' Here is one instance of this.
\begin{theorem} \label{thm:level-sub}
	Fix a smooth map $F\colon M\to N$ of constant rank $r$ between the $m$-manifold $M$ and $n$-manifold $N$. Then any $q\in\im F$ makes the level set $F^{-1}(\{q\})$ is a proper submanifold of $M$ of dimension $(m-r)$.
\end{theorem}
Morally, the dimensions of $M$ must go somewhere, and there are $r$ dimensions going out into $N$.
\begin{example}
	Consider the smooth function $f\colon\RR^2\to\RR$ given by $f(x,y)\coloneqq x^2-y^2$. Then
	\[df_{(x,y)}=\begin{bmatrix}
		2x & -2y
	\end{bmatrix},\]
	so the function $f|_{\RR^2\setminus(\RR\times\{0\}\cup\{0\}\times\RR)}$ is a smooth map of constant rank $1$, so \Cref{thm:level-sub} tells us that all of its fibers will be proper submanifolds of $M$ of dimension $2-1=1$.
\end{example}

\end{document}