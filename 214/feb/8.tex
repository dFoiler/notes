% !TEX root = ../notes.tex

\documentclass[../notes.tex]{subfiles}

\begin{document}

Here we go.

\subsection{Velocity Vectors}
Let's discuss a more geometric variant of tangent vectors.
\begin{definition}[velocity vector]
	Fix a smooth $n$-manifold $M$ and a point $p\in M$. Define the space $\mc J_pM$ to be the set of smooth curves $\gamma\colon(-\varepsilon,\varepsilon)\to M$ such that $\gamma(0)=p$ (and $\varepsilon>0$). We say that $\gamma_1,\gamma_2\in\mc J_p$ are equivalent, written $\gamma_1\sim\gamma_2$, if and only if $(f\circ\gamma_1)'(0)=(f\circ\gamma_2)'(0)$ for any $f\in C^\infty(M)$.
\end{definition}
\begin{remark}
	We won't bother checking that $\sim$ is an equivalence relation; it holds because we are basically checking equalities after passing to $\RR^{C^\infty(M)}$ by sending $\gamma\mapsto((f\circ\gamma)'(0))_f$.
\end{remark}
And here is how this relates to tangent vectors.
\begin{lemma}
	Fix a smooth $n$-manifold $M$ and a point $p\in M$. Then $T_pM$ is in natural bijection with $\mc J_pM/{\sim}$.
\end{lemma}
\begin{proof}
	In one direction, one can send some $[\gamma]\in(\mc J_pM/{\sim})$ to the derivation $v_{[\gamma]}\colon f\mapsto(f\circ\gamma)'(0)$. Note that this only depends on the class $[\gamma]$ rather than the representative $\gamma$ by definition of the equivalence relation $\sim$. This map is injective essentially by construction, and one can show by hand that it is surjective, for example by working locally on charts and then using lines as the needed curve to realize a differential in $T_pM$.
\end{proof}

\subsection{The Tangent Bundle}
Let's glue our tangent spaces together.
\begin{remark}
	Given $p,q\in\RR^n$, there is a natural identification $T_p\RR^n\to T_q\RR^n$. One can see this on velocity vectors by moving the curves over by hand. Alternatively, let $T\colon\RR^n\to\RR^n$ be the translation sending $T\colon p\mapsto q$, which is a diffeomorphism, and then we know we have an isomorphism $dT_p\colon T_p\RR^n\to T_q\RR^n$. (Recall functoriality of $T_p$ implies that diffeomorphisms produce isomorphisms.)
\end{remark}
In general, it is somewhat difficult to identify these tangent spaces naturally.
\begin{definition}[tangent bundle]
	Fix a smooth $n$-manifold $M$. Then \textit{tangent bundle} $TM$ is
	\[TM\coloneqq\bigsqcup_{p\in M}T_pM.\]
	Morally, $TM$ consists of all the tangent spaces glued together.
\end{definition}
\begin{proposition}
	Fix a smooth $n$-manifold $M$. Then $TM$ is a smooth $2n$-manifold.
\end{proposition}
\begin{proof}
	We will use \Cref{lem:build-manifold-from-set}. Quickly, note that we have a projection $\pi\colon TM\to M$ given by $\pi(p,v)\coloneqq p$.

	Now, for each smooth chart $(U,\varphi)$ on $M$, we define the chart $\left(\pi^{-1}U,\widetilde\varphi\right)$ on $TM$, where $\widetilde\varphi\colon\pi^{-1}U\to(\im\varphi)\times\RR^n$ is defined by
	\[\widetilde\varphi\colon\sum_{i=1}^nv_i\frac{\del}{\del x_i}\bigg|_p\mapsto(\varphi(p),(v_1,\ldots,v_n)).\]
	Recall $(\del/\del x_i)|_p=d\varphi^{-1}_{\varphi(p)}(\del/\del\widetilde x_i)$, where $(\widetilde x_1,\ldots,\widetilde x_n)$ are coordinates chosen on  is  We now have to check our various conditions. For example, $\widetilde\varphi$ is a bijection to an open subset of $\RR^n\times\RR^n=\RR^{n+1}$ by construction.
	\begin{listroman}
		\item Given two $(U,\varphi)$ and $(V,\psi)$, we need $\widetilde\varphi\left(\pi^{-1}U\cap\pi^{-1}V\right)$ to be open in $\RR^{2n}$. But this is $\widetilde\varphi\left(\pi^{-1}(U\cap V)\right)$, which is an open subset of $\RR^n\times\RR^n$ because $(U\cap V,\varphi|_{U\cap V})$ is a smooth chart on $M$, so the argument above applies.
		\item Given two $(U,\varphi)$ and $(V,\psi)$, we need the composite $\widetilde\varphi\circ\widetilde\psi^{-1}$ to be smooth, when suitably restricted. Well, one simply commutes the change-of-coordinates for the part on the tangent spaces, and on points, we simply use that $\varphi\circ\psi^{-1}$ is smooth already. Explicitly, one finds that this is
		\[(\widetilde x,v)\mapsto\left(\left(\varphi\circ\psi^{-1}\right)(\widetilde x),\sum_{i=1}^nv_i\frac{\del\ov y_\bullet}{\del\widetilde x_i}\frac{\del}{\del\widetilde y_\bullet}\right).\]
		\item A countable cover of $M$ by charts produces a countable cover of $TM$ by charts upon pulling back by $\pi$.
		\item Fix distinct $(p,v),(q,w)\in TM$. If $p\ne q$, then we can find disjoint smooth charts $(U,\varphi)$ and $(V,\psi)$ on $M$, so $\left(\pi^{-1}U,\widetilde\varphi\right)$ and $\left(\pi^{-1}V,\widetilde\psi\right)$ provided the needed disjoint charts. Otherwise, $p=q$, and then $p$ and $q$ are of course contained in the same chart $(U,\varphi)$, so $(p,v)$ and $(q,w)$ are contained in the same chart $\left(\pi^{-1}U,\widetilde\varphi\right)$.
		\qedhere
	\end{listroman}
\end{proof}
\begin{example}
	One has $T\RR^n=\RR^n\times\RR^n$.
\end{example}
\begin{example}
	One has $TS^1=S^1\times\RR$ and $TS^3=S^3\times\RR^3$ and even $TS^7=S^7\times\RR^7$.
\end{example}
\begin{example}
	For even $n$, one has $TS^n\ne S^n\times\RR^n$, which is essentially a consequence of the Hairy ball theorem: one would be able to produce $n$ linearly independent elements of $S^n\times\RR^n$ and then pull them back to $n$ linearly independent vector fields $TS^n$, which do not exist for even $n$. The same inequality holds for odd $n\notin\{1,3,7\}$.
\end{example}

\subsection{Maps of Constant Rank}
We are going to want some inverse function theorems. Here is the most basic case. Morally, the statement is that invertible derivative should mean locally invertible.
\begin{theorem}[Inverse function]
	Fix a smooth function $f\colon\RR^n\to\RR^n$. Given $x_0\in\RR^n$, if the map $(Tf)_{x_0}\colon T_{x_0}\RR^n\to T_{f(x_0}\RR^n$ is invertible, then there is an open neighborhood $U\subseteq\RR^n$ around $x_0$ such that $f|_U$ is a diffeomorphism.
\end{theorem}
By working on charts, the following result is basically immediate.
\begin{theorem}[Inverse function] \label{thm:inv-func}
	Fix a smooth function $f\colon M\to N$ of $n$-manifolds. Given $x_0\in\RR^n$, if the map $(Tf)_{x_0}\colon T_{x_0}M\to T_{f(x_0}N$ is invertible, then there is an open neighborhood $U\subseteq M$ around $x_0$ such that $f|_U$ is a diffeomorphism.
\end{theorem}
This condition is good enough to make into a definition.
\begin{definition}
	Fix a smooth function $F\colon M\to N$ of $n$-manifolds. Then $F$ is a \textit{local diffeomorphism} at $p$ if and only if $dF_p$ is invertible. Equivalently, by \Cref{thm:inv-func}, there is an open neighborhood $U$ of $p$ such that $F|_U$ is a diffeomorphism onto its image.
\end{definition}
\begin{remark}
	Of course, the converse direction (local diffeomorphism implies invertible derivative) is just by functoriality of the tangent space construction.
\end{remark}
\begin{remark}
	By gluing, if $F$ has invertible derivative at all points, and $F$ is a bijection, then one can see that $F^{-1}$ must be locally a diffeomorphism at all points, so in particular $F^{-1}$ is smooth, so $F$ is fully a diffeomorphism.
\end{remark}
\begin{example}
	The map $F\colon\RR\to S^1$ given by $x\mapsto(\cos x,\sin x)$ is not injective, but it is a local diffeomorphism.
\end{example}
More generally, one could require something weaker than full invertibility.
\begin{defihelper}[immersion, submersion, full rank, constant rank] \nirindex{immersion} \nirindex{submersion} \nirindex{full rank} \nirindex{constant rank}
	Fix a map $F\colon M\to N$ of smooth manifolds, where $m\coloneqq\dim M$ and $n\coloneqq\dim N$.
	\begin{itemize}
		\item $F$ is an \textit{immersion} if and only if $dF_p$ is injective for all $p\in M$.
		\item $F$ is a \textit{submersion} if and only if $dF_p$ is surjective for all $p\in M$.
		\item $F$ has \textit{full rank} if and only if $\op{rank}dF_p=\min\{m,n\}$ for all $p\in M$ (notably, this is as large as possible).
		\item $F$ has \textit{constant rank} if and only if $dF_p$ has the same rank for all $p\in M$ (notably, this is as large as possible).
	\end{itemize}
\end{defihelper}
We now state the following theorem.
\begin{theorem}
	Fix a map $F\colon M\to N$ of smooth manifolds. If $dF_p$ has full rank for some $p\in M$, then there is an open neighborhood $U$ of $p$ such that $F|_U$ has full rank.
\end{theorem}
\begin{proof}
	The condition that $dF_p$ having full rank is equivalent to the determinant of some largest submatrix being nonzero. So one has a map $M\to\RR^N$ for some large $N$ taking $p\in M$ to the list of determinants of these submatrices of $dF_p$, and this map is continuous, so the set of points not going to zero is open and contains $p$.
\end{proof}
\begin{example}
	Fix two manifolds $M$ and $N$, and fix some $y_0\in N$.
	\begin{itemize}
		\item The map $x\mapsto(x,y_0)$ is an immersion.
		\item The projection map $M\times N\to M$ is a submersion.
	\end{itemize}
\end{example}
\begin{example}
	Fix a smooth curve $\gamma\colon\RR\to\RR^2$ with non-vanishing derivative everywhere. Then $\gamma$ is an immersion.
\end{example}

\end{document}