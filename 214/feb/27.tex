% !TEX root = ../notes.tex

\documentclass[../notes.tex]{subfiles}

\begin{document}

Today we completed the proof of Sard's theorem. I have edited there for completeness.

\subsection{Applications of Sard's Theorem}
Here are some applications.
\begin{corollary} \label{cor:image-null-set}
	Fix a smooth map $F\colon M\to N$ where $\dim M<\dim N$. Then $F(M)$ has measure zero.
\end{corollary}
\begin{proof}
	Because $\dim M<\dim N$, it is required that every value of $F$ is critical: $dF_p\colon T_pM\to T_{F(p)}N$ can never be surjective! So we conclude by \Cref{thm:sard}. 
\end{proof}
For the next application, we need the following notion.
\begin{definition}[regular domain]
	A regular domain $D$ of a smooth manifold $M$ is a properly embedded codimension-$0$ smooth submanifold (possibly with boundary).
\end{definition}
\begin{corollary}
	Fix a closed subset $K$ of a smooth manifold $M$. Then there are descending regular domains $\{Q_i\}_{i\in\NN}$ such that
	\[M\supseteq Q_0\supseteq Q_1\supseteq\cdots\]
	and $K=\bigcap_{i\in\NN}Q_i$.
\end{corollary}
\begin{proof}
	To begin, we recall that we can find a nonnegative smooth function $f\in C^\infty(M)$ such that $f^{-1}(\{0\})=K$. Now, \Cref{thm:sard} allows us to find a regular sequence of values $\{s_i\}_{i\in\NN}$ such that $s_i\to0^+$ monotonically. Then $Q_i\coloneqq f^{-1}([0,s_i])$ will work. (We will not show that $f^{-1}([0,s_i])$ is a regular domain; this is essentially on the homework.)
\end{proof}

\subsection{The Whitney Embedding Theorem}
As another application, we will show that any smooth manifold can be embedded into some Euclidean space. To begin, we discuss how to decrease the dimensionality of the target space.
\begin{lemma}
	Fix a smooth $m$-manifold $M$ embedded in some $\RR^N$. For each $v\in\RR^N\setminus\RR^{N-1}$, let $\pi_v\colon\RR^N\to\RR^{N-1}$ denote the projection map with kernel $\RR v$. If $N>2m+1$, then there exists some $v$ for which $\pi_v|_M$ is an injective immersion $M\to\RR^{N-1}$.
\end{lemma}
\begin{proof}
	Injectivity of $\pi_v|_M$ is equivalent to asking for $p-q$ to never be parallel to $v$ for $p,q\in M$. Being a smooth immersion is equivalent to asking for $T_pM\cap\ker d(\pi_v)_p=0$; note $(\pi_v)_p=\pi_v$ up to the identification $T_p\RR^N=\RR^N$, so we are asking for $T_pM$ to not have any nonzero vectors parallel to $v$.

	We now build a smooth map to check these two facts. Set $\Delta_M\subseteq M\times M$ to be the diagonal subset $\{(p,p):p\in M\}$; this allows us to define $\kappa\colon(M\times M)\setminus\Delta_M\to\RP^{N-1}$ by $\kappa(x,y)\coloneqq[x-y]$. Analogously, we define $M_0\coloneqq\{(p,0)\in TM:p\in M\}$ by $\tau\colon TM\setminus M_0\to\RP^{N-1}$ by $\tau(p,w)\coloneqq[w]$. We are now choosing $v\in\RP^{N-1}$ to avoid the images of $\kappa$ and $\tau$, which are both null sets by \Cref{cor:image-null-set}, so we conclude.
\end{proof}
Next up, we show that we can embed compact manifolds.
\begin{lemma}
	Fix a smooth compact $m$-manifold $M$. Then $M$ can be embedded in $\RR^N$ for some $N>0$.
\end{lemma}
\begin{proof}
	Choose a finite smooth atlas $\{(U_i,\varphi_i)\}_{i=1}^d$. By adding in some more charts (and then using compactness to reduce), we may assume that $\im\varphi_i=B(0,1)\subseteq\RR^m$ by some shifting and that the open subsets $\varphi_i^{-1}(B(0,1/2))$ actually fully cover $M$. By smoothly extending, we are able to find some $\eta\colon B(0,1)\to[0,1]$ which is $0$ on $\del B(0,1)$ but $1$ on $B(0,1/2)$. We now define
	\[F\coloneqq((\eta\circ\varphi_1)\varphi_1,\ldots,(\eta\circ\varphi_m)\varphi_m).\]
	A quick counting argument tells us that the target is $\RR^{m(n+1)}$. Now one checks that $F$ is injective and an immersion and hence a smooth embedding (by compactness of $M$).
\end{proof}
\begin{remark}
	Please read the rest of the proof of the Whitney embedding theorem, which extends the above result to the general case.
\end{remark}
Here is the total result, whose proof we will not complete.
\begin{theorem}[Whitney embedding] \label{thm:whitney-embed}
	Fix a smooth $n$-manifold $M$. Then there is an embedding $M\to\RR^{2n+1}$.
\end{theorem}

\end{document}