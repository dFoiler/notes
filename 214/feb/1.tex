% !TEX root = ../notes.tex

\documentclass[../notes.tex]{subfiles}

\begin{document}

The second homework is due later today. We began class by completing a proof, so I edited directly into those notes.

\subsection{Partition of Unity for Manifolds}
We will show that partitions of unity exist for manifolds.
\begin{theorem} \label{thm:partition-of-unity}
	Fix a smooth manifold $M$. For any open cover $\{U_\alpha\}_{\alpha\in\kappa}$, there is a partition of unity $\{\varphi_\alpha\}_{\alpha\in\kappa}$ (of smooth functions) subordinate to $\{U_\alpha\}_{\alpha\in\kappa}$.
\end{theorem}
\begin{proof}
	We begin by constructing smooth functions $\{\widetilde\varphi\}_{\alpha\in\kappa}$ satisfying the following constraints.
	\begin{itemize}
		\item $\im\widetilde\varphi_\alpha\subseteq[0,\infty)$.
		\item $\op{supp}\widetilde\varphi_\alpha\subseteq U_\alpha$.
		\item The collection $\{\op{supp}\widetilde\varphi_\alpha\}_{\alpha\in\kappa}$ is a locally finite open cover of $M$.
	\end{itemize}
	Dividing out by the summation of the $\widetilde\varphi_\bullet$s completes the proof. Notably, for each $x\in M$, the sum
	\[\widetilde\varphi(x)\coloneqq\sum_{\alpha\in\kappa}\widetilde\varphi_\alpha(x)\]
	is finite ($x$ can only belong to finitely many of the supports); in fact, there is an open neighborhood $U$ of $x$ such that $U$ only intersects finitely many of the supports, so
	\[\widetilde\varphi|_U=\sum_{\substack{\alpha\in\kappa\\\op{supp}\widetilde\varphi_\alpha\cap U\ne\emp}}\widetilde\varphi_\alpha|_U\]
	is just a finite sum of smooth functions, so $\widetilde\varphi$ is smooth on $U$. Thus, by gluing, $\widetilde\varphi$ is smooth on $M$, and we note that it is nonzero because each $x\in M$ is in some support, so we can define $\varphi_\alpha\coloneqq\widetilde\varphi_\alpha/\varphi$ to satisfy all the needed conditions, most notable being that these functions are smooth, have support contained in $U_\alpha$, and $\sum_{\alpha\in\kappa}\varphi_\alpha=1$.

	It remains to construct the $\widetilde\varphi_\alpha$s. We proceed in steps.
	\begin{enumerate}
		\item We construct a nice open cover. For each $x\in M$, we can find some open neighborhood $U$ such that we have a homeomorphism $\varphi\colon U\to B(\varphi(x),2)$. Then $\left\{\varphi^{-1}(B(\varphi(x),1))\right\}_{x\in M}$ is an open cover of $M$, so we can refine this to a locally finite open cover $\mc U$ of precompact open sets. By looking down on compact, we may as well assume that $\mc U$ is made of coordinate balls $B(\varphi(x),r)$ contained in larger coordinate balls $B(\varphi(x),r')$ for $r'>r$.
		\item Now, for each coordinate ball $\varphi\colon U\cong B(0,r)$ for $U\in\mc U$ extending to $\varphi'\colon U'\cong B(0,r')$. Then we construct $f_U$ which is nonzero on $B(0,r)$ but vanishes on $B(0,r')$.

		Now, for each $U\in\mc U$, select $\alpha_U\in\kappa$ such that $\ov U\subseteq U_{\alpha_U}$. From here, we may set
		\[\widetilde\varphi_\alpha\coloneqq\sum_{\alpha_U=\alpha}f_U,\]
		which satisfies all the needed conditions. For example, one finds that the support of $\widetilde\varphi_\alpha$ is
		\[\ov{\bigcup_{U\subseteq U_\alpha}U}\subseteq\bigcup_{U\subseteq U_\alpha}\ov{U}\subseteq U.\]
		One needs local finiteness in order to verify the first inclusion; the point is that one can reduce this large union to a finite one around any given point, so the closures must agree.
		\qedhere
	\end{enumerate}
\end{proof}
Let's give some applications.
\begin{corollary} \label{cor:bump}
	Fix a smooth manifold $M$. For any closed set $A\subseteq M$ contained in an open set $U\subseteq M$, there exists a smooth function $\psi\colon M\to\RR$ such that $\psi|_A=1$ and $\psi|_{M\setminus U}=0$.
\end{corollary}
\begin{proof}
	Consider the open cover $\{U,M\setminus A\}$; this is an open cover because $U\cup(M\setminus A)=M$ is equivalent to $A\subseteq U$. Then \Cref{thm:partition-of-unity} produces two nonnegative smooth functions $\psi_0$ and $\psi_1$ such that $\op{supp}\psi_0\subseteq U$ and $\op{supp}\psi_1\subseteq M\setminus A$ and $\psi_0+\psi_1=1$ everywhere. But now $\psi_0$ is the desired function: $\op{supp}\psi_0\subseteq U$ implies $\psi_0|_{M\setminus U}$, and $\psi_0|_A+\psi_1|_A=1$, but $\psi_1|_A=0$ because $\op{supp}\psi_1\subseteq M\setminus A$.
\end{proof}
\begin{corollary}[Extension lemma] \label{cor:extension-lemma}
	Fix a smooth manifold $M$. Further, fix a closed subset $A\subseteq M$ contained in an open set $U\subseteq M$. Given a smooth function $f\colon A\to\RR^k$, there is a smooth function $\widetilde f\colon M\to\RR^k$ extending $f$ and with $\op{supp}\widetilde f\subseteq U$.
\end{corollary}
\begin{proof}
	Omitted.
\end{proof}
\begin{corollary}
	Fix a smooth manifold $M$. There is a nonnegative function $f\colon M\to\RR$ such that all the sets
	\[f^{-1}([0,c])\]
	are compact for any $c\ge0$.
\end{corollary}
\begin{proof}
	Fix a countable cover $\{U_n\}_{n\in\NN}$ of $M$ by precompact open subsets, and let $\{\psi_n\}_{n\in\NN}$ be the corresponding partition of unity. Then we set
	\[f\coloneqq\sum_{n=0}^\infty n\psi_n.\]
	Notably, for each $c\in\RR$, we see
	\[f^{-1}([0,c])\subseteq\bigcup_{n\le c}\op{supp}\psi_n,\]
	so $f^{-1}([0,c])$ is a closed subset of a finite union of compact sets (which is compact), so we are done.
\end{proof}
\begin{corollary}
	Fix a closed subset $K$ of a smooth manifold $M$. Then there is a nonnegative smooth function $f\colon M\to\RR$ such that $f^{-1}(\{0\})=K$.
\end{corollary}
\begin{proof}
	One begins with $M=\RR^n$ and then does the general case from there.
\end{proof}

\subsection{Diffeomorphisms}
Here is our definition.
\begin{definition}[diffeomorphism]
	Fix a map $F\colon M\to N$ of smooth manifolds, possibly with boundary. Then $F$ is a \textit{diffeomorphism} if and only if $F$ is bijective, smooth, and has smooth inverse.
\end{definition}
\begin{remark}
	Invariance of the boundary under smooth charts implies $F$ must send boundary points to boundary points.
\end{remark}
\begin{remark}
	If $F$ is a diffeomorphism, then $\dim M=\dim N$. Simply put, we can work locally on a chart, and then we are providing a diffeomorphism $\RR^m\to\RR^n$, but this can only happen when $m=n$. For example, it means that $DF$ and $DF^{-1}$ are invertible linear maps $\RR^m\to\RR^n$ and $\RR^n\to\RR^m$, respectively, which manifestly requires $m=n$.
\end{remark}
\begin{remark}
	It turns out that topological $n$-manifolds with a smooth structure admit a unique smooth structure up to diffeomorphism, for $n\ge3$. For $n\ge4$, even $\RR^4$ fails to have a unique smooth structure.
\end{remark}
\begin{remark}
	The collection $\op{Diff}(M)$ of diffeomorphisms $M\to M$ is a group, and one can give it a topology. For example, one can compute that $\op{Diff}\left(S^2\right)$ is homotopy equivalent to $O(3)$, given approximately by rotations.
\end{remark}

\subsection{Tangent Spaces}
Fix a smooth $n$-manifold $M$. One would like to provide each point $p\in M$ with an $n$-dimensional tangent vector space $T_pM$. If $M$ is embedded into Euclidean space reasonably, we can imagine using the embedding to realize the tangent space; for example, if $M$ is a (smooth) curve in $\RR^2$, we can imagine that the tangent vectors tell us what direction we are moving in. We would also like to actually be able to compute these things in charts.

Anyway, here is our definition of tangent vectors. This definition is a bit awkward to handle because we want to be invariant.
\begin{definition}[tangent space]
	Fix a smooth $n$-manifold $M$ and some point $p\in M$. A \textit{derivation at $p$} is an $\mathbb R$-linear map $v\colon C^\infty(M)\to\RR$ satisfying the Leibniz rule
	\[v(fg)=f(p)v(g)+g(p)v(f)\]
	for any $f,g\in C^\infty(M)$. Then the \textit{tangent space $T_p(M)$ at $p$} is the collection of derivations.
\end{definition}
\begin{remark}
	Note that $T_p(M)$ is an $\mathbb R$-subspace of the collection of linear maps $C^\infty(M)\to\RR$.
\end{remark}
\begin{example} \label{ex:directional-deriv}
	Fix $M\coloneqq\RR^n$ and some $p\in M$. Then any $v\in\RR^n$ has a ``directional derivative'' given by
	\[f\mapsto\sum_{i=1}^nv_i\frac{\del f}{\del x_i}\bigg|_p.\]
	This is simply by the product rule in multivariable calculus.
\end{example}

\end{document}