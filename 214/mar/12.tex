% !TEX root = ../notes.tex

\documentclass[../notes.tex]{subfiles}

\begin{document}

We continue discussing Lie groups. Today will be a little light on proofs.

\subsection{Lie Subgroups}
Here is our definition.
\begin{definition}[Lie subgroup]
	Fix a Lie group $G$. Then a \textit{Lie subgroup} is a subset $H\subseteq G$ which is the image of the injective Lie group homomorphism.
\end{definition}
\begin{example}
	If $H\subseteq G$ is an embedded submanifold and a subgroup of $G$, then the embedding $H\subseteq G$ provides the injective Lie group homomorphism making $H$ a Lie subgroup. For example, all the matrix groups in \Cref{ex:matrix-groups} are Lie subgroups of $\op{GL}$ (of suitable dimension).
\end{example}
\begin{remark}
	An injective Lie group homomorphism is an immersion by \Cref{rem:rank-theorem-for-lie}, so $H$ is an immersed submanifold.
\end{remark}
\begin{example}
	Consider the Lie group $T\coloneqq S^1\times S^1$. Then for $\alpha\in\RR$, there is a smooth map $F_\alpha\colon\RR\to T$ given by
	\[F_\alpha(t)\coloneqq\left(e^{2\pi it},e^{2\pi i\alpha t}\right).\]
	There are two cases.
	\begin{itemize}
		\item If $\alpha\in\QQ$, then $F$ fails to be injective; one can precisely compute the period $k$ as being the least positive integer so that $e^{2\pi ik}=e^{2\pi i\alpha k}=1$, which we can see is the denominator of $\alpha$. So one can define $\widetilde F_\alpha$ by restricting to $S^1$ as
		\[\widetilde F_\alpha(t)\coloneqq\left((e^{2\pi ikt},e^{2\pi i\alpha kt}\right),\]
		and now we see that $\im F_\alpha=\im\widetilde F_\alpha$ is a Lie subgroup.
		\item If $\alpha\notin\QQ$, then $F$ is injective, so $\im F$ is a Lie subgroup. Notably, it is dense in $T$, though we will not show it.
	\end{itemize}
\end{example}
Here's a quick check.
\begin{lemma} \label{lem:open-lie-subgroup}
	Suppose $H$ is an open Lie subgroup of $G$. Then $H$ is the union of connected components of $G$.
\end{lemma}
\begin{proof}
	Note that $H\subseteq G$ is a subgroup (it is the image of a group under a homomorphism), so we may partition
	\[G=\bigsqcup_{g\in G}gH\]
	into cosets. Each $gH$ is open because $L_g$ is a homeomorphism by \Cref{rem:translate-diffeo}, so the complement of $H$ is the union of open subsets of $G$, so $H$ is also closed. So $H$ is open and closed, and the result follows.
\end{proof}
\begin{proposition}
	Fix a connected Lie group $G$. Given an open neighborhood $U\subseteq G$ of $e$, the group $G$ is generated by $U$.
\end{proposition}
\begin{proof}
	Let $H$ be the subgroup generated by $U$. For example, $U\subseteq H$. Now, for any $g\in H$, we see that $L_g(U)$ is open by \Cref{rem:translate-diffeo} and lives inside $H$, so $H$ is open. Thus, \Cref{lem:open-lie-subgroup} tells us that $H$ is the union of connected components of $G$, so $H=G$ follows because $G$ is connected.
\end{proof}
This motivates us to work with the identity component of $e$ for disconnected groups.
\begin{definition}[identity component]
	Fix a Lie group $G$. Then the \textit{identity component} $G_\circ$ is the connected component of $G$ containing $e\in G$.
\end{definition}
\begin{proposition}
	Fix a Lie group $G$. Then $G_\circ$ is a properly embedded Lie subgroup.
\end{proposition}
\begin{proof}
	In fact, we claim that the open submanifold $G_\circ\subseteq G$ is itself a Lie group under the restricted multiplication and inversion. Namely, we must show that $m(G_0\times G_0)\subseteq G_0$ and $i(G_0)\subseteq G_0$. Well, $m$ and $i$ are continuous maps, so because $G_0\times G_0$ and $G_0$ are connected, their images are still connected. To finish, we note that $e=m(e,e)$ and $e=i(e)$ tells us that their images must land in the connected component of $e$, so $m(G_0\times G_0)\subseteq G_0$ and $i(G_0)\subseteq G_0$.
\end{proof}
\begin{example}
	Note that $\det\colon\op{GL}_n(\RR)\to\RR^\times$ is surjective, but the target $\RR^\times$ is disconnected (it's $\RR_{>0}\sqcup\RR_{<0}$), so $\op{GL}_n(\RR)$ must fail to be connected. But the pre-image of $\RR_{>0}$ is $\op{GL}^+_n(\RR)$, consisting of the invertible matrices with positive determinant, and $\op{GL}^+_n(\RR)$ turns out to be connected, so $\op{GL}^+_n(\RR)$. We will not show that it is connected here.
\end{example}
\begin{example}
	Similarly, one can check that $\op{SO}_n(\RR)$ is the connected component of the identity in $\op O_n(\RR)$.
\end{example}
We close with the following result.
\begin{proposition}
	Fix a Lie subgroup $H\subseteq G$ which is actually an embedded submanifold. Then $H\subseteq G$ is closed.
\end{proposition}
\begin{proof}[Sketch]
	As a sketch, one takes a sequence $\{h_i\}_{i\in\ZZ^+}$ in $H$ approaching $g\in G$, and we need to check that $g\in H$. One works in a slice chart of $g$ to conclude.
\end{proof}
\begin{remark}
	It turns out that if $H\subseteq G$ is a closed subgroup, then it turns out that $H$ is an embedded Lie subgroup, but we will not show this here.
\end{remark}

\subsection{Group Actions}
Groups will be known by their actions. Lie group actions should account for manifold structure, as the following definition establishes.
\begin{definition}[smooth action]
	Fix a Lie group $G$ and a manifold $M$. Then a \textit{smooth left action} $G$ on $M$ is a smooth map $\cdot\colon G\times M\to M$ satisfying the following.
	\begin{itemize}
		\item Associativity: $(g_1g_2)\cdot p=g_1\cdot(g_2\cdot p)$.
		\item Identity: $e\cdot p=p$.
	\end{itemize}
	A right Lie group action is defined analogously on the right via $\cdot\colon M\times G\to M$.
\end{definition}
\begin{example}
	If $M$ is a countable set, then we recover usual group actions of $G$ on sets.
\end{example}
\begin{example}
	Suppose $G$ and $H$ are Lie groups, and $H$ as a right action on $G$. Then we get a right action of $G$ on $H$ via
	\[p\cdot g\coloneqq g^{-1}\cdot p.\]
	(The right-hand side is the right action of $p$ on $g^{-1}$.)
\end{example}
\begin{example}
	Here are some actions of $\op{GL}_n(\RR)$ on $\RR^n$.
	\begin{itemize}
		\item Note $\op{GL}_n(\RR)$ has a smooth left action on $\RR^n$ by matrix-vector multiplication.
		\item Alternatively, one could define $A\cdot v\coloneqq\left(A^{-1}\right)^\intercal v$ to be a right action.
		\item There is also a smooth right action by $v\cdot A\coloneqq A^\intercal v$; notably, $(AB)^\intercal=B^\intercal A^\intercal$.
	\end{itemize}
\end{example}
\begin{example}
	Fix a Lie group $G$. Then here are some ways that the group $G$ could act on itself; they are all composites of multiplication and inversion, so they are smooth.
	\begin{itemize}
		\item $G$ has a smooth right and left action on $G$ by translation.
		\item $G$ has a smooth left action on $G$ by $g\cdot h\coloneqq hg^{-1}$.
		\item $G$ has a smooth left action on $G$ by $g\cdot h\coloneqq ghg^{-1}$.
	\end{itemize}
\end{example}
\begin{example}
	Fix a smooth manifold $M$. Then $\pi_1(M)$ has a smooth action on the universal cover $\widetilde M$ of $M$ by deck transformations. (Note $\pi_1(M)$ is a countable set, which we give the discrete topology, and it becomes a smooth $0$-manifold.)
\end{example}
Group actions take on the usual definitions.
\begin{defihelper}[orbit, isotropy] \nirindex{orbit} \nirindex{isotropy}
	Fix a Lie group $G$ with smooth action on the smooth manifold $M$.
	\begin{itemize}
		\item The \textit{orbit} of $p\in M$ is the set $G\cdot p\coloneqq\{gp:g\in G\}$. We let $G\backslash M$ denote the set of orbits.
		\item The \textit{isotropy subgroup} of $p\in M$ is the subgroup
		\[G_p\coloneqq\{g\in G:gp=p\}.\]
	\end{itemize}
\end{defihelper}
\begin{remark}
	The orbits $G\backslash M$ of $M$ partition $M$, by the usual abstract algebra argument.
\end{remark}
\begin{defihelper}[transitive, free] \nirindex{transitive} \nirindex{free}
	Fix a Lie group $G$ with smooth action on the smooth manifold $M$. The action is \textit{transitive} if and only if $G\cdot p=M$ for any $p\in M$. The action is \textit{free} if and only if $G_p=\{e\}$ for all $p\in M$.
\end{defihelper}
\begin{example}
	Consider the action of $\op{SO}_2(\RR)$ on $\RR^2$ by matrix-vector multiplication. Here, $\op{SO}_2(\RR)$ is the set of rotations of $\RR^2$. Thus, this action is not transitive (a point in $\RR^2$ only gets slid along a circle) and is not free (the isotropy subgroup of $0$ is all $\op{SO}_2(\RR)$).
\end{example}
\begin{example}
	Consider the action of $\op{GL}_n(\RR)$ on $\RR^n$ by matrix-vector multiplication. There are two orbits, given by $\{0\}$ and $\RR^n\setminus\{0\}$, so the action again is neither free nor transitive.
\end{example}
\begin{example}
	Let a Lie subgroup $H$ of $G$ act on the Lie group $G$ by left multiplication. Then the orbits are the right cosets $\{Hg:g\in G\}$.
\end{example}
\begin{example}
	Consider the action of the group $\op{GL}_n(\CC)$ on itself by conjugation. Then the orbits are classified by Jordan normal forms by some linear algebra over algebraically closed fields.
\end{example}
\begin{example}
	Consider the action of $\op{SO}_n(\RR)$ on $\op{GL}_n(\RR)$ by left multiplication. Then the orbits are given by the cosets, which one can show are in bijection with the group of upper triangular matrices $\op U_n(\RR)\subseteq\op{GL}_n(\RR)$. Indeed, for $A\in\op{GL}_n(\RR)$, one has a unique QR decomposition
	\[A=QR\]
	where $Q\in\op{SO}_n(\RR)$ and $R\in\op U_n(\RR)$.
\end{example}
\begin{example}
	Algebraic topology informs us that the orbits of the action of $\pi_1(M)$ on the universal cover $\widetilde M$ (by deck transformations) are given by points in $M$.
\end{example}
% \subsection{Equivariant Maps}
With group actions on a particular set, we want to understand maps between them.
\begin{definition}[equivariant]
	Fix a Lie group $G$ with smooth action on the smooth manifolds $M$ and $N$. Then a smooth map $F\colon M\to N$ is \textit{$G$-equivariant} if and only if
	\[F(g\cdot m)=g\cdot F(m)\]
	for any $g\in G$ and $m\in M$.
\end{definition}
\begin{example}
	Let $V$ be a vector space. Then the Lie group $\op{GL}(V)$ acts on $V$ by multiplication. One can define an action of $\op{GL}(V)$ on $V\otimes V$ by $g\cdot(v_1\otimes v_2)\coloneqq(gv_1\otimes gv_2)$. Then the diagonal embedding $F\colon V\to V\otimes V$ given by $v\mapsto v\otimes v$ is $G$-equivariant by construction.
\end{example}
\begin{remark}
	Please read some additional properties of equivariant maps.
\end{remark}
% \subsection{Representations}
Studying Lie groups gets interesting when one studies their representations, which are a special kind of group action. We won't say much here, but we can define them.
\begin{definition}
	Fix a Lie group $G$. Then a \textit{representation} of $G$ is a Lie group homomorphism $\rho\colon G\to\op{GL}(V)$ for some finite-dimensional vector space $V$ (over $\RR$ or $\CC$).
\end{definition}
\begin{remark}
	One can expand out what it means to be a Lie group homomorphism so that a representation simply means that $G$ has a smooth action on $V$ where each $g$ acts by a linear transformation on $V$.
\end{remark}
\begin{example}
	The identity map $\op{GL}_n(\RR)\to\op{GL}_n(\RR)$ is a representation, corresponding to matrix-vector multiplication.
\end{example}
\begin{example}
	The map $\op{GL}_n(\RR)\to\op{GL}_n(\RR)$ by $A\mapsto\left(A^{-1}\right)^\intercal$.
\end{example}
\begin{example}
	Matrix multiplication defines a smooth linear action of $\op{GL}_n(\RR)$ on $\RR^{n\times n}$, so we get a representation $\op{GL}_n(\RR)\to\op{GL}(\RR^{n\times n})$.
\end{example}
\begin{remark}
	It turns out that any compact Lie group $G$ has a faithful (i.e., injective) representation into a finite-dimensional vector space. Roughly speaking, one has $G$ act on $C^\infty(G)$ by $(g\cdot f)(x)\coloneqq f(x\cdot g)$ and then finds a way to reduce the dimension.
\end{remark}

\subsection{The Groups \texorpdfstring{$\op{SO}_3$}{SO3} and \texorpdfstring{$\op{SU}(2)$}{SU2}}
We spend some time showing how $\op{SO}_3(\RR)$ and $\op{SU}_2$ relate. Here, $\op{SU}(2)$ consists of the $2\times2$ matrices such that $A^\dagger A=1_2$ and $\det A=1$. This group has an action on the space $V$ of Hermitian matrices $H\in\CC^{2\times2}$ (namely, satisfying $H^\dagger=H$) with $\tr H=0$ by
\[U\cdot H\coloneqq UHU^\dagger.\]
Namely, one can check that $UHU^\dagger$ remains Hermitian and trace $0$ (for example, $\tr UHU^\dagger=\tr HU^\dagger U=\tr H$). Now, one can compute that $V$ has $\RR$-basis given by
\[\sigma_1\coloneqq\begin{bmatrix}
	0 & 1 \\
	1 & 0
\end{bmatrix},\qquad\sigma_2\coloneqq\begin{bmatrix}
	0 & -i \\
	i & 0
\end{bmatrix}\qquad\text{and}\qquad\sigma_3\coloneqq\begin{bmatrix}
	1 & 0 \\
	0 & -1
\end{bmatrix}.\]
The point is that $\op{SU}_2$ now gets a map to $\op{GL}_3(\RR)$. It will turn out that this lands in $\op{SO}_3(\RR)$.

\end{document}