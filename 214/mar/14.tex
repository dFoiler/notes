% !TEX root = ../notes.tex

\documentclass[../notes.tex]{subfiles}

\begin{document}

Midterm scores have been released. I did okay.

\subsection{Vector Fields}
We would like to attach a vector field of directions to a manifold $M$. Intuitively, this is smoothly attaching a vector to each point in $M$. Here is our definition.
\begin{definition}[vector field]
	Fix a smooth manifold $M$, and let $\pi\colon TM\to M$ be the canonical projection. Then a \textit{vector field} is a smooth section $X\colon M\to TM$ of $\pi$. In particular, $X$ is smooth, and $\pi(X(p))=p$ for each $p\in M$; i.e., $X(p)\in T_pM$ for each $p\in M$. A \textit{local vector field} is a vector field on an open subset $U\subseteq M$. We let $\mf X(M)$ denote the set of smooth vector fields on $M$.
\end{definition}
\begin{example}
	The map $X\colon M\to TM$ given by $X(p)\coloneqq(p,0)$ is a vector field.
\end{example}
\begin{example}
	Fix some index $i$. Given a smooth chart $(U,\varphi)$ on $M$ where $\varphi\colon U\to\RR^m$, we note that $X_i\colon U\to TU$ defined by
	\[X_i(p)\coloneqq\frac{\del}{\del x_i}\bigg|_p\]
	is a local vector field. Recall $\frac\del{\del x_i}\big|_p$ denotes $d\varphi_p^{-1}\left(\frac\del{\del x_i}\big|_{\varphi(p)}\right)$.
\end{example}
\begin{remark}
	The set $\mf X(M)$ is in fact a vector space, where we define $a_1X_1+a_2X_2$ by
	\[(a_1X_1+a_2X_2)(p)\coloneqq a_1X_1(p)+a_2X_2(p).\]
	Here, the linear combination is legal because it takes place in $T_pM$. More generally, given a smooth function $f\colon M\to\RR$, we see that $fX\colon M\to TM$ defined by $(fX)(p)\coloneqq f(p)X(p)$ is a smooth section of the projection $TM\to M$ and hence a vector field; one can check smoothness on a smooth chart. Thus, $\mf X(M)$ is a $C^\infty(M)$-module.
\end{remark}
\begin{remark}
	Suppose $X$ is a local vector field on the smooth chart $(U,\varphi)$ of $M$. Because any $T_pU$ has basis given by the $\frac{\del}{\del x_i}\big|_p$, so we can write
	\[X(p)=\sum_{i=1}^nf_i(p)\frac{\del}{\del x_i}\bigg|_p.\]
	Because projecting onto coordinate is smooth, we see that the $f_i$ are smooth if $X$ is. Conversely, if the $f_i$ are all smooth, then their linear combination to $X$ continues to be smooth. Because a function is smooth if and only if it is smooth on a cover of smooth charts, we see that we can check the smoothness of the vector field $X$ on a cover of smooth charts.
\end{remark}

\subsection{Frames}
It will be useful to have a notion of ``basis'' for $\mf X(M)$.
\begin{definition}[frame]
	Fix an open subset $U$ of a smooth manifold $M$.
	\begin{itemize}
		\item Local vector fields $X_1,\ldots,X_k$ on $U$ are \textit{linearly independent} if and only if $\{X_1(p),\ldots,X_k(p)\}$ is linearly independent for all $p\in U$.
		\item Local vector fields $X_1,\ldots,X_k$ on $U$ form a \textit{local frame} if and only if $\{X_1(p),\ldots,X_k(p)\}$ is a basis of $T_pM$ for all $p\in U$.
		\item A local frame is a \textit{global frame} if all the local vector fields are actually global vector fields.
	\end{itemize}
\end{definition}
The point is that a frame is locally a basis (of sorts) for $\mf X(U)$, though one cannot in general expect there to be a global frame at all. (Granted, one cannot in general expect there to be a global vector field at all.)
\begin{example}
	Let $(U,\varphi)$ be a smooth chart on the smooth $m$-manifold $M$. Then define the local vector field $X_i$ on $U$ by $X_i(p)\coloneqq\frac{\del}{\del x_i}\big|_p$. Then $\{X_1,\ldots,X_m\}$ is a local frame on $U$.
\end{example}
\begin{remark}
	Fix an open subset $U$ on the smooth $m$-manifold $M$. Given two local frames $\{X_i\}_{i=1}^m$ and $\{Y_i\}_{i=1}^m$ on $U$, we note that having a basis means that there are smooth functions $a_{ij}$ such that
	\[Y_j(p)=\sum_{i=1}^na_{ij}(p)X_i(p)\]
	for all $p\in U$.
\end{remark}
Here is a quick result on extending frames.
\begin{proposition}
	Fix an open subset $U$ on the smooth $m$-manifold $M$. Given $p\in U$ and linearly independent local vector fields $\{X_1,\ldots,X_k\}\subseteq\mc X(U)$ such that $\{X_1(p),\ldots,X_k(p),v_{k+1},\ldots,v_m\}$ is a full basis of $T_pM$, one can find an open neighborhood $V\subseteq U$ of $p$ and local vector fields $X_{k+1},\ldots,X_m$ of $V$ such that
	\[\{X_1,\ldots,X_m\}\]
	is a local frame over $V$ and $X_i(p)=v_i$ for $i>k$.
\end{proposition}
\begin{proof}
	Using coordinates and adjusting $\varphi$ suitably, we may assume that $X_i(p)=\frac\del{\del x_i}\big|_p$ for $i\le k$. Then define $X_i(q)\coloneqq\frac{\del}{\del x_i}\big|_p$ for $i>k$. Now, a set of frames being linearly independent is an open condition (we are asking for some determinant to fail to vanish), so there is an open neighborhood $V$ of $U$ in which the set $\{X_1,\ldots,X_m\}$ is linearly independent and hence a local frame.
\end{proof}
The existence of frames is nice enough for us to provide an adjective.
\begin{definition}
	A smooth manifold $M$ is \textit{parallelizable} if and only if $M$ has a global frame.
\end{definition}
\begin{remark}
	Fix a Lie group $G$. Then $G$ is parallelizable. Indeed, fix a basis $\{v_1,\ldots,v_m\}$ of $T_eG$, and then we can define
	\[X_i(g)\coloneqq(dL_g)_e(v_i).\]
	One can check that $X_i$ is in fact smooth because the $L_g$ are diffeomorphisms.
\end{remark}
\begin{example}
	The manifolds $\RR^n$, $S^1$, $\left(S^1\right)^n$, and $S^3\cong\op{SU}_2$ are all 
\end{example}

\subsection{Pushforward and Pullback}
There is some danger in pushforward because a smooth map $F\colon M\to N$ may fail to be injective, so we might be asking for the vector field $F_*X$ to take multiple directions in $N$. The correct definition is as follows.
\begin{definition}
	Fix a smooth map $F\colon M\to N$ of smooth manifolds. Then two vector fields $X\in\mf X(M)$ and $Y\in\mf Y(N)$ are \textit{$F$-related} if and only if $dF_p(X(p))=Y(p)$ for all $p\in M$.
\end{definition}
Here is our result for existence.
\begin{proposition}
	Fix a diffeomorphism $F\colon M\to N$ of smooth manifolds.
	\begin{listalph}
		\item For any $X\in\mf X(M)$, there is a unique $F$-related vector field $F_*X\in\mf X(N)$ such that
		\[(F_*X)(q)\coloneqq dF_{F^{-1}q}X(F^{-1}(q)).\]
		\item For any $Y\in\mf X(N)$, there is a unique $F$-related vector field $F^*X\in\mf X(M)$ such that
		\[(F^*X)(p)\coloneqq(dF_p)^{-1}Y(F(p)).\]
	\end{listalph}
\end{proposition}
\begin{proof}
	We have defined each of the vector fields on points, and one can see these definitions make them uniquely defined. It remains to show smoothness, which we omit.
\end{proof}
More generally, a smooth map permits us to understand vector fields between manifolds.
\begin{definition}[vector field]
	Fix a smooth map $F\colon M\to N$ of smooth manifolds, and let $\pi_N\colon TN\to N$ be the projection. Then a \textit{vector field of $N$ along $F$} is a map $X\colon M\to TN$ such that $\pi_N\circ X=F$; i.e., $X(p)\in T_{F(p)}N$ for each $p\in N$. We let $\mf X^F(N)$ denote this set of vector fields.
\end{definition}
For example, a vector field of $\RR^2$ along a curve $\gamma\colon\RR\to\RR^2$ is some smooth ways to place vectors along the curve $\gamma$.
\begin{remark}
	Please read about vector fields and smooth submanifolds.
\end{remark}

\subsection{Lie Bracket}
Given a vector field $X\in\mf X(M)$ and $v\in T_pM$, we would like to compute a directional derivative $\del vX$. For example, we might hope to take $\del_vX$ to be $(dX)_p(v)$, but this lives in $T_{X(p)}(TM)$ because $X$ maps $M\to TM$. Perhaps we want to project this down along $\pi\colon TM\to M$, but the composite $\pi\circ X=\id_M$, so we would just get $v\in T_pM$ back again.

Let's see an example to make explicit what's going on.
\begin{example}
	Take $M=\RR^2$. Then we have global frames $\left\{\frac\del{\del x_1},\frac\del{\del x_2}\right\}$ and $\left\{\frac\del{\del r},\frac\del{\del\theta}\right\}$. One should expect that $\del_{\del/\del x_1}\frac{\del}{\del x_2}=0$ because these are independent, but perhaps $\del_{\del/\del r}\frac\del{\del x}\ne0$ because these are not so orthogonal.
\end{example}
The point is that we really want to take 
\begin{definition}[Lie bracket]
	Fix vector fields $X$ and $Y$ on a smooth manifold $M$. Then there is a unique vector field $Z$ such that
	\[Zf=X(Yf)-Y(Xf)=Zf.\]
	We write $[X,Y]$ for $Z$ and name it the \textit{Lie bracket}.
\end{definition}
\begin{remark}
	Let's explain why $Z$ should exist. Well, for each $p\in M$, we are asking for $D_p\colon C^\infty(M)\to\RR$ defined by
	\[D_p(f)\coloneqq (X(Yf)-Y(Xf))(p)\]
	is a derivation and that sending $p\mapsto D_p$ is a smooth section of $TM\to M$. (This also explains that $Z$ is unique provided that it exists.) Linearity of $D_p$, and checking the Leibniz rule is a matter of writing everything out: note
	\[XY(f_1f_2)=X(f_1\cdot Y(f_2)+f_2\cdot Y(f_1))=X(f_1)\cdot Y(f_2)+f_1\cdot XY(f_2)+X(f_2)\cdot Y(f_1)+f_2\cdot XY(f_1).\]
	Writing this out for $YX$ and then subtracting produces the needed cancellation of the terms $X(f_1)\cdot Y(f_2)$ and $X(f_2)\cdot Y(f_1)$.
\end{remark}

\end{document}