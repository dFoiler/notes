% !TEX root = ../notes.tex

\documentclass[../notes.tex]{subfiles}

\begin{document}

Today we begin studying integration along curves.

\subsection{Trajectories}
We are going to want to understand trajectories.
\begin{definition}[trajectory]
	Fix a vector field $V\in\mf X(M)$. A smooth curve $\gamma\colon(a,b)\to M$ is a \textit{trajectory} or \textit{integral curve} of $V$ if and only if
	\[\gamma'=V\circ\gamma.\]
	Namely, the tangent vector along $\gamma(t)$ is the same as the one given by $V(\gamma(t))$.
\end{definition}
\begin{remark}
	Suppose we are in a local chart $(U,\varphi)$ where $\varphi=(x_1,\ldots,x_n)$. On one hand, we may write
	\[V=\sum_{i=1}^nV_i\frac{\del}{\del x_i}\]
	for smooth functions $V_1,\ldots,V_n$. On the other hand, a curve $\gamma$ with $\gamma(t)\in U$ will have
	\[\gamma'(t)=\sum_{i=1}^n\gamma_i'(t)\frac\del{\del x_i}\bigg|_{\gamma(t)},\]
	so having a trajectory amounts to solving the system
	\[\gamma'_i(t)=V_i(t)\text{ for }i\in\{1,\ldots,n\}.\]
\end{remark}
\begin{example}
	Take $M\coloneqq\RR^2$ so that we may identify $TM\cong\RR^2\times\RR^2$. Then $V(x)\coloneqq(x,x)$ can be solved directly for trajectories $\gamma$. Namely, we are asking for $\gamma'(t)=\gamma(t)$, so our curve must look like $\gamma(t)=ve^t$ where $v\in\RR^2$ is some vector.
\end{example}
\begin{example}
	If we replace $M\coloneqq\RR^2$ with $M\coloneqq B(0,1)$, then the same vector field $V(x)\coloneqq(x,x)$ will have basically the same trajectories, just perhaps limited in time.
\end{example}
\begin{example}
	Identify $M\coloneqq\CC$ with $\RR^2$, and consider the vector field $V(x)\coloneqq(x,ix)$. Then our trajectories look like $\gamma(t)=ve^{it}$ by solving the system in the usual way.
\end{example}
\begin{remark}
	Here is a quick aside: if $\gamma$ is a trajectory of $V$, and $t_0\in\RR$, then the function $\gamma_{t_0}(t)\coloneqq\gamma(t+t_0)$ is also a trajectory of $V$. This is simply because $\gamma'_{t_0}(t)=\gamma'(t+t_0)$.
\end{remark}
We would like for trajectories to exist and be unique. This is basically checked locally on charts, and then we will be able to glue along charts by some uniqueness.

% \subsection{Systems of Ordinary Differential Equations}
The following lemma is proven by working on charts.
\begin{lemma} \label{lem:trajectory-on-chart}
	Fix a smooth manifold $M$, a vector field $V\in\mf X(M)$, and some $p\in M$.
	\begin{listalph}
		\item Existence: we are granted open neighborhoods $U_0\subseteq M$ and $U\subseteq M$ for which there is $\varepsilon>0$ and a smooth map $\theta\colon(-\varepsilon,\varepsilon)\times U_0\to U$ such that any $q\in U_0$ makes
		\[\gamma_q(t)\coloneqq\theta(t,q)\]
		a trajectory of $V$ with $\gamma_q(0)=q$.
		\item Uniqueness: for any other trajectory $\widetilde\gamma\colon(a,b)\to U$ of $V$ with $\widetilde\gamma(0)=q$, we have $\widetilde\gamma=\gamma_q$ on $(-\varepsilon,\varepsilon)\cap(a,b)$.
	\end{listalph}
\end{lemma}
\begin{proof}
	This simply holds by passing to the chart $U_0$, where existence and uniqueness for systems of ordinary differential equations holds by general theory.
\end{proof}
We now glue together uniqueness.
\begin{lemma} \label{lem:local-uniq-traj}
	Fix a smooth manifold $M$, and suppose that we have two trajectories $\gamma_1\colon(a_1,b_1)\to M$ and $\gamma_2\colon(a_2,b_2)\to M$ of the same vector field $V$. If $\gamma_1(t)=\gamma_2(t)$ for any $t$, then $\gamma_1=\gamma_2$ on $(a_1,b_1)\cap(a_2,b_2)$.
\end{lemma}
\begin{proof}
	Let $I$ be the set of $t\in(a_1,b_1)\cap(a_2,b_2)$ where $\gamma_1(t)=\gamma_2(t)$. By definition, we see that $I$ is a closed subset of this interval, but by working in charts, we see that any point in $I$ has an open neighborhood in $I$ (by \Cref{lem:trajectory-on-chart}), so $I$ is also closed in the interval. Lastly, $I$ is nonempty by hypothesis, so connectivity forces $I$ to be the full interval.
\end{proof}
As such, we can glue together trajectories.
\begin{corollary}
	Fix a smooth manifold $M$, and suppose that we have two trajectories $\gamma_1\colon(a_1,b_1)\to M$ and $\gamma_2\colon(a_2,b_2)\to M$ of the same vector field $V$. If $\gamma_1(t_0)=\gamma_2(t_0)$ for any $t$, then
	\[\widetilde\gamma(t)\coloneqq\begin{cases}
		\gamma_1(t) & \text{if }t\in(a_1,b_1), \\
		\gamma_2(t) & \text{if }t\in(a_2,b_2),
	\end{cases}\]
	is also a trajectory of $V$.
\end{corollary}
\begin{proof}
	The function $\widetilde\gamma$ is well-defined by \Cref{lem:local-uniq-traj}, and it is a smooth trajectory because its restrictions to $(a_1,b_1)$ and $(a_2,b_2)$ is 
\end{proof}
To be more precise about our gluing, we will require a maximality notion.
\begin{definition}[maximal trajectory]
	Fix a vector field $V$ on a smooth manifold $M$. Given $p\in M$, there is a \textit{maximal trajectory} $\gamma\colon(a,b)\to M$ of $V$ such that $\gamma(0)=p$ in the following sense: any other trajectory $\widetilde\gamma\colon(\widetilde a,\widetilde b)\to M$ of $V$ satisfying $\widetilde\gamma(0)=p$ has $(\widetilde a,\widetilde b)\subseteq(a,b)$.
\end{definition}
\begin{remark}
	Let's show that these maximal trajectories exist. Indeed, consider the collection $\Gamma_p$ of all trajectories $\gamma\colon U_\gamma\to M$ of $V$ such $\gamma(0)=p$. Then let $U$ be the union of all the $U_\gamma$, and the uniqueness result of \Cref{lem:local-uniq-traj} allows us to define a trajectory $\widetilde\gamma\colon U\to M$ by saying $\widetilde\gamma(q)=\gamma(q)$ whenever $q\in U_{\gamma}$. (Namely, the lemma shows that this $\widetilde\gamma$ is well-defined in that it does not depend on the choice of $\gamma$ used to set $\gamma(q)$. That $\gamma$ is a smooth trajectory can be done because $\widetilde\gamma|_{U_\gamma}=\gamma$ for each $\gamma$.) This $\widetilde\gamma$ is maximal by construction: $U$ contains $U_{\gamma}$ for each $\gamma$!
\end{remark}
We would like for our maximal trajectories to always be defined over $\RR$, but the following example shows that this is not always the case.
\begin{example}
	Take $M\coloneqq\RR$, and let $V$ be the vector field $V(x_0)\coloneqq x_0^2\frac{\del}{\del x}\big|_{x_0}$. As such, we are trying to solve the ordinary differential equation $\gamma'=\gamma^2$, and we will also enforce $\gamma(0)=1$. Then solving produces $\gamma(t)=1/(1-t)$, so we see that the maximal trajectory must be $(-\infty,1)$. Namely, any other trajectory $\gamma_0\colon(a,b)\to\RR$ must agree with $\gamma$ on $(a,b)\cap(-\infty,1)$, but if $b>1$, then we note $\gamma_0(1)$ must be the limit of $\gamma(t)$ as $t\to1^-$ by continuity, which does not exist.
\end{example}
This is somehow a problem of the vector field, so we will produce a definition to fix it.
\begin{definition}[complete]
	A vector field $V$ on a smooth manifold $M$ is \textit{complete} if and only if any $p\in M$ has a trajectory $\gamma_p\colon\RR\to M$ of $V$ such that $\gamma_p(0)=p$. Note that $\gamma_p$ is then the maximal trajectory of $p$.
\end{definition}
The following lemma explains how a vector field might fail to be complete.
\begin{lemma}[Escape] \label{lem:escape}
	Fix a vector field $V$ on a smooth manifold $M$. Suppose that $\gamma\colon(a,b)\to M$ is a maximal trajectory of $V$ for some point $p\in M$. If $b<\infty$, then any compact $K\subseteq M$ and $t_0\in (a,b)$ will have $\gamma([t_0,b))\not\subseteq K$.
\end{lemma}
\begin{proof}[Sketch]
	Suppose for the sake of contradiction that $\gamma([t_0,b))$ lands fully inside $K$. Then the point is that we should be able to extend the trajectory to (an open neighborhood of) $\gamma(b)$, violating the maximality of $\gamma$.
\end{proof}
\begin{corollary} \label{cor:compact-support-is-complete}
	Fix a vector field $V$ on a smooth manifold $M$. If the support of $V$ is compact, then $V$ is complete.
\end{corollary}
\begin{proof}
	We proceed by contraposition, whereupon we get the result from \Cref{lem:escape}. Indeed, the maximal trajectory of some $p\in M$ must always stay inside the support of $V$, which is compact by hypothesis.
\end{proof}

\subsection{Flows}
We now glue together our maximal trajectories to see how vector fields flow.
\begin{definition}[flow]
	Fix a complete vector field $V$ on a smooth manifold $M$. For each $p$, let $\gamma_p\colon\RR\to M$ be the maximal trajectory so that $\theta_p(0)=p$. Then the \textit{flow} of $V$ is the corresponding function $\theta\colon\RR\times M\to M$ given by $\theta(t,p)\coloneqq\gamma_p(t)$.
\end{definition}
\begin{remark}
	By construction, we see $\theta_0=\id_M$ for each $p$.
\end{remark}
\begin{remark} \label{rem:additive-diffeo}
	One can check that $\theta_{t_1}\circ\theta_{t_2}=\theta_{t_1+t_2}$. This is simply how maximal trajectories work: by the uniqueness of trajectories forces
	\[\gamma_p(t_1+t_2)=\gamma_{\gamma_p(t_2)}(t_1)\]
	because both sides define a trajectory of $V$ giving the same point at $t_1=0$. As such, we see that $\RR$ has been given an action on $M$; for example, it follows that $\theta_t\colon M\to M$ is a diffeomorphism with inverse given by $\theta_{-t}$.
\end{remark}
\begin{remark}
	The previous remark establishes that $\theta$ is fully smooth. Indeed, by smooth variation of solutions, one sees that $\theta$ is smooth in some open neighborhood of $(0,p)$ for any $p\in M$, and then we get smoothness in general because
	\[\theta_t=\underbrace{\theta_{t/N}\circ\cdots\theta_{t/N}}_N\]
	for any $N>0$, so sending $N$ large is able to take the smoothness local at $(0,p)$ to smoothness everywhere.
\end{remark}
Here's an application, for fun.
\begin{proposition}
	Fix a connected smooth manifold $M$ and points $p,q\in M$. Then there is a compact subset $K$ containing $p$ and $q$ and a diffeomorphism $f\colon M\to M$ such that $f(p)=q$ and $f|_{M\setminus K}=\id_{M\setminus K}$.
\end{proposition}
\begin{proof}
	Let $G$ be the group of diffeomorphisms $M\to M$ which fix some the complement of some compact subset $K$; note that $G$ is indeed a group. (The interesting check is closure under composition, where the point is that the union of the two compact subsets whose complements are fixed will work.) We will show that the action of $G$ on $M$ is transitive; because $M$ is connected, it is enough to show that the orbits of $G$ are open (because the orbits partition $M$).

	In other words, for any $p\in M$, it suffices to show that there is an open neighborhood $U$ of $p$ contained in the orbit. We may assume that $U$ is a regular coordinate ball $B(0,1/2)\subseteq B(0,1)$ where $(U,\varphi)$ is a chart where $\varphi(p)=0$. Then we will show that any $q\in\varphi^{-1}(B(0,1/2))$ has a diffeomorphism $f$ sending $p$ to $q$ but fixing $M\setminus\varphi^{-1}(B(0,3/4))$.

	Well, choose a vector field $V$ to be supported on $\varphi^{-1}(B(0,3/4))$ (which is complete by \Cref{cor:compact-support-is-complete}) but on $B(0,1/2)$ is just given by having all tangent vectors point with unit length from $p$ to $q$. Then a maximal trajectory for $V$ will be able to have $\gamma(p)=q$, so using the flow of $V$ as in \Cref{rem:additive-diffeo} will complete the proof.
\end{proof}

\end{document}