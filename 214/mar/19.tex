% !TEX root = ../notes.tex

\documentclass[../notes.tex]{subfiles}

\begin{document}

We continue.

\subsection{More on the Lie Bracket}
Let's compute the Lie bracket in some examples.
\begin{remark}
	Intuitively, the Lie bracket amounts to taking the derivative of one vector field with respect to another vector field.
\end{remark}
\begin{example}
	In $\RR^m$, one has $\left[\frac\del{\del x_i},\frac\del{\del x_j}\right]=0$ whenever $i$ and $j$ are distinct indices. One sees this because partial derivatives commute in Euclidean space or more explicitly from \eqref{ex:local-lie-bracket}. As another example computation, we see
	\[\left[\frac\del{\del x_1},\frac\del{\del x_2}+f_1\frac\del{\del x_1}\right]=\frac\del{\del x_1}\left(\frac\del{\del x_2}+f_1\frac\del{\del x_1}\right)-\left(\frac\del{\del x_2}+f_1\frac\del{\del x_1}\right)\frac{\del}{\del x_1}\]
	collapses down to $\del f_1/\del x_1$ after the dust settles. This makes sense intuitively because we are taking the derivative $\frac\del{\del x_2}+f_1\frac\del{\del x_1}$ with respect to $x_1$.
\end{example}
We are essentially computing a commutator via the Lie bracket, so we have the following definition.
\begin{definition}[commute]
	Fix an $m$-manifold $M$. A set $S\subseteq\mf X(M)$ of global vector fields \textit{commutes} if and only if $[X,X']=0$ for any $X,X'\in S$.
\end{definition}
\begin{remark}
	Essentially by construction, we see that the Lie bracket is $\RR$-linear in both coordinates.
\end{remark}
\begin{remark}
	Note $[X,Y]=-[Y,X]$ by definition. In particular, $[X,X]=0$, so any vector field commutes with itself.
\end{remark}
\begin{remark}
	For $f\in C^\infty(M)$, we see that $[X,fY]=f[X,Y]+(Xf)Y$, essentially by the product rule. Explicitly, we find
	\[[X,fY]=X(fY)-fYX=(Xf)Y+fXY-fYX=f[X,Y]+(Xf)Y.\]
\end{remark}
\begin{remark}
	Another rather explicit computation shows
	\[[X,[Y,Z]]+[Y,[Z,X]]+[Z,[X,Y]]=0.\]
	For example, one sees that $[X,[Y,Z]]f=X[Y,Z]f-[Y,Z]Xf=X(YZf-ZYf)-(YZ-ZY)Xf=(XYZ-XZY-YZX-ZYX)f$ and then sums cyclically to make the total vanish.
\end{remark}
\begin{remark}
	We note that the Lie bracket does not depend on diffeomorphism class. Namely, if $F\colon M\to N$ is a diffeomorphism, and $X_1$ and $X_2$ is related to $Y_1$ and $Y_2$, then we find that $[X_1,X_2]$ and $[Y_1,Y_2]$ continue to be $F$-related. For example, one can show that
	\[F_*[X_1,X_2]=[F_*X_1,F_*X_2],\]
	though we will not write this out. This is a matter of working sufficiently locally everywhere and checking.
\end{remark}
Let's do a quick computation, for fun. Suppose we have two coordinate charts $(x_1,\ldots,x_m)$ and $(y_1,\ldots,y_m)$ on some open chart $U$ of a manifold. Let's compute the Lie brackets of $A,B\in\mf X(M)$ via both coordinate charts. Well, we will write
\[A\coloneqq\sum_{i=1}^ma_i\frac\del{\del x_i}\qquad\text{and}\qquad B\coloneqq\sum_{i=1}^mb_i\frac\del{\del x_i}.\]
Using change of coordinates, we may write
\[A=\sum_{i=1}^ma_i\sum_{j=1}^m\frac{\del y_j}{\del x_i}\frac{\del}{\del y_j}=\sum_{i=1}^m\widetilde a_j\frac{\del}{\del y_j}\]
where $\widetilde a_j$ collects terms as is necessary. We similarly write $\widetilde b_j$ so that $B=\sum_j\widetilde b_j\frac{\del}{\del y_j}$, and then we find that
\[[A,B]=\sum_{k=1}^m\left(\sum_{i=1}^ma_i\frac{\del b_k}{\del x_i}-b_i\frac{\del a_k}{\del x_i}\right)\frac\del{\del x_k}=\sum_{\ell=1}^m\left(\sum_{j=1}^m\widetilde a_j\frac{\del\widetilde b_k}{\del y_j}-\widetilde b_j\frac{\del\widetilde a_k}{\del y_j}\right)\frac{\del}{\del y_\ell}.\]
We know that these must be the same vector field, so taking the $\del/\del y_\ell$ coordinate reveals
\[\sum_{j=1}^m\widetilde a_j\frac{\del\widetilde b_k}{\del y_j}-\widetilde b_j\frac{\del\widetilde a_k}{\del y_j}=\sum_{k=1}^m\left(\sum_{i=1}^ma_i\frac{\del b_k}{\del x_i}-b_i\frac{\del a_k}{\del x_i}\right)\frac\del{y_\ell}{\del x_k}.\]

\subsection{Lie Algebras on Lie Groups}
On Euclidean space, we have a good notion of how to translate vectors around, which is able to produce lots of nice global vector fields like $\del/\del x_\bullet$. What is good about Euclidean space is that we have access to a group structure to translate vectors around, so a similar story will work on other Lie groups. To make sense of this, we have the following definition.
\begin{definition}[invariant]
	Fix a Lie group $G$. Then a vector field $Y\in\mf X(G)$ is \textit{left-invariant} if and only if $(L_g)_*X=X$ for all $g\in M$. In other words, for any $g'\in G$, we are asking for
	\[(dL_g)_{g'}X(g')=X(gg').\]
	We let $\op{Lie}G$ denote the vector space of left-invariant vector fields.
\end{definition}
For example, we are asking for $X(g)=(dL_g)_eX(e)$, so $X$ will be completely determined by $X(e)$. This is codified in the following lemma.
\begin{lemma} \label{lem:lie-is-tangent-space}
	Fix a Lie group $G$. Then $\op{Lie}G\cong T_eG$, where the isomorphism sends $X$ to $X(e)$.
\end{lemma}
\begin{proof}
	This map is certainly linear. To see that it is injective, suppose $X(e)=0$, and we want to show that $X$ itself vanishes. Well, for any $g\in G$, we see
	\[X(g)=(dL_g)_eX(e)=0,\]
	so $X=0$. Lastly, we want to check that the map is surjective. Well, for $v\in T_eG$, define $X\colon G\to TG$ by
	\[X(g)\coloneqq(dL_g)_eX(e).\]
	A direct computation shows that this definition is left-invariant, so it really just remains to show that $X$ is smooth, which is a check that we will omit. The main point is that $Xf$ is smooth for any smooth function $f$ by writing out everything explicitly, which is enough upon trying enough test functions $f$ on $X$.
\end{proof}
\begin{remark}
	The above lemma verifies that every Lie group has a global frame: let $v_1,\ldots,v_n$ be a basis of $T_eG$, and then \Cref{lem:lie-is-tangent-space} provides vector fields $X_1,\ldots,X_n$ such that $X_i(e)\coloneqq v_i$ for each $i$. Then we see that $\{X_1,\ldots,X_n\}$ is global frame because translating by $L_g$ for any $g$ preserves being a basis from $e$ to $g$.
\end{remark}
We now note that our Lie group structure descends.
\begin{lemma}
	Fix vector fields $X,Y\in\op{Lie}G$ where $G$ is a Lie group. Then $[X,Y]\in\op{Lie}G$.
\end{lemma}
\begin{proof}
	We are tasked with showing that $[X,Y]$ is left-invariant. Well, $L_g$ is a diffeomorphism, so
	\[(L_g)_*[X,Y]=[(L_g)_*X,(L_g)_*Y]=[X,Y],\]
	so we are done.
\end{proof}
This information is now packaged into a Lie algebra.
\begin{definition}[Lie algebra]
	A \textit{Lie algebra} is a vector space $V$ equipped with a Lie bracket $[\cdot,\cdot]$ which is bilinear, antisymmetric, and satisfying the Jacobi identity.
\end{definition}
\begin{example}
	Given a Lie group $G$, we have shown above that $\op{Lie}G$ becomes a Lie algebra.
\end{example}
\begin{example}
	Take the Lie group $G\coloneqq\RR^n$. Then $\op{Lie}G\cong T_0\RR^n\cong\RR^n$. Notably, $T_eG$ has basis given by $\del/\del x_i$, all of which commute with each other, so the Lie bracket vanishes on $\op{Lie}G$.
\end{example}
\begin{exe}
	Take the Lie group $G\coloneqq\op{GL}_n(\RR)$. We compute the Lie bracket.
\end{exe}
\begin{proof}
	We take coordinates given by $\del/\del x_{ij}$, and we will go ahead and compute our left-invariant vector fields. Notably, our left action is given by $L_XY=XY$ where $X,Y\in\op{GL}_n(\RR)$, which is linear in $Y$. Now, suppose $A\coloneqq\sum_{i,j}A_{ij}\frac\del{\del x_{ij}}$ is left-invariant. Then for $X\in G$ we have
	\[A(X)=(dL_X)_e(A(e))\stackrel*=XA(e)=\sum_{i,k=0}^nX_{ij}A_{jk}(e)\frac\del{\del x_{ik}}\bigg|_X,\]
	where $\stackrel*=$ is just because $L_X$ is the linear map given by multiplication by $X$, which goes down to the tangent space. The moral of the story is that any $A(e)\in M_n(\RR)$ produces the left-invariant vector field $A(X)\coloneqq XA(e)$.

	This allows us to compute the Lie bracket: fix $A,B\in\op{Lie}G$. Then, at some $X\in\op{GL}_n(\RR)$, using coordinates as before, we see
	\begin{align*}
		[A,B](X) &= \sum_{r,s=0}^n\left(\sum_{i,j=0}^nA_{ij}(X)\frac{\del B_{rs}}{\del x_{ij}}(X)-B_{ij}(X)\frac{\del A_{rs}(X)}{\del x_{ij}}\right)\frac{\del}{\del x_{rs}} \\
		&= \left(X_{ki}A_{ij}(e)\frac{\del X_{tr}B_{rs}(e)}{\del x_{ij}}-X_{ki}B_{ij}(e)\frac{\del X_{tr}A_{rs}(e)}{\del x_{ij}}\right)\frac\del{\del x_{rs}} \\
		&= \left(X_{ki}A_{ij}(e)B_{rs}(e)\frac{\del X_{tr}}{\del x_{ij}}-X_{ki}B_{ij}(e)A_{rs}(e)\frac{\del X_{tr}}{\del x_{ij}}\right)\frac\del{\del x_{rs}}.
	\end{align*}
	In total, after computing these derivatives, one sees that $[A,B]=AB-BA$.
\end{proof}

\end{document}