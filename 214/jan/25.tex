% !TEX root = ../notes.tex

\documentclass[../notes.tex]{subfiles}

\begin{document}

The first homework is due later today.

\subsection{A Couple Lemmas on Atlases}
Here are some basic properties of smooth manifolds which one can check.
\begin{lemma}
	Fix a smooth $n$-manifold $(M,\mc A)$. Given a chart $(U,\varphi)\in\mc A$, then for any open subset $U'\subseteq U$, we have $(U',\varphi|_{U'})\in\mc A$.
\end{lemma}
\begin{proof}
	By maximality of $\mc A$, it suffices to show that $\mc A\cup\{(U',\varphi|_{U'})\}$ is a smooth atlas. It contains $\mc A$, so this is at least an atlas of charts. For smooth compatibility, we pick up some $(V,\psi)\in\mc A$, and we must show that $(U',\varphi|_{U'})$ and $(V,\psi)$ are smoothly compatible. (The charts in $\mc A$ are already smoothly compatible with each other.) In other words, we must show that the transition functions are diffeomorphism: the transition maps are
	\[\varphi|_{U'\cap V}\circ\psi|_{U'\cap V}^{-1}=\left(\varphi|_{U\cap V}\circ\psi|_{U\cap V}^{-1}\right)|_{\psi(U'\cap V)}\]
	and
	\[\psi|_{U'\cap V}\circ\varphi|_{U'\cap V}^{-1}=\left(\psi|_{U\cap V}\circ\varphi|_{U\cap V}^{-1}\right)|_{\varphi(U'\cap V)},\]
	and these are both smooth as the restrictions of smooth maps. (Namely, we are using the fact that $(U,\varphi)$ and $(V,\psi)$ are smoothly compatible already.)
\end{proof}
\begin{lemma}
	Fix a smooth $n$-manifold $(M,\mc A)$. Given a chart $(U,\varphi)\in\mc A$ and diffeomorphism $\chi\colon\varphi(U)\to V$ for some open subset $V\subseteq\RR^n$, we have $(U,\chi\circ\varphi)\in\mc A$.
\end{lemma}
\begin{proof}
	The argument is similar to that of the above lemma. By maximality of $\mc A$, it suffices to show that $\mc A\cup\{(U,\chi\circ\varphi)\}$ is a smooth atlas. It contains $\mc A$, so this is at least an atlas. For smooth compatibility, we pick up some $(V,\psi)\in\mc A$, and we must show that $(V,\psi)$ and $(U,\chi\circ\varphi)$ are smoothly compatible. (Indeed, the charts in $\mc A$ are already smoothly compatible with each other.) Well, the transition maps are
	\[(\chi\circ\varphi)|_{U\cap V}\circ\psi|_{U\cap V}^{-1}=\chi|_{\varphi(U\cap V)}\circ\left(\varphi|_{U\cap V}\circ\psi|_{U\cap V}^{-1}\right)\]
	and
	\[\psi|_{U\cap V}\circ(\chi\circ\varphi)|_{U\cap V}^{-1}=\psi|_{U\cap V}\circ\varphi|_{U\cap V}^{-1}\circ\chi|_{\varphi(U\cap V)}^{-1},\]
	which are smooth maps because $(U,\varphi)$ and $(V,\psi)$ are already smoothly compatible, and $\chi$ is a diffeomorphism.
\end{proof}
\begin{lemma}
	Fix a smooth $n$-manifold $(M,\mc A)$. If $\varphi\colon U\to\RR^n$ is an injective map with $U\subseteq M$ is such that each $p\in U$ has some open neighborhood $U_p\subseteq U$ such that $(U_p,\varphi|_{U_p})\in\mc A$, then actually $(U,\varphi)\in\mc A$.
\end{lemma}
\begin{proof}
	Proceed as above via gluing.
\end{proof}

\subsection{Examples of Smooth Manifolds}
We go through some examples of smooth manifolds.
\begin{example} \label{ex:rn-sm-manifold}
	Recall from \Cref{lem:rn-manifold} that $\RR^n$ is a topological $n$-manifold. Then $\id\colon\RR^n\to\RR^n$ provides an atlas on $\RR^n$ consisting of a single chart, which is vacuously smooth; note \Cref{prop:get-max-smooth-atlas} then gives us a smooth structure.
\end{example}
More generally, we have the following.
\begin{proposition} \label{prop:open-sm-manifold}
	Fix a smooth $n$-manifold $(M,\mc A)$. For any open subset $M'\subseteq M$, we have that $M'$ is a topological $n$-manifold, and
	\[\mc A'\coloneqq\{(U,\varphi)\in\mc A:U\subseteq M'\}\]
	is a smooth structure on $M$.
\end{proposition}
\begin{proof}
	Omitted.
\end{proof}
\begin{example}
	Any open subset of $\RR^n$ is a smooth $n$-manifold by combining \Cref{ex:rn-sm-manifold} and \Cref{prop:open-sm-manifold}.
\end{example}
\begin{example}
	The charts on $S^n$ provided in \Cref{exe:sphere-top-man} provide a smooth atlas on $S^n$ and hence a smooth structure by \Cref{prop:get-max-smooth-atlas}. On the homework, we will see how to use stereographic projection to provide a smooth structure (in fact, the same smooth structure) on $S^n$.
\end{example}
\begin{example}
	Fix an $n$-dimensional $\RR$-vector space $V$. Then
	\[\mc A\coloneqq\left\{(V,\varphi):\varphi\text{ is an isomorphism to }\RR^n\right\}\]
	is a smooth atlas on $V$ and hence provides a smooth structure. These are smoothly compatible because the transition maps turn into linear isomorphisms $\RR^n\to\RR^n$, which are automatically smooth.
\end{example}
\begin{example}
	Fix the topological $1$-manifold $\RR$ of \Cref{lem:rn-manifold}. \Cref{ex:rn-sm-manifold} tells us $\mc A\coloneqq\{(\RR,{\id_\RR})\}$ provides a smooth atlas, and $\mc A'\coloneqq\{(\RR,\varphi)\}$ given by $\varphi\colon x\mapsto x^3$ is also a smooth atlas (again, smoothness is vacuous). However, $\mc A$ and $\mc A'$ provide smooth structures: otherwise, they would be contained in the same maximal smooth atlas, so $(\RR,{\id_\RR})$ and $(\RR,\varphi)$ would be smoothly compatible, but then the composite $\left({\id_\RR}\circ\varphi^{-1}\right)\colon x\mapsto\sqrt[3]x$ is not a smooth function $\RR\to\RR$.
\end{example}
\begin{remark}
	In fact, if a topological $n$-manifold has some smooth structure, there are uncountably many distinct smooth structures on $M$. On the other hand, for $n$-manifolds of small dimensions (e.g., $n\le3$), it turns out that these are diffeomorphic.
\end{remark}
\begin{remark}
	However, there do exist topological $n$-manifolds with no smooth structure, in dimensions $n\ge4$. Even worse, there are topological $n$-manifolds with distinct smooth structures up to diffeomorphism, again in dimensions $n\ge4$. Even for $S^n$, the story is complicated: there is only one smooth structure for $n\le3$, we don't understand $n=4$, and the story is complicated but somewhat understood for $n\ge5$.
\end{remark}
\begin{remark}
	Please read the examples on $\op{GL}_n(\RR)$, product manifolds, and some others.
\end{remark}

\subsection{Grassmannians}
The construction of smooth manifolds is rather long: we build a topological space, define some charts, and then check that the charts are smoothly compatible. Here's a lemma to do all of this at once.
\begin{lemma} \label{lem:build-manifold-from-set}
	Fix a set $M$ with a nonnegative integer $n\ge0$ and a collection of functions $\{(U_\alpha,\varphi_\alpha)\}_{\alpha\in\kappa}$ where $U_\alpha\subseteq M$ and $\varphi_\alpha\colon U_\alpha\to\RR^n$ is open. Further, suppose the following.
	\begin{listroman}
		\item $\varphi_\alpha(U_\alpha\cap U_\beta)\subseteq\RR^n$ is open for all $\alpha,\beta\in\kappa$.
		\item The composite $\varphi_\alpha|_{U_\alpha\cap U_\beta}\circ\varphi_\beta|_{U_\alpha\cap U_\beta}^{-1}$ is smooth for all $\alpha,\beta\in\kappa$.
		\item $M$ is covered by a countable subcollection of $\{U_\alpha\}_{\alpha\in\kappa}$.
		\item For distinct $p,q\in M$, either there is $\alpha\in\kappa$ such that $p,q\in U_\alpha$, or there are 
	\end{listroman}
	Then $M$ is a smooth $n$-manifold with smooth atlas given by $\{(U_\alpha,\varphi_\alpha)\}_{\alpha\in\kappa}$.
\end{lemma}
\begin{proof}
	We sketch the steps.
	\begin{enumerate}
		\item We provide $M$ with a topology. We would like for Well, we say that $A\subseteq M$ is open if and only if $\varphi_\alpha(A\cap U_\alpha)$ is open for all $\alpha\in\kappa$.
		\item Then condition (i) makes the $\varphi_\alpha$ into homeomorphisms onto their images. Thus, $\{(U_\alpha,\varphi_\alpha)\}_{\alpha\in\kappa}$ is an atlas.
		\item Condition (ii) implies that $\{(U_\alpha,\varphi_\alpha)\}$
		\item Condition (iii) implies that $M$ becomes second countable.
		\item Lastly, condition (iv) implies that $M$ is Hausdorff.
	\end{enumerate}
	We leave the checks to the reader.
\end{proof}
Let's see an example of this.
\begin{exe}
	Fix nonnegative integers $k\le n$. Then let $M\coloneqq\op{Gr}_k\left(\RR^n\right)$ denote the set of $k$-dimensional linear subspaces $V$ of $\RR^n$. We show that $M$ is a smooth $k(n-k)$-manifold.
\end{exe}
\begin{proof}
	We use \Cref{lem:build-manifold-from-set}. For concreteness, let we choose our index set $I$ to consist of pairs $(P,Q)$ of subspaces of $\RR^n$ such that $\RR^n=P\oplus Q$ and $\dim P=k$ and $\dim Q=n-k$. The point is that we are choosing a complement for our $k$-dimensional subspaces in order to help count them. In particular, we may define the subset
	\[U_\alpha\coloneqq\left\{V\in\op{Gr}_k\left(\RR^n\right):V\cap Q=\{0\}\right\}.\]
	Notably, for any $V\in U_\alpha$, there is a unique linear map $M_{P,Q,V}\colon P\to Q$ such that
	\[V=\left\{x+M_{P,Q,V}x\in P\oplus Q:x\in P\right\}.\]
	Approximately speaking, we are viewing $V$ as a graph. Anyway, this construction provides a map $\varphi_\alpha\colon U_\alpha\to\op{Hom}_\RR(P,Q)$ given by $V\mapsto M_{P,Q,V}$, where we identify $\op{Hom}_\RR(P,Q)\cong\RR^{k(n-k)}$. We now conclude by noting that we can check the properties from \Cref{lem:build-manifold-from-set}. For example, to see that the transition maps are smooth, suppose we have two pairs $(P,Q),(P',Q')\in I$, and the vector space $V$ decomposes into the two separate ways, and these matrices have rational functions in their coordinates, so smoothness follows. As another example, one can actually cover $M$ by finitely many charts, and the last check follows because any $k$-dimensional subspaces $V,V'\subseteq\RR^n$ has some $(n-k)$-dimensional subspace $Q\subseteq\RR^n$ such that $V\cap Q=V'\cap Q=\{0\}$.
\end{proof}

\subsection{Manifolds with Boundary}
Before moving on from our discussion of a single manifold, we discuss manifolds with boundary.
\begin{definition}[topological manifold with boundary]
	Fix a nonnegative integer $n$. A \textit{topological $n$-manifold with boundary} is a Hausdorff, second countable topological space $M$ with the following variant of being locally Euclidean: for any $p\in M$, there are open subsets $U\subseteq M$ and
	\[\widehat U\subseteq\HH\coloneqq\left\{(x_1,\ldots,x_n)\in\RR^n:x_n\ge0\right\}\]
	such that $p\in U$ and $U\cong\widehat U$. We continue to call $(U,\varphi)$ a chart.
\end{definition}
\begin{example}
	Any topological $n$-manifold is a topological $n$-manifold with boundary: one can simply make the charts output to $\HH^\circ$.
\end{example}
\begin{example}
	The space $\HH^n=\left\{(x_1,\ldots,x_n)\in\RR^n:x_n\ge0\right\}$ is a topological $n$-manifold with boundary.
\end{example}
The point is that we can pick up some ``boundary'' like the one in $\HH^n=\left\{(x_1,\ldots,x_n)\in\RR^n:x_n\ge0\right\}$. This notion of boundary is in fact fairly intrinsic.
\begin{defihelper}[boundary, interior] \nirindex{boundary point} \nirindex{interior point}
	Fix a topological $n$-manifold with boundary $M$ and a point $p\in M$.
	\begin{itemize}
		\item Then $p$ is a \textit{boundary point} if and only if there is a chart $(U,\varphi)$ such that $\varphi(p)\in\del\HH^n$.
		\item Then $p$ is an \textit{interior point} if and only if there is a chart $(U,\varphi)$ such that $\varphi(p)$ is in the interior of $\HH^n$.
	\end{itemize}
\end{defihelper}
We will show later that any point in $M$ is exactly one of a boundary point or an interior point.

Anyway, let's discuss smoothness. This requires understanding smoothness on $\del\HH^n$.
\begin{definition}
	Fix a subset $A\subseteq\RR^n$. A function $f\colon A\to\RR^m$ is \textit{smooth} if and only if there is an open subset $V\subseteq\RR^n$ containing $A$ and a smooth extension $\widetilde f\colon V\to\RR^n$ of $f$.
\end{definition}
\begin{remark}
	It turns out that (by Seeley's theorem) if $V\subseteq\HH^n$ is open, it is enough to check that the partial derivatives of some function $f\colon V\to\RR^m$ extend continuously to the boundary.
\end{remark}

\end{document}