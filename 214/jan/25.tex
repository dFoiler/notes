% !TEX root = ../notes.tex

\documentclass[../notes.tex]{subfiles}

\begin{document}

The first homework is due later today.

\subsection{A Couple Lemmas on Atlases}
Here are some basic properties of smooth manifolds which one can check.
\begin{lemma} \label{lem:res-chart}
	Fix a smooth $n$-manifold $(M,\mc A)$. Given a chart $(U,\varphi)\in\mc A$, then for any open subset $U'\subseteq U$, we have $(U',\varphi|_{U'})\in\mc A$.
\end{lemma}
\begin{proof}
	By maximality of $\mc A$, it suffices to show that $\mc A\cup\{(U',\varphi|_{U'})\}$ is a smooth atlas. It contains $\mc A$, so this is at least an atlas of charts. For smooth compatibility, we pick up some $(V,\psi)\in\mc A$, and we must show that $(U',\varphi|_{U'})$ and $(V,\psi)$ are smoothly compatible. (The charts in $\mc A$ are already smoothly compatible with each other.) In other words, we must show that the transition functions are diffeomorphism: the transition maps are
	\[\varphi|_{U'\cap V}\circ\psi|_{U'\cap V}^{-1}=\left(\varphi|_{U\cap V}\circ\psi|_{U\cap V}^{-1}\right)|_{\psi(U'\cap V)}\]
	and
	\[\psi|_{U'\cap V}\circ\varphi|_{U'\cap V}^{-1}=\left(\psi|_{U\cap V}\circ\varphi|_{U\cap V}^{-1}\right)|_{\varphi(U'\cap V)},\]
	and these are both smooth as the restrictions of smooth maps. (Namely, we are using the fact that $(U,\varphi)$ and $(V,\psi)$ are smoothly compatible already.)
\end{proof}
\begin{lemma}
	Fix a smooth $n$-manifold $(M,\mc A)$. Given a chart $(U,\varphi)\in\mc A$ and diffeomorphism $\chi\colon\varphi(U)\to V$ for some open subset $V\subseteq\RR^n$, we have $(U,\chi\circ\varphi)\in\mc A$.
\end{lemma}
\begin{proof}
	The argument is similar to that of the above lemma. By maximality of $\mc A$, it suffices to show that $\mc A\cup\{(U,\chi\circ\varphi)\}$ is a smooth atlas. It contains $\mc A$, so this is at least an atlas. For smooth compatibility, we pick up some $(V,\psi)\in\mc A$, and we must show that $(V,\psi)$ and $(U,\chi\circ\varphi)$ are smoothly compatible. (Indeed, the charts in $\mc A$ are already smoothly compatible with each other.) Well, the transition maps are
	\[(\chi\circ\varphi)|_{U\cap V}\circ\psi|_{U\cap V}^{-1}=\chi|_{\varphi(U\cap V)}\circ\left(\varphi|_{U\cap V}\circ\psi|_{U\cap V}^{-1}\right)\]
	and
	\[\psi|_{U\cap V}\circ(\chi\circ\varphi)|_{U\cap V}^{-1}=\psi|_{U\cap V}\circ\varphi|_{U\cap V}^{-1}\circ\chi|_{\varphi(U\cap V)}^{-1},\]
	which are smooth maps because $(U,\varphi)$ and $(V,\psi)$ are already smoothly compatible, and $\chi$ is a diffeomorphism.
\end{proof}
\begin{lemma}
	Fix a smooth $n$-manifold $(M,\mc A)$. If $\varphi\colon U\to\RR^n$ is an injective map with $U\subseteq M$ is such that each $p\in U$ has some open neighborhood $U_p\subseteq U$ such that $(U_p,\varphi|_{U_p})\in\mc A$, then actually $(U,\varphi)\in\mc A$.
\end{lemma}
\begin{proof}
	By the definition of being a maximal smooth atlas, it suffices to show that $(U,\varphi)$ is smoothly compatible with all charts in $\mc A$. Well, pick up some chart $(V,\psi)$, and we would like to show that the transition map
	\[\varphi|_{U\cap V}\circ\psi|_{U\cap V}^{-1}\]
	is a diffeomorphism. Well, we can being a diffeomorphism locally by checking it at all point $\psi(p)\in\psi(U\cap V)$ where $p\in U\cap V$. But for some fixed $p$, we are promised some open subset $U_p\subseteq U$ such that $(U_p,\varphi|_{U_p})\in\mc A$, so the map
	\[\varphi|_{U_p\cap V}\circ\psi|_{U_p\cap V}^{-1}=\left(\varphi|_{U\cap V}\circ\psi|_{U\cap V}^{-1}\right)|_{\psi(U_p\cap V)}\]
	is a diffeomorphism. So we produce smoothness at the images of $p$ of the function and its inverse.
\end{proof}

\subsection{Examples of Smooth Manifolds}
We go through some examples of smooth manifolds.
\begin{example} \label{ex:rn-sm-manifold}
	Recall from \Cref{lem:rn-manifold} that $\RR^n$ is a topological $n$-manifold. Then $\id\colon\RR^n\to\RR^n$ provides an atlas on $\RR^n$ consisting of a single chart, which is vacuously smooth; note \Cref{prop:get-max-smooth-atlas} then gives us a smooth structure.
\end{example}
More generally, we have the following.
\begin{proposition} \label{prop:open-sm-manifold}
	Fix a smooth $n$-manifold $(M,\mc A)$. For any nonempty open subset $M'\subseteq M$, we have that $M'$ is a topological $n$-manifold, and
	\[\mc A'\coloneqq\{(U,\varphi)\in\mc A:U\subseteq M'\}\]
	is a smooth structure on $M$.
\end{proposition}
\begin{proof}
	By \Cref{prop:sub-top-man}, we see that $M'$ is a topological $n$-manifold. It remains to show that $\mc A'$ is a smooth structure. Here are our checks.
	\begin{itemize}
		\item Chart: for any $x\in M'$, we know $\mc A$ is a chart on $M$, so there is a chart $(U,\varphi)\in\mc A$ with $x\in U$. Now, $U\subseteq M$ is open, so \Cref{lem:res-chart} tells us that $(U\cap M',\varphi|_{U\cap M'})$ is a chart in $\mc A$. But now $U\cap M'\subseteq M'$, so $(U\cap M',\varphi|_{U\cap M'})\in\mc A'$ by construction, so we conclude because $x\in U\cap M'$.
		\item Smooth: for any two charts $(U,\varphi),(V,\psi)\in\mc A'$, we note that these charts belong to the smooth atlas $\mc A$ already, so they are already smoothly compatible.
		\item Maximal: by definition of being a maximal smooth atlas, it suffices to show that if $(U,\varphi)$ is a chart of $M'$ smoothly compatible with $\mc A'$, then it must be in $\mc A'$. Well, $U\subseteq M'$ already, so it suffices to show that $(U,\varphi)\in\mc A$. Because $\mc A$ is already a maximal smooth atlas, it suffices to show that $(U,\varphi)$ is compatible with all the charts in $\mc A$. Well, for any chart $(V,\psi)\in\mc A$, we need the composite
		\[\varphi|_{U\cap V}\circ\psi|_{U\cap V}^{-1}\]
		to be a diffeomorphism. But we simply note that $(U\cap V,\psi|_{U\cap V})\in\mc A$ by \Cref{lem:res-chart} will live in $\mc A'$, so the above is a diffeomorphism because the hypothesis on $(U,\varphi)$ implies that it would be smoothly compatible with $(U\cap V,\psi|_{U\cap V})\in\mc A'$.
		\qedhere
	\end{itemize}
\end{proof}
\begin{example}
	Any nonempty open subset of $\RR^n$ is a smooth $n$-manifold by combining \Cref{ex:rn-sm-manifold} and \Cref{prop:open-sm-manifold}. For example,
	\[\op{GL}_n(\RR)\coloneqq\left\{M\in\RR^{n\times n}:\det M\ne0\right\}\]
	is an open subset of $\RR^{n\times n}$, so $\op{GL}_n(\RR)$ is a smooth manifold. (Notably, $\det\colon\RR^{n\times n}\to\RR$ is a polynomial and hence continuous, so the pre-image of $\RR\setminus\{0\}$ is open.)
\end{example}
\begin{example}
	From \Cref{ex:graph-top-man}, we know that the graph $\Gamma$ of a smooth function $f\colon V\to\RR^n$, where $V\subseteq\RR^m$ is open, is a topological $n$-manifold, where we have a chart given by the projection $\pi\colon\Gamma\to V$. Using this chart alone produces a smooth atlas and makes $\Gamma$ into a smooth $n$-manifold as well.
\end{example}
\begin{example}
	We claim that the charts on $S^n$ provided in \Cref{exe:sphere-top-man} provide a smooth atlas on $S^n$ and hence a smooth structure by \Cref{prop:get-max-smooth-atlas}. Indeed, we must show that the transition maps
	\[\varphi_i^\pm|_{U_i^\pm\cap U_j^\pm}\circ\varphi_j^\pm|_{U_i^\pm\cap U_j^\pm}^{-1}(x_1,\ldots,x_n)=\left(x_1,\ldots,\widehat x_j,\ldots,x_{i-1},\pm\sqrt{1-\left(x_1^2+\cdots+x_n^2\right)},x_i,\ldots,x_n\right)\]
	is a diffeomorphism (for any choice of signs). The above equation shows that our map is smooth for $i>j$, and the computation for $i<j$ simply switches the $i$th and $j$th coordinates. On the homework, we will see how to use stereographic projection to provide a smooth structure (in fact, the same smooth structure) on $S^n$.
\end{example}
\begin{example}
	Fix an $n$-dimensional $\RR$-vector space $V$. Then we claim
	\[\mc A\coloneqq\left\{(V,\varphi):\varphi\text{ is an isomorphism to }\RR^n\right\}\]
	is a smooth atlas on $V$ and hence provides a smooth structure. Indeed, certainly this is an atlas: there is some isomorphism $\varphi\colon V\to\RR^n$, and this chart will cover $V$. Further, these are smoothly compatible because the transition map between the two arbitrary charts $(V,\varphi)$ and $(V,\psi)$ is the linear isomorphism $\left(\varphi\circ\psi^{-1}\right)\colon\RR^n\to\RR^n$, which is linear and hence smooth.
\end{example}
\begin{example}
	Fix the topological $1$-manifold $\RR$ of \Cref{lem:rn-manifold}. \Cref{ex:rn-sm-manifold} tells us $\mc A\coloneqq\{(\RR,{\id_\RR})\}$ provides a smooth atlas, and $\mc A'\coloneqq\{(\RR,\varphi)\}$ given by $\varphi\colon x\mapsto x^3$ is also a smooth atlas (again, smoothness is vacuous). However, $\mc A$ and $\mc A'$ provide smooth structures: otherwise, they would be contained in the same maximal smooth atlas, so $(\RR,{\id_\RR})$ and $(\RR,\varphi)$ would be smoothly compatible, but then the composite $\left({\id_\RR}\circ\varphi^{-1}\right)\colon x\mapsto\sqrt[3]x$ is not a smooth function $\RR\to\RR$.
\end{example}
\begin{example}
	Recall that $\RP^n$ is a topological $n$-manifold by \Cref{exe:rpn-top-man}. We claim that the charts $(U_i,\varphi_i)$ actually form a smooth atlas on $\RP^n$, thus making $\RP^n$ into a smooth atlas. We already checked that these charts cover $\RP^n$, and they are smoothly compatible because we can compute the transition between $(U_i,\varphi_i)$ and $(U_j,\varphi_j)$ is
	\[\varphi_i|_{U_i\cap U_j}\circ\varphi_j|_{U_i\cap U_j}^{-1}(x_0,\ldots,\widehat x_j,\ldots,x_n)=\left(\frac{x_0}{x_i},\ldots,\frac{x_{j-1}}{x_i},\frac1{x_i},\frac{x_{j+1}}{x_i},\ldots,\frac{x_n}{x_i}\right),\]
	which we can see is a rational and hence smooth function.
\end{example}
\begin{example}
	Fix smooth manifolds $(M_1,\mc A_1),\ldots,(M_k,\mc A_k)$, where $M_i$ is a smooth $n_i$-manifold. The product $M\coloneqq M_1\times\cdots\times M_k$ is a smooth manifold by \Cref{prop:product-manifold}, and the proof implies that
	\[\mc A\coloneqq\left\{(U_1\times\cdots\times U_k,\varphi_1\times\cdots\times\varphi_k):(U_i,\varphi_i)\in\mc A_i\text{ for each }i\right\}\]
	is an atlas on $M$. In fact, this is a smooth atlas, thus providing $M$ with a smooth structure by \Cref{prop:get-max-smooth-atlas}. Well, the transition map between the charts $(U,\varphi)\coloneqq(U_1\times\cdots\times U_k,\varphi_1\times\cdots\times\varphi_k)$ and $(V,\psi)\coloneqq(V_1\times\cdots\times V_k,\psi_1\times\cdots\times\psi_k)$ is
	\[\varphi|_{U\cap V}\circ\psi|_{U\cap V}^{-1}=\left(\varphi_1|_{U_1\cap V_1}\circ\psi_1|_{U_1\cap V_1}^{-1}\right)^{-1}\times\cdots\times\left(\varphi_k|_{U_k\cap V_k}\circ\psi_k|_{U_k\cap V_k}^{-1}\right),\]
	which we can see is smooth as it is the product of smooth functions.
\end{example}
\begin{remark}
	In fact, if a topological $n$-manifold has some smooth structure, there are uncountably many distinct smooth structures on $M$. On the other hand, for $n$-manifolds of small dimensions (e.g., $n\le3$), it turns out that these are diffeomorphic.
\end{remark}
\begin{remark}
	However, there do exist topological $n$-manifolds with no smooth structure, in dimensions $n\ge4$. Even worse, there are topological $n$-manifolds with distinct smooth structures up to diffeomorphism, again in dimensions $n\ge4$. Even for $S^n$, the story is complicated: there is only one smooth structure for $n\le3$, we don't understand $n=4$, and the story is complicated but somewhat understood for $n\ge5$.
\end{remark}
% \begin{remark}
% 	Please read the examples on $\op{GL}_n(\RR)$, product manifolds, and some others.
% \end{remark}

\subsection{Grassmannians}
The construction of smooth manifolds is rather long: we build a topological space, define some charts, and then check that the charts are smoothly compatible. Here's a lemma to do all of this at once.
\begin{lemma} \label{lem:build-manifold-from-set}
	Fix a set $M$ with a nonnegative integer $n\ge0$ and a collection of functions $\{(U_\alpha,\varphi_\alpha)\}_{\alpha\in\kappa}$ where $U_\alpha\subseteq M$ and $\varphi_\alpha\colon U_\alpha\to\RR^n$ is open. Further, suppose the following.
	\begin{listroman}
		\item $\varphi_\alpha(U_\alpha\cap U_\beta)\subseteq\RR^n$ is open for all $\alpha,\beta\in\kappa$.
		\item The composite $\varphi_\alpha|_{U_\alpha\cap U_\beta}\circ\varphi_\beta|_{U_\alpha\cap U_\beta}^{-1}$ is smooth for all $\alpha,\beta\in\kappa$.
		\item $M$ is covered by a countable subcollection of $\{U_\alpha\}_{\alpha\in\kappa}$.
		\item For distinct $p,q\in M$, either there is $\alpha\in\kappa$ such that $p,q\in U_\alpha$, or there are 
	\end{listroman}
	Then $M$ is a smooth $n$-manifold with smooth atlas given by $\{(U_\alpha,\varphi_\alpha)\}_{\alpha\in\kappa}$.
\end{lemma}
\begin{proof}
	We sketch the steps.
	\begin{enumerate}
		\item We provide $M$ with a topology. We would like for Well, we say that $A\subseteq M$ is open if and only if $\varphi_\alpha(A\cap U_\alpha)$ is open for all $\alpha\in\kappa$.
		\item Then condition (i) makes the $\varphi_\alpha$ into homeomorphisms onto their images. Thus, $\{(U_\alpha,\varphi_\alpha)\}_{\alpha\in\kappa}$ is an atlas.
		\item Condition (ii) implies that $\{(U_\alpha,\varphi_\alpha)\}$
		\item Condition (iii) implies that $M$ becomes second countable.
		\item Lastly, condition (iv) implies that $M$ is Hausdorff.
	\end{enumerate}
	We leave the checks to the reader.
\end{proof}
Let's see an example of this.
\begin{exe}
	Fix nonnegative integers $k\le n$. Then let $M\coloneqq\op{Gr}_k\left(\RR^n\right)$ denote the set of $k$-dimensional linear subspaces $V$ of $\RR^n$. We show that $M$ is a smooth $k(n-k)$-manifold.
\end{exe}
\begin{proof}[Sketch]
	We use \Cref{lem:build-manifold-from-set}. For concreteness, let we choose our index set $I$ to consist of pairs $(P,Q)$ of subspaces of $\RR^n$ such that $\RR^n=P\oplus Q$ and $\dim P=k$ and $\dim Q=n-k$. The point is that we are choosing a complement for our $k$-dimensional subspaces in order to help count them. In particular, we may define the subset
	\[U_\alpha\coloneqq\left\{V\in\op{Gr}_k\left(\RR^n\right):V\cap Q=\{0\}\right\}.\]
	Notably, for any $V\in U_\alpha$, there is a unique linear map $M_{P,Q,V}\colon P\to Q$ such that
	\[V=\left\{x+M_{P,Q,V}x\in P\oplus Q:x\in P\right\}.\]
	Approximately speaking, we are viewing $V$ as a graph. Anyway, this construction provides a map $\varphi_\alpha\colon U_\alpha\to\op{Hom}_\RR(P,Q)$ given by $V\mapsto M_{P,Q,V}$, where we identify $\op{Hom}_\RR(P,Q)\cong\RR^{k(n-k)}$. We now conclude by noting that we can check the properties from \Cref{lem:build-manifold-from-set}. For example, to see that the transition maps are smooth, suppose we have two pairs $(P,Q),(P',Q')\in I$, and the vector space $V$ decomposes into the two separate ways, and these matrices have rational functions in their coordinates, so smoothness follows. As another example, one can actually cover $M$ by finitely many charts, and the last check follows because any $k$-dimensional subspaces $V,V'\subseteq\RR^n$ has some $(n-k)$-dimensional subspace $Q\subseteq\RR^n$ such that $V\cap Q=V'\cap Q=\{0\}$.
\end{proof}

\subsection{Manifolds with Boundary}
Before moving on from our discussion of a single manifold, we discuss manifolds with boundary.
\begin{definition}[topological manifold with boundary]
	Fix a nonnegative integer $n$. A \textit{topological $n$-manifold with boundary} is a Hausdorff, second countable topological space $M$ with the following variant of being locally Euclidean: for any $p\in M$, there are open subsets $U\subseteq M$ and
	\[\widehat U\subseteq\HH\coloneqq\left\{(x_1,\ldots,x_n)\in\RR^n:x_n\ge0\right\}\]
	such that $p\in U$ and $U\cong\widehat U$. We continue to call $(U,\varphi)$ a chart.
\end{definition}
\begin{example}
	Any topological $n$-manifold is a topological $n$-manifold with boundary: one can simply make the charts output to $\HH^\circ$.
\end{example}
\begin{example}
	The space $\HH^n=\left\{(x_1,\ldots,x_n)\in\RR^n:x_n\ge0\right\}$ is a topological $n$-manifold with boundary.
\end{example}
The point is that we can pick up some ``boundary'' like the one in $\HH^n=\left\{(x_1,\ldots,x_n)\in\RR^n:x_n\ge0\right\}$.

Anyway, let's discuss smoothness. This requires understanding smoothness on $\del\HH^n$.
\begin{definition}
	Fix a subset $A\subseteq\RR^n$. A function $f\colon A\to\RR^m$ is \textit{smooth} if and only if there is an open subset $V\subseteq\RR^n$ containing $A$ and a smooth extension $\widetilde f\colon V\to\RR^n$ of $f$.
\end{definition}
\begin{remark}
	It turns out that (by Seeley's theorem) if $V\subseteq\HH^n$ is open, it is enough to check that the partial derivatives of some function $f\colon V\to\RR^m$ extend continuously to the boundary.
\end{remark}
\begin{definition}[smooth manifold with boundary]
	Fix a nonnegative integer $n$. A \textit{smooth $n$-manifold with boundary} is a pair $(M,\mc A)$ where $M$ is a topological $n$-manifold with boundary, and $\mc A$ is a maximal smooth atlas, where we are taking atlas in the sense
\end{definition}
We will not bother to redo the proof of \Cref{prop:get-max-smooth-atlas} to explain that the notion of a maximal smooth atlas makes sense with subsets of $\HH^n$ in addition to subsets of $\RR^n$; all the proofs are the same.

Note that boundary is in fact an intrinsic notion.
\begin{defihelper}[boundary, interior] \nirindex{boundary point} \nirindex{interior point}
	Fix a smooth $n$-manifold with boundary $M$ and a point $p\in M$.
	\begin{itemize}
		\item Then $p$ is a \textit{boundary point} if and only if there is a smooth chart $(U,\varphi)$ such that $\varphi(p)\in\del\HH^n$.
		\item Then $p$ is an \textit{interior point} if and only if there is a smooth chart $(U,\varphi)$ such that $\varphi(p)$ is in the interior of $\HH^n$.
	\end{itemize}
\end{defihelper}
We will show in \Cref{thm:boundary-makes-sense} that any point in $M$ is exactly one of a boundary point or an interior point.

\end{document}