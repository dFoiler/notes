% !TEX root = ../notes.tex

\documentclass[../notes.tex]{subfiles}

\begin{document}

The first homework has been posted. It is mostly a review of point-set topology things. It is due on the 25th of January.
\begin{remark}
	Please read the section on fundamental groups of manifolds on your own. We will not discuss it in class.
\end{remark}
To review, our current goal is to define smooth manifolds. Thus far we have defined a topological space and provided enough adjectives to turn it into a topological manifold. To proceed, we need to add smoothness to our structure. We will do this later.

\subsection{Connectivity}
For now, we will content ourselves with some extra adjectives for our topological manifolds which will later be helpful. Here are two notions of connectivity.
\begin{definition}[connected]
	Fix a topological space $X$. Then $X$ is \textit{disconnected} if and only if there exist disjoint nonempty open subsets $U,V\subseteq X$ such that $X=U\sqcup V$. If $X$ is not disconnected, we say that $X$ is connected.
\end{definition}
\begin{example}
	The interval $[0,1]$ is connected. See \cite[Lemma~A.6]{elber-top}.
\end{example}
\begin{remark}
	Equivalently, we can say that $X$ is connected if and only if $X$ and $\emp$ are the only subsets of $X$ which are both open and closed.
\end{remark}
\begin{definition}[path-connected]
	Fix a topological space $X$. Then $X$ is \textit{path-connected} if and only if any two points $p,q\in X$ has some continuous map $\gamma\colon[0,1]\to X$ such that $\gamma(0)=p$ and $\gamma(1)=q$.
\end{definition}
\begin{example} \label{ex:ball-is-path-conn}
	The space $B(0,1)\subseteq\RR^n$ is path-connected. Indeed, we show that the path-connected component of $0$ is all of $B(0,1)$; see \cite[Definition~A.19]{elber-top}. In other words, we must exhibit a path from $0$ to $v$ for any $v\in B(0,1)$. Well, define $\gamma\colon[0,1]\to B(0,1)$ by $\gamma(t)\coloneqq tv$. This is continuous because it is linear, and it has $\gamma(0)=0$ and $\gamma(1)=v$ as desired.
\end{example}
In general, these two notions do not coincide.
\begin{example}
	Consider the topological space
	\[X\coloneqq\left\{(x,\sin(1/x)):x\in(0,1)\right\}\cup\{(0,y):y\in\RR\}.\]
	Then $X$ is connected, but it is not path-connected. See \cite[Exercise~A.20]{elber-top}.
\end{example}
But one does in general apply the other.
% \begin{lemma}
% 	Fix a connected topological space $X$. If $f\colon X\to Y$ is a continuous map, then $f(X)$ is connected.
% \end{lemma}
% \begin{proof}
% 	We proceed by contraposition. Suppose that $f(X)\subseteq Y$ is disconnected, so we may find disjoint nonempty open subsets $U,V\subseteq Y$ such that $f(X)=(U\cap Y)\sqcup(V\cap Y)$ and $U\cap f(X),V\cap f(X)\ne\emp$.
% \end{proof}
\begin{lemma} \label{lem:path-connected-to-connected}
	Fix a topological space $X$. If $X$ is path-connected, then $X$ is connected.
\end{lemma}
\begin{proof}
	See \cite[Lemma~A.16]{elber-top}, though we will sketch the proof. We proceed by contraposition. Suppose that $X$ is disconnected, so we may write $X=U\sqcup V$ where $U,V\subseteq X$ are disjoint nonempty open subsets. Now choose some $p\in U$ and $q\in V$, and we claim that there is no path $\gamma\colon[0,1]\to X$. Indeed, $\gamma^{-1}(U)$ and $\gamma^{-1}(V)$ would then be nonempty disjoint open subsets of $[0,1]$ covering $[0,1]$, which is a contradiction.
\end{proof}
However, for topological manifolds, these notions do coincide.
\begin{proposition}
	Fix a topological space $M$ which is locally Euclidean of dimension $n$. Then $M$ is path-connected if and only if it is connected.
\end{proposition}
\begin{proof}
	The forward direction is by \Cref{lem:path-connected-to-connected}. Thus, we focus on showing the converse. Fix some $p\in M$, and we define the subset
	\[U_p\coloneqq\{q\in M:\text{there exists a path from }p\text{ to }q\}.\]
	This is the path-connected component of $p$ in $M$; see \cite[Definition~A.19]{elber-top}. The main claim is that $U_p$ is open.
	
	Suppose $q\in M$, and we need to find an open neighborhood $B_q\subseteq M$ of $q$ living inside $U_p$. Noting then that $U_p=\bigcup_{q\in U_p}B_q$ will complete the proof of this claim. Well, $q$ has some open neighborhood $B\subseteq M$ equipped with a homeomorphism $\varphi\colon B\cong B(0,1)$ by \Cref{rem:better-chart}. Then $B(0,1)$ is path-connected by \Cref{ex:ball-is-path-conn}, so $B$ is path-connected by going back through the homeomorphism. Thus, because $U_p$ is an equivalence class, it is also the path-connected equivalence class of $q$, so $U_p$ must contain $B$.

	Now, let $\mathcal U$ denote the collection of path-connected components of $M$. This is a collection of disjoint open subsets covering $M$. Certainly it is nonempty, so select $U\in\mathcal U$. Then we write
	\[M=U\cup\bigcup_{U'\in\mc U\setminus\{U\}}U'.\]
	This is a decomposition of $M$ into disjoint open subsets, so because $M$ is connected, one of these must be empty. But $U$ is empty, so instead the union of the $U'$ must be nonempty. However, everything in $\mc U$ is nonempty, so instead we see that $\mc U\setminus\{U\}$ is empty, so $M=U$ is path-connected.
\end{proof}

\subsection{Local compactness}
Here is our definition.
\begin{definition}[local compactness]
	Fix a topological space $X$. Then $X$ is \textit{locally compact} if and only if any $x\in X$ has some open neighborhood $U\subseteq X$ such that there exists a compact subset $K\subseteq X$ containing $U$.
\end{definition}
\begin{remark} \label{rem:get-precompact}
	If $X$ is Hausdorff, then compact subsets are closed \cite[Corollary~4.13]{elber-top}, and closed subsets of a compact space are still compact \cite[Lemma~4.10]{elber-top}, so we may as well take $K=\ov U$ in the above definition.
\end{remark}
The above remark motivates the following definition.
\begin{definition}[precompact]
	Fix a topological space $X$. An open subset $U\subseteq X$ is \textit{precompact} if and only if $\ov U$ is compact.
\end{definition}
\begin{remark} \label{rem:smaller-precompact}
	Here is a quick check which will prove to be useful: if $X$ is Hausdorff and $U\subseteq X$ is precompact, and $V\subseteq U$, then $V$ is still precompact. Indeed, $\ov U$ is compact, and $\ov V\subseteq\ov U$ is a closed subset and hence compact \cite[Lemma~4.10]{elber-top}.
\end{remark}
\begin{example}
	The topological space $\RR$ is locally compact; see \cite[Example~4.71]{elber-top}.
\end{example}
\begin{nex}
	Infinite-dimensional normed vector spaces fail to be locally compact. Namely, open balls fail to be precompact, so local compactness fails.
\end{nex}
\begin{nex}
	The space $\QQ$ is not locally compact. Indeed, suppose for the sake of contradiction that we have a precompact nonempty open neighborhood $U\subseteq\QQ$ of $0\in\QQ$. Now, $\QQ$ is Hausdorff (it's a metric space), so we can find some $\varepsilon>0$ such that $(-\varepsilon,\varepsilon)\subseteq U$ while $\varepsilon\notin\QQ$, so \Cref{rem:smaller-precompact} tells us that $(\varepsilon/2,\varepsilon)$ is precompact so that $[\varepsilon/2,\varepsilon]$ is actually compact.

	However, this is false. Let $\{\alpha_i\}_{i\ge1}$ be an increasing sequence of irrationals in $[\varepsilon/2,\varepsilon]$ with $\alpha_i\to\varepsilon$. Explicitly, we can take $\alpha_i\coloneqq\frac i{i+1}\cdot\varepsilon$. Then we define
	\[U_i\coloneqq[\alpha_i,\alpha_{i+1}]\]
	for each $i\ge1$. Note $[\alpha_i,\alpha_{i+1}]=(\alpha_i,\alpha_{i+1})$, so the $U_\bullet$s provide a countable sequence of disjoint open subsets covering $[\varepsilon/2,\varepsilon]$. Thus, $[\varepsilon/2,\varepsilon]$ cannot be compact.
\end{nex}
One can check that manifolds are locally compact.
\begin{proposition} \label{prop:man-is-loc-cmpt}
	Fix a topological $n$-manifold $M$. Then $M$ is locally compact.
\end{proposition}
\begin{proof}
	This follows from being locally Euclidean. Fix $p\in M$, and then we are promised some open subset $U\subseteq M$ and $\widehat U\subseteq\RR^n$ with a homeomorphism $\varphi\colon U\cong\widehat U$. Then there is an open ball $B(\varphi(p),\varepsilon)\subseteq\widehat U$. Then $\overline{B(\varphi(p),\varepsilon/2)}\subseteq\widehat U$ is closed and bounded in $\RR^n$ and hence compact, so $\varphi^{-1}(B(\varphi(p),\varepsilon/2))$ is a subset of the compact subset $\varphi^{-1}(\overline{B(\varphi(p),\varepsilon/2)})$.
\end{proof}
Being locally compact approximately speaking allows one to understand a space by building it up from compact ones. Here is one way to do this.
\begin{definition}[exhaustion]
	Fix a topological space $X$. Then an \textit{exhaustion} of $X$ is a sequence $\{K_i\}_{i\in\NN}$ of compact subsets of $X$ satisfying the following.
	\begin{itemize}
		\item Ascending: $K_0\subseteq K_1\subseteq\cdots$.
		\item Covers: $X=\bigcup_{i\in\NN}K_i$.
		\item Not too close: $K_i\subseteq K_{i+1}^\circ$.
	\end{itemize}
\end{definition}
\begin{example}
	The space $\RR^n$ has an exhaustion by $K_i\coloneqq B(0,i)$.
\end{example}
Here is a way to build an exhaustion.
\begin{proposition} \label{prop:get-exhaustion}
	Fix a topological space $X$. If $X$ is second-countable, locally compact, and Hausdorff. Then $X$ has an exhaustion. In particular, topological $n$-manifolds have an exhaustion.
\end{proposition}
\begin{proof}
	The second claim follows from the first by \Cref{prop:man-is-loc-cmpt} and the definition of a manifold. So we will focus on showing the first claim.

	Fix a countable base $\mathcal B$ of $X$, and let $\mathcal B'$ be the subcollection of precompact open base elements. Quickly, we note that $\mathcal B'$ is still a base: certainly everything in $\mathcal B'$ is open, and then for any $p\in X$ and open neighborhood $U\subseteq X$, we need some $B'\in\mathcal B'$ such that $B'$ is precompact.
	
	Well, because $X$ is locally compact, there is a precompact open neighborhood $U'$ of $p$ by \Cref{rem:get-precompact}. Then $U\cap U$ is an open neighborhood of $p$, so we can find a base element $B\in\mathcal B$ containing $p$ and inside $U'\cap U$. Then $B\subseteq U'$ is precompact by \Cref{rem:smaller-precompact}.

	We now construct our exhaustion. Enumerate $\mc B=\{B_0,B_1,\ldots\}$, and we proceed as follows.
	\begin{enumerate}
		\item Set $K_0\coloneqq\ov{B_0}$, which is compact by construction of $B_0$.
		\item Now suppose we have a compact subset $K_i\subseteq X$, and we construct $K_{i+1}$. Note that $\mc B$ is an open cover of $K_i$, which can be reduced to a finite subcover, so there is some $M_{i+1}$ such that $K_i$ is covered by $\{B_i:i\le M_{i+1}\}$. We may as well suppose that $M_{i+1}\ge i+1$. Then we define
		\[K_{i+1}\coloneqq\bigcup_{i=1}^M\ov{B_i}.\]
		Note that the finite union of compact sets remains compact.
	\end{enumerate}
	The above construction produces an exhaustion. Here are our checks, which will complete the proof.
	\begin{itemize}
		\item Ascending: by construction, we see that
		\[K_{i+1}^\circ\supseteq\bigcup_{i=1}^MB_i\supseteq K_i.\]
		\item Covers: any $x\in X$ lives in some $B_i$, and by construction, we have $B_i\subseteq K_i$, so $x\in K_i$.
		\qedhere
	\end{itemize}
\end{proof}

\subsection{Paracompactness}
We will want to talk about covers in some more detail.
\begin{definition}[cover]
	Fix a topological space $X$. A \textit{cover} is a collection $\mathcal U\subseteq\mathcal P(X)$ such that
	\[X=\bigcup_{U\in\mathcal U}U.\]
\end{definition}
\begin{definition}[locally finite]
	Fix a topological space $X$. A cover $\mathcal U$ of $X$ is \textit{locally finite} if and only if any $p\in X$ has some open neighborhood $U\subseteq X$ intersecting at most finitely many elements of $\mathcal U$.
\end{definition}
\begin{definition}[refinement]
	Fix a cover $\mathcal U$ of a topological space $X$. Then a \textit{refinement} of $\mathcal U$ is a cover $\mathcal V$ such that any $V\in\mc U$ is contained in some $U\in\mc U$.
\end{definition}
And here is our definition.
\begin{definition}[paracompact]
	Fix a topological space $X$. Then $X$ is \textit{paracompact} if and only if every open cover has a locally finite open refinement.
\end{definition}
Approximately speaking, the point of desiring paracompactness is that it allows ``reducing to Euclidean'' arguments in the future will not have to deal with intersections which are infinitely bad. Anyway, here is our result.
\begin{proposition} \label{prop:get-paracompact}
	Fix a topological $n$-manifold $M$. Then $M$ is paracompact.
\end{proposition}
\begin{proof}
	In fact, we are only going to use the fact that $M$ has an exhaustion, proven in \Cref{prop:get-exhaustion}.

	Fix an open cover $\mathcal U$, and we want to produce a locally finite open refinement. To set us up, fix an exhaustion $\{K_i\}_{i\in\NN}$, which exists by \Cref{prop:get-exhaustion}, and define the following sets for each $i\in\NN$.
	\begin{itemize}
		\item For $i\ge-1$, define $V_i\coloneqq K_{i+1}\setminus K_i^\circ$, which is a closed subset of the compact set $K_{i+1}$ and hence compact \cite[Lemma~4.10]{elber-top}; take $K_{-1}=\emp$ without concern.
		\item For $i\ge0$, define $W_i\coloneqq K_{i+2}^\circ\setminus K_{i-1}$, which is open; here, take $K_{-1}=\emp$ without concern.
	\end{itemize}
	For intuition, we should think about the $W_\bullet$s as being a locally finite cover from which we will build the locally finite cover refinement of $\mathcal U$.

	For the construction, we fix some $j\ge0$ for the time being. For each $x\in V_j$, find some $U_x\in\mc U$ containing $x$. Note that $\{U_x\}_{x\in V_j}$  is an open cover of $V_j$, and because $V_j\subseteq W_j$, in fact $\{U_x\cap W_j\}_{x\in V_j}$ is an open cover. Because $V_j$ is compact, we can thus reduce this open cover to a finite subcover $\mc A_j$.

	Now letting $j$ vary, we define
	\[\mc V\coloneqq\bigcup_{j\ge0}\mc A_j.\]
	Here are our checks.
	\begin{itemize}
		\item Open cover: each $x\in X$ lives in some $K_{i+1}$ because we have an exhaustion, so lives in some $V_i$, so it lives in some open subset in $\mc A_j$, so it lives in some open subset in $\mc V$.
		\item Refinement: by construction, each open set in $\mathcal A_j$ is a subset in $\mathcal U$.
		\item Locally finite: this is essentially by construction. The main point is that any $x\in X$ lives in some $K_i$, so by choosing the least such $K_i$ places $x$ in some $V_i\subseteq W_i$. We now show that only finitely many open subsets in $\mathcal V$ intersect $W_i$. Note $W_i\subseteq K_{i+2}$, so $W_i\cap W_j=\emp$ for $j\ge i+2$. Thus, if $V\cap W_i\ne\emp$, we must have $V\in\mathcal A_j$ for $j<i+2$. But this is only finitely many indices, and each $\mathcal A_j$ is finite, so this is only finitely many candidates.
		\qedhere
	\end{itemize}
\end{proof}

\subsection{Products}
We now discuss an in-depth example.
\begin{proposition} \label{prop:product-manifold}
	Fix finitely many topological manifolds $M_1,\ldots,M_k$. Then the product
	\[M_1\times\cdots\times M_k\]
	is also a topological manifold of dimension $\dim M_1+\cdots+\dim M_k$.
\end{proposition}
We will do this via a sequence of lemmas.
\begin{lemma} \label{lem:prod-haus}
	Fix a collection of Hausdorff topological spaces $\{X_\alpha\}_{\alpha\in\Lambda}$. Then the product
	\[\prod_{\alpha\in\Lambda}X_\alpha\]
	is also Hausdorff.
\end{lemma}
\begin{proof}
	Fix distinct points $(x_\alpha)_{\alpha\in\Lambda}$ and $(y_\alpha)_{\alpha\in\Lambda}$ in the product. Then there is an index $\beta\in\Lambda$ such that $x_\beta\ne y_\beta$, so because $X_\beta$ is Hausdorff, there are disjoint open neighborhoods $U_\beta,V_\beta\subseteq X_\beta$ of $x_\beta$ and $y_\beta$, respectively. Then we define $U_\alpha=V_\alpha\coloneqq X_\alpha$ for $\alpha\ne\beta$, and we note that the open subsets
	\[\prod_{\alpha\in\Lambda}U_\alpha\qquad\text{and}\qquad\prod_{\alpha\in\Lambda}V_\alpha\]
	are disjoint open neighborhoods of $(x_\alpha)_{\alpha\in\Lambda}$ and $(y_\alpha)_{\alpha\in\Lambda}$, respectively, so we are done. (These are disjoint because any point in the intersection will have the $\beta$ coordinate in $U_\beta\cap V_\beta=\emp$.)
\end{proof}
\begin{lemma} \label{lem:prod-sec-cont}
	Fix finitely many second countable topological spaces $\{X_i\}_{i=1}^n$. Then the product
	\[\prod_{i=1}^nX_i\]
	is also second countable.
\end{lemma}
\begin{proof}
	Let the product be $X$. For each $i$, let $\mathcal B_i$ be a countable base for $X_i$. Then define
	\[\mathcal B\coloneqq\Bigg\{\prod_{i=1}^nB_i:B_i\in\mc B_i\text{ for each }i\Bigg\}.\]
	We claim that $\mc B$ is a base for the topology on the $X$. Indeed, suppose $(x_1,\ldots,x_n)\in X$ lives in some open subset $U\subseteq X$. From the standard base on $X$, we know that there are open subsets $U_i\subseteq X_i$ for each $i$ such that $(x_1,\ldots,x_n)\in U_1\times\cdots\times U_n$. Now, for each $U_i$, we note that $x_i\in U_i$ must have some $B_i\in\mc B_i$ such that $x_i\in B_i$ and $B_i\subseteq U_i$. But then
	\[(x_1,\ldots,x_n)\in B_1\times\cdots\times B_n\subseteq U,\]
	so $B_1\times\cdots\times B_n\in\mc B$ is the desired base element.
\end{proof}
We now prove \Cref{prop:product-manifold}.
\begin{proof}[Proof of \Cref{prop:product-manifold}]
	We get Hausdorff from \Cref{lem:prod-haus} and second countable from \Cref{lem:prod-sec-cont}. So it remains to check that we are locally Euclidean. For brevity, let $M$ be the product, and set $n_i\coloneqq\dim M_i$ for each $i$, and let $n\coloneqq n_1+\cdots+n_k$.

	Now, fix some point $(x_1,\ldots,x_k)\in M$. For each $i$, we get some open neighborhood $U_i\subseteq M_i$ of $x_i$ and some open $\widehat{U}_i\subseteq\RR^{n_i}$ with a homeomorphism $\varphi_i\colon U_i\cong\widehat U_i$. Now, we see that the product map
	\[(\varphi_1\times\cdots\times\varphi_k)\colon U_1\times\cdots\times U_k\to\widehat U_1\times\cdots\times\widehat U_k\]
	is still a homeomorphism, and the target is an open subset of
	\[\RR^{n_1}\times\cdots\times\RR^{n_k}\cong\RR^n,\]
	where this last homeomorphism is obtained by simply concatenating the coordinates. So we have constructed a composite homeomorphism from an open neighborhood of $(x_1,\ldots,x_k)$ to an open subset of $\RR^n$, as desired.
\end{proof}
\begin{example}
	\Cref{ex:s1-top-man} established $S^1$ as a topological $1$-manifold, so the $n$-torus
	\[T^n\coloneqq\underbrace{S^1\times\cdots\times S^1}_n\]
	is a topological $n$-manifold. Note that the covering space $p\colon\RR\to S^1$ will induce the covering space $p^n\colon\RR^n\to T^n$, so we can also view $T^n$ as $\RR^n/\ZZ^n$; in other words, we have the unsurprising homeomorphism $\RR^n/\ZZ^n\to(\RR/\ZZ)^n$.
\end{example}

\subsection{Open Submanifolds}
We proceed with a sequence of lemmas.
\begin{lemma} \label{lem:sub-haus}
	Suppose $X$ is a Hausdorff topological space. If $X'\subseteq X$ is a subspace, then $X'$ is still Hausdorff.
\end{lemma}
\begin{proof}
	Fix distinct points $p,q\in X'$. Then $X$ is Hausdorff, so there exist disjoint open neighborhoods $U,V\subseteq X$ of $p$ and $q$, respectively, so $U\cap X'$ and $V\cap X'$ are the needed disjoint open subsets of $X'$, respectively.
\end{proof}
\begin{lemma} \label{lem:sub-second-countable}
	Suppose that $X$ is a second countable topological space. Then for any subset $X'\subseteq X$, the topological (sub)space $X'$ is still second countable.
\end{lemma}
\begin{proof}
	Well, let $\mc B$ be a countable base for $X$, and we claim that the collection
	\[\mc B'\coloneqq\{B\cap X':B\in\mc B\}\]
	makes a countable base for $X'$. Note that $\mc B'$ is certainly countable because there is a surjective map $\mc B\to\mc B'$ by $B\mapsto(B\cap X')$, and $\mc B$ is countable. (This map is surjective by construction.)

	So it remains to show that $\mc B'$ is a base. Quickly, we claim that every $B'\in\mc B'$ is open in $X'$. Indeed, for any $B'\in\mc B'$, we can find some $B\in\mc B$ such that $B'=B\cap X'$. Now, $\mc B$ is a base, so $B\subseteq X$ is open, so $B'=B\cap X'$ is open in the subspace topology of $X'$.

	To finish checking that we have a base, fix some $x'\in X'$ and open $U'\subseteq X'$ containing $x'$. Then we need some $B'\in\mc B'$ such that $x'\in B'$ and $B'\subseteq U'$. Well, by the subspace topology, we can write $U'=U\cap X'$ for some open $U\subseteq X$, but then $x'\in U$, so there is some $B\in\mc B'$ such that $x'\in B$ and $B\subseteq U$. To finish, we set
	\[B'\coloneqq B\cap X',\]
	which is in $\mc B'$ by construction, and we have $x'\in B\cap X'=B'$ and $B'=B\cap X'\subseteq U\cap X'=U'$, so $B'$ is indeed the required basic open set.
\end{proof}
\begin{lemma} \label{lem:sub-loc-euclid}
	Suppose that $X$ is locally Euclidean of dimension $n$. Then for any open subset $X'\subseteq X$, the topological (sub)space $X'$ is locally Euclidean of dimension $n$.
\end{lemma}
\begin{proof}
	For any $x'\in X'$, we must find open subsets $U'\subseteq X'$ and $\widehat U'\subseteq\RR^n$ such that $x'\in U'$ and there is a homeomorphism $U'\cong\widehat U'$.

	Well, $x'\in X$, so there are open subsets $U\subseteq X$ and $\widehat U\subseteq\RR^n$ such that $x'\in U$ and there is a homeomorphism $\varphi\colon U\cong\widehat U$. Now, set
	\[U'\coloneqq U\cap X'.\]
	Then $\varphi$ is a homeomorphism, so $\varphi'\coloneqq\varphi|_{U'}$ continues to be a homeomorphism onto its image $\widehat U'\coloneqq\varphi(U')$. Indeed, the inverse of the bijection $\varphi|_{U'}\colon U'\to\widehat U'$ is $\varphi'|_{\widehat U'}$. Both of these maps are continuous by, so $\varphi|_{U'}$ is in fact a homeomorphism.
	
	Now, $U'\subseteq U$ is open, so because $\varphi$ is a homeomorphism, we see that $\varphi(U')\subseteq\widehat U$ is open: $\varphi(U')$ is the pre-image of the open subset $U'\subseteq U$ under the continuous map $\varphi^{-1}\colon\widehat U\to U$, so $\varphi(U')$ being open follows. Continuing, because $\widehat U\subseteq\RR^n$ is open, we conclude that $\widehat U'\subseteq\RR^n$ is open.\footnote{Namely, an open subset of an open subset $U$ is still an open subset. This sentence has some content because the larger open subset uses the subspace topology; the proof simply notes that being open in $U$ is equivalent to being the intersection of an open subset and $U$, which is open because finite intersections of open subsets continues to be open.} So $U'\subseteq X'$ is open (by the subspace topology), contains $x'$, and it is homeomorphic to an open subset $\widehat U'$ of $\RR^n$.
\end{proof}
\begin{proposition} \label{prop:sub-top-man}
	Fix a topological $n$-manifold $M$. For any nonempty open subset $U\subseteq M$, we have that $U$ is a topological $n$-manifold.
\end{proposition}
\begin{proof}
	Combine \Cref{lem:sub-haus,lem:sub-second-countable,lem:sub-loc-euclid}.
\end{proof}

\subsection{Charts}
The construction of our smooth structure will arise from more carefully understanding how a manifold is locally Euclidean. This arises from charts.
\begin{definition}[chart]
	Fix a topological $n$-manifold $M$. Then a \textit{coordinate chart} or just \textit{chart} is a pair $(U,\varphi)$ where $U\subseteq M$ is open and $\varphi\colon U\cong\widehat U$ is a homeomorphism where $\widehat U\subseteq\RR^n$ is open.
\end{definition}
Essentially, the content of $M$ being locally Euclidean is that it has an open cover by open subsets belonging to some chart. The reason we call it a chart is that we are (approximately speaking) providing ``local coordinates'' to an open subset of $M$.
\begin{definition}[coordinate function]
	Fix a chart $(U,\varphi)$ if a topological $n$-manifold $M$. Then we may write
	\[\varphi(p)\coloneqq\left(x^1(p),\ldots,x^n(p)\right)\in\RR^n\]
	for each $p\in M$. We call these functions $x^\bullet\colon U\to\RR$ the \textit{coordinate functions}.
\end{definition}
Note that these coordinate functions are continuous because they are simply the continuous function $\varphi$ composed with the projection $\RR^n\to\RR$.
\begin{example} \label{ex:graph-top-man}
	Fix an open subset $V\subseteq\RR^m$, and let $F\colon V\to\RR^n$ be a continuous function. Then the graph
	\[\Gamma\coloneqq\{(x,F(x)):x\in V\}\subseteq\RR^m\times\RR^n\]
	is a topological $n$-manifold. Because we are already a subspace of $\RR^m\times\RR^n\cong\RR^{m+n}$, we see that $\Gamma$ is also Hausdorff and second countable. (Subspaces inherit being Hausdorff directly, and we inherit being second countable by using the intersection of the given countable base.)
	
	The main content comes from being locally Euclidean. Namely, there is a projection map $\pi\colon\Gamma\to V$ by $(x,y)\mapsto x$ which in fact is a homeomorphism (it's continuous inverse is $({\id}\times F)\colon x\mapsto(x,F(x))$). So we have the single chart $(V,\pi)$, which establishes being a topological $n$-manifold.
\end{example}

\end{document}