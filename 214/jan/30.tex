% !TEX root = ../notes.tex

\documentclass[../notes.tex]{subfiles}

\begin{document}

Here we go.

\subsection{Smooth Manifolds with Boundary}
We would like for the boundary of a smooth manifold with boundary to make sense.
\begin{theorem} \label{thm:boundary-makes-sense}
	Fix a smooth $n$-manifold with boundary $M$, and fix some $p\in M$. Given two charts $(U,\varphi)$ and $(V,\psi)$ with $p\in U\cap V)$, then $\varphi(p)\in\del\HH^n$ if and only if $\psi(p)\in\del\HH^n$.
\end{theorem}
\begin{proof}
	Suppose this is not the case. Then, up to rearranging, we get $\varphi(p)\in(\HH^n)^\circ$ and $\psi(p)\in\del\HH^n$. Our transition maps are smooth, so we have produced a diffeomorphism from the open subsets $U'\subseteq\HH^n$ and $V'\subseteq\HH^n$ such that $U'\cap\del\HH^n=\emp$ but $V'\cap\del\HH^n\ne\emp$. Now, for smoothness, the transition map $\tau\colon V'\to U'$ must have an extension $\widetilde\tau\colon\widetilde V'\to\widetilde U'$. But then $\widetilde\tau$ is an invertible map, so the Inverse function theorem implies that $\tau$ is locally invertible and in particular must be an open map. But $V'$ goes to $U'$, which is not open in $\RR^n$, so we have our contradiction.
\end{proof}
\begin{remark}
	In fact,
	\[\psi\circ\varphi^{-1}|_{\del\HH^n\cap\varphi(U\cap V)}\colon\left(\del\HH^n\cap\varphi(U\cap V)\right)\to\left(\del\HH^n\cap\psi(U\cap V)\right)\]
	is a smooth transition map, though we will not check this here.
\end{remark}
\begin{remark}
	People in the modern day might allow $\del M$ to be a manifold with boundary itself, which is a ``manifold with corners.''
\end{remark}
\begin{remark}
	One can remove the smoothness assumption here as well, but it will require some cohomology or similar.
\end{remark}
The boundary/interior for a smooth manifold may not actually be its boundary/interior when embedded into a space.
\begin{example}
	Consider $M\coloneqq\left\{x\in\RR^n:x_n>0\right\}$. Then $M$ is a smooth manifold with boundary, but $\del M=\left\{x\in\RR^n:x_n=0\right\}$ when viewed as a subset of $\RR^n$.
\end{example}
\begin{example}
	Consider $M=S^n\subseteq\RR^{n+1}$. Then $M$ is a smooth manifold (without boundary), but as a subspace of $\RR^{n+1}$, we have $\del M=M$.
\end{example}
\begin{example}
	Consider $M\coloneqq\HH^n\cap B(0,1)$. Then $M$ is a smooth manifold whose boundary (as a manifold) is $\del\HH^n\cap B(0,1)$, but the topological boundary is $\del\HH^n\cup(\del B(0,1)\cap\HH^n)$.
\end{example}

\subsection{Smooth Maps to \texorpdfstring{$\RR^n$}{ Rn}}
We will define smooth maps in steps. To begin, we say what it means to have a smooth map $M\to\RR^n$. Basically, we look locally at the points on our manifold and check smoothness on charts.
\begin{definition}[smooth]
	Fix a smooth manifold $M$, possibly with boundary. Then a function $f\colon M\to\RR^m$ is \textit{smooth} if and only if each $p\in M$ has some smooth chart $(U,\varphi)$ with $p\in U$ and
	\[f|_U\circ\varphi|_U^{-1}\]
	is a smooth map $\varphi(U)\to\RR^m$.
\end{definition}
\begin{example}
	Any smooth map $f\colon U\to\HH^m$, where $U\subseteq\HH^n$ is open, is smooth in the above sense. Indeed, $U$ as an $n$-manifold has a smooth atlas given by $\{(U,{\id_U})\}$, and this witnesses the smoothness of $f$ for any $p\in U$.
\end{example}
Here is a quick sanity check: the charts don't matter.
\begin{lemma}
	Fix a smooth map $f\colon M\to\HH^m$, where $M$ is a smooth manifold, possibly with boundary. For any smooth chart $(V,\psi)$, the composition $f|_U\circ\varphi|_U^{-1}$ is smooth.
\end{lemma}
\begin{proof}
	This is a matter of tracking through all the definitions. Fix some $p\in V$, and we would to test smoothness around $p$. Well, $p$ has some smooth chart $(U,\varphi)$ such that $p\in U$ and $f|_U\circ\varphi|_U^{-1}$ is smooth. But now we write
	\[f|_{U\cap V}\circ\psi|_{U\cap V}^{-1}=\left(f|_{U\cap V}\circ\varphi|_{U\cap V}^{-1}\right)\circ\left(\varphi|_{U\cap V}\circ\psi|_{U\cap V}^{-1}\right),\]
	which is the composition of smooth maps: the left map is smooth by construction of $(U,\varphi)$, and the right map is smooth by compatibility of smooth charts.
\end{proof}
We are now ready to define smooth maps between manifolds. Approximately speaking, we simply add in a check locally on the target.
\begin{definition}[smooth]
	Fix smooth manifolds $M$ and $N$, possibly with boundary. A map $F\colon M\to N$ is \textit{smooth} if and only if each $p\in M$ has smooth charts $(U,\varphi)$ and $(V,\psi)$ such that $p\in U$ and $F(U)\subseteq V$ and the composite
	\[\psi\circ F|_U\circ\varphi|_U^{-1}\]
	is a smooth map $\HH^m\to\HH^n$. We may call the above composite a \textit{coordinate representation}.
\end{definition}
\begin{example}
	Any smooth map $F\colon U\to V$, where $U\subseteq\RR^m$ and $V\subseteq\RR^n$ are open, is smooth in the above sense. Indeed, $U$ and $V$ have smooth atlases given by $\{(U,{\id_U})\}$ and $\{(V,{\id_V})\}$ (respectively), and these charts witness that $F$ is smooth at each $p\in U$ because the composite
	\[{\id_V}\circ F\circ{\id_U}^{-1}=F\]
	is smooth by hypothesis.
\end{example}
Here's the same sanity check: the charts don't matter.
\begin{lemma}
	Fix a smooth map $F\colon M\to N$ of manifolds, possibly with boundary. If $(U,\varphi)$ and $(V,\psi)$ are smooth charts on $M$ and $N$, respectively, and $F(U)\subseteq V$, then the composite $\psi\circ F|_U\circ\varphi|_U^{-1}$ is smooth.
\end{lemma}
\begin{proof}
	Again, we track through locally, tracking through all the definitions. To check that $\psi\circ F|_U\circ\varphi|_U^{-1}$ is smooth, it suffices to check it an open cover of $\varphi(U)$. Pick $\varphi(p)\in\varphi(U)$ where $p\in U$, and we know that we have smooth charts $(U_p,\varphi_p)$ and $(V_p,\psi_p)$ in $M$ and $N$, respectively, such that $F(U_p)\subseteq V_p$ and the composite $\psi_p|_{F(U_p)}\circ F|_{U_p}\circ\varphi_p|_{U_p}^{-1}$ is smooth. Then we see that
	\[\psi\circ F|_U\circ\varphi|_U^{-1}=\left(\psi|_{V\cap V_p}\circ\psi_p|_{V\cap V_p}^{-1}\right)\circ\left(\psi_p\circ F|_U\circ\varphi_p|_{U_p}^{-1}\right)|_{\varphi_p(U\cap U_p)}\circ\left(\varphi_p|_{U\cap U_p}\circ\varphi|_{U\cap U_p}^{-1}\right)\]
	is smooth, where the left and right maps are smooth by smooth compatibility, and the middle map is smooth by construction.
\end{proof}
\begin{remark}
	One can write out the above proof diagrammatically by noting that having smooth charts $(U,\varphi)$ and $(U',\varphi')$ of $M$ and smooth charts $(V,\psi)$ and $(V',\psi')$ of $N$ such that $F(U)\subseteq V$ and $F(U')\subseteq V'$ will have the following diagram.
	% https://q.uiver.app/#q=WzAsNCxbMCwwLCJcXHZhcnBoaShVKSJdLFswLDEsIlxcdmFycGhpJyhVJykiXSxbMSwwLCJcXHBzaShWKSJdLFsxLDEsIlxccHNpJyhWJykiXSxbMSwwLCIiLDAseyJzdHlsZSI6eyJ0YWlsIjp7Im5hbWUiOiJhcnJvd2hlYWQifX19XSxbMywyLCIiLDAseyJzdHlsZSI6eyJ0YWlsIjp7Im5hbWUiOiJhcnJvd2hlYWQifX19XSxbMSwzLCJcXHBzaSdcXGNpcmMgRlxcY2lyYyhcXHZhcnBoaScpXnstMX0iLDJdLFswLDIsIlxccHNpXFxjaXJjIEZcXGNpcmNcXHZhcnBoaV57LTF9Il1d&macro_url=https%3A%2F%2Fraw.githubusercontent.com%2FdFoiler%2Fnotes%2Fmaster%2Fnir.tex
	\[\begin{tikzcd}
		{\varphi(U)} & {\psi(V)} \\
		{\varphi'(U')} & {\psi'(V')}
		\arrow[tail reversed, from=2-1, to=1-1]
		\arrow[tail reversed, from=2-2, to=1-2]
		\arrow["{\psi'\circ F\circ(\varphi')^{-1}}"', from=2-1, to=2-2]
		\arrow["{\psi\circ F\circ\varphi^{-1}}", from=1-1, to=1-2]
	\end{tikzcd}\]
	Here, the vertical maps are only defined on the corresponding intersections, but it is smooth when defined by the smooth compatibility.
\end{remark}
\begin{remark}
	Please read more of chapter 2 to get helpful properties of smooth maps.
\end{remark}

\subsection{Partition of Unity}
By way of motivation, suppose we have two smooth functions $f,g\colon\RR\to\RR$, and we want to build a smooth function $h\colon\RR\to\RR$ such that $f|_{(-\infty,-1)}=h|_{(-\infty,-1)}$ and $g|_{(1,\infty)}=h|_{(1,\infty)}$. One way to do this is to find smooth functions $\varphi,\psi\colon\RR\to\RR$ such that 
\[\begin{cases}
	\varphi|_{(-\infty,-1)}=1, \\
	\varphi|_{(1,\infty)}=0.
\end{cases}\]
Then $h\coloneqq\varphi f+(1-\varphi)g$ is smooth by construction, and it satisfies the restriction conditions also by construction. This idea of ``splitting up the $1$ function'' is known as partition of unity.
\begin{definition}[partition of unity]
	Fix a topological space $X$, and let $\{U_\alpha\}_{\alpha\in\kappa}$ be an open cover on $M$. Then a \textit{partition of unity subordinate to $\{U_\alpha\}_{\alpha\in\kappa}$} is a collection of continuous functions $\{\varphi_\alpha\}_{\alpha\in\kappa}$ on $X$ satisfying the following.
	\begin{itemize}
		\item $\im\varphi_\alpha\subseteq[0,1]$ always.
		\item $\op{supp}\varphi_\alpha\subseteq U_\alpha$ for each $\alpha$.
		\item The collection $\{\op{supp}\varphi_\alpha\}_{\alpha\in\kappa}$ is locally finite.
		\item For each $x\in X$, we have
		\[\sum_{\alpha\in\kappa}\varphi_\alpha(x)=1.\]
	\end{itemize}
\end{definition}
Of course, we must show that these exist. We will do this for manifolds.
\begin{theorem}
	Fix a smooth manifold $M$, possibly with boundary. For any open cover $\{U_\alpha\}_{\alpha\in\kappa}$, there is a partition of unity $\{\varphi_\alpha\}_{\alpha\in\kappa}$ (of smooth functions) subordinate to $\{U_\alpha\}_{\alpha\in\kappa}$.
\end{theorem}
\begin{proof}
	We begin by constructing smooth functions $\{\widetilde\varphi\}_{\alpha\in\kappa}$ satisfying the following constraints.
	\begin{itemize}
		\item $\im\widetilde\varphi_\alpha\subseteq[0,\infty)$.
		\item $\op{supp}\widetilde\varphi_\alpha\subseteq U_\alpha$.
		\item The collection $\{\op{supp}\widetilde\varphi_\alpha\}_{\alpha\in\kappa}$ is a locally finite open cover of $M$.
	\end{itemize}
	Dividing out by the summation of the $\widetilde\varphi_\bullet$s completes the proof; notably the sum is a smooth function vanishing nowhere on $M$, which can be checked locally.

	Approximately speaking, by refining, we may immediately assume that each $U_\alpha$ consists of coordinate balls and is a full-fledged basis of the topology on $M$, and \Cref{prop:get-paracompact} allows us to assume that we have a locally finite cover. Then on each coordinate ball, we let $\varphi_\alpha$ be some kind of bump function.
\end{proof}

\end{document}