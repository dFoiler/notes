% !TEX root = ../notes.tex

\documentclass[../notes.tex]{subfiles}

\begin{document}

Here we go.

\subsection{Smooth Manifolds with Boundary}
We would like for the boundary of a smooth manifold with boundary to make sense.
\begin{theorem} \label{thm:boundary-makes-sense}
	Fix a smooth $n$-manifold with boundary $M$, and fix some $p\in M$. Given two charts $(U,\varphi)$ and $(V,\psi)$ with $p\in U\cap V)$, then $\varphi(p)\in\del\HH^n$ if and only if $\psi(p)\in\del\HH^n$.
\end{theorem}
\begin{proof}
	Suppose this is not the case. Then, up to rearranging, we get $\varphi(p)\in(\HH^n)^\circ$ and $\psi(p)\in\del\HH^n$. Our transition maps are smooth, so we have produced a diffeomorphism from the open subsets $U'\subseteq\HH^n$ and $V'\subseteq\HH^n$ such that $U'\cap\del\HH^n=\emp$ but $V'\cap\del\HH^n\ne\emp$. Now, for smoothness, the transition map $\tau\colon V'\to U'$ must have an extension $\widetilde\tau\colon\widetilde V'\to\widetilde U'$. But then $\widetilde\tau$ is an invertible map, so the Inverse function theorem implies that $\tau$ is locally invertible and in particular must be an open map. But $V'$ goes to $U'$, which is not open in $\RR^n$, so we have our contradiction.
\end{proof}
\begin{remark}
	In fact,
	\[\psi\circ\varphi^{-1}|_{\del\HH^n\cap\varphi(U\cap V)}\colon\left(\del\HH^n\cap\varphi(U\cap V)\right)\to\left(\del\HH^n\cap\psi(U\cap V)\right)\]
	is a smooth transition map, though we will not check this here.
\end{remark}
\begin{remark}
	People in the modern day might allow $\del M$ to be a manifold with boundary itself, which is a ``manifold with corners.''
\end{remark}
\begin{remark}
	One can remove the smoothness assumption here as well, but it will require some cohomology or similar.
\end{remark}
The boundary/interior for a smooth manifold may not actually be its boundary/interior when embedded into a space.
\begin{example}
	Consider $M\coloneqq\left\{x\in\RR^n:x_n>0\right\}$. Then $M$ is a smooth manifold with boundary, but $\del M=\left\{x\in\RR^n:x_n=0\right\}$ when viewed as a subset of $\RR^n$.
\end{example}
\begin{example}
	Consider $M=S^n\subseteq\RR^{n+1}$. Then $M$ is a smooth manifold (without boundary), but as a subspace of $\RR^{n+1}$, we have $\del M=M$.
\end{example}
\begin{example}
	Consider $M\coloneqq\HH^n\cap B(0,1)$. Then $M$ is a smooth manifold whose boundary (as a manifold) is $\del\HH^n\cap B(0,1)$, but the topological boundary is $\del\HH^n\cup(\del B(0,1)\cap\HH^n)$.
\end{example}

\end{document}