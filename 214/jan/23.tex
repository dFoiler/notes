% !TEX root = ../notes.tex

\documentclass[../notes.tex]{subfiles}

\begin{document}

The first homework is due on Thursday. Today we discuss smooth structures.

\subsection{Examples of Topological Manifolds}
Let's provide a few more examples of topological manifolds.
\begin{exe}[sphere] \label{exe:sphere-top-man}
	We show that the $n$-sphere $S^n\subseteq\RR^{n+1}$ is a topological $n$-manifold.
\end{exe}
\begin{proof}
	Explicitly, for each $i\in\{1,\ldots,n+1\}$, we define
	\[U_i^\pm\coloneqq\{(x_1,\ldots,x_{n+1})\in S^n:\pm x_i>0\},\]
	which has a projection $\pi_i^\pm\colon U_i^\pm\to B(0,1)$ (for $B(0,1)\subseteq\RR^n$) given by erasing the $x_i$ coordinate. One can show that the $\pi_i^\pm$ are all homeomorphisms---certainly, it is continuous, and the inverse map is given by
	\[(x_1,\ldots,x_n)\coloneqq\left(x_1,\ldots,x_{i-1},\pm\sqrt{1-\left(x_1^2+\cdots+x_n^2\right)},x_i,\ldots,x_n\right),\]
	which is also continuous. (We won't bother checking that the maps are mutually inverse.) Lastly, we note that the $U_i^\pm$ is an open cover of $S^n$ because any point in $S^n$ has some nonzero coordinate, and this nonzero coordinate will have a sign.
\end{proof}
\begin{exe}[projective space]
	Define the space $\mathbb{RP}^n$ as ``lines in $\RR^{n+1}$'': it consists of equivalence classes of nonzero points in $\RR^{n+1}\setminus\{0\}$, where $x\sim y$ if and only if there is some $\lambda\in\RR^\times$ such that $x=\lambda y$. We show that $\RP^n$ is a topological $n$-manifold.
\end{exe}
\begin{proof}
	For notation, we let $[x_0:\cdots:x_n]$ denote the equivalence class of $(x_1,\ldots,x_n)$ in $\RP^n$. Note there is a projection $p\colon\left(\RR^{n+1}\setminus\{0\}\right)\to\RP^n$, and we give $\RP^n$ the induced (quotient) topology from $\RR^{n+1}\setminus\{0\}$.
	
	By \Cref{lem:get-second-countable}, to achieve second countable, it suffices to provide a finite open cover by open subsets homeomorphic to open subsets of $\RR^n$; this will also achieve locally Euclidean. Well, define
	\[U_i\coloneqq\left\{[x_0:\cdots:x_n]\in\RP^n:x_i\ne0\right\}.\]
	Note that the pre-image in $\RR^{n+1}\setminus\{0\}$ consists of the $(x_0,\ldots,x_n)\in\RR^{n+1}\setminus\{0\}$ with $x_i\ne0$, so $U_i\subseteq\RP^n$ is open. Now, by scaling, we can write elements of $U_i$ uniquely as $[y_0:\cdots:y_n]$ with $y_i=1$, which provides the required element in $\RR^n$. Explicitly, we define $\varphi_i\colon U_i\to\RR^n$ by
	\[\varphi_i\colon[x_0:\cdots:x_n]\mapsto\left(\frac{x_0}{x_i},\ldots,\widehat{\frac{x_i}{x_i}},\ldots,\frac{x_n}{x_i}\right).\]
	One sees that $\varphi_i$ is continuous: by the quotient topology, we are trying to show that $\varphi_i\circ\pi\colon\pi^{-1}U_i\to\RR^n$ is just $(x_0,\ldots,x_n)\mapsto(x_0/x_i,\ldots,\widehat{x_i/x_i},\ldots,x_n/x_i)$, which is continuous, so $\varphi_i$ is continuous because $\RP^n$ has the quotient topology. Lastly, one notes that the inverse of $\varphi_i$ is given by $(x_0,\ldots,\widehat x_i,\ldots,x_n)\mapsto[x_0:\ldots:x_{i-1}:1:x_{i+1}:\ldots:x_n]$, which is continuous because it is the composite of the map $\RR^n\to\RR^{n+1}$ given by $(x_0,\ldots,\widehat x_i,\ldots,x_n)\mapsto[x_0:\ldots:x_{i-1}:1:x_{i+1}:\ldots:x_n]$ and the projection $p\colon\left(\RR^{n+1}\setminus\{0\}\right)\to\RP^n$.

	Lastly, we show that $\RP^n$ is Hausdorff. Doing this in a slick way is surprisingly obnoxious. We claim that there is a $2$-to-$1$ covering space map
	\[p\colon S^n\to\RP^n.\]
	To see why this implies that $\RP^n$ is Hausdorff, fix two distinct points $x,y\in\RP^n$. Then there are lifts $x_1,x_2\in S^n$ of $x$ and $y_1,y_2\in S^n$. Because $S^n$ is already Hausdorff (it's a subspace of $\RP^n$), we can find disjoint open subsets $U_1,U_2,V_1,V_2\subseteq S^n$ around $x_1,x_2,y_1,y_2\in S^n$ respectively, and we can make them all small enough so that $p$ is a local homeomorphism. Then $p(U_1)\cap p(U_2)$ and $p(V_1)\cap p(V_2)$ are the desired open subsets.

	So we are left showing that we have a double cover $p$. The map is given by the composite
	\[S^n\subseteq\left(\RR^{n+1}\setminus\{0\}\right)\onto\RP^n,\]
	which we see is continuous automatically. To see that this is a $2$-to-$1$ local homeomorphism, we note that the pre-image of the standard open subset $U_i\subseteq\RP^n$ is
	\[\left\{(x_0,\ldots,x_n)\in\RR^{n+1}:x_i\ne0\right\},\]
	whose pre-image in $S^n$ splits into the two open subsets $U_i^\pm$. So we have our continuous map $U_i^+\sqcup U_i^-\to U_i$; it remains to show that $U_i^\pm\to U_i$ is a homeomorphism. We may as well assume $i=0$; then the inverse map is given by sending $[1:x_1:\cdots:x_n]$ to the point on the hemisphere of $S^n$ on this line, which is
	\[\pm\frac x{\left|x\right|},\]
	where the sign depends on $U_i^\pm$. This is continuous, so we are done.
\end{proof}
\begin{remark}
	Note $S^n$ is continuous, so the surjectivity of the covering space map $S^n\onto\RP^n$ implies that $\RP^n$ is compact.
\end{remark}

\subsection{Transition Functions}
Defining smooth structures will come out of transition maps between coordinate charts.
\begin{definition}[transition map]
	Fix charts $(U,\varphi)$ and $(V,\psi)$ on a topological $n$-manifold $M$. Then the \textit{transition map} is the map
	\[\psi\circ\varphi^{-1}\colon\varphi(U\cap V)\to\psi(U\cap V).\]
	Here, we are abusing notation a little: in order to make sense of $\psi\circ\varphi^{-1}$, we really want to work with the restrictions as $\psi|_{U\cap V}\circ(\varphi|_{U\cap V})^{-1}$.
\end{definition}
\begin{remark}
	Note $\varphi(U\cap V),\psi(U\cap V)\subseteq\RR^n$, so this is a homeomorphism from an open subset of $\RR^n$ to another open subset of $\RR^n$. Namely, $\varphi|_{U\cap V}$ and $\psi|_{U\cap V}$ are both homeomorphisms, so the above composition is still a homeomorphism.
\end{remark}
\begin{example}[polar coordinates]
	Consider the topological $2$-manifold $M\coloneqq\RR^2$. There is the identity chart $\id_M\colon M\to\RR^2$, and there is also ``polar coordinates'' on $U\coloneqq\RR^2\setminus(\RR_{\ge0}\times\{0\})$ with chart $\varphi\colon U\to\RR_+\times(0,\pi)$ defined by
	\[\varphi((x,y))\coloneqq\left(\sqrt{x^2+y^2},\arg(x,y)\right),\]
	where $\arg(x,y)$ is the angle of $(x,y)$ with the positive $x$-axis. Note the inverse map of $\varphi$ is given by $(r,\theta)\mapsto(r\cos\theta,r\sin\theta)$, so $\varphi$ is in fact a homeomorphism.

	Now, the transition map $\psi\circ\varphi^{-1}$ sends
	\[(r,\theta)\stackrel{\varphi^{-1}}\mapsto(r\cos\theta,r\sin\theta)\stackrel\psi\mapsto(r\cos\theta,r\sin\theta).\]
\end{example}
\begin{example}
	Consider the topological $2$-manifold $M\coloneqq S^2$ from \Cref{exe:sphere-top-man}. We compute the transition maps between $\varphi_1^+$ and $\varphi_3^+$, which overlap on the open set consisting of $(x_1,x_2,x_3)\in S^2$ such that $x_1,x_3>0$. Well, we can directly compute that $\varphi_3^+\circ\left(\varphi_1^+\right)^{-1}$ is given by
	\[(x_2,x_3)\stackrel{(\varphi_1^+)^{-1}}\to\left(\sqrt{1-x_2^2-x_3^2},x_2,x_3\right)\stackrel{\varphi_3^+}\to\left(\sqrt{1-x_2^2-x_3^2},x_2\right).\]
\end{example}
In the above examples, we can note that the maps between the Euclidean smooths are smooth on their domains. This becomes our notion of smoothness.
\begin{definition}[smoothly compatible]
	Two charts $(U,\varphi)$ and $(V,\psi)$ of a topological manifold $M$ are \textit{smoothly compatible} if and only if both transition maps $\psi\circ\varphi^{-1}$ and $\varphi\circ\psi^{-1}$ are smooth (i.e., infinitely differentiable). Notably, this condition is vacuously satisfied if $U\cap V=\emp$.
\end{definition}

\subsection{Smooth Structures}
We would like to cover $M$ with smoothly compatible charts, so it will be helpful to have a language for such covers.
\begin{definition}[atlas]
	Fix a topological manifold $M$. An \textit{atlas} $\mc A$ is a collection of charts ``covering $M$'' in the sense that
	\[M=\bigcup_{(U,\varphi)}U.\]
	An atlas is \textit{smooth} if and only if its charts are pairwise smoothly compatible. A smooth atlas is \textit{maximal} if and only if it is maximal in the sense of inclusion by smooth atlases.
\end{definition}
The point of using a maximal atlas is that we would like a way to say when two atlases provide the same smooth structure for a topological manifold, but it will turn out to be easier to provide a ``unique'' atlas to look at, which will be the maximal smooth atlas. Quickly, we note that maximal smooth atlases exist. One could argue this by Zorn's lemma, but we don't have to.
\begin{proposition} \label{prop:get-max-smooth-atlas}
	Fix a topological $n$-manifold $M$. Any smooth atlas $\mc A$ is contained in a unique maximal smooth atlas, denoted $\ov{\mc A}$.
\end{proposition}
\begin{proof}
	We have to show existence and uniqueness. We will construct this directly: define $\ov{\mc A}$ to be the collection of charts $(U,\varphi)$ which is smoothly compatible with each chart in $\mc A$. We show that $\ov{\mc A}$ is a maximal smooth atlas.
	\begin{itemize}
		\item Atlas: certainly $\ov{\mc A}\supseteq\mc A$, so $\ov{\mc A}$ covers $M$, so $\ov{\mc A}$ is an atlas.

		\item Smooth: fix any charts $(U_1,\varphi_1),(U_2,\varphi_2)\in\ov{\mc A}$, and we would like to show that they are smoothly compatible. If $U_1\cap U_2=\emp$, there is nothing to do, so we may assume that the intersection is nonempty. By symmetry, it will be enough to show that $\varphi_2\circ\varphi_1^{-1}$ is smooth.
		
		The point is that differentiability is a local notion: explicitly, fix some $q\in\varphi_1(U_1\cap U_2)$, and we want to show that $\varphi_2\circ\varphi_1^{-1}$ is smooth at $q$. This can be checked on a small open neighborhood of $q$; in particular, find the $p\in U_1\cap U_2$ such that $\varphi_1(p)=q$, and we can find some chart $(V,\psi)\in\mc A$ such that $p\in V$. Then we note that
		\[\varphi_2|_{U_1\cap U_2\cap V}\circ(\varphi_1|_{U_1\cap U_2\cap V})^{-1}=\left(\varphi_2|_{U_1\cap U_2\cap V}\circ(\psi|_{U_1\cap U_2\cap V})^{-1}\right)\circ\left(\psi|_{U_1\cap U_2\cap V}\circ(\varphi_1|_{U_1\cap U_2\cap V})^{-1}\right)\]
		is smooth on $\varphi_1(U_1\cap U_2\cap V)$ as it is the composition of smooth maps. So our left-hand side is smooth on $U_1\cap U_2\cap V$ and in particular at $q\in\varphi_1(U_1\cap U_2\cap V)$.

		\item Maximal: suppose $\mc A'$ is a smooth atlas containing $\mc A$. We must show that $\mc A'\subseteq\ov{\mc A}$; by supposing further that $\mc A'$ contains $\ov{\mc A}$, we achieve the maximality of $\ov{\mc A}$. Well, for each $(U,\varphi)\in\mc A'$, we see that $(U,\varphi)$ is smoothly compatible with each chart in $\mc A$, so $(U,\varphi)\in\ov{\mc A}$. Thus, $(U,\varphi)\in\ov{\mc A}$, so $\mc A'\subseteq\ov{\mc A}$.

		\item Unique: suppose $\mc A'$ is a maximal smooth atlas containing $\mc A$. Then the previous point establishes that $\mc A'\subseteq\ov{\mc A}$, but then we must have equality because $\mc A'$ is a maximal smooth atlas.
		\qedhere
	\end{itemize}
\end{proof}
So we may make the following definition.
\begin{definition}[maximal smooth atlas]
	Fix a topological $n$-manifold $M$. Given a smooth atlas $\mc A$ on $M$, we let $\ov{\mc A}$ denote the unique maximal smooth atlas containing $\mc A$, which we know exists and is unique by \Cref{prop:get-max-smooth-atlas}.
\end{definition}
\begin{corollary}
	Fix a topological $n$-manifold $M$. Given smooth atlases $\mc A_1$ and $\mc A_2$ such that $\mc A_1\cup\mc A_2$ is still a smooth atlas, then
	\[\ov{\mc A_1}=\ov{\mc A_2}.\]
\end{corollary}
\begin{proof}
	Define $\mc A\coloneqq\mc A_1\cup\mc A_2$. Then $\ov{\mc A}$ is a maximal smooth atlas containing $\mc A$ and hence both $\mc A_1$ and $\mc A_2$, so we see that $\ov{\mc A_1}=\ov{\mc A}$ and $\ov{\mc A_2}=\ov{\mc A}$. Notably, we are using the uniqueness of \Cref{prop:get-max-smooth-atlas}.
\end{proof}
At long last, here is our definition.
\begin{definition}[smooth manifold]
	Fix a topological $n$-manifold $M$. A \textit{smooth structure} on $M$ is a maximal smooth atlas on $M$. A \textit{smooth $n$-manifold} is a pair $(M,\mc A)$, where $\mc A$ is a smooth structure on $M$.
\end{definition}
\begin{remark}
	Adjusting the ``smoothness'' on the manifold $M$ produces different notions of manifold. For example, we can have twice differentiable manifolds, real analytic manifolds, complex manifolds, etc.
\end{remark}

\end{document}