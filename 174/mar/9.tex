% !TEX root = ../notes.tex

The fun continues but now in a different form.

\subsection{Products}
Let's do some examples to start because we are feeling kind today.
\begin{exe} \label{exe:produp0}
	We consider products $A\times B$ in $\mathrm{Set}$, defined as
	\[A\times B:=\{(a,b):a\in A\text{ and }b\in B\}\]
	in a categorical sense.
\end{exe}
\begin{remark}
	It turns out that products are limits.
\end{remark}
\begin{proof}
	We would like to make this definition more fit for category theory, for which we note that we have projection maps $\pi_A:A\times B\to A$ and $\pi_B:A\times B\to B$ by $\pi_A:(a,b)\mapsto a$ and $\pi_B:(a,b)\mapsto b$ respectively. In fact, we are universal in the following sense: for any object $A$ and maps $\varphi_A:C\to A$ and $\varphi_B:C\to B$, there is a unique (!) map $\varphi:C\to A\times B$ making the following diagram commute.
	% https://q.uiver.app/?q=WzAsNCxbMSwwLCJDIl0sWzEsMSwiQVxcdGltZXMgQiJdLFswLDIsIkEiXSxbMiwyLCJCIl0sWzAsMiwiXFx2YXJwaGlfQSIsMix7ImN1cnZlIjoyfV0sWzAsMywiXFx2YXJwaGlfQiIsMCx7ImN1cnZlIjotMn1dLFsxLDIsIlxccGlfQSJdLFsxLDMsIlxccGlfQiIsMl0sWzAsMSwiXFx2YXJwaGkiLDAseyJzdHlsZSI6eyJib2R5Ijp7Im5hbWUiOiJkYXNoZWQifX19XV0=
	\[\begin{tikzcd}
		& C \\
		& {A\times B} \\
		A && B
		\arrow["{\varphi_A}"', curve={height=12pt}, from=1-2, to=3-1]
		\arrow["{\varphi_B}", curve={height=-12pt}, from=1-2, to=3-3]
		\arrow["{\pi_A}", from=2-2, to=3-1]
		\arrow["{\pi_B}"', from=2-2, to=3-3]
		\arrow["\varphi", dashed, from=1-2, to=2-2]
	\end{tikzcd}\]
	To see uniqueness, we note that we must have
	\[\pi_A(\varphi(c)):=\varphi_A(c)\qquad\text{and}\qquad\pi_B(\varphi(c)):=\varphi_B(c),\]
	so we must define our map $\varphi$ as
	\[\varphi(c):=(\varphi_Ac,\varphi_Bc).\]
	In fact, we can see that this defined map does have $\pi_A\circ\varphi=\varphi_A$ and $\pi_B\circ\varphi=\varphi_B$, so the diagram does indeed commute.
\end{proof}
This example should feel similar to universal properties: whenever something occurs in our diagram, we have some unique induced map. To see this more formally, we have the following auxiliary exercise.
\begin{exe} \label{exe:produp}
	We exhibit the universal property for $A\times B$ in the category $\mathcal C:=\mathrm{Set}$.
\end{exe}
\begin{proof}
	We note that we are being granted a bijection between pairs of maps $(\varphi_A,\varphi_B)$ and our maps $\varphi$. In other words, there is a bijection
	\[\op{Mor}(C,A)\times\op{Mor}(C,B)\to\op{Mor}(C,A\times B).\]
	To turn this into a universal property, we consider the functor $F:\mathcal C\opp\to\mathrm{Set}$ by
	\[F:\op{Mor}(-,A)\times\op{Mor}(-,B).\]
	In particular, we send morphisms $f:S\to T$ to $F(f):\op{Mor}(T,A)\times\op{Mor}(T,B)\to\op{Mor}(S,A)\to\op{MOr}(S,B)$ by
	\[F(f):=(-\circ f)\times(-\circ f).\]
	We won't check this is a functor, but you can if you like doing that kind of thing.
	
	Now, we to get our universal property, we need to exhibit our natural isomorphism
	\[\eta:F\Rightarrow\op{Mor}(-,A\times B).\]
	We already gave a bijection of sets in the previous exercise, so we just need to show that it is natural. Well, pick up some morphism $f:T\to S$, and we have the following diagram to check.
	% https://q.uiver.app/?q=WzAsNCxbMCwwLCJ+fn5+flxcb3B7TW9yfShTLEEpXFx0aW1lc1xcb3B7TW9yfShTLEIpfn5+fn4iXSxbMSwwLCJ+fn5+flxcb3B7TW9yfShULEEpXFx0aW1lc1xcb3B7TW9yfShULEIpfn5+fn4iXSxbMCwxLCJcXG9we01vcn0oUyxBXFx0aW1lcyBCKSJdLFsxLDEsIlxcb3B7TW9yfShULEFcXHRpbWVzIEIpIl0sWzAsMiwiXFxldGFfUyIsMl0sWzEsMywiXFxldGFfVCJdLFsyLDMsIi1cXGNpcmMgZiIsMl0sWzAsMSwiKC1cXGNpcmMgZilcXHRpbWVzKC1cXGNpcmMgZikiXV0=&macro_url=https%3A%2F%2Fraw.githubusercontent.com%2FdFoiler%2Fnotes%2Fmaster%2Fnir.tex
	\[\begin{tikzcd}
		{~~~~~\op{Mor}(S,A)\times\op{Mor}(S,B)~~~~~} & {~~~~~\op{Mor}(T,A)\times\op{Mor}(T,B)~~~~~} \\
		{\op{Mor}(S,A\times B)} & {\op{Mor}(T,A\times B)}
		\arrow["{\eta_S}"', from=1-1, to=2-1]
		\arrow["{\eta_T}", from=1-2, to=2-2]
		\arrow["{-\circ f}"', from=2-1, to=2-2]
		\arrow["{(-\circ f)\times(-\circ f)}", from=1-1, to=1-2]
	\end{tikzcd}\]
	So now, pick up some pair $(h,g)\in\op{Mor}(S,A)\times\op{Mor}(S,B)$ and track through. Along the bottom, we go to $h\times g$ which then goes to $hf\times gf$. Along the top, we start with $(h,g)$ then go to $(hf,gf)$ which then goes to $hf\times gf$.
	
	Now let's compute our universal element. For this, we need to find what we are getting out of \autoref{thm:yoneda}, which is
	\[\eta_{A\times B}^{-1}(\id_{A\times B})=(\pi_A,\pi_B),\]
	which is fairly intuitive. In particular, we can track $\eta_{A\times B}((\pi_A,\pi_B))$ through and get $\id_{A\times B}$, which will give what we want.
\end{proof}
Here are some generalizing remarks.
\begin{remark}
	The universal property of products in \autoref{exe:produp} did not depend on $\mathrm{Set}$, but the construction did. Products are more of an ambient concept that might or might not happen in some given category.
\end{remark}
\begin{remark}
	We could just have easily defined arbitrary products
	\[\prod_{\alpha\in\lambda}S_\alpha\]
	for some sets $\{S_\alpha\}_{\alpha\in\lambda}$ by just increasing the number of terms. For example, our functor we want to represent is now
	\[F:\prod_{\alpha\in\lambda}\op{Mor}(-,S_\alpha).\]
	We can also write out the analogous universal property in terms of \autoref{exe:produp0}.
\end{remark}
\begin{example}
	The product of the one term $A$ equipped with the projection map $\id_A:A\to A$. Indeed, for any map $C\to A$, there is a unique map $C\to A$ making the following diagram commute.
	% https://q.uiver.app/?q=WzAsMyxbMSwxLCJBIl0sWzEsMCwiQSJdLFswLDAsIkMiXSxbMSwwLCIiLDAseyJsZXZlbCI6Miwic3R5bGUiOnsiaGVhZCI6eyJuYW1lIjoibm9uZSJ9fX1dLFsyLDBdLFsyLDEsIiIsMCx7InN0eWxlIjp7ImJvZHkiOnsibmFtZSI6ImRhc2hlZCJ9fX1dXQ==&macro_url=https%3A%2F%2Fraw.githubusercontent.com%2FdFoiler%2Fnotes%2Fmaster%2Fnir.tex
	\[\begin{tikzcd}
		C & A \\
		& A
		\arrow[Rightarrow, no head, from=1-2, to=2-2]
		\arrow[from=1-1, to=2-2]
		\arrow[dashed, from=1-1, to=1-2]
	\end{tikzcd}\]
\end{example}
\begin{example}
	The product of no terms at all is the final object $X$. Indeed, whenever we have no morphisms going anywhere, there is a unique map to $X$ making whatever diagram you want commute.
\end{example}

\subsection{Coproducts}
Next let's discuss coproducts. Let's just give the universal property.
\begin{definition}[Corproduct]
	Given two objects $A,B\in\mathcal C$, we define the \textit{coproduct} object $A\coprod B$ to be equipped with maps $\iota_A:A\to A\coprod B$ and $\iota_B:B\to A\coprod B$ such that, whenever we have an object $Z$ with maps $\varphi_A:A\to Z$ and $\varphi_B:B\to Z$, there is a unique map $A\coprod B\to Z$ making the following diagram commute.
	% https://q.uiver.app/?q=WzAsNCxbMCwwLCJBIl0sWzIsMCwiQiJdLFsxLDEsIkFcXGNvcHJvZCBCIl0sWzEsMiwiWiJdLFswLDIsIlxcaW90YV9BIl0sWzEsMiwiXFxpb3RhX0IiLDJdLFsxLDMsIlxcdmFycGhpX0IiLDAseyJjdXJ2ZSI6LTJ9XSxbMCwzLCJcXHZhcnBoaV9BIiwyLHsiY3VydmUiOjJ9XSxbMiwzLCIiLDEseyJzdHlsZSI6eyJib2R5Ijp7Im5hbWUiOiJkYXNoZWQifX19XV0=
	\[\begin{tikzcd}
		A && B \\
		& {A\coprod B} \\
		& Z
		\arrow["{\iota_A}", from=1-1, to=2-2]
		\arrow["{\iota_B}"', from=1-3, to=2-2]
		\arrow["{\varphi_B}", curve={height=-12pt}, from=1-3, to=3-2]
		\arrow["{\varphi_A}"', curve={height=12pt}, from=1-1, to=3-2]
		\arrow[dashed, from=2-2, to=3-2]
	\end{tikzcd}\]
\end{definition}
\begin{example}
	The disjoint union $A\sqcup B$ is the coproduct in $\mathrm{Set}$. Indeed, our maps are $\iota_A:a\mapsto(a,0)$ and $\iota_B:b\mapsto(b,1)$. To see the universal property, suppose that we have an object $C$ with maps $\varphi_A:A\to C$ and $\varphi_B:B\to C$. To see the uniqueness of $\varphi:A\sqcup B\to C$, we see that we must have
	\[\varphi(\iota_Aa)=\varphi_A(a)\qquad\text{and}\qquad\varphi(\iota_Bb)=\varphi_B(b)\]
	which exhausts all possible cases for elements of $A\sqcup B$. It is then not too hard to check that this does satisfy $\varphi\circ\iota_A=\varphi_A$ and $\varphi\circ\iota_B=\varphi_B$ by construction.
\end{example}
\begin{example}
	We can generalize to products with multiple terms. If we have one object, the coproduct of $A$ is just $A$. Similarly, if we have no objects, then the coproduct will be an initial object.
\end{example}

\subsection{More on Products}
Now let's generalize our examples. We begin by making the product even more categorical. At a high level, we might have lots of objects $\{A_\alpha\}_{\alpha\in\lambda}$, we are given maps $\pi_\alpha:\prod A\to A_\alpha$ in some universal way.
% https://q.uiver.app/?q=WzAsNCxbMSwwLCJcXHByb2QgQSJdLFswLDEsIkFfXFxhbHBoYSJdLFsxLDEsIlxcY2RvdHMiXSxbMiwxLCJBX1xcYmV0YSJdLFswLDEsIlxccGlfXFxhbHBoYSJdLFswLDMsIlxccGlfXFxiZXRhIiwyXV0=
\[\begin{tikzcd}
	& {\prod A} \\
	{A_\alpha} & \cdots & {A_\beta}
	\arrow["{\pi_\alpha}", from=1-2, to=2-1]
	\arrow["{\pi_\beta}"', from=1-2, to=2-3]
\end{tikzcd}\]
To make this more in terms of category theory, we note that we can formalize the bottom part of the diagram as the image of some functor
\[F:\mathcal J\to\mathcal C\]
for some discrete category $\mathcal J$. Namely, our objects $A_\alpha$ look like $F(\alpha)$ for various $\alpha\in\mathcal J$.

To put the product $\prod A$ on the same footing, we will similarly define the constant functor
\[C_x:\mathcal J\to\mathcal C\]
which sends all objects of $\mathcal J$ to $x$ and all morphisms to $\id_x$.

We would like to create arrows between our diagrams, we are asking for an arrow between our functors, so we are more or less asking for a natural transformation $\eta:C_x\Rightarrow F$. Namely, the component morphisms take some $\alpha\in\mathcal J$, we are being promised a morphism $\eta_\alpha:x\to A_\alpha$. If we wanted to check that $\eta$ is a natural transformation, we would pick up a morphism $\id_\alpha:\alpha\to\alpha$ in $\mathcal J$, which gives rise to the following diagram.
% https://q.uiver.app/?q=WzAsNCxbMCwwLCJ4Il0sWzEsMCwieCJdLFswLDEsIkYoXFxhbHBoYSkiXSxbMSwxLCJGKFxcYWxwaGEpIl0sWzAsMiwiXFxldGFfXFxhbHBoYSJdLFsxLDMsIlxcZXRhX1xcYWxwaGEiXSxbMCwxLCJcXGlkX3giXSxbMiwzLCJcXGlkX3tGKFxcYWxwaGEpfSIsMl1d
\[\begin{tikzcd}
	x & x \\
	{F(\alpha)} & {F(\alpha)}
	\arrow["{\eta_\alpha}", from=1-1, to=2-1]
	\arrow["{\eta_\alpha}", from=1-2, to=2-2]
	\arrow["{\id_x}", from=1-1, to=1-2]
	\arrow["{\id_{F(\alpha)}}"', from=2-1, to=2-2]
\end{tikzcd}\]
Notably, this commutes for free. If we wanted to add more structure to our products, we might want to change $\mathcal J$ to be not discrete and have $F$ be a more general diagram. This gives rise to limits.
\begin{restatable}[Cone]{defihelper}{conedef}
	Fix an index category $\mathcal J$ and a category $\mathcal C$ with an object $c\in\mathcal C$. Then a \textit{cone} is a natural transformation from the constant functor $C_c\Rightarrow F$, where $F:\mathcal J\to\mathcal C$ is some diagram.
\end{restatable}
The limit will be the object $\lim F\in\mathcal C$ which is a ``universal'' cone, in the same way that the product was universal with respect to a ``discrete cone.'' We will not discuss this more formally today, but we will discuss it more next lecture.