% !TEX root = ../notes.tex

Today we're having both Bryce and Chris lecture today (not in that order). We're in luck.

\subsection{More on Universal Properties}
We recall our definition.
\univpop*
\noindent We continue our discussion of \autoref{exe:univpoptensor}. Typically, one would think about the universal property as follows.
\begin{definition}[Universal property, II]
	An object $c\in\mathcal C$ \textit{satisfies the universal property} given by a functor $F:\mathcal C\to\mathrm{Set}$ if we have a universal element $x$ such that $(c, F, x)$ is a universal property.
\end{definition}
In particular, last time we talked about having the functor
\[\op{Bilin}(V,W,-):\mathrm{Vec}_k\to\mathrm{Set}\]
which we claim was represented by some object $V\otimes W$ and our universal element $\otimes\in\op{Bilin}(V,W,-)\to V\otimes W$. By \autoref{thm:yoneda}, we can unwind this to a natural transformation
\[\eta:\op{Hom}(V\otimes W,-)\Rightarrow\op{Bilin}(V,W,-),\]
which we are claiming is a natural isomorphism to be our universal property. Well, we have our object $\otimes$, so we now track everything through. Here is our diagram to unwind the isomorphism above for a morphism $\overline f:V\otimes W\to U$ corresponding to the bilinear map $f:V\times W\to U$.
% https://q.uiver.app/?q=WzAsNCxbMCwwLCJcXG9we0hvbX0oVlxcb3RpbWVzIFcsVlxcb3RpbWVzIFcpIl0sWzEsMCwiXFxvcHtIb219KFZcXG90aW1lcyBXLFUpIl0sWzAsMSwiXFxvcHtCaWxpbn0oVixXLFZcXG90aW1lcyBXKSJdLFsxLDEsIlxcb3B7QmlsaW59KFYsVyxVKSJdLFswLDIsIlxcZXRhX3tWXFxvdGltZXMgV30iLDJdLFswLDEsIlxcb3ZlcmxpbmUgZlxcY2lyYy0iXSxbMSwzLCJcXGV0YV9WIl0sWzIsMywiXFxvdmVybGluZSBmXFxjaXJjLSIsMl1d&macro_url=https%3A%2F%2Fraw.githubusercontent.com%2FdFoiler%2Fnotes%2Fmaster%2Fnir.tex
\[\begin{tikzcd}
	{\op{Hom}(V\otimes W,V\otimes W)} & {\op{Hom}(V\otimes W,U)} \\
	{\op{Bilin}(V,W,V\otimes W)} & {\op{Bilin}(V,W,U)}
	\arrow["{\eta_{V\otimes W}}"', from=1-1, to=2-1]
	\arrow["{\overline f\circ-}", from=1-1, to=1-2]
	\arrow["{\eta_U}", from=1-2, to=2-2]
	\arrow["{\overline f\circ-}"', from=2-1, to=2-2]
\end{tikzcd}\]
So to unwind what $\eta_U$ means, we plug into $\id_{V\otimes W}$. This makes the following diagram.
% https://q.uiver.app/?q=WzAsNCxbMCwwLCJcXGlkX3tWXFxvdGltZXMgV30iXSxbMSwwLCJcXG92ZXJsaW5lIGYiXSxbMCwxLCJcXG90aW1lcyJdLFsxLDEsImYiXSxbMCwyLCJcXGV0YV97Vlxcb3RpbWVzIFd9IiwyLHsic3R5bGUiOnsidGFpbCI6eyJuYW1lIjoibWFwcyB0byJ9fX1dLFswLDEsIlxcb3ZlcmxpbmUgZlxcY2lyYy0iXSxbMSwzLCJcXGV0YV9WIl0sWzIsMywiXFxvdmVybGluZSBmXFxjaXJjLSIsMix7InN0eWxlIjp7InRhaWwiOnsibmFtZSI6Im1hcHMgdG8ifX19XV0=&macro_url=https%3A%2F%2Fraw.githubusercontent.com%2FdFoiler%2Fnotes%2Fmaster%2Fnir.tex
\[\begin{tikzcd}
	{\id_{V\otimes W}} & {\overline f} \\
	\otimes & f
	\arrow["{\eta_{V\otimes W}}"', maps to, from=1-1, to=2-1]
	\arrow["{\overline f\circ-}", from=1-1, to=1-2]
	\arrow["{\eta_U}", from=1-2, to=2-2]
	\arrow["{\overline f\circ-}"', maps to, from=2-1, to=2-2]
\end{tikzcd}\]
Namely, we are told that bilinear maps $f:V\times W\to U$ correspond uniquely to a morphism $\overline f:V\otimes W\to U$ (by \autoref{thm:betterequiv}) in such a way that the following diagram commutes.
% https://q.uiver.app/?q=WzAsMyxbMCwwLCJWXFx0aW1lcyBXIl0sWzEsMCwiVlxcb3RpbWVzIFciXSxbMSwxLCJVIl0sWzAsMSwiXFxvdGltZXMiXSxbMCwyLCJmIiwyXSxbMSwyLCJcXG92ZXJsaW5lIGYiLDAseyJzdHlsZSI6eyJib2R5Ijp7Im5hbWUiOiJkYXNoZWQifX19XV0=
\[\begin{tikzcd}
	{V\times W} & {V\otimes W} \\
	& U
	\arrow["\otimes", from=1-1, to=1-2]
	\arrow["f"', from=1-1, to=2-2]
	\arrow["{\overline f}", dashed, from=1-2, to=2-2]
\end{tikzcd}\]
So this is our usual universal property for the tensor product.
\begin{remark}
	We will actually need to construct $V\otimes W$, which we did not do, in order to show that there exists a way to represent the functor
	\[\op{Bilin}(V,W,-):\mathrm{Vec}_k\to\mathrm{Set}.\]
	However, in practice, we only ever want to pay attention to the above universal property.
\end{remark}

\subsection{Category of Elements}
Today's discussion will not be discussion. Bryce is, reportedly, sorry. For brevity, we will take the following convention.
\begin{definition}[Universal]
	We say that an object $c\in\mathcal C$ is \textit{universal} if and only if it is either initial or final.
\end{definition}
Of course, $V\otimes W$ will not turn out to be universal in $\mathrm{Vec}_k$, but if we change our category, then it will be, which is nice.

With that said, here is our main character for today.
\begin{definition}[Category of elements]
	Fix $F:\mathcal C\to\mathrm{Set}$ a functor. Then the \textit{category of elements} of $F$, denoted $\int F$ is made of the following data.
	\begin{itemize}
		\item Objects are pairs $(c,x)$ where $c\in\mathcal C$ and $x\in Fc$. In practice, we should think about the object $x\in Fc$ on its own, but we will have to remember which $c$ it comes from.
		\item Morphisms $(c,x)\to(d,y)$ made of morphisms $f:c\to d$ which preserve our ``base points'' as $(Ff)(x)=y$. Importantly, we are keeping track of the arrows in $\mathcal C$, not in $\mathrm{Set}$; e.g., $F$ might not be injective on arrows, so we will keep track of these definitions.
		\item Identities are identities lifted from $\mathcal C$.
		\item Composition is composition in $\mathcal C$.
	\end{itemize}
\end{definition}
\begin{remark}
	There is a natural forgetful functor $\Pi:\int F\to\mathcal C$ by
	\[\Pi(c,x):=c\qquad\text{and}\qquad\Pi(f):=f.\]
	We bring this up because this is roughly why we are keeping track of the morphisms in $\mathcal C$ instead of $\mathrm{Set}$.
\end{remark}
There is also a contravariant version.
\begin{definition}[Category of elements, contravariant]
	Fix $F:\mathcal C\opp\to\mathrm{Set}$ a functor. Then the \textit{category of elements} of $F$, denoted $\int F$ is made of the following data.
	\begin{itemize}
		\item Objects are pairs $(c,x)$ where $c\in\mathcal C$ and $x\in Fc$.
		\item Morphisms $(c,x)\to(d,y)$ made of morphisms $f:c\to d$ which preserve our ``base points'' as $(Ff)(y)=x$. This flips because $F$ is contravariant.
		\item Identities are identities lifted from $\mathcal C$.
		\item Composition is composition in $\mathcal C$.
	\end{itemize}
\end{definition}
Let's see some examples.
\begin{example}
	Let $\mathcal C$ be a concrete category with faithful (forgetful) functor $U:\mathcal C\to\mathrm{Set}$. We work through $\int U$.
	\begin{itemize}
		\item Objects are pairs $(c,x)$ where $x\in Uc$.
		\item Morphisms are morphisms $f:(c,x)\to(d,y)$ such that $(Uf)(x)=y$.
	\end{itemize}
	In other words, $\int U$ is roughly the objects $c\in\mathcal C$ with an identified base point. Specifically, $\int (U:\mathrm{Top}\to\mathrm{Set})=\mathrm{Top}_*$.
\end{example}
\begin{example}
	Fix $\mathcal C$ a locally small category, which is how you know Bryce is lecturing, which permits a functor $\op{Mor}(c,-):\mathcal C\Rightarrow\mathrm{Set}$. We discuss $\int\op{Mor}(c,-)$.
	\begin{itemize}
		\item Objects are pairs $(d,f)$ where $f\in\op{Hom}(c,d)$. So our objects are morphisms.
		\item A morphism $\varphi:(d,f)\to(e,g)$ is a morphism $\varphi:d\to e$ in $\mathcal C$ such that $\varphi f=(\varphi\circ-)(f)=g$.
	\end{itemize}
	In other words, this gives the category under $\mathcal C$, denoted $c/\mathcal C$. The contravariant version gives $\mathcal C/c$.
\end{example}
\begin{exe}
	Fix $F:C\opp\to\mathrm{Set}$ a contravariant functor. We recover $\int F$ as a comma category.
\end{exe}
\begin{proof}
	To set up our discussion, we recall that \autoref{thm:yoneda} provides us with a sufficiently natural bijection
	\[\psi:Fc\cong\op{Mor}(\op{Mor}_\mathcal C(-,c),F).\]
	Now, objects in $\int F$ will naturally be objects $x\in Fc$. We would to track morphisms $f:(c,x)\to(d,y)$ through here as well, which means that we are going to need a morphism $\psi(x)\to\psi(y)$ in $\mathrm{Set}^{\mathcal C\opp}$. Roughly speaking, we are going to want the following diagram to commute.
	% https://q.uiver.app/?q=WzAsMyxbMCwwLCJcXG9we01vcn0oLSxDKSJdLFswLDEsIlxcb3B7TW9yfSgtLEMpIl0sWzEsMCwiRiJdLFswLDEsIj8iLDIseyJsZXZlbCI6Mn1dLFswLDIsIlxccHNpKHgpIiwwLHsibGV2ZWwiOjJ9XSxbMSwyLCJcXHBzaSh5KSIsMix7ImxldmVsIjoyfV1d
	\[\begin{tikzcd}
		{\op{Mor}(-,c)} & F \\
		{\op{Mor}(-,d)}
		\arrow["{(f\circ-)}"', Rightarrow, from=1-1, to=2-1]
		\arrow["{\psi(x)}", Rightarrow, from=1-1, to=1-2]
		\arrow["{\psi(y)}"', Rightarrow, from=2-1, to=1-2]
	\end{tikzcd}\tag{$*$}\label{eq:weirddiagram}\]
	In particular, \autoref{thm:yonedaembedding} tells us that all such morphisms between natural transformations take the form $(f\circ-)$ for some morphism $f$, from which we can track our base point.
	
	The point of all this is that we are going to have a nice correspondence between $\int F$ and the comma category
	\[\int F\cong\yo\downarrow\widetilde F,\]
	where $\widetilde F:\{*\}\to\mathrm{Set}^{\mathcal C\opp}$ is the constant functor taking $*\mapsto F$. Indeed, to quickly unwind our definition of the comma category, it is made of triplets $(c\in\mathcal C,*\in\{*\},f:\yo(c)\to F(*))$, where morphisms $h:(c,*,f)\to(c',*,f')$ require the following diagram to commute.
	% https://q.uiver.app/?q=WzAsNCxbMCwwLCJcXHlvKGMpIl0sWzAsMSwiXFx5byhjJykiXSxbMSwwLCJGIl0sWzEsMSwiRiJdLFswLDEsIkZoIiwyXSxbMCwyLCJmIl0sWzEsMywiZiciLDJdLFsyLDMsIiIsMCx7ImxldmVsIjoyLCJzdHlsZSI6eyJoZWFkIjp7Im5hbWUiOiJub25lIn19fV1d
	\[\begin{tikzcd}
		{\yo(c)} & F \\
		{\yo(c')} & F
		\arrow["Fh"', from=1-1, to=2-1]
		\arrow["f", from=1-1, to=1-2]
		\arrow["{f'}"', from=2-1, to=2-2]
		\arrow[Rightarrow, no head, from=1-2, to=2-2]
	\end{tikzcd}\]
	Notably, we only have to check the $\id_F:F\to F$ morphism because this is the only morphism carried from $\widetilde F:\{*\}\to\mathrm{Set}^{\mathcal C\opp}$. But this diagram above is exactly the one we asked for in \autoref{eq:weirddiagram}, so we are done.
\end{proof}

Next time we will discuss the following result.
\begin{proposition}
	Fix $F:\mathcal C\to\mathrm{Set}$ be a functor. Then $\int F$ has an initial object $(c,x)$ if and only if $F$ is representable by $c$ with universal element $x$.
\end{proposition}
\begin{proof}
	In one direction, take $(c,x)\in\int F$ initial. We would like a natural isomorphism $\eta:\op{Mor}(c,-)\Rightarrow F$. Well, by \autoref{thm:yoneda}, we get some natural transformation $\eta$ corresponding to $x$, where
	\[\eta_d(f):=(Ff)(x)\]
	by pushing through our definition in \autoref{thm:yoneda}. For this to be a natural isomorphism, we need the components $\eta_d:\op{Mor}(c,d)\to Fd$ to be isomorphisms. In other words, for each $d\in\mathcal C$ and $y\in Fd$, we need some $f:c\to d$ such that
	\[(Ff)(x)=\eta_d(f)=y.\]
	Equivalently, there is a unique morphism $f:(c,x)\to(d,y)$ in $\int F$, which is what we wanted.
\end{proof}
\begin{remark}
	In the dual case, $F$ will be contravariant, and our initial object becomes final.
\end{remark}