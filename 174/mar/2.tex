\documentclass[../notes.tex]{subfiles}

\begin{document}

% !TEX root = ../notes.tex

We continue. Chris did some review that the Yoneda embedding $\yo$ is full, which can be found in our proof of \autoref{thm:yonedaembedding}.

\subsection{Unique Representation}
For the attendance question, we have the following.
\begin{proposition}
	Fix a category $\mathcal C$. Then $x\cong y$ implies that $\op{Mor}(x,-)\simeq\op{Mor}(y,0)$.
\end{proposition}
\begin{proof}
	The point is that the Yoneda embedding $\yo$ must induce an isomorphism $\yo(f):\yo(x)\cong\yo(y)$, which is what we wanted.
\end{proof}
In fact, the converse is also true. We have the following definition.
\begin{definition}[Creates, reflects isomorphisms]
	A functor $F$ \textit{reflects isomorphisms} if and only if $Ff$ being an isomorphism implies that $f$ is an isomorphism. Similarly, a functor $F$ \textit{creates isomorphisms} if and only if $Fx\cong Fy$ forces $x\cong y$.
\end{definition}
\begin{example}
	The functor from $F:\mathrm Be+\mathrm Be\to\mathrm Be$ sending both points to $*$ will certainly reflect isomorphisms because we can only pull back. This $F$ however does not reflect isomorphisms because $Fe_1\cong Fe_2$ but $e_1$ is not isomorphic to $e_2$; there are no morphisms between them all.
\end{example}
We have the following result.
\begin{proposition}
	If a functor $F$ is fully faithful, then $F$ creates and reflects isomorphisms.
\end{proposition}
\begin{proof}
	This was on the homework.
\end{proof}
\begin{remark}
	Thus, noting that $\yo$ is fully faithful, we see that $\op{Mor}(x,-)\simeq\op{Mor}(y,0)$ forces $x\cong y$.
\end{remark}

The point of all of our discussion is as follows.
\begin{proposition}
	Suppose two objects $x$ and $y$ represent a functor $F:\mathcal C\to\mathrm{Set}$ by natural isomorphisms $\eta:\op{Mor}(x,-)\Rightarrow F$ and $\mu:\op{Mor}(y,-)\Rightarrow F$. Then there is a canonical isomorphism $x\cong y$ by $\eta^{-1}\mu$.
\end{proposition}
\begin{proof}
	Intuitively, $f:x\cong y$ induces $-\circ f:\op{Mor}(y,-)\to\op{Mor}(x,-)$, from we would like the following diagram to commute.
	% https://q.uiver.app/?q=WzAsMyxbMCwwLCJcXG9we01vcn0oeCwtKSJdLFsyLDEsIkYiXSxbMCwyLCJcXG9we01vcn0oeSwtKSJdLFsyLDAsIi1cXGNpcmMgZiIsMCx7ImxldmVsIjoyfV0sWzAsMSwiXFxldGEiLDAseyJsZXZlbCI6Mn1dLFsyLDEsIlxcbXUiLDIseyJsZXZlbCI6Mn1dXQ==
	\[\begin{tikzcd}
		{\op{Mor}(x,-)} \\
		&& F \\
		{\op{Mor}(y,-)}
		\arrow["{-\circ f}", Rightarrow, from=3-1, to=1-1]
		\arrow["\eta", Rightarrow, from=1-1, to=2-3]
		\arrow["\mu"', Rightarrow, from=3-1, to=2-3]
	\end{tikzcd}\]
	As such, we see that we want $f$ to induce a natural isomorphism $(-\circ f)=\eta^{-1}\mu:\op{Mor}(y,-)\Rightarrow\op{Mor}(x,-)$, which we can in fact pull back to $f$ because of \autoref{thm:yoneda}.
\end{proof}
\begin{idea}
	Thus, we can say that an object represents a functor is unique up to unique (commuting) isomorphism.
\end{idea}

\subsection{Universal Properties}
With our uniqueness in hand, we are ready to talk about universal properties.
\begin{restatable}[Universal property I, element]{definition}{univpop}
	A \textit{universal property} for an object $c\in\mathcal C$ is a representable functor $F:\mathcal C\Rightarrow\mathrm{Set}$ along with an element $x\in Fc$ such that $x$ induces (by the Yoneda lemma) a natural isomorphism $\op{Mor}(c,-)\Rightarrow F$. In such a triplet $(c,F,x)$, we call $x$ the \textit{universal element}.
\end{restatable}
\noindent To be explicit, $x\in Fc$ is inducing a natural transformation $\op{Mor}(c,-)\Rightarrow F$ by \autoref{thm:yoneda}, so the condition we are requiring is that we have a natural isomorphism.
\begin{exe}
	We discuss $\ZZ[x]$ as the free ring on $X$.
\end{exe}
\begin{proof}
	We will represent the forgetful functor $U:\mathrm{Ring}\to\mathrm{Set}$. To start, we need to show that $\ZZ[x]$ does actually represent $U$, for which we need a natural isomorphism
	\[\eta:\op{Mor}_{\mathrm{Ring}}(\ZZ[x],-)\Rightarrow U.\]
	In particular, fixing a ring $R$, we need an isomorphism $\op{Mor}_{\mathrm{Ring}}(\ZZ[x],R)\Rightarrow UR$, which we do by sending a morphism $f:\ZZ[x]\to R$ to $f(x)\in UR$. This is indeed a bijection because we can uniquely determine a morphism $\ZZ[x]\to R$ by where we send $x$.
	
	Lastly, our universal element is $x$. To see this, we track through \autoref{thm:yoneda} to compute
	\[\eta_{\ZZ[x]}(\id_{\ZZ[x]})=\id_{\ZZ[x]}(x)=x,\]
	which is what we wanted.
\end{proof}
\begin{remark}
	In words, the above proof says that $\ZZ[x]$ is the universal ring that has a distinguished element $x$. Being the ``universal ring'' is usually called being the ``free ring.''
\end{remark}
For our next example, we have the following definition.
\begin{definition}[Tensor products, I]
	Fix two $k$-vector spaces $V$ and $W$. Then $V\otimes W$ is made of formal sums
	\[\sum_{i=1}^nv_i\otimes w_i\]
	where $v_1,\ldots,v_n\in V$ and $w_1,\ldots,w_n\in W$. Further, $(v,w)\mapsto v\otimes w$ is $k$-bilinear.
\end{definition}
\begin{exe} \label{exe:univpoptensor}
	We discuss $V\otimes W$ by universal property.
\end{exe}
\begin{proof}
	The point is to consider the functor
	\[\op{Bilin}(V,W,-):\mathrm{Vec}_k\to\mathrm{Set}\]
	taking $U\mapsto\op{Bilin}(V,W,U)$, where $\op{Bilin}(V,W,U)$ consists of the $k$-bilinear maps $V\times W\to U$. Tensor products are actually intended to represent this functor. So here is a better definition for tensor products.
	\begin{definition}[Tensor products, II]
		Given vector spaces $V$ and $W$, we define $V\otimes W$ as the object that represents $\op{Bilin}(V,W,-)$.
	\end{definition}
	We do not actually know if $V\otimes W$ really exists, but we will do so shortly. Our universal element $x$ is intended to live in $\op{Bilin}(V,W,V\otimes W)$, so we are characterizing $V\otimes W$ by the bilinear map $V\times W\to V\otimes W$. In particular, we can now more or less say that $V\otimes W$ is the ``universal'' vector space with respect to a bilinear map $V\times W\to V\otimes W$.

	Unwinding a bit, we will name the map $V\times W\to V\otimes W$ as $\otimes$. In particular, we are hoping that this element induces a natural isomorphism
	\[\op{Mor}(V\otimes W,-)\Rightarrow\op{Bilin}(V,W,-).\]
	In particular, by the Yoneda lemma, we are hoping that bilinear maps $V\times W\to U$ are in natural bijection with linear maps $V\otimes W\to U$.
\end{proof}

\end{document}