% !TEX root = ../notes.tex

We do more limits today.

\subsection{Cones and Cocones}
We recall the following definition. For today, we fix $\mathcal J$ as our index category.
\conedef*
\begin{definition}[Apex]
	A cone over $F$ with \textit{summit} or \textit{apex} $c$ is a natural transformation $\lambda:c\Rightarrow F$ with components $\lambda_j:c\to Fj$. These components are called \textit{legs}.
\end{definition}
Informally, $\lambda$ consists of morphisms $\lambda_j:c\to Fj$ which commute with the morphisms promised by $\mathcal J$. Namely, for any morphism $f:i\to j$, we have the following naturality square diagram;
% https://q.uiver.app/?q=WzAsNCxbMCwwLCJjIl0sWzAsMSwiRmkiXSxbMSwxLCJGaiJdLFsxLDAsImMiXSxbMSwyLCJGZiIsMl0sWzAsMywiIiwyLHsibGV2ZWwiOjIsInN0eWxlIjp7ImhlYWQiOnsibmFtZSI6Im5vbmUifX19XSxbMCwxLCJcXGxhbWJkYV9pIiwyXSxbMywyLCJcXGxhbWJkYV9qIl1d
\[\begin{tikzcd}
	c & c \\
	Fi & Fj
	\arrow["Ff"', from=2-1, to=2-2]
	\arrow[Rightarrow, no head, from=1-1, to=1-2]
	\arrow["{\lambda_i}"', from=1-1, to=2-1]
	\arrow["{\lambda_j}", from=1-2, to=2-2]
\end{tikzcd}\]
Collapsing the top makes this look more like a triangle.
% https://q.uiver.app/?q=WzAsMyxbMSwwLCJjIl0sWzIsMSwiRmoiXSxbMCwxLCJGX2kiXSxbMCwxLCJcXGxhbWJkYV9qIl0sWzAsMiwiXFxsYW1iZGFfaSIsMl0sWzIsMSwiRmYiLDJdXQ==
\[\begin{tikzcd}
	& c \\
	{F_i} && Fj
	\arrow["{\lambda_j}", from=1-2, to=2-3]
	\arrow["{\lambda_i}"', from=1-2, to=2-1]
	\arrow["Ff"', from=2-1, to=2-3]
\end{tikzcd}\]
Of course, we also have a dual notion. We have the following definition.
\begin{definition}[Nadir]
	A cone under $F$ (or ``cocone'') with \textit{nadir} $c$ is a natural transformation $\lambda:F\Rightarrow c$ with components $\lambda_j:Fj\to c$. These components are (still) called \textit{legs}.
\end{definition}
\begin{remark}
	We use the word nadir because someone wanted to.
\end{remark}
This time our picture looks like the following.
% https://q.uiver.app/?q=WzAsMyxbMCwwLCJGaSJdLFsyLDAsIkZqIl0sWzEsMSwiYyJdLFswLDEsIkZmIl0sWzAsMiwiXFxsYW1iZGFfaSIsMl0sWzEsMiwiXFxsYW1iZGFfaiJdXQ==
\[\begin{tikzcd}
	Fi && Fj \\
	& c
	\arrow["Ff", from=1-1, to=1-3]
	\arrow["{\lambda_i}"', from=1-1, to=2-2]
	\arrow["{\lambda_j}", from=1-3, to=2-2]
\end{tikzcd}\]
Notably, the nadir $c$ is under $F$ this time.
\begin{example}
	Fix our index category $\mathcal J=\ZZ$. For a cone $F:\mathcal J\to\mathcal C$ over $c$, our diagram looks like the following.
	% https://q.uiver.app/?q=WzAsNixbMiwwLCJjIl0sWzEsMSwiRigtMSkiXSxbMiwxLCJGKDApIl0sWzMsMSwiRigxKSJdLFs0LDEsIlxcY2RvdHMiXSxbMCwxLCJcXGNkb3RzIl0sWzUsMV0sWzEsMl0sWzIsM10sWzMsNF0sWzAsMV0sWzAsMl0sWzAsM11d
	\[\begin{tikzcd}
		&& c \\
		\cdots & {F(-1)} & {F(0)} & {F(1)} & \cdots
		\arrow[from=2-1, to=2-2]
		\arrow[from=2-2, to=2-3]
		\arrow[from=2-3, to=2-4]
		\arrow[from=2-4, to=2-5]
		\arrow[from=1-3, to=2-2]
		\arrow[from=1-3, to=2-3]
		\arrow[from=1-3, to=2-4]
	\end{tikzcd}\]
\end{example}

\subsection{Limits and Colimits}
Intuitively, our limits will be the universal apex for a cone. It is the best cone; in some sense, it is the smallest or ``closest'' apex to the diagram. The diagram looks like the following.
% https://q.uiver.app/?q=WzAsNCxbMSwxLCJjIl0sWzEsMCwiYyciXSxbMCwyLCJGaSJdLFsyLDIsIkZqIl0sWzIsMywiRmYiLDJdLFswLDIsIlxcbGFtYmRhX2kiXSxbMCwzLCJcXGxhbWJkYV9qIiwyXSxbMSwyLCJcXGxhbWJkYSdfaSIsMix7ImN1cnZlIjoyfV0sWzEsMywiXFxsYW1iZGFfaiciLDAseyJjdXJ2ZSI6LTJ9XV0=
\[\begin{tikzcd}
	& {c'} \\
	& c \\
	Fi && Fj
	\arrow["Ff"', from=3-1, to=3-3]
	\arrow["{\lambda_i}", from=2-2, to=3-1]
	\arrow["{\lambda_j}"', from=2-2, to=3-3]
	\arrow["{\lambda'_i}"', curve={height=12pt}, from=1-2, to=3-1]
	\arrow["{\lambda_j'}", curve={height=-12pt}, from=1-2, to=3-3]
\end{tikzcd}\]
We have the following definition to induce our desired behavior.
\begin{definition}[Cone functor]
	Fix a diagram $F:\mathcal J\to\mathcal C$.
	\begin{itemize}
		\item We define the functor $\op{Cone}(-,F):\mathcal C\opp\to\mathrm{Set}$ by 
		\[c\mapsto\op{Cone}(c,F):=\op{Hom}_{\mathrm{Cat}}(c,F)\]
		which are the natural transformations $\lambda:c\Rightarrow F$. Then a morphism $f:c\to d$ goes to the morphism $F(f):\op{Hom}_{\mathrm{Cat}}(d,F)\to \op{Hom}_{\mathrm{Cat}}(c,F)$ so that a cone $\lambda:d\Rightarrow F$ gives rise to a cone
		\[F(f)(\lambda)=\lambda_\bullet\circ f\in\op{Hom}_{\mathrm{Cat}}(c,F).\]
		\item We define the functor $\op{Cone}(-,F):\mathcal C\to\mathrm{Set}$ by 
		\[c\mapsto\op{Cone}(F,c):=\op{Hom}_{\mathrm{Cat}}(F,c)\]
		which are the natural transformations $\lambda:c\Rightarrow F$. Then a morphism $f:c\to d$ goes to the morphism $F(f):\op{Hom}_{\mathrm{Cat}}(F,c)\to \op{Hom}_{\mathrm{Cat}}(F,d)$ so that a cone $\lambda:F\Rightarrow c$ gives rise to a cone
		\[F(f)(\lambda)=f\circ\lambda_\bullet\in\op{Hom}_{\mathrm{Cat}}(F,d).\]
	\end{itemize}
\end{definition}
Here is the image of $\op{Cone}(f)$ creating a cone with apex $c$ from a cone to apex $d$.
\[\begin{tikzcd}
	& d \\
	& c \\
	Fi && Fj
	\arrow["Ff"', from=3-1, to=3-3]
	\arrow["{\lambda_i}", from=2-2, to=3-1]
	\arrow["{\lambda_j}"', from=2-2, to=3-3]
	\arrow["{\lambda_i\circ f}"', curve={height=12pt}, from=1-2, to=3-1]
	\arrow["{\lambda_i\circ f}", curve={height=-12pt}, from=1-2, to=3-3]
	\arrow["f"{description}, from=1-2, to=2-2]
\end{tikzcd}\]
We will not check the functoriality of this functor, but surely it works: just look at it.

Our functors give the following definitions.
\begin{definition}[Limit, colimit]
	A \textit{limit} of a diagram $F:\mathcal J\to\mathcal C$ is a representation of $\op{Cone}(-,F)$; in other words, it is a natural isomorphism $\mathcal C(-,c)\simeq\op{Cone}(-,F)$. (Note $\op{Cone}(-,F)$ is the contravariant.) Dually, a \textit{colimit} is a representation of $\op{Cone}(F,-)$.
\end{definition}
We will mostly be talking about limits and leave the discussion of colimits to the curious.

Note that, by \autoref{thm:yoneda}, we see that a natural transformation
\[\alpha\in\op{Hom}(\mathcal C(-,c),\op{Cone}(-,F))\]
corresponds to some literal cone $\op{Cone}(c,F)$. From our discussion of the category of elements, we note that we can also think of a limit in the following way.
\begin{definition}[Limit, colimit]
	A \textit{limit} of a diagram $F:\mathcal J\to\mathcal C$ is a terminal object in $\int\op{Cone}(-,F)$.
\end{definition}
To review, our objects of $\int\op{Cone}(-,F)$ look like pairs $(c,\lambda)\in\int\op{Cone}(-,F)$ where $\lambda:c\Rightarrow F$. Then our morphisms $(c,\lambda)\to(c,\mu)$ have the data $f:c\to d$ such that
\[\op{Cone}(f,F)(\mu)=\lambda.\]
In other words, we require $\mu_\bullet\circ f=\lambda_\bullet$, which is equivalent to the commutativity of the following diagram.
% https://q.uiver.app/?q=WzAsNCxbMSwxLCJjIl0sWzEsMCwiZCJdLFswLDIsIkZpIl0sWzIsMiwiRmoiXSxbMiwzLCJGZiIsMl0sWzAsMiwiXFxsYW1iZGFfaSJdLFswLDMsIlxcbGFtYmRhX2oiLDJdLFsxLDIsIlxcbXVfaSIsMix7ImN1cnZlIjoyfV0sWzEsMywiXFxtdV9qIiwwLHsiY3VydmUiOi0yfV0sWzEsMCwiZiIsMV1d
\[\begin{tikzcd}
	& d \\
	& c \\
	Fi && Fj
	\arrow["Ff"', from=3-1, to=3-3]
	\arrow["{\lambda_i}", from=2-2, to=3-1]
	\arrow["{\lambda_j}"', from=2-2, to=3-3]
	\arrow["{\mu_i}"', curve={height=12pt}, from=1-2, to=3-1]
	\arrow["{\mu_j}", curve={height=-12pt}, from=1-2, to=3-3]
	\arrow["f"{description}, from=1-2, to=2-2]
\end{tikzcd}\]
Thus, $(c,\lambda)$ being terminal in $\int\op{Cone}(-,F)$ means that any pair of objects $(d,\mu)$ will have the morphism $f$ be unique.

At this point, we can see that our limits are indeed unique up to unique isomorphism because our terminal objects are unique up to unique isomorphism. Alternatively, we can give the following argument.
\begin{prop}
	The limit of a diagram $F:\mathcal J\to\mathcal C$ is unique up to unique isomorphism.
\end{prop}
\begin{proof}
	The point is to stack our limits on top of each other. So that $(c,\lambda)$ and $(c',\lambda')$ are both limits of $F$. Then we place them in the following diagram and note that we have unique maps $f$ and $g$ induced by the diagram above.
	% https://q.uiver.app/?q=WzAsNSxbMSwyLCJjIl0sWzEsMSwiYyciXSxbMCwzLCJGaSJdLFsyLDMsIkZqIl0sWzEsMCwiYyJdLFsyLDMsIkZmIiwyXSxbMCwyLCJcXGxhbWJkYV9pIl0sWzAsMywiXFxsYW1iZGFfaiIsMl0sWzEsMiwiXFxsYW1iZGFfaSciLDJdLFsxLDMsIlxcbGFtYmRhX2onIl0sWzEsMCwiZiIsMV0sWzQsMiwiXFxsYW1iZGFfaSIsMix7ImN1cnZlIjoyfV0sWzQsMywiXFxsYW1iZGFfaiIsMCx7ImN1cnZlIjotMn1dLFs0LDEsImciLDFdXQ==
	\[\begin{tikzcd}
		& c \\
		& {c'} \\
		& c \\
		Fi && Fj
		\arrow["Ff"', from=4-1, to=4-3]
		\arrow["{\lambda_i}", from=3-2, to=4-1]
		\arrow["{\lambda_j}"', from=3-2, to=4-3]
		\arrow["{\lambda_i'}"', from=2-2, to=4-1]
		\arrow["{\lambda_j'}", from=2-2, to=4-3]
		\arrow["f"{description}, from=2-2, to=3-2]
		\arrow["{\lambda_i}"', curve={height=12pt}, from=1-2, to=4-1]
		\arrow["{\lambda_j}", curve={height=-12pt}, from=1-2, to=4-3]
		\arrow["g"{description}, from=1-2, to=2-2]
	\end{tikzcd}\]
	By uniqueness, we see that $f\circ g$ must be the identity (there is only one such morphism from $c\to c$ making the diagram commute, and $\id_c$ works), so $f$ and $g$ are must be inverses by redoing the stacking with $f$ to show $g\circ f=\id_d$.
\end{proof}
\begin{notation}
	From now on, we will write $\lim F$ for the limit of $F$ and $\op{colim}F$ for $F$.
\end{notation}
\begin{remark}
	Not all categories have all their limits and colimits. For example, $\mathrm{Field}$ does not have an initial object (we cannot inject into both $\FF_2$ and $\FF_3$), so $\mathrm{Field}$ is missing the limit of the diagram from the empty category.
\end{remark}
We close with an example.
\begin{exe}
	We show that product are limits from the discrete category.
\end{exe}
\begin{proof}
	Fix our functor $F:\mathcal J\to\mathcal C$ with apex $(P,\pi)$, which means that we have morphisms $\pi_j:P\to Fj$. Note that we have no commutativity among the $j\in\mathcal J$ because $\mathcal J$ has no non-identity morphisms.

	To translate the universal property, we see that whenever we have another apex $(c,\lambda)$, there is a unique morphism $f:c\to P$ making the following diagram commute.
	% https://q.uiver.app/?q=WzAsNCxbMSwxLCJQIl0sWzEsMCwiYyJdLFswLDIsIkZpIl0sWzIsMiwiRmoiXSxbMiwzLCJGZiIsMl0sWzAsMiwiXFxwaV9pIl0sWzAsMywiXFxwaV9qIiwyXSxbMSwyLCJcXGxhbWJkYV9pIiwyLHsiY3VydmUiOjJ9XSxbMSwzLCJcXGxhbWJkYV9qIiwwLHsiY3VydmUiOi0yfV0sWzEsMCwiZiIsMV1d
	\[\begin{tikzcd}
		& c \\
		& P \\
		Fi && Fj
		\arrow["Ff"', from=3-1, to=3-3]
		\arrow["{\pi_i}", from=2-2, to=3-1]
		\arrow["{\pi_j}"', from=2-2, to=3-3]
		\arrow["{\lambda_i}"', curve={height=12pt}, from=1-2, to=3-1]
		\arrow["{\lambda_j}", curve={height=-12pt}, from=1-2, to=3-3]
		\arrow["f"{description}, from=1-2, to=2-2]
	\end{tikzcd}\]
	This is what we wrote down in \autoref{exe:produp}.
\end{proof}