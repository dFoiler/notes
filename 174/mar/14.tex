\documentclass[../notes.tex]{subfiles}

\begin{document}

% !TEX root = ../notes.tex

The fun, as they say, never stops.

\subsection{More Examples}
Chris is back, so today is just examples.
\begin{exe}
	We discuss the limit of the diagram
	\[A\stackrel f\to B.\]
\end{exe}
\begin{proof}
	The limit will be an object $X$ with a map $\iota:X\to A$ such that, for any object $Y$, there is a unique map $Y\to X$ making the following diagram commute.
	% https://q.uiver.app/?q=WzAsNCxbMCwxLCJBIl0sWzEsMSwiQiJdLFsxLDAsIlgiXSxbMCwwLCJZIl0sWzAsMSwiZiJdLFsyLDBdLFsyLDFdLFszLDIsIiIsMCx7InN0eWxlIjp7ImJvZHkiOnsibmFtZSI6ImRhc2hlZCJ9fX1dLFszLDAsIlxcdmFycGhpIiwyXV0=
	\[\begin{tikzcd}
		Y & X \\
		A & B
		\arrow["f", from=2-1, to=2-2]
		\arrow[from=1-2, to=2-1]
		\arrow[from=1-2, to=2-2]
		\arrow[dashed, from=1-1, to=1-2]
		\arrow["\varphi"', from=1-1, to=2-1]
	\end{tikzcd}\]
	Well, we simply set $X:=A$ with $X\to A$ simply as the identity map $\id_A:X\to A$. Then we are forced to have $Y\to X$ be $\varphi$ by the diagram commuting, which finishes.
\end{proof}
\begin{exe}
	We exhibit a product where the projection maps are not epimorphisms.
\end{exe}
\begin{proof}
	This is somewhat hard because faithful functors preserve epimorphisms, so concrete categories won't work here. So we consider the following category.
	% https://q.uiver.app/?q=WzAsNCxbMSwwLCJCIl0sWzAsMSwiQSJdLFsyLDEsIkMiXSxbMywxLCJEIl0sWzIsMywiIiwwLHsiY3VydmUiOi0xfV0sWzIsMywiIiwyLHsiY3VydmUiOjF9XSxbMCwyXSxbMCwxXSxbMCwzLCIiLDEseyJjdXJ2ZSI6LTIsInN0eWxlIjp7ImJvZHkiOnsibmFtZSI6ImRhc2hlZCJ9fX1dXQ==
	\[\begin{tikzcd}
		& B \\
		A && C & D
		\arrow[curve={height=-6pt}, from=2-3, to=2-4]
		\arrow[curve={height=6pt}, from=2-3, to=2-4]
		\arrow[from=1-2, to=2-3]
		\arrow[from=1-2, to=2-1]
		\arrow[curve={height=-12pt}, dashed, from=1-2, to=2-4]
	\end{tikzcd}\]
	It is not too hard to see that $B$ is the product of $A$ and $C$ (the only object with map to both $A$ and $C$ is $B$ itself, so it is our only object to check), but the map $B\to C$ is not an epimorphism because of the problems with $C\to D$.
\end{proof}

\subsection{Equalizers}
\begin{defi}[Equalizer]
	An \textit{equalizer} is a limit of the following diagram.
	% https://q.uiver.app/?q=WzAsMixbMCwwLCJcXGJ1bGxldCJdLFsxLDAsIlxcYnVsbGV0Il0sWzAsMSwiIiwwLHsib2Zmc2V0IjoxfV0sWzAsMSwiIiwyLHsib2Zmc2V0IjotMX1dXQ==
	% https://q.uiver.app/?q=WzAsMixbMCwwLCJBIl0sWzEsMCwiQiJdLFswLDEsImciLDIseyJvZmZzZXQiOjF9XSxbMCwxLCJmIiwwLHsib2Zmc2V0IjotMX1dXQ==
	\[\begin{tikzcd}
		A & B
		\arrow["g"', shift right=1, from=1-1, to=1-2]
		\arrow["f", shift left=1, from=1-1, to=1-2]
	\end{tikzcd}\]
	We denote this by $\op{eq}(f,g)$.
\end{defi}
More concretely, we set up our diagram as follows.
% https://q.uiver.app/?q=WzAsMyxbMCwxLCJBIl0sWzEsMSwiQiJdLFswLDAsIkUiXSxbMCwxLCJnIiwyLHsib2Zmc2V0IjoxfV0sWzAsMSwiZiIsMCx7Im9mZnNldCI6LTF9XSxbMiwwLCJlIiwyXSxbMiwxLCJlJyJdXQ==
\[\begin{tikzcd}
	E \\
	A & B
	\arrow["g"', shift right=1, from=2-1, to=2-2]
	\arrow["f", shift left=1, from=2-1, to=2-2]
	\arrow["e"', from=1-1, to=2-1]
	\arrow["{e'}", from=1-1, to=2-2]
\end{tikzcd}\]
By commutativity of the diagram, we want $fe=ge=e'$, so we will ignore the morphism $e'$ entirely: it's induced by the rest of the diagram.

Now, for $E$ to be universal, we are saying that, for any morphism $h:X\to A$, there is a unique morphism $X\to E$ making the following diagram commute.
% https://q.uiver.app/?q=WzAsNCxbMSwxLCJBIl0sWzIsMSwiQiJdLFswLDEsIkUiXSxbMCwwLCJYIl0sWzAsMSwiZyIsMix7Im9mZnNldCI6MX1dLFswLDEsImYiLDAseyJvZmZzZXQiOi0xfV0sWzIsMCwiZSJdLFszLDIsImgiLDIseyJzdHlsZSI6eyJib2R5Ijp7Im5hbWUiOiJkYXNoZWQifX19XSxbMywwLCJcXHZhcnBoaSJdXQ==
\[\begin{tikzcd}
	X \\
	E & A & B
	\arrow["g"', shift right=1, from=2-2, to=2-3]
	\arrow["f", shift left=1, from=2-2, to=2-3]
	\arrow["e", from=2-1, to=2-2]
	\arrow["h"', dashed, from=1-1, to=2-1]
	\arrow["\varphi", from=1-1, to=2-2]
\end{tikzcd}\]
This is not an obvious limit; here is an example.
\begin{exe}
	We compute equalizers in $\mathrm{Set}$.
\end{exe}
\begin{proof}
	As a starting example, we note that we do have a ``trivial'' cone with $X=\emp$. This does not use the other information of our limit, so we simply define
	\[E:=\{a\in A:f(a)=g(a)\}\]
	with inclusion morphism $\iota:E\subseteq A$. Certainly $f\iota=g\iota$ by construction.

	Now, to show the universal property, any other object $X$ with a morphism $h:X\to A$ such that $fh=gh$, we see that $h(x)\in E$ for each $x\in X$. Thus, $h$ does map into $E$, so we have our induced map
	\[\widetilde h:X\to E\]
	by simply restricting the codomain. This morphism is unique because any such morphism $\widetilde h$ must have $\iota\widetilde h=h$, so $\widetilde h(x)=h(x)$ for each $x\in X$.
\end{proof}
\begin{remark}
	I think the same construction will work for equalizers in any concrete category.
\end{remark}
\begin{exe}
	Working in $\mathrm{Ab}$, we consider the equalizer of the following diagram, where $f:A\to B$ is some morphism.
	% https://q.uiver.app/?q=WzAsMixbMCwwLCJBIl0sWzEsMCwiQiJdLFswLDEsIjAiLDIseyJvZmZzZXQiOjF9XSxbMCwxLCJmIiwwLHsib2Zmc2V0IjotMX1dXQ==
	\[\begin{tikzcd}
		A & B
		\arrow["0"', shift right=1, from=1-1, to=1-2]
		\arrow["f", shift left=1, from=1-1, to=1-2]
	\end{tikzcd}\]
	In particular, we claim that the equalizer is the kernel.
\end{exe}
\begin{proof}
	By essentially doing the same proof as in $\mathrm{Set}$, the equalizer will be the set
	\[E:=\{a\in A:f(a)=0(a)=0\},\]
	which is $\ker f$.
\end{proof}
\begin{remark}
	It follows that $\op{eq}(f,g)=\op{eq}(f-g,0)=\ker(f-g)$ by tracking through what we need for our diagrams to commute.
\end{remark}
Here is a nice result on equalizers.
\begin{prop}
	Given two morphisms $f,g:A\to B$ and an equalizer $e:E\to A$, the map $e$ is always monic
\end{prop}
\begin{proof}
	Fix two maps $h,k:X\to E$ such that $eh=ek$. This has the following diagram.
	% https://q.uiver.app/?q=WzAsNCxbMiwwLCJBIl0sWzMsMCwiQiJdLFsxLDAsIkUiXSxbMCwwLCJYIl0sWzAsMSwiZyIsMix7Im9mZnNldCI6MX1dLFswLDEsImYiLDAseyJvZmZzZXQiOi0xfV0sWzIsMCwiZSJdLFszLDIsImgiLDAseyJvZmZzZXQiOi0xfV0sWzMsMiwiayIsMix7Im9mZnNldCI6MX1dXQ==
	\[\begin{tikzcd}
		X & E & A & B
		\arrow["g"', shift right=1, from=1-3, to=1-4]
		\arrow["f", shift left=1, from=1-3, to=1-4]
		\arrow["e", from=1-2, to=1-3]
		\arrow["h", shift left=1, from=1-1, to=1-2]
		\arrow["k"', shift right=1, from=1-1, to=1-2]
	\end{tikzcd}\]
	Then we see that $eh$ and $ek$ both have $f(eh)=f(ek)=g(eh)=g(ek)$, so there is a unique map $x:X\to E$ such that $ex=eh=ek$. But then we see that $h$ and $k$ both work, so $h=k$ is forced.
\end{proof}
\begin{remark}
	This is notably different from projections failing to be epic because we are really only told that $p_Af=p_Bf$ or $p_Bf=p_Bg$ when looking at just one projection. However, we need both of these for $f=g$.
\end{remark}

\subsection{Coequalizers}
Of course, there is also a dual notion of an equalizer.
\begin{defi}[Coequalizer]
	A \textit{coequalizer} is a colimit of the following diagram.
	% https://q.uiver.app/?q=WzAsMixbMCwwLCJcXGJ1bGxldCJdLFsxLDAsIlxcYnVsbGV0Il0sWzAsMSwiIiwwLHsib2Zmc2V0IjoxfV0sWzAsMSwiIiwyLHsib2Zmc2V0IjotMX1dXQ==
	% https://q.uiver.app/?q=WzAsMixbMCwwLCJBIl0sWzEsMCwiQiJdLFswLDEsImciLDIseyJvZmZzZXQiOjF9XSxbMCwxLCJmIiwwLHsib2Zmc2V0IjotMX1dXQ==
	\[\begin{tikzcd}
		A & B
		\arrow["g"', shift right=1, from=1-1, to=1-2]
		\arrow["f", shift left=1, from=1-1, to=1-2]
	\end{tikzcd}\]
	We denote this by $\op{coeq}(f,g)$.
\end{defi}
From essentially the same discussion as before, the only data we need for a cocone of the diagram
\[\begin{tikzcd}
	A & B
	\arrow["g"', shift right=1, from=1-1, to=1-2]
	\arrow["f", shift left=1, from=1-1, to=1-2]
\end{tikzcd}\]
is an object $Q$ with a morphism $q:B\to Q$. The universal property is saying that any object $X$ with a morphism $\varphi:B\to X$ has a unique induced morphism as follows.
% https://q.uiver.app/?q=WzAsNCxbMCwwLCJBIl0sWzEsMCwiQiJdLFsyLDAsIlEiXSxbMiwxLCJYIl0sWzAsMSwiZyIsMix7Im9mZnNldCI6MX1dLFswLDEsImYiLDAseyJvZmZzZXQiOi0xfV0sWzEsMiwicSJdLFsyLDMsIiIsMCx7InN0eWxlIjp7ImJvZHkiOnsibmFtZSI6ImRhc2hlZCJ9fX1dLFsxLDMsIlxcdmFycGhpIiwyXV0=
\[\begin{tikzcd}
	A & B & Q \\
	&& X
	\arrow["g"', shift right=1, from=1-1, to=1-2]
	\arrow["f", shift left=1, from=1-1, to=1-2]
	\arrow["q", from=1-2, to=1-3]
	\arrow[dashed, from=1-3, to=2-3]
	\arrow["\varphi"', from=1-2, to=2-3]
\end{tikzcd}\]
And now for examples.
\begin{exe}
	We compute coequalizers in $\mathrm{Set}$.
\end{exe}
\begin{proof}
	The ``dual'' to a subset is a quotient, so we have reason to believe that the coequalizer should be a quotient. Thus, we define the equivalence relation $\sim$ in $B$ generated by $f(a)\sim g(a)$.\footnote{For example, we can take the intersection of all equivalence relations $B\times B$ which contain the requirements $f(a)\sim g(a)$ for each $a\in A$.} It will happen that the canonical projection map $B\onto B/\sim$ is our coequalizer.
\end{proof}

\end{document}