% LTeX: enabled=false

\documentclass[openany]{book}
\usepackage[utf8]{inputenc}

\newcommand{\nirpdftitle}{174 Notes}
\usepackage{import}
\inputfrom{..}{nir}

\pagestyle{contentpage}

\title{174: Category Theory}
\author{Nir Elber}
\date{Spring 2022}
\rhead{\textit{174: CAT. THEORY}}

\begin{document}

\maketitle

\toctrue
\tableofcontents
\tocfalse

\newpage

\chapter{Basic Definitions}

\epigraph{Category theory is much easier once you realize that it is designed to formalize and abstract things you already know.}
{---Ravi Vakil}
% http://math.stanford.edu/~vakil/216blog/FOAGnov1817public.pdf

\foreach \n in {19,21,24}
{
	\section{January \n}
	\input{jan/\n}
}

\chapter{Functors and Natural Transformations}

\epigraph{Mathematics is the art of giving the same names to different things}
{---Henri Poincar\'e}

\foreach \n in {26,31}
{
	\section{January \n}
	\input{jan/\n}
}

\foreach \n in {2,7,9,11,14,16,18}
{
	\section{February \n}
	\input{feb/\n}
}

\chapter{Universal Properties}

\epigraph{The Yoneda embedding, contravariant it is.}
{---Mike Stay}
% http://math.stanford.edu/~vakil/216blog/FOAGnov1817public.pdf

\foreach \n in {23,25,28}
{
	\section{February \n}
	\input{feb/\n}
}

\foreach \n in {2,4,7}
{
	\section{March \n}
	\input{mar/\n}
}

\chapter{Limits and Colimits}

\epigraph{It's true that many pieces of categorical terminology do come from analysis, but maybe all that says is that analysis is an old and venerable subject.}
{---Tom Leinster}
% https://mathoverflow.net/a/6564/473811

\foreach \n in {9,11,14,16,18,28,30}
{
	\section{March \n}
	\input{mar/\n}
}

\foreach \n in {1,4,6,8}
{
	\section{April \n}
	\input{apr/\n}
}

\chapter{Adjoints}

\epigraph{By Insane Moon Logic, it sort of makes sense.}
{---Scott Alexander}

\foreach \n in {11,13,15,18,20,22}
{
	\section{April \n}
	\input{apr/\n}
}

\chapter{Kan Extensions}

\epigraph{You take the red pill, you stay in wonderland, and I show you how deep the rabbit hole goes.}
{---Morpheus}

\foreach \n in {25,27}
{
	\section{April \n}
	\input{apr/\n}
}

% Yoneda to limits in Set are "Universal Properties"?
% Everything after probably fits into "Adjunctions"

\nirprintindex

\end{document}