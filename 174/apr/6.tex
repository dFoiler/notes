\documentclass[../notes.tex]{subfiles}

\begin{document}

% !TEX root = ../notes.tex

Today is almost the last day of discussing limits.

\subsection{Limit Functoriality}
We begin by talking about functoriality of limits.
\begin{restatable}{proposition}{limfunc}
	Fix a category $\mathcal C$ with all limits of shape $\mathcal J$. Upon choosing a limit and associated limit cone for each diagram $\mathcal J\to\mathcal C$ defines the action of objects of a functor
	\[\lim_\mathcal J:\op{Fun}(\mathcal J,\mathcal C)\to\mathcal C.\]
\end{restatable}
\begin{proof}
	As promised, for each $F:\mathcal J\to\mathcal C$, we choose the limit cone $\lambda:\lim_\mathcal JF\Rightarrow F$. For the resulting argument, we need three diagrams $F,G,H$ with cones $\lambda,\mu,\eta$ over $F,G,H$ respectively.

	We know $\lim_\mathcal J$ is supposed to do on objects, so we pick up a morphism $\alpha:F\Rightarrow G$ and take it to a morphism of limit objects. Well, note that we can consider the composite of natural transformations
	\[\lim_\mathcal JF\stackrel\lambda\Rightarrow F\stackrel\alpha\Rightarrow G.\]
	In particular, this cone over $G$ induces a unique arrow
	\[\lim_\mathcal JF\to\lim_\mathcal J,\]
	which we suggestively call $\lim_\mathcal J\alpha$. It remains to run our functoriality checks.
	\begin{itemize}
		\item Identity: we check $\lim_\mathcal J\id_F=\id_{\lim_\mathcal JF}$. We have the following diagram.
		% https://q.uiver.app/?q=WzAsNCxbMCwwLCJcXGRpc3BsYXlzdHlsZVxcbGltX3tcXG1hdGhjYWwgSn1GIl0sWzAsMSwiRiJdLFsxLDEsIkYiXSxbMSwwLCJcXGRpc3BsYXlzdHlsZVxcbGltX3tcXG1hdGhjYWwgSn1GIl0sWzEsMiwiXFxpZF9GIiwyLHsibGV2ZWwiOjJ9XSxbMCwxLCJcXGxhbWJkYSIsMix7ImxldmVsIjoyfV0sWzMsMiwiXFxsYW1iZGEiLDAseyJsZXZlbCI6Mn1dLFswLDMsIiIsMCx7InN0eWxlIjp7ImJvZHkiOnsibmFtZSI6ImRhc2hlZCJ9fX1dXQ==
		\[\begin{tikzcd}
			{\displaystyle\lim_{\mathcal J}F} & {\displaystyle\lim_{\mathcal J}F} \\
			F & F
			\arrow["{\id_F}"', Rightarrow, from=2-1, to=2-2]
			\arrow["\lambda"', Rightarrow, from=1-1, to=2-1]
			\arrow["\lambda", Rightarrow, from=1-2, to=2-2]
			\arrow[dashed, from=1-1, to=1-2]
		\end{tikzcd}\]
		If we place $\id_{\lim_\mathcal JF}$ on the top arrow, then the diagram commutes, so we conclude that
		\[\lim_\mathcal J\id_F=\id_{\lim_\mathcal JF}\]
		by uniqueness.
		\item Associativity: given morphisms $\alpha:F\Rightarrow G$ and $\beta:G\Rightarrow H$, which looks like the following diagram.
		% https://q.uiver.app/?q=WzAsNixbMCwwLCJcXGRpc3BsYXlzdHlsZVxcbGltX3tcXG1hdGhjYWwgSn1GIl0sWzAsMSwiRiJdLFsxLDEsIkciXSxbMSwwLCJcXGRpc3BsYXlzdHlsZVxcbGltX3tcXG1hdGhjYWwgSn1HIl0sWzIsMCwiXFxkaXNwbGF5c3R5bGVcXGxpbV97XFxtYXRoY2FsIEp9SCJdLFsyLDEsIkgiXSxbMSwyLCJcXGFscGhhIiwyLHsibGV2ZWwiOjJ9XSxbMCwxLCJcXGxhbWJkYSIsMix7ImxldmVsIjoyfV0sWzMsMiwiXFxtdSIsMCx7ImxldmVsIjoyfV0sWzAsMywiIiwwLHsic3R5bGUiOnsiYm9keSI6eyJuYW1lIjoiZGFzaGVkIn19fV0sWzMsNCwiIiwwLHsic3R5bGUiOnsiYm9keSI6eyJuYW1lIjoiZGFzaGVkIn19fV0sWzIsNSwiXFxiZXRhIiwyLHsibGV2ZWwiOjJ9XSxbNCw1LCJcXGV0YSIsMCx7ImxldmVsIjoyfV1d
		\[\begin{tikzcd}
			{\displaystyle\lim_{\mathcal J}F} & {\displaystyle\lim_{\mathcal J}G} & {\displaystyle\lim_{\mathcal J}H} \\
			F & G & H
			\arrow["\alpha"', Rightarrow, from=2-1, to=2-2]
			\arrow["\lambda"', Rightarrow, from=1-1, to=2-1]
			\arrow["\mu", Rightarrow, from=1-2, to=2-2]
			\arrow[dashed, from=1-1, to=1-2]
			\arrow[dashed, from=1-2, to=1-3]
			\arrow["\beta"', Rightarrow, from=2-2, to=2-3]
			\arrow["\eta", Rightarrow, from=1-3, to=2-3]
		\end{tikzcd}\]
		This diagram commutes, so the arrow $\lim_{\mathcal J}F\to\lim_{\mathcal J}H$ can be made $(\lim_{\mathcal J}\beta)(\lim_{\mathcal J}\alpha)$ or $(\lim_{\mathcal J}\beta\alpha)$, so they are the same by uniqueness.
	\end{itemize}
	This finishes our functoriality check.
\end{proof}
\begin{remark}
	We might be able to have $\lim_\mathcal J$ output into cones instead of $\mathcal C$, but we won't bother.
\end{remark}
\begin{remark}
	We have to choose all the limit cones in advance, which is somewhat annoying.
\end{remark}

\subsection{Limits of Limits}
We also talk a little about limits of limits. Note that a functor $F:\mathcal I\times\mathcal J\to\mathcal C$ more or less induces two different functors
\[F:\mathcal I\to\op{Fun}(\mathcal J,\mathcal C)\]
by taking $i\in\mathcal I$ to the functor $j\mapsto F(i,j)$. If we wanted to be careful, we would note that we should send morphisms $f:j\to j'$ to $F(\id_i,f)$.
\begin{remark}
	Formally speaking, we are kind of saying that
	\[\op{Fun}^2(\mathcal I\times\mathcal J)\simeq\op{Fun}^2(\mathcal I,\op{Fun}(\mathcal J,\mathcal C)).\]
	So the category of categories is Cartesian-closed. I only wrote down that sentence for its meme value.
\end{remark}
Anyway, here is our statement.
\begin{theorem} \label{thm:limlim}
	Fix a locally small category $\mathcal C$. If the limits
	\[\lim_{i\in\mathcal I}\lim_{j\in\mathcal J}F(i,j)\qquad\text{and}\qquad\lim_{j\in\mathcal J}\lim_{i\in\mathcal I}F(i,j)\]
	exist, then they isomorphic and isomorphic to $\lim_{\mathcal I\times\mathcal J}F$.
\end{theorem}
Formally, we are viewing the limit
\[\lim_{i\in\mathcal I}\lim_{j\in\mathcal J}F(i,j)\]
as first taking the limit of $F$ over $\mathcal J$ (viewing $F$ as a functor from $\mathcal J$ to $\op{Fun}(\mathcal I,\mathcal C$) and then taking the limit resulting limit over $\mathcal I$.

We start with some motivating exercise.
\begin{exe}
	We draw the diagrams for equalizers commuting with pullbacks.
\end{exe}
\begin{proof}
	Consider the following two categories.
	% https://q.uiver.app/?q=WzAsNyxbMCwwLCJcXG1hdGhjYWwgSSJdLFsxLDAsIlxcYnVsbGV0Il0sWzIsMCwiXFxidWxsZXQiXSxbMCwxLCJcXG1hdGhjYWwgSiJdLFsxLDEsIlxcYnVsbGV0Il0sWzIsMSwiXFxidWxsZXQiXSxbMywxLCJcXGJ1bGxldCJdLFsxLDIsIiIsMCx7Im9mZnNldCI6MX1dLFsxLDIsIiIsMix7Im9mZnNldCI6LTF9XSxbNCw1XSxbNiw1XV0=
	\[\begin{tikzcd}
		{\mathcal I} & \bullet & \bullet \\
		{\mathcal J} & \bullet & \bullet & \bullet
		\arrow[shift right=1, from=1-2, to=1-3]
		\arrow[shift left=1, from=1-2, to=1-3]
		\arrow[from=2-2, to=2-3]
		\arrow[from=2-4, to=2-3]
	\end{tikzcd}\]
	Then the product can be realized as follows.
	% https://q.uiver.app/?q=WzAsNixbMCwxLCJcXGJ1bGxldCJdLFsxLDIsIlxcYnVsbGV0Il0sWzEsMSwiXFxidWxsZXQiXSxbMiwyLCJcXGJ1bGxldCJdLFsxLDAsIlxcYnVsbGV0Il0sWzIsMSwiXFxidWxsZXQiXSxbMCwyXSxbNCwyXSxbMSwzXSxbNSwzXSxbMCwxLCIiLDIseyJvZmZzZXQiOjF9XSxbMSwwLCIiLDEseyJvZmZzZXQiOjF9XSxbMiwzLCIiLDEseyJvZmZzZXQiOjF9XSxbMywyLCIiLDEseyJvZmZzZXQiOjF9XSxbNSw0LCIiLDIseyJvZmZzZXQiOjF9XSxbNCw1LCIiLDEseyJvZmZzZXQiOjF9XV0=
	\[\begin{tikzcd}
		& \bullet \\
		\bullet & \bullet & \bullet \\
		& \bullet & \bullet
		\arrow[from=2-1, to=2-2]
		\arrow[from=1-2, to=2-2]
		\arrow[from=3-2, to=3-3]
		\arrow[from=2-3, to=3-3]
		\arrow[shift right=1, from=2-1, to=3-2]
		\arrow[shift right=1, from=3-2, to=2-1]
		\arrow[shift right=1, from=2-2, to=3-3]
		\arrow[shift right=1, from=3-3, to=2-2]
		\arrow[shift right=1, from=2-3, to=1-2]
		\arrow[shift right=1, from=1-2, to=2-3]
	\end{tikzcd}\]
	There are a few ways to compute this limit.
	\begin{itemize}
		\item We might take the limit over the two copies of $\mathcal J$ first (i.e., take the pullbacks first), which with our two limits will end up looking like the following.
		% https://q.uiver.app/?q=WzAsMixbMCwwLCJcXGJ1bGxldCJdLFsxLDEsIlxcYnVsbGV0Il0sWzAsMSwiIiwwLHsib2Zmc2V0IjoxfV0sWzAsMSwiIiwyLHsib2Zmc2V0IjotMX1dXQ==
		\[\begin{tikzcd}
			\bullet \\
			& \bullet
			\arrow[shift right=1, from=1-1, to=2-2]
			\arrow[shift left=1, from=1-1, to=2-2]
		\end{tikzcd}\]
		Then we can compute the limit over this diagram.
		\item We might take the limit over each of the three copies of $\mathcal I$ first (i.e., take the equalizers first), which will give us a diagram that looks like the following.
		% https://q.uiver.app/?q=WzAsMyxbMCwxLCJcXGJ1bGxldCJdLFsxLDEsIlxcYnVsbGV0Il0sWzEsMCwiXFxidWxsZXQiXSxbMCwxXSxbMiwxXV0=
		\[\begin{tikzcd}
			& \bullet \\
			\bullet & \bullet
			\arrow[from=2-1, to=2-2]
			\arrow[from=1-2, to=2-2]
		\end{tikzcd}\]	
		Then we can compute the pullbacks here.
	\end{itemize}
	The theorem is saying that the two ways to compute the limit actually coincide.
\end{proof}
Anyway, here is the proof of our theorem.
\begin{proof}[Proof of \autoref{thm:limlim}]
	By \autoref{thm:yonedaembedding}, it suffices to show that
	\[\op{Mor}\left(X,\lim_{\mathcal I}\lim_{\mathcal J}F\right)\cong\op{Mor}\left(X,\lim_{\mathcal I\times\mathcal J}F\right)\]
	naturally. We will show the left isomorphism, and the other one follows by symmetry.
	
	Further, taking the limits out of the morphism sets, we can assume that everything is in set, where we know how to compute limits already. In particular, using the construction in \autoref{prop:setlims}, we have
	\[\lim_{\mathcal I\times\mathcal J}F=\op{Cone}(*,F)\qquad\text{and}\qquad\lim_{\mathcal I}\lim_{\mathcal J}F=\op{Cone}\left(*,\lim_{j\in\mathcal J}F(-,j)\right),\]
	where the second is by tracking what $F$ means in this case. Explicitly, $\lim_{j\in\mathcal J}F(-,j)$ is a functor $\mathcal I\to\mathcal C$. We construct these isomorphisms by hand.
	\begin{itemize}
		\item We construct a morphism
		\[\varphi:\op{Cone}(*,F)\to\op{Cone}\left(*,\lim_{j\in\mathcal J}F(-,j)\right).\]
		Well, picking up a cone $\lambda:*\Rightarrow F$, we see we are asking for a collection of morphisms
		\[\varphi(\lambda)_i:*\to\lim_{j\in\mathcal J}F(i,j)\]
		for each $i\in\mathcal I$. As such, fix some $i\in\mathcal I$. Now, the cone $\lambda$ promises us morphisms $\lambda_{i,j}:*\to F(i,j)$. In particular, $j\mapsto\lambda_{(i,j)}$ gives us a cone which looks like the following.
		% https://q.uiver.app/?q=WzAsMyxbMCwxLCJGKGksaikiXSxbMCwwLCIqIl0sWzEsMCwiXFxkaXNwbGF5c3R5bGVcXGxpbV97alxcaW5cXG1hdGhjYWwgSn1GKGksaikiXSxbMSwwLCJcXGxhbWJkYV97aSxqfSIsMix7ImxldmVsIjoyfV0sWzIsMCwiIiwwLHsibGV2ZWwiOjJ9XSxbMSwyLCIiLDAseyJzdHlsZSI6eyJib2R5Ijp7Im5hbWUiOiJkYXNoZWQifX19XV0=
		\[\begin{tikzcd}
			{*} & {\displaystyle\lim_{j\in\mathcal J}F(i,j)} \\
			{F(i,j)}
			\arrow["{\lambda_{i,j}}"', Rightarrow, from=1-1, to=2-1]
			\arrow[Rightarrow, from=1-2, to=2-1]
			\arrow[dashed, from=1-1, to=1-2]
		\end{tikzcd}\]
		Notably, the above diagram has $j$ be variable. We now define $\varphi(\lambda)_i$ to be the induced map in the above diagram.

		We now check that $\varphi(\lambda)$ is really a cone. Well, set $f:i\to i'$ to be a morphism in $\mathcal I$. Thus, we need to check
		\[\lim{j\in\mathcal J}F(f,\id_j)\circ\varphi(\lambda)_i\stackrel?=\varphi(\lambda)_{i'}.\]
		To check this, we draw the following associated diagram.
		% https://q.uiver.app/?q=WzAsNSxbMSwwLCJcXGRpc3BsYXlzdHlsZVxcbGltX3tqXFxpblxcbWF0aGNhbCBKfUYoaSxqKSJdLFsyLDAsIlxcZGlzcGxheXN0eWxlXFxsaW1fe2pcXGluXFxtYXRoY2FsIEp9RihpJyxqKSJdLFsxLDEsIkYoaSxqKSJdLFsyLDEsIkYoaScsaikiXSxbMCwwLCIqIl0sWzAsMiwiIiwwLHsibGV2ZWwiOjJ9XSxbMSwzLCIiLDAseyJsZXZlbCI6Mn1dLFswLDEsIiIsMSx7InN0eWxlIjp7ImJvZHkiOnsibmFtZSI6ImRhc2hlZCJ9fX1dLFsyLDMsIkYoZixcXGlkX2opIiwyLHsibGV2ZWwiOjJ9XSxbNCwwLCJcXHZhcnBoaShcXGxhbWJkYV9pKSJdLFs0LDIsIlxcbGFtYmRhX3tpLGp9IiwyLHsibGV2ZWwiOjJ9XV0=
		\[\begin{tikzcd}
			{*} & {\displaystyle\lim_{j\in\mathcal J}F(i,j)} & {\displaystyle\lim_{j\in\mathcal J}F(i',j)} \\
			& {F(i,j)} & {F(i',j)}
			\arrow[Rightarrow, from=1-2, to=2-2]
			\arrow[Rightarrow, from=1-3, to=2-3]
			\arrow[dashed, from=1-2, to=1-3]
			\arrow["{F(f,\id_j)}"', Rightarrow, from=2-2, to=2-3]
			\arrow["{\varphi(\lambda_i)}", from=1-1, to=1-2]
			\arrow["{\lambda_{i,j}}"', Rightarrow, from=1-1, to=2-2]
		\end{tikzcd}\]
		Now, our functoriality is giving the map on the top right as $\lim_{\mathcal J}F(f,\id_j)$. The triangle and the square both commute by construction of each, so the full diagram commutes, but then we see that $\varphi(\lambda)_{i'}$ is the only map from $*$ to $\lim_{\mathcal J}F(i',j)$ making the diagram commute, so we get
		\[\lim{j\in\mathcal J}F(f,\id_j)\circ\varphi(\lambda)_i\stackrel?=\varphi(\lambda)_{i'}.\]
		\item We construct a morphism
		\[\psi:\op{Cone}\left(*,\lim_{\mathcal J}F(-,j)\right)\to\op{Cone}(*,F).\]
		This is somewhat hard. Let $\mu:*\Rightarrow\lim_{\mathcal J}F(-,j)$ be some cone so that we want a morphism
		\[\psi(\mu)_{i,j}:*\to F(i,j).\]
		As such, we let $\rho_{i,j}:\lim_{j\in\mathcal J}F(i,j)\to F(i,j)$ be the canonical projection. As such, we define
		\[\psi(\mu)_{i,j}:=\rho_{i,j}\mu_i\]
		by simply composing our two cones.

		It remains to show that we actually have a cone. Well, pick up $(f,g):(i,j)\to(i',j')$ some morphism. Then we compute
		\[F(f,g)\psi(\mu)_{i,j}=F(f,\id_{j'})F(\id_{i'},g)\rho_{i,j}\mu_i\]
		by moving around our definitions. Now, $F(\id_{i'},g)\rho_{i,j}=\rho_{i,j'}$ because $\rho$ is a cone. So we so far are dealing with
		\[F(f,\id_{j'})\rho_{i,j'}\mu_i.\]
		For this, we draw the following diagram.
		% https://q.uiver.app/?q=WzAsNSxbMSwwLCIqIl0sWzAsMSwiXFxsaW1fe1xcbWF0aGNhbCBKfUYoaSxqKSJdLFsyLDEsIlxcbGltX3tcXG1hdGhjYWwgSn1GKGknLGopIl0sWzIsMiwiRihpJyxqKSJdLFswLDIsIkYoaSxqKSJdLFswLDEsIlxcbXVfaSIsMl0sWzAsMiwiXFxtdV97aSd9Il0sWzIsMywiXFxyaG9fe2knLGp9IiwwLHsibGV2ZWwiOjJ9XSxbMSw0LCJcXHJob197aSxqfSIsMix7ImxldmVsIjoyfV0sWzEsMiwiXFxsaW1fe1xcbWF0aGNhbCBKfUYoZixqJykiLDJdLFs0LDMsIkYoZixqKSIsMl1d
		\[\begin{tikzcd}
			& {*} \\
			{\lim_{\mathcal J}F(i,j)} && {\lim_{\mathcal J}F(i',j)} \\
			{F(i,j')} && {F(i',j')}
			\arrow["{\mu_i}"', from=1-2, to=2-1]
			\arrow["{\mu_{i'}}", from=1-2, to=2-3]
			\arrow["{\rho_{i',j'}}", from=2-3, to=3-3]
			\arrow["{\rho_{i,j'}}"', from=2-1, to=3-1]
			\arrow["{\lim_{\mathcal J}F(f,\id_j)}"', from=2-1, to=2-3]
			\arrow["{F(f,\id_{j'})}"', from=3-1, to=3-3]
		\end{tikzcd}\]
		The top triangle commutes because $\mu$ is a cone, and the bottom square commutes by definition of the top map of the square. As such, we see that
		\[F(f,\id_{j'})\rho_{i,j'}\mu_{i}=\rho_{i',j'}\mu_{i'}=\psi(\mu)_{i',j'}\]
		by the commutativity of the diagram. This finishes showing that $\psi(\mu)$ assembles into a cone.
	\end{itemize}
	It remains to check that $\varphi$ and $\psi$ are mutually inverse and natural, but we won't bother proving these.
\end{proof}

\end{document}