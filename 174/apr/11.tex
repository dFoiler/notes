% !TEX root = ../notes.tex

We finally begin our discussion of adjoints.
\begin{remark}
	Being shocked for thirty minutes after lecturing is strongly cringe.
\end{remark}

\subsection{Introducing Adjunctions}
Given two categories $\mathcal C$ and $\mathcal D$, we can ask for the following relations, listed in order of strength.
\begin{itemize}
	\item We can ask for categories to be equal. This is very strong.
	\item We can ask for categories to be isomorphic. This is evil.
	\item We can ask for categories to be equivalent. This is a good notion, but it is also somewhat strong.
\end{itemize}
For example, we might want to think of $\mathrm{Set}$ and $\mathrm{Grp}$ to be related. Of course, they are not equivalent (for example, $\mathrm{Grp}$ has a zero object $\{e\}$ while $\mathrm{Set}$ does not).

The goal for today is to talk about a weaker notion than equivalence (so that we have lots of interesting examples) that is still powerful enough to be useful. Here is our definition.
\begin{definition}
	Fix two categories $\mathcal C$ and $\mathcal D$. An \textit{adjunction} is a pair of functors $F:\mathcal C\to\mc D$ and $G:\mc D\to\mc C$ with isomorphisms (of sets)
	\[\op{Mor}_{\mc D}(Fc,d)\cong\op{Mor}_{\mc C}(c,Gd)\]
	natural in both arguments. We will call $F$ a \textit{left adjoint} to $G$ and $G$ a \textit{right adjoint} to $F$.
\end{definition}
\begin{notation}
	We will write the pair of functors $F:\mathcal C\to\mc D$ and $G:\mc D\to\mc C$ as $F:\mc C\rightleftharpoons\mc D:G$.
\end{notation}
\begin{remark}
	It turns out that adjoints are unique up to some notion of isomorphism, so we may write ``the'' adjoint. However, we will not prove this uniqueness for a while.
\end{remark}
Let's actually write out our naturality square. For $c$, if $f:c\to c'$ is a morphism, we need the following diagram to commute.
% https://q.uiver.app/?q=WzAsNCxbMCwwLCJcXG9we01vcn1fe1xcbWMgRH0oRmMnLGQpIl0sWzAsMSwiXFxvcHtNb3J9X3tcXG1jIER9KEZjLGQpIl0sWzEsMCwiXFxvcHtNb3J9X3tcXG1jIEN9KGMnLEdkKSJdLFsxLDEsIlxcb3B7TW9yfV97XFxtYyBDfShjLEdkKSJdLFswLDEsIi1cXGNpcmMgRmYiLDJdLFsyLDMsIi1cXGNpcmMgZiJdLFswLDIsIlxcY29uZyIsMSx7InN0eWxlIjp7ImJvZHkiOnsibmFtZSI6Im5vbmUifSwiaGVhZCI6eyJuYW1lIjoibm9uZSJ9fX1dLFsxLDMsIlxcY29uZyIsMSx7InN0eWxlIjp7ImJvZHkiOnsibmFtZSI6Im5vbmUifSwiaGVhZCI6eyJuYW1lIjoibm9uZSJ9fX1dXQ==&macro_url=https%3A%2F%2Fraw.githubusercontent.com%2FdFoiler%2Fnotes%2Fmaster%2Fnir.tex
\[\begin{tikzcd}
	{\op{Mor}_{\mc D}(Fc',d)} & {\op{Mor}_{\mc C}(c',Gd)} \\
	{\op{Mor}_{\mc D}(Fc,d)} & {\op{Mor}_{\mc C}(c,Gd)}
	\arrow["{-\circ Ff}"', from=1-1, to=2-1]
	\arrow["{-\circ f}", from=1-2, to=2-2]
	\arrow["\cong"{description}, draw=none, from=1-1, to=1-2]
	\arrow["\cong"{description}, draw=none, from=2-1, to=2-2]
\end{tikzcd}\]
And here is the other naturality square, for a morphism $g:d\to d'$.
% https://q.uiver.app/?q=WzAsNCxbMCwwLCJcXG9we01vcn1fe1xcbWMgRH0oRmMsZCkiXSxbMCwxLCJcXG9we01vcn1fe1xcbWMgRH0oRmMsZCcpIl0sWzEsMCwiXFxvcHtNb3J9X3tcXG1jIEN9KGMsR2QpIl0sWzEsMSwiXFxvcHtNb3J9X3tcXG1jIEN9KGMsR2QnKSJdLFswLDEsImdcXGNpcmMtIiwyXSxbMiwzLCJHZ1xcY2lyYy0iXSxbMCwyLCJcXGNvbmciLDEseyJzdHlsZSI6eyJib2R5Ijp7Im5hbWUiOiJub25lIn0sImhlYWQiOnsibmFtZSI6Im5vbmUifX19XSxbMSwzLCJcXGNvbmciLDEseyJzdHlsZSI6eyJib2R5Ijp7Im5hbWUiOiJub25lIn0sImhlYWQiOnsibmFtZSI6Im5vbmUifX19XV0=&macro_url=https%3A%2F%2Fraw.githubusercontent.com%2FdFoiler%2Fnotes%2Fmaster%2Fnir.tex
\[\begin{tikzcd}
	{\op{Mor}_{\mc D}(Fc,d)} & {\op{Mor}_{\mc C}(c,Gd)} \\
	{\op{Mor}_{\mc D}(Fc,d')} & {\op{Mor}_{\mc C}(c,Gd')}
	\arrow["{g\circ-}"', from=1-1, to=2-1]
	\arrow["{Gg\circ-}", from=1-2, to=2-2]
	\arrow["\cong"{description}, draw=none, from=1-1, to=1-2]
	\arrow["\cong"{description}, draw=none, from=2-1, to=2-2]
\end{tikzcd}\]
Here is one last piece of notation.
\begin{notation}
	We will notate $f^\sharp:Fc\to d$ being ``adjoint'' or ``transpose'' to the morphism $f^\flat:c\to Gd$, by the isomorphism promised by the adjunction.
\end{notation}
Let's make the naturality a little faster; it turns out that the naturality is equivalent to having a natural isomorphism as follows.
% https://q.uiver.app/?q=WzAsMixbMCwwLCJcXG1jIENcXG9wcFxcdGltZXNcXG1jIEQiXSxbMiwwLCJcXG1hdGhybXtTZXR9Il0sWzAsMSwiXFxtYyBEKEYtLC0pIiwwLHsiY3VydmUiOi0yfV0sWzAsMSwiXFxtYyBDKC0sRy0pIiwyLHsiY3VydmUiOjJ9XSxbMiwzLCJcXGNvbmciLDAseyJzaG9ydGVuIjp7InNvdXJjZSI6MjAsInRhcmdldCI6MjB9fV1d&macro_url=https%3A%2F%2Fraw.githubusercontent.com%2FdFoiler%2Fnotes%2Fmaster%2Fnir.tex
\[\begin{tikzcd}
	{\mc C\opp\times\mc D} && {\mathrm{Set}}
	\arrow[""{name=0, anchor=center, inner sep=0}, "{\mc D(F-,-)}", curve={height=-12pt}, from=1-1, to=1-3]
	\arrow[""{name=1, anchor=center, inner sep=0}, "{\mc C(-,G-)}"', curve={height=12pt}, from=1-1, to=1-3]
	\arrow["\cong", shorten <=3pt, shorten >=3pt, Rightarrow, from=0, to=1]
\end{tikzcd}\]
Take a moment to verify that the functors actually type-check.

Here is another way to promote the efficiency.
\begin{lemma} \label{lem:betteradjoint}
	Fix functors $F:\mc C\rightleftharpoons\mc D:G$ with given isomorphisms $\mc D(Fc,d)\cong\mc C(c,Gd)$ for $c\in\mc C$ and $d\in\mc D$. Then, naturality is equivalent to the following: one of the squares below commutes if and only if the other does, for any morphisms making the diagrams well-defined.
	% https://q.uiver.app/?q=WzAsOCxbMCwwLCJGYyJdLFswLDEsIkZjJyJdLFsxLDAsImQiXSxbMSwxLCJkJyJdLFsyLDAsImMiXSxbMywwLCJHZCJdLFsyLDEsImMnIl0sWzMsMSwiR2QnIl0sWzAsMiwiZl5cXHNoYXJwIl0sWzEsMywiZ15cXHNoYXJwIiwyXSxbMCwxLCJGaCIsMl0sWzIsMywiayJdLFs0LDYsImgiLDJdLFs0LDUsImZeXFxmbGF0Il0sWzYsNywiZ15cXGZsYXQiLDJdLFs1LDcsIkdrIl1d&macro_url=https%3A%2F%2Fraw.githubusercontent.com%2FdFoiler%2Fnotes%2Fmaster%2Fnir.tex
	\[\begin{tikzcd}
		Fc & d & c & Gd \\
		{Fc'} & {d'} & {c'} & {Gd'}
		\arrow["{f^\sharp}", from=1-1, to=1-2]
		\arrow["{g^\sharp}"', from=2-1, to=2-2]
		\arrow["Fh"', from=1-1, to=2-1]
		\arrow["k", from=1-2, to=2-2]
		\arrow["h"', from=1-3, to=2-3]
		\arrow["{f^\flat}", from=1-3, to=1-4]
		\arrow["{g^\flat}"', from=2-3, to=2-4]
		\arrow["Gk", from=1-4, to=2-4]
	\end{tikzcd}\]
\end{lemma}
\begin{proof}
	This is supposedly on the homework.
\end{proof}
In general, the above condition

\subsection{Examples}
We are feeling kind today, so let's see some examples.
\begin{exe}
	The functor $F:\mathrm{Set}\to\mathrm{Grp}$ sending a group $X$ to the free group on $X$ is left adjoint to the forgetful functor $U:\mathrm{Grp}\to\mathrm{Set}$.
\end{exe}
\begin{proof}
	The point is that, given a set $X$ and group $G$, a morphism $FX\to G$ is in some sense the ``same data'' as a morphism $X\to FG$. For example, given a morphism $f^\flat:X\to UG$, we define $f^\sharp:FX\to G$ by
	\[f^\sharp(x_1x_2\cdots x_n)=f^\flat(x_1)f^\flat(x_2)\cdots f^\flat(x_n).\]
	This turns out to be an isomorphism, and it turns out that we can show naturality everywhere, but we will not bother.
\end{proof}
\begin{remark}
	There are many examples of ``free-forgetful'' adjunctions.
\end{remark}
\begin{remark}
	Philosophically, a right adjoint poses a question which the left adjoint answers. For example, $U:\mathrm{Grp}\to\mathrm{Set}$ is asking how to get maps of groups from sets, which the free functor tells us how to do.
\end{remark}
\begin{exe}
	The functor $F:\mathrm{Set}\to\mathrm{Top}$ taking a set $X$ to the topological space $X$ equipped with the discrete topology is left adjoint to the forgetful functor $U:\mathrm{Top}\to\mathrm{Set}$.

	Similarly, the functor $G:\mathrm{Set}\to\mathrm{Top}$ taking a set $X$ to the topological space $X$ equipped with the indiscrete topology is right adjoint to $U$.
\end{exe}
\begin{proof}
	For the first claim, we are saying that, given a set $X$ and a topological space $T$, continuous maps $FX\to T$ have the same data as $X\to UT$. However, all maps (of sets) $FX\to T$ are continuous for free because $FX$ has the discrete topology.

	On the other hand, for the second claim, we are saying that, given a set $X$ and a topological space $T$, maps (of spaces) $UT\to X$ have the same data as continuous maps $T\to GX$. This is true because all maps $T\to GX$ are automatically continuous because we are asking for the preimages of $\emp$ and $GX$ to be open, which are true for free.
\end{proof}
\begin{exe}
	We give the embedding functor between the poset categories $\iota:(\ZZ,\le)\to(\RR,\le)$ a right adjoint.
\end{exe}
\begin{proof}
	We claim that the right adjoint is $r\mapsto\floor r$. This is well-defined and does give a functor. To get a right adjoint, we are saying that $\iota n\le r$ if and only if $n\le\floor r$, which is true by some analysis. Notably, we only care about the existence of morphisms to get our bijections because we are working in poset categories.
\end{proof}
\begin{remark}
	Of course, $\iota$ also has a left adjoint as $\ceil{\cdot}$.
\end{remark}
\begin{exe}
	We give the embedding functor $\iota:\mathrm{Groupoids}\to\mathrm{Cat}$ a left and right adjoint.
\end{exe}
\begin{proof}
	Recall that a groupoid is a category where all morphisms are isomorphisms. There are two ways to do this.
	\begin{itemize}
		\item We can construct the ``maximal subgroupoid'' $\max\mathcal C$ by merely taking $\mathcal C$ and throwing out all morphisms which are not isomorphisms. This turns out to be a right adjoint.
		\item We can force all morphisms to be isomorphisms by adding in all the necessary inverses and quotient out by what we need to remain a category; this process is called ``localization,'' in analogy with localizing a ring. This turns out to be a left adjoint.
	\end{itemize}
	We won't actually check that these are adjoints (because the checks are painful), so we will declare that we are done.
\end{proof}

\subsection{Units and Counits}
We are now done talking about examples. As such, have a lemma.
\begin{lemma} \label{lem:getunits}
	Fix adjunctions $F:\mc C\rightleftharpoons\mc D:G$. Then there exists a natural transformation $\eta:\id_{\mc C}\Rightarrow GF$, where $\eta_c:c\to GFc$ is defined as the transpose of $\id_{Fc}:Fc\to Fc$ along the isomorphism
	\[\op{Mor}_{\mc D}(Fc,Fc)\cong\op{Mor}_{\mc C}(c,GFc)\]
	promised by the adjunction.
\end{lemma}
\begin{proof}
	We only have to check naturality. As such, fix some morphism $f:c\to c'$, and we need the following diagram to commute.
	% https://q.uiver.app/?q=WzAsNCxbMCwwLCJjIl0sWzAsMSwiYyciXSxbMSwwLCJHRmMiXSxbMSwxLCJHRmMnIl0sWzAsMSwiZiIsMl0sWzIsMywiR0ZmIl0sWzAsMiwiXFxldGFfYyJdLFsxLDMsIlxcZXRhX3tjJ30iLDJdXQ==&macro_url=https%3A%2F%2Fraw.githubusercontent.com%2FdFoiler%2Fnotes%2Fmaster%2Fnir.tex
	\[\begin{tikzcd}
		c & GFc \\
		{c'} & {GFc'}
		\arrow["f"', from=1-1, to=2-1]
		\arrow["GFf", from=1-2, to=2-2]
		\arrow["{\eta_c}", from=1-1, to=1-2]
		\arrow["{\eta_{c'}}"', from=2-1, to=2-2]
	\end{tikzcd}\]
	Note that $\id_{c}$ is the $f^\flat$ of $\id_{Fc}$, so by \autoref{lem:betteradjoint}, it suffices to check that the following diagram commutes.
	% https://q.uiver.app/?q=WzAsNCxbMCwwLCJGYyJdLFswLDEsIkZjJyJdLFsxLDAsIkZjIl0sWzEsMSwiRmMnIl0sWzAsMiwiXFxpZF97RmN9Il0sWzEsMywiXFxpZF97RmMnfSJdLFswLDEsIkZmIiwyXSxbMiwzLCJGZiJdXQ==&macro_url=https%3A%2F%2Fraw.githubusercontent.com%2FdFoiler%2Fnotes%2Fmaster%2Fnir.tex
	\[\begin{tikzcd}
		Fc & Fc \\
		{Fc'} & {Fc'}
		\arrow["{\id_{Fc}}", from=1-1, to=1-2]
		\arrow["{\id_{Fc'}}", from=2-1, to=2-2]
		\arrow["Ff"', from=1-1, to=2-1]
		\arrow["Ff", from=1-2, to=2-2]
	\end{tikzcd}\]
	This commutes because look at it.
\end{proof}
The $\eta$ in the lemma is special.
\begin{definition}[Unit]
	Work in the context of \autoref{lem:getunits}. Then $\eta$ is called the \textit{unit}.
\end{definition}
Our lemma also has the following dual.
\begin{lemma} \label{lem:getcounits}
	Fix adjunctions $F:\mc C\rightleftharpoons\mc D:G$. Then there exists a natural transformation $\varepsilon:FG\Rightarrow\id_{\mc D}$, where $\varepsilon_d:FGd\to d$ is defined as the transpose of $\id_{Gd}:Gd\to Gd$ along the isomorphism
	\[\op{Mor}_{\mc D}(FGd,d)\cong\op{Mor}_{\mc C}(Gd,Gd)\]
	promised by the adjunction.
\end{lemma}
\begin{proof}
	Duality.
\end{proof}
And here is our corresponding word.
\begin{definition}[Unit]
	Work in the context of \autoref{lem:getcounits}. Then $\varepsilon$ is called the \textit{counit}.
\end{definition}
We close class with an example.
\begin{exe}
	We compute units and counits for the free-forgetful adjoints between $\mathrm{Set}$ and $\mathrm{Grp}$.
\end{exe}
\begin{proof}
	Our unit $\eta_X:X\to UFX$ sends an element $x\in X$ to $x\in UFX$, the length-one word. On the other hand, counit $\varepsilon_G:FUG\to G$ sends an element $g_1\cdots g_n\in FUG$ made of letters of $G$ to its evaluation in $G$.
\end{proof}