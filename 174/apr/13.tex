\documentclass[../notes.tex]{subfiles}

\begin{document}

% !TEX root = ../notes.tex

We continue talking about adjoints.

\subsection{More Examples}
We recall the following definition.
\adjdefi*
\noindent From here we were able to define the unit and counit. Another way to view our construction last time is to apply the Yoneda lemma to the natural isomorphism
\[\op{Mor}_\mc D(Fc,-)\cong\op{Mor}_{\mc C}(c,G-)\]
of functors $\mc D\to\mathrm{Set}$. In particular, \autoref{thm:yoneda} grants us an object $\eta_c:\op{Mor}_\mc C(c,GFc)$ representing this isomorphism, which assembles into our unit $\eta:\id_\mc C\Rightarrow GF$.

Let's see some more examples.
\begin{exe}
	We discuss the product--hom adjunction in $\mathrm{Set}$.
\end{exe}
\begin{proof}
	The point is that there is a natural bijection
	\[\op{Mor}(X\times Y,Z)\cong\op{Mor}(X,\op{Mor}(Y,Z))\label{eq:curry}\tag{$*$}\]
	by taking the function $f:X\times Y\to Z$ to the function $\widetilde f$ defined by $x\mapsto(y\mapsto f(x,y))$. To see that this is a bijection, there is an inverse taking $g\in\op{Mor}(X,\op{Mor}(Y,Z))$ by $\hat g(x,y):=g(x)(y)$, and we can check that these are inverses by hand.
	
	This will assemble into an adjunction of the functors $F:=-\times Y$ and $G:=\op{Mor}(Y,-)$. Thus, \autoref{eq:curry} turns into a natural isomorphism
	\[\op{Mor}(FX,Z)\cong\op{Mor}(X,GY),\]
	which is what we need for $F\adjoint G$.

	We close our discussion by tracking through our unit and counit.
	\begin{itemize}
		\item For the unit, we need to transpose $\id_{Fc}$ through $\op{Mor}_\mc D(Fc,Fc)\cong\op{Mor}(c,GFc)$. In particular, we are tracking through $\id_{X\times Y}$ through
		\[\op{Mor}(X\times Y,X\times Y)\cong\op{Mor}(X,\op{Mor}(Y,X\times Y)).\]
		Thus, $\eta_X(x)(y)$ should be $\eta_X(x)(y)=\id_{X\times Y}(x,y):=(x,y)$ after moving everything through.
		\item For the counit, we need to transpose $\id_{Gd}$ through $\op{Mor}_\mc D(FGd,d)\cong\op{Mor}(Gd,Gd)$. In particular, we are tracking through $\id_{\op{Mod}(Y,Z)}$ through
		\[\op{Mor}(\op{Mod}(Y,Z)\times Y,Z)\cong\op{Mor}(\op{Mod}(Y,Z),\op{Mod}(Y,Z)).\]
		Thus, we can find $\varepsilon_Z(f,y):=f(y)$, which finishes.
	\end{itemize}
	This finishes our discussion.
\end{proof}
\begin{exe}
	We discuss the hom--tensor adjunction for $k$-vector spaces.
\end{exe}
\begin{proof}
	It is false that
	\[\op{Hom}(V\times W,U)\cong\op{Hom}(V,\op{Hom}(W,U)).\]
	This does not make sense because we don't really want to talk about linear maps $V\times W\to U$ but rather being bilinear in both arguments, so we want an isomorphism
	\[\op{Bilin}(V\times W,U)\cong\op{Hom}(V,\op{Hom}(W,U))\]
	instead. Thus, using the universal property for tensor products, we would have
	\[\op{Hom}(V\otimes W,U)\cong\op{Hom}(V,\op{Hom}(W,U)).\]
	We can compute that the unit is still $\varepsilon_U(f\otimes w):=f(w)$. Chris does not remember the counit precisely.
\end{proof}

\subsection{The Triangle Equations}
We should probably prove something today, so let's prove something.
\begin{proposition} \label{prop:triangleequations}
	Fix adjoint functors $F\adjoint G$ between categories $\mc C$ and $\mc D$ with unit $\eta$ and counit $\varepsilon$. Then we have the triangle equations.
	% https://q.uiver.app/?q=WzAsNixbMCwwLCJGIl0sWzEsMCwiRkdGIl0sWzEsMSwiRiJdLFsyLDAsIkciXSxbMywwLCJHRkciXSxbMywxLCJHIl0sWzAsMSwiRlxcZXRhIiwwLHsibGV2ZWwiOjJ9XSxbMSwyLCJcXHZhcmVwc2lsb24gRiIsMCx7ImxldmVsIjoyfV0sWzAsMiwiXFxpZF9GIiwyLHsibGV2ZWwiOjJ9XSxbMyw0LCJcXGV0YSBHIiwwLHsibGV2ZWwiOjJ9XSxbNCw1LCJHXFx2YXJlcHNpbG9uIiwwLHsibGV2ZWwiOjJ9XSxbMyw1LCJcXGlkX0ciLDIseyJsZXZlbCI6Mn1dXQ==
	\[\begin{tikzcd}
		F & FGF & G & GFG \\
		& F && G
		\arrow["F\eta", Rightarrow, from=1-1, to=1-2]
		\arrow["{\varepsilon F}", Rightarrow, from=1-2, to=2-2]
		\arrow["{\id_F}"', Rightarrow, from=1-1, to=2-2]
		\arrow["{\eta G}", Rightarrow, from=1-3, to=1-4]
		\arrow["G\varepsilon", Rightarrow, from=1-4, to=2-4]
		\arrow["{\id_G}"', Rightarrow, from=1-3, to=2-4]
	\end{tikzcd}\]
\end{proposition}
\begin{proof}
	Expanding out on objects, our first triangle takes some $c\in\mc C$ and writes down the following.
	% https://q.uiver.app/?q=WzAsMyxbMCwwLCJGYyJdLFsxLDAsIkZHRmMiXSxbMSwxLCJGYyJdLFswLDEsIkYoXFxldGFfYykiXSxbMSwyLCJcXHZhcmVwc2lsb25fe0ZjfSJdLFswLDIsIlxcaWRfe0ZjfSIsMl1d
	\[\begin{tikzcd}
		Fc & FGFc \\
		& Fc
		\arrow["{F(\eta_c)}", from=1-1, to=1-2]
		\arrow["{\varepsilon_{Fc}}", from=1-2, to=2-2]
		\arrow["{\id_{Fc}}"', from=1-1, to=2-2]
	\end{tikzcd}\label{eq:triangle1}\tag{1}\]
	Namely, the top arrow is a whiskering. Similarly, the other triangle looks like the following on some $d\in\mc D$.
	% https://q.uiver.app/?q=WzAsMyxbMCwwLCJHZCJdLFsxLDAsIkdGR2QiXSxbMSwxLCJHZCJdLFswLDIsIlxcaWRfe0dkfSIsMl0sWzAsMSwiXFxldGFfe0dkfSJdLFsxLDIsIkcoXFx2YXJlcHNpbG9uX2QpIl1d
	\[\begin{tikzcd}
		Gd & GFGd \\
		& Gd
		\arrow["{\id_{Gd}}"', from=1-1, to=2-2]
		\arrow["{\eta_{Gd}}", from=1-1, to=1-2]
		\arrow["{G(\varepsilon_d)}", from=1-2, to=2-2]
	\end{tikzcd}\label{eq:triangle2}\tag{2}\]
	To prove the result, we recall the following lemma.
	\lembetteradjoint*
	\noindent As such, we start with the following diagram. Note that the following diagram commutes because look at it.
	% https://q.uiver.app/?q=WzAsNCxbMCwwLCJjIl0sWzEsMCwiR0ZjIl0sWzAsMSwiR0ZjIl0sWzEsMSwiR0ZjIl0sWzAsMiwiXFxldGFfYyIsMl0sWzAsMSwiXFxldGFfYyJdLFsxLDMsIlxcaWRfe0dGY30iXSxbMiwzLCJcXGlkX3tHRmN9IiwyXV0=
	\[\begin{tikzcd}[ampersand replacement=\&]
		c \& GFc \\
		GFc \& GFc
		\arrow["{\eta_c}"', from=1-1, to=2-1]
		\arrow["{\eta_c}", from=1-1, to=1-2]
		\arrow["{\id_{GFc}}", from=1-2, to=2-2]
		\arrow["{\id_{GFc}}"', from=2-1, to=2-2]
	\end{tikzcd}\]
	Thus, \autoref{lem:betteradjoint} tells us that the following diagram commutes.
	% https://q.uiver.app/?q=WzAsNCxbMCwwLCJGYyJdLFsxLDAsIkZjIl0sWzAsMSwiRkdGYyJdLFsxLDEsIkZjIl0sWzAsMSwiXFxpZF97RmN9Il0sWzEsMywiXFxpZF97RmN9Il0sWzIsMywiXFx2YXJlcHNpbG9uX3tGY30iLDJdLFswLDIsIkZcXGV0YV9jIiwyXV0=
	\[\begin{tikzcd}[ampersand replacement=\&]
		Fc \& Fc \\
		FGFc \& Fc
		\arrow["{\id_{Fc}}", from=1-1, to=1-2]
		\arrow["{\id_{Fc}}", from=1-2, to=2-2]
		\arrow["{\varepsilon_{Fc}}"', from=2-1, to=2-2]
		\arrow["{F\eta_c}"', from=1-1, to=2-1]
	\end{tikzcd}\]
	This is \autoref{eq:triangle1}.

	For the other one, we start with the following square which commutes because look at it.
	% https://q.uiver.app/?q=WzAsNCxbMCwwLCJGR2QiXSxbMSwwLCJGR2QiXSxbMCwxLCJGR2QiXSxbMSwxLCJkIl0sWzAsMSwiXFxpZF97RkdkfSJdLFswLDIsIlxcaWRfe0ZHZH0iLDJdLFsyLDMsIlxcdmFyZXBzaWxvbl9kIiwyXSxbMSwzLCJcXHZhcmVwc2lsb25fZCJdXQ==
	\[\begin{tikzcd}[ampersand replacement=\&]
		FGd \& FGd \\
		FGd \& d
		\arrow["{\id_{FGd}}", from=1-1, to=1-2]
		\arrow["{\id_{FGd}}"', from=1-1, to=2-1]
		\arrow["{\varepsilon_d}"', from=2-1, to=2-2]
		\arrow["{\varepsilon_d}", from=1-2, to=2-2]
	\end{tikzcd}\]
	Again applying \autoref{lem:betteradjoint}, we get the following commutative diagram.
	% https://q.uiver.app/?q=WzAsNCxbMCwwLCJGR2QiXSxbMSwwLCJHRkdkIl0sWzAsMSwiRkdkIl0sWzEsMSwiR2QiXSxbMCwxLCJcXGV0YV97R2R9Il0sWzAsMiwiXFxpZF97R2R9IiwyXSxbMiwzLCJcXGlkX3tHZH0iLDJdLFsxLDMsIkcoXFx2YXJlcHNpbG9uX2QpIl1d
	\[\begin{tikzcd}[ampersand replacement=\&]
		FGd \& GFGd \\
		FGd \& Gd
		\arrow["{\eta_{Gd}}", from=1-1, to=1-2]
		\arrow["{\id_{Gd}}"', from=1-1, to=2-1]
		\arrow["{\id_{Gd}}"', from=2-1, to=2-2]
		\arrow["{G(\varepsilon_d)}", from=1-2, to=2-2]
	\end{tikzcd}\]
	This is \autoref{eq:triangle2}. This finishes the proof.
\end{proof}
As such, we have the following nice result for adjoints.
\begin{restatable}{theorem}{adjointbyunits}
	Fix functors $F:\mc C\rightleftharpoons\mc D:G$ with natural transformations $\eta:\id_\mc C\Rightarrow GF$ and $\varepsilon:FG\Rightarrow\id_\mc D$ satisfying the triangles in \autoref{prop:triangleequations}. Then $F\adjoint G$.
\end{restatable}
\begin{proof}
	Given some $f^\sharp$ in $\mc C$, we set $f^\flat$ equal to $Gf^\sharp\circ\eta_c$; conversely, we send $g^\flat$ in $\mc D$ to $g^\sharp$ equal to $\varepsilon_d\circ Fg^\flat$. The point of showing that these commute is by drawing the following diagram.
	% https://q.uiver.app/?q=WzAsNSxbMCwwLCJGYyJdLFsxLDAsIkZHRmMiXSxbMiwwLCJGR2QiXSxbMywwLCJkIl0sWzIsMSwiRmMiXSxbMCw0LCJcXGlkX3tGY30iLDJdLFsxLDQsIlxcdmFyZXBzaWxvbl97RmN9Il0sWzMsNCwiZl5cXHNoYXJwIl0sWzAsMSwiRlxcZXRhX2MiXSxbMSwyLCJGR2ZeXFxzaGFycCJdLFsyLDMsIlxcdmFyZXBzaWxvbl9kIl1d
	\[\begin{tikzcd}[ampersand replacement=\&]
		Fc \& FGFc \& FGd \& d \\
		\&\& Fc
		\arrow["{\id_{Fc}}"', from=1-1, to=2-3]
		\arrow["{\varepsilon_{Fc}}", from=1-2, to=2-3]
		\arrow["{f^\sharp}", from=1-4, to=2-3]
		\arrow["{F\eta_c}", from=1-1, to=1-2]
		\arrow["{FGf^\sharp}", from=1-2, to=1-3]
		\arrow["{\varepsilon_d}", from=1-3, to=1-4]
	\end{tikzcd}\]
	This diagram commutes by some effort, which will give the inverse conditions.
	
	We can show that these are mutually inverse, and then they are natural in both arguments because of course they are.
\end{proof}

\end{document}