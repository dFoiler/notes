\documentclass[../notes.tex]{subfiles}

\begin{document}

% !TEX root = ../notes.tex

It's time to get started, but it is discussion section, for some definition of discussion.

\subsection{}
We pick up the following lemma.
\begin{lemma}
	Let $F\adjoint G$ be an adjunction between categories $\mc C$ and $\mc D$, with unit $\eta$ and counit $\varepsilon$. Then $G$ is faithful/full/both if and only if each $d\in\mc D$ has $\varepsilon_d$ is an epimorphism/split monomorphism/isomorphism.
\end{lemma}
\begin{proof}
	We go one at a time.
	\begin{listroman}
		\item Fix $d,d'\in\mc D$ with parallel morphisms $f,g\colon d\to d'$. The main point is to look at
		\[f\circ\varepsilon_d,g\circ\varepsilon_d.\]
		Passing to the transpose, we go to $(f\varepsilon_d)^\sharp=G(f\varepsilon_d)\eta_{Gd}$ by our discussion of units and things. Distributing this is
		\[(f\varepsilon_d)^\sharp=Gf\]
		by applying the triangle equalities. Similarly, $(g\varepsilon_d)^\sharp=Gg$, so $f\varepsilon_d=g\varepsilon_d$ if and only if $Gf=Gg$. Thus, if $\varepsilon_d$ is an epimorphism, this means $f=g$ if and only if $Gf=Gg$. Similarly, if $G$ is faithful, then $f=g$ if and only if $Gf=Gg$ if and only if $f\varepsilon_d=g\varepsilon_d$, so $\varepsilon_d$ is an epimorphism.
		\item We proceed by force. Suppose $G$ is full. We would like $\varepsilon_d\colon FGd\to d$ to be a split monomorphism, so we need a retraction $r\colon d\to FGd$ so that $r\varepsilon)d=\id_{FGd}$.

		As such, we simply pick up $\eta_{Gd}\colon Gd\to GFGd$, but $G$ is full, so we can lift this to some morphism $r\colon d\to FGd$ so that $Gr=\eta_{Gd}$. We now show
		\[r\varepsilon_d=\id_{FGd}.\]
		Taking the transpose, we are showing
		\[(r\varepsilon_d)^\flat=G(r\varepsilon_d)\eta_{Gd}\stackrel*=Gr=\eta_{Gd}\]
		where we are as usual using the triangle inequalities in $\stackrel*=$. Transposing back finishes.

		In the other direction, suppose that each of the $\varepsilon_d$ are split monomorphisms. For each $d\in\mc D$, we are promised a retraction $r_d\colon d\to FGd$ so that $r_d\varepsilon_d=\id_{FGd}$.

		We now lift by hand: fix $d,d'\in\mc D$ with $f\colon Gd\to Gd'$, and we want to lift it by hand. Namely, define
		\[g\coloneqq f^\sharp r_d=\varepsilon_d\circ Ff\circ r_d,\]
		which we can see is a map $d\to d'$. We don't want to hit this with $G$ directly because $GFf$ is sad; roughly speaking, we want to put $\varepsilon_d$ on the other side of $Ff$. As such, we compute
		\[(Gg)^\sharp=\varepsilon_{d'}\circ FGg=\varepsilon_{d'}\circ FG\varepsilon_{d'}\circ FGFf\circ FGr_d.\]
		Now, we use the naturality of the following square.
		% https://q.uiver.app/?q=WzAsNCxbMCwwLCJGR0ZHZCJdLFsxLDAsIkZHZCciXSxbMSwxLCJkJyJdLFswLDEsIkZHZCciXSxbMCwzLCJcXHZhcmVwc2lsb24gRkdkJyIsMl0sWzAsMSwiRkdcXHZhcmVwc2lsb24gZCciXSxbMSwyLCJcXHZhcmVwc2lsb24gZCciXSxbMywyLCJcXHZhcmVwc2lsb24gZCciLDJdXQ==&macro_url=https%3A%2F%2Fraw.githubusercontent.com%2FdFoiler%2Fnotes%2Fmaster%2Fnir.tex
		\[\begin{tikzcd}
			FGFGd & {FGd'} \\
			{FGd'} & {d'}
			\arrow["{\varepsilon FGd'}"', from=1-1, to=2-1]
			\arrow["{FG\varepsilon d'}", from=1-1, to=1-2]
			\arrow["{\varepsilon d'}", from=1-2, to=2-2]
			\arrow["{\varepsilon d'}"', from=2-1, to=2-2]
		\end{tikzcd}\]
		This gives us
		\[\varepsilon_{d'}\circ\varepsilon_{FGd'}\circ FGFf\circ FGr_d.\]
		Next, we use naturality of another square, as follows.
		% https://q.uiver.app/?q=WzAsNCxbMCwwLCJGR0ZHZCJdLFsxLDAsIkZHRkdkJyJdLFswLDEsIkZHZCJdLFsxLDEsIkZHZCciXSxbMCwyLCJcXHZhcmVwc2lsb24gRkdkIiwyXSxbMCwxLCJGR0ZmIl0sWzIsMywiRmYiLDJdLFsxLDMsIlxcdmFyZXBzaWxvbiBGR2QiXV0=&macro_url=https%3A%2F%2Fraw.githubusercontent.com%2FdFoiler%2Fnotes%2Fmaster%2Fnir.tex
		\[\begin{tikzcd}
			FGFGd & {FGFGd'} \\
			FGd & {FGd'}
			\arrow["{\varepsilon FGd}"', from=1-1, to=2-1]
			\arrow["FGFf", from=1-1, to=1-2]
			\arrow["Ff"', from=2-1, to=2-2]
			\arrow["{\varepsilon FGd}", from=1-2, to=2-2]
		\end{tikzcd}\]
		This gives
		\[\varepsilon_d\circ Ff\circ \varepsilon_{FGd}\circ FGr_d.\]
		Moving things over, we want to put the retraction on the other side, so we draw the following naturality square.
		% https://q.uiver.app/?q=WzAsNCxbMCwwLCJGR2QiXSxbMSwwLCJGR0ZHZCJdLFswLDEsImQiXSxbMSwxLCJGR2QiXSxbMiwzLCJyX2QiLDJdLFswLDEsIkZHcl9kIl0sWzAsMiwiXFx2YXJlcHNpbG9uX2QiLDJdLFsxLDMsIlxcdmFyZXBzaWxvbiBGR2QiXV0=&macro_url=https%3A%2F%2Fraw.githubusercontent.com%2FdFoiler%2Fnotes%2Fmaster%2Fnir.tex
		\[\begin{tikzcd}
			FGd & FGFGd \\
			d & FGd
			\arrow["{r_d}"', from=2-1, to=2-2]
			\arrow["{FGr_d}", from=1-1, to=1-2]
			\arrow["{\varepsilon_d}"', from=1-1, to=2-1]
			\arrow["{\varepsilon FGd}", from=1-2, to=2-2]
		\end{tikzcd}\]
		In total, we are left with
		\[\varepsilon_{d'}\circ Ff\circ r_d\circ\varepsilon_d\]
		which retracts properly to $f^\sharp$
		\item This follows from adding together (i) and (ii) because isomorphisms are the same as being epic and split monic.
		\qedhere
	\end{listroman}
\end{proof}
\begin{remark}
	There is also the following dual statement for $F$. Namely, $F$ is faithful/full/both if and only if each $d\in\mc D$ has $\varepsilon_d$ is a monomorphism/split epimorphism/isomorphism. In particular, we can pass to the opposite category to change our adjoints.
\end{remark}

\subsection{The Category of Categories}
We would like to understand the category of (small) categories. We pick up the following definition.
\begin{definition}
	A \textit{reflective} subcategory $\mc C$ is a full subcategory $\mc D$ of $\mc C$ such that there is a left adjoint $L\colon\mc C\to\mc D$ of the embedding $\mc C\into\mc D$. This $L$ is called the \textit{reflector} or the \textit{localization}.
\end{definition}
The point is that we have a reflector $L$ which gives us a fairly natural way to fix $\mc D$: the embedding $\mc C\into\mc D$ is fully faithful, so each of the counit morphisms $\varepsilon_d\colon L_id\to d$ are all isomorphisms. In other words, for each $d\in\mc D$, we essentially have
\[Ld\cong\id_\mc D\]
by viewing $d\in\mc C$ via the embedding.
\begin{example}
	The embedding $\iota\colon\mathrm{Ab}\to\mathrm{Grp}$ gives a full subcategory. This is reflective, with its reflector $L$ taking a group $G$ to its abelianization $G/[G,G]$. We won't check that this is a left adjoint, but it follows because mapping into an abelian group is the same thing as mapping from the abelianization.
\end{example}
\begin{example}
	The embedding $\mathrm{Sh}(X)\into\mathrm{PSh}(X)$ gives a full subcategory, and its reflector is sheafification.
\end{example}
\begin{exe}
	There is an embedding of $N\colon\mathrm{Cat}\into\mathrm{sSet}$. Here, $\mathrm{sSet}$ is the set of presheaves on $\Delta$, where $\Delta$ is the ``simplex'' category with the following data.
	\begin{itemize}
		\item The objects of $\Delta$ are sets $[n]:=\{0,1,2,\ldots,n\}$.
		\item The morphisms of $\Delta$ are non-decreasing maps $[n]\to[m]$.
	\end{itemize}
	Approximately speaking, the objects of $\Delta$ look like $n$-splices. This embedding makes a reflective subcategory.
\end{exe}
\begin{proof}
	We first describe the embedding. Fix a category $\mc C$. For notation, given an object $X_\bullet\colon\Delta\opp\to\mathrm{Set}$ and let $X_n:=X([n])$. The way that we are going to embed is by
	\[N(\mc C)_n:=\op{Fun}([n],\mc C).\]
	The point is that we have a left adjoint $h\colon\mathrm{sSet}\to\mathrm{Cat}$ by just restricting down to the $0$th component for objects and the $1$st component for morphisms.
\end{proof}
As such, we have the following result.
\begin{proposition}
	Suppose that we have an inclusion $\iota\colon\mc D\to\mc C$ making $\mc D$ a reflective subcategory.
	\begin{listroman}
		\item The inclusion $\iota$ creates limits (that $\mc C$ admits).
		\item $\mc D$ has all colimits that $\mc C$ admits by applying the reflector.
	\end{listroman}
\end{proposition}
The point is that understanding $\mc C$ will give us understanding of $\mc D$.
\begin{proof}
	We proceed as follows.
	\begin{listroman}
		\item This follows by muttering something about monads.
		\item We will actually show this. Let $F\colon\mc J\to\mc D$ be a diagram such that $\mc C$ has colimits of shape $\mc J$. Now, let $\lambda\colon \iota F\Rightarrow c$ be a colimit cone in $\mc C$, and we note that $L\lambda\colon L\iota F\Rightarrow Lc$ is a colimit cone in $\mc D$ because left adjoints preserve colimits. However, $L\iota F\cong F$, so we have found our colimit.
		\qedhere
	\end{listroman}
\end{proof}
\begin{corollary}
	The category $\mathrm{Cat}$ is complete and cocomplete.
\end{corollary}
\begin{proof}
	By synthesizing the above discussion, we see that it suffices to show that
	\[\mathrm{sSet}=\mathrm{Psh}(\Delta)=\op{Fun}(\Delta\opp,\mathrm{Set})\]
	is complete and cocomplete. However, $\mathrm{Set}$ is complete and cocomplete, and we know how to compute limits and colimits in functor categories (namely, pointwise) pulling from their codomain. This finishes.
\end{proof}

\end{document}