\documentclass[../notes.tex]{subfiles}

\begin{document}

% !TEX root = ../notes.tex

Today we began with review, which was not recorded.

\subsection{Group Objects}
The main idea is as follows.
\begin{idea}
	We want to talk about ``groups in a category $\mathcal C$.''
\end{idea}
One way to do this is to simply require a functor $\mathrm BG\to\mathcal C$, but we would like to have an object in $\mathcal C$ into a group, on its own.

As such, here is our definition. Fix a category $\mathcal C$ with finite products (and hence a terminal object $T$). Then a \textit{group object $G$} in $\mathcal C$ has the following data.
\begin{itemize}
	\item $G\in\op{ob}\mathcal C$.
	\item We have a morphism $\nabla:G\times G\to\mathcal C$ for our operation.
	\item We have an identity element $\eta:T\to G$ (via the morphisms-as-elements philosophy).
	\item Lastly, there is an inverse map $s:G\to G$.
\end{itemize}
We also require the following diagrams to commute.
\begin{itemize}
	\item Associativity.
	% https://q.uiver.app/?q=WzAsNCxbMCwwLCJHXFx0aW1lcyBHXFx0aW1lcyBHIl0sWzEsMCwiR1xcdGltZXMgRyJdLFswLDEsIkdcXHRpbWVzIEciXSxbMSwxLCJHIl0sWzAsMSwiXFxuYWJsYVxcdGltZXNcXGlkX0ciXSxbMCwyLCJcXGlkX0dcXHRpbWVzXFxuYWJsYSIsMl0sWzEsMywiXFxuYWJsYSJdLFsyLDMsIlxcbmFibGEiLDJdXQ==&macro_url=https%3A%2F%2Fraw.githubusercontent.com%2FdFoiler%2Fnotes%2Fmaster%2Fnir.tex
	\[\begin{tikzcd}
		{G\times G\times G} & {G\times G} \\
		{G\times G} & G
		\arrow["{\nabla\times\id_G}", from=1-1, to=1-2]
		\arrow["{\id_G\times\nabla}"', from=1-1, to=2-1]
		\arrow["\nabla", from=1-2, to=2-2]
		\arrow["\nabla"', from=2-1, to=2-2]
	\end{tikzcd}\tag{Ass}\label{eq:ass}\]
	This is intended to say that $(ab)c=a(bc)$ for $a,b,c\in G$.
	\item Left identity.
	% https://q.uiver.app/?q=WzAsMyxbMCwwLCJUXFx0aW1lcyBHIl0sWzEsMCwiR1xcdGltZXMgRyJdLFsxLDEsIkciXSxbMCwxLCJcXGV0YVxcdGltZXNcXGlkX0ciXSxbMSwyLCJcXG5hYmxhIl0sWzAsMiwiXFxwaV9HIiwyXV0=
	\[\begin{tikzcd}
		{T\times G} & {G\times G} \\
		& G
		\arrow["{\eta\times\id_G}", from=1-1, to=1-2]
		\arrow["\nabla", from=1-2, to=2-2]
		\arrow["{\pi_G}"', from=1-1, to=2-2]
	\end{tikzcd}\tag{LId}\label{eq:lid}\]
	This is intended to say that $1\cdot g=g$ for any $g\in G$.
	\item Right identity.
	% https://q.uiver.app/?q=WzAsMyxbMCwwLCJHXFx0aW1lcyBHIl0sWzAsMSwiRyJdLFsxLDAsIkdcXHRpbWVzIFQiXSxbMCwxLCJcXG5hYmxhIiwyXSxbMiwwLCJcXGlkX0dcXHRpbWVzXFxldGEiLDJdLFsyLDEsIlxccGlfRyJdXQ==
	\[\begin{tikzcd}
		{G\times G} & {G\times T} \\
		G
		\arrow["\nabla"', from=1-1, to=2-1]
		\arrow["{\id_G\times\eta}"', from=1-2, to=1-1]
		\arrow["{\pi_G}", from=1-2, to=2-1]
	\end{tikzcd}\tag{RId}\label{eq:rid}\]
	This is intended to say that $g\cdot1=g$ for any $g\in G$.
\end{itemize}
The data so far assemble into a monoid object. To create a group, we introduce the canonical map $\varepsilon:G\to T$ (the ``counit'' map) and the diagonal map $\Delta:G\to G\times G$ by $\id_G\times\id_G$. So here is our last diagram.
\begin{itemize}
	\item Inverse.
	% https://q.uiver.app/?q=WzAsNyxbMCwxLCJHIl0sWzAsMCwiR1xcdGltZXMgRyJdLFsxLDEsIlQiXSxbMiwwLCJHXFx0aW1lcyBHIl0sWzIsMSwiRyJdLFswLDIsIkdcXHRpbWVzIEciXSxbMiwyLCJHXFx0aW1lcyBHIl0sWzIsNCwiXFxldGEiLDJdLFswLDIsIlxcdmFyZXBzaWxvbiIsMl0sWzMsNCwiXFxuYWJsYSJdLFswLDEsIlxcRGVsdGEiXSxbMSwzLCJzXFx0aW1lc1xcaWRfRyJdLFs2LDQsIlxcbmFibGEiLDJdLFswLDUsIlxcRGVsdGEiLDJdLFs1LDYsIlxcaWRfR1xcdGltZXMgcyIsMl1d
	\[\begin{tikzcd}
		{G\times G} && {G\times G} \\
		G & T & G \\
		{G\times G} && {G\times G}
		\arrow["\eta"', from=2-2, to=2-3]
		\arrow["\varepsilon"', from=2-1, to=2-2]
		\arrow["\nabla", from=1-3, to=2-3]
		\arrow["\Delta", from=2-1, to=1-1]
		\arrow["{s\times\id_G}", from=1-1, to=1-3]
		\arrow["\nabla"', from=3-3, to=2-3]
		\arrow["\Delta"', from=2-1, to=3-1]
		\arrow["{\id_G\times s}"', from=3-1, to=3-3]
	\end{tikzcd}\tag{Inv}\label{eq:inv}\]
	The top diagram takes $g\in G$ and asserts that $g^{-1}\cdot g$ is equal to the identity element as required by $T$. The bottom diagram is doing the same for $g\cdot g^{-1}$.
\end{itemize}
So we now state our definition.
\begin{definition}[Group object]
	Fix a category $\mathcal C$ with finite products (and hence a terminal object $T$). Then a \textit{group object $G$} in $\mathcal C$ has the following data.
	\begin{itemize}
		\item $G\in\op{ob}\mathcal C$.
		\item We have a morphism $\nabla:G\times G\to\mathcal C$ for our operation.
		\item We have an identity element $\eta:T\to G$ (via the morphisms-as-elements philosophy).
		\item Lastly, there is an inverse map $s:G\to G$.
	\end{itemize}
	We also require \autoref{eq:ass}, \autoref{eq:lid}, \autoref{eq:rid}, and \autoref{eq:inv} to all commute.
\end{definition}
\begin{example}
	A group object in $\mathrm{Set}$ is a group.
\end{example}
\begin{example}
	A group object in $\mathrm{Grp}$ is an abelian group. This is by the Eckmann--Hamilton argument because we have made the group $G$ a monoid in two different ways, and we get told that these monoid structures must coincide. However, we have also required that the inverse map $s:G\to G$ to be a group homomorphism. The fact that this inverse map is a group homomorphism requires $G$ to be abelian.
\end{example}
We quickly outline the Eckmann--Hamilton argument. Because $\eta:\{e\}\to G$ is a group homomorphism, we do indeed realize $\eta$ as the actual identity for $G$. It remains to show that the $\nabla$ map is correct, so suppose we have a second morphism $\nabla':G\times G\to G$.

Well, we note that a group homomorphism $f:G\to H$ consists of the data of the following data commuting.
% https://q.uiver.app/?q=WzAsNCxbMCwwLCJHXFx0aW1lcyBHIl0sWzEsMSwiSCJdLFsxLDAsIkciXSxbMCwxLCJIXFx0aW1lcyBIIl0sWzAsMywiZlxcdGltZXMgZiIsMl0sWzIsMSwiZiJdLFswLDIsIlxcbmFibGFfRyJdLFszLDEsIlxcbmFibGFfSCIsMl1d
\[\begin{tikzcd}
	{G\times G} & G \\
	{H\times H} & H
	\arrow["{f\times f}"', from=1-1, to=2-1]
	\arrow["f", from=1-2, to=2-2]
	\arrow["{\nabla_G}", from=1-1, to=1-2]
	\arrow["{\nabla_H}"', from=2-1, to=2-2]
\end{tikzcd}\]
In particular, $\id_G:G\to G$ will provide the following data.
% https://q.uiver.app/?q=WzAsNCxbMCwwLCIoR1xcdGltZXMgRylcXHRpbWVzKEdcXHRpbWVzIEcpIl0sWzEsMCwiR1xcdGltZXMgRyJdLFsxLDEsIkciXSxbMCwxLCJHXFx0aW1lcyBHIl0sWzAsMywiXFxuYWJsYVxcdGltZXNcXG5hYmxhIiwyXSxbMCwxLCJcXG5hYmxhJ1xcdGltZXNcXG5hYmxhJyJdLFsxLDIsIlxcbmFibGEiXSxbMywyLCJcXG5hYmxhJyIsMl1d
\[\begin{tikzcd}
	{(G\times G)\times(G\times G)} & {G\times G} \\
	{G\times G} & G
	\arrow["\nabla\times\nabla"', from=1-1, to=2-1]
	\arrow["{\nabla'\times\nabla'}", from=1-1, to=1-2]
	\arrow["\nabla", from=1-2, to=2-2]
	\arrow["{\nabla'}"', from=2-1, to=2-2]
\end{tikzcd}\]
Tracking through $((x,1),(1,y))\in G\times G$ shows that $\nabla'(x,y)=xy$, which is what we wanted.

\end{document}