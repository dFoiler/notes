% !TEX root = ../notes.tex

Welcome to the last day of class.

\subsection{Ultrafilters}
Last time we were able to write down Kan extensions for the diagram
% https://q.uiver.app/?q=WzAsMyxbMCwwLCJcXG1jIEMiXSxbMCwxLCJcXG1jIEQiXSxbMSwwLCJcXG1jIEUiXSxbMCwyLCJGIl0sWzAsMSwiSyIsMl1d&macro_url=https%3A%2F%2Fraw.githubusercontent.com%2FdFoiler%2Fnotes%2Fmaster%2Fnir.tex
\[\begin{tikzcd}
	{\mc C} & {\mc E} \\
	{\mc D}
	\arrow["F", from=1-1, to=1-2]
	\arrow["K"', from=1-1, to=2-1]
\end{tikzcd}\]
by writing down
\[(\op{Lan}_KF)(d)\coloneqq\colim\left(\mc C\vec\times_\mc Dd\to\mc C\stackrel F\to\mc E\right).\]
Let's see another example: we're going to talk about ultrafilters.
\begin{definition}[Ultrafilter]
	Fix a set $S$. An \textit{ultrafilter} on $S$ is a collection $\mc U\subseteq\mc P(S)$ with the following coherence conditions.
	\begin{listalph}
		\item $\emp\notin\mc U$.
		\item Upwards closed: if $A,B\in\mc P(S)$, then $A\in\mc U$ and $A\subseteq B$ implies $B\in\mc U$.
		\item Intersection: $A,B\in\mc P(S)$ implies $A\cap B\in\mc P(S)$.
		\item For each $A\in\mc P(S)$, exactly one of $A\in\mc U$ or $(S\setminus A)\in\mc U$ is true.
	\end{listalph}
	We let $\beta(S)$ denote the set of all ultrafilters on $S$.
\end{definition}
\begin{remark}
	Note that, if we threw out the first condition, then $\emp\in\mc U$ would imply $\emp\subseteq S\subseteq S$, thus giving $S\in\mc U$, violating the last condition. So we can actually derive $\emp\notin\mc U$ from the other three.
\end{remark}
Ultrafilters are important for point-set topology, for various reasons; it turns out that they are the correct indexing set.
\begin{example}
	Given $s\in S$, then we can build
	\[\delta_s\coloneqq\{A\subseteq S:s\in A\},\]
	which is an ultrafilter.
\end{example}
It happens that the map $s\mapsto\delta_s$ is a bijection if $S\to\beta(S)$ when $S$ is finite. However, when $S$ is infinite, there might be other ultrafilters, but the axiom of choice is needed in their construction.
\begin{remark}
	Here are some use cases for ultrafilters.
	\begin{itemize}
		\item Logic, as in \L{}o\'s's theorem.
		\item Topology, as for limits.
		\item And more: geometric group theory, dynamics, and so on.
	\end{itemize}
\end{remark}
There is also a functoriality of ultrafilters.
\begin{proposition}
	Fix a function $f\colon S\to T$, there is a map $f_*\colon\beta(S)\to\beta(T)$ by
	\[\mc U\mapsto\left\{\beta\subseteq T:f^{-1}(B)\in\mc U\right\}\]
\end{proposition}
\begin{proof}
	Note that $f^{-1}$ is closed under intersection and unions, so we can just check the axioms one by one.
\end{proof}
So we have a functoriality and can now do category theory.
\begin{theorem}
	The diagram
	% https://q.uiver.app/?q=WzAsMyxbMCwwLCJcXG1hdGhybXtmaW5TZXR9Il0sWzAsMSwiXFxtYXRocm17U2V0fSJdLFsxLDAsIlxcbWF0aHJte1NldH0iXSxbMCwyLCIiLDAseyJzdHlsZSI6eyJ0YWlsIjp7Im5hbWUiOiJob29rIiwic2lkZSI6InRvcCJ9fX1dLFswLDEsIiIsMix7InN0eWxlIjp7InRhaWwiOnsibmFtZSI6Imhvb2siLCJzaWRlIjoidG9wIn19fV0sWzEsMiwiXFxiZXRhIiwyLHsic3R5bGUiOnsiYm9keSI6eyJuYW1lIjoiZGFzaGVkIn19fV1d&macro_url=https%3A%2F%2Fraw.githubusercontent.com%2FdFoiler%2Fnotes%2Fmaster%2Fnir.tex
	\[\begin{tikzcd}
		{\mathrm{finSet}} & {\mathrm{Set}} \\
		{\mathrm{Set}}
		\arrow[hook, from=1-1, to=1-2]
		\arrow[hook, from=1-1, to=2-1]
		\arrow["\beta"', dashed, from=2-1, to=1-2]
	\end{tikzcd}\]
	makes $\beta$ a right Kan extension.
\end{theorem}
\begin{proof}
	The point is to check
	\[\beta(S)=\lim_{S\to F}F,\]
	where we have interpreted $S\to F$ in the correct way.
\end{proof}
Now, having a right Kan extension tells us lots of things by abstract nonsense. For example, by using the functor $\id\colon\mathrm{Set}\to\mathrm{Set}$, the universal property gives us a natural transformation
\[\eta\colon\id\Rightarrow\beta.\]
In the same way, there is also a natural transformation
\[\mu\colon\beta^2\Rightarrow\beta.\]
Having these two notions lets us write down strange commutative diagrams, like the following.
% https://q.uiver.app/?q=WzAsNCxbMCwwLCJcXGJldGFeMyJdLFsxLDAsIlxcYmV0YV4yIl0sWzEsMSwiXFxiZXRhIl0sWzAsMSwiXFxiZXRhXjIiXSxbMCwxLCJcXG11XFxiZXRhIiwwLHsibGV2ZWwiOjJ9XSxbMSwyLCJcXG11IiwwLHsibGV2ZWwiOjJ9XSxbMywyLCJcXG11IiwyLHsibGV2ZWwiOjJ9XSxbMCwzLCJcXGJldGFcXG11IiwyLHsibGV2ZWwiOjJ9XV0=&macro_url=https%3A%2F%2Fraw.githubusercontent.com%2FdFoiler%2Fnotes%2Fmaster%2Fnir.tex
\[\begin{tikzcd}
	{\beta^3} & {\beta^2} \\
	{\beta^2} & \beta
	\arrow["\mu\beta", Rightarrow, from=1-1, to=1-2]
	\arrow["\mu", Rightarrow, from=1-2, to=2-2]
	\arrow["\mu"', Rightarrow, from=2-1, to=2-2]
	\arrow["\beta\mu"', Rightarrow, from=1-1, to=2-1]
\end{tikzcd}\]
Observe that this would be essentially impossible to verify by hand (or at least very annoying), but it follows directly from uniqueness of the natural transformation $\beta^3\Rightarrow\beta$ by universal property.
\begin{remark}
	It turns out that $\eta$ and $\mu$ will specify the data of a monad.
\end{remark}

\subsection{Category Theory via Kan Extensions}
Today, we will explain why all concepts are Kan extensions.\footnote{Alternatively, ``all Kancepts are Kan extensions.''} Namely, we are going to reinterpret many concepts in category theory as Kan extensions. But for now, we will talk about ultrafilters.

Let's see an example concept.
\begin{proposition}
	Fix a functor $F\colon\mc C\to\mc D$.
	\begin{enumerate}
		\item The colimit $\colim F$ exists if and only if the left Kan extension of $!\colon\mc C\to1$ exists.
		\item Under the hypothesis of (a), $\op{Lan}_!F\cong\colim F$.
	\end{enumerate}
\end{proposition}
\begin{proof}
	Let's talk about why we might expect (a) to be true. Namely, we are looking for functors $d\colon1\Rightarrow\mc D$ is asking for the following diagram.
	% https://q.uiver.app/?q=WzAsMyxbMCwwLCJcXG1jIEMiXSxbMiwwLCJcXG1jIEQiXSxbMSwxLCIxIl0sWzAsMiwiISIsMl0sWzIsMSwiZCIsMl0sWzAsMSwiRiJdLFs1LDIsIlxcZ2FtbWEiLDAseyJzaG9ydGVuIjp7InNvdXJjZSI6MjAsInRhcmdldCI6MTB9fV1d&macro_url=https%3A%2F%2Fraw.githubusercontent.com%2FdFoiler%2Fnotes%2Fmaster%2Fnir.tex
	\[\begin{tikzcd}
		{\mc C} && {\mc D} \\
		& 1
		\arrow["{!}"', from=1-1, to=2-2]
		\arrow["d"', from=2-2, to=1-3]
		\arrow[""{name=0, anchor=center, inner sep=0}, "F", from=1-1, to=1-3]
		\arrow["\gamma", shorten <=3pt, shorten >=2pt, Rightarrow, from=0, to=2-2]
	\end{tikzcd}\]
	However, this is equivalent data to a cone $\gamma\colon F\Rightarrow d$. If we have a left Kan extension, then this is again equivalent data to natural transformations $\op{Lan}_!F\Rightarrow d$.
	
	But because we are doing everything over $1$, this is the same data as a morphism $\op{Mor}_\mc D(\op{Lan}_!F,d)$. So $\op{Lan}_!F$ will exactly represent $\op{Cone}(F,-)$, meaning that colimits and the left Kan extension have exactly the same data.

	This proof technically gives us (b) for free because we showed that we have the same data from both objects, but we can also write down the formula
	\[\op{Lan}_!F\cong\colim\left(\mc C\vec\times_1\{*\}\stackrel F\to\mc D\right),\]
	but of course $\mc C\vec\times_1\{*\}$ has is the same data as $\mc C$.
\end{proof}
\begin{remark}
	We can also interpret limits as right Kan extensions, by flipping the arrows.
\end{remark}
Next let's see adjoints.
\begin{proposition}
	The data of an adjunction $F\adjoint G$ with unit $\eta\colon\id_\mc C\Rightarrow GF$ and counit $\varepsilon\colon FG\Rightarrow\id_\mc D$ has the same data of left and right Kan extensions of the following two diagrams, respectively.
	% https://q.uiver.app/?q=WzAsNixbMCwwLCJcXG1jIEMiXSxbMSwwLCJcXG1jIEMiXSxbMCwxLCJcXG1jIEQiXSxbMywwLCJcXG1jIEQiXSxbNCwwLCJcXG1jIEQiXSxbMywxLCJcXG1jIEMiXSxbMCwxLCIiLDAseyJsZXZlbCI6Miwic3R5bGUiOnsiaGVhZCI6eyJuYW1lIjoibm9uZSJ9fX1dLFswLDIsIkYiLDJdLFsyLDEsIkciLDJdLFszLDUsIkciLDJdLFszLDQsIiIsMCx7ImxldmVsIjoyLCJzdHlsZSI6eyJoZWFkIjp7Im5hbWUiOiJub25lIn19fV0sWzUsNCwiRiIsMl0sWzYsOCwiXFxldGEiLDIseyJzaG9ydGVuIjp7InNvdXJjZSI6MjAsInRhcmdldCI6MjB9fV0sWzExLDEwLCJcXHZhcmVwc2lsb24iLDAseyJzaG9ydGVuIjp7InNvdXJjZSI6MjAsInRhcmdldCI6MjB9fV1d&macro_url=https%3A%2F%2Fraw.githubusercontent.com%2FdFoiler%2Fnotes%2Fmaster%2Fnir.tex
	\[\begin{tikzcd}
		{\mc C} & {\mc C} && {\mc D} & {\mc D} \\
		{\mc D} &&& {\mc C}
		\arrow[""{name=0, anchor=center, inner sep=0}, Rightarrow, no head, from=1-1, to=1-2]
		\arrow["F"', from=1-1, to=2-1]
		\arrow[""{name=1, anchor=center, inner sep=0}, "G"', from=2-1, to=1-2]
		\arrow["G"', from=1-4, to=2-4]
		\arrow[""{name=2, anchor=center, inner sep=0}, Rightarrow, no head, from=1-4, to=1-5]
		\arrow[""{name=3, anchor=center, inner sep=0}, "F"', from=2-4, to=1-5]
		\arrow["\eta"', shorten <=2pt, shorten >=2pt, Rightarrow, from=0, to=1]
		\arrow["\varepsilon", shorten <=2pt, shorten >=2pt, Rightarrow, from=3, to=2]
	\end{tikzcd}\]
	In particular, the right adjoint $G$ is the left Kan extension $\op{Lan}_F\id_\mc C$, and the left adjoint $F$ is the right Kan extension $\op{Ran}_G\id_\mc D$.
\end{proposition}
\begin{proof}
	We will only show one direction; suppose $F\adjoint G$. The main point is that we get an adjunction
	% https://q.uiver.app/?q=WzAsMixbMCwwLCJcXG9we0Z1bn0oXFxtYyBELFxcbWMgQykiXSxbMSwwLCJcXG9we0Z1bn0oXFxtYyBDLFxcbWMgQykiXSxbMCwxLCItXFxjaXJjIEYiLDIseyJvZmZzZXQiOjJ9XSxbMSwwLCJHXFxjaXJjLSIsMix7Im9mZnNldCI6Mn1dLFszLDIsIlxcYWRqb2ludCIsMyx7InNob3J0ZW4iOnsic291cmNlIjoyMCwidGFyZ2V0IjoyMH0sInN0eWxlIjp7ImJvZHkiOnsibmFtZSI6Im5vbmUifSwiaGVhZCI6eyJuYW1lIjoibm9uZSJ9fX1dXQ==&macro_url=https%3A%2F%2Fraw.githubusercontent.com%2FdFoiler%2Fnotes%2Fmaster%2Fnir.tex
	\[\begin{tikzcd}
		{\op{Fun}(\mc D,\mc C)} & {\op{Fun}(\mc C,\mc C)}
		\arrow[""{name=0, anchor=center, inner sep=0}, "{-\circ F}"', shift right=2, from=1-1, to=1-2]
		\arrow[""{name=1, anchor=center, inner sep=0}, "{G\circ-}"', shift right=2, from=1-2, to=1-1]
		\arrow["\adjoint"{marking}, Rightarrow, draw=none, from=1, to=0]
	\end{tikzcd}\]
	from \autoref{thm:whiskering}. However, the uniqueness of adjoints combined with what we already know about the adjoint of $(-\circ F)$ from \autoref{cor:getadjointprecomp}, we are able to conclude
	\[\op{Lan}_F(-)\adjoint(-\circ F).\]
	In particular, $\op{Lan}_F\id_\mc C\cong G^*(\id_\mc C)=G$, so we are done.
\end{proof}

\subsection{The Kan Extension Formula}
To set up our discussion, fix $F\colon\mc C\to\mc D$ such that $\mc D$ has all colimits and that $\mc C$ is small. Then, by the formula in \autoref{cor:kanformulaex}, we know that we can extend $F$ as in the following Kan extension diagram.
% https://q.uiver.app/?q=WzAsMyxbMCwwLCJcXG1jIEMiXSxbMSwwLCJcXG1jIEQiXSxbMCwxLCJcXG1jIEMiXSxbMCwxLCJGIl0sWzAsMiwiIiwyLHsibGV2ZWwiOjIsInN0eWxlIjp7ImhlYWQiOnsibmFtZSI6Im5vbmUifX19XSxbMiwxLCIiLDIseyJzdHlsZSI6eyJib2R5Ijp7Im5hbWUiOiJkYXNoZWQifX19XV0=&macro_url=https%3A%2F%2Fraw.githubusercontent.com%2FdFoiler%2Fnotes%2Fmaster%2Fnir.tex
\[\begin{tikzcd}
	{\mc C} & {\mc D} \\
	{\mc C}
	\arrow["F", from=1-1, to=1-2]
	\arrow[Rightarrow, no head, from=1-1, to=2-1]
	\arrow[dashed, from=2-1, to=1-2]
\end{tikzcd}\]
However, this extension somewhat clearly should be $F$, so the formula in \autoref{cor:kanformulaex} tells us
\[Fc\cong\colim\left(\mc C\vec{{}\times{}}_\mc Cc\to\mc C\stackrel F\to\mc D\right)\]
Here are some consequences.
\begin{theorem}[Co-Yoneda lemma]
	Fix everything as above. Then we can the diagram
	% https://q.uiver.app/?q=WzAsMyxbMCwwLCJcXGRpc3BsYXlzdHlsZVxcY29wcm9kX3t5XFx0byB4XFx0byBjfUYoeSkiXSxbMiwwLCJGYyJdLFsxLDAsIlxcZGlzcGxheXN0eWxlXFxjb3Byb2Rfe3hcXHRvIGN9RngiXSxbMiwxXSxbMCwyLCIiLDAseyJvZmZzZXQiOjJ9XSxbMCwyLCIiLDAseyJvZmZzZXQiOi0yfV1d&macro_url=https%3A%2F%2Fraw.githubusercontent.com%2FdFoiler%2Fnotes%2Fmaster%2Fnir.tex
	\[\begin{tikzcd}
		{\displaystyle\coprod_{y\to x\to c}F(y)} & {\displaystyle\coprod_{x\to c}Fx} & Fc
		\arrow[from=1-2, to=1-3]
		\arrow[shift right=2, from=1-1, to=1-2]
		\arrow[shift left=2, from=1-1, to=1-2]
	\end{tikzcd}\]
	is a coequalizer diagram; here one of the maps being coequalized is the inclusion from $Fy$, and the other is the inclusion from $Fx$.
\end{theorem}
\begin{theorem}[Density, I]
	Fix a functor $F\colon\mc C\to\mathrm{Set}$ from a small category $\mc C$. Then $F$ is the colimit of the diagram
	\[\left(\int_\mc CF\right)\opp\to\mc C\opp\stackrel\yo\into\op{Fun}(\mc C,\mathrm{Set}).\]
	Dually, a functor $F\colon\mc C\opp\to\mathrm{Set}$ is the colimit of the diagram
	\[\int_{\mc C\opp}F\to\mc C\stackrel\yo\to\op{PSh}(\mc C).\]
\end{theorem}
\begin{proof}
	This follows quickly from the Co-Yoneda lemma, with a little elbow grease.
\end{proof}
\begin{theorem}[Density, II]
	The diagram
	% https://q.uiver.app/?q=WzAsMyxbMCwwLCJcXG1jIEMiXSxbMSwwLCJcXG9we1BTaH1cXG1jIEMiXSxbMCwxLCJcXG9we1BTaH1cXG1jIEMiXSxbMCwyLCJcXHlvIiwyLHsic3R5bGUiOnsidGFpbCI6eyJuYW1lIjoiaG9vayIsInNpZGUiOiJ0b3AifX19XSxbMCwxLCJcXHlvIiwwLHsic3R5bGUiOnsidGFpbCI6eyJuYW1lIjoiaG9vayIsInNpZGUiOiJ0b3AifX19XSxbMiwxLCIiLDIseyJsZXZlbCI6Miwic3R5bGUiOnsiaGVhZCI6eyJuYW1lIjoibm9uZSJ9fX1dXQ==&macro_url=https%3A%2F%2Fraw.githubusercontent.com%2FdFoiler%2Fnotes%2Fmaster%2Fnir.tex
	\[\begin{tikzcd}
		{\mc C} & {\op{PSh}\mc C} \\
		{\op{PSh}\mc C}
		\arrow["\yo"', hook, from=1-1, to=2-1]
		\arrow["\yo", hook, from=1-1, to=1-2]
		\arrow[Rightarrow, no head, from=2-1, to=1-2]
	\end{tikzcd}\]
	gives a left Kan extension.
\end{theorem}
\begin{proof}
	Using the formula, we can just compute
	\[(\op{Lan}_\yo\yo)(F)\cong\colim\left(\mc C\vec{{}\times{}}_{\op{PSh}\mc C}F\to\mc C\stackrel\yo\into\op{PSh}\mc C\right).\]
	However, $\mc C\vec{{}\times{}}_{\op{PSh}\mc C}F$ is $\int_{\mc C\opp}F$, so the previous density theorem tells us that we get $F$ out of this colimit.
\end{proof}