\documentclass[../notes.tex]{subfiles}

\begin{document}

% !TEX root = ../notes.tex

Bryce's advisor will be giving next week's lectures. So it's time to speed-run adjunctions.

\subsection{Contravariant Adjoints}
Let's discuss trying to use contravariant functors to make adjunctions. We might start with two contravariant functors $F:\mc C\opp\to\mc D$ and $G:\mc D\opp\to\mc C$, but these are not compatible because they don't go both ways properly. As such, we want to turn $G$ into a functor
\[\mc D\to\mc C\opp.\]
Indeed, functors $\mc A\to\mc B$ can become functors $\mc A\opp\to\mc B\opp$ by just reversing all the arrows, so we can indeed view $G$ as a functor $G:\mc D\to\mc C\opp$, as desired.

Now, to make out adjoint, we might just try to require the isomorphism
\[\op{Mor}_\mc D(Fc,d)\cong\op{Mor}_\mc C\opp(c,Gd)=\op{Mor}_\mc C(Gd,c),\]
but now both functors are on the left side, so this is a little weird. Nonetheless, we have the following definition.
\begin{definition}[Mutually adjoint]
	Two contravariant functors $F:\mc C\opp\to\mc D$ and $G:\mc D\opp\to\mc C$ (thought of as a functor $\mc D\to\mc C\opp$) are \textit{mutually left adjoint} if and only if there are natural isomorphisms
	\[\op{Mor}_\mc D(Fc,d)\cong\op{Mor}_\mc C(Gd,c).\]
	They are \textit{mutually right adjoint} if and only if there are natural isomorphisms
	\[\op{Mor}_\mc D(d,Fc)\cong\op{Mor}_\mc C(c,Gd).\]
\end{definition}
\begin{exe}
	Fix $\mc P:\mathrm{Set}\opp\to\mathrm{Set}$ sending $A\mapsto\mc P(A)$ and $f\mapsto f^{-1}$. We claim that $\mf P$ is mutually right adjoint with itself.
\end{exe}
\begin{proof}
	We are requiring a natural isomorphism
	\[\op{Mor}_{\mathrm{Set}}(Y,\mc PX)\stackrel?\cong\op{Mor}_{\mathrm{Set}}(X,\mc PY).\]
	The main point is that $\mathrm{Set}$ is Cartesian-closed, roughly meaning that we can curry as
	\[\op{Mor}_{\mathrm{Set}}(X\times Y,Z)\cong\op{Mor}_{\mathrm{Set}}(X,\op{Mor}(Y,Z)).\]
	Now, recall that $\mf P$ is represented by $\Omega:=\{\mathrm T,\mathrm F\}$, in that $\mc P\cong\op{Mor}_{\mathrm{Set}}(-,\Omega)$. Thus,
	\begin{align*}
		\op{Mor}_{\mathrm{Set}}(X,\mc PY) &\cong \op{Mor}_{\mathrm{Set}}(X,\op{Mor}(Y,\Omega)) \\
		&\cong \op{Mor}_{\mathrm{Set}}(X\times Y,\Omega) \\
		&\cong \op{Mor}_{\mathrm{Set}}(Y\times X,\Omega) \\
		\op{Mor}_{\mathrm{Set}}(Y,\op{Mor}(X,\Omega)) \\
		\op{Mor}_{\mathrm{Set}}(Y,\mc PX),
	\end{align*}
	and everything is natural, so we are done.
\end{proof}

\subsection{Uniqueness of Adjoints}
We take a moment clean up after ourselves and quickly justify why we have been saying ``the adjoint.''
\begin{proposition}
	Fix functors $F,F'$ which are both left adjoint to $G$. Then there is a unique natural isomorphism $\theta:F\cong F$ such that the following two triangles commute.
	% https://q.uiver.app/?q=WzAsNixbMCwwLCJcXGlkX1xcbWMgQyJdLFsxLDAsIkdGIl0sWzEsMSwiR0YnIl0sWzIsMCwiRkciXSxbMiwxLCJGJ0ciXSxbMywwLCJcXGlkX1xcbWMgRCJdLFswLDEsIlxcZXRhIiwwLHsibGV2ZWwiOjJ9XSxbMSwyLCJHXFx0aGV0YSIsMCx7ImxldmVsIjoyfV0sWzAsMiwiXFxldGEnIiwyLHsibGV2ZWwiOjJ9XSxbMyw0LCJcXHRoZXRhIEciLDIseyJsZXZlbCI6Mn1dLFszLDUsIlxcdmFyZXBzaWxvbiIsMCx7ImxldmVsIjoyfV0sWzQsNSwiXFx2YXJlcHNpbG9uJyIsMix7ImxldmVsIjoyfV1d&macro_url=https%3A%2F%2Fraw.githubusercontent.com%2FdFoiler%2Fnotes%2Fmaster%2Fnir.tex
	\[\begin{tikzcd}
		{\id_\mc C} & GF & FG & {\id_\mc D} \\
		& {GF'} & {F'G}
		\arrow["\eta", Rightarrow, from=1-1, to=1-2]
		\arrow["G\theta", Rightarrow, from=1-2, to=2-2]
		\arrow["{\eta'}"', Rightarrow, from=1-1, to=2-2]
		\arrow["{\theta G}"', Rightarrow, from=1-3, to=2-3]
		\arrow["\varepsilon", Rightarrow, from=1-3, to=1-4]
		\arrow["{\varepsilon'}"', Rightarrow, from=2-3, to=1-4]
	\end{tikzcd}\]
\end{proposition}
\begin{proof}
	We start by exhibiting $\theta$, which we do by hand. As such, we define
	\[\theta:=\varepsilon F'\circ F\eta'\qquad\text{and}\qquad\theta':=\varepsilon'F\circ F'\eta.\]
	We show that these are inverse by hand; note that they are natural transformations as composition and whiskering of natural transformations. We can type-check that $F\eta':F\Rightarrow FGF$ and then $\varepsilon F':FGF'\Rightarrow F'$ and similar for $\theta'$.

	To show these are inverse, we show that
	\[\theta'\theta\stackrel?=\id_F,\]
	which we do on components as
	\[(\theta'\theta)_c=\id_c\]
	for some object $c$. To do this, we show that the transposes are equal, for which we compute the transpose $(\theta'\theta)_c$ as
	\[G(\theta'\theta)\circ\eta_c=G\varepsilon'F\circ GF'\eta\circ G\varepsilon F'\circ GF\eta'\circ\eta,\]
	where we have dropped the $c$ out of laziness. We now use naturality. Namely, we see
	% https://q.uiver.app/?q=WzAsNCxbMCwwLCJcXGlkX3tcXG1jIEN9Il0sWzEsMCwiR0YiXSxbMSwxLCJHRkdGJyJdLFswLDEsIkdGJyJdLFswLDMsIlxcZXRhJyIsMix7ImxldmVsIjoyfV0sWzAsMSwiXFxldGEiLDAseyJsZXZlbCI6Mn1dLFsxLDIsIkdGXFxldGEnIiwwLHsibGV2ZWwiOjJ9XSxbMywyLCJcXGV0YSBHRiciLDIseyJsZXZlbCI6Mn1dXQ==&macro_url=https%3A%2F%2Fraw.githubusercontent.com%2FdFoiler%2Fnotes%2Fmaster%2Fnir.tex
	\[\begin{tikzcd}
		{\id_{\mc C}} & GF \\
		{GF'} & {GFGF'}
		\arrow["{\eta'}"', Rightarrow, from=1-1, to=2-1]
		\arrow["\eta", Rightarrow, from=1-1, to=1-2]
		\arrow["{GF\eta'}", Rightarrow, from=1-2, to=2-2]
		\arrow["{\eta GF'}"', Rightarrow, from=2-1, to=2-2]
	\end{tikzcd}\]
	gives
	\[G\varepsilon'F\circ GF'\eta\circ G\varepsilon F'\circ (GF\eta'\circ\eta)=G\varepsilon'F\circ GF'\eta\circ G\varepsilon F'\circ \eta GF'\circ\eta'.\]
	Using the triangle equalities, we note that $G\varepsilon F'\circ\eta GF'=(G\varepsilon\circ\eta G)F'$ vanishes,\footnote{Namely, $G\varepsilon F'\circ\eta G$ is an identity natural transformation, so whiskering with $F'$ on the right gives the morphism $\id_{F'(-)}$, which does indeed vanish} so we are left with
	\[G\varepsilon'F\circ GF'\eta\circ\eta'.\]
	Using naturality in the same way as last time but changing the primes around, we get
	\[G\varepsilon'F\circ GF'\eta\circ\eta,\]
	which again by the triangle equalities simply vanishes into $\eta$. So indeed, the transpose of $(\theta'\theta)_c$ is $\eta_c$, which is the transpose of $\id_c$.

	We now show that our triangles commute as needed. For the left triangle, we compute
	\[G\theta\circ\eta=G(\varepsilon F'\circ F\eta')\circ\eta=G\varepsilon F'\circ GF\eta'\circ\eta.\]
	By the naturality from before, we have $GF\eta'\circ\eta=\eta GF'\circ\eta'$, which gives
	\[G\varepsilon F'\circ\eta GF'\circ\eta'=(G\varepsilon\circ\eta G)\circ\eta'=\eta'\]
	where we have used the triangle equalities. The other triangle follows similarly, which we omit.

	It remains to show uniqueness of $\theta$. Pick up an object $c$, and we show that $\theta_c:Fc\to F'c$ is unique. Well, to make the left triangle commute, we need the following triangle to commute.
	% https://q.uiver.app/?q=WzAsMyxbMCwwLCJjIl0sWzEsMCwiR0ZjIl0sWzEsMSwiR0YnYyJdLFswLDEsIlxcZXRhX2MiXSxbMSwyLCJHXFx0aGV0YV9jIl0sWzAsMiwiXFxldGEnX2MiLDJdXQ==
	\[\begin{tikzcd}
		c & GFc \\
		& {GF'c}
		\arrow["{\eta_c}", from=1-1, to=1-2]
		\arrow["{G\theta_c}", from=1-2, to=2-2]
		\arrow["{\eta'_c}"', from=1-1, to=2-2]
	\end{tikzcd}\]
	Now, we can compute the transpose of $\theta_c:Fc\to F'c$ (through $F\adjoint G$) as $G\theta_c\circ\eta_c$, which is $\eta_c'$ from the above triangle. As such, $\theta_c$ must be the transpose of $\eta'_c$, which is exactly how we constructed $\theta$ to begin with anyway.
\end{proof}

\subsection{Composing Adjunctions}
Adjunctions can also create more adjunctions, provided they cohere.
\begin{proposition}
	Fix $F\adjoint G$ and $F'\adjoint G'$ in the following diagram.
	% https://q.uiver.app/?q=WzAsMyxbMCwwLCJcXG1jIEMiXSxbMSwwLCJcXG1jIEQiXSxbMiwwLCJcXG1jIEUiXSxbMSwwLCJHIiwwLHsib2Zmc2V0IjotMn1dLFsxLDIsIkYnIiwwLHsib2Zmc2V0IjotMn1dLFsyLDEsIkcnIiwwLHsib2Zmc2V0IjotMn1dLFswLDEsIkYiLDAseyJvZmZzZXQiOi0yfV0sWzYsMywiXFxhZGpvaW50IiwzLHsic2hvcnRlbiI6eyJzb3VyY2UiOjIwLCJ0YXJnZXQiOjIwfSwibGV2ZWwiOjEsInN0eWxlIjp7ImJvZHkiOnsibmFtZSI6Im5vbmUifSwiaGVhZCI6eyJuYW1lIjoibm9uZSJ9fX1dLFs0LDUsIlxcYWRqb2ludCIsMyx7InNob3J0ZW4iOnsic291cmNlIjoyMCwidGFyZ2V0IjoyMH0sInN0eWxlIjp7ImJvZHkiOnsibmFtZSI6Im5vbmUifSwiaGVhZCI6eyJuYW1lIjoibm9uZSJ9fX1dXQ==&macro_url=https%3A%2F%2Fraw.githubusercontent.com%2FdFoiler%2Fnotes%2Fmaster%2Fnir.tex
	\[\begin{tikzcd}
		{\mc C} & {\mc D} & {\mc E}
		\arrow[""{name=0, anchor=center, inner sep=0}, "G", shift left=2, from=1-2, to=1-1]
		\arrow[""{name=1, anchor=center, inner sep=0}, "{F'}", shift left=2, from=1-2, to=1-3]
		\arrow[""{name=2, anchor=center, inner sep=0}, "{G'}", shift left=2, from=1-3, to=1-2]
		\arrow[""{name=3, anchor=center, inner sep=0}, "F", shift left=2, from=1-1, to=1-2]
		\arrow["\adjoint"{marking}, draw=none, from=3, to=0]
		\arrow["\adjoint"{marking}, Rightarrow, draw=none, from=1, to=2]
	\end{tikzcd}\]
	Then $F'F\adjoint GG'$.
\end{proposition}
\begin{proof}
	We compute
	\begin{align*}
		\op{Mor}_\mc E(F'Fc,e) &\cong \op{Mor}_\mc D(Fc,G'e) \\
		&\cong \op{Mor}_\mc C(c,GG'e)
	\end{align*}
	by tracking through our various adjunctions. This finishes.
\end{proof}
And here is the last thing we will prove today.
\begin{proposition}
	Fix an equivalence of categories $F:\mc C\simeq\mc D:G$ with its promised natural isomorphisms $\eta:\id_\mc C\Rightarrow GF$ and $\varepsilon:\id_{\mc D}\Rightarrow FG$. We can replace these with an adjoint equivalence by modifying either one of $\delta$ or $\varepsilon$ (notably not changing $F$ and $G$!).
\end{proposition}
\begin{proof}
	This proof is, reportedly, a little long. Fix $\eta$ and we modify $\varepsilon$. (For the other result, swap $F$ and $G$ and $\varepsilon$ and $\eta$.) Now, we fix
	\[\gamma:=G\varepsilon\circ\eta G\]
	which is a natural transformation from $G\Rightarrow G$ (namely, $\eta G:G\Rightarrow GFG$ and $G\varepsilon:GFG\Rightarrow G$). Notably, this is a natural isomorphism because $G\varepsilon$ and $\eta G$ are both isomorphisms at each unit.

	Our goal is to kill $\gamma$ to fix our triangle equalities. As such, we set
	\[\varepsilon':=\varepsilon\circ F\gamma^{-1},\]
	which is now a natural isomorphism $FG\cong\id_\mc D$ by a similar computation to before: we send $FG\Rightarrow FG\Rightarrow\id_\mc D$.

	We now check our triangle equalities. Note that the following diagram commutes.
	% https://q.uiver.app/?q=WzAsNSxbMCwwLCJHIl0sWzEsMCwiR0ZHIl0sWzEsMSwiR0ZHIl0sWzAsMSwiRyJdLFsxLDIsIkciXSxbMSwyLCJHRlxcZ2FtbWFeey0xfSIsMCx7ImxldmVsIjoyfV0sWzAsMywiXFxnYW1tYV57LTF9IiwyLHsibGV2ZWwiOjJ9XSxbMywyLCJcXGV0YSBHIiwwLHsibGV2ZWwiOjJ9XSxbMiw0LCJHXFx2YXJlcHNpbG9uIiwwLHsibGV2ZWwiOjJ9XSxbMyw0LCJcXGdhbW1hIiwyLHsibGV2ZWwiOjJ9XSxbMCwxLCJcXGV0YSBHIiwwLHsibGV2ZWwiOjJ9XV0=&macro_url=https%3A%2F%2Fraw.githubusercontent.com%2FdFoiler%2Fnotes%2Fmaster%2Fnir.tex
	\[\begin{tikzcd}
		G & GFG \\
		G & GFG \\
		& G
		\arrow["{GF\gamma^{-1}}", Rightarrow, from=1-2, to=2-2]
		\arrow["{\gamma^{-1}}"', Rightarrow, from=1-1, to=2-1]
		\arrow["{\eta G}", Rightarrow, from=2-1, to=2-2]
		\arrow["G\varepsilon", Rightarrow, from=2-2, to=3-2]
		\arrow["\gamma"', Rightarrow, from=2-1, to=3-2]
		\arrow["{\eta G}", Rightarrow, from=1-1, to=1-2]
	\end{tikzcd}\]
	The bottom triangle commutes by definition of $\gamma$, and the square commutes by naturality of $\eta G$---simply chase an element all the way through. As such, we can collapse this diagram down to the following.
	% https://q.uiver.app/?q=WzAsMyxbMCwwLCJHIl0sWzEsMCwiR0ZHIl0sWzEsMSwiRyJdLFswLDEsIlxcZXRhIEciLDAseyJsZXZlbCI6Mn1dLFsxLDIsIkdcXHZhcmVwc2lsb24nIiwwLHsibGV2ZWwiOjJ9XSxbMCwyLCJcXGdhbW1hXFxjaXJjXFxnYW1tYV57LTF9IiwyLHsibGV2ZWwiOjIsInN0eWxlIjp7ImhlYWQiOnsibmFtZSI6Im5vbmUifX19XV0=&macro_url=https%3A%2F%2Fraw.githubusercontent.com%2FdFoiler%2Fnotes%2Fmaster%2Fnir.tex
	\[\begin{tikzcd}
		G & GFG \\
		& G
		\arrow["{\eta G}", Rightarrow, from=1-1, to=1-2]
		\arrow["{G\varepsilon'}", Rightarrow, from=1-2, to=2-2]
		\arrow["{\gamma\circ\gamma^{-1}}"', Rightarrow, no head, from=1-1, to=2-2]
	\end{tikzcd}\]
	This is our first triangle inequality, so we are done.

	For the second triangle inequality, we draw the following huge diagram.
	% https://q.uiver.app/?q=WzAsOCxbMCwwLCJGIl0sWzEsMCwiRkdGIl0sWzIsMCwiRiJdLFswLDEsIkZHRiJdLFsxLDEsIkZHRkdGIl0sWzIsMSwiRkdGIl0sWzEsMiwiRkdGIl0sWzIsMiwiRiJdLFszLDYsIiIsMCx7ImxldmVsIjoyLCJzdHlsZSI6eyJoZWFkIjp7Im5hbWUiOiJub25lIn19fV0sWzYsNywiXFx2YXJlcHNpbG9uJ0YiLDIseyJsZXZlbCI6Mn1dLFs1LDcsIlxcdmFyZXBzaWxvbidGIiwwLHsibGV2ZWwiOjJ9XSxbNCw2LCJGR1xcdmFyZXBzaWxvbidGIiwwLHsibGV2ZWwiOjJ9XSxbMCwxLCJGXFxldGEiLDAseyJsZXZlbCI6Mn1dLFsxLDIsIlxcdmFyZXBzaWxvbidGIiwwLHsibGV2ZWwiOjJ9XSxbMiw1LCJGXFxldGEiLDAseyJsZXZlbCI6Mn1dLFswLDMsIkZcXGV0YSIsMix7ImxldmVsIjoyfV0sWzMsNCwiRlxcZXRhIEdGIiwwLHsibGV2ZWwiOjJ9XSxbMSw0LCJGR0ZcXGV0YSIsMCx7ImxldmVsIjoyfV0sWzQsNSwiXFx2YXJlcHNpbG9uJ0ZHRiIsMCx7ImxldmVsIjoyfV1d&macro_url=https%3A%2F%2Fraw.githubusercontent.com%2FdFoiler%2Fnotes%2Fmaster%2Fnir.tex
	\[\begin{tikzcd}
		F & FGF & F \\
		FGF & FGFGF & FGF \\
		& FGF & F
		\arrow[Rightarrow, no head, from=2-1, to=3-2]
		\arrow["{\varepsilon'F}"', Rightarrow, from=3-2, to=3-3]
		\arrow["{\varepsilon'F}", Rightarrow, from=2-3, to=3-3]
		\arrow["{FG\varepsilon'F}", Rightarrow, from=2-2, to=3-2]
		\arrow["F\eta", Rightarrow, from=1-1, to=1-2]
		\arrow["{\varepsilon'F}", Rightarrow, from=1-2, to=1-3]
		\arrow["F\eta", Rightarrow, from=1-3, to=2-3]
		\arrow["F\eta"', Rightarrow, from=1-1, to=2-1]
		\arrow["{F\eta GF}", Rightarrow, from=2-1, to=2-2]
		\arrow["FGF\eta", Rightarrow, from=1-2, to=2-2]
		\arrow["{\varepsilon'FGF}", Rightarrow, from=2-2, to=2-3]
	\end{tikzcd}\]
	We quickly review why this diagram commutes.
	\begin{itemize}
		\item The top-left square commutes by the naturality of $\eta$ applied to a square of $F$s.
		\item The top-right square commutes by naturality of $\varepsilon'$ applied to a square of $F$s.
		\item The bottom-right square commutes by the naturality of $\varepsilon'$ applied to a square of $F$s.
		\item The bottom-left triangle commutes by the triangle equality we showed for $\varepsilon'$ already.
	\end{itemize}
	This diagram now collapses to
	\[(\varepsilon'F\circ F\eta)^2=\varepsilon'F\circ F\eta,\]
	which forces $\varepsilon'F\circ F\eta=\id$ because these are isomorphisms, which is what we wanted.
\end{proof}

\end{document}