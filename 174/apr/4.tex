% !TEX root = ../notes.tex

Chris is back!

\subsection{Elements in Categories}
Fix a concrete category $\mathcal C$ with its forgetful functor $U:\mathcal C\to\mathrm{Set}$.
\begin{example}
	In $\mathrm{Set}$, we can view elements $x\in X$ for a set $X$ as morphisms from $\{*\}$ to $X$.
\end{example}
\begin{example}
	In $\mathrm{Grp}$, we can view elements $g\in G$ for a group $G$ as morphisms from $\ZZ$ to $G$, by essentially tracking where $1$ goes.
\end{example}
Can we work more generally?
\begin{nex}
	In $\mathrm{FinGrp}$, there is no way for this to occur. Roughly speaking, we are asking for $\op{Mor}(T,-)\simeq U$, but this functor $U$ need not be representable.
\end{nex}
It turns out that there is no way to make this precise for general categories. As such, we have the following definition.
\begin{definition}[Generalized element]
	Fix a terminal object $T$ in a category $\mathcal C$. Then we call a \textit{generalized element} of an object $X$ to be a morphism $T\to X$. More generally, a \textit{generalized element of shape} $Y\in\mathcal C$ is a morphism $Y\to X$.
\end{definition}
It turns out that arguments about elements can often be (philosophically) transformed into arguments about generalized elements.

We are feeling benevolent, so here are some examples.
\begin{example}
	Fix morphisms $f,g:A\to B$. We can translate the statement
	\[f=g\iff f(x)=g(x)\text{ for all }a\in A\]
	into asserting that
	\[f=g\iff fx=gx\text{ for all }x:X\to A.\]
	To see this that this is true, we note the forward direction is by substitution, and the reverse direction is by setting $X=A$ and $x=\id_A$ so that we are told $f=fx=gx=g$.
\end{example}
\begin{example}
	A morphism $f:A\to B$ is a monomorphism if and only if, for each object $X$, we have $fg=fh$ for $g,h:X\to A$ implies $g=h$. This is merely saying that $fg=fh$ for ``generalized elements,'' which correctly generalizes our notion of injectivity.
\end{example}
\begin{nex}
	A morphism $f:A\to B$ is an epimorphism if and only if, for each object $X$, we have $gf=hf$ for $g,h:B\to X$ implies $g=h$. This does not look like surjectivity for sets: we would be asking that any morphism $g:X\to B$ has a morphism $h:X\to A$ making the following diagram commute.
	% https://q.uiver.app/?q=WzAsMyxbMSwxLCJYIl0sWzAsMCwiQSJdLFsxLDAsIkIiXSxbMSwyLCJmIl0sWzAsMiwiZyIsMl0sWzAsMSwiaCIsMCx7InN0eWxlIjp7ImJvZHkiOnsibmFtZSI6ImRhc2hlZCJ9fX1dXQ==
	\[\begin{tikzcd}
		A & B \\
		& X
		\arrow["f", from=1-1, to=1-2]
		\arrow["g"', from=2-2, to=1-2]
		\arrow["h", dashed, from=2-2, to=1-1]
	\end{tikzcd}\]
	This is a fairly strong condition.
\end{nex}

\subsection{Sheaves}
For our next example, we define a sheaf.
\begin{definition}[Presheaf]
	Fix a category $\mathcal C$. A $\mathcal D$-valued \textit{presheaf} on $\mathcal C$ is a contravariant functor $\mathcal C\opp\to\mathrm{Set}$. If $\mathcal D$ is omitted, we will assume that $\mathcal D=\mathrm{Set}$.
\end{definition}
To define a sheaf, we need to talk about topological spaces, which is made of a set $X$ and a topology $T$. Then we have a category $\mathcal O(X)$, which is a preorder by the containment of open sets; i.e., we have exactly one morphism $U\to V$ for open sets $U,V\subseteq X$ if and only if $U\subseteq V$.

In particular, given a presheaf $\mathcal F:\mathcal O(X)\opp\to\mathrm{Set}$, we see that $U\subseteq V$ induces a morphism $\mathcal F(V)\to\mathcal F(U)$, which we call restriction because we will think about this as restriction of functions. We will denote $f\in\mathcal F(V)$ as going to $f|_U\in\mathcal F(U)$; there is no ambiguity to the morphism $\mathcal F(V)\to\mathcal F(U)$ because the morphism $U\to V$ is unique.

Now, here is our definition.
\begin{defi}[Sheaf]
	Fix a topological space $(X,T)$ to pick up a presheaf $F:\mathcal O(X)\opp\to\mathrm{Set}$. Then $F$ is a ($\mathrm{Set}$-valued) \textit{sheaf} if and only if it satisfies the following extra conditions.
	\begin{itemize}
		\item Locality: fix an open set $U$ and an open cover $\{U_\alpha\}_{\alpha\in\lambda}$. If $f|_{U_\alpha}=g|_{U_\alpha}$ for all $\alpha$, we have $f=g$.
		\item Gluing: fix an open set $U$ and an open cover $\{U_\alpha\}_{\alpha\in\lambda}$. If we have local elements $f_\alpha\in\mathcal O(U_\alpha)$ such that $f_\alpha|_{U_\alpha\cap U_\beta}=f_\beta|_{U_\alpha\cap U_\beta}$ for all $\alpha,\beta\in\lambda$, then there is a global element $f\in\mathcal O(U)$ such that $f|_{U_\alpha}=f_\alpha$ for each $\alpha$.
	\end{itemize}
\end{defi}
Visually, locality is saying that an open cover of an open set $U$ can completely determine an element of $V$. Here is the image.
\begin{center}
	\begin{asy}
		unitsize(1cm);
		draw((0,0) .. (1,0) ..  (0,2) .. (-1,1) .. (-2,1) .. cycle);
		pair[] pts = {(0,0), (1,0), (0.5,1), (-0.2,1.4), (-1.3,1), (1.6,0.5), (-2,0.4), (1.4,1.5), (0.4,2), (-1,0.5)};
		for(int i = 0; i < pts.length; ++i)
		{
			draw(circle(pts[i],0.7), dotted);
		}
	\end{asy}
\end{center}
Now, if we have two elements $f$ and $g$ we are equal on the entire open set, then locality requires that $f=g$. Gluing is roughly saying that if we have a family $\{f_\alpha\}_{\alpha\in\lambda}$ on our open cover in such a way that we restrict properly to intersections, then we can glue our elements together.
\begin{example}
	Given topological spaces $X$ and $Y$, we can define the sheaf of continuous functions $X\to Y$ by taking an open set $U\subseteq X$ to the set of continuous functions $\mathcal O(U)$ from $U$ to $Y$. This satisfies locality and gluing somewhat intuitively.
	\begin{itemize}
		\item Locality: if two continuous functions are equal when restricted to an open cover, then they are equal as functions: simply track where all the elements go.
		\item Gluing: continuous functions can be built up from local functions if they agree on intersection. This is called the pasting lemma.
	\end{itemize}
\end{example}
Here is an alternate definition of sheaves.
\begin{defi}[Sheaf]
	Fix a topological space $(X,T)$ to pick up a $\mathcal C$-valued presheaf $F:\mathcal O(X)\opp\to\mathrm{Set}$. For any open set $U\subseteq X$ with open cover $\{U_\alpha\}_{\alpha\in\lambda}$, the morphism
	\[\mathcal F(U)\to\prod_{\alpha\in\lambda}\mathcal F(U_\alpha)\]
	can be induced (component-wise) by the restriction morphisms $\mathcal F(U)\to\mathcal F(U_\alpha)$. By restricting further, we can build two maps
	\[\pi_{\op{res}\alpha},\pi_{\op{res}\beta}\prod_{\alpha\in\lambda}\mathcal F(U_\alpha)\to\prod_{\alpha\in\lambda}\prod_{\beta\in\lambda}\mathcal F(U_\alpha\cap U_\beta)\]
	by taking $\{f_\alpha\}_{\alpha\in\lambda}$ by either going to $\pi_{\op{res}\alpha}:\{f_\alpha\}_{\alpha\in\lambda}\mapsto f_\alpha|_{U_\alpha\cap U_\beta}$ or $\pi_{\op{res}\alpha}:\{f_\alpha\}_{\alpha\in\lambda}\mapsto f_\beta|_{U_\alpha\cap U_\beta}$. Now, $\mathcal F$ is a \textit{sheaf} if and only if
	% https://q.uiver.app/?q=WzAsMyxbMCwwLCJcXG1hdGhjYWwgRihVKSJdLFsxLDAsIlxcZGlzcGxheXN0eWxlXFxwcm9kX3tcXGFscGhhXFxpblxcbGFtYmRhfVxcbWF0aGNhbCBGKFVfXFxhbHBoYSkiXSxbMiwwLCJcXGRpc3BsYXlzdHlsZVxccHJvZF97XFxhbHBoYVxcaW5cXGxhbWJkYX1cXHByb2Rfe1xcYmV0YVxcaW5cXGxhbWJkYX1cXG1hdGhjYWwgRihVX1xcYWxwaGFcXGNhcCBVX1xcYmV0YSkiXSxbMCwxXSxbMSwyLCJcXG9we3Jlc31cXGFscGhhIiwwLHsib2Zmc2V0IjotMn1dLFsxLDIsIlxcb3B7cmVzfVxcYmV0YSIsMCx7Im9mZnNldCI6Mn1dXQ==
	\[\begin{tikzcd}
		{\mathcal F(U)} & {\displaystyle\prod_{\alpha\in\lambda}\mathcal F(U_\alpha)} & {\displaystyle\prod_{\alpha\in\lambda}\prod_{\beta\in\lambda}\mathcal F(U_\alpha\cap U_\beta)}
		\arrow[from=1-1, to=1-2]
		\arrow["{\op{res}\alpha}", shift left=2, from=1-2, to=1-3]
		\arrow["{\op{res}\beta}"', shift right=2, from=1-2, to=1-3]
	\end{tikzcd}\]
	is an equalizer diagram.
\end{defi}
We would like to show that our second definition correctly generalizes the first definition in the case where $\mathcal C=\mathrm{Set}$.

Indeed, suppose that $\mathcal F:\mathcal O(X)\opp\to\mathrm{Set}$ is a sheaf in the first notion. Then, to show that we have an equalizer diagram, we first show that the following diagram commutes.
% https://q.uiver.app/?q=WzAsMyxbMCwwLCJcXG1hdGhjYWwgRihVKSJdLFsxLDAsIlxcZGlzcGxheXN0eWxlXFxwcm9kX3tcXGFscGhhXFxpblxcbGFtYmRhfVxcbWF0aGNhbCBGKFVfXFxhbHBoYSkiXSxbMiwwLCJcXGRpc3BsYXlzdHlsZVxccHJvZF97XFxhbHBoYVxcaW5cXGxhbWJkYX1cXHByb2Rfe1xcYmV0YVxcaW5cXGxhbWJkYX1cXG1hdGhjYWwgRihVX1xcYWxwaGFcXGNhcCBVX1xcYmV0YSkiXSxbMCwxXSxbMSwyLCJcXG9we3Jlc31cXGFscGhhIiwwLHsib2Zmc2V0IjotMn1dLFsxLDIsIlxcb3B7cmVzfVxcYmV0YSIsMCx7Im9mZnNldCI6Mn1dXQ==
\[\begin{tikzcd}
	{\mathcal F(U)} & {\displaystyle\prod_{\alpha\in\lambda}\mathcal F(U_\alpha)} & {\displaystyle\prod_{\alpha\in\lambda}\prod_{\beta\in\lambda}\mathcal F(U_\alpha\cap U_\beta)}
	\arrow[from=1-1, to=1-2]
	\arrow["{\op{res}\alpha}", shift left=2, from=1-2, to=1-3]
	\arrow["{\op{res}\beta}"', shift right=2, from=1-2, to=1-3]
\end{tikzcd}\]
In the case of functions, pick up some $f\in\mathcal F(U)$, and which goes to
\[\{f|_{U_\alpha}\}_{\alpha\in\lambda}\]
in the middle. Then tracking through $\op{res}\alpha$ and $\op{res}\beta$, we are asking for
\[f|_{U_\alpha}|_{U_\alpha\cap U_\beta}=f|_{U_\beta}|_{U_\alpha\cap U_\beta}.\]
However, this is true by functoriality. This kind of argument holds more generally, even if we are not talking about a specific function $f\in\mathcal F(U)$.

We now show that we have an equalizer diagram.
% https://q.uiver.app/?q=WzAsNCxbMCwxLCJcXG1hdGhjYWwgRihVKSJdLFsxLDEsIlxcZGlzcGxheXN0eWxlXFxwcm9kX3tcXGFscGhhXFxpblxcbGFtYmRhfVxcbWF0aGNhbCBGKFVfXFxhbHBoYSkiXSxbMiwxLCJcXGRpc3BsYXlzdHlsZVxccHJvZF97XFxhbHBoYVxcaW5cXGxhbWJkYX1cXHByb2Rfe1xcYmV0YVxcaW5cXGxhbWJkYX1cXG1hdGhjYWwgRihVX1xcYWxwaGFcXGNhcCBVX1xcYmV0YSkiXSxbMCwwLCJYIl0sWzAsMSwiXFxpb3RhIl0sWzEsMiwiXFxvcHtyZXN9XFxhbHBoYSIsMCx7Im9mZnNldCI6LTJ9XSxbMSwyLCJcXG9we3Jlc31cXGJldGEiLDAseyJvZmZzZXQiOjJ9XSxbMywxXSxbMywwLCIiLDEseyJzdHlsZSI6eyJib2R5Ijp7Im5hbWUiOiJkYXNoZWQifX19XV0=
\[\begin{tikzcd}
	X \\
	{\mathcal F(U)} & {\displaystyle\prod_{\alpha\in\lambda}\mathcal F(U_\alpha)} & {\displaystyle\prod_{\alpha\in\lambda}\prod_{\beta\in\lambda}\mathcal F(U_\alpha\cap U_\beta)}
	\arrow["\iota", from=2-1, to=2-2]
	\arrow["{\op{res}\alpha}", shift left=2, from=2-2, to=2-3]
	\arrow["{\op{res}\beta}"', shift right=2, from=2-2, to=2-3]
	\arrow[from=1-1, to=2-2]
	\arrow[dashed, from=1-1, to=2-1]
\end{tikzcd}\]
In particular, we are more or less granted an infinite tuple of morphisms $\{f_\alpha\}_{\alpha\in\lambda}$ from $f_\alpha:X\to\mathcal F(U_\alpha)$. The gluing axiom, now, looks like the existence of the induced map above. In particular, requiring that $f_\alpha|_{U_\alpha\cap U_\beta}=f_\beta|_{U_\alpha\cap U_\beta}$, which is the hypothesis for gluing, and then we get a full $f:X\to\mathcal F(U)$ such that $\iota(f)=\{f|_{U_\alpha}\}_{\alpha\in\lambda}$ restricts as $f_\alpha=f|_{U_\alpha}$.

The locality axiom is the uniqueness in the equalizer diagram. Namely, if we have two such morphisms $f,g:X\to\mathcal F(U)$ such that $\iota f=\iota g$, then the uniqueness of the diagram forces $f=g$. This translates precisely into saying that $f|_{U_\alpha}=g|_{U_\alpha}$ for any $\alpha$ and $\beta$ forces $f=g$, which is what locality wants.