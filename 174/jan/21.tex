% !TEX root = ../notes.tex

By the way, this course is being run by Bryce (interested in category theory, homological algebra, and algebraic topology) and Chris (interested in representation theory and category theory).

\subsection{Small Correction}
Last class we discussed trying to a total order $(\mathcal P,\le)$ into an indiscrete category. One way to do this is to say to give a morphism between two objects $a,b\in\mathcal P$ if and only if one of $a<b$ or $b<a$ or $a=a$ is true. Observe that the order does not actually matter here because any two objects have exactly one morphism anyways.

\subsection{Groupoids}
Reportedly, there will usually not be a lecture to begin out our discussion sections, but here is a lecture to begin out our first discussion section.

Last time we left off talking about indiscrete categories. Here is a nice fact.
\begin{proposition}
	Fix $\mathcal C$ an indiscrete category. Then all maps are isomorphisms.
\end{proposition}
\begin{proof}
	Fix any morphism $f:x\to y$. There is also a morphism $g:y\to x$, and we see that $gf\in\op{Mor}(x,x)$. But $\id_x\in\op{Mor}(x,x)$ as well, so we are forced to have $gf=\id_x$ by uniqueness of morphisms. Similar shows that $fg=\id_y$, finishing the proof.
\end{proof}
\begin{remark}
	This statement is also true for discrete categories but only because all identity morphisms are isomorphisms immediately.
\end{remark}
The property of the proposition is nice enough to deserve a definition.
\begin{definition}[Groupoid]
	A category in which all morphisms are isomorphisms is called a \textit{groupoid}.
\end{definition}
\begin{example}
	Viewing groups as one-element categories, we see that groups are groupoids because all elements (i.e., morphisms of the one-object set) have inverses and hence are isomorphisms.
\end{example}
Intuitively, a groupoid is a group but more ``spread out.''

\subsection{Arrow Words}
We close out with some miscellaneous definitions for our morphisms.
\begin{definition}[Endo-, automorphism]
	Fix a category $\mathcal C$. A morphism $f:x\to y$ is an \textit{endomorphism} if and only if $x=y$. A morphism $f:x\to y$ is an \textit{automorphism} if and only if it is an isomorphism and an endomorphism.
\end{definition}
\begin{example}
	In the category of abelian groups, the map $\ZZ\to\ZZ$ given by multiplication by $2$ is an endomorphism but not an automorphism.
\end{example}
\begin{definition}[Monic, epic]
	Fix a category $\mathcal C$ and a morphism $f:x\to y$.
	\begin{itemize}
		\item We say $f$ is a \textit{monomorphism} (or is \textit{monic}) if and only if $fg=fh$ implies $g=h$ for any morphisms $g,h:c\to x$. In other words, the map
		\[\op{Mor}(c,x)\stackrel{f\circ-}\to\op{Mor}(c,y)\]
		is injective. (This map is called ``post-composition.'') We might write $f:x\into y$ for emphasis.
		\item We say $f$ is an \textit{epimorphism} (or is \textit{epic}) if and only if $gf=hf$ implies $g=h$ for any morphisms $g,h:y\to c$. In other words, the map
		\[\op{Mor}(y,c)\stackrel{-\circ f}\to\op{Mor}(x,c)\]
		is injective. (This map is called ``pre-composition.'') We might write $f:x\onto y$ for emphasis.
	\end{itemize}
\end{definition}
Intuitively, the monomorphism condition looks like the injectivity condition (namely, $f(x)=f(y)$ implies $x=y$), so monic is supposed to be a generalization for injective.
\begin{example}
	In the category of sets, monic is equivalent to injective, and epic is equivalent to surjective. Then it happens that being monic and epic implies being isomorphic. We will not fill in the details here.
\end{example}
\begin{warn}
	It is not always true that being monic and epic implies being isomorphic. It is true in $\mathrm{Set},\mathrm{Ab},\mathrm{Grp}$ but not in, say, $\mathrm{Ring}$ as the below example shows.
\end{warn}
\begin{ex}
	The inclusion $f:\ZZ\into\QQ$ in $\mathrm{Ring}$ is both epic and monic but not an isomorphism. We run some checks.
	\begin{itemize}
		\item We show monic. Suppose $g,h:R\to\ZZ$ are morphisms with $fg=fh.$ We claim $g=h$. Well, for any $r\in R$, we see $g(r)=f(g(r))$ and $h(r)=f(h(r))$ because $f$ is merely an inclusion, so $g(r)=h(r)$ follows.
		\item We show epic. Suppose $g,h:\QQ\to R$ are morphisms with $gf=hf$. We claim $g=h$. We start by noting any $m\in\ZZ\setminus\{0\}$ and $n\in\ZZ$ will have
		\[g\left(n/m\right)\cdot g(m)=g(n)\]
		and similar for $h$. However, $g(m)=g(f(m))=h(f(m))=h(m)$ and $g(n)=h(n)$ for the same reason, so $g\left(\frac nm\right)=g(n)/g(m)=h(n)/h(m)=h\left(\frac nm\right)$, and we are done because any rational can be expressed as some $\frac nm$.
		\item Lastly, $f$ is not an isomorphism because $\ZZ$ and $\QQ$ are not isomorphic. For example, $2x-1$ has a solution in $\QQ$ but not in $\ZZ$.
	\end{itemize}
\end{ex}
And now discussion begins.