% !TEX root = ../notes.tex








Reportedly there is a lot of material that Bryce would like to cover today.

\subsection{Our Definition}
We're doing category theory, so let's define what a category is.
\begin{definition}[Category]
	A category $\mathcal C$ is a pair of objects and morphisms $(\op{Ob}\mathcal C,\op{Mor}\mathcal C)$ satisfying the following.
	\begin{itemize}
		\item $\op{Ob}\mathcal C$ is a collection of \textit{objects}. By abuse of notation, when we write $c\in\mathcal C$ 
		\item $\op{Mor}\mathcal C$ is a collection of \textit{morphisms}. Morphisms might also be called arrows or maps or functions or continuous functions or similar.

		A morphism is written $f:x\to y$ where $x,y\in\op{Ob}\mathcal C$. Here, $c$ is the \textit{domain}, and $d$ is the \textit{codomain}.
	\end{itemize}
	These morphisms have a little extra structure.
	\begin{itemize}
		\item For each $x\in\mathcal C$, there is a morphism $\id_x:x\to x$.
		\item Given any pair of morphisms $f:x\to y$ and $g:y\to z$, there exists a \textit{composition} $gf:x\to z$. Importantly, the codomain of $f$ is the domain of $g$.
	\end{itemize}
	Additionally, morphisms satisfy the following coherence conditions.
	\begin{itemize}
		\item Associativity: for any morphisms $f:a\to b$ and $g:b\to c$ and $h:c\to d$, we have that $h(gf)=(hg)f$.
		\item Identity: given any morphism $f:a\to b$, we have $\id_bf=f$ and $f\id_a=f$.
	\end{itemize}
\end{definition}
\noindent Yes, this is a long definition. For reference, it is on page 3 of Riehl.

The intuition to have here is that we have objects to be thought of as points a whole bunch of morphisms which are to be thought of arrows between them. Here is an example of some morphisms in a category.
% https://tikzcd.yichuanshen.de/#N4Igdg9gJgpgziAXAbVABwnAlgFyxMJZABgBpiBdUkANwEMAbAVxiRAB12AjJhhmHCAC+pdJlz5CKAIzkqtRizacefAcNEgM2PASJlp8+s1aIO3Xv0EixOyUQBMpQ9WNKzKy+qHyYUAObwRKAAZgBOEAC2SGQgOBBIDjYg4VGJ1PFIAMzJqdGIsZmI0rkR+bJxCYhJmnlIFUU5tWUxGVWxDBAQaI4A7GQhjHAw8gx0XDAMAAriulIgYVj+ABbWzWnFbfXUnd1E0gAcA0MjO+OTM3Z6Zosra6EtiFlbTztdPSjHDMOj59Oz9huS1WGgeGyclXSIF2HxIpEG31O0L+lwk1wWwOsFCEQA
\[\begin{tikzcd}
	\bullet \arrow[d] \arrow[r] \arrow[loop, distance=2em, in=215, out=145] & \bullet \arrow[ld] \arrow[rd] \arrow[loop, distance=2em, in=125, out=55] &                                                     \\
	\bullet \arrow[rr] \arrow[loop, distance=2em, in=305, out=235]          &                                                                          & \bullet \arrow[loop, distance=2em, in=305, out=235]
\end{tikzcd}\]
The loops are identity morphisms. As an aside, it is reasonable to think that definition of a category is overly abstract. Most of the time we will be thinking about some concrete category.

Before continuing, we bring in the following definition.
\begin{definition}[Hom-sets]
	Fix a category $\mathcal C$. Then, given objects $x,y\in\mathcal C$, we write $\mathcal C(x,y)$ or $\op{Hom}_\mathcal C(x,y)$ or $\op{Hom}(x,y)$ or $\op{Mor}(x,y)$ for the set of morphisms $f:x\to y$. I personally prefer $\op{Mor}(x,y)$.
\end{definition}
\noindent Note that two objects need not have a morphism between them. For example, the following is a category even though the two objects have a morphism between them.
% https://tikzcd.yichuanshen.de/#N4Igdg9gJgpgziAXAbVABwnAlgFyxMJZABgBpiBdUkANwEMAbAVxiRAB12AjJhhmHCAC+pdJlz5CKAIzkqtRizacefAcPkwoAc3hFQAMwBOEALZIyIHBAvUGECGiIAmAOxkDjODHkM6XGAYABXE8AjYjLG0AC0ERQxNzRFkrG2S7BycUAE4PLx87f0CQ7DCpEEiYuIohIA
\[\begin{tikzcd}
	\bullet \arrow[loop, distance=2em, in=215, out=145] & \bullet \arrow[loop, distance=2em, in=35, out=325]
\end{tikzcd}\]
As a less contrived example, there is no morphism between $\FF_2$ and $\FF_3$ in the category of fields.

\subsection{Examples}
Let's talk about examples.
\begin{example}
	The category $\mathrm{Set}$ has objects which are all sets and its morphisms are the functions between sets.
\end{example}
\begin{example}
	The category $\mathrm{Grp}$ has objects which are all groups and its morphisms are group homomorphisms. Similarly, $\mathrm{Ab}$ has abelian groups.
\end{example}
\begin{example}
	The category $\mathrm{Ring}$ has objects which are all rings (with identity) and its morphisms are group homomorphisms.
\end{example}
\begin{example}
	The category $\mathrm{Field}$ has objects which are all fields and its morphisms are field/ring homomorphisms.
\end{example}
\begin{ex}
	The category $\mathrm{Vec}_k$ has objects which are all $k$-vector spaces and its morphisms are $k$-linear transformations.
\end{ex}
Those are the good examples. We like them because they are with familiar objects.

Here are some weirder examples.
\begin{example}[Walking arrow]
	The diagram
	% https://tikzcd.yichuanshen.de/#N4Igdg9gJgpgziAXAbVABwnAlgFyxMJZABgBpiBdUkANwEMAbAVxiRAB12AjJhhmHCAC+pdJlz5CKAIzkqtRizacefAcPkwoAc3hFQAMwBOEALZIyIHBCTShFIUA
	\[\begin{tikzcd}
		\bullet \arrow[r] & \bullet
	\end{tikzcd}\]
	indues a category with a single non-identity morphism.
\end{example}
Note that we will stop writing down all of the identity morphisms and all induced morphisms because they're annoying to writ eout.
\begin{example}[Walking isomorphism]
	The diagram
	% https://q.uiver.app/?q=WzAsMixbMCwwLCJcXGJ1bGxldCJdLFsxLDAsIlxcYnVsbGV0Il0sWzAsMSwiIiwwLHsiY3VydmUiOi0xfV0sWzEsMCwiIiwwLHsiY3VydmUiOi0xfV1d
	\[\begin{tikzcd}
		\bullet & \bullet
		\arrow[curve={height=-6pt}, from=1-1, to=1-2]
		\arrow[curve={height=-6pt}, from=1-2, to=1-1]
	\end{tikzcd}\]
	induces a category with two non-identity morphisms. We declare that any composition of the two non-identity morphisms is the identity.
\end{example}
There are also such things as a poset category, but for this we should define a poset first.
\begin{defi}[Poset]
	A \textit{poset} $(\mathcal P,\le)$ is a set $\mathcal P$ and a relation $\le$ on $\mathcal P$ which satisfies the following; let $a,b,c\in\mathcal P$.
	\begin{itemize}
		\item Reflexive: $a\le a$.
		\item Antisymmetric: $a\le b$ and $b\le a$ implies $a=b$.
		\item Transitive: $a\le b$ and $b\le c$ implies $a\le c$.
	\end{itemize}
\end{defi}
\noindent Now, it turns out that all posets induce a category.
\begin{example}[Poset category]
	Given any poset $(\mathcal P,\le)$, we can define the poset category as follows.
	\begin{itemize}
		\item The objects are elements of $\mathcal P$.
		\item For $x,y\in\mathcal P$, there is a morphism $x\to y$ if and only if $x\le y$, and there is only one morphism.
	\end{itemize}
\end{example}
Checking that the poset category is in fact a category is not very interesting. The identity law comes from reflexivity, where $\id_a$ witnesses $a\le a$.

Additionally, transitivity defines our composition: if $a\le b$ and $b\le c$, then $a\le c$, and the morphism representing $a\le c$ is unambiguous because there is at most one moprhism $a\to c$. This uniqueness is in fact crucial for our composition: if $f:a\to b$ and $g:b\to c$ and $h:c\to d$ are morphisms, then $h(gf)=(hg)f$ because they are both morphisms $a\to d$, of which there is at most one.

We continue with our examples. We will not check that these are actually categories formally; perhaps the reader can do the checks on their own time.
\begin{example}[Groups]
	Given a group $G$, we can define the category $bG$ to have one object $*$ and morphisms $g:*\to*$ given by group elements $g\in G$. Composition in the category is group multiplication; the identity morphism $\id_*$ needed is the identity element of $G$; and the associativity check comes from associativity in $G$.
\end{example}
\begin{example}[Pointer sets]
	We define the category of pointed sets $\mathrm{Set}_*$ to consist of objects which are orderd pairs $(X,x)$ where $X$ is a set and $x\in X$ is an element. Then morphism are ``based maps'' $f:(X,x)\to(Y,y)$ to consist of the data of a function $f:X\to Y$ such that $f(x)=y$.
\end{example}
\begin{example}
	Given any set $S$, we can define a category consisting of objects which are elements of $S$ and morphisms which are only the required identity morphisms.
\end{example}
This last example generalizes.
\begin{definition}[Discrete, indiscrete]
	Fix a category $\mathcal C$. Then $\mathcal C$ is \textit{discrete} if and only if the only morphisms are identity morphisms. Additionally, $\mathcal C$ is \textit{indiscrete} if and only if $\op{Mor}(x,y)$ has exactly one element for each pair of objects $(x,y)$.
\end{definition}
\begin{warn}
	A total order with more than one element is not a category. Namely, if we have distinct objects $x$ and $y$, then we cannot have both $x\le y$ and $y\le x$, so not both $\op{Mor}(x,y)$ and $\op{Mor}(y,x)$ inhabited.
\end{warn}

\subsection{Size Issues}
Let's briefly talk about why we are calling $\op{Ob}\mathcal C$ and $\op{Mor}\mathcal C$ ``collections.'' In short, we cannot have a set that contains all sets, but we would still like a category which contiains all categories. There are a few ways around this; here are two.
\begin{itemize}
	\item Grothendeik inaccessible categories: we essentially upper-bound the size of our sets and then let $\mathrm{Set}$ contain all of our sets.
	\item Proper classes: we add in things called ``classes'' to foundational mathematics we are allowed to be bigger than sets.
\end{itemize}
We will avoid doing anything like this in this course, so here is a definition making this concrete.
\begin{definition}[Small, locally small]
	Fix $\mathcal C$ a category. Then $\mathcal C$ is \textit{small} if and only if $\op{Mor}\mathcal C$. Alternatively, $\mathcal C$ is \textit{locally small} if and only if $\op{Mpr}(x,y)$ is a set.
\end{definition}
\begin{example}
	The category $\mathrm{Set}$ is locally small, but it is not small. To see that it is not small, note that there is an injective map from sets $S$ to morphisms $\{1\}\to S$.
\end{example}
It turns out that most of our categories will be locally small. It is a very nice property to have.

\subsection{Isomorphism}
In algebra (e.g., group theory), we are interested in when two objects are the same. In category theory, we focus on the morphisms between objects, so we need to be careful how we define this. Here is our definition.
\begin{definition}[Isomorphism]
	Fix a category $\mathcal C$. Then a morphism $f:x\to y$ is an \textit{isomorphism} if and only if there is a morphism $g:y\to x$ such that $fg=\id_y$ and $gf=\id_x$. We call $g$ the \textit{inverse} of $f$ and often notate it $f^{-1}$.
\end{definition}
\noindent This is fairly intuitive: isomorphisms are those morphisms with a way to reverse them.

Observe that we called $g$ ``the'' inverse of $f$, and we may do so because inverses are unique.
\begin{proposition}
	Fix a category $\mathcal C$. Inverses of morphisms, if they exist, are unique.
\end{proposition}
\begin{proof}
	Fix $f:x\to y$ some isomorphism, and suppose that we have found two inverse morphisms $g,h:y\to x$. Then
	\[g=g\id_y=g(fh)=(gf)h=\id_xh=h,\]
	so indeed the inverse morphisms that we found are the same.
\end{proof}
Anyways, here are some examples.
\begin{example}
	In $\mathrm{Set}$, the isomorphisms are the bijective maps. For this we would have to show that bijective maps have inverse maps, which is not too hard to show.
\end{example}
\begin{example}
	In $\mathrm{Grp}$, the isomorphisms are group isomorphisms. Similarly, isomorphisms in $\mathrm{Ring}$ are ring isomorphisms.
\end{example}
As a warning, we will say now that lots of categories do not have a good categorial notion of injectivity or surjectivity, so we will not be able to say that isomorphisms are merely ``bijective'' morphisms.