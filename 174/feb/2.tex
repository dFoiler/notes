% !TEX root = ../notes.tex

Today we are talking about product categories and the \textrm{Hom} bifunctor.

\subsection{\textrm{Hom} Bifunctor}
Here is our definition.
\begin{definition}[Product category]
	Fix categories $\mathcal C$ and $\mathcal D$. Then we define the \textit{product category} $\mathcal C\times\mathcal D$ as follows.
	\begin{itemize}
		\item We define $\op{Ob}\mathcal C\times\mathcal D$ to be the collection of ordered pairs $(c,d)$ with $c\in\mathcal C$ and $d\in\mathcal D$.
		\item We define $\op{Mor}((c,d),(c',d'))$ to be the collection of ordered pairs $(f,g)$ with $f:c\to c'$ a morphism in $\mathcal C$ and $g:d\to d'$ a morphism in $\mathcal D$.
	\end{itemize}
	Lastly, we define identity to be the identity on each object and composition by composition component-wise.
\end{definition}
From yesterday, we have the following functors.
\begin{definition}[Functors represented by objects]
	Fix $\mathcal C$ a locally small category and $x\in\mathcal C$ an object. Then we have the functors
	\[\op{Mor}_\mathcal C(x,-):\mathcal C\to\mathrm{Set}\qquad\text{and}\qquad\op{Mor}_\mathcal C(-,x):\mathcal C^\mathrm{op}\to\mathrm{Set}.\]
	The former functor is the \textit{covariant functor represented by $x$}, and the latter is the \textit{contravariant functor represented by $x$}.
\end{definition}
We would like to codify the structure that having two functors gives us, so we have the following definition.
\begin{defi}[Bifunctor]
	A \textit{bifunctor} is a functor whose domain is a product of categories.
\end{defi}
In particular, here is our standard example.
\begin{defihelper}[\textrm{Hom} bifunntor] \nirindex{Hom bifunctor@\textrm{Hom} bifunctor}
	Fix $\mathcal C$ a locally small category. Then \textit{\textrm{Hom} bifunctor} is the functor given by the functors representing a particular object $x\in\mathcal C$. Namely, we have
	\[\op{Mor}_\mathcal C(-,-):\mathcal C^\mathrm{op}\times\mathcal C\to\mathrm{Set}\]
	by taking $(x,y)\mapsto\op{Mor}_\mathcal C(x,y)$.
\end{defihelper}
We will not check that this is actually a functor, but it is.

\subsection{Category Isomorphism}
We would like a notion of two categories being the same, but this is somewhat subtle. Here is a first approximation.
\begin{definition}[Isomorphism]
	A functor $F:\mathcal C\to\mathcal D$ is an \textit{isomorphism of categories} if and only if there is an inverse functor $G:\mathcal D\to\mathcal C$ so that $GF=\id_\mathcal C$ and $FG=\id_\mathcal D$. In this case we say that $\mathcal C$ and $\mathcal D$ are \textit{isomorphic}.
\end{definition}
\begin{remark}
	As usual, isomorphisms are unique and whatnot.
\end{remark}
Let's make this definition a little more concrete.
\begin{proposition} \label{prop:isoisobjectbijective}
	An isomorphism $F:\mathcal C\to\mathcal D$ descends to a bijective (i.e., injective and surjective) map $\op{Ob}\mathcal C\to\op{Ob}\mathcal D$.
\end{proposition}
\begin{remark}
	We are attempting to care about set-theoretic issues in our phrasing because Bryce cares about set-theoretic issues.
\end{remark}
\begin{proof}[Proof of \autoref{prop:isoisobjectbijective}]
	Let $G$ be the inverse morphism for $F$. Then we claim that the induced map $G:\op{Ob}\mathcal D\to\op{Ob}\mathcal C$ will be the inverse for the induced map for $F$. This is clear because $GF=\id_\mathcal C$ and $FG=\id_\mathcal D$.
\end{proof}
It turns out that isomorphisms are a little too strong: there are categories we want to be the same but are not actually isomorphic.
\begin{example}
	The category
	% https://q.uiver.app/?q=WzAsMixbMCwwLCJcXGJ1bGxldCJdLFsxLDAsIlxcYnVsbGV0Il0sWzAsMSwiIiwwLHsiY3VydmUiOi0xfV0sWzEsMCwiIiwwLHsiY3VydmUiOi0xfV1d
	\[\begin{tikzcd}
		\bullet & \bullet
		\arrow[curve={height=-6pt}, from=1-1, to=1-2]
		\arrow[curve={height=-6pt}, from=1-2, to=1-1]
	\end{tikzcd}\]
	is not isomorphic to
	% https://q.uiver.app/?q=WzAsMSxbMCwwLCJcXGJ1bGxldCJdXQ==
	\[\begin{tikzcd}
		\bullet
	\end{tikzcd}\]
	because there are a different number of objects, so there is no bijection.
\end{example}

\subsection{Natural Transformation}
To salvage our notion of categorical isomorphism, we need a notion of naturality. Naturality is more of something that we can feel as mathematicians rather than something we like to formalize.
\begin{example}
	Any two trivial groups have a canonical isomorphism between them. In fact, there is only one homomorphism at all.
\end{example}
\begin{nex}
	There is no ``natural'' or ``canonical'' isomorphism $\ZZ/3\ZZ\to A_3$, though the groups are isomorphic.
\end{nex}
\begin{nex}
	Given a two-dimensional $\RR$-vector space named $V$, there is no canonical isomorphism $\RR^2\to V$.
\end{nex}
We would like maps to preserve all the structure we could want. So here is our notion of naturality for functors.
\begin{definition}[Natural transformation]
	Fix functors $F,G:\mathcal C\to\mathcal D$. A \textit{natural transformation} $\eta:F\Rightarrow G$ consists of the data of a morphism $\eta_X:Fc\to Gc$ for each $c\in\mathcal C$ such that the following diagram always commutes for any morphism $f:c\to c'$ in $\mathcal C$.
	% https://q.uiver.app/?q=WzAsNCxbMCwwLCJGYyJdLFswLDEsIkZjJyJdLFsxLDAsIkdjIl0sWzEsMSwiR2MnIl0sWzAsMSwiRmYiLDJdLFsyLDMsIkdmIl0sWzAsMiwiXFxldGFfYyJdLFsxLDMsIlxcZXRhX3tjJ30iLDJdXQ==
	\[\begin{tikzcd}
		Fc & Gc \\
		{Fc'} & {Gc'}
		\arrow["Ff"', from=1-1, to=2-1]
		\arrow["Gf", from=1-2, to=2-2]
		\arrow["{\eta_c}", from=1-1, to=1-2]
		\arrow["{\eta_{c'}}"', from=2-1, to=2-2]
	\end{tikzcd}\]
	The maps $\varphi_c$ are called the \textit{components} of $\varphi$.
\end{definition}
\begin{quot}
	Burn this square into your minds. It is the most important square in this class.
\end{quot}
As usual, we start with examples.
\begin{exe}
	We work in $\mathrm{Vec}_k$. Then we consider the functor $-^{**}:\mathrm{Vec}_k\to\mathrm{Vec}_k$ by $V\mapsto V^{**}$. Then we claim that there is a natural transformation from $-^{**}$ to $\id$, using the natural transformation
	\[\op{ev}_V:V\to V^{**}\]
	by $\op{ev}_V(x):=(\lambda\in V^*\mapsto\lambda x)$.
\end{exe}
\begin{proof}
	We need to check that the following diagram commutes.
	% https://q.uiver.app/?q=WzAsNCxbMCwwLCJWIl0sWzAsMSwiVyJdLFsxLDAsIlZeeyoqfSJdLFsxLDEsIldeeyoqfSJdLFswLDEsImYiLDJdLFsyLDMsImZeeyoqfSJdLFswLDIsIlxcb3B7ZXZ9X1YiXSxbMSwzLCJcXG9we2V2fV9XIiwyXV0=
	\[\begin{tikzcd}
		V & {V^{**}} \\
		W & {W^{**}}
		\arrow["f"', from=1-1, to=2-1]
		\arrow["{f^{**}}", from=1-2, to=2-2]
		\arrow["{\op{ev}_V}", from=1-1, to=1-2]
		\arrow["{\op{ev}_W}"', from=2-1, to=2-2]
	\end{tikzcd}\]
	Very quickly, we recall that $f^{**}:V^{**}\to W^{**}$ is by
	\[f(\varphi)=(\lambda\in W^*\mapsto\varphi(\lambda f)).\]
	Namely, $\lambda:W\to k$, so $\lambda f:V\to k$ lives in $V^*$, so $\varphi(\lambda f)\in k$.

	Now we check the commutativity of the square. Fix some $x\in V$ and a linear functional $\lambda:W\to k$. Then we can carefully compute, after many tears and groans, that
	\[f^{**}(\op{ev}_V(x))(\lambda)=\op{ev}_V(x)(\lambda f)=\lambda f(x)=\op{ev}_W(f(x))(\lambda).\]
	Because $\lambda$ was arbitrary, we see that $f^{**}\op{ev}_V(\lambda)=\op{ev}_Wf(x)$, which then gives us $f^{**}\op{ev}_V=\op{ev}_Wf$.
\end{proof}
We have the following definition.
\begin{definition}[Natural isomorphism]
	A natural transformation $\eta:F\to C$ is a \textit{natural isomorphism} if and only if its component morphisms are isomorphisms.
\end{definition}
\begin{example}
	In $\mathrm{finVec}_k$, the above $\op{ev}$ is a natural isomorphism because $\op{ev}_V:V\Rightarrow V^{**}$ is an isomorphism when $V$ is finite-dimensional.
\end{example}

Here is a quick proposition.
\begin{proposition}
	Let $\varphi:F\Rightarrow G$ be a natural isomorphism. Then the inverse morphisms $\psi_c:=\varphi_c^{-1}$ assemble to make a natural transformation $\psi:G\Rightarrow F$.
\end{proposition}
\begin{proof}
	We will be brief. Given a morphism $f:x\to y$, we need to check that the following diagram commutes.
	% https://q.uiver.app/?q=WzAsNCxbMCwwLCJHeCJdLFswLDEsIkd5Il0sWzEsMCwiRngiXSxbMSwxLCJGeSJdLFswLDEsIkdmIiwyXSxbMiwzLCJGZiJdLFswLDIsIlxccHNpX3giXSxbMSwzLCJcXHBzaV95IiwyXV0=
	\[\begin{tikzcd}
		Gx & Fx \\
		Gy & Fy
		\arrow["Gf"', from=1-1, to=2-1]
		\arrow["Ff", from=1-2, to=2-2]
		\arrow["{\psi_x}", from=1-1, to=1-2]
		\arrow["{\psi_y}"', from=2-1, to=2-2]
	\end{tikzcd}\]
	In other words, we need to know that $\psi_yFf=Gf\psi_x$. Well, we already know that
	\[\varphi_yFf=Gf\varphi_x\]
	by naturality, so
	\[Ff\psi_x=\psi_y\varphi_yFf\psi_x=\psi_yGf\varphi_x\psi_x=\psi_yGf\]
	after checking through.
\end{proof}