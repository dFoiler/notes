% !TEX root = ../notes.tex

\subsection{Examples of Natural Transformations}
We're talking about more natural transformations today. For our first example, consider the covariant power set functor $\mathcal P:\mathrm{Set}\to\mathrm{Set}$ by $S\mapsto\mathcal P(S)$ and $f:S\to T$ to $\mathcal P(f)(U):=f(U)$ for $U\subseteq S$.
\begin{exe}
	 We define a natural transformation $\eta_\bullet:\id_{\mathrm{Set}}\Rightarrow\mathcal P$ a function $\eta_S:S\to\mathcal P(S)$ by
	\[\eta_S(x):=\{x\}\]
\end{exe}
\begin{proof}
	Fix $f:S\to T$ a morphism in $\mathrm{Set}$. After plugging everything in, we need the following diagram to commute.
	% https://q.uiver.app/?q=WzAsNCxbMCwwLCJTIl0sWzEsMCwiVCJdLFswLDEsIlxcbWF0aGNhbCBQKFMpIl0sWzEsMSwiXFxtYXRoY2FsIFAoVCkiXSxbMiwzLCJcXG1hdGhjYWwgUChmKSIsMl0sWzAsMiwiXFxldGFfUyIsMl0sWzEsMywiXFxldGFfVCJdLFswLDEsImYiXV0=
	\[\begin{tikzcd}
		S & T \\
		{\mathcal P(S)} & {\mathcal P(T)}
		\arrow["{\mathcal P(f)}"', from=2-1, to=2-2]
		\arrow["{\eta_S}"', from=1-1, to=2-1]
		\arrow["{\eta_T}", from=1-2, to=2-2]
		\arrow["f", from=1-1, to=1-2]
	\end{tikzcd}\]
	To see this commutes, fix some $x\in S$, and we run it through the diagram as follows.
	% https://q.uiver.app/?q=WzAsNCxbMCwwLCJ4Il0sWzEsMCwiZih4KSJdLFswLDEsIlxce3hcXH0iXSxbMSwxLCJcXHtmKHgpXFx9Il0sWzAsMiwiXFxldGFfUyIsMix7InN0eWxlIjp7InRhaWwiOnsibmFtZSI6Im1hcHMgdG8ifX19XSxbMSwzLCJcXGV0YV9UIiwwLHsic3R5bGUiOnsidGFpbCI6eyJuYW1lIjoibWFwcyB0byJ9fX1dLFswLDEsImYiLDAseyJzdHlsZSI6eyJ0YWlsIjp7Im5hbWUiOiJtYXBzIHRvIn19fV0sWzIsMywiXFxtYXRoY2FsIFAoZikiLDIseyJzdHlsZSI6eyJ0YWlsIjp7Im5hbWUiOiJtYXBzIHRvIn19fV1d
	\[\begin{tikzcd}
		x & {f(x)} \\
		{\{x\}} & {\{f(x)\}}
		\arrow["{\eta_S}"', maps to, from=1-1, to=2-1]
		\arrow["{\eta_T}", maps to, from=1-2, to=2-2]
		\arrow["f", maps to, from=1-1, to=1-2]
		\arrow["{\mathcal P(f)}"', maps to, from=2-1, to=2-2]
	\end{tikzcd}\]
	So indeed, the diagram does commute.
\end{proof}
\begin{remark}
	We may call the second diagram an ``internal'' diagram because it is looking internally at our objects.
\end{remark}
For our next example, recall we defined a functorial $G$-action on some object $c\in\mathcal C$ by a functor $F:\mathrm BG\to\mathcal C$. Our goal is to define a $G$-equivariant map between objects.
\begin{exe}
	We track the data between two $G$-representations $F,G:\mathrm BG\to\mathrm{Vec}_k$ by a natural transformation $\eta_\bullet:\mathrm{Vec}_k\Rightarrow\mathrm{Vec}_k$.
\end{exe}
\begin{proof}
	Because $\mathrm BG$ has only one object $*$, we set $V:=F(*)$ and $W:=G(*)$ and need to check the commutativity of the following diagram, for some $g:*\to *$ in $G$.
	% https://q.uiver.app/?q=WzAsNCxbMCwwLCJWIl0sWzEsMCwiViJdLFswLDEsIlciXSxbMSwxLCJXIl0sWzAsMiwiXFxldGFfKiIsMl0sWzEsMywiXFxldGFfKiJdLFswLDEsIkZnIl0sWzIsMywiRnciLDJdXQ==
	\[\begin{tikzcd}
		V & V \\
		W & W
		\arrow["{\eta_*}"', from=1-1, to=2-1]
		\arrow["{\eta_*}", from=1-2, to=2-2]
		\arrow["Fg", from=1-1, to=1-2]
		\arrow["Fw"', from=2-1, to=2-2]
	\end{tikzcd}\]
	Note that the natural transformation $\eta_\bullet$ really only consists of the map $\eta_*$, which is a linear map $V\to W$ which respects the group action: $\eta_*(gv)=g\eta_*(v)$.
\end{proof}
These $G$-equivariant maps can be turned into a category.
\begin{definition}[\texorpdfstring{$\mathrm{Rep}_G$}{RepG}]
	We define the category of \textit{$G$-representations} to be the category consisting of objects which are functors $F:\mathrm BG\to\mathrm{Vec}_k$ and morphisms which are natural transformations between the functors.
\end{definition}
\begin{exe}
	We check that there is a category whose objects are functors $\mathcal C\to\mathcal D$ and whose 
\end{exe}
\begin{proof}
	To define our morphisms, suppose $F,G,H:\mathcal C\to\mathcal D$ with natural transformations $\eta_\bullet:F\Rightarrow G$ and $\nu_\bullet:G\Rightarrow H$. Lastly, we define our composition by
	\[(\nu\eta)_X:=\eta_X\nu_X.\]
	To check that $(\eta\nu)_\bullet:F\Rightarrow H$ is in fact a natural transformation, we have the following ladder.
	% https://q.uiver.app/?q=WzAsNixbMCwwLCJGeCJdLFsxLDAsIkZ5Il0sWzAsMSwiR3giXSxbMSwxLCJHeSJdLFswLDIsIkh4Il0sWzEsMiwiSHkiXSxbMCwyLCJcXGV0YV94IiwyXSxbMiw0LCJcXG51X3giLDJdLFsxLDMsIlxcZXRhX3kiXSxbMyw1LCJcXG51X3kiXSxbMCwxLCJGZiJdLFsyLDMsIkdmIl0sWzQsNSwiSGYiLDJdLFswLDQsIihcXG51XFxldGEpX3giLDIseyJvZmZzZXQiOjIsImN1cnZlIjoyfV0sWzEsNSwiKFxcbnVcXGV0YSlfeSIsMCx7Im9mZnNldCI6LTIsImN1cnZlIjotMn1dXQ==
	\[\begin{tikzcd}
		Fx & Fy \\
		Gx & Gy \\
		Hx & Hy
		\arrow["{\eta_x}"', from=1-1, to=2-1]
		\arrow["{\nu_x}"', from=2-1, to=3-1]
		\arrow["{\eta_y}", from=1-2, to=2-2]
		\arrow["{\nu_y}", from=2-2, to=3-2]
		\arrow["Ff", from=1-1, to=1-2]
		\arrow["Gf", from=2-1, to=2-2]
		\arrow["Hf"', from=3-1, to=3-2]
		\arrow["{(\nu\eta)_x}"', shift right=2, curve={height=12pt}, from=1-1, to=3-1]
		\arrow["{(\nu\eta)_y}", shift left=2, curve={height=-12pt}, from=1-2, to=3-2]
	\end{tikzcd}\]
	Each square commutes, so the $2\times1$ rectangle will also commute. We check associativity by drawing a $3\times1$ rectangle and seeing that it commutes.

	To define our identity maps for our category, we take $(\id_F)_X:=\id_{F(x)}:Fx\to Fx$. We can check that this works with our composition without too many tears.
\end{proof}
\begin{definition}[Functor category]
	The category of the above exercise is the \textit{functor category}, notated $\mathcal D^{\mathcal C}$.
\end{definition}
\begin{example}
	We have that $\mathrm{Rep}_G=\mathrm{Vec}_k^{\mathrm BG}$.
\end{example}

\subsection{Yoneda, Contravariant It Is}
For the discussion that follows, we fix $\mathcal C$ locally small and $f:w\to x$ and $h:y\to z$ some morphisms in $\mathcal C$. From this we get the following square.
% https://q.uiver.app/?q=WzAsNCxbMCwwLCJcXG9we01vcn0oeCx5KSJdLFsxLDAsIlxcb3B7TW9yfSh4LHopIl0sWzAsMSwiXFxvcHtNb3J9KHcseSkiXSxbMSwxLCJcXG9we01vcn0odyx6KSJdLFswLDIsIi1mIiwyXSxbMCwxLCJoLSJdLFsxLDMsIi1mIl0sWzIsMywiaC0iLDJdXQ==
\[\begin{tikzcd}
	{\op{Mor}(x,y)} & {\op{Mor}(x,z)} \\
	{\op{Mor}(w,y)} & {\op{Mor}(w,z)}
	\arrow["{-f}"', from=1-1, to=2-1]
	\arrow["{h-}", from=1-1, to=1-2]
	\arrow["{-f}", from=1-2, to=2-2]
	\arrow["{h-}"', from=2-1, to=2-2]
\end{tikzcd}\]
We can check that this square commutes. Here is the internal square.
% https://q.uiver.app/?q=WzAsNCxbMCwwLCJnIl0sWzEsMCwiaGciXSxbMCwxLCJnZiJdLFsxLDEsImhnZiJdLFswLDIsIi1mIiwyLHsic3R5bGUiOnsidGFpbCI6eyJuYW1lIjoibWFwcyB0byJ9fX1dLFswLDEsImgtIiwwLHsic3R5bGUiOnsidGFpbCI6eyJuYW1lIjoibWFwcyB0byJ9fX1dLFsxLDMsIi1mIiwwLHsic3R5bGUiOnsidGFpbCI6eyJuYW1lIjoibWFwcyB0byJ9fX1dLFsyLDMsImgtIiwyLHsic3R5bGUiOnsidGFpbCI6eyJuYW1lIjoibWFwcyB0byJ9fX1dXQ==
\[\begin{tikzcd}
	g & hg \\
	gf & hgf
	\arrow["{-f}"', maps to, from=1-1, to=2-1]
	\arrow["{h-}", maps to, from=1-1, to=1-2]
	\arrow["{-f}", maps to, from=1-2, to=2-2]
	\arrow["{h-}"', maps to, from=2-1, to=2-2]
\end{tikzcd}\]
Hooray, it commutes. The point is that $h-$ and $-f$ are going to induce natural transformations of our $\mathrm{Mor}$ functors.
\begin{itemize}
	\item The functors $\op{Mor}(x,-),\op{Mor}(w,-):\mathcal C\to\mathrm{Set}$. Then any morphism $f:w\to x$ induces a natural transformation $-f:\op{Mor}(x,-)\Rightarrow\op{Mor}(w,-)$. The naturality check is the commutativity of the above square.
	\item Similarly, the functors $\op{Mor}(-,y),\op{Mor}(-,z):\mathcal C\to\mathrm{Set}$. Then any morphism $h:x\to y$ induces a natural transformation $h-:\op{Mor}(-,y)\Rightarrow\op{Mor}(-,w)$. The naturality check is again the commutativity of the above square.
\end{itemize}
We won't be more explicit about our squares because my head hurts.

\begin{remark}
	Later in life we will talk about the Yoneda embedding, which is essentially about the embedding $\mathcal C^\mathrm{op}\to\mathrm{Set}^\mathcal C$, which takes $x\mapsto\op{Mor}(x,-)$ and $f:x\to y$ to the natural transformation $-f:\op{Mor}(x,-)\Rightarrow\op{Mor}(y,-)$. This will turn out to be a functor and very good. We will not say more for now.
\end{remark}

\subsection{Categorification}
The category $\mathrm{Set}$ has some nice operations: we can talk about products $A\times B$, disjoint unions $A\sqcup B$, and functions $A^C=\{f:C\to A\}$. Note that these notations are suggestive of multiplication, addition (depending on whom you talk to), and exponentiation. For example,
\[\#(A\times B)=\#A\times\#B,\quad\#(A\sqcup B)=\#A+\#B,\quad\#\left(A^C\right)=\#A^{\#C}.\]
This gives us some notion of a ``cardinality functor'' $\#:\mathrm{FinSet}\to\NN$, which we can check does some things.

This lets us define ``categorification.'' We will not give a formal definition of this, but here are some instructive examples.
\begin{example}
	The functor $\#:\mathrm{FinSet}\to\NN$ is a decategorification functor. For example, we can categorify $a\times(b+c)=a\times b+a\times c$ in $\NN$ to some natural isomorphism
	\[A\times(B\sqcup C)\simeq (A\times B)\sqcup (A\times C).\]
\end{example}
\begin{example}
	There is a decategorification functor $\dim:\mathrm{fdRep}_G\to\NN$.
\end{example}

\subsection{Equivalence: Advertisement}
Let's close class by defining an equivalence of categories. Recall that we called a functor $F:\mathcal C\to\mathcal D$ an isomorphism if and only if it has an inverse functor $G:\mathcal D\to\mathcal C$ such that $FG=\id_\mathcal D$ and $GF=\id_\mathcal C$.

This is a bad notion of saying two categories are the same.
\begin{example}
	The categories of $k$-matrices and $k$-vector spaces are not isomorphic (they don't have the same), even though we often think about vector spaces as merely being some dimensional space.
\end{example}
Here is the fix
\begin{definition}[Equivalence]
	Two categories $\mathcal C$ and $\mathcal D$ are \textit{equivalent} if and only if there exist functors $F:\mathcal C\to\mathcal D$ and $G:\mathcal D\to\mathcal C$ such that $FG\simeq\id_{\mathcal D}$ and $GF\simeq\id_\mathcal C$.
\end{definition}