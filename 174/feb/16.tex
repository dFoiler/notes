\documentclass[../notes.tex]{subfiles}

\begin{document}

% !TEX root = ../notes.tex

Here we go.

\subsection{House-Keeping}
Let's start with the attendance question from last class because it was a little tricky.
\begin{exe} \label{exe:indiscreteissmall}
	All nonempty indiscrete categories are equivalent.
\end{exe}
\begin{proof}
	The first part of this problem is remembering that indiscrete categories are ones that have all morphism sets are singletons. The second part of the problem is recognizing the following lemma.
	\begin{lemma} \label{lem:equivtoone}
		Fix $\mathcal C$ be a nonempty indiscrete category. Then $\mathcal C$ is equivalent to $\mathrm Be$, where $e$ is the single-element group.
	\end{lemma}
	\begin{proof}
		We use the functor $F:\mathcal C\to\mathrm Be$ sending all objects to $*$ and all morphisms to $\id_*$. It is surjective on objects because there is only one object to hit, and $\mathcal C$ is nonempty. Further, $F$ is fully faithful because, for any $c,c'\in\mathcal C$, the induced map
		\[F:\op{Mor}(c,c')\to\op{Mor}(*,*)\]
		is a bijection because both of these are singletons. It follows from \autoref{thm:betterequiv} that $F$ is an equivalence.
	\end{proof}
	So transitivity promises that all indiscrete categories are equivalent, finishing the proof.
\end{proof}
\begin{remark}
	In fact, one can use essentially the same proof to show that any functor between indiscrete categories is an equivalence. In particular, the (weak) inverse to the equivalence generated by \autoref{lem:equivtoone} is not canonical.
\end{remark}

\subsection{Diagram-Chasing Philosophy}
We recall that we proved \autoref{lem:changepaths} last time, which philosophically means that we should not try to show equalities of morphisms where there is some overlap between the morphisms. For example, to compare all paths in the rectangle
% https://q.uiver.app/?q=WzAsNixbMCwwLCJcXGJ1bGxldCJdLFsxLDAsIlxcYnVsbGV0Il0sWzAsMSwiXFxidWxsZXQiXSxbMSwxLCJcXGJ1bGxldCJdLFsyLDAsIlxcYnVsbGV0Il0sWzIsMSwiXFxidWxsZXQiXSxbMCwxXSxbMSwzXSxbMiwzXSxbMCwyXSxbMSw0XSxbMyw1XSxbNCw1XV0=
\[\begin{tikzcd}
	\bullet & \bullet & \bullet \\
	\bullet & \bullet & \bullet
	\arrow[from=1-1, to=1-2]
	\arrow[from=1-2, to=2-2]
	\arrow[from=2-1, to=2-2]
	\arrow[from=1-1, to=2-1]
	\arrow[from=1-2, to=1-3]
	\arrow[from=2-2, to=2-3]
	\arrow[from=1-3, to=2-3]
\end{tikzcd}\]
above, we merely have to check the commutativity of the squares.

We would like to have some tools to prove that diagrams commute.
\begin{remark}
	We remark that the following force commutative diagrams immediately.
	\begin{itemize}
		\item Any diagram indexed by a preorder commutes.
		\item Any diagram in a preorder commutes because any two morphisms between objects must be equal, so we get the commuting in the image of the index category.
	\end{itemize}
\end{remark}

\subsection{Initial and Final Objects}
Let's keep building up our theory.
\begin{restatable}[Initial, final]{definition}{initialfinaldefi}
	Fix a category $\mathcal C$.
	\begin{itemize}
		\item An object $i\in\mathcal C$ is \textit{initial} if and only if, for every $c\in\mathcal C$, there is a unique morphism in $\op{Mor}(i,c)$.
		\item The dual notion is that an object $t\in\mathcal C$ is \textit{final} or \textit{terminal} if and only if, for every $c\in\mathcal C$, there is a unique morphism in $\op{Mor}(c,t)$.
	\end{itemize}
\end{restatable}
\begin{remark}
	It is true that initial and final objects are unique up to unique isomorphism. We will not show this here because it might appear on the homework.
\end{remark}
And here are many, many examples.
\begin{example}
	We work in $\mathrm{Set}$.
	\begin{itemize}
		\item We have $\emp$ is initial. Namely, there is only one function $\emp\to S$ for any set $S$ by taking all elements of $\emp$ to whatever one's heart desires in $S$, and there is only one way to do this because any two such functions always have the same outputs.
		\item The singleton set $\{*\}$ is final. Indeed, any set $S$ has a unique function $S\to\{*\}$ by sending all elements of $S$ to $*$.
	\end{itemize}
\end{example}
\begin{example}
	In $\mathrm{Top}$, the initial object is $\emp$ and the final object is $\{*\}$.
\end{example}
\begin{example}
	We work in $\mathrm{Set}_*$, which are ordered pairs $(S,s)$ where $s\in S$. Morphisms $(S,s)\to(T,t)$ are functions $f:S\to T$ such that $f(s)=t$. Singleton sets $\{*\}$ is both initial and final. It's final for the same reason as in $\mathrm{Set}$, and it is initial because any pointed set $(S,s)$ has the unique morphism $*\mapsto s$.
\end{example}
\begin{example}
	We work in $\mathrm{Ab}$ or $\mathrm{Grp}$. Then the trivial group $0$ is the initial and final object by sending identities to identities.
\end{example}
\begin{nex}
	The object $\ZZ/2\ZZ$ is not initial in $\mathrm{Ring}$: there is no morphism $\ZZ/2\ZZ\to\ZZ$. Funnily enough, there is at most one morphism from $\ZZ/2\ZZ$ to anywhere.
\end{nex}
\begin{nex}
	We work in $\mathrm{Ring}$.
	\begin{itemize}
		\item The object $\ZZ$ is initial by sending $1\mapsto1_R$ (which is forced) for any ring $R$, and this uniquely determines the rest of the morphism.
		\item The zero ring $0$ is final in $\mathrm{Ring}$ because there is only one function $R\to0$ for any ring $R$, and it is in fact a ring homomorphism.
	\end{itemize}
\end{nex}
\begin{example}
	The category $\mathrm{Field}$ has no initial or final object. There is no final object because all morphisms are injections, and we cannot embed all fields into one large field. There is no initial object because there are no morphisms between fields of different characteristic. (One can fix this problem by considering the fields of characteristic $p$, where $\FF_p$ is the initial object.)
\end{example}
\begin{quot}
	I hate this category, and you should too.
\end{quot}
\begin{example}
	Let $\mathcal P$ be a preorder category.
	\begin{itemize}
		\item We claim that global minimums are equivalent to initial objects. To be explicit, there is surely at most one morphism between any two elements, so the object $m\in\mathcal P$ is an initial object if and only if there is a morphism $m\to x$ for each $x\in\mathcal P$ if and only if $m\le x$ for each $x$ if and only if $m$ is a global minimum.
		\item Dually, global maximums are equivalent to final objects.
	\end{itemize}
\end{example}
These new definitions give us a quick criterion for diagram-chasing.
\begin{lemma}
	Fix $f_1,\ldots,f_n$ and $g_1,\ldots,g_m$ be ``parallel'' paths in $\mathcal C$; i.e., $s:=\op{dom}f_1=\op{dim}g_1$ and $t:=\op{cod}f_n=\op{cod}g_m$. If $s$ is initial or $t$ is final, then
	\[f_n\cdots f_1=g_m\cdots g_1.\]
\end{lemma}
\begin{proof}
	We have two cases.
	\begin{itemize}
		\item Take $s$ initial. Then $f_n\cdots f_1$ and $g_m\cdots g_1$ are both maps $s\to t$, of which there is a unique map by $s$ being initial, so these are equal.
		\item Take $t$ final. Then repeat the above sentence using the fact $t$ is final instead of $s$ being initial.
		\qedhere
	\end{itemize}
\end{proof}

\subsection{Concrete Categories}
We have the following definition.
\begin{definition}[Concrete]
	A category $\mathcal C$ is \textit{concrete} if and only if there is a fully faithful functor $U:\mathcal C\to\mathrm{Set}$. We call $U$ the \textit{forgetful functor}.
\end{definition}
For example, this asserts that two morphisms $f,g:x\to y$ in $\mathcal C$ are equal if and only if their ``restrictions'' down in $\mathrm{Set}$ are equal, for which we can do an element-wise check on elements of sets.
\begin{lemma} \label{lem:faithfulcheck}
	Fix $U:\mathcal C\to\mathcal D$ be faithful functors. A diagram in $\mathcal C$ commutes if and only if its image through $U$ commutes.
\end{lemma}
\begin{proof}
	Fix $J$ our index category with the diagram $K:J\to\mathcal C$. We already know that $K$ commuting implies that $UK$ commutes by \autoref{prop:functcommdiag}.

	In the other direction, suppose that $UK$ commutes. Then pick up $k,k':i\to j$ in $J$ so that $UKk=UKk'$, but then $U$ being faithful forces
	\[Kk=Kk',\]
	which is exactly what we need to commute.
\end{proof}
And here is why we care
\begin{corollary}
	Commutativity of a diagram in a concrete category can be checked on ``elements.''
\end{corollary}
\begin{proof}
	Essentially we use the forgetful functor in \autoref{lem:faithfulcheck}. To be explicit, checking on ``elements'' is doing the diagram-chase in $\mathrm{Set}$, which we can then pull back to the original concrete category through the forgetful functor via \autoref{lem:faithfulcheck}.
\end{proof}
In other words, we can diagram-chase by working everything in set.

\subsection{Commutative Rectangles}
We have the following warning.
\begin{warn} \label{warn:rect}
	Consider the following rectangle.
	% https://q.uiver.app/?q=WzAsNixbMCwwLCJcXGJ1bGxldCJdLFsxLDAsIlxcYnVsbGV0Il0sWzAsMSwiXFxidWxsZXQiXSxbMSwxLCJcXGJ1bGxldCJdLFsyLDAsIlxcYnVsbGV0Il0sWzIsMSwiXFxidWxsZXQiXSxbMCwxXSxbMSwzXSxbMiwzXSxbMCwyXSxbMSw0XSxbMyw1XSxbNCw1XV0=
	\[\begin{tikzcd}
		\bullet & \bullet & \bullet \\
		\bullet & \bullet & \bullet
		\arrow[from=1-1, to=1-2]
		\arrow[from=1-2, to=2-2]
		\arrow[from=2-1, to=2-2]
		\arrow[from=1-1, to=2-1]
		\arrow[from=1-2, to=1-3]
		\arrow[from=2-2, to=2-3]
		\arrow[from=1-3, to=2-3]
	\end{tikzcd}\]
	We know that the squares commuting implies that the rectangle commutes. The converse is not true.
\end{warn}
\begin{example}
	We work in $\mathrm{Ab}$. The outer rectangle of the diagram
	% https://q.uiver.app/?q=WzAsNixbMCwwLCJcXFpaIl0sWzEsMCwiXFxaWiJdLFswLDEsIjAiXSxbMSwxLCJcXFpaIl0sWzIsMCwiMCJdLFsyLDEsIlxcWloiXSxbMCwxLCIiLDAseyJsZXZlbCI6Miwic3R5bGUiOnsiaGVhZCI6eyJuYW1lIjoibm9uZSJ9fX1dLFsyLDNdLFswLDJdLFsxLDRdLFszLDUsIiIsMix7ImxldmVsIjoyLCJzdHlsZSI6eyJoZWFkIjp7Im5hbWUiOiJub25lIn19fV0sWzQsNV0sWzEsMywiIiwwLHsibGV2ZWwiOjIsInN0eWxlIjp7ImhlYWQiOnsibmFtZSI6Im5vbmUifX19XV0=
	\[\begin{tikzcd}
		\ZZ & \ZZ & 0 \\
		0 & \ZZ & \ZZ
		\arrow[Rightarrow, no head, from=1-1, to=1-2]
		\arrow[from=2-1, to=2-2]
		\arrow[from=1-1, to=2-1]
		\arrow[from=1-2, to=1-3]
		\arrow[Rightarrow, no head, from=2-2, to=2-3]
		\arrow[from=1-3, to=2-3]
		\arrow[Rightarrow, no head, from=1-2, to=2-2]
	\end{tikzcd}\]
	will commute, but the inner squares do not. (The zero map is not the identity map.)
\end{example}
We can salvage \autoref{warn:rect} as follows.
\begin{lemma}
	Fix a rectangle as follows.
	% https://q.uiver.app/?q=WzAsNixbMCwwLCJhIl0sWzEsMCwiYiJdLFsyLDAsImMiXSxbMCwxLCJhJyJdLFsxLDEsImInIl0sWzIsMSwiYyciXSxbMCwxLCJlIl0sWzAsMywiZiIsMl0sWzEsMiwiayJdLFsxLDQsImciXSxbMiw1LCJoIl0sWzMsNCwiaiIsMl0sWzQsNSwibSIsMl1d
	\[\begin{tikzcd}
		a & b & c \\
		{a'} & {b'} & {c'}
		\arrow["e", from=1-1, to=1-2]
		\arrow["f"', from=1-1, to=2-1]
		\arrow["k", from=1-2, to=1-3]
		\arrow["g", from=1-2, to=2-2]
		\arrow["h", from=1-3, to=2-3]
		\arrow["j"', from=2-1, to=2-2]
		\arrow["m"', from=2-2, to=2-3]
	\end{tikzcd}\]
	Suppose the outer rectangle commutes. Then the diagram commutes if
	\begin{itemize}
		\item the right square commutes and $m$ is monic, or
		\item the left square commutes and $e$ is epic.
	\end{itemize}
\end{lemma}
\begin{proof}
	We have separate cases.
	\begin{itemize}
		\item Suppose the right square commutes and $m$ is monic. The right square commutes, so $hk=mg$. Similarly, the outer rectangle commutes, so $hke=mjf$. But then
		\[mge=hke=mjf,\]
		so $ge=jf$ because $m$ is monic. This shows the left square commutes, so we are done.
		\item This holds by running the proof of the above in the opposite category, where the main point is that the left and right squares flip, and $m$ being monic turns into $e$ being epic.
		\qedhere
	\end{itemize}
\end{proof}

\end{document}