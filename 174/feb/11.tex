% !TEX root = ../notes.tex

\subsection{A Better Equivalence}

Today we will be talking about the following theorem for our discussion.
\begin{restatable}{theorem}{betterequiv} \label{thm:betterequiv}
	Fix $F:\mathcal C\to\mathcal D$ a functor. Then the following are true.
	\begin{listalph}
		\item If $F$ is an equivalence, then $F$ is fully faithful and essentially surjective.
		\item Assuming a strong form of the axiom of choice, the converse holds.
	\end{listalph}
\end{restatable}
\begin{remark}
	The strong form of the Axiom of choice is for, not sets, but classes/categories depending on how we choose to construct our categories.
\end{remark}
\begin{proof}[Proof of (a) in \autoref{thm:betterequiv}]
	We will want some lemmas.
	\begin{lemma} \label{lem:optimize}
		Fix a category $\mathcal C$. Further, fix a morphism $f:c\to d$ and isomorphisms $\varphi:c\cong c'$ and $\psi:d\cong d'$. Then there is a unique morphism $f':c'\to d'$ such that one (or equivalently, all) of the following four squares commute.
		% https://q.uiver.app/?q=WzAsNCxbMCwwLCJjJyJdLFsxLDAsImMiXSxbMCwxLCJkJyJdLFsxLDEsImQiXSxbMCwyLCJmJyIsMix7InN0eWxlIjp7ImJvZHkiOnsibmFtZSI6ImRhc2hlZCJ9fX1dLFswLDEsIiIsMCx7Im9mZnNldCI6LTF9XSxbMSwwLCJcXHZhcnBoaSIsMCx7Im9mZnNldCI6LTF9XSxbMiwzLCIiLDIseyJvZmZzZXQiOi0xfV0sWzMsMiwiXFxwc2kiLDAseyJvZmZzZXQiOi0xfV0sWzEsMywiZiJdXQ==
		\[\begin{tikzcd}
			{c'} & c \\
			{d'} & d
			\arrow["{f'}"', dashed, from=1-1, to=2-1]
			\arrow[shift left=1, from=1-1, to=1-2]
			\arrow["\varphi", shift left=1, from=1-2, to=1-1]
			\arrow[shift left=1, from=2-1, to=2-2]
			\arrow["\psi", shift left=1, from=2-2, to=2-1]
			\arrow["f", from=1-2, to=2-2]
		\end{tikzcd}\]
		Here, the four squares are achieved by changing the direction of $\varphi$ and $\psi$.
	\end{lemma}
	\begin{proof}
		This is on the homework.
	\end{proof}
	We now return to the proof of the theorem. In the easier direction, suppose that $F$ is an equivalence with its inverse equivalence $G:\mathcal D\to\mathcal C$, witnessed by natural isomorphisms $\eta_\bullet:\id_\mathcal C\Rightarrow GF$ and $\varepsilon:GF\Rightarrow\id_\mathcal D$. We have the following checks.
	\begin{itemize}
		\item We show that $F$ is essentially surjective. Indeed, for any object $d\in\mathcal D$, we set $c:=Gd$. Then we see $FGd\cong d$ is witnessed by the component isomorphism $\varepsilon_d$.

		\item We show that $F$ is faithful, for which we have to use \autoref{lem:optimize}. Indeed, suppose that we have morphisms $f,g:c\to d$ such that $Ff=Fg$. Then in fact $GFf=Gfg$, so the following diagrams will commute.
		% https://q.uiver.app/?q=WzAsNCxbMCwwLCJjIl0sWzEsMCwiZCJdLFswLDEsIkdGYyJdLFsxLDEsIkdGZCJdLFswLDEsImYiXSxbMCwyLCJcXGV0YV9jIiwyXSxbMSwzLCJcXGV0YV9kIl0sWzIsMywiR0ZmPUdGZyIsMl1d
		\[\begin{tikzcd}
			c & d \\
			GFc & GFd
			\arrow["f", from=1-1, to=1-2]
			\arrow["{\eta_c}"', from=1-1, to=2-1]
			\arrow["{\eta_d}", from=1-2, to=2-2]
			\arrow["{GFf=GFg}"', from=2-1, to=2-2]
		\end{tikzcd}\qquad
		\begin{tikzcd}
			c & d \\
			GFc & GFd
			\arrow["g", from=1-1, to=1-2]
			\arrow["{\eta_c}"', from=1-1, to=2-1]
			\arrow["{\eta_d}", from=1-2, to=2-2]
			\arrow["{GFf=GFg}"', from=2-1, to=2-2]
		\end{tikzcd}\]
		It follows from \autoref{lem:optimize} that there $f$ and $g$ are uniquely determined, so $f=g$.
	\end{itemize}
	We quickly remark that, by symmetry, $G$ is also faithful.
	\begin{itemize}
		\item We show that $F$ is full, which will use the lemma as well as the fact that $G$ is faithful (!). Well, suppose that we have some morphism $g:Fc\to Fd$. Passing through to $G$, we get a morphism $Gg:GFg\to GFg$, so by \autoref{lem:optimize}, there is a unique morphism $f:c\to d$ so that the following diagram commutes.
		% https://q.uiver.app/?q=WzAsNCxbMCwwLCJjIl0sWzAsMSwiZCJdLFsxLDAsIkdGYyJdLFsxLDEsIkdGZCJdLFswLDEsImYiLDIseyJzdHlsZSI6eyJib2R5Ijp7Im5hbWUiOiJkYXNoZWQifX19XSxbMiwzLCJHZyJdLFswLDIsIlxcZXRhX2MiXSxbMSwzLCJcXGV0YV9kIiwyXV0=
		\[\begin{tikzcd}
			c & GFc \\
			d & GFd
			\arrow["f"', dashed, from=1-1, to=2-1]
			\arrow["Gg", from=1-2, to=2-2]
			\arrow["{\eta_c}", from=1-1, to=1-2]
			\arrow["{\eta_d}"', from=2-1, to=2-2]
		\end{tikzcd}\]
		Now, both $GFf$ and $Gg$ make the following diagram commute.
		% https://q.uiver.app/?q=WzAsNCxbMCwwLCJjIl0sWzAsMSwiZCJdLFsxLDAsIkdGYyJdLFsxLDEsIkdGZCJdLFswLDEsImYiLDJdLFsyLDMsIkdnLEdGZiIsMCx7InN0eWxlIjp7ImJvZHkiOnsibmFtZSI6ImRhc2hlZCJ9fX1dLFswLDIsIlxcZXRhX2MiXSxbMSwzLCJcXGV0YV9kIiwyXV0=
		\[\begin{tikzcd}
			c & GFc \\
			d & GFd
			\arrow["f"', from=1-1, to=2-1]
			\arrow["{Gg,GFf}", dashed, from=1-2, to=2-2]
			\arrow["{\eta_c}", from=1-1, to=1-2]
			\arrow["{\eta_d}"', from=2-1, to=2-2]
		\end{tikzcd}\]
		Thus, by \autoref{lem:optimize}, we see $GFf=Gg$, so $Ff=g$ by the faithfulness of $G$. This finishes.
		\qedhere
	\end{itemize}
\end{proof}
\begin{proof}[Proof of (b) in \autoref{thm:betterequiv}]
	Fix $F:\mathcal C\to\mathcal D$ a fully faithful and essentially surjective functor. We need to construct a $G:\mathcal D\to\mathcal C$ with some natural isomorphisms. We do this by hand.
	\begin{itemize}
		\item For each $d\in\mathcal D$, we callously choose\footnote{Note we are using some fuzzy form of the axiom of choice here. We will not say more about this.} $Gd$ to be any $c\in\mathcal C$ together with an isomorphism $\varepsilon_d:GFd\to d$. Indeed, such a $d$ with isomorphism $\varepsilon_d$ exists because $F$ is essentially surjective.
		\item For each $f:d\to d'$ in $\mathcal D$, we use \autoref{lem:optimize} to choose $h$ to be the unique morphism making the following diagram commute.
		% https://q.uiver.app/?q=WzAsNCxbMCwwLCJkIl0sWzAsMSwiZCciXSxbMSwwLCJGR2QiXSxbMSwxLCJGR2QnIl0sWzIsMywiR2YiLDAseyJzdHlsZSI6eyJib2R5Ijp7Im5hbWUiOiJkYXNoZWQifX19XSxbMCwxLCJmIiwyXSxbMSwzLCJcXHZhcmVwc2lsb25fe2QnfSIsMl0sWzAsMiwiXFx2YXJlcHNpbG9uX2QiXV0=
		\[\begin{tikzcd}
			d & FGd \\
			{d'} & {FGd'}
			\arrow["h", dashed, from=1-2, to=2-2]
			\arrow["f"', from=1-1, to=2-1]
			\arrow["{\varepsilon_{d'}}"', from=2-1, to=2-2]
			\arrow["{\varepsilon_d}", from=1-1, to=1-2]
		\end{tikzcd}\]
		But because $F$ is fully faithful, there will be a unique morphism which we call $Gf$ such that $F(Gf)=h$.
	\end{itemize}
	We would like to check that $G$ is in fact our inverse equivalence. However, we don't even know if $G$ is a functor yet.
	\begin{itemize}
		\item Fix $d\in\mathcal D$ and we compute $G(\id_d)$. We run through the definition. Well, we note that $\id_{FGd}$ makes the following diagram commute, so it will be the morphism generated by \autoref{lem:optimize}.
		\[\begin{tikzcd}
			d & FGd \\
			{d} & {FGd'}
			\arrow["{\id_{FGd}}", dashed, from=1-2, to=2-2]
			\arrow["{\id_d}"', from=1-1, to=2-1]
			\arrow["{\varepsilon_{d'}}"', from=2-1, to=2-2]
			\arrow["{\varepsilon_d}", from=1-1, to=1-2]
		\end{tikzcd}\]
		But now we see that $F(\id_{Gd})=\id_{FGd}$, so $\id_{Gd}$ must be the corresponding morphism promised by the fullness and faithfulness of $F$. In particular, by definition, $G(\id_d)=\id_{Gd}$.
		\item Suppose we have $f:d\to d'$ and $g:d'\to d''$. We want to show that $G(gf)=Gg\circ Gf$. For this, we have the following very big diagram.
		% https://q.uiver.app/?q=WzAsNixbMCwwLCJGR2QiXSxbMCwxLCJGR2QnIl0sWzAsMiwiRkdkJyciXSxbMSwwLCJkIl0sWzEsMSwiZCciXSxbMSwyLCJkJyciXSxbMCwxLCJGR2YiXSxbMSwyLCJGR2ciXSxbMCwyLCJGKEdmXFxjaXJjIEdmKSIsMix7Im9mZnNldCI6MSwiY3VydmUiOjJ9XSxbMCwzLCJcXHZhcmVwc2lsb25fZCIsMl0sWzEsNCwiXFx2YXJlcHNpbG9uX3tkJ30iLDJdLFsyLDUsIlxcdmFyZXBzaWxvbl97ZCcnfSIsMl0sWzMsNCwiZyJdLFs0LDUsImYiXSxbMyw1LCJnZiIsMCx7Im9mZnNldCI6LTEsImN1cnZlIjotMn1dXQ==
		\[\begin{tikzcd}
			FGd & d \\
			{FGd'} & {d'} \\
			{FGd''} & {d''}
			\arrow["FGf", from=1-1, to=2-1]
			\arrow["FGg", from=2-1, to=3-1]
			\arrow["{F(Gf\circ Gf)}"', shift right=1, curve={height=12pt}, from=1-1, to=3-1]
			\arrow["{\varepsilon_d}"', from=1-1, to=1-2]
			\arrow["{\varepsilon_{d'}}"', from=2-1, to=2-2]
			\arrow["{\varepsilon_{d''}}"', from=3-1, to=3-2]
			\arrow["g", from=1-2, to=2-2]
			\arrow["f", from=2-2, to=3-2]
			\arrow["gf", shift left=1, curve={height=-12pt}, from=1-2, to=3-2]
		\end{tikzcd}\]
		This diagram does commute, from which we see that the left arrow can be either $F(Gg\circ Gf)$ (by funtoriality of $F$) or $F(G(gf))$. So by \autoref{lem:optimize}, we have $F(Gg\circ Gf)=F(G(gf))$, so faithfulness of $F$ implies $Gg\circ Gf=G(gf)$.
	\end{itemize}
	Now we construct our natural isomorphisms.
	\begin{itemize}
		\item By construction of the $\varepsilon$s, the following diagram commutes.
		% https://q.uiver.app/?q=WzAsNCxbMCwwLCJGR2QiXSxbMSwwLCJkIl0sWzAsMSwiRkdkJyJdLFsxLDEsImQnIl0sWzAsMiwiRkdmIiwyXSxbMSwzLCJmIl0sWzAsMSwiXFx2YXJlcHNpbG9uX2QiXSxbMiwzLCJcXHZhcmVwc2lsb25fe2QnfSIsMl1d
		\[\begin{tikzcd}
			FGd & d \\
			{FGd'} & {d'}
			\arrow["FGf"', from=1-1, to=2-1]
			\arrow["f", from=1-2, to=2-2]
			\arrow["{\varepsilon_d}", from=1-1, to=1-2]
			\arrow["{\varepsilon_{d'}}"', from=2-1, to=2-2]
		\end{tikzcd}\]
		\item For the other direction, we note that if $Fx\cong Fy$ in $\mathcal D$, then $x\cong y$, which we will prove on the homework.\footnote{Yes, I know.} In particular, to create an isomorphism $\eta_c:c\to GFc$, it suffices to create an isomorphism $Fc\to FGFc$, for which we use $F\eta_c:=\varepsilon_{Fc}^{-1}$. For naturality, we suppose we have a morphism $f:c\to c'$, and we note that the following diagram commutes.
		% https://q.uiver.app/?q=WzAsNixbMCwwLCJGYyJdLFswLDEsIkZjJyJdLFsxLDAsIkZHRmMiXSxbMSwxLCJGR0ZjJyJdLFsyLDAsIkZjIl0sWzIsMSwiRmMnIl0sWzAsMiwiRlxcZXRhX2MiXSxbMCwxLCJGZiIsMl0sWzEsMywiRlxcZXRhX3tjJ30iLDJdLFsyLDMsIkZHRmYiXSxbNCw1LCJGZiJdLFsyLDQsIlxcdmFyZXBzaWxvbl97RmN9Il0sWzMsNSwiXFx2YXJlcHNpbG9uX3tGYyd9IiwyXV0=
		\[\begin{tikzcd}
			Fc & FGFc & Fc \\
			{Fc'} & {FGFc'} & {Fc'}
			\arrow["{F\eta_c}", from=1-1, to=1-2]
			\arrow["Ff"', from=1-1, to=2-1]
			\arrow["{F\eta_{c'}}"', from=2-1, to=2-2]
			\arrow["FGFf", from=1-2, to=2-2]
			\arrow["Ff", from=1-3, to=2-3]
			\arrow["{\varepsilon_{Fc}}", from=1-2, to=1-3]
			\arrow["{\varepsilon_{Fc'}}"', from=2-2, to=2-3]
		\end{tikzcd}\]
		Indeed, the outer rectangle commutes by definition of the $\eta_\bullet$s, and the right square commutes by naturality of the $\varepsilon_\bullet$s. Then this forces the left square to commute by an argument by noting
		\[\varepsilon_{Fc'}\circ F\eta_{c'}\circ Ff=\varepsilon_{Fc'}\circ FGFf\circ F\eta_c\]
		by the commutativity of the outer diagram, so we get the commutativity by inverting along $\varepsilon_{Fc'}$.
		\qedhere
	\end{itemize}
\end{proof}