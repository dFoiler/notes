% !TEX root = ../notes.tex

\documentclass[../notes.tex]{subfiles}

\begin{document}

\section{January 20}

I'm only here for a few weeks, but here we go anyway.

\subsection{Overview}
There will be three parts to the course, approximately going around the trace formula.
\begin{enumerate}
	\item Representations of real and $p$-adic groups: we will not cover structure theory of reductive groups, but we will try to say some interesting things.
	
	For $p$-adic groups, we will mainly follow Bernstein's Harvard notes from 1993; Cassleman's notes are also good but are more elementary. Here are some topics: we will talk about the Bernstein center following some notes of Deligne, and we will talk about the Plancherel formula approximately following the statements of Sakellaridis--Venkatesh. Note that the Plancherel formula gains importance because it more or less computes $L^2(G)$.

	For real groups, there is a ``lifting'' problem of topological representations compared to algebraic representations (i.e., $(\mf g,K)$-modules). We may also say something about the construction of representations (focusing on discrete series and inductions) and cohomology of representations. In particular, $(\mf g,K)$-cohomology has an application to locally symmetric spaces. Time-permitting, we can say something about localization theory.

	\item The (relative) trace formula: automorphic representations are found in $L^2([G])$, where $[G]$ is some adelic quotient. Then the trace formula classically computes the functional $f\mapsto\tr Rf$, where $f$ is some element of a Hecke algebra, and $R$ is a representation. When $[G]$ is compact, this automatically makes sense and is classical; however, when $[G]$ is not compact, this trace need not be finite. Our goal, then, is to explain how to make sense of such traces. For example, we will discuss Arthur's truncation.

	\item Endoscopy and stabilization: we want to say something about how the trace formula can be used to prove Langlands's functoriality conjecture for the standard representations $^LG\to\op{GL}_n$, where $G$ is a classical group. It turns out that this is done via a comparison of trace formulae: on one hand, there is the stable trace formula for $G$; on the other hand, we can compare this to some part of the twisted trace formula for $\op{GL}_n$. The comparison between these trace formulae is done via a ``stabilization,'' which we will only discuss in the non-twisted case.

	Stabilization starts with the following observation: for a local field $F\in\{\RR,\QQ_p\}$, there are elements of $\op{SL}_2(F)$ which are not conjugate but become conjugate over $\ov F$. (This is not true for $\op{GL}_n$!) Conjugacy over $\ov F$ is known as stable conjugacy. It was discovered that certain representations of $\op{SL}_2(F)$ come in pairs $\{\pi_+,\pi_-\}$ such that the difference of the characters looks like a character of a torus (e.g., supported on the image of a torus in $\op{SL}_2(F)$). This is the most basic instance of endoscopy.
\end{enumerate}
There will be notes embedded into the automorphic project.

\subsection{Reciprocity}
There's not much better to do today, so let's say something about reciprocity. Fix a global field $K$.
\begin{itemize}
	\item There ought to be a reciprocity map sending Galois representations $\op{Gal}(\ov K/K)\to\op{GL}_n(\CC)$ to automorphic representations of $\op{GL}_n$. In fact, these automorphic representations ought to be the ones with vanishing infinitesimal character.
	\item There ought to be a reciprocity map sending representations of the motivic Galois group to automorphic representations. In fact, these automorphic representations ought to be the ones with integral infinitesimal characters.
	\item Lastly, there ought to be some conjectural $\Gamma$ for which its representations correspond to all automorphic representations.
\end{itemize}

\end{document}