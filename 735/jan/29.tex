% !TEX root = ../notes.tex

\documentclass[../notes.tex]{subfiles}

\begin{document}

\section{January 29}
Last class was cancelled due to snow!

\subsection{Inductions and Restrictions}
Today, we will take $G$ to be some reductive group over a nonarchimedean local field $F$. Today, we will classify the compact representations of $G(F)$. This will imply that $\mc M(G(F))$ splits into a cuspidal part and parabolically induced part. This eventually produces the theory of the Bernstein center, which is a finer decomposition of $\mc M(G(F))$.

Quickly, recall that (maximal) parabolic subgroups $P\subseteq G$ can be described as those for which $G/P$ is compact. They can be parameterized by cocharacters $\lambda\colon\mathbb G_m\to G$: the character $\lambda$ induces a grading on the Lie algebra $\mf g$; the Lie algebra of the parabolic is then given by the Lie subalgebra of the nonnegatively graded pieces. Accordingly, we let $P_\lambda$ be the associated parabolic and $U_\lambda$ be the associated unipotent radical. (It turns out that $\op{Lie}U_\lambda$ consists of the subspace of $\mf g$ with positive degree.) There is also a Levi $L_\lambda\subseteq G$ for which $P_\lambda=U_\lambda L_\lambda$.

To motivate the notion of a cuspidal representation, we recall Frobenius reciprocity.
\begin{defihelper}[induction, restriction]
	Fix a topological subgroup $H$ of a td group $G$. Then the natural \textit{restriction} functor $\op{Res}^G_H\colon\mc M(G)\to\mc M(H)$ admits a right adjoint $\op{Ind}_H^G\colon\mc M(H)\to\mc M(G)$, called \textit{induction}. If $H$ is open, then there is a left adjoint, denoted $\op{ind}^G_H$ or $\op{cInd}^G_H$, and it is called \textit{compact induction}.
\end{defihelper}
\begin{remark}
	One can construct $\op{Ind}^G_HW$ as the smooth sections of the equivariant sheaf $W\times^HG$ on $H\backslash G$. Concretely, these are the smooth functions $f\colon G\to W$ which are $H$-invariant.
\end{remark}
\begin{remark}
	It turns out that $\op{cInd}^G_HW$ admits the same description, with the extra condition that $\op{supp}f/H$ is compact.
\end{remark}
\begin{remark}
	If $H$ is not open, then one can still define a functor $\op{cInd}^G_H$, but it is no longer adjoint; one should also change the definition of $\op{cInd}^G_H$ as measures on $H\backslash G$ valued in the sheaf $W\times^HG$.
\end{remark}
\begin{remark}
	The isomorphism
	\[\op{Hom}_G(V,\op{Ind}^G_HW)\to\op{Hom}_H(\op{Res}^G_HV,W)\]
	is given by evaluating the function in $\op{Ind}^G_HW$ at the identity.
\end{remark}
\begin{definition}[parabolic induction]
	Fix a reductive group $G$ over a nonarchimedean local field $F$, and choose a parabolic $P\subseteq G$ with Levi subgroup $L$. Then we define \textit{parabolic induction} $I^G_P$ as the composite functor
	\[\mc M(L(F))\stackrel{\op{Inf}}\to\mc M(P(F))\stackrel{\op{Ind}}\to\mc M(G(F)).\]
\end{definition}
\begin{definition}[Jacquet]
	Fix a reductive group $G$ over a nonarchimedean local field $F$, and choose a parabolic $P\subseteq G$. Then we define the \textit{Jacquet functor} $R^G_P$ to be the left adjoint of $I^G_P$.
\end{definition}
\begin{remark}
	It turns out that $R^G_P$ admits an explicit description as taking the $U$-coinvariants, where $U\subseteq P$ is the unipotent radical.
\end{remark}
It turns out that taking coinvariants always right exact, for any subgroup $H\subseteq G$. In the unipotent case, one can do better.
\begin{theorem}
	Fix a reductive group $G$ over a nonarchimedean local field $F$, and choose a parabolic $P\subseteq G$. Then $R^G_P$ is exact.
\end{theorem}
\begin{proof}
	We need to show that taking coinvariants by a unipotent subgroup $U$ is exact. The main point is that a unipotent group $U$ admits a filtration $\{U_n\}_n$ by compact unipotent subgroups.
	\begin{example}
		We will not prove this latter claim, but we will note that it is not so hard to see for $\op{GL}_n$, where the unipotent radical is simply given by the upper-triangular matrices. Then one can take the subgroup to be the matrices generated by ones coming from coefficients in $\mf p^{-\bullet}\OO$ just above the diagonal. One can upgrade this to work in general because unipotent groups are in general filtered by $\mathbb G_a$ (in characteristic $0$).
	\end{example}
	But there is a natural isomorphism $(-)_H\to(-)^H$ for any compact group $H$ given by integration: send a vector $v$ to $\int_Hhv\,dh$. The result now follows by using the filtration.

	Let's explain the last deduction. We need to show that an embedding $A\into B$ induces an injection $A_U\into B_U$. Then we want to show that the kernel of the composite
	\[A\to B\to B_U\]
	is exactly the vectors of the form $\{ua-a:a\in A\}$. Well, choose some $a\in A$ in the kernel, so we are granted $b\in B$ and $u\in U_n$ (for $n$ large) so that $a=b-ub$. This means that $a$ is in the kernel of the averaging map by $U_n$, so the result follows.
\end{proof}
\begin{remark}
	It may occasionally be convenient to normalize $I^G_P$ by a factor of $\delta_P^{1/2}$, where $\delta_P$ is the modular character. This is convenient because it sends us to half-densities instead of sections of the line bundle. Explicitly, this normalized parabolic induction preserves unitarity (namely, one can consider square integrals). On the other hand, it is occasionally inconvenient to be forced to choose a square root.
\end{remark}

\subsection{Cuspidal Representations}
Here is the main definition for this part of the course.
\begin{defihelper}[quasicuspidal, cuspidal] \nirindex{quasicuspidal} \nirindex{cuspidal}
	Fix a reductive group $G$ over a nonarchimedean local field $F$. A smooth representation $W$ of $G(F)$ is \textit{quasicuspidal} if and only if $R^G_PW=0$ for all proper parabolic subgroups $P\subseteq G$. We say that $W$ is \textit{cuspidal} if and only if $W$ is quasicuspidal and finitely generated. Lastly, $W$ is \textit{supercuspidal} if and only if $W$ is quasicuspidal and irreducible.
\end{defihelper}
\begin{example}
	Consider the group $G=\mathbb G_m$. Then the compact induction $\op{cInd}_{\OO^\times}^{F^\times}\CC$ is cuspidal but not finite length. This same example works for any torus.
\end{example}
\begin{theorem}
	Fix a reductive group $G$ over a nonarchimedean local field $F$. Then a representation $W$ of $G(F)$ is compact modulo center if and only if it is quasicuspidal.
\end{theorem}
The proof is based on a ``$U_p$-operator.''
\begin{example}
	Consider the group $G=\op{GL}_2$, which admits Iwahori subgroup
	\[J=\begin{bmatrix}
		\OO^\times & \OO \\
		\mf p & \OO^\times
	\end{bmatrix}.\]
	Then there is an operator $U_p$ (in the Iwahori Hecke algebra) which is the characteristic function on
	\[J\op{diag}(p,1)J.\]
	It turns out that $U_p^n$ is (up to scalar) $J\op{diag}(p^n,1)J$. These operators turn out to provide a subalgebra of the Iwahori Hecke algebra $\mc H(J\backslash G/J)$ isomorphic to $\CC[\NN]$.
\end{example}
To generalize this, we recall the following facts about reductive groups.
\begin{theorem}[Weak Cartan decomposition]
	Fix a reductive group $G$ over a nonarchimedean local field $F$. Choose a minimal parabolic $P_0\subseteq G$ with Levi subgroup $L_0$, and let $A_0\subseteq L_0$ be the maximal split (central) torus. Then $\Lambda\coloneqq X_*(A_0)$ is a free abelian group, and there is a dominant subset $\Lambda^+\coloneqq\Lambda$ consisting of those elements acting nonnegatively on the weights of $P_0$. Then there is a compact subset $K\subseteq G$ for which
	\[G=\cup_{\lambda\in\Lambda^+}K\lambda(\varpi)K.\]
\end{theorem}
\begin{example}
	For $G=\op{GL}_n$, the usual Cartan decomposition reads
	\[\op{GL}_n(F)=\bigsqcup_{\lambda\in\Lambda^+}K\lambda(\varpi)K,\]
	were $K=G(\OO)$, and $\Lambda^+$ consists of the dominant cocharacters.
\end{example}
\begin{theorem}
	Fix a reductive group $G$ over a nonarchimedean local field $F$ with minimal parabolic $P_0$ with Levi decomposition $P_0=L_0U_0$. Then $G(F)$ admits a basis of open compact subgroups $J_n$ with Iwahori factorization
	\[J_n=J_{n+}J_{n0}J_{n-}\]
	with respect to $P_0$, where $J_{n0}\coloneqq J_n\cap L_0$ and $J_{n+}\coloneqq J_n\cap U_0$ and $J_{n-}\coloneqq J_n\cap U_0^-$.
\end{theorem}
\begin{remark}
	One can interchange the subgroups in the Iwahori factorization.
\end{remark}
\begin{remark}
	It turns out that any such $J_n$ admits an embedding $\CC[\Lambda^+]\into\mc H(J_n\backslash G/J_n)$ given by (up to some scalar)
	\[\lambda\mapsto1_{J_n\lambda(\varpi)J_n}.\]
	Indeed, up to some scalar, it is enough to check that
	\[J_n\lambda(\varpi)J_n\cdot J_n\lambda'(\varpi)J_n\stackrel?=J_n\lambda\lambda'(\varpi)J_n.\]
	The point is that the $0$ parts commute with the $\lambda(\varpi)$ and $\lambda'(\varpi)$, and the positive and negative parts can be moved around with some explicit factors.
\end{remark}

\end{document}