% !TEX root = ../notes.tex

\documentclass[../notes.tex]{subfiles}

\begin{document}

\section{January 22}
Today we start discussing the complex representation theory of $p$-adic groups. Specifically, we are hoping to say something about the Bernstein center over the next few lectures. The original paper 
\begin{remark}
	Everything we say will also be fine over base fields of characteristic $0$, not necessarily algebraically closed.
\end{remark}

\subsection{Smooth Representations}
For now, $G$ will be a totally disconnected, locally compact, second countable topological group.
\begin{definition}[td group]
	A \textit{td group} is a totally disconnected, locally compact, second countable topological group.
\end{definition}
\begin{example}
	For any affine algebraic group $G$ over a local field $K$, the topological group $G(K)$ is totally disconnected, locally compact, and second countable.
\end{example}
\begin{example}
	The group $G=\ZZ$ is td.
\end{example}
We are currently interested in representation theory of td groups.
\begin{definition}[smooth]
	Fix a td group $G$. Then a (complex) representation $V$ of $G$ is \textit{smooth} if and only if the stabilizer of any $v\in V$ is open in $G$. The category of smooth representations is denoted by $\mc M(G)$.
\end{definition}
\begin{remark}
	To say a little about where we are going, the theory of the Bernstein center will allow us to decompose the category $\mc M(G)$ into subcategories indexed by pairs of a Levi subgroup and a supercuspidal representation (up to equivalence). (Such a pair is frequently referred to as a cuspidal support.)
\end{remark}
It will be worthwhile to be able to view the category of (smooth) representations of $G$ as a module category. This is the purpose of the Hecke algebra.
\begin{notation}
	Fix a td group $G$. Then $S(G)$ denotes the collection $C_c^\infty(G)$ of compactly supported, locally compact functions on $G$. The collection $C^\infty(G)$ consists of the uniformly smooth functions, which are those which are locally constant and smooth as a representation of $G\times G$.
\end{notation}
\begin{definition}[Hecke algebra]
	Fix a td group $G$. Then the \textit{Hecke algebra} $\mc H(G)$ is the convolution algebra of complex, compactly supported, smooth Borel measures on $G$ (meaning that they are invariant under translation by an open subgroup of $G$).
\end{definition}
\begin{remark}
	One can show that $\mc H(G)=S(G)\,dg$, where $dg$ is some choice of Haar measure. Note that the definition of $\mc H(G)$ descends canonically to $\QQ$ upon choosing $dg$ to have integral measure on some chosen open subgroup of $G$.
\end{remark}
\begin{remark}
	The algebra $\mc H(G)$ is an ``idempotent algebra,'' meaning that any finite collection $\mc F\subseteq\mc H(G)$ admits an idempotent $e$ for which
	\[ef=fe=f\]
	for all $f\in\mc F$. For example, one can take $e$ to be the characteristic measure of some small compact open subgroup built from the $f$s. It follows that
	\[\mc H(G)=\lim_{K\subseteq G}e_K\mc H(G)e_K.\]
\end{remark}
\begin{definition}
	Fix a td group $G$. Then an $\mathcal H(G)$-module $M$ is \textit{non-degenerate} if and only if the natural map
	\[\mc H(G)\otimes_{\mc H(G)}M\to M\]
	is surjective.
\end{definition}
\begin{remark}
	It looks like the natural map should always be an isomorphism, but this is not the case because $\mc H(G)$ is not necessarily unital!
\end{remark}
\begin{remark}
	It turns out that the surjectivity of this map is equivalent to being an isomorphism.
\end{remark}
\begin{proposition}
	Fix a td group $G$. The natural functor from $\mc M(G)$ to the category of non-degenerate $\mc H(G)$-modules is an equivalence. The functor sends a representation $V$ of $G$ to the Hecke module $V$ with Hecke action given by
	\[(f\,dg)\cdot v\coloneqq\int_Gf(g)(gv)\,dg.\]
\end{proposition}
\begin{proof}
	We will say nothing of content, but let's describe the inverse functor. Given the isomorphism
	\[\mc H(G)\otimes_{\mc H(G)}V\to V,\]
	one can construct a $G$-action on $V$ by figuring out how $\delta_g$ should act on $V$, which is done by taking a limit back in $\mc H(G)\otimes_{\mc H(G)}V$.
\end{proof}
\begin{corollary}
	Fix a td group $G$. Then $\mc M(G)$ is abelian and has enough projectives.
\end{corollary}
\begin{proof}
	The category of $\mc H(G)$-modules has these properties, and one can check that they descend to the subcategory of non-degenerate modules. The difficult point is to check that there are enough projectives, so let's explain this. Fix a non-degenerate module $M$. Then each $m\in M$ admits an idempotent $e_m\in\mc H(G)$ with $e_mm=m$. Then there is a natural surjection
	\[\bigoplus_{m\in M}\mc H(G)e_m\to M\]
	where $\mc He_m$ goes to $e_mm=m$. Thus, it is enough to note that $\mc H(G)e_m$ is projective, which is true because the functor $\op{Hom}_{\mc H(G)}(\mc H(G)e_m,-)$ is simply the functor which takes $e_m$-invariants.
\end{proof}
\begin{example}
	In an abelian category like $\mc M(G)$, a morphism which is monic and epic is automatically an isomorphism. But we do not expect this from our representation theoretic categories in general: for example, in the category of Banach spaces, it is quite possible to find vector spaces with dense subspaces.
\end{example}
\begin{remark}
	When the td group $G$ is the $\QQ_p$-points of an algebraic group, one can further find that $\mc M(G)$ admits enough injectives.
\end{remark}
While we're here, we note that there is a Schur's lemma.
\begin{lemma}[Schur]
	Fix a smooth irreducible representation $V$ of a td group $G$. Then
	\[\op{End}_GV=\CC.\]
\end{lemma}
\begin{proof}
	This requires the assumption that $G$ is second countable. We proceed in steps.
	\begin{enumerate}
		\item Note that $\dim V$ is at most countable: smoothness implies that
		\[V=\bigcup_{K\subseteq G}V^K.\]
		This union can be made countable because $G$ is second countable! Now, upon choosing a vector $v\in V$, we see that $V^K$ is $e_K\mc H(G)e_K\cdot v$, which is at most countable dimensional because $e_K\mc H(G)e_K$ is.
		\item Now, choose some $G$-invariant operator $T$ of $V$. We will show that $T$ admits an eigenvalue $\lambda$, which means that $\ker(T-\lambda\id_V)$. Well, if $T$ has no eigenvalue, then $T-\lambda\id_V$ is invertible for all $\lambda\in\CC$, so we have an uncountable family of operators
		\[\left\{\frac1{T-\lambda\id_V}:\lambda\in\CC\right\}.\]
		However, $\op{End}_GV$ is countable, so these operators admit a relation. Clearing denominators implies that $T$ is the root of a polynomial of finite degree, which is a contradiction to admitting no eigenvalues.
		\qedhere
	\end{enumerate}
\end{proof}
\begin{remark}
	This proof works fine as long as the base field is uncountable. If we want to work over $\ov\QQ$ or number fields, then there is a counterexample: take $G=\ov\QQ^\times$ and $V=\ov\QQ$. This problem will be fixed when we pass to $p$-adic groups.
\end{remark}
% K = Kbar
% A is an algebra over K which is at most countable dimension
% V is countable dimension over K
% V is irreducible as an A-module
% Is End_A(V) = C?

\subsection{Finiteness Conditions}
It will be worthwhile to have more finiteness criteria on our representations.
\begin{definition}[admissible]
	Fix a td group $G$. Then a representation $V$ of $G$ is \textit{admissible} if and only if
	\[\dim V^K<\infty\]
	for all open subgroups $K\subseteq G$.
\end{definition}
\begin{example}
	Finite-dimensional representations of $\ZZ$ are admissible, and these are the only ones because $\{0\}$ is a compact open subgroup. For example, the regular representation fails to be admissible, which indicates that the category admissible representations is too small!
\end{example}
Here is an application of admissibility.
\begin{definition}[contragredient]
	Fix a td group $G$. Then the \textit{contragredient} $\widetilde V$ of a representation $V$ of $G$ consists of the smooth vectors of the dual of $V$.
\end{definition}
\begin{remark}
	If $V$ is smooth, then there is a canonical embedding $V\into\widetilde{\widetilde V}$ given by $v\mapsto(\varphi\mapsto\varphi(v))$. This embedding is an isomorphism if and only if $V$ is admissible. To see this last claim, note that $\widetilde V$ is smooth, so
	\[\widetilde V=\bigcup_{K\subseteq G}\widetilde V^K.\]
	Now, some functional $\ell\in\widetilde V^K$ factors through the $K$-coinvariants of $V$. However, by fixing a Haar measure, coinvariants are identified with invariants (send a vector $v$ in the coinvariants to its $K$-average), so we find that $\widetilde V^K$ is identified with $(V^K)^\lor$. Thus, $\widetilde V$ is identified with $\bigcup_K(V^*)^K$, and the claim follows from usual representation theory.
\end{remark}
Here is another finiteness condition.
\begin{definition}[compact]
	A smooth representation $V$ of a td group $G$ is \textit{compact} if and only if its matrix coefficients are compactly supported. Similarly, $V$ is compactly supported modulo the center if and only if its matrix coefficients are compactly supported modulo the center. Here, a matrix coefficient is a function in the image of the natural evaluation map
	\[\arraycolsep=1.4pt\begin{array}{cccccccc}
		\widetilde V &\otimes& V &\to& C^\infty(G) \\
		\ell &\otimes& v &\mapsto& (g\mapsto\ell(gv))
	\end{array}\]
	outputting uniformly smooth functions.
\end{definition}
\begin{remark} \label{rem:alt-compactness}
	It turns out that $V$ is compact if and only if, for each open compact subgroup $K\subseteq G$ and vector $v\in V$, the functional
	\[g\mapsto e_Kgv\]
	has compact support. (Here, $e_K$ is the idempotent for $K\subseteq G$ in the Hecke algebra.)
\end{remark}
\begin{remark}
	One can translate everything into Hecke modules as follows: one may identify $C^\infty(G)$ with $\widetilde{\mc H(G)}$ because functions pair with measures to produces scalars. Then the matrix coefficient map sends $\ell\otimes v$ to $(T\mapsto\ell(Tv))$.
\end{remark}
Now is as good a time as any to introduce the following idea.
\begin{idea}
	Harmonic analysis of a group $G$ has categorical interpretations.
\end{idea}
Here's an example, which uses the notion of compactness: compact representations split $\mc M(G)$.
\begin{proposition}
	Fix a unimodular td group $G$, and choose a compact representation $V$.
	\begin{listalph}
		\item There is a splitting
		\[\mc H(G)=\mc H(G)_V\oplus\mc H(G)_{\lnot V}\]
		of $\mc H(G)$ into two-sided ideals.
		\item There is a splitting of categories
		\[\mc M(G)=\mc M(G)_V\oplus\mc M(G)_{\lnot V}.\]
	\end{listalph}
\end{proposition}
\begin{proof}
	Note (b) follows from (a) by taking module categories from the two ideals: send $M\in\mc M(G)$ to the decomposition $\mc H(G)_VM\oplus\mc H(G)_{\lnot V}M$. Let $m$ be the matrix coefficient map for $V$. It remains to show (a). Here, $\mc H(G)_V$ is defined to be the image of $m\,dg$, which we can see is a two-sided ideal. The complement $\mc H(G)_{\lnot V}$ is defined by embedding $\mc H(G)$ into $L^2(G)$ (by choosing a Haar measure) and then the inner product on $L^2(G)$ yields the desired complement.
\end{proof}
\begin{remark}
	If $V$ is irreducible (and Schur's lemma holds), there is an ``algebraic'' argument which does not require us to embed $\mc H(G)$ into $L^2(G)$. Note that there is a composite
	\[\widetilde V\otimes V\stackrel m\to S(G)\stackrel{dg}\to\mc H(G)\to\op{End}(V)^\infty=\widetilde V\otimes V.\]
	Here, $(-)^\infty$ denotes smooth endomorphisms. As soon as we choose a compact open subgroup $K\subseteq G$, we can take invariants for $K\times K$ everywhere, which provides an identity of $\op{End}(V)^\infty$. Then $\mc H(G)_V$ is taken to be the projection on this identity, and $\mc H(G)_{\lnot V}$ is the kernel of this identity. (As a technical point, we note that the irreducibility of $V$ is used to show that the splitting has the required properties: the map $\mc H(G)\to\op{End}(V)^{K\times K}$ is surjective.)
\end{remark}
Our two finiteness conditions are related as follows.
\begin{lemma}
	Fix a td group $G$. Then finitely generated compact representations of $G$ are admissible.
\end{lemma}
\begin{proof}
	Fix such a representation $V$. For simplicity, we assume that $V$ has only a single generator $v$. It follows that $V^K$ is spanned by
	\[\{e_Kgv:g\in G\}.\]
	However, the map $g\mapsto e_Kgv$ is compactly supported by \Cref{rem:alt-compactness}, so the set in the previous sentence is finite, so $V^K$ is finite-dimensional.
\end{proof}

\end{document}