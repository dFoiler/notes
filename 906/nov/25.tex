% !TEX root = ../notes.tex

\documentclass[../notes.tex]{subfiles}

\begin{document}

\section{November 25}
It is likely that there will be no more homeworks.

\subsection{The Adams Spectral Sequence}
Recall that the Dold--Kan correspondence provides us with an equivalence $\mathrm{sSet}(\ZZ)\to\op{Ch}(\ZZ)_{\ge0}$, so we can form a composite
\[\mathrm{sSet}(\ZZ)\to\op{Ch}(\ZZ)_{\ge0}\to\mc D(\ZZ)=\mathrm{Mod}(\ZZ).\]
Thus, simplicial abelian groups $\Delta\opp\to\mathrm{Ab}$ produce spectra by taking the colimit of the composite $\Delta\opp\to\mathrm{Ab}\to\mc D(\ZZ)$, where the second functor is given by taking Eilenberg--MacLane spaces.

Analogously, give a cosimplicial abelian group $\Delta\to\mathrm{Ab}$, we can obtain an object in $\mc D(\ZZ)$ again by taking the colimit.
\begin{remark}
	Note that there is also a Dold--Kan correspondence which takes cosimplicial abelian groups to $\op{Ch}(\ZZ)_{\le0}$, defined similarly. Namely, given $A\colon\Delta\to\mathrm{Ab}$, we can define a sequence of maps
	\[A_0\to A_1\to A_2\to\cdots\]
	by taking alternating sums of the various maps provided by the cosimplicial structure on $A$.
\end{remark}
We would like to upgrade this story to $\mathbb E_\infty$-rings. To work relatively, fix a map $A\to B$ of $\mathbb E_\infty$-rings. %, such as the canonical map $\mathbb S\to\FF_2$.

One way to generalize the above story is to let $\mc I$ denote the fiber of $A\to B$, and then $A$ admits a filtration by
\[\cdots\to\mc I\otimes_A\mc I\otimes_A\mc I\to\mc I\otimes_A\mc I\to\mc I\to A.\]
Here, the map $\mc I\otimes\mc I\to\mc I$ is induced by taking $\mc I\otimes-$ of the map $\mc I\to A$, and then the higher maps are defined inductively. For example, because tensor products preserve cofibers, we see that the map $\mc I\otimes\mc I\to\mc I$ has cofiber $\mc I\otimes_AB$.

Now, because this is a filtration, this will produce a spectral sequence.
\[E_1^{pq}=\pi_p\op{cofiber}\left(\mc I^{\otimes(q+1)}\to\mc I^{\otimes q}\right).\]
This is called the Adams (or descent) spectral sequence. We would like for it to converge to $\pi_*A$, but we are not ready to state this as a theorem yet.
% \begin{theorem}[Adams spectral sequence]
% 	Fix a map $A\to B$ of $\mathbb E_\infty$-rings with fiber $\mc I$. Then there is a spectral sequence
% 	\[E_1^{pq}=\pi_p\op{cofiber}\left(\mc I^{\otimes(q+1)}\to\mc I^{\otimes q}\right).\]
% \end{theorem}
% \begin{remark}
% 	This is also called the descent spectral sequence.
% \end{remark}
\begin{example}
	Consider the map $\mathbb S\to\FF_2$. Then the cofiber of $\mc I^{\otimes2}\to\mc I$ is given by
	\[\pi_*(\mc I\otimes_{\mathbb S}\FF_2)=\pi_*\op{fiber}(\FF_2\to\FF_2\otimes_{\mathbb S}\FF_2),\]
	which is something that we can compute. In general, one may need to compute some fibers for powers of $\FF_2$, which is still doable.
\end{example}
It is quite non-obvious if the Adams spectral sequence convergence. To explain this, we construct this spectral sequence differently. Note that there are two maps $B\to B\otimes_AB$ given by taking $-\otimes_AB$ or $B\otimes_A-$ with the unit, and there is also a map in the reverse direction given by multiplication. Similarly, there are three maps $B\otimes_AB\to B\otimes_AB\otimes_AB$, and there are two different multiplications backwards. Continuing, we get some diagram
% https://q.uiver.app/#q=WzAsNCxbMCwwLCJCIl0sWzEsMCwiQlxcb3RpbWVzX0FCIl0sWzIsMCwiQlxcb3RpbWVzX0FCXFxvdGltZXNfQUIiXSxbMywwLCJcXGNkb3RzIl0sWzEsMF0sWzAsMSwiIiwyLHsib2Zmc2V0IjozfV0sWzAsMSwiIiwxLHsib2Zmc2V0IjotM31dLFsyLDFdLFsxLDIsIiIsMSx7Im9mZnNldCI6M31dLFsxLDIsIiIsMSx7Im9mZnNldCI6LTN9XSxbMiwxLCIiLDEseyJvZmZzZXQiOjV9XSxbMiwxLCIiLDEseyJvZmZzZXQiOi01fV1d&macro_url=https%3A%2F%2Fraw.githubusercontent.com%2FdFoiler%2Fnotes%2Fmaster%2Fnir.tex
\[\begin{tikzcd}[cramped]
	B & {B\otimes_AB} & {B\otimes_AB\otimes_AB} & \cdots
	\arrow[shift right=2, from=1-1, to=1-2]
	\arrow[shift left=2, from=1-1, to=1-2]
	\arrow[from=1-2, to=1-1]
	\arrow[shift right=2, from=1-2, to=1-3]
	\arrow[shift left=2, from=1-2, to=1-3]
	\arrow[from=1-3, to=1-2]
	\arrow[shift right=4, from=1-3, to=1-2]
	\arrow[shift left=4, from=1-3, to=1-2]
\end{tikzcd}\]
which is a cosimplicial $A$-module; i.e., it is a functor $\Delta\to\op{Mod}(A)$. All rightward maps are given by units, and all leftward maps are given by multiplications.
\begin{remark}
	This is the \v{C}ech resolution. Geometrically, we may imagine taking $\Spec B$ to be some disjoint union $\bigsqcup_iU_i$ covering $A$, and then $\Spec B\otimes_AB$ is the disjoint union of the various intersections, $\Spec B\otimes_AB\otimes_AB$ is the disjoint union of the various $3$-fold intersections, and so on.
\end{remark}
\begin{remark}
	The limit of this diagram turns out to be the universal approximation of $A$ using $B$-modules. We will not make this precise, but it is possible to do so.
\end{remark}
\begin{definition}[completion]
	Fix a morphism $A\to B$ of $\mathbb E_\infty$-rings. Then we define the \textit{$B$-completion of $A$} as the limit of the following \v{C}ech resolution.
	\[\begin{tikzcd}[cramped]
		B & {B\otimes_AB} & {B\otimes_AB\otimes_AB} & \cdots
		\arrow[shift right=2, from=1-1, to=1-2]
		\arrow[shift left=2, from=1-1, to=1-2]
		\arrow[from=1-2, to=1-1]
		\arrow[shift right=2, from=1-2, to=1-3]
		\arrow[shift left=2, from=1-2, to=1-3]
		\arrow[from=1-3, to=1-2]
		\arrow[shift right=4, from=1-3, to=1-2]
		\arrow[shift left=4, from=1-3, to=1-2]
	\end{tikzcd}\]
\end{definition}
\begin{example}
	If $\mc I$ has $\pi_*$ concentrated in degrees above $1$, then $A\to A_B$ is an equivalence. The point is that the powers of $\mc I$ 
\end{example}
\begin{example}
	It turns out that $\mathbb S_{\FF_2}=\mathbb S_2^\land$.
\end{example}
\begin{remark}
	The completion $\mathbb S_{\mathrm{KU}}$ is completely understood and is quite interesting. For example, for each prime $p$, the canonical map
	\[(\mathbb S_{\mathrm{KU}})^\land_p\to\mathrm{KU}^\land_p\]
	is Galois with Galois group $\ZZ_p^\times$.
\end{remark}
\begin{notation}
	Given a morphism $A\to B$ of $\mathbb E_\infty$-rings, we define
	\[\mathrm{fil}^n_{B/A}\coloneqq\lim_{[k]\in\Delta}\tau_{\ge n}B^{\otimes(k+1)}\]
	for each integer $n$.
\end{notation}
Then there is a filtration
\[\cdots\to\mathrm{fil}^{n-1}_{B/A}\to\mathrm{fil}^n_{B/A}\to\mathrm{fil}^{n-1}_{B/A}\to\cdots\]
which has limit equal to $A_B$. This now gives rise to the Adams spectral sequence.
\begin{theorem}[Adams spectral sequence]
	Fix a map $A\to B$ of $\mathbb E_\infty$-rings, and let $\mc I$ be the fiber. Then there is a spectral sequence
	\[E_1^{pq}=\mathrm H_p(\pi_qB\to\pi_q(B\otimes_AB)\to\pi_q(B\otimes_AB\otimes_AB)\to\cdots)\Rightarrow\pi_*A_B.\]
\end{theorem}
\begin{proof}[Sketch]
	The associated graded piece $\op{gr}^n\coloneqq\op{cofiber}\left(\mathrm{fil}^{n+1}\to\mathrm{fil}^n\right)$ can be seen as the limit of the diagram
	% https://q.uiver.app/#q=WzAsNCxbMCwwLCJcXFNpZ21hXm5cXHBpX25CIl0sWzEsMCwiXFxTaWdtYV5uXFxwaV9uKEJcXG90aW1lc19BQikiXSxbMiwwLCJcXFNpZ21hXm5cXHBpX24oQlxcb3RpbWVzX0FCXFxvdGltZXNfQUIpIl0sWzMsMCwiXFxjZG90cyJdLFsxLDBdLFswLDEsIiIsMix7Im9mZnNldCI6M31dLFswLDEsIiIsMSx7Im9mZnNldCI6LTN9XSxbMiwxXSxbMSwyLCIiLDEseyJvZmZzZXQiOjN9XSxbMSwyLCIiLDEseyJvZmZzZXQiOi0zfV0sWzIsMSwiIiwxLHsib2Zmc2V0Ijo1fV0sWzIsMSwiIiwxLHsib2Zmc2V0IjotNX1dXQ==&macro_url=https%3A%2F%2Fraw.githubusercontent.com%2FdFoiler%2Fnotes%2Fmaster%2Fnir.tex
	\[\begin{tikzcd}[cramped]
		{\Sigma^n\pi_nB} & {\Sigma^n\pi_n(B\otimes_AB)} & {\Sigma^n\pi_n(B\otimes_AB\otimes_AB)} & \cdots
		\arrow[shift right=3, from=1-1, to=1-2]
		\arrow[shift left=3, from=1-1, to=1-2]
		\arrow[from=1-2, to=1-1]
		\arrow[shift right=3, from=1-2, to=1-3]
		\arrow[shift left=3, from=1-2, to=1-3]
		\arrow[from=1-3, to=1-2]
		\arrow[shift right=5, from=1-3, to=1-2]
		\arrow[shift left=5, from=1-3, to=1-2]
	\end{tikzcd}\]
	but now $\Sigma^n$ comes out, so we are computing $\Sigma^n$ of the $\ZZ$-module coming from the chain complex
	\[\pi_nB\to\pi_n(B\otimes_AB)\to\pi_n(B\otimes_AB\otimes_AB)\to\cdots,\]
	which is still the \v{C}ech resolution.
\end{proof}
For example, when $B=\FF_2$ and $A$ is the sphere, we can actually compute all these objects!
\begin{example}
	The canonical map $\mathbb S\to\FF_2$ produces a spectral sequence
	\[E_1^{pq}=\mathrm H_p\left(\pi_q\FF_2\to\pi_q(\FF_2\otimes_{\mathbb S}\FF_2)\to\pi_q(\FF_2\otimes_{\mathbb S}\FF_2\otimes_{\mathbb S}\FF_2)\to\cdots\right)\Rightarrow\pi_*\mathbb S_{\FF_2}.\]
	It turns out that this $E_1$ can be seen to be $\op{Ext}_{\mc A}(\FF_2,\FF_2)$, where $\mc A$ is the Steenrod algebra.
\end{example}

\subsection{Introducing Formal Groups}
It is difficult to compute $\op{Ext}_{\mc A}(\FF_2,\FF_2)$, so our next goal is to factor $\mathbb S\to\FF_2$ into a composite $\mathbb S\to E\to\FF_2$, and then one can use the Adams spectral sequence twice, thereby hopefully making the calculation easier (done in two steps). We are going to take $E=\mathrm{MU}$.

Informally, $\mathrm{MU}$ has the following data. (We will give a formal definition later.)
\begin{itemize}
	\item The objects are $0$-dimensional complex manifolds.
	\item The morphisms are $1$-dimensional complex bordisms with boundary.
	\item The $2$-morphisms are $2$-dimensinal complex bordisms of $1$-dimensional manifolds.
\end{itemize}
One then continues this process. The $\mathbb E_\infty$-ring structure arises from disjoint unions and products.

It will turn out that $\pi_*\mathrm{MU}$ (and the homotopy groups of the higher tensor powers) will be concentrated in even degrees, making it an even $\mathbb E_\infty$-ring. As such, we want to say something more about even $\mathbb E_\infty$-rings.
\begin{example}
	Recall from \Cref{ex:even-cp-inf-cohomology} that there is an isomorphism
	\[E^*\CP^\infty\cong(\pi_{-*}E)[[t]].\]
\end{example}
\begin{definition}[complex orientation]
	Fix an even $\mathbb E_\infty$-ring $E$. Then a \textit{complex orientation} of $E$ is a choice of isomorphism
	\[E^*\CP^\infty\to(\pi_{-*}E)[[t]].\]
\end{definition}
\begin{remark}
	Such an isomorphism amounts to a special choice of $t$. Because $\deg t=2$, this amounts to choice of map $t\colon\Sigma^\infty\CP^\infty\to\Sigma^2t$. In order to get an isomorphism, one requires that this map $t$ makes the composite
	\[S^2=\Sigma^\infty\CP^1\to\Sigma^\infty\stackrel t\to\Sigma^2E\]
	is the unit of $E$ (upon undoing $\Sigma^2$).
\end{remark}
It turns out that the Atiyah--Hirzebruch spectral sequence also collapses for the product $\CP^i\times\CP^j$. Running the same calculations for this product, if $E$ admits a complex orientation, then there is also a canonical isomorphism
\[E^*\left(\CP^\infty\times\CP^\infty\right)=(\pi_{-*}E)[[x,y]].\]
Thus, the multiplication on $\CP^\infty=K(\ZZ,2)$ gives rise to a multiplication
\[E^*\CP^\infty\to E^*(\CP^\infty\times\CP^\infty),\]
so we end up sending $t\in E^*\CP^\infty$ to some power series $f\in(\pi_{-*}E)[[x,y]]$.
\begin{example}
	For $E=\ZZ$, we showed that $f(x,y)=x+y$ on the homework by computing the Chern class of a tensor product.
\end{example}
\begin{example}
	If $E=\mathrm{KU}$, then there is a complex orientation for which $f(x,y)=x+y+\beta xy$.
\end{example}
This $f$ turns out to provide a formal group law in $(\pi_{-*}E)[[x,y]]$, once we have forgotten about the grading on $\pi_{-*}E$. Note that $E$ being even implies that $\pi_{-*}E$ is a commutative ring.
\begin{remark} \label{rem:get-formal-group}
	In general, $f$ satisfies $f(x,y)=f(y,x)$ and $f(x,f(y,z))=f(f(x,y),z)$. Indeed, these two identities follow because the multiplication on $\CP^\infty$ is commutative and associative. Furthermore,
	\[f(x,y)\equiv x+y\pmod{xy}\]
	because the multiplication on $\CP^\infty$ is unital. (Indeed, setting $y=0$ had better give $f(x,0)=x$ because the multiplication is unital, and similar for the other coordinate.)
\end{remark}
\begin{definition}[commutative formal group]
	Suppose that $R$ is a discrete commutative ring. Consider the affine formal scheme
	\[\widehat{\mathbb A}^1_R\coloneqq\colim_n\Spec R[t]/\left(t^n\right).\]
	Then a power series $f\in R[[x,y]]$ is a \textit{commutative formal group law} if and only if it satisfies the following.
	\begin{itemize}
		\item Unital: $f(x,y)\equiv x+y\pmod{xy}$.
		\item Commutative: $f(x,y)=f(y,z)$.
		\item Associative: $f(x,f(y,z))=f(f(x,y),z)$.
	\end{itemize}
\end{definition}
\begin{remark}
	Equivalently, we may view a formal group as a group object in the category of formal schemes, non-canonically isomorphic to $\widehat{\AA}^1_R$ with some group structure.
\end{remark}
\begin{example}
	\Cref{rem:get-formal-group} produces a canonical formal group law from an even $\mathbb E_\infty$-ring $E$ equip\-ped with a complex orientation. If we forget about the choice of complex orientation, then only the formal group $E^*\CP^\infty$ is canonical (not the law).
\end{example}
\begin{example}
	For any one-dimensional group scheme $G$ over $R$, the completion at the origin is a formal group $\widehat G$.
	\begin{itemize}
		\item Taking $G=\mathbb G_a$ gives the formal group law $f(x,y)=x+y$. Note that this comes from $E=\ZZ$.
		\item Taking $G=\mathbb G_m$ gives the formal group law $f(x,y)=x+y+xy$ because $(1+x)(1+y)=1+(x+y+xy)$. Note that this comes from $E=\mathrm{KU}$.
	\end{itemize}
\end{example}
Here is a more complicated example.
\begin{theorem}[Hopkins--Miller, Lurie] \label{thm:formal-group-from-elliptic-curve}
	Let $\mc C$ be the category of pairs $(A,\mc E)$ where $\mc E$ is an elliptic curve over the discrete ring $A$ which is a versal deformation over every geometric point. Then there is a functor from the discrete category $\mc C$ to the category of even $\mathbb E_\infty$-rings taking the pair $(A,\mc E)$ to an $\mathbb E_\infty$-ring $R$ for which $\pi_*R=A[\beta^\pm]$ and $\pi_0R=A$ and the associated formal group is $\widehat{\mc E}$.
\end{theorem}
Being a versal deformation is some kind of smoothness condition.
\begin{remark}
	There is an $\mathbb E_\infty$-ring $\mathrm{TMF}$ of topological modular forms. It turns out that the limit of the functor in \Cref{thm:formal-group-from-elliptic-curve} is $\mathrm{TMF}$.
\end{remark}
However, it turns out that there are formal groups which do not arise from even $\mathbb E_\infty$-rings.

\end{document}