% !TEX root = ../notes.tex

\documentclass[../notes.tex]{subfiles}

\begin{document}

\section{November 4}
Here we go.

\subsection{The Towers}
Let's do a calculation.
\begin{example}
	Let $F$ be the fiber of $\op{Sq}^1\colon\FF_2\to\Sigma\FF_2$. Then we show that $F$ is the spectrum $\ZZ/4\ZZ$.
\end{example}
\begin{proof}
	The homotopy long exact sequence reads
	\[\pi_{i+1}\FF_2\to\pi_{i}\FF_2\to\pi_iF\to\pi_i\FF_2\to\pi_{i-1}\FF_2\]
	Because the homotopy groups of $\FF_2$ are concentrated in degree $0$, we see that $F$ must have its homotopy groups concentrated in degree $0$, where it is some extension of $\FF_2$ by $\FF_2$. Thus, $F$ is an Eilenberg--MacLane space, so it is either $\ZZ/4\ZZ$ or $(\ZZ/2\ZZ)^2$.

	To determine which one, we note that $F\to\FF_2\to\Sigma\FF_2$ is a fiber sequence and therefore also a cofiber sequence. However, any map $(\ZZ/2\ZZ)^2\to\FF_2$ is either zero (which cannot be the case) or is surjective; in the second case, the projection $\FF_2^{\oplus2}\to\FF_2$ is the direct sum of the zero map and the identity to see that the corresponding cone $\FF_2\to\Sigma\FF_2$ is trivial. Thus, we must instead have $F$ in order to gain a nontrivial map $\FF_2\to\Sigma^2\FF_2$.
\end{proof}
\begin{example}
	Similarly, one can use the homotopy long exact sequence to see that the fiber $E$ of $\op{Sq}^8\colon\FF_2\to\Sigma^8\FF_2$ has
	\[E_n\left(\Delta^0\right)=\pi_n(E)=\begin{cases}
		\FF_2 & \text{if }i\in\{0,7\}, \\
		0 & \text{otherwise}.
	\end{cases}\]
	Similarly,
	\[E^n\left(\Delta^0\right)=\pi_{-n}(E)=\begin{cases}
		\FF_2 & \text{if }i\in\{0,-7\}, \\
		0 & \text{otherwise}.
	\end{cases}\]
\end{example}
\begin{example}
	Continue with the previous example. If $X$ is any space then we get a long exact sequence the fiber sequence
	\[\underline{\op{Hom}}(\Sigma_+^\infty X,E)\to\underline{\op{Hom}}(\Sigma_+^\infty X,\FF_2)\to\underline{\op{Hom}}(\Sigma_+^\infty X,\Sigma^8\FF_2),\]
	so taking the homotopy long exact sequence produces a long exact sequence
	\[\cdots\to E^n(X)\to\mathrm H^n(X;\FF_2)\to\mathrm H^{n+8}(X;\FF_2)\to E^{n+1}(X)\to\cdots.\]
\end{example}
The previous examples more or less tell us that we can try to compute some generalized cohomology if we can realize a spectrum as fibers of some Eilenberg--MacLane spaces.
\begin{proposition}
	Fix a spectrum $E$.
	\begin{listalph}
		\item The Whitehead tower: $E$ is the colimit of the diagram
		\[\cdots\to\tau_{\ge1}E\to\tau_{\ge0}E\to\tau_{\ge-1}E\to\cdots.\]
		\item The Postnikov tower: $E$ is the limit of the diagram
		\[\cdots\to\tau_{\le1}E\to\tau_{\le0}E\to\tau_{\le-1}E\to\cdots.\]
	\end{listalph}
\end{proposition}
\begin{remark}
	In the Whitehead tower, there are fiber sequences
	\[\tau_{\ge i+1}E\to\tau_{\ge i}\to\Sigma^{i}\pi_iE,\]
	which can be seen by calculating the long exact sequence from the cofiber. (Namely, one finds homotopy groups concentrated in a single degree.) Similarly, in the Postnikov tower, there are fiber sequences
	\[\Sigma^{i}\pi_{i}E\to\tau_{\le i}E\to\tau_{\le i-1}E.\]
\end{remark}
\begin{remark}
	Let's apply this. If $X$ is a space and $E$ is a spectrum, then $E\otimes\Sigma_+^\infty X$ is the colimit of the diagram
	\[\cdots\to\tau_{\ge1}E\otimes\Sigma_+^\infty X\to\tau_{\ge0}E\otimes\Sigma_+^\infty X\to\tau_{\ge-1}E\otimes\Sigma_+^\infty X\to\cdots\]
	because tensor products commute with colimits. Now, because tensor products commute with cofiber sequences, we get fiber sequences
	\[\tau_{\ge n}E\otimes\Sigma_+^\infty X\to\tau_{\ge n-1}E\otimes\Sigma_+^\infty X\to\Sigma^{n-1}\pi_{n-1}E\otimes\Sigma_+^\infty X.\]
	The homotopy group $\pi_i$ of the cofiber here is $\mathrm H_{i-n+1}(X;\pi_{n-1}E)$, which can be seen after being sufficiently careful with degree shifts.
\end{remark}

\subsection{The Atiyah--Hirzebruch Spectral Sequence}
The previous remark suggests that we may compute generalized homology as a sort of sequential colimit of long exact sequences. One can use this to prove the following.
\begin{theorem}[Atiyah--Hirzebruch] \label{thm:ah-ss}
	Fix a spectrum $E$ and a space $X$. Then there is a spectral sequence
	\[E^2_{pq}=\mathrm H_{p+q}(X;\pi_qE)\Rightarrow E_{p+q}(X).\]
\end{theorem}
We will derive this from the following generalization of the Serre spectral sequence by taking $F$ to be a point and $X\to B$ to be the identity.
\begin{theorem}[Atiyah--Hirzebruch] \label{thm:general-serre-ss}
	Fix a homotopy fiber sequence $F\to X\to B$ of pointed spaces, and suppose that $B$ is simply connected. Then for any spectrum $E$, there is a spectral sequence
	\[E^2_{pq}=\mathrm H_p(B;E_qF)\Rightarrow E_{p+q}X.\]
\end{theorem}
\begin{remark}
	There turns out to not be issue of convergence for this spectral sequence even though $E_qF$ may be supported in negative degrees. What saves us is that $\mathrm H_p(B;-)$ is supported only in nonnegative degrees, so there are only finitely many nonzero differentials out of any particular $E_{pq}$.
\end{remark}
\begin{proof}
	The fiber sequence tells us that $X$ is the colimit of some $B\to\mathrm{Spaces}_*$, where the second map goes to $F$. It follows that
	\begin{align*}
		E\otimes\Sigma_+^\infty X &= \colim_B\left(E\otimes\Sigma_+^\infty F\right) \\
		&= \colim_B\colim_n\tau_{\ge n}\left(E\otimes\Sigma_+^\infty F\right) \\
		&= \colim_n\colim_B\tau_{\ge n}\left(E\otimes\Sigma_+^\infty F\right).
	\end{align*}
	There is now a formal process to compute the cohomology of such a sequential colimit. Note that the colimit over $B$ collapses because it is simply connected: the cones in the Whitehead tower only have homotopy groups in a single degree, so the functor out of $B$ factors through some shifted embedding $\mathrm{Ab}\to\mathrm{Spectra}$, which is trivial.
\end{proof}
\begin{remark}
	Let's recall the formal process. Given a sequence
	\[\cdots\to A_1\to A_0\to A_{-1}\to\cdots,\]
	there is a spectral sequence
	\[\bigoplus_i\pi_n\op{cone}(A_i\to A_{i-1})\Rightarrow\pi_n\op{colim}A_i.\]
\end{remark}
\begin{remark}
	Technically, \Cref{thm:ah-ss} only follows from \Cref{thm:general-serre-ss} when $X$ is simply connected. However, one can rerun the same proof because $F$ is just a point: the corresponding functor out of $B$ is automatically factoring through abelian groups.
\end{remark}
Here is the corresponding statement for cohomology.
\begin{theorem}[Atiyah--Hirzebruch]
	Fix a homotopy fiber sequence $F\to X\to B$ of pointed spaces, and suppose that $B$ is simply connected. Then for any spectrum $E$, there is a spectral sequence with $E_2^{pq}=\mathrm H^p(B;E^qF)$. This converges to $E^{p+q}(X)$ if $X$ has finitely many homology nonzero groups or if $E$ has finitely many homotopy groups.
\end{theorem}
\begin{proof}
	Once again, the fiber sequence tells us that $X$ is the colimit of some $B\to\mathrm{Spaces}_*$, where the second map goes to $F$. It follows that
	\begin{align*}
		\underline{\op{Hom}}(\Sigma_+^\infty X,E) &= \underline{\op{Hom}}\left(\colim_B\Sigma_+^\infty F,E\right) \\
		&= \lim_B\underline{\op{Hom}}(\Sigma_+^\infty F,E) \\
		&= \lim_B\colim_n\tau_{\ge n}\underline{\op{Hom}}(\Sigma_+^\infty F,E).
	\end{align*}
	To apply the same formal process now produces a spectral sequence as soon as we figure out how to interchange the limit and colimit, which can be done under some conditions. Thus, we only sometimes receive cohomology at the end. The given convergence conditions arise by asking for some conditions to make the differentials (out of any particular box) of the resulting spectral sequence to vanish in sufficiently large pages.
	% There is now a formal process to compute the cohomology of such a sequential colimit. Note that the colimit over $B$ collapses because it is simply connected: the cones in the Whitehead tower only have homotopy groups in a single degree, so the functor out of $B$ factors through some shifted embedding $\mathrm{Ab}\to\mathrm{Spectra}$, which is trivial.
\end{proof}
\begin{remark}
	It turns out that we are allowed to interchange the limit and colimit always when $E$ is an abelian group, which explains the cohomological Serre spectral sequence.
\end{remark}

\end{document}