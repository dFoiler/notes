% !TEX root = ../notes.tex

\documentclass[../notes.tex]{subfiles}

\begin{document}

\section{November 18}
Reportedly, we did pretty well on the exam. There may or may not be two more homeworks, by popular demand.

\subsection{Modules}
There is a category of modules.
\begin{definition}
	Fix an $\mathbb E_1$-ring $R$. Then there is an $\infty$-category $\mathrm{Mod}_R$ of left $R$-modules. Approximately speaking, a left $R$-module is a spectrum $M$ equipped with an action map $a\colon R\otimes_{\mathbb S}M\to M$ and a choice of homotopy witnessing the commutativity of the square
	% https://q.uiver.app/#q=WzAsNCxbMCwwLCJSXFxvdGltZXNfe1xcbWF0aGJiIFN9Ulxcb3RpbWVzX3tcXG1hdGhiYiBTfU0iXSxbMSwwLCJSXFxvdGltZXNfe1xcbWF0aGJiIFN9TSJdLFswLDEsIlJcXG90aW1lc197XFxtYXRoYmIgU31NIl0sWzEsMSwiTSJdLFswLDEsIjFcXG90aW1lcyBhIl0sWzAsMiwiYVxcb3RpbWVzIDEiLDJdLFsyLDMsImEiXSxbMSwzLCJhIl1d&macro_url=https%3A%2F%2Fraw.githubusercontent.com%2FdFoiler%2Fnotes%2Fmaster%2Fnir.tex
	\[\begin{tikzcd}[cramped]
		{R\otimes_{\mathbb S}R\otimes_{\mathbb S}M} & {R\otimes_{\mathbb S}M} \\
		{R\otimes_{\mathbb S}M} & M
		\arrow["{1\otimes a}", from=1-1, to=1-2]
		\arrow["{a\otimes 1}"', from=1-1, to=2-1]
		\arrow["a", from=1-2, to=2-2]
		\arrow["a", from=2-1, to=2-2]
	\end{tikzcd}\]
	as well as higher coherences. The category $\mathrm{Mod}_R$ is stable.
\end{definition}
\begin{example}
	It turns out that $\mathrm{Spectra}$ is $\mathrm{Mod}_{\mathbb S}$.
\end{example}
\begin{example}
	It turns out that the derived category $\mc D(\ZZ)$ is $\mathrm{Mod}_{\mathbb Z}$. Here, the derived category $\mc D(\ZZ)$ is the category of chain complexes of abelian groups, localized along quasi-isomorphisms.
\end{example}
\begin{remark} \label{rem:free-forget-module}
	There is a functor $(-\otimes_{\mathbb S}R)\colon\mathrm{Mod}_{\mathbb S}\to\mathrm{Mod}_R$ which is left adjoint to the forgetful functor $\mathrm{Mod}_R\to\mathrm{Mod}_{\mathbb S}$.
\end{remark}
\begin{example} \label{ex:homs-in-dz}
	Let's compute homology groups in $\mc D(\ZZ)$. For a $\ZZ$-module $M$, a direct calculation shows $\mathrm H_n(M)=\pi_0\op{Hom}_{\mc D(\ZZ)}(\ZZ[n],M)$. On the other hand, $\pi_0\op{Hom}_{\mc D(\ZZ)}(\ZZ[n],M)$ is
	\[\pi_0\op{Hom}_{\mc D(\ZZ)}\left(\Sigma^n\ZZ[0],M\right)=\pi_0\op{Hom}_{\mc D(\ZZ)}\left(\ZZ\otimes_{\mathbb S}\Sigma^n\mathbb S,M\right)\]
	because $\Sigma$ passes through tensor products. By the adjunction of \Cref{rem:free-forget-module}, this last homotopy group is $\pi_0\underline{\op{Hom}}_{\mathbb S}(S^n,M)=\pi_nM$, where we are now viewing $M$ as its underlying spectrum.
\end{example}
% \begin{remark}
% 	If $\mc C$ is a stable category, then any object $c\in C$ produces a spectrum $\underline{\op{Hom}}_{\mc C}(c,c)$, and composition turns it into an $\mathbb E_1$-ring. Accordingly, to say that $M$ is an $R$-module is equivalent to exhibiting a map $R\to\underline{\op{Hom}}(M,M)$ of spectra.
% \end{remark}
\begin{example}
	The canonical map
	\[R\to\underline{\op{Hom}}_{\mathrm{Mod}_R}(R,R)\]
	is an isomorphism of $\mathbb E_1$-rings.
\end{example}
It turns out that we can realize ``nice'' stable categories as module categories, as one expects from abelian categories.
\begin{definition}[compact]
	An object $c$ in some $\infty$-category $\mc C$ is \textit{compact} if and only if $\underline{\op{Hom}}_{\mc C}(c,-)$ commutes with filtered colimits.
\end{definition}
\begin{theorem}[Schwede--Shipley]
	Fix a presentable stable category $\mc C$. If $\mc C$ admits a compact generator $c$, then $\mc C$ is equivalent to modules over $\underline{\op{Hom}}_{\mc C}(c,c)$.
\end{theorem}
We have not explained what ``presentable'' means, and we will not do so.
\begin{example}
	One can check that $\mc D(\ZZ)$ has a compact generator given by $\ZZ[0]$, and one can compute (via \Cref{ex:homs-in-dz}) that
	\[\underline{\op{Hom}}_{\mc D(\ZZ)}(\ZZ[0],\ZZ[0])\]
	is the spectrum $\ZZ$ because it has homotopy groups concentrated in $0$, where it is $\ZZ$. It follows that $\mc D(\ZZ)=\mathrm{Mod}_\ZZ$.
\end{example}
We may want to have general tensor products.
\begin{definition}
	Fix an $\mathbb E_\infty$-ring $R$. Given $R$-modules $A$ and $B$, we define $A\otimes_RB$ by
	\[\underline{\op{Hom}}_R(A\otimes_RB,-)=\underline{\op{Hom}}_R(A,\underline{\op{Hom}}_R(B,-)).\]
\end{definition}
\begin{remark}
	It turns out that the tensor product is well-defined and provides a symmetric monoidal structure on $\mathrm{Mod}_R$.
\end{remark}
\begin{remark}
	The tensor product allows us to enrich $R$-modules over themselves. In general, if $R$ is merely $\mathbb E_n$, then $\mathrm{Mod}_R$ is $\mathbb E_{n-1}$.
\end{remark}
\begin{example}
	If $R$ is a discrete commutative ring, then one can pass to the derived category to show that
	\[\op{Tor}_i^R(M,N)=\pi_i(M\otimes_RN)\qquad\text{and}\qquad\op{Ext}^i_R(M,N)=\pi_{-i}\underline{\op{Hom}}_R(M,N).\]
	(The tensor product here is derived.) For example, $\op{Tor}^\ZZ_1(\FF_2,\FF_2)=\pi_1(\FF_2\otimes_\ZZ\FF_2)$. Viewing $\FF_2$ as the kernel of $2\colon\ZZ\to\ZZ$, one can then compute as cohomology of a total complex as usual. Implicitly, we are using the fact that $\op{Ch}(R)\to\mc D(R)$ preserves tensor products with ``level-wise free'' chain complexes.
\end{example}
\begin{remark}
	Professor Jeremy Hahn is not aware of any non-derived tensor products of discrete $R$-modules of discrete commutative rings.
\end{remark}

\subsection{Skew Fields}
Having a module theory allows us to define fields.
\begin{definition}[free]
	Fix an $\mathbb E_1$-ring $R$. Then an $R$-module $M$ is \textit{free} if and only if it is isomorphic to an $R$-module of the form
	\[\bigoplus_{i\in I}\Sigma^{n_i}R\]
	for some index set $I$ and sequence of integers $\{n_i\}_{i\in I}$.
\end{definition}
\begin{definition}[skew field]
	An $\mathbb E_1$-ring $R$ is a \textit{skew field} if and only if every module is free.
\end{definition}
\begin{proposition} \label{prop:skew-field-on-pi-is-skew}
	Fix an Eilenberg--MacLane space $k$ corresponding to a discrete skew field. Then $k$ is a skew field.
\end{proposition}
\begin{proof}
	Fix a $k$-module $M$. Then $\pi_*M$ is a graded $\pi_*k$-module, which means that $\pi_*M$ is a graded vector space over the skew field $\pi_*k$. It follows that $\pi_*M$ is isomorphic to a direct sum of copies of $\pi_*k$, possibly with some shifts. Accordingly, we have an isomorphism
	\[\bigoplus_{i\in I}\pi_*k[n_i]\to\pi_*M.\]
	Each index $n_i$ now induces an element in the homotopy group $S^{n_i}\to M$, so we have a map
	\[\bigoplus_{i\in I}S^{n_i}\to M\]
	which hits a basis of $\pi_*M$ on taking homotopy. Using the free-forgetful adjunction, we get a map
	\[\bigoplus_{i\in I}k\otimes_{\mathbb S}\Sigma^{n_i}\mathbb S\to M\]
	of $k$-modules which is an isomorphism on homotopy groups. We conclude that $M$ is free.
\end{proof}
\begin{example}
	The same argument of \Cref{prop:skew-field-on-pi-is-skew} shows that any $\mathbb E_1$-ring $R$ is a skew field if $\pi_*R$ is a graded skew field.
\end{example}
\begin{example} \label{ex:ku-skew-field}
	There is an $\mathbb E_1$-ring $R=KU/p$ which has homotopy
	\[\pi_*R\cong\FF_p[\beta^{\pm}],\]
	where $\beta$ is in degree $2$. It turns out that every graded $\pi_*R$-module is free, so the argument of \Cref{prop:skew-field-on-pi-is-skew} applies and shows that $R$ is a skew field.
\end{example}
Skew fields have a good notion of characteristic.
\begin{definition}[characteristic]
	Fix skew fields $R_1$ and $R_2$. Then $R_1$ and $R_2$ have the same \textit{characteristic} if and only if $R_1\otimes_{\mathbb S}R_2\ne0$.
\end{definition}
\begin{remark} \label{rem:reflexive-characteristic}
	In general, for any $\mathbb E_1$-ring $R$, we have that $R\otimes_{\mathbb S}R\ne0$. Indeed, there is a nonzero composite
	\[\mathbb S=\mathbb S\otimes_{\mathbb S}\mathbb S\to R\otimes_{\mathbb S}R\stackrel m\to R\]
	which is the multiplicative unit. It follows that $R\otimes_{\mathbb S}R$ is nonzero. In particular, this shows that skew fields have the same characteristic as themselves.
\end{remark}
\begin{remark}
	Having the same characteristic is an equivalence relation. Symmetry has little content, and one sees that $R\otimes_{\mathbb S}R\ne0$ from \Cref{rem:reflexive-characteristic}. For the transitivity, if $R_1\otimes_{\mathbb S}R_2$ and $R_2\otimes_{\mathbb S}R_3$ are both nonzero, then one notes that $R_1\otimes_{\mathbb S}R_2$ is a free module over $R_2$, so
	\[(R_1\otimes_{\mathbb S}R_2)\otimes_{\mathbb S}R_3\ne0.\]
	By rearranging, this forces $R_1\otimes_{\mathbb S}R_3\ne0$.
\end{remark}
\begin{remark}
	It turns out that if $R_1$ and $R_2$ have the same characteristic, then for any space $X$, one has
	\[\op{rank}_{R_1^*\left(\Delta^0\right)}R_1^*(X)=\op{rank}_{R_2^*\left(\Delta^0\right)}R_2^*(X).\]
	This is supposed to follow by noting that $R_1\otimes_{\mathbb S}R_2$ is nonzero and free over both $R_1$ and $R_2$.
\end{remark}
\begin{example}
	Every discrete skew field has the same characteristic as $\QQ$ or as $\FF_p$ for some $p$.
\end{example}
\begin{example}
	Consider the skew field $KU/p$ of \Cref{ex:ku-skew-field}. Then we claim that $KU/p$ has different characteristic from $\QQ$ and from any $\FF_p$. Indeed, consider $\mathrm BC_p=K(\FF_p,1)$. Then the generalized cohomology group
	\[(KU/p)^*(\mathrm BC_p)\]
	is a module over $(KU/p)^*\left(\Delta^0\right)$ of dimension $p$ by considering some line bundles. On the other hand, calculations with the Serre spectral sequence can show that $\FF_p^*(\mathrm BC_p)$ is a vector space of infinite rank over $\FF_p$, and $\FF_\ell^*(\mathrm BC_p)$ is a vector space of rank $1$, and $\QQ^*(\mathrm BC_p)$ is also a vector space of rank $1$.
\end{example}
\begin{remark}
	There is still a classification of characteristics, though there are more of them. It turns out that any skew field $R$ has the same characteristic as $\QQ$ or $\FF_p$ for some $p$, or $\pi_0R$ has characteristic $p$ for some prime $p$. In the latter case, there is some $h$ for which $R^*(\mathrm BC_p)$ is free of rank $p^h$ over $R^*\left(\Delta^0\right)$, and this $h$ uniquely determines the characteristic of $R$.
\end{remark}

\subsection{New Rings from Old Ones}
We close class by noting that we have some of the usual ways to build new rings from old ones.
\begin{definition}[localization]
	Fix an $\mathbb E_\infty$-ring $R$, and choose some $x\in\pi_pR$. Then $x$ defines a map $x\colon S^p\to R$ of spectra, which by the free-forgetful adjunction is equivalent to providing a map $x\colon R\to\Sigma^{-p}R$. Then we define the \textit{localization} as the $R$-module
	\[R\left[x^{-1}\right]\coloneqq\colim\left(R\stackrel x\to\Sigma^{-p}R\stackrel x\to\Sigma^{-2p}R\to\cdots\right).\]
\end{definition}
\begin{remark}
	It turns out that $\pi_*\left(R\left[x^{-1}\right]\right)=(\pi_*R)\left[x^{-1}\right]$.
\end{remark}
\begin{remark}
	It is a theorem that $R\left[x^{-1}\right]$ is an $\mathbb E_\infty$-ring. One can define a multiplication by hand, but it is not easy to write down the higher coherences.
\end{remark}
In contrast, quotients may not produce rings.
\begin{definition}[quotient]
	Fix an $\mathbb E_\infty$-ring $R$, and choose some $x\in\pi_pR$. Then we define the \textit{quotient} $R/x$ as the cofiber of the induced map $x\colon\Sigma^pR\to R$. In other words, $R/x$ is the colimit of the diagram $0\from\Sigma^pR\stackrel x\to R$.
\end{definition}
\begin{remark}
	There is a long exact sequence
	\[\cdots\to\pi_*R\stackrel x\to\pi_{*+p}R\to\pi_{*+p}R/x\to\pi_{*+p-1}\stackrel x\to\pi_{*+2p-1}\to\cdots\]
	of homotopy groups.
\end{remark}
\begin{example}
	The quotient $\mathbb S/2$ is not a ring. Roughly speaking, the problem is that $\pi_*(\mathbb S/2)$ admits elements of order $4$.
\end{example}
\begin{example}
	The quotient $KU/p$ is an $\mathbb E_1$-ring, but it is not an $\mathbb E_\infty$-ring.
\end{example}
However, even though quotients may not produce rings, completions produce rings.
\begin{remark}
	In general, even though $R/x$ may merely be a module, $R/x^2$ is always an $\mathbb E_1$-ring, $R/x^3$ is always an $\mathbb E_2$-ring, and so on.
\end{remark}
\begin{definition}[completion]
	Fix an $\mathbb E_\infty$-ring $R$, and choose some $x\in\pi_pR$. Then we define the \textit{completion} as the module
	\[R^\land_x=\lim\left(R/x\from R/x^2\from R/x^3\from\cdots\right).\]
	It turns out that $R^\land_x$ is an $\mathbb E_\infty$-ring.
\end{definition}
\begin{remark}
	For completeness, we note that the map $R/x^2\to R/x$ is induced by universal property: one merely needs to show that the composite $\Sigma^{2p}R\stackrel{x^2}\to R\to R/x$ vanishes, which is true. Equivalently, this map is induced by the diagram
	% https://q.uiver.app/#q=WzAsNixbMCwwLCJcXFNpZ21hXnsycH1SIl0sWzEsMCwiUiJdLFsyLDAsIlIveF4yIl0sWzIsMSwiUi94Il0sWzEsMSwiUiJdLFswLDEsIlxcU2lnbWFecCJdLFswLDUsIngiLDJdLFswLDEsInheMiJdLFs1LDQsIngiXSxbMSw0LCIiLDEseyJsZXZlbCI6Miwic3R5bGUiOnsiaGVhZCI6eyJuYW1lIjoibm9uZSJ9fX1dLFsxLDJdLFs0LDNdLFsyLDMsIiIsMSx7InN0eWxlIjp7ImJvZHkiOnsibmFtZSI6ImRhc2hlZCJ9fX1dXQ==&macro_url=https%3A%2F%2Fraw.githubusercontent.com%2FdFoiler%2Fnotes%2Fmaster%2Fnir.tex
	\[\begin{tikzcd}[cramped]
		{\Sigma^{2p}R} & R & {R/x^2} \\
		{\Sigma^p} & R & {R/x}
		\arrow["{x^2}", from=1-1, to=1-2]
		\arrow["x"', from=1-1, to=2-1]
		\arrow[from=1-2, to=1-3]
		\arrow[equals, from=1-2, to=2-2]
		\arrow[dashed, from=1-3, to=2-3]
		\arrow["x", from=2-1, to=2-2]
		\arrow[from=2-2, to=2-3]
	\end{tikzcd}\]
	where we have implicitly recalled the colimit defining $R/x$ and $R/x^2$.
\end{remark}

\end{document}