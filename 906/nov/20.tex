% !TEX root = ../notes.tex

\documentclass[../notes.tex]{subfiles}

\begin{document}

\section{November 20}
Today we continue discussing rings.

\subsection{New Modules from Old Ones}
We can upgrade our ring constructions to module constructions.
\begin{definition}[quotient]
	Fix an $\mathbb E_\infty$-ring $R$, and choose some $x\in\pi_pR$, which we think of as a map $\Sigma^pR\to R$. Given an $R$-module $M$, we define the \textit{quotient module} $M/x$ as the cofiber of the induced map $x\colon \Sigma^pM\to M$.
\end{definition}
\begin{remark}
	By definition, we see that $M/x=R/x\otimes_RM$ because tensor products
\end{remark}
\begin{definition}[localization]
	Fix an $\mathbb E_\infty$-ring $R$, and choose some $x\in\pi_pR$, which we think of as a map $R\to\Sigma^{-p}R$. Given an $R$-module $M$, we define the \textit{localized module} $M\left[x^{-1}\right]$ as the sequential colimit
	\[M\left[x^{-1}\right]\coloneqq\colim\left(M\stackrel x\to\Sigma^{-p}M\stackrel x\to\Sigma^{-2p}M\to\cdots\right).\]
\end{definition}
\begin{remark}
	By definition, we see that $M\left[x^{-1}\right]=R\left[x^{-1}\right]\otimes_RM$.
\end{remark}
\begin{remark}
	Lurie shows that $R\left[x^{-1}\right]$ is an $\mathbb E_\infty$-ring by explicitly computing its module category and checking that it is symmetric monoidal.
\end{remark}
\begin{definition}[completion]
	Fix an $\mathbb E_\infty$-ring $R$, and choose some $x\in\pi_pR$, which we think of as a map $R\to\Sigma^{-p}R$. Given an $R$-module $M$, we define the \textit{completed module} $M^\land_x/x$ as the sequential limit
	\[M\left[x^{-1}\right]\coloneqq\colim\left(\cdots\to M/x^2\to M/x\right).\]
\end{definition}
\begin{remark}
	Because tensor products do not commute with limits, it is not in general true that $M^\land_x$ and $R_x^\land\otimes_RM$ are the same. Unsurprisingly, there are some finiteness conditions one could place on $M$ to make this statement true.
\end{remark}
\begin{example}
	One can construct $\mathrm{KU}$ as $\Sigma_+^\infty\CP^\infty\left[\beta^{-1}\right]$.
\end{example}

\subsection{Many Calculations}
Let's do some calculations.
\begin{example}
	Let's be more concrete. Suppose we want to compute $\pi_*(\FF_p\otimes_\ZZ\FF_p)$. View $\FF_p$ as the cofiber of $p\colon\ZZ\to\ZZ$. Then $\FF_p\otimes_\ZZ\FF_p$ is the cofiber of $p\colon\FF_p\to\FF_p$ by taking the tensor product with $\FF_p$, so we have a homotopy fiber sequence
	\[\FF_p\stackrel p\to\FF_p\to\FF_p\otimes_\ZZ\FF_p.\]
	Taking the long exact sequence, we find that $\pi_*(\FF_p\otimes_\ZZ\FF_p)$ is supported in degrees $0$ and $1$, where it is $\FF_p$. This sort of process allows us to compute $\op{Tor}_i^R(M,N)=\pi_i(M\otimes_RN)$.
\end{example}
\begin{example}
	Let's calculate $\underline{\op{Hom}}_\ZZ(\FF_p,\ZZ)$. Again, we see that $\FF_p$ is the cofiber of $p\colon\ZZ\to\ZZ$, so we are computing the fiber of $p\colon\underline{\op{Hom}}_\ZZ(\ZZ,\ZZ)\to\underline{\op{Hom}}_\ZZ(\ZZ,\ZZ)$. Now, $\underline{\op{Hom}}_\ZZ(\ZZ,\ZZ)$ is $\underline{\op{Hom}}_{\mathbb S}(\mathbb S,\ZZ)=\ZZ$, so we are computing the fiber of $p\colon\ZZ\to\ZZ$, which is the loops of the cofiber of $p\colon\ZZ\to\ZZ$. In total, we find that
	\[\underline{\op{Hom}}_\ZZ(\FF_p,\ZZ)=\Sigma^{-1}\FF_p.\]
	(A similar calculation will work for general modules.) For example, we see that there is a nontrivial map $\Sigma^{-1}\mathbb S\to\underline{\op{Hom}}_\ZZ(\FF_p,\ZZ)$, which is coming from the data of our map $\Sigma^{-1}\FF_p\to\ZZ$.
\end{example}
\begin{example}
	One can use the homotopy long exact sequence to see that the fiber of the canonical projection $\ZZ\to\FF_p$ is $\Sigma^{-1}\FF_p$. Accordingly, the cofiber is $\Sigma\FF_p$, so there is a ``Blockstein'' map $\FF_p\to\Sigma\FF_p$, whose cofiber can be calculated to be $\ZZ/p^2\ZZ$ by some calculation. (This is analogous to our calculation that $\ZZ/4\ZZ$ is the fiber of $\op{Sq}^2\colon\FF_2\to\Sigma\FF_2$.)
\end{example}
\begin{remark}
	In general, one does not expect the co/fibers of maps $\FF_2\to\Sigma^n\FF_2$ to be discrete abelian groups when $n\notin\{0,1\}$. For example, $\op{Sq}^3\colon\FF_2\to\Sigma^3\FF_2$ is not a morphism of $\ZZ$-modules (even though it is a morphism of $\mathbb S$-modules). Indeed, one can see this because $\ZZ$-module morphisms will be required to have very constrained homotopy groups.
\end{remark}
We can also derive some important examples.
\begin{example} \label{ex:spectra-uct}
	Cohomology is dual to homology. Work over a skew field $k$ for sanity. For any spectrum $X$, we know that $k\otimes_{\mathbb S}X$ is free over $k$ (because $k$ is a skew field), so any choice of basis $\mc B$ of $\mathrm H_*(X;k)$, we find that
	\[X\otimes_{\mathbb S}k=\bigoplus_{x\in\mc B}\Sigma^{-\deg x}k.\]
	Thus,
	\[\underline{\op{Hom}}_{\mathbb S}(X,k)=\underline{\op{Hom}}_{k}(k\otimes_{\mathbb S}X,k),\]
	so upon taking homotopy, we see that cohomology is the linear dual of homology.
\end{example}
\begin{remark}
	Similarly, it turns out that the homotopy groups determine a $\ZZ$-module, and this is true for any principal ideal domain.
\end{remark}
\begin{example}
	There is a K\"unneth theorem. Indeed, the point is that, for any field $k$, we have
	\[k\otimes_{\mathbb S}X\otimes_{\mathbb S}Y=(k\otimes_{\mathbb S}X)\otimes_{k}(k\otimes_{\mathbb S}Y),\]
	so we get the K\"unneth theorem by taking homotopy. Indeed, the point is that $k\otimes_{\mathbb S}X$ expands into a sum of $k$s over a basis of the homology (as in \Cref{ex:spectra-uct}), so we just need to take tensor products of our bases.
\end{example}

\subsection{A Quotient but Not a Ring}
For our next few examples, we study $\mathbb S/2$.
\begin{example}
	Let's compute the cohomology $\mathrm H^*(\mathbb S/2;\FF_2)=\pi_{-*}\underline{\op{Hom}}_{\mathbb S}(\mathbb S/2,\FF_2)$. Now, $\mathbb S/2$ is the cofiber of $2\colon\mathbb S\to\mathbb S$, so we see that we are interested in the homotopy groups of the fiber of
	\[2\colon\underbrace{\underline{\op{Hom}}_{\mathbb S}(\mathbb S,\FF_2)}_{\mathbb F_2}\to\underbrace{\underline{\op{Hom}}_{\mathbb S}(\mathbb S,\FF_2)}_{\FF_2}.\]
	Thus, $\mathrm H^*(\mathbb S/2;\FF_2)$ is supported in degrees $\{0,1\}$, where it is $\FF_2$. It turns out that $\op{Sq}^1\colon\mathrm H^0(\mathbb S/2;\FF_2)\to\mathrm H^1(\mathbb S/2;\FF_2)$ is an isomorphism, which follows by noting that the unit maps provide a morphism
	% https://q.uiver.app/#q=WzAsNixbMSwwLCJcXG1hdGhiYiBTIl0sWzEsMSwiXFxaWiJdLFsyLDAsIlxcbWF0aGJiIFMiXSxbMiwxLCJcXFpaIl0sWzAsMSwiXFxTaWdtYV57LTF9XFxGRl8yIl0sWzAsMCwiXFxTaWdtYV57LTF9XFxtYXRoYmIgUy8yIl0sWzAsMiwiMiJdLFsxLDMsIjIiXSxbNCwxXSxbNSw0XSxbMCwxXSxbNSwwXSxbMiwzXV0=&macro_url=https%3A%2F%2Fraw.githubusercontent.com%2FdFoiler%2Fnotes%2Fmaster%2Fnir.tex
	\[\begin{tikzcd}[cramped]
		{\Sigma^{-1}\mathbb S/2} & {\mathbb S} & {\mathbb S} \\
		{\Sigma^{-1}\FF_2} & \ZZ & \ZZ
		\arrow[from=1-1, to=1-2]
		\arrow[from=1-1, to=2-1]
		\arrow["2", from=1-2, to=1-3]
		\arrow[from=1-2, to=2-2]
		\arrow[from=1-3, to=2-3]
		\arrow[from=2-1, to=2-2]
		\arrow["2", from=2-2, to=2-3]
	\end{tikzcd}\]
	of short exact sequences, where the induced left map comes from the nontrivial element in $\mathrm H^0(\mathbb S/2;\FF_2)$. However, $\op{Sq}^1\colon\Sigma^{-1}\FF_2\to\FF_2$ arises from the projection $\ZZ\to\FF_2$, so the claim follows.
\end{example}
\begin{example}
	Using the K\"unneth theorem (and taking the dual to compute cohomology), we find that $\mathrm H^*(\mathbb S/2\otimes_{\mathbb S}\mathbb S/2;\FF_2)$ is four-dimensional, and it is
	\[\mathrm H^*(\mathbb S/2;\FF_2)\otimes_{\FF_2}\mathrm H^*(\mathbb S/2;\FF_2).\]
	Let $x$ and $y$ be the generators of $\mathrm H^*(\mathbb S/2;\FF_2)$ in degrees $0$ and $1$, respectively. Then we know $\op{Sq}^1x=y$, so $\op{Sq}^1(x\otimes y)=y\otimes y$ and $\op{Sq}^1(y\otimes x)=y\otimes y$. One can use the Cartan formula to compute
	\[\op{Sq}^2(x\otimes x)=\op{Sq}^0x\otimes\op{Sq}^2x+\op{Sq}^1x\otimes\op{Sq}^1x+\op{Sq}^2x\otimes\op{Sq}^0x=y\otimes y.\]
\end{example}
\begin{example}
	We show that $\mathbb S/2$ is not a unital ring. Indeed, this would provide us with a canonical map $\mathbb S\to\mathbb S/2$ (providing the unit), which then would induce an $\mathbb S/2$-module composite
	\[\mathbb S/2=\mathbb S/2\otimes_{\mathbb S}S\to\mathbb S/2\otimes_{\mathbb S}\mathbb S/2\stackrel m\to\mathbb S/2,\]
	so it follows that $\mathbb S/2$ is a retract of $\mathbb S/2\otimes_{\mathbb S}\mathbb S/2$. Thus, we see that $\mathrm H^*(\mathbb S/2;\FF_2)$ needs to be a retract of $\mathrm H^*(\mathbb S/2\otimes_{\mathbb S}\mathbb S/2;\FF_2)$, but this is not possible by considering the calculations of $\op{Sq}^2$ above. Indeed, this would imply that we have an embedding
	\[\mathrm H^*(\mathbb S/2;\FF_2)\into\mathrm H^*(\mathbb S/2\otimes_{\mathbb S}\mathbb S/2;\FF_2)\]
	which preserves the grading (and the Steenrod operations), but $\mathrm{Sq}^2$ vanishes on $\mathrm H^*(\mathbb S/2;\FF_2)$ and is nonzero on $\mathrm H^*(\mathbb S/2\otimes_{\mathbb S}\mathbb S/2;\FF_2)$.
\end{example}
\begin{remark}
	Alternatively, one can compute the long exact sequence
	\[\pi_*\mathbb S\stackrel2\to\pi_*\mathbb S\to\pi_*\mathbb S/2\to\pi_{*-1}\mathbb S\stackrel2\to\pi_{*-1}\mathbb S.\]
	It turns out that one gets $4$-torsion in the middle, which would not be possible if $\pi_*\mathbb S/2$ were actually a (graded) ring.
\end{remark}

\subsection{The Dual Steenrod Algebra}
Let's next say something about the Steenrod algebra.
\begin{remark}
	The Steenrod algebra $\mc A^*$ is $\pi_{-*}\underline{\op{Hom}}_{\mathbb S}(\FF_2,\FF_2)$. This implies that $\underline{\op{Hom}}_{\mathbb S}(\FF_2,\FF_2)$ is not an $\mathbb E_2$-ring because its homotopy groups are not commutative. However, it is still an $\mathbb E_1$-ring, where the multiplicative structure is given by composition.
\end{remark}
\begin{definition}[dual Steenrod operation]
	The \textit{dual Steenrod algebra} is
	\[\mc A_*\coloneqq\pi_*(\FF_2\otimes_{\mathbb S}\FF_2).\]
\end{definition}
\begin{remark}
	As before, we find that the Steenrod algebra $\mc A^*$ is
	\[\pi_{-*}\underline{\op{Hom}}_{\mathbb S}(\FF_2,\FF_2)=\pi_{-*}\underline{\op{Hom}}_{\mathbb F_2}(\FF_2\otimes_{\mathbb S}\FF_2,\FF_2),\]
	so $\mc A$ is the linear dual of $\mc A_*$. Because $\mc A$ has countable dimension and is finite-dimensional in each degree, we find that $\mc A_*$ is also finite-dimensional in each degree, so $\mc A_*$ is also the linear dual of $\mc A$.
\end{remark}
\begin{remark}
	There is a multiplication map $\FF_2\otimes_{\mathbb S}\FF_2$ which on homotopy gives rise to a multiplication
	\[\mc A_*\otimes_{\FF_2}\mathbb A_*\to\mc A_*.\]
	Taking the dual needs to give rise to a ``comultiplication''
	\[\mc A^*\to\mc A^*\otimes_{\FF_2}\mc A^*.\]
	It turns out that this is a graded associative comultiplication (so we get a Hopf algebra), so there are not so many options for what this map can be. It is given by $\op{Sq}^n\mapsto\sum_{i+j=n}\mathrm{Sq}^i\otimes\mathrm{Sq}^k$. (This is due to Milnor, and it is roughly equivalent to the Cartan formula for the Steenrod squares.)
\end{remark}
\begin{theorem}[Milnor]
	The dual Steenrod algebra $\mc A_*$ is the free polynomial ring on generators $\xi_i$ for $i\ge1$, where $\deg\xi_i=2^i-1$.
\end{theorem}
This is basically equivalent to unwinding the Cartan formula.
\begin{remark}
	The element $\xi_1$ is dual to $\op{Sq}^1$. It then turns out that $\xi_2$ is dual to $\left[\mathrm{Sq}^2,\mathrm{Sq}^1\right]$, and $\xi_4$ is dual to $\left[\mathrm{Sq}^4,\left[\mathrm{Sq}^2,\mathrm{Sq}^1\right]\right]$. This is weird because it needs to encode the Adem relations. It turns out that these commutators tend to be important cohomology operations, and this explains why.
\end{remark}
\begin{example}
	The spectrum $\tau_{\le2}\tau_{\ge0}\mathrm{KU}/2$ has homotopy groups only in degrees $\{0,2\}$, where it is $\FF_2$. Thus, $\mathrm{KU}/2$ is the fiber of some map $\FF_2\to\Sigma^3\FF_2$ using the towers: one has a map to $\tau_{\le0}\tau_{\ge0}\mathrm{KU}/2$, which is $\FF_2$, and the fiber is concentrated in degree $3$. One can ask what the map $\FF_2\to\Sigma^3\FF_2$ is, and it turns out that it is $\left[\mathrm{Sq}^1,\mathrm{Sq}^2\right]$.
\end{example}
\begin{example}
	If we want to compute the stable homotopy groups of spheres, then we should compute the homotopy groups of $\mathbb S$. Outside degree $0$, these homotopy groups are finite, so we may as well complete at the various primes $p$. Let's start with the prime $2$, which is already hard. To approach this, we may let $\mc I$ be the fiber of the canonical map $\mathbb S_2^\land\to\FF_2$. Then we could hope to build a filtration by powers of $\mc I$.
\end{example}
% \begin{remark}
% 	Here is another way to argue that $\mc A_*$ is finite-dimensional in each degree. There is a construction of Eilenberg--MacLane spaces via the Dold--Kan correspondence, which tells us that they have finitely many non-degenerate simplices in each dimension. It follows that $\mathrm H_*(\FF_2;\FF_2)$ is finite-dimensional in each degree, which means that $\pi_*(\FF_2\otimes_{\mathbb S}\FF_2)$ in each degree.
% \end{remark}

\end{document}