% !TEX root = ../notes.tex

\documentclass[../notes.tex]{subfiles}

\begin{document}

\section{November 6}
Next class is the exam. Please show up on time. The exam will feature a Serre spectral sequence, so here are some things to know.
\begin{itemize}
	\item There is a cohomological Serre spectral sequence for a homotopy fiber sequence $F\to E\to B$ which looks like
	\[E_2^{pq}=\mathrm H^p(B;\mathrm H^q(F:\FF_2))\Rightarrow\mathrm H^{p+q}(E;\FF_2).\]
	\item If $F$ is connected, then there are interpretations via transgressions for the edge maps with respect to the natural maps $\mathrm H^*(B;\FF_2)\to\mathrm H^*(E;\FF_2)$ and $\mathrm H^*(E;\FF_2)\to\mathrm H^*(F;\FF_2)$.
	\item One should know what $\op{Sq}^i\colon K(\FF_2,i)\to K(\FF_2,2i)$ means.
	\item One should know something about the cohomology of $\RP^\infty$.
\end{itemize}
\begin{remark}
	Professor Jeremy Hahn does not know where the quadrants are. He is waiting until his child learns to tell him.
\end{remark}

\subsection{\texorpdfstring{$\mathbb E_1$}{ E1}-Rings}
Here is our definition.
\begin{definition}[homotopy ring]
	A spectrum $E$ is an \textit{$\mathbb E_1$-ring} or a \textit{homotopy ring} if and only if it has an associative map $E\otimes_{\mathbb S}E\to E$. If the multiplication is further commutative, then $E$ is an $\mathbb E_\infty$-ring.
\end{definition}
\begin{remark}
	In particular, distributivity is automatic because the multiplication map is defined from the tensor product.
\end{remark}
\begin{remark}
	As usual, ``associative'' and ``commutative'' have higher coherences.
\end{remark}
\begin{example}
	Any discrete associative ring provides an $\mathbb E_1$-ring. Any discrete commutative ring provides an $\mathbb E_\infty$-ring.
\end{example}
\begin{remark}
	Fix $E$ an $\mathbb E_1$-ring. For any space $X$, one can multiply two maps $f\colon\Sigma^\infty_+X\to\Sigma^iE$ and $g\colon\Sigma^\infty_+X\to\Sigma^jE$ by letting $f\otimes g$ be the composite
	\[\Sigma_+^\infty X\stackrel\Delta\to\Sigma_+^\infty(X\times X)=\Sigma_+^\infty X\otimes_{\mathbb S}\Sigma_+^\infty X\to\Sigma^iX\otimes_{\mathbb S}\Sigma^jE\to\Sigma^{i+j}E.\]
	This provides a cup product (i.e., ring) structure on $E^*(X)$. However, this product (a priori) has no reason to be associative, commutative, or unital.
\end{remark}
\begin{remark}
	It turns out that $\mathbb S$ is the initial $\mathbb E_1$-ring (and the initial $\mathbb E_\infty$-ring). This makes $\mathbb S$ analogous to $\ZZ$. Notably, the groupoid of finite sets (with bijections) is some $\mathbb E_\infty$-space, and its groupoid completion turns out to be $\Omega^\infty\mathbb S$. Thus, we can view counting as ``accidentally'' replacing finite sets by the discrete category, thereby losing access to the interesting structure $\mathbb S$!
\end{remark}
It will be useful to add some properties to our rings.
\begin{definition}[homotopy unital]
	A homotopy ring $E$ is \textit{homotopy unital} if and only if there is a map $\mathbb S\to E$ such that there the maps
	\[E=E\otimes\mathbb S\to E\otimes E\to E\qquad\text{and}\qquad E=\mathbb S\otimes E\to E\otimes E\to E\]
	are both homotopic to the identity.
\end{definition}
\begin{remark}
	It may seem evil that we are not specifying the homotopies to the identity in the definition. In general, it turns out that the property of the existence of some identity makes the corresponding homotopy unique up to contractible choice.
\end{remark}
\begin{definition}[homotopy associative]
	A homotopy ring $E$ is \textit{homotopy associative} if and only if the diagram
	% https://q.uiver.app/#q=WzAsNCxbMCwwLCJFXFxvdGltZXMgRVxcb3RpbWVzIEUiXSxbMSwwLCJFXFxvdGltZXMgRSJdLFswLDEsIkVcXG90aW1lcyBFIl0sWzEsMSwiRSJdLFswLDEsIm1cXG90aW1lczEiXSxbMCwyLCIxXFxvdGltZXMgbSIsMl0sWzIsMywibSJdLFsxLDMsIm0iXV0=&macro_url=https%3A%2F%2Fraw.githubusercontent.com%2FdFoiler%2Fnotes%2Fmaster%2Fnir.tex
	\[\begin{tikzcd}[cramped]
		{E\otimes E\otimes E} & {E\otimes E} \\
		{E\otimes E} & E
		\arrow["{m\otimes1}", from=1-1, to=1-2]
		\arrow["{1\otimes m}"', from=1-1, to=2-1]
		\arrow["m", from=1-2, to=2-2]
		\arrow["m", from=2-1, to=2-2]
	\end{tikzcd}\]
	commutes up to homotopy.
\end{definition}
\begin{defihelper}[$\mathbb E_1$-ring, $\mathbb E_\infty$-ring] \nirindex{E1 ring@$\mathbb E_1$-ring} \nirindex{Einf ring@$\mathbb E_\infty$-ring}
	An $\mathbb E_1$-ring is a homotopy associative and unital ring $E$ equipped with specified homotopies for the various associators. An $\mathbb E_\infty$-ring is an $\mathbb E_1$-ring where the multiplication commutes (with higher compatibilities, as usual).
\end{defihelper}
\begin{remark}
	Here is another way to view the extra associativity conditions on an $\mathbb E_1$-ring: if $E$ is an $\mathbb E_1$-ring, then $\Omega^\infty E$ is a group-like $\mathbb E_\infty$-space under the additive structure, and it separately an $\mathbb E_1$-space under the multiplicative structure (which has no reason to be group-like or $\mathbb E_\infty$).
\end{remark}
\begin{remark}
	Similarly, when $E$ is an $\mathbb E_\infty$-ring, then the multiplication structure grants $\Omega^\infty$ a second structure of an $\mathbb E_\infty$-space.
\end{remark}
\begin{remark}
	If $R$ is an $\mathbb E_1$-ring, then $\pi_*R$ is an associative ring because this is just the cohomology ring $R^{-*}\Delta^0$. Similarly, if $R$ is an $\mathbb E_\infty$-ring, then $\pi_*R$ is further (graded) commutative. One finds again that the generalized cohomology $R^*(X)$ becomes graded commutative when $R$ is an $\mathbb E_\infty$-ring.
\end{remark}
\begin{example}
	Any Eilenberg--MacLane spectrum (from a ring) is an $\mathbb E_\infty$-ring.
\end{example}
\begin{example}
	The sphere spectrum $\mathbb S$ is the initial $\mathbb E_\infty$-ring; it is the free group-like $\mathbb E_\infty$-space on a point. On the other hand, it turns out that $\mathrm{FinSet}$ (with morphsims which are bijections) is the free $\mathbb E_\infty$-space on a point, so $\mathbb S$ is the group completion of $\mathrm{FinSet}$. (In theory, this could let us access the homotopy groups of spheres using difficult combinatorics of finite sets.) For example, this implies that the homotopy groups of spheres forms an associative graded commutative ring. As an example, $\pi_0\mathbb S=\ZZ$, and $\pi_1\mathbb S=\ZZ/2\ZZ\eta$ and $\pi_2\mathbb S=\ZZ/2\ZZ\eta^2$. In general, it turns out that if $x\in\pi_{2i+1}\mathbb S$, then $x^2=-x^2$, so $x^2\in\pi_{4i+2}\mathbb S[2]$.
\end{example}
\begin{example}
	There is an $\mathbb E_\infty$ ring $KU$, where $KU$ is as in \Cref{ex:ku-spectra}. It turns out that $\tau_{\ge0}KU$ is the group completion of complex vector spaces whose operations come from $\oplus$ and $\otimes$; this explains some of the $\mathbb E_\infty$-ring structure. For example,
	\[KU=(\mathrm{BU}\times\ZZ,\mathrm U,\mathrm{BU}\times\ZZ,\mathrm U,\ldots),\]
	so $\Omega^2KU=KU$. In particular, $KU$ has $2$-periodic homotopy groups. As such, the calculations $\pi_0KU=\pi_0(\mathrm{BU}\times\ZZ)=\ZZ$ and $\pi_1KU=\pi_1(\mathrm{BU}\times\ZZ)=0$. It follows that $\pi_*KU$ is the ring $\ZZ[\beta^\pm]$, where $\beta$ lives in degree $2$.
\end{example}
There is also some module theory, which we will not say much about, but we give two examples.
\begin{example}
	The category of $\mathbb S$-modules is $\mathrm{Spectra}$.
\end{example}
\begin{example}
	The category of $\ZZ$-modules is $\mc D(\ZZ)$.
\end{example}
The point is that a category with a compact generator turns out to be modules over the endomorphism ring of that generator. In our $\infty$-categorical setting, this is called the Schwede--Shipley theorem.

\subsection{Even Cohomology}
We now give many calculations.
\begin{example} \label{ex:ku-of-cpn}
	Let's do a calculation. Note that $\CP^n$ has finitely many homology groups, so there is a spectral sequence
	\[E_2^{pq}=\mathrm H^p\left(\CP^n;KU^q(\Delta^0)\right)\Rightarrow KU^{p+q}(\CP^n).\]
	The left-hand side can be computed to be $\ZZ[\beta^\pm][t]/\left(t^{n+1}\right)$, where $t\in\mathrm H^p(\CP^n;\ZZ)$ lives in bidegree $(2,0)$, and $\beta\in\pi_2\Delta^0$ is in $KU^{-2}(\Delta^0)$ and therefore bidegree $(0,-2)$. It follows that all differentials vanish automatically for parity reasons, so we gain an isomorphism of rings
	\[KU^*(\CP^n)=\frac{\ZZ[\beta^\pm][t]}{\left(t^{n+1}\right)}.\]
\end{example}
The above calculation had its key input come from parity considerations, so we make the following convenient definition.
\begin{definition}[even]
	An $\mathbb E_\infty$-ring $R$ is \textit{even} if and only if $\pi_*R$ vanishes in odd degrees.
\end{definition}
\begin{example}
	All Eilenberg--MacLane spectra from rings are even $\mathbb E_\infty$-rings.
\end{example}
\begin{nex}
	The sphere spectrum $\mathbb S$ is not even. For example, $\pi_1\mathbb S$ is nonzero.
\end{nex}
\begin{example} \label{ex:even-cpn}
	The same calculation of \Cref{ex:ku-of-cpn} proves a ring isomorphism
	\[E^*(\CP^n)\cong\frac{\pi_{-*}E[t]}{\left(t^{n+1}\right)}.\]
\end{example}
\begin{remark}
	Heuristically, one expects generalized cohomology of $\mathbb E_\infty$-rings to be about as comparable as computing with the usual Eilenberg--MacLane spectra.
\end{remark}
Computing homotopy groups of colimits is not so bad.
\begin{lemma}
	Fix a sequence
	\[\cdots\to A_1\to A_0\to A_{-1}\to A_{-2}\to\cdots\]
	of spectra. Then $\pi_i(\colim A_\bullet)=\colim\pi_i(A_\bullet)$ for any $i$.
\end{lemma}
\begin{proof}[Sketch]
	This follows because $\mathbb S$ is a compact object, meaning that maps out of $\mathbb S$ commute with sequential colimits. The claim follows.
\end{proof}
\begin{remark}
	Of course, there is an analogous statement for filtered colimits.
\end{remark}
For limits, we need some kind of Mittag-Leffler condition.
\begin{lemma}[Milnor exact sequence] \label{lem:milnor-ses}
	Fix a sequence
	\[\cdots\to A_1\to A_0\to A_{-1}\to A_{-2}\to\cdots\]
	of spectra. Then for any $i$, there is an exact sequence
	\[0\to{\lim}^1\pi_{i+1}A_\bullet\to\pi_i(\lim A_\bullet)\to\lim\pi_iA_\bullet\to0.\]
	For example, if the maps $\pi_{i+1}A_n\to\pi_{i+1}A_{n-1}$ are surjective for sufficiently large $n$, then $\lim^1\pi_{i+1}A_\bullet=0$.
\end{lemma}
Here, $\lim^1$ is the first right-derived functor of the limit functor.
\begin{proof}[Sketch]
	Let $s\colon\prod_iA_i\to\prod_iA_i$ be the shifting map. Then $\lim A_i$ is the pullback of the diagram
	% https://q.uiver.app/#q=WzAsMyxbMCwxLCJcXGRpc3BsYXlzdHlsZVxccHJvZF9pQV9pIl0sWzEsMSwiXFxkaXNwbGF5c3R5bGVcXHByb2RfaShBX2lcXHRpbWVzIEFfaSkiXSxbMSwwLCJcXGRpc3BsYXlzdHlsZVxccHJvZF9pQV9pIl0sWzAsMSwiKHtcXGlkfSxzKSJdLFsyLDEsIih7XFxpZH0se1xcaWR9KSJdXQ==&macro_url=https%3A%2F%2Fraw.githubusercontent.com%2FdFoiler%2Fnotes%2Fmaster%2Fnir.tex
	\[\begin{tikzcd}[cramped]
		& {\displaystyle\prod_iA_i} \\
		{\displaystyle\prod_iA_i} & {\displaystyle\prod_i(A_i\times A_i)}
		\arrow["{({\id},{\id})}", from=1-2, to=2-2]
		\arrow["{({\id},s)}", from=2-1, to=2-2]
	\end{tikzcd}\]
	from which the result follows by using the long exact sequence of a pullback.
\end{proof}
\begin{example}
	Fix an even $\mathbb E_\infty$-ring $E$. Then there is an isomorphism of graded rings
	\[E^*(\CP^\infty)=\pi_{-*}E[[t]],\]
	where $t$ lives in degree $2$. Importantly, this is a limit of graded power series rings, meaning that the limit is taken in each graded piece separately.
\end{example}
\begin{proof}
	Note that $\CP^\infty$ is the colimit of $\CP^n$. Accordingly, for any even $\mathbb E_\infty$-ring $E$, we find that
	\begin{align*}
		E^k\CP^\infty &= \pi_0\underline{\op{Hom}}(\CP^\infty,\Sigma^kE) \\
		&= \pi_0\underline{\op{Hom}}(\colim\CP^n,\Sigma^kE) \\
		&= \pi_0\lim\underline{\op{Hom}}(\CP^n,\Sigma^*E).
	\end{align*}
	We would like to pass $\pi_0$ through the limit, which follows because the internal maps
	\[E^{k+1}(\CP^n)\to E^k(\CP^n)\]
	is surjective by the calculation of these cohomology groups of \Cref{ex:even-cpn}. It follows that
	\[E^*\CP^\infty=\lim\frac{E^k(\CP^n)[t]}{\left(t^{n+1}\right)},\]
	so we are done.
\end{proof}
\begin{example}
	If $E$ is $\ZZ$, then we find $E^*\CP^\infty=\ZZ[[t]]$ where $t$ lives in degree $2$. It does follow that this is the same as $\ZZ[t]$, as we can see by computing the limit in each graded piece.
\end{example}
\begin{example}
	If $E=KU$, then $KU^0(\CP^\infty)$ is $\ZZ[[\beta t]]$, where now this is an honest power series ring because the limit at the $0$th graded piece is now doing something interesting! Indeed, $KU^0(\CP^\infty)$ is
	\[\lim\left(\frac{\ZZ[\beta^\pm,t]}{\left(t^{n+1}\right)}\right)_0=\lim\frac{\ZZ[\beta t]}{\left((\beta t)^{n+1}\right)}.\]
\end{example}

\subsection{Line Bundles}
Let's give a line bundle example.
\begin{definition}[line bundle]
	Fix a space $X$. Then a \textit{line bundle} is a map $\mc L\colon X\to\op{BU}(1)$.
\end{definition}
\begin{remark}
	Recall that $\op{BU}(1)\cong\CP^\infty$. As such, the multiplicative structure on $K(\ZZ,2)$ allows us to take the tensor product of line bundles.
\end{remark}
\begin{definition}[Chern class]
	Fix a space $X$. Given a line bundle $\mc L\colon X\to\CP^\infty$, we define the \textit{first Chern class} $c_1(\mc L)\in E^2(X)$ to be the image of the canonical generator along the map
	\[\mc L\colon E^2(\CP^\infty)\to E^2(X).\]
\end{definition}
\begin{remark}
	If $\mc L_1$ and $\mc L_2$ are line bundles, then one can use the construction of the tensor product (and the multiplication on $\CP^\infty$) to show that there is a formula for $c_1(\mc L_1\otimes\mc L_2)$ in terms of $c_1(\mc L_1)$ and $c_1(\mc L_2)$. In general, this formula is a formal group law!
\end{remark}
\begin{example}
	If $E=KU$, then one finds
	\[c_1(\mc L_1\otimes\mc L_2)=c_1(\mc L_1)+c_1(\mc L_2)+\beta c_1(\mc L_1)c_2(\mc L_2).\]
\end{example}

\end{document}