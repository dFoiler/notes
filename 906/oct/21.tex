% !TEX root = ../notes.tex

\documentclass[../notes.tex]{subfiles}

\begin{document}

\section{October 21}
There will be a correction posted to the problem set.

\subsection{\texorpdfstring{$\mathbb E_1$}{ E1}-Spaces}
Recall that we have embedded $\mathrm{Set}$ into the $\infty$-category $\mathrm{Spaces}$, and we have embedded the category $\mathrm{Mod}_R$ into the $\infty$-category $D(R)_{\ge0}$. Notably, it is hard to understand $\mathrm{Spaces}$, but it is perhaps manageable to understand the analogous construction $D(R)_{\ge0}$ because these are just chain complexes (by the Dold--Kan correspondence). One then uses $D(R)_{\ge0}$ to understand $\mathrm{Spaces}$ by constructing some functors $\mathrm{Spaces}\to D(R)_{\ge0}$ (for which it is enough to specify the image of $\Delta^0$).

We may wish to weaken $\mathrm{Mod}_R$ in order to build more flexible functors out of $\mathrm{Spaces}$. For this, we use the weakest algebraic structures we can find, which are monoids.
\begin{definition}[discrete monoid]
	A \textit{discrete monoid} is a set $X$ equipped with a multiplication $m\colon X\times X\to X$ which is unital and associative. We let $\mathrm{Mon}$ denote the category of monoids.
\end{definition}
\begin{example}
	There is a full subcategory $\mathrm{Grp}$ of $\mathrm{Mon}$.
\end{example}
\begin{example}
	Monoids are not required to have inverses, so $\ZZ_{\ge0}$ is a monoid.
\end{example}
We are about to construct $\mathbb E_1$-spaces, which include monoids. Heuristically, they consist of a space $X$ with some multiplication $m\colon X\times X\to X$ and an identity $e\colon\Delta^0\to X$ for which any $x,y,z\in X$ admit a path $(xy)z\to x(yz)$ and a path $1x\to x$ and $x1\to x$ and so on. There are also lots of different ways to associate, so we add in higher-order homotopies to encode these coherences.
\begin{defihelper}[$\mathbb E_1$-space] \nirindex{E1 space@$\mathbb E_1$-space}
	Given an $\infty$-groupoid $X$, an $\mathbb E_1$-structure on $X$ is a functor $M\colon\Delta\opp\to\mathrm{Spaces}$ such that $M_0=\Delta^0$ and $M_1=X$ and the injections $[1]\to[n]$ (given by $i\le i+1$) produce maps $M_n\to M_1^n$ which is an equivalence.
\end{defihelper}
\begin{remark}
	In particular, we see that $M_n=X^n$ for each $n$, and there is a specific map witnessing this identification.
\end{remark}
\begin{example} \label{ex:classical-monoid-to-e-1}
	Given a classical monoid $X$, we can produce the functor $M$ as factoring through $\mathrm{Set}$. For example, the boundaries $X\to\Delta^0$ should be constants, and the face $\Delta^0\to X$ should be the identity. More generally, the differential $d^i\colon X^{n+1}\to X^n$ should multiply in the $i$th coordinate, and the face maps include the identity at a particular index. In fewer words, this is the nerve of the one-object category produced from the monoid $M$.
\end{example}
\begin{example} \label{ex:top-group-is-e1}
	Consider a (nice) topological group $G$. Then the construction of \Cref{ex:classical-monoid-to-e-1} produces a functor $\Delta\opp\to\mathrm{Top}$ (via the same checks), which then produces a functor $\Delta\opp\to\mathrm{Spaces}$ by composing through $\mathrm{Sing}$.
\end{example}
\begin{example}
	Given a Kan complex $X$, we recall that we have a Kan complex $\underline{\op{Mor}}(X,X)$, which is also an $\mathbb E_1$-space. The point is that we can compose two endomorphisms to produce a third endomorphism, which is our multiplication map.
\end{example}
\begin{example}
	If $X$ is a Kan complex, then $\Omega X$ is an $\mathbb E_1$-space. The point is that composition of loops is associative and unital, up to ``coherent'' homotopy. Namely, the choice of homotopies witnessing the associative and unital properties are more or less unique (up to some higher choices, which are unique up to higher choices, and so on). In fact, $\pi_0(\Omega X)$ is a group.
\end{example}
\begin{definition}[group-like]
	An $\mathbb E_1$-space is \textit{group-like} if and only if $\pi_0X$ is a group.
\end{definition}
\begin{remark}
	It turns out that $\mathbb E_1$-spaces form a full subcategory of $\underline{\op{Mor}}(\Delta\opp,\mathrm{Spaces})$, where we are silently viewing $\Delta\opp$ as a quasicategory. In particular, it may look like adding $\mathbb E_1$-structure looks like extra structure, but it does not restrict the morphisms. There is then the fully faithful subcategory of group-like $\mathbb E_1$-spaces. As such, being group-like is a property of an $\mathbb E_1$-space, not extra data.
\end{remark}
\begin{definition}
	Fix an $\mathbb E_1$-space $X$. Then $BM$ is the colimit of $M\colon\Delta\opp\to\mathrm{Spaces}$.
\end{definition}
\begin{example}
	If $G$ is a discrete group, then $BG=K(G,1)$. Indeed, the colimit of this functor $\Delta\opp\to\mathrm{Spaces}$ factors through $\mathrm{Set}$, so it will be given by the functor $\Delta\opp\to\mathrm{Set}$ which is the ordinary category $BG$. Similarly, if $M$ is a discrete monoid, then $BM=BG$, where $G$ is the group completion of $M$ (given by adding formal inverses). Namely, we would expect to get the $1$-object category given by $M$, but this is just $BG$ up to weak equivalence.
\end{example}
\begin{example}
	Consider a topological Lie group $G$. Then we receive an $\mathbb E_1$-space $\Delta\opp\to\mathrm{Spaces}$ (via \Cref{ex:top-group-is-e1}), so we can define $BG$.
\end{example}
\begin{definition}[vector bundle]
	A \textit{vector bundle} on a space $X$ is a map $X\to B\op U(n)$ of simplicial sets.
\end{definition}
\begin{remark}
	When $X$ is a nice topological space, then this is analogous to defining a vector bundle as a representation of $\pi_1(X)$.
\end{remark}
\begin{remark}
	More generally, if $G$ is a topological group, then $[X,BG]$ classifies principal $G$-bundles up to isomorphism.
\end{remark}
It turns out that $\mathbb E_1$-spaces are complicated enough for most of our purposes.
\begin{theorem} \label{thm:one-object-groupoid}
	The functors $B$ and $\Omega$ are inverse equivalences between the quasicategories of group-like $\mathbb E_1$-spaces and connected pointed spaces.
\end{theorem}
\begin{remark}
	Intuitively, we are saying that $1$-object groupoids (i.e., connected $\infty$-groupoids) are groups (i.e., group-like $\mathbb E_1$-spaces). Namely, one can recover a group $G$ from $BG$ by taking loops. Of course, the showing the inverse equivalence is hard!
\end{remark}
\begin{corollary}[van Kampen]
	The functor $\pi_1$ on pointed connected spaces preserves pushouts.
\end{corollary}
\begin{proof}
	We know $\pi_1X=\pi_0\Omega X$, so we are equivalently asking for $\pi_0$ to preserve pushouts on the category of group-like $\mathbb E_1$-spaces.
\end{proof}
\begin{corollary}
	Fix an $\mathbb E_1$-space $M$. Then $\Omega BM$ is the group completion of $M$.
\end{corollary}
\begin{proof}
	For any group $G$, we see that
	\begin{align*}
		\op{Hom}_{\mathbb E_1}(M,G) &= \op{Hom}_{\mathbb E_1}(\Omega BM,G),
	\end{align*}
	but one can check directly that $BM$ is insensitive to the group completion. As such, $\Omega BM$ must be the same as its group completion.
\end{proof}
\begin{corollary}
	The free group-like $\mathbb E_1$-space on a space $X$ is $\Omega\Sigma X$. In other words, $\Omega\Sigma$ is left adjoint to the forgetful functor on $\mathbb E_1$-spaces.
\end{corollary}
\begin{proof}
	We can calculate
	\begin{align*}
		\op{Hom}_{\mathbb E_1}(\Omega\Sigma X,G) &= \op{Hom}_{\mathrm{ConSpaces}_*}(\Sigma X,BG) \\
		&= \op{Hom}_{\mathrm{Spaces}_*}(\Sigma X,BG) \\
		&= \op{Hom}_{\mathrm{Spaces}_*}(X,\Omega BG) \\
		&= \op{Hom}_{\mathrm{Spaces}_*}(X,G),
	\end{align*}
	as required.
\end{proof}
\begin{example}
	If $X$ is a group-like $\mathbb E_1$-space, then we can think of $\pi_nX$ as $\mathbb E_1$-space maps $\Omega S^{n+1}\to X$ (because $\Omega S^{n+1}=\Omega\Sigma S^n$). This roughly explains why we may care about $\Omega S^{n+1}$.
\end{example}

\end{document}