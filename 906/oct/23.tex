% !TEX root = ../notes.tex

\documentclass[../notes.tex]{subfiles}

\begin{document}

\section{October 23}
Let's quickly say something about the homework.
\begin{remark}
	Given a homotopy fiber sequence $F\to E\to B$, the functor $\op{\mathrm{Mor}}(X,-)$ preserves limits, so we get a homotopy fiber sequence
	\[\op{\mathrm{Mor}}(X,F)\to\op{\mathrm{Mor}}(X,E)\to\op{\mathrm{Mor}}(X,B),\]
	and taking the long exact sequence in homotopy produces a long exact sequence
	\[[X,\Omega B]\to[X,F]\to[X,E]\to[X,B].\]
	Accordingly, given a map $f\colon X\to E$, lifting it to $X\to F$ is exactly given by the data of a homotopy from $X\to E\to B$ and the constant. (This can be seen directly.) Furthermore, given two null-homotopies of the composite $X\to B$ glue together to a homotopy from the constant $X\to B$ to itself, which is given by an element of $\Omega\underline{\op{Mor}}(X,B)=\underline{\op{Mor}}(X,\Omega B)$. The point is that one can ``subtract'' two lifts in $[X,F]$ to an element in $[X,\Omega B]$, which amounts to saying that there is a group action of $\pi_1[X,B]$ on $[X,F]$.
\end{remark}
\begin{example}
	Here is an interesting example of a Wilson space. There is an embedding $\op U(n)\to\op U(n+1)$ given by $g\mapsto\op{diag}(g,1)$. This produces a sequence $\op U(1)\to\op U(2)\to\op U(3)\to\cdots$ of group homomorphisms, so may take the colimit of
	\[B\op U(1)\to B\op U(2)\to B\op U(3)\to\cdots,\]
	which produces some space $B\op U$. This turns out to be a Wilson space.
\end{example}

\subsection{\texorpdfstring{$\mathbb E_\infty$}{ E infinity}-Algebras}
Here is a starting definition.
\begin{defihelper}[$\mathbb E_k$-space] \nirindex{Ek space@$\mathbb E_k$-space}
	Fix some $\infty$-category $\mc C$ with products. Then an \textit{$\mathbb E_1$-algebra} consists of the functors $F\colon\Delta\opp\to\mc C$ for which $F_0$ is the empty product, and the induced maps $F_n\to F_1^n$ are isomorphisms. Then, in general, for each $k\ge1$, we define an $\mathbb E_{k+1}$-space to be an $\mathbb E_1$-algebra valued in $\mathbb E_k$-spaces.
\end{defihelper}
\begin{remark}
	One should check that $\mathbb E_k$-spaces has products, which it does.
	% The category of $\mathbb E_1$-spaces has products, so we can define an $\mathbb E_2$-space is an $\mathbb E_1$-algebra in the category of $\mathbb E_1$-spaces. Then we continue: an $\mathbb E_3$-space is an $\mathbb E_1$-algebra in the category of $\mathbb E_2$-spaces, and so on.
\end{remark}
\begin{remark}
	Note that an $\mathbb E_2$-space has two forgetful functors to $\mathbb E_1$-spaces according to which operation we forget about. However, the Eckmann--Hilton argument applies in this case, so the two $\mathbb E_1$-structures turn out to be equivalent (although this equality is non-canonical).
\end{remark}
\begin{definition}[group-like]
	An $\mathbb E_k$-space $X$ is \textit{group-like} if and only if $\pi_0X$ is a group.
\end{definition}
\begin{theorem}[May recognition] \label{thm:may-recog}
	Let $\mathrm{Spaces}_{*,\ge k}$ be the full subcategory of pointed spaces with trivial homotopy groups vanish in degree less than $k$. Then
	\[\Omega^k\colon\mathrm{Spaces}_{*,\ge k}\to\mathrm{Spaces}(\mathbb E_k)\]
	is fully faithful and has image given by the group-like $\mathbb E_k$-spaces. The inverse functor is given by $B^{k-1}$.
\end{theorem}
\begin{remark}
	It turns out the category of group-like $\mathbb E_k$-spaces is equivalent to the category of pointed spaces whose homotopy groups vanish before $k$. This is a direct generalization of \Cref{thm:one-object-groupoid}.
\end{remark}
The reason we have introduced $\mathbb E_k$-spaces is that we want to define $\mathbb E_\infty$-spaces.
\begin{defihelper}[$\mathbb E_\infty$-space] \nirindex{E infinity@$\mathbb E_\infty$-space}
	An \textit{$\mathbb E_\infty$-space} is a space with a collection of compatible $\mathbb E_k$-space structure for all $k$. A \textit{group-like} $\mathbb E_\infty$-space $X$ is one where $\pi_0X$ is a group.
\end{defihelper}
\begin{remark}
	A group-like $\mathbb E_\infty$ has the data of a sequence of spaces $\{X_i\}$ with equivalences $X_i=\Omega X_{i+1}$ and $X_i$ has no nontrivial homotopy groups in degree less than $i$; in this situation, $X_0$ is the $\mathbb E_\infty$-space. Indeed, one can just iteratively apply \Cref{thm:may-recog}. We may call such a sequence an infinite loop space.
\end{remark}
\begin{remark}
	Where $\mathbb E_1$-spaces generalize monoids and group-like $\mathbb E_1$-spaces generalize groups, we now say that $\mathbb E_\infty$-spaces generalize abelian groups. The point is that the infinite coherences remember the commutativity. What is amazing is that we have encoded commutativity as a piece of data instead of as a property!
\end{remark}
\begin{example}
	If $A$ is a discrete abelian group, then
	\[A=\Omega K(A,1)=\Omega K(A,2)=\Omega K(A,3)=\cdots,\]
	so $A$ embeds as a discrete group-like $\mathbb E_\infty$-space. It turns out that a discrete abelian monoid upgrades to an $\mathbb E_\infty$-space.
\end{example}
\begin{nex}
	Note that $\Omega^2S^3$ is an $\mathbb E_2$-space, but it turns out that it is not an $\mathbb E_\infty$-space. On the other hand, it does admit an $\mathbb E_3$-structure because it is $\Omega^3B\op{SU}(2)$.
\end{nex}
\begin{example}
	Any symmetric monoidal classical groupoid is an $\mathbb E_\infty$-space. For example, $\mathrm{FinSet}$ (equip\-ped with bijections for morphisms) is can be thought of as
	\[\bigsqcup_{n\ge0}BS_n\]
	because $BS_n$ is the single object, but we are remembering that it ``should'' have $n$ elements. Thus, this admits an $\mathbb E_\infty$-structure, where the operation is given by $\sqcup$ of the finite sets.
\end{example}
\begin{example}
	Fix a classical commutative ring $R$. Then the category of finitely generated free $R$-mod\-ules (still with isomorphisms as morphisms) is also an $\mathbb E_\infty$-space with the operation given by $\oplus$.
\end{example}
\begin{example}
	There is an $\mathbb E_\infty$-space
	\[\bigsqcup_{n\ge0}B\op U(n),\]
	which is the groupoid of finite-dimensional complex vector spaces. The $\mathbb E_\infty$-space structure comes from $\oplus$.
\end{example}
\begin{remark}
	The functor $M\mapsto\Omega BM$ is left adjoint to the forgetful functor from group-like $\mathbb E_\infty$-spaces to $\mathbb E_\infty$-spaces. However, this operation can be complicated (and interesting!).
\end{remark}
\begin{example}
	If $M=\bigsqcup_{n\ge0}B\op U(n)$, then $\Omega BM=\op BU\times\ZZ$, which we can think about as ``formal differences'' of vector spaces (i.e., ``virtual'' vector spaces).
\end{example}
\begin{example}
	If $M$ is given by the finitely generated free $R$-modules, then $\pi_i\Omega BM$ are called the algebraic $K$-groups $K_i(R)$ of $R$. For example, it turns out that the Kummer--Vandiver conjecture is equivalent to $K_{4i}(\ZZ)=0$ for all $i\ge0$.
\end{example}
\begin{example}
	If $M$ is given by the finite sets under disjoint union, then $\pi_i(\Omega BM)$ is the $k$th homotopy group of spheres.
\end{example}
\begin{example}
	We will construct a group-like $\mathbb E_\infty$-space $MO$, whose objects are $0$-dimensional manifolds, morphisms are cobordisms of objects, $2$-morphisms are cobordisms of cobordisms, and so on. Disjoint union provides our $\mathbb E_\infty$-structure, where the basepoint is $\emp$; it turns out $\pi_kMO$ is closed $k$-manifolds up to cobordism. Lastly, to see that it is a group-like, we note that $*\sqcup*$ is equivalent to $\emp$ by an explicit cobordism, so $\pi_0MO$ is a group (in fact, $2$-torsion).
\end{example}
\begin{example}
	Given any pointed space $X$, there are maps
	\[X\to\Omega\Sigma X\to\Omega^2\Sigma^2\to\cdots.\]
	The first map is induced by \Cref{thm:freud}. The second map is $\Sigma$ applied to the canonical map $\Sigma X\to\Omega\Sigma(\Sigma X)$, and one can continue this process. It follows that the colimit $\Omega^\infty\Sigma^\infty X$ is a group-like $\mathbb E_\infty$-space, where $\pi_i\Omega^\infty\Sigma^\infty X$ is the $i$th stable homotopy group of $X$ (which makes sense by \Cref{thm:freud}). It turns out that $\Omega\Sigma\mathrm{FinSet}$ is $\Omega^\infty\Sigma^\infty S^0$.
\end{example}

\subsection{Spectra}
We can generalize our notion of group-like $\mathbb E_\infty$-spaces being thought of as infinite loop spaces.
\begin{definition}[spectrum]
	A \textit{spectrum} is a sequence $(X_0,X_1,\ldots)$ of pointed spaces along with equivalences $X_i\to\Omega X_{i+1}$.
\end{definition}
\begin{example}
	Any group-like $\mathbb E_\infty$-space produces a spectrum.
\end{example}
Here is a nontrivial example.
\begin{theorem}[Bott periodicity] \label{thm:bott-periodicity}
	Let $\op U$ be the infinite unitary group, which is the colimit of the finite unitary groups. It turns out that $\Omega(B{\op U}\times\ZZ)\simeq\op U$ and $\Omega U\simeq B\op U\times\ZZ$.
\end{theorem}
\begin{example}
	There is a spectrum $KU\coloneqq(B{\op U}\times\ZZ,{\op U},B{\op U}\times\ZZ,{\op U},\ldots)$.
\end{example}

\end{document}