% !TEX root = ../notes.tex

\documentclass[../notes.tex]{subfiles}

\begin{document}

\section{October 9}
Today we continue computing homotopy groups.

\subsection{Mod \texorpdfstring{$\mc C$}{C} Hurewicz's Theorem}
Here is another application: we can characterize Eilenberg--MacLane spaces.
\begin{proposition} \label{prop:characterize-eilenberg-maclane}
	Fix an abelian group $A$ and a nonnegative integer $n\ge2$. For a simply connected pointed Kan complex $X$ such that
	\[\pi_iX\cong\begin{cases}
		0 & \text{if }i\ne n, \\
		A & \text{if }i=n,
	\end{cases}\]
	there is a homotopy equivalence $X\to K(A,n)$.
\end{proposition}
\begin{proof}
	By \Cref{thm:hurewisz}, we know that $\mathrm H^n(X;\ZZ)=A$ (and $\mathrm H^i(X,\ZZ)=0$ for $i<n$), so the universal property of $K(A,n)$ provides a map $f\colon X\to K(A,n)$ which is an isomorphism on $\mathrm H^n(-;\ZZ)$. It follows by \Cref{thm:hurewisz} that $f$ is an isomorphism on $\pi_n$, and it is an isomorphism on $\pi_i$ for $i\ne n$ because the relevant groups vanish, so the result follows from \Cref{thm:whitehead}.
\end{proof}
\begin{remark}
	In fact, the homotopy equivalence $X\to K(A,n)$ is now unique up to ``$(-1)$-connected'' choice, meaning that it is canonical up to a choice of isomorphism $\pi_nX\cong A$.
\end{remark}
Here is an application of the Whitehead tower: we will be able to show that homotopy groups are frequently finitely generated.
\begin{notation} \label{not:class-c}
	Throughout, $\mc C$ denotes a subclass of abelian groups which are either finitely generated or the subclass of finite abelian groups which are $P$-torsion for a set of primes $P$.
\end{notation}
\begin{example}
	If $\mc P$ is the set of all primes, then $\mc C$ becomes all finite abelian groups. But if $\mc P=\{2\}$, then these are $2$-groups.
\end{example}
\begin{lemma} \label{lem:c-is-ideal}
	Consider one of the full subcategories $\mc C$ of abelian groups.
	\begin{listalph}
		\item The category $\mc C$ admits finite limits and colimits.
		\item The category $\mc C$ is closed under extensions.
		\item For $A,B\in\mc C$, then $A\otimes_\ZZ B$ and $\op{Tor}_1(A,B)$ are in $\mc C$.
	\end{listalph}
\end{lemma}
\begin{proof}
	These are all purely formal.
\end{proof}
\begin{lemma} \label{lem:homotopy-fiber-to-c}
	Fix a homotopy fiber sequence $F\to E\to B$ of pointed Kan complexes with $B$ simply connected. If any two of the families
	\[\left\{\mathrm H_i(B;\ZZ)\right\}_i,\qquad\left\{\mathrm H_i(E;\ZZ)\right\}_i,\qquad\text{and}\qquad\left\{\mathrm H_i(F;\ZZ)\right\}_i\]
	are in $\mc C$, then so is the third.
\end{lemma}
\begin{proof}
	We will only show one of the three implications because the other two are similar. In particular, suppose that $\widetilde H_i(B;\ZZ)$ and $\widetilde H_i(F;\ZZ)$ are in $\mc C$ for all $i$, and we want to show the same for $\widetilde H_i(E;\ZZ)$. For this, we use the Serre spectral sequence (\Cref{thm:serre-spectral-seq})
	\[E^2_{pq}=\mathrm H_p(B;\mathrm H_q(F;\ZZ))\Rightarrow\mathrm H_{p+q}(E;\ZZ).\]
	We can handle $\widetilde H^0(E;\ZZ)$ directly by staring at the corner of the Serre spectral sequence, so we suppose that $i>0$.

	Now, by the filtration on $E^\infty$, it is enough by \Cref{lem:c-is-ideal} to merely show that the individual terms $E^\infty_{pq}$ are in $\mc C$. Because $E^\infty$ is constructed from $E^2$ by taking some repeated kernels and quotients, it is enough by \Cref{lem:c-is-ideal} to show that $E^2_{pq}$ is in $\mc C$ when one of $p$ or $q$ is positive. One can handle $p=0$ or $q=0$ by hand, so we assume that $p$ and $q$ are positive. Then we note there is a short exact sequence
	\[0\to\mathrm H_p(B)\otimes\mathrm H_q(F)\to\mathrm H_p(B;\mathrm H_q(F))\to\op{Tor}_1(\mathrm H_{p-1}(B),\mathrm H_{q-1}(F))\to0\]
	by the Universal coefficients theorem, so we are done by \Cref{lem:c-is-ideal}.
\end{proof}
\begin{lemma} \label{lem:k-a-n-in-c}
	Fix an abelian group $A$ in $\mc C$. Then $\mathrm H_i(K(A,n);\ZZ)$ is in $\mc C$ for all $i,n>0$.
\end{lemma}
\begin{proof}
	If $n\ge2$, then the homotopy fiber sequence $K(A,n-1)\to\Delta^0\to K(A,n)$ allows us to reduce to the case $n-1$ by \Cref{lem:homotopy-fiber-to-c}, so we may assume that $n=1$. (Note that $n\ge2$ makes $K(A,n)$ simply connected!) We now have to study the homology of $K(A,1)$. Well, writing
	\[K(A,1)=\ZZ^{\oplus r}\bigoplus_{p\text{ prime}}\bigoplus_{\nu\ge0}(\ZZ/p^\nu\ZZ)^{r_{p,\nu}}\]
	for some nonnegative integers $r$ and $r_{p,\nu}$. Now, homotopy commutes with taking products of spaces, so it follows that $K(A,1)$ splits as a finite product of $K(-,1)$s of the above spaces, so by the K\"unneth formula, it is enough to handle $A$ equal to $\ZZ$ or some $p$-power cyclic group $\ZZ/p^\nu\ZZ$. For $A=\ZZ$, we have $K(\ZZ,1)=S^1$, whose homology in positive degree is supported in degree $1$, and it outputs $\ZZ$.

	It remains to handle $A=\ZZ/p^\nu\ZZ$. Note that the map $p^n\colon\mathrm H^2(K(\ZZ,2);\ZZ)\to\mathrm H^2(K(\ZZ,2);\ZZ)$ induces a map $p^n\colon K(\ZZ,2)\to K(\ZZ,2)$ by the universal property. Now, let $F$ be the fiber of this latter $p^n$ map, so the homotopy long exact sequence produces exact sequences
	\[\pi_{i+1}K(\ZZ,2)\stackrel{p^n}\to\pi_{i+1}K(\ZZ,2)\to\pi_iF\to\pi_iK(\ZZ,2)\stackrel{p^n}\to\pi_iK(\ZZ,2),\]
	so $\pi_iF$ is supported in degree $1$, where it is isomorphic to $A$. Thus, \Cref{prop:characterize-eilenberg-maclane} tells us $F$ is $K(A,1)$, so we get a homotopy fiber sequence
	\[K(\ZZ,1)\to K(A,1)\to K(\ZZ,2)\]
	by taking loops and using \Cref{prop:characterize-eilenberg-maclane}. The result now follows by expanding the Serre spectral sequence to compute (co)homology.
\end{proof}
\begin{theorem} \label{thm:mod-c-hur}
	Fix a simply connected pointed Kan complex $X$, and choose a class $\mc C$ as in \Cref{not:class-c}. If $\pi_iX\in\mc C$ for $0<i<n$, then the natural map $\pi_nX\to\mathrm H_n(X;\ZZ)$ is an isomorphism modulo $\mc C$, meaning that the kernel and cokernel live in $\mc C$.
\end{theorem}
\begin{proof}
	We use the Whitehead tower $\{\tau_{\ge i}X\to K(\pi_iX,i)\}$ for $i\in\{2,\ldots,n\}$. Because $X$ is simply connected, we see that $X=\tau_{\ge2}X$, so this will be enough. Because $\pi_nX=\mathrm H_n(\tau_{\ge n}X;\ZZ)$ by construction of the Whitehead tower, we see that we need to show
	\[\mathrm H^n(\tau_{\ge n}X;\ZZ)\to\mathrm H^n(\tau_{\ge2}X;\ZZ)\]
	is an isomorphism modulo $\mc C$. We will do this by an induction, so it is enough to just do this in one step for $\tau_{\ge j}X\to\tau_{\ge j-1}X$, where $j-1\ge2$. Here, we have the homotopy fiber sequence $\tau_{\ge j}X\to\tau_{\ge j-1}X\to K(\pi_{j-1}X,j-1)$ by construction of the tower, so we get a Serre spectral sequence (\Cref{thm:serre-spectral-seq}) given by
	\[E^2_{pq}=\mathrm H_p(K(\pi_{j-1}X,j-1);\mathrm H_q(\tau_{\ge j}X;\ZZ))\Rightarrow\mathrm H_{p+q}(\tau_{\ge j-1}X;\ZZ).\]
	Now, $\mathrm H_p(K(\pi_{j-1}X,j-1);\ZZ)$ is always in $\mc C$ by \Cref{lem:k-a-n-in-c}, so the result follows from staring hard at the relevant Serre spectral sequence, where the point is that the differentials only manage to go to and from things in $\mc C$.
\end{proof}
\begin{corollary}
	For any $i,n\ge0$, the group $\pi_i(S^n)$ is finitely generated.
\end{corollary}
\begin{proof}
	For $n\in\{0,1\}$, there is nothing to do. For $n\ge2$, we know $S^n$ is simply connected, so we may apply \Cref{thm:mod-c-hur}. Then let $\mc C$ be the class of finitely generated abelian groups, so we know that the least $i$ for which $\pi_iX\notin\mc C$ has that the morphism $\pi_iX\to\mathrm H_i(X;\ZZ)$ has kernel and cokernel in $\mc C$. But then $\mathrm H_i(X;\ZZ)$ is always in $\mc C$ (for example, by using cellular homology), which implies that $\pi_iX$ is in $\mc C$, which is our contradiction.
\end{proof}
\begin{corollary}
	For $i>3$, the group $\pi_iS^3$ is finite.
\end{corollary}
\begin{proof}
	As in \Cref{ex:pi4-s3}, there is a homotopy fiber sequence $F\to S^3\to K(\ZZ,3)$, where $S^3\to K(\ZZ,3)$ is given by an isomorphism $\mathrm H^3\left(S^3;\ZZ\right)\cong\ZZ$. Thus, $\pi_iS^3$ being finite for $i>3$ is equivalent to showing that $\pi_iF$ is finite for all $i$ (because the homotopy long exact sequence erases this ``problem'' at $i=3$). Now, $F$ is simply connected by a long exact sequence calculation, so we may apply \Cref{thm:mod-c-hur} to see that it is equivalent to show that $\mathrm H_i(F;\ZZ)$ is finite for all $i$.

	However, \Cref{lem:homotopy-fiber-to-c} does do enough to tell us that $\mathrm H_i(F;\ZZ)$ is finitely generated, so it is enough to show that $\mathrm H_i(F;\ZZ)_\QQ=0$, which can be done via the Serre spectral sequence and calculating the rational homology of $S^3$ and $K(\ZZ,3)$.
\end{proof}

\subsection{The Suspension}
We are interested in understanding $\mathrm{Spaces}$, which we note can be understood as nice topological spaces up to homotopy equivalence or Kan complexes up to homotopy equivalence or even simplicial sets up to weak equivalence. We have already said something about homotopy pullbacks, so let's discuss homotopy pushouts.
\begin{definition}[homotopy pushout]
	Given maps $C\from A\to B$ of simplicial sets, the \textit{homotopy push\-out} is the pushout of the diagram
	% https://q.uiver.app/#q=WzAsMyxbMCwwLCJBXFxzcWN1cCBBIl0sWzEsMCwiQlxcc3FjdXAgQyJdLFswLDEsIkFcXHRpbWVzXFxEZWx0YV4xIl0sWzAsMl0sWzAsMV1d&macro_url=https%3A%2F%2Fraw.githubusercontent.com%2FdFoiler%2Fnotes%2Fmaster%2Fnir.tex
	\[\begin{tikzcd}[cramped]
		{A\sqcup A} & {B\sqcup C} \\
		{A\times\Delta^1}
		\arrow[from=1-1, to=1-2]
		\arrow[from=1-1, to=2-1]
	\end{tikzcd}\]
	up to weak equivalence. For example, the map $A\times\Delta^1\to B\times\{1\}$ is the product of the maps $A\to B$ and $\Delta^1\onto\{1\}$.
\end{definition}
\begin{remark}
	There is something subtle here that one may take the pushout and not receive something which is not a Kan complex in general!
\end{remark}
\begin{remark}
	There is also a pointed version, where we have to additionally produce a basepoint on the homotopy pushout, which we do by modding out by $\{a\}\times\Delta^1$.
\end{remark}
\begin{remark}
	Intuitively, we are making $A$ into a cylinder as $A\times\Delta^1$, and one ``face'' of the cylinder is being glued onto $C$ (via the map $A\to C$) and the other ``face'' of the cylinder is being glued onto $B$ (via the map $A\to B$).
\end{remark}
We will not need to work in full levels of generality for the homotopy pushout.
\begin{definition}[suspension]
	Fix a space $X$ with basepoint $x$. Then we define the \textit{suspension} $\Sigma X$ is the quotient
	\[\frac{X\times\Delta^1}{(X\times\{0,1\})\cup\left(\{x\}\times\Delta^1\right)},\]
	where the quotient means that we are collapsing $X\times\{0\}$ and $X\times\{1\}$ to a point, and we are collapsing $\{x\}\times[0,1]$ to a point.
\end{definition}
\begin{remark}
	Equivalently, $\Sigma X$ is the homotopy pushout of $\Delta^0\from X\to\Delta^0$ in pointed spaces. (The pointed-ness is why we have to collapse $\{x\}\times[0,1]$ in this definition.)
\end{remark}
\begin{remark}
	Comparing pullbacks and pushouts, there is an adjunction $\underline{\op{Mor}}(\Sigma X,Y)\simeq\underline{\op{Mor}}(X,\Omega Y)$ as pointed spaces. In fact, using our explicit constructions, there is an adjuction
	\[\underline{\op{Mor}}(\Sigma X,Y)\cong\underline{\op{Mor}}(X,\Omega Y)\]
	in simplicial sets!
\end{remark}
\begin{example}
	One can show that $\Sigma S^n$ is homotopy equivalent to $ S^{n+1}$: we basically take one copy of $S^n$ and think of it is as a diameter of $S^n$, and the suspension adds two cones to the top and bottom of $S^n$, which become the two hemispheres of $S^{n+1}$! (Collapsing the basepoint at the end does not change the homotopy type.)
\end{example}
\begin{example} \label{ex:suspension-sphere}
	In the category of pointed spaces, we see that $\underline{\op{Mor}}_{\mathrm{Spaces}_*}(\Sigma S^n,X)$ is the pushout of the diagram
	% https://q.uiver.app/#q=WzAsMyxbMCwxLCJcXHVuZGVybGluZXtcXG9we01vcn19X3tcXG1hdGhybXtTcGFjZXN9Xyp9KFxcRGVsdGFeMCxYKSJdLFsxLDAsIlxcdW5kZXJsaW5le1xcb3B7TW9yfX1fe1xcbWF0aHJte1NwYWNlc31fKn0oXFxEZWx0YV4wLFgpIl0sWzEsMSwiXFx1bmRlcmxpbmV7XFxvcHtNb3J9fV97XFxtYXRocm17U3BhY2VzfV8qfShTXm4sWCkiXSxbMCwyXSxbMSwyXV0=&macro_url=https%3A%2F%2Fraw.githubusercontent.com%2FdFoiler%2Fnotes%2Fmaster%2Fnir.tex
	\[\begin{tikzcd}[cramped]
		& {\underline{\op{Mor}}_{\mathrm{Spaces}_*}(\Delta^0,X)} \\
		{\underline{\op{Mor}}_{\mathrm{Spaces}_*}(\Delta^0,X)} & {\underline{\op{Mor}}_{\mathrm{Spaces}_*}(S^n,X)}
		\arrow[from=1-2, to=2-2]
		\arrow[from=2-1, to=2-2]
	\end{tikzcd}\]
	in $\mathrm{Spaces}_*$. But one can see that this is the same as the homotopy pullback of the diagram $\Delta^0\to\Omega^nX\from\Delta^0$, which is $\Omega^{n+1}X=\underline{\mathrm{Mor}}_{\mathrm{Spaces}_*}\left(S^{n+1},X\right)$. This allows us to compute $\Sigma S^n=S^{n+1}$ again!
\end{example}

\end{document}