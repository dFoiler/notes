% !TEX root = ../notes.tex

\documentclass[../notes.tex]{subfiles}

\begin{document}

\section{October 14}
Today we continue.

\subsection{The Fruedenthal Suspension Theorem}
For our next result, we note that the adjunction provides a canonical map $X\to\Omega\Sigma X$.
\begin{theorem}[Fruedenthal suspension] \label{thm:freud}
	Fix a pointed space $X$ for which $\pi_iX=0$ for $i<n$. Then the canonical map
	\[\pi_iX\to\pi_i(\Omega\Sigma X)\to\pi_{i+1}(\Sigma X)\]
	is an isomorphism for $i\le2n-2$ and a surjection for $i=2n-1$.
\end{theorem}
\begin{proof}
	There is a homotopy fiber sequence $\Omega\Sigma X\to\Delta^0\to\Sigma X$. For $n\ge1$, one can sow that $\Sigma X$ is simply connected, so we get a Serre spectral sequence
	\[E_2=\mathrm H_p(\Sigma X;\mathrm H_q(\Omega\Sigma X;\ZZ))\Rightarrow\mathrm H_{p+q}\left(\Delta^0;\ZZ\right).\]
	Now, by the Mayer--Vietoris sequence, we can calculate $\mathrm H_p(\Sigma X;\ZZ)=\mathrm H_{p-1}(X;\ZZ)$, so \Cref{thm:hurewisz} tells us that $\mathrm H_{p-1}(X;\ZZ)$ vanishes for $0<p-1<n$. Thus, the $E_2$ page vanishes in columns $p\in\{1,2,\ldots,n\}$. Similarly, \Cref{thm:hurewisz} tells us that $\mathrm H_q(\Omega\Sigma X;\ZZ)$ vanishes for small $q\in\{1,2,\ldots,n-1\}$ because we can compare it to $\pi_q(\Omega\Sigma X)=\pi_{q+1}(\Sigma X)$.

	Thus, because our $E_\infty$ page needs to vanish except at the origin, we see that the only differentials which matter for small values are $E^{i0}\to E^{0,i-1}$ for $i<2n+1$, which we see is
	\[\mathrm H_i(\Sigma X)\to\mathrm H_{i-1}(\Omega\Sigma X).\]
	Thus, these have to be isomorphisms.
	
	We now argue as in \Cref{thm:whitehead-homology} to turn isomorphisms in homology to isomorphisms in homotopy. Examining the Serre spectral sequence arising from the homotopy fiber sequence $F\to X\to\Omega\Sigma X$, we find that $\mathrm H_i(F;\ZZ)$ needs to vanish for $i$ less than about $2n$, so $\pi_iF$ needs to be trivial in the same region by \Cref{thm:hurewisz}, so the required isomorphism follows by taking the long exact sequence in homotopy.
\end{proof}

\subsection{Stable Homotopy Groups of Spheres}
Here is an application.
\begin{definition}[stable homotopy]
	Fix some integer $k$. Then we define
	\[\pi_k^{\mathrm{st}}\coloneqq\lim_{n\to\infty}\pi_{n+k}\left(S^n\right).\]
	Here, the limit is taken with respect to the canonical maps $\pi_{n+k}\left(S^n\right)\to\pi_{n+k+1}\left(S^{n+1}\right)$ defined in \Cref{thm:freud} (via \Cref{ex:suspension-sphere}). Note that this limit stabilizes for large $n$ by \Cref{thm:freud}.
\end{definition}
\begin{remark}
	In \Cref{ex:pi4-s3}, we computed $\pi_4S^3\cong\ZZ/2\ZZ$, so $\pi_1^{\mathrm{st}}\cong\ZZ/2\ZZ$.
\end{remark}
There is a lot known about stable homotopy groups of spheres. Here is a table of some known values.
\[\begin{array}{c|ccccccccccc}
	i & 0 & 1 & 2 & 3 & 4 & 5 & 6 & 7 & 8 \\
	\pi_i^{\mathrm{st}} & 0 & \ZZ & \ZZ/2\ZZ & \ZZ/24\ZZ & 0 & 0 & \ZZ/2\ZZ & \ZZ/240\ZZ & (\ZZ/2\ZZ)^2
\end{array}\]
By using Serre's approach to compute homotopy groups by going up the Whitehead tower, one can show the following.
\begin{theorem}[Adams spectral sequence]
	For each prime $p$, there is a spectral sequence $E$ for which
	\[E_2^{st}=\mathrm{Ext}_A^{s,t}(\FF_p,\FF_p)\Rightarrow\pi_{t-s}^{\mathrm{st}}\otimes\ZZ_p,\]
	where $A$ is the graded Steenrod algebra, defined using the cohomology of Eilenberg--MacLane spaces $K(\ZZ/p\ZZ;n)$s.
\end{theorem}
Here is a picture of the Adams spectral sequence for $p=2$.
% https://q.uiver.app/#q=WzAsMTIsWzAsNSwiXFxidWxsZXQiXSxbMCw0LCJcXGJ1bGxldCJdLFswLDMsIlxcYnVsbGV0Il0sWzAsMiwiXFxidWxsZXQiXSxbMCwxLCJcXGJ1bGxldCJdLFswLDAsIlxcdmRvdHMiXSxbMSw0LCJcXGJ1bGxldCJdLFsyLDMsIlxcYnVsbGV0Il0sWzMsMiwiXFxidWxsZXQiXSxbMywzLCJcXGJ1bGxldCJdLFszLDQsIlxcYnVsbGV0Il0sWzYsMywiXFxidWxsZXQiXSxbMCwxLCIiLDAseyJzdHlsZSI6eyJoZWFkIjp7Im5hbWUiOiJub25lIn19fV0sWzEsMiwiIiwwLHsic3R5bGUiOnsiaGVhZCI6eyJuYW1lIjoibm9uZSJ9fX1dLFsyLDMsIiIsMCx7InN0eWxlIjp7ImhlYWQiOnsibmFtZSI6Im5vbmUifX19XSxbMyw0LCIiLDAseyJzdHlsZSI6eyJoZWFkIjp7Im5hbWUiOiJub25lIn19fV0sWzAsNiwiIiwyLHsic3R5bGUiOnsiaGVhZCI6eyJuYW1lIjoibm9uZSJ9fX1dLFs2LDcsIiIsMix7InN0eWxlIjp7ImhlYWQiOnsibmFtZSI6Im5vbmUifX19XSxbNyw4LCIiLDIseyJzdHlsZSI6eyJoZWFkIjp7Im5hbWUiOiJub25lIn19fV0sWzAsMTAsIiIsMix7InN0eWxlIjp7ImhlYWQiOnsibmFtZSI6Im5vbmUifX19XSxbMiw4LCIiLDIseyJzdHlsZSI6eyJoZWFkIjp7Im5hbWUiOiJub25lIn19fV0sWzEsOSwiIiwyLHsic3R5bGUiOnsiaGVhZCI6eyJuYW1lIjoibm9uZSJ9fX1dLFsxMCwxMSwiIiwyLHsic3R5bGUiOnsiaGVhZCI6eyJuYW1lIjoibm9uZSJ9fX1dLFsxMCw5LCIiLDIseyJzdHlsZSI6eyJoZWFkIjp7Im5hbWUiOiJub25lIn19fV0sWzksOCwiIiwxLHsic3R5bGUiOnsiaGVhZCI6eyJuYW1lIjoibm9uZSJ9fX1dXQ==&macro_url=https%3A%2F%2Fraw.githubusercontent.com%2FdFoiler%2Fnotes%2Fmaster%2Fnir.tex
\[\begin{tikzcd}[cramped,sep=tiny]
	\vdots \\
	\bullet \\
	\bullet &&& \bullet \\
	\bullet && \bullet & \bullet &&& \bullet \\
	\bullet & \bullet && \bullet \\
	\bullet
	\arrow[no head, from=3-1, to=2-1]
	\arrow[no head, from=4-1, to=3-1]
	\arrow[no head, from=4-1, to=3-4]
	\arrow[no head, from=4-3, to=3-4]
	\arrow[no head, from=4-4, to=3-4]
	\arrow[no head, from=5-1, to=4-1]
	\arrow[no head, from=5-1, to=4-4]
	\arrow[no head, from=5-2, to=4-3]
	\arrow[no head, from=5-4, to=4-4]
	\arrow[no head, from=5-4, to=4-7]
	\arrow[no head, from=6-1, to=5-1]
	\arrow[no head, from=6-1, to=5-2]
	\arrow[no head, from=6-1, to=5-4]
\end{tikzcd}\]
(In this small region, one finds $E_2=E_\infty$, but in general there will be a lot of cancellation.) The $n$th column communicates information about $\pi_n^{\mathrm{st}}\otimes\ZZ_p$. For example, the $0$th column should be $\ZZ_2$, so it goes infinitely vertically. In particular, each dot produces a $\ZZ/2\ZZ$, and the vertical lines tell us how to assemble each $\ZZ/2\ZZ$ into a group. For example, one can read $\pi_3^{\mathrm{st}}=$.
\begin{remark}[Lin--Wang--Xu]
	One can show that there is a class $h_6^2$ in $\pi_{126}^{\mathrm{st}}\otimes\ZZ_p$. This is a very recent result!
\end{remark}
\begin{remark} \label{rem:high-adams}
	One can see that there is a well-controlled subgroup of the stable homotopy groups defined as follows. Namely, for any $n$, there is a map $J\colon\op O(n)\to\underline{\op{Mor}}_{\mathrm{Spaces}_*}(S^n,S^n)$ by viewing $S^n$ as the one-point compactification of $\RR^n$. This latter group is $\Omega^nS^n$, so there are maps
	\[\pi_k\op O(n)\to\pi_k\Omega^nS^n=\pi_{k+n}S^n.\]
	Thus, the image of $J$ provides an interesting subgroup of $\pi_k^{\mathrm{st}}$. This image is well-understood (meaning that they are understood combinatorially, apparently in terms of Bernoulli numbers). It turns out that they approximately correspond to parts of the Adams sequence which are in ``high'' filtration.
\end{remark}
Here is one reason to care about stable homotopy groups.
\begin{theorem}[Kervaire--Milnor]
	For each $n\ge5$, there is an exact sequence
	\[0\to bP_{n+1}\to\Theta_n\to\frac{\pi_n^{\mathrm{st}}}{\im J}\to KI\to0.\]
	Here, $\im J\subseteq\pi_n^{\mathrm{st}}$ is constructed in \Cref{rem:high-adams}, and $\Theta_n$ is the group of exotic $n$-spheres up to diffeomorphism. Continuing, $bP_{n+1}$ is given by the group of exotic $n$-spheres which are boundaries of framed manifolds, and it is a cyclic group understood combinatorially by denominators of Bernoulli numbers. Lastly, $KI$ is an explicit group which is either $0$ or $\ZZ/2\ZZ$.
\end{theorem}
Here is some of what it is known about $KI$.
\begin{itemize}
	\item A theorem of Browder shows that $KI$ is trivial unless $n$ is $2$ less than a power of $2$, and Browder showed that it is $\ZZ/2\ZZ$ if and only if a certain class $h_j^2$ from the Adams spectral sequence (for the prime $p=2$) survives to $E_\infty$.
	\item Then Kervaire--Milnor showed that it is $\ZZ/2\ZZ$ if $n\in\{2,6,14\}$.
	\item Mahowald--Tangora--Jones showed that it is $\ZZ/2\ZZ$ for $n\in\{30,62\}$.
	\item Lin--Wang--Xu recently proved that it is $\ZZ/2\ZZ$ when $n=126$.
	\item Lastly, Hill--Hopkins--Ravenel showed that $KI$ vanishes for $n>126$. This result uses chromatic homotopy theory at height $4$.
\end{itemize}
The moral is that differential topology in dimensions $n\ge5$ can be turned into homotopy theory (up to combinatorial difficulties), and this moral has other incarnations.
\begin{remark}
	A reasonable question to ask is how many dots are there on $E_\infty$ of the Adams spectral sequence at $p=2$ in the columns from $1$ to $n$. The best known upper bound is $n^{\log(n)^2}$, which comes directly from counting dots on the $E_2$ page. The image of $J$ provides a lower bound of $\sqrt n$; this was improved by Oka to $n$ by producing explicit elements of the Adams spectral sequence. This was recently improved (by Jermey Hahn and some others) to $n\log n$.
\end{remark}

\end{document}