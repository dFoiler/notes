% !TEX root = ../notes.tex

\documentclass[../notes.tex]{subfiles}

\begin{document}

\section{October 30}
I have regained access to my computer. The exam will feature computations with the Serre spectral sequence.

\subsection{Examples of Spectra}
This class has been interested in the $\infty$-categories $\mathrm{Spaces}$, $\mathrm{Spaces}_*$, and $\mathrm{Spectra}$. There is a forgetful functor $\mathrm{Spaces}_*\to\mathrm{Spaces}$, which is adjoint to adding a disjoint basepoint $(-)_+\colon\mathrm{Spaces}\to\mathrm{Spaces}_*$. There is are also adjoint functors $\Omega^\infty\colon\mathrm{Spectra}\to\mathrm{Spaces}_*$ and $\Sigma^\infty\colon\mathrm{Spaces}_*\to\mathrm{Spectra}$. We also recall that $\Sigma^\infty$ outputs to the $\infty$-category of $\mathbb E_\infty$-spaces.
\begin{example}
	If $X$ is a pointed space, then $\Sigma^\infty X$, then we recall that
	\[\pi_i\Sigma^\infty X=\pi_i^{\mathrm{st}}X=\pi_i\left(\colim_n\Omega^n\Sigma^n X\right).\]
	Now, $\pi_i$ passes through the filtered colimit because $S^i$ is a compact object, so this is $\colim_n\pi_i\Omega^n\Sigma^nX$.
\end{example}
\begin{example}
	As an instance of the above example, we set $\mathbb S\coloneqq\Sigma^\infty S^0$ to be the sphere spectrum. It's called the sphere spectrum because
	\[S^n=\Omega^\infty\Sigma^\infty S^n=\Omega^\infty\Sigma^n\mathbb S,\]
	To explain why it may be fundamental, recall $\Sigma^\infty\Delta^0=0$, so the suspension of just one point is uninteresting. But $\Sigma^\infty_+\Delta^0=\Sigma^\infty\Delta^0_+=\mathbb S$ has a chance of being interesting. In particular, it is the ``free object on a point'' in homotopy theory, so it holds the same interest as $\ZZ$ would for algebraists.
\end{example}
Recall that $\mathrm{Spectra}$ has many nice properties: it is stable (which implies that pushout squares and pullback squares are the same), admits all limits and colimits, and has a zero object.
\begin{remark}
	It turns out that stability is in fact equivalent to pushout squares being the same as pullback squares.
\end{remark}
It is also Cartesian: for spectra $E$ and $F$, one has
\[\underline{\op{Hom}}(E,F)=\Omega\underline{\op{Hom}}(E,\Sigma F)=\cdots,\]
which can be unwound to give a definition of a spectrum. In other words, $\underline{\op{Hom}}(E,F)$ has an underlying space given by viewing $E$ and $F$ as spaces. Then
\[\underline{\op{Hom}}(E,F)=\underline{\op{Hom}}(E,\Omega\Sigma F)=\Omega\underline{\op{Hom}}(E,\Sigma F)\]
where $\Omega$ comes out of the limit because it is a limit. This process can be iterated to turn $\underline{\op{Hom}}(E,F)$ into a spectrum: the $n$th term of the spectrum is the space $\underline{\op{Hom}}(E,\Sigma^nF)$.
\begin{example}
	We see $\underline{\op{Hom}}(\mathbb S,E)=E$ by expanding everything out. Indeed, $\mathbb S=\Sigma^\infty S^0$, so this is $\underline{\op{Hom}}(S^0,\Omega^\infty\Sigma^nE)$, which is $\Omega^\infty\Sigma^nE$ (because we are working with pointed spaces!).
\end{example}
\begin{definition}[Eilenberg--MacLane spectrum]
	If $A$ is a discrete abelian group, then we let $\Sigma^nA$ be an \textit{Eilenberg--MacLane spectrum}.
\end{definition}
\begin{example}
	By definition, we see that
	\[\Omega^\infty\Sigma^nA=\begin{cases}
		K(A,n) & \text{if }n\ge0, \\
		\Delta^0 & \text{if }n<0.
	\end{cases}\]
\end{example}

\subsection{Generalized Cohomology}
Here is our definition.
\begin{definition}[generalized cohomology]
	Fix a spectrum $E$. Then we define the \textit{generalized reduced $E$-coho\-mo\-logy groups} $\widetilde E^n\colon\mathrm{Spaces}_*\to\mathrm{Ab}$ by
	\[\widetilde E^n(X)\coloneqq\pi_{-n}\underline{\op{Hom}}_{\mathrm{Spectra}}(\Sigma^\infty X,E).\]
	(This definition immediately extends to the case where $\Sigma^\infty X$ is replaced by a general spectrum.) We further define the \textit{generalized $E$-cohomology} $E^n\colon\mathrm{Spaces}_*\to\mathrm{Ab}$ by $E^n(X)\coloneqq\widetilde E^n(X_+)$.
\end{definition}
The moral is that a spectrum gives rise to a generalized cohomology theory. Here are some instances of this.
\begin{remark}
	Equivalently, we see $\widetilde E^n(X)$ is also
	\[\pi_0\underline{\op{Hom}}_{\mathrm{Spectra}}(\Sigma^\infty X,\Sigma^nE)=\pi_0\underline{\op{Hom}}_{\mathrm{Spaces}_*}(X,\Omega^\infty\Sigma^nE).\]
	For example, for the Eilenberg--MacLane spectrum $A$, we find that $\widetilde E^n(X)$ is $\mathrm H^n(X;A)$.
\end{remark}
\begin{remark}
	One can check that $E^n(X)=\widetilde E^n(X)\oplus E\left(\Delta^0\right)$, as one would expect from ordinary cohomology.
\end{remark}
\begin{remark} \label{rem:cohomology-on-point}
	One sees that $E^n\left(\Delta^0\right)$ is $\widetilde E^n\left(S^0\right)$, which is $\pi_0\underline{\op{Hom}}_{\mathrm{Spectra}}(S^n,E)$. One can further calculate this as $\pi_0\underline{\op{Hom}}(\Sigma^n\mathbb S,E)$, which is $\pi_0\underline{\op{Hom}}(\mathbb S,\Omega^nE)$, which is $\pi_0\underline{\op{Hom}}_{\mathrm{Spaces}_*}(S^0,\Omega^\infty\Omega^nE)$, which is
	\[\pi_0\underline{\op{Hom}}_{\mathrm{Spaces}}(\Delta^0,\Omega^\infty\Omega^nE)=\pi_nE.\]
	Thus, the sphere spectrum (and more generally, the cohomology theory) of $E$ remembers the homotopy groups of $E$.
\end{remark}
\begin{lemma} \label{lem:e-infinity-spectra-as-colim}
	Fix a group-like $\mathbb E_\infty$-space $E$. Then
	\[E=\colim_n\Sigma^{-n}\Sigma^\infty\Omega^\infty\Sigma^nE.\]
\end{lemma}
\begin{proof}
	The comparison maps are induced by moving the functor $\Sigma^{-1}=\Omega$ inside the limit $\Sigma^\infty\Omega^\infty$. (This also explains how we can construct a map from the colimit to $E$.) Note $\pi_i\Omega^\infty\Sigma^nE=\op{Hom}(S^i,\Omega^\infty\Sigma^nE)=\op{Hom}_{\mathrm{Spectra}}(S^i,\Sigma^nE)=\pi_{i-n}E$, so $\pi_i\Omega^\infty\Sigma^nE$ vanishes for $i<n$. Now, by the Freudenthal suspension theorem, we see that the stable homotopy groups of $\Omega^\infty\Sigma^nE$ agree with the homotopy groups up to degree $2n-1$. Thus, both sides of the colimit have the same homotopy groups up to degree $n$ or so, so we are done by sending $n$ to infinity.
\end{proof}
\begin{remark}
	The moral of \Cref{lem:e-infinity-spectra-as-colim} is that we can write a group-like $\mathbb E_\infty$-space $E$ as a colimit of suspension spectra.
\end{remark}
\begin{example}
	View $E\coloneqq\FF_2$ as an Eilenberg--MacLane spectrum. The $\mathbb F_2$-cohomology of $\FF_2$ is the Steenrod algebra.
\end{example}
\begin{proof}
	By \Cref{lem:e-infinity-spectra-as-colim}, we see that the cohomology is
	\[\widetilde E^*(\FF_2)=\widetilde E^*\left(\colim_n\Sigma^{-n}\Sigma^\infty\Omega^\infty\Sigma^n\FF_2\right)=\lim_n\widetilde E^*(\Sigma^{-n}\Sigma^\infty\underbrace{\Omega^\infty\Sigma^n\FF_2}_{K(\FF_2n)}).\]
	We can now compute this directly as the limit of $\pi_0\mathrm{Hom}_{\mathrm{Spectra}}(\Sigma^{-n}\Sigma^\infty K(\FF_2,n),\Sigma^*\FF_2)$, which is then the limit of $\pi_0\underline{\op{Hom}}(K(\FF_2,n),K(\FF_2,n+*))=\mathrm H^{*+n}(K(\FF_2,n);\FF_2)$ after undoing adjunctions. Now, Serre's theorems recasts this as the Steenrod algebra, where $\mathrm{Sq}^i$ lives in degree $i$.
\end{proof}
\begin{remark}
	One way to recast the algebra structure on the Steenrod algebra is that $\mathrm{Sq}^i$ is some map $\FF_2\to\Sigma^i\FF_2$, which can be composed as follows: $\op{Sq}^i\op{Sq}^j$ is the composite
	\[\FF_2\stackrel{\op{Sq}^j}\to\Sigma^i\FF_2\stackrel{\Sigma^i\op{Sq}^j}\to\Sigma^{i+j}\FF_2.\]
	Notably, the cohomology of spectra still has these Steenrod operations, but we do not have cup products!
\end{remark}
Thus far, we have explained that the cohomology theory of a spectrum remembers interesting information about the spectrum. However, it also turns out that behaved cohomology theories all come from spectra.
\begin{theorem}[Brown representability]
	Fix a sequence of functors $\widetilde h^n\colon\mathrm{Spaces}_{*,\ge1}\to\mathrm{Ab}$ satisfying Mayer--Vietoris. Then there is a spectrum $E$ with $\widetilde E^n=\widetilde h^n$.
\end{theorem}
\begin{remark}
	It is possible for different spectra to give the same cohomology theory. However, giving rise to the same cohomology theory is a very strong constraint; for example, the homotopy groups must agree by \Cref{rem:cohomology-on-point}. In general, there is a somewhat extensive theory of ``phantom maps'' explaining how such things happen.
\end{remark}
\begin{remark}
	Here is a fairly banal analogue of the homological Whitehead theorem: if $f\colon X\to Y$ induces isomorphisms $\widetilde E^n(X)\to\widetilde E^n(Y)$ for all spectra $E$ and $n\in\ZZ$, then $f$ induces an equivalence $\Sigma^\infty X\to\Sigma^\infty Y$. Indeed, one can just plug in $E=\Sigma^\infty X$ and apply the Yoneda lemma. However, we remark that it is possible for $X\not\cong Y$ while $\Sigma^\infty X\cong\Sigma^\infty Y$.
\end{remark}

\subsection{The Tensor Product}
We would like to form a tensor product $E\otimes F$ of spectra, which we do via adjunctions.
\begin{definition}[tensor product]
	Given spectra $A$ and $B$, we define $A\otimes B$ by the Yoneda lemma via
	\[\underline{\op{Hom}}(A\otimes B,-)=\underline{\op{Hom}}(A,\underline{\op{Hom}}(B,-)).\]
	Existence follows from an adjoint functor theorem: the functor $-\otimes B$ is constructed as the left adjoint to $\underline{\op{Hom}}(B,-)$. We may write this as $A\otimes_{\mathbb S}B$ for emphasis.
\end{definition}
\begin{remark}
	There are things to check about this tensor product, such as commutativity and associativity. Because we defined $-\otimes_{\mathbb S}B$, we also see that the tensor product commutes with colimits.
\end{remark}
\begin{example}
	We see
	\[\underline{\op{Hom}}(A\otimes\mathbb S,-)=\underline{\op{Hom}}(A,\underline{\op{Hom}}(\mathbb S,-))=\underline{\op{Hom}}(A,-).\]
	Thus, $A\otimes\mathbb S=A$.
\end{example}
\begin{remark}
	If $X$ and $Y$ are unpointed spaces, then
	\[\underline{\op{Hom}}(X\times Y,-)=\underline{\op{Hom}}(X,\underline{Hom}(Y,-)),\]
	so $\Sigma^\infty_+(X\times Y)=\Sigma_+^\infty X\otimes\Sigma^\infty_+Y$. This explains how to compute tensor products of suspension spectra. Thus, if $E$ and $F$ are spectra with no negative homotopy groups, then the tensor product $E\otimes F$ arises as a colimit via \Cref{lem:e-infinity-spectra-as-colim}. This explains why it may be commutative.
\end{remark}
\begin{example}
	If $A$ and $B$ are discrete abelian groups, then $A\otimes_{\mathbb S}B$ has no reason to be the discrete abelian group $A\otimes_\ZZ B$ (though they will agree in $\pi_0$). One could even do a derived tensor product $A\otimes_\ZZ^{\mathrm{L}}B$, which agrees in $\pi_1$ with $A\otimes_{\mathbb S}B$ but not higher.
\end{example}
\begin{definition}[generalized homology]
	Fix a spectrum $E$ and a spectrum $X$. Then we define the \textit{generalized $E$-homology} by
	\[\widetilde E_*(X)\coloneqq\pi_*(E\otimes_{\mathbb S}X).\]
	If $X$ is a pointed space, we may abbreviate $\widetilde E_*(\Sigma^\infty X)$ to $\widetilde E_*(X)$. If $X$ is an unpointed space, we define $E_*(X)\coloneqq\widetilde E_*(X_+)$.
\end{definition}
\begin{example}
	Taking $E=\mathbb S$ recovers stable homotopy groups.
\end{example}
\begin{example}
	Because $\Sigma^\infty_+$ and tensor products both preserve colimits, we see that we will be able to produce a long exact sequence in generalized $E$-homology.
\end{example}

\end{document}