% !TEX root = ../notes.tex

\documentclass[../notes.tex]{subfiles}

\begin{document}

\section{October 7}
Today we will continue using the Serre spectral sequence.

\subsection{Hurewicz's Theorem}
Throughout, $F\to E\to B$ is a homotopy fiber sequence of pointed Kan complexes with $B$ simply connected. For any ring $R$, we then receive a Serre spectral sequence
\[E^2_{pq}=\mathrm H_p(B;\mathrm H_q(F;R))\Rightarrow\mathrm H_{p+q}(E;R).\]
We will also assume that $F$ is connected, which implies that $\mathrm H_0(F;R)=R$, so $E^2_{0q}=\mathrm H_q(F;R)$ and $E^2_{p0}=\mathrm H_p(B;R)$, which we visualize as follows.
% https://q.uiver.app/#q=WzAsMTUsWzAsMywiUiJdLFsxLDMsIlxcbWF0aHJtIEhfMShCO1IpIl0sWzIsMywiXFxtYXRocm0gSF8yKEI7UikiXSxbMywzLCJcXGNkb3RzIl0sWzEsMiwiPyJdLFsxLDEsIj8iXSxbMiwxLCI/Il0sWzIsMiwiPyJdLFszLDIsIlxcY2RvdHMiXSxbMywxLCJcXGNkb3RzIl0sWzEsMCwiXFx2ZG90cyJdLFsyLDAsIlxcdmRvdHMiXSxbMCwwLCJcXHZkb3RzIl0sWzAsMiwiXFxtYXRocm0gSF8xKEY7UikiXSxbMCwxLCJcXG1hdGhybSBIXzIoRjtSKSJdXQ==&macro_url=https%3A%2F%2Fraw.githubusercontent.com%2FdFoiler%2Fnotes%2Fmaster%2Fnir.tex
\[\begin{tikzcd}[cramped,sep=tiny]
	\vdots & \vdots & \vdots \\
	{\mathrm H_2(F;R)} & {?} & {?} & \cdots \\
	{\mathrm H_1(F;R)} & {?} & {?} & \cdots \\
	R & {\mathrm H_1(B;R)} & {\mathrm H_2(B;R)} & \cdots
\end{tikzcd}\]
The direction of the arrows implies that the $E^\infty$ page will contain a quotient of $\mathrm H_q(F;R)$ and a subset of $\mathrm H_p(B;R)$.
\begin{remark} \label{rem:filtration-pieces-serre-spectral}
	It turns out that the aforementioned quotient of $\mathrm H_q(F;R)$ is the image from $\mathrm H_q(E;R)$, and the aforementioned subset of $\mathrm H_p(B;R)$ is the image of $\mathrm H_p(E;R)$. Both of these follow from unwinding the construction of the spectral sequence. For example, $E^\infty_{0q}$ should come from the filtered complex $C_\bullet(E;R)$ as the pieces which come from the zeroth part of the filtration, which is $C_\bullet(F;R)$.
\end{remark}
Let's explain how we can try to compute some homotopy groups. Hopefully one can handle $\pi_0$ and $\pi_1$. (Perhaps $\pi_1$ is not easy, but at least its abelianization is $\mathrm H_1$.) If we wanted to compute $\pi_2(X)$ for a simply connected Kan complex $X$, then we see that
\[\pi_2(X)=\pi_2(X)^{\mathrm{ab}}=\pi_1(\Omega X)^{\mathrm{ab}}=\mathrm H^1(\Omega X;\ZZ),\]
where we are using that the higher homotopy groups are abelian (via \Cref{lem:eh}) and \Cref{rem:homotopy-of-omega}. But now we can try to compute $\mathrm H_1(\Omega X;\ZZ)$ using the homotopy fiber sequence $\Omega X\to\Delta^0\to X$ of \Cref{lem:loop-as-fiber}. Then the Serre spectral sequence (\Cref{thm:serre-spectral-seq}) produces the following $E_2$ page.
% https://q.uiver.app/#q=WzAsMTYsWzAsMywiXFxaWiJdLFsxLDMsIlxcbWF0aHJtIEhfMShYKSJdLFsyLDMsIlxcbWF0aHJtIEhfMihYKSJdLFszLDMsIlxcbWF0aHJtIEhfMyhYKSJdLFswLDIsIlxcbWF0aHJtIEhfMShcXE9tZWdhIFgpIl0sWzAsMSwiXFxtYXRocm0gSF8yKFxcT21lZ2EgWCkiXSxbMCwwLCJcXG1hdGhybSBIXzMoXFxPbWVnYSBYKSJdLFsxLDIsIj8iXSxbMiwyLCI/Il0sWzMsMiwiPyJdLFsxLDAsIj8iXSxbMiwwLCI/Il0sWzMsMCwiPyJdLFszLDEsIj8iXSxbMiwxLCI/Il0sWzEsMSwiPyJdXQ==&macro_url=https%3A%2F%2Fraw.githubusercontent.com%2FdFoiler%2Fnotes%2Fmaster%2Fnir.tex
\begin{equation}
	\begin{tikzcd}[cramped,sep=tiny]
		{\mathrm H_3(\Omega X)} & {?} & {?} & {?} \\
		{\mathrm H_2(\Omega X)} & {?} & {?} & {?} \\
		{\mathrm H_1(\Omega X)} & {?} & {?} & {?} \\
		\ZZ & {\mathrm H_1(X)} & {\mathrm H_2(X)} & {\mathrm H_3(X)}
	\end{tikzcd} \label{eq:serre-on-omega}
\end{equation}
Note $\mathrm H_1(X)=\pi_1(X)^{\mathrm{ab}}=0$ because $X$ is simply connected. We thus see that $E^\infty_{01}$ needs to vanish (because it is computing homology of $\Delta^0$), so $d_2\colon\mathrm H_2(X)\to\mathrm H_1(\Omega X)$ must be an isomorphism. We have thus proven the following.
\begin{proposition}
	Suppose $X$ is a simply connected pointed Kan complex. Then
	\[\pi_2(X)=\mathrm H_2(X;\ZZ).\]
\end{proposition}
Here is the more general result.
\begin{theorem}[Hurewicz] \label{thm:hurewisz}
	Fix a pointed Kan complex $X$ and a positive integer $n\ge2$. If $\pi_k(X)=0$ for $k<n$, then the natural map
	\[\pi_n(X)\to\mathrm H_n(X;\ZZ)\]
	is an isomorphism.
\end{theorem}
\begin{proof}
	The natural map sends a loop to the corresponding cycle; in other words, it sends the element $\Delta^n\to X$ to the image in $\mathrm H_n(S^n;\ZZ)\to\mathrm H_n(X;\ZZ)$.
	
	The proof is by an induction exactly as above. Suppose we have the statement at level $n$, and we would like to show it for $n+1$. In particular, we know that $\pi_k(X)=0$ for $k<n+1$, so the induction tells us that $\mathrm H_k(X)$ for $k<n+1$. Now, we once again note that
	\[\pi_{n+1}(X)=\pi_n(\Omega X)\cong\mathrm H_n(\Omega X),\]
	and we would like to show that this right-hand group is isomorphic to $H_{n+1}(X)$. For example, we get to know that $\mathrm H_k(\Omega X)=0$ for $k<n$. As usual, we will use the Serre spectral sequence (\Cref{thm:serre-spectral-seq}), which outputs the $E_2$ page \eqref{eq:serre-on-omega}. However, basically everything vanishes, and when we want to compute $\mathrm H_n(\Delta)$, we find that everything has vanished except $\mathrm H_n(\Omega X)$, which only admits the differential
	\[d\colon\mathrm H_{n+1}(X)\to\mathrm H_n(\Omega X).\]
	Thus, to vanish in $E_\infty$, this must be an isomorphism, and the result follows. One can check that the various composites involved do produce the above natural map by unwinding the spectral sequence in this simple case.
\end{proof}
\begin{example}
	One can show that $\pi_0S^3$ and $\pi_1S^3$ are both trivial, so \Cref{thm:hurewisz} tells us that
	\[\pi_2S^3=\mathrm H^2\left(S^3;\ZZ\right)=0.\]
	Thus, \Cref{thm:hurewisz} kicks in to tell us that
	\[\pi_3S^3=\mathrm H^3\left(S^3;\ZZ\right)\cong\ZZ.\]
	In general, one finds that $\pi_kS^n$ is trivial for $k<n$ by an induction. Then $\pi_nS^n\cong\mathrm H^n(S^n;\ZZ)=\ZZ$ for $n\ge1$.
\end{example}
\begin{example} \label{ex:pi4-s3}
	We compute $\pi_4S^3\cong\ZZ/2\ZZ$.
\end{example}
\begin{proof}
	There is a map $f\colon S^3\to K(\ZZ,3)$ given by choosing a generator of $\mathrm H^3(S^3;\ZZ)\cong\ZZ$. This is an isomorphism on $\mathrm H^i$ for $i\le3$ (the groups are trivial at $i<3$), so the Universal coefficient theorem tells us that it is also an isomorphism on $\mathrm H_i$ for $i\le3$, and \Cref{thm:hurewisz} then tells us that it is an isomorphism on $\pi_*$ for $i\le3$. Now, let $F$ be the homotopy fiber of $f$, so the long exact sequence in homotopy (\Cref{rem:homotopy-les}) produces the exact sequence
	\[\pi_{i+1}K(\ZZ,3)\to\pi_iF\to\pi_iS^3\to\pi_iK(\ZZ,3),\]
	so plugging in various vanishings shows that $\pi_iF$ is trivial for $i\in\{0,1,2,3\}$ and $\pi_4F\cong\pi_4S^3$.
	
	But now we can use \Cref{thm:hurewisz} to see that $\pi_4F\cong\mathrm H_4(F;\ZZ)$! We can now use the Serre spectral sequence. Comparing with the Universal coefficient theorem, one can compute some homology of $K(\ZZ,3)$ via comparing with its cohomology as computed in \Cref{exe:cohomology-k-z-3}. In particular, one finds that
	\[\mathrm H_i(K(\ZZ,3);\ZZ)\cong\begin{cases}
		\ZZ & \text{if }i\in\{0,4,6\}, \\
		0 & \text{if }i\in\{1,2,4,6\}, \\
		\ZZ/2\ZZ & \text{if }i=5, \\
		\ZZ/3\ZZ & \text{if }i=6.
	\end{cases}\]
	(Note that there is some degree shift due to the presence of $\op{Ext}$ groups in the Universal coefficient theorem.) In particular, $E^2_{pq}=\mathrm H_p(K(\ZZ,3);\mathrm H_q(F))$ looks like the following.
	% https://q.uiver.app/#q=WzAsMzAsWzAsNCwiXFxaWiJdLFsxLDQsIjAiXSxbMiw0LCIwIl0sWzMsNCwiXFxaWiJdLFs0LDQsIjAiXSxbNSw0LCJcXFpaLzJcXFpaIl0sWzEsMywiMCJdLFsyLDMsIjAiXSxbMSwyLCIwIl0sWzIsMiwiMCJdLFs0LDMsIjAiXSxbNCwyLCIwIl0sWzQsMSwiMCJdLFswLDMsIjAiXSxbMCwyLCIwIl0sWzAsMSwiMCJdLFszLDMsIjAiXSxbNSwzLCIwIl0sWzMsMiwiMCJdLFs1LDIsIjAiXSxbMSwxLCIwIl0sWzIsMSwiMCJdLFszLDEsIjAiXSxbNSwxLCIwIl0sWzAsMCwiXFxwaV80U14zIl0sWzEsMCwiPyJdLFsyLDAsIj8iXSxbMywwLCI/Il0sWzQsMCwiPyJdLFs1LDAsIj8iXV0=&macro_url=https%3A%2F%2Fraw.githubusercontent.com%2FdFoiler%2Fnotes%2Fmaster%2Fnir.tex
	\[\begin{tikzcd}[cramped,sep=tiny]
		{\pi_4S^3} & {?} & {?} & {?} & {?} & {?} \\
		0 & 0 & 0 & 0 & 0 & 0 \\
		0 & 0 & 0 & 0 & 0 & 0 \\
		0 & 0 & 0 & 0 & 0 & 0 \\
		\ZZ & 0 & 0 & \ZZ & 0 & {\ZZ/2\ZZ}
	\end{tikzcd}\]
	Thus, we see that $\pi_4S^3$ will only interact with the differential $d_5\colon\ZZ/2\ZZ\to\pi_4S^3$, but in $E^\infty$ these groups compute $\mathrm H^4\left(S^3\right)=\mathrm H^5\left(S^3\right)=0$, so it follows that $d_5$ is an isomorphism, so we conclude that $\pi_4S^3\cong\ZZ/2\ZZ$.
\end{proof}
\begin{remark}
	Heuristically, we thus see that the homotopy groups of spheres ``know about'' the (co)\-homology of Eilenberg--MacLane spaces. This is one reason to care about them.
\end{remark}

\subsection{The Whitehead Tower}
Fix a pointed Kan complex $X$, and we will build a ``tower'' of spaces.
\begin{enumerate}
	\item We define $\tau_{\ge1}X$ to be the path component of the basepoint, which we can rigorously think of as the full $\infty$-subcategory generated by the objects isomorphic to the basepoint. Then we see that the inclusion $\tau_{\ge1}X$ is a $\pi_*$-isomorphism in degrees bigger than $0$, but $\pi_0\tau_{\ge1}X=0$.
	\item Then let $\tau_{\ge2}X$ be the universal cover of $\tau_{\ge1}X$, and we see that $\tau_{\ge2}X\to\tau_{\ge1}X$ is a $\pi_*$ isomorphism in degrees bigger than $1$, but $\pi_1(\tau_{\ge2}X)$ vanishes.
	\item To continue, note that $\tau_{\ge2}X$ is simply connected, so \Cref{thm:hurewisz} tells us that $A\coloneqq\pi_2(\tau_{\ge2}X)$ is isomorphic to $\mathrm H_2(\tau_{\ge2}X;\ZZ)\cong\mathrm H^2(\tau_{\ge2}X;\ZZ)$ and thus produces a map
	\[\tau_{\ge2}X\to K(A,2)\]
	which is further an isomorphism on $\pi_2$. We then define $\tau_{\ge3}X$ as the homotopy fiber of this map so that the inclusion $\tau_{\ge3}X\to\tau_{\ge2}X$ is a $\pi_*$-isomorphism in degrees at least $3$, but $\pi_2(\tau_{\ge3}X)$ vanishes.
\end{enumerate}
Iterating the process in the last step, we see that we have produced a tower
% https://q.uiver.app/#q=WzAsNyxbMCw0LCJYIl0sWzAsMywiXFx0YXVfe1xcZ2UxfVgiXSxbMCwyLCJcXHRhdV97XFxnZTJ9WCJdLFsxLDIsIksoXFxwaV8yWCwyKSJdLFsxLDEsIksoXFxwaV8yWCwyKSJdLFswLDEsIlxcdGF1X3tcXGdlM31YIl0sWzAsMCwiXFx2ZG90cyJdLFsyLDNdLFs1LDRdLFs2LDVdLFs1LDJdLFsyLDFdLFsxLDBdXQ==&macro_url=https%3A%2F%2Fraw.githubusercontent.com%2FdFoiler%2Fnotes%2Fmaster%2Fnir.tex
\begin{equation}
	\begin{tikzcd}[cramped]
		\vdots \\
		{\tau_{\ge3}X} & {K(\pi_2X,2)} \\
		{\tau_{\ge2}X} & {K(\pi_2X,2)} \\
		{\tau_{\ge1}X} \\
		X
		\arrow[from=1-1, to=2-1]
		\arrow[from=2-1, to=2-2]
		\arrow[from=2-1, to=3-1]
		\arrow[from=3-1, to=3-2]
		\arrow[from=3-1, to=4-1]
		\arrow[from=4-1, to=5-1]
	\end{tikzcd} \label{eq:whitehead}
\end{equation}
where $\tau_{\ge n+1}X\to\tau_{\ge n}X\to K(\pi_nX,n)$ is a homotopy fiber sequence for each $n$, and this implies that
\[\pi_k(\tau_{\ge n}X)=\begin{cases}
	0 & \text{if }k<n, \\
	\pi_k(X) & \text{if }k\ge n.
\end{cases}\]
\begin{definition}[Whitehead tower]
	Given a pointed Kan complex $X$, we define the \textit{Whitehead tower} as the tower constructed in \eqref{eq:whitehead}.
\end{definition}
\begin{remark}
	Because we have built these things as homotopy fiber sequences, the entire tower is unique up to homotopy.
\end{remark}
\begin{remark}
	The point is that the $\tau_{\ge k}X$ have lots of earlier vanishing homotopy groups, so one can hope to use the method of \Cref{ex:pi4-s3} to compute homotopy groups if only we understood the homology of the $\tau_{\ge k}$s well enough.
\end{remark}
\begin{remark}
	There is also a ``Postnikov tower''
	\[\cdots\to\tau_{\le2}X\to\tau_{\le1}X\to\tau_{\le0}X\]
	for any unpointed Kan complex $X$; it turns out that $X$ is the limit of this diagram. If $X$ is pointed, then one can reconstruct $\tau_{\ge n+1}X$ as the homotopy fiber of the canonical map $X\to\tau_{\le n}X$. In general, $\tau_{\le0}X=\bigsqcup_{\pi_0X}\Delta^0$ and $\tau_{\le1}X$ is the homotopy category of $X$; going further, $\tau_{\le n}X$ should be the ``homotopy $n$-category.''
\end{remark}

\subsection{Whitehead's Theorem for Homology}
We conclude with a theorem.
\begin{theorem}[Whitehead, homology]
	Fix a map $f\colon X\to Y$ of simply connected pointed Kan complexes. Then $f$ is a homotopy equivalence if and only if $\mathrm H_*(f;\ZZ)$ is an isomorphism of graded groups.
\end{theorem}
\begin{proof}
	Let $F$ be the homotopy fiber of $f$. To show that $f$ is a homotopy equivalence, it is equivalent to show that $\pi_*f$ is an isomorphism in all degrees (by \Cref{thm:whitehead}) which is equivalent to showing that $\pi_*F$ is trivial in all degrees. By \Cref{thm:hurewisz}, it is now enough to just check that $\pi_1F=0$ and $\mathrm H_n(F)=0$ for all $n\ge1$.
	
	For one, $\pi_1(F)=0$ by the homotopy long exact sequence (\Cref{rem:homotopy-les}), which works because $X$ and $Y$ are simply connected. This is a little involved, so let's explain. Note $\pi_0F$ trivializes because of the exact sequence $\pi_1Y\to\pi_0F\to\pi_0X$. For $\pi_1F$, we write
	\[\pi_2X\to\pi_2Y\to\pi_1F\to\pi_1X.\]
	The last group vanishes because $X$ is simply connected, so it remains to show that the map $\pi_2X\to\pi_2Y$ is surjective. Well, by \Cref{thm:hurewisz}, we see that the map $\pi_2X\to\pi_2Y$ is isomorphic to the map $\mathrm H_2(X;\ZZ)\to\mathrm H_2(Y;\ZZ)$, which is an isomorphism by hypothesis on $f$.

	It remains to show that $\mathrm H_n(F)=0$ for $n\ge1$, for which we use the Serre spectral sequence (\Cref{thm:serre-spectral-seq}). Well, \Cref{rem:filtration-pieces-serre-spectral} has informed us that $E^\infty_{p0}$ is the image of $\mathrm H_p(X)\to\mathrm H_p(Y)$. But this is an isomorphism, so there cannot be any differentials outside this, and everything must vanish on $E^\infty$ in order to get the right answer. One then inductively finds that $\mathrm H_n(F)=0$ for $n\ge1$ by studying how to get the Serre spectral sequence to never modify $E^\bullet_{p0}$s.
\end{proof}
\begin{remark}
	This is amazing, and suggests that we may as well spend out time studying homology. However, it turns out that other fields of mathematics find themselves caring about homotopy groups anyway, so it is worthwhile to have tools to compute them.
\end{remark}

\end{document}