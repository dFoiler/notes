% !TEX root = ../notes.tex

\documentclass[../notes.tex]{subfiles}

\begin{document}

\section{October 28}
Today, we will continue discussing spectra.

\subsection{Spectra}
We can generalize our notion of group-like $\mathbb E_\infty$-spaces being thought of as infinite loop spaces.
\begin{definition}[spectrum]
	A \textit{spectrum} is a sequence $(X_0,X_1,\ldots)$ of pointed spaces along with equivalences $X_i\to\Omega X_{i+1}$.
\end{definition}
\begin{remark}
	A morphism $X\to Y$ of spectra is a morphism $X_i\to Y_i$ of the underlying pointed spaces, along with homotopies of the equivalences $X_i\to\Omega X_{i+1}$. Diagrammatically, we are providing vertical arrows in the diagram
	% https://q.uiver.app/#q=WzAsOCxbMCwwLCJYXzAiXSxbMSwwLCJcXE9tZWdhIFhfMSJdLFsyLDAsIlxcT21lZ2FeMlhfMiJdLFszLDAsIlxcY2RvdHMiXSxbMCwxLCJZXzAiXSxbMSwxLCJcXE9tZWdhIFlfMSJdLFsyLDEsIlxcT21lZ2FeMiBZXzIiXSxbMywxLCJcXGNkb3RzIl0sWzAsNF0sWzAsMV0sWzEsMl0sWzIsM10sWzQsNV0sWzUsNl0sWzYsN10sWzEsNV0sWzIsNl1d&macro_url=https%3A%2F%2Fraw.githubusercontent.com%2FdFoiler%2Fnotes%2Fmaster%2Fnir.tex
	\[\begin{tikzcd}[cramped]
		{X_0} & {\Omega X_1} & {\Omega^2X_2} & \cdots \\
		{Y_0} & {\Omega Y_1} & {\Omega^2 Y_2} & \cdots
		\arrow[from=1-1, to=1-2]
		\arrow[from=1-1, to=2-1]
		\arrow[from=1-2, to=1-3]
		\arrow[from=1-2, to=2-2]
		\arrow[from=1-3, to=1-4]
		\arrow[from=1-3, to=2-3]
		\arrow[from=2-1, to=2-2]
		\arrow[from=2-2, to=2-3]
		\arrow[from=2-3, to=2-4]
	\end{tikzcd}\]
	along with ``square homotopies'' in each of the given squares.
\end{remark}
\begin{example} \label{ex:e-inf-is-spectra}
	Any group-like $\mathbb E_\infty$-space produces a spectrum.
\end{example}
\begin{example}
	As an instance of \Cref{ex:e-inf-is-spectra}, we recall that a discrete abelian group $A$ gives rise to the spectrum
	\[A=\Omega K(A,1)=\Omega^2K(A,2)=\cdots.\]
	In fact, \Cref{ex:discrete-abelian-group-as-e-inf} shows that this is already an $\mathbb E_\infty$-space.
\end{example}
\begin{example} \label{ex:space-to-spec}
	Given a space $X$, we may follow \Cref{ex:space-to-e-inf} to construct a group-like $\mathbb E_\infty$-space $\Sigma^\infty X$ from a pointed space $X$. Indeed, we define $\Sigma^\infty X$ as the colimit
	\[X\to\Omega\Sigma X\to\Omega^2\Sigma^2X\to\cdots,\]
	where the internal maps arise from the adjunction of $\Omega$ and $\Sigma$. Namely, we define $X_0$ as the above colimit, and $X_i$ in general is the colimit of $\Sigma^iX\to\Omega\Sigma^{i+1}X\to\Omega^2\Sigma^{i+2}X\to\cdots$. It follows that $X_i=\Omega X_{i+1}$ because $\Omega=\op{Mor}\left(S^1,-\right)$ preserves filtered colimits (because $S^1$ is a compact object).
\end{example}
\begin{example}
	As an instance of \Cref{ex:space-to-spec}, we note that $\Delta^0=\Omega\Delta^0$, so $\Delta^0$ is a spectrum. We may call this the ``zero'' spectrum $0$ because it is both initial and terminal. (Indeed, $\Delta^0$ is already initial and terminal in pointed spaces.)
\end{example}
\begin{remark}
	The functor $\Sigma^\infty\colon\mathrm{Spaces}_*\to\mathrm{Specta}$ admits a right adjoint $\Omega^\infty$ defined by $\Omega^\infty X\coloneqq X_0$. Approximately speaking, $\Omega^\infty$ should be thought of as a forgetful functor removing all ``higher'' information.
\end{remark}
Here is a nontrivial example.
\begin{theorem}[Bott periodicity] \label{thm:bott-periodicity}
	Let $\op U$ be the infinite unitary group, which is the colimit of the finite unitary groups. It turns out that $\Omega(B{\op U}\times\ZZ)\simeq\op U$ and $\Omega U\simeq B\op U\times\ZZ$.
\end{theorem}
\begin{example}
	There is a spectrum $KU\coloneqq(B{\op U}\times\ZZ,{\op U},B{\op U}\times\ZZ,{\op U},\ldots)$.
\end{example}
\begin{remark}
	It is not a joke that both spectra and $\mathbb E_\infty$-spaces should be thought of as analogous to abelian groups. For a taste of how this works, we know that the Dold--Kan correspondence (\Cref{thm:dold-kan}) explains that the category of simplicial abelian groups is equivalent to the derived $\infty$-category of chain complexes supported in nonnegative degrees, so one can more or less view $\mathbb E_\infty$ as analogous this derived category. Then spectra are analogous to allowing chain complexes in all integral degrees.
\end{remark}

\subsection{Stability}
The $\infty$-category of spectra will turn out to have many nice properties. The most important of these is stability, which we now explain.
\begin{notation}
	Let $\mc C$ be an $\infty$-category with a $0$ object and finite limits. Then we define the functor $\Omega_{\mc C}\colon\mc C\to\mc C$ by sending an object $x\in\mc C$ to the limit of the following diagram.
	% https://q.uiver.app/#q=WzAsMyxbMSwwLCIwIl0sWzAsMSwiMCJdLFsxLDEsIngiXSxbMSwyXSxbMCwyXV0=&macro_url=https%3A%2F%2Fraw.githubusercontent.com%2FdFoiler%2Fnotes%2Fmaster%2Fnir.tex
	\[\begin{tikzcd}[cramped]
		& 0 \\
		0 & x
		\arrow[from=1-2, to=2-2]
		\arrow[from=2-1, to=2-2]
	\end{tikzcd}\]
	If there is no risk of confusion, will abbreviate $\Omega_{\mc C}$ to $\Omega$.
\end{notation}
\begin{example} \label{ex:spectra-is-stable}
	The category of spectra has $\Omega$ arise from the usual loops: it sends $(X_0,X_1,X_2,\ldots)$ to $(\Omega X_0,X_0,X_1,\ldots)$. In fact, we see that $\Omega$ is an auto-equivalence on spectra. The inverse functor is given by
	\[(X_0,X_1,\ldots)\mapsto(X_1,X_2,\ldots)\]
	because $X_0=\Omega X_1$. The point is that a spectrum can always be recovered from its ``higher'' terms.
\end{example}
\begin{definition}[stable]
	Let $\mc C$ be an $\infty$-category with a $0$ object and finite limits. Then we say that $\mc C$ is \textit{stable} if and only if $\Omega_{\mc C}$ is an equivalence. In this situation, we may write $\Sigma_{\mc C}$ for the inverse equivalence, which we will abbreviate to $\Sigma$ whenever possible.
\end{definition}
\begin{example}
	Consider the derived category $\mc D(\ZZ)$ of chain complexes of abelian groups. It turns out that $\Omega$ shifts a chain complex to the right by $1$, so $\Omega$ is an auto-equivalence.
\end{example}
\begin{remark}
	In \Cref{ex:spectra-is-stable}, we checked that the category of spectra is stable. However, the corresponding functor $\Sigma$ is not the usual suspension of pointed spaces.
\end{remark}
The following result explains why stability is a nice notion.
\begin{proposition} \label{prop:stability-is-good}
	Fix a stable $\infty$-category $\mc C$.
	\begin{listalph}
		\item The $\infty$-category $\mc C$ admits finite colimits.
		\item The functor $\Sigma$ sends an object $x$ to the colimit of the diagram $0\from c\to 0$.
		\item A square in $\mc C$ is a pullback if and only if it is a pushout.
	\end{listalph}
\end{proposition}
\begin{remark}
	One should think of (c) as analogous to the result for additive categories that products and coproducts of two objects exist and are the same. For example, it follows that the canonical map $X\sqcup Y\to X\times Y$ must be an isomorphism.
\end{remark}
\begin{lemma} \label{lem:use-stability}
	Fix a stable $\infty$-category $\mc C$. Let $\mc P\subseteq\underline{\op{Hom}}\left(\Delta^1\times\Delta^1,\mc C\right)$ be the full subcategory of pullback squares. Then the forgetful functor $\mc P\to\underline{\op{Hom}}(\Lambda^2_0,\mc C)$ is an equivalence.
\end{lemma}
\begin{proof}
	We exhibit an inverse ``up to loops.'' Namely, given some object $C\from A\to B$ in $\underline{\op{Hom}}(\Lambda^2_0,\mc C)$, we construct $F$, $G$, and $X$ to fit into the diagram
	% https://q.uiver.app/#q=WzAsMTMsWzIsMywiQyJdLFsyLDIsIkEiXSxbMywyLCJCIl0sWzEsMywiMCJdLFszLDEsIjAiXSxbMiwxLCJGIl0sWzEsMiwiRyJdLFsxLDEsIlgiXSxbMSwwLCJcXE9tZWdhIEMiXSxbMiwwLCIwIl0sWzAsMSwiXFxPbWVnYSBCIl0sWzAsMiwiMCJdLFswLDAsIlxcT21lZ2EgQSJdLFs3LDZdLFs2LDNdLFs2LDFdLFszLDBdLFsxLDBdLFs3LDVdLFs1LDFdLFs1LDRdLFs0LDJdLFsxLDJdLFs4LDldLFs4LDddLFsxMCw3XSxbMTAsMTFdLFsxMSw2XSxbOSw1XSxbMTIsOF0sWzEyLDEwXV0=&macro_url=https%3A%2F%2Fraw.githubusercontent.com%2FdFoiler%2Fnotes%2Fmaster%2Fnir.tex
	\[\begin{tikzcd}[cramped]
		{\Omega A} & {\Omega C} & 0 \\
		{\Omega B} & X & F & 0 \\
		0 & G & A & B \\
		& 0 & C
		\arrow[from=1-1, to=1-2]
		\arrow[from=1-1, to=2-1]
		\arrow[from=1-2, to=1-3]
		\arrow[from=1-2, to=2-2]
		\arrow[from=1-3, to=2-3]
		\arrow[from=2-1, to=2-2]
		\arrow[from=2-1, to=3-1]
		\arrow[from=2-2, to=2-3]
		\arrow[from=2-2, to=3-2]
		\arrow[from=2-3, to=2-4]
		\arrow[from=2-3, to=3-3]
		\arrow[from=2-4, to=3-4]
		\arrow[from=3-1, to=3-2]
		\arrow[from=3-2, to=3-3]
		\arrow[from=3-2, to=4-2]
		\arrow[from=3-3, to=3-4]
		\arrow[from=3-3, to=4-3]
		\arrow[from=4-2, to=4-3]
	\end{tikzcd}\]
	where all squares are pullback squares (and as usual should be filled in with homotopies). Thus, we have functorially constructed $X$ is the pullback of the maps $\Omega C\from\Omega A\to\Omega B$.

	Now, let $\varphi\colon\mc P\to\underline{\op{Hom}}(\Lambda^2_0,\mc C)$ be the forgetful functor, and let $\psi\colon\underline{\op{Hom}}(\Lambda^2_0,\mc C)$ be the functor described in the previous paragraph. A quick calculation (and the uniqueness of pullbacks) shows that both composites $\varphi\circ\psi$ and $\psi\circ\varphi$ amount to taking $\Omega$ everywhere. Because $\mc C$ is stable, $\Omega$ admits an inverse, so we are done!
\end{proof}
\begin{proof}[Proof of \Cref{prop:stability-is-good}]
	This is a repeated application of \Cref{lem:use-stability}. For example, to exhibit finite colimits, it is enough to exhibit pushouts of two elements (because we already have a $0$ object). Well, given some diagram $C\from A\to B$, we see that maps to some test object $T$ are equivalent to maps of diagrams to $T=T=T$. The equivalence of \Cref{lem:use-stability} then provides us with an object $D$ fitting into a pullback square
	% https://q.uiver.app/#q=WzAsNCxbMCwwLCJBIl0sWzEsMCwiQiJdLFswLDEsIkMiXSxbMSwxLCJEIl0sWzAsMV0sWzEsM10sWzIsM10sWzAsMl1d&macro_url=https%3A%2F%2Fraw.githubusercontent.com%2FdFoiler%2Fnotes%2Fmaster%2Fnir.tex
	\[\begin{tikzcd}[cramped]
		A & B \\
		C & D
		\arrow[from=1-1, to=1-2]
		\arrow[from=1-1, to=2-1]
		\arrow[from=1-2, to=2-2]
		\arrow[from=2-1, to=2-2]
	\end{tikzcd}\]
	so that maps from this square to the square given by $T$ are the same as maps from $C\from A\to B$ to $T$. We conclude that $D$ is the pushout, which (a) follows.
	
	Continuing, the above argument implies that the pushout of the diagram $0\from\Omega D\to0$ is just $D$, so (b) follows. Lastly, to show (c), we note that we have already shown that pushout squares are pullback squares by their construction. Conversely, a pullback square can forget about its lower-right corner, and then one can re-remember the lower-right corner is the pushout via the preceding paragraph. This shows that pullback squares are pushout squares.
\end{proof}
\begin{remark}
	One can unwind the proof of \Cref{lem:use-stability} to show that property (c) is equivalent to the stability of $\mc C$.
\end{remark}

\subsection{Stable Homotopy Groups}
Here is our main definition.
\begin{definition}[stable homotopy]
	Fix a spectrum $X$ and an integer $i\in\ZZ$. Then we define the \textit{$n$th stable homotopy group} $\pi_i^{\mathrm{st}}(X)$ as
	\[\pi_i^{\mathrm{st}}(X)\coloneqq\pi_{i+j}X_j,\]
	provided that $i+j\ge0$. Whenever possible, we will abbreviate $\pi_i^{\mathrm{st}}(X)$ to $\pi_iX$.
\end{definition}
\begin{remark}
	Note that the definition is independent of the choice of $i$ and $j$ because $\Omega X_{j+1}=X_{j}$, so $\pi_{i-1}X_{j+1}=\pi_{i-1}\Omega X_j=\pi_iX_j$.
\end{remark}
\begin{remark}
	A spectrum $X$ arises from a group-like $\mathbb E_\infty$-space if and only if $\pi_nX=0$ for all $n<0$ because this means that $X_i$ is $i$-connected for each $i$ (namely, $\pi_jX_i=0$ for $j<i$), so one can use \Cref{thm:may-recog} to recover $\mathbb E_\infty$-space structure.
\end{remark}
As with pointed spaces, we have a Whitehead's theorem.
\begin{theorem}[Whitehead for spectra] \label{thm:whitehead-spectra}
	Fix a morphism $f\colon X\to Y$ of spectra. Then $f$ is an equivalence if and only if $\pi_if\colon\pi_iX\to\pi_iY$ is an isomorphism for all $i\in\ZZ$.
\end{theorem}
\begin{proof}
	This immediately reduces to Whitehead's theorem for pointed spaces upon unwinding the definition of a morphism of spectra.
\end{proof}
\begin{remark}
	If $X$ and $Y$ are in fact group-like $\mathbb E_\infty$-spectra, then one knows that $\pi_iX=\pi_iY$ for $i<0$, so checking the conclusion of \Cref{thm:whitehead-spectra} only requires $i\ge0$.
\end{remark}
As with homotopy groups, we may hope for a long exact sequence.
\begin{definition}[fiber sequence]
	Fix a stable $\infty$-category $\mc C$. Then a sequence $F\to E\to B$ is a \textit{fiber sequence} if and only if
	% https://q.uiver.app/#q=WzAsNCxbMCwwLCJGIl0sWzEsMCwiRSJdLFsxLDEsIkIiXSxbMCwxLCIwIl0sWzAsMV0sWzEsMl0sWzMsMl0sWzAsM11d&macro_url=https%3A%2F%2Fraw.githubusercontent.com%2FdFoiler%2Fnotes%2Fmaster%2Fnir.tex
	\[\begin{tikzcd}[cramped]
		F & E \\
		0 & B
		\arrow[from=1-1, to=1-2]
		\arrow[from=1-1, to=2-1]
		\arrow[from=1-2, to=2-2]
		\arrow[from=2-1, to=2-2]
	\end{tikzcd}\]
	is a pullback square. We may also call this a \textit{cofiber sequence}.
\end{definition}
\begin{remark}
	We may identify the notions of fiber sequences and cofiber sequences because of \Cref{prop:stability-is-good}.
\end{remark}
\begin{theorem}
	Fix a fiber sequence $F\to E\to B$ of spectra. Then there is a long exact sequence of stable homotopy groups
	\[\cdots\to\pi_{i+1}B\to\pi_iF\to\pi_iE\to\pi_iB\to\pi_{i-1}F\to\cdots.\]
\end{theorem}
\begin{proof}
	The same proof as in spaces shows that a fiber sequence $F\to E\to B$ gives rise to a fiber sequence $\Omega B\to F\to E$. Because $\Omega$ is now an auto-equivalence, we also receive a fiber sequence $E\to B\to\Sigma F$. Thus, we can see that it really amounts to checking exactness at $\pi_0$, which amounts to noting that $\Omega^\infty$ preserves limits (because it is a right adjoint) so that the fiber sequence $F\to E\to B$ of spectra goes to a fiber sequence $F_0\to E_0\to B_0$ of spaces. We now may apply the homotopy long exact sequence for pointed spaces.
\end{proof}
\begin{remark}
	Here is an amusing application: if $F\to E\to B$ is a cofiber sequence of pointed spaces, then $\Sigma^\infty F\to\Sigma^\infty E\to\Sigma^\infty B$ continues to be a cofiber sequence ($\Sigma^\infty$ preserves colimits because it is a left adjoint), so we get a long exact sequence of stable homotopy groups
	\[\cdots\to\pi_{i+1}^{\mathrm{st}}B\to\pi_i^{\mathrm{st}}F\to\pi_i^{\mathrm{st}}E\to\pi_i^{\mathrm{st}}B\to\pi_{i-1}^{\mathrm{st}}F\to\cdots.\]
	This is purely a statement about pointed spaces!
\end{remark}

\end{document}