% !TEX root = ../notes.tex

\documentclass[../notes.tex]{subfiles}

\begin{document}

\section{October 2}
The next problem set will be released shortly.
\begin{remark}
	If $B$ is connected, then changing basepoints of our pointed Kan complexes will not change the homotopy fiber $F$, though changing the choice of equivalence of the homotopy fibers depends on the choice of paths. We may say that this choice is ``not contractible''; notably, even if $B$ is simply connected, then the choice of path may be canonical, but it is still not a contractible choice!
\end{remark}
\begin{example}
	The most basic homotopy fiber sequence is $F\to F\times B\to B$. In this case, one finds that
	\[\mathrm H_n(F\times B;R)=\bigoplus_{p+q=n}\mathrm H_p(B;\mathrm H_q(F;R))\]
	by something like the K\"unneth formula. When $\mathrm H_q(F:R)$ is free over $R$ (for example, if $R$ is a field), then the right-hand summands are $\mathrm H_p(B;R)\otimes_R\mathrm H_q(F;R)$. This should be compared against \Cref{thm:serre-spectral-seq} (and indeed, this can be proved with the Serre spectral sequence), where some higher differentials may cause the homology of a general $E$ to be smaller!
\end{example}

\subsection{Computing Cup Products}
Quickly, we note that there is a Serre spectral sequence in cohomology, whose proof is totally dual.
\begin{theorem}
	Fix a homotopy fiber sequence $F\to E\to B$ of Kan complexes. Then there is a cohomologically graded spectral sequence $E$ with $E_2^{pq}=\mathrm H^p\left(B;\mathrm H^q(F;R)\right)$ and $E_\infty^{pq}=\mathrm H^{p+q}(E;R)$.
\end{theorem}
\begin{remark} \label{rem:cup-product-on-spectral-seq}
	If we have $\mathrm H^p(B;\mathrm H^q(F;R))=\mathrm H^p(B;R)\otimes_R\mathrm H^q(F;R)$, then there is a cup product structure on $E_2$ given by taking
	\[(x\otimes y)\cup(x'\otimes y')=(-1)^{\deg(y)\deg(x')}(x\cup x')(y\cup y').\]
	It further turns out that $d_r(x\cup y)=d_rx\cup y+(-1)^{p+q}x\cup d_ry$ when $x$ is in bidegree $(p,q)$.
\end{remark}
\begin{exe}
	As in \Cref{exe:loops-s3-homology}, one can check that
	\[\mathrm H^i\left(\Omega S^3;\ZZ\right)=\begin{cases}
		\ZZ & \text{if }i\text{ is even}, \\
		0 & \text{if }i\text{ is odd}.
	\end{cases}\]
	We compute the cup products.
\end{exe}
\begin{proof}
	As discussed in \Cref{exe:loops-s3-homology}, our spectral sequence looks like
	% https://q.uiver.app/#q=WzAsMjAsWzAsMywiXFxaWiJdLFsxLDMsIjAiXSxbMiwzLCIwIl0sWzMsMywiXFxaWiJdLFswLDIsIjAiXSxbMSwyLCIwIl0sWzIsMiwiMCJdLFsyLDEsIjAiXSxbMSwxLCIwIl0sWzMsMiwiMCJdLFs0LDMsIjAiXSxbNCwyLCIwIl0sWzQsMSwiMCJdLFs0LDAsIjAiXSxbMywwLCIwIl0sWzIsMCwiMCJdLFsxLDAsIjAiXSxbMCwwLCIwIl0sWzAsMSwiXFxaWiJdLFszLDEsIlxcWloiXSxbMTgsM11d&macro_url=https%3A%2F%2Fraw.githubusercontent.com%2FdFoiler%2Fnotes%2Fmaster%2Fnir.tex
	\[\begin{tikzcd}[cramped]
		0 & 0 & 0 & 0 & 0 \\
		\ZZ & 0 & 0 & \ZZ & 0 \\
		0 & 0 & 0 & 0 & 0 \\
		\ZZ & 0 & 0 & \ZZ & 0
		\arrow[from=2-1, to=4-4]
	\end{tikzcd}\]
	in both the second and third page; we have drawn in a single $d_3$ arrow, which we know must be an isomorphism in order to cause the $E_\infty=E_4$ page to give the correct answer. Note that $\mathrm H^\bullet\left(S^3;\ZZ\right)$ is $\ZZ[e]/\left(e^2\right)$ with $e$ sitting in degree $3$, which we can see directly from its cohomology.

	Now, for each $i$, we define $x_i$ to be a generator of $\mathrm H^{2i}\left(\Omega S^3;\ZZ\right)$. We will require that $d_3x_{i+1}=x_ie$ for each $i$, where the product is the one described in \Cref{rem:cup-product-on-spectral-seq}; notably, multiplication by $e$ does give an isomorphism from bidegree $(0,2i)$ to $(2,2i)$ and shows that $x_ie$ is in fact a generator.

	For example, $x_1^2$ is some multiple of $x_2$, so we compute
	\[d_3\left(x_1^2\right)=d_3(x_1)x_1+x_1d_3(x_1)=2x_1.\]
	(Note that everything in sight is commutative because at least one of the degrees is even at all times.) On the other hand, $d_3(x_2)=x_1$, so we conclude $x_1^2=2x_2$. Similarly,
	\[d_3(x_1x_2)=d_3(x_1)x_2+x_1d_3(x_2)=ex_2+x_1(2x_1)=3x_2e,\]
	so $x_1x_2=3x_2e$. In general, one has that $x_ix_j=\binom{i+j}ix_{i+j}$, which can be checked with a quick induction, so this is a divided power algebra.
\end{proof}
\begin{exe}
	On the homework, we will show that
	\[\mathrm H^\bullet\left(K(\ZZ,2);\ZZ\right)=\ZZ\left[x\right],\]
	where $x$ lives in degree $2$.
\end{exe}
\begin{exe} \label{exe:cohomology-k-z-3}
	We compute some parts of the ring $\mathrm H^\bullet(K(\ZZ,3);\ZZ)$.
\end{exe}
\begin{proof}
	There is a homotopy fiber sequence $K(\ZZ,2)\to\Delta^0\to K(\ZZ,3)$, so \Cref{thm:serre-spectral-seq} provides us with a spectral sequence $E$ with
	\[E_2^{pq}=\mathrm H^p\left(K(\ZZ,3);\mathrm H^q(K(\ZZ,2);\ZZ)\right).\]
	Here are some example calculations.
	\begin{itemize}
		\item Note
		\[E_2^{0q}=\mathrm H^0\left(K(\ZZ,3);\mathrm H^q(K(\ZZ,2);\ZZ)\right)=\mathrm H^q\left(\CP^\infty;\ZZ\right),\]
		which we know is  $\ZZ$ in even degrees and $0$ otherwise.
		\item To compute $E_2^{1q}$, we note that it is the dual of $\mathrm H_1$, which is the abelianization of $\pi_1$, which vanishes.
		\item Note $E_2^{pq}$ vanishes for odd $q$ because $\mathrm H^q(K(\ZZ,2);\ZZ)=0$.
	\end{itemize}
	Our $E_2$ page now looks like the following.
	% https://q.uiver.app/#q=WzAsMjAsWzAsNSwiXFxaWiJdLFswLDQsIjAiXSxbMCwzLCJcXFpaIHgiXSxbMCwyLCIwIl0sWzAsMSwiXFxaWiB4XjIiXSxbMCwwLCIwIl0sWzEsNSwiMCJdLFsxLDQsIjAiXSxbMSwzLCIwIl0sWzEsMiwiMCJdLFsxLDEsIjAiXSxbMSwwLCJcXGJ1bGxldCJdLFsyLDQsIjAiXSxbMyw0LCIwIl0sWzQsNCwiMCJdLFs1LDQsIjAiXSxbMiwyLCIwIl0sWzMsMiwiMCJdLFs0LDIsIjAiXSxbNSwyLCIwIl1d&macro_url=https%3A%2F%2Fraw.githubusercontent.com%2FdFoiler%2Fnotes%2Fmaster%2Fnir.tex
	\[\begin{tikzcd}[cramped,sep=tiny]
		0 & 0 \\
		{\ZZ x^2} & 0 \\
		0 & 0 & 0 & 0 & 0 & 0 \\
		{\ZZ x} & 0 \\
		0 & 0 & 0 & 0 & 0 & 0 \\
		\ZZ & 0
	\end{tikzcd}\]
	For example, we now see that $E_2^{2,0}$ vanishes because no differentials can interact with this term, and we need to get the cohomology of $\Delta^0$. This corresponds to computing $\mathrm H^2(K(\ZZ,3);\ZZ)$, so the entire column also vanishes. Similarly, the fourth column vanishes.

	Continuing, the differential $d_3\colon E^{02}\to E^{30}$ shows that $\mathrm H^3(K(\ZZ,3);\ZZ)$ is $\ZZ$. This produces the following.
	% https://q.uiver.app/#q=WzAsMjksWzAsNSwiXFxaWiJdLFswLDQsIjAiXSxbMCwzLCJcXFpaIHgiXSxbMCwyLCIwIl0sWzAsMSwiXFxaWiB4XjIiXSxbMCwwLCIwIl0sWzEsNSwiMCJdLFsxLDQsIjAiXSxbMSwzLCIwIl0sWzEsMiwiMCJdLFsxLDEsIjAiXSxbMSwwLCIwIl0sWzIsNCwiMCJdLFszLDQsIjAiXSxbNCw0LCIwIl0sWzUsNCwiMCJdLFsyLDIsIjAiXSxbMywyLCIwIl0sWzQsMiwiMCJdLFs1LDIsIjAiXSxbMiw1LCIwIl0sWzIsMywiMCJdLFsyLDEsIjAiXSxbMiwwLCIwIl0sWzMsNSwiXFxaWiBlIl0sWzMsMywiXFxaWiB4ZSJdLFszLDEsIlxcWlogeF4yZSJdLFszLDAsIjAiXSxbNCw1LCI/Il1d&macro_url=https%3A%2F%2Fraw.githubusercontent.com%2FdFoiler%2Fnotes%2Fmaster%2Fnir.tex
	\[\begin{tikzcd}[cramped,sep=tiny]
		0 & 0 & 0 & 0 \\
		{\ZZ x^2} & 0 & 0 & {\ZZ x^2e} \\
		0 & 0 & 0 & 0 & 0 & 0 \\
		{\ZZ x} & 0 & 0 & {\ZZ xe} \\
		0 & 0 & 0 & 0 & 0 & 0 \\
		\ZZ & 0 & 0 & {\ZZ e} & {?}
	\end{tikzcd}\]
	As a harder calculation, we note that $d_3\colon E^{04}\to E^{32}$ is the map $\ZZ x^2\to\ZZ xe$, and we can calculate that $x^2$ maps to $2xe$. Accordingly, one can calculate that $E_\infty^{50}=0$, so the fifth column also vanishes.
	% https://q.uiver.app/#q=WzAsNDAsWzAsNSwiXFxaWiJdLFswLDQsIjAiXSxbMCwzLCJcXFpaIHgiXSxbMCwyLCIwIl0sWzAsMSwiXFxaWiB4XjIiXSxbMCwwLCIwIl0sWzEsNSwiMCJdLFsxLDQsIjAiXSxbMSwzLCIwIl0sWzEsMiwiMCJdLFsxLDEsIjAiXSxbMSwwLCIwIl0sWzIsNCwiMCJdLFszLDQsIjAiXSxbNCw0LCIwIl0sWzUsNCwiMCJdLFsyLDIsIjAiXSxbMywyLCIwIl0sWzQsMiwiMCJdLFs1LDIsIjAiXSxbMiw1LCIwIl0sWzIsMywiMCJdLFsyLDEsIjAiXSxbMiwwLCIwIl0sWzMsNSwiXFxaWiBlIl0sWzMsMywiXFxaWiB4ZSJdLFszLDEsIlxcWlogeF4yZSJdLFszLDAsIjAiXSxbNCw1LCIwIl0sWzQsMywiMCJdLFs0LDEsIjAiXSxbNCwwLCIwIl0sWzUsNSwiMCJdLFs1LDMsIjAiXSxbNSwxLCIwIl0sWzUsMCwiMCJdLFs2LDQsIjAiXSxbNiwyLCIwIl0sWzYsMCwiMCJdLFs2LDUsIj8iXV0=&macro_url=https%3A%2F%2Fraw.githubusercontent.com%2FdFoiler%2Fnotes%2Fmaster%2Fnir.tex
	\[\begin{tikzcd}[cramped,sep=tiny]
		0 & 0 & 0 & 0 & 0 & 0 & 0 \\
		{\ZZ x^2} & 0 & 0 & {\ZZ x^2e} & 0 & 0 \\
		0 & 0 & 0 & 0 & 0 & 0 & 0 \\
		{\ZZ x} & 0 & 0 & {\ZZ xe} & 0 & 0 \\
		0 & 0 & 0 & 0 & 0 & 0 & 0 \\
		\ZZ & 0 & 0 & {\ZZ e} & 0 & 0 & {?}
	\end{tikzcd}\]
	Because $d_3\colon\ZZ x^2\to\ZZ xe$ is multiplication by $2$, we see that $E_\infty^{60}=\ZZ/2\ZZ$, so we get the following.
	% https://q.uiver.app/#q=WzAsNDIsWzAsNSwiXFxaWiJdLFswLDQsIjAiXSxbMCwzLCJcXFpaIHgiXSxbMCwyLCIwIl0sWzAsMSwiXFxaWiB4XjIiXSxbMCwwLCIwIl0sWzEsNSwiMCJdLFsxLDQsIjAiXSxbMSwzLCIwIl0sWzEsMiwiMCJdLFsxLDEsIjAiXSxbMSwwLCIwIl0sWzIsNCwiMCJdLFszLDQsIjAiXSxbNCw0LCIwIl0sWzUsNCwiMCJdLFsyLDIsIjAiXSxbMywyLCIwIl0sWzQsMiwiMCJdLFs1LDIsIjAiXSxbMiw1LCIwIl0sWzIsMywiMCJdLFsyLDEsIjAiXSxbMiwwLCIwIl0sWzMsNSwiXFxaWiBlIl0sWzMsMywiXFxaWiB4ZSJdLFszLDEsIlxcWlogeF4yZSJdLFszLDAsIjAiXSxbNCw1LCIwIl0sWzQsMywiMCJdLFs0LDEsIjAiXSxbNCwwLCIwIl0sWzUsNSwiMCJdLFs1LDMsIjAiXSxbNSwxLCIwIl0sWzUsMCwiMCJdLFs2LDQsIjAiXSxbNiwyLCIwIl0sWzYsMCwiMCJdLFs2LDUsIlxcWlovMlxcWlogZiJdLFs2LDMsIlxcWloyL1xcWlogeGYiXSxbNiwxLCJcXFpaMi9cXFpaIHheMmYiXV0=&macro_url=https%3A%2F%2Fraw.githubusercontent.com%2FdFoiler%2Fnotes%2Fmaster%2Fnir.tex
	\[\begin{tikzcd}[cramped,sep=tiny]
		0 & 0 & 0 & 0 & 0 & 0 & 0 \\
		{\ZZ x^2} & 0 & 0 & {\ZZ x^2e} & 0 & 0 & {\ZZ2/\ZZ x^2f} \\
		0 & 0 & 0 & 0 & 0 & 0 & 0 \\
		{\ZZ x} & 0 & 0 & {\ZZ xe} & 0 & 0 & {\ZZ2/\ZZ xf} \\
		0 & 0 & 0 & 0 & 0 & 0 & 0 \\
		\ZZ & 0 & 0 & {\ZZ e} & 0 & 0 & {\ZZ/2\ZZ f}
	\end{tikzcd}\]
	One can similarly see that the seventh column vanishes. But we can now calculate $E_2^{80}$ is $\ZZ/3\ZZ$, so things seem to be getting complicated.
\end{proof}
\begin{remark}
	The presence of $\mathrm H^6(K(\ZZ,3);\ZZ)\cong\ZZ/2\ZZ$ means that there is a map $K(\ZZ,3)\to K(\ZZ,6)$ given by the universal property. Applying the universal property again tells us that there is a natural map
	\[\mathrm H^3(-;\ZZ)\to\mathrm H^6(-;\ZZ),\]
	which turns out to be squaring (with the cup product). Because $\alpha\cup\alpha$ is always $2$-torsion for $\alpha\in\mathrm H^3(X;\ZZ)$ (because it is equal to its negative by the graded commutative), we find that this class is $2$-torsion! Similarly, we see that there is no natural transformation $\mathrm H^3(-;\ZZ)\to\mathrm H^7(-;\ZZ)$.
\end{remark}
\begin{remark}
	More generally, one sees that the cup product amounts to a map
	\[K(\ZZ,p)\times K(\ZZ,q)\to K(\ZZ,p+q).\]
\end{remark}
\begin{remark}
	The induced natural transformation $\mathrm H^3(-;\ZZ)\to\mathrm H^8(-;\ZZ)$ is a Steenrod operation, which we will discuss later. The moral of our story is that the cohomology of $K(\ZZ,n)$s tell us all natural transformations on cohomology.
\end{remark}

\end{document}