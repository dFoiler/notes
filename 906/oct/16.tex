% !TEX root = ../notes.tex

\documentclass[../notes.tex]{subfiles}

\begin{document}

\section{October 16}
Today is the last day which can appear on the exam. We will discuss the cohomology of the Eilenberg--MacLane spaces.

\subsection{The Steenrod Square}
We are going to compute $\mathrm H^*(K(\FF_2,n);\FF_2)$. There is an analogue for everything over $\FF_p$, but we will not discuss it.
\begin{remark} \label{rem:cohom-em-by-nat-trans}
	Recall that $\mathrm H^i(K(\FF_2,n);\FF_2)$ consists of homotopy classes of maps
	\[K(\FF_2,n)\to K(\FF_2,i)\]
	by the universal property, which in turn are natural transformations $\mathrm H^n(-;\FF_2)\Rightarrow\mathrm H^i(-;\FF_2)$.
\end{remark}
By \Cref{rem:cohom-em-by-nat-trans}, we see that we are interested in defining some natural transformations.
\begin{notation}
	For each $n\ge0$, we define $i_n\in\mathrm H^n(K(\FF_2,n);\FF_2)$ to correspond by the universal property to the identity map $K(\FF_2,n)\to K(\FF_2,n)$.
\end{notation}
\begin{remark}
	Roughly speaking, to define a natural transformation, it basically amounts to define what it does for the ``universal'' elements $i_n$.
\end{remark}
\begin{proposition}[Steenrod square]
	For each $i\ge0$, there is a natural transformation $\mathrm{Sq}^i\colon\mathrm H^*(X;\FF_2)\to\mathrm H^{*+i}(X;\FF_2)$ satisfying the following. For example, we ought to have the following.
	\begin{listalph}
		\item Zero: $\mathrm{Sq}^0=\id$.
		\item Top degree: $\mathrm{Sq}^i(x)=x\cup x$ if $i=\deg x$. In particular, $\op{Sq}^ni_n=i_n^2$.
		\item Higher degree: $\mathrm{Sq}^i(x)=0$ if $i>\deg x$.
		\item The natural transformation $\op{Sq}^i$ is a homomorphism.
		\item We have
		\[\op{Sq}^k(ab)=\sum_{i+j=k}\op{Sq}^i(a)\op{Sq}^j(b).\]
		\item The map $\op{Sq}^i$ commutes with the suspension isomorphism $\widetilde{\mathrm H}^k(X;\FF_2)\to\widetilde{\mathrm H}^{k+1}(\Sigma X;\FF_2)$.
		% \item We have $\op{Sq}^ni_n\coloneqq i_n^2$.
	\end{listalph}
\end{proposition}
\begin{proof}
	It remains to describe $\mathrm{Sq}^ni_{n+k}$ for $k\ge1$, for which we use the Serre spectral sequence $E$. Let's start with $k=1$, where we work with the homotopy fiber sequence $K(\FF_2,n)\to\Delta^0\to K(\FF_2,n+1)$, which has
	\[E_2^{pq}=\mathrm H^p(K(\FF_2,n+1);\FF_2)\otimes\mathrm H^q(K(\FF_2,n);\FF_2)\Rightarrow\mathrm H^{p+q}\left(\Delta^0;\FF_2\right).\]
	Many of these columns vanish: by \Cref{thm:hurewisz}, we know that
	\[\mathrm H_i(K(\FF_2,n+1);\ZZ)=\begin{cases}
		\ZZ & \text{if }i=0, \\
		0 & \text{if }0<i<n+1, \\
		\FF_2 & \text{if }i=n+1.
	\end{cases}\]
	By the Universal coefficients theorem, we thus find that $E_2^{p0}$ vanishes in the region $0<i<n+1$, and $E_2^{n+1,0}=\FF_2$. The exact same calculation explains that $E_2^{0q}=0$ for $0<q<n$ and $E_2^{0n}=\FF_2$. For example, we receive an isomorphism
	\[d_n\colon E_2^{0n}\to E_2^{n+1,0}.\]
	Here, $E_2^{0n}$ is generated by $i_n$, so we are being told that $d_n(i_n)=i_{n+1}$. Similarly, we can calculate
	\[d_n\left(i_n^2\right)=i_n(d_ni_n)+(di_n)i_n=2i_n(di_n)=0.\]
	Thus, we see that the only other differential which can kill $d_n$ must be
	\[d_{2n}\colon E^{0,2n}\to E^{2n+1,0},\]
	and we may define $\op{Sq}^ni_{n+1}\coloneqq d_{2n}(i_n^2)$. The above construction can be iterated to define $\op{Sq}^ni_{n+k}$ for each $k\ge2$ by differentials in the Serre spectral sequence.
	
	We will not check the last compatibilities. For (e), we remark that this corresponds to $(a+b)^2=a^2+b^2$ in degree $i$, so it is important that we are working over $\FF_2$.
\end{proof}
\begin{example} \label{ex:cohom-k-1}
	It turns out that $\RP^\infty$ is a $K(\FF_2,1)$, so $\mathrm H^*(K(\FF_2,1);\FF_2)$ is $\FF_2[x]$ where $x$ lives in degree $1$. We compute some Steenrod squares.
\end{example}
\begin{proof}
	For example, we see that
	\[\op{Sq}^ix=\begin{cases}
		x & \text{if }i=0, \\
		x^2 & \text{if }i=1, \\
		0 & \text{if }i>1.
	\end{cases}\]
	Similarly,
	\[\op{Sq}^ix^2=\begin{cases}
		x^2 & \text{if }i=0, \\
		0 & \text{if }i=1\text{ or }i>2, \\
		x^4 & \text{if }i=2.
	\end{cases}\]
	Here, the interesting calculation is $\op{Sq}^1x^2=\op{Sq}^0x\cdot\op{Sq}^1x+\op{Sq}^1x\cdot\op{Sq}^0x=0$ (in $\FF_2$). We can also calculate $\op{Sq}^1x^3$ is
	\[\op{Sq}^1x^2\cdot\op{Sq}^0x+\op{Sq}^0x^2\cdot\op{Sq}^1x=x^4.\]
	In general, one can show that $\op{Sq}^ix^k=\binom kix^{k+i}$ after some combinatorics.
\end{proof}

\subsection{The Cohomology of \texorpdfstring{$K(\FF_2,n)$}{ K(F2,n)}}
Let's try to provide a more concrete description of this long differential. Fix a homotopy fiber sequence $F\to E\to B$ with $B$ simply connected, so we get
\[E_2^{pq}=\mathrm H^p(B;\FF_2)\otimes\mathrm H^q(F;\FF_2)\Rightarrow\mathrm H^{p+q}(E;\FF_2).\]
\Cref{rem:filtration-pieces-serre-spectral} describes $E_\infty^{p0}$ as the image of $\mathrm H^p(B;\FF_2)\to\mathrm H^p(E;\FF_2)$ and describes $E_\infty^{0q}$ as the image of the map $\mathrm H^q(E;\FF_2)\to\mathrm H^q(F;\FF_2)$.
\begin{definition}[transgressive]
	Fix a homotopy fiber sequence $F\to E\to B$ with $F$ connected and $B$ simply connected. Then a pair $(x,y)\in\mathrm H^{i+1}(B;\FF_2)\times\mathrm H^i(F;\FF_2)$ is \textit{transgressive} if and only if the image of $x\in\widetilde{\mathrm H}^{i+1}(B)$ and $y\in\mathrm H^i(F)$ having the same image in $\mathrm H^{i+1}(E,F)$, where $\mathrm H^{i+1}(E,F)$ is the homotopy pushout of the following.
	% https://q.uiver.app/#q=WzAsMyxbMCwwLCJGIl0sWzEsMCwiRSJdLFswLDEsIlxcRGVsdGFeMCJdLFswLDJdLFswLDFdXQ==&macro_url=https%3A%2F%2Fraw.githubusercontent.com%2FdFoiler%2Fnotes%2Fmaster%2Fnir.tex
	\[\begin{tikzcd}[cramped]
		F & E \\
		{\Delta^0}
		\arrow[from=1-1, to=1-2]
		\arrow[from=1-1, to=2-1]
	\end{tikzcd}\]
\end{definition}
\begin{remark}
	It turns out that $(x,y)$ being transgressive is equivalent to $y$ surviving to $E_\infty$ and
	\[d_{i+1}(y)=x.\]
\end{remark}
\begin{corollary}[Kudo transgression] \label{cor:kudo-transgression}
	Fix a homotopy fiber sequence $F\to E\to B$ with $F$ connected and $B$ simply connected; let $E$ be the associated cohomological Serre spectral sequence. Then if $y\in\mathrm H^{i+1}(F;\FF_2)$ survives to $E_{i+1}$ and $d_{i+1}(y)=x$, then $\op{Sq}^n(y)$ survives to the $E_{i+n+1}$ page and $d_{i+n+1}(y)=\op{Sq}^n(x)$.
\end{corollary}
\begin{proof}
	Track through the definition of the Steenrod squares, which we know already come from some transgressive pairs.
\end{proof}
One can use \Cref{cor:kudo-transgression} to fully compute the cohomology of $K(\FF_2,2)$.
\begin{proposition}[Serre]
	The cohomology ring $\mathrm H^*(K(\FF_2,2);\FF_2)$ is the free polynomial algebra
	\[\FF_2\left[i_2,\op{Sq}^1i_2,\op{Sq}^2\op{Sq}^1i_2,\op{Sq}^4\op{Sq}^2\op{Sq}^1i_2,\ldots\right].\]
\end{proposition}
\begin{example} \label{ex:cup-product-k-f2}
	There is a cup product map $K(\FF_2,1)\times K(\FF_2,1)\to K(\FF_2,2)$. Namely, on cohomology, we find that $\mathrm H^*(K(\FF_2,1)\times K(\FF_2,1);\FF_2)=\FF_2[x,y]$ by the K\"unneth isomorphism (and \Cref{ex:cohom-k-1}). Thus, there is a ring homomorphism
	\[\FF_2\left[i_2,\op{Sq}^1i_2,\op{Sq}^2\op{Sq}^1i_2,\op{Sq}^4\op{Sq}^2\op{Sq}^1i_2,\ldots\right]\to\FF_2[x,y].\]
	For example, the ambient ring structure tells us that $i_2\mapsto xy$. The rest can be calculated using the Steenrod squares: for example, $\op{Sq}^1(i_2)$ goes to $\op{Sq}^1(xy)=x^2y+xy^2$. It turns out that this ring homomorphism is injective in degree less than $4$. For example, this injectivity could be used to calculate $\op{Sq}^1\op{Sq}^1i_2$: indeed, $\op{Sq}^1\op{Sq}^1xy=0$ implies $\op{Sq}^1\op{Sq}^1i_2=0$. This identity $\mathrm{Sq}^1\mathrm{Sq}^1=0$ must now hold in general!
\end{example}
One can extend these notions as follows.
\begin{theorem}[Serre]
	The ring $\mathrm H^*(K(\FF_2,n);\FF_2)$ is a polynomial ring with generators of the form
	\[\op{Sq}^{j_1}\cdots\op{Sq}^{j_r}i_n\]
	for which $j_k\ge2j_{k+1}$ and $\sum_k(j_k-2j_{k+1})<n$.
\end{theorem}
\begin{remark}
	This is a great theorem! It tells us that any natural transformation on cohomology is induced by some polynomial in the Steenrod squares.
\end{remark}
\begin{theorem}[Adem relations] \label{thm:squares-on-k-f2}
	For any space $X$, for indices $i<2j$, we have
	\[\mathrm{Sq}^i\mathrm{Sq}^j=\sum_{k=0}^{i/2}\binom{j-k-1}{i-2k}\mathrm{Sq}^{i+j-k}\mathrm{Sq}^k.\]
\end{theorem}
\begin{remark}
	\Cref{thm:squares-on-k-f2} is proved using the technique of \Cref{ex:cup-product-k-f2}.
\end{remark}
\begin{example}
	One can calculate that $\mathrm{Sq}^1\mathrm{Sq}^2=\mathrm{Sq}^3$ and $\mathrm{Sq}^1\mathrm{Sq}^4=\mathrm{Sq}^5$ and $\mathrm{Sq}^2\mathrm{Sq}^4+\mathrm{Sq}^5\mathrm{Sq}^1=\mathrm{Sq}^6$. In general, one finds that one can generate all Steenrod squares using just the Steenrod squares of powers of $2$.
\end{example}
\begin{remark}
	The Steenrod algebra $\mc A$ is the algebra generated by formal symbols $\mathrm{Sq}^i$ with the Adem relations. This is the Steenrod algebra $\mc A$ appearing in the Adams spectral sequence of \Cref{thm:adams-ss}. It turns out that this algebra also appears in the theory of formal group laws.
\end{remark}
% Now, say a pair $(x,y)$ with $x\otimes y\in E_2^{i+1,i}$ is transgressive if and only if the image of $x\in\widetilde{\mathrm H}^{i+1}(B)$ and $y\in\mathrm H^i(F)$ having the same image in $\mathrm H^{i+1}(E,F)$ (where the latter is the cohomology of the homotopy pushout of $\Delta^0\from F\to E$). It turns out that $(x,y)$ being transgressive is equivalent to $y$ surviving to $E_\infty$ and $d_{i+1}(y)=x$.

\end{document}