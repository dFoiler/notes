% !TEX root = ../notes.tex

\documentclass[../notes.tex]{subfiles}

\begin{document}

\section{September 11}
We continue discussing our quasicategories.

\subsection{More on Kan Complexes}
Let's say a little more about Kan complexes.
\begin{remark}
	Fix a Kan complex $C$ and a morphism $f\colon c\to c'$ in $C_1$. Then one can construct an inverse for $f$ as follows: the outer horn
	% https://q.uiver.app/#q=WzAsMyxbMSwwLCJjJyJdLFswLDEsImMiXSxbMiwxLCJjIl0sWzEsMiwiIiwwLHsibGV2ZWwiOjIsInN0eWxlIjp7ImhlYWQiOnsibmFtZSI6Im5vbmUifX19XSxbMSwwLCJmIl1d&macro_url=https%3A%2F%2Fraw.githubusercontent.com%2FdFoiler%2Fnotes%2Fmaster%2Fnir.tex
	\[\begin{tikzcd}[cramped]
		& {c'} \\
		c && c
		\arrow["f", from=2-1, to=1-2]
		\arrow[equals, from=2-1, to=2-3]
	\end{tikzcd}\]
	to $C$ must fill out to a map $\Delta^2\to C$, which provides us with the data of some map $g\colon B\to A$. Then the composite $g\circ f$ can be seen to be homotopic to the identity via the map $\Delta^2\to C$.
\end{remark}
Here is a more general notion.
\begin{definition}[Kan fibration]
	A morphism $X\to Y$ of simplicial sets is a \textit{Kan fibration} if and only if, for any $n$ and $i\in[n]$, any two maps $\Lambda^n_i\to X$ and $\Delta^n\to Y$ admits a lifting map $\Delta^n\to X$ making the following diagram commute.
	% https://q.uiver.app/#q=WzAsNCxbMSwwLCJYIl0sWzEsMSwiWSJdLFswLDAsIlxcTGFtYmRhXm5faSJdLFswLDEsIlxcRGVsdGFebiJdLFsyLDNdLFszLDFdLFsyLDBdLFswLDFdLFszLDAsIiIsMSx7InN0eWxlIjp7ImJvZHkiOnsibmFtZSI6ImRhc2hlZCJ9fX1dXQ==&macro_url=https%3A%2F%2Fraw.githubusercontent.com%2FdFoiler%2Fnotes%2Fmaster%2Fnir.tex
	\[\begin{tikzcd}[cramped]
		{\Lambda^n_i} & X \\
		{\Delta^n} & Y
		\arrow[from=1-1, to=1-2]
		\arrow[from=1-1, to=2-1]
		\arrow[from=1-2, to=2-2]
		\arrow[dashed, from=2-1, to=1-2]
		\arrow[from=2-1, to=2-2]
	\end{tikzcd}\]
\end{definition}
\begin{example}
	A simplicial set $X$ has a canonical map to the terminal object $\Delta^0$, so the map $X\to\Delta^0$ is a Kan fibration if and only if $X$ is a Kan complex.
\end{example}
For today, we will be interested in what sorts of maps $A\to B$ admit lifts against Kan fibrations. Of course, this includes the horn inclusions $\Lambda^n_i\to\Delta^n$, but there are more ways to generate such morphisms.
\begin{definition}[saturated]
	A nonempty class $\Sigma$ of morphisms of simplicial sets is \textit{saturated} if and only if it is closed under pushouts, retracts, coproducts, composition, and transfinite composition.
\end{definition}
We will explain the terms in this definition shortly, but let's start by explaining why we should care.
\begin{definition}[anodyne]
	The smallest saturated class of maps containing all horn inclusions is the class of \textit{anodyne} maps. The smallest saturated class of maps containing all inner horn inclusions is the class of \textit{inner anodyne} maps.
\end{definition}
\begin{proposition}
	Suppose that a map $f\colon A\to B$ is anodyne. Then any Kan fibration $g\colon X\to Y$ fitting into the solid square
	% https://q.uiver.app/#q=WzAsNCxbMCwwLCJBIl0sWzAsMSwiQiJdLFsxLDAsIlgiXSxbMSwxLCJZIl0sWzAsMSwiZiIsMl0sWzEsM10sWzAsMl0sWzIsM10sWzEsMiwiIiwyLHsic3R5bGUiOnsiYm9keSI6eyJuYW1lIjoiZGFzaGVkIn19fV1d&macro_url=https%3A%2F%2Fraw.githubusercontent.com%2FdFoiler%2Fnotes%2Fmaster%2Fnir.tex
	\[\begin{tikzcd}[cramped]
		A & X \\
		B & Y
		\arrow[from=1-1, to=1-2]
		\arrow["f"', from=1-1, to=2-1]
		\arrow[from=1-2, to=2-2]
		\arrow[dashed, from=2-1, to=1-2]
		\arrow[from=2-1, to=2-2]
	\end{tikzcd}\]
	will admit a dashed map $B\to X$ making the diagram commute.
\end{proposition}
\begin{remark}
	It turns out that a morphism $A\to B$ is anodyne if and only if it is monic and the induced map $\left|A\right|\to\left|B\right|$ is a homotopy equivalence. We will not prove this, so we will not use it.
\end{remark}
Let's now explain what it means for a class $\mc P$ of morphisms to be saturated.
\begin{definition}
	Let $\mc P$ be a class of morphisms in a category. If $f\colon A\to B$ is in $\mc P$, and we are given any map $A\to A'$ living in a pushout diagram as follows.
	% https://q.uiver.app/#q=WzAsNCxbMCwwLCJBIl0sWzEsMCwiQSciXSxbMCwxLCJCIl0sWzEsMSwiQiciXSxbMCwyLCJmIiwyXSxbMiwzXSxbMCwxXSxbMSwzLCJmJyJdXQ==&macro_url=https%3A%2F%2Fraw.githubusercontent.com%2FdFoiler%2Fnotes%2Fmaster%2Fnir.tex
	\[\begin{tikzcd}[cramped]
		A & {A'} \\
		B & {B'}
		\arrow[from=1-1, to=1-2]
		\arrow["f"', from=1-1, to=2-1]
		\arrow["{f'}", from=1-2, to=2-2]
		\arrow[from=2-1, to=2-2]
	\end{tikzcd}\]
	Then $\mc P$ is \textit{closed under pushouts} if we always have $f'\in\mc P$ for all such diagrams.
\end{definition}
\begin{example}
	We claim that the class of anodyne maps is closed under pushouts. Indeed, given any map $A\to A'$ producing a pushout map $f'\colon A'\to B'$, we would like to know if we can always fill in for the dashed arrow.
	% https://q.uiver.app/#q=WzAsNixbMCwwLCJBIl0sWzEsMCwiQSciXSxbMCwxLCJCIl0sWzEsMSwiQiciXSxbMiwwLCJYIl0sWzIsMSwiWSJdLFswLDIsImYiLDJdLFsyLDNdLFswLDFdLFsxLDMsImYnIl0sWzEsNF0sWzMsNV0sWzQsNV0sWzMsNCwiIiwyLHsic3R5bGUiOnsiYm9keSI6eyJuYW1lIjoiZGFzaGVkIn19fV1d&macro_url=https%3A%2F%2Fraw.githubusercontent.com%2FdFoiler%2Fnotes%2Fmaster%2Fnir.tex
	\[\begin{tikzcd}[cramped]
		A & {A'} & X \\
		B & {B'} & Y
		\arrow[from=1-1, to=1-2]
		\arrow["f"', from=1-1, to=2-1]
		\arrow[from=1-2, to=1-3]
		\arrow["{f'}", from=1-2, to=2-2]
		\arrow[from=1-3, to=2-3]
		\arrow[from=2-1, to=2-2]
		\arrow[dashed, from=2-2, to=1-3]
		\arrow[from=2-2, to=2-3]
	\end{tikzcd}\]
	Well, we are granted a map $B\to X$ making the diagram commute, so there is a map $B'\to X$ making the diagram commute because $B'$ is a pushout.
\end{example}
\begin{definition}[retract]
	A map $f'\colon A'\to B'$ is a \textit{retract} of a map $f\colon A\to B$ if and only if it fits into a diagram as follows.
	% https://q.uiver.app/#q=WzAsNixbMCwwLCJBJyJdLFswLDEsIkInIl0sWzEsMCwiQSJdLFsxLDEsIkIiXSxbMiwwLCJBJyJdLFsyLDEsIkInIl0sWzAsMl0sWzIsNF0sWzAsMSwiZiciLDJdLFsxLDNdLFszLDVdLFsyLDMsImYiXSxbNCw1LCJmJyJdLFswLDQsIiIsMSx7ImN1cnZlIjotMiwibGV2ZWwiOjIsInN0eWxlIjp7ImhlYWQiOnsibmFtZSI6Im5vbmUifX19XSxbMSw1LCIiLDEseyJjdXJ2ZSI6MiwibGV2ZWwiOjIsInN0eWxlIjp7ImhlYWQiOnsibmFtZSI6Im5vbmUifX19XV0=&macro_url=https%3A%2F%2Fraw.githubusercontent.com%2FdFoiler%2Fnotes%2Fmaster%2Fnir.tex
	\[\begin{tikzcd}[cramped]
		{A'} & A & {A'} \\
		{B'} & B & {B'}
		\arrow[from=1-1, to=1-2]
		\arrow[curve={height=-12pt}, equals, from=1-1, to=1-3]
		\arrow["{f'}"', from=1-1, to=2-1]
		\arrow[from=1-2, to=1-3]
		\arrow["f", from=1-2, to=2-2]
		\arrow["{f'}", from=1-3, to=2-3]
		\arrow[from=2-1, to=2-2]
		\arrow[curve={height=12pt}, equals, from=2-1, to=2-3]
		\arrow[from=2-2, to=2-3]
	\end{tikzcd}\]
\end{definition}
\begin{definition}[coproduct]
	Given a collection of maps $f_i\colon A_i\to B_i$ (as $i$ varies over an index set $I$), then the coproduct map is defined as the induced map
	\[\bigsqcup_{i\in I}f_i\colon\bigsqcup_{i\in I}A_i\to\bigsqcup_{i\in I}B_i.\]
\end{definition}
\begin{definition}[transfinite composition]
	Suppose that there is a diagram
	\[A_0\to A_1\to A_2\to\cdots\]
	of maps. Then the transfinite composition is the colimit of the diagram
	\[A_0\to\colim_iA_i.\]
\end{definition}
One can check that the class of anodyne maps has all the above closure properties.

\subsection{Pushout Products}
Here is are a few more useful facts about anodyne maps.
\begin{remark}
	All anodyne maps are monomorphisms.
\end{remark}
Next, we will want to prove that anodyne maps are closed under products. For this, we pick up the following definition.
\begin{definition}[pushout product]
	Given two maps $f\colon A\to B$ and $g\colon C\to D$ in a category $\mc C$, the \textit{pushout product} is the induced map
	\[(B\times C)\bigsqcup_{A\times C}(A\times D)\to (B\times D).\]
	We may write this map as $f\boxtimes g$.
\end{definition}
\begin{example}
	The pushout product of a map $f\colon A\to B$ and the initial map $\emp\to C$ is the induced map
	\[A\times C\to B\times C.\]
\end{example}
\begin{remark}
	Pushout products are associative in a suitable way, which we will not prove in general.
\end{remark}
\begin{theorem} \label{thm:anodyne-pushout-product}
	Fix a monomorphism $f\colon A\to B$ of simplicial sets and an anodyne (respectively, inner anodyne) map $g\colon C\to D$. Then the pushout product $f\boxtimes g$ is anodyne (respectively, inner anodyne).
\end{theorem}
The rest of this class will be spent proving this result. We will focus on the anodyne case; the proof of the inner anodyne case is basically the same.
\begin{remark}[Jeremy Hahn]
	If you try to visualize this with geometric realizations, then you will recover some exercise that is in Hatcher somewhere.
\end{remark}
Let's give the main claim for \Cref{thm:anodyne-pushout-product}.
\begin{lemma} \label{lem:weird-anodyne}
	The class of anodyne maps is the smallest saturated class containing all maps of the form $f\boxtimes i$, where $f\colon A\to B$ is monic, and $i\colon\Delta^0\to\Delta^1$ is some map.
\end{lemma}
\begin{proof}[Proof of \Cref{thm:anodyne-pushout-product} from \Cref{lem:weird-anodyne}]
	Fix a monomorphism $i$, and we consider the class $\mc P$ of maps $j$ such that $i\boxtimes j$ is anodyne. We want to check that $\mc P$ contains all anodyne maps. One can check that this class is saturated, so by \Cref{lem:weird-anodyne}, it remains to prove that $\mc P$ contains the maps $i\boxtimes(j\boxtimes k)$ where $j\colon A\to B$ is monic, and $k\colon\Delta^0\to\Delta^1$ is some map. Well,
	\[i\boxtimes(j\boxtimes k)=(i\boxtimes j)\boxtimes k,\]
	and $i\boxtimes j$ continues to be monic, so this is anodyne by \Cref{lem:weird-anodyne}!
\end{proof}
It remains to prove \Cref{lem:weird-anodyne}. Let $\mc P$ be the saturated class of maps in the statement, and we would like to show that it coincides with the saturated class of anodyne maps. This will be done via two inclusions, so it suffices to show that each class contains the other's generators.
\begin{lemma}
	Let $\mc P$ be the smallest saturated class containing all maps of the form $f\boxtimes i$, where $f\colon A\to B$ is monic, and $i\colon\Delta^0\to\Delta^1$ is some map. Then every anodyne map is in $\mc P$.
\end{lemma}
\begin{proof}
	It is enough to show that the horn inclusions $\Lambda^n_i\to\Delta^n$ are in $\mc P$, which amounts to some explicit combinatorics. By symmetry, we may assume that $i<n$. We now claim that the diagram
	% https://q.uiver.app/#q=WzAsNixbMCwwLCJcXExhbWJkYV5uX2kiXSxbMCwxLCJcXERlbHRhXm4iXSxbMSwwLCJcXGRpc3BsYXlzdHlsZShcXExhbWJkYV5uX2lcXHRpbWVzXFxEZWx0YV4xKVxcYmlnc3FjdXBfe1xcTGFtYmRhXm5faVxcdGltZXNcXERlbHRhXjB9KFxcRGVsdGFeblxcdGltZXNcXERlbHRhXjApIl0sWzEsMSwiXFxEZWx0YV5uXFx0aW1lc1xcRGVsdGFeMSJdLFsyLDAsIlxcTGFtYmRhX2lebiJdLFsyLDEsIlxcRGVsdGFebiJdLFswLDEsImkiLDJdLFsyLDMsImlcXGJveHRpbWVzXFx2YXJlcHNpbG9uXzAiLDJdLFswLDJdLFsxLDNdLFszLDVdLFsyLDRdLFs0LDVdXQ==&macro_url=https%3A%2F%2Fraw.githubusercontent.com%2FdFoiler%2Fnotes%2Fmaster%2Fnir.tex
	\[\begin{tikzcd}[cramped]
		{\Lambda^n_i} & {\displaystyle(\Lambda^n_i\times\Delta^1)\bigsqcup_{\Lambda^n_i\times\Delta^0}(\Delta^n\times\Delta^0)} & {\Lambda_i^n} \\
		{\Delta^n} & {\Delta^n\times\Delta^1} & {\Delta^n}
		\arrow[from=1-1, to=1-2]
		\arrow["i"', from=1-1, to=2-1]
		\arrow[from=1-2, to=1-3]
		\arrow["{i\boxtimes\varepsilon_0}"', from=1-2, to=2-2]
		\arrow[from=1-3, to=2-3]
		\arrow[from=2-1, to=2-2]
		\arrow[from=2-2, to=2-3]
	\end{tikzcd}\]
	is a retract diagram, which will complete the proof. Because the vertical maps are all monic, we only have to construct the bottom maps, and then one needs to check that one can induce the top maps accordingly. Well, the first map $\Delta^n\to\Delta^n\times\Delta^1$ is just ${\id_{\Delta^n}}\times\varepsilon_1$. The second map requires some more work to define. In a picture, it is defined via the $n$-simplex
	% https://q.uiver.app/#q=WzAsMTYsWzAsMCwiMCJdLFsxLDAsIjEiXSxbMiwwLCJcXGNkb3RzIl0sWzMsMCwiaS0xIl0sWzQsMCwiaSJdLFs1LDAsImkiXSxbNiwwLCJcXGNkb3RzIl0sWzAsMSwiMCJdLFsxLDEsIjEiXSxbMiwxLCJcXGNkb3RzIl0sWzMsMSwiaS0xIl0sWzQsMSwiaSJdLFs1LDEsImkrMSJdLFs2LDEsIlxcY2RvdHMiXSxbNywwLCJpIl0sWzcsMSwibiJdLFswLDddLFs3LDhdLFs4LDldLFs5LDEwXSxbMTAsMTFdLFsxMSwxMl0sWzEyLDEzXSxbMTMsMTVdLFsxNCwxNV0sWzAsMV0sWzEsMl0sWzIsM10sWzMsNF0sWzQsNV0sWzUsNl0sWzYsMTRdLFsxLDhdLFszLDEwXSxbNCwxMV0sWzUsMTJdXQ==&macro_url=https%3A%2F%2Fraw.githubusercontent.com%2FdFoiler%2Fnotes%2Fmaster%2Fnir.tex
	\[\begin{tikzcd}[cramped]
		0 & 1 & \cdots & {i-1} & i & i & \cdots & i \\
		0 & 1 & \cdots & {i-1} & i & {i+1} & \cdots & n
		\arrow[from=1-1, to=1-2]
		\arrow[from=1-1, to=2-1]
		\arrow[from=1-2, to=1-3]
		\arrow[from=1-2, to=2-2]
		\arrow[from=1-3, to=1-4]
		\arrow[from=1-4, to=1-5]
		\arrow[from=1-4, to=2-4]
		\arrow[from=1-5, to=1-6]
		\arrow[from=1-5, to=2-5]
		\arrow[from=1-6, to=1-7]
		\arrow[from=1-6, to=2-6]
		\arrow[from=1-7, to=1-8]
		\arrow[from=1-8, to=2-8]
		\arrow[from=2-1, to=2-2]
		\arrow[from=2-2, to=2-3]
		\arrow[from=2-3, to=2-4]
		\arrow[from=2-4, to=2-5]
		\arrow[from=2-5, to=2-6]
		\arrow[from=2-6, to=2-7]
		\arrow[from=2-7, to=2-8]
	\end{tikzcd}\]
	of $\Delta^n\times\Delta^1$.
\end{proof}
For the second inclusion, we will use the following result.
\begin{proposition} \label{prop:decompose-sset-monic}
	Suppose $A\to B$ is a monomorphism of simplicial sets. Then there is a canonical sequence of morphisms
	\[A\to A_0\to A_1\to\cdots\]
	with $A=A_{-1}$ such that $B$ is the colimit of this diagram, and there are pushout squares as follows.
	% https://q.uiver.app/#q=WzAsNCxbMCwwLCJcXGRpc3BsYXlzdHlsZVxcYmlnc3FjdXBfe0lfaX1cXGRlbFxcRGVsdGFeaSJdLFsxLDAsIkFfe2ktMX0iXSxbMSwxLCJBX2kiXSxbMCwxLCJcXGRpc3BsYXlzdHlsZVxcYmlnc3FjdXBfe0lfaX1cXERlbHRhXmkiXSxbMCwzXSxbMywyXSxbMCwxXSxbMSwyXV0=&macro_url=https%3A%2F%2Fraw.githubusercontent.com%2FdFoiler%2Fnotes%2Fmaster%2Fnir.tex
	\[\begin{tikzcd}[cramped]
		{\displaystyle\bigsqcup_{I_i}\del\Delta^i} & {A_{i-1}} \\
		{\displaystyle\bigsqcup_{I_i}\Delta^i} & {A_i}
		\arrow[from=1-1, to=1-2]
		\arrow[from=1-1, to=2-1]
		\arrow[from=1-2, to=2-2]
		\arrow[from=2-1, to=2-2]
	\end{tikzcd}\]
\end{proposition}
\begin{proof}
	Intuitively, we are attaching $i$-simplices to $A_{i-1}$ to produce $B$. More precisely, one sets $A_i$ to be the smallest simplicial subset of $B$ containing $A$ and for which the maps $A_i(j)\to B(j)$ are isomorphisms for $j\le i$.
\end{proof}
\begin{lemma}
	Let $\mc P$ be the smallest saturated class containing all maps of the form $f\boxtimes i$, where $f\colon A\to B$ is monic, and $i\colon\Delta^0\to\Delta^1$ is some map. Then every map in $\mc P$ is anodyne.
\end{lemma}
\begin{proof}
	Given any monomorphism $f\colon A\to B$ and map $i\colon\Delta^0\to\Delta^1$, we must check that $f\boxtimes i$ is anodyne. By \Cref{prop:decompose-sset-monic}, we reduce to the case where $f$ is the inclusion $\del\Delta^n\to\Delta^n$. This is rather complicated, so we will content ourselves with $n=1$. Here, we are looking at the following map.
	\[(\del\Delta^1\times\Delta^1)\bigsqcup_{\del\Delta^1\times\Delta^0}(\Delta^1\times\Delta^0)\to(\Delta^1\times\Delta^1).\]
	The right-hand side is a square, and the left-hand side is the following boundary.
	% https://q.uiver.app/#q=WzAsNCxbMCwxLCJcXGJ1bGxldCJdLFsxLDEsIlxcYnVsbGV0Il0sWzAsMCwiXFxidWxsZXQiXSxbMSwwLCJcXGJ1bGxldCJdLFswLDFdLFsxLDNdLFswLDJdXQ==&macro_url=https%3A%2F%2Fraw.githubusercontent.com%2FdFoiler%2Fnotes%2Fmaster%2Fnir.tex
	\[\begin{tikzcd}[cramped]
		\bullet & \bullet \\
		\bullet & \bullet
		\arrow[from=2-1, to=1-1]
		\arrow[from=2-1, to=2-2]
		\arrow[from=2-2, to=1-2]
	\end{tikzcd}\]
	To show that this is anodyne, we note that we can fill in the lower-right triangle via a pushout against $\Lambda_1^2\to\Delta^2$, and then we can fill in the upper-left triangle by pushing out against $\Lambda_0^2\to\Delta^2$. Thus, our inclusion is anodyne.
\end{proof}
This concludes the proof of \Cref{lem:weird-anodyne} and thus the proof of \Cref{thm:anodyne-pushout-product}.

\subsection{Internal \texorpdfstring{$\op{Mor}$}{ Mor}}
Let's explain an application.
\begin{definition}
	Fix simplicial sets $X$ and $Y$. Then we define the simplicial set $\underline{\op{Mor}}(X,Y)$ as having $n$-simplices
	\[\underline{\op{Mor}}(X,Y)_n\coloneqq\op{Mor}_{\mathrm{sSet}}(X\times\Delta^n,Y).\]
\end{definition}
\begin{remark} \label{rem:curry}
	This gives $\mathrm{sSet}$ ``internal $\op{Mor}$s,'' making it a Cartesian closed category. Namely, one can see that we have a natural isomorphism
	\[\op{Mor}_{\mathrm{sSet}}(A,\underline{\op{Mor}}(B,C))\simeq\op{Mor}_{\mathrm{sSet}}(A\times B,C).\]
	To motivate our definition, we note that requiring this would require
	\[\underline{\op{Mor}}(X,Y)_n=\op{Mor}_{\mathrm{sSet}}(\Delta^n,\underline{\op{Mor}}(X,Y))=\op{Mor}_{\mathrm{sSet}}(\Delta^n\times X,Y),\]
	which is the given definition up to commutativity of the product.
\end{remark}
\begin{proposition}
	Fix simplicial sets $X$ and $Y$ such that $Y$ is a Kan complex. Then $\underline{\op{Mor}}(X,Y)$ is a Kan complex. Similarly, if $Y$ is a quasicategory, then $\underline{\op{Mor}}(X,Y)$ is a quasicategory.
\end{proposition}
\begin{proof}
	We show the first statement because the proof of the second is almost identical. Given any map $\Lambda^n_i\to\underline{\op{Mor}}(X,Y)$, we need to exhibit a lifted map $\Delta^n\to\underline{\op{Mor}}(X,Y)$. By \Cref{rem:curry}, we need to exhibit a lift in the following diagram.
	% https://q.uiver.app/#q=WzAsMyxbMCwwLCJcXExhbWJkYV9pXm5cXHRpbWVzIFgiXSxbMCwxLCJcXERlbHRhXm5cXHRpbWVzIFgiXSxbMSwwLCJZIl0sWzAsMl0sWzAsMV0sWzEsMl1d&macro_url=https%3A%2F%2Fraw.githubusercontent.com%2FdFoiler%2Fnotes%2Fmaster%2Fnir.tex
	\[\begin{tikzcd}[cramped]
		{\Lambda_i^n\times X} & Y \\
		{\Delta^n\times X}
		\arrow[from=1-1, to=1-2]
		\arrow[from=1-1, to=2-1]
		\arrow[from=2-1, to=1-2]
	\end{tikzcd}\]
	However, the left-hand map is anodyne by \Cref{thm:anodyne-pushout-product}, so we are done!
\end{proof}
\begin{remark}
	This is an incarnation of the fact that the category of Kan complexes is much nicer than $\mathrm{Top}$: we have internal $\op{Mor}$s, which is much harder to come by for topological spaces.
\end{remark}
Let's keep working with Kan complexes.
\begin{definition}[isomorphism]
	Two objects in a Kan complex are \textit{isomorphic} if and only if there is a map between them.
\end{definition}
\begin{definition}[equivalence]
	A map $f\colon X\to Y$ of Kan complexes is a \textit{homotopy equivalence} if and only if there is a map $g\colon Y\to X$ such that $f\circ g$ is isomorphic to $\id_Y$ in $\underline{\op{Mor}}(Y,Y)$ and that $g\circ f$ is isomorphic to $\id_X$ in $\underline{\op{Mor}}(X,X)$.
\end{definition}
\begin{definition}
	Fix a Kan complex $X$. Then $\pi_0(X)$ is the set of isomorphism classes of objects in $X$.
\end{definition}

\end{document}