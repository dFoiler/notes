% !TEX root = ../notes.tex

\documentclass[../notes.tex]{subfiles}

\begin{document}

\section{September 4}
The first problem set will be posted in about a day.

\subsection{More on Simplicial Sets}
It is worthwhile to explain nerves a little more.
\begin{exe}
	Fix a category $\mc C$. We work our $N(\mc C)_i$ for $i\in\{0,1,2\}$.
\end{exe}
\begin{proof}
	Here we go.
	\begin{itemize}
		\item We see $N(\mc C)_0$ consists of functors from the category $\{\bullet\}$, which are just objects of $\mc C$.
		\item Similarly, $N(\mc C)_1$ consists of functors from the category $\{\bullet\to\bullet\}$ to $\mc C$, which are just morphisms of $\mc C$.
		\item Lastly, we note $N(\mc C)_2$ consists of functors from the category $\{\bullet\to\bullet\to\bullet\}$ to $\mc C$, which amounts to the data of a diagram
		% https://q.uiver.app/#q=WzAsNyxbMywwLCJjIl0sWzQsMCwiYyciXSxbNCwxLCJjJyciXSxbMCwwLCJcXGJ1bGxldCJdLFsxLDAsIlxcYnVsbGV0Il0sWzEsMSwiXFxidWxsZXQiXSxbMiwwLCJcXFJpZ2h0YXJyb3ciXSxbMyw0XSxbNCw1XSxbMyw1XSxbMCwxLCJmIl0sWzEsMiwiZyJdLFswLDIsIihnXFxjaXJjIGYpIiwyXV0=&macro_url=https%3A%2F%2Fraw.githubusercontent.com%2FdFoiler%2Fnotes%2Fmaster%2Fnir.tex
		\[\begin{tikzcd}[cramped]
			\bullet & \bullet & \Rightarrow & c & {c'} \\
			& \bullet &&& {c''}
			\arrow[from=1-1, to=1-2]
			\arrow[from=1-1, to=2-2]
			\arrow[from=1-2, to=2-2]
			\arrow["f", from=1-4, to=1-5]
			\arrow["{(g\circ f)}"', from=1-4, to=2-5]
			\arrow["g", from=1-5, to=2-5]
		\end{tikzcd}\]
		so that the nerve is required to know something about composition!
	\end{itemize}
	Let's work out $N$ on some morphisms. For example, the canonical map $\sigma_0\colon[n]\to[0]$ picks out the identity diagram, and the maps $\delta^i$ pick out some sub-diagrams.
\end{proof}
It turns out that $\mathrm{sSet}$ is a presheaf category.
\begin{definition}[presheaf]
	Fix a category $\mc C$. Then a \textit{presheaf} on $\mc C$ is a functor $\mc C\opp\to\mathrm{Set}$. Accordingly, the presheaf category $\op{PSh}(\mc C)$ is the functor category $\op{Fun}(\mc C\opp,\mathrm{Set})$.
\end{definition}
\begin{example}
	We see that $\mathrm{sSet}=\op{PSh}(\Delta)$.
\end{example}
These categories are nice because they admit Yoneda embeddings.
\begin{lemma}[Yoneda]
	Fix a category $\mc C$. Then there is a functor $\yo\colon\mc C\to\mathrm{PSh}(\mc C)$ which is defined on objects by
	\[\yo(c)\coloneqq\op{Mor}_{\mc C}(-,c).\]
	Furthermore, $\yo$ is fully faithful.
\end{lemma}
\begin{remark}
	Another way to state the last conclusion is that there is a canonical bijection between $\op{Mor}_{\mc C}(c_1,c_2)$ and natural transformations $\yo(c_1)\Rightarrow\yo(c_2)$.
\end{remark}
\begin{definition}[representable]
	A presheaf $\mc F$ on a category $\mc C$ is \textit{representable} if and only if there is an object $c\in\mc C$ for which $\mc F$ is isomorphic to $\yo(c)$.
\end{definition}
We are now allowed to remove the absolute value bars from our $\Delta^n$.
\begin{notation}
	We define $\Delta^n$ as the simplicial set $\yo([n])$.
\end{notation}
\begin{example}
	We see that $\left(\Delta^n\right)_m$ consists of the order-preserving maps $[m]\to[n]$. For example, $\left(\Delta^2\right)_0$ has three elements, and $\left(\Delta^2\right)_1$ has six elements. Here are three of the elements of $\left(\Delta^2\right)_1$.
	% https://q.uiver.app/#q=WzAsMyxbMCwwLCIwIl0sWzEsMCwiMSJdLFsxLDEsIjIiXSxbMCwxXSxbMSwyXSxbMCwyXV0=&macro_url=https%3A%2F%2Fraw.githubusercontent.com%2FdFoiler%2Fnotes%2Fmaster%2Fnir.tex
	\[\begin{tikzcd}[cramped]
		0 & 1 \\
		& 2
		\arrow[from=1-1, to=1-2]
		\arrow[from=1-1, to=2-2]
		\arrow[from=1-2, to=2-2]
	\end{tikzcd}\]
\end{example}
Here is another important simplicial set.
\begin{definition}[boundary]
	For each $n\ge0$, we define the \textit{boundary} $\del\Delta^n\in\mathrm{sSet}$ to be the subfunctor of $\Delta^n$ with $\del\Delta^n(m)$ given by the non-surjective maps $[m]\to[n]$.
\end{definition}
\begin{definition}[horn]
	For each $i\in[n]$, we define the \textit{$i$th horn} $\Lambda_i^n\in\mathrm{sSet}$ to be the subfunctor of $\Delta^n$ with $\Lambda^i_n$ given by the maps $[m]\to[n]$. We say that $\Lambda_i^n$ is an \textit{inner horn} if and only if $0<i<n$; otherwise, $\Lambda_i^n$ is an outer horn.
\end{definition}
\begin{remark}
	There are canonical inclusions $\Lambda^i_n\subseteq\del\Delta^n\subseteq\Delta^n$ for each relevant $i$ and $n$.
\end{remark}
\begin{example}
	Intuitively, $\Lambda^n_i$ deletes the face opposite $i$. For example, here is $\Lambda_0^2$.
	% https://q.uiver.app/#q=WzAsMyxbMCwwLCIwIl0sWzEsMCwiMSJdLFsxLDEsIjIiXSxbMCwxXSxbMCwyXV0=&macro_url=https%3A%2F%2Fraw.githubusercontent.com%2FdFoiler%2Fnotes%2Fmaster%2Fnir.tex
	\[\begin{tikzcd}[cramped]
		0 & 1 \\
		& 2
		\arrow[from=1-1, to=1-2]
		\arrow[from=1-1, to=2-2]
	\end{tikzcd}\]
	One can similarly draw $\Lambda^2_1$ (which omits $0\to2$) and $\Lambda^2_2$ (which omits $0\to1$).
\end{example}

\subsection{Lifting Horns}
These horns allow us to state a special property of nerves.
\begin{proposition} \label{prop:nerve-has-inner-horn-filler}
	Fix a category $\mc C$. Then any map $\Lambda_i^n\to N(\mc C)$ from an inner horn $\Lambda^n_i$ extends uniquely to a map $\Delta^n\to N(\mc C)$.
	% https://q.uiver.app/#q=WzAsMyxbMCwwLCJcXExhbWJkYV9pXm4iXSxbMSwwLCJOKFxcbWMgQykiXSxbMCwxLCJcXERlbHRhXm4iXSxbMCwyLCIiLDAseyJzdHlsZSI6eyJ0YWlsIjp7Im5hbWUiOiJob29rIiwic2lkZSI6InRvcCJ9fX1dLFswLDFdLFsyLDEsIiEiLDEseyJzdHlsZSI6eyJib2R5Ijp7Im5hbWUiOiJkYXNoZWQifX19XV0=&macro_url=https%3A%2F%2Fraw.githubusercontent.com%2FdFoiler%2Fnotes%2Fmaster%2Fnir.tex
	\[\begin{tikzcd}[cramped]
		{\Lambda_i^n} & {N(\mc C)} \\
		{\Delta^n}
		\arrow[from=1-1, to=1-2]
		\arrow[hook, from=1-1, to=2-1]
		\arrow["{!}"{description}, dashed, from=2-1, to=1-2]
	\end{tikzcd}\]
\end{proposition}
\begin{remark}
	In fact, a simplicial set is the nerve of a category if and only if it satisfies the conclusion of \Cref{prop:nerve-has-inner-horn-filler}. Thus, we have a characterization of the image of the fully faithful nerve functor! Amusingly, this allows one to give an alternate definition of a category in terms of simplicial sets; this is not circular because one can define simplicial sets as combinatorial simplicial sets.
\end{remark}
\begin{example}
	We show that any map $\Lambda_2^1\to N(\mc C)$ admits a unique extension to $\Delta^2$. Well, $\Lambda_2^1$ specifies two maps $f\colon c_0\to c_1$ and $g\colon c_1\to c_2$, which we complete to a map from $\Delta^2$ by defining the map $c_0\to c_2$ to be the composite.
\end{example}
\begin{example}
	It turns out that extending maps $\Lambda_1^3\to N(\mc C)$ and $\Lambda_2^3\to N(\mc C)$ to $\Delta^3$ encodes associativity of composition.
\end{example}
\begin{nex}
	One does not expect any map $\Lambda_0^2\to N(\mc C)$ to always extend to $\Delta^2$. Indeed, $\Lambda_0^2$ only has the maps $0\to 1$ and $0\to 2$, but there is no obvious way to then produce a map $1\to 2$ in the nerve!
\end{nex}
\begin{remark} \label{rem:groupoid-is-kan}
	One can check that a category is a groupoid if and only if the outer horns also admit horn fillings. The point is that being a groupoid allows one to reverse all the arrows, so coherence of composition allows one to do the filling.
\end{remark}
We now turn to $\op{Sing}$.
\begin{proposition}
	The functor $\op{Sing}\colon\mathrm{sSet}\to\mathrm{Top}$ admits a left adjoint $\left|\cdot\right|\colon\mathrm{Top}\to\mathrm{sSet}$. In fact, $\left|\Delta^n\right|$ is defined to be the topological $n$-simplex.
\end{proposition}
It is worthwhile to know how to construct adjoints.
\begin{theorem}
	Fix a category $\mc C$. Then $\op{PSh}(\mc C)$ has all limits and colimits.
\end{theorem}
\begin{theorem}
	Suppose that $\mc C$ and $\mc D$ are categories, where $\mc D$ admits colimits. For any functor $F\colon\mc C\to\mc D$, there is a unique functor $G\colon\mathrm{PSh}(\mc C)\to\mc D$ preserving colimits for which the composite
	\[\mc C\stackrel{\yo}\to\mathrm{PSh}(\mc C)\stackrel G\to\mc D.\]
	In fact, $G$ is a left adjoint.
\end{theorem}
\begin{remark}
	This property characterizes $\op{Sing}$: indeed, for any topological space $Y$, we need to have $\op{Sing}(Y)(n)$ to be
	\[\op{Mor}_{\mathrm{sSet}}(\Delta^n,\op{Sing}(Y))=\op{Mor}_{\mathrm{Top}}(\left|\Delta^n\right|,Y).\]
\end{remark}
We are now able to characterize the image of $\op{Sing}$.
\begin{proposition} \label{prop:sing-is-kan}
	Fix a topological space $Y$. Then any map $\Lambda_i^n\to\op{Sing}Y$ admits a lift to a map $\Delta^n\to\op{Sing}Y$.
\end{proposition}
\begin{proof}
	By the adjunction, it is enough to lift a map $\left|\Lambda_i^n\right|\to Y$ to a map $\left|\Delta^n\right|\to Y$. But this is not hard because there are projection maps $\left|\Delta^n\right|\to\left|\Lambda_i^n\right|$.
\end{proof}

\subsection{Kan Complexes}
\Cref{prop:sing-is-kan} motivates the following definition.
\begin{definition}[Kan complex]
	A \textit{Kan complex} is a simplicial set $X$ in which every $\Lambda^n_i\to X$ admits a lift to a map $\Delta^n\to X$.
\end{definition}
\begin{example}
	By \Cref{prop:sing-is-kan}, we see that $\op{Sing}Y$ is always a Kan complex.
\end{example}
\begin{example}
	By \Cref{rem:groupoid-is-kan}, we see that $N(\mc C)$ is a Kan complex if and only if $\mc C$
\end{example}
At long last, we may define $\infty$-categories, which is intended to simultaneously generalize nerves and Kan complexes.
\begin{defihelper}[$\infty$-category, quasicategory] \nirindex{quasicategory} \nirindex{infinity category@$\infty$-category}
	An \textit{$\infty$-category} \textit{quasicategory} is a simplicial set $X$ where every inner horn $\Lambda^n_i\to X$ admits a lift to $\Delta^n\to X$. We may call $X_0$ the \textit{objects}, call $X_1$ the \textit{morphisms}, and call $X_n$ the $n$-morphisms for $n\ge1$. More concretely, for any $E\in\mc C_2$, we may say that $d_1E$ exhibits a \textit{$2$-isomorphism} between $d_0E$ and $d_2E$.
\end{defihelper}
\begin{definition}[homotopic]
	Two maps $f,g\colon X\to Y$ are \textit{homotopic} if and only if there is a map $h\colon X\times\Delta^1\to Y$ such that the composites with $d_0\colon X\times\Delta^0\to X\times\Delta^1$ and $d_1\colon X\times\Delta^0\to X\times\Delta^1$ are $g$ and $f$, respectively.
\end{definition}
\begin{remark}
	It turns out that being homotopic is an equivalence relation; the symmetry check uses the fact that $Y$ is a Kan complex.
\end{remark}
\begin{definition}[homotopy equivalent]
	Two Kan complexes $X$ and $Y$ are homotopy equivalent if and only if there are maps $f\colon X\to Y$ and $g\colon Y\to X$ such that $f\circ g$ and $g\circ f$ are both homotopic to the identities.
\end{definition}
We will make use of the following hard(!) theorem.
\begin{theorem}[Quillen]
	If $X$ is a CW complex, then $\left|\op{Sing}X\right|$ is homotopy equivalent to $X$. Similarly, if $X$ is a Kan complex, then $\op{Sing}\left|X\right|$ is homotopy equivalent to $X$.
\end{theorem}
\begin{corollary}
	The homotopy category of topological spaces is equivalent to the homotopy category of Kan complexes.
\end{corollary}
This theorem is a purely motivational statement: it allows us to pass from topological spaces to just Kan complexes.

\end{document}