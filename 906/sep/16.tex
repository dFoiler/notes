% !TEX root = ../notes.tex

\documentclass[../notes.tex]{subfiles}

\begin{document}

\section{September 16}
There is a grader now.

\subsection{Some Fibrations}
Just as Kan fibrations admit lifts against anodyne maps, we may define inner fibrations.
\begin{definition}
	A map $f\colon X\to Y$ of simplicial sets is an \textit{inner fibration} if and only if every inner horn $\Lambda^n_i\to\Delta^n$ with a map $\Lambda^n_i\to X$ and $\Delta^n\to Y$ (commuting with $f$) admits a lifting making the following diagram commute.
	% https://q.uiver.app/#q=WzAsNCxbMSwwLCJYIl0sWzEsMSwiWSJdLFswLDAsIlxcTGFtYmRhXm5faSJdLFswLDEsIlxcRGVsdGFebiJdLFsyLDBdLFszLDFdLFsyLDNdLFswLDEsImYiXSxbMywwLCIiLDAseyJzdHlsZSI6eyJib2R5Ijp7Im5hbWUiOiJkYXNoZWQifX19XV0=&macro_url=https%3A%2F%2Fraw.githubusercontent.com%2FdFoiler%2Fnotes%2Fmaster%2Fnir.tex
	\[\begin{tikzcd}[cramped]
		{\Lambda^n_i} & X \\
		{\Delta^n} & Y
		\arrow[from=1-1, to=1-2]
		\arrow[from=1-1, to=2-1]
		\arrow["f", from=1-2, to=2-2]
		\arrow[dashed, from=2-1, to=1-2]
		\arrow[from=2-1, to=2-2]
	\end{tikzcd}\]
\end{definition}
\begin{remark}
	Not every anodyne map is inner anodyne. For example, arbitrary $\infty$-categories are not expected to be $\infty$-groupoids.
\end{remark}
\begin{example}
	A simplicial set $X$ is an $\infty$-category if and only if the canonical map $X\to\Delta^0$ is an inner fibration.
\end{example}
\begin{example}
	Let $J$ be the nerve of the category with two isomorphic points. Then the inclusion $\Delta^1\to J$ is anodyne but not inner anodyne, which we can see because arbitrary nerves are not expected to receive lifts of the map $\Delta^1\to J$.
\end{example}
\begin{remark}
	Every monomorphism is in the saturated class containing the boundary maps $\del\Delta^n\to\Delta^n$. This is because any monomorphism can be written as a (transfinite) composition of cell attachments.
\end{remark}
\begin{definition}[trivial fibration]
	A map $f\colon X\to Y$ of simplicial sets is a \textit{trivial fibration} if and only if every boundary horn $\del\Delta^n\to\Delta^n$ with maps $\del\Delta^n\to X$ and $\Delta^n\to Y$ (commuting with $f$) admits a lifting making the following diagram commute.
	% https://q.uiver.app/#q=WzAsNCxbMSwwLCJYIl0sWzEsMSwiWSJdLFswLDAsIlxcTGFtYmRhXm5faSJdLFswLDEsIlxcRGVsdGFebiJdLFsyLDBdLFszLDFdLFsyLDNdLFswLDEsImYiXSxbMywwLCIiLDAseyJzdHlsZSI6eyJib2R5Ijp7Im5hbWUiOiJkYXNoZWQifX19XV0=&macro_url=https%3A%2F%2Fraw.githubusercontent.com%2FdFoiler%2Fnotes%2Fmaster%2Fnir.tex
	\[\begin{tikzcd}[cramped]
		{\del\Delta^n} & X \\
		{\Delta^n} & Y
		\arrow[from=1-1, to=1-2]
		\arrow[from=1-1, to=2-1]
		\arrow["f", from=1-2, to=2-2]
		\arrow[dashed, from=2-1, to=1-2]
		\arrow[from=2-1, to=2-2]
	\end{tikzcd}\]
\end{definition}
\begin{remark}
	Not every anodyne map is inner anodyne. For example, arbitrary $\infty$-categories are not expected to be $\infty$-groupoids.
\end{remark}

\subsection{Common Names for Fibrations}
Here are some definitions for quasicategories.
\begin{definition}[functor]
	A \textit{functor} of quasicategories $\mc C$ and $\mc D$ is a map of the underlying simplicial sets.
\end{definition}
\begin{definition}[natural transformation]
	A \textit{natural transformation} of quasicategories $\mc C$ and $\mc D$ is a morphism in $\underline{\op{Mor}}(\mc C,\mc D)$, which has the same data as a map $\mc C\times\Delta^1\to\mc D$. If this morphism factors through the two-element category $J$ with two isomorphic elements, then we say that the morphism is a \textit{natural isomorphism}
\end{definition}
\begin{definition}[isomorphism]
	A morphism $f$ in a quasicategory $\mc C$ is an \textit{isomorphism} if and only if the induced map $\Delta^1\to\mc C$ factors through the two-element category $J$ with two isomorphic elements.
\end{definition}
\begin{definition}[equivalence]
	A functor $F\colon\mc C\to\mc D$ of quasicategories is an \textit{equivalence} if and only if there is a functor $G\colon\mc D\to\mc C$ for which $FG$ and $GF$ are both isomorphic to the identities in $\underline{\op{Mor}}(\mc D,\mc D)$ and $\underline{\op{Mor}}(\mc C,\mc C)$, respectively. If $\mc C$ and $\mc D$ are in fact $\infty$-groupoids, then 
\end{definition}
\begin{proposition} \label{prop:triv-fibration-to-equiv}
	Fix a functor $F\colon\mc C\to\mc D$. If $F$ is a trivial fibration, then $F$ is an equivalence.
\end{proposition}
\begin{proof}
	Note that the canonical map $\emp\to\mc D$ is monic, so we gain a functor $G\colon\mc D\to\mc C$ making the diagram
	% https://q.uiver.app/#q=WzAsNCxbMCwwLCJcXGVtcCJdLFswLDEsIlxcbWMgRCJdLFsxLDAsIlxcbWMgQyJdLFsxLDEsIlxcbWMgRCJdLFswLDEsIiIsMCx7InN0eWxlIjp7InRhaWwiOnsibmFtZSI6Imhvb2siLCJzaWRlIjoidG9wIn19fV0sWzIsMywiRiJdLFsxLDMsIiIsMSx7ImxldmVsIjoyLCJzdHlsZSI6eyJoZWFkIjp7Im5hbWUiOiJub25lIn19fV0sWzAsMiwiIiwxLHsic3R5bGUiOnsidGFpbCI6eyJuYW1lIjoiaG9vayIsInNpZGUiOiJ0b3AifX19XSxbMSwyLCJHIiwxLHsic3R5bGUiOnsiYm9keSI6eyJuYW1lIjoiZGFzaGVkIn19fV1d&macro_url=https%3A%2F%2Fraw.githubusercontent.com%2FdFoiler%2Fnotes%2Fmaster%2Fnir.tex
	\[\begin{tikzcd}[cramped]
		\emp & {\mc C} \\
		{\mc D} & {\mc D}
		\arrow[hook, from=1-1, to=1-2]
		\arrow[hook, from=1-1, to=2-1]
		\arrow["F", from=1-2, to=2-2]
		\arrow["G"{description}, dashed, from=2-1, to=1-2]
		\arrow[equals, from=2-1, to=2-2]
	\end{tikzcd}\]
	commute. By construction, we see that $FG$ is equal to the identity in $\underline{\op{Mor}}(\mc D,\mc D)$. To compute $GF$, we draw the diagram
	% https://q.uiver.app/#q=WzAsNSxbMCwwLCIoXFx7MFxcfVxcdGltZXNcXG1jIEMpXFxzcWN1cChcXHsxXFx9XFx0aW1lc1xcbWMgQykiXSxbMCwxLCJcXERlbHRhXjFcXHRpbWVzXFxtYyBDIl0sWzEsMSwiXFxtYyBDIl0sWzIsMSwiXFxtYyBEIl0sWzIsMCwiXFxtYyBDIl0sWzEsMiwiXFxvcHtwcn1fMiIsMl0sWzIsMywiRiIsMl0sWzAsMSwiIiwyLHsic3R5bGUiOnsidGFpbCI6eyJuYW1lIjoiaG9vayIsInNpZGUiOiJ0b3AifX19XSxbMSw0LCJcXGV0YSIsMSx7InN0eWxlIjp7ImJvZHkiOnsibmFtZSI6ImRhc2hlZCJ9fX1dLFs0LDMsIkYiXSxbMCw0LCIoR0Yse1xcaWRfe1xcbWMgQ319KSJdXQ==&macro_url=https%3A%2F%2Fraw.githubusercontent.com%2FdFoiler%2Fnotes%2Fmaster%2Fnir.tex
	\[\begin{tikzcd}[cramped]
		{(\{0\}\times\mc C)\sqcup(\{1\}\times\mc C)} && {\mc C} \\
		{\Delta^1\times\mc C} & {\mc C} & {\mc D}
		\arrow["{(GF,{\id_{\mc C}})}", from=1-1, to=1-3]
		\arrow[hook, from=1-1, to=2-1]
		\arrow["F", from=1-3, to=2-3]
		\arrow["\eta"{description}, dashed, from=2-1, to=1-3]
		\arrow["{\op{pr}_2}"', from=2-1, to=2-2]
		\arrow["F"', from=2-2, to=2-3]
	\end{tikzcd}\]
	where the outer square commutes because $FG=\id_{\mc D}$, so we get to induce a lift $\eta\colon\Delta^1\times\mc C\to\mc C$, which is a natural isomorphism from $GF$ to $\id_{\mc C}$. To check that $\eta$ is actually a natural isomorphism, we note that the commutativity of the outer square in
	% https://q.uiver.app/#q=WzAsNSxbMCwwLCJcXERlbHRhXjFcXHRpbWVzXFxtYyBDIl0sWzAsMSwiSlxcdGltZXNcXG1jIEMiXSxbMSwxLCJcXG1jIEMiXSxbMiwxLCJcXG1jIEQiXSxbMiwwLCJcXG1jIEMiXSxbMCwxLCIiLDAseyJzdHlsZSI6eyJ0YWlsIjp7Im5hbWUiOiJob29rIiwic2lkZSI6InRvcCJ9fX1dLFsxLDIsIlxcb3B7cHJ9XzIiLDJdLFsyLDMsIkYiLDJdLFs0LDMsIkYiXSxbMCw0LCJcXGV0YSJdLFsxLDQsIiIsMCx7InN0eWxlIjp7ImJvZHkiOnsibmFtZSI6ImRhc2hlZCJ9fX1dXQ==&macro_url=https%3A%2F%2Fraw.githubusercontent.com%2FdFoiler%2Fnotes%2Fmaster%2Fnir.tex
	\[\begin{tikzcd}[cramped]
		{\Delta^1\times\mc C} && {\mc C} \\
		{J\times\mc C} & {\mc C} & {\mc D}
		\arrow["\eta", from=1-1, to=1-3]
		\arrow[hook, from=1-1, to=2-1]
		\arrow["F", from=1-3, to=2-3]
		\arrow[dashed, from=2-1, to=1-3]
		\arrow["{\op{pr}_2}"', from=2-1, to=2-2]
		\arrow["F"', from=2-2, to=2-3]
	\end{tikzcd}\]
	shows that we get a lifting map $J\times\mc C\to\mc C$, witnessing that $\eta$ is a natural isomorphism.
\end{proof}
\begin{proposition} \label{prop:anodyne-of-kan-is-equiv}
	Fix a map $f\colon X\to Y$ of Kan complexes. If $f$ is anodyne, then $f$ is an equivalence.
\end{proposition}
\begin{proof}
	We will show that the induced map $\underline{\op{Mor}}(Y,Z)\to\underline{\op{Mor}}(X,Z)$ is a trivial fibration for each Kan complex $Z$. Plugging in $Z\in\{X,Y\}$ shows that $\underline{\op{Mor}}(Y,Y)\to\underline{\op{Mor}}(X,Y)$ and $\underline{\op{Mor}}(Y,Z)\to\underline{\op{Mor}}(X,X)$ are trivial fibrations, so they are equivalence of Kan complexes. To check this, we need to show that we have lifts in a square of the form
	% https://q.uiver.app/#q=WzAsNCxbMCwwLCJcXGRlbFxcRGVsdGFebiJdLFsxLDAsIlxcdW5kZXJsaW5le1xcb3B7TW9yfX0oWSxaKSJdLFsxLDEsIlxcdW5kZXJsaW5le1xcb3B7TW9yfX0oWCxaKSJdLFswLDEsIlxcRGVsdGFebiJdLFswLDFdLFszLDJdLFsxLDJdLFswLDMsIiIsMSx7InN0eWxlIjp7InRhaWwiOnsibmFtZSI6Imhvb2siLCJzaWRlIjoidG9wIn19fV1d&macro_url=https%3A%2F%2Fraw.githubusercontent.com%2FdFoiler%2Fnotes%2Fmaster%2Fnir.tex
	\[\begin{tikzcd}[cramped]
		{\del\Delta^n} & {\underline{\op{Mor}}(Y,Z)} \\
		{\Delta^n} & {\underline{\op{Mor}}(X,Z)}
		\arrow[from=1-1, to=1-2]
		\arrow[hook, from=1-1, to=2-1]
		\arrow[from=1-2, to=2-2]
		\arrow[from=2-1, to=2-2]
	\end{tikzcd}\]
	which is equivalent to finding a lift in the square
	% https://q.uiver.app/#q=WzAsMyxbMCwwLCJcXGRpc3BsYXlzdHlsZShcXGRlbFxcRGVsdGFeblxcdGltZXMgWSlcXGJpZ3NxY3VwX3tcXGRlbFxcRGVsdGFeblxcdGltZXMgWH0oXFxEZWx0YV5uXFx0aW1lcyBYKSJdLFsxLDAsIloiXSxbMCwxLCJcXERlbHRhXm5cXHRpbWVzIFkiXSxbMiwxLCIiLDAseyJzdHlsZSI6eyJib2R5Ijp7Im5hbWUiOiJkYXNoZWQifX19XSxbMCwxXSxbMCwyXV0=&macro_url=https%3A%2F%2Fraw.githubusercontent.com%2FdFoiler%2Fnotes%2Fmaster%2Fnir.tex
	\[\begin{tikzcd}[cramped]
		{\displaystyle(\del\Delta^n\times Y)\bigsqcup_{\del\Delta^n\times X}(\Delta^n\times X)} & Z \\
		{\Delta^n\times Y}
		\arrow[from=1-1, to=1-2]
		\arrow[from=1-1, to=2-1]
		\arrow[dashed, from=2-1, to=1-2]
	\end{tikzcd}\]
	by using \Cref{rem:curry}. But this exists because the left map is inner anodyne by \Cref{lem:weird-anodyne}.
\end{proof}
\begin{proposition}
	Fix a map $f\colon X\to Y$ of quasicategories. If $f$ is inner anodyne, then $f$ is an equivalence.
\end{proposition}
\begin{proof}
	The proof is the same as in \Cref{prop:anodyne-of-kan-is-equiv}.
\end{proof}

\subsection{Homotopy Groups}
We now begin doing some algebraic topology. We are interested in constructing (and computing) algebraic invariants of topological spaces, which for this class amount to Kan complexes. For example, a topological space $X$ with basepoint $x\in X$ has a fundamental group $\pi_1(X,x)$ of loops in $X$ based at $x$ (up to homotopy), and there is a group operation given by path concatenation.

To do this for simplicial sets, we pick up the following notation.
\begin{notation}
	Fix morphisms $A\to B$ and $X\to Y$ of simplicial sets. Then we define the simplicial set $\underline{\op{Mor}}((B,A),(Y,X))$ as the pullback in the following diagram.
	% https://q.uiver.app/#q=WzAsNCxbMCwwLCJcXHVuZGVybGluZXtcXG9we01vcn19KChCLEEpLChZLFgpKSJdLFsxLDAsIlxcdW5kZXJsaW5le1xcb3B7TW9yfX0oQixZKSJdLFswLDEsIlxcdW5kZXJsaW5le1xcb3B7TW9yfX0oQSxYKSJdLFsxLDEsIlxcdW5kZXJsaW5le1xcb3B7TW9yfX0oQSxZKSJdLFsyLDNdLFsxLDNdLFswLDFdLFswLDJdLFswLDMsIiIsMSx7InN0eWxlIjp7Im5hbWUiOiJjb3JuZXIifX1dXQ==&macro_url=https%3A%2F%2Fraw.githubusercontent.com%2FdFoiler%2Fnotes%2Fmaster%2Fnir.tex
	\[\begin{tikzcd}[cramped]
		{\underline{\op{Mor}}((B,A),(Y,X))} & {\underline{\op{Mor}}(B,Y)} \\
		{\underline{\op{Mor}}(A,X)} & {\underline{\op{Mor}}(A,Y)}
		\arrow[from=1-1, to=1-2]
		\arrow[from=1-1, to=2-1]
		\arrow["\lrcorner"{anchor=center, pos=0.125}, draw=none, from=1-1, to=2-2]
		\arrow[from=1-2, to=2-2]
		\arrow[from=2-1, to=2-2]
	\end{tikzcd}\]
\end{notation}
\begin{remark}
	If $A\to B$ and $X\to Y$ are monomorphisms of Kan complexes, then $\underline{\op{Mor}}((B,A),(Y,X))$ is a Kan complex. This more or less follows from \Cref{thm:anodyne-pushout-product}.
\end{remark}
\begin{definition}[fundamental group]
	Fix a Kan complex $X$ and a point $x\in X_0$. Then we define the \textit{fundamental group} $\pi_1(X,x)$ is the isomorphism classes of
	\[\underline{\op{Mor}}((\Delta^1,\{0\}\sqcup\{1\}),(X,\{x\})).\]
\end{definition}
To go to higher homotopy groups, it will be technically convenient to have some more simplicial sets.
\begin{notation}
	For each nonnegative integer $n$, we define $\square^n$ as the $n$-fold product $\left(\Delta^1\right)^{\times n}$ and its boundary $\del\square^n$ as the union
	\[\bigcup_{j=0}^{n-1}\left(\Delta^1\right)^{\times j}\times(\{0\}\sqcup\{1\})\times\left(\Delta^1\right)^{\times(n-j-1)}.\]
\end{notation}
\begin{example}
	Here is a picture of $\del\square^1\subseteq\Delta^1$.
	% https://q.uiver.app/#q=WzAsNSxbMCwwLCJcXGJ1bGxldCJdLFsxLDAsIlxcYnVsbGV0Il0sWzMsMCwiXFxidWxsZXQiXSxbNCwwLCJcXGJ1bGxldCJdLFsyLDAsIlxcaW50byJdLFsyLDNdXQ==&macro_url=https%3A%2F%2Fraw.githubusercontent.com%2FdFoiler%2Fnotes%2Fmaster%2Fnir.tex
	\[\begin{tikzcd}[cramped]
		\bullet & \bullet & \into & \bullet & \bullet
		\arrow[from=1-4, to=1-5]
	\end{tikzcd}\]
	Here is a picture of $\del\square^2$.
	% https://q.uiver.app/#q=WzAsNCxbMCwwLCJcXGJ1bGxldCJdLFsxLDAsIlxcYnVsbGV0Il0sWzAsMSwiXFxidWxsZXQiXSxbMSwxLCJcXGJ1bGxldCJdLFswLDFdLFsxLDNdLFswLDJdLFsyLDNdXQ==&macro_url=https%3A%2F%2Fraw.githubusercontent.com%2FdFoiler%2Fnotes%2Fmaster%2Fnir.tex
	\[\begin{tikzcd}[cramped]
		\bullet & \bullet \\
		\bullet & \bullet
		\arrow[from=1-1, to=1-2]
		\arrow[from=1-1, to=2-1]
		\arrow[from=1-2, to=2-2]
		\arrow[from=2-1, to=2-2]
	\end{tikzcd}\]
	We note that $\square^2$ would also include the inner $2$-simplex.
\end{example}
We are now able to define higher homotopy groups.
\begin{definition}[homotopy groups]
	Fix a Kan complex $x$ and a point $x\in X_0$. For each nonnegative integer $n$, we define the \textit{homotopy group} $\pi_n(X,x)$ is the isomorphism classes of
	\[\underline{\op{Mor}}((\square^n,\del\square^n),(X,\{x\})).\]
\end{definition}
\begin{remark}
	One could get away with using the inclusions $\del\Delta^n\into\Delta^n$ instead of the squares, which we will prove later.
\end{remark}
Let's check that we have actually defined groups.
\begin{lemma}
	Fix a Kan complex $X$ and a point $x\in X_0$. Then there is a unital multiplication
	\[\pi_n(X,x)\to\pi_n(X,x)\to\pi_n(X,x)\]
	by lifting as follows.
	% https://q.uiver.app/#q=WzAsNCxbMSwwLCJcXExhbWJkYV8xXjJcXHRpbWVzXFxzcXVhcmVee24tMX0iXSxbMiwwLCJYIl0sWzEsMSwiXFxEZWx0YV4yXFx0aW1lc1xcc3F1YXJlXntuLTF9Il0sWzAsMSwiXFxEZWx0YV4xXFx0aW1lc1xcc3F1YXJlXntuLTF9Il0sWzAsMSwiKGYsZykiXSxbMywyLCJkXzEiXSxbMiwxLCIiLDEseyJzdHlsZSI6eyJib2R5Ijp7Im5hbWUiOiJkYXNoZWQifX19XSxbMCwyXV0=&macro_url=https%3A%2F%2Fraw.githubusercontent.com%2FdFoiler%2Fnotes%2Fmaster%2Fnir.tex
	\[\begin{tikzcd}[cramped]
		& {\Lambda_1^2\times\square^{n-1}} & X \\
		{\Delta^1\times\square^{n-1}} & {\Delta^2\times\square^{n-1}}
		\arrow["{(f,g)}", from=1-2, to=1-3]
		\arrow[from=1-2, to=2-2]
		\arrow["{d_1}", from=2-1, to=2-2]
		\arrow[dashed, from=2-2, to=1-3]
	\end{tikzcd}\]
\end{lemma}
\begin{proof}
	The motivation here is that the map $\Lambda_1^2\times\square^{n-1}\to X$ amounts to stacking the two square maps $f\colon\square^n\to X$ and $g\colon\square^n\to X$ on top of each other. Then the whole gluing process basically extends 
\end{proof}
There are actually $n$ different unital multiplications given by how we glue our boxes together. The fact that there are at least different multiplications for $n\ge2$ implies that these multiplications are all equal, associative, and commutative. This follows from the Eckmann--Hilton argument.
\begin{lemma}[Eckmann--Hilton]
	Fix a set $A$ with two binary operations $\cdot_1$ and $\cdot_2$ which are equipped with two-sided units $1_1$ and $1_2$. Further, suppose that
	\[(a\cdot_1b)\cdot_2(c\cdot_1d)=(a\cdot_2c)\cdot_1(b\cdot_2d).\]
	Then ${\cdot_1}={\cdot_2}$ and $1_1=1_2$. In fact, $\cdot_1$ and $\cdot_2$ are associative and commutative.
\end{lemma}
\begin{proof}
	We will write $\begin{bsmallmatrix}
		a \\ b
	\end{bsmallmatrix}$ for $a\cdot_2b$ and $\begin{bsmallmatrix}
		a & b
	\end{bsmallmatrix}$ for $a\cdot_1b$. The given identity amounts to saying that $\begin{bsmallmatrix}
		a & b \\ c & d
	\end{bsmallmatrix}$ is well-defined.

	Let's start by checking that the units are the same. For this, we write
	\begin{align*}
		1_2 &= \begin{bmatrix}
			1_2 \\ 1_2
		\end{bmatrix} \\
		&= \begin{bmatrix}
			1_2 & 1_1 \\
			1_1 & 1_2
		\end{bmatrix} \\
		&= \begin{bmatrix}
			1_1 & 1_1
		\end{bmatrix} \\
		&= 1_1.
	\end{align*}
	Thus, we may just write $1$ for our unit. For the commutativity, we write
	\begin{align*}
		\begin{bmatrix}
			a & b
		\end{bmatrix} &= \begin{bmatrix}
			a & 1 \\
			1 & b
		\end{bmatrix} \\
		&= \begin{bmatrix}
			a \\ b
		\end{bmatrix} \\
		&= \begin{bmatrix}
			1 & a \\
			b & 1
		\end{bmatrix} \\
		&= \begin{bmatrix}
			b & a
		\end{bmatrix}.
	\end{align*}
	Note that this has also shown that the two operations are the same. Lastly, to check associativity, we write
	\begin{align*}
		\begin{bmatrix}
			a & \begin{bmatrix}
				b & c
			\end{bmatrix}
		\end{bmatrix} &= \begin{bmatrix}
			a & \begin{bmatrix}
				b \\ c
			\end{bmatrix}
		\end{bmatrix} \\
		&= \begin{bmatrix}
			a & b \\
			1 & c
		\end{bmatrix} \\
		&= \begin{bmatrix}
			\begin{bmatrix}
				a & b
			\end{bmatrix} \\
			c
		\end{bmatrix} \\
		&= \begin{bmatrix}
			\begin{bmatrix}
				a & b
			\end{bmatrix} & c
		\end{bmatrix}.
	\end{align*}
	This finishes the proof!
\end{proof}
\begin{remark}
	Let's say a bit about motivation for these homotopy groups. One reason, of course, is that many meaningful invariants in mathematics turn out to be related to homotopy groups.
\end{remark}

\end{document}