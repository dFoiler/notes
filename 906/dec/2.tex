% !TEX root = ../notes.tex

\documentclass[../notes.tex]{subfiles}

\begin{document}

\section{December 2}
Today, we compute the homotopy groups of $\mathrm{MU}$.
\begin{remark}
	Professor Hahn sees the time ticking for his absurd task.
\end{remark}

\subsection{Computing Homology of \texorpdfstring{$\mathrm{MU}$}{ MU}}
Let's give a homotopy-theoretic construction of $\mathrm{MU}$. Taking the one-point compactification of $\CC^n$ produces a group homomorphism
\[\op U(n)\to\op{Aut}\left(S^{2n}\right).\]
Thus, we get a morphism of $\mathbb E_1$-spaces $\op{BU}(n)\to\op{BAut}\left(S^{2n}\right)$, and taking the colimit over $n$ produces a map $\mathrm{BU}\to\mathrm{BGL}_1(\mathbb S)$, where $\op{BGL}_1(\mathbb S)\subseteq\mathrm{Mod}(\mathbb S)$ is the one-object subcategory of spectra whose morphisms are all invertible.
\begin{definition}
	We define the spectrum $\op{MU}$ as the colimit
	\[\mathrm{MU}\coloneqq\colim\left(\mathrm{BU}\to\mathrm{BGL}_1(\mathbb S)\to\mathrm{Spectra}\right).\]
\end{definition}
\begin{theorem}[Thom]
	The homotopy groups $\pi_*\mathrm{MU}$ are related to bordism classes of stably almost complex (real) manifolds.
\end{theorem}
\begin{remark}
	Note that $\Sigma^\infty S^{n+m}=\Sigma^\infty S^n\otimes\Sigma^\infty S^m$ by pulling out $\Sigma^\infty$ from the tensor products and moving suspensions around. It follows that $\mathrm{BU}\to\mathrm{BGL}_1(\mathbb S)$ is a morphism of $\mathbb E_\infty$-spaces, where $\mathrm{BU}$ has its extra $\mathbb E_\infty$-structure given by taking sums. Thus, $\mathrm{MU}$ is an $\mathbb E_\infty$-ring by passing this $\mathbb E_\infty$-structure to the colimit.
\end{remark}
We would like to understand $\mathrm{MU}$, so let's start by trying to understand $\mathrm H_*(\mathrm{MU};\ZZ)=\pi_*(\ZZ\otimes_{\mathbb S}\mathrm{MU})$. The reason why this may be tractable is that there is a commuting diagram
% https://q.uiver.app/#q=WzAsNCxbMCwwLCJcXG9we0JHTH1fMShcXG1hdGhiYiBTKSJdLFsxLDAsIlxcb3B7QkdMfV8xKFxcWlopIl0sWzEsMSwiXFxtYXRocm17TW9kfShcXFpaKSJdLFswLDEsIlxcbWF0aHJte01vZH0oXFxtYXRoYmIgUykiXSxbMCwxLCItXFxvdGltZXNfe1xcbWF0aGJiIFN9XFxaWiJdLFszLDIsIi1cXG90aW1lc197XFxtYXRoYmIgU31cXFpaIl0sWzAsMywiIiwxLHsic3R5bGUiOnsidGFpbCI6eyJuYW1lIjoiaG9vayIsInNpZGUiOiJ0b3AifX19XSxbMSwyLCIiLDEseyJzdHlsZSI6eyJ0YWlsIjp7Im5hbWUiOiJob29rIiwic2lkZSI6InRvcCJ9fX1dXQ==&macro_url=https%3A%2F%2Fraw.githubusercontent.com%2FdFoiler%2Fnotes%2Fmaster%2Fnir.tex
\[\begin{tikzcd}[cramped]
	{\op{BGL}_1(\mathbb S)} & {\op{BGL}_1(\ZZ)} \\
	{\mathrm{Mod}(\mathbb S)} & {\mathrm{Mod}(\ZZ)}
	\arrow["{-\otimes_{\mathbb S}\ZZ}", from=1-1, to=1-2]
	\arrow[hook, from=1-1, to=2-1]
	\arrow[hook, from=1-2, to=2-2]
	\arrow["{-\otimes_{\mathbb S}\ZZ}", from=2-1, to=2-2]
\end{tikzcd}\]
so we can try to instead compute a colimit along
\[\mathrm{BU}\to\mathrm{BGL}_1(\mathbb S)\to\mathrm{BGL}_1(\ZZ)\to\mathrm{Mod}(\ZZ).\]
Well, this map $\mathrm{BU}\to\mathrm{BGL}_1(\ZZ)=B(\ZZ/2\ZZ)=K(\ZZ/2\ZZ,1)$ amounts to some class in $\mathrm H^1(\mathrm{BU};\FF_2)$. But because $\mathrm H^*(\mathrm{BU};\ZZ)=\ZZ[c_1,c_2,\ldots]$ where $\deg c_i=2i$, we find that $\mathrm H^1(\mathrm{BU};\FF_2)=0$. Thus, the map $\mathrm{BU}\to\mathrm{BGL}_1(\ZZ)$ is nullhomotopic! As such, we may compute our homology as
\[\pi_*(\ZZ\otimes_{\mathbb S}\mathrm{MU})=\mathrm H_*(\mathrm{BU};\ZZ)=\mathrm H^*(\mathrm{BU};\ZZ)^\lor.\]
Indeed, the first equality follows because we are basically computing the homotopy groups of the composite $\mathrm{BU}\to*\to\mathrm{Mod}(\ZZ)$, which is the homology groups of $\mathrm{BU}$ by the Dold--Kan correspondence. Thus, we have proven the following result.
\begin{proposition}
	There is an isomorphism
	\[\mathrm H_*(\mathrm{MU};\ZZ)=\pi_*(\ZZ\otimes_{\mathbb S}\mathrm{MU})\cong\mathrm H_*(\mathrm{BU};\ZZ).\]
\end{proposition}
\begin{remark}
	The summation structure on vector spaces makes $\mathrm H_*(\mathrm{BU};\ZZ)$ into a ring, and one can check that the isomorphism
	\[\pi_*(\ZZ\otimes_{\mathbb S}\mathrm{MU})\to\mathrm H_*(\mathrm{BU};\ZZ)\]
	given above is in fact an isomorphism of rings.
\end{remark}
\begin{remark}
	Recall the homology calculation
	\[\mathrm H_*(\mathrm{BU};\ZZ)=\ZZ[b_1,b_2,\ldots],\]
	where the right-hand side is a polynomial ring as described in the previous remark. On the other hand, there is an inclusion $\CP^\infty=\mathrm{BU}(1)\to\mathrm{BU}$, and it turns out that these generators $b_i$ are simply given by the image of generators from $\mathrm H_*(\CP^\infty;\ZZ)\to\mathrm H_*(\mathrm{BU};\ZZ)$.
\end{remark}

\subsection{Using the Adams Spectral Sequence}
We now know that
\[\mathrm H_*(\mathrm{MU};\ZZ)=\pi_*(\ZZ\otimes_{\mathbb S}\mathrm{MU})\cong\ZZ[b_1,b_2,\ldots],\]
so
\[\mathrm H_*(\mathrm{MU};\FF_2)\cong\FF_2[b_1,b_2,\ldots].\]
For example, we could now try to understand the Steenrod squares acting on this homology ring. Because we are now working with homology, this really means that we want to understand the image of the $b_\bullet$s along the ``coaction'' map
\[\mathrm H_*(\mathrm{MU};\FF_2)\to\mathrm H_*(\mathrm{MU};\FF_2)\otimes_{\FF_2}\mc A_*.\]
Here, $\mc A_*$ is the dual Steenrod algebra.
\begin{remark}
	The coaction map for $\mathrm{MU}$ is not the same as the coaction map on $\mathrm H_*(\mathrm{BU};\FF_2)$.
\end{remark}
To understand this coaction map on the $b_\bullet$s, we see that we are allowed to restrict to $\CP^\infty$. Let's explain this. Consider the composite functor $F$ given by
\[\CP^\infty\to\mathrm{BU}\to\mathrm{BGL}_1(\mathbb S)\to\mathrm{Spectra}.\]
As such, there is a map $\colim F\to\mathrm{MU}$, which fits into the diagram
% https://q.uiver.app/#q=WzAsNCxbMCwwLCJcXG1hdGhybSBIXyooXFxjb2xpbSBGO1xcRkZfMikiXSxbMSwwLCJcXG1hdGhybSBIXyooXFxtYXRocm17TVV9O1xcRkZfMikiXSxbMCwxLCJcXG1hdGhybSBIXyooXFxDUF5cXGluZnR5O1xcRkZfMikiXSxbMSwxLCJcXG1hdGhybSBIXiooXFxtYXRocm17QlV9O1xcRkZfMikiXSxbMCwxLCIiLDAseyJzdHlsZSI6eyJ0YWlsIjp7Im5hbWUiOiJob29rIiwic2lkZSI6InRvcCJ9fX1dLFswLDJdLFsxLDNdLFsyLDMsIiIsMix7InN0eWxlIjp7InRhaWwiOnsibmFtZSI6Imhvb2siLCJzaWRlIjoidG9wIn19fV1d&macro_url=https%3A%2F%2Fraw.githubusercontent.com%2FdFoiler%2Fnotes%2Fmaster%2Fnir.tex
\[\begin{tikzcd}[cramped]
	{\mathrm H_*(\colim F;\FF_2)} & {\mathrm H_*(\mathrm{MU};\FF_2)} \\
	{\mathrm H_*(\CP^\infty;\FF_2)} & {\mathrm H^*(\mathrm{BU};\FF_2)}
	\arrow[hook, from=1-1, to=1-2]
	\arrow[from=1-1, to=2-1]
	\arrow[from=1-2, to=2-2]
	\arrow[hook, from=2-1, to=2-2]
\end{tikzcd}\]
where the vertical maps are ring isomorphisms but do not respect the coaction. However, the top horizontal map does respect to the coaction, so we are nonetheless motivated to understand the coaction on $\mathrm H_*(\colim F;\FF_2)$. To do this, it turns out that
\[\colim_{\CP^\infty} F=\Sigma^{-2}\Sigma^\infty\CP^\infty.\]
This is not quick to prove and requires some geometric input; roughly speaking, this requires understanding the tautological line bundle $\mc L\colon\CP^\infty\to\mathrm{BU}$, which one shows is related back to $\CP^\infty$. (At ``finite level,'' this is related to the fact that the tautological line bundle on $\CP^n$ is related to $\CP^{n+1}$.)
\begin{remark}
	A straightforward calculation provides an isomorphism
	\[\mathrm H_*\left(\Sigma^{-2}\Sigma^\infty\CP^\infty;\FF_2\right)\cong\mathrm H_*(\Sigma^\infty_+\CP^\infty;\FF_2)\]
	of vector spaces. The point is that the right-hand side is ``pointed'' while the left-hand side shifts by $2$, so one can compute the homology on both sides is still $\FF_2$ for the positive even degrees.
\end{remark}
Thus, we can compute the coaction on $\mathrm H_*(\Sigma^{-2}\Sigma^\infty\CP^\infty;\FF_2)$ by shifting the coaction on $\mathrm H_*(\Sigma^\infty\CP^\infty;\FF_2)$ and then shifting by $2$. The final answer is a bit complicated, so we will avoid actually writing it down.

We may now use the Adams spectral sequence! Define $\op{fil}^n\mathrm{MU}^\land_2$ (over $\ZZ^\land_2$) as before, and we then receive a spectral sequence
\[\op{Ext}_{\mc A}(\mathrm H^*(\mathrm{MU};\FF_2),\FF_2)\Rightarrow\pi_*\mathrm{MU}^\land_2.\]
Milnor computed this spectral sequence to have starting page concentrated in those degrees with no differentials. Milnor also did a similar process for each prime $p$. This produces the following theorem.
\begin{theorem}[Milnor]
	We consider the homotopy groups of $\pi_*\mathrm{MU}$.
	\begin{listalph}
		\item The homotopy groups of $\pi_*\mathrm{MU}$ are concentrated in even degrees.
		\item The induced ring map
		\[\pi_*(\mathrm{MU})\to\pi_*(\ZZ\otimes_{\mathbb S}\mathrm{MU})\cong\ZZ[b_1,b_2,\ldots]\]
		is injective.
		\item The image of the ring map in (b) includes each $b_i$ for which $i+1$ is not a prime-power. If $i+1$ is the prime-power $p^k$, then the image includes $pb_i$ plus some lower-order terms.
	\end{listalph}
\end{theorem}
\begin{example}
	It turns out that $b_1$ is not in the image of $\pi_*(\mathrm{MU})$.
\end{example}
\begin{remark}
	The fact that $\QQ\otimes_{\mathbb S}\mathrm{MU}=\QQ\otimes_{\mathbb Z}\mathbb Z\otimes_{\mathbb S}\mathrm{MU}$ implies that
	\[\pi_*(\mathrm{MU})_\QQ\to\pi_*(\ZZ\otimes_{\mathbb S}\mathrm{MU})_\QQ\cong\QQ[b_1,b_2,\ldots]\]
	is an isomorphism. Thus, the content in Milnor's theorem is really the ``torsion'' of when $b_i$ is actually in the image ``on the nose.''
\end{remark}

\subsection{Using the Formal Group Law}
To go further, we use our formal groups. Because $\pi_*\mathrm{MU}$, the generalized cohomology $\mathrm{MU}^*(\CP^\infty)$ is a formal group. In fact, there is a canonical choice of formal group law $f_{\mathrm{MU}}\in\pi_*\mathrm{MU}[[x,y]]$. Let's explain this. It amounts to find a canonical isomorphism
\[\mathrm{MU}^*(\CP^\infty)\cong\pi_*(\mathrm{MU})[[t]].\]
Equivalently, we need a canonical generator of $t_{\mathrm{MU}}\in\mathrm{MU}^2(\CP^\infty)$, or equivalently a canonical complex orientation. Well, recall that we already had an inclusion $\CP^\infty=\op{BU}(1)\into\mathrm{BU}$, which then induces a map $\Sigma^{-2}\Sigma^\infty\CP^\infty\to\mathrm{MU}$. This is exactly the data of a complex orientation $t_{\mathrm{MU}}\colon\Sigma^\infty\CP^\infty\to\Sigma^2\mathrm{MU}$ we needed to produce! Thus, we obtain a canonical formal group law.

This formal group law turns out to be quite special.
\begin{definition}[Lazard ring]
	We define $L$ to be the free commutative ring $\ZZ[a_{ij}]$ modulo exactly the relations required for
	\[f(x,y)=\sum_{i,j}a_{ij}x^iy^j\]
	to be a commutative formal group law.
\end{definition}
\begin{example}
	The requirements of a formal group imply that $a_{i0}=0$ and $a_{0j}=0$ for $i>0$ and $j>0$.
\end{example}
\begin{remark}
	Because $L$ is universal, we obtain a map $L\to\pi_*\mathrm{MU}$ which maps $a_{ij}$ to the coefficients of the formal group law $f_{\mathrm{MU}}$.
\end{remark}
We are now ready to state our main result.
\begin{theorem}[Quillen] \label{thm:quillen}
	Consider the composite map
	\[L\to\pi_*\mathrm{MU}\to\pi_*(\ZZ\otimes_{\mathbb S}\mathrm{MU})\cong\ZZ[b_1,b_2,\ldots].\]
	Then this map is injective. Its image contains all the $b_i$s for which $i+1$ is not a prime-power. If $i+1$ is a power of the prime $p$, the image contains some explicit terms of the form $pb_i$ plus some lower-order terms.
\end{theorem}
Being explicit about the ``lower-order'' terms (which we have not done in the statement of the theorem but was known to Quillen), we receive the following corollary.
\begin{corollary}
	The universal map $L\to\pi_*\mathrm{MU}$ is an isomorphism of rings, and $f_{\mathrm{MU}}$ corresponds to the universal formal group law on $L$.
\end{corollary}
\begin{proof}[Sketch of \Cref{thm:quillen}]
	Our previous calculations showed that $\ZZ\otimes_{\mathbb S}\mathrm{MU}$ is an even $\mathbb E_\infty$-ring, and in fact, there is a canonical complex orientation
	\[\Sigma^{-2}\Sigma^\infty\CP^\infty\to\mathrm{MU}=\mathbb S\otimes_{\mathbb S}\mathrm{MU}\to\ZZ\otimes_{\mathbb S}\mathrm{MU},\]
	which we will abusively also denote $t_{\mathrm{MU}}$. Thus, we receive a canonical isomorphism
	\[\left(\ZZ\otimes_{\mathbb S}\mathrm{MU}\right)^*(\CP^\infty)\cong\ZZ[b_1,b_2,\ldots][[t_{\mathrm{MU}}]].\]
	However, there is another canonical complex orientation given by the composite
	\[\Sigma^{-2}\Sigma^\infty\CP^\infty\to\ZZ=\ZZ\otimes_{\mathbb S}\mathbb S\to\ZZ\otimes_{\mathbb S}\mathrm{MU},\]
	which we will denote $t_\ZZ$. So we now have canonical isomorphisms
	\[\ZZ[b_1,b_2,\ldots][[t_{\ZZ}]]=\left(\ZZ\otimes_{\mathbb S}\mathrm{MU}\right)^*(\CP^\infty)\cong\ZZ[b_1,b_2,\ldots][[t_{\mathrm{MU}}]].\]
	Let's compare these generators, which will be the geometric input to our theorem.
	\begin{lemma}
		Fix everything as above. Then
		\[t_{\mathrm{MU}}=t_{\ZZ}+b_1t_{\ZZ}^2+b_2t_\ZZ^3+\cdots.\]
	\end{lemma}
	\begin{proof}
		Observe that $t_{\mathrm{MU}}$ induces a map
		\[\ZZ\otimes_{\mathbb S}\Sigma^{-2}\Sigma^\infty\CP^\infty\to\ZZ\otimes_{\mathbb S}\mathrm{MU}\]
		of $\ZZ$-modules, which then fits into a diagram
		% https://q.uiver.app/#q=WzAsNCxbMCwwLCJcXFpaXFxvdGltZXNfe1xcbWF0aGJiIFN9XFxTaWdtYV57LTJ9XFxTaWdtYV5cXGluZnR5XFxDUF5cXGluZnR5Il0sWzAsMSwiXFxaWlxcb3RpbWVzX3tcXG1hdGhiYiBTfVxcU2lnbWFfK15cXGluZnR5XFxDUF5cXGluZnR5Il0sWzEsMSwiXFxaWlxcb3RpbWVzX3tcXG1hdGhiYiBTfVxcU2lnbWFfK15cXGluZnR5XFxtYXRocm17QlV9Il0sWzEsMCwiXFxaWlxcb3RpbWVzX3tcXG1hdGhiYiBTfVxcbWF0aHJte01VfSJdLFswLDNdLFswLDFdLFsxLDJdLFszLDJdXQ==&macro_url=https%3A%2F%2Fraw.githubusercontent.com%2FdFoiler%2Fnotes%2Fmaster%2Fnir.tex
		\[\begin{tikzcd}[cramped]
			{\ZZ\otimes_{\mathbb S}\Sigma^{-2}\Sigma^\infty\CP^\infty} & {\ZZ\otimes_{\mathbb S}\mathrm{MU}} \\
			{\ZZ\otimes_{\mathbb S}\Sigma_+^\infty\CP^\infty} & {\ZZ\otimes_{\mathbb S}\Sigma_+^\infty\mathrm{BU}}
			\arrow[from=1-1, to=1-2]
			\arrow[from=1-1, to=2-1]
			\arrow[from=1-2, to=2-2]
			\arrow[from=2-1, to=2-2]
		\end{tikzcd}\]
		where the vertical maps are the isomorphisms discussed before. However, the bottom map can be explicitly computed: the left-hand side is some graded vector space supported with generators $\beta_i$ in even degrees $2i$, and $\ZZ\otimes\Sigma_+^\infty\mathrm{BU}=\ZZ[b_1,b_2,\ldots]$. Computing this bottom map and then using the coherence of the $\ZZ$-module structure completes the proof.
	\end{proof}
	We are now ready to start computing our formal group law, which is some algebra. Recall that this is induced by the map
	\[(\mathrm{MU}\otimes\ZZ)^*(\CP^\infty)\to(\mathrm{MU}\otimes\ZZ)^*(\CP^\infty)^2=\ZZ[b_1,b_2,\ldots][[x_{\mathrm{MU}},y_{\mathrm{MU}}]],\]
	and we want to know the image of $t_{\mathrm{MU}}$. Let this image be the power series $\sum_{ij}a_{ij}x_{\mathrm{MU}}^iy_{\mathrm{MU}}^j$.

	On the other hand, $t_{\ZZ}$ will just go to $x_\ZZ+b_\ZZ$ because this is the formal group law for $\CP^\infty$. Thus, $t_{\mathrm{MU}}$ has image
	\[(x_\ZZ+y_\ZZ)+b_1(x_\ZZ+y_\ZZ)^2+b_2(x_\ZZ+y_\ZZ)^3+\cdots.\]
	Expanding this power series out completes the proof upon comparing with Milnor's work! For example, one can compute that $a_{11}=2b_1$.
\end{proof}
% We will spend the rest of class saying something about \Cref{thm:quillen}. 

\end{document}