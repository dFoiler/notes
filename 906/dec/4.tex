% !TEX root = ../notes.tex

\documentclass[../notes.tex]{subfiles}

\begin{document}

\section{December 4}
We're going to say a few things.

\subsection{Formal Group Laws}
Let's try to build a moduli space of formal groups. Let $L$ be the Lazard ring.
\begin{remark}
	Formal group laws over $R$ are in bijection with maps $L\to R$. Indeed, send a map $c\colon L\to R$ to the formal group law
	\[f_c(x,y)\coloneqq\sum_{i,j}c(a_{ij})x^iy^j.\]
\end{remark}
\begin{remark}
	For any power series $g\in R[[t]]$ with $g(t)\equiv t\pmod{t^2}$, one can show that $g$ admits an inverse $g^{-1}$ satisfying $g\circ g^{-1}=t$. This is basically shown by induction.
\end{remark}
\begin{definition}[strict isomorphism]
	Fix a discrete ring $R$. For any $g\in R[[t]]$ with $g(t)\equiv t\pmod{t^2}$, one can take a formal group law $f$ and produce another formal group law
	\[g^{-1}\circ f\circ (g,g).\]
	We say that two formal group laws $f_1$ and $f_2$ are \textit{strictly isomorphic} if there is $g$ relating $f_1$ to $f_2$ as above.
\end{definition}
\begin{remark}
	Roughly speaking, strict isomorphism corresponds to changing the coordinate of $\widehat{\AA}^1$.
\end{remark}
\begin{definition}
	We define the moduli stack $\mc M_{\mathrm{fg}}^{\mathrm{strict}}$ to be the functor which takes discrete rings $R$ to formal group laws over $R$ up to strict isomorphism.
\end{definition}
\begin{remark}
	For any scalar $\lambda\in R^\times$, one can take a formal group law $f$ and define a new formal group law $\lambda f\left(\lambda^{-1}x,\lambda^{-1}y\right)$. This descends to an action of $\mathbb G_m$ on $\mc M_{\mathrm{fg}}^{\mathrm{strict}}$.
\end{remark}
\begin{definition}
	We define the moduli stack $\mc M_{\mathrm{fg}}$ to be the quotient $\mc M_{\mathrm{fg}}^{\mathrm{strict}}/\mathbb G_m$.
\end{definition}
\begin{remark}
	It turns out that all isomorphisms are now included, so one should think of $\mc M_{\mathrm{fg}}$ as the actual moduli space of formal groups.
\end{remark}
\begin{example}
	Fix an algebraically closed field $k$.
	\begin{itemize}
		\item If $\op{char}k=0$, then all formal group laws over $k$ are isomorphic to $f(x,y)=x+y$.
		\item If $\op{char}k=p$ is positive, then every formal group law over $k$ admits a height $h\in[1,\infty)$, and this height characterizes the formal group law up to isomorphism. To compute the height, one studies the power series
		\[\underbrace{f(f(f(\cdots f}_f(x,x),\cdots ,x),x),x)=px+\sum_{n\ge1}\ell_nx^n.\]
		Then the height is at least $h$ if and only if the coefficient of $x^{p^h}$ is not a unit. This height is an isomorphism invariant.
	\end{itemize}
\end{example}
\begin{remark}
	Let's geometrize this diagram.
	\begin{itemize}
		\item We have a very non-closed point over $\QQ$, whose geometric points are all isomorphic.
		\item For each prime $p$, there is a chain of specializations
		\[\QQ\to(h=1)\to(h=2)\to\cdots\to\FF_p,\]
		where $(h=i)$ refers to the formal group law of height $i$.
	\end{itemize}
\end{remark}

\subsection{The Adams--Novikov Spectral Sequence}
Let's now compute some homotopy groups of spheres. To start, let's try to compute $\mathrm{MU}_*\mathrm{MU}$. Using the Postnikov filtration, one finds
\[\pi_*(\mathrm{MU})[b_1,b_2,\ldots],\]
where these generators basically come from our homology calculation $\ZZ_*\mathrm{MU}=\ZZ[b_1,b_2,\ldots]$.

Now, to compute $\pi_*\mathbb S$, we note that $\mathbb S=\mathbb S_{\mathrm{MU}}$ because the augmentation ideal is concentrated in positive degrees, so the \v{C}ech nerve will actually output $\mathbb S$. Thus, we can use descent with
\[\mathbb S_{\mathrm{MU}}=\lim\mathrm{MU}^{\otimes(\bullet+1)},\]
so we find that we will be interested is some spectral sequence which starts with the homology of the chain complex% https://q.uiver.app/#q=WzAsNCxbMCwwLCJcXHBpXypcXG1hdGhybXtNVX0iXSxbMSwwLCJcXHBpXyooXFxtYXRocm17TVV9XFxvdGltZXNfe1xcbWF0aGJiIFN9XFxtYXRocm17TVV9KSJdLFsyLDAsIlxccGlfKihcXG1hdGhybXtNVX1cXG90aW1lc197XFxtYXRoYmIgU31cXG1hdGhybXtNVX1cXG90aW1lc197XFxtYXRoYmIgU31cXG1hdGhybXtNVX0pIl0sWzMsMCwiXFxjZG90cyJdLFswLDEsIiIsMCx7Im9mZnNldCI6LTF9XSxbMCwxLCIiLDIseyJvZmZzZXQiOjF9XSxbMSwyXSxbMSwyLCIiLDIseyJvZmZzZXQiOi0yfV0sWzEsMiwiIiwyLHsib2Zmc2V0IjoyfV1d&macro_url=https%3A%2F%2Fraw.githubusercontent.com%2FdFoiler%2Fnotes%2Fmaster%2Fnir.tex
\[\begin{tikzcd}[cramped]
	{\pi_*\mathrm{MU}} & {\pi_*(\mathrm{MU}\otimes_{\mathbb S}\mathrm{MU})} & {\pi_*(\mathrm{MU}\otimes_{\mathbb S}\mathrm{MU}\otimes_{\mathbb S}\mathrm{MU})} & {\cdots.}
	\arrow[shift left, from=1-1, to=1-2]
	\arrow[shift right, from=1-1, to=1-2]
	\arrow[from=1-2, to=1-3]
	\arrow[shift left=2, from=1-2, to=1-3]
	\arrow[shift right=2, from=1-2, to=1-3]
\end{tikzcd}\]
For example, we may want to understand maps out of this complex.
\begin{itemize}
	\item A map $\pi_*\mathrm{MU}\to R$ is a formal group law over $R$.
	\item A map $\pi_*(\mathrm{MU}\otimes_{\mathbb S}\mathrm{MU})\to R$ corresponds to a pair of formal group laws (one from each $\pi_*\mathrm{MU}$) equipped with a strict isomorphism between them (which we see from the homology calculation of $\pi_*(\mathrm{MU}\otimes_{\mathbb S}\mathrm{MU})=\pi_*\mathrm{MU}[b_1,b_2,\ldots]$).
	\item The same sort of argument shows that a map $\pi_*(\mathrm{MU}\otimes_{\mathbb S}\mathrm{MU}\otimes_{\mathbb S}\mathrm{MU})\to R$ is a diagram
	\[f_1\stackrel{g_1}\to f_2\stackrel{g_2}f_3.\]
\end{itemize}
One can continue this process, and it shows that the starting page for our descent spectral sequence is
\[\mathrm H^*(\mc M_{\mathrm{fg}}^{\mathrm{strict}};\OO).\]
This looks like it only has a single grading (from the cohomology), but the second grading comes from the $\mathbb G_m$-action on $\mc M_{\mathrm{fg}}^{\mathrm{strict}}$. We will not spell out exactly what the bigrading is.
\begin{proposition}[Adams--Novikov spectral sequence]
	There is a spectral sequence
	\[\mathrm H^*(\mc M_{\mathrm{fg}}^{\mathrm{strict}};\OO)\Rightarrow\pi_*\mathbb S.\]
\end{proposition}
\begin{remark}
	This spectral sequence also arises from motivic homotopy theory.
\end{remark}
This is not the usual spectral sequence one uses: it turns out to be easier to localize.
\begin{remark}
	For any spectrum $X$, there is a coaction
	\[\mathrm{MU}_*X\to\mathrm{MU}_*X\otimes_{\pi_*\mathrm{MU}}\mathrm{MU}_*\mathrm{MU}\]
	which is akin to the Steenrod operations. (Indeed, such a thing should exist for any homology theory.) This turns out to give $\mathrm{MU}_*X$ the structure of a quasicoherent sheaf on $\mc M_{\mathrm{fg}}^{\mathrm{strict}}$ (namely, we are defining some module over $L$ equipped with some descent), which descends to a quasicoherent sheaf on $\mc M_{\mathrm{fg}}$.
\end{remark}
Thus, we have a functor
\[\mathrm{Spectra}\to\mathrm{QCoh}(\mc M_{\mathrm{fg}})\]
given by $X\mapsto\op{MU}_*X$, and we can imagine doing sheaf-y things. For example,  $X\mapsto X\otimes_{\mathbb S}\QQ$ to focus on the $\QQ$-point of $\mc M_{\mathrm{fg}}$. We could also imagine focusing on the $\FF_p$-points, but there is actually a full chain of points for each height $h$. Thus, for a fixed prime $p$ and height $h$, we hope to be able to localize at this formal group of height $h$. Descending the quasicoherent sheaves, we would hope to be able to define a localization functor
\[L_{K(h)}\colon\mathrm{Spectra}\to\mathrm{Spectra}.\]
(It is customary to avoid mentioning the prime $p$.) For example,
\[L_{K(1)}X\coloneqq\left(X\otimes\lim\mathrm{KU}^{\otimes(\bullet+1)}\right)^\land_p.\]
One may now hope that $\pi_*L_{K(n)}\mathbb S$ is easier to compute.
\begin{proposition}[Adams--Novikov spectral sequence]
	Fix a prime $p$ and height $n$. There is a spectral sequence
	\[\mathrm H^*(\mc M_{\mathrm{fg}}^{\mathrm{strict}};\OO_{K(n)})\Rightarrow\pi_*L_{K(n)}\mathbb S.\]
\end{proposition}
One has used this spectral sequence to understand $\pi_*L_{K(1)}\mathbb S$ and $\pi_*L_{K(2)}\mathbb S$, but otherwise, only partial information is known.
\begin{theorem}[Hopkins--Ravenel]
	The Adams--Novikov spectral sequence has finitely many pages, and the $E_\infty$-page is concentrated in an explicit ``band'' of degrees.
\end{theorem}
\begin{remark}
	Studying homotopy groups by focusing on the localizations $L_{K(n)}$ is the subject of chromatic homotopy theory.
\end{remark}

\subsection{A Little Derived Algebraic Geometry}
We end the course with another approach to the Adams--Novikov spectral sequence.
\begin{definition}
	Given an $\mathbb E_\infty$-ring $R$, we define the functor $\op{Spev}$ to take $\mathbb E_\infty$-rings $A$ to
	\[\op{Spev}R(A)\coloneqq\op{Hom}_{\mathbb E_\infty}(R,A).\]
\end{definition}
Though $\op{Spev}$ is some kind of bizarre homotopy-coherent object, it has an underlying classical stack by taking a Kan extension along the following diagram.
% https://q.uiver.app/#q=WzAsMyxbMCwwLCJcXHtcXHRleHR7ZXZlbiB9XFxtYXRoYmIgRV9cXGluZnR5XFx0ZXh0ey1yaW5nc31cXH0iXSxbMCwxLCJcXHtcXHRleHR7ZGlzY3JldGUgcmluZ3N9XFx9Il0sWzEsMCwiXFxtYXRocm17U3BhY2VzfSJdLFswLDEsIlxccGlfezJrKn0iLDJdLFsxLDIsIiIsMCx7InN0eWxlIjp7ImJvZHkiOnsibmFtZSI6ImRhc2hlZCJ9fX1dLFswLDIsIlxcb3B7U3Bldn0iXV0=&macro_url=https%3A%2F%2Fraw.githubusercontent.com%2FdFoiler%2Fnotes%2Fmaster%2Fnir.tex
\[\begin{tikzcd}[cramped]
	{\{\text{even }\mathbb E_\infty\text{-rings}\}} & {\mathrm{Spaces}} \\
	{\{\text{discrete rings}\}}
	\arrow["{\op{Spev}}", from=1-1, to=1-2]
	\arrow["{\pi_{2k*}}"', from=1-1, to=2-1]
	\arrow[dashed, from=2-1, to=1-2]
\end{tikzcd}\]
Note that the left functor has a grading, so the Kan extension has an implicit grading.
\begin{theorem}
	The functor $F$ produced by the Kan extension
	% https://q.uiver.app/#q=WzAsMyxbMCwwLCJcXHtcXHRleHR7ZXZlbiB9XFxtYXRoYmIgRV9cXGluZnR5XFx0ZXh0ey1yaW5nc31cXH0iXSxbMCwxLCJcXHtcXHRleHR7Y29tbS4gZGlzY3JldGUgcmluZ3N9XFx9Il0sWzEsMCwiXFxtYXRocm17U3BhY2VzfSJdLFswLDEsIlxccGlfezJrKn0iLDJdLFsxLDIsIkYiLDIseyJzdHlsZSI6eyJib2R5Ijp7Im5hbWUiOiJkYXNoZWQifX19XSxbMCwyLCJcXG9we1NwZXZ9Il1d&macro_url=https%3A%2F%2Fraw.githubusercontent.com%2FdFoiler%2Fnotes%2Fmaster%2Fnir.tex
	\[\begin{tikzcd}[cramped]
		{\{\text{even }\mathbb E_\infty\text{-rings}\}} & {\mathrm{Spaces}} \\
		{\{\text{comm. discrete rings}\}}
		\arrow["{\op{Spev}}", from=1-1, to=1-2]
		\arrow["{\pi_{2k*}}"', from=1-1, to=2-1]
		\arrow["F"', dashed, from=2-1, to=1-2]
	\end{tikzcd}\]
	is exactly $\mc M_{\mathrm{fg}}^{\mathrm{strict}}$.
\end{theorem}
Here is an application, which does not mention Kan extensions.
\begin{theorem}
	Define
	\[\op{fil}^n\mathbb S\coloneqq\lim_{\substack{\mathbb S\to E\\\text{even }\mathbb E_\infty\text{-ring }E}}\tau_{\ge2n}E.\]
	Then this filtration of $\mathbb S$ gives rise to the Adams--Novikov spectral sequence.
\end{theorem}
More generally, the same argument shows the following.
\begin{theorem}[motivic spectral sequence]
	Fix an $\mathbb E_\infty$-ring $R$, and define
	\[\op{fil}^nR\coloneqq\lim_{\substack{R\to E\\\text{even }\mathbb E_\infty\text{-ring }E}}\tau_{\ge2n}R.\]
	This gives rise to the ``motivic'' spectral sequence
	\[\lim_{\substack{R\to E\\\text{even }\mathbb E_\infty\text{-ring }E}}\pi_{2*}E\Rightarrow\pi_*R.\]
\end{theorem}
We can use this spectral sequence to define de Rham cohomology.
\begin{definition}[Hochschild homology]
	Fix a smooth discrete commutative ring $R$. Then consider the diagram
	\[\mathrm BS^1=\CP^\infty\to\Delta^0\to\op{Alg}_\ZZ(\mathbb E_\infty).\]
	Then the \textit{Hochschild homology} $\mathrm{HP}(R)$ is the limit over $\CP^\infty$ (from the $S^1$-action) of the colimit of this diagram.
\end{definition}
\begin{theorem}
	The homotopy group $\pi_0\mathrm{HP}(R)$ is the de Rham cohomology of $R$, whose grading is recovered from the motivic spectral sequence.
\end{theorem}
\begin{remark}
	Information from the higher homotopy groups recovers the Hodge filtration.
\end{remark}
\begin{remark}
	By replacing $\op{Alg}_\ZZ(\mathbb E_\infty)$ with $\op{Alg}_{\mathbb S}(\mathbb E_\infty)$ in the above discussion, we recover prismatic cohomology. In fact, this was the original definition due to Scholze.
\end{remark}

\end{document}