% !TEX root = ../notes.tex

















Here we go again.

\subsection{Integrability Conditions}
We're still studying integrability. Last time we defined integrability and gave some easier conditions for being integrable without actually computing Darboux integrals, and we will continue with that story today.
\begin{proposition}
	Fix $f:[a,b]\to\RR$ is a monotonic function. Then $f$ is integrable on $[a,b].$
\end{proposition}
Before we go into proving this, we should point out that monotonic functions are pretty nice. Granted, they aren't terrible---a monotonic function will have only many countably many discontinuities.\todo{} Though ``countably many'' is a bit weak of an assertion, for any given countable set has a corresponding monotonic function with discontinuities on that set.
\begin{proof}
	We are going to need to somewhat start from scratch. Take $f$ increasing so that our Darboux sums are, for a given partition $P=\{t_k\}_{k=0}^n\subseteq[a,b],$
	\[U(f,P)=\sum_{k=1}^nM(f,[t_{k-1},t_k])(t_k-t_{k-1})=\sum_{k=1}^nf(t_k)(t_k-t_{k-1}),\]
	and
	\[L(f,P)=\sum_{k=1}^nm(f,[t_{k-1},t_k])(t_k-t_{k-1})=\sum_{k=1}^nf(t_{k-1})(t_k-t_{k-1}).\]
	Now we notice that
	\[U(f,P)-L(f,P)=\sum_{k=1}^n\big(f(t_k)-f(t_{k-1})\big)(t_k-t_{k-1}).\]
	Fix $\varepsilon>0,$ and set $\delta:=\frac{\varepsilon}{f(b)-f(a)}$ so that, if the mesh of $P$ is less than $\delta,$ we have
	\[U(f,P)-L(f,P)<\sum_{k=1}^n\big(f(t_k)-f(t_{k-1})\big)\cdot\frac{\varepsilon}{f(b)-f(a)}=\varepsilon,\]
	where we have telescoped to evaluate the sum. This shows that $f$ is integrable.
\end{proof}
\begin{prop}
	Fix $f:[a,b]\to\RR$ is a continuous function. Then $f$ is integrable on $[a,b].$
\end{prop}
\begin{proof}
	Take $f$ continuous so that our Darboux sums are, for a given partition $P=\{t_k\}_{k=0}^n\subseteq[a,b],$
	\[U(f,P)=\sum_{k=1}^nM(f,[t_{k-1},t_k])(t_k-t_{k-1}),\]
	and
	\[L(f,P)=\sum_{k=1}^nm(f,[t_{k-1},t_k])(t_k-t_{k-1}).\]
	Now we notice that
	\[U(f,P)-L(f,P)=\sum_{k=1}^n\big(M(f,[t_{k-1},t_k])-m(f,[t_{k-1},t_k])\big)(t_k-t_{k-1}).\]
	Fix $\varepsilon>0,$ and we would like small mesh to make the above difference less than $\varepsilon.$ Now, $f$ is continuous on $[a,b],$ so it is uniformly continuous (this is the key trick!), so we may find $\delta>0$ such that
	\[|x_1-x_2|<\delta\implies|f(x_1)-f(x_2)|<\frac{\varepsilon}{b-a}.\]
	In particular, if the mesh of $P$ is less than our $\delta,$ we find
	\[M(f,[t_{k-1},t_k])-m(f,[t_{k-1},t_k]<\frac{\varepsilon}{b-a}\]
	because $|t_k-t_{k-1}|<\delta,$ where we are using the fact that $f$ achieves its maximum and minimum at some $x_1$ and $x_2,$ so $|x_1-x_2|<\delta$ gives the result. Thus,
	\[U(f,P)-L(f,P)<\sum_{k=1}^n\frac{\varepsilon}{2(b-a)}\cdot(t_k-t_{k-1})=\varepsilon,\]
	where we have telescoped to evaluate the sum. This shows that $f$ is integrable.
\end{proof}
\begin{remark}
	Intuitively, thinking of continuous as ``locally monotone'' and building up continuity from the monotone result is probably fine. Of course, this is not technically fine because 
\end{remark}
\begin{remark}
	It is somewhat impressive that the notion of uniform continuity came up crucially in the above proof. Namely, it is explicitly needed because we need the same $\delta$ to work over the entire interval.
\end{remark}

\subsection{Integral Rules}
Let's talk through some integral rules. Here are the main statements.
\begin{proposition}
	Fix $f$ and $g$ integrable functions $[a,b]\to\RR$ with $c\RR.$
	\begin{listalph}
		\item $\int_a^bcf(x)\,dx=c\int_a^bf(x)\,dx.$
		\item $\int_a^b(f+g)(x)\,dx=\int_a^bf(x)\,dx+\int_a^bg(x)\,dx.$
	\end{listalph}
\end{proposition}
\begin{proof}
	Omitted. The main idea is to imitate the proof of the corresponding limit laws.
\end{proof}
Similarly, here is a bounding result.
\begin{proposition}
	Fix integrable functions $f,g:[a,b]\to\RR$ such that $f(x)\ge g(x)$ for each $x\in[a,b].$ Then
	\[\int_a^bf(x)\,dx\ge\int_a^bg(x)\,dx.\]
\end{proposition}
\begin{proof}
	The main idea is to compare the Darboux sums by hand. To make this technically easier, we show that
	\[\int_a^b(f-g)\ge0.\]
	Namely, writing down a Darboux sum, we find that, for a given partition $P\subseteq[a,b],$ we find that
	\[U(f-g,P)=\sum_{k=1}^n\underbrace{M(f-g,[t_{k-1},t_k])}_{\ge0}(t_k-t_{k-1})\ge0,\]
	so it follows $U(f)=\inf_PU(f-g,P)\ge0,$ which is what we need because we know $f-g$ is integrable.
\end{proof}

\subsection{Pushing Integrability}
Integrable functions tend to be pretty well-behaved. We have the following definition.
\begin{definition}[Piecewise continuous]
	A function $f:[a,b]\to\RR$ is said to be \textit{piecewise continuous} if and only if there is a partition $\{t_k\}_{k=0}^n$ of $[a,b]$ such that $f$ is uniformly continuous (!) on each $(t_{k-1},t_k).$
\end{definition}
In theory, we'd like to say that $f$ is continuous on each $[t_{k-1},t_k]$ (giving uniform continuity), but continuity on two sets implies continuity on their union, so this would make piecewise continuity imply continuity. More concretely, we would like functions like
\[f(x)=\begin{cases}
	1 & x > 0, \\
	x = 0,
\end{cases}\]
for $f:[0,1]\to\RR$ to be piecewise continuous, but there is no partition of $[0,1]$ will fix the problem at $0.$

As for why we want uniform continuity, the intuition is that it imitates continuity. More formally, we want $f$ restricted to $(t_{k-1},t_k)$ to extend to a continuous function on $[t_{k-1},t_k]$ to really earn the name ``piecewise.'' More concretely, we are trying to prevent infinite discontinuities.

Similarly, we have the following result.
\begin{definition}[Piecewise montonic]
	A function $f:[a,b]\to\RR$ is said to be \textit{piecewise monotonic} if and only if there is a partition $\{t_k\}_{k=0}^n$ of $[a,b]$ such that $f$ is monotonic on each $(t_{k-1},t_k).$
\end{definition}
Again, we want the open intervals here for approximately the same reason as before: functions such as
\[f(x)=\begin{cases}
	x & x<0, \\
	x+1 & x\ge0,
\end{cases}\]
with bad jumps would not be piecewise monotonic.
\begin{warn}
	Piecewise monotonic does not imply bounded. For example, a patched version of $\tan.$
\end{warn}

These covers most functions we care about. Of course, it does not cover all of them.
\begin{nex}
	The function
	\[f(x)=\begin{cases}
		0 & x=0, \\
		x\sin\left(\frac1x\right) & x\ne0,
	\end{cases}\]
	is continuous on $\RR$ but not locally monotone at $0.$ The function $f(x)+\frac x2$ even has $f'(x)>0.$ The function $\int_0^xf(t)\,dt$ is even continuously differentiable at $0$ but not locally monotone.
\end{nex}
However, it is true that, if $f$ is continuously differentiable, then $f'(x)>0$ ($f'(x)<0$) implies that $f$ is locally increasing (decreasing) around $x.$ Importantly, this assertion is agnostic about $f'(x)=0,$ as discussed in the above example.

Anyways, we have the following result.
\begin{proposition}
	Fix $f:[a,b]\to\RR$ which is piecewise continuous or bounded and piecewise monotonic. Then $f$ is integrable.
\end{proposition}
\begin{proof}
	We take these one at a time.
	\begin{itemize}
		\item Because $f$ is piecewise continuous, find our partition $\{t_k\}_{k=0}^n$ such that $f$ is uniformly continuous on each $(t_{k-1},t_k).$ Then extend $f$ to a continuous function $[t_{k-1},t_k],$ and we note that $f$ is bounded and hence integrable here automatically. Then taking the union of these integrals will finish.
		\item Because $f$ is piecewise monotonic, find a partition $\{t_k\}_{k=0}^n$ such that $f$ is monotonic on each $(t_{k-1},t_k).$ Then extend $f$ to a monotonic function $[t_{k-1},t_k]$ by taking
		\[f(t_{k-1}):=\lim_{t\to t_{k-1}^+}f(t)\qquad\text{and}\qquad f(t_{k-1}):=\lim_{t\to t_{k}^-}f(t)\]
		Now, we note that $f$ is bounded and hence integrable here automatically. Then taking the union of these integrals will finish.
		\qedhere
	\end{itemize}
\end{proof}

\subsection{Some Extra Bits}
This is a result we might care about.
\begin{theorem}[Intermediate value for integrals]
	Fix $f$ continuous on $[a,b].$ Then there exists $c\in(a,b)$ such that
	\[f(c)=\frac1{b-a}\int_a^bf(t)\,dt.\]
\end{theorem}
\begin{proof}
	Roughly speaking, this comes from using the intermediate value theorem on
	\[F(x):=\int_a^xf(t)\,dt\]
	to finish.
\end{proof}
Ross also mentions the Dominated convergence theorem, but we don't care.

Anyways, let's jump into some exercises.
\begin{definition}[Step]
	A function $f:[a,b]\to\RR$ is said to be \textit{step} if and only if there is a partition $\{t_k\}_{k=0}^n$ of $[a,b]$ such that $f$ is constant on each $(t_{k-1},t_k).$
\end{definition}
We note that step functions are piecewise monotonic and hence integrable.

\subsection{Fundamental Theorem of Calculus}
Here is the first part.
\begin{theorem}[Fundamental theorem of calculus, I]
	Fix $f$ a continuous function on $[a,b]$ which is differentiable on $(a,b)$ such that $g'$ is integrable on $[a,b].$ Then
	\[\int_a^bg'(x)\,dx=g(b)-g(a).\]
\end{theorem}
We remark that the typical requirement is that $g'$ is continuous, but we have weakened it to integrable. Granted, I am not sure how to create a function whose derivative is bad enough to not be integrable.
\begin{remark}
	It is true that an integrable function must be continuous somewhere.
\end{remark}