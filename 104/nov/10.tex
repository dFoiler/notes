% !TEX root = ../notes.tex















So a midterm ocurred.

\subsection{Midterm Housekeeping}
Here are some notes.
\begin{itemize}
	\item Some people tried to use the Ratio test on the first question. It is difficult to make such an argument rigorous because the 
	\item The check of the endpoint $x=-2$ on the first question is a bit tricky. The point here is that the subsequence
	\[\sum_{n\text{ odd}}\frac{(-2)^n}{n2^n}=\sum_{n\text{ odd}}\frac{(-1)^n}n\]
	actually diverges even though it looks like it converges: the sum is only over a single sign. Additionally, one must keep track of the positive even terms to make sure they do not counteract the negative divergence.
	\item People did pretty well on 2(a). On 2(b), many people looked at
	\[\lim_{n\to\infty}\sup\left\{\left|\sin\left(\frac xn\right)\right|:x\in[0,1]\right\},\]
	which is correct. However, some people tried to reverse the limit and the supremum to say that this goes to $0$ for free, which is not correct: such an argument would work for $x\in\RR,$ where the convergence is not uniform.
	\item For the later questions, some people had trouble keeping track of quantifiers, especially upon negation. For example, the difference between continuity and uniform continuity is subtle but important.
	\item Professor Sharma expected more people to use sequences on number 3 because this turns discontinuity into a tangible object.
	\item Professor Sharma is surprised that people approached (a) and (b) different even though they are the same question: the telemetric on $\ZZ$ induces the same topology on $\ZZ$ as the usual one.
	\item The common idea for number 5 was to do something like
	\[U=\left\{(x,y)\in\QQ^2:x^2<2\right\}\]
	to disconnect $\QQ^2.$ However, some people did not prove that $U$ and $\QQ^2\setminus U$ are open, which is not completely obvious. In general, proving a disconnection requires many checks.
	\item Some people tried to use the fact that the image of a connected set is connected for number 5, which is difficult to rigorize into a proof.
	\item Keeping track of symbols for number 6 was a bit nontrivial. Namely, we need a third symbol in this proof to make things function, which was not present in all submissions.
\end{itemize}
This exam was harder than the first one, on average. People had the most trouble with number 5, probably because connectedness is a weird concept. Last time the average was about 80\%; this time it was about 75\% (about 89/120). It is likely that there will be some minor curve, though how much there is remains unclear.

\subsection{Derivatives}
We are talking about derivatives, for now. We'll do integrals as the last part of this class.
\begin{ex}
	Fix $1_\QQ:\RR\to\QQ$ the $\QQ$-indicator. It's not continuous anywhere, which we had a few ways of showing. But can we still compute
	\[\int_3^4f(x)\,dx?\]
	The answer will be no, but it's not obvious why or why not because typical calculus (such as the Fundamental theorem of calculus) ideas do not apply.
\end{ex}
To attack these kinds of questions, we will need rigor. So that is where we are going.

The first section of chapter 5 reviews derivatives, proving the various derivative rules. The proofs are not terribly interesting, but they are good to ``better understand'' the derivative. At the very least they require technical skill.
\begin{definition}[Derivative]
	Fix $f:S\to\RR$ a real-valued function. Then we define the \textit{derivative}
	\[f'(x):=\lim_{h\to0}\frac{f(x+h)-f(x)}h.\]
	If, given $a\in S,$ we have $f'(a)\in\RR,$ then we say that $f$ is \textit{differentiable} at $a.$
\end{definition}
In practice, the limit definition might be annoying to use, or even infeasible. For example, it is reasonable to expect
\[f(x):=\begin{cases}
	\sqrt{x^4+\sin^4x}/x & x\ne0, \\
	0 & x=0.
\end{cases}\]
We expect that $f'(0)=0,$ but this is not obvious. If we could show that the derivative is continuous, we could use the quotient rule and go to $0,$ but it is not obvious that the derivative is in fact continuous. For a worse example,
\[f(x)=\begin{cases}
	x^2\sin(1/x) & x\ne0, \\
	0 & x=0,
\end{cases}\]
is differentiable on all of $\RR$ but not continuously differentiable at $x=0.$

Anyways, let's prove a fairly basic result.
\begin{proposition}
	Fix $f:S\to\RR$ a function which is differentiable at some point $a\in S.$ Then $f$ is continuous at $a\in S.$
\end{proposition}
\begin{proof}
	We are given that the limit
	\[\lim_{h\to0}\frac{f(a+h)-f(a)}h=f'(a)\in\RR\]
	exists and is finite. We want to show that
	\[\lim_{a\to0}f(x)=f(a).\]
	Intuitively, if we lose continuity, then the numerator in the limit for $f'(a)$ will not go to $0$ even though the denominator does, which will imply that $f'(a)\notin\RR.$ We will not make this more rigorous here.
	% To make this rigorous, as usual fix $\varepsilon>0$ so that we want to show there exists $\delta>0$ to have
	% \[|x-a|<\delta\stackrel?\implies|f(x)-f(a)|<\varepsilon.\]
	% However, by the differentiability, we are given that there exists $\delta_0>0$ so that
	% \[|x-a|<\delta\stackrel?\implies|f(x)-f(a)|<\varepsilon|x-a|.\]\todo{}
\end{proof}
Of course, there are continuous functions which are not differentiable.
\begin{example}
	The function $f(x):=|x|$ is continuous but not differentiable at $0$ because the corresponding limit does not exist.
\end{example}
\begin{example}
	The function $f(x):=\sqrt[3]x$ is continuous but not differentiable at $0$ because the corresponding limit is infinite.
\end{example}

\subsection{Derivative Rules}
Here are some derivative rules.
\begin{proposition}
	Fix $f,g:S\to\RR,$ and suppose that $f$ and $g$ are both differentiable at $a\in S.$ Then $(f+g)'(a)$ exists and is equal to $f'(a)+g'(a).$
\end{proposition}
\begin{proof}
	We see that
	\begin{align*}
		(f+g)'(a) &= \lim_{h\to0}\frac{(f+g)(a+h)-(f+g)(a)}h \\
		&= \lim_{h\to0}\frac{f(a+h)-f(x)}h+\lim_{h\to0}\frac{g(a+h)-g(a)}h \\
		&= f'(a)+g'(a),
	\end{align*}
	where the key step was to split up the sum of limits into the individual limits.
\end{proof}
\begin{proposition}
	Fix $f,g:S\to\RR,$ and suppose that $f$ and $g$ are both differentiable at $a\in S.$ Then $(f+g)'(a)$ exists and is equal to $f'(a)g(a)+f(a)g'(a).$
\end{proposition}
\begin{proof}
	We see that
	\[(fg)'(a)=\lim_{h\to0}\frac{(fg)(a+h)-(fg)a}h=\lim_{h\to0}\frac{f(a+h)g(a+h)-f(a)g(a)}h.\]
	Now, the key point is to add and subtract $f(a+h)g(a)$ in the numerator, as we did for the corresponding product rule for limits. We get
	\begin{align*}
		(fg)'(a) &= \lim_{h\to0}\frac{f(a+h)g(a+h)-f(a+h)g(a)+f(a+h)g(a)-f(a)g(a)}h \\
		&= \lim_{h\to0}\frac{f(a+h)g(a+h)-f(a+h)g(a)}h+\lim_{h\to0}\frac{f(a+h)g(a)-f(a)g(a)}h \\
		&= \lim_{h\to0}f(a+h)\cdot\lim_{h\to0}\frac{g(a+h)-g(a)}h+g(a)\cdot\lim_{h\to0}\frac{f(a+h)-f(a)}h \\
		&= f(a)f'(a)+g(a)f'(a).
	\end{align*}
	Importantly, in the last equality we have used the fact that $f$ is continuous at $a,$ which is true because $f$ is differentiable at $a.$
\end{proof}
And here is an exercise.
\begin{exe}
	Fix $f$ a function on an open interval continuous on some $a\in\RR.$ Further, take $g$ defined on an open interval containing $f(a).$ Then we show $g\circ f$ is defined on some open interval containing $a.$
\end{exe}
\begin{proof}
	Without loss of generality, we assert that $g$ is defined on $(f(a)-\varepsilon,f(a)+\varepsilon),$ for some $\varepsilon>0.$ Then, because $f$ is continuous at $a,$ there exists $\delta>0$ such that each $x\in\op{dom}f$ has
	\[|x-a|<\delta\implies|f(x)-f(a)|<\varepsilon.\]
	So we see that $x\in\op{dom}f$ with $x\in(a-\delta,a+\delta)$ implies $f(x)\in(f(a)-\varepsilon,f(a)+\varepsilon),$ which implies that $(g\circ f)(x)$ is well-defined.

	To finish, we know there exists $\delta'$ such that $f$ is defined on $(a-\delta',a+\delta').$ Then $\delta'':=\min\{\delta,\delta'\}$ will have $x\in(a-\delta'',a+\delta'')\subseteq\op{dom}f$ and $x\in(a-\delta,a+\delta),$ so indeed $(a-\delta'',a+\delta'')\in\op{dom}(g\circ f).$
\end{proof}
\begin{remark}
	Keeping track of the semantics is annoying but important: it is easy to miss (as I did) the last step of intersecting the interval inside of $\op{dom}f$ with our $(a-\delta,a+\delta).$ This is exacerbated by the fact we usually write continuity omitting the condition $x\in\op{dom}f.$
\end{remark}
Next time we will talk about the Mean value theorem.