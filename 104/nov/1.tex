\documentclass[../notes.tex]{subfiles}

\begin{document}

% !TEX root = ../notes.tex













Finally done with October. One last push I guess.

\subsection{Uniform Convergence}
Last class we were looking at the sequence of functions
\[f_n(x):=x^n\]
for $f_n:[0,1]\to[0,1].$ Then we say that, for $x\in\left[0,1\right),$ we see that
\[\lim_{n\to\infty}f_n(x)=0,\]
but
\[\lim_{n\to\infty}f_n(1)=1.\]
With this in mind, we define
\[f(x):=\begin{cases}
	0 & 0\le x<1, \\
	1 & x=1.
\end{cases}\]
We would like to say that $f_n\to f$ as $n\to\infty.$ This notion is more rigorously called ``pointwise'' convergence.
\begin{definition}[Pointwise convergence]
	Given a sequence of functions $f_n:S\to\RR,$ we say that $f_n$ \textit{converges pointwise to} to some $f:S\to\RR$ if and only if, for each $x\in S,$ we have that $f_n(x)\to f(x)$ as $n\to\infty.$
\end{definition}
This is perhaps the weakest and worst form of convergence. For example, it has the defect that continuous functions can pointwise converge to a discontinuous function.
\begin{remark}
	The high-level reason why this is failing is that our convergence is not very uniform: points $x$ close to $1$ will have $f_n(x)\to0$ very slowly.
\end{remark}
So we need to strengthen our notion of convergence. This gives uniform convergence.
\begin{definition}[Uniform convergence]
	Fix a sequence of functions $f_n:S\to\RR.$ Then $f_n\to f$ \textit{uniformly converges to} $f:S\to\RR$ if and only if, for each $\varepsilon>0,$ there is $N$ such that $n>N$ implies
	\[|f_n(x)-f(x)|<\varepsilon\]
	for each $x\in\RR.$
\end{definition}
This is different from pointwise convergence because now $\varepsilon$ is not allowed to vary with $x.$ This is similar to uniform continuity in that, again, $\varepsilon$ was not allowed to depend on which point whose continuity we were looking at.
\begin{nex}
	It is not the case that $f_n(x):=x^n$ converges to $f(x)=1_{x=1}$ uniformly. Well, take $\varepsilon:=1/2.$ Then there is no $N$ such that $n>N$ implies
	\[\left|x^n\right|<1/2\]
	for each $x\in\left[0,1\right).$ Indeed, for any $N,$ choose any $n>N$ and then take $x:=\sqrt[n]{2/3}$ so that $|x^n|=2/3>1/2.$
\end{nex}
Anyways, let's see uniform convergence do something useful.
\begin{proposition}
	Suppose that $f_n:S\to\RR$ is a sequence of continuous functions converging uniformly to $f:S\to\RR.$ Then $f$ is continuous.
\end{proposition}
\begin{proof}
	Fix any $a\in S$ so that we want to show $f$ is continuous at $a.$ Well, fix any $\varepsilon>0.$ Then there exists $N$ such that $n>N$ has
	\[|f_n(x)-f(x)|<\frac\varepsilon3\]
	for each $x\in S.$ Now, fix any $n>N.$ Because $f_n$ is continuous, there exists $\delta$ such that
	\[|x-a|<\delta\implies|f_n(x)-f_n(a)|<\frac\varepsilon3.\]
	The point of these estimates is the following manipulation: for $x\in S$ with $|x-a|<\delta,$ we see
	\begin{align*}
		|f(x)-f(a)| &\le |f(x)-f_n(x)|+|f_n(x)-f_n(a)|+|f_n(a)-f(a)| \\
		&< \frac\varepsilon3+\frac\varepsilon3+\frac\varepsilon3 \\
		&=\varepsilon,
	\end{align*}
	which establishes the needed continuity.
\end{proof}
\begin{remark}
	This sort of ``double triangle inequality'' idea is fairly common. For example, a similar argument shows that uniformly continuous functions that converge uniformly will converge to a uniformly convergent function. (Dropping the ``uniformly'' in either hypothesis makes this false.)
\end{remark}
We quickly remark that we also have the following equivalent definition of uniform convergence.
\begin{proposition}
	Fix $f_n:S\to\RR$ and $f:S\to\RR.$ Then $f_n\to f$ converges uniformly if and only if
	\[\limsup_{n\to\infty}|f(x)-f_n(x)|=0.\]
\end{proposition}
\begin{proof}
	Omitted; see the book.
\end{proof}
Let's do some exercises.
\begin{exercise}
	We study the sequence of functions $f_n(x):=\frac1{1+x^n}$ on $\left[0,\infty\right)$ for $n\in\NN.$
\end{exercise}
\begin{proof}
	We start by evaluating the pointwise limit.
	\begin{itemize}
		\item For $0\le x<1,$ we see that $x^n\to0,$ so $1+x^n\to1,$ so $f_n(x)\to 1.$
		\item For $x=1,$ we see that this is constantly $f_n(x)=\frac12.$
		\item For $x>1,$ we see that $x^n\to\infty,$ so $f_n(x)\to0.$
	\end{itemize}
	So our convergent function is
	\[f(x)=\begin{cases}
		1 & 0\le x<1, \\
		1/2 & x=1, \\
		0 & x>1.
	\end{cases}\]
	For example, this implies that $f_n\to f$ is not uniform on $[0,1]$ because $f$ is not continuous on $[0,1].$

	Explicitly, we fix $x\in\left[0,1\right],$ and we can evaluate
	\[f_n(x)-f(x)=\frac{-x^n}{1+x^n},\]
	so we see that
	\[\sup_{x\in\left[0,1,\right)}\left|f_n(x)-f(x)\right|=\frac12\]
	after doing a bit of legwork. So indeed, $f_n\to f$ is not uniform.

	How about $[0,1/2]$? Here we can evaluate
	\[\sup_{x\in\left[0,1\right)}|f_n(x)-f(x)|=\frac{(1/2)^n}{1+(1/2)^n}\]
	because this difference is increasing. So this vanishes as $n\to\infty,$ and we are safe.
\end{proof}
\begin{remark}
	Uniform continuity does not really care for individual points because it is a global concept. For example, $f_n\to f$ is not uniform even on $\left[0,1\right),$ using the same function.
\end{remark}

\subsection{Being Integrable}
It feels like there ought to be some property that pointwise convergence preserves. For example, does pointwise convergence preserve being integrable?
\begin{example}
	Consider
	\[f_n(x)=\sum_{k=0}^nx^k\]
	on $(0,1).$ These functions will converge to $\frac1{1-x}$ as $n\to\infty,$ which is not integrable on $(0,1)$ (even though $f$ is not continuous).
\end{example}
Well, what about uniform continuity?
\begin{proposition}
	Fix $f_n:S\to\RR$ a sequence of functions converging uniformly to a function $f:S\to\RR.$ Then $f$ is integrable on bounded domains.
\end{proposition}
\begin{proof}
	We take a few properties of integration on faith because we have not defined what it means to be integrable. Then, if we are integrating of over a bounded domain $T\subseteq S\cap[-M,M],$ then, for any $\varepsilon>0,$ there exists $N$ such that $n>N$ has
	\[\left|\int_Tf(x)\,dx-\int_Tf_n(x)\,dx\right|\le\int_T|f(x)-f_n(x)|\,dx\le\int_{[-M,M]}\varepsilon\,dx=2M\varepsilon.\]
	Sending $\varepsilon\to0$ shows that $\int_Tf(x)$ is well-defined as a real number.
\end{proof}
However, we need to take some care for unbounded domains. I'm honestly not sure what is the case.

\subsection{Uniformly Cauchy}
Here is another notion.
\begin{definition}
	A sequence of functions $f_n:S\to\RR$ is \textit{uniformly Cauchy} if and only if there exists $\varepsilon>0$ such that $N\in\NN$ has $n,m>N$ implies
	\[|f_n(x)-f_m(x)|<\varepsilon\]
	for each $x\in S.$
\end{definition}
\begin{remark}
	This is essentially asserting that $\{f_n\}_{n\in\NN}$ converges in the space of functions $\RR\to\RR.$
\end{remark}
We would like uniformly Cauchy sequences of functions to converge uniformly to some function. Indeed this is the case.
\begin{prop}
	A uniformly Cauchy sequence $f_n:S\to\RR$ converges uniformly to a function $f:S\to\RR.$
\end{prop}
\begin{proof}
	We start by exhibiting our function $f:S\to\RR.$ We do this pointwise: for each $x\in S,$ we see that $\{f_n(x)\}_{n\in\NN}$ is a Cauchy sequence and hence converges to some $f(x).$

	Now we show that $f_n\to f$ uniformly. Well, for any $\varepsilon>0,$ there exists $N$ so that $n,m>N$ has
	\[|f_n(x)-f_m(x)|<\frac\varepsilon2\]
	for each $x\in S.$ Now we notice that each $f_n(x)$ is inside of $\left(f_m(x)-\frac{\varepsilon}{2},f_m(x)+\frac{\varepsilon}{2}\right),$ so the limit of the $f_n$ must live inside of this interval as well. In particular, have that our $N$ has $m>N$ implying
	\[|f(x)-f_m(x)|\le\frac{\varepsilon}{2}<\varepsilon\]
	for each $x\in S.$
\end{proof}

\subsection{Series of Functions}
Of course, we are really interested in power series. So, for example, does
\[\sum_{k=0}^n\frac1{k!}x^k\to e^x\]
converge uniformly? More explicitly, we define
\[f_n(x):=\sum_{k=0}^n\frac1{k!}x^k.\]
We would like this to converge to a real function, and we would also like to know if this converge is uniform. The answer is no: very negative values can go over the rails, even for large $n.$ But we can save ourselves by looking at a bounded interval, where our convergence should be better-behaved. We won't write out the details here.

It turns out that restricting to a bounded interval is also not good enough, however.
\begin{example}
	The series
	\[-\log(1-x)=\sum_{k=1}^\infty\frac{x^k}k\]
	will have the partial sums converge but not uniformly.
\end{example}
What about preserving differentiability?
\begin{example}
	The series
	\[f(x):=\sum_{k=1}^\infty\frac{\sin\left(n^2x\right)}{n^2}\]
	has its partial sums converge uniformly to a continuous function. However, its derivative, taken pointwise, is
	\[f'(x)=\sum_{k=1}^\infty\cos\left(n^2x\right),\]
	which does not seem to converge anywhere. So this function seems to be continuous everywhere and differentiable nowhere.
\end{example}
For completeness, here is the graph of the above function.
\begin{center}
	\begin{asy}
		unitsize(2cm);
		import graph;
		ngraph=1000;
		real f(real x)
		{
			int n = 100;
			real total = 0;
			for(int i = 1; i < n; ++i)
			{
				total += sin(i*i*x) / (i*i);
			}
			return total;
		}
		draw(graph(f,-4,4), blue);
		draw((-4,0)--(4,0));
		draw((0,-2)--(0,2));
	\end{asy}
\end{center}
Anyways, we would like to talk about series of functions. Here is a good test for this purpose.
\begin{theorem}[Weierstrass \texorpdfstring{$M$}{}-test]
	Fix a sum
	\[\sum_{k=1}^\infty g_k(x).\]
	Suppose that there exists $M_k\ge0$ such that $|g_k(x)|\le M_k$ for each $k$ while $\sum_{k=1}^\infty M_k$ converges uniformly. Then the partial sums uniformly converge.
\end{theorem}
\begin{example}
	We can use the above test to show that $e^x$ converges uniformly on any bounded interval. Essentially, even though $\frac{x^k}{k!}$ might not be bounded on all of $\RR,$ it is at least bounded on the bounded interval, which is good enough for the above test.
\end{example}
And here is an exercise to close us out.
\begin{exercise}
	Suppose that $f_n:S\to\RR$ is a bounded function converging uniformly to some $f:S\to\RR.$ Then $f$ is bounded.
\end{exercise}
\begin{proof}
	Fix $\varepsilon=1,$ and then we are promised $N$ such that $n>N$ has
	\[|f_n(x)-f(x)|<\varepsilon\]
	for each $x\in S.$ Now fix some $n>N.$ We see that $f_n$ is bounded so that $f_n(x)\in[-M,M]$ for some $M\in\RR.$ But now $f(x)\in[-M-1,M+1]$ for each $x\in S,$ so we get that $f$ is bounded.
\end{proof}

\end{document}