\documentclass[../notes.tex]{subfiles}

\begin{document}

Let's go ahead and begin.

\subsection{Logistics}
Email is \texttt{asharma18@berkeley.edu}. The course website is \href{https://www.ocf.berkeley.edu/~asharma/Math104/}{\texttt{ocf.berkeley.edu/\~asharma/Math104}}. Namely, we are not using bCourses. Office hours are (tentatively) 3:15PM--4:45PM in 833 Evans and Saturday 2PM--4PM online. Check the website for the Zoom link.

We're using \textit{Elementary Analysis} by Ross, and we will follow the book pretty closely. Most of the things we do will be relatively known, but we will be adding rigor as we go through. For example, the homework is from the book and largely in order.

Do attempt to turn in the homework nontrivially early due to technical issues. Turning in will be done completely online. Please put different sections of the homework to different pages. This will help if a section needs to be moved to a later homework. Also, please take the homework seriously; it will gauge if you are keeping up with the homework. Similarly, working together is somewhat discouraged if you are not actually learning.

Rigor will be important in this class. For example, if asked to find the derivative of $f(x)=x^2$ using the limit definition of the derivative, then one should know the limit definition. These sorts of things are important in this class.
\begin{ex}
	We can compute
	\[\int_1^2\frac1x\,dx=\log2-\log1=\log2.\]
	However, we cannot compute $\int_{-1}^21/x\,dx$ because it has problems at $0.$
\end{ex}
The point of this example is that $\int_{-1}^21/x\,dx$ seems like it should be $\log2,$ but this is very false because $\int1/x\,dx$ is an entire class of functions. Definitions are important here.

\subsection{Foreshadowing}
We will be doing induction proofs in this class. Here's a different kind of thing.
\begin{thm}[Extreme value theorem]
	A continuous function $f:I\to\RR$ defined on a closed and bounded interval $I$ achieves an absolute maximum and minimum.
\end{thm}
Explaining why this is true requires some thought. We can show that the image of $f$ is bounded because the domain is compact, so if it has no maximum, we could create an infinitely ascending sequence with no bound in the image of $f.$

This turns out to be quite subtle. Upgrading from closed and bounded intervals is nontrivial because finding the condition ``compact'' is quite difficult to do. For example, we cannot set $D=\QQ\cap[0,1]$ because of something like $f(x)=|x-\sqrt2|,$ which achieves no absolute minimum on $D.$

As another kind of example, we can create a function which is nowhere continuous, which is somewhat nonintuitive given most people's graphical understanding of functions. For example,
\[f(x)=\begin{cases}
	0 & x\in\QQ, \\
	1 & x\notin\QQ.
\end{cases}\]
The main problem here is that
\[\lim_{x\to x_0}f(x)\]
does not exist here (this requires a $\delta$-$\varepsilon$ proof, sadly), so it cannot be continuous by the definition of continuous.

Can we create a function which is continuous everywhere but somewhere not differentiable? Here $f(x)=|x|.$ This construction can be generalized to any finite set of points $S$ we want to not be differentiable by
\[\sum_{a\in S}|x-a|.\]
\begin{ques}
	How bad can we make our set of not differentiable points? Can we make our bad points $\QQ$? How about all of $\RR$?
\end{ques}

Here's another question. Consider the sequence $3,3.1,3.14,3.141,\ldots,$ which consists of the truncated  Does this sequence converge? Well, we need a good definition of convergence. Cauchy sequences, for example, will do the trick here for $\RR,$ but this does not converge in $\QQ.$ Does $\RR$ have this same problem? More precisely, this is the question we are asking.
\begin{ques}
	Suppose a sequence in $\RR$ ``converges.'' Must it converge to a real number?
\end{ques}
Using fancy words, we are asking if $\RR$ is metrically complete. We could ask the same question for $\ZZ.$

One issue here is that we don't have a good definition of a limit point without a number we actually converge to. For example, perhaps we want convergence to means something like ``the terms get closer and closer together,'' which is connected to Cauchy sequences.

\end{document}