% !TEX root = ../notes.tex

















Here we go. We're talking about \S13 today. It's optional but important.

\subsection{Metrics}
One of the main idea is to expand the theory we've built over $\RR$ to work more generally. What is something nice that $\RR$ has? Well, $\RR$ has a good notion of ``distance.''
\begin{definition}[Metric space]
    Given a set $X,$ we say that a function $d:X^2\to\RR_{\ge0}$ is a \textit{metric} if and only if it satisfies the following conditions; fix any $x,y,z\in X.$
    \begin{itemize}
        \item Distance-zero: $d(x,x)=0,$ and $d(x,y)>0$ if $x\ne y.$ In other words, $d(x,y)=0$ if and only if $x=y.$
        \item Symmetry: $d(x,y)=d(y,x).$
        \item Triangle inequality: $d(x,z)\le d(x,y)+d(y,z).$
    \end{itemize}
    In this case, we call $X$ a \textit{metric space}.
\end{definition}
\begin{warn}
    To be a metric, we must satisfy all of the above conditions. They are annoying, but they are necessary.
\end{warn}
A while ago, we defined other distance functions; they were metrics. Recall the following examples.
\begin{example}
    On $\RR^2,$ the function
    \[d_{\text{Euclid}}\big((x_1,y_1),(x_2,y_2)\big)=\sqrt{(x_1-x_2)^2+(y_1-y_2^2)}\]
    is the usual, Euclidean metric.
\end{example}
\begin{example}
    On $\RR^2,$ the function
    \[d_{\text{taxi}}\big((x_1,y_1),(x_2,y_2)\big)=|x_1-x_2|+|y_1-y_2|\]
    is called the taxicab metric. Physically speaking, this is the distance we have to go if we can only walk along ``streets'' parallel the axes.
\end{example}
\begin{example}
    On $\RR^2,$ the function
    \[d_{\text{tele}}\big((x_1,y_1),(x_2,y_2)\big)=\begin{cases}
        1 & (x_1,y_1)\ne(x_2,y_2), \\
        0 & (x_1,y_1)=(x_2,y_2).
    \end{cases}\]
    is called the tele-metric. Physically speaking, this is the distance we have to go in the internet: it's all a click away.
\end{example}
We want these metrics to be nice with each other. For example, if a sequence of points converges to the origin using $d_{\text{Euclid}},$ then does it converge to the origin in $d_{\text{taxi}}$? What about $d_{\text{tele}}$?

Anyways, let's do a more nontrivial example.
\begin{exercise}[Ross 13.3]
    Let $B$ be the set of bounded sequences in $\RR.$ Then, given two sequences $\{x_k\}_{k\in\NN},\,\{y_k\}_{k\in\NN}\in B,$ we define
    \[d(x,y)=\sup_k|x_k-y_k|.\]
    Then $d$ is a metric on $B.$
\end{exercise}
\begin{proof}
    We can check the conditions one at a time.
    \begin{itemize}
        \item Distance-zero: the only way to make the supremum of the differences zero is to make everything zero, so all elements are equal.
        \item Symmetry: the absolute value respects negation.
        \item Triangle inequality: this is checked by force.
        \qedhere
    \end{itemize}
\end{proof}

\subsection{Convergence Ideas}
To answer these questions, we need a good notion of convergence. Recall the definition in $\RR.$
\begin{definition}[Convergence in \texorpdfstring{$\RR$}{}]
    A sequence $\{x_n\}_{n\in\NN}\subseteq\RR$ \textit{converges} to $x\in\RR$ if and only if, for each $\varepsilon>0,$ there exists $N$ such that $n>N$ implies
    \[|x-x_n|<\varepsilon.\]
\end{definition}
To generalize this to more general metric spaces, we note that $|x-x_n|$ is really just the distance between $x$ and $x_n.$ Here is the general notion for metric spaces.
\begin{definition}[Convergence in metric spaces]
    Fix $X$ a metric space with metric $d.$ Then a sequence $\{x_n\}_{n\in\NN}\subseteq\RR$ \textit{converges} to $x\in X$ if and only if, for each $\varepsilon>0,$ there exists $N$ such that $n>N$ implies
    \[d(x,x_n)<\varepsilon.\]
\end{definition}
Note this is essentially the same definition as in $\RR.$

How about Cauchy sequences? Here was our definition in $\RR.$
\begin{definition}[Cauchy in \texorpdfstring{$\RR$}{}]
    A sequence $\{x_n\}_{n\in\NN}\subseteq\RR$ is \textit{Cauchy} if and only if, for each $\varepsilon>0,$ there exists $N$ such that $m,n>N$ implies
    \[|x_m-x_n|<\varepsilon.\]
\end{definition}
Again, to generalize, we swap out distance in $\RR$ with general distance.
\begin{definition}[Cauchy in metricspaces]
    A sequence $\{x_n\}_{n\in\NN}\subseteq\RR$ is \textit{Cauchy} if and only if, for each $\varepsilon>0,$ there exists $N$ such that $m,n>N$ implies
    \[d(x_m,x_n)<\varepsilon.\]
\end{definition}

\subsection{Completeness}
Completeness of $\RR$ was another part of our story here. In some sense, this came down to all Cauchy sequences converged, which are notions we have defined. So we have the following.
\begin{definition}[Complete]
    A metric space $X$ is \textit{complete} if every Cauchy sequence converged.
\end{definition}
Note that all convergent sequences are Cauchy\footnote{If $a_n\to a,$ then for any $\varepsilon>0,$ find $N$ for which $n>N$ implies $|a_n-a|<\varepsilon.$ Then $n,m>N$ implies $|a_n-a_m|<\varepsilon.$}, so the reverse direction is the kicker.
\begin{example}
    We showed that $\RR$ was complete, which roughly came from the Completeness axiom.
\end{example}
\begin{nex}
    We know that $\QQ$ is not complete: take the sequence $\{\floor{\pi n}/n\}_{n\in\NN}.$
\end{nex}
\begin{remark}
    We can define $\RR$ as equivalence classes of Cauchy sequences in $\QQ,$ if we wanted.
\end{remark}
What about $\RR^2$ (with the Euclidean metric)? is it complete? In particular, does every Cauchy sequence converge?
\begin{proposition}
    Fix $D\in\NN.$ Then $\RR^D$ is a complete metric space.
\end{proposition}
\begin{proof}
    Suppose that $\{x_k\}_{k\in\NN}$ is a Cauchy sequence so that we need to show it converges. For concreteness, given $x_k\in\RR^D,$ we let $\pi_\ell x_k$ be the $\ell$th coordinate.

    We claim that $\{\pi_\ell x_k\}_{k\in\NN}$ is a Cauchy sequence. Indeed, for any $\varepsilon>0,$ we know there is some $N$ so that $n,m>N$ implies
    \[d(x_n,x_m)<\varepsilon.\]
    But
    \[|\pi_\ell x_n-\pi_\ell x_m|\le\sqrt{\sum_{k=1}^D(\pi_kx_m-\pi_kx_n)^2}=d(x_n,x_m)<\varepsilon,\]
    so indeed, $|\pi_\ell x_n-\pi_\ell x_m|<\varepsilon,$ making $\{\pi_\ell x_k\}_{k\in\NN}$ a Cauchy sequence.

    Now, because each coordinate projects into a Cauchy sequence, each coordinate will converge eventually because $\RR$ is complete, so we combine these converging sequences into
    \[y:=(y_\ell)_{\ell=1}^D,\]
    where $\pi_\ell x_n\to y_\ell$ as $n\to\infty.$ We claim that $x_n\to y$ as $n\to\infty.$ Now, for any $\varepsilon>0,$ we know that we can find an $N_\ell$ for each coordinate such that $n>N_\ell$ implies
    \[|\pi_\ell x_n-y_\ell|<\frac{\varepsilon}{\sqrt D}.\]
    Then, for $n>\max_\ell\{N_\ell\},$ we have
    \[d(x_n,y)=\sqrt{\sum_{k=1}^D(\pi_\ell x_\ell-y_\ell)^2}<\sqrt{\sum_{k=1}^D\left(\frac{\varepsilon}{\sqrt D}\right)^2}=\sqrt{D\cdot\frac{\varepsilon^2}D}=\varepsilon,\]
    verifying that $x_n\to y$ as $n\to\infty.$
\end{proof}

We can also define bounded; we just give the definition.
\begin{definition}[Bounded]
    A subset $S\subseteq X$ is ``bounded'' if and only if there exists $x_0\in X$ and $r\in R$ such that $x\in S$ implies $d(x,x_0)<r.$ In other words, we can put $S$ in a box.
\end{definition}
One important thing we did in $\RR$ is that every bounded sequence has a convergent subsequence. Is this true in $\RR^n$? Sure: we can find a convergent subsequence for the first coordinate, then find a convergent subsequence of that for the second coordinate, and so on. We'll leave this area with a question.
\begin{ques}
    Is this true for general complete, metric spaces?
\end{ques}
The answer turns out to be no; see \href{https://math.stackexchange.com/a/399927/869257}{here}. In short, we can use the tele-metric; here the sequence
\[1,2,3,\ldots\]
is bounded because all elements of $\RR$ are a distance of $1$ away from (say) $0.$ However, this sequence does not converge because it is not Cauchy: for any $N,$ we can find unequal $n,m>N$ so that $d(n,m)=1.$

\subsection{Open Sets}
We want to generalize ``open'' and ``closed'' intervals in $\RR,$ which connects much deeper inside topology. Starting easy, the open interval $(-1,1)$ seems like it should generalize to the unit circle minus the edge in $\RR^2.$ Explicitly, we want
\[\left\{x\in\RR^2:d(x,0)<1\right\}.\]
So now we're using distances instead of order again. So we have the following definition.
\begin{definition}[Open sphere]
    Fix $X$ a metric space, and take $x\in X$ and $r>0.$ Then we define the \textit{open sphere} centered at $x$ of radius $r$ as
    \[s_r(x)=\{y\in X:d(x,y)<r\}.\]
\end{definition}
We are using ``sphere'' here because we might want to work in very funny metric spaces.

These can be funny. For example, here is a ``sphere'' in the taxicab metric of $\RR^2.$
\begin{center}
    \begin{asy}
        unitsize(1cm);
        fill((0,1)--(1,0)--(0,-1)--(-1,0)--cycle, lightgray);
        draw((0,1)--(1,0)--(0,-1)--(-1,0)--cycle, dashed);
        draw((-1.5,0)--(1.5,0));
        draw((0,-1.5)--(0,1.5));
    \end{asy}
\end{center}
And in the tele-metric, the spheres are points or the full space.

Anyways, we now move towards defining and ``open'' set.
\begin{definition}[Interior]
    Fix $X$ a metric space. Given a subset $S\subseteq X,$ we say that $x\in S$ is in the \textit{interior} of $S$ if and only if there exists $r>0$ such that $s_r(x)\subseteq S.$ The set of interior points is notated $S^\circ.$
\end{definition}
And here we are.
\begin{definition}[Open]
    Fix $X$ a metric space. Then $U$ is an open set if and only if $U$ is its interior.
\end{definition}
\begin{example}
    The open interval $(0,1)$ is open: given any $x\in (0,1),$ we can set $r:=\frac12\min\{x,1-x\}$ so that $s_x(r)\subseteq(0,1).$ So each element of $(0,1)$ is in its interior.
\end{example}
\begin{nex}
    The interval $[0,1)$ is not open: there is no $r$ such that $s_r(0)\subseteq[0,1).$
\end{nex}
Let's actually prove something.
\begin{prop}
    Fix $X$ a metric space. Then $X$ and $\emp$ are both open.
\end{prop}
\begin{proof}
    We do these one at a time.
    \begin{itemize}
        \item For any $x\in X,$ we note that
        \[s_1(x)=\{y\in X:d(x,y)<1\}\subseteq X\]
        by definition. So all points of $X$ are in the interior of $X.$
        \item $\emp$ is open because all elements of $\emp$ are in the interior (and bananas).
        \qedhere
    \end{itemize}
\end{proof}
We also have the following, which we don't do in class (as they are homework).
\begin{prop} \label{prop:opensetops}
    An arbitrary union of open sets is open. A finite intersection of open sets is open.
\end{prop}
\begin{proof}
    This is on the homework. We do remark that intersections are finite because, in some sense, we want the ``smallest'' interior sphere over all of our open sets, but we can only take minimums in finite ses. For example,
    \[\bigcap_{k=1}^\infty(-1/k,1/k)=\{0\}\]
    is not open.
\end{proof}

\subsection{Closed}
Now we can define ``closed.''
\begin{definition}[Closed]
    Fix a metric space $X.$ Then $V\subseteq X$ is \textit{closed} if and only if $X\setminus V$ is open.
\end{definition}
\begin{example}
    The interval $[0,1]$ is closed because its complement is $(-\infty,0)\cup(1,\infty),$ which is the union of two open intervals and hence is open.
\end{example}
\begin{nex}
    The interval $[0,1)$ is not open, from earlier. Also, its complement is $(-\infty,0)\cup[1,\infty)$ is also not open because of $1.$ So $[0,1)$ is neither open nor closed.
\end{nex}
While we're here, we note that we can turn around \autoref{prop:opensetops}.
\begin{prop}
    Any finite union of closed sets is closed. Any arbitrary intersection of closed sets is closed.
\end{prop}
\begin{proof}
    Take the complement of the statements in \autoref{prop:opensetops}.
\end{proof}

In light of the above proposition, we have the following.
\begin{definition}[Closure]
    Fix $X$ a metric space and $S$ a subset. Then we define the \textit{closure} of $S$ to be
    \[\overline S:=\bigcap_{S\subseteq V}V,\]
    where the intersection is over closed sets $V$ containing $S.$
\end{definition}
Because this is an arbitrary intersection of closed sets, $\overline S$ is closed, and we note that $\overline S$ will contain any closed set around $S,$ so $\overline S$ is in some sense the ``smallest'' closed set around $S.$

We have the following result.
\begin{proposition}
    A subset $S\subseteq X$ of a metric space $X$ is closed if and only if $\overline S=S.$
\end{proposition}
\begin{proof}
    We leave this as an exercise.
\end{proof}
What happens when there's some discrepancy between the two?
\begin{definition}[Boundary]
    Fix $X$ a metric space and $S$ a subset. Then we define the \textit{boundary points} of $S$ to be
    \[\del S:=\overline S\setminus S^\circ.\]
\end{definition}
\begin{example}
    If $S=(0,1)$ is closed, $\del S=\{0,1\}.$
\end{example}
Why do we care? We have the following.
\begin{proposition}
    Fix $X$ a metric space and $S$ a subset. Then any $x\in\overline S$ has a sequence of points in $S$ converging to $x.$
\end{proposition}
\begin{proof}
    For any $\varepsilon,$ we claim that find $y\in S$ such that $d(x,y)<\varepsilon.$ Indeed, we show that every $y\in S$ has $d(x,y)\ge\varepsilon,$ then $x\notin\overline S.$ Consider the ball
    \[s_\varepsilon(x)=\{x'\in X:d(x,x')<\varepsilon\}.\]
    By hypothesis, $S\cap s_\varepsilon(x)=\emp,$ so $X\setminus s_\varepsilon(x)$ is a closed set containing $S.$ So
    \[\overline S\subseteq X\setminus s_\varepsilon(x),\]
    which does not contain $x.$ So indeed, $x\notin\overline S.$

    So for each $n\in\NN,$ we can find $x_n\in S$ such that
    \[d(x,x_n)<1/n.\]
    Then we can see that the sequence $\{x_n\}_{n\in\NN}$ converges to $x.$ Indeed, for any $\varepsilon>0,$ we can set $N:=1/\varepsilon$ so that $n>N$ implies
    \[d(x,x_n)<\frac1n<\frac1N=\varepsilon,\]
    which is what we wanted.
\end{proof}

\subsection{Compactness}
We start with the following assertion.
\begin{prop}
    Fix $X$ a complete metric space. Fix $V_1\supseteq V_2\supseteq\cdots$ a descending sequence of nonempty, bounded, closed sets. Then
    \[V:=\bigcap_{k=1}^\infty V_k\]
    is also nonempty, bounded, and closed.
\end{prop}
\begin{proof}
    Closed is by arbitrary intersection. Bounded is by taking a bound of $V_1$ to bound $V.$

    The meat here is showing that $V$ is nonempty. Well, find some $x_k\in V_k$ for each $k.$ Then $\{x_k\}_{k\in\NN}$ lives in a complete metric space and hence has a convergent subsequence, which by abuse of notation we call $\{y_k\}_{k\in\NN}.$ Say it converges to $y.$

    We claim $y\in V.$ The point is that, after long enough, $\{y_k\}_{k\in\NN}$ lives completely in any fixed $V_\bullet,$ so $y\in V_\bullet$ for any fixed $V_\bullet.$ Thus, $y\in V,$ finishing.
\end{proof}
We now define compact and will return to the above shorty.
\begin{defi}[Compact]
    Fix $X$ a metric space and $S\subseteq X.$ Then $S$ is \textit{compact} if and only if, for every open cover $\mathcal U$ on top of $S,$ there exists a finite subcover $\mathcal U_0$ which still covers $S.$
\end{defi}
\begin{nex}
    We have that $\RR$ is not compact. For example, the open cover
    \[\{(-n,n):n\in\NN\}\]
    has no finite subcover: any finite number will miss sufficiently large real numbers.
\end{nex}
\begin{nex}
    The interval $(0,1)$ is not compact. For example,
    \[\{(1/n,1):n\in\NN\}\]
    has no finite subcover: any finite number will miss sufficiently small real numbers.
\end{nex}
\begin{example}
    Ihe interval $[0,1]$ is compact. This is deep, and we will not write out the proof here.
\end{example}
Showing that something is compact is quite difficult: we need to deal with all open covers at once, which is hard to handle. Regardless, there is the following theorem.
\begin{theorem}[Heine--Borel]
    In $\RR^n,$ a subset is compact if and only if it is closed and bounded.
\end{theorem}
This is amazing! Compactness was very hard to handle, but we have good feelings for what closed and bounded should mean.