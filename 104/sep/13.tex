% !TEX root = ../notes.tex












Alright, take two using VS Code. Hopefully this goes better.

\subsection{The Archimedean Property}
Recall the completeness axiom.
\begin{ax}[Completeness]
    Every set of real numbers bounded above as a supremum.
\end{ax}
Recall that this is also equivalent to every set bounded below having an infimum.

Let's relate this to the following.
\begin{proposition}[Density of \texorpdfstring{$\QQ$}{}] \label{prop:denseq}
    Between any two distinct real numbers, there is a rational number.
\end{proposition}
We would like to know why this is true. This turns out to be related to the following.
\begin{proposition}[Archimedean] \label{prop:arch}
    For any $a,b$ positive real numbers, there exists a positive integer $n$ such that $na>b.$
\end{proposition}
In other words, the multiples of $a$ ``just keep going.'' Let's prove this.
\begin{proof}[Proof of \autoref{prop:arch}]
    Suppose for the sake of contradiction that this is false. Then the set
    \[S=\{na:n\in\NN\}\]
    has $b$ as an upper bound: $na\le b$ for each $n\in\NN.$ It follows that $S$ has a supremum, which we name $M.$ In particular, $M-a$ is not an upper bound for $S,$ so there exists some $n_0\in\NN$ such that $n_0a>M-a.$ But then
    \[M<(n_0+1)a\in S,\]
    which contradicts the fact that $M$ was expected to be an upper bound.
\end{proof}
\begin{warn}
    Note that the Archimedean property is not true in all metric spaces. For example, this fails in $\QQ_p.$
\end{warn}
Now, let's use the Archimedean property to show \autoref{prop:denseq}.
\begin{proof}[Proof of \autoref{prop:denseq}]
    This requires some care. The idea is to ``tile'' the real numbers by some rational number less than $b-a>0.$ Indeed, $b-a>0$ implies that there exists some $n\in\NN$ such that $n(b-a)>1.$ (Yes, $1>0.$)

    Thus, $nb>1+na.$ We would like to place an integer between $na$ and $nb.$ Rigorizing this requires some care, so we will not do so here, but picking up $k\in\ZZ$ between $na$ and $nb,$ we see
    \[an<k<bn,\]
    So $a<\frac kn<b.$
\end{proof}
In fact, density of $\QQ$ gives us the following.
\begin{proposition}
    Any real number is the limit of some sequence of rational numbers.
\end{proposition}
\begin{proof}
    Fix $r$ our real number. Then, given any positive integer $n,$ we define $q_n$ as a real number between $r-\frac1n$ and $r+\frac1n.$ Then we see that
    \[\left|r-q_n\right|<\frac1n\]
    for each $n\in\NN.$ So this distance goes to $0,$ yielding convergence.
\end{proof}

This turns out to be quite useful because it gives us some small handle on real numbers: at the very least they are all the limit of some sequence of rational numbers.
\begin{remark}
    Our metric here matters. If we use the lonely metric, where
    \[d(x,y)=\begin{cases}
        1 & x\ne y, \\
        0 & x=y.
    \end{cases}\]
    Here, the only convergent sequences are ones which are eventually constant, so $\QQ$ is not dense in $\RR.$
\end{remark}
Anyways, let's do some example problems.
\begin{proposition}[Ross 4.5]
    Fix $S$ a nonempty subset of $\RR$ with $\sup S\in S.$ Then $\sup S=\max S.$
\end{proposition}
\begin{proof}
    Note $s\in S$ implies that $s\le\sup S$ because $\sup S$ is an upper bound for $S.$ However, $\sup S\in S$ implies that $\sup S=\max S$ as well because maximum is unique.
\end{proof}
\begin{proposition}[Ross 4.11]
    Fix $a,b\in\RR$ with $a<b.$ Prove that there are infinitely many rational numbers in $(a,b).$
\end{proposition}
\begin{proof}
    Suppose that there are only finitely many rational numbers in $(a,b).$ Surely there is at least one, named $q,$ and surely there are at least two because there is a rational number in $(a,q).$

    Now, because there are only finitely many rational numbers in $(a,b),$ this set of rational numbers has a minimum, named $q_0.$ But then we know there is a rational number in $(a,q_0),$ which violates the minimality of $q_0.$
\end{proof}
\begin{remark}
    We can remove the contradiction by actually exhibiting the sequence of rational numbers with $q_0\in (a,b)$ and then recursively defining $q_{k+1}\in(a,q_k).$
\end{remark}

\begin{proposition}[Ross 4.15]
    Suppose that $a\le b+1/n$ for each $n\in\NN.$ Then $a\le b.$
\end{proposition}
\begin{proof}
    We show the contrapositive: suppose $a>b,$ and we show that there exists $n\in\NN$ such that $a\le b+1/n.$ It follows $a-b>0$ so that there exists $n\in\NN$ with $n(a-b)>1,$ which implies $a>b+\frac1n$ for some $n\in\NN.$
\end{proof}

\subsection{Talking \texorpdfstring{$+\infty$ and $-\infty$}{}}
It's going to be convenient to be able to talk about $+\infty$ and $-\infty$ in this class, mostly for the sake of intervals and bounding.
\begin{definition}[Intervals with infinities]
    We define the interval $(a,+\infty):=\{x\in\RR:x>a\}$ and the other intervals with $+\infty$ and $-\infty$ similarly.
\end{definition}
Note that there are dangers here: we cannot really do arithmetic with $\pm\infty.$ Sometimes we can (e.g., $2\cdot+\infty=+\infty$), but sometimes we cannot; for example, what is $0\cdot\infty$?
\begin{warn}
    % Do not write $x\to\infty^+.$ You cannot approach $\infty$ from the right, in the same way that it is redundant to write $x\to\infty^-.$
    Do not write $[5,\infty].$ You cannot have a closed interval of real numbers actually include $\infty.$
\end{warn}
Anyways, what we are getting out of our $\infty$ is full completeness.
\begin{definition}[Supremum and infimum, II]
    Fix $S$ a nonempty set of real numbers. If $S$ is bounded above, we use the definition of $\sup S$ from earlier. Otherwise, we define $\sup S:=+\infty.$

    Similarly, if $S$ is bounded below, we use the definition of $\inf S$ from earlier. Otherwise, we define $\inf S:=-\infty.$
\end{definition}
It follows that every set has a supremum and an infimum.

Something else that $\pm\infty$ does is that it helps us disambiguate what ``limit doesn't exists'' means. For example, being told that a function $f(x)$ has
\[\lim_{x\to0}f(x)\notin\RR\]
could mean all sorts of things: it could be $\infty,-\infty,$ too oscillatory, etc. Being given this information is just one way that we can track of this information.

Anyways, let's do some examples.
\begin{proposition}
    Given nonempty sets $A,B\subseteq\RR,$ we have that $\sup(A+B)=\sup A+\sup B,$ where $A+B:=\{a+b:a\in A\text{ and }b\in B\}.$
\end{proposition}
\begin{proof}
    We have to do casework on the supremums being finite or infinite.
    \begin{itemize}
        \item In one case, suppose that at least one of $A$ or $B$ has infinite supremum. Without loss of generality, $\sup A=+\infty,$ which is equivalent to $A$ not being bounded above. We claim that $\sup(A+B)=+\infty$ as well.
        
        Well, for any real number $r\in\RR,$ and fixing some $b\in B,$ there exists $a\in A$ such that $a>r-b$ because $A$ is not bounded above. But then $a+b>r,$ so $a+b\in A+B$ exceeds any finite bound $r\in\RR.$

        \item Otherwise, we may fix $\alpha=\sup S$ and $\beta=\sup S$ real numbers. Note that $\alpha+\beta$ is an upper bound for $A+B$ because, for any $a\in A$ and $b\in B,$ we have $a\le\alpha$ and $b\le\beta$ so that
        \[a+b\le\alpha+\beta.\]
        It follows that $\sup A+\sup B\ge\sup(A+B)$ because $\sup(A+B)$ is the least upper bound.

        In the other direction, the trick is to show that $\sup(A+B)-\sup A\ge\sup B,$ for which it suffices to show that $\sup(A+B)-\sup A$ is an upper bound for $B.$ Well, for any $b\in B$ and $a\in A,$ we see that
        \[a+b\le\sup(A+B)\]
        definitionally. It follows that $\sup(A+B)-b$ is an upper bound for $A$ always, so $\sup A\le\sup(A+B)-b,$ so $b\le\sup(A+B)-\sup A.$ So indeed, $\sup(A+B)-\sup A$ is an upper bound for $B,$ finishing.
        \qedhere
    \end{itemize}
\end{proof}
\begin{proposition}
    Fix $S\subseteq\RR$ a nonempty subset. Then $\inf S\le\sup S.$
\end{proposition}
\begin{proof}
    Fix some $s\in S.$ Then we have $\inf S\le s$ always (even in case of infinities) as well as $s\le\sup S$ (even in case of infinities), so $\inf S\le\sup S$ by transitivity (even in case of infinities). We won't rigorize this, but it would essentially have to be casework.
\end{proof}

\subsection{Philosophy}
The final section of Chapter 1 in Ross is \S6. Roughly speaking, it has to do with the construction of $\RR$ from Dedekind cuts. Essentially what this does is prove the Completeness axiom for a particular set so that we can be sure that the real numbers exist, but this confidence will not be relevant to our story. So next class we are talking about sequences and convergence.
