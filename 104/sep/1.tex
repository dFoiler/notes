








I went to office hours today; it was pretty fun. I exhibited a continuous, surjective function $[0,1)\to(0,1)$ and felt smart.

\subsection{Philosophy about Induction}
Let's study the sequence
\[S_n=\sum_{k=1}^n\sin k.\]
We would like to have closed form for this series, but it is not at all obvious how to obtain one. And, for example, induction doesn't really help us find such a formula: induction only helps us verify truth, not discover it. This is important to keep in mind.

\subsection{More Classes of Numbers}
From $\NN,$ the next class we care about is $\ZZ:=\{a-b:a,b\in\NN\}.$ Then we have $\QQ:=\{p/q:p,q\in\QQ\}.$ Formally, we define
\[\QQ:=\frac{\left\{(p,q)\in\ZZ^2:q\ne0\right\}}{\{(p_1,q_1)\sim(p_2,q_2):\exists c,d:cp_1=cp_2\text{ and }dq_1=dq_2\}}.\]
Alternatively, a number is rational if and only if its decimal expansion is eventually periodic.

Then there are numbers which are not rational: for example, $\sqrt2\notin\QQ.$ We would like to prove this; let's start with a smaller question.
\begin{prop}
    There does not exist an integer $n$ for which $n^2=50.$
\end{prop}
\begin{proof}
    We can bound $7^2<50<8^2,$ so $7<\sqrt{50}<8.$
\end{proof}
Why can't we try the same thing for $\QQ$? Well, there's just too many (infinitely many) rational numbers between any given two rational numbers. This is not a finite computation here: $\ZZ$ is nice because it is discrete.

Anyways, let's prove $\sqrt2\notin\QQ.$
\begin{prop} \label{prop:sqrt2}
    We show $\sqrt2\notin\QQ.$
\end{prop}
\begin{proof}
    Suppose for the sake of contradiction $\sqrt2\in\QQ$ so that $\sqrt2=p/q$ such that $p,q\in\NN$ and $\gcd(p,q)=1$ (the fraction is reduced). Technically, it suffices to assume that at least one of $m,n$ is odd. But now $q\sqrt2=p$ implies that
    \[p^2=2q^2.\]
    Now, $p^2$ is even, so $p$ is even, so $p=2p_0.$ But then we can rearrange to
    \[2p_0^2=q^2,\]
    so $q^2$ is even, so $q$ is even. However, this violates our assumption that $p$ and $q$ were both even, which is a contradiction.
\end{proof}
\begin{remark}
    The above more or less convinces us that $\RR$ is a useful thing to look at: it has numbers like $\sqrt2$ which we want but are not immediately accessible via $\QQ.$
\end{remark}
Similar logic could show that $\sqrt3$ or $\sqrt[3]2$ are irrational.

After $\QQ,$ the next class that we care about is the set of algebraic numbers.
\begin{defi}[Algebraic]
    We say that a real number $\alpha$ is \textit{algebraic} if and only if there exists a polynomial $p(x)\in\ZZ[x]$ such that $p(\alpha)=0.$
\end{defi}
\begin{ex}
    We see that all rational numbers are algebraic: for $\alpha:=a/b\in\QQ,$ we see that $\alpha$ is a root of $p(x):=bx-a\in\ZZ[x].$
\end{ex}
\begin{ex}
    We see that there are algebraic numbers which are not rational: for $\alpha:=\sqrt2\notin\QQ,$ we see that $\alpha$ is a root of $p(x):=x^2-2\in\ZZ[x].$
\end{ex}
So we see that $\QQ\subsetneq\AA.$ In fact, most of our friends are algebraic, such as $\sqrt[3]6$ or $\sqrt{\sqrt[3]5-\sqrt[4]2}$ similar.

\subsection{Rational Root Theorem}
The super-powered version of \autoref{prop:sqrt2} is the Rational root theorem, which turns sieving through $\QQ$ into a finite computation.
\begin{thm}[Rational root] \label{thm:rrt}
    Fix a polynomial
    \[f(x)=\sum_{k=0}^nc_kx^k\in\ZZ[x]\]
    with $c_n\ne0$ and $c_0\ne0.$ Now suppose that $q=a/b$ is a root of $f(x),$ where $a,b\in\ZZ$ and $\gcd(a,b)=1.$ Then we claim $a\mid c_n$ and $b\mid c_0.$
\end{thm}
\begin{proof}
    The main point is to plug in $a/b$ into $f(x).$ We see that
    \[0=\sum_{k=0}^nc_k\left(\frac ab\right)^k,\]
    so
    \[0=\sum_{k=0}^nc_ka^kb^{n-k}.\]
    Now, the idea is to isolate
    \[c_na^n=b\cdot\sum_{k=0}^{n-1}-c_ka^kb^{n-k-1},\]
    so it follows $b\mid c_na^n.$ However, $\gcd(a,b)=1$ now forces $\boxed{b\mid c_n}.$
    
    For the other divisibility, we similarly isolate
    \[c_0b^n=a\cdot\sum_{k=1}^n-c_ka^{k-1}b^{n-k},\]
    so $a\mid c_0b^n.$ But again, $\gcd(a,b)=1$ forces $\boxed{a\mid c_0},$ finishing.
\end{proof}
\begin{cor} \label{cor:rrt}
    Suppose $\alpha\in\QQ$ is the root of a monic polynomial $f(x)\in\ZZ[x].$ Then $\alpha\in\ZZ.$
\end{cor}
\begin{proof}
    Writing $\alpha=a/b$ as a reduced fraction, we see that we must have $b\mid\pm1$ by \autoref{thm:rrt}, so $b=\pm1.$ Thus, $\alpha=\pm a\in\ZZ,$ finishing.
\end{proof}
This is surprisingly powerful. In particular, the Rational root theorem gives us a way to determine if an algebraic integer is rational by turning it into a finite computation: we only have to check the rational numbers with a bounded numerator or denominator. Here are some examples.
\begin{ex}
    We show that $\sqrt[3]6\notin\QQ.$ Well, it's a root of the monic polynomial
    \[x^3-6,\]
    so $\sqrt[3]6\in\QQ$ implies $\sqrt[3]6\notin\ZZ,$ which is false because $2<\sqrt[3]6<3.$
\end{ex}
\begin{cor}
    Suppose that $n\in\ZZ$ has $\sqrt n\notin\ZZ.$ Then $\sqrt n\notin\QQ.$
\end{cor}
\begin{proof}
    We show the contrapositive. Indeed, $\sqrt n\in\QQ$ would imply $\sqrt n\in\ZZ$ by \autoref{cor:rrt}: $\sqrt n$ is a root of the monic polynomial $x^2-n.$
\end{proof}
\begin{ex}
    We show that $\alpha:=\sqrt{2+\sqrt2}\notin\QQ.$ Note that $\alpha^2=2+\sqrt2,$ so $\left(\alpha^2-2\right)^2=2.$ Thus, $\alpha$ is a root of the (monic) polynomial
    \[f(x)=\left(x^2-2\right)^2-2=x^4-4x^2+2.\]
    By \autoref{cor:rrt}, we see that any rational root is in $\{\pm1,\pm2\},$ but we can check by hand that none of them work.
\end{ex}
\begin{ex}
    We note that $\alpha:=\sqrt{4+2\sqrt3}-\sqrt3$ looks very irrational, but it's actually $1.$ Note that $\left(1+\sqrt3\right)^2=4+2\sqrt3,$ so $\alpha=(1+\sqrt3)-\sqrt3=1.$
    
    If we wanted to, we could write out a polynomial for the sake of completeness. Note $\alpha+\sqrt3=\sqrt{4+2\sqrt3},$ so
    \[\alpha^2+2\alpha\sqrt3+3=4+2\sqrt3.\]
    Then $2\sqrt3(\alpha-1)=1-\alpha^2,$ and we can square both sides to get the polynomial $12(\alpha-1)^2-\left(\alpha-1\right)^2=0.$
\end{ex}
Note that even our attempt to find a polynomial, which comes from trying to show that $\sqrt{4+2\sqrt3}-\sqrt3$ is irrational, we can look at the polynomial we created, named
\[12(x-1)^2-\left(x-1\right)^2,\]
has a double root of $x=1,$ so this guides us that maybe the number we were looking at what $1.$

\subsection{Transcendental Talk}
People spent a while thinking that $\QQ$ has everything we could ever want. Now we are brought to place where $\AA$ seems to have everything we could ever want. However, there are real numbers which are not algebraic. The numbers $\pi$ and $e$ are the typical examples, but how about $2^{\sqrt2}$? This is known as the Gelfond--Schneider constant, and it turns out to be not algebraic as well.

The simplest example to prove is
\[L:=\sum_{k=1}^\infty\frac1{10^{n!}}.\]
(This is a ``Liouville number.'') The problem here is that $L$ has really great rational approximations, even better than any algebraic number could hope for. This sum even provides us a sequence (via the partial sums) that are each algebraic (in fact, rational) that converges to a transcendental number.

So it looks like we are repeatedly able to have sequences which seem to converge (in some decimal expansion sense) but are not converging to a number in our set: it happened for $\QQ,$ and now it happened for $\AA.$ So now we have the real numbers to fix this problem.
\begin{defi}[Real numbers]
    We define the real numbers as any decimal expansion.
\end{defi}
The point here is that we could define $\QQ$ as eventually periodic decimals, and then algebraic numbers also have good decimal expansions (or at least rational approximations), but we hope that adding in all real numbers---all decimal expansions---we note that all of our sequences which converge (in some decimal expansion sense) will actually converge to a real number. This is more or less what it means to converge (in some decimal expansion sense).
\begin{ex}
    There are a lot of real numbers. For example, $0.1234567891011121314151617\ldots$ is not a rational number: its decimal expansion can never repeat. (If it repeated, then the form of our natural numbers would be too restricted.) However, this number is not very useful.
\end{ex}
\begin{ex}
    Another real number we don't care about is
    \[\sum_{k=1}^n\frac1{10^{k^2}}\]
    is also irrational because its decimal expansion is again not eventually periodic: for any period length, we can always find a string of constant zeroes of length twice the period, so the period must only consist of zeroes, but this number has $1$s as far down in the decimal expansion as we please.
\end{ex}
\begin{warn}
    We make the first homework due on September 12th.
\end{warn}