% !TEX root = ../notes.tex














Here we go again.

\subsection{Clarifications}
We quickly return to some terminology. For a sequence $\{a_k\}_{k\in\NN}$ of real numbers, we say that the sequence ``has a limit'' if
\[\lim_{n\to\infty}a_n\in\RR\cup\{\pm\infty\},\]
and $\{a_k\}_{k\in\NN}$ ``converges'' if
\[\lim_{n\to\infty}a_n\in\RR.\]
The ambiguous terminology is what it means for a limit to ``exist.'' Does ``exist'' force $\RR$ or $\RR\cup\{\pm\infty\}$?
\begin{warn}
	For this course, we will say that the limit of a sequence $\{a_k\}_{k\in\NN}$ \textit{exists} if and only if
	\[\lim_{n\to\infty}a_n\in\RR\cup\{\pm\infty\}.\]
\end{warn}
So this is another thing we have to keep track of. The main distinction here matters most on the homework; on exams, just ask the professor what is meant.

\subsection{Monotonic Sequences}
The idea here is that sequences whose terms ``get closer'' as the sequence goes on should mean that the sequence should converge in $\RR.$ However, this is a bit annoying to prove because actually finding the limit is a bit annoying, and we still need to rigorize ``get closer.''

To rigorize ``get closer,'' we begin by talking about monotonic sequences.
\begin{defi}[Flavors of monotonic]
	Fix $\{a_k\}_{k\in\NN}$ a sequence of real numbers.
	\begin{itemize}
		\item We say $\{a_k\}_{k\in\NN}$ is \textit{increasing} if $a_k\le a_{k+1}$ for each $k.$
		\item We say $\{a_k\}_{k\in\NN}$ is \textit{strictly increasing} if $a_k<a_{k+1}$ for each $k.$
		\item We say $\{a_k\}_{k\in\NN}$ is \textit{decreasing} if $a_k\ge a_{k+1}$ for each $k.$
		\item We say $\{a_k\}_{k\in\NN}$ is \textit{strictly decreasing} if $a_k>a_{k+1}$ for each $k.$
	\end{itemize}
	If any of the above are satisfied, we say that $\{a_k\}_{k\in\NN}$ is \textit{monotonic}.
\end{defi}
Here is an important result.
\begin{theorem} \label{thm:boundmon}
	Any bounded, monotonic sequence in $\RR$ converges.
\end{theorem}
\begin{proof}
	Let's say $\{a_k\}_{k\in\NN}$ is bounded and increasing (without loss of generality). Note the sequence is upper-bounded and hence has a supremum, which we name $a.$ We show that
	\[\lim_{n\to\infty}a_n\stackrel?=a.\]
	Well, take any $\varepsilon>0.$ Then $(a-\varepsilon,a]$ must contain some $a_N,$ for otherwise $a_k\le a-\varepsilon<a$ for each $a_k,$ so $a-\varepsilon$ would be a lesser upper bound than $a.$

	Now, for each $n>N,$ we have
	\[a-\varepsilon<a_N<a_n\le a,\]
	so it follows that $|a_n-a|<\varepsilon,$ completing the proof.
\end{proof}
So the above result is nice, but it's pretty restricted in scope: most sequences are not going to be monotonic. Of course, the above logic still works if our sequence is ``eventually'' monotonic, but still most sequences are not monotonic.

However, if we expand our view, we can ask if every sequence has some monotonic subsequence, which would mean that every sequence has some notion of getting close to somewhere. Sure, it's possible that different subsequences converge to different places, which is annoying, but it's still information. At a high level, our goal is to describe how sequences approach numbers, and keeping track of multiple limits is one way this happens.
\begin{remark}
	\autoref{thm:boundmon} fails in $\QQ$ because we don't have completeness: $3,3.1,3.14,3.141$ is a bounded sequence of rational numbers which does not converge to a rational number.
\end{remark}
\begin{exercise}
	Consider the sequence $\{s_k\}_{k\in\NN}$ defined recursively by $s_1=1$ and
	\[s_{n+1}=\sqrt{2+s_n}\]
	for $n\in\NN.$ Then we can prove $\lim s_n=2.$
\end{exercise}
\begin{proof}
	The proof is in two steps.
	\begin{enumerate}
		\item We show that the limit exists, also in two steps: we show that $\{s_k\}_{k\in\NN}$ is bounded and monotonic.
		\begin{enumerate}
			\item We claim that $1\le s_n\le 2$ for each $n,$ by induction. Of course this is true for $n=1.$ Then, if $1\le s_n\le 2,$ it follows
			\[3\le2+s_n\le4,\]
			so
			\[1\le\sqrt3\le s_{n+1}\le2,\]
			which finishes.
			\item We show that $\{s_k\}_{k\in\NN}$ is increasing. The trick is to show that
			\[\sqrt{2+x}\stackrel?\ge x\]
			for any $x\in[1,2].$ Well, this is equivalent to $x+2\ge x^2,$ which is equivalent to $(x+1)(x-2)\le0,$ which is true on $x\in[-1,2]$ because $x+1\ge0$ and $x+2\le0$ here.

			It follows that $s_{n+1}=\sqrt{2+s_n}\ge s_n$ from the bounding above.
		\end{enumerate}
		\item Now let the limit be $L$ because we know it exists. Then taking both sides
		\[s_{n+1}=\sqrt{2+s_n}\]
		as $n\to\infty,$ we find $L=\sqrt{2+L}.$ Thus, $0=L^2-(2-L)=(L+1)(L-2),$ so $L\in\{-1,2\}.$ But the $s_k$ are all larger than $1,$ so $L\ge1,$ so we have $\boxed{L=2}.$
		\qedhere
	\end{enumerate}
\end{proof}
\begin{remark}
	This is quite remarkable! Actually finding this limit by some explicit formula would be quite annoying to do, and indeed, we did not have to.
\end{remark}
\begin{remark}
	\autoref{thm:boundmon} kind of shows that all decimal expansions are real numbers. If we accept that rational numbers are real, then the fact that all bounded, monotonic sequences converge, then
	\[3,3.1,3.14,3.141,\ldots\]
	must converge to some real number, no matter the decimals. (In particular, this sequences is increasing.)
\end{remark}
Here is the other side of \autoref{thm:boundmon}.
\begin{proposition}
	Any unbounded, increasing sequence must go to $\infty.$
\end{proposition}
\begin{proof}
	We show the contrapositive; take $\{a_k\}_{k\in\NN}$ which does not go to $\infty.$ Then there exists some $M$ such that there is no $N$ such that $n>N$ implies
	\[a_n>M.\]
	In particular, pushing our quantifier exchange through, for all $N,$ there is some $n>N$ such that $a_n\le M.$ However, $n>N$ gives $a_n\le a_N,$ so for all $N,$ we conclude that $a_N\le M.$ (Here we used that $\{a_k\}_{k\in\NN}$ is increasing!)
	
	Thus, $\{a_k\}_{k\in\NN}$ is upper-bounded by $M.$ We can take $a_1$ as a lower bound to finish.
\end{proof}
So it follows after some more work that any monotonic sequence ``has a limit.''

\subsection{Introducing \text{limsup} and \text{liminf}}
We hope to use monotone sequences to gain some control on all sequences. Here is one idea: fix $\{s_k\}_{k\in\NN}$ a sequence of real numbers and define
\[t_N:=\sup\{s_n:n>N\}.\]
Note that it's possible for $t_N$ to be $\infty,$ and in fact, if one of them is $\infty,$ then all of them are $\infty.$\footnote{Roughly speaking, this is because the ``finite truncation'' that $t_N$ does will not mess with the long-term behavior of $\{s_k\}_{k\in\NN}.$}

So for now, we fix $\{s_k\}_{k\in\NN}$ to be bounded above so that each $\{s_n:n>N\}$ is also bounded above, so each $t_N$ is a real number. Now we consider
\[\lim_{N\to\infty}t_N.\]
This certainly ``exists'' because $\{t_N\}_{N\in\NN}$ is decreasing and hence monotonic: the fact that $N_1\le N_2$ implies $\{s_n:n>N_2\}\subseteq\{s_n:n>N_1\}$ implies $t_{N_1}\le t_{N_2}$ because $t_{N_1}$ will upper-bound $\{s_n:n>N_2\}.$

This gives us the following definition.
\begin{definition}[\texorpdfstring{$\limsup$ and $\liminf$}{}]
	Fix $\{s_k\}_{k\in\NN}$ a sequence of real numbers. Then we define
	\[\limsup_{n\to\infty}s_k:=\lim_{N\to\infty}\sup\{s_n:n>N\}.\]
	Similarly,
	\[\liminf_{n\to\infty}s_k:=\lim_{N\to\infty}\inf\{s_n:n>N\}.\]
\end{definition}
\begin{warn}
	If $\{s_k\}_{k\in\NN}$ is not bounded above, then $\limsup$ will be $+\infty.$ If $\{s_k\}_{k\in\NN}$ is something like $s_k=-k,$ then 
\end{warn}
Regardless of the above warning, we do know that $\limsup$ and $\liminf$ will always exist from the preceding discussion.

Let's do an example.
\begin{proposition}
	Suppose that $\{s_k\}_{k\in\NN}$ is a sequence with $s_k\to s$ as $k\to\infty.$ Then we claim that
	\[\limsup s_n=\liminf s_n=s.\]
\end{proposition}
\begin{proof}
	We show that $\limsup s_n=s,$ and the $\liminf$ argument is similar. We also won't talk about $s\in\{\pm\infty\}$: $s=+\infty$ just means that $s_n$ is not bounded above, and $s=-\infty$ can be done with some effort as well.

	So let's say $s\in\RR.$ Fix any $\varepsilon>0.$ Then there is some $N$ for which $n>N$ implies $s-\varepsilon<s_n<s+\varepsilon,$ which implies that
	\[\sup\{s_n:n>N\}\in(s-\varepsilon,s+\varepsilon).\]
	This shows that $\lim_{N\to\infty}\sup\{s_n:n>N\}=s.$
\end{proof}
In fact, the converse is also true.
\begin{proposition}
	Suppose that $\{s_k\}_{k\in\NN}$ is a sequence with
	\[\limsup_{n\to\infty}s_n=\liminf_{n\to\infty}s_n=s.\]
	Then $s_k\to s$ as well.
\end{proposition}
\begin{proof}
	We outline. Again, we ignore $\pm\infty,$ though they are hard. The idea is that, for any $\varepsilon>0,$ there is some $N_1$ and $N_2$ such that $N>\max\{N_1,N_2\}$ implies
	\[s-\varepsilon<\inf\{s_n:n>N\}\le\sup\{s_n:n>N\}<s+\varepsilon.\]
	Then any $n>N$ implies $s-\varepsilon<s_n<s+\varepsilon$ as well, finishing.
\end{proof}
In general, the best we can say is that
\[\inf\{s_n:n>N\}\le\sup\{s_n:n>N\},\]
so it follows that $\limsup s_n\ge\liminf s_n.$

\subsection{Cauchy Sequences}
We are beginning to realize our goal of tracking how sequences converge. For example, if we are told that $\{s_k\}_{k\in\NN}$ has $\limsup s_k=1$ and $\liminf s_k=0.$ Then, roughly speaking, $\{s_k\}_{k\in\NN}$ should ``oscillate'' in some sense between $0$ and $1,$ even in the long term.

So, for example, $\{s_k\}_{k\in\NN}$ has infinitely many terms at least $3/4$ and at most $1/4,$ so there are infinitely many pairs with distance at most $1/2.$

With this in mind, the next big idea here is that of a Cauchy sequence.
\begin{definition}[Cauchy sequence]
	Fix $\{s_k\}_{k\in\NN}$ a sequence of real numbers. We say that $\{s_k\}_{k\in\NN}$ is a \textit{Cauchy sequence} if and only if, for each $\varepsilon>0,$ there exists some $N$ such that $n,m>N$ implies
	\[|s_n-s_m|<\varepsilon.\]
\end{definition}
The advantage here is that Cauchy sequences let us focus locally on the the terms only without having to ``guess'' a limit as we've been doing. However, this is somewhat worse because we now have two indices $n$ and $m$ to keep track of.

We note the following.
\begin{proposition}
	Any converging sequence is a Cauchy sequence.
\end{proposition}
\begin{proof}
	Suppose that $\{s_k\}_{k\in\NN}$ converges to some $s\in\RR.$ Then, for any $\varepsilon,$ there exists some $N$ such that $n>N$ implies
	\[|s_n-s|<\varepsilon/2.\]
	This factor of $1/2$ is a trick that will help us momentarily. Indeed, for $n,m>N,$ it follows
	\[|s_n-s_m|\le|s_n-s|+|s-s_m|<\frac\varepsilon2+\frac\varepsilon2=\varepsilon,\]
	which proves $\{s_k\}_{k\in\NN}$ is converging.
\end{proof}
\begin{remark}
	The above proof does not use the completeness axiom, so this is also true in, say, $\QQ.$
\end{remark}
What about the converse? Indeed, it is true.
\begin{proposition}
	Any Cauchy sequence converges to some real number.
\end{proposition}
\begin{proof}
	We outline; fix $\{s_k\}_{k\in\NN}$ a Cauchy sequence. The trick is to show that
	\[\limsup_{n\to\infty}s_n=\liminf_{n\to\infty}s_n,\]
	which roughly holds because, after long enough, the $\sup\{s_n:n>N\}$ and $\inf\{s_n:n>N\}$ need to get closer together because of the Cauchy condition.
\end{proof}
\begin{remark}
	The above proof does use completeness, and it is not true in $\QQ,$ using the typical examples. It is an interesting question which kinds of spaces have that all Cauchy sequences converge.
\end{remark}

Let's do an exercise.
\begin{exercise}[Ross 10.7]
	Fix $S\subseteq\RR$ a bounded, nonempty subset such that $\sup S\notin S.$ Then there is a sequence $\{s_k\}_{k\in\NN}\subseteq S$ which converges to $\sup S.$
\end{exercise}
\begin{proof}
	Note that, for each $k\in\NN,$ we can fix $\varepsilon=1/k$ and find some $s_k\in S$ such that $\sup S-1/k<s_k$ because $\sup S-1/k$ cannot be an upper bound. Then we can show
	\[\lim_{k\to\infty}s_k=\sup S\]
	in the usual manner. The point is that
	\[\sup S-\frac1k<s_k<\sup S+\frac1k,\]
	so we can take $N=1/\varepsilon$ in our formal proof. We won't write this out fully.
\end{proof}
\begin{remark}
	The above proof is not very interesting, but it's leading towards the following question: if $\{s_k\}_{k\in\NN},$ can we find a subsequence converging to $\limsup s_k$?
\end{remark}
\begin{exercise}[Ross 10.11]
	Define the sequence $\{t_k\}_{k\in\NN}$ by $t_1:=1$ and
	\[t_{n+1}=\left(1-\frac1{4n^2}\right)t_n\]
	for $n\in\NN.$
\end{exercise}
\begin{proof}
	We can show inductively that this sequence is positive, and it is decreasing because $1-\frac1{4n^2}<1.$
\end{proof}
It's an interesting question to evaluate this limit. I think it turns out to be $2/\pi$ due to some kind of Weierstrass factorization argument.