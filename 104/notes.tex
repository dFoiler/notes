% LTeX: enabled=false

\documentclass[openany]{book}
\usepackage[utf8]{inputenc}

\usepackage{import}
\inputfrom{..}{nir}

\pagestyle{contentpage}

\title{104: Introduction to Analysis}
\author{Nir Elber}
\date{Fall 2021}
\rhead{\textit{104: INTRO. TO ANALYSIS}}

\begin{document}

\maketitle

\toctrue
\tableofcontents
\tocfalse

\newpage

\chapter{Pinning Down the Reals}

\foreach \n in {25,30}
{
	\section{August \n}
	\input{aug/\n}
}

\foreach \n in {1,8,13}
{
	\section{September \n}
	\input{sep/\n}
}

\chapter{Convergence}

\foreach \n in {15,20,22,27,29}
{
	\section{September \n}
	\input{sep/\n}
}

\foreach \n in {4,6}
{
	\section{October \n}
	\input{oct/\n}
}

\chapter{Continuity}

\foreach \n in {13,18,20,25,27}
{
	\section{October \n}
	\input{oct/\n}
}

\foreach \n in {1,3}
{
	\section{November \n}
	\input{nov/\n}
}

\chapter{Differentiation}

\foreach \n in {10,15,17}
{
	\section{November \n}
	\input{nov/\n}
}

\chapter{Integration}

\foreach \n in {22,29}
{
	\section{November \n}
	\input{nov/\n}
}

\foreach \n in {1}
{
	\section{December \n}
	\input{dec/\n}
}

% role of choice in analysis?
% \todo{} does convergence become Cauchy for free on compact intervals?
% \todo{} isn't this essentially just defining Cauchy sequences in C(R)?
% converging on a compact interval => uniformly Cauchy?

% general recipe for nowhere differentiable functions?

\nirprintindex

\end{document}