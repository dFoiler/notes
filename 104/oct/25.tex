\documentclass[../notes.tex]{subfiles}

\begin{document}

% !TEX root = ../notes.tex














Let's have some fun today.

\subsection{Limits}
We're going to talk about limits, with many of the same ideas from calculus or sequences. For example, they prove in the book the limit laws and so on, and we will not do all of these formally. Regardless, here is our definition.
\begin{definition}[Limits]
	Fix $f:S\to\RR.$ We say that the \textit{limit of $f(x)$ as $x\to a$ is $L$ along $S$}, notated
	\[\lim_{x\to a^S}f(x)=L,\]
	if and only if for every sequence $\{a_n\}_{n\in\NN}\subseteq S$ converging to $a$ has $f(a_n)$ converge to $L.$
\end{definition}
Note that we have added a subset $S$ to our definition. One reason this is a good thing to do is that it lets us talk about limits of functions which are not defined over $\RR.$ For example, it's not that the limit
\[\lim_{x\to-\infty}\sqrt x\]
does not exist---the limit does not even make sense because the interesting values with $x\to-\infty$ aren't defined for $\sqrt x.$ If we let $S:=\RR_{\ge0},$ then
\[\lim_{x\to-\infty^S}\sqrt x\]
now at least will compile, though $-\infty$ perhaps doesn't make sense with this $S.$ Similarly,
\[\lim_{x\to\infty^S}\frac{\tan x}{x(|\tan x|+1)}\]
where $S$ is the domain of $\tan x$ will at least make sense and equal $0,$ though without the $S$ here, things make less sense.
\begin{remark}
	We have that
	\[\lim_{x\to a^+}f(x)=\lim_{x\to a^{(a,\infty)}}f(x),\]
	so our limits generalize left and right limits. This also explains our notation.
\end{remark}
As usual, we can modify our sequences definition of the limit to an $\varepsilon$-$\delta$ definition.
\begin{proposition}
	Fix $f:S\to\RR.$ We have that
	\[\lim_{x\to a^S}f(x)=L\]
	if and only if, for each $\varepsilon>0,$ there exists $\delta>0$ such that each $x\in S$ satisfying $|x-a|<\delta$ have $|f(x)-L|<\varepsilon.$
\end{proposition}
\begin{proof}
	We have two implications.
	\begin{itemize}
		\item Suppose that, for each $\varepsilon>0,$ there exists $\delta>0$ such that $x\in S$ has $|x-a|<\delta$ implying $|f(x)-L|<\varepsilon.$ Now, take any sequence $\{a_n\}_{n\in\NN}\subseteq S$ which converges to $a\in S.$

		Then, for any $\varepsilon>0,$ there exists $\delta$ such that $x\in S$ has $|x-a|<\delta$ implying $|f(x)-L|<\varepsilon.$ However, there is an $N$ for which $n>N$ implies $|a_n-a|<\delta,$ so this $N$ has $n>N$ implies $|f(a_n)-f(a)|<\varepsilon$ as well.

		\item Conversely, suppose that there is $\varepsilon>0$ for which no $\delta>0$ has $|x-a|<\delta$ implies $|f(x)-f(a)|<\varepsilon.$ Then, for each $n\in\NN,$ there is some $x_n$ with $|x_n-a|<1/n$ and $|f(x_n)-f(a)|\ge\varepsilon.$

		But now the sequence $x_n$ converges to $a$ while $f(x_n)$ never goes within $\varepsilon$ of $f(a),$ so $f(x_n)$ does not converge to $f(a).$ So it follows $\lim_{x\to a^S}f(x)$ is not $L.$
		\qedhere
	\end{itemize}
\end{proof}
As an example, let's show the following.
\begin{proposition}
	Fix $f:\RR\to\RR.$ We have that $\lim_{x\to a}f(x)=L$ if and only if
	\[\lim_{x\to a^+}f(x)=\lim_{x\to a^-}f(x)=L.\]
\end{proposition}
\begin{proof}
	We have two implications here.
	\begin{itemize}
		\item Suppose that $\lim_{x\to a}f(x)=L.$ We will show $\lim_{x\to a^+}f(x)=L,$ and the other limit is similar. Now, for any $\varepsilon>0,$ there exists $\delta>0$ such that $x\in\RR$ has
		\[|x-a|<\delta\implies|f(x)-f(a)|<\varepsilon.\] 
		Then for any $a\in(a,\infty),$ we have $|x-a|<\delta$ implies $|f(x)-f(a)|<\varepsilon$ still, finishing.
		\item Suppose that $\lim_{x\to a^+}f(x)=\lim_{x\to a^-}f(x)=L.$ Then, for any $\varepsilon>0,$ there exists $\delta^+>0$ such that $x\in(a,\infty)$ has
		\[|x-a|<\delta^+\implies|f(x)-f(a)|<\varepsilon.\]
		Similarly, there exists $\delta^->0$ such that $x\in(-\infty,a)$ has the same. Then set $\delta:=\min\{\delta^+,\delta^-\}.$ Then because $x\in\RR$ implies $x=a$ or $x\in(a,\infty)$ or $x\in(-\infty,a),$ we have in all cases that
		\[|x-a|<\delta\implies|f(x)-f(a)|<\varepsilon.\]
		This is what we wanted.
		\qedhere
	\end{itemize}
\end{proof}
\begin{remark}
	It is possible to do the first implication with ideas from sequences: any sequence approaching $a$ from the left will be some sequence and hence have the output converging to $L.$
\end{remark}
To finish, let's do an exercise.
\begin{exercise}
	Suppose $f_1,f_2,f_3\in(a,b)$ with $f_1(x)\le f_2(x)\le f_3(x)$ on the domain. Then
	\[\lim_{x\to a^+}f_1(x)=\lim_{x\to a^+}f_3(x)=L\]
	implies
	\[\lim_{x\to a^+}f_2(x)=L.\]
\end{exercise}
\begin{proof}
	There might be ways to do this with sequences, but we can do this with $\varepsilon$-$\delta$ style ideas. Fix $\varepsilon>0$ so that there exists $\delta_1>0$ such that $x\in(a,b)$ has
	\[|x-a|<\delta_1\implies|f_1(x)-L|<\varepsilon.\]
	Similarly, there exists $\delta_3>0$ such that $x\in(a,b)$ has
	\[|x-a|<\delta_2\implies|f_3(x)-L|<\varepsilon.\]
	Now, as usual, take $\delta:=\min\{\delta_1,\delta_2\}.$ Then any $x\in(a,b)$ will have
	\[-\varepsilon<f_1(x)-L\le f_2(x)-L\le f_3(x)-L<\varepsilon\]
	by using our definitions of $\delta_1$ and $\delta_3$ on the left and right. This finishes.
\end{proof}

\subsection{Upgrading to Metric Spaces}
Let's start by moving continuity up to a metric space. The main point is that the condition
\[|x-a|<\delta\implies|f(x)-f(a)|<\varepsilon\]
can be moved into a continuity condition by noting that $|x-a|$ is really our distance metric. So here is our definition.
\begin{definition}[Continuity for metric spaces]
	Fix $(X,d_X)$ and $(Y,d_Y)$ metric spaces. Then a function $f:X\to Y$ is \textit{continuous} at $a\in X$ if and only if, for each $\varepsilon>0,$ we have some $\delta>0$ such that $x\in X$ has
	\[d_X(x,a)<\delta\implies d_Y(f(x),f(a))<\varepsilon.\]
\end{definition}
\begin{example}
	Using the telemetric on $\RR,$ all functions are continuous. The main point is that for any $\varepsilon>0,$ we can take $\delta=1/2$ so that $d_X(x,a)<\delta$ implies $x=a.$ Alternatively, any sequence converging to $a$ must be eventually constant, so it of course lifts to a sequence converging to $f(a)$ upon pushing through $f.$
\end{example}
\begin{example}
	The taxicab metric on $\RR^2,$ continuity actually looks the same as for Euclidean continuity $\RR^2\to\RR^2$! Namely, the metrics induce the same topology on $\RR^2,$ though the taxicab metric does look different.
\end{example}
And we can go to uniformly continuous by moving around our quantifiers, as we did in $\RR.$
\begin{definition}[Uniform continuity for metric spaces]
	Fix $(X,d_X)$ and $(Y,d_Y)$ metric spaces. Then a function $f:X\to Y$ is \textit{uniformly continuous} if and only if, for each $\varepsilon>0,$ we have $\delta>0$ such that $x,y\in X$ has
	\[d_X(x,y)<\delta\implies d_Y(f(x),f(y))<\varepsilon.\]
\end{definition}
Let's build towards a more topological definition of continuity.
\begin{definition}[Pre-image]
	Let $f:S\to T$ be a function. Then we define the \textit{pre-image}, for $B\subseteq T,$
	\[f^{-1}(B):=\{s\in S:f(S)\in B\}.\]
\end{definition}
\begin{warn}
	This $f^{-1}$ is not the same as an inverse function! Here, $f^{-1}:\mathcal P(T)\to\mathcal P(S)$ is defined for arbitrary functions (not necessarily one-to-one).
\end{warn}
% \todo{actually read S21 oops}

\subsection{A Better Continuity}
And here is our topological definition of continuity.
\begin{theorem}
	Fix $(X,d_X)$ and $(Y,d_Y)$ metric spaces. We have that $f:X\to Y$ is continuous if and only if, $f^{-1}$ sends open sets of $Y$ to open sets of $X.$
\end{theorem}
\begin{proof}
	We have two implications.
	\begin{itemize}
		\item Let's start by taking $f$ continuous. Fix $U\subseteq Y$ an open set so that we want to show $f^{-1}(U)$ is open. Well, find any $a\in f^{-1}(U),$ and we want to show that $a$ is in the interior of $f^{-1}(U).$

		Now, $f(a)\in U$ is in an open set, so there is a $\varepsilon$ such that
		\[\{y\in Y:d_Y(y,f(a))<\varepsilon\}\subseteq U.\]
		By the continuity of $f,$ we are promised $\delta>0$ such that $d_X(x,a)<\delta$ implies $d_Y(f(x),f(a))<\varepsilon.$ It follows that
		\[\{x\in X:d_X(x,a)<\delta\}\subseteq\{x\in X:d_Y(f(x),f(a))<\varepsilon\}\subseteq f^{-1}(U)\]
		because $d_Y(f(x),f(a))<\varepsilon$ implies $f(x)\in U.$ Thus, $a$ is indeed in the interior of $f^{-1}(U),$ finishing.
		\item Now suppose that $f$ takes open sets to open sets. Take $a\in X$ so that we want to show $f$ is continuous at $a.$ Now, for any $\varepsilon>0,$ we note that
		\[B:=\{y\in Y:d_Y(a,y)<\varepsilon\}\]
		is an open set of $Y,$ so it follows that
		\[f^{-1}(B)\]
		is also open. But $a\in X$ has $f(a)\in B,$ so $a\in f^{-1}(B)$ is in the interior of $f^{-1}(B),$ so there is $\delta>0$ such that
		\[\{x\in X:d_X(a,x)<\delta\}\subseteq f^{-1}(B).\]
		It follows that $d_X(a,x)<\delta$ implies $f(x)\in B$ implies $d_Y(f(a),f(x))<\varepsilon,$ which shows that $f$ is indeed continuous at $x=a.$
		\qedhere
	\end{itemize}
\end{proof}
\begin{remark}
	What is good about this definition is that it works nicely and is quite simple for more general topological spaces.
\end{remark}
\begin{remark}
	I think we can actually strengthen the above statement to say that $f$ is continuous at $x=a$ if and only if each open set $U_Y\subseteq Y$ containing $f(a)$ has a pre-image $f^{-1}(U)$ containing some open subset $U_X\subseteq f^{-1}(U)$ containing $a\in U_X.$ The point is that the above proof is very ``local'' at $a.$
\end{remark}
Here is an example of something that we get from this characterization of continuity.
\begin{proposition}
	Fix $f:X\to Y$ a continuous map of metric spaces. Then if $X$ is compact, then $f(X)$ is compact.
\end{proposition}
\begin{proof}
	The idea is that any open cover of $f(X)$ can be pulled back along $f$ to an open cover of $X.$ Then the open cover of $X$ has a finite subcover, which tells us which sets of the open cover of $f(X)$ we ``really need'' to cover.

	Formally, suppose that $\{U_\alpha\}_{\alpha\in\lambda}$ is an open cover of $f(X).$ Then, for each $x\in X,$ we have that $f(x)\in U_\alpha$ for some $\alpha\in\lambda,$ so each $x\in X$ belongs to some $f^{-1}(U_\alpha)$ for some $\alpha.$ It follows that
	\[\mathcal U_X:=\left\{f^{-1}(U_\alpha):U_\alpha\in\lambda\right\}\]
	will fully cover $X,$ and this is in fact an open cover because $f$ is continuous. Now, $\mathcal U_X$ is an open cover of $X,$ so compactness of $X$ promises us some sequence $\{\alpha_k\}_{k=1}^n$ which is yields a finite subcover of $X.$

	To finish, we claim that the $\{U_{\alpha_k}\}_{k=1}^n$ fully covers $f(X).$ Indeed, for any $y\in f(X),$ there is $x$ such that $f(x)=y.$ Now, $x\in f^{-1}(U_{\alpha_k})$ for some $k$ because the $f^{-1}(U_{\alpha_\bullet})$ fully cover $X.$ Thus, $y\in f(x)\in U_{\alpha_k},$ so indeed our finite subcover does cover.
\end{proof}
And as usual, we can show the following.
\begin{proposition}
	Fix $(X,d_X)$ and $(Y,d_Y)$ metric spaces. If $X$ is compact, and if $f:X\to Y$ is continuous, then $f$ is uniformly continuous.
\end{proposition}
\begin{proof}
	Fix $\varepsilon>0.$ Now, for each $a\in X,$ there is some $\delta_a$ such that
	\[\{f(x)\in Y:d_X(a,x)<\delta_a\}\subseteq\{y\in Y:d_Y(y,a)<\varepsilon/2\}.\]
	Now set $U_a:=\{a\in X:d_X(a,x)<\delta_a/2\},$ which is open in $X.$ Then we see that $a\in U_a,$ so
	\[X=\bigcup_{x\in X}\{x\}\subseteq\bigcup_{x\in X}U_x,\]
	so the $U_\bullet$ provide an open cover of $X.$ But now compactness of $X$ implies that we may choose some finite subcover $\{U_{x_k}\}_{k=1}^n$ of $X,$ and because this is finite, we may fix
	\[\delta:=\frac12\min_{1\le k\le n}\delta_k.\]
	Fix $a,b\in X$ with $d_X(a,b)<\delta.$ Because of our open cover, we can place $x_1\in U_k,$ but now
	\[d_X(b,x_k)\le d(a,b)+d(b,x_k)<\delta+\frac12\delta_k<\delta_{x_k}.\]
	But also $d_X(a,x_k)<\delta_{x_k},$ so we see that $d_Y(f(a),f(x_k))<\varepsilon/2$ and $d_Y(f(b),x_k)<\varepsilon/2$ by construction of the $U_\bullet.$ It follows $d_Y(f(a),f(b))<\varepsilon.$
\end{proof}
\begin{remark}
	This is essentially the generalization of the statement that any uniformly continuous function on a closed interval is uniformly continuous.
\end{remark}

Let's close with some exercises.
\begin{exercise}
	Fix $(X,d)$ a metric space, and fix $x_0\in X.$ Then show $f:X\to\RR$ defined by
	\[f(x):=d(x,x_0)\]
	is uniformly continuous.
\end{exercise}
\begin{proof}
	Fix $\varepsilon>0.$ Then we want to define $\delta>0$ such that
	\[d(x,y)<\delta\stackrel?\implies|d(x,x_0)-d(y,x_0)|<\varepsilon.\]
	Well, the point is that
	\[d(x,x_0)\le d(x,y)+d(y,x_0),\]
	so $d(x,x_0)-d(y,y_0)\le d(x,y).$ Similarly, we see that
	\[d(y,x_0)\le d(y,x)+d(x,x_0),\]
	so $d(y,y_0)-d(x,x_0)\le d(x,y).$ It follows that we can take $\delta:=\varepsilon$ so that
	\[d(x,y)<\delta\implies|d(x,x_0)-d(y,y_0)|\le d(x,y)<\delta<\varepsilon,\]
	which is exactly what we wanted.
\end{proof}
\begin{exercise}[Ross 21.9]
	Find a continuous, surjective function $[0,1]^2\onto[0,1].$
\end{exercise}
\begin{proof}
	Consider the function $\pi:[0,1]^2\to[0,1]$ defined by
	\[\pi(x,y):=x.\]
	This is well-defined because the first coordinate does live in $[0,1].$ It remains to check continuity; we show that $\pi$ is uniformly continuous. Fix $\varepsilon>0.$ Now, for any two points $(x_1,y_1),(x_2,y_2)\in[0,1]^2$ take $\delta:=\varepsilon$ so that
	\[\sqrt{(x_1-x_2)^2+(y_1-y_2)^2}<\delta=\varepsilon\]
	implies that $(x_1-x_2)^2<\varepsilon-(y_1-y_2)^2\le\varepsilon^2$ implies that $|x_1-x_2|<\varepsilon.$
\end{proof}

\end{document}