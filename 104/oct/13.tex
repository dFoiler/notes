% !TEX root = ../notes.tex















So a midterm ocurred. I think it went ok.

\subsection{Midterm Notes}
Here are some general comments about the content.
\begin{itemize}
	\item On the first question, many people tried to have $\lim s_n\ne-\infty$ imply $\lim s_n\in\RR\cup\{+\infty\},$ but this is false. Namely, in all interesting cases the limit should not exist at all.
	\item The easiest way to solve the first problem is to note
	\[-\infty=\limsup s_n\ge\liminf s_n\]
	and finish because $\liminf s_n=\limsup s_n=-\infty.$ I didn't know if this was something we could cite.
	\item We have to be somewhat careful with what $\lim s_n\ne-\infty$ means. It tells us that there exists some $M$ for which no $N$ has $n>N$ implies $s_n>M.$ This is not the same thing as saying there exists some $M$ for which $s_n>M$ for each $n.$
	\item On the second question, we have to be careful with what ``$A$ is not open'' means. Pushing quantifiers through, all we get is that there is some $a\in A$ for which all $r>0$ has
	\[\{x\in X:d(a,x)<r\}\not\subseteq A.\]
	Here, $\not\subseteq$ is an annoying symbol to work with.
	\item The class has the hardest time with number 2.
	\item For the third question, some people wrote down
	\[\int_1^\infty\frac1{x^x}\,dx=\sum_{n=1}^\infty\frac1{x^x},\]
	which is untrue. What we know is that they converge together or diverge together.
	\item For the fourth question, the class did best. There was some discussion about ``solving'' for the inductive step.
	\item For the fifth question, the best solution is to make one pair of distinct points have distance $1.$
	\item For the last question, people again did well.
\end{itemize}
Here are some notes on statistics.
\begin{itemize}
	\item Class average was about 96 points out of 120, which is about 80\%.
	\item In particular, the curve will likely be minor.
\end{itemize}

\subsection{Continuity, Advertisement}
As a warning, metric spaces, which people seem to understand only mediocrely, will appear in the future though not in a major sense. For example, we will be talking about continuity for a little while, and perhaps this should change along with our metric. So as our metric changes, we might want to be careful with how our notion of continuity changes.

Anyways, we will work with the normal distance metric on $\RR.$ Let's imagine we're trying to graph some function. When things are continuous, we can just guess a few points and connect the dots. For discontinuities, these tend to stand out in the graph: they might look like jumps or asymptotes or oscillations or similar.

To test for continuity, here is our definition.
\begin{definition}[Continuity, I]
	Fix a function $f:S\to\RR$ for some $S\subseteq\RR.$ Then we say that $f$ is \textit{continuous at $a$} if and only if
	\[\lim_{x\to a}f(x)=f(a).\]
\end{definition}
We will skirt around what $\lim$ means for real numbers. In practice most of our examples are for piecewise functions or just actually continuous, so these limits are fine to evaluate. Namely, for most functions we care about---polynomials, $\exp,$ $\sin,$ etc.---are all continuous.

But sometimes life is not so good.
\begin{example}
	The function
	\[f(x)=\begin{cases}
		1 & x\in\QQ, \\
		0 & x\notin\QQ,
	\end{cases}\]
	is continuous at no $a\in\RR.$
\end{example}
We will be able to prove this shortly. But the point here is that proving this is somewhat obnoxious without rigorous definitions of our terms.

\subsection{Continuity, Rigorously}
So here is our real definition of continuity.
\begin{definition}[Continuity, II]
	We say that a function $f:\RR\to\RR$ is \textit{continuous at $x=a$} if and only if, for each $\varepsilon>0,$ there exists $\delta>0$ such that
	\[|x-a|<\del\implies|f(x)-f(a)|<\varepsilon.\]
\end{definition}
Geometrically, are imagining that we have a given error bound of $\varepsilon$ and an open interval $(f(a)-\varepsilon,f(a)+\varepsilon).$ Then we want to find some small open interval given by $\delta$ for which $(a-\delta,a+\delta)$ will go into the error bound interval.

As an aside, this definition is potentially more annoying to work with than it was for convergence in sequences, but this is because we are now upgrading to reals. Things will get harder; so it goes.

Here is one way to access continuity.
\begin{proposition}
	Suppose that $f(x)$ is continuous at $x=a.$ Then a sequence $\{a_k\}_{k\in\NN}$ converging to $a$ will have $\{f(a_k)\}_{k\in\NN}$ converge to $a.$
\end{proposition}
\begin{proof}
	Fix any $\varepsilon>0.$ By continuity of $f$ at $a,$ there is some $\delta>0$ for which $|x-a|<\delta$ implies $|f(x)-f(a)|<\varepsilon.$ But because $a_\bullet\to a,$ there exists $N$ for which
	\[n>N\implies|a_n-a|<\delta\implies|f(a_n)-f(a)|<\varepsilon.\]
	So we see that $f(a_n)\to f(a).$
\end{proof}
In fact, the converse is also true.
\begin{proposition} \label{prop:seqimpcont}
	Fix $a\in\RR.$ Suppose that $f:\RR\to\RR$ has the property that, for each sequence $\{a_k\}_{k\in\NN}$ with $a_k\to a,$ we have $f(a_k)\to a.$ Then $f$ is continuous at $a.$
\end{proposition}
\begin{proof}
	We proceed by contraposition. Suppose that $f$ is not continuous at $a,$ and we will exhibit a sequence $a_k\to a$ for which $f(a_k)$ does not converge to $f(a).$ We know there exists $\varepsilon>0$ for which no $\delta$ has
	\[|x-a|<\delta\implies|f(x)-f(a)|<\varepsilon.\]
	In particular, for each $n\in\NN,$ we can find some $a_n$ for which $|a_n-a|<\frac1n$ while $|f(a_n)-f(a)|\ge\varepsilon.$

	We claim that this is the sequence that we want. Indeed, we have that $a_n\to a$ because, for any $\varepsilon_0>0,$ we can set $N:=1/\varepsilon_0$ so that $n>N$ implies
	\[|a_n-a|<\frac1n<\frac1N=\varepsilon_0.\]
	However, $f(a_n)$ does not converge to $f(a).$ Indeed, for the given $\varepsilon>0,$ we know there is no $N$ for which $n>N$ implies $|f(a_n)-f(a)|<\varepsilon$ because $|f(a_n)-f(a)|\ge\varepsilon$ for each $n\in\NN.$
\end{proof}
So \autoref{prop:seqimpcont} tells us that we can reduce study of continuity to study of sequences, but there is still some amount of cumbersome work because we would have to account for all such sequences; doing single sequences is not enough.
\begin{example}
	The function $\sin\left(\frac1x\right)$ is not continuous at $x=0,$ but the sequence $\left\{\frac1{2\pi k}\right\}_{k\in\NN}$ has values of only $0$s, which do converge.
\end{example}
We remark that the definition of continuity we are working with focuses on differences of absolute values, which is really just the standard metric in disguise. So let's try changing the metric.
\begin{example}
	In the telemetric (distances between distinct points is always $1$), sequences converges if and only if it is eventually constant. Namely, if $a_k\to a,$ then take $\varepsilon=\frac12$ so that there is some $N$ for which $n>N$ has
	\[d_{\op{tele}}(a,a_k)<\frac12,\]
	but this forces $a_k=a.$ And conversely, if a sequence is eventually constant, then of course it converges.

	But now, all functions are continuous everywhere! Indeed, for any sequence $a_k\to a,$ we have $f(a_k)$ will eventually be exclusively terms of $f(a),$ which converges to $f(a),$ as needed.
\end{example}
Things get more complicated, for example, if the domain and range have different metrics and notions of continuity.