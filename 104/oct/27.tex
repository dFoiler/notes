% !TEX root = ../notes.tex














Here we go again.

\subsection{Connectedness}
Let's talk about connectedness. It matters later in life. We have the following definition.
\begin{definition}[Connected, I]
	A metric space $(X,d)$ is \textit{connected} if and only if has no proper nonempty subset which is both open and closed in $X.$ If $(X,d)$ is not connected, then we say that it is \textit{disconnected}.
\end{definition}
\begin{remark}
	Of course, both $\emp$ and $X$ are 
\end{remark}
There are some equivalent variations of this definitions. Here is an example.
\begin{definition}[Connected, II]
	A metric space $(X,d)$ is \textit{disconnected} if and only if we can write $X:=U_1\sqcup U_2$ for nonempty subsets $U_1$ and $U_2.$ If this is impossible, we say that $X$ is connected.
\end{definition}
With this in mind, we have the following.
\begin{definition}
	If some proper nonempty open subset $U\subseteq X$ with $U$ both open and closed, then we say that $U$ \textit{disconnects} $X.$
\end{definition}
\begin{remark}
	This is best seen geometrically: usually what is happening with disconnections is seen with $X:=[0,1]\cup[2,3],$ where the subset $[0,1]$ is both open and closed.
\end{remark}
\begin{example}
	In $\ZZ,$ any nonempty proper subset will disconnect $\ZZ.$ Indeed, points are open in $\ZZ,$ so all subsets are open.
\end{example}
\begin{example}
	In $\QQ,$ the subset
	\[U:=\{x\in\QQ:x>\sqrt2\}\]
	is open, but its complement is
	\[U^c=\{x\in\QQ:x<\sqrt2\}\]
	because $\sqrt2\notin\QQ,$ so we find that $U^c$ is also open, making $U$ closed.
\end{example}
In general, it is much harder to prove that a space is connected than disconnected because disconnection merely requires us to exhibit the disconnecting subset.

Let's use this definition for something.
\begin{proposition}
	Fix $f:X\to Y$ a continuous function between the metric spaces $(X,d_X)$ and $(Y,d_Y).$ If $X$ is connected, then $f(X)$ is connected.
\end{proposition}
\begin{proof}
	We can think about this topologically, similar to the proof that continuity preserves compactness. Anyways, we show this by contraposition: take $f(X)$ disconnected, and we show that $X$ is disconnected. This means that we have a proper nonempty subset $U\subseteq f(X)$ which is both open and closed so that
	\[f(X)=U\cup U^c\]
	shows that $f(X)$ can be disjointedly decomposed into two nonempty open sets. By continuity, we see that $f^{-1}(U)$ and $f^{-1}(U^c)$ are both open. Further, they union to $X$ because each $x\in X$ has $f(x)$ in one of $U$ and $U^c,$ so $x$ lives in one of $f^{-1}(U)$ and $f^{-1}(U^c).$

	So we see that $f^{-1}(U)$ is both open and closed; it remains to show that $f^{-1}(U)$ is nonempty and proper. Well, $U$ was nonempty, so take $y\in U\subseteq f(X)$ so that there exists $x\in X$ such that $f(x)=y\in U.$ It follows that $x\in f^{-1}(U),$ so $f^{-1}(U)$ is nonempty. Then because $U^c$ was nonempty, it follows $f^{-1}(U^c)$ is nonempty as well, so $f^{-1}(U)$ is proper. This finishes.
\end{proof}
\begin{remark}
	We can use this as something to determine what continuous functions might look like. For example, we know immediately that there is no continuous surjection from $[0,1]\to[0,1]\cup[2,3]$ once we know that $[0,1]$ is connected while $[0,1]\cup[2,3]$ is not.
\end{remark}
Anyways, we should probably give an example of a nontrivial connected set.
\begin{proposition}
	A subset $S\subseteq\RR$ is connected if and only if $S$ is an interval.
\end{proposition}
\begin{proof}
	We have two implications.
	\begin{itemize}
		\item Suppose that $S$ is not an interval. This means that there exists $t\in\RR$ with $t\notin S$ such that $S\cap(t,\infty)\ne\emp$ and $S\cap(-\infty,t)\ne\emp.$\footnote{This seems annoying to show, but I don't want to think about it.} But now these exact two sets will disconnect $S$! Indeed, we set
		\[U_1:=S\cap(t,\infty)\qquad\text{and}\qquad U_2:=S\cap(-\infty,t).\]
		We see that $U_1\cup U_2=S$ because $t\notin S,$ and we know they are both nonempty, so it remains to show that they are open. This is immediate by the induced topology, but we can use sphere arguments here: each point $s\in U_1$ will live in the interior of $U_1$ by using $r:=s-t>0$ as our radius.
	\end{itemize}
	\begin{remark}
		This is more or less generalizing the proof that $\QQ$ is disconnected.
	\end{remark}
	\begin{itemize}
		\item It remains to show that intervals are connected. We start by reducing to the case where interval has the form $[a,b].$ Fix $I\subseteq\RR$ an interval, and suppose for the sake of contradiction we can write
		\[I:=U_1\cup U_2,\]
		where $U_1$ and $U_2$ are disjoint, nonempty open subsets. Now find $a\in U_1$ and $b\in U_2,$ and without loss of generality we may take $a<b.$ Now we see that
		\[(U_1\cap[a,b])\cup(U_2\cap[a,b])=I\cap[a,b]=[a,b].\]
		Here both of these are open by the indued topology, they are nonempty because $a\in U_1\cap[a,b]$ and $b\in U_2\cap[a,b],$ and these are disjoint because $U_1$ and $U_2$ are in fact disjoint. Thus, $[a,b]$ is also disconnected.

		We now forget what we were doing and show that closed intervals $[a,b]$ are disconnected. Indeed, suppose that we can write
		\[[a,b]=U_1\sqcup U_2,\]
		where $U_1$ and $U_2$ are disjoint open sets. Without loss of generality, $a\in U_1$ and $b\in U_2.$

		Now we use the order topology on $\RR$ by applying the pushing trick from showing $[0,1]$ is compact. We see that $\sup U_1>a$ because there is a sphere around $a$ inside of $U_1,$ and we see that $\sup U_2<b$ because there is a sphere around $b$ inside of $U_2.$
		
		But now we notice that $\sup U_1\notin U_1$ because if so then we could place an open ball around $\sup U_1$ to get larger. On the other hand, $\sup U_1\notin U_2$ because if so we could place an open ball around $\sup U_1$ in $U_2$ to force the elements of $U_1$ to be smaller.
		\qedhere
	\end{itemize}
\end{proof}
We remark that the Intermediate value theorem follows from this statement: if a continuous function $f:S\to\RR$ has domain an interval, then its domain is connected, so its image is connected, so its image is an interval.
\begin{remark}
	There is also a notion of path-connectedness mentioned in the book, but we will not care about it in this class.
\end{remark}
Let's do an exercise and then move on.
\begin{exercise}[Ross 22.3]
	Fix $E\subseteq(X,d)$ a connected subset of a metric space. Show that the closure $\overline E$ is also connected.
\end{exercise}
\begin{proof}
	We show the contraposition: suppose $\overline E$ is disconnected, and we show $E$ is disconnected. Then fix disjoint nonempty open sets $U_1,U_2\subseteq E$ which union to $E.$ But now we see that
	\[(U_1\cap E)\cup(U_2\cap E)=E\]
	is a disjoint union of open sets in $E$ which union to $E.$

	It remains to show that $U_1$ and $U_2$ are nonempty. Well, suppose for the sake of contradiction that (say) $U_1\cap E$ contains $E$ so that $U_1$ contains $E.$ But now this implies that $U_1$ is a proper closed subset of $\overline E$ which contains $E,$ which violates the fact that $\overline E$ is the smallest closed set. Rigorizing this would be somewhat painful (we have to write out $\overline E=\bigcap_{V\supseteq E}V$ and talk about that definition).
\end{proof}

\subsection{Power Series}
Let's preview some of chapter 4. Some of this will be review from calculus.
\begin{definition}[Power series]
	Given a sequence of real numbers $\{a_n\}_{n=0}^\infty\subseteq\RR,$ we define the (formal) \textit{power series}
	\[\sum_{n=0}^\infty a_nx^n\]
	to more or less represents a function $\RR\to\\RR.$
\end{definition}
\begin{remark}
	I have defined the above formally so that we can plug in values of $x$ and then ask if the series converges. This prevents us from wondering if the series ``exists'' at all.
\end{remark}
Intuitively, it might feel like if
\[\sum_{n=0}^\infty a_k\cdot(-100)^k\]
converges then
\[\sum_{k=0}^\infty a_k\cdot20^k\]
should also converge because the second series seems ``uniformly'' smaller in magnitude, even though perhaps the alternating series has an effect. But of course such intuition requires care because these series potentially feel very different.

In general, a series has a few possibilities.
\begin{itemize}
	\item The series can converge at $x=0$ only. All series must converge here because the series just looks like $a_0$ here.
	\item The series converges for all $x\in\RR.$
	\item The series converges for each $|x|<R$ for some finite $R\in\RR,$ but the series diverges for $|x|>R.$ (The behavior at $x=\mp R$ is intentionally unspecified.)
\end{itemize}
And here are some examples.
\begin{example}
	The series
	\[\exp(z)=\sum_{k=0}^\infty\frac1{k!}x^k\]
	will converge everywhere. We can kind of feel that the coefficients get really small (smaller than exponential) fast, so the series ought converge.
\end{example}
\begin{example}
	The series
	\[\sum_{k=0}^\infty2^{k^2}x^k\]
	will converge nowhere outside of $0.$ We can show this using the root test because
	\[\sqrt[n]{|2^{n^2}x^n|}=|x|\cdot\sqrt[n]{2^{n^2}}=|x|\cdot2^n\to\infty,\]
	so this always diverges for $x\ne0.$
\end{example}
\begin{example}
	The series
	\[\frac1{1-x}=\sum_{k=0}^\infty x^k\]
	will converge for $|x|<1$ and diverge for $|x|>1.$
\end{example}
To evaluate the radius of convergence, as in the third case, it is most helpful to use the Root test. The ratio test is potentially helpful for particular series, but if the $a_\bullet$ are poorly behaved locally, then the Ratio test is also potentially poorly behaved. With this in mind, we have the following definition.
\begin{definition}[Radius of convergence]
	Given a power series
	\[\sum_{k=0}^\infty a_kx^k,\]
	we define the \textit{radius of convergence} to be
	\[R:=\left(\limsup_{n\to\infty}\sqrt[n]{|a_n|}\right)^{-1}.\]
	By convention, if the $\limsup$ is $0,$ we set $R:=\infty.$
\end{definition}
This behaves like the radius of convergence essentially by the Root test. We will not write out the proof here, but we will say that the main idea is that
\[\limsup_{n\to\infty}\sqrt[n]{|a_nx^n|}=|x|\cdot\limsup_{n\to\infty}\sqrt[n]{|a_n|}=\frac{|x|}R,\]
so we converge (absolutely!) for $\frac{|x|}R<1$ and diverge for $\frac{|x|}R>1.$ Namely, the fact that the Root test is inconclusive for $\pm1$ turns into the fact that we are unsure what happens for $x=\pm R.$

\subsection{Uniform Continuity: A Prelude}
We might want to formally integrate and differentiate a power series. For example, with $f(x):=\arctan x,$ we have
\[f'(x)=\frac1{1+x^2}=\frac1{1-\left(-x^2\right)}=\sum_{k=0}^\infty(-1)^kx^{2k}.\]
Integrating would tell us that
\[f(x)=\sum_{k=0}^\infty\frac{(-1)^k}{2k+1}x^{2k+1}.\]
These sorts of tricks are nice; for example, we are able to compute $\arctan$ from this, but the series for $\arcsin$ is quite worse.

However, these sorts of ideas require some care. For example, the series
\[\ln(1+x)=\sum_{k=1}^\infty\frac{(-1)^{k+1}x^k}k\]
will converge at $x=1$ because it is the alternating harmonic series, and in fact this converges to $\ln2.$ But when we differentiate this series, we see
\[\frac1{1+x}=\sum_{k=0}^\infty(-1)^kx^k,\]
which diverges at $x=1.$ So we do not get what we want without paying attention. What happened here?

On one hand, this problem will not occur worse: if we consider the series
\[f(x)=\sum_{k=0}^\infty a_kx^k,\]
then we see that
\[\limsup_{n\to\infty}\sqrt[n]{|a_n|}=\limsup_{n\to\infty}\sqrt[n]{|(n+1)a_{n+1}|}\]
because $\sqrt[n]{n+1}\to1.$ Thus, the radii of convergence between $f(x)$ and $f'(x)$ are the same. The point of this is to say that strengthening our convergence (to absolute) helps here, but absolute convergence is not actually necessary to make this work. The correct notion is uniform continuity.

Let's do some exercises before we talk more about uniform continuity.
\begin{exe}
	We find the interval of convergence for
	\[\sum_{n=1}^\infty n^2x^n.\]
\end{exe}
\begin{proof}
	The main point is to compute
	\[\limsup_{n\to\infty}\sqrt[n]{n^2}.\]
	Well, we see that $\log\sqrt[n]{n^2}=\frac2n\log n,$ which goes to $0$ as $n\to\infty.$ So we have that
	\[\lim_{n\to\infty}\sqrt[n]{n^2}=e^1=1,\]
	so our radius of convergence is $R=1^{-1}=1.$ Alternatively, the book proves $n^{1/n}\to1$ somewhere, so it follows $n^{2/n}\to1$ follows.

	It remains to deal with the endpoints. Well, $\sum(-1)^nn^2$ will always diverge by the Divergence test because $\left|n^2\right|=n^2\to\infty.$ So our interval of convergence is $\boxed{(-1,1)}.$
\end{proof}
\begin{exercise}
	We find the interval of convergence of
	\[\sum_{n=1}^\infty\left(\frac xn\right)^n.\]
\end{exercise}
\begin{proof}
	The main point is to compute
	\[\limsup_{n\to\infty}\sqrt[n]{\left(\frac1n\right)^n}=\limsup_{n\to\infty}\frac1n=0,\]
	so the radius of convergence is $+\infty.$ Thus, our series converges $\boxed{\text{everywhere}}.$
\end{proof}
\begin{remark}
	This makes sense because $n^{-n}$ gets small very fast, faster than for $\frac1{n!}.$
\end{remark}
\begin{exercise}
	We compute the interval of convergence of
	\[\sum_{n=1}^\infty\frac{2^n}{n^2}x^n.\]
\end{exercise}
\begin{proof}
	The main point is to compute
	\[\limsup_{n\to\infty}\sqrt[n]{\frac{2^n}{n^2}}=2\cdot\limsup_{n\to\infty}n^{-2/n}=2,\]
	so our radius of convergence is $1/2.$ As for the endpoints, we find that
	\[\sum_{n=1}^\infty\left|\frac{2^n}{n^2}\left(\frac12\right)^n\right|=\sum_{n=1}^\infty\frac1{n^2}\]
	converges, so the series converge absolutely. It follows that our interval of convergence is $\boxed{[-1/2,1/2]}.$
\end{proof}
\begin{exe}
	We compute the interval of convergence of
	\[\sum_{n=1}^\infty\frac{n^3}{3^n}x^n.\]
\end{exe}
\begin{proof}
	We can see that
	\[\limsup_{n\to\infty}\sqrt[n]{\frac{n^3}{3^n}}=\frac13,\]
	so it follows that our radius of convergence is is $3.$ The endpoints give $\sum(\pm1)^nn^3,$ which fail the divergence test, so our interval of convergence is $\boxed{(-3,3)}.$
\end{proof}

\subsection{Uniform Continuity: Another Prelude}
Let's see another reason we might want stronger continuity. Fix
\[f_n(x):=x^n\]
for $f_n:[0,1]\to[0,1]$ and $n\ge0.$ But now we see that, for $x\in\left[0,1\right),$ we have that
\[\lim_{n\to\infty}f_n(x)=\lim_{n\to\infty}x^n=0\]
while
\[\lim_{n\to\infty}f_n(1)=\lim_{n\to\infty}1=1.\]
So we let
\[f(x):=\begin{cases}
	1 & x=1, \\
	0 & x\ne1,
\end{cases}\]
so that $f_n\to f$ as $n\to\infty.$ But now the $f_n$ are continuous while converging to a discontinuous function, which is sad. The problem, again, is that our convergence is somehow ``too weak,'' and the stronger form of convergence--``uniform convergence''--is what we want.