% !TEX root = ../notes.tex

\documentclass[../notes.tex]{subfiles}

\begin{document}

\section{Local Fields}

In this section, we introduce the main characters of our intermission, which are the characteristic-$0$ local fields.

\subsection{The \texorpdfstring{$p$}{ p}-adics}
Fix a prime $p$ in the following discussion. The goal of the present subsection is to define the ring of $p$-adic integers $\ZZ_p$. Approximately speaking, the point will be that $\ZZ_p$ is able to capture the information of $\ZZ/p\ZZ$ and $\ZZ/p^2\ZZ$ and $\ZZ/p^3\ZZ$ and so on, all at once.

To see why this might be helpful, we recall the idea of ``local obstructions'' we saw much earlier in this course: an equation might have solutions $\ZZ/p\ZZ$ but lose solutions in $\ZZ/p^2\ZZ$, or it might have solutions in $\ZZ/p^3\ZZ$ but lose solutions in $\ZZ/p^4\ZZ$.
\begin{example}
	The equation $x^2+1=0$ has solutions in $\ZZ/2\ZZ$ but no solutions in $\ZZ/4\ZZ$.
\end{example}
\begin{example}
	The equation $4x^2+4=0$ has solutions in $\ZZ/8\ZZ$ but no solutions in $\ZZ/16\ZZ$.
\end{example}
The ring $\ZZ_p$ will be able to keep track of all this modular information at once. For example, thinking about $\ZZ_p$ will allow us to cleanly thing about statements such as the following one.
\begin{proposition}
	Let $a$ be an integer. The following are equivalent.
	\begin{listalph}
		\item The equation $x^2\equiv a\pmod 8$ has solutions.
		\item The equation $x^2\equiv a\pmod{2^\nu}$ for any power $2^\nu$.
	\end{listalph}
\end{proposition}
With this in mind, a good first guess for $\ZZ_p$ would be the infinite product ring
\[R_p\coloneqq\prod_{\nu=0}^\infty\ZZ/p^\nu\ZZ.\]
However, this ring has many problems. For example, $R_p$ is not an integral domain: $(1,0,0,\ldots)\cdot(0,1,0,\ldots)=(0,0,0,\ldots)$. A worse problem is that it somehow fails to actually care about modular information: a central property of the $\ZZ/p^\nu\ZZ$s is that if an equation has solutions in $\ZZ/p^{\nu+1}\ZZ$, then it will have solutions in $\ZZ/p^\nu\ZZ$ because of the reduction map
\begin{equation}
	\ZZ/p^{\nu+1}\ZZ\to\ZZ/p^\nu\ZZ. \label{eq:zpz-proj}
\end{equation}
This reduction map would allow us to recover $\ZZ/p^\nu\ZZ$-solutions from ``higher-order'' solutions, and we would like $\ZZ_p$ to mirror this structure.

In particular, we would like projection maps $\ZZ_p\to\ZZ/p^\nu\ZZ$ for each $p^\nu$, and we want these maps to commute with the maps \eqref{eq:zpz-proj}. This allows us to fix $R_p$ into the following ring.
\begin{defihelper}[$p$-adic integers] \nirindex{p-adics integers@$p$-adic integers}
	Fix a prime $p$. Then we define the ring of \textit{$p$-adic integers} by
	\[\ZZ_p\coloneqq\Bigg\{(a_\nu)_{\nu=0}^\infty\in\prod_{\nu=0}^\infty\ZZ/p^\nu\ZZ:a_{\nu+1}\equiv a_\nu\pmod{p^\nu}\Bigg\}.\]
	In practice, it is helpful to view $(a_\nu)_\nu$ as a sequence of integers, but it is important to remember that we only care about $a_\nu\pmod{p^\nu}$.
\end{defihelper}
\begin{remark} \label{rem:upgrade-zp-equiv}
	Quickly, note that $a_{\nu+1}\equiv a_\nu\pmod{p^\nu}$ implies (by induction) that
	\[a_{\nu+N}\equiv a_\nu\pmod{p^\nu}\]
	for any nonnegative integer $N\ge0$. Indeed, there is nothing to say for $N=0$, and then the inductive step recognizes that $a_{\nu+N+1}\equiv a_{\nu+N}\pmod{p^{\nu+N}}$ and hence $a_{\nu+N+1}\equiv a_{\nu+N}\equiv a_\nu\pmod{p^\nu}$.
\end{remark}
\begin{example} \label{ex:power-series-z2}
	Fix $p=2$ and define $a_\nu\coloneqq1+2+2^2+\cdots+2^{\nu-1}$. Then $(a_\nu)_\nu\in\ZZ_p$.
\end{example}
\begin{proof}
	Certainly we have defined a sequence of elements $a_\nu\in\ZZ/2^\nu\ZZ$. It remains to check that
	\[a_{\nu+1}=\sum_{k=0}^\nu2^k\equiv\sum_{k=0}^{\nu-1}2^k=a_\nu\pmod{2^\nu},\]
	so we are done.
\end{proof}
Of course, we can write elements of $\ZZ_p$ as sequences in $\prod_\nu\ZZ/p^\nu\ZZ$, but \Cref{ex:power-series-z2} suggests a more intuitive way to look at our elements: any element of $\ZZ_p$ can be thought of as a power series
\[c_0+c_1p+c_2p^2+\cdots\]
where $c_0,c_1,c_,\ldots\in\{0,1,\ldots,p-1\}$. Let's explain this bijection.
\begin{proposition} \label{prop:p-adic-power-series}
	Fix a prime $p$. Then there is a unique bijection $\{0,1,\ldots,p-1\}^\NN\to\ZZ_p$ given by sending $(c_\nu)_{\nu=0}^\infty$ to the sequence
	\[\Bigg(\sum_{k=0}^{\nu-1}c_kp^k\Bigg)_{\nu=0}^\infty\in\ZZ_p.\]
\end{proposition}
\begin{proof}
	Let the function $\{0,1,\ldots,p-1\}^\NN\to\ZZ_p$ be called $\varphi$. Here are our checks.
	\begin{itemize}
		\item Well-defined: given $(c_\nu)_{\nu=0}^\infty$, we must show that $\varphi\left((c_\nu)_{\nu=0}^\infty\right)\in\ZZ_p$. Well, expanding out, it is enough to note that
		\[\sum_{k=0}^{\nu}c_kp^k\equiv\sum_{k=0}^{\nu-1}c_kp^k\pmod{p^\nu},\]
		completing the argument.
		\item Injective: suppose we have two sequences $(c_\nu)_{\nu=0}^\infty$ and $(d_\nu)_{\nu=0}^\infty$ going to the same element in $\ZZ_p$. Then checking in $\ZZ_p$ tells us that
		\[\sum_{k=0}^{\nu-1}c_kp^k\equiv\sum_{k=0}^{\nu-1}d_kp^k\pmod{p^\nu}\]
		for any $\nu$. We now claim that $c_\nu=d_\nu$ for each $\nu\ge0$ by induction on $k$. If $\nu=0$, then we note $c_0\equiv d_0\pmod p$ enforces $c_0=d_0$ because $c_0,d_0\in\{0,1,\ldots,p-1\}$. Now, for the inductive step, suppose $c_k=d_k$ for $k<\nu$, then we see that
		\[\sum_{k=0}^{\nu}c_kp^k\equiv\sum_{k=0}^{\nu}d_kp^k\pmod{p^{\nu+1}},\]
		but we already know that
		\[\sum_{k=0}^{\nu-1}c_kp^k=\sum_{k=0}^{\nu-1}d_kp^k\]
		by the inductive hypothesis, so we conclude $c_\nu p^\nu\equiv d_\nu p^\nu\pmod{p^{\nu+1}}$, so $c_\nu\equiv d_\nu\pmod p$, so $c_\nu=d_\nu$.
		\item Surjective: fix any $(a_\nu)_{\nu=0}^\infty\in\ZZ_p$. Now, we would like to have
		\begin{equation}
			a_\nu\stackrel?\equiv\sum_{k=0}^{\nu-1}c_kp^k\pmod{p^\nu} \label{eq:desired-power-series}
		\end{equation}
		for each $\nu\ge0$. We will define $(c_\nu)_{\nu=0}^\infty$ inductively to satisfy these. We begin by choosing $c_0\equiv a_1\pmod p$, which is possible because we may choose $c_0\in\{0,1,\ldots,p-1\}$.

		For the inductive step, suppose we are given $c_0,\ldots,c_{\nu-1}$ so that \eqref{eq:desired-power-series} holds. Observe that this also implies
		\[a_{\nu+1}\equiv\sum_{k=0}^{\nu-1}c_kp^k\pmod{p^\nu},\]
		so we define $c_\nu\in\{0,1,\ldots,p-1\}$ by
		\[c_\nu\equiv\frac1{p^\nu}\Bigg(a_{\nu+1}-\sum_{k=0}^{\nu-1}c_kp^k\Bigg)\pmod p.\]
		Rearranging yields
		\[a_{\nu+1}\equiv\sum_{k=0}^\nu c_kp^k\pmod{p^{\nu+1}},\]
		completing the inductive step.
		\qedhere
	\end{itemize}
\end{proof}

\subsection{Ring Theory of \texorpdfstring{$\ZZ_p$}{ Zp}}
Here are some basic checks on $\ZZ_p$.
\begin{lemma}
	Fix a prime $p$. Then $\ZZ_p$ is a subring of $R_p$.
\end{lemma}
\begin{proof}
	We have the following checks.
	\begin{itemize}
		\item Identities: note that $(0,0,\ldots)\in\ZZ_p$ and $(1,1,\ldots)\in\ZZ_p$ because $a\equiv a\pmod{p^\nu}$ for any $p^\nu$ and $a\in\{0,1\}$.
		\item Closure: given $(a_\nu)_\nu,(b_\nu)_\nu\in\ZZ_p$, we note that
		\begin{align*}
			a_{\nu+1}+b_{\nu+1} &\equiv a_\nu+b_\nu\pmod{p^\nu}, \\
			a_{\nu+1}\cdot b_{\nu+1} &\equiv a_\nu\cdot b_\nu\pmod{p^\nu},
		\end{align*}
		so $(a_\nu+b_\nu)_\nu,(a_\nu+b_\nu)\in\ZZ_p$ as well.
		\qedhere
	\end{itemize}
\end{proof}
We complained that $R_p$ is not an integral domain, but $\ZZ_p$ is.
\begin{lemma} \label{lem:zp-int-domain}
	Fix a prime $p$. Then $\ZZ_p$ is an integral domain.
\end{lemma}
\begin{proof}
	The point is to use the coherence of the coordinates of in elements of $\ZZ_p$. Suppose we have nonzero $(a_\nu)_\nu,(b_\nu)_\nu\in\ZZ_p$, and we show that the product is nonzero. Because $(a_\nu)_\nu$ and $(b_\nu)_\nu$ are nonzero, we can find some specific nonnegative integers $m$ and $n$ such that $a_m\not\equiv0\pmod{p^m}$ and $b_n\not\equiv0\pmod{p^n}$.
	
	Thus, the largest power of $p$ dividing $a_m$ is less than $m$, and the same then holds for $a_{m+n}$; similarly, and the largest power of $p$ dividing $b_n$ is less than $n$, and the same then holds for $b_{m+n}$. Thus, the largest power of $p$ dividing $a_{m+n}b_{m+n}$ is less than $m+n$, so
	\[a_{m+n}b_{m+n}\not\equiv0\pmod{p^{m+n}},\]
	so $(a_\nu)_\nu\cdot(b_\nu)_\nu\ne0$.
\end{proof}
It is notable that there is a copy of $\ZZ$ in $\ZZ_p$, and $\ZZ$ is ``dense'' in $\ZZ_p$ in some sense. This tells us that $\ZZ_p$ is indeed a fairly good approximation of $\ZZ$, but do remember that $\ZZ_p$ only cares about$\pmod{p^\nu}$ information.
\begin{lemma}
	Fix a prime $p$. For any $(a_\nu)_\nu\in\ZZ_p$ and positive integer $N$, there exists an integer $a\in\ZZ$ such that
	\[a\equiv a_\nu\pmod{p^\nu}\]
	for all $\nu\le N$.
\end{lemma}
\begin{proof}
	Choose $a\coloneqq a_N$. Then any $\nu\le N$ has
	\[a_N\equiv a_\nu\pmod{p^\nu}\]
	by construction of $\ZZ_p$, so we are done.
\end{proof}
\begin{lemma} \label{lem:z-into-zp}
	Fix a prime $p$. The map $i_p\colon\ZZ\to\ZZ_p$ given by $i_p(a)\coloneqq(a)_{\nu=0}^\infty$ is an injective ring homomorphism.
\end{lemma}
\begin{proof}
	Note that the maps $\ZZ\to\ZZ/p^\nu\ZZ$ for each $p^\nu$ are all (surjective) ring homomorphisms, so we have a ring map
	\[\ZZ\to\prod_{\nu=0}^\infty\ZZ/p^\nu\ZZ\]
	given by $a\mapsto(a)_{\nu=0}^\infty$. So to check that $i_p$ is a ring homomorphism, we merely need to check that the above map lands in $\ZZ_p$. Well, we see that
	\[a\equiv a\pmod{p^\nu}\]
	for any $p^\nu$, so indeed $i_p(a)\in\ZZ_p$.

	Lastly, we check that $i_p$ is injective. It suffices to check that $i_p$ has trivial kernel. Well, suppose we have some nonzero integer $n\in\ZZ$. By unique prime factorization in $\ZZ$, there is a some largest nonnegative integer $\nu$ such that $p^\nu\mid n$. But then
	\[n\not\equiv0\pmod{p^{\nu+1}},\]
	so it follows that $i_p(n)\ne0$.
\end{proof}
\begin{remark}
	\Cref{lem:z-into-zp} is somewhat remarkable in that each individual map $\ZZ\to\ZZ/p^\nu\ZZ$ has lost information (it has nontrivial kernel), but gluing them all together does not lose any information about $\ZZ$! Typically, we think about taking modding out as forgetting some information, but we can technically recover $\ZZ$ from within $\ZZ_p$.
\end{remark}
We now make a few remarks on the ring structure of $\ZZ_p$, simply because it is so nice. To emphasize that $\ZZ_p$ only cares about the prime $p$, we note that all other primes are invertible in $\ZZ_p$. In fact, we have the following stronger statement.
\begin{proposition} \label{prop:units-in-zp}
	Fix a prime $p$ and some $(a_\nu)_\nu\in\ZZ_p$. The following are equivalent.
	\begin{listalph}
		\item $(a_\nu)_\nu$ is a unit.
		\item $a_\nu\pmod{p^\nu}$ is a unit for all $\nu\ge1$.
		\item $a_\nu\pmod{p^\nu}$ is a unit for some $\nu\ge1$.
	\end{listalph}
\end{proposition}
\begin{proof}
	We show our implications in sequence.
	\begin{itemize}
		\item We show (a) implies (b). Suppose that $(a_\nu)_\nu$ is a unit so that we have some $(b_\nu)_\nu\in\ZZ_p$ such that $(a_\nu)_\nu\cdot(b_\nu)_\nu=1$. Here, $1$ is the sequence $(1,1,\ldots)$, so we see that
		\[a_\nu\cdot b_\nu\equiv1\pmod{p^\nu}\]
		for each $\nu$, which is (b).
		\item Note that (b) implies (c) with no work.
		\item We show (c) implies (b). Suppose that $a_N\pmod{p^N}$ is a unit where $N\ge1$. Now, for any $\nu\ge1$, we see that
		\[a_N\equiv a_\nu\pmod{p^{\min\{N,\nu\}}},\]
		so $a_N\equiv a_\nu\pmod p$, so $p\nmid a_N$ implies that $p\nmid a_\nu$. But $p\nmid a_\nu$ implies that $a_\nu$ is a unit in $(\ZZ/p^\nu\ZZ)^\times$, which is what we wanted.
		\item We show that (b) implies (a). By hypothesis, we may define
		\[b_\nu\equiv\begin{cases}
			1\pmod1 & \text{if }\nu=0, \\
			a_\nu^{-1}\pmod{p^\nu} & \text{if }\nu\ge1.
		\end{cases}\]
		Quickly, note that $(a_\nu)_\nu\cdot(b_\nu)_\nu=1$, so it only remains to verify that $(b_\nu)_\nu$ lives in $\ZZ_p$. Well, for any $n\ge m$, we note that
		\[1\equiv b_na_n\equiv b_na_m\pmod{p^m},\]
		so $b_n\equiv a_m^{-1}\equiv b_m\pmod{p^m}$, as needed.
		\qedhere
	\end{itemize}
\end{proof}
\begin{lemma} \label{lem:principal-ideals-zp}
	Fix a prime $p$. For any integer $n\ge1$, we have
	\[p^n\ZZ_p=\left\{(a_\nu)_\nu\in\ZZ_p:a_n\equiv0\pmod{p^n}\right\}.\]
\end{lemma}
\begin{proof}
	There are two inclusions to show. In one direction, for any $(a_\nu)_\nu\in\ZZ_p$, we see that $p^n(a_\nu)_\nu=\left(p^na_\nu\right)_\nu$ has $p^na_\nu\equiv0\pmod{p^n}$. In the other direction, suppose that $(a_\nu)_\nu\in\ZZ_p$ has $a_n\equiv0\pmod{p^n}$. Then we use \Cref{prop:p-adic-power-series} to expand $(a_\nu)_\nu$ as a power series
	\[\sum_{k=0}^\infty c_kp^k\]
	where each $c_k$ lives in $\{0,1,\ldots,p-1\}$. The condition on $a_n$ requires that $c_k=0$ for $k<n$. We now define $(b_\nu)_\nu$ as from the power series
	\[\frac1{p^n}\sum_{k=0}^\infty c_kp^k=\sum_{k=0}^\infty c_{n+k}p^k.\]
	In particular, for each $\nu\ge1$, we see that
	\[p^nb_\nu\equiv\sum_{k=0}^{\nu-1}c_{n+k}p^{n+k}\equiv\sum_{k=0}^{\nu+n-1}c_kp^k\equiv a_\nu\pmod{p^\nu},\]
	so we conclude that $(a_\nu)_\nu\in p^n\ZZ_p$.
\end{proof}
\begin{corollary}
	Fix a prime $p$. Then the ring $\ZZ_p$ has a unique maximal ideal, which is $p\ZZ_p$.
\end{corollary}
\begin{proof}
	We claim that all proper ideals are contained in $p\ZZ_p$, which will imply the claim. In fact, we will show the contraposition of this statement: we claim that any ideal not contained in $p\ZZ_p$ is not a proper ideal. Well, suppose that $I\subseteq\ZZ_p$ is an ideal not contained in $p\ZZ_p$. Then there is some $(a_\nu)_\nu\in I$ which is not in $p\ZZ_p$. Well, \Cref{lem:principal-ideals-zp} tells us that $p\ZZ_p$ exactly consists of the $(b_\nu)_\nu\in\ZZ_p$ such that $p\mid b_\nu$ for each $\nu\ge1$, so \Cref{prop:units-in-zp} tells us that $\ZZ_p\setminus p\ZZ_p=\ZZ_p^\times$. As such $(a_\nu)_\nu\in I$ is a unit, which implies $I=\ZZ_p$.
\end{proof}
Having a unique maximal ideal which is principal is incredibly nice, but in fact the ideal structure of $\ZZ_p$ is as nice as it could possibly be.
% \begin{lemma}
% 	Fix a prime $p$, and define the function $v\colon\ZZ_p\setminus\{0\}\to\ZZ$ by
% 	\[v((a_\nu)_\nu)\coloneqq\min\left\{\nu\ge1:a_\nu\not\equiv0\pmod{p^\nu}\right\}.\]
% \end{lemma}
\begin{proposition}
	Fix a prime $p$. Then the ring $\ZZ_p$ is a principal ideal domain. In fact, for any nonzero ideal $I\subseteq\ZZ_p$, there exists a nonnegative integer $n$ such that
	\[I=p^n\ZZ_p.\]
\end{proposition}
\begin{proof}
	Note that $\ZZ_p$ is already an integral domain by \Cref{lem:zp-int-domain}. Now, let $I\subseteq\ZZ_p$ be a nonzero ideal. The main point of the proof is the construction of $n$. Because $I$ is nonzero, there exists some nonzero $(a_\nu)_\nu\in I$, and being nonzero implies that there is some greatest integer $n((a_\nu)_\nu)$ such that $a_n\not\equiv0\pmod{p^n}$. Define $n$ to be the least such integer over all nonzero $(a_\nu)_\nu\in I$.

	It remains to show that $I=p^n\ZZ_p$. One inclusion is not so bad: for each nonzero $(a_\nu)_\nu\in I$, the construction of $n$ requires that $a_n\equiv0\pmod{p^n}$, so $(a_\nu)_\nu\in p^n\ZZ_p$ by \Cref{lem:principal-ideals-zp}. For the other inclusion, we must show that $p^n\in I$. Well, by the construction of $n$, we are promised some $(a_\nu)_\nu\in I$ such that $a_n\equiv0\pmod{p^n}$ but $a_{n+1}\not\equiv0\pmod{p^{n+1}}$. By \Cref{lem:principal-ideals-zp} again, we are able to write
	\[(a_\nu)_\nu=p^n(b_\nu)_\nu\]
	for some $(b_\nu)_\nu\in\ZZ_p$. We would like to cancel out the $(b_\nu)_\nu$ from both sides.
	
	Well, if $b_{n+1}\equiv0\pmod p$, then $p^nb_{n+1}\equiv0\pmod{p^{n+1}}$, which contradicts the construction of $(a_\nu)_\nu$. Thus, we must instead have $b_{n+1}\in\left(\ZZ/p^{n+1}\ZZ\right)^\times$, so $(b_\nu)_\nu$ is a unit, and we may find $(c_\nu)_\nu\in\ZZ_p$ such that $(b_\nu)_\nu\cdot(c_\nu)_\nu=1$. Finishing up, we see
	\[p^n=(a_\nu)_\nu\cdot(c_\nu)_\nu\]
	lives in $I$, which completes the proof.
\end{proof}

\todo{}

\subsection{Hensel's Lemma}

% \subsection{Newton's Method}

\subsection{Problems}
\begin{prob}[3 points]
	Fix a prime $p$, and choose a subset $S\subseteq\ZZ$ containing exactly one representative of each class in $\ZZ/p\ZZ$. For any $a\in\ZZ_p$, show that there is a unique sequence $(c_\nu)_{\nu=0}^\infty\in S^\NN$ such that
	\[a=\Bigg(\sum_{k=0}^{\nu-1}c_kp^k\Bigg)_{\nu=0}^\infty.\]
\end{prob}
\begin{prob}[4 points] \label{prob:223-in-z3}
	Show that $x^2-223y^2=-3$ has a solution in $\ZZ_3$ by showing that there exists $y\in\ZZ_3$ such that $y^2=4/223$. Use \Cref{prob:norm-223} to conclude that the equation
	\[x^2-223y^2\equiv-3\pmod n\]
	has a solution for any positive integer $n$.
\end{prob}

% show Z_p is a PID with a unique maximal ideal

% define Z_p algebraically, and get Q_p as a quotient
% show that Z is dense in Z_p, but Z_p has more elements than Z
% exhibit the absolute value
% give redefinition of Qp
% ostrowski
% product formula

% 10-adics?
% power series? mahler expansions?
% other fields? I don't know if there are applications ...

% strassman's theorem to solve x^3 - 2y^3 = 1

\end{document}