% !TEX root = ../notes.tex

\documentclass[../notes.tex]{subfiles}

\begin{document}

\section{Local Fields}

In this section, we introduce the main characters of our intermission, which are the characteristic-$0$ local fields.

\subsection{The \texorpdfstring{$p$}{ p}-adics, Algebraically}
Fix a prime $p$ in the following discussion. The goal of the present subsection is to define the ring of $p$-adic integers $\ZZ_p$. Approximately speaking, the point will be that $\ZZ_p$ is able to capture the information of $\ZZ/p\ZZ$ and $\ZZ/p^2\ZZ$ and $\ZZ/p^3\ZZ$ and so on, all at once.

To see why this might be helpful, we recall the idea of ``local obstructions'' we saw much earlier in this course: an equation might have solutions $\ZZ/p\ZZ$ but lose solutions in $\ZZ/p^2\ZZ$, or it might have solutions in $\ZZ/p^3\ZZ$ but lose solutions in $\ZZ/p^4\ZZ$.
\begin{example}
	The equation $x^2+1=0$ has solutions in $\ZZ/2\ZZ$ but no solutions in $\ZZ/4\ZZ$.
\end{example}
\begin{example}
	The equation $4x^2+4=0$ has solutions in $\ZZ/8\ZZ$ but no solutions in $\ZZ/16\ZZ$.
\end{example}
The ring $\ZZ_p$ will be able to keep track of all this modular information at once. For example, thinking about $\ZZ_p$ will allow us to cleanly thing about statements such as the following one.
\begin{proposition}
	Let $a$ be an integer. The following are equivalent.
	\begin{listalph}
		\item The equation $x^2\equiv a\pmod 8$ has solutions.
		\item The equation $x^2\equiv a\pmod{2^\nu}$ for any power $2^\nu$.
	\end{listalph}
\end{proposition}
With this in mind, a good first guess for $\ZZ_p$ would be the infinite product ring
\[R_p\coloneqq\prod_{\nu=0}^\infty\ZZ/p^\nu\ZZ.\]
However, this ring has many problems. For example, $R_p$ is not an integral domain: $(1,0,0,\ldots)\cdot(0,1,0,\ldots)=(0,0,0,\ldots)$. A worse problem is that it somehow fails to actually care about modular information: a central property of the $\ZZ/p^\nu\ZZ$s is that if an equation has solutions in $\ZZ/p^{\nu+1}\ZZ$, then it will have solutions in $\ZZ/p^\nu\ZZ$ because of the reduction map
\begin{equation}
	\ZZ/p^{\nu+1}\ZZ\to\ZZ/p^\nu\ZZ. \label{eq:zpz-proj}
\end{equation}
This reduction map would allow us to recover $\ZZ/p^\nu\ZZ$-solutions from ``higher-order'' solutions, and we would like $\ZZ_p$ to mirror this structure.

In particular, we would like projection maps $\ZZ_p\to\ZZ/p^\nu\ZZ$ for each $p^\nu$, and we want these maps to commute with the maps \eqref{eq:zpz-proj}. This allows us to fix $R_p$ into the following ring.
\begin{defihelper}[$p$-adic integers] \nirindex{p-adics integers@$p$-adic integers}
	Fix a prime $p$. Then we define the ring of \textit{$p$-adic integers} by
	\[\ZZ_p\coloneqq\Bigg\{(a_\nu)_{\nu=0}^\infty\in\prod_{\nu=0}^\infty\ZZ/p^\nu\ZZ:a_{\nu+1}\equiv a_\nu\pmod{p^\nu}\Bigg\}.\]
	In practice, it is helpful to view $(a_\nu)_\nu$ as a sequence of integers, but it is important to remember that we only care about $a_\nu\pmod{p^\nu}$.
\end{defihelper}
Here are some basic checks on $\ZZ_p$.
\begin{lemma}
	Fix a prime $p$. Then $\ZZ_p$ is a subring of $R_p$.
\end{lemma}
\begin{proof}
	We have the following checks.
	\begin{itemize}
		\item Identities: note that $(0,0,\ldots)\in\ZZ_p$ and $(1,1,\ldots)\in\ZZ_p$ because $a\equiv a\pmod{p^\nu}$ for any $p^\nu$ and $a\in\{0,1\}$.
		\item Closure: given $(a_\nu)_\nu,(b_\nu)_\nu\in\ZZ_p$, we note that
		\begin{align*}
			a_{\nu+1}+b_{\nu+1} &\equiv a_\nu+b_\nu\pmod{p^\nu}, \\
			a_{\nu+1}\cdot b_{\nu+1} &\equiv a_\nu\cdot b_\nu\pmod{p^\nu},
		\end{align*}
		so $(a_\nu+b_\nu)_\nu,(a_\nu+b_\nu)\in\ZZ_p$ as well.
		\qedhere
	\end{itemize}
\end{proof}
We complained that $R_p$ is not an integral domain, but $\ZZ_p$ is.
\begin{lemma}
	Fix a prime $p$. Then $\ZZ_p$ is an integral domain.
\end{lemma}
\begin{proof}
	The point is to use the coherence of the coordinates of in elements of $\ZZ_p$. Suppose we have nonzero $(a_\nu)_\nu,(b_\nu)_\nu\in\ZZ_p$, and we show that the product is nonzero. Because $(a_\nu)_\nu$ and $(b_\nu)_\nu$ are nonzero, we can find some specific nonnegative integers $m$ and $n$ such that $a_m\not\equiv0\pmod{p^m}$ and $b_n\not\equiv0\pmod{p^n}$.
	
	Thus, the largest power of $p$ dividing $a_m$ is less than $m$, and the same then holds for $a_{m+n}$; similarly, and the largest power of $p$ dividing $b_n$ is less than $n$, and the same then holds for $b_{m+n}$. Thus, the largest power of $p$ dividing $a_{m+n}b_{m+n}$ is less than $m+n$, so
	\[a_{m+n}b_{m+n}\not\equiv0\pmod{p^{m+n}},\]
	so $(a_\nu)_\nu\cdot(b_\nu)_\nu\ne0$.
\end{proof}
It is notable that there is a copy of $\ZZ$ in $\ZZ_p$, and $\ZZ$ is ``dense'' in $\ZZ_p$ in some sense. This tells us that $\ZZ_p$ is indeed a fairly good approximation of $\ZZ$, but do remember that $\ZZ_p$ only cares about$\pmod{p^\nu}$ information.\todo{}

\subsection{Hensel's Lemma}

\subsection{Ostrowski's Theorem}

\subsection{Problems}
\begin{prob} \label{prob:223-in-z3}
	Show that $x^2-223y^2=-3$ has a solution in $\ZZ_3$ by showing that there exists $y\in\ZZ_3$ such that $y^2=4/223$. Use \Cref{prob:norm-223} to conclude that the equation
	\[x^2-223y^2\equiv-3\pmod n\]
	has a solution for any positive integer $n$.
\end{prob}

% show Z_p is a PID with a unique maximal ideal

% define Z_p algebraically, and get Q_p as a quotient
% show that Z is dense in Z_p, but Z_p has more elements than Z
% exhibit the absolute value
% give redefinition of Qp
% ostrowski
% product formula

% 10-adics?
% power series? mahler expansions?
% other fields? I don't know if there are applications ...

% strassman's theorem to solve x^3 - 2y^3 = 1

\end{document}