% !TEX root = ../notes.tex

\documentclass[../notes.tex]{subfiles}

\begin{document}

\section{Dirichlet's Unit Theorem}

In this section, we will prove the Dirichlet unit theorem, at long last.

\subsection{Dirichlet's Unit Theorem: Set Up}
We are almost at a point where we can state our main theorem. Approximately speaking, our goal is to generalize \Cref{prop:solve-pell-norm-one} to more general number fields. We now have enough machinery to explain where $x^2-dy^2=1$ is coming from: given a non-square positive integer $d$, in the field $\QQ(\sqrt d)$, by \Cref{cor:norm-tr-by-embeds}, the norm map $\op N_{\QQ(\sqrt d)/\QQ}$ is given by
\[\op N_{\QQ(\sqrt d)/\QQ}\left(x+y\sqrt d\right)=\left(x+y\sqrt d\right)\left(x-y\sqrt d\right)=x^2-dy^2.\]
This also explains why we kept factoring $x^2-dy^2$ into $\left(x-y\sqrt d\right)\left(x+y\sqrt d\right)$. It will shortly be helpful for us to have a more algebraic description of these elements.
\begin{lemma} \label{lem:unit-by-norm-one}
	Fix a number field $K$. Then an element $u\in\OO_K$ is a unit (i.e., has a multiplicative inverse in $\OO_K$) if and only if $\left|\op N_{K/\QQ}(u)\right|=1$.
\end{lemma}
\begin{proof}
	We have two implications to show.
	\begin{itemize}
		\item Suppose that $u\in\OO_K$ is a unit. Then we have some $v\in\OO_K$ such that $uv=1$. Taking norms of this equation, we see
		\[\op N_{K/\QQ}(u)\cdot\op N_{K/\QQ}(v)=1.\]
		However, $\op N_{K/\QQ}(u),\op N_{K/\QQ}(v)\in\ZZ$ by \Cref{rem:norm-and-tr-ints}, and the only way to have two integers multiply to $1$ is for them to be $\pm1$. Thus, $\op N_{K/\QQ}(u)=\pm1$, as desired.
		\item Suppose that $\op N_{K/\QQ}(u)=\pm1$ so that $\op N_{K/\QQ}(u)^2=1$. The point is to expand the norm using \Cref{cor:norm-tr-by-embeds} to get an equation of the form $uv=1$. Indeed, by \Cref{cor:norm-tr-by-embeds}, we see that
		\[\prod_{i=1}^n\sigma_i(u)^2=1\]
		where the $\sigma_1,\ldots,\sigma_n\colon K\into\CC$ are the embeddings of \Cref{prop:embeddings-to-c}. Identifying $K$ with its image under $\sigma_1\colon K\into\CC$ (for example), we see that
		\[\sigma_1(u)\cdot\underbrace{\sigma_1(u)\prod_{i=2}^n\sigma_i(u)^2}_{v\coloneqq}=1.\]
		Now, $uv=1$, so we will be done once we establish that $v\in\OO_K$. Well, $u\in\OO_K$, so letting $f(x)$ be a monic polynomial with integer coefficients such that $f(u)=0$, we see that $f(\sigma_i(u))=0$ for all $i$, so $\sigma_i(u)$ is an algebraic integer for each $\sigma_i$, so $v$ is also an algebraic integer. Further, $v=1/u\in K$, so it follows $v\in\OO_K$.
		\qedhere
	\end{itemize}
\end{proof}
So integer pairs $(x,y)$ satisfying $x^2-dy^2=1$ will be units in $\OO_{\QQ(\sqrt d)}$. Note that the $-1$ case is also explained \Cref{lem:unit-by-norm-one} because being a unit permits $\op N_{K/\QQ}(x+y\sqrt d)=-1$.

To continue, we observe that there is something a little off with \Cref{prop:solve-pell-norm-one}. Namely, the proposition is only solving units in $\OO_{\QQ(\sqrt d)}^\times$ of the form $x+y\sqrt d$ where $x,y\in\ZZ$, but we saw in \Cref{prop:quad-o-k} that sometimes we have a denominator of $2$ present. Explicitly, one could use \Cref{prop:solve-pell-norm-one} to look for units in $\ZZ[\sqrt 5]^\times$ even though $\ZZ[\sqrt 5]\ne\OO_{\QQ(\sqrt5)}$. As such, in the statement we prove, we will not want to only focus on the rings $\OO_K$ but generalizes of them.
\begin{definition}[order]
	Fix a number field $K$ of degree $n$ over $\QQ$. Then an \textit{order} $\OO$ is a subring of $\OO_K$ which is a free abelian group of rank $n$. As with $\OO_K$, we will abuse notation and write
	\[\disc\OO\coloneqq\disc(\alpha_1,\ldots,\alpha_n)\]
	for any basis $\alpha_1,\ldots,\alpha_n$ of $\OO$. Note $\disc\OO$ is still well-defined by \Cref{lem:disc-int-change-of-basis}.
\end{definition}
\begin{example}
	Fix a number field $K$. Then $\OO_K$ is itself an order by \Cref{thm:o-k-is-free}.
\end{example}
\begin{example}
	Let $d$ be a non-square integer, and set $K\coloneqq\QQ(\sqrt d)$. Then
	\[\ZZ[\sqrt d]\coloneqq\{a+b\sqrt d:a,b\in\ZZ\}\]
	is a subring of algebraic integers which is a free abelian group of rank $2$ (with basis given by $1$ and $\sqrt d$).
\end{example}
We are now almost ready to state our result. For technical reasons, we will want the notion of a signature. 
\begin{definition}[signature]
	Fix a number field $K$ of degree $n$ over $\QQ$, and let $\sigma_1,\ldots,\sigma_n\colon K\into\CC$ be the $n$ embeddings of \Cref{prop:embeddings-to-c}.
	\begin{itemize}
		\item If an embedding $\sigma\colon K\into\CC$ outputs to $\RR$, we call $\sigma$ a \textit{real embedding}.
		\item Otherwise, $\sigma\colon K\into\CC$ has output to $\CC\setminus\RR$ and is called a \textit{complex embedding}.
	\end{itemize}
	Among the $n$ embeddings $\sigma\colon K\into\CC$, we let $r_1$ denote the number of real embeddings and $2r_2$ denote the number of complex embeddings, and we let $(r_1,r_2)$ be the \textit{signature} of $K$.
\end{definition}
\begin{remark} \label{rem:complex-conjugate-complex-embed}
	It is worth explaining why the number of complex embeddings $\sigma\colon K\into\CC$ is even. Well, for any complex embedding $\sigma\colon K\into\CC$, there is a complex conjugate $\overline\sigma(\alpha)\coloneqq\overline{\sigma(\alpha)}$ embedding. Because there is $\alpha\in K$ with $\sigma(\alpha)\in\CC\setminus\RR$, we see that $\overline\sigma(\alpha)\ne\sigma(\alpha)$, so $\overline\sigma\ne\sigma$, so these are in fact distinct embeddings. Thus, $\sigma\mapsto\overline\sigma$ defines a map from complex embeddings to complex embeddings, and $\overline{\overline\sigma}=\sigma$ implies that this is an involution, so it follows that the number of complex embeddings is even: we may pair a complex embedding $\sigma$ off with its complex conjugate embedding!
\end{remark}
At long last, here is our result.
\begin{theorem}[Dirichlet unit] \label{thm:dirichlet-unit}
	Fix a number field $K$ of signature $(r_1,r_2)$. Let $\mu(K)$ denote the group of roots of unity in $K$. Let $\OO\subseteq\OO_K$ be an order, and let $\mu(\OO)$ be the roots of unity in $\OO$. Then
	\[\OO^\times\cong\mu(\OO)\times\ZZ^{r_1+r_2-1}.\]
	In other words, there is a set of units $u_1,\ldots,u_{r_1+r_2-1}$ such that, for any unit $u\in\OO_K^\times$, there is a unique root of unity $\zeta$ and integers $n_1,\ldots,n_{r_1+r_2-1}$ such that $u=\zeta u_1^{n_1}\cdots u_{r_1+r_2-1}^{n_{r_1+r_2-1}}$.
\end{theorem}
We are not going to prove \Cref{thm:dirichlet-unit} at all in this section; we will postpone until we have discussed a little Minkowski theory. For now, we satisfy ourselves with an example.
\begin{example}
	Let's show that \Cref{thm:dirichlet-unit} appropriately generalizes \Cref{prop:solve-pell-norm-one}. Fix a non-square positive integer $d$. Then $K\coloneqq\QQ(\sqrt d)$ has signature $(2,0)$, and $\mu(K)=\{\pm1\}$ because $\{\pm1\}$ are the only roots of unity in $\RR$. Thus, \Cref{thm:dirichlet-unit} implies that the order $\ZZ[\sqrt d]$ has
	\[\ZZ[\sqrt d]^\times\cong\{\pm1\}\times\ZZ.\]
	Tracking \Cref{lem:unit-by-norm-one} backward tells us that any solution $x^2-dy^2=\pm1$ has $x+y\sqrt d=\pm\left(x_0+y_0\sqrt d\right)^n$ for some unique sign $\pm$ and integer $n$. One can then reduce to $x^2-dy^2=1$ as a subgroup of $\ZZ[\sqrt d]^\times$.
\end{example}

\subsection{Dirichlet's Unit Theorem: Upper Bound} \label{subsec:dirichlet-upper}
In this subsection, we prove what we can from \Cref{thm:dirichlet-unit} without using any Minkowski theory. The goal, roughly speaking, is to explain what the $r_1+r_2-1$ is doing in the statement. In the discussion which follows, let $K$ be a number field of degree $n$ and signature $(r_1,r_2)$, and we let $\rho_1,\ldots,\rho_{r_1}\colon K\into\CC$ denote the real embeddings, and we let $\sigma_1,\ldots,\sigma_{r_2}$ be a subset of complex embeddings so that $\sigma_1,\ldots,\sigma_{r_2},\overline{\sigma_1},\ldots,\overline{\sigma_{r_2}}$ provides all complex embeddings. (See \Cref{rem:complex-conjugate-complex-embed}.) Additionally, let $\OO$ be an order.

The conclusion of \Cref{thm:dirichlet-unit} features the additive group $\ZZ$, but $\OO^\times$ is a largely multiplicative object. We would thus like to turn our multiplicative problem and turn it into an additive one, which is done by taking $\log$s. To begin, the multiplicative problem we are interested in solving is essentially trying to ensure the equation
\[\prod_{i=1}^{r_1}\left|\rho_i(u)\right|\cdot\prod_{i=1}^{r_2}\left|\sigma_i(u)\right|^2=1,\]
which for $u\in\OO_K$ we know is equivalent to $u\in\OO_K^\times$ by \Cref{lem:unit-by-norm-one} and \Cref{cor:norm-tr-by-embeds}. To make this equation additive, we note that it is equivalent to
\begin{equation}
	\sum_{i=1}^{r_1}\log\left|\rho_i(u)\right|+\sum_{i=1}^{r_2}\log\left|\sigma_i(u)\right|^2=0, \label{eq:log-of-norm-unit}
\end{equation}
provided that $u\in K^\times$. Let's break down what just happened into two steps.
\begin{enumerate}
	\item We use the embeddings to map $K$ into some Euclidean space. With our enumeration, the most obvious thing to do is via the map $K\to\RR^{r_1}\times\CC^{r_2}$ given by $\alpha\mapsto(\rho_1(\alpha),\ldots,\rho_{r_1}(\alpha),\sigma_1(\alpha),\ldots,\sigma_{r_2}(\alpha))$. However, we would like to work with real vector spaces, so we use the basis $\{1,i\}$ of $\CC$ as an $\mathbb R$-vector space to define $j\colon K\to\RR^n$ by
	\[j\colon\alpha\mapsto(\rho_1(\alpha),\ldots,\rho_{r_1}(\alpha),\op{Re}\sigma_1(\alpha),\op{Im}\sigma_1(\alpha),\ldots,\op{Re}\sigma_{r_2}(\alpha),\op{Im}\sigma_{r_2}(\alpha)).\]
	By construction, we see that $j\colon K\to\RR^n$ is a homomorphism of additive groups.
	\item After mapping $j\colon K\to\RR^n$, we would like to take logarithms, so we define the map $\op{Log}\colon(\RR^\times)^n\to\RR^{r_1+r_2}$ by
	\[\op{Log}(x_1,\ldots,x_{r_1},a_1,b_1,\ldots,a_{r_2},b_{r_2})\coloneqq\left(\log\left|x_1\right|,\ldots,\log\left|x_{r_2}\right|,\log\left|a_1^2+b_1^2\right|,\ldots,\log\left|a_{r_2}^2+b_{r_2}^2\right|\right).\]
	Observe that this is the same $\op{Log}$ map that we saw in \Cref{prop:general-pell}, and it will be useful for approximately the same reason: this map does a good job of measuring the ``multiplicative'' height of a nonzero element in $\OO$.

	Anyway, for $\alpha\in K^\times$, we have $\sigma(\alpha)\ne0$ for each embedding $\sigma$, so $\mathrm{Log}(j(\alpha))$ is well-defined. And by construction (and by properties of $\log$), we see that the composite $({\op{Log}}\circ j)\colon K^\times\to\RR^{r_1+r_2}$ is a homomorphism of groups.
	
	If in addition $\alpha\in\OO_K^\times$, then \eqref{eq:log-of-norm-unit} tells us that $\op{Log}(j(\alpha))$ lands in
	\[H\coloneqq\left\{(x_1,\ldots,x_{r_1+r_2}):\sum_{i=1}^{r_1+r_2}x_i=0\right\}\subseteq\RR^{r_1+r_2},\]
	which we call the ``trace-$0$ hyperplane.'' Note that this is a hyperplane of $\RR^{r_1+r_2}$ cut out by a single equation, so $\dim H=r_1+r_2-1$. This is where the $r_1+r_2-1$ in \Cref{thm:dirichlet-unit} will come from.
\end{enumerate}
In order to justify our use of lattices, we note that $\OO\subseteq K$ should in some sense feel like a ``discrete subgroup'' (compare this with $\ZZ\subseteq\QQ$), so it is reasonable to expect that $j(\OO)\subseteq\RR^n$ and $\op{Log}(j(\OO^\times))\subseteq H$ are discrete subgroups of Euclidean space and thus lattices. We show this now.
\begin{proposition} \label{prop:order-lattice}
	Fix a number field $K$, and fix notation as above. Then $j(\OO)\subseteq\RR^n$ is a lattice of rank $n$ with covolume $\op{vol}(\RR^n/j(\OO))=\frac1{2^{r_2}}\sqrt{\left|\disc\OO\right|}$.
\end{proposition}
\begin{proof}
	By definition, $\OO$ is a free abelian group of rank $n$, so produce a basis $\alpha_1,\ldots,\alpha_n$. We claim that $j(\alpha_1),\ldots,j(\alpha_n)$ provides a basis for $j(\OO)\subseteq\RR^n$. Certainly these elements span $j(\OO)$ because $j\colon\OO\to\RR^n$ is a homomorphism of additive groups, implying any $\alpha\in\OO$ can be written as
	\[j(\alpha)=j(c_1\alpha_1+\cdots+c_n\alpha_n)=c_1j(\alpha_1)+\cdots+c_nj(\alpha_n)\]
	for some integers $c_1,\ldots,c_n\in\ZZ$. 
	
	Now, to compute the covolume, we need to compute the determinant of the (transpose of the) matrix
	\[\begin{bmatrix}
		\rho_1(\alpha_1) & \cdots & \rho_{r_1}(\alpha_1) & \op{Re}\sigma_1(\alpha_1) & \op{Im}\sigma_1(\alpha_1) & \cdots & \op{Re}\sigma_{r_2}(\alpha_1) & \op{Im}\sigma_{r_2}(\alpha_1) \\
		\vdots & \ddots & \vdots & \vdots & \vdots & \ddots & \vdots & \vdots \\
		\rho_1(\alpha_n) & \cdots & \rho_{r_1}(\alpha_n) & \op{Re}\sigma_1(\alpha_n) & \op{Im}\sigma_1(\alpha_n) & \cdots & \op{Re}\sigma_{r_2}(\alpha_n) & \op{Im}\sigma_{r_2}(\alpha_n) \\
	\end{bmatrix}.\]
	We would like to make this matrix look like the matrix for $\disc(\alpha_1,\ldots,\alpha_n)$. Multiply each real part column by $i$ times the imaginary column, which makes our determinant equal
	\[\det\begin{bmatrix}
		\rho_1(\alpha_1) & \cdots & \rho_{r_1}(\alpha_1) & \sigma_1(\alpha_1) & \op{Im}\sigma_1(\alpha_1) & \cdots & \sigma_{r_2}(\alpha_1) & \op{Im}\sigma_{r_2}(\alpha_1) \\
		\vdots & \ddots & \vdots & \vdots & \vdots & \ddots & \vdots & \vdots \\
		\rho_1(\alpha_n) & \cdots & \rho_{r_1}(\alpha_n) & \sigma_1(\alpha_n) & \op{Im}\sigma_1(\alpha_n) & \cdots & \sigma_{r_2}(\alpha_n) & \op{Im}\sigma_{r_2}(\alpha_n) \\
	\end{bmatrix}.\]
	We now multiply each imaginary part column $\op{Im}\sigma_i(\alpha_j)$ by $-2i$ and then add the corresponding $\sigma_i(\alpha_j)$ term to produce $\overline{\sigma_i}(\alpha_j)$, thus making our determinant
	\[\frac1{(-2i)^{r_2}}\det\begin{bmatrix}
		\rho_1(\alpha_1) & \cdots & \rho_{r_1}(\alpha_1) & \sigma_1(\alpha_1) & \overline{\sigma_1}(\alpha_1) & \cdots & \sigma_{r_2}(\alpha_1) & \overline{\sigma_{r_2}}(\alpha_1) \\
		\vdots & \ddots & \vdots & \vdots & \vdots & \ddots & \vdots & \vdots \\
		\rho_1(\alpha_n) & \cdots & \rho_{r_1}(\alpha_n) & \sigma_1(\alpha_n) & \overline{\sigma_1}(\alpha_n) & \cdots & \sigma_{r_2}(\alpha_n) & \overline{\sigma_{r_2}}(\alpha_n) \\
	\end{bmatrix}.\]
	Taking absolute values, we see that this is $\frac1{2^{r_2}}\sqrt{\left|\disc(\alpha_1,\ldots,\alpha_n)\right|}$, as needed.
\end{proof}
\begin{proposition} \label{prop:log-j-o-is-lattice}
	Fix a number field $K$, and fix notation as above. Then $\op{Log}(j(\OO^\times))\subseteq H$ is a lattice.
\end{proposition}
\begin{proof}
	The point is to apply \Cref{prop:how-to-lattice}. Fix any $M>0$, and we show that $\op{Log}(j(\OO^\times))\cap[-M,M]^{r_1+r_2}$ is finite, which will be enough by \Cref{prop:how-to-lattice} because we already know that $\op{Log}(j(\OO^\times))\subseteq H$ from the discussion above.

	Well, we simply pull back to $\RR^n$. Indeed, $\op{Log}(x_1,\ldots,x_{r_1},a_1,b_1,\ldots,a_{r_2},b_{r_2})\in[-M,M]^n$ implies that $\left|x_\bullet\right|\le e^M$ and $\left|a_\bullet\right|,\left|b_\bullet\right|<e^{M/2}$ always. Thus, $\op{Log}(j(\alpha))\in[-M,M]^n$ implies that
	\[j(\alpha)\in[-\exp(M),\exp(M)]^n,\]
	but only finitely many $\alpha$ satisfy this by \Cref{prop:how-to-lattice} because $j(\OO)\subseteq\RR^n$ is a lattice by \Cref{prop:order-lattice}.
\end{proof}
We know that $\op{Log}(j(\OO^\times))$ is a lattice, so it is a free abelian group of rank at most $\dim H=r_1+r_2-1$. However, the statement of \Cref{thm:dirichlet-unit} includes some roots of unity. Where did they go?
\begin{lemma} \label{lem:ker-log-is-mu-k}
	Fix a number field $K$, and fix notation as above. Then $\ker({\op{Log}}\circ j)$ is the set $\mu(\OO)$ of roots of unity in $\OO$, and $\mu(\OO)$ is finite.
\end{lemma}
\begin{proof}
	Quickly, note that some $\alpha\in\OO_K^\times$ lives in $\ker({\op{Log}}\circ j)$ if and only if $\left|\sigma(\alpha)\right|=1$ for all embeddings $\sigma\colon K\into\CC$. Let $S\subseteq\OO^\times$ be the set of such $\alpha$. In one direction, certainly $\mu(\OO)\subseteq S$: for any $\zeta\in\mu(\OO)$, we have $\zeta^n=1$ for some $n$, so $\left|\sigma(\zeta)\right|^n=1$ and thus $\left|\sigma(\zeta)\right|=1$ for all embeddings $\sigma\colon K\into\CC$.

	For the other direction, we will show that $S$ is finite; this will complete the proof because $S\subseteq\OO^\times$ is a subgroup, meaning that any $\alpha\in S$ has $\alpha^{\#S}=1$ and thus $\alpha\in\mu(\OO)$. Anyway, to show that $S$ is finite, we use the proof of \Cref{prop:log-j-o-is-lattice}: note that any $\alpha$ in the kernel must have $\left|\sigma(\alpha)\right|=1$ for all embeddings $\sigma\colon K\into\CC$. But then
	\[\ker({\op{Log}}\circ j)\subseteq j(\OO)\cap[-1,1]^n\]
	is finite by \Cref{prop:order-lattice}, so we are done.
\end{proof}
\begin{remark}
	One can more directly show that any $\alpha\in\OO_K$ such that $\left|\sigma(\alpha)\right|=1$ for all embeddings $\sigma\colon K\into\CC$ must be a root of unity. Here is the argument. Let $S\subseteq\OO_K$ be the set of such $\alpha$. For each $\alpha$, let $f(x)\in\ZZ[x]$ be the corresponding monic irreducible polynomial (see \Cref{lem:monic-irred-of-alg-int}) so that
	\[f(x)=\prod_{\sigma\colon\QQ(\alpha)\into\CC}(x-\sigma(\alpha)).\]
	However, a direct expansion reveals that the $x^r$ coefficient of this polynomial is the sum of $\binom d{d-r}$ complex numbers of absolute value $1$, where $d\le n$ is the degree of $f(x)$. Thus, the set of polynomials $f(x)\in\ZZ[x]$ which can correspond to some $\alpha\in S$ is finite, so $S$ is finite. However, we can see that $S\subseteq K^\times$ is a subgroup, so it follows that $\alpha^{\#S}=1$ for any $\alpha\in S$.
\end{remark}
We now combine \Cref{lem:ker-log-is-mu-k} with \Cref{prop:log-j-o-is-lattice} to achieve the result, which is quite close to \Cref{thm:dirichlet-unit}.
\begin{proposition} \label{prop:almost-dirichlet-unit}
	Fix a number field $K$, and fix notation as above. Let $r$ be the rank of the lattice $\op{Log}(j(\OO^\times))\subseteq H$. Then
	\[\OO^\times\cong\mu(\OO^\times)\times\ZZ^r.\]
\end{proposition}
\begin{proof}
	Let $v_1,\ldots,v_r\in\op{Log}(j(\OO^\times))$ be a basis, and find any $\alpha_1,\ldots,\alpha_r\in\OO^\times$ so that $\op{Log}(j(\alpha_i))=v_i$ for each $i$. Then we define the function
	\[\varphi\colon\mu(\OO^\times)\times\ZZ^r\to\OO^\times\]
	by $\varphi(\zeta,(e_1,\ldots,e_r))\coloneqq\zeta\alpha_1^{e_1}\cdots\alpha_r^{e_r}$. We claim that $\varphi$ is a group isomorphism, which will complete the proof. Here are our checks.
	\begin{itemize}
		\item Homomorphism: we check
		\[\varphi\left(\zeta\zeta',(e_1+e_1',\ldots,e_r+e_r')\right)=\zeta\zeta'\cdot\alpha_1^{e_1+e_1'}\cdots\alpha_r^{e_r+e_r'}=\varphi\left(\zeta,(e_1,\ldots,e_r)\right)\varphi\left(\zeta',(e_1',\ldots,e_r')\right).\]
		\item Injective: suppose $\varphi(\zeta,(e_1,\ldots,e_r))=1$. Then
		\[\alpha_1^{e_1}\cdots\alpha_r^{e_r}=\zeta^{-1},\]
		so passing through ${\op{Log}}\circ j$ tells us that $e_1v_1+\cdots+e_rv_r=0$ by \Cref{lem:ker-log-is-mu-k}, so $(e_1,\ldots,e_r)=(0,\ldots,0)$ because the $v_\bullet$ are linearly independent. But then the above equation implies $\zeta=1$ as well.
		\item Surjective: fix $u\in\OO^\times$. Then $\op{Log}(j(u))\in\op{Log}(j(\OO^\times))$ has some $(e_1,\ldots,e_r)\in\ZZ^r$ such that
		\[\op{Log}(j(u))=e_1v_1+\cdots+e_rv_r.\]
		But then $\op{Log}\left(j(u\alpha_1^{-e_1}\cdots\alpha_r^{-e_r})\right)=0$, so $u\alpha_1^{-e_1}\cdots\alpha_r^{-e_r}\in\mu(\OO)$, so $u=\zeta\alpha_1^{e_1}\cdots\alpha_r^{e_r}$ for some $\zeta\in\mu(\OO)$.
		\qedhere
	\end{itemize}
\end{proof}
% https://math.stackexchange.com/q/246157
% nevermind I think that's too complicated ...

\subsection{Dirichlet's Unit Theorem: Lower Bound}
We continue with the notation of the previous subsection; for example, $K$ is a number field, and $\OO\subseteq\OO_K$ is an order. From \Cref{prop:almost-dirichlet-unit}, it remains to compute the rank of the lattice $\op{Log}(j(\OO^\times))\subseteq H$.
\begin{proposition} \label{prop:compute-unit-rank}
	Fix a number field $K$ and an order $\OO\subseteq\OO_K$, and fix notation as in \cref{subsec:dirichlet-upper}. Then $\op{Log}(j(\OO^\times))\subseteq H$ is a lattice of rank $r_1+r_2-1$.
\end{proposition}
Note \Cref{thm:dirichlet-unit} would then follow immediately from \Cref{prop:almost-dirichlet-unit,prop:compute-unit-rank}. It remains to prove \Cref{prop:compute-unit-rank}, which is the goal of this subsection. This requires exhibiting many units. For this, our argument will be similar in spirit to \Cref{prop:pell-has-fund-sol}: we will produce lots of elements of small norm, and we will use quotients of these in order to produce the required units.

To begin, we need to know that there are not many elements of small norm.
\begin{lemma} \label{lem:element-norm-is-ideal-norm}
	Fix a number field $K$ and an order $\OO\subseteq\OO_K$. For any nonzero $\alpha\in\OO$, we have $[\OO:(\alpha)]=\left|\op N_{K/\QQ}(\alpha)\right|$.
\end{lemma}
\begin{proof}
	By definition, $\op N_{K/\QQ}(\alpha)$ is the determinant of the multiplication-by-$\alpha$ map $\mu_\alpha\colon\OO\to\OO$, whose image is exactly $(\alpha)$. But \Cref{lem:lattice-index-is-covol} then tells us that $\left|\det\mu_\alpha\right|=[\OO:(\alpha)]$, so the result follows.
\end{proof}
\begin{lemma} \label{lem:finite-ideals-of-index}
	Fix a number field $K$ and an order $\OO\subseteq\OO_K$. For any positive integer $N$, there are only finitely many ideals $I\subseteq\OO$ such that $[\OO:I]=N$.
\end{lemma}
\begin{proof}
	In fact, there are only finitely many additive subgroups $I$ of $\OO\cong\ZZ^n$ of index $N$. Well, any subgroup $I\subseteq\ZZ^n$ of index $N$ makes the quotient $\ZZ^n/I$ annihilated by $N$, so $NI\supseteq\ZZ^n$. Thus, $N\ZZ^n\subseteq I\subseteq\ZZ^n$, so $I$ may be recovered by its image $I/N\ZZ^n\subseteq\ZZ^n/N\ZZ^n$, but $\ZZ^n/N\ZZ^n$ is a finite group, so there are only finitely many options for $I$.
\end{proof}
\begin{proposition} \label{prop:finite-element-of-norm}
	Fix a number field $K$ and an order $\OO\subseteq\OO_K$. For any positive integer $N$, there are only finitely many $\alpha\in\OO/\OO^\times$ such that $\left|\op N_{K/\QQ}(\alpha)\right|=N$.
\end{proposition}
\begin{proof}
	Quickly, note $\left|\op N_{K/\QQ}(\alpha)\right|$ is well-defined for $\alpha\in\OO/\OO^\times$ because $\alpha\in\OO_K^\times$ is equivalent to $\left|\op N_{K/\QQ}(\alpha)\right|=1$ by \Cref{lem:unit-by-norm-one}.

	Now, we note that $\alpha\cdot\OO^\times=\alpha'\cdot\OO^\times$ if and only if $(\alpha)=(\alpha')$ by tracking through our principal ideals. Thus, \Cref{lem:element-norm-is-ideal-norm} tells us that we are asking for finitely (principal) ideals $I\subseteq\OO$ such that $[\OO:I]=N$. This finiteness follows from \Cref{lem:finite-ideals-of-index}.
\end{proof}
We now use \Cref{prop:finite-element-of-norm} to produce units. To begin, we need many elements of small norm.
\begin{lemma} \label{lem:decrease-coords-bound-norm}
	Fix a number field $K$ and an order $\OO\subseteq\OO_K$, and fix notation as in \cref{subsec:dirichlet-upper}. Further, fix an index $1\le i_0\le r_1+r_2$. Then there is an absolute constant $C(\OO)>0$ with the following property: for any nonzero $\alpha\in\OO$, there is a nonzero $\alpha'\in\OO$ such that $\left|\op N_{K/\QQ}(\alpha')\right|\le C(\OO)$ and writing
	\[\op{Log}(j(\alpha))=(a_1,\ldots,a_{r_1+r_2})\qquad\text{and}\qquad\op{Log}(j(\alpha'))=(a'_1,\ldots,a'_{r_1+r_2})\]
	requires $a'_i<a_i$ for each $i\ne i_0$.
\end{lemma}
\begin{proof}
	We use \Cref{thm:mink}. Fix $C(\OO)>0$ to be determined later. Our lattice will be $j(\OO)\subseteq\RR^n$, which we know to be of rank $n$ by \Cref{prop:order-lattice}. To achieve $a'_i<a_i$, we define $S\subseteq\RR^n$ by
	\[S\coloneqq\left\{(x_1,\ldots,x_n):\left|x_1\right|<e^{c_1},\ldots,\left|x_{r_1}\right|<e^{c_{r_1}},\left|x_{r_1+1}^2+x_{r_1+2}^2\right|<e^{c_{r_1+1}},\ldots,\left|x_{n-1}^2+x_n^2\right|<e^{c_{r_1+r_2}}\right\},\]
	where we require $c_i=a_i$ for each $i\ne i_0$ and
	\[e^{c_{i_0}}\coloneqq\frac{C(\OO)}{\prod_{i\ne i_0}e^{c_i}}.\]
	Notably, $S$ is the product of $r_1$ intervals and $r_2$ circles, so its volume is
	\[\op{vol}(S)=2^{r_1}e^{c_1+\cdots+c_{r_1}}\cdot\pi^{r_2}e^{c_{r_1+1}\cdots c_{r_1+r_2}}=2^{r_1}\pi^{r_2}C(\OO).\]
	For $C(\OO)$ big enough, we see $2^{r_1}\pi^{r_2}C(\OO)>\op{vol}(\RR^n/j(\OO))$ (note that $C(\OO)$ only needs to depend on $\OO$), so \Cref{thm:mink} yields a nonzero $j(\alpha')\in j(\OO)$ such that $\alpha'\in S$. Looking at the construction of $S$, we see that the inequalities on $\op{Log}(j(\alpha'))$ hold by construction (look at $c_i=a_i$ for $i\ne i_0$), and $\left|\op N_{K/\QQ}(\alpha')\right|\le C(\OO)$ again hold by construction (look at $c_{i_0}$).
\end{proof}
And here are our units.
\begin{lemma} \label{lem:special-negative-unit}
	Fix a number field $K$ and an order $\OO\subseteq\OO_K$, and fix notation as in \cref{subsec:dirichlet-upper}. For any index $1\le i_0\le r_1+r_2$, there is a unit $\gamma\in\OO^\times$ such that writing
	\[\op{Log}(j(\gamma))=(u_1,\ldots,u_{r_1+r_2})\]
	has $u_i<0$ for $i\ne i_0$. In fact, $u_{i_0}>0$.
\end{lemma}
\begin{proof}
	Quickly, note that having $u_i<0$ for $i\ne i_0$ implies $u_{i_0}>0$ because the coordinates need to sum to $0$ because $\op{Log}(j(u))\subseteq H$.

	Anyway, fix $C(\OO)$ as in \Cref{lem:decrease-coords-bound-norm}. We now inductively use \Cref{lem:decrease-coords-bound-norm} to produce many elements of norm bounded by $C(\OO)$, which will be required to give us units by \Cref{prop:finite-element-of-norm}. To begin, we may assume $C(\OO)>1$, and we set $\alpha_1\coloneqq1$. Then we apply \Cref{lem:decrease-coords-bound-norm} on $\alpha_1$ to produce some $\alpha_2$ with norm at most $C(\OO)$ as well, and then we do this again to produce $\alpha_3$, and so on.

	This produces an infinite list of elements of norm at most $C(\OO)$, but the set of classes in $\OO/\OO^\times$ represented by such elements is finite by \Cref{prop:finite-element-of-norm}, so we may find distinct elements of the sequence $\alpha$ and $\alpha'$ which differ by a unit so that $\alpha\gamma=\alpha'$. We claim that $u$ is the desired unit: by construction of the sequence, we see that writing
	\begin{align*}
		\op{Log}(j(\alpha)) &= (a_1,\ldots,a_{r_1+r_2}) \\
		\op{Log}(j(\alpha')) &= (a'_1,\ldots,a'_{r_1+r_2}) \\
		\op{Log}(j(\gamma)) &= (u_1,\ldots,u_{r_1+r_2})
	\end{align*}
	requires $a'_i<a_i$ for each $i\ne i_0$, so we are done upon noting $u_i=a_i'-a_i$ for each $i$.
\end{proof}
Applying \Cref{lem:special-negative-unit} to each unit, we produce units $\gamma_1,\ldots,\gamma_{r_1+r_2}\in\OO^\times$ such that exactly the $i$th component of the $\op{Log}(j(\gamma_i))$ is positive. Now, to prove \Cref{prop:compute-unit-rank}, it is enough to produce $r_1+r_2-1$ linearly independent vectors, meaning we want to show
\[\op{rank}\begin{bmatrix}
	| & & | \\
	\op{Log}(j(\gamma_1)) & \cdots & \op{Log}(j(\gamma_{r_1+r_2-1})) \\
	| & & |
\end{bmatrix}=r_1+r_2-1.\]
It turns out to be easier to consider the transpose matrix, whereupon the result becomes the following piece of linear algebra.
\begin{lemma}
	Let $A=(a_{ij})_{1\le i,j,\le n}$ be a matrix such that the rows sum to zero and $a_{ij}<0$ for each $i\ne j$ and $a_{ii}>0$ for each $i$. Then $\op{rank}A=n-1$.
\end{lemma}
\begin{proof}
	Certainly $\op{rank}A<n$ because $(1,\ldots,1)\in\RR^{n\times n}$ is in the kernel. To establish $\op{rank}A\ge n-1$, we will show that the first $n-1$ columns of $A$ are linearly independent. Enumerate the columns of $A$ as $v_1,\ldots,v_n$. Any nontrivial linear relation among the first $n-1$ of these columns
	\[\sum_{i=1}^{n-1}c_iv_i=0\]
	may find some $\left|c_{i_0}\right|>0$ as large as possible and then divide the entire equation by $c_{i_0}$ so that the new equation has $c_{i_0}=1$ while $\left|c_i\right|\le1$ for each $i$. However, row $i_0<n$ now has
	\[0=\sum_{i=1}^{n-1}c_ia_{i_0i}\ge\sum_{i=1}^na_{i_0i}>\sum_{i=1}^na_{i_0i}=0,\]
	which is a contradiction.
\end{proof}
\Cref{prop:compute-unit-rank} now follows from the lemma and the discussion immediately preceding it. This completes the proof of \Cref{thm:dirichlet-unit}.

\subsection{A Harder Problem Revisited}
We are now equipped to understand and prove \Cref{prop:pell-cbrt-2}. The following lemma will be useful.
\begin{lemma} \label{lem:cubic-fund-unit}
	Fix a cubic number field $K$ with exactly one real embedding $\rho\colon K\into\RR$. For any order $\OO\subseteq\OO_K$, we have $\OO^\times\cong\mu(\OO)\times\ZZ$, and any unit $u\in\OO^\times$ with $\rho(u)>1$ will in fact have
	\[\rho(u)^3+7>\frac14\left|\disc\OO\right|.\]
\end{lemma}
\begin{proof}
	The first claim is a direct consequence of \Cref{thm:dirichlet-unit}: having exactly one real embedding implies that the signature $(r_1,r_2)$ is $(1,r_2)$ where $1+2r_2=[K:\QQ]=3$, meaning the signature is actually $(r_1,r_2)=(1,1)$. Then $r_1+r_2-1=1$, so we are done by \Cref{thm:dirichlet-unit}.

	It remains to show the inequality. The point is that $u\in\OO$ implies that $\left|\disc(1,u,u^2)\right|\ge\left|\disc\OO\right|$ by \Cref{lem:disc-change-of-basis}. To compute $\disc(1,u,u^2)$, let $\sigma\colon K\into\CC$ be a complex embedding so that our embeddings are $\rho,\sigma,\overline\sigma\colon K\into\CC$. Define $r\coloneqq1/\sqrt{\rho(u)}<1$ so that $\left|\sigma(u)\right|^2=\left|\op N_{K/\QQ}(u)\right|/\rho(u)=r^2$ (we are using \Cref{lem:unit-by-norm-one}) and so $\sigma(u)=re^{i\theta}$ for some $\theta$. As such, we compute
	\begin{align*}
		\disc\left(1,u,u^2\right) &= \det\begin{bmatrix}
			1 & \rho(u) & \rho(u)^2 \\
			1 & \sigma(u) & \sigma(u)^2 \\
			1 & \overline\sigma(u) & \overline\sigma(u)^2
		\end{bmatrix}^2 \\
		&= \left(\left(\sigma(u)\overline\sigma(u)^2-\sigma(u)^2\overline\sigma(u)\right)-\left(\rho(u)\overline\sigma(u)^2-\rho(u)^2\sigma(u)\right)+\left(\rho(u)\sigma(u)^2-\rho(u)^2\sigma(u)\right)\right)^2 \\
		&= \left((\rho(u)-\sigma(u))(\sigma(u)-\overline\sigma(u))(\overline\sigma(u)-\rho(u))\right)^2 \\
		&= (\sigma(u)-\overline\sigma(u))^2\left|\rho(u)-\sigma(u)\right|^2 \\
		&= -4r^2(\sin\theta)^2\left|1/r^2-r\cos\theta-ir\sin\theta\right|^4 \\
		&= -4r^2(\sin\theta)^2\left((1/r^2-r\cos\theta)^2+(r\sin\theta)^2\right)^2 \\
		&= -4r^2(\sin\theta)^2\left(1/r^4-2(1/r)\cos\theta+r^2\right)^2 \\
		&= -4(\sin\theta)^2\left(r^3+1/r^3-2\cos\theta\right)^2.
	\end{align*}
	This is negative, so we take absolute values to see
	\[\frac14\left|\disc\OO\right|<(\sin\theta)^2\left(r^3+\frac1{r^3}-2\cos\theta\right)^2.\]
	It remains to bound the right-hand side. To manipulate, set $c\coloneqq\cos\theta$ and $s\coloneqq r^3+1/r^3$; note $s\ge2$ because $r^3+1/r^3-2\ge\left(r^3-1/r^3\right)\ge0$. As such,
	\begin{align*}
		(\sin\theta)^2\left(r^3+\frac1{r^3}-2\cos\theta\right) &= \left(1-c^2\right)(s-2c)^2 \\
		&= \left(1-c^2\right)\left(s^2-4sc+4c^2\right) \\
		&= s^2-4sc+4c^2-s^2c^2+4sc^3-4c^4 \\
		&= s^2+4-\left(sc+2-2c^2\right)^2.
	\end{align*}
	Thus,
	\[\frac14\left|\disc\OO\right|<\left(r^3+\frac1{r^3}\right)^2+4=r^6+\frac1{r^6}+6<\rho(u)^3+7,\]
	as needed.
\end{proof}
\begin{proposition} \label{prop:units-cbrt-2}
	The cubic number field $K\coloneqq\QQ(\sqrt[3]2)$ has exactly one real embedding, and $\OO\coloneqq\ZZ[\sqrt[3]2]$ is an order. The element $u\coloneqq1+\sqrt[3]2+\sqrt[3]4$ is a unit, and any element of $\OO^\times$ can be written uniquely in the form $\pm u^n$ for some integer $n$.
\end{proposition}
\begin{proof}
	By the argument of \Cref{prop:embeddings-to-c}, $K\cong\QQ[x]/\left(x^3-2\right)$ has the real embedding $\sqrt[3]2\mapsto\sqrt[3]2$ and two complex embeddings $\sqrt[3]2\mapsto e^{2\pi i/3}\sqrt[3]2$ and $\sqrt[3]2\mapsto e^{4\pi i/3}\sqrt[3]2$. In the argument which follows, we identify $K$ with its embedding in $\RR$. To finish up the claims of the first sentence, we see that $\OO$ has basis given by $\left\{1,\sqrt3[2],\sqrt[3]4\right\}$: these certainly generate $\OO$, and they are a $\QQ$-linearly independent basis of $K$, so they are certainly linearly independent over $\ZZ$.

	Next, we note that $u$ is in fact a unit because $\op N_{K/\QQ}(u)=1+2+4-6=1$ by \Cref{ex:norm-k-cbrt-2}. It remains to show that any other unit in $\OO$ can be written in the form $\pm u^n$. This follows by \Cref{thm:dirichlet-unit} and \Cref{lem:cubic-fund-unit}. Because $\OO\subseteq\RR$, we see that $\mu(\OO)=\{\pm1\}$. (All other roots of unity in $\CC$ do not live in $\RR$.) It remains to find a unit $u_0\in\OO^\times$ such that
	\[\OO^\times=\{\pm1\}\times u_0^\ZZ,\]
	which exists by \Cref{thm:dirichlet-unit}. By adjusting the sign of $u_0$, we may assume $u_0>0$. By replacing $u_0$ with $1/u_0$ as needed, we may assume $u_0>1$.

	Now, we see that we have $u=u_0^n$ for some positive integer $n$. We claim that $u=u_0$, which will complete the proof. It is enough to show that $u<u_0^2$, or equivalently, $u^3<u_0^6$. Well, recall $\disc\OO=-108$, so $u_0^3>\frac14\cdot108-7=20$. But now $u^3<(1+2+4)^3=343<400$, so $u^3<u_0^6$, so we are done.
\end{proof}
At long last, \Cref{prop:units-cbrt-2} follows from \Cref{prop:pell-cbrt-2}.
\pellcbrttwo*
\begin{proof}
	By \Cref{ex:norm-k-cbrt-2}, any solution to $x^3+2y^3+4z^3-6xyz=1$ is really a norm-$1$ unit of $x+y\sqrt[3]2+z\sqrt[3]4$. Looking at the units provided by \Cref{prop:units-cbrt-2}, we see that the norm-$1$ units are of the form
	\[x_n+y_n\sqrt[3]2+z_n\sqrt[3]4=(1+\sqrt[3]2+\sqrt[3]4)^n\]
	for an integer $n\in\ZZ$. Explicitly, the sign $-1$ has norm $-1$, and $\op N_{\QQ(\sqrt[3]2)/\QQ}(1+\sqrt[3]2+\sqrt[3]4)=1$, so the units of the form $(1+\sqrt[3]2+\sqrt[3]4)^n$ have norm $1$, and the units of the form $-(1+\sqrt[3]2+\sqrt[3]4)^n$ have norm $-1$. To finish, and the recursion provided by \Cref{prop:units-cbrt-2} exactly describes the needed triples $(x_n,y_n,z_n)$, so we are done!
\end{proof}

\subsection{Problems}
Do ten points worth of the following exercises and then complete the survey.
% \begin{prob}[1 point]
% 	Find a basis $\{(x_1,y_1),(x_2,y_2)\}$ of the lattice $\ZZ^2\subseteq\RR^2$ such that $x_1,y_1,x_2,y_2>10$.
% \end{prob}
% \begin{prob}[2 points]
% 	Let $\Lambda\subseteq\RR^n$ be a lattice. Show that there is a vector $v\in\Lambda\setminus\{0\}$ such that $\norm v$ is minimized among all values in $\Lambda\setminus\{0\}$.
% \end{prob}
\begin{prob}[1 point]
	Complete at least one of the following.
	\begin{listalph}
		\item Take a walk outside for at least $5$ minutes.
		\item Do $10$ jumping jacks.
		\item Breathe deeply $10$ times.
	\end{listalph}
\end{prob}
\begin{prob}[2 points]
	Let $K$ be a number field with signature $(r_1,r_2)$. Show that the image of the map $j\colon K\to\RR^{r_1+2r_2}$ is dense.
\end{prob}
\begin{prob}[2 points]
	Suppose that $K$ is a number field and has at least one real embedding $\rho\colon K\into\RR$. Show that the only roots of unity in $K$ are $\{\pm1\}$.
\end{prob}
\begin{prob}[3 points]
	Compute the signature of the following number fields $K$.
	\begin{listalph}
		\item $K=\QQ(\sqrt{-2},\sqrt{-3})$.
		\item $K=\QQ(\sqrt2,\sqrt[3]3)$.
		\item $K=\QQ(i,\sqrt[3]3)$.
	\end{listalph}
	It is sufficient to simply list the embeddings.
\end{prob}
\begin{prob}[3 points]
	Let $a,b>1$ be coprime squarefree integers, and set $K\coloneqq\QQ(\sqrt{-a},\sqrt{-b})$ and $\OO\coloneqq\ZZ[\sqrt{-a},\sqrt{-b}]$. Show that $\OO\subseteq\OO_K$ is an order, and show that
	\[\OO^\times\cong\mu(\OO)\times\ZZ.\]
\end{prob}
\begin{prob}[3 points]
	Let $a_1,a_2,b_1,b_2,c_1,c_2$ be positive real numbers, and define the matrix
	\[M\coloneqq\begin{bmatrix}
		a_1+a_2 & -a_1 & -a_2 \\
		-b_1 & b_1+b_2 & -b_2 \\
		-c_1 & -c_2 & c_1+c_2
	\end{bmatrix}.\]
	Show that $M$ has rank $2$ by showing it is not invertible but has an invertible $2\times2$ minor.
\end{prob}
\begin{prob}[4 points]
	Let $S^1\subseteq\CC^\times$ denote the subgroup of elements all of whose absolute values are $1$, and let $G\subseteq S^1$ be a finite subgroup 
	\begin{listalph}
		\item Consider the map $\pi\colon\RR\to S^1$ given by $\pi(t)\coloneqq\exp(2\pi it)$. Show that $\pi^{-1}(G)$ is lattice in $\RR$.
		\item Use (a) to show that $G$ is cyclic.
		\item Use (b) to show that $\mu(\OO)$ is cyclic for any order $\OO$ of a number field $K$.
	\end{listalph}
\end{prob}
% \begin{prob}[3 points]
% 	Let $\Lambda'$ and $\Lambda$ be lattices in $\RR^n$ of rank $n$. Suppose $\Lambda'\subseteq\Lambda$.
% 	\begin{listalph}
% 		\item Show that there is a positive integer $m$ such that $m\op{vol}\left(\RR^n/\Lambda\right)=\op{vol}\left(\RR^n/\Lambda'\right)$.
% 		\item If $\op{vol}\left(\RR^n/\Lambda\right)=\op{vol}\left(\RR^n/\Lambda'\right)$, show that $\Lambda=\Lambda'$.
% 	\end{listalph}
% \end{prob}
\begin{prob}[6 points]
	Show that $\ZZ[\sqrt[3]7]\subseteq\QQ(\sqrt[3]7)$ is an order, and compute its discriminant. Then compute $u\in\ZZ[\sqrt[3]7]^\times$ such that
	\[\ZZ[\sqrt[3]7]^\times=\{\pm1\}\times u^\ZZ.\]
\end{prob}
\begin{prob}[7 points]
	Classify the integer solutions to
	\[x^3+3y^3+9z^3-9xyz=1\]
	in a way akin to \Cref{prop:pell-cbrt-2}.
\end{prob}
% \begin{prob}[4 points]
% 	Fix a prime $p$ such that there exists an integer $x$ such that $x^2\equiv-2\pmod p$. Show that there is a pair of integers $(a,b)$ such that $p=a^2+2b^2$.
% \end{prob}
% \begin{prob}[5 points]
% 	The following problem requires the notion of a closed set: a subset $S\subseteq\RR^n$ is said to be ``closed'' if and only if any convergent sequence $\{a_n\}_{n=0}^\infty$ contained in $S$ has limit in $S$.
% 	\begin{listalph}
% 		\item (0 points) For experience, show that $[-1,1]^n\subseteq\RR^n$ is a closed set for any positive integer $n$.
% 		\item (5 points) Let $\Lambda\subseteq\RR^n$ be a lattice of rank $n$. For any closed, convex, symmetric about the origin subset $S\subseteq\RR^n$ such that
% 		\[\op{vol}(S)\ge2^n\op{vol}\left(\RR^n/\Lambda\right),\]
% 		show that $S$ contains a nonzero lattice point of $\Lambda$.
% 	\end{listalph}
% \end{prob}
\begin{prob}[0 points]
	Please rate the speed of the following lectures, from ``much too slow'' to ``much too fast.''
	\begin{itemize}
		\item October 23: Equivalence of Forms
		\item October 25: Reduction of Forms
		\item October 27: Quadratic Residues
	\end{itemize}
	Please also rate the difficulty of the problems on the homework you solved.
\end{prob}

\end{document}