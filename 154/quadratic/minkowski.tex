% !TEX root = ../notes.tex

\documentclass[../notes.tex]{subfiles}

\begin{document}

\section{Minkowski Theory}

Having spent a long time building the theory of number rings, we will take a break to discuss some geometry of numbers. We will then return and prove Dirichlet's unit theorem, \Cref{thm:dirichlet-unit}.

\subsection{Minkowski's Theorem}

\subsection{Sample Applications of Minkowski's Theorem}

\subsection{Dirichlet's Unit Theorem: Upper Bound}
In this subsection, we prove what we can from \Cref{thm:dirichlet-unit} without using any Minkowski theory. The goal, roughly speaking, is to explain what the $r_1+r_2-1$ is doing there. In the discussion which follows, let $K$ be a number field of degree $n$ and signature $(r_1,r_2)$, and we let $\rho_1,\ldots,\rho_{r_1}\colon K\into\CC$ denote the real embeddings, and we let $\sigma_1,\ldots,\sigma_{r_2}$ be a subset of complex embeddings so that $\sigma_1,\ldots,\sigma_{r_2},\overline{\sigma_1},\ldots,\overline{\sigma_{r_2}}$ provides all complex embeddings. (See \Cref{rem:complex-conjugate-complex-embed}.)

The conclusion of \Cref{thm:dirichlet-unit} features the additive group $\ZZ$, but $\OO^\times$ is a largely multiplicative object. We would thus like to turn our multiplicative problem and turn it into an additive one, which is done by taking $\log$s. To begin, the multiplicative problem we are interested in solving is essentially trying to ensure the equation
\[\prod_{i=1}^{r_1}\left|\rho_i(u)\right|\cdot\prod_{i=1}^{r_2}\left|\sigma_i(u)\right|^2=1,\]
which for $u\in\OO_K$ we know is equivalent to $u\in\OO_K^\times$ by \Cref{lem:unit-by-norm-one} and \Cref{cor:norm-tr-by-embeds}. To make this equation additive, we note that it is equivalent to
\[\sum_{i=1}^{r_1}\log\left|\rho_i(u)\right|+2\sum_{i=1}^{r_2}\log\left|\sigma_i(u)\right|=0,\]
provided that $u\in K^\times$. Let's break down what just happened into two steps.
\begin{enumerate}
	\item We use the embeddings to map $K$ into some Euclidean space. With our enumeration, the most obvious thing to do is via the map $K\to\RR^{r_1}\times\CC^{r_2}$ given by $\alpha\mapsto(\rho_1(\alpha),\ldots,\rho_{r_1}(\alpha),\sigma_1(\alpha),\ldots,\sigma_{r_2}(\alpha))$. However, we would like to work with real vector spaces, so we use the basis $\{1,i\}$ of $\CC$ as an $\mathbb R$-vector space to define $j\colon K\to\RR^n$ by
	\[j\colon\alpha\mapsto(\rho_1(\alpha),\ldots,\rho_{r_1}(\alpha),\op{re}\sigma_1(\alpha),\op{Im}\sigma_1(\alpha),\ldots,\op{Re}\sigma_{r_2}(\alpha),\op{Im}\sigma_{r_2}(\alpha)).\]
	\item After mapping $j\colon K\to\RR^n$, we would like to take logarithms, so we define the map $\op{Log}\colon\RR^n\to\RR^{r_1+r_2}$ by
	\[\op{Log}(x_1,\ldots,x_{r_1},a_1,b_1,\ldots,a_{r_2},b_{r_2})\coloneqq\left(\log\left|x_1\right|,\ldots,\log\left|x_{r_2}\right|,\log\left|a_1^2+b_1^2\right|,\ldots,\log\left|a_{r_2}^2+b_{r_2}^2\right|\right).\]
	Notably, for $\alpha\in K^\times$, we have $\sigma(\alpha)\ne0$ for each embedding $\sigma$, so
	\[\]
\end{enumerate}
Note that $\OO$ is a free abelian group of rank $n$, so $j(\OO)$ becomes a free abelian group of rank $n$ sitting inside $\RR^n$. We would like $j(\OO)$ to be a lattice, a term which we now define.
\begin{definition}[lattice]
	Fix a finite-dimensional real vector $V$. Then a \textit{lattice} $\Lambda\subseteq V$ is the set of vectors of the form
	\[\{a_1v_1+\cdots+a_mv_m:a_1,\ldots,a_m\in\ZZ\},\]
	where $v_1,\ldots,v_m$ are some linearly independent vectors.
\end{definition}
\begin{remark}
	Notably, linear independence implies that $\ZZ^m\cong\Lambda$ by the isomorphism $(a_1,\ldots,a_m)\mapsto a_1v_1+\cdots+a_mv_m$.
\end{remark}
% https://math.stackexchange.com/q/246157

\subsection{Dirichlet's Unit Theorem: Lower Bound}


\end{document}