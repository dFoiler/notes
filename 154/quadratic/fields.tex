% !TEX root = ../notes.tex

\documentclass[../notes.tex]{subfiles}

\begin{document}

\section{Number Rings}

In our solutions to Pell equations, we frequently ran into numbers of the form
\[a+b\sqrt d\]
where $a,b\in\ZZ$ and $d$ is a positive integer which is not a square. But in our explanation of \Cref{ex:pell-5} we even ran into numbers of the form
\[\frac{a+b\sqrt 5}2\]
where $a$ and $b$ had the same parity. The goal of this section is to contextualize what is going on here. Starting in this section, we will assume basic ring theory, on the level of any reasonable abstract algebra text. We refer to \Cref{app:fields} for the necessary field theory.

% Throughout, our focus will be on generalizing our story with $\sqrt d$s to general field extensions of $\QQ$, which are called number fields.
% \begin{definition}[number field]
% 	A \textit{number field} $K$ is a finite field extension of $\QQ$. In other words, the degree $[K:\QQ]\coloneqq\dim_\QQ K$ of the field extension is finite.
% \end{definition}

\subsection{Normal Domains}
It is a question of classical interest in number theory to take an integral domain and ask if it is a unique factorization domain: in some sense, unique factorization domains are ``the best'' rings (e.g., they are computationally nice to work with because their multiplicative structure can be well-understood). However, it is in general somewhat difficult to do such a check, and one spends a good part of an algebraic number theory learning how to do so.

To start off, a basic hypothesis is that the integral domain be ``normal.'' For motivation, we recall the Rational root theorem.
\begin{theorem}[rational root for $\ZZ$]
	Let $f(x)\in\ZZ[x]$ be a monic polynomial with integer coefficients. If $q$ is a rational root of $f(x)$, then $q$ is an integer.
\end{theorem}
More generally, we will prove the following more general result.
\begin{theorem}[rational root] \label{thm:rrt-for-ufd}
	Let $A$ be a unique factorization domain with fraction field $K$, and let $f(x)\in A[x]$ be a monic polynomial with coefficients in $A$. If $q\in K$ is a root of $f(x)$, then $q\in A$.
\end{theorem}
\begin{proof}
	Quickly, if $q=0$, there is nothing to say. Otherwise, by using unique factorization, we may write $q=a/b$ where $a$ and $b$ have no irreducible factors in common. Explicitly, by unique factorization, we may express $q$ as a quotient of nonzero elements in $A$ by writing
	\[q=\dfrac{u\prod_{i=1}^np_i^{\alpha_i}}{\prod_{j=1}^np_i^{\beta_i}},\]
	where $u$ is a unit, $\alpha_\bullet$ and $\beta_\bullet$ are nonnegative integers, and each $p_\bullet$ is a unique irreducible (not equal to the product of any other $p_\bullet$ and a unit). Then we may write
	\[q=\underbrace{u\prod_{\substack{1\le i\le n\\\alpha_i>\beta_i}}p^{\alpha_i-\beta_i}}_{a\coloneqq}\bigg/\underbrace{\prod_{\substack{1\le i\le n\\\alpha_i<\beta_i}}p^{\beta_i-\alpha_i}}_{b\coloneqq}\]
	Notably, if $\alpha_i=\beta_i$, then we may remove $p_i$.

	We now proceed with the usual proof of the Rational root theorem. Write
	\[f(x)=x^d+\sum_{k=0}^{d-1}r_kx^k\]
	where $d=\deg f$ and $r_0,r_1,\ldots,r_{d-1}\in A$. We are given that $f(a/b)=0$. As such, the main point is that we can manipulate $f(a/b)=0$ into
	\[0=a^d+\sum_{k=0}^{d-1}r_ka^kb^{d-k}=a^d+b\sum_{k=0}^{d-1}r_ka^kd^{(d-1)-k}.\]
	To show that $q\in A$, we will show that $b$ is a unit, which indeed implies that $q=ab^{-1}\in A$. Because $A$ is a unique factorization domain, it suffices to show that no irreducible element $p$ divides $b$. Well, if $p\mid b$, then $p$ divides the sum above, so reducing$\pmod p$ requires $p\mid a^d$, and so $p\mid a$ because $p$ is prime by \Cref{lem:prime-is-irred-for-ufd}; however, no irreducible divides both $a$ and $b$ by their construction, so we are done.
\end{proof}
Let's see how \Cref{thm:rrt-for-ufd} can detect when a ring is fails to be a unique factorization domain.
\begin{example} \label{ex:z-root-5-not-normal}
	Construct the ring $\ZZ[\sqrt 5]\coloneqq\left\{a+b\sqrt 5:a,b\in\ZZ\right\}$ where addition and multiplication are as expected. Then $\ZZ[\sqrt 5]$ is not a unique factorization domain.
\end{example}
\begin{solution}
	Quickly, we check that $\ZZ[\sqrt 5]$ is in fact a subring of (say) $\CC$: we have all the needed identities, and we are closed under the needed operations because
	\begin{align*}
		\left(a+b\sqrt 5\right)+\left(a'+b'\sqrt 5\right) &= (a+a')+(b+b')\sqrt 5, \\
		\left(a+b\sqrt 5\right)\cdot\left(a'+b'\sqrt 5\right) &= (aa'+5bb')+(ab'+ba')\sqrt 5.
	\end{align*}
	(We will not check that $\ZZ[\alpha_1,\ldots,\alpha_n]$ is a ring in the future.) Anyway, the main content of this example is to consider the polynomial
	\[f(x)\coloneqq x^2-x-1,\]
	which is monic and has integer coefficients (in particular, the coefficients are in $\ZZ[\sqrt 5]$). Using the quadratic formula, we see that $\frac{1+\sqrt5}2$ is a root of $f(x)$ which lives in the quotient field of $\ZZ[\sqrt 5]$ but not in $\ZZ[\sqrt 5]$. Thus, \Cref{thm:rrt-for-ufd} tells us that $\ZZ[\sqrt 5]$ cannot be a unique factorization domain!
\end{solution}
\begin{exe}
	More generally, let $d\equiv1\pmod4$ be an integer which is not a square. Then show that the set
	\[\ZZ[\sqrt d]\coloneqq\left\{a+b\sqrt d:a,b\in\ZZ\right\}\]
	is a subring of $\CC$ but fails to be a unique factorization domain because $\frac{1+\sqrt d}2$ is the root of a monic polynomial with integer coefficients.
\end{exe}
The test we are applying is worth turning into an adjective.
\begin{definition}[normal]
	Fix an integral domain $A$ with fraction field $K$. Then $A$ is said to be \textit{normal} if and only if the following holds: for any monic polynomial $f(x)\in A[x]$, if $q\in K$ is a root of $f(x)$, then $q\in A$.
\end{definition}
\begin{example}
	\Cref{thm:rrt-for-ufd} is equivalent to the statement that unique factorization domains are normal.
\end{example}
\begin{nex}
	The content of \Cref{ex:z-root-5-not-normal} is showing that $\ZZ[\sqrt5]$ is not normal.
\end{nex}
It does turn out that the rings $\ZZ[\sqrt2]$ and $\ZZ[\sqrt3]$ are normal, but we will hold off showing this until we have built a little more theory.\footnote{For example, it is possible to show that these rings are unique factorization domains directly. This approach does not generalize to further $\ZZ[\sqrt d]$ because, for example, $\ZZ[\sqrt{15}]$ is normal but not a unique factorization domain.}

\subsection{Number Rings}
Notably, we showed that $\ZZ[\sqrt5]$ is not normal by only looking at monic polynomials with coefficients in $\ZZ$. This is surprising because the definition of a normal domain allows our coefficients to live in the full ring $\ZZ[\sqrt5]$, but we only used $\ZZ$! It will turn out that using $\ZZ$ is enough, though showing this requires a little effort. Regardless, we are motivated to make the following definitions.
\begin{definition}[algebraic integer]
	Fix a field extension $K$ of $\QQ$. An element $\alpha\in K$ is an \textit{algebraic integer} if and only if $\alpha$ is the root of some monic polynomial in $\ZZ[x]$.
\end{definition}
\begin{definition}[number ring]
	Fix a finite field extension $K$ of $\QQ$. Then the \textit{number ring} $\mathcal O_K$ of $K$ consists of all the algebraic integers in $K$.
\end{definition}
\begin{example} \label{ex:o-q-is-z}
	We see that $\OO_\QQ=\ZZ$ by \Cref{thm:rrt-for-ufd}. Explicitly, any element $n\in\ZZ$ is the root of the monic polynomial $x-n\in\ZZ[x]$. On the other hand, if $\alpha\in\QQ$ is an algebraic integer, then $\alpha$ is the root of a monic polynomial in $\ZZ[x]$, so $\alpha\in\ZZ$ by \Cref{thm:rrt-for-ufd}.
\end{example}
Though we will not dwell on it too much, it is worth acknowledging that the following generalized (relative) notion is more correct to work with.
\begin{definition}[integral]
	Fix an embedding of rings $A\subseteq B$. An element $\alpha\in B$ is \textit{integral over $A$} if and only if $\alpha$ is the root of some monic polynomial in $A[x]$.
\end{definition}
\begin{example}
	Fix a field extension $K$ of $\QQ$. Then $\alpha\in K$ is an algebraic integer if and only if $\alpha$ is integral over $\ZZ$.
\end{example}
We do need to check that $\OO_K$ is in fact a ring. Of course, $0$ and $1$ are algebraic integers (they are the roots of the monic polynomials $x$ and $x-1$, respectively), so it remains to show that $\OO_K$ is closed under addition and multiplication. This requires a little work. The following result is known as the ``determinant trick'' in commutative algebra. Note that this result (and its corollaries) is basically \Cref{prop:how-to-algebraic}.
\begin{proposition} \label{prop:how-to-integral}
	Fix an embedding of rings $A\subseteq B$ and some $\alpha\in B$. Then the following are equivalent.
	\begin{listalph}
		\item $\alpha$ is integral over $A$.
		\item The ring $A[\alpha]$ is finitely generated as an $A$-module.
		\item There is a subring $A'\subseteq B$ finitely generated as an $A$-module containing both $A$ and $\alpha$.
	\end{listalph}
\end{proposition}
Here, a ring $A'$ containing $A$ is finitely generated as an $A$-module (or ``finitely generated over $A$'') if and only if there are finitely many elements $r_1',\ldots,r_n'$ such that each $r'\in A'$ can be written as
\[r'=\sum_{k=1}^na_kr_k'\]
for some $a_1,\ldots,a_n\in A$. In other words, each element of $A'$ is an $A$-linear combination of some fixed finite set in $A'$.
\begin{proof}
	We show the implications separately, following \Cref{prop:how-to-algebraic}.
	\begin{itemize}
		\item To see that (a) implies (b), we suppose $\alpha$ is the root of the monic polynomial $f(x)\in A[x]$ of degree $d$, written as
		\[f(x)=x^d+\sum_{k=0}^{d-1}a_kx^k.\]
		Then, for any $n\ge d$, we can express $\alpha^n$ as a $\ZZ$-linear combination of lower powers because
		\[\alpha^n=-\sum_{k=0}^{d-1}a_k\alpha^{k+d-n}.\]
		It follows that $A[\alpha]$ is generated by the elements $1,\alpha,\alpha^2,\ldots,\alpha^{d-1}$.
		\item Note that (b) implies (c) by setting $A'=A[\alpha]$.
		\item Checking that (c) implies (a) is harder. If $A'=0$, then $\alpha=0$, so there is nothing to say; otherwise, $A'\ne0$. Suppose $A'$ is generated by the elements $r_1',r_2',\ldots,r_n'$. Note that $\alpha r_i'\in A'$ for each $r_i'$, so we may write
		\[\alpha r_i'=\sum_{j=1}^na_{ij}r_j'\]
		for some elements $a_{ij}\in A$. In other words, the matrix $T\coloneqq(a_{ij})_{i,j=1}^n$ has
		\[\alpha\begin{bmatrix}
			r_1' \\
			\vdots \\
			r_n'
		\end{bmatrix}=T\begin{bmatrix}
			r_1' \\
			\vdots \\
			r_n'
		\end{bmatrix}.\]
		Thus, $T-\alpha I_n$ is an $n\times n$ matrix with entries in $A'$, and it has the nonzero vector $(r_1',\ldots,r_n')$ in its kernel, so $\det(T-\alpha I_n)=0$. Expanding out the polynomial $\det(\alpha I_n-T)=0$ makes $\alpha$ the root of a monic polynomial (of degree $n$) with coefficients in $A$, so $\alpha$ is indeed integral over $A$.
		\qedhere
	\end{itemize}
\end{proof}
In our application, we will also want the following lemma.
\begin{lemma} \label{lem:finite-extension-transitive}
	Let $A\subseteq B\subseteq C$ be embeddings of rings. If $C$ is finitely generated as a $B$-module, and $B$ is finitely generated as an $A$-module, then $C$ is finitely generated as an $A$-module.
\end{lemma}
\begin{proof}
	The point is to concatenate our generating sets together; indeed, this argument is basically the same as \Cref{lem:deg-in-towers}. Say that $B$ is generated over $A$ by the elements $b_1,\ldots,b_m$, and say that $C$ is generated over $B$ by the elements $c_1,\ldots,c_n$. Then any $c\in C$ has some $b_1',\ldots,b_n'\in B$ such that
	\[c=\sum_{\ell=1}^nb_\ell'c_\ell,\]
	but now each $b_\ell'$ can be expanded over $A$ as
	\[c=\sum_{\ell=1}^n\sum_{k=1}^ma_{k\ell}b_kc_\ell\]
	for some $a_{k\ell}\in A$. Thus, the elements $b_kc_\ell$ for $1\le k\le m$ and $1\le\ell\le n$ generate $C$ over $A$, so we are done.
\end{proof}
\begin{corollary} \label{cor:o-k-normal-ring}
	Fix a finite field extension $K$ of $\QQ$. Then $\OO_K$ is a normal ring.
\end{corollary}
\begin{proof}
	This argument is essentially \Cref{cor:alg-elements-form-field}. We run our checks separately.
	\begin{itemize}
		\item We check that $\OO_K$ is a ring. As discussed previously, $0,1\in\OO_K$ because these elements are the roots of the polynomials $x$ and $x-1$, respectively. It remains to show that, for any $\alpha,\beta\in\OO_K$, we have $\alpha+\beta,\alpha\beta\in\OO_K$. The main point is to show that $\ZZ[\alpha,\beta]$ is finitely generated as a $\ZZ$-module, which will complete the proof by \Cref{prop:how-to-integral} because $\alpha+\beta,\alpha\beta\in\ZZ[\alpha,\beta]$.
	
		Well, let $\alpha$ and $\beta$ be the roots of the monic polynomials $f(x),g(x)\in\ZZ[x]$ respectively. Then by \Cref{prop:how-to-integral} shows that $\ZZ[\beta]$ is finitely generated as a $\ZZ$-module, and $f(\alpha)=0$ shows that $\alpha$ is integral over $\ZZ[\beta]$, so $\ZZ[\alpha,\beta]$ is finitely generated as a $\ZZ[\beta]$-module. We conclude $\ZZ[\alpha,\beta]$ is finitely generated as a $\ZZ$-module by \Cref{lem:finite-extension-transitive}.

		\item We check that $\OO_K$ is normal. Suppose that $\alpha\in K$ is the root of the monic polynomial $f(x)\in\OO_K[x]$; we show that $\alpha\in\OO_K$ by showing that $\alpha$ is an algebraic integer. Well, expand $f(x)$ as
		\[f(x)=x^d+\sum_{k=0}^{d-1}a_kx^k\]
		for some $a_0,\ldots,a_{d-1}\in\OO_K$. Each $a_\bullet$ is integral over $\ZZ$, so \Cref{prop:how-to-integral} tells us that $\ZZ[a_\bullet]$ for each $a_\bullet$. As such, as in the previous check, we may build the tower
		\[\ZZ\subseteq\ZZ[a_0]\subseteq\ZZ[a_0,a_1]\subseteq\cdots\subseteq\ZZ[a_0,\ldots,a_{d-1}],\]
		where each ring is finitely generated over the previous one by \Cref{prop:how-to-integral}. Then \Cref{lem:finite-extension-transitive} tells us that $\ZZ[a_0,\ldots,a_{d-1}]$ is finitely generated as a $\ZZ$-module. Lastly, $f(\alpha)=0$ tells us that $\alpha$ is integral over $\ZZ[a_0,\ldots,a_{d-1}]$, so $\ZZ[a_0,\ldots,a_{d-1},\alpha]$ is finitely generated as a $\ZZ[a_0,\ldots,a_{d-1}]$-module and hence finitely generated as a $\ZZ$-module by \Cref{lem:finite-extension-transitive}, meaning that $\alpha$ is integral over $\ZZ$ by \Cref{prop:how-to-integral}.
		\qedhere
	\end{itemize}
\end{proof}
As another application of \Cref{prop:how-to-integral}, we dispose of the following annoying technicality: as discussed in \Cref{lem:minimal-poly}, any algebraic element in a field extension will be the root of some unique irreducible polynomial. When integral, we would like this polynomial to have integral coefficients, but this is not technically obvious, so we place this check into the following lemma.
\begin{lemma}
	Fix a finite field extension $K$ of $\QQ$ and some $\alpha\in\OO_K$. Let $f(x)\in\QQ[x]$ be the unique monic irreducible polynomial with $f(\alpha)=0$. Then $f(x)\in\ZZ[x]$.
\end{lemma}
\begin{proof}
	Embed $K$ into $\CC$, via (say) \Cref{prop:embeddings-to-c}. Then $f(x)$ will factor in $\CC$ as
	\[f(x)=\prod_{i=1}^n(x-\alpha_i).\]
	Now, we know that $\alpha\in\OO_K$ is the root of some monic $g(x)\in\ZZ[x]$, so \Cref{lem:minimal-poly} tells us that $g(x)=f(x)q(x)$ for some $q(x)\in\QQ[x]$; comparing leading coefficients, we see that the leading coefficient of $q$ is also $1$. Now, for any root $\alpha_i\in\CC$ of $f$, we see that $g(\alpha_i)=0$, so $\alpha_i\in\CC$ is also an algebraic integer.

	Thus, we consider the ring $A\coloneqq\ZZ[\alpha_1,\ldots,\alpha_n]$. Iteratively applying \Cref{prop:how-to-integral}, the fact that each $\alpha_i$ is integral implies that $A$ is finitely generated as a $\ZZ$-module, and so every element of $A$ is integral over $\ZZ$. But direct expansion shows that the coefficients of $f$ live in $A$ and hence are all integral over $\ZZ$ and hence are integers by \Cref{ex:o-q-is-z}.
\end{proof}
\begin{remark} \label{rem:norm-and-tr-ints}
	It follows from \Cref{prop:norm-and-tr-by-min-poly} and \Cref{cor:norm-tr-by-embeds} that the trace and norm of algebraic integer is an integer. Indeed, the stated results show that these values arise as integer multiples of coefficients of the minimal polynomial, and the above lemma shows that the minimal polynomial has integer coefficients.
\end{remark}
\begin{example} \label{ex:norm-k-cbrt-2}
	Fix $K\coloneqq\QQ(\sqrt[3]2)$. Then
	\[\op N_{K/\QQ}\left(a+b\sqrt[3]2+c\sqrt[3]4\right)=a^3+2b^3+4c^3-6abc.\]
	In particular, if $a,b,c\in\ZZ$, then this is an integer.
\end{example}
\begin{solution}
	One way to solve this is via \Cref{cor:norm-tr-by-embeds} and direct expansion. A slightly more conceptual way to do this is by directly appealing to the definition of the norm. Set $\alpha\coloneqq a+b\sqrt[3]2+c\sqrt[3]4$ for brevity. Then note that the elements $1,\sqrt[3]2,\sqrt[3]4$ form a basis for $K$ as a $\QQ$-vector space (see, for example, \Cref{lem:quotient-of-poly-ring}), so the computations
	\begin{align*}
		\alpha &= a+b\sqrt[3]2+c\sqrt[3]4, \\
		\alpha\sqrt[3]2 &= 2c+a\sqrt[3]2+b\sqrt[3]4, \\
		\alpha\sqrt[3]4 &= 2b+2c\sqrt[3]2+a\sqrt[3]4,
	\end{align*}
	tell us that we can represent the mutliplication-by-$\alpha$ map $K\to K$ by the matrix
	\[\begin{bmatrix}
		a & 2c & 2b \\
		b & a & 2c \\
		c & b & a
	\end{bmatrix}.\]
	One can directly compute that the determinant of this matrix is $a\left(a^2-2bc\right)-b\left(2ac-2b^2\right)+c\left(4c^2-2ab\right)$, which simplifies correctly.
\end{solution}

\subsection{Number Rings of Quadratic Extensions}
After doing all that theory, we are owed an example, so we will compute $\OO_K$ for quadratic field extensions $K=\QQ(\sqrt d)$ of $\QQ$. We will want the following lemma, which we have been using in some guise for quite a bit of the previous section.
\begin{lemma} \label{lem:quadratic-conjugation}
	Fix $K\coloneqq\QQ(\sqrt d)$ where $d$ is a non-square integer. Then the function $\sigma\colon K\to K$ given by $\sigma\left(a+b\sqrt d\right)\coloneqq a-b\sqrt d$ is a ring homomorphism.
\end{lemma}
\begin{proof}
	To begin, note that $\sigma$ is well-defined because $a+b\sqrt d=a'+b'\sqrt d$ implies that $(a-a')=(b'-b)\sqrt d$ and so $a=a'$ and $b=b'$ because $\sqrt d$ is irrational. To check that $\sigma$ is a homomorphism, we note that $\sigma(0)=0$ and $\sigma(1)=1$ and
	\begin{align*}
		\sigma\left((a+b\sqrt d)+(a'+b'\sqrt d)\right) &= (a+a')-(b+b')\sqrt d \\
		&= \sigma(a+b\sqrt d)+\sigma(a'+b'\sqrt d) \\
		\sigma\left((a+b\sqrt d)(a'+b'\sqrt d)\right) &= (aa'+dbb')-(ab'+ba')\sqrt d \\
		&= \sigma\left(a+b\sqrt d\right)\sigma\left(a'+b'\sqrt d\right),
	\end{align*}
	completing the proof.
\end{proof}
\begin{example}
	Fix $K\coloneqq\QQ(\sqrt2)$. Then $\ZZ[\sqrt2]=\OO_K$. In particular, $\ZZ[\sqrt2]$ is normal.
\end{example}
\begin{proof}
	Note that $\sqrt2\in\OO_K$ because it is the root of the polynomial $x^2-2=0$. Thus, $\ZZ[\sqrt2]$ is finitely generated as a $\ZZ$-module by \Cref{prop:how-to-integral}, and we see $\ZZ[\sqrt2]\subseteq\OO_K$.

	We now show that $\OO_K\subseteq\ZZ[\sqrt2]$, which is harder. Suppose $a+b\sqrt 2\in\OO_K$, where we allow $a,b\in\QQ$. We want to show that $a,b\in\ZZ$. Well, $a+b\sqrt2$ is the root of some monic polynomial $f(x)\in\ZZ[x]$, so $\sigma\left(a+b\sqrt2\right)=a-b\sqrt2$ is also the root of $f(x)$ by \Cref{lem:quadratic-conjugation}, so $a-b\sqrt2$ is also an algebraic integer. Thus,
	\begin{align*}
		(a+b\sqrt2)+(a-b\sqrt2)&=2a, \\
		\left((a+b\sqrt2)-(a-b\sqrt2)\right)\sqrt2&=4b, \\
		(a+b\sqrt2)(a-b\sqrt2)&=a^2-2b^2,
	\end{align*}
	are also algebraic integers. But they are also rational and hence actually integers by \Cref{ex:o-q-is-z}, so we may write $a=a_0/2$ and $b=b_0/4$ for some integers $a_0$ and $b_0$. Then
	\[a^2-2b^2=\frac{a_0^2}4-\frac{b_0^2}8=\frac{2a_0^2-b_0^2}8\]
	needs to be an integer, so $2a_0^2\equiv b_0^2\pmod8$, which can only happen if $a_0\equiv0\pmod2$ and $b_0\equiv0\pmod4$. Thus, $a$ and $b$ are in fact integers, so we conclude.
\end{proof}
More generally, we can show the following statement.
\begin{proposition} \label{prop:quad-o-k}
	Fix a non-square squarefree integer $d$, and fix $K\coloneqq\QQ(\sqrt d)$. Then we have
	\[\OO_K=\begin{cases}
		\ZZ[\sqrt d] & \text{if }d\equiv2,3\pmod4, \\
		\ZZ\left[\frac{1+\sqrt d}2\right] & \text{if }d\equiv1\pmod4.
	\end{cases}\]
\end{proposition}
\begin{proof}
	We proceed as in the example. Of course, $\sqrt d\in\OO_K$ because it is the root of $x^2-d=0$, and if $d\equiv1\pmod4$, then $\frac{1+\sqrt d}2\in\OO_K$ because it is the root of
	\[x^2-x-\frac{d-1}4=0.\]
	Thus, \Cref{prop:how-to-integral} then assures us that $\ZZ[\sqrt d]\subseteq\OO_K$ and $\ZZ\left[\frac{1+\sqrt d}2\right]\subseteq\OO_K$ when $d\equiv1\pmod4$.

	It remains to show our other inclusion. Well, fix some $a+b\sqrt d\in\OO_K$ where $a,b\in\QQ$. Because $a+b\sqrt d$ is the root of some monic polynomial with integer coefficients, we see that $a-b\sqrt d$ is as well by \Cref{lem:quadratic-conjugation}, so $a-b\sqrt d$. Thus,
	\begin{align*}
		(a+b\sqrt d)+(a-b\sqrt d)&=2a, \\
		\left((a+b\sqrt d)-(a-b\sqrt d)\right)\sqrt d&=2bd, \\
		(a+b\sqrt d)(a-b\sqrt d)&=a^2-db^2,
	\end{align*}
	are also algebraic integers. But they are also rational and hence actually integers by \Cref{ex:o-q-is-z}, so we may write $a=a_0/2$ and $b=b_0/(2d)$ for some integers $a_0$ and $b_0$. For example, we see that
	\[4\left(a^2-db^2\right)=a_0^2-\frac{b_0^2}{d}\]
	must be an integer, so because $d$ is squarefree, we conclude that $d\mid b_0$, so we write $b=b_1/2$ for some integer $b_1$. To continue the argument, we split into cases.
	\begin{itemize}
		\item Suppose $d\equiv2,3\pmod4$. Then we see
		\[a^2-db^2=\frac{a_0^2-db_1^2}{4}\]
		must be an integer. By checking$\pmod4$, we see that both $a_0$ and $b_1$ must be even, so $a$ and $b$ are both integers, so $a+b\sqrt d\in\ZZ[\sqrt d]$.
		\item Suppose $d\equiv1\pmod4$. Then we see
		\[a^2-db^2=\frac{a_0^2-db_1^2}{4}\]
		must be an integer. By checking$\pmod4$, we see that both $a_0$ and $b_0$ must have the same parity, so
		\[a+b\sqrt d=\frac{a_0-b_0}2+a_0\cdot\frac{1+\sqrt d}2\in\ZZ\left[\frac{1+\sqrt d}2\right],\]
		establishing the needed inclusion.
		\qedhere
	\end{itemize}
\end{proof}
Note that \Cref{prop:quad-o-k} fulfills a curiosity of \Cref{ex:pell-5}, namely about where the denominator of $2$ came from and why it was so controlled!

To continue our journey of generalization to compute other rings of integers, we need to generalize aspects of the above proof. For example, the $\sigma$ arises because $a+b\sqrt d\mapsto a-b\sqrt d$ is the nontrivial field embedding $\QQ(\sqrt d)\into\CC$ promised by \Cref{prop:embeddings-to-c}. Further, $2a$ and $a^2-db^2$ are the trace and norm from \cref{subsec:norm-tr}, which we knew had to be integers by \Cref{rem:norm-and-tr-ints}. However, it is not so clear where the denominator of $2d$ so quickly or why was it so annoying to argue beyond that.

\subsection{The Discriminant}
The discriminant is an invariant which roughly speaking measures the size of a number field.
\begin{definition}[discriminant]
	Fix a number field $K$ of degree $n$ over $\QQ$, and let $\sigma_1,\ldots,\sigma_n\colon K\into\CC$ denote the $n$ embeddings of \Cref{prop:embeddings-to-c}. Given $\alpha_1,\ldots,\alpha_n\in K$, we define the \textit{discriminant} to be
	\[\disc(\alpha_1,\ldots,\alpha_n)\coloneqq\det\begin{bmatrix}
		\sigma_1(\alpha_1) & \cdots & \sigma_1(\alpha_n) \\
		\vdots & \ddots & \vdots \\
		\sigma_n(\alpha_1) & \cdots & \sigma_n(\alpha_n)
	\end{bmatrix}^2.\]
\end{definition}
\begin{example} \label{ex:quad-disc}
	Let $K\coloneqq\QQ(\sqrt d)$ for some squarefree $d$. Then $\sigma_1,\sigma_2\colon K\into\CC$ are given by $a+b\sqrt d\mapsto a+b\sqrt d$ and $a+b\sqrt d\mapsto a-b\sqrt d$. Then
	\[\disc(1,\sqrt d)=\det\begin{bmatrix}
		1 & \sqrt d \\
		1 & -\sqrt d
	\end{bmatrix}=(-2\sqrt d)^2=4d.\]
\end{example}
As stated, it is somewhat annoying to compute the discriminant or to check that it is nonzero. We will spend some time explaining how to compute this and some of its basic properties. To begin, we check that the discriminant lands in $\QQ$ and in $\ZZ$ when the $\alpha_i$ live in $\OO_K$.
\begin{lemma} \label{lem:disc-by-tr}
	Fix a number field $K$ of degree $n$ over $\QQ$. Given $\alpha_1,\ldots,\alpha_n\in K$, we have
	\[\disc(\alpha_1,\ldots,\alpha_n)=\det\begin{bmatrix}
		\op T_{K/\QQ}(\alpha_1\alpha_1) & \cdots & \op T_{K/\QQ}(\alpha_1\alpha_n) \\
		\vdots & \ddots & \vdots \\
		\op T_{K/\QQ}(\alpha_n\alpha_1) & \cdots & \op T_{K/\QQ}(\alpha_n\alpha_n)
	\end{bmatrix}.\]
\end{lemma}
\begin{proof}
	Let $\sigma_1,\ldots,\sigma_n\colon K\into\CC$ denote the $n$ embeddings of \Cref{prop:embeddings-to-c}, and set $a_{ij}\coloneqq\sigma_i(\alpha_j)$ and $A\coloneqq(a_{ij})_{i,j=1}^n$ so that $\disc(\alpha_1,\ldots,\alpha_n)=\det A^2$. Now, $\det A=\det A^\intercal$, so we define $B\coloneqq A^\intercal A$ so that
	\[B_{ik}=\sum_{j=1}^nA_{ij}^\intercal A_{jk}=\sum_{j=1}^n\sigma_j(\alpha_i)\sigma_j(\alpha_k)=\op T_{K/\QQ}(\alpha_i\alpha_k).\]
	Thus, the result follows because $\disc(\alpha_1,\ldots,\alpha_n)=\det B$.
\end{proof}
\begin{corollary} \label{cor:disc-is-z}
	Fix a number field $K$ of degree $n$ over $\QQ$ and $\alpha_1,\ldots,\alpha_n\in K$. Then $\disc(\alpha_1,\ldots,\alpha_n)\in\QQ$. If $\alpha_1,\ldots,\alpha_n\in\OO_K$, then $\disc(\alpha_1,\ldots,\alpha_n)\in\ZZ$.
\end{corollary}
\begin{proof}
	We use \Cref{lem:disc-by-tr}. By definition, we see that $\op T_{K/\QQ}(\alpha_i\alpha_j)\in\QQ$ for each $i$ and $j$, so the first claim follows. If $\alpha_i\in\OO_K$ for each $i$, then $\op T_{K/\QQ}(\alpha_i\alpha_j)\in\ZZ$ for each $i$ and $j$ by \Cref{rem:norm-and-tr-ints}, establishing the second claim.
\end{proof}
\begin{proposition} \label{prop:disc-nonzero}
	Fix a number field $K$ of degree $n$ over $\QQ$. Given $\alpha_1,\ldots,\alpha_n\in K$, then $\disc(\alpha_1,\ldots,\alpha_n)\ne0$ if and only if $\alpha_1,\ldots,\alpha_n$ are $\QQ$-linearly independent.
\end{proposition}
\begin{proof}
	If the $\alpha_i$ have a $\QQ$-linear relation $a_1\alpha_1+\cdots+a_n\alpha_n$, then in fact
	\[a_1\begin{bmatrix}
		\sigma_1(\alpha_1) \\ \vdots \\ \sigma_n(\alpha_1)
	\end{bmatrix}+\cdots+a_n\begin{bmatrix}
		\sigma_1(\alpha_n) \\ \vdots \\ \sigma_n(\alpha_n)
	\end{bmatrix}=0,\]
	so the rows of the matrix defining $\disc$ are linearly dependent, implying that $\disc(\alpha_1,\ldots,\alpha_n)=0$.

	Conversely, suppose that the $\alpha_1,\ldots,\alpha_n$ are $\QQ$-linearly independent and hence form a basis of $K$ as a $\QQ$-vector space. Supposing for contradiction that $\disc(\alpha_1,\ldots,\alpha_n)=0$, then \Cref{lem:disc-by-tr} provides a nonzero vector $(a_1,\ldots,a_n)\in\QQ^n$ such that
	\[\begin{bmatrix}
		\op T_{K/\QQ}(\alpha_1\alpha_1) & \cdots & \op T_{K/\QQ}(\alpha_1\alpha_n) \\
		\vdots & \ddots & \vdots \\
		\op T_{K/\QQ}(\alpha_n\alpha_1) & \cdots & \op T_{K/\QQ}(\alpha_n\alpha_n)
	\end{bmatrix}\begin{bmatrix}
		a_1 \\ \vdots \\ a_n
	\end{bmatrix}=0,\]
	so setting $\alpha\coloneqq a_1\alpha_1+\cdots+a_n\alpha_n\ne0$, additivity of $\op T_{K/\QQ}$ implies $\op T_{K/\QQ}(\alpha_i\alpha)=0$ for each $i$. Because the $\alpha_i$ provide a basis for $K/\QQ$, we conclude $\op T_{K/\QQ}(\beta\alpha)=0$ for all $\beta\in K$, but this is  a contradiction: set $\beta=\alpha^{-1}$ so that $\op T_{K/\QQ}(1)=\sigma_1(1)+\cdots+\sigma_n(1)=n$ must vanish, which is impossible.
\end{proof}
So the discriminant can detect a basis of $K/\QQ$. In fact, it can detect change of basis as well.
\begin{lemma} \label{lem:disc-change-of-basis}
	Fix a number field $K$ of degree $n$ over $\QQ$. Given $\QQ$-linearly independent sets $\{\alpha_1,\ldots,\alpha_n\}$ and $\{\beta_1,\ldots,\beta_n\}$, let $A\in\QQ^{n\times n}$ be the change of basis matrix with $(\alpha_1,\ldots,\alpha_n)=A(\beta_1,\ldots,\beta_n)$ for each $i$. Then
	\[\disc(\alpha_1,\ldots,\alpha_n)=(\det A)^2\disc(\beta_1,\ldots,\beta_n).\] 
\end{lemma}
\begin{proof}
	Let $\sigma_1,\ldots,\sigma_n\colon K\into\CC$ denote the embeddings of \Cref{prop:embeddings-to-c}. Because $A$ has coordinates in $\QQ$, we see that
	\[\begin{bmatrix}
		\sigma_i(\alpha_1) \\
		\vdots \\
		\sigma_i(\alpha_n)
	\end{bmatrix}=A\begin{bmatrix}
		\sigma_i(\beta_1) \\
		\vdots \\
		\sigma_i(\beta_n)
	\end{bmatrix}\]
	for each $i$ because $\sigma_i$ fixes the coordinates of $A$. Thus, we produce the matrix equation
	\[\begin{bmatrix}
		\sigma_1(\alpha_1) & \cdots & \sigma_n(\alpha_1) \\
		\vdots & \ddots & \vdots \\
		\sigma_1(\alpha_n) & \vdots & \sigma_n(\alpha_n)
	\end{bmatrix}=A\begin{bmatrix}
		\sigma_1(\beta_1) & \cdots & \sigma_n(\beta_1) \\
		\vdots & \ddots & \vdots \\
		\sigma_1(\beta_n) & \vdots & \sigma_n(\beta_n)
	\end{bmatrix}.\]
	Taking determinants and squaring completes the proof.
\end{proof}
% https://math.stackexchange.com/a/2946828

\subsection{Number Ring Structure}
We use what we know about the discriminant to talk about number rings. The following will be our main result.
\begin{theorem} \label{thm:o-k-is-free}
	Fix a number field $K$. Then $\OO_K$ is a free abelian group of rank $[K:\QQ]$.
\end{theorem}
We will show \Cref{thm:o-k-is-free} by showing that $\OO_K$ contains and is contained in a free abelian group of rank $[K:\QQ]$. This will complete the proof by the following result.
\begin{lemma} \label{lem:subgroup-of-free}
	Fix a nonnegative integer $n$, and let $G$ be a subgroup of $\ZZ^n$. Then $G$ is a free abelian group of rank at most $n$.
\end{lemma}
\begin{proof}
	We induct on $n$. If $n=0$, there is nothing to say. If $n=1$, then we note that any subgroup $G\subseteq\ZZ$ is closed under $\ZZ$-linear combination and is therefore an ideal and so principal by \Cref{prop:z-pid}, which makes $G=d\ZZ$ for some $d\in\ZZ$. Now, if $G\cong0\ZZ$, then $G$ is free of rank $0$; and if $d\ne0$, then $G\cong\ZZ$ is free of rank $1$.

	For our induction, suppose $n\ge1$, and let $\pi\colon\ZZ^n\to\ZZ$ be the projection onto the last coordinate, and we note $\ker\pi=\ZZ^{n-1}\times\{0\}$ is isomorphic to $\ZZ^{n-1}$. Now, the main claim is that
	\begin{equation}
		H\cong(H\cap\ker\pi)\oplus\pi(H). \label{eq:reduce-h}
	\end{equation}
	This will complete the proof because $H\cap\ker\pi\subseteq\ker\pi\cong\ZZ^{n-1}$ must be a free abelian group of rank at most $n-1$, and $\pi(H)\subseteq\ZZ$ must be a free abelian group of rank $1$.

	It remains to show \eqref{eq:reduce-h}. If $\pi(H)=\{0\}$, then $H\subseteq\ker\pi$, so $H=H\cap\ker\pi$, and there is nothing to say. Otherwise, $\pi(H)\subseteq\ZZ$ contains a nonzero element and is isomorphic to $\ZZ$, so let $h_0\in H$ be such that $\pi(h_0)\in\pi(H)$ generates $\pi(H)$. We now show $H\cong(H\cap\ker\pi)\oplus\ZZ$: define the homomorphism $\varphi\colon(H\cap\ker\pi)\oplus\ZZ\to H$ by
	\[\varphi(h,n)\coloneqq h+nh_0.\]
	We won't bother checking that $\varphi$ is a homomorphism, but we will check that it is a bijection, which will complete the proof.
	\begin{itemize}
		\item Injective: we show trivial kernel. If $\varphi(h,n)=0$, then $h+nh_0=0$, so pushing through $\pi$ shows that $n\pi(h_0)=0$, so $n=0$. But then $h=0$ follows.
		\item Surjective: for any $h\in H$, set $n\coloneqq\pi(h)/\pi(h_0)$, which is an integer by hypothesis on $h_0$. Then
		\[\varphi(h-nh_0,n)=h-nh_0+nh_0=h,\]
		so $h\in\im\varphi$.
		\qedhere
	\end{itemize}
\end{proof}
\begin{corollary} \label{cor:get-free-rank-n}
	Let $G$ be a group which both is contained in and contains a free abelian group of rank $n$. Then $G$ is a free abelian group of rank $n$.
\end{corollary}
\begin{proof}
	Write $G\subseteq\ZZ^n$. By \Cref{lem:subgroup-of-free}, $G$ is a free group of some rank $r\le n$, so we can provide $G$ with a basis $v_1,\ldots,v_r$. Further, $G$ contains a subgroup isomorphic to $\ZZ^n$, so let $w_1,\ldots,w_n$ be the image of the basis vectors of $\ZZ^n$ in $G$.

	We will appeal to some linear algebra to complete the proof. Embed $G$ into $\QQ^n$, and let $V$ be the span of the elements of $G$. Notably, this is the span of the elements $v_1,\ldots,v_r$. Additionally, $V$ contains the $\ZZ$-lienarly independent elements $w_1,\ldots,w_n$, but being $\ZZ$-lienarly independent implies that they must be $\QQ$-linearly independent by clearing denominators on any relation. Thus, $r\ge\dim V\ge n$, so we are done.
\end{proof}
Now, showing one direction, it is not so bad to show that $\OO_K$ contains a free group of rank $n$. The point is to take any basis of $K/\QQ$ and scale the basis vectors using the following lemma.
\begin{lemma} \label{lem:clear-fraction-alg-int}
	Fix a number field $K$. For any $\alpha\in K$, there exists some positive integer $n$ such that $n\alpha\in\OO_K$.
\end{lemma}
\begin{proof}
	By \Cref{lem:finite-is-alg}, we see that $\alpha$ is at least algebraic over $K$, so we can find some monic polynomial
	\[f(x)=x^d+a_{d-1}x^{d-1}+\cdots+a_1x+a_0\]
	such that $f(\alpha)=0$, where $a_0,a_1,\ldots,a_{d-1}\in\QQ$.
	
	The idea is to clear the denominators of $f$. For each $i$, we can write $a_i=p_i/q_i$ where $q_i$ is a positive integer, and define $n\coloneqq  q_0q_1\cdots q_{d-1}$. Then we see that
	\begin{align*}
		0 &= n^d\left(\alpha^d+a_{d-1}\alpha^{d-1}+a_{d-2}\alpha^{d-2}+\cdots+a_1\alpha+a_0\right) \\
		&= (n\alpha)^d+na_{d-1}(n\alpha)^{d-1}+n^2a_{d-2}(n\alpha)^{d-2}+\cdots+n^{d-1}a_1(n\alpha)+n^da_0,
	\end{align*}
	so we define
	\[g(x)\coloneqq x^d+na_{d-1}x^{d-1}+n^2a_{d-2}+\cdots+n^{d-1}a_1x+n^da_0\]
	so that $g(x)\in\ZZ[x]$. Thus, $g(n\alpha)=0$ shows that $n\alpha\in\OO_K$.
\end{proof}
\begin{corollary} \label{cor:o-k-contains-rank-n}
	Fix a number field $K$. Then $\OO_K$ contains a free abelian group of rank $[K:\QQ]$.
\end{corollary}
\begin{proof}
	For brevity, set $n\coloneqq[K:\QQ]$, and let $\alpha_1,\ldots,\alpha_n\in K$ be a basis for $K$ as a $\QQ$-vector space. Multiplying each $\alpha_i$ by a positive integer will not change the fact that this is a basis, so \Cref{lem:clear-fraction-alg-int} allows us to assume that $\alpha_1,\ldots,\alpha_n\in\OO_K$. Now,
	\[\ZZ\alpha_1\oplus\cdots\oplus\ZZ\alpha_n\subseteq K\]
	is a free abelian group of rank $n$ (indeed, this follows because the $\alpha_i$ are $\QQ$-linearly independent and hence $\ZZ$-linearly independent), and each element lives in $\OO_K$ by \Cref{cor:o-k-normal-ring}.
\end{proof}
The other direction requires more effort. In particular, we now use the discriminant.
\begin{lemma} \label{lem:bound-denom-o-k}
	Fix a number field $K$ of degree $n$, and let $\alpha_1,\ldots,\alpha_n\in\OO_K$ generate a free abelian group of rank $n$ as promised in \Cref{cor:o-k-contains-rank-n}. Then each $\alpha\in\OO_K$ can be written as
	\[\alpha=\frac{c_1\alpha_1+\cdots+c_n\alpha_n}{\disc(\alpha_1,\ldots,\alpha_n)}\]
	for some $c_1,\ldots,c_n\in\ZZ$.
\end{lemma}
\begin{proof}
	Note that the conclusion makes sense by \Cref{prop:disc-nonzero} because the $\alpha_i$ are $\ZZ$-linearly independent and hence $\QQ$-linearly independent. So let $d\coloneqq\disc(\alpha_1,\ldots,\alpha_n)$, which we know is a nonzero integer (see also \Cref{cor:disc-is-z}).

	The idea of the proof is to use Cram\'er's rule to solve for the $c_i$. For any $\alpha\in\OO_K$, we do know that we can at least write
	\[\alpha=q_1\alpha_1+\cdots+q_n\alpha_n\]
	for some $q_1,\ldots,q_n\in\QQ$. One equation is not enough to determine the $q_i$, but we can produce more: for each of the embeddings $\sigma_1,\ldots,\sigma_n\colon K\into\QQ$ promised by \Cref{prop:embeddings-to-c}, we get an equation
	\[\sigma_i(\alpha)=q_1\sigma_i(\alpha_1)+\cdots+q_n\sigma_i(\alpha_n).\]
	We can now use Cram\'er's rule or the adjugate matrix to solve for the $x_i$ from these $n$ different equations: we find that $q_i=a_i/b$ where $a_i,b\in\OO_K$ and in particular $b=\det(\sigma_i(\alpha_j))_{i,j=1}^n$.

	To complete the proof, we note that $b^2=d$ by definition of $d$, so $q_i=ba_i/d$, so we want to set $c_i\coloneqq ba_i=q_id$. Note $c_i=q_id\in\QQ$ and $c=ba_i\in\OO_K$, so $c\in\OO_\QQ=\ZZ$ by \Cref{ex:o-q-is-z}. This completes the proof.
\end{proof}
\begin{corollary} \label{cor:o-k-in-rank-n}
	Fix a number field $K$. Then $\OO_K$ is contained in a free abelian group of rank $n$.
\end{corollary}
\begin{proof}
	Using the notation of \Cref{lem:bound-denom-o-k}, we see that
	\[\OO_K\subseteq\ZZ\frac{\alpha_1}d\oplus\cdots\oplus\ZZ\frac{\alpha_n}d\]
	where $d\coloneqq\disc(\alpha_1,\ldots,\alpha_n)$. This is what we wanted.
\end{proof}
\Cref{thm:o-k-is-free} now follows from combining \Cref{cor:get-free-rank-n,cor:o-k-contains-rank-n,cor:o-k-in-rank-n}.
\begin{remark}
	Note that the proof of \Cref{lem:bound-denom-o-k} has essentially automated the first part of the argument of \Cref{prop:quad-o-k}; notably, $\disc(1,\sqrt d)=4d$ by \Cref{ex:quad-disc}, so \Cref{lem:bound-denom-o-k} tells us immediately that any element of $\OO_{\QQ(\sqrt d)}$ takes the form $\frac{a+b\sqrt d}{4d}$. Of course, more precise arguments are able to improve this.
\end{remark}
We are now able to make the following definition.
\begin{definition}[integral basis]
	Fix a number field $K$ of degree $n$. By \Cref{thm:o-k-is-free}, $\OO_K$ is freely generated by $n$ elements $\alpha_1,\ldots,\alpha_n\in\OO_K$; any such list of generators is called an \textit{integral basis}. By abuse of notation, we will write
	\[\disc\OO_K\coloneqq\disc(\alpha_1,\ldots,\alpha_n).\]
\end{definition}
Perhaps we should check that $\disc\OO_K$ does not depend on the choice of integral basis.
\begin{lemma} \label{lem:disc-int-change-of-basis}
	Fix a number field $K$ of degree $n$. Choose $\alpha_1,\ldots,\alpha_n\in\OO_K$ and $\beta_1,\ldots,\beta_n\in\OO_K$ which are $\QQ$-linearly independent and satisfy
	\[\underbrace{\ZZ\alpha_1\oplus\cdots\oplus\ZZ\alpha_n}_{\mathcal A\coloneqq}\subseteq\underbrace{\ZZ\beta_1\oplus\cdots\oplus\ZZ\beta_n}_{\mathcal B\coloneqq}.\]
	Then there exists an integer $d=[\mathcal B:\mathcal A]$ such that $\disc(\alpha_1,\ldots,\alpha_n)=d^2\disc(\beta_1,\ldots,\beta_n)$.
\end{lemma}
\begin{proof}
	This follows from \Cref{lem:disc-change-of-basis}. Let $A\in\QQ^{n\times n}$ be the matrix with $(\alpha_1,\ldots,\alpha_n)=A(\beta_1,\ldots,\beta_n)$; note this gives $\left|\det A\right|=[\mathcal B:\mathcal A]$. Because $\alpha_1,\ldots,\alpha_n\in\ZZ\beta_1\oplus\cdots\oplus\ZZ\beta_n$, we see that $A\in\ZZ^{n\times n}$. Thus, $\det A\in\ZZ$, so \Cref{lem:disc-change-of-basis} finishes.
\end{proof}
In particular, if $\alpha_1,\ldots,\alpha_n\in\OO_K$ and $\beta_1,\ldots,\beta_n\in\OO_K$ are both integral bases, then their discriminants must be equal.

After all this theory, we should do another example.
\begin{example}
	Fix the field $K\coloneqq\QQ(\sqrt[3]2)$. Then $\OO_K=\ZZ[\sqrt[3]2]$, and $\disc\OO_K=-108$.
\end{example}
\begin{proof}
	This is fairly involved, so we take a deep breath. The elements $1,\sqrt[3]2,\sqrt[3]4$ are surely algebraic integers, so $\ZZ[\sqrt[3]2]\subseteq\OO_K$. It remains to show the reverse inclusion. We use \Cref{lem:bound-denom-o-k}: to compute the discriminant, let $\zeta_3$ denote the third root of unity, and we see
	\begin{align*}
		\disc(1,\sqrt[3]2,\sqrt[3]4) &= \det\begin{bmatrix}
			1 & \sqrt[3]2 & \sqrt[3]4 \\
			1 & \zeta_3\sqrt[3]2 & \zeta_3^2\sqrt[3]4 \\
			1 & \zeta_3^2\sqrt[3]2 & \zeta_3\sqrt[3]4
		\end{bmatrix}^2 \\
		&= \left(1\left(2\zeta_3^2-2\zeta_3^4\right)-1\left(2\zeta_3-2\zeta_3^2\right)+1\left(2\zeta_3^2-2\zeta_3\right)\right)^2 \\
		&= 4\left(\zeta_3^2-\zeta_3-\zeta_3+\zeta_3^2+\zeta_3^2-\zeta_3\right)^2 \\
		&= 36\left(\zeta_3^2-\zeta_3\right)^2 \\
		&= -108.
	\end{align*}
	Thus, we can write any $\alpha\in\OO_K$ as $\frac1{108}(a+b\sqrt[3]2+\sqrt[3]4)$ for some $a,b,c\in\ZZ$. We want to show that each of $a,b,c\in\ZZ$ is divisible by $108$. There are two computations.
	\begin{itemize}
		\item If $\alpha=\frac13(a+b\sqrt[3]2+\sqrt[3]4)\in\OO_K$ for integers $a,b,c\in\ZZ$, we show that $3\mid a,b,c$. By subtracting out from $1,\sqrt[3]2,\sqrt[3]4$, we may assume that $a,b,c\in\{-1,0,1\}$. However, by \Cref{ex:norm-k-cbrt-2} tells us that the norm of $\alpha$ is
		\[-1<-\frac{1+2+4+6}{27}\le\frac{a^3+2b^3+4c^3-6abc}{27}\le\frac{1+2+4+6}{27}<1,\]
		so we must have $\op N_{K/\QQ}(\alpha)=0$, which forces $\alpha=0$ by \Cref{cor:norm-tr-by-embeds}.
		\item If $\alpha=\frac13(a+b\sqrt[3]2+\sqrt[3]4)\in\OO_K$ for integers $a,b,c\in\ZZ$, we show that $3\mid a,b,c$. This is essentially the same argument. By subtracting out from $1,\sqrt[3]2,\sqrt[3]4$, we may assume that $a,b,c\in\{0,1\}$. However, by \Cref{ex:norm-k-cbrt-2} tells us that the norm of $\alpha$ is
		\[0\le\frac{a^3+2b^3+4c^3-6abc}{8}\le\frac{1+2+4}{8}<1,\]
		so we must have $\op N_{K/\QQ}(\alpha)=0$, which forces $\alpha=0$ by \Cref{cor:norm-tr-by-embeds}.
	\end{itemize}
	We now complete the argument. Any $\alpha\in\OO_K$ can be written as $\alpha\coloneqq\frac1{108}(a+b\sqrt[3]2+\sqrt[3]4)$. Then $36\alpha=\frac13(a+b\sqrt[3]2+\sqrt[3]4)$, so $3\mid a,b,c$ by the above, so we can write $\alpha=\frac1{36}(a'+b'\sqrt[3]2+c'\sqrt[3]4)$. Considering $12\alpha$ shows that $3\mid a',b',c'$ still, and we can continue downwards until $\alpha=x+y\sqrt[3]2+z\sqrt[3]4$ for integers $x,y,z$. This completes the proof.
\end{proof}

\subsection{Dirichlet's Unit Theorem: Set Up}
We are almost at a point where we can state our main theorem. Approximately speaking, our goal is to generalize \Cref{prop:solve-pell-norm-one} to more general number fields. We now have enough machinery to explain where $x^2-dy^2=1$ is coming from: given a non-square positive integer $d$, in the field $\QQ(\sqrt d)$, by \Cref{cor:norm-tr-by-embeds}, the norm map $\op N_{\QQ(\sqrt d)/\QQ}$ is given by
\[\op N_{\QQ(\sqrt d)/\QQ}\left(x+y\sqrt d\right)=\left(x+y\sqrt d\right)\left(x-y\sqrt d\right)=x^2-dy^2.\]
This also explains why we kept factoring $x^2-dy^2$ into $\left(x-y\sqrt d\right)\left(x+y\sqrt d\right)$. It will shortly be helpful for us to have a more algebraic description of these elements.
\begin{lemma} \label{lem:unit-by-norm-one}
	Fix a number field $K$. Then an element $u\in\OO_K$ is a unit (i.e., has a multiplicative inverse in $\OO_K$) if and only if $\left|\op N_{K/\QQ}(u)\right|=1$.
\end{lemma}
\begin{proof}
	We have two implications to show.
	\begin{itemize}
		\item Suppose that $u\in\OO_K$ is a unit. Then we have some $v\in\OO_K$ such that $uv=1$. Taking norms of this equation, we see
		\[\op N_{K/\QQ}(u)\cdot\op N_{K/\QQ}(v)=1.\]
		However, $\op N_{K/\QQ}(u),\op N_{K/\QQ}(v)\in\ZZ$ by \Cref{rem:norm-and-tr-ints}, and the only way to have two integers multiply to $1$ is for them to be $\pm1$. Thus, $\op N_{K/\QQ}(u)=\pm1$, as desired.
		\item Suppose that $\op N_{K/\QQ}(u)=\pm1$ so that $\op N_{K/\QQ}(u)^2=1$. The point is to expand the norm using \Cref{cor:norm-tr-by-embeds} to get an equation of the form $uv=1$. Indeed, by \Cref{cor:norm-tr-by-embeds}, we see that
		\[\prod_{i=1}^n\sigma_i(u)^2=1\]
		where the $\sigma_1,\ldots,\sigma_n\colon K\into\CC$ are the embeddings of \Cref{prop:embeddings-to-c}. Identifying $K$ with its image under $\sigma_1\colon K\into\CC$ (for example), we see that
		\[\sigma_1(u)\cdot\underbrace{\sigma_1(u)\prod_{i=2}^n\sigma_i(u)^2}_{v\coloneqq}=1.\]
		Now, $uv=1$, so we will be done once we establish that $v\in\OO_K$. Well, $u\in\OO_K$, so letting $f(x)$ be a monic polynomial with integer coefficients such that $f(u)=0$, we see that $f(\sigma_i(u))=0$ for all $i$, so $\sigma_i(u)$ is an algebraic integer for each $\sigma_i$, so $v$ is also an algebraic integer. Further, $v=1/u\in K$, so it follows $v\in\OO_K$.
		\qedhere
	\end{itemize}
\end{proof}
So integer pairs $(x,y)$ satisfying $x^2-dy^2=1$ will be units in $\OO_{\QQ(\sqrt d)}$. Note that the $-1$ case is also explained \Cref{lem:unit-by-norm-one} because being a unit permits $\op N_{K/\QQ}(x+y\sqrt d)=-1$.

To continue, we observe that there is something a little off with \Cref{prop:solve-pell-norm-one}. Namely, the proposition is only solving units in $\OO_{\QQ(\sqrt d)}^\times$ of the form $x+y\sqrt d$ where $x,y\in\ZZ$, but we saw in \Cref{prop:quad-o-k} that sometimes we have a denominator of $2$ present. Explicitly, one could use \Cref{prop:solve-pell-norm-one} to look for units in $\ZZ[\sqrt 5]^\times$ even though $\ZZ[\sqrt 5]\ne\OO_{\QQ(\sqrt5)}$. As such, in the statement we prove, we will not want to only focus on the rings $\OO_K$ but generalizes of them.
\begin{definition}[order]
	Fix a number field $K$ of degree $n$ over $\QQ$. Then an \textit{order} $\OO$ is a subring of $\OO_K$ which is a free abelian group of rank $n$.
\end{definition}
\begin{example}
	Fix a number field $K$. Then $\OO_K$ is itself an order by \Cref{thm:o-k-is-free}.
\end{example}
\begin{example}
	Let $d$ be a non-square integer, and set $K\coloneqq\QQ(\sqrt d)$. Then
	\[\ZZ[\sqrt d]\coloneqq\{a+b\sqrt d:a,b\in\ZZ\}\]
	is a subring of algebraic integers which is a free abelian group of rank $2$ (with basis given by $1$ and $\sqrt d$).
\end{example}
We are now almost ready to state our result. For technical reasons, we will want the notion of a signature. 
\begin{definition}[signature]
	Fix a number field $K$ of degree $n$ over $\QQ$, and let $\sigma_1,\ldots,\sigma_n\colon K\into\CC$ be the $n$ embeddings of \Cref{prop:embeddings-to-c}.
	\begin{itemize}
		\item If an embedding $\sigma\colon K\into\CC$ outputs to $\RR$, we call $\sigma$ a \textit{real embedding}.
		\item Otherwise, $\sigma\colon K\into\CC$ has output to $\CC\setminus\RR$ and is called a \textit{complex embedding}.
	\end{itemize}
	Among the $n$ embeddings $\sigma\colon K\into\CC$, we let $r_1$ denote the number of real embeddings and $2r_2$ denote the number of complex embeddings, and we let $(r_1,r_2)$ be the \textit{signature} of $K$.
\end{definition}
\begin{remark} \label{rem:complex-conjugate-complex-embed}
	It is worth explaining why the number of complex embeddings $\sigma\colon K\into\CC$ is even. Well, for any complex embedding $\sigma\colon K\into\CC$, there is a complex conjugate $\overline\sigma(\alpha)\coloneqq\overline{\sigma(\alpha)}$ embedding. Because there is $\alpha\in K$ with $\sigma(\alpha)\in\CC\setminus\RR$, we see that $\overline\sigma(\alpha)\ne\sigma(\alpha)$, so $\overline\sigma\ne\sigma$, so these are in fact distinct embeddings. Thus, $\sigma\mapsto\overline\sigma$ defines a map from complex embeddings to complex embeddings, and $\overline{\overline\sigma}=\sigma$ implies that this is an involution, so it follows that the number of complex embeddings is even: we may pair a complex embedding $\sigma$ off with its complex conjugate embedding!
\end{remark}
At long last, here is our result.
\begin{theorem}[Dirichlet unit] \label{thm:dirichlet-unit}
	Fix a number field $K$ of signature $(r_1,r_2)$. Let $\mu(K)$ denote the group of roots of unity in $K$. Let $\OO\subseteq\OO_K$ be an order, and let $\mu(\OO)$ be the roots of unity in $\OO$. Then
	\[\OO^\times\cong\mu(\OO)\times\ZZ^{r_1+r_2-1}.\]
	In other words, there is a set of units $u_1,\ldots,u_{r_1+r_2-1}$ such that, for any unit $u\in\OO_K^\times$, there is a unique root of unity $\zeta$ and integers $n_1,\ldots,n_{r_1+r_2-1}$ such that $u=\zeta u_1^{n_1}\cdots u_{r_1+r_2-1}^{n_{r_1+r_2-1}}$.
\end{theorem}
We are not going to prove \Cref{thm:dirichlet-unit} at all in this section; we will postpone until we have discussed a little Minkowski theory. For now, we satisfy ourselves with an example.
\begin{example}
	Let's show that \Cref{thm:dirichlet-unit} appropriately generalizes \Cref{prop:solve-pell-norm-one}. Fix a non-square positive integer $d$. Then $K\coloneqq\QQ(\sqrt d)$ has signature $(2,0)$, and $\mu(K)=\{\pm1\}$ because $\{\pm1\}$ are the only roots of unity in $\RR$. Thus, \Cref{thm:dirichlet-unit} implies that the order $\ZZ[\sqrt d]$ has
	\[\ZZ[\sqrt d]^\times\cong\{\pm1\}\times\ZZ.\]
	Tracking \Cref{lem:unit-by-norm-one} backward tells us that any solution $x^2-dy^2=\pm1$ has $x+y\sqrt d=\pm\left(x_0+y_0\sqrt d\right)^n$ for some unique sign $\pm$ and integer $n$. One can then reduce to $x^2-dy^2=1$ as a subgroup of $\ZZ[\sqrt d]^\times$.
\end{example}

\subsection{Problems}
Do ten points worth of the following exercises.
\begin{prob}[2 points]
	Show that $\sqrt[3]{5}/2$ is not an algebraic integer.
\end{prob}
\begin{prob}[3 points]
	We show that $\ZZ[\sqrt{-5}]$ is not a unique factorization domain.
	\begin{listalph}
		\item Show that the elements $2$, $3$, $1+\sqrt{-5}$, and $1-\sqrt{-5}$ are irreducible. You may find it helpful to use norms.
		\item Show that there is no unit $u\in\ZZ[\sqrt{-5}]$ such that $2u=1+\sqrt{-5}$ or $2u=1-\sqrt{-5}$.
		\item Finish the proof by noting $2\cdot3=\left(1+\sqrt{-5}\right)\left(1-\sqrt{-5}\right)$.
	\end{listalph}
\end{prob}
\begin{prob}[4 points]
	Show that the following complex numbers $\alpha$ are algebraic integers by finding a monic polynomial $f(x)\in\ZZ[x]$ such that $f(\alpha)=0$.
	\begin{listalph}
		\item $\alpha=\sqrt2$.
		\item $\alpha=\sqrt2+\sqrt3$.
		\item $\alpha=\frac13\left(1+\sqrt[3]{10}+\sqrt[3]{100}\right)$.
	\end{listalph}
\end{prob}
\begin{prob}[4 points]
	Let $m$ and $n$ be squarefree coprime integers such that $m\equiv n\equiv1\pmod4$, and set $K\coloneqq\QQ\left(\sqrt m,\sqrt n\right)$. Show that
	\[\OO_K=\ZZ\left[\frac{1+\sqrt m}2,\frac{1+\sqrt n}2\right].\]
	Notably, $\left(\frac{1+\sqrt m}2\right)\left(\frac{1+\sqrt n}2\right)\in\OO_K$.
\end{prob}
\begin{prob}[5 points]
	Let $K\coloneqq\QQ(\sqrt[3]3)$. Show that $\OO_K=\ZZ[\sqrt[3]3]$, and compute $\disc\OO_K$.
\end{prob}

\end{document}