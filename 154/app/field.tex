% !TEX root = ../notes.tex

\documentclass[../notes.tex]{subfiles}

\begin{document}

\section{A Little Field Theory} \label{app:fields}
The notes assume ring and group theory, but we will spend this appendix establishing the field theory that we will need.

\subsection{Basic Notions}
Here is our following definition.
\begin{definition}[field]
	A \textit{field} $K$ is a ring where each nonzero element has a multiplicative inverse.
\end{definition}
We will be interested in how fields relate to each other.
\begin{definition}[field extension]
	A \textit{field extension} $L/K$ is when one field $K$ is contained in another $L$. The \textit{degree} $[L:K]$ of the extension is the dimension of $L$ as a $K$-vector space. The field extension is said to be \textit{finite} if and only if $[L:K]<\infty$.
\end{definition}
\begin{example}
	Fix a field extension $L/K$. Given $\alpha\in L$, we can construct the field $K(\alpha)$ which is the quotient field of the integral domain generated by $K$ and $\alpha\in L$. Formally, $K(\alpha)$ is the quotient field of the ring $K[\alpha]$ which is defined as the image of the ring homomorphism
	\[\operatorname{ev}_\alpha\colon K[x]\to L\]
	defined by sending the polynomial $f(x)\in K[x]$ to $f(\alpha)$.
\end{example}
It might feel a little weird that we have jumped directly to one field being contained in another instead of a tamer notion of homomorphism. The following result explains why.
\begin{lemma}
	Let $\varphi\colon K\to L$ be a ring homomorphism of fields. Then $\varphi$ is injective.
\end{lemma}
\begin{proof}
	It suffices to check that $\ker\varphi$ is trivial. Well, $\ker\varphi$ is an ideal of $K$, but if nontrivial, then $\ker\varphi$ contains a unit and therefore must be all of $K$. However, $\varphi(1)=1$, so $\ker\varphi\ne K$, so we see that we must have $\ker\varphi=0$.
\end{proof}
A basic fact about our extensions is how degrees behave in extensions.
\begin{lemma} \label{lem:deg-in-towers}
	Let $M/L$ and $L/K$ be extensions of fields. Then
	\[[M:L][L:K]=[M:K].\]
\end{lemma}
\begin{proof}
	If $M/L$ or $L/K$ are infinite, then the $K$-vector space $M$ will have an infinitely linearly independent set, meaning $[M:K]$ is also infinite. Otherwise, we take $[M:L]=[L:K]$ to be finite. Let $\{m_1,\ldots,m_r\}$ and $\{\ell_1,\ldots,\ell_s\}$ be bases for $M/L$ and $L/K$, respectively. We claim that
	\[\{m_i\ell_j\}_{1\le i\le r,1\le j\le s}\]
	is a basis for $M/K$.
	\begin{itemize}
		\item We show that the $m_i\ell_j$ span: any $m\in M$ can be expressed as
		\[m=\sum_{i=1}^ra_im_i\]
		where $a_k\in L$, but then each $a_k\in L$ can be expressed as
		\[m=\sum_{i=1}^r\sum_{j=1}^sb_{ij}m_i\ell_j,\]
		where $b_{ij}\in K$. This is what we wanted.
		\item We show that the $m_i\ell_j$ are linearly independent: suppose
		\[\sum_{i=1}^r\sum_{j=1}^sb_{ij}\ell_jm_i=\sum_{i=1}^r\sum_{j=1}^sb_{ij}'\ell_jm_i.\]
		Then
		\[\sum_{i=1}^r\sum_{j=1}^s(b_{ij}-b_{ij}')\ell_jm_i=0,\]
		so because $\{m_1,\ldots,m_r\}$ is a basis of $M/L$, we see
		\[\sum_{j=1}^s(b_{ij}-b_{ij}')\ell_j=0\]
		for each $i$, but because $\{\ell_1,\ldots,\ell_s\}$ is a basis of $L/K$, we see $b_{ij}=b'_{ij}$ for each $i$ and $j$.
		\qedhere
	\end{itemize}
\end{proof}

\subsection{Polynomial Rings}
In this subsection, we show that $K[x]$ is a unique factorization domain for any field $K$. We will not bother to show the usual facts about degree over integral domains, such as $\deg fg=\deg f+\deg g$ for nonzero $f,g\in K[x]$. However, we will show the following result, whose proof is more technical than one would like. Thankfully, we will not have to use this result for a while.
\begin{lemma} \label{lem:c-is-sep}
	Fix an irreducible polynomial $f\in\CC[x]$. Then $f$ has no repeated roots.
\end{lemma}
\begin{proof}
	Note $\deg f\ge1$ because $\deg f=1$ implies that $f$ is a unit and thus not irreducible. Because $\CC$ is algebraically closed, we note that $f(x)$ factors as
	\[f(x)=c\prod_{i=1}^n(x-\alpha_i)\]
	for some complex numbers $c,\alpha_1,\ldots,\alpha_n\in\CC$. Now, if $\alpha_i=\alpha_j$ for $i\ne j$, then $f(x)$ has a double root at $\alpha_i$, so a direct computation shows that $f'(\alpha_i)=0$, so $f(x)$ and $f'(x)$ have a root in common, so $\gcd(f(x),f'(x))$ is a non-constant polynomial with degree strictly less than $f(x)$ but dividing $f(x)$, which contradicts $f(x)$ being irreducible.
\end{proof}
Anyway, the main point to showing that $K[x]$ is a unique factorization domain is the following result.
\begin{proposition}[division] \label{prop:division-for-poly}
	Fix a field $K$, and let $a,b\in K[x]$ be polynomials with $b\ne0$. Then there exist polynomials $q,r\in K[x]$ such that
	\[a=bq+r\]
	where $r=0$ or $0\le\deg r<\deg b$.
\end{proposition}
\begin{proof}
	We induct on $\deg a$. If $a=0$ or $\deg a<\deg b$, we set $q=0$ and $r=a$. Otherwise, $\deg a\ge\deg b$. Let the leading coefficient of $b$ be $b_nx^n$, and let the leading coefficient of $a$ be $a_mx^m$. Then
	\[c(x)\coloneqq a(x)-\frac{a_m}{b_n}x^{m-n}\cdot b\]
	cancels out the leading coefficient of $a$, so $c=0$ or $\deg c<\deg a$. So we can apply the result to $c$ by the induction, writing
	\[c=bq_c+r_c,\]
	so
	\[a(x)=b(x)\left(q_c(x)+\frac{a_m}{b_n}x^{m-n}\right)+r(x),\]
	finishing.
\end{proof}
\begin{theorem}
	Fix a field $K$. Then the field $K[x]$ is a principal ideal domain and hence a unique factorization domain.
\end{theorem}
\begin{proof}
	If we show that $K[x]$ is a principal ideal domain, we finish immediately by \Cref{thm:pid-is-ufd}. So we want to show that $K[x]$ is a principal ideal domain.

	Well, let $I\subseteq K[x]$ be a principal ideal domain. If $I=\{0\}$, then $I=(0)$, so there is nothing to say. Otherwise, $I$ has a nonzero element, so we let $f\in I$ denote any element of least degree. We claim $I=(f)$. Certainly $(f)\subseteq I$, so we want to show that $a\in I$ lives in $(f)$. Well, by \Cref{prop:division-for-poly}, we may write
	\[a=fq+r\]
	where $r=0$ or $0\le\deg r<\deg f$. However, $r=a-fq\in I$ has $\deg r<\deg f$, so minimality of $\deg f$ requires $r=0$, meaning $a=fq$, so $a\in(f)$.
\end{proof}

\subsection{Algebraic Elements} \label{subsec:alg-elements}
In this subsection, we show some basic properties of algebraic elements. Here is our definition.
\begin{definition}[algebraic]
	Fix a field extension $L/K$. An element $\alpha\in L$ is \textit{algebraic over $K$} if and only if $\alpha$ is the root of some nonzero polynomial in $K[x]$.
\end{definition}
\begin{example}
	Any element of $K$ is algebraic over $K$ because $\alpha\in K$ is the root of the polynomial $x-\alpha\in K[x]$.
\end{example}
Finite extensions provide a wealth of algebraic elements.
\begin{lemma} \label{lem:finite-is-alg}
	Let $L/K$ be a finite extension of fields. Then each $\alpha\in L$ is algebraic over $K$.
\end{lemma}
\begin{proof}
	The elements $1,\alpha,\alpha^2,\ldots$ form an infinite set in $L$, so they cannot be $K$-linearly independent because $\dim_KL<\infty$. Thus, there is a relation of the form
	\[\sum_{k=0}^na_k\alpha^k=0\]
	where $a_k\in K$ are not all zero. As such, the polynomial $f(x)\coloneqq\sum_{k=0}^na_kx^k$ will do.
\end{proof}
It will shortly be helpful to limit the polynomial attached to $\alpha$ somewhat.
\begin{lemma} \label{lem:minimal-poly}
	Fix a field extension $L/K$, and let $\alpha\in L$ be algebraic over $K$. Then $\alpha$ is the root of a unique monic irreducible polynomial $f(x)\in K[x]$. In fact, for any polynomial $g\in K[x]$ with $g(\alpha)=0$, we have $f\mid g$.
\end{lemma}
\begin{proof}
	We begin by showing existence. We know that $\alpha$ is the root of some nonzero polynomial $f(x)\in K[x]$, so we choose $f(x)$ to have the smallest degree possible. By dividing out the leading coefficient (which is nonzero because $f$ is nonzero), we may assume that $f$ is monic. It remains to show that $f$ is irreducible. Well, suppose that
	\[f=ab\]
	for $a,b\in K[x]$. Note neither $a$ nor $b$ is zero because this would imply $f=0$; additionally, if both are units, then $f$ is a unit and hence a constant polynomial, which also makes no sense. Now, evaluating at $\alpha$, we see that $f(\alpha)=0$ requires $a(\alpha)=0$ or $b(\alpha)=0$, so by minimality of $f$, we must have $\deg a\ge\deg f$ or $\deg b\ge\deg f$. Without loss of generality take $\deg a\ge\deg f$, but $f=ab$ then forces $\deg a=\deg f$, and $b$ is a constant polynomial.

	We now show that $g(\alpha)=0$ implies $f\mid g$ for any $g\in K[x]$. By \Cref{prop:division-for-poly}, we may write
	\[g=fq+r\]
	where $r=0$ or $0\le\deg r<\deg f$. By plugging in $\alpha$, we see that $f(\alpha)=g(\alpha)=0$ implies $r(\alpha)=0$. But if nonzero $\deg r<\deg f$, violating minimality of $f$, so we instead have $r=0$, implying $g=fq$ and so $f\mid g$.

	To finish up, we show that $f$ is unique. Well, if $g$ is another monic irreducible polynomial with $g(\alpha)=0$, then $f\mid g$ by the above argument. But $f$ is nonzero, so $f\mid g$ requires $g=fu$ for a unit $u$. Being a unit means that $u$ is a constant polynomial in $K$, so for example $\deg f=\deg g$, and because $f$ and $g$ have the same leading coefficient, we must have $u=1$. Thus, $f=g$, as needed.
\end{proof}
\begin{lemma} \label{lem:quotient-of-poly-ring}
	Fix a field extension $L/K$, and let $\alpha\in L$ be algebraic over $K$ and in particular the root of a monic irreducible polynomial $f(x)\in K[x]$. Then
	\[\frac{K[x]}{(f(x))}\cong K[\alpha].\]
	In particular, $K[\alpha]$ is a field of degree $\deg f$ over $K$.
\end{lemma}
\begin{proof}
	Quickly, note that the last sentence follows from the isomorphism because 
	
	For the first claim, note that there is a surjective ring homomorphism $\operatorname{ev}_\alpha\colon K[x]\to K[\alpha]$ by sending $g(x)\mapsto g(\alpha)$ for any $g(x)\in K[x]$. We want to show that $\ker{\operatorname{ev}_\alpha}=(f)$. Certainly $f\in\ker\op{ev}_\alpha$. For the other inclusion, we note that any $g\in\ker\op{ev}_\alpha$ has $g(\alpha)=0$ and hence $f\mid g$ by \Cref{lem:minimal-poly}.

	For the second claim, note that $K[\alpha]$ is not a field because $K[x]/(f(x))$ is a field by \Cref{rem:pid-is-dim-1}. As for the degree computation, write
	\[f(x)=x^d+\sum_{k=0}^{d-1}a_kx^k.\]
	Then for each $n\ge d$, we can express $\alpha^n$ in terms of $\alpha^k$ with $k<n$: indeed, $f(\alpha)=0$ implies
	\[\alpha^n=-\sum_{k=0}^{d-1}a_k\alpha^{k+n-d}.\]
	Thus, $1,\alpha,\alpha^2,\ldots,\alpha^{d-1}$ spans $K[\alpha]$, so $\dim_KK[\alpha]\le d$. In fact, these $\alpha^k$ are linearly independent because any nontrivial relation involving them becomes a polynomial $g(x)$ with $\alpha$ is a root which is either zero or has degree less than $\deg f$, but \Cref{lem:minimal-poly} enforces $g=0$.
\end{proof}
As a last aside, we note that the sum and product of algebraic elements remains algebraic. This requires a trick known as the ``determinant trick.''
\begin{proposition} \label{prop:how-to-algebraic}
	Fix a field extension $L/K$ and some $\alpha\in L$. Then the following are equivalent.
	\begin{listalph}
		\item $\alpha$ is algebraic over $K$.
		\item The field $K[\alpha]$ is a finite extension of $K$.
		\item There is a subfield $K'\subseteq L$ finite over $K$ which contains $\alpha$.
	\end{listalph}
\end{proposition}
\begin{proof}
	We show the implications separately.
	\begin{itemize}
		\item Here, (a) implies (b) is proven in \Cref{lem:quotient-of-poly-ring}.
		\item Note (b) implies (c) by setting $K'\coloneqq K[\alpha]$.
		\item Checking that (c) implies (a) is harder. Suppose $K'$ is generated by the elements $\alpha_1',\alpha_2',\ldots,\alpha_n'$. Note that $\alpha\alpha_i\in A'$ for each $\alpha_i'$, so we may write
		\[\alpha\alpha_i'=\sum_{j=1}^na_{ij}\alpha_j'\]
		for some elements $a_{ij}\in K'$. In other words, the matrix $T\coloneqq(a_{ij})_{i,j=1}^n$ has
		\[\alpha\begin{bmatrix}
			\alpha_1' \\
			\vdots \\
			\alpha_n'
		\end{bmatrix}=T\begin{bmatrix}
			\alpha_1' \\
			\vdots \\
			\alpha_n'
		\end{bmatrix}.\]
		Thus, $T-\alpha I_n$ is an $n\times n$ matrix with entries in $K'$, and it has the nonzero vector $(\alpha_1',\ldots,\alpha_n')$ in its kernel, so $\det(T-\alpha I_n)=0$. Expanding out the polynomial $\det(\alpha I_n-T)=0$ makes $\alpha$ the root of a monic polynomial (of degree $n$) with coefficients in $A$, so $\alpha$ is indeed integral over $A$.
		\qedhere
	\end{itemize}
\end{proof}
\begin{corollary}
	Fix a field extension $L/K$, and let $K'$ denote the set elements of $L$ algebraic over $K$. Then $K'$ is a subfield of $L$. In fact, for any $\alpha\in L$ algebraic over $K'$, we have $\alpha\in K'$.
\end{corollary}
\begin{proof}
	We run our checks separately.
	\begin{itemize}
		\item We check that $K'$ is a field. Note $0,1\in K'$ because these elements are the roots of the polynomials $x$ and $x-1$, respectively. It remains to show that, for any $\alpha,\beta\in K'$, we have $\alpha+\beta,\alpha\beta\in K'$ and $\alpha/\beta\in K'$ if $\beta\ne0$. The main point is to show that $K[\alpha,\beta]$ is a finite extension of $K$, which will complete the proof by \Cref{prop:how-to-algebraic}.
	
		Well, let $\alpha$ and $\beta$ be the roots of the monic polynomials $f(x),g(x)\in K[x]$ respectively. Then by \Cref{prop:how-to-algebraic} shows that $K[\beta]$ is a finite extension of $K$, and $f(\alpha)=0$ shows that $\alpha$ is integral over $K[\beta]$, so $K[\alpha,\beta]$ is finite field extension of $K[\beta]$. We conclude $K[\alpha,\beta]$ is a finite field extension of $K$ by \Cref{lem:deg-in-towers}.

		\item Suppose that $\alpha\in L$ is the root of the monic polynomial $f(x)\in K'[x]$ (monic by \Cref{lem:minimal-poly}); we show that $\alpha\in K'$. Well, expand $f(x)$ as
		\[f(x)=x^d+\sum_{k=0}^{d-1}a_kx^k\]
		for some $a_0,\ldots,a_{d-1}\in K'$. Each $a_\bullet$ is algebraic over $K$, so \Cref{prop:how-to-algebraic} tells us that $K[a_\bullet]$ for each $a_\bullet$. As such, as in the previous check, we may build the tower
		\[K\subseteq K[a_0]\subseteq K[a_0,a_1]\subseteq\cdots\subseteq K[a_0,\ldots,a_{d-1}],\]
		where each field is finite over the previous one by \Cref{prop:how-to-algebraic}. Then \Cref{lem:deg-in-towers} tells us that $K[a_0,\ldots,a_{d-1}]$ is finite over $K$. Lastly, $f(\alpha)=0$ tells us that $\alpha$ is algebraic over $K[a_0,\ldots,a_{d-1}]$, so $\ZZ[a_0,\ldots,a_{d-1},\alpha]$ is finite over $K[a_0,\ldots,a_{d-1}]$- and hence finite over $K$ by \Cref{lem:deg-in-towers}, meaning that $\alpha$ is algebraic over $K$ by \Cref{prop:how-to-algebraic}.
		\qedhere
	\end{itemize}
\end{proof}

\subsection{Enough Galois Theory to be Dangerous}
We are going to derive a lot of mileage from the following result in field theory. It leads towards Galois theory; even though Galois theory is a beautiful subject, it is one that we can avoid somewhat.
\begin{proposition} \label{prop:embeddings-to-c}
	Let $L/K$ be a finite field extension, where $L$ is a subfield of $\CC$. Then each embedding $\sigma\colon K\to\CC$ extends to exactly $[L:K]$ embeddings $\widetilde\sigma\colon L\into\CC$.
\end{proposition}
Here, we are using the term ``embedding'' to refer to an (injective) ring homomorphism.
\begin{proof}
	We induct on $[L:K]$, which is legal because $[L:K]<\infty$. If $[L:K]=1$, then $L=K$, and there is nothing to say because we must have $\widetilde\sigma=\sigma$.

	Otherwise, suppose $[L:K]>1$. Then fix $\alpha\in L\setminus K$. By \Cref{lem:finite-is-alg}, $\alpha$ is algebraic over $K$, so by \Cref{lem:minimal-poly}, $\alpha$ is the root of some monic irreducible polynomial $f(x)$. Now, $\CC$ is algebraically closed, so we note that $f(x)$ factors as
	\[f(x)=\prod_{i=1}^n(x-\alpha_i)\]
	for some complex numbers $\alpha_1,\ldots,\alpha_n\in\CC$; note that the $\alpha_\bullet$ are distinct by \Cref{lem:c-is-sep}. Thus, given an embedding $\sigma\colon K\into\CC$, there are $n$ exactly extensions to $\sigma_i\colon K[\alpha]\into\CC$ by sending $\sigma_i(\alpha)\coloneqq\alpha_i$. We have a number of checks to make this sentence make sense.
	\begin{itemize}
		\item Setting $\sigma_i(\alpha)=\alpha_i$ defines a unique embedding $K[\alpha]\into\CC$. The embedding here is uniquely defined because we need to have $\sigma_i|_K=\sigma$, and then any polynomial in $K[\alpha]$ will have its output determined by where $\alpha$ goes. To show that $\sigma_i$ is well-defined, we note that it is simply the composite
		\[K[\alpha]\cong\frac{K[x]}{(f(x))}\cong K[\alpha_i]\subseteq\CC,\]
		where the left isomorphism is by \Cref{lem:quotient-of-poly-ring}.
		\item We have in fact defined $n$ embeddings because the roots $\alpha_i$ are distinct.
		\item Each extension $\widetilde\sigma\colon K[\alpha]\to\CC$ of $\sigma$ must take this form. It suffices by our first point to check that $\widetilde\sigma(\alpha)=\alpha_i$ for some $\alpha_i$. Well, note that
		\[f(\widetilde\sigma(\alpha))=\widetilde\sigma(f(\alpha))=0\]
		because $\widetilde\sigma$ is a ring homomorphism. The result follows.
	\end{itemize}
	Now, by induction each of the $\sigma_i$ extend to exactly $[L:K[\alpha]]<[L:K]$ distinct embeddings $L\into K$, totaling to
	\[[L:K[\alpha]]\cdot[K[\alpha]:K]=[L:K]\]
	embeddings $L\into\CC$, where we have used \Cref{lem:deg-in-towers}. Let $\sigma_i$ extend to $\sigma_{i1},\ldots,\sigma_{im}$ where $m=[L:K[\alpha]]$. We have the following checks on the $\sigma_{ij}$.
	\begin{itemize}
		\item Note that $\sigma_{ij}$ must be distinct: if $\sigma_{ij}=\sigma_{i'j'}$, then restricting to $K[\alpha]$ reveals that $\sigma_{ij}|_{K[\alpha]}=\sigma_i$, so $\sigma_i=\sigma_{i'}$, so $i=i'$. But then the uniqueness of extending from $K[\alpha]$ to $L$ means that $\sigma_{ij}=\sigma_{ij'}$ implies $j=j'$.
		\item We show that all extensions $\widetilde\sigma\colon L\into\CC$ of $\sigma$ take the form $\sigma_{ij}$. Well, restricting $\widetilde\sigma$ to $K[\alpha]$ shows that $\widetilde\sigma|_{K[\alpha]}=\sigma_i$ for some $i$. Then $\widetilde\sigma=\sigma_{ij}$ for some $j$ by construction of the $\sigma_{ij}$.
	\end{itemize}
	The above checks show that the $\sigma_{ij}$ provide all extensions of $\sigma\colon K\into\CC$, counted uniquely, finishing.
\end{proof}

\subsection{Norm and Trace}
\Cref{prop:embeddings-to-c} allows us to make sense of the norm and trace of an algebraic element, which we now define.
\begin{definition}
	Let $L/K$ be a finite extension of fields. Then for $\alpha\in L$, let $\mu_\alpha\colon L\to L$ denote the multiplication-by-$\alpha$ map, which is $K$-linear by the distributive law. Then we define the \textit{trace} of $\alpha$ as $\op T_{L/K}(\alpha)\coloneqq\tr\mu_\alpha$ and the \textit{norm} of $\alpha$ as $\op N_{L/K}(\alpha)\coloneqq\det\mu_\alpha$.
\end{definition}
\begin{example}
	Let $L/K$ be a finite extension of fields. Then any $\alpha\in K$ has $\mu_\alpha$ given by the matrix $\alpha I_{[L:K]}$, so $\op T_{L/K}(\alpha)=[L:K]\alpha$ and $\op N_{L/K}(\alpha)=\alpha^{[L:K]}$.
\end{example}
% The norm and trace satisfy the obvious tower laws.
% \begin{lemma} \label{lem:tr-norm-in-towers}
% 	Let $M/L$ and $L/K$ be finite extensions of fields. Then
% 	\[{\op T_{L/K}}\circ{\op T_{M/L}}={\op T_{M/K}}\qquad\text{and}\qquad{\op N_{L/K}}\circ{\op N_{M/L}}={\op N_{M/K}}.\]
% \end{lemma}
% \begin{proof}
% 	Let $\{m_1,\ldots,m_r\}$ and $\{\ell_1,\ldots,\ell_s\}$ be bases for $M/L$ and $L/K$, respectively, so that $m_i\ell_j$ provides a basis for $M/K$ by the proof of \Cref{lem:deg-in-towers}.
% 	Now, fix some $\alpha\in M$. 
% \end{proof}
Here is our key example of the norm and trace.
\begin{proposition} \label{prop:norm-and-tr-by-min-poly}
	Let $L/K$ be an extension of fields. Then let $\alpha\in K$ be algebraic over $K$ which is the root of the monic irreducible polynomial $f(x)\in K[x]$. Writing $f(x)=x^d+\sum_{k=0}^{d-1}a_kx^k$, we have
	\[\op T_{K[\alpha]/K}(\alpha)=-a_{d-1}\qquad\text{and}\qquad\op N_{K[\alpha]/K}(\alpha)=(-1)^da_0.\]
\end{proposition}
\begin{proof}
	Note $K[\alpha]$ is finite over $K$ by \Cref{lem:quotient-of-poly-ring}, where we actually showed that $1,\alpha,\alpha^2,\ldots,\alpha^{d-1}$ is a basis of $L$ as a $K$-vector space. Then $\mu_\alpha$ with respect to this (ordered) basis looks like the $d\times d$ matrix
	\begin{equation}
		\begin{bmatrix}
			0 & 0 & \cdots & 0 & -a_0 \\
			1 & 0 & \cdots & 0 & -a_1 \\
			0 & 1 & \cdots & 0 & -a_2 \\
			\vdots & \vdots & \ddots & \vdots & \vdots \\
			0 & 0 & \cdots & 1 & -a_{d-1}
		\end{bmatrix}. \label{eq:mu-alpha-in-k-alpha}
	\end{equation}
	The trace of this matrix is $-a_{d-1}$, and its determinant is $(-1)^da_0$ by expansion by minors.
\end{proof}
\begin{corollary}
	Let $K/\QQ$ be a finite extension of fields of degree $n$. Fix $\alpha\in K$, and let $\sigma_1,\ldots,\sigma_n$ denote the embeddings $K\into\CC$. Then
	\[\op T_{K/\QQ}(\alpha)=\sum_{i=1}^n\sigma_i(n)\qquad\text{and}\qquad\op N_{K/\QQ}(\alpha)=\prod_{i=1}^n\sigma_i(n).\]
\end{corollary}
\begin{proof}
	We use \Cref{prop:norm-and-tr-by-min-poly}. Note $\alpha\in K$ is algebraic over $\QQ$ by \Cref{lem:finite-is-alg}, so let $\tau_1,\ldots,\tau_d$ denote the embeddings $K[\alpha]\into\CC$. By the proof of \Cref{prop:embeddings-to-c}, we see that
	\[f(x)=\prod_{i=1}^d(x-\tau_i(\alpha)),\]
	where $f(x)\in\QQ[x]$ is the unique monic irreducible polynomial with $f(\alpha)=0$ provided by \Cref{lem:minimal-poly}. Thus, \Cref{prop:norm-and-tr-by-min-poly} tells us that
	\[\op T_{\QQ[\alpha]/\QQ}(\alpha)=\sum_{i=1}^d\tau_i(d)\qquad\text{and}\qquad\op N_{\QQ[\alpha]/\QQ}(\alpha)=\prod_{i=1}^d\tau_i(\alpha).\]
	To complete the proof, we must extend up from $K[\alpha]/K$ to $L/K$. Well, let $\ell_1,\ldots,\ell_{n/d}$ denote a basis for $L$ as a $K[\alpha]$-vector space, where we are implicitly using \Cref{lem:deg-in-towers}. Then the proof of \Cref{lem:deg-in-towers} shows us that $\ell_i\alpha^j$ provides a basis for $L/\QQ$, so writing out $\mu_\alpha$ according to this basis looks like $n/d$ blocks of \eqref{eq:mu-alpha-in-k-alpha}. Because each $\tau_i$ extends to exactly $n/d$ embeddings $K\into\CC$ by \Cref{prop:embeddings-to-c}, the result follows.
\end{proof}

\end{document}