% !TEX root = ../notes.tex

\documentclass[../notes.tex]{subfiles}

\begin{document}

\section{February 8}

Last class we showed that every curve can be embedded into $\PP^3_k$.
\begin{remark}
	We will not show this, but one can show with a little more work that any curve is birational to a curve in the plane with at worst nodes as singularities.
\end{remark}

\subsection{The Canonical Embedding}
Throughout, $X$ is a $k$-curve, where $k$ is algebraically closed. We will set $g\coloneqq g(X)$ and let $K$ denote the canonical divisor.
\begin{example}
	Notably, $g(X)=0$ forces $\dim|K|=-1$, so $|K|$ is empty. (After all, $X\cong\PP^1_k$.)
\end{example}
\begin{example}
	If $g(X)=1$, then we are looking at $\dim|K|=0$, and here $\deg K=0$, so we are just looking at the mapping from $X$ to a point.
\end{example}
In higher genus, things get more interesting.
\begin{lemma} \label{lem:k-is-base-point-free}
	Fix a $k$-curve $X$ with $g(X)\ge2$. Then the canonical divisor $K$ is base-point-free.
\end{lemma}
\begin{proof}
	By \Cref{prop:div-to-proj-embed}, it suffices to show that $\ell(K-P)=\ell(K)-1=g-1$ for any $P\in X$. Now, because $g(X)\ge2>0$, we know that $X\not\cong\PP^1_k$, so $\dim|P|=0$ is forced. Now, using \Cref{thm:rh}, we solve
	\[\ell(P)=\ell(K-P)+\deg(P)+(1-g),\]
	so $\ell(K-P)=g-1$, which is what we wanted.\todo{What?}
\end{proof}
In our study of curves, the following curves will make a somewhat large class.
\begin{definition}[hyperelliptic]
	A $k$-curve $X$ is \textit{hyperelliptic} if and only if it admits a degree-$2$ map to $\PP^1_k$.
\end{definition}
\begin{remark}
	Given any divisor $D$ on $X$ with $\ell(D)=2$ and $\deg D=2$, one gets a rational map $X\dashrightarrow\PP^1_k$ determined by $D$. (One can show that all such divisors are base-point-free using \Cref{prop:div-to-proj-embed}.) Notably, $D$ is linearly equivalent to an effective divisor.
\end{remark}
\begin{remark}
	If $g(X)=2$, then $X$ is hyperelliptic. This was shown on the homework. In brief, $K$ is base-point-free and has degree $2$, so it determines a degree-$2$ map $X\to\PP^1_k$.
\end{remark}
One can improve \Cref{lem:k-is-base-point-free} as follows.
\begin{proposition} \label{prop:k-is-very-ample}
	Fix a $k$-curve $X$ with $g(X)\ge2$. Then $K$ is very ample if and only if $X$ is not hyperelliptic.
\end{proposition}
\begin{proof}
	By \Cref{prop:div-to-proj-embed}, we are interested in the condition
	\[\ell(K-P-Q)=\ell(K)-2\stackrel?=g-2\]
	for any $P,Q\in X$. As such, by \Cref{thm:rh}, we compute
	\[\ell(P+Q)=\ell(K-P-Q)+\deg(P+Q)+(1-g)=\ell(K-P-Q)+3-g,\]
	so $\ell(K-P-Q)=g-2$ is equivalent to
	\[\ell(P+Q)=1.\]
	Now, if $X$ is hyperelliptic, then one can find a divisor $D$ with $\dim|D|=1$ and $\deg D=2$, so $D$ linearly equivalent to an effective divisor which looks like $P+Q$. But then $\ell(P+Q)=2>1$, which is a problem. In the other direction, if $X$ is not hyperelliptic, then each effective divisor $D$ with degree $2$ must have $D=P+Q$, which forces $\ell(P+Q)<2$ to retain being not hyperelliptic.
\end{proof}
Thus, we are interested in the embedding induced by this canonical divisor.
\begin{definition}[canonical morphism]
	Fix a $k$-curve $X$ of genus $g(X)\ge2$. Then the canonical divisor $K$ determines the \textit{canonical morphism} $X\to\PP^{g-1}_k$ by \Cref{lem:k-is-base-point-free}. If $X$ is not hyperelliptic curves, then we in fact get a \textit{canonical embedding} by \Cref{prop:k-is-very-ample}.
\end{definition}
\begin{remark}
	If $\deg\mc L=d$, and $\mc L$ is very ample, then $\mc L$ grants a closed embedding of $X$ to $\PP^{r-1}_k$ where $r\coloneqq\dim_k\Gamma(X,\mc L)$ (using the canonical embedding), and this embedding retains $X$ as a curve of degree $d$.
\end{remark}

\end{document}