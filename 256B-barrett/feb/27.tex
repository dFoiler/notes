% !TEX root = ../notes.tex

\documentclass[../notes.tex]{subfiles}

\begin{document}

\section{February 27}

We continue.

\subsection{Cohomology of Quasicoherent Sheaves}
We are going to show that quasicoherent sheaves over an affine scheme have vanishing cohomology. This is analogous to the result that Stein spaces have vanishing cohomology.
\begin{theorem} \label{thm:qcoh-cohom}
	Fix a Noetherian ring $A$ and affine scheme $X=\Spec A$. If $\mc F$ is a quasicoherent sheaf, then $H^i(X,\mc F)=0$ for $i>0$.
\end{theorem}
The main input to this result is the following result.
\begin{restatable}{proposition}{quasicohinjisflasque} \label{prop:inj-qcoh-sheaf-is-flasque}
	Fix a Noetherian ring $A$ and an injective $A$-module $I$. Then the quasicoherent sheaf $\widetilde I$ on $\Spec A$ is flasque.
\end{restatable}
\noindent Let's see how \Cref{prop:inj-qcoh-sheaf-is-flasque} implies \Cref{thm:qcoh-cohom}.
\begin{proof}[Proof of \Cref{thm:qcoh-cohom}]
	We may write $\mc F\coloneqq\widetilde M$ for some $A$-module $M$, and we give $M$ an injective resolution
	\[0\to M\to I^0\to I^1\to I^2\to\cdots.\]
	Now, the functor $\widetilde\cdot$ is exact (this can be checked on stalks because localization is flat), so we get the resolution
	\[0\to\widetilde M\to\widetilde{I^0}\to\widetilde{I^1}\to\widetilde{I^2}\to\cdots,\]
	which is a flasque resolution by \Cref{prop:inj-qcoh-sheaf-is-flasque}. Now, computing cohomology with this flasque resolution, we recover the original exact injective resolution for $A$. In particular, we see that we remain exact and therefore have vanishing higher cohomology.
\end{proof}
Thus, we turn to the proof of \Cref{prop:inj-qcoh-sheaf-is-flasque}. We require the following algebraic ingredient.
\begin{theorem}[Krull] \label{thm:krull-a-adic}
	Fix a Noetherian ring $A$. Given an ideal $\mf a\subseteq A$ and finite $A$-modules $M\subseteq N$, then the $\mf a$-adic topology on $M$ is induced by that on $N$. In other words, for any $n>0$, there exists $n'>0$ such that $\mf a^nM\supseteq(\mf a^{n'}N)\cap M$.
\end{theorem}
Note that $(\mf a^{n'}N)\cap M\supseteq\mf a^{n'}M$ automatically, so our system of fundamental neighborhoods yielding topology on $A$ to in fact produce the same topology on $M$.

To show \Cref{thm:krull-a-adic}, we will need a lemma.
\begin{definition}[sheaf with support]
	Given a topological space $X$ and a closed subset $Z\subseteq X$, we define for any sheaf $\mc F$ of abelian groups the set
	\[\Gamma_Z(X,\mc F)\coloneqq\{s\in\Gamma(X,\mc F):s|_p=0\text{ for }p\notin Z\}.\]
	This is the \textit{sheaf of sections with support contained in $Z$}.
\end{definition}
\begin{definition}
	Fix a ring $A$ and an ideal $\mf a$. Then we define
	\[\Gamma_\mf a(M)\coloneqq\{m\in M:a^nm=0\text{ for some }n>0\text{ for all }a\in\mf a\}.\]
\end{definition}
Notably, $U\mapsto\Gamma_{Z\cap U}(U,\mc F)$ defines a sheaf on $\mc F$, which is left exact and produces the cohomology groups $H_Z^i(X,\mc F)$.

In our affine case, we have the following.
\begin{lemma}
	Fix a Noetherian ring $A$. Then for $Z=V(\mf a)$,
	\[\Gamma_Z(X,\widetilde M)\coloneqq\left\{m\in M:a^nm=0\text{ for some }n>0\text{ for all }a\in\mf a\right\}.\]
\end{lemma}
\begin{proof}
	Quickly, let the right-hand set be $\Gamma_\mf a(M)$.

	In one direction, for $m\in\Gamma_\mf a(M)$, we see $\mf a^nm=0$ for some $n>0$ because $A$ is Noetherian (so that $\mf a$ is finitely generated, allowing us to find $n$ large enough so that all generators to some sufficiently higher power $n$ lie in $\op{Ann}m$). Thus, if $\mf p\notin Z$, then $\mf a\not\subseteq\mf p$, so we see there is $f\in\mf a\setminus\mf p$ which is invertible in $M_\mf p$. In particular, $f^nm=0$, so $m|_\mf p=0$, so we see that $\mf p$ is not in the support of $m$. Thus, the support of $m$ is contained in $Z$.

	In the other direction, suppose $m\in M$ has support in $Z$. Now, for $\mf p\notin Z$, we have $m|_\mf p=0$ by hypothesis, so there is $f\in A\setminus\mf p$ such that $f^{n_f}m=0$ for some $n_f>0$. In particular, it follows that $\op{Ann}m\not\subseteq\mf p$ as witnessed by $f$. In other words, we see that $\op{Ann}m\subseteq\mf p$ implies that $\mf p\in Z$ and so $\mf a\subseteq\mf p$, so we conclude that
	\[\mf a\subseteq\rad\mf a=\bigcap_{\mf p\supseteq\mf a}\mf p\subseteq\bigcap_{\mf p\supseteq\op{Ann}m}\mf p=\rad\op{Ann}m.\]
	Because $\mf a$ is finitely generated (as $A$ is Noetherian), we conclude $\mf a^n\subseteq\op{Ann}m$ for some $n>0$ by looking at each generator. It follows that $m\in\Gamma_\mf a(M)$.
\end{proof}
\begin{remark} \label{rem:h0-qcoh}
	Thus, we would like to show that $H^0_Z(X,\mc F)$ is quasicoherent when $\mc F$ is quasicoherent; notably, from the above, we will see that $H^0_Z(X,\mc F)=\widetilde{\Gamma_\mf a(M)}$ by taking global sections.

	Well, we note that we have an exact sequence
	\[0\to H^0_Z(X,\mc F)\to\mc F\to j_*\mc F|_{X\setminus Z},\]
	where $j\colon(X\setminus Z)\subseteq X$ is the inclusion. Indeed, exactness here can be checked at stalks. But now this is a scheme-theoretic result: the kernel of a morphism of quasicoherent sheaves is in fact quasicoherent, so it will be enough to see that $j_*\mc F|_{X\setminus Z}$ is quasicoherent. But in our setting above, the direct image sheaf of a quasicoherent sheaf is still quasicoherent when the relevant scheme morphism is separated and quasicompact (when the source is Noetherian), but of course all these hypotheses are satisfied.
\end{remark}

We now investigate this functor $\Gamma_\mf a$ further.
\begin{lemma}
	Fix a Noetherian ring $A$ and an ideal $\mf a\subseteq M$. Then the functor $\Gamma_\mf a$ preserves injectives.
\end{lemma}
\begin{proof}
	Fix an injective $I$, and set $J\coloneqq\Gamma_\mf a(I)$. By the homework, it is enough to show that any ideal $\mf b\subseteq A$ with a morphism $\varphi\colon\mf b\to J$ can be extended to a morphism $A\to J$.

	Now, for each $b\in\mf b$, we note that $\mf a^n\varphi(b)=0$ for some $n>0$. In particular, $\mf a^nb\subseteq\ker\varphi$. By choosing generators for $\mf b$, we may actually assume that $\mf a^n\mf b\subseteq\ker\varphi$. However, by \Cref{thm:krull-a-adic}, we may thus find $n'$ such that $\mf a^{n'}\cap\mf b\subseteq\mf a^n\mf b$. In total, we have the following diagram.
	% https://q.uiver.app/?q=WzAsNixbMCwwLCJBIl0sWzAsMSwiXFxtZiBiIl0sWzEsMSwiXFxkaXNwbGF5c3R5bGVcXGZyYWN7XFxtZiBifXtcXG1mIGFee24nfVxcY2FwXFxtZiBifSJdLFsxLDAsIlxcZGlzcGxheXN0eWxlXFxmcmFjIEF7XFxtZiBhXntuJ31cXGNhcFxcbWYgYn0iXSxbMiwxLCJKIl0sWzMsMSwiSSJdLFsxLDAsIiIsMCx7InN0eWxlIjp7InRhaWwiOnsibmFtZSI6Imhvb2siLCJzaWRlIjoidG9wIn19fV0sWzIsMywiIiwwLHsic3R5bGUiOnsidGFpbCI6eyJuYW1lIjoiaG9vayIsInNpZGUiOiJ0b3AifX19XSxbMCwzLCIiLDEseyJzdHlsZSI6eyJoZWFkIjp7Im5hbWUiOiJlcGkifX19XSxbMSwyLCIiLDEseyJzdHlsZSI6eyJoZWFkIjp7Im5hbWUiOiJlcGkifX19XSxbMiw0XSxbMyw1LCIiLDAseyJzdHlsZSI6eyJib2R5Ijp7Im5hbWUiOiJkYXNoZWQifX19XSxbNCw1XV0=&macro_url=https%3A%2F%2Fraw.githubusercontent.com%2FdFoiler%2Fnotes%2Fmaster%2Fnir.tex
	\[\begin{tikzcd}
		A & {\displaystyle\frac A{\mf a^{n'}}} \\
		{\mf b} & {\displaystyle\frac{\mf b}{\mf a^{n'}\cap\mf b}} & J & I
		\arrow[hook, from=2-1, to=1-1]
		\arrow[hook, from=2-2, to=1-2]
		\arrow[two heads, from=1-1, to=1-2]
		\arrow[two heads, from=2-1, to=2-2]
		\arrow[from=2-2, to=2-3]
		\arrow[dashed, from=1-2, to=2-4]
		\arrow[from=2-3, to=2-4]
	\end{tikzcd}\]
	Well, we would like to show that the induced map $A/\mf a^{n'}\to I$ factors through $J$, but this can just be checked directly from the construction. Indeed, the element $x\in A/\mf a^{n'}$ in $I$ must be killed by $\mf a^{n'}$, so the image of $1$ does in fact live in $\Gamma_\mf a(I)=J$.
\end{proof}
We now move towards showing \Cref{prop:inj-qcoh-sheaf-is-flasque}.
\begin{lemma} \label{lem:inj-localization-is-surj}
	Fix a Noetherian ring $A$. Given an injective $A$-module $I$, any $f\in A$ has the localization map $I\to I_f$ surjective.
\end{lemma}
\begin{proof}
	For each $n>0$, we set $\mf b_n\coloneqq\op{Ann}f^n$. Notably, this grants the ascending chain of ideals
	\[\mf b_1\subseteq\mf b_2\subseteq\mf b_3\subseteq\cdots,\]
	so this must stabilize at some $\mf b_r$. Setting $\theta\colon I\to I_f$ to be the localization map, pick any $y/f^n\in I_f$ for $y\in I$, which we would like to be in the image of $I$.

	Now, note there is a map $\left(f^{n+r}\right)\to I$ by $f^{n+r}\mapsto f^ry$. This is well-defined by construction of $r$. In particular, if $af^{n+r}=a'f^{n'+r}$, we can check that $af^ry=a'f^ry$ by computing some annihilators. Thus, the injectivity of $I$ lifts this map to some
	\[\psi\colon A\to I\]
	such that $\psi\left(f^{n+r}\right)=f^ry$. Notably, setting $z\coloneqq\psi(1)$, we see $f^{n+r}z=\psi\left(f^{n+r}\right)=f^ry$. Rearranging, we see that $z$ gets sent to $y/f^n$ in $I_f$, which is what we wanted.
\end{proof}
We will continue the proof next class.

\end{document}