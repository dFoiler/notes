% !TEX root = ../notes.tex

\documentclass[../notes.tex]{subfiles}

\begin{document}

\section{February 1}

We continue discussing the Frobenius morphism.

\subsection{Relative Frobenius}
We begin class with a few remarks on the Frobenius automorphism when $k$ is not perfect. Roughly speaking, the issue is that a Frobenius morphism $F\colon X_p\to X$ is not $p$. In general, one sees $[K(X):kK(X)^p]=p$ (note we have taken the composite with $p$!), so the extension $\left[K(X):K(X)^p\right]>p$.

The idea is to generalize \eqref{eq:frob-square}. Namely, we construct our Frobenius $X^{(p)}\colon X$ by pullback in the following square.
% https://q.uiver.app/?q=WzAsNSxbMSwxLCJYXnsocCl9Il0sWzIsMSwiWCJdLFsxLDIsIlxcU3BlYyBrIl0sWzIsMiwiXFxTcGVjIGsiXSxbMCwwLCJYIl0sWzQsMSwiRiIsMCx7ImN1cnZlIjotMn1dLFs0LDIsIiIsMix7ImN1cnZlIjoyfV0sWzQsMCwiRl97WC9rfSJdLFswLDFdLFswLDJdLFsyLDMsIkYiXSxbMSwzXV0=&macro_url=https%3A%2F%2Fraw.githubusercontent.com%2FdFoiler%2Fnotes%2Fmaster%2Fnir.tex
\[\begin{tikzcd}
	X \\
	& {X^{(p)}} & X \\
	& {\Spec k} & {\Spec k}
	\arrow["F", curve={height=-12pt}, from=1-1, to=2-3]
	\arrow[curve={height=12pt}, from=1-1, to=3-2]
	\arrow["{F_{X/k}}", from=1-1, to=2-2]
	\arrow[from=2-2, to=2-3]
	\arrow[from=2-2, to=3-2]
	\arrow["F", from=3-2, to=3-3]
	\arrow[from=2-3, to=3-3]
\end{tikzcd}\]
Here $F_{X/k}$ is the relative Frobenius. Notably, when $k$ is perfect, we see that the canonical projection $X^{(p)}\to X$ is an isomorphism of $k$-schemes because the Frobenius on the bottom is an isomorphism. One can even make this isomorphism explicit by arguing as in \Cref{ex:perf-xp-is-x}.

We quickly check that this roughly generalizes our earlier construction.
\begin{example}
	Set $X=\Spec A$ for a $k$-algebra $A$. We track the diagram in this case.
\end{example}
\begin{proof}
	Then $X^{(p)}=\Spec(A\otimes_kk)$, where the $k$-action on $k$ is given by the Frobenius $F\colon k\to k$. Thus, when $k$ is perfect, we are indeed looking $k$ acting on $A$ by $f\otimes\alpha^p=\alpha f\otimes1$, so we have an isomorphism $A\otimes_kk\cong A$, where the $k$-action is given by
	\[g^{(p)}(\alpha)a=g(\alpha^{1/p})a,\]
	where $g\colon k\to A$ is the structure morphism.

	To understand the relative Frobenius $X^{(p)}\to X$, we note that it sends $f\otimes\alpha\mapsto\alpha f^p$ by tracking the diagram. As such, when $k$ is perfect, we may think of our morphism as
	\[\alpha^{1/p}f\mapsto\alpha f^p,\]
	so the map is in fact $k$-linear.
\end{proof}
\begin{remark}
	Thus, when $k$ is not perfect, we see that $X^{(p)}$ need not be isomorphic to $X$, even absolutely. For example, take $A=k[x]/\left(x^2-\alpha\right)$ for some $\alpha\in k$. In $A\otimes_kk$, we see that
	\[(x\otimes1)^2=\alpha\otimes1=1\otimes\alpha^p.\]
	Thus, $A\otimes_kk$, even though it is a two-dimensional $k$-algebra, is isomorphic to $k[x_p]/\left(x_p^2-\alpha^p\right)$, which is not the same as $A$ when $k$ is not perfect!
\end{remark}
Anyway, let's see the analogue of \Cref{thm:all-is-frob} in our setting.
\begin{theorem}
	Fix a finite morphism $f\colon X\to Y$ of (smooth, proper, integral) $k$-curves, where $k$ is a field of characteristic $p>0$. Further, suppose that $f^\sharp\colon K(Y)\to K(X)$ is purely inseparable of degree $p$. Then $Y\cong X^{(p)}$ for some $r$, and $f$ is the relative Frobenius under this isomorphism.
\end{theorem}

\subsection{Inseparability for Fun and Profit}
We will not prove this, but it's fun to know.
\begin{corollary}
	Fix a finite morphism $f\colon X\to Y$ of $k$-curves, where $k$ is algebraically closed. Then $g(X)\ge g(Y)$.
\end{corollary}
\begin{proof}
	Factor the field extension $K(Y)\subseteq K(X)$ into a purely inseparable extension followed by a separable extension. Genus does not change when we are dealing with an inseparable extension by \Cref{thm:all-is-frob}, so it suffices to show that the genus does not fall with separated morphisms. Well, by \Cref{thm:rh}, we see
	\[2g(X)-2=(\deg f)(2g(Y)-2)+\deg R\ge2g(Y)-2,\]
	so $g(X)\ge g(Y)$ follows.
\end{proof}
\begin{remark}
	For our equality cases, we see that one either has an isomorphism or an unramified finite morphism of elliptic curves. Such unramified maps of elliptic curves (which are not isomorphisms) do exist.\todo{Classify them?}
\end{remark}

\subsection{Embeddings of Curves}
We now return to the case where $k$ is algebraically closed. Our next goal is to show that every $k$-curve can be embedded into $\PP^3_k$. As such, we are roughly speaking interested in showing that certain linear systems separate points (which means that we are base-point-free) and tangent vectors.

Let's begin by discussing being base-point-free.
\begin{lemma}
	Fix a divisor $D$ on a $k$-curve $X$. Then $\OO_X(D)$ is base-point-free if and only if
	\[\dim|D|=\dim|D-P|+1.\]
\end{lemma}
\begin{proof}
	We are essentially showing $\ell(D)=\ell(D-P)+1$. Notably, we do always have $\ell(D)\le\ell(D-P)+1$ by staring at the short exact sequence
	\[0\to\OO_X(-P)\to\OO_X\to k(P)\to0\]
	and then tensoring by $\OO_X(D)$ to give
	\[0\to\mc L(D-P)\to\mc L(D)\to k(P)\to0.\]
	We now discuss the equality. Note $\ell(D)=\ell(D-P)+1$ is just telling us that there is a global section in $\Gamma(X,\OO_X(D))\setminus\Gamma(X,\OO_X(D-P))$. In other words, we have some $f\in K(X)$ such that $D+\op{div}(f)$ is effective, but $D-P+\op{div}(f)$ is not. Thus, we see that we are saying $P$ is not in the support of $f$, so $P$ is not a base-point for $D$.
\end{proof}
Next time we will make a similar dimension condition to be ample and very ample.

\end{document}