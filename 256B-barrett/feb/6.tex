% !TEX root = ../notes.tex

\documentclass[../notes.tex]{subfiles}

\begin{document}

\section{February 6}

As is to be expected, we sleep with one eye open.
\begin{remark}
	The next homework will be due Sunday night. The hope is that it is fun.
\end{remark}

\subsection{Small Projective Embeddings}
We are still trying to embed curves into $\PP^3_k$. Our main tool continues to be \Cref{prop:div-to-proj-embed}.
\begin{corollary} \label{cor:ample-is-positive}
	Fix a divisor $D$ on a $k$-curve $X$ is ample if and only if $\deg D>0$.
\end{corollary}
\begin{proof}
	Certainly if $\deg D\le0$, then $\deg(nD)=0$ and thus $\ell(nD)=0$ for all positive integers $n$, so $nD$ is never very ample, so $D$ is not ample. (Alternatively, if $nD$ is very ample, then $nD$ is the pull-back of $\OO_{\PP^1_k}(1)$ some closed embedding $f\colon X\to\PP^1_k$ and hence has positive degree.) Conversely, if $\deg D>0$, then $\deg(2g+1)D\ge2g+1$, so $(2g+1)D$ is very ample by \Cref{cor:big-is-very-ample}, so $D$ is ample.
\end{proof}
\begin{example}
	With $X=\PP^1_k$, we note that $\OO_{\PP^1_k}(1)$ is very ample by the (identity) embedding $X\cong\PP^1_k$. It follows that $\OO_{\PP^1_k}(n)$ are all very ample (and hence ample) for all $n>0$. By \Cref{cor:ample-is-positive}, these are all the ample divisors of $X$.
\end{example}
\begin{remark}
	It turns out that a very ample divisor $D$ on $X$ yielding a closed embedding $f\colon X\to\PP^n_k$ has $\deg D=\deg f(X)$. Here, $\deg f(X)$ is defined using some intersection theory; for example, if $n=2$, then this degree is the degree of the polynomial cutting out $X$. For example, if $g(X)=1$ and $D$ is a divisor of degree $3$ (and hence very ample by \Cref{cor:ample-is-positive}). Further,
	\[\ell(D)=\ell(K-D)+\deg D+(1-g)=0+3+0=3,\]
	so we define a closed embedding to $\PP^2_k$. We conclude that $X$ has an embedding as a cubic curve in $\PP^2_k$. Conversely, Exercise~I.7.2(b) in \cite{hartshorne} tells us that any cubic plane curve has genus $\frac12(3-1)(3-2)=1$. Note that adjusting the divisor's linear equivalence class can give us different embeddings to $\PP^2_k$ (which are not the same up to an automorphism of $\PP^2_k$).
\end{remark}
Anyway, we are now almost ready to prove our main result.
\begin{theorem} \label{thm:x-to-p3}
	Fix a $k$-curve $X$.\todo{K algebraically closed?} Then $X$ has an embedding to $\PP^3_k$.
\end{theorem}
The outline here is as follows. To begin, fix some closed embedding $X\to\PP^n_k$ for some $n>0$. Then if $n>3$, we will show that we can project down from $\PP^n_k$ to $\PP^{n-1}_k$ in a way which preserves us having a closed embedding. Inducting $n$ downwards like this will complete the proof.

We know how to do this first step by \Cref{cor:big-is-very-ample}, so we focus on the second step.
\begin{proposition} \label{prop:project-curve-down}
	Fix a $k$-curve $X$ embedded into $\PP^n_k$ for some positive integer $n$. Given $O\in\PP^n_k\setminus X$, then the projection $\varphi\colon X\to\PP^{n-1}_k$ from $O$ is a closed embedding if and only if the following both hold.
	\begin{itemize}
		\item $O$ does not belong on any secant line.
		\item $O$ does not belong on any tangent line.
	\end{itemize}
\end{proposition}
Note that the line we are projecting from $O$ onto does not matter so much because it merely adjusts the map by an automorphism of the ambient space $\PP^n_k$, which turns into an automorphism of $\PP^{n-1}_k$ on the embedding.
\begin{example}
	One can compute that the projection from $O=[0:\cdots:0:1]$ in $\PP^n_k$ to $V(x_n)$ is given by the projection
	\[[a_0:\cdots:a_n]\mapsto[a_0:\cdots:a_{n-1}].\]
	Namely, the line connecting $O$ and $[a_0:\cdots:a_{n-1}]$ is parameterized by $[t_0:t_1]\mapsto[t_0a_0:\cdots:t_0a_{n-1}:t_1a_n]$, which tells us what the intersection with $V(x_n)$ should be.

	In our theory of linear systems, we note that the global sections $x_0,\ldots,x_{n-1}$ span some subspace $V\subseteq\Gamma(\PP^n_k,\OO_{\PP^n_k}(1))$, and we can see that the only base-point here is the point $O$ because this is the only point of simultaneous vanishing. Thus, our theory grants us a morphism $(\PP^n_k\setminus\{0\})\to\PP^{n-1}_k$ given by the above formula! Explicitly, on the affine chart $U_i\subseteq\PP^n_k$ where $x_i\ne x_0$ doesn't vanish, we are looking at the ring map
	\[k\left[\frac{y_0}{y_i},\ldots,\frac{y_{n-1}}{y_i}\right]\to k\left[\frac{x_0}{x_i},\ldots,\frac{x_n}{x_i}\right]\]
	given by $y_j/y_i\mapsto x_j/x_i$.
\end{example}
\begin{proof}
	Roughly speaking, we don't want $O$ to be on any secant line so that the projection doesn't send two points to the same point. Additionally, we don't want $O$ to be on any tangent line so that the closed embedding separates tangent vectors. We omit the remainder of the proof aside from this general intuition.
\end{proof}
We now turn to the proof of \Cref{thm:x-to-p3}.
\begin{proof}[Proof of \Cref{thm:x-to-p3}]
	By \Cref{cor:big-is-very-ample}, we have some closed embedding $X\to\PP^n_k$ for some $n$ large enough. Now, if $n>3$, we use \Cref{prop:project-curve-down} to project down to $\PP^{n-1}_k$. Thus, it suffices to find a point $O$ not on any secant line or tangent line. Well, the ``secant variety'' $\op{Sec}X$ defined by the image of the obvious map
	\[(X\times X\setminus\Delta_X)\times\PP^1_k\to\PP^n_k\]
	we can see has dimension of the image at most $3$. Explicitly, on points, this map is given by
	\[([a_0:\cdots:a_n],[b_0:\cdots:b_n],[t_0:t_1])\mapsto[a_0t_0+b_0t_1:\cdots:a_nt_0+b_nt_1].\]
	Similarly, the ``tangent variety'' $\op{Tan}X$ defined by the image of the obvious map
	\[X\times\PP^1_k\to\PP^n_k\]
	we can see has dimension of the image at most $2$. Thus, with $n>3$, these closed subschemes are proper and so have dense complement, meaning that we can find can point $O$ in the complement of these varieties. This completes the proof.
\end{proof}

\end{document}