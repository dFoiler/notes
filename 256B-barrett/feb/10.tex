% !TEX root = ../notes.tex

\documentclass[../notes.tex]{subfiles}

\begin{document}

\section{February 10}

Last class, we were in the middle of describing the canonical morphism of hyperelliptic curves. I have moved the entire proof of today because we did more work on it today.

\subsection{Hyperelliptic Curves}
We will spend the rest of class understanding hyperelliptic curves. The following is our statement.
\begin{theorem} \label{thm:hyper-canonical-divisor}
	Fix a hyperelliptic curve $X$ of genus $g(X)\ge2$.
	\begin{listalph}
		\item Then $X$ has a unique divisor class $g^1_2$ yielding the double-cover $\pi\colon X\to\PP^1_k$.
		\item The canonical morphism $f\colon X\to\PP^{g-1}$ can be written as
		\[X\to\PP^1_k\to\PP^{g-1}_k,\]
		where the last map is the $(g-1)$-uple embedding.
		\item Every effective canonical divisor can be written as the sum of $(g-1)$ effective divisors linearly equivalent to $g^1_2$.
	\end{listalph}
\end{theorem}
\begin{proof}
	For brevity, let $X'$ be the image $f(X)$, and we fix some $g^1_2$ on $X$ yielding a double-cover $X\to\PP^1_k$.
	
	Now, for any effective divisor $P+Q$ linearly equivalent to $g^1_2$, then we claim $Q$ is a base-point of $K-P$: indeed, we know $\ell(K-P)=g-1$ because $K$ is base-point-free by \Cref{lem:k-is-base-point-free}, but $K$ is not very ample witnessed by $P$ and $Q$ by the proof of \Cref{prop:k-is-very-ample}, so $\ell(K-P-Q)=\ell(K-P)$ is forced. Thus, $Q$ is in fact a base-point.

	We now claim that $f$ is not birational. There are two cases.
	\begin{itemize}
		\item In the case where $P\ne Q$, we see that $K$ does not separate the points $P$ and $Q$. Explicitly, we see that $s\in\Gamma(X,\OO_X(K-P))$ is equivalent to $s\in\Gamma(X,\OO_X(K-P-Q))$ is equivalent to $s\in\Gamma(X,\OO_X(K-Q))$ (by symmetry). In other words, we are saying that any divisor $K+\op{div}(s)$ which retains $P$ in its support will also retain $Q$ in its support.
		
		It follows that $f(P)=f(Q)$: otherwise $f(P)\ne f(Q)$ lets us separate the two points in $\PP^{g-1}_k$ and hence find a basis of $\Gamma(\PP^{g-1}_k,\OO_{\PP^{g-1}}(1))$ separating them. But then we can pull back this basis to $\Gamma(X,\OO_X)$ to find sections separating $P$ and $Q$.

		\item Otherwise, we in the case where $P=Q$, we are now given that $P$ is a base-point of $K-P$. Explicitly, we know that $s\in\Gamma(X,\OO_X(K-P))$ is equivalent to $s\in\Gamma(X,\OO_X(K-2P))$, which now means that $f$ does not separate tangent vectors at $P$---namely, the image of $\Gamma(X,\OO_X(K))$ does not generate $\mf m_P\mc O_X(K)_P/\mf m_P/\OO_X(K)_P$. However, $\OO_{\PP^{g-1}_k}(1)$ does separate these tangent spaces by hyperplanes, so $f$ is again not a closed embedding at the local ring at $P$.
	\end{itemize}
	By adjusting $P+Q$ appropriately in its linear equivalence class, we see that $f$ fails to be a closed embedding at infinitely many points, so $f$ is not a birational morphism.

	As such, $\mu\coloneqq\deg f$ is at least $2$.\footnote{Alternatively, note that $\dim|g^1_2|=1$ forces there to be infinitely many representatives of the form $P+Q$ with $P\ne Q$, for otherwise there are infinitely many representatives of the form $2P$, which would mean that our double-cover $X\to\PP^1_k$ is ramified at infinitely many points, which is impossible.} However, letting $d$ denote the degree of $X'\subseteq\PP^{g-1}_k$, we claim
	\[d\mu\stackrel?=2g-2.\]
	For this, we must understand $d=\deg X'$, which we note is the number of intersections (counted with multiplicity) of a hyperplane section of $\PP^{g-1}_k$ intersecting $Y=X'\setminus\op{Sing}X'$, where $\op{Sing}X'$ is the singular locus. Letting $H\cap Y$ denote such an intersection (with $d$ points and hence degree $d$ as a divisor), we restrict $f\colon X\to X'$ to $f\colon X\to\widetilde{X'}$, granting
	\[\deg f^*(H\cap \widetilde{X'})=(\deg f)d=\mu d\]
	by explicitly writing down what does $f^*$ does on points on the finite morphism of smooth curves $f|_{\widetilde{X'}}$. In total, we conclude that $d\le g-1$.

	Now, the composite $\widetilde{X'}\to X'\subseteq\PP^{g-1}_k$ is a projective morphism and so grants a divisor $\mf d$ on $\widetilde{X'}$, of dimension $g-1$. To compute the degree, we intersect $\widetilde{X'}$ with a generic hyperplane (for example, avoiding the singular locus of $X'$), which still has $d$ intersections after passing to the birational curve $X'$. As such, we see $\deg\mf d=d$. But from the homework, we have the inequality
	\[\dim\mf d\le\deg D\le g-1,\]
	so we conclude that equalities must hold everywhere, which in turn requires $\mf d$ to be a complete linear system. Further, the homework also tells us that the equality case forces $g(X')=0$, so we must have $X\cong\PP^1_k$ (note the degree of $\mf d$ is positive!), so comparing degrees enforces $\mf d=\OO_{\PP^1_k}(g-1)$. Notably, we also get $\deg\mu=2$ here.

	Now, the map $\widetilde{X'}\to X'\subseteq\PP^{g-1}_k$ is given by the line bundle $\OO_{\PP^1_k}(g-1)$ and $g$ spanning sections, but there is a basis of the $g$ monomials of degree $g-1$ in the variables $x_0$ and $x_1$, so our morphism we can write out explicitly as
	\[[x_0:x_1]\mapsto\left[x_0^{g-1}:x_0^{g-2}x_1:\cdots:x_0x_1^{g-2}:x_1^{g-1}\right]\]
	by expanding out our sections. (One can also show that this is a closed embedding.) This is exactly the $(g-1)$-uple embedding (or ``Veronese'' embedding). We conclude that the canonical morphism $X\to\PP^{g-1}_k$ has been factored as
	\[X\to\widetilde{X'}\cong\PP^1_k\to\PP^{g-1}_k,\]
	where the map $\PP^1_k\to\PP^{g-1}_k$. In particular, the image of the $(g-1)$-uple embedding can be checked to be nonsingular, so $X'$ is successfully nonsingular, and we have $\widetilde{X'}\cong X'$.

	Roughly speaking, it remains to discuss the uniqueness of our double-cover $\pi\colon X\to\PP^1_k$. Notably, for any $P+Q$ linearly equivalent to $g^1_2$, pulling back the divisor $f(P)\in X'\cong\PP^1_k$ up to $P+Q$ enforces $f(P)=f(Q)$. Running the same argument with $\pi$, which comes from a particular representative of $g^1_2$, we conclude that $f$ and $\pi$ coincide for infinitely many points of $X$, so they must coincide because $\op{eq}(f,\pi)$ has positive codimension in $X$ and therefore must vanish. It follows that $\pi$ and $f$ are uniquely determined by representatives of $g^1_2$. And quickly, we note that $g^1_2$ is also unique: it factors through the canonical morphism $f\colon X\to\PP^{g-1}_k$ (followed by an embedding), which we already know to be unique, so the precise morphism $\pi$ is unique (of course, up to automorphism of $\PP^1_k$).

	As such, we have shown (a) and (b). To show (c), we note that all effective canonical divisors are realizes as pullbacks of the hyperplane line bundle $\OO_{\PP^{g-1}_k}(1)$ on $X'\subseteq\PP^{g-1}_k$. But we can pull this back along $f$ factored as
	\[X\stackrel\pi\to\PP^1_k\cong X'\subseteq\PP^{g-1}_k.\]
	Well, this hyperplane intersects $X'$ at $g-1$ points, and then each of those points in $\PP^1_k$ will pull back to an effective divisor which represents $g^1_2$.
\end{proof}

\subsection{Special Divisors}
We pick up the following definition.
\begin{definition}[special]
	Fix a divisor $D$ on a $k$-curve $X$. Then $D$ is \textit{special} if and only if $\ell(K-D)>0$.
\end{definition}
\begin{remark}
	If $\ell(K-D)=0$, then \Cref{thm:rh} tells us
	\[\ell(D)=\deg D+1-g,\]
	so we understand the divisor pretty well.
\end{remark}
We will show the following result.
\begin{proposition}[Clifford] \label{prop:clifford}
	Fix an effective special divisor $D$ on the $k$-curve $X$. Then
	\[\dim|D|\le\frac12\deg D.\]
\end{proposition}
To show this, we have the following lemma.
\begin{lemma} \label{lem:add-divisors-dim}
	Fix effective divisors $D$ and $E$ on a $K$-curve $X$. Then
	\[\dim|D|+\dim|E|\le\dim|D+E|.\]
\end{lemma}
\begin{proof}
	Note that there is a map $\varphi\colon|D|\times|E|\to|D+E|$ by sending a pair $(D',E')$ of effective divisors in the correct linear equivalence classes to $D'+E'$. Note that any effective divisor in $|D+E|$ can be written as a sum of effective divisors in only finitely many ways, so this map has finite fibers.

	Going further, we can upgrade $\varphi$ to a $k$-bilinear map of $k$-vector spaces given by
	\[\Gamma(X,\mc L(D))\times\Gamma(X,\mc L(E))\to\Gamma(X,\mc L(D+E))\]
	by sending $(f,g)\mapsto fg$. In particular, this grants us a map on projective spaces
	\[\PP\Gamma(X,\mc L(D))\times\PP\Gamma(X,\mc L(E))\to\PP\Gamma(X,\mc L(D+E))\]
	agreeing with our earlier definition of $\varphi$ by tracking through how $\PP\Gamma(X,\mc L(D))$ agrees with $|D|$. Notably, this map still has finite fibers, so it is quasifinite, so taking dimensions (this morphism is basically finite) lets us conclude $\dim|D|+\dim|E|\le\dim|D+E|$.
\end{proof}
We are now ready to show \Cref{prop:clifford}.
\begin{proof}[Proof of inequality in \Cref{prop:clifford}]
	Because $D$ is effective and special, we know $\ell(K-D)\ne0$, so we can find an effective divisor $E$ linearly equivalent to $K-D$. As such, \Cref{lem:add-divisors-dim} grants
	\[\dim|D|+\dim|K-D|=\dim|D|+\dim|K-D|\le\dim|K+E|=\dim|K|=g-1,\]
	but \Cref{thm:rh} tells us
	\[\dim|D|-\dim|K-D|\le\deg D+1-g,\]
	so rearranging grants the inequality.
\end{proof}
\begin{remark} \label{rem:clifford-eqs}
	We can also check some equality cases, quickly. If $D=0$, there is nothing to say. When $D=K$, we simply note $\dim|K|=g-1$, but $\deg K=2g-2$, so we are done. Lastly, suppose $X$ is hyperelliptic and $D\sim kg^1_2$ for some $k$. We only care about linear equivalence class, so we might as well assume $D=kg^1_2$ and proceed by induction on $k$. At $k=1$, we note $\dim g^1_2=1$ and $\deg g^1_2=2$, so equality holds. Then for our induction, we note \Cref{lem:add-divisors-dim} gives
	\[\dim\left|(k+1)g^1_2\right|\ge\dim\left|kg^1_2\right|+\dim\left|kg^1_2\right|=k+1,\]
	but we just showed that $\dim\left|(k+1)g^1_2\right|\le\frac12\deg\left((g+1)g^1_2\right)=k+1$ above, so we must have equalities.
\end{remark}

\end{document}