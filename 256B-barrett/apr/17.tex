% !TEX root = ../notes.tex

\documentclass[../notes.tex]{subfiles}

\begin{document}

\section{April 17}

Today we would like to finish flatness.

\subsection{Flatness via Locally Free}
Let's finally prove that our two notions of associated primes coincide.
\begin{proposition}
	Fix a nonzero finitely generated module $M$ over a Noetherian ring $R$. Then $A(\widetilde M)=A(M)$.
\end{proposition}
\begin{proof}
	Certainly we have $A(M)\subseteq A(\widetilde M)$. On the other hand, $x\in A(\widetilde M)$ means that we have some section $s\in M$ annihilated by an ideal primary to $\mf m_x$. It follows that every element of $\mf m_x$ is a zero-divisor on $M$ by taking powers, so we can write
	\[\mf m_x=\bigcup_{\mf q\in A_{\OO_{X,x}}(M)}\mf q,\]
	so prime avoidance enforces $\mf m_x$ to live in $A_{\OO_{X,x}}(M_x)$, so undoing the localization confirms $x\in A(M)$. Namely, it is true more generally that
	\[A_{R\left[S^{-1}\right]}M\left[S^{-1}\right]=A_R(M)\cap\Spec R\left[S^{-1}\right]\]
	for any multiplicative subset $S\subseteq R$, which completes the proof.
\end{proof}
To solidify our idea that flatness means behaved fibers, we pick up two results.
\begin{proposition}
	Fix a finite morphism $f\colon X\to Y$ of Noetherian schemes. Then a coherent sheaf $\mc F$ on $X$ is flat over $Y$ if and only if $f_*\mc F$ is locally free.
\end{proposition}
\begin{proof}
	Flatness is a stalk-local property, so we may assume that $Y$ is affine, whereupon taking the pre-image along $f$ forces $X$ to be affine. Thus, we let $X=\Spec B$ and $Y=\Spec A$ so that $f\colon A\to B$ makes $B$ into a finite $A$-algebra. In algebra, $\mc F$ becomes some $B$-module $M$ flat over $A$, and $f_*\mc F$ is the corresponding $A$-module (under restriction).

	Now, $\mc F_x$ is a localization of $(f_*\mc F)_y$ by tracking through the definitions (namely, $f$ might not be injective). Thus, in the backward direction, we note that $\mc F_x$ being flat over $\OO_{Y,y}$ ensures that $\mc F_x$ is free over $\OO_{Y,y}$, which does not change upon a further localization.

	In the forward direction, we are given that $M$ is flat over $A$, which means $M_\mf p$ is flat over $A_\mf p=\OO_{Y,y}$. Now, finitely generated modules over Noetherian local rings being flat implies that they are free, so $M_\mf p$ is free over $A_\mf p$, which is what we wanted. To be explicit, one can find a surjection $\OO_V^{\oplus n}\to\mc F$ (for an open subset $V\subseteq Y$) which is an isomorphism at $y\in V$, so the kernel and cokernel vanish at the point $y\in Y$, so this will be an isomorphism in some open subset of $V$, so we are indeed locally free.
\end{proof}
\begin{proposition}
	Fix a finite morphism $f\colon X\to Y$ of Noetherian schemes; further, suppose that $Y$ is integral. Then a coherent sheaf $\mc F$ on $X$ is flat over $Y$ if and only if the function
	\[y\mapsto\dim_{k(y)}((f_*\mc F)_y\otimes k(y))\]
	is constant.
\end{proposition}
\begin{proof}
	Again, everything is being checked stalk-locally, so we may reduce and adopt notation as in the previous proof. As a sanity check, note that the point here is that
	\[(f_*\mc F)_y\otimes k(y)=M\otimes_Ak(y)\]
	is a finite-dimensional $k(y)$-vector space, so at the very least it makes sense to discuss its dimension.

	The forward direction is given by the previous proposition because $\mc F$ being flat implies locally free, which means locally constant rank. In the reverse direction, we are trying to show the following algebraic statement: fix a Noetherian local integral domain $A$. Then a finitely generated $A$-module $M$ is free if and only if
	\[\dim_{\op{Frac}A}M\otimes_A\op{Frac}A=\dim_{A/\mf m}M\otimes_A(A/\mf m).\]
	Notably, we are comparing the generic point of $A$ to the closed point, which will give our constancy. Anyway, the forward direction here has nothing to say. For the backward direction, fix a basis of $M/\mf mM$ (over $A/\mf m$), and choosing lifts to $M$ yields by Nakayama's lemma a surjection $\pi\colon A^d\to M$ which reduce to an isomorphism$\pmod{\mf m}$.\footnote{To see this application, recall that Nakayama's lemma asserts that $IM=M$ for $I$ contained in the maximal ideal of our local ring $A$ implies that $M=0$. Then the point is that our surjection $A^d\to M$ has cokernel vanishing when reduced$\pmod{\mf m}$, so the cokernel must fully vanish by Nakayama's lemma.} At the current moment, we have an exact sequence
	\[0\to\ker\pi\to A^d\to M\to0.\]
	But now we can localize at the generic point to receive the exact sequence
	\[0\to K\otimes_A\ker\pi\to K^d\to K\otimes_AM\to0\]
	where $K\coloneqq\op{Frac}A$. However, the right-hand map is a surjection of two $K$-vector spaces of the same dimension (by hypothesis!), so it is an isomorphism, so $K\otimes_A\ker\pi$ vanishes, so $\ker\pi$ vanishes because it is torsion-free (as the submodule of the torsion-free module $A^d$). Thus, $A^d\cong M$ follows.
\end{proof}
Let's see some examples.
\begin{example}
	Fix an algebraically closed field $k$, and set $X\coloneqq\Spec k[x]$ and $Y\coloneqq\Spec k[y]$, and let $f\colon X\to Y$ be the map given by $y\mapsto x^2$. Then $f$ is flat. There are three cases.
	\begin{itemize}
		\item To check that $\OO_X$ is flat at the point $(y-\alpha)$ for $\alpha\ne0$, we write $\alpha=\beta^2$ and compute
		\[\dim_{k(y)}\left(\frac{k[x]}{\left(x^2-\alpha\right)}\otimes k(y)\right)=2.\]
		\item At the point $(y)$, then we are looking at $k[x]/\left(x^2\right)$, which still has dimension $2$ over residue fields $k$.
		\item At the generic point, we are looking at $k(y)[x]/\left(x^2-y\right)$, which still has dimension $2$ over the residue field $k(y)$.
	\end{itemize}
\end{example}
\begin{remark}
	For varieties, it should be enough to just check flatness at closed points by some kind of upper semi-continuity result. Alternatively, recall that the flat locus is open.
\end{remark}
\begin{example}
	Fix an algebraically closed field $k$, and set $X\coloneqq\Spec k[x,y]$ and $Y\coloneqq\Spec k\left[x^2,xy,y^2\right]$ with the natural map $X\to Y$ given by the inclusion $k\left[x^2,xy,y^2]\right]\subseteq k[x,y]$. Then $f$ is not going to be flat. For example, at the closed point $\left(x-\alpha,y-\beta\right)\in X$, so the image in $Y$ is the maximal ideal $\mf q\coloneqq\left(x^2-\alpha^2,xy-\alpha\beta,y^2-\beta^2\right)$. Reversing, we can compute our fiber as
	\[\frac{k[x,y]}{\mf pk[x,y]}=\frac{k[x,y]}{(x-\alpha,y-\beta)}\oplus\frac{k[x,y]}{(x+\alpha,y+\beta)},\]
	so the fiber over $p$ has the closed points $(\alpha,\beta)$ and $(-\alpha,-\beta)$, provided that $\alpha$ and $\beta$ are nonzero. However, in the case that $\alpha=\beta=0$, our fiber essentially collapses down a dimension.
\end{example}

\end{document}