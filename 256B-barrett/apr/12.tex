% !TEX root = ../notes.tex

\documentclass[../notes.tex]{subfiles}

\begin{document}

\section{April 12}

For the next few lectures, we are going to lecture from Mumford.

\subsection{A Corollary of Flat Base-Change}
Last class we roughly proved \Cref{thm:flat-base-change}. Here is a nice corollary.
\begin{corollary}
	Fix a finite-type separated morphism $f\colon X\to Y$ of Noetherian schemes. Further, suppose that $Y$ is affine, and $y$ is a point in $Y$, and we let $X_y\coloneqq X\times_Yk(y)$ denote the fiber of $f$ over $y\in Y$. Letting $\underline{k(y)}$ be the constant sheaf on $\overline{\{y\}}$, we have
	\[H^i(X_y,\mc F_y)\cong H^i(X,\mc F\otimes_k\underline{k(y)}).\]
\end{corollary}
\begin{proof}
	Set $Y'\coloneqq\overline{\{y\}}$ be the reduced closed subscheme in $Y$. Let $i\colon Y'\to Y$ and $j\colon X'\to X$ be the corresponding embeddings. Then we see that both sides of our equation depend only on $\mc F_y=i^*\mc F$ and not on all of $\mc F$; namely, we can compute that
	\[H^i(X,\mc F\otimes\underline{k(y)})=H^i(X,i_*(i^*\mc F\otimes\underline{k(y)}))=H^i(X',i^*\mc F\otimes\underline{k(y)})\]
	because of the support of $\underline{k(y)}$ is zero outside $Y'$; namely, we are using the fact that the closed embedding $X'\to X$ will not change our cohomology. Thus, we may replace $X$ with $X'$ and $\mc F$ with $i^*(\mc F\otimes k(y))$ and $Y$ with $Y'$. In particular, we are now assuming that $Y$ is integral with generic point given by $y$. Observe that $Y$ continues to be affine because $Y'$ is the closed subscheme of the affine scheme $Y$.

	The point is that $u\colon\Spec k(y)\to Y$ is a flat map (namely, we are embedding an integral domain into its fraction field). Now, let $Y=\Spec A$, and we compute
	\[u^*R^if_*\mc F=\widetilde{H^i(X,\mc F)\otimes_Ak(y)}\qquad\text{and}\qquad R^if'_*(u')^*\mc F=\widetilde{H^i(X_y,\mc F_y)}=\widetilde{H^i(X_y,\mc F\otimes_Ak(y))}.\]
	Now, by our reduction replacing $\mc F$ with $i^*(\mc F\otimes k(y))$, we see that $\mc F$ is in fact a sheaf of $k(y)$-modules, so the tensor product on both sides does not really do anything. So we finish by \Cref{thm:flat-base-change}.
\end{proof}
\begin{example}
	If $y$ is a closed point, then we can just directly compute
	\[H^i(X_y,\mc F_y)=H^i(X_y,i^*\mc F)=H^i(X,i_*i^*\mc F)=H^i(X,\mc F\otimes\underline{k(y)}),\]
	where we have used the fact that closed embeddings do not change cohomology.
\end{example}

\subsection{Associated Primes}
We are going to define associated points for sheaves of modules. The rough idea is that the associated points are ``generic points'' for the sheaf. For example, the associated points of the structure sheaf of a closed subscheme are going to be the generic points of the irreducible components of the closed subscheme. Our goal, for now, is to prove the following result.
\begin{restatable}{proposition}{associatedforflatness} \label{prop:flatness-by-associated}
	Fix a morphism $f\colon X\to Y$ of Noetherian schemes. If the coherent sheaf $\mc F$ is flat over $Y$, then associated points of $\mc F$ get mapped by $f$ to associated points of $\OO_Y$.
\end{restatable}
\noindent Anyway, here is our definition of associated point.
\begin{definition}[associated point]
	Fix a coherent sheaf $\mc F$ on a Noetherian scheme $X$. Then the set of \textit{associated points} of $\mc F$ is
	\[A(\mc F)\coloneqq\{x\in X:\text{there is }s\in\mc F_x\text{ annihilated by }I\subseteq\OO_{X,x}\text{ where }\rad I=\mf m_x\}.\]
\end{definition}
\begin{remark}
	Quickly, $x\in A(\mc F)$ is equivalent to having an open neighborhood $U\subseteq X$ of $x$ and section $s\in\OO_X(U)$ such that $\op{supp}s=\overline{\{x\}}$. Indeed, having $s\in\mc F_x$ annihilated with annihilator which is not $\mf m_x$-primary---i.e., $\op{Ann}s$ is not contained in $\mf m_x$---means that $s$ is supported on a closed subscheme of $\Spec\OO_{X,x}$, which means that the support of $s$ is smaller than $\overline{\{x\}}$. On the other hand, if $s$ is not supported at $x$, then it vanishes on the stalk.\todo{What is going on here?}
\end{remark}
\begin{remark}
	The generic points of $\op{Supp}\mc F$ will thus be contained in $A(\mc F)$, but the containment might be strict. However, if $\mc F=\OO_Z$ for some reduced closed subscheme $Z\subseteq X$, then $A(\OO_Z)$ will indeed be the generic points of $Z$.
\end{remark}
\begin{remark}
	We also note that $A(\mc F)$ is always finite.
\end{remark}
Associated points also relate to depth.
\begin{lemma} \label{lem:associated-has-no-depth}
	Fix a Noetherian scheme $Y$. Then $y\in A(\OO_Y)$ if and only if $\op{depth}\OO_{Y,y}=0$.
\end{lemma}
\begin{proof}
	Saying that $\op{depth}\OO_{Y,y}=0$ is really requiring that all non-units in $\OO_{Y,y}$ to fail to be able to be in a regular sequence, which means that all non-units are zero-divisors. In one direction, if $y\in A(\OO_Y)$, then we may find $s\in\OO_{Y,y}$ annihilated by $I\subseteq\OO_{X,x}$ with $\rad I=\mf m_x$. In other words, for any $f\in\mf m_y$, we have $f^n\in I$ for some $n$ large enough has $f^ns=0$, meaning that $f$ is a zero-divisor. We omit the proof of the other direction, for now.
\end{proof}
In commutative algebra, we also have an analogous notion of associated primes.
\begin{definition}[associated prime]
	Fix an $R$-module $M$. Then a prime $\mf p$ is \textit{associated to $M$} if and only if there exists $m\in M$ such that $\mf p=\op{Ann}m$. The set of all associated primes will be denoted $A(M)$.
\end{definition}
\begin{remark}
	If $R$ is Noetherian and $M$ is finitely generated over $R$, then we can show that $A(M)$ is finite. Additionally, $A(M)$ still contains the generic points of $\op{Supp}M$.
\end{remark}
We are going to want the following results.
\begin{lemma}
	Fix a Noetherian ring $R$. Then a finitely generated module $M$ is nonzero if and only if $A(M)\ne\emp$.
\end{lemma}
\begin{proof}
	Of course, if $M$ is zero, then $A(M)=\emp$. However, if $M$ is nonzero, then $A(M)$ will contain the generic points of $\op{Supp}M$. Proving this previous claim is a little technical. Roughly speaking, the idea is to build a filtration of $M$ as
	\[(0)=M_0\subseteq M_1\subseteq\cdots\subseteq M_n=M,\]
	where $M_i/M_{i+1}$ is $R/\mf p_i$ for some prime ideal $\mf p_i$. (Roughly speaking, one can do this by an induction.) This filtration tells us what the associated primes should be.
\end{proof}

\end{document}