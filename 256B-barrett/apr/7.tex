% !TEX root = ../notes.tex

\documentclass[../notes.tex]{subfiles}

\begin{document}

\section{April 7}

Today is Friday. We're finally done with Serre duality. It's going to be a good day.

\subsection{Higher Direct Images}
Fix a continuous map $f\colon X\to Y$ of topological spaces. This then produces a functor $f_*\colon\mathrm{Sh}_X\to\mathrm{Sh}_Y$, which is our direct-image functor. This functor is left-exact because taking sections is left-exact, and we can check left-exactness on the level of sections. To be explicit, given a left-exact sequence
\[0\to\mc F'\to\mc F\to\mc F'',\]
we get the left-exact sequence
\[0\to\mc F'(U)\to\mc F(U)\to\mc F''(U)\]
for any open subset $U\subseteq X$. Thus, we have the left-exact sequence
\[0\to f_*\mc F'(V)\to f_*\mc F(V)\to f_*\mc F''(V)\]
for any open subset $V\subseteq Y$, so we conclude that $f_*\mc F'=\ker(f_*\mc F)\to f_*\mc F'')$ by definition of the kernel. Anyway, the point is that we can right-derive.
\begin{definition}
	Fix a continuous map $f\colon X\to Y$ of topological spaces. Then we define $R^if_*$ to be the $i$th right-derived functor of the left-exact functor $f_*\colon\mathrm{Sh}_X\to\mathrm{Sh}_Y$.
\end{definition}
Now, these sheaves are nice because they turn out to be essentially sheafifications of cohomology.
\begin{proposition} \label{prop:higher-direct-image-by-cohom}
	Fix a continuous map $f\colon X\to Y$ of topological spaces and sheaf $\mc F$ on $X$. Then for each $i\ge0$, $R^if_*\mc F$ is the sheafification of the presheaf (on $Y$) given by
	\[V\mapsto H^i\left(f^{-1}V,\mc F\right).\]
\end{proposition}
\begin{proof}
	If $i=0$, we are looking at $f_*\mc F(V)=\Gamma\left(f^{-1}V,\mc F\right)=H^0\left(f^{-1}V,\mc F\right)$. So it remains to show that each functor is universal. The right-derived functor is a universal $\delta$-functor because it is derived.

	Next, define $\mc H^i$ to be the functor defined in the proposition. One can see that $R^i$ is a right-derived functor and hence universal. Before sheafification, we send short exact sequences of presheaves to long exact sequences of presheaves, but sheafification is exact: sheafification is a left adjoint of the forgetful functor, so it preserves finite colimits. But also sheafification preserves finite products (which are finite coproducts), so it remains to check that it preserves kernels. This can be checked directly.
	
	Anyway, the point is that $\mc H^i$ is a $\delta$-functor by taking our long exact sequence of presheaves and taking sheafification. Next, to be effaceable, we note that our category has enough injectives, so we may embed $\mc F$ into an injective sheaf $\mc I$, but then the map
	\[H^i\left(f^{-1}V,\mc F\right)\to H^i\left(f^{-1}V,\mc I\right)\]
	is the zero map because $\mc I$ is injective (and will remain flasque and hence acyclic on any open subset $f^{-1}V\subseteq X$). Taking sheafification will retain this as the zero map (indeed, $\mc H^i(\mc I)=0$ still), so we get that $\mc H^i$ is effaceable.
\end{proof}
\begin{corollary} \label{cor:restrict-higher-direct-image}
	Fix a continuous map $f\colon X\to Y$ of topological spaces and sheaf $\mc F$ on $X$. If $V\subseteq Y$ is open, then $\left(R^if_*\mc F\right)|_V=R^if_*'\left(\mc F|_{f^{-1}V}\right)$, where $f'\colon f^{-1}V\to V$ is the restricted map.
\end{corollary}
\begin{proof}
	Sheafification commutes with restriction (just look directly at the construction in sheafification as taking compatible stalks), so we see that $\left(R^if_*\mc F\right)|_V$ and $R^if_*'\left(\mc F|_{f^{-1}V}\right)$ are both sheafifications of the presheaf
	\[V'\mapsto H^i\left(f^{-1}V',\mc F\right)\]
	by \Cref{prop:higher-direct-image-by-cohom}.
\end{proof}
\begin{corollary} \label{cor:flasque-is-acyclic-direct-image}
	Fix a continuous map $f\colon X\to Y$ of topological spaces and sheaf $\mc F$ on $X$. If $\mc F$ is flasque, then $R^if_*\mc F=0$ for any $i>0$.
\end{corollary}
\begin{proof}
	By \Cref{prop:higher-direct-image-by-cohom}, we see that $R^if_*\mc F$ is the sheafification of the presheaf
	\[V\mapsto H^i\left(f^{-1}V,\mc F\right).\]
	However, the right-hand side here always vanishes because $\mc F$ is flasque (and remains so upon restriction to $f^{-1}V$), so we are looking at the zero presheaf, which yields the zero sheaf upon sheafification.
\end{proof}
\begin{corollary}
	Fix a continuous map $f\colon X\to Y$ of ringed spaces and sheaf $\mc F$ on $X$. Then $R^if_*$ is the $i$th right-derived of $f_*\colon\mathrm{Mod}_{\OO_X}\to\mathrm{Mod}_{\OO_Y}$.
\end{corollary}
\begin{proof}
	We certainly have $R^0f_*=f_*$ as needed, and both functors can use the same acyclic resolutions to compute cohomology, namely flasque resolutions by \Cref{cor:flasque-is-acyclic-direct-image}. This is what we wanted.
\end{proof}
\begin{proposition} \label{prop:get-higher-direct-image-qcoh}
	Fix a Noetherian scheme $X$, and let $f\colon X\to Y$ be a map to an affine scheme $Y$. For a quasicoherent sheaf $\mc F$ on $X$, we have $R^if_*\mc F=\widetilde{H^i(X,\mc F)}$.
\end{proposition}
\begin{proof}
	In degree $0$, we know that $f_*$ sends quasicoherent sheaves to quasicoherent sheaves because $X$ is Noetherian. Thus, we use that we are quasicoherent to observe that
	\[f_*\mc F=\widetilde{\Gamma(Y,f_*\mc F)}=\widetilde{\Gamma(X,\mc F)}=\widetilde{H^0(X,\mc F)},\]
	so we have our isomorphism in degree-$0$. Thus, it remains to show that we have universal $\delta$-functors. On one hand, $R^if_*$ is a right-derived functor and hence a universal $\delta$-functor.
	
	On the other hand, $\widetilde{H^i(X,-)}$ does send short exact sequences to first the long exact sequence for $H^i(X,-)$ and then to a long exact sequence for $\widetilde{H^i(X,-)}$ by taking the exact functor $\widetilde\cdot$ everywhere. Lastly, to be effaceable, we embed $\mc F$ into an injective quasicoherent sheaf $\mc I$, and then the morphism
	\[\widetilde{H^i(X,\mc F)}\to\widetilde{H^i(X,\mc I)}\]
	is the zero map because the map $H^i(X,\mc F)\to H^i(X,\mc I)$ is the zero map because $H^i(X,\mc I)=0$ because $\mc I$ is injective.
\end{proof}
\begin{corollary}
	Fix a Noetherian scheme $X$, and let $f\colon X\to Y$ be a map to an affine scheme $Y$. For a quasicoherent sheaf $\mc F$ on $X$, $R^if_*\mc F$ remains quasicoherent.
\end{corollary}
\begin{proof}
	For any affine open subscheme $V$, we use \Cref{cor:restrict-higher-direct-image} to see that
	\[(R^if_*\mc F)|_V=R^if_*\left(\mc F|_{f^{-1}V}\right).\]
	Now, $\mc F|_{f^{-1}V}$ is quasicoherent, and now $f'\colon f^{-1}V\to V$ is a map to an affine scheme, so we use \Cref{prop:get-higher-direct-image-qcoh} to write $R^if_*\left(\mc F|_{f^{-1}V}\right)$ as $\widetilde M$ for some module $M$ on $\OO_V(V)$.
\end{proof}

\subsection{Higher Direct Images for Projective Morphisms}
Recall the following definition.
\begin{definition}[projective]
	A scheme morphism $f\colon X\to Y$ is \textit{projective} if and only if it factors through a closed immersion into $\PP^n_Y=\PP^n_\ZZ\times_\ZZ Y$ followed by the projection to $Y$.
\end{definition}
And here is our result.
\begin{theorem}
	Fix a projective morphism $f\colon X\to Y$ of Noetherian schemes; let $\OO(1)$ be a very ample line bundle on $X$, and let $\mc F$ be a coherent sheaf on $X$.
	\begin{listalph}
		\item The map $f^*f_*\mc F(n)\to\mc F(n)$ is surjective for $n$ large enough.
		\item $R^if_*\mc F$ are coherent sheaves.
		\item For any $i>0$, we have $R^if_*\mc F(n)=0$ for $n$ large enough.
	\end{listalph}
\end{theorem}
\begin{proof}
	Here we go.
	\begin{itemize}
		\item To show (b) and (c), we note that \Cref{cor:restrict-higher-direct-image} lets us reduce to an affine open subscheme $\Spec A$ of $Y$ and then pull it back to an affine open subscheme in the pre-image in $X$. Then with $Y=\Spec A$, we see
		\[R^if_*\mc F=\widetilde{H^i(X,\mc F)}.\]
		Now, by \Cref{thm:very-ample-kills-cohom}, we see that $H^i(X,\mc F)$ is in fact finitely generated over $A$ and vanishes in high enough twisting. This gives (b) automatically, and (c) follows by twisting far enough to kill cohomology on each affine open subscheme of some fixed finite affine open cover of the quasicompact scheme $X$.
		\item To show (a), we again may assume that $Y$ is affine because surjectivity may be checked at stalks in $X$ and is thus local on $X$ and thus on $Y$. So set $Y=\Spec B$, and on some affine open subscheme $U=\Spec A$ of $X$, we see
		\[(f^*f_*\mc F(n))(U)=\Gamma(U,\mc F(n))\otimes_BA\]
		by computing global sections. The point is that for $n$ large enough we are looking at the map
		\[\Gamma(U,\mc F(n))\otimes_BA\to\Gamma(U,\mc F(n)),\]
		which is indeed surjective for $n$ large enough, which I guess can be checked directly.\todo{What?}
		\qedhere
	\end{itemize}
\end{proof}
Continuing, we note that we can even compute our higher direct images via some version of \v Cech cohomology.
\begin{proposition}
	Fix a map $f\colon X\to Y$ of separated Noetherian schemes. Fix a quasicoherent sheaf $\mc F$ on $X$, and equip $X$ with affine open cover $\mf U\coloneqq\{U_\alpha\}_{\alpha\in\lambda}$ so that we have the \v{C}ech complex $C^\bullet(\mf U,\mc F)$. Let $C^\bullet(\mf U,\mc F)\sh$ be the natural sheafification. Then
	\[R^pf_*\mc F=H^p\left(f_*C^\bullet(\mf U,\mc F)\sh\right).\]
\end{proposition}
\begin{proof}
	To be explicit, on an open subset $U\subseteq X$, one has
	\[C^p(\mf U,\mc F)\sh=\prod_{i_0<\cdots<i_p}j_*\left(\mc F|_{U_{i_0\cdots i_p}}\right),\]
	where $j$ is the relevant extension by zero. (Note that $U_{i_0\cdots i_p}$ is affine because $X$ is separated.) The point is that these remain to be quasicoherent sheaves because the restriction and then extension by zero (from a Noetherian scheme) of a quasicoherent sheaf remain quasicoherent. For example, if $Y$ is affine, then we use the fact that
	\[H^p(X,\mc F)=H^pC^\bullet(\mf U,\mc F)\]
	from our discussion of \v Cech cohomology, so it follows that
	\[R^pf_*\mc F=\widetilde{H^p(X,\mc F)}=H^p\widetilde{C^\bullet(\mf U,\mc F)},\]
	which gives the desired result after comparing with the quasicoherent complex $f_*C^\bullet\left(\mf U,\mc F\right)\sh$. (Explicitly, one should compute that everything matches on global sections and then use the fact that our sheaves are quasicoherent.) It remains to reduce to the case where $Y$ is affine, but this is not hard: for affine open subscheme $V\subseteq Y$, we see that $f^{-1}V\cap U_{i_0\cdots i_p}$ is also affine. (Namely, a separated morphism out of an affine scheme is affine, so the map out of $U_{i_0\cdots i_p}\to Y$ is affine, so taking the pre-image of $V$ will give an affine open subscheme.)
\end{proof}

\end{document}