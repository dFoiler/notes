% !TEX root = ../notes.tex

\documentclass[../notes.tex]{subfiles}

\begin{document}

\section{April 28}

It's the last lecture of Math 256B. We're finishing our discussion of smoothness.

\subsection{Smoothness Factors through \'Etale}
Let's get some utility out of our Jacobi criterion.
\begin{proposition} \label{prop:smooth-by-fibers}
	Fix a morphism $f\colon X\to Y$ of $S$-schemes. Further, fix $x\in X$ and $y\coloneqq f(x)$ where $X$ is smooth over $S$ at $x$, and $Y$ is smooth over $S$ at $y$. Then the following are equivalent.
	\begin{listalph}
		\item $f$ is smooth at $x$.
		\item The map $(f^*\Omega_{Y/S})_x\to(\Omega_{X/S})_x$ is an isomorphism onto a direct factor.
		\item The map $f^*\Omega_{Y/S}\otimes k(x)\to\Omega_{X/S}\otimes k(x)$ is injective (on the fiber).
	\end{listalph}
\end{proposition}
\begin{proof}
	Here we go.
	\begin{itemize}
		\item We show (a) implies (b). This follows from noting that
		\[0\to f^*\Omega_{Y/S}\to\Omega_{X/S}\to\Omega_{X/Y}\to0\]
		will be split-exact at $x$ because $f$ is smooth at $x$, as we showed in the argument on Monday.
		\item Note (b) implies (c) by Nakayama's lemma.
		\item We show (c) implies (a). To begin, we suppose that $Y=\AA^s_S$. As such, $f$ is really determined by the global sections $\overline f_1,\ldots,\overline f_s\in\Gamma(X,\OO_X)$. (Note we may shrink $X$ and $Y$ as much as we want throughout this argument because smoothness is defined stalk-locally.)
		
		Now, $\Omega_{\AA^s_S/S}$ is freely generated by $dT_1,\ldots,dT_s$ where $Y=\AA^n_s=\Spec_S\OO_S[T_1,\ldots,T_s]$. By hypothesis, we see that
		\[df(dT_1)(x),\ldots,df(dT_s)(x)\]
		are linearly independent due to our injectivity of a map of vector spaces. Notably, $df(dT_i)=df_i$ for each $i$ by construction of the $f_i$.

		Continuing, because $X$ is smooth, we may embed $X\into\AA^m_S$ cut out by the ideal sheaf $\mc I\subseteq\OO_{\AA^m_S}$ locally generated at $x$ by the global sections $h_{r+1},\ldots,h_m$ (we may assume global after shrinking $X$) such that
		\[dh_{r+1}(x),\ldots,dh_m(x)\]
		are linearly independent vectors in $\Omega_{\AA^m_S/S}$. At this point, we consider the composite map
		\[X\to X\times_SY\to\AA^m_S\times_S\AA^s_S=\AA^{m+s}_S=\AA^m_Y.\]
		We must now begin a computation. Set $S=\Spec R$ and $X=\Spec A$ so that $Y=\Spec R[T_1,\ldots,T_s]$. We would like to compute the ideal sheaf of the above composite of closed embeddings. Well, begin by lifting the $\overline f_\bullet$ to legitimate polynomials $f_\bullet$ in $\OO_{\AA^m_S}$. Thus, the above closed embedding will be cut out by the global sections $T_\bullet-f_\bullet$ plus the ones $h_{r+1},\ldots,h_m$ from earlier.

		Notably, the map $X\to\AA^m_Y$ at $x$ is being cut out by the correct number of sections as we just showed, so it remains to check that our differentials
		\[dh_{r+1}(x,y),\ldots,dh_m(x,y),(dT_1-df_1)(x,y),\ldots,(dT_s-df_s)(x,y)\]
		are linearly independent in $\Omega_{\AA^m_Y/Y}\otimes k(x,y)$. To begin, note $T_i\in\OO_Y$, so $dT_i=0$ immediately.

		We now check linear independence in steps. To begin, consider the projection $p\colon Y\to S$, and label the map $X\to\AA^m_S$ by $\iota\colon X\to\AA^m_S$. Now, we see that our differentials $df_1(x),\ldots,df_s(x)$ in $\Omega_{\AA^m_S/S}\otimes k(x)$ will map under $d\iota$ to the differentials $d\overline f_\bullet(x)$. However, we know that we have the split-exact sequence
		\[0\to(\mc I/\mc I^2)_x\to(\Omega_{\AA^m_S/S}|_X)_x\to(\Omega_{X/S})_x\to0\]
		because $X$ is smooth over $S$. Notably, the elements $\overline h_{r+1},\ldots,\overline h_m$ generate the free $\OO_{X,x}$-module $(\mc I/\mc I^2)_x$ more or less by construction, so the splitting assures us that $d h_{r+1}(x),\ldots,d h_m(x)$ generate the image of $\mc I/\mc I^2\otimes k(x)$ in $\Omega_{\AA^m_S/S}|_X\otimes k(x)$.

		To conclude, we see that $d\overline f_1(x),\ldots,d\overline f_s(x)$ are linearly independent sections on the other side $\Omega_{X/S}\otimes k(x)$ of our split-exact sequence, so we are done by taking the union of our linearly independent sections.
		
		\item For (c) implies (a), it remains to deal with the case where $Y$ is a general scheme over $S$. Smoothness grants sections $g_1,\ldots,g_s$ of $\OO_Y$ such that the differentials $dg_1,\ldots,dg_s$ generate $(\Omega_{Y/S})_y$. Then the hypothesis (c) tells us that
		\[df(dg_1)(x),\ldots,df(dg_s)(x)\]
		are linearly independent differentials in $\Omega_{X/S}\otimes k(x)$. Continuing, we may append global sections $h_{s+1},\ldots,h_r\in\OO_X$ (shrink $X$ as necessary) such that the elements
		\[df(dg_1)(x),\ldots,df(dg_s)(x),dh_{s+1}(x),\ldots,dh_r(x)\]
		form a basis of $\Omega_{X/S}\otimes k(x)$. Now, the $g_\bullet$ determine a map to affine space given by $Y\to\AA^s_S$, and the $h_\bullet$ determine a map to affine space $X\to\AA^{r-s}_S$.

		As such, we build the following diagram.
		% https://q.uiver.app/?q=WzAsNCxbMCwwLCJYIl0sWzEsMCwiWVxcdGltZXNfU1xcQUFee3Itc31fUyJdLFsyLDAsIlkiXSxbMSwxLCJcXEFBXnJfUyJdLFsxLDIsIiIsMCx7InN0eWxlIjp7ImhlYWQiOnsibmFtZSI6ImVwaSJ9fX1dLFswLDEsIihmLGgpIl0sWzAsMywiKGdmLGgpIiwyXSxbMSwzXV0=&macro_url=https%3A%2F%2Fraw.githubusercontent.com%2FdFoiler%2Fnotes%2Fmaster%2Fnir.tex
		\[\begin{tikzcd}
			X & {Y\times_S\AA^{r-s}_S} & Y \\
			& {\AA^r_S}
			\arrow[two heads, from=1-2, to=1-3]
			\arrow["{(f,h)}", from=1-1, to=1-2]
			\arrow["{(gf,h)}"', from=1-1, to=2-2]
			\arrow[from=1-2, to=2-2]
		\end{tikzcd}\]
		Notably, the map $X\to\AA^r_S$ satisfies the conditions of (c) mapping to affine space, so we conclude that it is smooth and in fact \'etale because the relative dimension is $0$. The same argument shows that the map $Y\times_S\AA^{r-s}_S\to\AA^r_S$ is \'etale at $x$, so one can show that $(f,h)$ is \'etale at $x$ by some kind of cancellation theorem argument (we want some pointwise version of \Cref{lem:etale-cancellation}, which can be shown), so the composite $X\to Y\times_S\AA^{r-s}_S\to Y$ is smooth, as needed.
		\qedhere
	\end{itemize}
\end{proof}
\begin{lemma} \label{lem:etale-cancellation}
	Fix \'etale $S$-schemes $f\colon X\to S$ and $g\colon Y\to S$. Given an $S$-morphism $h\colon X\to Y$, then $h$ is \'etale.
\end{lemma}
\begin{proof}
	We show formal lifting. Fix $S$-schemes $Z$ and a closed subscheme $Z'$ defined by an ideal sheaf $\mc I$ such that $\mc I^2=0$. This gives the following rectangle.
	% https://q.uiver.app/?q=WzAsNixbMSwwLCJYIl0sWzEsMSwiWSJdLFsxLDIsIlMiXSxbMCwyLCJaIl0sWzAsMSwiWiJdLFswLDAsIlonIl0sWzMsMl0sWzQsMywiIiwwLHsibGV2ZWwiOjIsInN0eWxlIjp7ImhlYWQiOnsibmFtZSI6Im5vbmUifX19XSxbNSw0XSxbNSwwXSxbMCwxXSxbMSwyXV0=&macro_url=https%3A%2F%2Fraw.githubusercontent.com%2FdFoiler%2Fnotes%2Fmaster%2Fnir.tex
	\[\begin{tikzcd}
		{Z'} & X \\
		Z & Y \\
		Z & S
		\arrow[from=3-1, to=3-2]
		\arrow[Rightarrow, no head, from=2-1, to=3-1]
		\arrow[from=1-1, to=2-1]
		\arrow[from=1-1, to=1-2]
		\arrow[from=1-2, to=2-2]
		\arrow[from=2-2, to=3-2]
	\end{tikzcd}\]
	Now, the map $Z\to S$ has a unique lift to $Z\to Y$ making the diagram commute and a unique lift to $Z\to X$ because $X$ and $Y$ are \'etale over $S$. Now, uniqueness makes the map $X\stackrel\alpha\to Y$ equal to the map $Z\stackrel\beta\to X\to Y$, so our lifting is unique, so we are done.
\end{proof}
\begin{remark}
	The same statement holds for being unramified or smooth, though the above proof doesn't work.
\end{remark}
We are now ready to prove the advertisement for smoothness given at the beginning.
\begin{corollary} \label{cor:etale-via-fibers}
	Let $f\colon X\to Y$ be a morphism of smooth schemes. The following are equivalent.
	\begin{listalph}
		\item $f$ is \'etale at $x$.
		\item The map $f^*\Omega_{Y/S}\otimes k(x)\to\Omega_{X/S}\otimes k(x)$ is a bijection.
	\end{listalph}
\end{corollary}
\begin{proof}
	Plug in relative dimension $0$ into \Cref{prop:smooth-by-fibers}.
\end{proof}
\begin{proposition}
	Fix a morphism $f\colon X\to S$ of schemes, and fix some $x\in X$. The following are equivalent.
	\begin{listalph}
		\item The scheme morphism $f\colon X\to S$ is smooth at $x$.
		\item There is an open neighborhood $U$ of $x$ in $X$ and a commutative triangle
		% https://q.uiver.app/?q=WzAsMyxbMCwwLCJVIl0sWzEsMCwiXFxBQV5uX1MiXSxbMSwxLCJTIl0sWzAsMiwiZnxfVSIsMl0sWzEsMiwiIiwwLHsic3R5bGUiOnsiaGVhZCI6eyJuYW1lIjoiZXBpIn19fV0sWzAsMV1d&macro_url=https%3A%2F%2Fraw.githubusercontent.com%2FdFoiler%2Fnotes%2Fmaster%2Fnir.tex
		\[\begin{tikzcd}
			U & {\AA^n_S} \\
			& S
			\arrow["{f|_U}"', from=1-1, to=2-2]
			\arrow[two heads, from=1-2, to=2-2]
			\arrow[from=1-1, to=1-2]
		\end{tikzcd}\]
		where the map $U\to\AA^n_S$ is \'etale.
	\end{listalph}
\end{proposition}
\begin{proof}
	That (b) implies (a) holds because the composition of smooth morphisms remains smooth. So we spend our time showing (a) implies (b).

	As usual, we may shrink $X$ around $x$ as much as we please throughout this argument. Let $g_1,\ldots,g_n$ be global sections such that $dg_1(x),\ldots,dg_n(x)$ generate $\Omega_{X/S}\otimes k(x)$, extracted via smoothness. These global sections induce a map $X\to\AA^n_S$ which is a bijection at the level of differentials (note that we do have relative dimension $n$), which is thus \'etale by \Cref{cor:etale-via-fibers}. This then factors through the map $f$ by the construction of the $g_\bullet$ as generating the needed ideal sheaf, so we are done.
\end{proof}

\end{document}