% !TEX root = ../notes.tex

\documentclass[../notes.tex]{subfiles}

\begin{document}

\section{April 14}

We continue discussing flatness. Throughout, $X$ and $Y$ are Noetherian schemes, and $\mc F$ is a coherent sheaf on $X$.

\subsection{Flatness and Associated Primes}
We would like to show the opposite direction of \Cref{lem:associated-has-no-depth}. In the process, we are building some notion of associated primes. We have the following proposition.
\begin{proposition}
	Let $M$ be a finitely generated module $M$ over a Noetherian ring $R$. Then
	\[\bigcup_{\mf q\in A(M)}\mf q\]
	is the set of zero-divisors on $M$.
\end{proposition}
The key point of the above proposition is that $A(M)$ is finite, so we are taking a finite union!
\begin{proof}
	Of course, if $x\in\mf q$ for $\mf q\in A(M)$, then we can find $m_\mf q\in M$ with annihilator given by $\mf q$, so $x\cdot m_\mf q=0$, so $x$ is a zero-divisor on $M$.

	Conversely, if $x\in R$ is a zero-divisor on $M$, define
	\[N\coloneqq\{m\in M:xm=0\}.\]
	But now $N$ is nonzero by construction of $x$, so it has an associated prime $\mf p$. Thus, we can embed
	\[R/\mf p\into N\subseteq M,\]
	so $\mf p$ is an associated prime of $M$. Lastly, we note $x\in\mf p$ because $\mf p$ is the annihilator of some $m\in N$, but $x\in\op{Ann}m$ because $m\in N$, so $x\in\mf p$ follows.
\end{proof}
And here is our opposite direction.
\begin{proof}[Proof of converse of \Cref{lem:associated-has-no-depth}]
	Suppose that $\mf m_y$ consists only of zero-divisors. Then we see
	\[\mf m_y=\bigcup_{\mf q\in\op{Ass}\OO_{Y,y}}\mf q,\]
	so $\mf m_y$ is a finite union of primes. Thus, by prime avoidance, we see that $\mf m_y$ must actually be contained in one of these primes, so because $\mf m_y$ is a maximal ideal, we conclude that $\mf m_y=\mf q$ for some $\mf q\in\op{Ass}\OO_{Y,y}$. Thus, $\mf m_y$ is associated, so $y\in A(\OO_Y)$.
\end{proof}
We are now ready to prove \Cref{prop:flatness-by-associated}.
\associatedforflatness*
\begin{proof}[Proof of \Cref{prop:flatness-by-associated}]
	Fix $x\in A(\OO_X)$ and $y\coloneqq f(x)$. Suppose for contradiction that $f(x)\notin A(\OO_Y)$. By \Cref{lem:associated-has-no-depth}, we see that there is a non-zero-divisor $a\in\mf m_y$. Thus, the embedding $a\colon\OO_{Y,y}\to\OO_{Y,y}$ remains injective after tensoring with $\mc F_x$, so $a\colon\mc F_x\to\mc F_x$ is injective. Inductively, we note that any $n>0$ has
	\[a^n\colon\mc F_x\to\mc F_x\]
	also has trivial kernel and is therefore injective. We are now essentially done. Indeed, we show there is no $s\in\mc F_x$ killed by an ideal primary to $\mf m_x$, for we can see that any $s$ is annihilated by $I\subseteq\OO_{X,x}$, but this ideal ought to be contained in $\mf m_x$ and hence maps to $\mf m_y$ but will then contain a pre-image of a power of $a$. Thus, $\rad I\ne\mf m_x$ because no power of $a\in\mf m_x$ can land in $I$.
\end{proof}
\begin{remark}
	Fix a Noetherian affine scheme $X=\Spec A$. Then we claim that $A(\OO_X)=A(A)$. Of course $A(M)\subseteq A(\widetilde M)$ because having a prime actually annihilate an element $a\in M$ yields our section $a_\mf p\in A_\mf p$ killed by exactly the maximal ideal $\mf pA_\mf p\subseteq A_\mf p$. Conversely, if $x\in A(\OO_X)$, then \Cref{lem:associated-has-no-depth}, we note $\mf m_xA_x$ consists of zero-divisors on $A$, so $\mf m_x$ is in $\op{Ass}_{\OO_{X,x}}\OO_{X,x}$. This implies that $x\in\op{Ass}_AA$ as soon as we note that
	\[\op{Ass}_AM\cap\Spec A\left[S^{-1}\right]=\op{Ass}_AM\left[S^{-1}\right].\]
\end{remark}

\subsection{Flatness Facts}
As an aside, there is a strengthening of \Cref{prop:flatness-by-associated}.
\begin{proposition}
	Fix a morphism $f\colon X\to Y$ of Noetherian schemes. If the coherent sheaf $\mc F$ is flat over $Y$, then $x\in A(\mc F)$ if and only if $f(x)\in A(\OO_Y)$ and $x\in A(\mc F\otimes k(f(x)))$.
\end{proposition}
We will not prove this.
\begin{remark}
	In fact, under the same set-up, one can be a little more precise: one can compute
	\[\op{depth}_{\OO_{X,x}}\mc F_x=\op{depth}\OO_{Y,f(x)}+\op{depth}_{\OO_{X,x}/\mf m_y\OO_{X,x}}(\mc F_x\otimes k(y))\]
\end{remark}
Here is a last result.
\begin{theorem}[Miracle flatness]
	Fix a homomorphism $\varphi\colon R\to S$ of local Noetherian rings. If $R$ is regular, and $S$ is Cohen--Macaulay, and $\dim S=\dim R+\dim S/\mf m_RS$, then the map $\varphi$ is flat.
\end{theorem}
We will also not prove this. Here are two useful points about flatness.
\begin{proposition}
	Fix a scheme morphism $f\colon X\to Y$ is a morphism of schemes. Then the ``flat locus'' is open: if $f$ is flat at $x\in X$, then $f$ is flat on an open neighborhood of $x$. Further, if $f$ is flat and locally of finite presentation, then $f$ is open.
\end{proposition}
\begin{proof}
	Again, omitted, even though this is quite a nice result. The first result is a spreading our argument, and the second result uses going-down for flat ring maps.
\end{proof}
Here is another strengthening of \Cref{prop:flatness-by-associated}.
\begin{proposition}
	Fix a morphism $f\colon X\to Y$ of Noetherian schemes, where $Y$ is a regular curve. Further, fix a coherent sheaf $\mc F$ on $X$. If $x\in A(\mc F)$ implies $f(x)$ is the generic point of $Y$, then $f$ is flat.
\end{proposition}
\begin{proof}
	The point is to apply force based on what we know about curves. Suppose that $\mc F$ is not flat over the scheme $Y$ so that $\mc F$ is not flat over the point $x\in X$. Namely, $\mc F_x$ is not flat over $\OO_{Y,y}$. Because $Y$ is a regular scheme, $\OO_{Y,y}$ is a field or discrete valuation ring, but all modules over fields are free and hence flat, so $\OO_{Y,y}$ must be a discrete valuation ring. In particular, $y$ is not the generic point.

	We now try to find a point in $A(\mc F)$ which is in the fiber over $y$, which will complete the proof. By using the classification of modules over a principal ideal domain, we know that $\mc F_x$ is the sum of torsion modules and a free part, but the free part is flat over $\OO_{Y,y}$, so we see that $\mc F_x$ must have torsion. Namely, letting $\pi\in\OO_{Y,y}$ be a uniformizer, we have $s\in\mc F_x$ killed by a power of $\pi$; say $\pi^ns=0$. By adjusting $s$ to a power, we may assume that $n=1$.

	Continuing, we set $I\coloneqq\op{Ann}_{\OO_{X,x}}(s)$, and we let $\mf p$ be a minimal prime of $\OO_{X,x}$ containing $I$. This prime $\mf p$ corresponds to $x'$ such that $x\in\overline{\{x'\}}$. In total, we have composite
	\[\OO_{Y,y}\to\OO_{X,x}\to\OO_{X,x'}.\]
	Label this composite by $\varphi$. Notably, $\varphi^{-1}\mf m_{x'}$ contains $I$ and therefore will contain $\pi$, so $\varphi^{-1}\mf m_{x'}=\mf m_y$, which is what $f(x')=y$ means according to our scheme morphism.

	Now, we claim that $I\OO_{X,x'}$ is primary for $\mf m_{x'}$ and kills $s$. Then $x'\in A(\mc F)$ by definition associated points, so we will finish. To prove this claim, choose $\mf p$ minimal over $I$, but then
	\[\OO_{X,x'}/I\OO_{X,x'}=(\OO_{X,x})_\mf p/I(\OO_{X,x})_\mf p,\]
	meaning that $n$ consists of nilpotents, so $\sqrt{I\OO_{X,x}}=\mf m_{x'}$ follows.
\end{proof}

\end{document}