% !TEX root = ../notes.tex

\documentclass[../notes.tex]{subfiles}

\begin{document}

\section{April 24}

Today we talk about formal smoothness.

\subsection{Formally Unramified and Friends}
Here is our result for today.
\begin{theorem} \label{thm:formal-smooth}
	Fix a scheme morphism $f\colon X\to S$ locally of finite presentation. The following are equivalent.
	\begin{listalph}
		\item $f$ is unramified/smooth/\'etale.
		\item For all affine $S$-schemes $Y$ and closed subschemes $Y'\subseteq Y$ defined by an ideal sheaf $\mc I$ such that $\mc I^2=0$, the canonical map
		\[\op{Hom}_S(Y,X)\to\op{Hom}_S(Y',X)\]
		is injective/surjective/bijective.
	\end{listalph}
\end{theorem}
For example, for \'etale morphisms, we are drawing a square
% https://q.uiver.app/?q=WzAsNCxbMCwwLCJZJyJdLFsxLDAsIlgiXSxbMSwxLCJTIl0sWzAsMSwiWSJdLFswLDNdLFszLDJdLFsxLDJdLFswLDFdLFszLDEsIiIsMSx7InN0eWxlIjp7ImJvZHkiOnsibmFtZSI6ImRhc2hlZCJ9fX1dXQ==&macro_url=https%3A%2F%2Fraw.githubusercontent.com%2FdFoiler%2Fnotes%2Fmaster%2Fnir.tex
\[\begin{tikzcd}
	{Y'} & X \\
	Y & S
	\arrow[from=1-1, to=2-1]
	\arrow[from=2-1, to=2-2]
	\arrow[from=1-2, to=2-2]
	\arrow[from=1-1, to=1-2]
	\arrow[dashed, from=2-1, to=1-2]
\end{tikzcd}\]
and claiming that there exists a unique map $Y'\to X$ making the diagram commute. We will prove the statement for smoothness for time reasons.
\begin{proof}[]
	It turns out that (b) can have everything reduce to affine open subschemes. The concern here is mostly that we need the maps to glue reasonably, but we will not say more. Certainly the statement in (a) is defined at stalks and is thus local, so everything is local, so we may assume that $S=\Spec R$ and $X=\Spec B$ is a closed subscheme of an open subscheme $\Spec A$ of $\AA^n_S$, using the fact that $f$ is locally of finite presentation; namely, we may say $B\coloneqq A/I$ where $I$ is finitely generated.
	\begin{itemize}
		\item Now, in one direction, suppose that $f$ is smooth. Here, $\Omega_{\AA^n_S/S}$ is free of rank $n$. Because we are working locally, we may use the definition of smoothness to find $g_1,\ldots,g_n\in A$ such that $dg_1,\ldots,dg_n$ generate the $A$-module $\Omega_{A/R}=\Omega_{\AA^n_S/S}|_{\Spec A}$ and $g_{r+1},\ldots,g_n$ generate $I\subseteq A$. Notably, the $B$-module $I/I^2$ is generated by $\overline g_{r+1},\ldots,\overline g_n$. Now, we claim that
		\[0\to I/I^2\to\Omega_{A/R}\otimes_AB\to\Omega_{B/R}\to0\]
		is a split exact sequence. Indeed, this is just a computation: the generators $\overline g_{r+1},\ldots,\overline g_n$ go to the free generators $dg_{r+1},\ldots,dg_n$ in $\Omega_{A/R}\otimes_AB$. Thus, we get exactness on the left, and we can compute the kernel of the right map is exactly generated by these elements by examining the map $\Omega_{A/R}\otimes_AB\to\Omega_{B/R}$. So we get what we want. (Note that we have left-exactness of this sequence even without $f$ being smooth!)
	
		We now approach our proof directly. Fix $Y\coloneqq\Spec C$ where $C$ is an $A$-algebra, and let $Y'\coloneqq\Spec C/J$ with $J\subseteq C$ an ideal with $J^2=0$. Now, our affine diagram is as follows.
		% https://q.uiver.app/?q=WzAsNSxbMCwwLCJDL0oiXSxbMSwwLCJBL0kiXSxbMSwxLCJBIl0sWzEsMiwiUiJdLFswLDEsIkMiXSxbMyw0XSxbMywyXSxbMiwxXSxbMSwwXSxbNCwwXSxbMSw0LCIiLDEseyJzdHlsZSI6eyJib2R5Ijp7Im5hbWUiOiJkYXNoZWQifX19XV0=&macro_url=https%3A%2F%2Fraw.githubusercontent.com%2FdFoiler%2Fnotes%2Fmaster%2Fnir.tex
		\[\begin{tikzcd}
			{C/J} & {A/I} \\
			C & A \\
			& R
			\arrow[from=3-2, to=2-1]
			\arrow[from=3-2, to=2-2]
			\arrow[from=2-2, to=1-2]
			\arrow[from=1-2, to=1-1]
			\arrow[from=2-1, to=1-1]
			\arrow[dashed, from=1-2, to=2-1]
		\end{tikzcd}\]
		Explicitly, given some $\overline\varphi$, we want to fill in the dotted line $\varphi$. Now, $A$ is a free algebra over a localization $R_f$ of $R$; say $\Spec A\subseteq\Spec R[x_1,\ldots,x_n]$. Then we can track the image of $x_\bullet$ in $C/J$, choose some arbitrary lift to $C$, and then we have constructed a map $R[x_1,\ldots,x_n]\to C$ making the above diagram commute. In fact, we are defining a map $\psi\colon A\to C$, which is safe because $\Spec C/J\cong\Spec C$; notably, being homeomorphic allows us to restrict appropriately to $\Spec A\subseteq\AA^n_S$ because we started with a map $\Spec C/J\to\Spec A$.
	
		Now, $\psi(I)\subseteq J$ because $A\to C\to C/J$ factors as $A\to A/I\to C/J$, which vanishes on $I$; notably, this provides a $B$-module map $\psi'\colon I/I^2\to J$. We now want to use the universal property of $\Omega_{A/R}$ as universal among derivations $d\colon A\to\Omega_{A/R}$. Using the splitting of the injection $I/I^2\to\Omega_{A/R}\otimes_AB$, we produce a map $\psi''$ fitting in the following diagram.
		% https://q.uiver.app/?q=WzAsOSxbMCwyLCIwIl0sWzEsMiwiSS9JXjIiXSxbMiwyLCJcXE9tZWdhX3tBL1J9XFxvdGltZXNfQUIiXSxbMywyLCJcXE9tZWdhX3tCL1J9Il0sWzQsMiwiMCJdLFsyLDEsIlxcT21lZ2Ffe0EvUn0iXSxbMiwzLCJKIl0sWzIsMCwiQSJdLFsxLDAsIkkiXSxbMSw2LCJcXHBzaSciLDJdLFsyLDYsIlxccHNpJyciXSxbMCwxXSxbMSwyLCJkIl0sWzIsM10sWzMsNF0sWzcsNSwiZCIsMl0sWzUsMl0sWzgsN10sWzgsMV1d&macro_url=https%3A%2F%2Fraw.githubusercontent.com%2FdFoiler%2Fnotes%2Fmaster%2Fnir.tex
		\[\begin{tikzcd}
			& I & A \\
			&& {\Omega_{A/R}} \\
			0 & {I/I^2} & {\Omega_{A/R}\otimes_AB} & {\Omega_{B/R}} & 0 \\
			&& J
			\arrow["{\psi'}"', from=3-2, to=4-3]
			\arrow["{\psi''}", from=3-3, to=4-3]
			\arrow[from=3-1, to=3-2]
			\arrow["d", from=3-2, to=3-3]
			\arrow[from=3-3, to=3-4]
			\arrow[from=3-4, to=3-5]
			\arrow["d"', from=1-3, to=2-3]
			\arrow[from=2-3, to=3-3]
			\arrow[from=1-2, to=1-3]
			\arrow[from=1-2, to=3-2]
		\end{tikzcd}\]
		A direct computation shows that the rectangle commutes; notably, we are applying the derivation $d$ once each way around the rectangle. Now, $\psi''$ in the above diagram is producing an $R$-derivation $\delta\colon A\to J$ by composing with $d\colon A\to\Omega_{A/R}$ as above, which we see from the above commutativity has $\delta|_I=\psi|_I$.

		Quickly, observe that the map $A\stackrel\delta\to C\to C/J$ vanishes because $\im\delta\subseteq J$. Additionally, we note that $\psi-\delta$ vanishes on $I$ because $\delta|_I=\psi|_I$. In total, $\psi-\delta$ produces a map
		\[\varphi\colon A/I\to C.\]
		Quickly, $\psi$ is a map of $R$-algebras, and $\delta$ is an $R$-derivation, so $\varphi$ is in fact a map of $R$-algebras. It remains to check that
		% https://q.uiver.app/?q=WzAsNCxbMCwwLCJDL0oiXSxbMSwwLCJBL0kiXSxbMCwxLCJDIl0sWzEsMSwiQSJdLFsxLDAsIlxcb3ZlcmxpbmVcXHZhcnBoaSIsMl0sWzIsMF0sWzEsMiwiXFx2YXJwaGkiLDFdLFszLDFdLFszLDIsIlxccHNpIl1d&macro_url=https%3A%2F%2Fraw.githubusercontent.com%2FdFoiler%2Fnotes%2Fmaster%2Fnir.tex
		\[\begin{tikzcd}
			{C/J} & {A/I} \\
			C & A
			\arrow["\overline\varphi"', from=1-2, to=1-1]
			\arrow[from=2-1, to=1-1]
			\arrow["\varphi"{description}, from=1-2, to=2-1]
			\arrow[from=2-2, to=1-2]
			\arrow["\psi", from=2-2, to=2-1]
		\end{tikzcd}\]
		commutes. Well, because $A\to A/I$ is surjective and thus epic, it suffices to show that the composite $A\to A/I\to C\to C/J$ and $A\to A/I\to C/J$ are equal. However, we can see that these are equal by the construction of $\varphi$. Namely, the bottom triangle commutes by construction of $\varphi$ (as discussed: $\delta$ does no change on $A/I$), and the outer square commutes by construction of $\psi$.
		\item We sketch the converse. As usual, we may assume that $X$ is a closed subscheme of $\AA^n_S$ defined by the sheaf of ideals $\mc I$. It suffices to check that
		\[0\to\mc I/\mc I^2\to\Omega_{\AA^n_S/S}\otimes\OO_X\to\Omega_{X/S}\to0\]
		is locally split-exact. Indeed, if split-exact at some $x\in X$, then the inclusion $\mc I_x\subseteq\mf m_x$ grants us some generators $\overline g_{r+1},\ldots,\overline g_n$ of $\left(\mc I/\mc I^2\right)_x$, whose lifts will generate $\mc I_x$; split-exactness tells us that we are surjecting onto a free submodule $dg_{r+1},\ldots,dg_n$ in $\Omega_{\AA^n_S/S}$, which provide the needed sections for smoothness.

		We now continue with the notation previously established. We now build the following diagram.
		% https://q.uiver.app/?q=WzAsNSxbMCwxLCJBL0leMiJdLFswLDAsIkEvSSJdLFsxLDAsIkEvSSJdLFsxLDEsIkEiXSxbMSwyLCJSIl0sWzIsMSwiIiwwLHsibGV2ZWwiOjIsInN0eWxlIjp7ImhlYWQiOnsibmFtZSI6Im5vbmUifX19XSxbNCwzXSxbMywyXSxbMCwxXSxbNCwwXSxbMiwwLCIiLDEseyJzdHlsZSI6eyJib2R5Ijp7Im5hbWUiOiJkYXNoZWQifX19XV0=&macro_url=https%3A%2F%2Fraw.githubusercontent.com%2FdFoiler%2Fnotes%2Fmaster%2Fnir.tex
		\[\begin{tikzcd}
			{A/I} & {A/I} \\
			{A/I^2} & A \\
			& R
			\arrow[Rightarrow, no head, from=1-2, to=1-1]
			\arrow[from=3-2, to=2-2]
			\arrow[from=2-2, to=1-2]
			\arrow[from=2-1, to=1-1]
			\arrow[from=3-2, to=2-1]
			\arrow[dashed, from=1-2, to=2-1]
		\end{tikzcd}\]
		In other words, the formal splitting grants us an $R$-algebra map $\varphi\colon A/I\to A/I^2$ making the above diagram commutes. One can see that this means that $\varphi$ produces a splitting of
		\[0\to I/I^2\to A/I^2\stackrel\nu\to A/I\to0,\]
		so the splitting here produces a section $\tau\colon A/I^2\to I/I^2$ as ${\id_{A/I^2}}-\varphi\circ\nu$. One can compute from here that $\tau$ is an $R$-derivation upon writing it out; thus, our universal property of $\Omega_{A/R}$ produces an $A$-module map $q\colon\Omega_{A/R}\to I/I^2$, which one can check is the desired splitting.
		\qedhere
	\end{itemize}
\end{proof}

\end{document}