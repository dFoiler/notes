% !TEX root = ../notes.tex

\documentclass[../notes.tex]{subfiles}

\begin{document}

\section{January 20}

We continue moving towards Riemann--Roch.

\subsection{Linear Systems}
Let's discuss linear systems. Let $X$ be a non-singular projective irreducible variety over a field $k$, and let $D$ be a divisor of $X$.

Recall that a Cartier divisor $D=\{(U_i,f_i)\}$ on $X$ is associated to the line bundle $\mc L(D)$ which is locally trivial on each $U_i$, given as $f_i^{-1}\OO_X|_{U_i}$. Conversely, suppose that $\mc L$ is a line bundle on $X$. Then we pick up some nonzero global section $\Gamma(X,\mc L)$. Give $\mc L$ a trivializing open cover $\{U_i\}$, where we are given isomorphisms $\varphi_i\colon\mc L|_{U_i}\simeq\OO_X|_{U_i}$. Setting $f_i\coloneqq\varphi_i(s)$ recovers an (effective) Cartier divisor $\{(U_i,f_i)\}$ on $X$. We call this line bundle $\op{div}(\mc L,s)$.

This thinking gives the following result.
\begin{proposition}
	Let $X$ be a non-singular projective integral variety over a field $k$. Given a Cartier divisor $D_0$, and let $\mc L\coloneqq\mc L(D_0)$ be the corresponding line bundle.
	\begin{listalph}
		\item For each nonzero section $s\in\Gamma(X,\mc L)$, the divisor $\op{div}(\mc L,s)$ is an effective divisor linearly equivalent to $D_0$.
		\item Every effective divisor linearly equivalent tot $D_0$ is obtained in this way.
		\item If $k$ is algebraically closed, we have $\op{div}(\mc L,s)=\op{div}(\mc L,s')$ if and only if $s$ and $s'$ differ by a scalar in $k^\times$.
	\end{listalph}
\end{proposition}
The above result essentially says that we can study $\Gamma(X,\mc L)$ as a $k$-vector space instead of trying to understand linear equivalence of divisors. For example, if $\Gamma(X,\mc L)=0$, then $D$ is not equivalent to any effective divisor!
\begin{proof}
	We go one at a time.
	\begin{listalph}
		\item Embed $\mc L\subseteq\mc K_X$ as usual. Then $s\in\Gamma(X,\mc L)$ becomes a rational function in $K(X)$. By the construction of $\mc L$, we have an open cover $\{U_i\}$ and some $f_i$ so that $\mc L|_{U_i}=f_i^{-1}\OO_X|_U$. Because we have a global section, we may write $\varphi_i(s)=f_if$ for some fixed $f$, and then tracking through our Cartier divisor, we get
		\[\op{div}(\mc L,s)=D_0+\op{div}(f),\]
		as needed.
		\item Suppose $D$ is an effective divisor with $D=D_0+\op{div}(f)$. Then we see $(f)\ge-D_0$, so $f$ determines a nonzero global section of $\mc L\mc L(D_0)$ by tracking through the above constructions: namely, set $s|_{U_i}=f_i^{-1}f$ and glue. (In particular, $(f)\ge-D_0$ means $f/f_i\in \OO_X(U_i)$ for each $i$.) So we see $D=\op{div}(\mc L,s)$.
		\item One can see directly that $s=cs'$ for $c\in k^\times$ will have $\op{div}(\mc L,s)=\op{div}(\mc L,s)$. Conversely, if $\op{div}(\mc L,s)=\op{div}(\mc L,s')$, then under the embedding $\mc L\subseteq\mc K_X$, we may correspond $s$ and $s'$ to $f,f'\in K(X)^\times$. Thus, $f/f'\in\Gamma(X,\OO_X^\times)$. But because $k$ is algebraically closed and $X$ is proper over $k$, we have $\Gamma(X,\OO_X)=k$, so we are done.
		\qedhere
	\end{listalph}
\end{proof}
\begin{remark}
	More generally, we have the following: let $k$ be a field, and let $X$ be a proper, geometrically reduced scheme over $k$. Then $\Gamma(X,\OO_X)=k$ if and only if $X$ is geometrically reduced.
\end{remark}
So we have the following.
\begin{corollary}
	Let $X$ be a non-singular projective integral variety over a field $k$. The set $|D_0|$ of effective divisors linearly equivalent to a given divisor $D_0$ is in natural bijection with $(\Gamma(X,\mc L(D))\setminus\{0\})/k^\times$.
\end{corollary}
With this in mind, we set the following notation.
\begin{notation}
	Let $X$ be a non-singular projective integral variety over a field $k$. Given a divisor $D_0$ of $X$, we define $\ell(D_0)\coloneqq\dim_k\Gamma(X,\mc L(D))$ and $\dim D_0\coloneqq\ell(D_0)-1$.
\end{notation}
The Riemann--Roch theorem is interested in the values of $\ell(D_0)$. Here is a quick lemma.
\begin{lemma}
	Let $X$ be a non-singular projective integral variety over a field $k$. Fix a divisor $D$ of $X$.
	\begin{listalph}
		\item If $\ell(D)\ne0$, then $\deg D\ge0$.
		\item If $\ell(D)\ne0$ and $\deg D=0$, then $D$ is linearly equivalent to $0$.
	\end{listalph}
\end{lemma}
\begin{proof}
	Note $\ell(D)\ne0$ enforces $D\sim D_0$ for some effective divisor $D$, so $\deg D=\deg D_0\ge0$, which shows (a). Then for (b), we note $\deg D_0=0$ forces $D_0=0$.
\end{proof}

\subsection{Riemann--Roch for Curves}
We now force $\dim X=1$, meaning that $X$ is a curve. Let $\Omega_{X/k}$ denote the sheaf of differentials, which is equal to the canonical sheaf $\omega_X=\bigwedge^{\dim X}\Omega_{X/k}$. Any divisor linearly equivalent to $\Omega_{X/k}$ will be denoted $K$ and is called the ``canonical divisor.'' Note that the canonical divisor is really a canonical divisor class.
\begin{theorem}[Riemann--Roch] \label{thm:rr}
	Let $D$ be a divisor on a $k$-curve $X$, and let $g$ be the genus of $X$. Further, suppose $k$ is algebraically closed. Then
	\[\ell(D)-\ell(K-D)=\deg D+1-g.\]
\end{theorem}
\begin{proof}
	Set $\mc L\coloneqq\mc L(D)$ for brevity. Note $\mc L(K-D)\cong\omega_X\otimes\mc L^\lor$, so Serre duality implies
	\[\ell(K-D)=\dim_k\Gamma(\omega_X\otimes\mc L^\lor)=\dim_kH^1(X,\mc L).\]
	Thus, our left-hand side is $\chi(\mc L)\coloneqq\dim H^0(X,\mc L)-\dim H^1(X,\mc L)$.\footnote{This is the Euler characteristic of $\mc L(D)$ because our higher cohomology groups vanish.} Quickly, note $D=0$ can be seen directly by
	\[\dim_kH^0(X,\OO_X)-\dim_kH^1(X,\OO_X)=\dim k-g=1-g,\]
	which is what we wanted.

	We now perturb $D$ by a point. We show the formula holds for $D$ if and only if the formula holds for $D+p$, where $p\in X$ is some closed point. Note we have a short exact sequence
	\[0\to\mc L(-p)\to\OO_X\to k(p)\to0,\]
	where $k(p)$ refers to the skyscraper sheaf which is the structure sheaf about $p$. Tensoring with $\mc L(D+p)$, we get
	\[0\to\mc L(D)\to\mc L(D+p)\to k(p)\to0.\]
	Now, $\chi$ is additive in short exact sequences by using the long exact sequence in cohomology, so
	\[\chi(\mc L(D))=\chi(\mc L(D+p))+\chi(k(P)),\]
	but $\chi(k(p))=\dim_k\Gamma(X,k(p))=\dim_kk=1$ because $k$ is algebraically closed. The conclusion now follows because $\deg(D+p)=\deg D+1$.
\end{proof}

\end{document}