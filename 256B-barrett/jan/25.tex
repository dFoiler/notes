% !TEX root = ../notes.tex

\documentclass[../notes.tex]{subfiles}

\begin{document}

\section{January 25}

Today we discuss ramification.

\subsection{Ramification}
Given a finite morphism $f\colon X\to Y$ of $k$-curves, there are some numbers we can attach to $f$. For example, we have the degree $\deg f\coloneqq[K(X):K(Y)]$. Additionally, for each $y\in Y$, we have a map
\[\OO_{Y,y}\to\prod_{x\in f^{-1}(\{y\})}\OO_{X,x}.\]
Each of these discrete valuation rings have residue field $k$, so when $k$ is algebraically closed, this is pretty simple to understand. Indeed, this gives rise to ``ramification'' information.
\begin{definition}[ramification]
	Let $f\colon X\to Y$ be a morphism of $k$-curves. Then for $x\in X$, we define the \textit{ramification index} as $e_x\coloneqq v_x(t_y)$, where $t_y\in\OO_{Y,y}$ is a uniformizer for $\OO_{Y,y}$, and $v_x(t_y)$ refers to the valuation of $t_y$ embedded in $\OO_{X,x}$.
\end{definition}
Note $e_x>0$ because $\OO_{Y,y}\to\OO_{X,x}$ is a map of local rings.

When $k$ is not algebraically closed, things get a little more complicated, and we will also want to keep track of the degree of the corresponding residue field extension.
\begin{definition}[ramification] \label{defi:gen-ramification}
	Let $\varphi\colon(A,\mf m)\to(B,\mf n)$ be a map of discrete valuation rings. Fix a uniformizer $\varpi\in\mf m$, and define $e\coloneqq v_B(\varphi(\varpi))$ and $f\coloneqq[B/\mf n:A/\mf m]$ and $p\coloneqq\op{char}(A/\mf m)$.
	\begin{itemize}
		\item $\varphi$ is \textit{unramified} if and only if $e=1$ and $B/\mf n$ is separable over $A/\mf m$.
		\item $\varphi$ is \textit{tamely ramified} if and only if $p\nmid e$ and $B/\mf n$ is separable over $A/\mf m$.
		\item Otherwise, $\varphi$ is \textit{wildly ramified}.
	\end{itemize}
\end{definition}
\begin{remark}
	Algebraic number theory has a lot to say about how the above process works for local fields (or even just number rings).
\end{remark}
When $k$ is algebraically closed, we see that the extension $\OO_{Y,y}/\mf m_{Y,y}=\OO_{X,x}/\mf m_{X,x}=k$, so this extension is of course separable, so \Cref{defi:gen-ramification} simplifies somewhat in our situation.
\begin{definition}
	Let $f\colon X\to Y$ be a finite morphism of $k$-curves. Then we say that $x\in X$ is \textit{unramified/tamely ramified/wildly ramified} if and only if the corresponding map $f^\sharp\colon\OO_{Y,y}\to\OO_{X,x}$ is unramified/tamely ramified/wildly ramified.
\end{definition}
Notably, last class we recalled that
\[f^*[y]=\sum_{x\in f^{-1}(\{y\})}e_x[x].\]
Now, even in our algebraically closed situation, we will want to care about separable extensions.
\begin{definition}[separable]
	Let $f\colon X\to Y$ be a finite morphism of $k$-curves. Then $f$ is \textit{separable} if and only if the extension $K(Y)\subseteq K(X)$ is separable.
\end{definition}
Now, we would like to keep track of our ramification information all at once.
\begin{lemma}
	Let $f\colon X\to Y$ be a finite separable morphism of $k$-curves. Then
	\[0\to f^*\Omega_{Y/k}\to\Omega_{X/k}\to\Omega_{X/Y}\to0\]
	is an exact sequence of line bundles on $X$.\todo{}
\end{lemma}
\begin{proof}
	We know from last semester that this map is exact on the right, so we need the map $f^*\Omega_{Y/k}\to\Omega_{X/k}$ is injective. Well, we may check exactness of quasicoherent sheaves on affine open subschemes, so we may assume that $X=\Spec A$, where the map looks like $A\to A$. Letting $I$ denote the kernel of this map, we get an embedding $A/I\subseteq A$, but if $I$ is nontrivial, then this means that the map $A\to A$ is zero at the generic point.

	Thus, we want to check that the map $A\to A$ is nonzero at the generic point. Everything is compatible with localization, so we are now looking at
	\[f^*\Omega_{K(Y)/k}\to\Omega_{K(X)/k}\to\Omega_{K(X)/K(Y)}\to0.\]
	Thus, to show that the map on the left is nonzero, it suffices to show that $\Omega_{K(X)/K(Y)}=0$. We now use the fact that $K(X)/K(Y)$ is separable, for which the statement is true.\todo{}
\end{proof}
The point here is that $\Omega_{X/Y}$ precisely measures the ``difference'' between $\Omega_{Y/k}$ and $\Omega_{X/k}$.

In what follows, we fix the following notation. Let $f\colon X\to Y$ be a finite morphism of $k$-curves. Given $x\in X$ and $y\coloneqq f(x)$, let $\varpi_x$ be a uniformizer for $\OO_{X,x}$ and $\varpi_y$ be a uniformizer for $\Omega_{Y,y}$. Then we note $d\varpi_x$ generates $(\Omega_{X/k})_x$, and $d\varpi_y$ generates $(\Omega_{Y/k})_y$, and we have a map
\[(f^*\Omega_{Y/k})_x\to(\Omega_{X/k})_x\simeq\OO_{X,x}\]
by sending $f^*\colon d\varpi_y\mapsto d\varpi_y/d\varpi_x\cdot d\varpi_x$, where $d\varpi_y/d\varpi_x$ is an element of $\OO_{X,x}$. (This is the definition of $d\varpi_y/d\varpi_x$.)
\begin{proposition}
	Let $f\colon X\to Y$ be a finite separable morphism of $k$-curves.
	\begin{listalph}
		\item $\Omega_{X/Y}$ is supported on exactly the set of ramification points of $f$, so the set of ramified points is finite.
		\item For each $x\in X$, we have $(\Omega_{Y/X})_x$ is a principal $\OO_{X,x}$-module of length $v_x(d\varpi_y/d\varpi_x)$.
		\item If $f$ is tamely ramified at $x$, then the length of $(\Omega_{Y/X})_x$ is $e_x-1$; if it's wildly ramified, then the length is larger.
	\end{listalph}
\end{proposition}
\begin{proof}
	We show these one at a time. As a warning, all uniformizers might be swapped here.
	\begin{listalph}
		\item Recall that $\Omega_{Y/X}$ is generically zero. Now, $(\Omega_{Y/X})_p=0$ if and only if the map
		\[(f^*\Omega_{Y/k})_x\to(\Omega_{X/k})_x\]
		is an isomorphism, which means that a uniformizer for $\OO_{Y,f(x)}$ is going to a uniformizer of $\OO_{X,x}$, which is equivalent to $f$ being unramified at $x$.\todo{What?} The point here is that the set of ramified points correspond to some dimension-zero subset and is therefore finite.
		\item The length of $(\Omega_{Y/X})_p$ is its $k$-dimension, which we can compute as
		\[\op{length}(\Omega_{X/k})_x-\op{length}(f^*\Omega_{Y/K})_x,\]
		which we can compute is $e_x$ by hand. Notably, this has to do with how we identify $(\Omega_{X/k})_x$ with $\OO_{X,x}$.
		\item Letting $e$ denote our ramification index, we may set $\varpi_y=a\varpi_x^e$ where $a\in\OO_{X,x}^\times$, which upon taking differentials reveals
		\[d\varpi_y=ea\varpi_x^{e-1}d\varpi_x+\varpi_x^eda.\]
		Now, if $f$ is tamely ramified at $x$, we see $\op{char}k\nmid e$, so we see that the valuation here is in fact $e_x-1$. The statement for wild ramification follows similarly.
		\qedhere
	\end{listalph}
\end{proof}
\begin{remark}
	The length of the modules here coincides with the dimension as a $k$-vector space. This is because $\OO_{X,x}$ is a discrete valuation ring with residue field $k$.
\end{remark}

\end{document}