% !TEX root = ../notes.tex

\documentclass[../notes.tex]{subfiles}

\begin{document}

\section{January 18}

Here we go.

\subsection{House-Keeping}
Here are some notes on the course.
\begin{itemize}
	\item We will continue to use \cite{hartshorne}. Note that \cite{rising-sea} is also popular, as is \cite{stacks}.
	\item Office hours will probably be after class on Wednesday and Friday.
	\item There is a \href{https://bcourses.berkeley.edu/courses/1522162}{\texttt{bCourses}}.
	\item In the course, we plan to cover curves, some coherent cohomology (and maybe on Zariski sheaves), and some surfaces if we have time.
	\item Grading will be homework and a term paper. Homework will be challenging, so collaboration is encouraged.
\end{itemize}
In this course, we will discuss coherent cohomology, but we will begin by talking about curves.

\subsection{Serre Duality Primer}
For the next few weeks, we will focus on non-singular curves over an algebraically closed field. Here is our definition.
\begin{definition}[curve]
	Fix a field $k$. A \textit{$k$-curve} is an integral, proper, normal scheme of dimension $1$. Note that being normal is equivalent to being smooth, so we are requiring our curves to be smooth!
\end{definition}
We will want to talk about genus a little. Here is a working definition.
\begin{definition}[arithmetic genus]
	Fix a projective $k$-variety $X$. Then the \textit{arithmetic genus} $p_a(X)$ is defined
\end{definition}
\begin{definition}[geometric genus]
	Fix an irreducible $k$-variety $X$. Then the \textit{geometric genus} is $p_g(X)\coloneqq\dim_k\Gamma(X,\omega_X)$, where $\omega_X$ is the canonical sheaf. Explicitly, $\omega_X$ is the top exterior power of the sheaf of differential forms on $X$.
\end{definition}
In general, the above notions are not the same, but they will be for curves.
\begin{proposition}
	Fix a $k$-curve $X$. Then $p_g(X)=p_a(X)$. We denote this genus by $g(X)$ or $g$ when the curve is clear.
\end{proposition}
We would like to actually compute some genera, but this is a bit difficult. One goal of the class is to build a cohomology theory $H^i(X,\mc F)$ for coherent sheaves $\mc F$ on $X$, and it turns out we can use these cohomology groups to compute the genus of $X$. Roughly speaking, we will derive (on the right) the left-exact functor $\Gamma(X,\cdot)$, so the cohomology will in some sense measure the difference between global sections and local sections. For example, flasque sheaves will have trivial cohomology.

For now, we will black-box various things. Here is an example of something we will prove.
\begin{proposition}
	Fix a projective $k$-variety $X$, and let $\mc F$ be a coherent sheaf. Then $H^i(X,\mc F)=0$ for $i>\dim X$, and $H^i(X,\mc F)$ are finite-dimensional $k$-vector spaces for all $i\ge0$.
\end{proposition}
To show the Riemann--Roch theorem, we will black-box Serre duality, which we will prove much later. In the case of curves, it says the following.
\begin{theorem}[Serre duality]
	Fix a $k$-curve $X$. Then, for any vector bundle $\mc L$ on $X$, there is a duality
	\[H^i(X,\mc L^\lor\otimes\omega_X)\otimes_kH^{1-i}(X,\mc L)\to k,\]
	where $i\in\{0,1\}$.
\end{theorem}
\begin{remark}
	Notably, we see $p_g(X)=\dim_k\Gamma(X,\omega_X)=\dim_kH^0(X,\omega_X)=\dim_kH^1(X,\OO_X)$.
\end{remark}
We will also want the following fact.
\begin{proposition}
	Fix a closed embedding $i\colon X\to Y$ of schemes. Given a sheaf $\mc F$ of abelian groups on $Y$, then
	\[H^i(X,i_*\mc F)=H^i(Y,\mc F).\]
\end{proposition}

\subsection{Divisors Refresher}
We also want to recall a few facts about divisors. We begin with Weil divisors.
\begin{definition}[Weil divisor]
	Fix an irreducible $k$-variety $X$. A \textit{Weil divisor} $\op{Div}(X)$ are $\ZZ$-linear combinations of codimension-$1$ irreducible closed subschemes. Then the \textit{principal divisors} are the image of the map $\op{div}\colon K(X)\to\op{Div}(X)$, where $\op{div}$ takes rational functions to poles. The \textit{class group} $\op{Cl}X$ is the quotient.
\end{definition}
More generally, we have Cartier divisors.
\begin{definition}[Cartier divisor]
	Fix a scheme $X$. A \textit{Cartier divisor} in $\op{CaDiv}X$ is a global section of $\Gamma(X,\mc K^\times/\OO_X^\times)$, where $K^\times$ is the sheafification of the presheaf $U\mapsto\op{Frac}\OO_X(U)$. The \textit{principal divisors} are the image of $\Gamma(X,\mc K^\times)$, and the \textit{class group} $\op{CaCl}X$ is the quotient.
\end{definition}
Notably, if $X$ is an integral sheaf, then $\mc K$ is the constant sheaf $K(X)$. Then a global section is given by the pair $(\{U_i\},\{f_i\})$ where the $U_i$ cover $X$, and $f_i\in K(X)^\times$ so that $f_i/f_j\in\OO_X(U_i\cap U_j)^\times$. (The coherence condition allows the Cartier divisors to glue.) Notably, each $f\in K(X)$ grants a principal divisor $(\{X\},\{f\})$, which are exactly the principal divisors.

Here is the main result on these divisors.
\begin{proposition}
	If $X$ is an integral, separated, Noetherian, and locally factorial (notably, regular in codimension $1$), then Weil divisors are in canonical isomorphism with Cartier divisors. Further, the principal divisors are in correspondence, and so the class groups are also isomorphic.
\end{proposition}
\begin{example}
	Non-singular $k$-curves have all the required adjectives. Namely, codimension-$1$ means we are looking at points, and being smooth implies being regular, so all the local rings are dimension-$1$ regular local rings, which are discrete valuation rings. Notably, discrete valuation rings are
\end{example}
Yet another connection to divisors comes from invertible sheaves. Namely, for integral schemes $X$, the group of invertible sheaves $\op{Pic}X$ is isomorphic to $\op{CaCl}X$. The point here is that invertible shaves can be embedded into $\mc K^\times$ when $X$ is integral.

We will be interested in some special divisors.
\begin{definition}[effective]
	Fix a $k$-curve $X$. Then an \textit{effective Weil divisor} is a $\ZZ_{\ge0}$ linear combination of closed points of $X$. Note that the collection of effective Weil divisors forms a submonoid of $\op{Div}X$. We might be interested in knowing how many effective divisors are equivalent to some given divisor; the set of these is denoted $|D|$.
\end{definition}
When our schemes $X$ have enough adjectives, we note that the above correspondences tell us that there is a way to send a Cartier divisor $(\{U_i\},\{f_i\})$ to a line bundle $\mc L$ embedded in $\mc K^\times$. Explicitly, we build $\mc L(D)$ by $\mc L(D)|_{U_i}\cong\OO_X|_{U_i}\subseteq\mc K$, where the last isomorphism is by sending $1\mapsto f_i^{-1}$. Notably, if $D$ is effective, then the global section $1$ of $\mc K^\times$ can be pulled back along to a nonzero global section on $\mc L(D)$ which is $f_i$ on each $U_i$.

\end{document}