% !TEX root = ../notes.tex

\documentclass[../notes.tex]{subfiles}

\begin{document}

\section{March 10}

Today we talk about cohomology of projective space.

\subsection{Cohomology of Projective Space}
Fix a Noetherian ring $A$, and we work with $\PP^r_A=\Proj A[x_0,\ldots,x_r]$. Here is our result; note that we are looking at our cohomology as $A$-modules, which we know is legal because we can take flasque resolutions and value everything in $\mathrm{Mod}_A$ instead of merely $\mathrm{Ab}$.
\begin{theorem} \label{thm:proj-cohom}
	Fix a Noetherian ring $A$, and set $X\coloneqq\PP^r_A$ with $r>0$. Further, set $S\coloneqq A[x_0,\ldots,x_n]$ and $\mc F\coloneqq\bigoplus_{r\in\ZZ}\OO(n)$.
	\begin{listalph}
		\item We have $H^0(X,\mc F)=S$. In other words, $H^0(X,\OO(n))=S_n$ for any $n$.
		\item For any $0<i<r$, we have $H^i(X,\mc F)=0$.
		\item We have $H^r(X,\OO(-r-1))\cong A$.
		\item For all $n$, there is a perfect $A$-pairing
		\[H^0(X,\OO(n))\otimes_AH^r(X,\OO(-n-r-1))\to H^r(X,\OO(-r-1))=A.\]
		In other words, $H^0(X,\OO(n))\cong\op{Hom}_A(H^r(X,\OO(-n-r-1)),A)$.
	\end{listalph}
\end{theorem}
\begin{remark}
	Because $\PP^r_A$ can be covered by $r+1$ affine open subschemes in the standard affine open cover $\mf U$, we see
	\[H^i(X,\mc F)=\check H^i(\mf U,\mc F)=0\]
	for $i>r$ by definition of \v Cech cohomology. Thus, we have no higher cohomology than given in \Cref{thm:proj-cohom}.
\end{remark}
\begin{remark}
	To properly state the perfect pairing, we note that a global section $s\in H^0(X,\OO(n))$ produces a map $s\colon\OO(-n-r-1)\to\OO(-r-1)$ by multiplication and checking degrees, which by functoriality produces a map $s\colon H^r(X,\OO(-n-r-1))\to H^r(X,\OO(-r-1))$. The perfect pairing is asserting that the induced map
	\[H^0(X,\OO(n))\cong\op{Hom}_A(H^r(X,\OO(-n-r-1)),A)\]
	is an isomorphism. Notably, by adjunction, we are really talking about a map $s\otimes t\mapsto st$, which is what is included in the statement.
\end{remark}
\begin{proof}[Proof of \Cref{thm:proj-cohom}]
	We showed (a) last class in \Cref{prop:h0-proj-space}. Now, we use \v Cech cohomology. Set $U_i\coloneqq D_+(x_i)\cong\Spec A[x_{0/i},\ldots,x_{n/i}]/(x_{i/i}-1)$ so that $\mf U\coloneqq\{U_i\}$ is an open cover of $X$. Notably, we see that
	\[U_{i_0\cdots i_p}=D_+(x_{i_0}\cdots x_{i_p}).\]
	Computing our sections now, we see
	\[\Gamma(U_{i_0\cdots i_p},\mc F)=\bigoplus_{n\in\ZZ}\Gamma(U_{i_0\cdots i_p},\OO(n))=\bigoplus){n\in\ZZ}\Gamma(D_+(x_{i_0}\cdots x_{i_p}),\OO(n)).\]
	Now, accounting for our shifting, $\Gamma(D_+(x_{i_0}\cdots x_{i_p}),\OO(n))$ is asking for degree-$0$ elements of $(S_n)_{x_{i_0}\cdots x_{i_p}}$, which are really the degree-$n$ elements of $S_{x_{i_0}\cdots x_{i_p}}$. As such,
	\[\Gamma(U_{i_0\cdots i_p},\mc F)=\bigoplus_{n\in\ZZ}(S_{x_{i_0}\cdots x_{i_p}})_n=S_{x_{i_0}\cdots x_{i_p}}.\]
	As such, our \v Cech complex $C^\bullet(\mf U,\mc F)$ looks like
	\begin{equation}
		0\to\prod_{i=0}^rS_{x_i}\xrightarrow{d^0}\prod_{0<i_0<i_1\le r}S_{x_{i_0}x_{i_1}}\xrightarrow{d^1}\cdots\to\prod_{0<i_0<\cdots<i_r\le r}S_{x_{i_0}\cdots x_{i_r}}\xrightarrow{d^{r-1}}S_{x_0\cdots x_r}\to0. \label{eq:proj-cech}
	\end{equation}
	For example, $\im d^{r-1}$ is the span of the monomials $x_0^{\nu_1}\cdots x_r^{\nu_r}$ where at least one of the $\nu_\bullet$ is nonnegative because it must arise from some $S_{x_0\cdots\widehat{x_k}\cdots x_r}$, which means that $x_k$ is not used. Now, we can compute
	\[\check H^r(\mf U,\mc F)=\frac{S_{x_0\cdots x_r}}{\im d^{r-1}}.\]
	In particular, the numerator is a free $A$-module generated by monomials $x_0^{\nu_1}\cdots x_r^{\nu_r}$, but the denominator is a free $A$-module generated by monomials with at least one positive coordinate. Thus, by taking the quotient appropriately, $\check H^r(\mf U,\mc F)$ is a free $A$-module generated by $x_0^{\nu_0}\cdots x_r^{\nu_r}$ where all the terms are (strictly) negative!

	For example, if we insist on looking at $\check H^r(\mf U,\OO(-r-1))$, then we are looking at an $A$-module freely generated by $x_0^{\nu_0}\cdots x_r^{\nu_r}$ where each $\nu_\bullet$ is negative, and $\nu_0+\cdots+\nu_r=-r-1$. However, the only way to do this is for everything to be $-1$, so we see
	\[H^r(X,\OO(-r-1))=\check H^r(\mf U,\OO(-r-1))=Ax_0^{-1}\cdots x_r^{-1}.\]
	This is (c).

	One can see (d) similarly. In particular, $H^0(X,\OO(n))$ has an $A$-basis of monomials $x_0^{\nu_0'}\cdots x_r^{\nu_r'}$ with $\nu_\bullet'\ge0$ and total degree $n$; on the other hand, $H^r(X,\OO(-n-r-1))$ has an $A$-basis of monomials $x_0^{\nu_0}\cdots x_r^{\nu_r}$ with $\nu_\bullet\le0$ and total degree $-n-r-1$, so we can simply multiply these homogeneous polynomials to produce a homogeneous polynomial of degree $-r$, which we can put into $H^r(X,\OO(-r-1))$. In other words, we are sending
	\[x_0^{\nu_0'}\cdots x_r^{\nu_r'}\cdot x_0^{\nu_0}\cdots x_r^{\nu_r}\mapsto x_0^{\nu_0'+\nu_0}\cdots x_r^{\nu_r'+\nu_r}\pmod{\im d^{r-1}}.\]
	It remains to show that we are producing a perfect pairing in this way. In other words, given some morphism $H^r(X,\OO(-n-r-1))\to H^r(X,\OO(-r-1))$, it must arise from some unique element of $S$ via the above multiplication; it suffices to check this on our basis of monomials. One can see this from a direct computation: given some monomial $x_0^{\nu_0}\cdots x_r^{\nu_r}$ satisfying
	\[\sum_{i=0}^r(-\ell_i-1)=n,\]
	we need to produce a monomial $x_0^{\nu_0'}\cdots x_{\nu_r'}$ such that the corresponding products land us in degree $-r-1$. However, we see that we must have $0\le\nu_i\le-\nu_i-1$ everywhere to ensure that we land in a monomial with negative degrees everywhere (up to some residue class---anything with a nonnegative degree somewhere will vanish entirely). But then equalities must hold everywhere to get the correct total degree, which makes our chosen monomial unique.

	Lastly, we must show (b). Let's compute $H^i(X,\mc F)_{x_r}$. Here, our \v Cech complex \eqref{eq:proj-cech} can be localized by $x_r$ to compute $H^i(\mf U,\mc F)_{x_r}$ because localization is an exact functor and therefore preserves quotients. However, we can also view this localization as restricting the open cover $\mf U$ by intersecting everything to $D_+(x_r)$, so we are really looking at $H^i(X,\mc F)_{x_r}=H^i(\mf U,\cap D_+(x_r),\mc F|_{D_+(x_r)})$, which vanishes by \Cref{thm:qcoh-cohom}.

	However, we claim that 
	\[H^i(X,\mc F)\xrightarrow{x_r}H^i(X,\mc F)\]
	is an isomorphism, which will complete the proof by embedding into the localization (which vanishes); we show this by induction. Well, we note that we have an exact sequence of graded $S$-modules given by
	\begin{equation}
		0\to S(-1)\stackrel{x_r}\to S\to S/(x_r)\to0. \label{eq:desired-ses-mult-xr}
	\end{equation}
	Taking $\widetilde\cdot$ produces an exact sequence of sheaves by
	\[0\to\OO_X(-1)\to\OO_X\to\iota_*\OO_H\to0,\]
	where $H$ is a hyperplane $\iota\PP^{r-1}_A\subseteq\PP^r_A$. Twisting and summing appropriately, we get the exact sequence
	\[0\to\mc F(-1)\to\mc F\to\iota_*\mc F_H\to0\]
	where $\mc F_H=\bigoplus_{n\in\ZZ}\OO_H(n)$. Now, for $1<i<r-1$, the long exact sequence provides
	\begin{equation}
		0\to H^i(X,\mc F(-1))\stackrel{x_r}\to H^1(X,\mc F)\to0 \label{eq:dim-shift-mult-xr}
	\end{equation}
	by induction (inducting on $r$ noting that $H$ has dimension $r-1$), finishing. It remains to deal with degree $0$ and degree $r-1$. In degree $0$, we note that we already have exactness
	\[0\to\Gamma(X,\mc F(-1))\to\Gamma(X,\mc F)\to\Gamma(X,\iota_*\OO_H)\to0\]
	because this is just \eqref{eq:desired-ses-mult-xr} on global sections. Thus, we still get \eqref{eq:dim-shift-mult-xr} in degree $0$, so we are safe.

	Lastly, in degree $r$, we have
	\[0\to H^{r-1}(X,\mc F(-1))\to H^{r-1}(X,\mc F)\to H^{r-1}(X,\mc F_H)\stackrel\delta\to H^r(X,\mc F(-1))\to H^r(X,\mc F)\to0.\]
	We want to show that $\delta$ is injective to recover \eqref{eq:dim-shift-mult-xr} in degree $r-1$. The main idea is that \v Cech cohomology produces an exact sequence of complexes: because $\mc F(-1)$ is quasicoherent, any sequence of indices has the exact sequence
	\[0\to\mc F(-1)(U_{i_0\cdots i_p})\to\mc F(U_{i_0\cdots i_p})\to\mc F_H(U_{i_0\cdots i_p})\to0,\]
	so summing over everything gives
	\[0\to\underbrace{\mc C^\bullet(\mf U,\mc F(-1))}_A\to\underbrace{\mc C^\bullet(\mf U,\mc F)}_B\to\underbrace{\mc C^\bullet(\mf U,\mc F_H)}_C\to0.\]
	Now, we want to compute the connecting homomorphism $\delta_{r-1}$. Well, it is induced by the Snake lemma in the following diagram.
	% https://q.uiver.app/?q=WzAsOCxbMiwwLCJCXntyLTF9L1xcaW0gZF57ci0xfV9CIl0sWzMsMCwiQ157ci0xfS9cXGltIGRee3ItMX1fQyJdLFs0LDAsIjAiXSxbMSwxLCJcXGtlciBkXnJfQSJdLFsyLDEsIlxca2VyIGRecl9CIl0sWzEsMCwiQV57ci0xfS9cXGltIGRfQV57ci0xfSJdLFszLDEsIlxca2VyIGRecl9BIl0sWzAsMSwiMCJdLFs3LDNdLFszLDRdLFs0LDZdLFs1LDBdLFswLDFdLFsxLDJdLFs1LDNdLFswLDRdLFsxLDZdXQ==&macro_url=https%3A%2F%2Fraw.githubusercontent.com%2FdFoiler%2Fnotes%2Fmaster%2Fnir.tex
	\[\begin{tikzcd}
		& {A^{r-1}/\im d_A^{r-2}} & {B^{r-1}/\im d^{r-2}_B} & {C^{r-1}/\im d^{r-2}_C} & 0 \\
		0 & {\ker d^r_A} & {\ker d^r_B} & {\ker d^r_A}
		\arrow[from=2-1, to=2-2]
		\arrow[from=2-2, to=2-3]
		\arrow[from=2-3, to=2-4]
		\arrow[from=1-2, to=1-3]
		\arrow[from=1-3, to=1-4]
		\arrow[from=1-4, to=1-5]
		\arrow[from=1-2, to=2-2]
		\arrow[from=1-3, to=2-3]
		\arrow[from=1-4, to=2-4]
	\end{tikzcd}\]
	Well, an element of $C^{r-1}/\im d_C^{r-2}$ is
	\[\prod_{k=0}^r(S/(x_r))_{x_0\cdots\widehat{x_k}\cdots x_r},\]
	but this is just $(S/(x_r))_{x_0\cdots x_{r-1}}$ and is therefore given by a monomial $x_0^{\ell_0}\cdots x_{r-1}^{\ell_{r-1}}$ with nonnegative degree in each coordinate. We can lift this monomial directly to a class in
	\[B^{r-1}=\prod_{k=0}^rS_{x_0\cdots\widehat{x_k}\cdots x_r},\]
	where the terms with $k\ne r$ we'll just take to vanish. As such, applying $d_B^{r-1}$ to our monomial, we compute
	\[d_B^{r-1}\left(x_0^{\ell_0}\cdots x_{r-1}^{\ell_{r-1}}\right)=(-1)^r\left[x_0^{\ell_0}\cdots x_{r-1}^{\ell_{r-1}}\right]\]
	because all other terms of our alternating sum vanish. Now pulling this back to $\ker d^r_A$, we have to divide by $x_r$ and thus produce the class represented by the monomial $x_0^{\ell_0}\cdots x_{r-1}^{\ell_{r-1}}x_r^{-1}$. Thus, our connecting homomorphism $\delta$ is simply division by $x_r$, which is indeed injective. Indeed, it has ``inverse'' given by the next map $x_r\colon H^r(X,\mc F(-1))\to H^r(X,\mc F)$ in our long exact sequence because then we are dealing with a monomial of nonnegative degree, which vanishes.
\end{proof}

\end{document}