% !TEX root = ../notes.tex

\documentclass[../notes.tex]{subfiles}

\begin{document}

\section{March 1}

We continue talking about quasicoherent sheaves.

\subsection{Finishing Cohomology of Quasicoherent Sheaves}
Here is our result.
\quasicohinjisflasque*
\begin{proof}
	We use Noetherian induction on the closed set $Y(I)\coloneqq\overline{\op{Supp}\widetilde I}$. As our base case, if $Y(I)$ is a singleton $\{x\}$, then the closed embedding $i\colon\{x\}\to X$ tells us that
	\[\widetilde I=i_*i^*\widetilde I\]
	by checking the obvious map at stalks. And now one can see that any skyscraper sheaf will be flasque by simply looking at it.

	In general, we would like to show that $\Gamma(X,\mc I)\to\Gamma(U,\mc I)$ is surjective by simply always lifting to a global section before restricting. Now, if $Y\cap U=\emp$, then $\Gamma(U,\mc I)=0$, so there is nothing to say; as such, we may assume that $Y\cap U\ne\emp$. Using the standard base on the affine scheme, we may find $f\in A$ such that $D(f)\subseteq U$ and $D(f)\cap Y\ne\emp$.

	Continuing, we set $Z\coloneqq V(f)$ so that we have the following commutative diagram.
	% https://q.uiver.app/?q=WzAsNSxbMCwwLCJcXEdhbW1hX1ooWCxcXHdpZGV0aWxkZSBJKSJdLFsxLDAsIlxcR2FtbWFfe1pcXGNhcCBVfShVLFxcd2lkZXRpbGRlIEkpIl0sWzAsMSwiXFxHYW1tYShYLFxcd2lkZXRpbGRlIEkpIl0sWzEsMSwiXFxHYW1tYShVLFxcd2lkZXRpbGRlIEkpIl0sWzIsMSwiXFxHYW1tYShEKGYpKSxcXHdpZGV0aWxkZSBJKSJdLFswLDFdLFsxLDMsIiIsMCx7InN0eWxlIjp7InRhaWwiOnsibmFtZSI6Imhvb2siLCJzaWRlIjoidG9wIn19fV0sWzAsMiwiIiwyLHsic3R5bGUiOnsidGFpbCI6eyJuYW1lIjoiaG9vayIsInNpZGUiOiJ0b3AifX19XSxbMiwzXSxbMyw0XV0=&macro_url=https%3A%2F%2Fraw.githubusercontent.com%2FdFoiler%2Fnotes%2Fmaster%2Fnir.tex
	\[\begin{tikzcd}
		{\Gamma_Z(X,\widetilde I)} & {\Gamma_{Z\cap U}(U,\widetilde I)} \\
		{\Gamma(X,\widetilde I)} & {\Gamma(U,\widetilde I)} & {\Gamma(D(f)),\widetilde I)}
		\arrow[from=1-1, to=1-2]
		\arrow[hook, from=1-2, to=2-2]
		\arrow[hook, from=1-1, to=2-1]
		\arrow[from=2-1, to=2-2]
		\arrow[from=2-2, to=2-3]
	\end{tikzcd}\]
	Now, the bottom composite is surjective by \Cref{lem:inj-localization-is-surj}.

	We now proceed directly. Given $t\in\Gamma(U,\widetilde I)$, we see that we can lift $t|_{D(f)}$ to some $s\in\Gamma(X,\mc I)$ by this surjectivity. But then $s|_U-t$ will have support in $X\setminus D(f)=Z$. As such, if we could just show that the map
	\[\Gamma_Z(X,\widetilde I)\to\Gamma_{Z\cap U}(U,\widetilde I)\]
	is surjective, then we could lift $s|_U-t$ to some $r\in\Gamma_Z(X,\widetilde I)$ so that $(s-r)|_{D(f)}=s|_{D(f)}=t$, but $(s-r)|_U-t=(s|_U-t)-r|_U=0$. Thus, we want to show that the above map is surjective, which amounts to showing that the sheaf $\Gamma_Z(-,\mc I)=\widetilde{\Gamma_{(f)}(I)}$ is flasque, where this equality is by \Cref{rem:h0-qcoh}.

	However, the support of $\widetilde{\Gamma_{(f)}(I)}$ is contained in $Z\cap Y$, where it is contained in $Y$ by construction of $I$ and contained in $Z$ by construction of $(f)$. But now we are done by Noetherian induction because $Z\cap Y$ is a smaller closed subset. Explicitly, we may suppose the result for all strictly smaller closed subsets than $Y$, for otherwise we would be able to build some infinite descending chain of closed sets in $\Spec A$, which is not possible because $A$ is Noetherian.
\end{proof}
\begin{corollary}
	Fix a Noetherian scheme $X$. Then any quasicoherent sheaf $\mc F$ can be embedded into a flasque quasicoherent sheaf.
\end{corollary}
\begin{proof}
	Over affine opens, we can embed into an injective module, which is flasque by the above. Taking the direct image sheaf to $X$ and then taking the product produces the desired flasque sheaf and embedding.
\end{proof}
\begin{remark}
	Professor Barrett is unsure if injectives in the category of quasicoherent sheaf are actually acyclic for $\Gamma(X,-)$.
\end{remark}

\subsection{Serre's Criterion for Affines}
Let's give a cohomological criterion for being affine.
\begin{theorem}[Serre] \label{thm:serre}
	Fix a Noetherian scheme $X$. Then the following are equivalent.
	\begin{listalph}
		\item $X$ is affine.
		\item $H^i(X,\mc F)=0$ for all $i>0$ and quasicoherent sheaves $\mc F$ on $X$.
		\item $H^1(X,\mc F)=0$ for all coherent sheaves $\mc F$ of ideals.
	\end{listalph}
\end{theorem}
For this, we want the following lemma.
\begin{lemma} \label{lem:affine-cover-is-affine}
	A scheme $X$ is affine if and only if there exist global sections $f_1,\ldots,f_r\in\Gamma(X,\OO_X)$ which generate $\Gamma(X,\OO_X)$ such that all the subschemes $D(f_i)$ are affine.
\end{lemma}
\begin{proof}
	This is by Exercise~II.2.17 of \cite{hartshorne}. We did this for homework last semester.
\end{proof}
Now let's show the theorem.
\begin{proof}[Proof of \Cref{thm:serre}]
	We show our implications in sequence. Note (a) implies (b) by \Cref{thm:qcoh-cohom}, and (b) implies (c) with no work, so the hard part is showing (c) implies (a). We would like to use \Cref{lem:affine-cover-is-affine}. Well, for each $p\in X$, we let $p\in U$ be some affine open neighborhood and set $Y\coloneqq X\setminus U$. Then we get some exact sequence
	\[0\to\mc I_{Y\cup\{p\}}\to\mc I_Y\to k(p)\to0,\]
	where we can check exactness at stalks. Now, by the long exact sequence we see
	\[\Gamma(X,\mc I_Y)\to k(p)\to H^1(X,\mc I_{Y\cup\{p\}}),\]
	where the last cohomology group vanishes by hypothesis. In total, our surjectivity produces $f\in\Gamma(X,\mc I_Y)\subseteq\Gamma(X,\OO_X)$ such that $f|_p=1$ in $k(p)$. Further, $f$ has support contained in $Y$, so $X_f\subseteq U$, so $X_f=X_f\cap U=U_f$ is an affine open neighborhood of $U$.

	Thus, we have produced an affine open cover $\{X_f\}$ of $X$. Because $X$ is Noetherian, we see that $X$ is quasicompact, so we can reduce this to a finite affine open cover $\{X_{f_i}\}_{i=1}^r$. By \Cref{lem:affine-cover-is-affine}, it suffices to show that the $f_i$ generate $\Gamma(X,\OO_X)$. Well, each $f_i$ produces some map $\OO_X\to\OO_X$, so we get a map
	\[\pi\colon\OO_X^r\to\OO_X\]
	given by $(a_1,\ldots,a_r)\mapsto a_1f_1+\cdots+a_rf_r$. Notably, because the $X_{f_i}$ cover $X$, we see that this map is surjective by checking at stalks because we do not have $f_i|_p\in\mf m_p$ for all $i$ at any fixed point $p$. Letting $\mc F\coloneqq\ker\pi$, we get the short exact sequence
	\[0\to\mc F\to\OO_X^r\stackrel\pi\to\OO_X\to0,\]
	so it now suffices to show that $H^1(X,\ker\pi)$ because this will give us surjectivity on the right upon taking global sections. Well, we consider the filtration
	\[\mc F=\mc F\cap\OO_X^r\supseteq\mc F\cap\OO_X^{r-1}\supseteq\cdots\supseteq\mc F\cap\OO_X\supseteq0.\]
	Note that coherent sheaves are closed under taking kernels and finite products, so each of these intersections remains a coherent sheaf. Well, we do this by induction. Note $\mc F\cap\OO_X$ is a coherent sheaf of ideals, so its $H^1$ vanishes by hypothesis; this is our base case. To go up, we look at the following short exact sequence.
	\[0\to\mc F\cap\OO_X^n\to\mc F\cap\OO_X^{n+1}\to\mc C_n\to0.\]
	Thus, it now suffices to show that $\mc C_n$ has vanishing $H^1$ by the long exact sequence. Well, we note that we have a morphism of exact sequences.
	% https://q.uiver.app/?q=WzAsMTAsWzEsMCwiXFxtYyBGXFxjYXBcXE9PX1hebiJdLFsyLDAsIlxcbWMgRlxcY2FwXFxPT19YXntuKzF9Il0sWzMsMCwiXFxtYyBDX24iXSxbMSwxLCJcXE9PX1hebiJdLFsyLDEsIlxcT09fWF57bisxfSJdLFszLDEsIlxcT09fWCJdLFswLDAsIjAiXSxbNCwwLCIwIl0sWzQsMSwiMCJdLFswLDEsIjAiXSxbNiwwXSxbMCwxXSxbMSwyXSxbMiw3XSxbOSwzXSxbMyw0XSxbNCw1XSxbNSw4XSxbMiw1LCIiLDEseyJzdHlsZSI6eyJ0YWlsIjp7Im5hbWUiOiJob29rIiwic2lkZSI6InRvcCJ9fX1dLFsxLDQsIiIsMSx7InN0eWxlIjp7InRhaWwiOnsibmFtZSI6Imhvb2siLCJzaWRlIjoidG9wIn19fV0sWzAsMywiIiwxLHsic3R5bGUiOnsidGFpbCI6eyJuYW1lIjoiaG9vayIsInNpZGUiOiJ0b3AifX19XV0=&macro_url=https%3A%2F%2Fraw.githubusercontent.com%2FdFoiler%2Fnotes%2Fmaster%2Fnir.tex
	\[\begin{tikzcd}
		0 & {\mc F\cap\OO_X^n} & {\mc F\cap\OO_X^{n+1}} & {\mc C_n} & 0 \\
		0 & {\OO_X^n} & {\OO_X^{n+1}} & {\OO_X} & 0
		\arrow[from=1-1, to=1-2]
		\arrow[from=1-2, to=1-3]
		\arrow[from=1-3, to=1-4]
		\arrow[from=1-4, to=1-5]
		\arrow[from=2-1, to=2-2]
		\arrow[from=2-2, to=2-3]
		\arrow[from=2-3, to=2-4]
		\arrow[from=2-4, to=2-5]
		\arrow[hook, from=1-4, to=2-4]
		\arrow[hook, from=1-3, to=2-3]
		\arrow[hook, from=1-2, to=2-2]
	\end{tikzcd}\]
	In particular, we see that $\mc C_n$ is coherent by the top, and it embeds into $\OO_X$ by the Snake lemma, so it is a coherent sheaf of ideals and therefore has vanishing $H^1$ by hypothesis.
\end{proof}

\end{document}