% !TEX root = ../notes.tex

\documentclass[../notes.tex]{subfiles}

\begin{document}

\section{February 9}
Here we go.

\subsection{Duality for Local Fields}
In light of the previous remark, it is useful to fix some characters.
\begin{notation}
	Fix a place $v$ of $\QQ$.
	\begin{itemize}
		\item If $v=p$ is finite, then we define the character $\psi_p\colon\QQ_p\to S^1$ by the composite $\QQ_p\to\QQ_p/\ZZ_p\cong\mu_{p^\infty}$, where the second isomorphism sends $a/p^n$ to $e^{2\pi ia/p^n}$.
		\item If $v=\infty$ is infinite, then we define the character $\psi_\infty\colon\RR\to S^1$ by $\psi_\infty(x)\coloneqq e^{-2\pi ix}$.
	\end{itemize}
	In general, for a place $w$ of a number field $F$ lying over place $v$ of $\QQ$, we define $\psi_v\coloneqq\psi_p\circ\tr_{F_v/\QQ_p}$.
\end{notation}
\begin{remark}
	The character $\psi_p$ has the property that it is trivial on $\ZZ_p$ but nontrivial on $p^{-1}\ZZ_p$. In general, for finite places $v$ of a field $F$, one finds that $\psi_v|_{\OO_v}=1$, but $\OO_v$ may not be the largest subgroup with this property, though this is true for all but finitely many $v$.
\end{remark}
\begin{remark}
	The choice of $\psi_\infty$ is done so that the assembled character $\psi_\QQ\colon\AA_\QQ\to S^1$ vanishes on $\QQ$. In fact, the character $\psi_F\coloneqq\prod_v\psi_v$ also multiplies to $1$ because we can just take a trace of everything to $\QQ$.
\end{remark}
Self-duality allows us to define Fourier transforms sending functions on $F_v$ to functions on $F_v$. Let's be careful about what sorts of functions we want to integrate.
\begin{definition}[Schwartz]
	Fix a local field $F_v$.
	\begin{itemize}
		\item If $F_v$ is nonarchimedean, then a \textit{Schwartz function} is a locally constant, compactly supported function $F_v\to\CC$.
		\item If $F_v$ is archimedean, then a \textit{Schwartz function} is a smooth function, all of whose derivatives vanish faster than a polynomial at $\infty$.
	\end{itemize}
	The space of all Schwartz functions is denoted $\mc S(F_v)$.
\end{definition}
\begin{notation}
	Fix a Schwartz $f$ on a local field $F_v$. Then we define the \textit{Fourier transform} by
	\[\widehat f(y)\coloneqq\int_{F_v}f(x)\psi_v(xy)\,d_vx.\]
\end{notation}
\begin{remark}
	One can check that the Fourier transform of a Schwartz function is Schwartz. We could of course integrate any function in $L^2(F_v)$, but we will not have a reason to.
\end{remark}
\begin{remark}
	By \Cref{rem:self-dual-haar}, there is a unique choice of Haar measure $d_vx$ so that the second Fourier transform of a function $f$ is $f(-x)$. For $F_v=\RR$, this turns out to be the Lebesgue measure, and for $F_v=\CC$, this turns out to be twice the Lebesgue measure. One can check this by plugging in the function $f$ to be a Gaussian, which turns out to be self-dual for the given measures.
\end{remark}
The previous remark (in the case of $\CC$) has indicated that the self-dual Haar measure is potentially interesting. Let's explain what we receive for nonarchimedean local fields.
\begin{example}
	For $\QQ_p$, one can check that $\int_{\ZZ_p}d_px=1$. Indeed, suppose that $V=\int_{\ZZ_p}d_px$, and consider the indicator function $1_{\ZZ_p}$. Then the Fourier transform
	\[\widehat1_{\ZZ_p}(y)=\int_{\ZZ_p}\psi_p(xy)\,d_px.\]
	Now, for $y\in\ZZ_p$, this integral is constant, so we receive $V$. For $y\notin\ZZ_p$, we see that this character $x\mapsto\psi_p(xy)$ is nontrivial on $\ZZ_p$, so the integral vanishes. (To be explicit, say $y=a/p^n$ where $\gcd(a,p)=1$ and $n\ge1$, and then $\int_{\ZZ_p}\psi_p(ax/p^n)\,d_px=\int_{\ZZ_p}\psi_p(a(x+1)/p^n)\,d_px=\psi_p(a/p^n)\int_{\ZZ_p}\psi_p(ax/p^n)\,d_px$. Because $\psi_p(a/p^n)\ne0$, so the integral vanishes.) Thus, $\widehat1_{\ZZ_p}=V1_{\ZZ_p}$, so the second Fourier transform is $V^21_{\ZZ_p}$. So for $d_px$ to be self-dual, we are required to have $V=1$. As a by-product, we note that we have also shown that the function $1_{\ZZ_p}$ is self-dual!
\end{example}
\begin{remark}
	On the homework, we will compute $\int_{\OO_v}d_vx$ for places $v$ lying over $p$, which is done by a similar procedure. Namely, we find that
	\[\widehat1_{\OO_v}=\int_{\ZZ_p}\psi_p(xy)\,d_px\]
	is $V$ times an indicator of the inverse different ideal
	\[\mc D_v^{-1}\coloneqq\{y\in F_v:\tr y\OO_v\subseteq\ZZ_p\}.\]
	This contains $\OO_v$, but it may in general be bigger. If $F_v/\QQ_p$ is unramified, then $\mc D_v^{-1}=\ZZ_p$, so in fact our self-dual measure should give $\int_{\OO_v}d_vx=1$. Otherwise, one finds that setting $\int_{\OO_v}d_vx=1$ need not be self-dual: one needs to multiply or divide by some square root of the index of $\mc D_v$.
\end{remark}

\subsection{Duality for the Ad\'eles}
We are now ready to put a measure on $\AA_F$.
\begin{definition}[Schwartz]
	Fix a number field $F$. A function $f\colon\AA_F\to\CC$ is \textit{Schwartz} if and only if it lives in the restricted tensor product
	\[\mc S(\AA_F)\coloneqq\bigotimes_v(\mc S(F_v),1_{\OO_v}),\]
	where it is restricted in the sense that almost all factors of a pure tensor are equal to $1_{\OO_v}$.
\end{definition}
\begin{definition}
	Fix a number field $F$. Then we define a measure $dx$ on $\AA_F$ to be the product $\prod_vd_vx$, where the measurable subsets consist of finite unions of basic open sets (except at the infinite places).
\end{definition}
\begin{remark}
	This measure is well-defined because any subset in the Borel algebra will produce a factor $\OO_v$ at all but finitely many places $v$, and all but finitely many of those places have $\int_{\OO_v}d_vx=1$, so the entire product turns out to be finite on any Borel subset.
\end{remark}
\begin{remark}
	Let's be more precise about this construction. There is a unique Haar measure $dx$ on $\AA_F$ such that the measure of $\prod_{v\nmid\infty}\OO_v\times\prod_{v\mid\infty}B_v(0,1)$ is the expected product of the given measures. The character $\psi_F\colon\AA_F\to S^1$ produces an isomorphism $\AA_F\to\AA_F^*$ (by gluing together the local isomorphisms), and the corresponding self-dual measure can be checked to be $dx$ by computing the Fourier transform of the indicator of $\bigotimes_vf_v$, where $f_v=1_{\OO_v}$ at finite places and the Gaussian at infinite places.
\end{remark}
The reason we want to be able to work globally is that there is a Poisson summation formula for the subgroup $F\subseteq\AA_F$.
\begin{theorem}[Poisson summation]
	Fix a number field $F$. For $f\in\mc S(\AA_F)$,
	\[\sum_{x\in F}f(x)=\sum_{y\in F}\widehat f(y).\]
\end{theorem}
Let's explain the general story here.
\begin{definition}[cocompact]
	Fix a topological abelian group $X$. A closed subgroup $\Gamma\subseteq X$ is \textit{cocompact} if and only if the quotient $\Gamma\backslash X$ is compact.
\end{definition}
\begin{remark}
	It turns out that $X$ is compact if and only if $X^*$ is discrete. Thus, if $\Gamma\subseteq X$ is discrete and cocompact, then the dual subgroup
	\[\Gamma^\perp\coloneqq\{\chi\in X^*:\chi|_\Gamma=1\}\]
	is a discrete cocompact subgroup of $X^*$.
\end{remark}
\begin{example}
	The subgroup $\ZZ\subseteq\RR$ is discrete and cocompact because $\RR/\ZZ$ is the circle group. Upon identifying $\RR^*$ with $\RR$ via the character $\psi_\infty$, the dual subgroup $\ZZ^*$ of $\RR$ is exactly $\ZZ$: indeed, we are asking for $x\in\RR$ for which the character $y\mapsto\psi_\infty(xy)$ is trivial on $\ZZ$, which is equivalent to having $x\in\ZZ$ because $\psi_\infty(xy)=e^{-2\pi ixy}$.
\end{example}
\begin{theorem}[Poisson summation]
	Fix a locally compact topological group $X$, and let $\Gamma\subseteq X$ be a discrete cocompact subgroup. For any $f\in L^2(X)$, we have
	\[\op{vol}(\Gamma\backslash X;dx)\sum_{x\in\Gamma}f(x)=\sum_{y\in\Gamma^\perp}\widehat f(y)\]
	provided that the left-hand side converges absolutely and uniformly. Here, $X^*$ has been given the dual measure of \Cref{thm:fourier-inversion}.
\end{theorem}
\begin{proof}
	The idea is to consider the function
	\[\varphi(x)=\sum_{\gamma\in\Gamma}f(x+\gamma).\]
	Provided convergence, this function descends to a function on $\Gamma\backslash X$. This then has a Fourier transform $\widehat\varphi$, which is a function on $(\Gamma\backslash X)^*=\Gamma^\perp$. Fourier inversion now provides the equality
	\[\varphi(0)=\sum_{y\in\Gamma^\perp}\widehat\varphi(y).\]
	We will be done as soon as we can check that $\widehat\varphi(y)=\widehat f(y)$, which is some explicit calculation. Namely, the self-duality $X\cong X^*$ comes from a pairing $X\times X\to S^1$, which gives our Fourier transform the form
	\begin{align*}
		\widehat f(y) &= \int_Xf(x)\langle x,y\rangle\,dx \\
		&\stackrel*= \int_{\Gamma\backslash X}f(x)\langle x,y\rangle\,dx \\
		&= \widehat\varphi(y),
	\end{align*}
	where $\stackrel*=$ holds because the inner product $\langle x,y\rangle$ only depends on the class in $\Gamma\backslash X$ (because $y\in\Gamma^\perp$).
\end{proof}
\begin{remark}
	If we have an isomorphism $X\cong X^*$ which sends $\Gamma$ to $\Gamma^\perp$, and we choose a self-dual Haar measure under the isomorphism, then one can check that the induced volume of $\Gamma\backslash X$ is $1$ by plugging in $f$ and its Fourier transform into the Poisson summation formula!
\end{remark}
\begin{example}
	For the application to $F\subseteq\AA_F$, one needs to check that $F^\perp\subseteq\AA_F$ is identified with $F$ in the isomorphism $\AA_F\cong\AA_F^*$ given by the character $\psi_F$. Certainly $F\subseteq F^\perp$ because $\psi_F(a)=1$ for all $a\in F$. For the other inclusion, we see that $F^\perp\subseteq\AA_F$ is a discrete subgroup of $\AA_F$ including $F$, which we show must be $F$ on the homework. (Here, $F^\perp$ is discrete because its dual is the compact quotient $\AA_F/F$.)
\end{example}
\begin{remark}
	One should also check that the function
	\[y\mapsto\sum_{x\in F}f(x+y)\]
	converges absolutely and uniformly for any $f\in\mc S(\AA_F)$. Well, we may descend to a function of the form $\prod_vf_v$, so we are looking at some indicator on a set which is a product of compacts. Using the finite places, we see that we are requiring some bounded valuation at every place, so we are summing over a fractional ideal. But $f$ is Schwartz at the infinite places, so the desired converges follows.
\end{remark}

\subsection{Duality for Function Fields}
Let's say something about duality for function fields $F=\FF_q(X)$. Then the correct dual object for $\AA_F$ turns out to be differential forms.
\begin{notation}
	Fix a function field $F$. Then we define
	\[\AA_\omega\coloneqq\prod_v\left(\omega_{F_v},\omega_{\OO_v}\right).\]
	Here, $\omega$ is the $F$-bundle of $1$-forms on $X$, $\omega_v$ is the stalk at $v$, $\omega_{\OO_v}$ is its completion, and $\omega_{F_v}$ is the base-change $\omega_{\OO_v}\otimes_{\OO_v}F_v$.
\end{notation}
Let's see how this produces duality.
\begin{definition}[residue]
	Fix a closed point $v$ of a smooth projective geometrically connected curve $X$. Then we define the \textit{residue} $\omega_{F_v}\to\FF_q(v)$ defined as follows: for a differential form $\theta$, choose a local coordinate $dt$ and expand
	\[\theta=\sum_{n\in\ZZ}a_nt^n\,dt.\]
	Then the residue is $\op{res}_v\theta\coloneqq a_{-1}$.
\end{definition}
\begin{remark}
	It turns out that the residue is independent of the choice of coordinate $t$. Indeed, any other coordinate is of the form $s=ut$ for a unit $u$, and we see that $dt/t=ds/s$.
\end{remark}
Here is the local duality.
\begin{proposition}
	Fix a function field $F=\FF_q(X)$, and choose a nontrivial character $\psi\colon\FF_p\to S^1$. Then the pairing $F_v\times\omega_{F_v}\to S^1$ defined by
	\[\langle f,\theta\rangle_v\coloneqq\psi(\tr_{\FF_q(v)/\FF_p}f\theta)\]
	realizes an isomorphism $F_v^*\cong\omega_{F_v}$. Furthermore, $\OO_v^\perp$ is identified with $\omega_{\OO_v}$.
\end{proposition}
Here is the global duality, which we prove on the homework.
\begin{proposition}
	Fix a function field $F=\FF_q(X)$, and choose a nontrivial character $\psi\colon\FF_p\to S^1$. Then the product pairing $\AA_F\times\AA_{\omega_F}\to S^1$ defined by
	\[(a,\theta)\mapsto\prod_v\langle a_v,\theta_v\rangle_v\]
	produces an isomorphism $\AA_F^*\to\AA_{\omega_F}$.
\end{proposition}
\begin{remark}
	A choice of $\theta\in\omega_F$ allows us to identify $\AA_{\omega_F}$ with $\AA_F$ and thus identify $\AA_F$ with itself. This is analogous to choosing a full character of $\QQ_p$ for each $p$.
\end{remark}
\begin{remark}
	On the homework, we will show that $F^\perp=\omega_F$. The inclusion $\omega_F\subseteq F^\perp$ follows from the residue theorem: for any $\theta\in\omega_F$, we have
	\[\sum_v\tr_{\FF_q(v)/\FF_q}\op{res}_v\theta=0.\]
	The other inclusion uses a compactness argument.
\end{remark}
Now that we are refusing to write down any self-dualities, our choices of Haar measure now depend on a scalar. Instead, we will take by convention that $\op{vol}(\OO_v,d_vx)=1$ for each $v\in X$, and we take the product to produce a measure on $\AA_F$.
\begin{example}
	Recall that there is an exact sequence
	\[\FF_q\to\prod_v\OO_v\to F\backslash\AA_F\to\mathrm H^1(X;\OO_X)\to0.\]
	This shows that the measure of $F\backslash\AA_F$ is $q^{g-1}$.
\end{example}
\begin{example}
	Analogously, one can show that there is an identification of
	\[\omega_F\backslash\AA_{\omega_F}\mathbin{\bigg/}\prod_v\OO_v\]
	with $\mathrm H^1(X;\OO_X)$, and there is a stabilizer of $\mathrm H^0(X;\OO_X)$. Thus, $\omega_F\backslash\AA_{\omega_F}$ has measure $q^{1-g}$.
\end{example}
\begin{remark}
	It turns out that the corresponding Poisson summation formula is an incarnation of the Riemann--Roch formula, if one plugs in a special choice of function.
\end{remark}

\end{document}