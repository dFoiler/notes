% !TEX root = ../notes.tex

\documentclass[../notes.tex]{subfiles}

\begin{document}

\section{February 11}
Here we go.

\subsection{Multiplicative Measures}
We are now ready to move from ad\'eles to id\'eles. Let's start by trying to fix a measure.
\begin{remark}
	Fix some local field $F_v$. By definition of $\left|\cdot\right|_v$, we see that $d_vx/\left|\cdot\right|_v$ is a Haar measure on $F_v^\times$.
\end{remark}
Now, for a global field $F$, one may attempt to put a measure on $\AA_F^\times$ by multiplying together all the local measures. However, we are going to want to integrate indicators on basic open subsets of $\AA_F^\times$. For example, we could try to integrate $\prod_{v\nmid\infty}1_{\OO_v}\prod_{v\mid\infty}1_{B_v(0,1)}$, whose integral will be a scalar times
\[\prod_{v\nmid\infty}\int_{\OO_v^\times}d_v^\times x.\]
But this integral is $\mu_v(\OO_v)-\mu_v(\varpi_v\OO_v)=\left(1-\left|\varpi_v\right|_v\right)\mu_v(\OO_v)$. For all but finitely many $v$, we see that $\mu_v(\OO_v)=1$, so we see that the above product has all but finitely many of its factors not equal to $1$! In fact, it vanishes for $F=\QQ$. Thus, this normalization is not suitable for our purposes. Instead, we divide out by this factor $1-\left|\varpi_v\right|_v$.
\begin{definition}
	Fix a nonarchimedean local field $F_v$. Then we define
	\[d_v^\times x\coloneqq\frac1{1-q_v^{-1}}\cdot\frac{d_vx}{\left|x\right|_v},\]
	and we define $d^\times x$ on $\AA_F^\times$ as the product measure.
\end{definition}
\begin{remark}
	It follows from the preceding calculation that $d^\times x$ is a finite product on basic open subsets.
\end{remark}
An interesting question is to calculate the volume of $F^\times\backslash\AA_F^{\times,1}$ according to the measure $d^\times x$. For number fields $F$, recall from the proof of \Cref{thm:idele-class-group} that we have an exact sequence
\[1\to\mu(F)\to\OO_F^\times\to(F\otimes_\QQ\RR)^{\times,1}\to F^\times\backslash\AA_F^{\times,1}/\prod_{v\nmid\infty}\OO_v^\times\to\op{Cl}F\to0.\]
One eventually finds the following.
\begin{proposition}
	Fix a number field $F$ with signature $(r_1,r_2)$. Then the volume of $F^\times\backslash\AA_F^{\times,1}$ is
	\[\frac{h_F\op{Reg}_F}{w}\cdot\frac{2^{r_1}(2\pi)^{r_2}}{\sqrt{\left|\op{disc}\OO_F\right|}}.\]
	Here, $h_F$ is the class number, $\op{Reg}_F$ is the regulator (which is an appropriately measured covolume of $\OO_F^\times$ sitting in $(F\otimes_\QQ\RR)^{\times,1}$), and $w$ is the number of roots of unity.
\end{proposition}
\begin{remark}
	The second factor basically arises from how we have chosen our additive measures.
\end{remark}
We will shortly see that the volume of $F^\times\backslash\AA_F^{\times,1}$ is also related to the Dedekind $\zeta_F$-function, thereby proving the analytic class number formula.
\begin{defihelper}[Dedekind $\zeta$-function] \nirindex{Dedekind zeta function@Dedeking $\zeta$-function}
	Fix a global field $F$. Then we define the \textit{Dedekind $\zeta_F$-function} as
	\[\zeta_F(s)\coloneqq\sum_{I\subseteq\OO_F}\frac1{\op N(I)}.\]
	This sum converges absolutely and uniformly on compacts for $\op{Re}s>1$.
\end{defihelper}
\begin{remark}
	Unique prime factorization of ideals produces an Euler product
	\[\zeta_F(s)=\prod_{v\nmid\infty}\frac1{1-q_v^{s}}.\]
	This is remarkable because it looks like ``$\zeta_F(1)$'' is the scale factor between the failed ``product'' measure $dx/\left|x\right|$ on $\AA_F^\times$ and our successful measure $d^\times x$.
\end{remark}

\subsection{Our \texorpdfstring{$L$}{ L}-functions}
The goal of Tate's thesis is to reprove some general results on the functional equations of $L$-functions. For example, we will be able to reprove the functional equation of the Dirichlet $L$-functions, defined by
\[L(s,\chi)\coloneqq\sum_{\substack{n=1\\\gcd(n,N)=1}}^\infty\frac{\chi(n)}{n^s},\]
where $\chi\colon(\ZZ/N\ZZ)^\times\to\CC^\times$ is some character.
\begin{example}
	Taking $\chi=1$ (and $N=1$) recovers Riemann's $\zeta$-function.
\end{example}
Roughly speaking, our functional equation will equate $L(s,\chi)$ and $L(1-s,\ov\chi)$, after we ``fix'' these $L$-functions slightly.

We will be able to work over general global fields. Let's start by relating the discussion of the previous example with our ad\'elic discussion.
\begin{example}
	It turns out that $\AA_\QQ^\times=\QQ^\times\times\RR^+\times\widehat{\ZZ}^\times$ (recall $\widehat{\ZZ}=\prod_p\ZZ_p$), which one can see by taking some successive quotients and appealing to \Cref{thm:idele-class-group}. The moral is that a Dirichlet character $(\ZZ/N\ZZ)^\times\to\CC^\times$ can be viewed as a continuous character $\AA_\QQ^\times\to\CC^\times$ which ``happens'' to factor through $\widehat{\ZZ}^\times$.
\end{example}
\begin{definition}[id\'ele class character]
	Fix a global field $F$. Then an \textit{id\'ele class character} is a character $\chi\colon F^\times\backslash\AA_F^\times\to\CC^\times$. It is \textit{unitary} if its image is contained in $S^1$.
\end{definition}
\begin{remark}
	The unitary characters (by definition) live in the Pontryagin dual of $F^\times\backslash\AA_F^\times$.
\end{remark}
\begin{example}
	Note that there is a norm character $\left|\cdot\right|\colon F^\times\backslash\AA_F^\times\to\RR^+$. In fact, for any $s\in\RR$, we receive a character $\left|\cdot\right|^s$.
\end{example}
% \begin{remark} \label{rem:unitary-norm-decomposition}
% 	For any id\'ele class character $\chi$, we note that $\left|\chi\right|=\left|\cdot\right|^s$ for some unique real number $s$.\todo{} Thus, we may write
% 	\[\chi=\frac{\chi}{\left|\chi\right|}\cdot\left|\chi\right|,\]
% 	which decomposes $\chi$ into a product of a unitary character and a power of the norm. This decomposition is unique! In this way, we can give the space of id\'ele class characters a topology.
% \end{remark}
\begin{remark}
	Suppose $F$ is a number field. Then $\left|\cdot\right|\colon F^\times\backslash\AA_F^\times\to\RR^+$ is surjective, so we can take $\op{Hom}(-,\CC^\times)$ of the short exact sequence
	\[1\to F^\times\backslash\AA_F^{\times,1}\to F^\times\backslash\AA_F^\times\to\RR^+\to1\]
	to see that $\left(F^\times\backslash\AA_F^\times\right)^*$ is an extension of $\op{Hom}(\RR^+,\CC^\times)=\CC$ by a discrete group $\left(F^\times\backslash\AA_F^{\times,1}\right)^*$. (We have used the compactness of $F^\times\backslash\AA_F^{\times,1}$ to show that any map to $\CC^\times$ factors through $S^1$. Note that exactness holds on the right after the duality because already $(-)^*$ is exact.) We are thus able to conclude that the space of id\'ele class characters inherits a topology of a complex manifold. %; in fact, by gluing the aforementioned $i\RR$ by the $\RR$ in \Cref{rem:unitary-norm-decomposition}, we get a complex manifold: it is just a discrete disjoint union of many copies of $\CC$, indexed by the choice of unitary character up to a power of $\left|\cdot\right|^{is}$.
\end{remark}
\begin{remark}
	Suppose $F$ is a function field $\FF_q(X)$. Then the norm $\left|\cdot\right|\colon F^\times\backslash\AA_F^\times\to\RR^+$ surjects instead onto the discrete group $q^\ZZ$. We thus see that $\op{Hom}\left(F^\times\backslash\AA_F^\times,\CC^\times\right)$ is an extension of $\op{Hom}(\ZZ,\CC^\times)=\CC^\times$ by a discrete group. The space of id\'ele class characters continues to be a complex manifold by the same argument.
\end{remark}
\begin{remark}
	By continuity of $\chi$, we see that $\chi|_{\OO_v^\times}=1$ for almost all $v$. Indeed, this follows from a ``no small subgroups'' argument applied to continuous maps on the group $\prod_{v\nmid\infty}\OO_v^\times$.
\end{remark}
\begin{definition}[unramified]
	An id\'ele class character $\chi$ is \textit{unramified} at a finite place $v$ if and only if $\chi|_{\OO_v}=1$; otherwise, we see that it is \textit{ramified} at $v$.
\end{definition}
We are now ready to define our $L$-functions.
\begin{definition}
	Fix an id\'ele class character $\chi$ of a global field $F$, and choose a finite place $v$. Then we define
	\[L_v(\chi_v)\coloneqq\begin{cases}\frac1{1-\chi_v(\varpi_v)} & \text{if }\chi_v|_{\OO_v}=1, \\0&\text{otherwise},\end{cases}\]
	and
	\[L(\chi)\coloneqq\prod_{v\nmid\infty}L_v(s,\chi_v).\]
	We may write $L_v(s,\chi_v)$ and $L(s,\chi)$ for $L_v\left(\chi_v\left|\cdot\right|_v^s\right)$ and $L\left(\chi\left|\cdot\right|^s\right)$, respectively.
\end{definition}
\begin{remark}
	We can expand $L(\chi)$ out as a sum
	\[L(\chi)=\sum_{I\subseteq\OO_F}\frac{\chi(I)}{\op N(I)},\]
	where $\chi(I)$ means $\prod_v\chi(\varpi_v)^{\nu_v(I)}$, and $\chi(\varpi_v)$ in this expression means $0$ if $\chi$ is ramified at $v$.
\end{remark}
\begin{remark}
	If $\chi$ is unitary, then the function $s\mapsto L(s,\chi)$ can be checked to converge absolutely and uniformly on compacts in the region $\Re s>1$.
\end{remark}
\begin{example}
	Taking $\chi=1$ recovers Dedekind $\zeta$-functions.
\end{example}

\subsection{Global Duality}
Tate's main global result is a duality statement.
\begin{definition}[global integral]
	Fix an id\'ele class character $\chi$ on a number field $F$. For $f\in\mc S(\AA_F)$, we define
	\[Z(\chi,f)\coloneqq\int_{\AA_F^\times}\chi(a)f(a)\,d^\times a.\]
	We may also write $Z(s,\chi,f)\coloneqq Z(\chi\left|\cdot\right|^s,f)$.
\end{definition}
\begin{theorem}[Tate]
	Fix a global field $F$ on a number field $F$ and some $f\in\mc S(\AA_F)$. The function $\chi\mapsto Z(\chi,f)$ admits a meromorphic continuation and a functional equation
	\[Z(\chi,f)=Z\left(\chi^{-1}\left|\cdot\right|,\widehat f\right).\]
\end{theorem}
\begin{remark}
	Equivalently, one may say that the function $s\mapsto Z(s,\chi,f)$ admits a meromorphic continuation for any choice of $\chi$.
\end{remark}
Let's explain how this relates to our $L$-functions.
\begin{theorem}
	Fix a number field $F$. Then there is a functional equation relating $\zeta_F(s)$ and $\zeta_F(1-s)$.
\end{theorem}
\begin{proof}
	The idea is to make $f$ and $\widehat f$ the same at almost all places. Define a function $(f_v)$ in $\mc S(\AA_F)$ as follows.
	\begin{itemize}
		\item For finite $v$, we define $f_v=1_{\OO_v}$. In particular, at unramified places $v$, we recall that $1_{\OO_v}$ is self-dual.
		\item If $F_v=\RR$ and $\chi_v=1$, then the function $e^{-\pi x^2}$ is self-dual. The case where $\chi_v=\op{sgn}$ takes the function $xe^{-\pi x^2}$.
		\item If $F_v=\CC$ and $\chi_v=1$, then the function $e^{-\pi\left|z\right|^2}$ is self-dual. If $\chi_v$ is more general, then some slightly different recipe is used.
	\end{itemize}
	One can now compute
	\[Z(s,1,f) = \prod_v\int_{F_v^\times}\left|a_v\right|^sf_v(a_v)\,d^\times_va.\]
	If $v$ is finite, then we are being asked to compute
	\begin{align*}
		\int_{\OO_v\setminus\{0\}}\left|a\right|^s\,d^\times_va &= \sum_{n\ge0}^\infty\int_{\varpi^n\OO_v}\left|a\right|^s\,d^\times_va \\
		&= \sum_{n\ge0}q_v^{-ns}\op{vol}\left(\OO_v^\times\right) \\
		&= \frac{\op{vol}\left(\OO_v^\times\right)}{1-q_v^{-s}}.
	\end{align*}
	If $v$ is unramified over $\QQ_p$, then the volume on top is $1$; if it is ramified, then we are computing some square root of the norm of the different (which is the discriminant). Thus, up to these contributions from rational factors, we find that $Z(s,1,f)$ is
	\[\zeta_F(s)\prod_{\text{real }v}\pi^{-s/2}\Gamma(s/2)\prod_{\text{complex }v}2(2\pi)^{1-s}\Gamma(s).\]
	We are thus able to produce a functional equation for $\zeta_F(s)$. We did not work this out in class, and I do not have time to do it on my own currently.
\end{proof}
For function fields $F=\FF_q(X)$, we have been careful to avoid identifying $\AA_F$ with itself, so we don't have a self-duality.
\begin{example}
	In this case, note that the Euler product of $\zeta_F(s)$ expands into
	\[\sum_{\text{effective }D\subseteq X}q^{-s\deg D}=\sum_{d\ge0}q^{-ds}\cdot\#X^{(d)}(\FF_q),\]
	where $X^{(d)}(\FF_q)$ refers to the number of effective divisors on $D$ of degree $d$ defined over $\FF_q$. We have used the notation $X^{(d)}$ to indicate that this could be thought of as the stack $X^d/\Sigma_d$.
\end{example}
\begin{example}
	Suppose that $\chi$ is an unramified id\'ele class character, meaning that it factors through $F^\times\backslash\AA_F^\times/\prod_v\OO_v^\times$, which we recall is $\op{Pic}X$. One can calculate as before that
	\[L(s,\chi)=\sum_{D\subseteq X}q^{-s\deg D}\chi(\OO_X(D)).\]
	Further suppose that $\chi|_{\op{Pic}^0X}\ne1$. Then we claim that $L(s,\chi)$ is a polynomial in $q^{-s}$. Indeed, in light of the above expansion, it is enough to show that all line bundles of large degree $d$ appear an equal number of times as $\OO_X(D)$ as $D$ varies over effective divisors of $d$. But this is true because $d>2g-2$ makes $X^{(d)}$ a $\PP^{d-g}$-bundle over $\op{Pic}^dX$: by an argument with linear systems, the fibers are all copies of $\PP^{d-g}=\PP\mathrm H^0(X;\OO_X(D))$.
\end{example}

\end{document}