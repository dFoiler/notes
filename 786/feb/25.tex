% !TEX root = ../notes.tex

\documentclass[../notes.tex]{subfiles}

\begin{document}

\section{February 25}
Today we continue our nonarchimedean representation theory.

\subsection{Representations of the Hecke Algebra}
Let's take a moment to explain how $\mc H$ acts on representations.
\begin{lemma} \label{lem:hecke-action}
	Fix a totally disconnected group $G$. There is a fully faithful embedding $\op{Rep}_\CC^{\mathrm{sm}}(G)\to\op{Mod}(\mc H)$.
	% Fix a smooth representation $V$ of a totally disconnected group $G$.
	% \begin{listalph}
	% 	\item Then $\mc H$ admits a natural action on $V$.
	% 	\item For any $v\in V$, there is a compact open subgroup $K\subseteq G$ for which $e_Kv=v$.
	% \end{listalph}
\end{lemma}
\begin{proof}
	For (a), the action is given by
	\[\mu\cdot v\coloneqq\int_Ggv\,d\mu(g).\]
	Note that the integral always converges because $\mu$ has compact support. By an explicit calculation with $f\,dg$s, one can check that this action makes $V$ into an $\mathcal H$-module. This functor is of course faithful (because linear maps are just sent to themselves).

	It remains to check that our functor is full. Fix a linear map $\varphi$ between two smooth representations $V$ and $W$, and suppose that $\varphi$ preserves the $\mc H$-module structure. We would like to show that $g\varphi(v)=\varphi(gv)$ for any $g\in G$ and $v\in V$. Well, for any sufficiently small compact open subgroup $K$ of $g$, we see that $e_Kv=v$, so the given equation is equivalent to $\varphi$ preserving the action by the translated measure $ge_K$.
\end{proof}
\begin{remark}
	Further, by shrinking $K$, we may assume that $v\in V^K$ and that $\mu=f\,dg$ where $f$ is $K$-invariant. It follows that
	\[\mu\cdot v=\sum_{g\in G/K}f(g)\op{vol}(K;dg)(gv)\]
	by factoring the integral.
\end{remark}
We use this to give a definition of the ``smooth'' $\mc H$-modules.
\begin{definition}[smooth]
	Fix a totally disconnected group $G$. Then an $\mathcal H$-module $M$ is \textit{smooth} if and only if
	\[M=\bigcup_{K\subseteq G}e_KM.\]
\end{definition}
\begin{remark}
	If $V$ is a smooth representation of $G$, then one can check that $e_K$ projects $V$ onto $V^K$. Indeed, $e_K$ certainly fixes $V^K$, and for any $k\in K$ and $v\in V$, we see that $k(e_Kv)=e_Kv$ by pushing the linear map $k$ through the integral
	\[e_Kv=\frac1{\op{vol}(K;dg)}\int_Kgv\,dg.\]
	Thus, smooth representations of $G$ go to smooth $\mc H$-modules.
\end{remark}
\begin{remark}
	Conversely, given a smooth $\mc H$-module $M$, we can write
	\[M=\colim_{K\subseteq G}e_KM.\]
	Then each $e_KM$ upgrades from an $\mathcal H$-module to a representation of $G/K$ and thus by inflation to a representation of $G$. Thus, the colimit upgrades to a smooth representation of $G$ because the category of smooth representations is closed under such colimits. This construction provides an inverse functor to the one in \Cref{lem:hecke-action}, thereby providing an equivalence of categories from smooth representations to smooth modules.
\end{remark}
It is notable that all of our discussion can also be made to work globally for function fields.
\begin{example}
	If $G$ is an affine algebraic group over a function field $F$, then $\AA_F$ is totally disconnected, so $G(\AA_F)$ is a totally disconnected. For example, it turns out that
	\[\op{GL}_n(\AA_F)=\prod_v(\op{GL}_n(F_v),\op{GL}_n(\OO_v)),\]
	which one can see directly from the construction of the topology on $G(\AA_F)\subseteq\AA_F^{n^2+1}$. The basic open subsets of $\op{GL}_n(\AA_F)$ can thus be indexed by effective divisors $D$ on $X$: indeed, given such a $D=\sum_vn_vv$, we can define $K_D$ as the kernel of the natural projection
	\[\prod_v\op{GL}_n(\OO_v)\onto\prod_{n_v\ne0}\op{GL}_n(\OO_v/\varpi_v^{n_v}).\]
	In particular, because this map is surjective, we see that $K_D$ has finite index in $\prod_v\op{GL}_n(\OO_v)$, where the index is given by the size of the right-hand group.
\end{example}
\begin{remark}
	Fix an affine algebraic group $G$ over a global field $F$. Then one may want to write down a decomposition $G(\AA_F)$ as a restricted product, but this requires us to make sense of $G(\OO_v)$ for all but finitely many places $v$. Luckily, this is not too hard: one simply can choose any model $\mc G$ over $\Spec\OO_F$, and then we pass to the open subset of $\Spec\OO_F$ where $G$ remains an affine algebraic group. Because any two models $\mc G$ and $\mc G'$ are generically isomorphic to $G$ (namely, over $F$), one can spread out the composite isomorphism $\mc G_F\to G\from\mc G'_F$ to show that $\mc G$ and $\mc G'$ are isomorphic at all but finitely many places. Thus, the restricted product $\prod_v(G(F_v),\mc G(\OO_v))$ does not depend on the choice of model $\mc G$.
\end{remark}
Thus, for an affine algebraic group $G$ over a function field $F$, one has a Hecke algebra $\mc H_{G(\AA_F)}$. We will later want a Hecke algebra also for number fields, which requires us to find a Hecke algebra for archimedean groups. We will return to this later.

Let's do some more representation theory.
\begin{proposition}[Schur's lemma] \label{prop:schur}
	Fix admissible irreducible representations $V_1$ and $V_2$ of a totally disconnected group $G$. Then
	\[\dim_\CC\op{Hom}_G(V_1,V_2)=1_{V_1\cong V_2}.\]
\end{proposition}
\begin{proof}
	Note that any nonzero map $V_1\to V_2$ has neither full kernel nor full cokernel, so the irreducibility implies that any nonzero map is an isomorphism. Thus, if $V_1$ and $V_2$ are not isomorphic, $\dim\op{Hom}_G(V_1,V_2)=0$.

	It remains to work in the case $V_1\cong V_2$. We will write $V$ for both representations, for brevity, so we want to show that $\op{End}_G(V)=\CC$. The previous paragraph has shown that any nonzero element of $D\coloneqq\op{End}_G(V)$ is invertible, so this is a division algebra. Furthermore, $D$ contains the scalars in $\CC$, so $D$ is a division algebra over $\CC$.

	The rest of the argument will be by counting, using the bound that $V$ has countable dimension (because it is a countable union of finite subspaces).
	\begin{itemize}
		\item On one hand, suppose for the sake of contradiction that $D\setminus\CC$ is nonempty. Because $\CC$ is algebraically closed, this means that $D$ has a transcendental element $T$ over $\CC$, so $D$ contains $\CC(T)$. But the elements $\{1/(T-a):a\in\CC\}$ are linearly independent, so we conclude that $\dim_\CC D$ is uncountable.
		\item On the other hand, $\dim_\CC D$ is countable because $\dim V$ is countable: indeed, any nonzero vector $v\in V$ generates $V$ under the $G$-action, so a $G$-equivariant map $V\to V$ is uniquely determined by the image of $v$. Thus, the map $\op{ev}_v\colon D\to V$ is injective, so $\dim_\CC D$ is countable.
	\end{itemize}
	The above two points produce contradiction if $D\ne\CC$, so we conclude that $D=\CC$.
\end{proof}
\begin{remark}
	Suppose that $G$ is a reductive group over a nonarchimedean local field $F$. It turns out that all irreducible smooth representations $G(F)$ are automatically admissible. This is difficult to prove, so we will not.
\end{remark}
\begin{remark}
	Again, suppose that $G$ is a reductive group over a nonarchimedean local field $F$. Then \Cref{prop:schur} remains true for representations over $\overline\QQ$, but a different argument is required. In short, one can still make a size argument by using the fact that the relevant universal enveloping algebra of the Lie algebra is filtered and has finitely generated associated graded algebra.
\end{remark}

\subsection{Parabolic Induction}
We now fix a reductive group $G$ over a nonarchimedean local field $F$.
\begin{definition}[parabolic]
	Fix a reductive group $G$ over a field $F$. Then a subgroup $P\subseteq G$ is \textit{parabolic} if and only if $G/P$ is proper.
\end{definition}
\begin{remark}
	It is equivalent for $P_{\overline F}$ to contain a Borel subgroup $B_{\overline F}$, which is a maximal connected solvable subgroup of $G$. For example, $G_{\overline F}/B_{\overline F}$ can be seen to be proper because this is some sort of flag variety. Indeed, $G_{\overline F}/B_{\overline F}$ is the space of Borel subgroups, which are all conjugate to each other.
\end{remark}
\begin{remark}[Levi decomposition]
	Given a parabolic subgroup $P$ with unipotent radical $U$, the quotient $L\coloneqq P/U$ is reductive. It turns out that the quotient map $P\onto L$ admits a (non-canonical) splitting, so we may view $L$ as a subgroup of $P$.
\end{remark}
\begin{example}
	A Borel subgroup $B$ of $\op{GL}_n$ is given by the upper-triangular matrices, and this is the only one up to conjugation. Thus, one can see that parabolic subgroups of $\op{GL}_n$ look like
	\[\left({\op{GL}_{n_1}}\times\cdots{\op{GL}_{n_d}}\right)B,\]
	where $(n_1,\ldots,n_d)$ is some sequence of positive integers with sum $n$.
\end{example}
Let's attempt to define parabolic induction. Choose a smooth representation $V$ of the Levi quotient $L$. Then we may inflate $V$ into a representation of $P$ via the quotient map $P\onto L$. We could attempt to define the induction to $G$ as consisting of those functions $f\colon G\to V$ such that
\[f(px)=pf(x),\]
where the $G$-action is given by $(gf)(x)\coloneqq f(xg)$. This is certainly a well-defined representation, but it has a couple defects, which we now fix.
\begin{itemize}
	\item This representation has no reason to be smooth. To fix this, we will want to work with those functions $f\colon G\to V$ admitting an open compact subgroup $K\subseteq G$ for which $kf=f$ for all $k\in K$.\footnote{This really is extracting those locally constant maps on $P(F)\backslash G(F)$ because $P(F)\backslash G(F)$ is a compact space.}
	\item It is useful to normalize by the modular character.
\end{itemize}
We thus have the following definition.
\begin{definition}[parabolic induction]
	Fix a reductive group $G$ over a nonarchimedean field $F$, and let $P$ be a parabolic subgroup with Levi quotient $L$. Given a smooth representation $V$ of $L(F)$, we define the \textit{parabolic induction} $\op{Ind}_P^GV$ to consist of those functions $f\colon G(F)\to V$ with the following two properties.
	\begin{itemize}
		\item Smooth: there is an open compact subgroup $K\subseteq G(F)$ such that $f(xk)=f(x)$ for all $x\in G(F)$ and $k\in K$.
		\item Induction: for all $p\in P(F)$ and $x\in G(F)$, we have $f(px)=\delta_P(p)^{1/2}\tau(p)f(x)$, where $\delta_P$ is the modular character.
	\end{itemize}
	The $G$-action on $\op{Ind}_P^GV$ is given by $(gf)(x)\coloneqq f(xg)$.
\end{definition}
Let's explain the presence of the modular character $\delta_P$, and thereby also recall its definition. (The following discussion works for all local fields $F$.) Note that $P(F)\backslash G(F)$ is a (nonarchimedean) manifold, so it admits an integration theory. Indeed, any manifold $X$ of dimension $n$ admits an integration theory as follows: if a top differential form $\theta\in\omega_X$ is supported on a single chart $U$ with local coordinates $\theta=f\,dx_1\land\cdots\land dx_n$, then we can define
\[\int_X\left|\theta\right|\coloneqq\int_U\left|f\right|\,dx_1\ldots dx_n.\]
This expression is independent of the choice of chart, which one can check by some Jacobian calculation. From here, one can use a partition of unity to compute $\int_X\left|\theta\right|$ even if $\theta$ is not supported on a single chart. (The expression will further be independent of a choice of partition of unity, which we can see by further refining the partition of unity given two such choices.)

We now apply our integration theory to $X\coloneqq P(F)\backslash G(F)$. The action of $P$ on $G$ is free, so we see that $T_e(P(F)\backslash G(F))=\mf p\backslash\mf g$. It follows that
\[TX=(\mf p\backslash\mf g)\times^PG.\]
In particular, $TX$ admits a global frame (by choosing a basis of $\mf p\backslash\mf g$), so the line bundle of top differential forms is trivial, which we see by writing
\[\omega_X=\land^{\dim X}(\mf p\backslash\mf g)^*\times^PG.\]
But now $P$ acts on $\land^{\mathrm{top}}(\mf p\backslash\mf g)^*$ by the scalar
\[\det(p;(\mf p\backslash\mf g)^*)=\frac{\det(p;\mf p)}{\det(p;\mf g)}.\]
The absolute value of this is denoted $\delta_P(p)$.

The relevance of $\delta_P$ to our integration theory is as follows: our discussion above explained how to integrate expressions of the form $\left|\theta\right|$ where $\theta\in\omega_X$. But we can merely view $\left|\omega_X\right|$ as some real line bundle. By choosing some top differential form, we can then identify sections of $\left|\omega_X\right|$ with functions $f\colon G(F)\to\RR$ such that
\[f(px)=\delta(p)f(x)\]
for all $p\in P$ and $x\in G$. Thus, $\op{Ind}_P^GV$ consists of sections of the line bundle $\left|\omega_X\right|^{1/2}$, which are known as ``half-densities.'' The relevance of this for us is that two half-densities $f,g\in\op{Ind}_P^GV$ will have a product which can be integrated. Eventually, this will imply that the parabolic induction of a unitary representation remains unitary.

\end{document}