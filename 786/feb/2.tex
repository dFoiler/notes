% !TEX root = ../notes.tex

\documentclass[../notes.tex]{subfiles}

\begin{document}

\section{February 2}
Here we go.

\subsection{Logistic Notes}
Here are some logistic notes.
\begin{itemize}
	\item The book \textit{Automorphic Forms and Representations} by Bump \cite{bump-forms} will be our main reference. We will focus on the third chapter.
	\item There are two quizzes, which will count as about 30\% of the final grade. The rest of the grade will come from the homework.
	\item The course requires knowledge of number theory and some representation theory. Most notably, we need control of Lie groups and Lie algebras.
\end{itemize}
The story of automorphic forms begins with modular forms. Roughly speaking, a modular form is a function on the upper-half plane which is symmetric for $\op{SL}_2(\CC)$. There is an exposition in the last chapter of \cite{course-arithmetic}. However, our story will start with $\op{GL}_1$ instead of $\op{GL}_2$. The perspective we take is Tate's thesis, who used Fourier analysis to reprove the analytic properties of the relevant (automorphic) $L$-functions.

\subsection{Places of Global Fields}
To do our Fourier analysis, we need to decompose our number field at each place, for which we will need the ring of ad\'eles.
\begin{definition}[global field]
	A \textit{global field} $F$ is either a number field or a function field. Here, a number field is a finite extension of $\QQ$, and a function field refers to the function field of a smooth, projective, geometrically connected curve $X$ over a finite field $\FF_q$.
\end{definition}
\begin{example}
	The field $\FF_q(t)$ is a global field.
\end{example}
\begin{definition}[place]
	Fix a global field $F$. Then a \textit{place} is an equivalence class of multiplicative absolute values of $F$.
	\begin{itemize}
		\item If $F$ is a number field, then the \textit{finite} (or \textit{nonarchimedean}) places are those in bijection with $\OO_F$, and the \textit{infinite places} (or \textit{archimedean}) are those in bijection with the embeddings $F\into\CC$ (up to conjugation).
		\item If $F$ is the function field of a curve $X$, then the places are in bijection with the closed points of the curve $X$.
	\end{itemize}
	We let $V(F)$ be the set of all places, and we let $V(F)_\infty$ denote the set of infinite places.
\end{definition}
\begin{example}
	For $F=\QQ$, the place at a finite prime $p$ is represented by $\left|q\right|_p\coloneqq p^{-\nu_p(q)}$, where $\nu_p(q)$ is the number of $p$s appearing in the prime factorization of $q$.
\end{example}
\begin{example}
	For $F=\FF_q(t)$, we have the curve $X=\PP^1$, so there is one place at infinity, and the rest of the points come from $\AA^1$. The places in $\AA^1$ are parameterized by the monic irreducible polynomials of $\FF_q[t]$.
\end{example}
\begin{notation}
	Fix a global field $F$. For each place $v$, we let $F_v$ be the completion of $F$ along a norm represented by $v$. We let $\OO_v$ denote the elements with norm at most $1$; we let $\OO_v$ denote the elements with norm $1$, and we let $\mf p_v$ denote the elements with norm less than $!$.
\end{notation}
\begin{remark}
	If $v$ is nonarchimedean, then it turns out that $\OO_v$ is a discrete valuation ring with maximal ideal $\mf p_v$. It also turns out that there is an exact sequence
	\[1\to\OO_v^\times\to F_v^\times\to\ZZ\to0,\]
	where the map $F_v^\times\to\ZZ$ is the valuation map.
\end{remark}
It is helpful to normalize our absolute values. Let's start with the global fields.
\begin{notation}
	Fix a place $v$ of a global field $F$. We normalize a choice of absolute value $\left|\cdot\right|_v$ as follows.
	\begin{itemize}
		\item For $F=\QQ$, each prime $p$ produces the absolute value $\left|q\right|_p\coloneqq p^{-\nu_p(q)}$. The infinite place $\infty$ produces the absolute value $\left|x\right|_\infty$ which is the usual one (in $\RR$).
		\item For a finite extension $F$ of $\QQ$, say that $v$ lies over $v_0$ of $\QQ$, and we define
		\[\left|x\right|_v\coloneqq\left|\op N_{F_v/\QQ_{v_0}}(x)\right|_{v_0}.\]
	\end{itemize}
\end{notation}
\begin{example}
	For $F_v=\CC$, we see that $\left|x\right|_v=\left|x\ov x\right|_\RR$ is the square of the usual absolute value on $\CC$. Note that this norm does not obey the triangle inequality.
\end{example}
For a function field $\FF_q(X)$, there is not a canonical embedding $\FF_q(t)$ into $\FF_q(X)$, so it does not seem suitable to proceed as above by taking norms. Instead, we normalize directly.
\begin{notation}
	Fix a function field $F$ of a smooth, projective, geometrically connected curve $X$ over $\FF_q$, and choose a place $v\in X$. Then the completion $F_v$ is isomorphic to $k_v((t))$, where $t\in\OO_v$ is a choice of uniformizer and $k_v/\FF_q$ is a finite extension. Then we normalize our norm $\left|\cdot\right|_v$ by $\left|t\right|_v\coloneqq(\#k_v)^{-1}$.
\end{notation}
These choices of normalization obey a product formula.
\begin{proposition} \label{prop:product-formula}
	Fix a global field $F$. For each $x\in F$,
	\[\prod_{v\in V(F)}\left|x\right|_v=1.\]
\end{proposition}
\begin{proof}[Sketch]
	This is included in a standard first course in number theory, so we will be brief. For number fields, this is checked directly by passing to $\QQ$, where it is a consequence of unique prime factorization. For function fields $\FF_q(X)$, we may think of $f\in\FF_q(X)$ as a rational function $X$, and $\left|f\right|_v=q^{\deg(v)\cdot-\op{ord}_v(f)}$, where $\op{ord}_v$ is the order of vanishing. Thus, the product formula more or less amounts to the statement that the sum of the zeroes and poles of $f$ all cancel out (over the algebraic closure).
\end{proof}

\subsection{Ad\'eles}
We now define the ad\'eles by gluing together our localizations.
\begin{definition}[ad\'eles]
	Fix a global field $F$. Then the ring of \textit{ad\'eles} $\AA_F$ is defined as the restricted product
	\[\AA_F\coloneqq\prod_{v\in V(F)}(F_v,\OO_v),\]
	meaning that $\AA_F$ consists of sequences of elements in $F_v$ which are in $\OO_v$ for all but finitely many $v$.
\end{definition}
\begin{remark}
	By construction, we see that
	\[\AA_F=\bigcup_{\substack{\text{finite }S\subseteq V(F)\\S\supseteq V(F)_\infty}}\left(\prod_{v\notin S}\OO_v\times\prod_{v\in S}F_v\right).\]
	Thus, $\AA_F$ is a colimit of (product) topological rings, so $\AA_F$ is a topological ring.
\end{remark}
\begin{remark}
	A basis neighborhood basis of $0\in\AA_F$ is given as follows: for any choice of finite $S\subseteq V(F)$ containing $V(F)_\infty$, choose open neighborhoods $U_v\subseteq\OO_v$ of $0$, and then we have the open subset
	\[\prod_{v\notin S}\OO_v\times\prod_{v\in S}U_v.\]
	One can further require that the open subsets $U_v$ take the form $\mf p_v^{m_v}$, where $m_v$ is some integer.
\end{remark}
Tate's thesis is about $\op{GL}_1(\AA_F)$, so the following group will be important to us.
\begin{definition}
	Fix a global field $F$. Then the group of \textit{id\'eles} $\AA_F^\times$ is defined as the restricted product
	\[\AA_F^\times\coloneqq\prod_{v\in V(F)}(F_v^\times,\OO_v^\times),\]
	meaning that $\AA_F^\times$ consists of sequences of elements in $F_v^\times$ which are in $\OO_v^\times$ for all but finitely many $v$.
\end{definition}
Notably, $\op{GL}_1(\AA_F)=\AA_F^\times$.
\begin{remark}
	This is not the set of nonzero elements in $\AA_F$ because we require the inverse to also be an ad\'ele!
\end{remark}
\begin{remark}
	One should not give the subset $\AA_F^\times\subseteq\AA_F$ the subspace topology. Instead, the topology should be given by the restricted product, whose open subsets can be smaller. Thus, an element of the neighborhood basis of $1\in\AA_F^\times$ can be described as follows: for any choice of finite $S\subseteq V(F)$ containing $V(F)_\infty$, choose open neighborhoods $U_v\subseteq\OO_v$ of $0$, and then we have the open subset
	\[\prod_{v\notin S}\OO_v^\times\times\prod_{v\in S}U_v.\]
	One can further require that the open subsets $U_v$ take the form $\mf p_v^{m_v}$, where $m_v$ is some integer.
\end{remark}
Later in the course, we will even want to study groups like $\op{GL}_2(\AA_F)$ or $\op{GL}_n(\AA_F)$. Let's be explicit about what this notation means.
\begin{definition}[general linear group]
	Fix a ring $R$. Then we define $\op{GL}_n(R)$ to be the group of invertible $n\times n$ matrices. Explicitly, this can be described as the group of $n\times n$ matrices with entries in $R$ whose determinant is invertible.
\end{definition}
\begin{remark}
	Fix a global field $F$. One can check that
	\[\op{GL}_n(\AA_F)=\prod_{v\in V(F)}(\op{GL}_n(F_v),\op{GL}_n(\OO_v)),\]
	which also tells us what the topology should be.
\end{remark}
\begin{remark} \label{rem:embed-to-get-topology}
	Here is another way to construct the topology: $\op{GL}_n$ can embed (as a scheme) as a closed subspace of $\left(n^2+1\right)$-dimensional space $A$, where the embedding sends $g\in\op{GL}_n(R)$ to the tuple of coordinates follows by the inverse of the determinant. This is a closed embedding, essentially by definition of $\op{GL}_n$. Then we can give $\op{GL}_n(\AA_F)$ the natural topology given as a closed subspace of $A(\AA_F)$. It is not too hard (but rather annoying) to check that these definitions agree.
\end{remark}
\begin{example}
	The determinant map $\det\colon\op{GL}_n(\AA_F)\to\AA_F^\times$ is continuous. One can see this via \Cref{rem:embed-to-get-topology} because the determinant and its inverse are both continuous maps to $\AA_F$. But the topology on $\AA_F^\times$ is given as a closed subspace of $\AA_F\times\AA_F$ (where the embedding is given by $x\mapsto(x,1/x)$).
\end{example}
This course is interested in the representation theory of $\op{GL}_n(\AA_F)$, focusing on the cases $n\in\{1,2\}$. If we think about such representations appropriately, it turns out that such a representation $\pi$ will decompose into a tensor product $\bigotimes_v'\pi_v$, where $\pi_v$ is a representation of $\op{GL}_n(F_v)$. More than half of the course will thus be interested in the representation theory of $\op{GL}_n(F_v)$ because we will want to study the finite and infinite places separately.

\subsection{Characters on the Ad\'eles}
We will need more structure theory of the ad\'eles.
\begin{proposition}
	Fix a global field $F$. The diagonal embedding $F\into\AA_F$ embeds $F$ as a discrete subgroup.
\end{proposition}
\begin{proof}
	Fix distinct $a,b\in F$. By examining the open subsets we have access to, we need to show that $\left|a-b\right|_v\ge1$ for some $v$, which follows from \Cref{prop:product-formula}.
\end{proof}
\begin{corollary} \label{cor:units-embed-adele}
	Fix a global field $F$. The diagonal embedding $F^\times\into\AA_F^\times$ embeds $F$ as a discrete subgroup.
\end{corollary}
\begin{proof}
	For each $a\in F^\times$, we need to know that there is an open subset $U$ of $\AA_F^\times$ for which $U\cap F^\times=\{a\}$. But there is such an open subset of $\AA_F$, which continues to be open in $\AA_F^\times$.
\end{proof}
We will be interested in characters on $\AA_F^\times$.
\begin{notation}
	Fix a place $v$ of a global field $F$. Given a continuous character $\chi\colon\AA_F^\times\to\CC^\times$, we let $\chi_v\colon F_v^\times\to\CC^\times$ denote the induced character.
\end{notation}
\begin{remark}
	The continuity of $\chi$ forces $\chi_v|_{\OO_v^\times}=1$ for all but finitely many $v$. Conversely, given a family $\{\chi_v\}_{v\in V(F)}$ of continuous characters for which $\chi_v|_{\OO_v^\times}=1$ for all but finitely many $v$, one can check that there is a unique continuous character $\chi$ on $\AA_F^\times$ gluing them together.\todo{}
\end{remark}
The previous remark motivates the following definition.
\begin{definition}[unramified]
	Fix a place $v$ of a global field $F$. Then a character $\chi_v\colon F_v^\times\to\CC^\times$ is \textit{unramified} if and only if $\chi_v|_{\OO_v^\times}=1$.
\end{definition}
\begin{example}
	By definition, $\chi_v$ factors through $F_v^\times/\OO_v^\times\cong\ZZ$. Thus, $\chi_v$ can be described as $\chi_v=\left|\cdot\right|_v^s$ for some $s\in\CC$.
\end{example}
Not all characters are interesting to us because we want our characters $\chi_v$ to talk to each other.
\begin{definition}[Hecke character]
	Fix a global field $F$. A \textit{Hecke character} is a continuous character $\chi\colon\AA_F^\times\to\CC^\times$ which vanishes on $F^\times$.
\end{definition}
\begin{remark}
	It is equivalent to ask for $\chi$ to be continuous on $F^\times\backslash\AA_F^\times$ by \Cref{cor:units-embed-adele}.
\end{remark}

\end{document}