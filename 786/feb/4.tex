% !TEX root = ../notes.tex

\documentclass[../notes.tex]{subfiles}

\begin{document}

\section{February 4}
Here we go.

\subsection{Adelic Quotients}
Thus, it will be worthwhile to know something about the quotient $F^\times\backslash\AA_F^\times$. Let's start with the additive group.
\begin{theorem}[approximation] \label{thm:approximate-ga}
	Fix a number field $F$. Then
	\[\AA_F=F+\prod_{v\notin V(F)_\infty}\OO_v+\prod_{v\in V(F)_\infty}F_v.\]
\end{theorem}
\begin{proof}
	Given an ad\'ele $(a_v)_v\in\AA_F$, we see that we may ignore the infinite places. Then we are asked to find $a\in F$ for which $a\equiv a_v\pmod{\OO_v}$ for all $v$. After multiplying out some denominators, this amounts to the Chinese remainder theorem for $\OO_F$.
\end{proof}
Here is an analog for function fields.
\begin{proposition}
	Fix a function field $F=\FF_q(X)$. Then one has
	\[F\backslash\AA_F\mathbin{\bigg/}\prod_{v\in V(F)}\OO_v\cong\mathrm H^1(X;\OO_X).\]
\end{proposition}
\begin{proof}
	The idea is to use the ``two-step complex'' $F\to\AA_F/\prod_v\OO_v$ to compute the cohomology of $\OO_X$. Note that $\AA_F/\prod_v\OO_v$ is the restricted product
	\[\prod_{v\in V(F)}\left(\frac{F_v}{\OO_v},\frac{\OO_v}{\OO_v}\right)=\bigoplus_{v\in V(F)}F_v/\OO_v.\]
	Now, there is a short exact sequence
	\[0\to\OO_X\to\mc K\to\bigoplus_{v\in V(F)}i_{v*}(F_v/\OO_v)\to0,\]
	where $\mc K$ is the constant sheaf of rational functions. Taking the long exact sequence in cohomology produces an exact sequence
	\[F\to\bigoplus_{v\in V(F)}F_v/\OO_v\to\mathrm H^1(X;\OO_X)\to\mathrm H^1(X;\mc K).\]
	Because $X$ is irreducible, the constant sheaf $\mc K$ is flasque, so $\mathrm H^1(X;\mc K)=0$. The result now follows.
\end{proof}
\begin{remark}
	As an application, the right-hand side will frequently have more than one element: it has dimension over $\FF_q$ equal to the genus of $X$, so the cohomology group has one element if and only if $X$ is $\PP^1_{\FF_q}$!
\end{remark}
\begin{remark}
	If one expands one of the $\OO_v$s to $F_v$s, then it turns out that the quotient is trivial.
\end{remark}
\begin{remark}
	One can check that the stabilizer of a double coset of the $F$-action on a double coset is exactly $\FF_q=\mathrm H^0(X;\OO_X)$.
\end{remark}
Returning to number fields, we see that \Cref{thm:approximate-ga} grants us a surjection
\[F\otimes_\QQ\RR\onto F\backslash\AA_F\mathbin{\bigg/}\prod_{v\notin V(F)}\OO_v.\]
The kernel is exactly given by the elements $x\in F$ for which $v\in\OO_v$ for all $v$, which is exactly $\OO_F$. Thus, there is an isomorphism
\[\frac{F\otimes_\QQ\RR}{\OO_F}\to F\backslash\AA_F\mathbin{\bigg/}\prod_{v\notin V(F)}\OO_v\]
of topological groups. Here, the left-hand side is a torus of dimension $[F:\QQ]$: it is isomorphic as a topological group to $\RR^d/\ZZ^d$.

\subsection{Idelic Quotients}
Of course, we are more interested in $\AA_F^\times$, so let's turn our attention there. As usual, arguing with function fields is easier.
\begin{proposition}
	Fix a function field $F\coloneqq\FF_q(X)$. Then one has
	\[F^\times\backslash\AA_F^\times\mathbin{\bigg/}\prod_{v\in V(F)}\OO_v^\times\cong\op{Pic}X.\]
\end{proposition}
\begin{proof}
	Once again, we use the ``two-step complex'' $F^\times\to\AA_F^\times/\prod_v\OO_v^\times$. Here, $\AA_F^\times/\prod_{v\in V(F)}\OO_v^\times$ is the restricted product
	\[\prod_{v\in V(F)}\left(\frac{F_v^\times}{\OO_v^\times},\frac{\OO_v^\times}{\OO_v^\times}\right)\cong\bigoplus_{v\in V(F)}\ZZ.\]
	Here, the last isomorphism occurs by taking valuations. Now, this latter group is isomorphic to $\op{Div}(X)$, and we can see that $F^\times$ embeds via these isomorphisms as the principal divisors. The result follows.
\end{proof}
\begin{remark}
	It turns out that $\op{Pic}$ upgrades into a group scheme $\op{Pic}_X$ with a connected component $\op{Pic}_X^n$ for each degree. The Jacobian $\op{Jac}X$ is exactly $\op{Pic}_X^0$. Thus, $\op{Pic}(X)$ is infinite, but the degree-zero part $\op{Jac}X(\FF_q)$ is some group, which has about $q^g$ points by the Weil conjectures.
\end{remark}
\begin{remark}
	The kernel of the map $F^\times\to\op{Div}X$ is exactly the constant functions $\FF_q^\times$. This reflects the fact that line bundles have some action by $\FF_q^\times$.
\end{remark}
And now we move to number fields. Here is a starting result.
\begin{lemma}
	Fix a number field $F$. The map
	\[\prod_{v\mid\infty}F_v^\times\to F^\times\backslash\AA_F^\times\mathbin{\bigg/}\prod_{v\nmid\infty}\OO_v^\times\]
	has cokernel isomorphic to the class group of $F$.
\end{lemma}
\begin{proof}
	The cokernel is
	\[F^\times\backslash\AA_{F,f}^\times\mathbin{\bigg/}\prod_{v\nmid\infty}\OO_v^\times,\]
	where $\AA_{F,f}^\times$ is the ring of finite ad\'eles. The right-hand quotient is $\bigoplus_{v\nmid\infty}F_v^\times/\OO_v^\times$, which is isomorphic to the group of fractional ideals (or equivalently, $\op{Div}(\op{Spec}\OO_F)$). Taking a further quotient by $F^\times$ shows that the cokernel is the class group.
\end{proof}
\begin{remark}
	The kernel of the map is exactly the elements of $F^\times$ which are units at every place, which is exactly $\OO_F^\times$. It follows that we have an exact sequence
	\[1\to\OO_F^\times\to(F\otimes_\QQ\RR)^\times\to F^\times\backslash\AA_F^\times\mathbin{\bigg/}\prod_{v\nmid\infty}\OO_v^\times\to\op{Cl}F\to0.\]
\end{remark}
To continue cutting down the size of the quotient, note that both $F^\times$ and $\prod_v\OO_v^\times$ have global norm in $\AA_F^\times\to\RR^+$ equal to $1$.
\begin{notation}
	Fix a number field $F$. Then we define $\AA_{F}^{\times,1}$ to be the subset of elements with global norm $1$.
\end{notation}
\begin{remark}
	The map
	\[F^\times\backslash\AA_F^\times\mathbin{\bigg/}\prod_{v\nmid\infty}\OO_v^\times\to F^\times\backslash\AA_{F,f}^\times\mathbin{\bigg/}\prod_{v\nmid\infty}\OO_v^\times\]
	continues to be surjective because we can always choose the archimedean part of an ad\'ele so that the ad\'ele has norm $1$.
\end{remark}
Thus, our exact sequence now looks like
\[1\to\OO_F^\times\to(F\otimes_\QQ\RR)^{\times,1}\to F^\times\backslash\AA_F^{1,\times}\mathbin{\bigg/}\prod_{v\nmid\infty}\OO_v^\times\to\op{Cl}F\to0,\]
where $(F\otimes_\QQ\RR)_1$ refers to the subgroup whose product is $1$. Taking $\log\left|\cdot\right|_v$ (with $\left|\cdot\right|_v$ chosen as before) maps $(F\otimes_\QQ\RR)^{\times,1}$ into a Euclidean space isomorphic to $\RR^{r_1+r_2-1}$, where $(r_1,r_2)$ is the signature of $F$. Note that the kernel of $\log\left|\cdot\right|_v$ is given by the elements of archimedean norm $1$, which when restricted to $\OO_F^\times$ is exactly the group $\mu(F)$ of roots of unity.

Now, by Dirichlet's unit theorem, we see that $\OO_F^\times$ embeds as a lattice of full rank into $\RR^{r_1+r_2-1}$, so the quotient is a compact torus. We have thus proven the following result.
\begin{theorem}
	Fix a number field $F$ with signature $(r_1,r_2)$ The double quotient
	\[F^\times\backslash\AA_F^{\times,1}\mathbin{\bigg/}\prod_{v\nmid\infty}\OO_v^\times\]
	is isomorphic to an extension of $(\RR/\ZZ)^{r_1+r_2-1}$ by the class group $\op{Cl}F$. In particular, it is compact.
\end{theorem}
\begin{remark}
	Thus, we see that
	\[F^\times\backslash\AA_F^{\times}\mathbin{\bigg/}\prod_{v\nmid\infty}\OO_v^\times\]
	is not compact, but it is an extension of a compact abelian group by $\RR^+$.
\end{remark}

\subsection{Pontryagin Duality}
Our next task is to do some Fourier analysis on $\AA_F$ and $\AA_F^\times$. Let's first recall generalities of Fourier analysis on locally compact abelian topological groups.
\begin{definition}[Pontryagin dual]
	Fix a locally compact abelian group $X$. Then its \textit{Pontryagin dual} $X^*$ is the set of homomorphisms $X\to S^1$, equipped with the compact open topology.
\end{definition}
\begin{remark}
	There is a functoriality as follows: for any homomorphism $f\colon X\to Y$, we have a homomorphism $f^*\colon Y^*\to X^*$ given by pre-composition.
\end{remark}
Here are some theorems about this construction.
\begin{theorem}[Duality]
	There is a natural isomorphism ${\id}\Rightarrow(-)^{**}$. For a given group $G$, it is given by sending $g\in G$ to the character $\op{ev}_g\colon G^*\to S^1$ defined by $\op{ev}_g\colon\chi\mapsto\chi(g)$.
\end{theorem}
\begin{theorem}[Exact]
	The functor $(-)^*$ is exact.
\end{theorem}
Let's see some examples.
\begin{example}
	If $X=\ZZ$, then its Pontryagin dual is just $S^1$. On the other hand, all continuous homomorphisms $S^1\to S^1$ take the form $z\mapsto z^n$, so $\left(S^1\right)^*=\ZZ$.
\end{example}
\begin{example}
	Homomorphisms $\RR\to S^1$ all take the form $\chi_\xi\colon t\mapsto e^{i\xi t}$, where $\xi\in\RR$ is some real number. Thus, $\RR^*=\RR$.
\end{example}
\begin{example}
	In general, given a finite-dimensional real vector space $V$, we may identify the dual $V^*$ with the Pontryagin dual, where one sends $\varphi\colon V\to\RR$ to the character $v\mapsto e^{i\varphi(v)}$.
\end{example}
\begin{example}
	Homomorphisms $\ZZ/n\ZZ\to S^1$ are uniquely determined by where they send $1$, so
	\[(\ZZ/n\ZZ)^*\cong\mu_n.\]
	Conversely, all homomorphisms $\mu_n\to\mu_n$ are given by $z\mapsto z^k$ for some $k$, so $\mu_n^*\cong\ZZ/n\ZZ$. In particular, we see that $(\ZZ/n\ZZ)^*$ is non-canonically isomorphic to $\ZZ/n\ZZ$, but the isomorphism $\ZZ/n\ZZ\to(\ZZ/n\ZZ)^{**}$ is canonical! In general, for any finite abelian group $G$, we see that $G^*$ is non-canonically isomorphic to $G$.
\end{example}
\begin{example}
	Exactness of the functor $(-)^*$ implies that
	\[\ZZ_p^*=\left(\lim\ZZ/p^\bullet\ZZ\right)^*=\colim\mu_{p^\bullet}=\mu_{p^\infty}.\]
\end{example}
\begin{example}
	Once again, exactness of the functor $(-)^*$ implies that
	\[\QQ_p^*=\left(\colim\left(\ZZ_p\stackrel p\to\ZZ_p\stackrel p\to\cdots\right)\right)^*=\lim\left(\mu_{p^\infty}\stackrel p\to\mu_{p^\infty}\stackrel p\to\cdots\right).\]
	Thus, this limit is some kind of coherent sequence of taking $p$th roots, which is then isomorphic to $\QQ_p$. Indeed, $\mu_{p^\infty}$ is isomorphic to $\QQ_p/\ZZ_p$, which we see by sending $a/p^n\in\QQ_p/\ZZ_p$ to $\exp(2\pi ia/p^n)$. In fact, it turns out that the exact sequence
	\[0\to\ZZ_p\to\QQ_p\to\QQ_p/\ZZ_p\to0\]
	dualizes to an isomorphism
	% https://q.uiver.app/#q=WzAsMTAsWzAsMCwiMCJdLFsxLDAsIlxcWlpfcCJdLFsyLDAsIlxcUVFfcCJdLFszLDAsIlxcUVFfcC9cXFpaX3AiXSxbNCwwLCIwIl0sWzAsMSwiMSJdLFsxLDEsIlxcbGltXFxtdV97cF5ufSJdLFsyLDEsIlxcUVFfcF4qIl0sWzMsMSwiXFxtdV97cF5cXGluZnR5fSJdLFs0LDEsIjEiXSxbMCwxXSxbMSwyXSxbMiwzXSxbMyw0XSxbNSw2XSxbNiw3XSxbNyw4XSxbOCw5XSxbMSw2XSxbMiw3XSxbMyw4XV0=&macro_url=https%3A%2F%2Fraw.githubusercontent.com%2FdFoiler%2Fnotes%2Fmaster%2Fnir.tex
	\[\begin{tikzcd}[cramped]
		0 & {\ZZ_p} & {\QQ_p} & {\QQ_p/\ZZ_p} & 0 \\
		1 & {\lim\mu_{p^n}} & {\QQ_p^*} & {\mu_{p^\infty}} & 1
		\arrow[from=1-1, to=1-2]
		\arrow[from=1-2, to=1-3]
		\arrow[from=1-2, to=2-2]
		\arrow[from=1-3, to=1-4]
		\arrow[from=1-3, to=2-3]
		\arrow[from=1-4, to=1-5]
		\arrow[from=1-4, to=2-4]
		\arrow[from=2-1, to=2-2]
		\arrow[from=2-2, to=2-3]
		\arrow[from=2-3, to=2-4]
		\arrow[from=2-4, to=2-5]
	\end{tikzcd}\]
	sending $x\in\ZZ_p$ to the sequence $\{\zeta_{p^n}^x\}_n$.
\end{example}
Thus, we see that all local fields are identified with their Pontryagin duals. In fact, all of our constructions amount to identifying a space with its dual upon choosing a single character.
\begin{remark}
	Explicitly, given a choice of nontrivial character $\psi_p\in\QQ_p^*$, there is a map $\QQ_p\to\QQ_p^*$ given by taking $x$ to the character $y\mapsto\psi_p(xy)$. It turns out that this map is an isomorphism, so we have more or less defined a non-degenerate bilinear form $\QQ_p\times\QQ_p\to S^1$. This procedure also works for $\RR$!
\end{remark}
In light of the previous remark, it is useful to fix some characters.
\begin{notation}
	Fix a place $v$ of $\QQ$.
	\begin{itemize}
		\item If $v=p$ is finite, then we define the character $\psi_p\colon\QQ_p\to S^1$ by the composite $\QQ_p\to\QQ_p/\ZZ_p\cong\mu_{p^\infty}$, where the second isomorphism sends $a/p^n$ to $e^{2\pi ia/p^n}$.
		\item If $v=\infty$ is infinite, then we define the character $\psi_\infty\colon\RR\to S^1$ by $\psi_\infty(x)\coloneqq e^{-2\pi ix}$.
	\end{itemize}
\end{notation}
\begin{remark}
	The choice of $\psi_\infty$ is done so that the assembled character $\psi\colon\AA_\QQ\to S^1$ vanishes on $\QQ$.
\end{remark}

\subsection{Fourier Theory}
To do Fourier analysis, we need a notion of measure.
\begin{theorem}[Haar]
	Fix a locally compact group $X$. Then there is a left-invariant Radon measure $dx$ on $X$ which is unique up to scalar.
\end{theorem}
Now, here is our Fourier transform.
\begin{definition}[Fourier transform]
	Fix a locally compact abelian group $X$. The \textit{Fourier transform} sends a function $f\in L^1(X)$ to the function $\widehat f\colon X^*\to\CC$ given by
	\[\widehat f(\xi)=\int_Xf(x)\overline\xi(x)\,dx.\]
\end{definition}
It is a large theorem that there is an inversion.
\begin{theorem}[Fourier inversion]
	Fix a locally compact abelian group $X$, and let $dx$ be a Haar measure on $X$. Then there is a Haar measure $d\chi$ on $X^*$ such that
	\[f(x)=\int_{X^*}\widehat f(\chi)\chi(x)\,d\chi\]
	for any $f\in L^1(X)$ for which $\widehat f\in L^1(X^*)$.
\end{theorem}
\begin{remark}
	It turns out that the Fourier transform extends to an isomorphism $L^2(G)\to L^2(G^*)$.
\end{remark}
\begin{remark}
	Equivalently, we see that the double Fourier transform of $f$ is $f(-x)$.
\end{remark}
\begin{remark}
	If $X$ admits an isomorphism $X\cong X^*$, then the Haar measure $d\chi$ is not necessarily equal to the Haar measure $dx$ because it might be off by a scalar: indeed, replacing $dx$ with $c\,dx$ replaces $\widehat f$ with $c\widehat f$, and so we see that we end up replacing $d\chi$ with $c^{-1}\,d\chi$. Thus, there is a unique measure $dx$ on $X$ (even up to scalar!) which is ``Fourier self-dual.''
\end{remark}

\end{document}