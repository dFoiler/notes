% !TEX root = ../notes.tex

\documentclass[../notes.tex]{subfiles}

\begin{document}

\section{January 22}

In this course, we would like to do computations.

\subsection{CW Complexes}
We would like to compute the homology and cohomology of interesting spaces. CW complexes are a good mix between being simple enough to work with while being flexible enough to have many interesting examples.
\begin{definition}[CW complex]
	A \textit{CW complex} is a topological space $X$ equipped with an ascending chain of subspaces
	\[\emp=X_{-1}\subseteq X_0\subseteq X_1\subseteq\cdots\subseteq\]
	satisfying the following.
	\begin{listroman}
		\item $X=\bigcup_{k\ge-1}X_k$.
		\item $X_{k+1}$ is obtained inductively be adjoining $(k+1)$-disks $\{D_\alpha^{k+1}\}_\alpha$ to $X_k$ as follows: some $x\in S^k_\alpha$ is identified with its image in $X_k$ via some continuous map $f\colon\bigsqcup_\alpha S_\alpha^k\to X_k$.
		\item The topology is on $X$ is given by asserting $U\subseteq X$ is open if and only if $U\cap X_k$ is open for all $k$.
	\end{listroman}
	We call the interior $e^n_\alpha\coloneqq(D_\alpha^n)^\circ$ an \textit{$n$-cell}, and we call $X_n$ the \textit{$n$-skeleton}. If the total number of cells if finite, then $X$ is \textit{finite}; if merely each $X_n$ is finite, then $X$ is \textit{finite type}.
\end{definition}
\begin{example}
	Compact manifolds have the structure of a CW-complex with only finitely many disks.
\end{example}
\begin{remark}
	Note that we only ever identify points on the new disks for $X_{k+1}$ along boundaries. In particular, we see that $X$ can be seen as a union of the spaces $e^n_\alpha$ as $n$ and $\alpha$ vary; thus, $e_\alpha^n\cap e_{\alpha'}^{n'}=\emp$ whenever $(n,\alpha)\ne(n',\alpha')$.
\end{remark}
The point is that a CW complex allows us to work combinatorially with many topological spaces.
\begin{example}
	We realize $S^n$ as a CW complex as follows: simply attach $D^n$ to a point $\{*\}$ by attaching the entire boundary to the point. Note that this is certainly continuous by definition of the quotient topology, and one can check that this is a bijection by doing casework at and away from $\{*\}$. Thus, we have a continuous bijection between compact Hausdorff spaces, so we have produced a homeomorphism.
\end{example}
\begin{example}
	Here is another way to realize $S^n$. We do this inductively: note $n=0$ has no content (simply take a point). Then supposing that we have already produced $S^{n-1}$, we simply attach a top and bottom hemisphere of $n$-cells to produce $S^n$.
\end{example}
\begin{example}
	We can realize $\RP^n$ as a CW complex. Simply realize $\RP^n$ as $S^n$ modulo the antipodal map, so we see that we can identify $\RP^n$ as $\RP^{n-1}$ union with a single $n$-cell (as in the previous example) but identified ``twice-over'' on the boundary.
\end{example}
\begin{example}
	We realize $T^2=S^1\times S^1$ as a CW complex. Think of $T^2$ as a rectangle with the bottom/top and left/right edge pairs identified. Then we have a single point $e_0$ as our $0$-cell, two $1$-cells corresponding to the two edges (after identification), and then there is the $2$-cell $e^2$ embedded. Here is the picture.
	\begin{center}
		\begin{asy}
			unitsize(2cm);
			dot("$e_0$", (0,0), SW);
			dot("$e_0$", (1,0), SE);
			dot("$e_0$", (1,1), NE);
			dot("$e_0$", (0,1), NW);
			draw((0,0)--(0,1)--(1,1)--(1,0)--cycle);
			label("$e_2$", (0.5,0.5));
			label("$e_1$", (0,0.5), W);
			label("$e_1$", (1,0.5), E);
			label("$e_1'$", (0.5,0), S);
			label("$e_1'$", (0.5,1), N);
		\end{asy}
	\end{center}
\end{example}
\begin{example}
	We realize $\CP^n$ as a CW complex. The idea is that we can think about complex lines in $\CC^{n+1}$ as intersecting
	\[S^{2n+1}=\left\{(z_0,\ldots,z_n)\in\CC^{n+1}:\left|z_0\right|^2+\cdots+\left|z_n\right|^2=1\right\},\]
	and $p,q\in S^{n+1}$ both live on the same line if and only if $p/q\in S^1$. This explains that $\CP^{n+1}$ should be topologically a quotient $S^{2n+1}/S^1$.
	
	We are now ready to give our CW structure, which we do inductively, beginning by noting that $\CP^0$ is a point. We would now like to attach $D^{2n}$ to $\CP^{n-1}$ to produce $\CP^n$. To begin, note we can realize $S^{2n}$ inside $S^{2n+1}$ as above by adding the requirement $z_n\in\RR$, and then we can realize $D^{2n}$ inside this $S^{2n}$ by further requiring $z_n\ge0$. Then note that $D^{2n}$ surjects onto $S^{2n+1}/S^1$ (by using the $S^1$-action), so we induce a homeomorphism of a quotient of $D^{2n}$ onto $S^{2n+1}/S^1$. We take a moment to note that this quotient behaves as follows: distinct points $(z_0,\ldots,z_n)$ and $(w_0,\ldots,w_n)$ in $D^{2n}$ are identified if and only if $z_n=w_n=0$ (because we are only multiplying by scalars in $S^1$) and further $(z_0,\ldots,z_{n-1})$ and $(w_0,\ldots,w_{n-1})$ define the same line in $\CP^{n-1}$. In this way, we see that $\CP^{2n}$ is realized as $\CP^{2n-1}$ together with a $D^{2n}$ attached as described above.
\end{example}

\end{document}