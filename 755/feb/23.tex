% !TEX root = ../notes.tex

\documentclass[../notes.tex]{subfiles}

\begin{document}

\section{February 23}
Here we go.

\subsection{Spin Groups}
Here is the easiest example of the spin representation.
\begin{example}
	By the ``belt trick,'' one has that $\pi_1(\op{SO}_3(\CC))\cong\ZZ/2\ZZ$, and it turns out that its universal cover is $\op{SL}_2(\CC)$. (Indeed, $\mf{sl}_2=\mf{so}_3$.) In particular, the adjoint representation defines a covering $\op{SL}_2(\CC)\to\op{SO}_3(\CC)$, and it turns out that the kernel is $\{\pm1\}$. (We will also show that $\op{SL}_2(\CC)$ is simply connected in \Cref{ex:sl2-simply-connected}.) Thus, $\op{Spin}_3(\CC)=\op{SL}_2(\CC)$, and the spin representation is just the standard representation.
\end{example}
It remains to handle $n\ge3$.
\begin{proposition} \label{prop:pi1-so-n}
	For $n\ge3$, we have $\pi_1(\op{SO}_n(\CC))\cong\ZZ/2\ZZ$.
\end{proposition}
We will want the following lemma.
\begin{lemma} \label{lem:complex-sphere-pi}
	Let $X_n\subseteq\CC^n$ be the hypersurface cut out by the equation
	\[z_1^2+\cdots+z_n^2=1.\]
	For any $k\in\{1,\ldots,n-2\}$, one has $\pi_k(X_n)=0$.
\end{lemma}
\begin{proof}
	One can retract $X_n(\CC)$ to $X_n(\RR)$, which is $S^{n-1}$, and then the result follows from calculations of lower-dimensional homotopy groups. To do the contraction, define $f_\bullet\colon X_n\times[0,1]\to X_n$ by
	\[f_t(x+iy)=\frac{x+tiy}{\sqrt{x^2-t^2y^2}}.\]
	Indeed, note that $x+iy\in X_n$ (where $x,y\in\RR^n$) is equivalent to having $\left|x\right|^2-\left|y\right|^2=1$ and $xy=0$, and we can see that the output continues to have this property. But now $f_1$ is the identity and $f_0$ has real output, so we are done.
\end{proof}
\begin{example} \label{ex:sl2-simply-connected}
	Adjusting basis, one sees that
	\[\op{SL}_2(\CC)=\{(a,b,c,d):ad-bc=1\}\]
	is simply connected (by taking $n=4$). In fact, we also see that $\pi_2(\op{SL}_2(\CC))=0$.
\end{example}
\begin{example} \label{ex:so3-has-z2}
	At $n=3$, there is a short exact sequence
	\[1\to\mu_2\to\op{SL}_2(\CC)\to\op{SO}_3(\CC)\to1.\]
	This is a homotopy fiber sequence because the latter action is free, so taking the long exact sequence in homotopy and using \Cref{ex:sl2-simply-connected} shows that $\pi_1(\op{SO}_3(\CC))=\ZZ/2\ZZ$.
\end{example}
\begin{proof}[Proof of \Cref{prop:pi1-so-n}]
	We proceed by induction; \Cref{ex:so3-has-z2} allows us to assume $n\ge4$.

	We now show the general case. Note that $\op{SO}_n(\CC)$ acts transitively on $X_n$, and the stabilizer of the vector $(1,0,\ldots,0)$ is exactly $\op{SO}_{n-1}(\CC)$ embedded diagonally. Thus, we have a homotopy fiber sequence
	\[\op{SO}_{n-1}(\CC)\to\op{SO}_n(\CC)\to X_n.\]
	Taking the long exact sequence in homotopy yields the exact sequence
	\[\pi_2(X_n)\to\pi_1(\op{SO}_{n-1}(\CC))\to\pi_1(\op{SO}_n(\CC))\to\pi_1(X_n).\]
	The end terms vanish for $n\ge4$ by \Cref{lem:complex-sphere-pi}, so we are done by induction.
\end{proof}
We may thus make the following definition.
\begin{definition}[spin]
	For a positive integer $n\ge3$, let $\op{Spin}_n(\CC)$ be the universal cover of $\op{SO}_n(\CC)$ for each $n$.
\end{definition}
\begin{remark}
	By \Cref{prop:pi1-so-n}, we see that the canonical projection $\op{Spin}_n(\CC)\to\op{SO}_n(\CC)$ is a double cover.
\end{remark}
\begin{example}
	The short exact sequence in \Cref{ex:so3-has-z2} shows that $\op{Spin}_3(\CC)=\op{SL}_2(\CC)$.
\end{example}
\begin{example}
	Note that $\mf{so}_4=\mf{sl}_2\oplus\mf{sl}_2$ (e.g., by looking at Dynkin diagrams), so $\op{Spin}_4(\CC)\cong\op{SL}_2(\CC)\times\op{SL}_2(\CC)$ because there is a unique simply connected Lie group attached to a given finite-dimensional Lie algebra. One can calculate that the spin representations are given by $S_+=\CC^2\boxtimes\CC$ and $S_-=\CC\boxtimes\CC^2$. By tracking through the isomorphism $\mf{so}_4=\mf{sl}_2\oplus\mf{sl}_2$, one finds that the projection $\op{SL}_2(\CC)\times\op{SL}_2(\CC)\to\op{SO}_4(\CC)$ has kernel $(-1,-1)$. Additionally, the standard representation of $\op{SO}_4(\CC)$ is $\CC^2\boxtimes\CC^2$.
\end{example}
\begin{example}
	One has that $\mf{so}_5\cong\mf{sp}_4$, which one can see by looking at Dynkin diagrams. One can show as before that $\op{Sp}_{2n}$ is simply connected, so $\op{Spin}_5(\CC)=\op{Sp}_4(\CC)$, and the double cover $\op{Sp}_4(\CC)\to\op{SO}_5(\CC)$ has kernel $\{\pm1\}$. (This is the kernel because the kernel needs to be central, and the nontrivial element needs to act by $-1$ on the spin representations in order to avoid descending to $\op{SO}$.) The spin representation of $\op{SO}_5(\CC)$ turns out to be the standard representation of $\op{Sp}_4(\CC)$.
\end{example}
\begin{example}
	One has $\mf{so}_6\cong\mf{sl}_4$, so $\op{Spin}_6(\CC)=\op{SL}_4(\CC)$ because $\op{SL}_4(\CC)$ is simply connected. The double cover $\op{SL}_4(\CC)\to\op{SO}_6(\CC)$ has kernel $\{\pm1\}$. The spin representations $S_+$ and $S_-$ are the standard representation and its dual. (Note that these are the only possible irreducible representations of $\op{SL}_4(\CC)$ with the correct dimension.) On the other hand, the standard representation of $\op{SO}_6(\CC)$ is the second exterior power of the standard representation.
\end{example}
For $n\ge7$, the group $\op{Spin}_n(\CC)$ is harder to describe, and it is our next goal.

\subsection{Clifford Algebras}
Throughout, $V$ is an inner product space over a field $k$ which has characteristic not equal to $2$.
\begin{definition}
	Fix an inner product space $V$ over a field $k$ with $\op{char}k\ne2$. Then the \textit{Clifford algebra} $\op{Cl}V$ is the quotient of the universal tensor algebra by the ideal generated by the elements
	\[v\otimes v-\frac12(v,v).\]
\end{definition}
\begin{remark}
	For any $a,b\in V$, it follows that $2(ab+ba)=(a+b)^2-a^2-b^2=(a+b,a+b)-(a,a)-(b,b)=2(a,b)$, so $ab+ba=(a,b)$. Setting $a=b$ recovers the relation $2a^2=(a,a)$.
\end{remark}
\begin{remark}
	The Clifford algebra admits a natural filtration $\{F_i\}$ induced by the standard filtration on the tensor algebra. In particular, $F_0=k$, $F_1=k+V$, $F_2=(k+V)\cdot(k+V)$, and so on.
\end{remark}
\begin{remark}
	Because the relations we took a quotient by for $\op{Cl}V$ live in even degree, there is a $\ZZ/2\ZZ$-grading on $\op{Cl}V$, which we label $\op{Cl}_+V\oplus\op{Cl}_-V$.
\end{remark}
\begin{remark}
	Recall that one can define the exterior algebra $\land V$ as the quotient of the tensor algebra $TV$ by the ideal generated by the elements $v^2$. Thus, the natural map $TV\to\op{Cl}V$ descends to a map $\land V\to\op{gr}\op{Cl}V$. This map is further a surjection: fixing an orthonormal basis $\{e_i\}$ of $V$, we see that $\op{Cl}V$ is spanned by the vectors of the form
	\[e_{i_1}\cdots e_{i_r}.\]
	Using the relations $e_ie_j+e_je_i=1_{i=j}$, we see that we may even assume $i_1<\cdots<i_r$. But then these vectors are found from $\land V$, so we are done.
\end{remark}
Here is a basis, which is a variant of the Poincar\'e--Birkhoff--Witt theorem.
\begin{theorem} \label{thm:calculate-clifford}
	Fix an inner product space $V$.
	\begin{listalph}
		\item If $\dim V=2n$, then $\op{Cl}V\cong M_{2^n}(\CC)$.
		\item If $\dim V=2n+1$, then $\op{Cl}V\cong M_{2^n}(\CC)\oplus M_{2^n}(\CC)$.
	\end{listalph}
\end{theorem}
\begin{proof}
	Let's start with the case $\dim V=2n$, and choose a basis $\{a_1,\ldots,a_n,b_1,\ldots,b_n\}$ so that $(a_i,a_j)=0$, $(b_i,b_j)=0$ and $(a_i,b_j)=1_{i=j}$. We now define operators on $M\coloneqq\land(a_1,\ldots,a_n)$ as follows: for each $i$, we define $A_iv\coloneqq a_iv$ and $B_iv\coloneqq\frac{\del}{\del a_i}v$, where
	\[\frac{\del}{\del a_i}(a_{i_1}\land\cdots\land a_{i_n})\coloneqq\begin{cases}
		0  & \text{if }i\notin\{i_1,\ldots,i_n\}, \\
		(-1)^{k-1}(a_{i_1}\land\cdots\land\widehat a_{i_k}\land\cdots\land a_{i_n}) & \text{if }i=i_k.
	\end{cases}\]
	One can check that $A_iB_j+B_jA_i=1_{i-j}$ and $A_iA_j+A_jA_i=0$ and $B_jB_i+B_iB_j=0$, so we get a natural map $\rho\colon\op{Cl}V\to\op{End}M$ given by $\rho(a_i)\coloneqq A_i$ and $\rho(b_i)\coloneqq B_i$.

	Now, on the homework, we will show that increasing sequences $\{i_\bullet\}$ and $\{j_\bullet\}$ produce linearly independent elements
	\[A_{i_1}\cdots A_{i_r}B_{j_1}\cdots B_{j_s}.\]
	It follows that $\rho$ is an embedding because the elements $a_{i_1}\cdots a_{i_r}b_{j_1}\cdots b_{j_s}$ are already known to be a spanning set, and we can see that they are sent to a linearly independent set. But then note that $\dim\op{Cl}V=2^{2n}$ while $\dim M=2^{2n}$, so it actually follows that $\rho$ is an isomorphism.

	We now turn to the case where $\dim V=2n+1$. Then we can pick a basis $\{a_1,\ldots,a_n,b_1,\ldots,b_n,z\}$, where the $a_\bullet$s and $b_\bullet$s have relations as above, and $(a_i,z)=(b_i,z)=0$ for all $i$ and $(z,z)=2$. In particular, we see that $za_i=-a_iz$ and $zb_i=-b_iz$ and $z^2=1$. Now, set $M\coloneqq\land(a_1,\ldots,a_n)$. There are two ways to extend the action of $\op{Cl}V_0$ (where $V_0\coloneqq\op{span}\{a_1,\ldots,a_n,b_1,\ldots,b_n\}$) to $\op{Cl}V$. Indeed, either one can take
	\[z(a_{i_1}\cdots a_{i_r})=\pm(-1)^r(a_1\land\cdots\land a_{i_r}).\]
	Notably, the choice of sign $\pm$ produces non-isomorphic modules $M_\pm$ for $\op{Cl}V$ because there is a unique vector $v$ with $b_iv=0$ for all $i$, but then $zv=\pm v$ in $M_\pm$.

	Thus, one receives a natural map
	\[\rho\colon\op{Cl}V\to\op{End}M_+\oplus\op{End}M_-,\]
	and one checks as before that it is an isomorphism. For example, one can check that $M_+$ and $M_-$ are irreducible, which makes $\rho$ a surjection, and then it becomes an isomorphism via some dimension calculations.
\end{proof}
\begin{remark}
	The above proof has actually shown that the natural map $\land V\to\op{gr}\op{Cl}V$ is an isomorphism.
\end{remark}
\begin{remark}
	Here is another approach: one can try to check that $M$ is an irreducible module for $\op{Cl}V$ instead of trying to show that the elements $A_{i_1}\cdots A_{i_r}B_{j_1}\cdots B_{j_s}$ are linearly independent. In particular, irreducibility implies that $\rho$ is surjective by some density theorem.
\end{remark}

\subsection{Spin Representations}
We are now ready to construct some spin representations.
\begin{proposition}
	Fix an inner product space $V$ over a field of characteristic not $2$. View $\mf{so}(V)$ as $\land^2V$ via the identification $V\cong V^*$. Then the natural map $\xi\colon\land^2V\to\op{Cl}V$ defined by
	\[\xi(a\land b)=\frac12(ab-ba)\]
	is a Lie algebra homomorphism. Here, $\op{Cl}V$ has been given the natural Lie bracket from being an associative algebra.
\end{proposition}
\begin{proof}
	This is a direct calculation. Note
	\begin{align*}
		[\xi(a\land b),\xi(c\land d)] &= \left[ab-\frac12(a,b),cd-\frac12(c,d)\right] \\
		&\stackrel*= [ab,cd] \\
		&= abcd-cdab,
	\end{align*}
	where $\stackrel*=$ has used the Clifford algebra relations. Now, one can successively switch vectors around with some error terms. For example, $abcd=abcd+acbd-acbd=(b,c)ad-acbd$. Continuing this process until we get to the permutation $cdab$, we see that this equals
	\[(b,c)ad-(b,d)ac+(a,c)db-(a,d)cb,\]
	which can be seen to be
	\[(b,c)\xi(a\land b)-(b,d)\xi(a\land c)+(a,c)\xi(d\land b)-(a,d)\xi(c\land b),\]
	which is $\xi([a\land b,c\land d])$ in $\mf{so}(V)$.
\end{proof}
Notably, one now sees that the map $U\mf{so}(V)\to\op{Cl}_+V$ is surjective, so we can use the modules $M$ constructed in \Cref{thm:calculate-clifford} to find our spin representations.
\begin{itemize}
	\item When $\dim V=2n$, it turns out that the $\ZZ/2\ZZ$-grading on $\op{Cl}V$ produces a $\ZZ/2\ZZ$-grading $M=M_0\oplus M_1$ on $M$. Thus, we may view $M_0\oplus M_1$ as a representation of $\mf{so}_{2n}$, and these representations have dimension $2^{n-1}$. By finding a suitable highest weight vector and comparing dimensions, one finds that these representations are $S_+$ and $S_-$, respectively.
	\item When $\dim V=2n+1$, it turns out that both $M_+$ and $M_-$ are both the same spin representation $S$, which one again checks by find a suitable highest weight vector and comparing dimensions. Notably, it turns out that $M_+$ and $M_-$ remain irreducible when restricted $\mf{so}(V)$.
\end{itemize}

\subsection{Dual Representations}
We return to the setting where $\mf g$ is a finite-dimensional simple Lie algebra over $\CC$.
\begin{remark} \label{rem:dual-weight}
	Fix an irreducible representation $L_\lambda$ of $\mf g$, where (as usual) $\lambda$ is some dominant integral weight. Then $L_\lambda^*\cong L_{\ov\lambda}$ for some other dominant weight $\ov\lambda$. Notably, the highest weight of $L^*_\lambda$ is just the negative of the lowest weight of $L_\lambda$. To compute this, recall that there is a maximal element $w_0\in W$ of length $\left|\Phi_+\right|$, meaning that $w_0(\Phi_+)=\Phi_-$. Thus, $w_0$ maps the dominant chamber $P_+$ to $-P_+$. It follows that $w_0\lambda$ is exactly the lowest weight of $L_\lambda$, so $\ov\lambda=-w_0\lambda$. (Indeed, writing $w_0\mu=\lambda-\sum_ik_i\alpha_i$ for nonnegative $k_i$, then $\mu=w_0\lambda-\sum_ik_iw_0\alpha_i=w_0\lambda+\sum_ik_i\alpha_{\sigma(i)}$, where $\sigma$ is some permutation of the simple roots induced by $w_0$. The minimality of $w_0\lambda$ follows.)
\end{remark}
It is potentially interesting to calculate $w_0$ in some special cases. Note that $w_0$ maps the positive roots to the negative roots, so $-w_0$ permutes the simple roots, so it is a permutation of the vertices of the Dynkin diagram. Further, because Weyl group elements preserve the inner product, we see that $-w_0$ will further preserve the edges of the Dynkin diagram. Thus, $-w_0$ is a graph automorphism!
\begin{example}
	For the types $A_1$, $B_n$, $C_n$, $G_2$, $F_4$, $E_7$, and $E_8$, the Dynkin diagram admits no automorphism, so $-w_0$ is the identity.
\end{example}
The remaining cases are slightly trickier. Let's remove a few more. Note that $W$ acts trivially on the quotient $P/Q$: indeed, we see that
\[s_i\lambda=\lambda-(\lambda,\alpha_i^\lor)\alpha_i,\]
and the latter term is in $Q$. Thus, $-w_0$ acts by an inversion on $P/Q$, so we conclude that $-w_0\ne1$ whenever $P/Q$ is not a product of $(\ZZ/2\ZZ)$s.
\begin{example}
	In type $A_{n-1}$ for $n\ge3$, we see that $P/Q\cong\ZZ/n\ZZ$, so $-w_0$ must act by flipping the entire Dynkin diagram. One can also see explicitly that this longest Weyl element is the full flip in $S_n$.
\end{example}
The remaining cases in $D_n$ are harder. If $n$ is odd, then $P/Q\cong(\ZZ/4\ZZ)$, and so $-w_0$ acts by $-1$ already, so it is nontrivial. If $n$ is even, then $P/Q\cong(\ZZ/2\ZZ)^2$, but one can calculate that $-w_0$ acts by $1$. This last calculation must be done explicitly. The Weyl group in this case is $S_n\ltimes\{\pm1\}^n_0$, where $\{\pm1\}^n_0$ is the kernel of the multiplication map $\{\pm1\}^n\to\{\pm1\}$. Then $w_0$ is the element sending positive roots to negative roots, which when $n$ is even is simply $-\id$. (When $n$ is odd, this element is not in the Weyl group!) Thus, $-w_0$ is trivial!
\begin{example}
	For $\mf{so}(4k+2)$, one can calculate $S_+^*=S_-$. But for $\mf{so}(4k)$, one finds that $S_+^*=S_+$ and $S_-^*=S_-$.
\end{example}
\begin{example}
	For type $E_6$, one finds that $P/Q\cong\ZZ/3\ZZ$, so $-w_0$ is nontrivial and flips the whole Dynkin diagram. In particular, it switches the two miniscule weights.
\end{example}

\subsection{Maximal Root}
We have already recalled a little work with the maximal root when we discussed miniscule representations, but we will now study the maximal root in more detail.
\begin{definition}[maximal root]
	Fix a simple Lie algebra $\mf g$. Then the \textit{maximal root} $\theta$ of $\mf g$ is the highest weight of the adjoint representation of $\mf g$.
\end{definition}
\begin{remark}
	Because the adjoint representation has weights equal to the root system, we see that $\theta$ is in fact a root. Its maximality follows because it is the highest weight among the roots.
\end{remark}
\begin{example}
	In type $A_{n-1}$, we can see that $\theta=(1,0,\ldots,0,-1)$, which is not fundamental.
\end{example}
\begin{example}
	In type $C_n$, one sees that the adjoint representation $\mf g$ is $S^2V$, which is $L_{2\omega_1}$. Thus, $\theta$ is not fundamental.
\end{example}
\begin{proposition}
	For all simple Lie algebras $\mf g$ which are not in types $A$ or $C$, the maximal root $\theta$ is fundamental.
\end{proposition}
\begin{proof}
	In the case $\mf g=\mf{so}_N$, we need to have $N\ge7$ to avoid the exceptional isomorphisms. In this case, one finds that $\mf g=\land^2V$, which is $L_{\omega_2}$. We will not say much about the exceptional types, but they can be computed explicitly.
\end{proof}
\begin{remark}
	There is a notion of extended Dynkin diagram which allows one to provide a more algebraic proof of this fact.
\end{remark}

\end{document}