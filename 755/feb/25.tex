% !TEX root = ../notes.tex

\documentclass[../notes.tex]{subfiles}

\begin{document}

\section{February 25}
We continue.

\subsection{The Principal \texorpdfstring{$\mf{sl}_2$}{sl2}-triple}
Let $\mf g$ be a simple finite-dimensional Lie algebra.
\begin{definition}
	Let $\mf g$ be a semisimple finite-dimensional Lie algebra. Then we define $e\coloneqq\sum_ie_i$ and $h\coloneqq2\rho^\lor$. There are coefficients
	\[f\coloneqq\sum_ic_if_i\]
	such that $(e,f,h)$ is an $\mathfrak{sl}_2$-triple. This is the \textit{principal $\mf{sl}_2$-triple}.
\end{definition}
\begin{remark}
	The coefficients are given by $c_i\coloneqq(2\rho^\lor,\omega_i)$, which one can check because this is what is needed for $[e,f]=\sum_ic_ih_i$ to equal $h=\sum_i(2\rho^\lor,\omega_i)h_i$.
\end{remark}
\begin{example}
	For $\mf{gl}_n$, one finds that
	\[e=\begin{bmatrix}
		0 & 1 & 0 & \cdots & 0 & 0 \\
		0 & 0 & 1 & \cdots & 0 & 0 \\
		\vdots & \vdots & \vdots & \ddots & \vdots \\
		0 & 0 & 0 & \cdots & 0 & 1 \\
		0 & 0 & 0 & \cdots & 0 & 0
	\end{bmatrix}\]
	and $h=\op{diag}(n-1,n-3,\ldots,3-n,1-n)$. On the homework, we will calculate $f$.
\end{example}
It is somewhat important to decompose $\mf g$ as a representation of this principal $\mf{sl}_2$.
\begin{definition}[height]
	Fix a semisimple finite-dimensional Lie algebra $\mf g$. Then the \textit{height} $\left|\alpha\right|$ of a root $\alpha$ is the sum of the coefficients when $\alpha$ is expressed a sum of simple roots. We will let $r_m$ be the number of positive roots with height $m$. By convention, we let $r_0$ be the rank of $\mf g$.
\end{definition}
\begin{remark}
	View $\mf g$ as a representation of the principal $\mf{sl}_2$-triple. One can check that $(2\rho^\lor,\alpha)=\left|\alpha\right|$, so $[h,e_\alpha]=2\left|\alpha\right|e_\alpha$. Thus, the weights of $\mf g$ are all even, and
	\[r_m=\dim\mf g[2m].\]
	Thus, the representation theory of $\mf{sl}_2$ assures us that $\{r_m\}$ is a decreasing sequence.
\end{remark}
Tracking through the representation of $\mf{sl}_2$, we are thus motivated to study the differences $r_m-r_{m+1}$.
\begin{definition}[exponent]
	Fix a semisimple finite-dimensional Lie algebra $\mf g$. Then an \textit{exponent} of $\mf g$ is a positive integer $m$ for which $r_m>r_{m+1}$. Its \textit{multiplicity} is $r_m-r_{m+1}$. We may enumerate the exponents (with multiplicity) by $m_1,m_2,\ldots$.
\end{definition}
\begin{example}
	Note that $r_1$ is the rank of $\mf g$ because roots of height one are (by definition) simple.
\end{example}
\begin{example}
	Suppose $\mf g$ is simple. Note that $r_2$ is the number of roots of the form $\alpha_i+\alpha_j$, which happens exactly when $i$ and $j$ are connected. (Indeed, if there is no connection, then the roots are perpendicular.) But the Dynkin diagram is a tree, so $r_2=r_1-1$, so $1$ is an exponent with multiplicity $1$.
\end{example}
\begin{example}
	For $n$ large enough, we see that $r_n=0$, so the number of exponents (enumerated with multiplicity) is exactly $r$.
\end{example}
\begin{example}
	The maximal root $\theta$ corresponds to the exponent $m_r=(\theta,\rho^\lor)$. By definition, $(\theta,\rho^\lor)+1$ is the ``Coxeter number.''
\end{example}
\begin{example}
	For $\mf g=\mf{sl}_n$, we see that roots of height $\ell$ are those of the form $e_i-e_{i+\ell}$. Thus, the exponents are $\{1,2,\ldots,n-1\}$, and the Coxeter number is $n$.
\end{example}
\begin{remark}
	By the representation theory of $\mf{sl}_2$, we see that
	\[\mf g=\bigoplus_{i=1}^rL_{2m_i}.\]
	For example, this implies that $\sum_{i=1}^r(2m_i+1)=\dim\mf g$, so $\sum_{i=1}^rm_i$ is the number of positive roots.
\end{remark}
\begin{example}
	For example, $\mf{sl}_n=L_2\oplus L_4\oplus\cdots\oplus L_{2n-2}$. Another way to see this decomposition is to apply the Clebsch--Gordan rule to compute $L_{n-1}\otimes L_{n-1}^*-\CC$, where ``$-\CC$'' refers to the fact that we are removing the trivial representation.
\end{example}
Here are the exponents for the other Lie algebras.
\begin{remark}
	It turns out that the exponents depends only on the undirected graph isomorphism of the Dynkin diagram, which one can see by Chevalley's theorem.
\end{remark}
\begin{example}
	Here are the exponents for the classical types.
	\begin{itemize}
		\item The exponents of $\mf{sp}_{2n}$ are $\{1,3,\ldots,2n-1\}$.
		\item The exponents of $\mf{so}_{2n+1}$ are $\{1,3,\ldots,2n-1\}$.
		\item The exponents of $\mf{so}_{2n+2}$ are $\{1,3,\ldots,2n-1\}\sqcup\{n\}$, where the extra $n$ at the end is added with multiplicity when $n$ is odd.
	\end{itemize}
\end{example}
\begin{example}
	Here are the exponents for the exceptional types.
	\begin{itemize}
		\item The exponents of $F_4$ are $\{1,5,7,11\}$.
		\item The exponents of $E_6$ are $\{1,4,5,7,,11\}$.
		\item The exponents of $E_7$ are $\{1,5,7,9,11,13,17\}$.
		\item The exponents of $E_8$ are $\{1,7,11,13,17,19,\}$. These are the 
	\end{itemize}
\end{example}
\begin{remark}
	Here is a curious property of these exponents: they are symmetric, in the sense that $m_i+m_{r-i}$ is constant. We will prove this later.
\end{remark}
We close our discussion by saying a little more about the aforementioned Coxeter number.
\begin{definition}[Coxeter number]
	Fix a simple Lie algebra $\mf g$. Then the \textit{Coxeter number} $h_{\mf g}$ is $(\theta,\rho^\lor)+1$, which is one more than the height of the maximal root $\theta$. The \textit{dual Coxeter number} $h_{\mf g}^\lor$ is $(\theta^\lor,\rho)+1$.
\end{definition}
\begin{remark}
	If $(-,-)$ is normalized so that $(\theta,\theta)=2$, then we see that the dual Coxeter number is also $(\theta,\rho)+1$. This normalization is natural because the quadratic Casimir element $C$ for this normalization acts on the adjoint representation by
	\[(\theta,\theta+2\rho)=2h_{\mf g}^\lor.\]
\end{remark}
\begin{remark}
	If $\mf g$ is simply laced (i.e., among the types $A$, $D$, or $E$), then $h_{\mf g}=h_{\mf g}^\lor$ because $\theta$ and $\theta^\lor$ are the same in these simply laced cases. However, they are not the same in general. Indeed, $h^\lor_{B_n}=2n-1$, $h^\lor_{C_n}=n+1$, $h^\lor_{G_2}=4$, and $h^\lor_{F_4}=9$. Notably, this dual Coxeter number is also not the dual Coxeter number of the dual!
\end{remark}

\subsection{Real, Complex, and Quaternionic Representations}
As an application of our principal $\mf{sl}_2$, we say something about dividing our representations into various types.
\begin{defihelper}[real, complex, quaternionic] \nirindex{real} \nirindex{complex} \nirindex{quaternionic}
	Fix a finite-dimensional irreducible representation of a group $G$. Then $V$ has \textit{complex type} if and only if $V\not\cong V^*$. Continuing, given an isomorphism $A\colon V\to V^*$, this induces a dual map $A^*\colon V^*\to V$, so there is a scalar $\lambda$ such that $A^*=\lambda A$ after these identifications. If $A^*=A$, then $V$ has \textit{real type}, and if $A^*=-A$, then $V$ has \textit{quaternionic type}.\todo{}
\end{defihelper}
\begin{remark}
	If we forget about the complex structure of $V$, merely viewing it in $\op{Rep}_\RR(V)$, then it turns out that $\op{End}_{\RR[G]}(V)$ is $\CC$ for complex type, $M_2(\RR)$ for the real type, and $\mathbb H$ for quaternionic type. We will prove this on the homework.
\end{remark}
\begin{remark}
	On the homework, we will show that when $G$ is finite, $V$ has real type if and only if $V$ admits a basis in which all elements of $G$ act by matrices with real coefficients.
\end{remark}
\begin{example} \label{ex:sl2-types}
	Consider $G=\op{SL}_2(\CC)$, and consider the irreducible representation $L_n=\op{Sym}^n\CC^2$. Then $L_1$ is of quaternionic type, so one can check that $L_n$ has quaternionic type for $n$ odd and real type for $n$ even.
\end{example}
\begin{example}
	For $G=\op{SL}_n(\CC)$ for $n\ge3$, one has that $V$ and $V^*$ are non-isomorphic, so $V$ has complex type.
\end{example}
It is thus interesting to know which representations of a Lie group admit which types.
\begin{proposition} \label{prop:check-type}
	Fix a Lie simply connected Lie group $G$ with Lie algebra $\mf g$, and choose a dominant weight $\lambda$.
	\begin{listalph}
		\item If $\lambda\ne-w_0\lambda$, then $L_\lambda$ is of complex type.
		\item If $\lambda=-w_0\lambda$, then $L_\lambda$ is of real type if $(\lambda,2\rho^\lor)$ is even and is of quaternionic type if $(\lambda,2\rho^\lor)$ is odd.
	\end{listalph}
\end{proposition}
\begin{proof}
	Part (a) follows because $L_\lambda^*=L_{-w_0\lambda}$ by \Cref{rem:dual-weight}.

	For (b), we are granted an isomorphism $A\colon L_\lambda\to L_\lambda^*$. We now restrict $L_\lambda$ to the principal $\mf{sl}_2$-triple $(e,f,h)$ of $\mf g$. Then the largest eigenvalue of $h$ on $L_\lambda$ comes from its maximal weight, so this eigenvalue is $m\coloneqq(2\rho^\lor,\lambda)$; furthermore, we see that $L_\lambda[m]$ is one-dimensional. The other weight vectors in $L_\lambda$ have weights
	\[(2\rho^\lor,\lambda-\beta)=m-2\sum_i(\omega_i,\beta),\]
	which we note is strictly less than $m$. Thus, $A$ must carry $L_m$ to $L_m^*$, so we are done by \Cref{ex:sl2-types}!
\end{proof}
Let's use \Cref{prop:check-type} to determine the types of the spin representations.
\begin{example} \label{ex:even-so-type}
	Consider $\mf g=\mf{so}_{2n}$ so that $\rho^\lor=\rho=(n-1,n-2,\ldots,0)$. Then one sees that $S_+=L_{\omega_n}$ and $S_-=L_{\omega_{n-1}}$, where $\omega_n=(1/2,\ldots,1/2)$ and $\omega_{n-1}=(1/2,\ldots,1/2,-1/2)$. We know that $S_+$ and $S_-$ are self-dual only when $n$ is even (by looking at the action of $-w_0$). For example, we can calculate
	\[(\omega_n,2\rho^\lor)=\frac{n(n-1)}2,\]
	so $S_+$ is of real type for $n\equiv0\pmod4$ and of quaternionic type for $n\equiv2\pmod4$.
\end{example}
\begin{example} \label{ex:odd-so-type}
	Consider $\mf g=\mf{so}_{2n+1}$ so that $\rho^\lor=(n,n-1,\ldots,1)$ and $S=L_{\omega_n}$ has $\omega_n=(1/2,\ldots,1/2)$. Then $S$ is always self-dual because $-\omega_0$ is trivial, and
	\[(\omega_n,2\rho^\lor)=\frac{n(n+1)}2.\]
	Thus, $S$ is of quaternionic type when $n\equiv1,2\pmod4$ and is of real type when $n\equiv0,3\pmod4$.
\end{example}
The above two examples combine into the following.
\begin{theorem}[Real Bott periodicity]
	The type of the spin representations of $\mf{so}_n$ is uniquely determined by $n\pmod 8$.
\end{theorem}
\begin{proof}
	Combine \Cref{ex:even-so-type,ex:odd-so-type}.
\end{proof}
\begin{remark}
	This is related to Bott periodicity in homotopy theory, but I do not know how.
\end{remark}

\subsection{Review of Topology}
We end our class by saying something about integration.
\begin{definition}[locally compact]
	Fix a Hausdorff topological space $X$. Then $X$ is \textit{locally compact} if and only if any $p\in X$ admits an open neighborhood $U$ for which the closure $\overline U$ is compact.
\end{definition}
\begin{example}
	The base of balls in $\RR^n$ shows that $\RR^n$ is locally compact. Because manifolds are locally homeomorphic to $\RR^n$, we see that manifolds are locally compact.
\end{example}
\begin{lemma} \label{lem:filter-locally-compact-space}
	Fix a second countable locally compact space $X$. Then $X$ admits a filtration $\{K_i\}$ of compact sets for which $X=\bigcup_iK_i$, and each $x\in X$ admits an open neighborhood contained in some $K_i$.
\end{lemma}
\begin{proof}
	The idea is that $X$ being second countable means that any open cover can be refined (and shrunken) into a countable one. Indeed, for an open cover $\mc U$, define a function from the base $\mc B$ to $\mc U$ by sending a basic open subset $B$ to one in $\mc U$ which contains it, if one exists. The image of this map is the required subcover.
	
	Thus, for each $x\in X$, we find an open neighborhood $U_x$ such that $\overline{U}_x$ is compact. Then $\{U_x\}_{x\in X}$ can be refined to a countable subcover $\{U_i\}_{i=1}^n$, and defining
	\[K_j\coloneqq\bigcup_{i\le j}\overline{U_i}\]
	will do the trick.
\end{proof}
\begin{definition}[locally finite]
	An open cover $\mc U$ of a topological space $X$ is \textit{locally finite} if and only if each $x\in X$ admits an open neighborhood $V$ for which
	\[\{U\in\mc U:U\cap V\ne\emp\}\]
	is finite for each $x\in U$.
\end{definition}
\begin{lemma}
	Fix a second countable locally compact topological space $X$. Then every base of $X$ has a countable, locally finite subcover.
\end{lemma}
\begin{proof}
	By \Cref{lem:filter-locally-compact-space}, we may filter $X$ as $X=\bigcup_iK_i$ by compact sets as described.

	Let our base be $\mc B$, and we will build our subcover $\mc U$ by removing elements from $\mc B$, starting with $\mc U=\mc B$. We may immediately assume that it is countable because $X$ is second countable, so we may enumerate $\mc U$ as $\{U_i\}_{i\in\NN}$. We now apply the following inductive process for each $n$, starting with $n=1$.
	\begin{enumerate}
		\item Because $K_n$ is compact, we are granted a positive integer $N$ for which $\{U_i\}_{i\le N}$ covers $K_n$, which we keep in $\mc U$.
		\item We then remove from $\mc U$ all other open subsets which intersect with $K_n$, and go back to the previous step now with $K_{n+1}$. Note that because $\mc B$ was a base, any element $x\notin K_n$ will admit an open neighborhood avoiding $K_n$, so $\mc U$ continues to be a base on $X\setminus K_n$.
	\end{enumerate}
	At the end of this process, we see that $\mc U$ is certainly a countable cover of $X$, and it is locally finite because any point $x\in X$ admits an open neighborhood $U$ contained in one of the $K_n$s, and each $K_n$ only has intersection with finitely many open subsets in $\mc U$.
\end{proof}
Let's also say something about differential forms.
\begin{definition}[differential form]
	Fix a manifold $M$ of dimension $n$. For a positive integer $k\le n$, we define $\Omega^k(M)$ to be the space of \textit{differential forms} of degree $k$. Explicitly, $\Omega^k(M)\coloneqq\land^kT^*M$.
\end{definition}
\begin{remark}
	In local coordinates $(x_1,\ldots,x_n)$ of $M$, we see that a local frame of $T^*M$ is given by $\{dx_1,\ldots,dx_n\}$. Thus, some $\omega\in\Omega^k(M)$ takes the form
	\[\omega=\sum_{i_1<\cdots<i_k}f_{i_1\ldots i_k}(x)\,dx_{i_1}\land\cdots\land dx_{i_k}.\]
	If we wanted to change coordinates to some $(y_1,\ldots,y_n)$, where we now think of $x_i$ as a function of the $y$s, then
	\[\omega=\sum_{\substack{i_1<\cdots<i_k\\j_1<\cdots<j_k}}f_{i_1\ldots i_k}\det\left[\frac{\del(x_{i_1},\ldots,x_{i_k}}{\del(y_{j_1},\ldots,y_{j_k})}\right]dy_{j_1}\land\cdots\land dy_{j_k}.\]
\end{remark}
\begin{definition}[de Rham cohomology]
	Fix a manifold $M$. There is a differential map $d\colon\OO(M)\to\Omega(M)$ given by sending a function $f$ to the covector $df$; this extends to a map $d\colon\Omega^k(M)\to\Omega^{k+1}(M)$. Then the \textit{$k$th de Rham cohomology} of $M$ is given by
	\[\mathrm H^k_{\mathrm{dR}}(M)=\frac{\ker\left(d\colon\Omega^k(M)\to\Omega^{k+1}(M)\right)}{\im\left(d\colon\Omega^{k-1}(M)\to\Omega^k(M)\right)}.\]
\end{definition}

\end{document}