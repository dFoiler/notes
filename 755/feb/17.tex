% !TEX root = ../notes.tex

\documentclass[../notes.tex]{subfiles}

\begin{document}

\section{February 17}
Today we discuss miniscule weights.

\subsection{Miniscule Weights}
Here is our definition.
\begin{definition}[miniscule]
	Fix a semisimple Lie algebra $\mf g$. A dominant integral weight $\omega$ of $\mf g$ is \textit{miniscule} if and only if $(\omega,\beta)\le1$ for all coroots $\beta$.
\end{definition}
\begin{remark}
	It is enough to check the condition $(\omega,\beta)\le1$ for positive coroots.
\end{remark}
\begin{example}
	The weight $0$ is always miniscule.
\end{example}
\begin{example} \label{ex:miniscule-rep-gln}
	For $\mf g=\mf{gl}_n(\CC)$, all the fundamental weights $\omega_i$ are miniscule. Indeed, the fundamental weights are
	\[\omega_i=(\underbrace{1,\ldots,1}_i,0,\ldots,0).\]
	But now $(\omega_i,e_j-e_k)\in\{0,+1\}$ for each $j<k$, depending on the relative placement of $i$ between $j$ and $k$. Indeed, it is zero if $j$ and $k$ are both below or both above $i$, and it is one if $i$ is between them.
\end{example}
\begin{remark} \label{rem:miniscule-to-fundamental}
	Any nonzero miniscule weight $\omega$ is fundamental; we may assume that $\mf g$ is simple. Let $\{\alpha_1,\ldots,\alpha_r\}$ be a collection of simple roots, and let $\theta^*$ be the maximal coroot, which takes the form $\sum_im_i\alpha_i^\lor$ where each $m_i$ is positive. (One way to find this (co)weight is as the highest weight of the (co)adjoint representation. It then turns out that this $\theta^*$ is a coroot---the weights of the adjoint representation are all roots---for which $\theta^*-\alpha_i^\lor$ is a positive coroot always.) Then
	\[(\omega,\theta^*)=\sum_im_i(\omega,\alpha_i^\lor).\]
	Thus, there is at most one index $i$ with $(\omega,\alpha_i^\lor)=1$, which means that $\omega=0$ or $\omega$ equals one of the $\omega_i$s (because the fundamental weights form a dual basis).
\end{remark}
It is helpful to characterize our miniscule weights.
\begin{lemma}
	Fix a semisimple Lie algebra $\mf g$, and choose a fundamental weight $\omega_i$. Then $\omega_i$ is miniscule if and only if $m_i=1$, where $m_i$ comes from the expansion $\theta^*=\sum_jm_j\alpha_j^\lor$ of the maximal coroot.
\end{lemma}
\begin{proof}
	Certainly if $\omega_i$ is miniscule, then $(\omega,\theta^*)\le1$ implies that $m_i\le1$ and so $m_i=1$. Conversely, if $m_i=1$, then for any positive coroot
	\[\beta=\sum_jn_j\alpha_j^\lor,\]
	we see that we must have $n_j\le m_j$ for each $j$, so $(\omega_i,\beta)\le1$ follows.
\end{proof}
\begin{example}
	A direct calculation of the maximal coroot shows that the Lie algebras with exceptional types $E_2$, $F_4$, and $E_8$ admit no miniscule weights.
\end{example}
\begin{lemma} \label{lem:miniscule-not-root}
	Fix a semisimple Lie algebra $\mf g$. Suppose that we have $\omega\in Q$ for which $(\omega,\beta)\le1$ for all coroots $\beta$. Then $\omega=0$.
\end{lemma}
\begin{proof}
	Because $\omega$ is in the root lattice, we may write
	\[\omega=\sum_ik_i\alpha_i\]
	for some integers $k_\bullet$. For the sake of contradiction, assume that $\omega$ is nonzero; we may assume that $\omega$ is a minimal counterexample, in the sense that it minimizes $\sum_i\left|k_i\right|$. Because the inner product is positive-definite, we know that $(\omega,\omega)$ is positive, so
	\[\sum_ik_i(\omega,\alpha_i)>0.\]
	Thus, one of the summands is positive, so we may find an index $j$ such that $k_j$ and $(\omega,\alpha_j)$ have the same sign; by possibly replacing $\omega$ with $-\omega$, we may assume that $k_j>0$ and $(\omega,\alpha_j)>0$. It follows that $(\omega,\alpha_j^\lor)>0$, so $(\omega,\alpha_j^\lor)=1$ by hypothesis.
	
	Thus, $s_i\omega=\omega-\alpha_i$, so $s_i\omega$ is another vector in the root lattice, and we can see that $s_i\omega$ also satisfies the hypothesis. Indeed, the hypothesis $(\omega,\beta)\le1$ and being in the root lattice is invariant under the Weyl group because the Weyl group preserves the set of coroots and the inner product. As such, we have arrived at our contradiction because the sum of the coordinates of $s_i\omega$ is strictly smaller than the sum of the coordinates of $\omega$.
\end{proof}

\subsection{Miniscule Representations}
Miniscule weights are interesting to us because their corresponding representations are easy to understand.
\begin{proposition} \label{prop:miniscule-rep}
	Fix a semisimple Lie algebra $\mf g$, and choose a dominant integral weight $\omega$. Then the following are equivalent.
	\begin{listalph}
		\item The weight $\omega$ is miniscule.
		\item All weights of $L_\omega$ are contained in the orbit $W\omega$.
		\item If $\lambda$ is a dominant integral weight such that $\omega-\lambda\in Q_+$, then $\lambda=\omega$.
	\end{listalph}
\end{proposition}
\begin{proof}
	We show our implications in sequence.
	\begin{itemize}
		\item We show (a) implies (c), which we will do by induction on the rank of $\mf g$. (There is nothing to do if the rank is zero.) We proceed in steps.
		\begin{enumerate}
			\item If $\omega=0$, then $\lambda\in-Q_+$, so $(\lambda,\rho)\le0$; but because $\lambda$ is dominant, if it is nonzero, then $(\lambda,\rho)>0$.

			We may now suppose that $\omega$ is nonzero.

			\item Then $\omega$ is a fundamental weight $\omega_i$ (by \Cref{rem:miniscule-to-fundamental}). Now, write $\omega-\lambda=\sum_jn_j\alpha_j$, where the $n_j$s are nonnegative. If $n_j$ vanishes for some $j$ with $j\ne i$, then we may pass to a smaller Lie algebra (by deleting the $j$th vertex of the Dynkin diagram), so we are done by the induction.

			Thus, we may assume that $n_j>0$ for each $j\ne i$.

			\item Now, for any positive coroot $\beta$, we know that
			\[(\omega_i-\lambda,\beta)=(\omega_i,\beta)-(\lambda,\beta)\le(\omega_i,\beta)\le1.\]
			Additionally, if $\alpha_i^\lor$ is not found in the expansion of $\beta$, then we must actually have $(\omega_i,\beta)\le0$; for example, $(\omega_i-\lambda,\alpha_j^\lor)\le0$ for all $j\ne i$.

			\item Now, observe that
			\[(\omega_i-\lambda,\omega_i-\lambda)=\sum_jn_j(\omega_i-\lambda,\alpha_i^\lor).\]
			If this is nonpositive, then $\omega_i-\lambda$ vanishes, and we are done. But each of the $n_j$s are nonnegative for each $j\ne i$, and we just showed $(\omega_j-\lambda,\alpha_j^\lor)\le0$ for $j\ne i$, so the only way to avoid nonpositivity is for $n_i>0$ and $(\omega_i-\lambda_i,\alpha_i^\lor)>0$.

			We are thus reduced to the case that $n_j>0$ for all $j$, including $j=i$. We are now ready to conclude. Recall $\theta^*$ is a dominant coweight, so $(\omega_i-\lambda,\theta^*)\ge1$ because we just showed that $(\omega_i-\lambda,\alpha_i^\lor)>0$ for all $i$. (In a few more words, one can expand $\omega_i-\lambda=\sum_jn_j\alpha_j$ and $\theta^*=\sum_jp_j\omega_j^\lor$, where the $n_j$s are positive and the $p_j$s are nonnegative, with at least one nonzero.) On the other hand, $(\omega_i,\theta^*)\le1$, so $(\lambda,\theta^*)\le0$, so $\lambda=0$, so $\omega_i\in Q_+$, which contradicts \Cref{lem:miniscule-not-root}.
		\end{enumerate}

		\item We show that (c) implies (b). Let $\mu$ be a weight of $L_\omega$. Then we can find $w\in W$ to push $\mu$ into the dominant chamber, meaning that $\lambda\coloneqq w\mu$ is a dominant weight. But the construction of $L_\omega$ implies that $\omega-\lambda\in Q_+$, so $\omega=\lambda$ follows by (c).

		\item We show that (b) implies (a) by contraposition. Indeed, suppose that $\omega$ is not miniscule, which implies that there is a positive root $\alpha$ such that $(\omega,\alpha^\lor)>1$. Unravelling $\alpha^\lor$, we see that $2(\omega,\alpha)>(\alpha,\alpha)$. Thus, $\omega-\alpha$ is a weight of $L_\omega$, witnessed by the weight vector $f_\alpha v_\omega$, which we note is nonzero by the representation theory of $\mf{sl}_{2,\alpha}$ (because $2(\omega,\alpha)>(\alpha,\alpha)$).

		We will be done as soon as we can check that $\omega-\alpha$ is not Weyl-conjugate to $\omega$, but this is not hard: indeed,
		\[(\omega-\alpha,\omega-\alpha)=(\omega,\omega)-2(\omega,\alpha)+(\alpha,\alpha)<(\omega,\omega).\]
	\end{itemize}
	The above implications complete the proof.
\end{proof}
\begin{corollary}
	Fix a miniscule weight $\omega$ of a semisimple Lie algebra $\mf g$. Then
	\[\op{ch}L_\omega=\sum_{\gamma\in W\omega}e^\gamma.\]
\end{corollary}
\begin{proof}
	This follows from part (b) of \Cref{prop:miniscule-rep}. Note that the multiplicities of these weight spaces must be $1$ because the multiplicity of the highest weight space needs to be $1$, and the Weyl group permutes the weight spaces $L[\gamma]$ as $\gamma$ varies over $W\omega$ (by the definition of the Weyl group action).
\end{proof}
\begin{proposition}
	Fix a dominant integral weight $\omega$ of a semisimple Lie algebra $\mf g$. Then $\omega$ is miniscule if and only if the restriction of $L_\omega$ to any simple $\mf{sl}_{2,\alpha}\subseteq\mf g$ is a direct sum of one-dimensional and two-dimensional representations for each $\alpha$.
\end{proposition}
\begin{proof}
	In the forward direction, we may suppose that $\omega$ is miniscule. Then any weight vector $v\in L_\omega$ which is a highest weight vector for some $\mf{sl}_{2,\alpha}$ has $h_\alpha v=(\omega,\alpha^\lor)v$. But then $\left|(\omega,\alpha^\lor)\right|\le1$, so the result follows.

	In the backward direction, if $\omega$ is not miniscule, then we may find a positive root $\alpha$ such that $(\omega,\alpha^\lor)\ge2$. Then the highest weight vector $v\in L_\omega$ admits $h_\alpha v=(\omega,\alpha^\lor)v$, so $v$ generates a $\mf{sl}_{2,\alpha}$-representation of dimension $(\omega,\alpha^\lor)+1$.
\end{proof}

\subsection{Applications of Miniscule Representations}
We will get significant mileage out of the following application of the Weyl character formula.
\begin{corollary}
	Fix a miniscule weight $\omega$ of a semisimple Lie algebra $\mf g$. Then for all dominant integral weights $\lambda$,
	\[L_\omega\otimes L_\lambda=\bigoplus_{\gamma\in W\omega}L_{\lambda+\gamma},\]
	where $L_{\lambda+\gamma}$ vanishes (by convention) if and only if $\lambda+\gamma$ is not dominant.
\end{corollary}
\begin{proof}
	By iteratively removing highest weight vectors, it is enough to check that the two representations have the same character. Well, by the Weyl character formula, we see that
	\begin{align*}
		\op{ch}(L_\omega\otimes L_\lambda) &= \op{ch}(L_\omega)\op{ch}(L_\lambda) \\
		&= \sum_{\gamma\in W\omega}e^\gamma\cdot\frac{\sum_{w\in W}\op{sgn}(w)e^{w(\lambda+\rho)}}{\prod_{\alpha\in\Phi+}(e^{\alpha/2}-e^{-\alpha/2})} \\
		&= \frac1{\prod_{\alpha\in\Phi+}(e^{\alpha/2}-e^{-\alpha/2})}\sum_{\substack{w\in W\\\gamma\in W\omega}}\op{sgn}(w)e^{w(\lambda+\rho)+\gamma}.
	\end{align*}
	We now replace $\gamma$ with $w\gamma$, merely moving around some orbit, which gives
	\[\op{ch}(L_\omega\otimes L_\lambda)=\sum_{\gamma\in W\omega}\frac1{\prod_{\alpha\in\Phi+}(e^{\alpha/2}-e^{-\alpha/2})}\sum_{w\in W}\op{sgn}(w)e^{w(\lambda+\rho+\gamma)}.\]
	Each internal sum does look like $\op{ch}L_{\lambda+\gamma}$, so we will be done as soon as we check that it vanishes if $\lambda+\gamma$ fails to be dominant. Well, suppose that $\lambda+\gamma$ is not dominant; then we can find $\alpha_i^\lor$ such that $(\lambda+\gamma,\alpha_i^\lor)<0$. On the other hand, $(\gamma,\alpha_i^\lor)\ge-1$ because $\gamma\in W\omega$, and $\omega$ is miniscule, so $(\lambda+\gamma,\alpha_i^\lor)=-1$, so $(\lambda+\gamma+\rho,\alpha_i^\lor)=0$. Thus, $s_i$ fixes $\lambda+\gamma+\rho$, so the terms $\op{sgn}(w)e^{w(\lambda+\rho+\gamma)}$ and $\op{sgn}(s_iw)e^{w(\lambda+\rho+\gamma)}$ in our sum will cancel each other out.
\end{proof}
Let's apply this to $\mf{sl}_n(\CC)$.
\begin{example} \label{ex:gln-tensor-std}
	We work with $\mf g=\mf{sl}_n(\CC)$. Let $V$ be the standard representation of $\op{GL}_n(\CC)$, which is a miniscule representation by \Cref{ex:miniscule-rep-gln}. The weights of $V$ are the standard basis of the diagonal Cartan $\mf h\subseteq\mf{sl}_n(\CC)$. Then for any partition $\lambda$, we see that
	\[V\otimes L_\lambda=\bigoplus_{i=1}^nL_{\lambda+e_i},\]
	where $L_{\lambda+e_i}$ vanishes by convention if $\lambda+e_i$ fails to be a partition. In other words, $V\otimes L_\lambda$ consists of those $L_\mu$s, where the Young diagram for $\mu$ is obtained by adding a box (suitably) to the Young diagram for $\lambda$. For example,
	\[V\otimes L_{(6,4,4,2,1)}=L_{(7,4,4,2,1)}\oplus L_{(6,5,4,2,1)}\oplus L_{(6,4,4,3,1)}\oplus L_{(6,4,4,2,2)}\oplus L_{(6,4,4,2,1,1)}.\]
\end{example}
\begin{example}
	Continue in the setting of \Cref{ex:gln-tensor-std}. Then $\land^iV$ is the miniscule representation attached to the fundamental weight $\omega_i$ (see \Cref{ex:miniscule-rep-gln}). Thus, for any partition $\lambda$, we see that $\land^iV\otimes L_\lambda$ contains those permutations $\mu$ which are obtained by adding one to $i$ different coordinates in the partition $\lambda$, in a way that ensures that $\mu$ is still a partition. Combinatorially, we are trying to add $k$ different boxes to different rows to the Young diagram of $\lambda$ in a way that ensures that we still have a Young diagram afterward.
\end{example}
We can now combine this discussion with Schur--Weyl duality.
\begin{corollary} \label{cor:add-box-pi-lambda}
	Fix some positive integer $N$.
	\begin{listalph}
		\item For any partition $\lambda\vdash N$, we have
		\[\CC[S_{N+1}]\otimes_{\CC[S_N]}\pi_\lambda=\bigoplus_{\mu\in\lambda+\square}\pi_\mu,\]
		where $\mu$ varies over those Young diagrams obtained by suitably adding a box to the Young diagram of $\lambda$.
		\item For any partition $\mu\vdash N+1$, we have
		\[\pi_\mu|_{S_N}=\bigoplus_{\lambda\in\mu-\square}\pi_\lambda,\]
		where $\lambda$ varies over those Young diagrams obtained by suitably removing a box to the Young diagram of $\lambda$.
	\end{listalph}
\end{corollary}
\begin{proof}
	We quickly explain that (a) implies (b) by Frobenius reciprocity. (A similar argument also shows that (b) implies (a).) Indeed, by Frobenius reciprocity, some $\pi_\lambda$ has multiplicity $m$ in the restriction $\pi_\mu|_{S_N}$ if and only if $\pi_\mu$ has multiplicity $m$ in the induction $\CC[S_{N+1}]\otimes_{\CC[S_N]}\pi_\lambda$. But (a) tells us exactly that this multiplicity vanishes except when $\mu$ is obtained by adding a single box to $\lambda$ (in which case the multiplicity is one). Unwinding this produces the required decomposition of $\pi_\mu$.

	Thus, we will content ourselves with showing (a). Well, define $V\coloneqq\CC^n$, where $n$ is large (e.g., larger than $N+1$). Frobenius reciprocity implies that
	\begin{align*}
		\op{Hom}_{S_{N+1}}\left(\CC[S_{N+1}]\otimes_{\CC[S_N]}\pi_\lambda,V^{\otimes(N+1)}\right) &= \op{Hom}_{S_{N}}\left(\pi_\lambda,V\otimes V^{\otimes N}\right),
	\end{align*}
	and this latter representation is simply $V\otimes S^\lambda V$ by definition of the Schur functor $S^\lambda$ (and the fact that $S_N$ is acting trivially on $V$). On the other hand,
	\[\op{Hom}_{S_{N+1}}\Bigg(\bigoplus_{\mu\in\lambda+\square}\pi_\mu,V^{\otimes(N+1)}\Bigg)=\bigoplus_{\mu\in\lambda+\square}S^\mu V\]
	by definition of the Schur functors. Now, \Cref{ex:gln-tensor-std} has shown that our two calculations have produced the same representation of $\op{GL}(V)$, so \Cref{thm:schur-weyl} tells us that the corresponding representations of $S_{N+1}$ must also be the same.
\end{proof}

\end{document}