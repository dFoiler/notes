% !TEX root = ../notes.tex

\documentclass[../notes.tex]{subfiles}

\begin{document}

\section{February 9}
Today we discuss Schur--Weyl duality.

\subsection{Representations of \texorpdfstring{$\op{GL}_n(\CC)$}{ GLn(C)}}
Here is our result.
\begin{proposition}
	Irreducible representations of $\op{GL}_n(\CC)$ are parameterized by pairs $(m,\lambda)$, where $m\in\ZZ$ and $\lambda=(\lambda_1,\ldots,\lambda_n)$ is a decreasing sequence of nonnegative integers with $\lambda_n=0$, and
	\[m=\sum_i\lambda_i+nr\]
	for some integer $r$. The highest weight of the corresponding irreducible representation $L_{m,\lambda}$ is $(\lambda_1+r,\ldots,\lambda_n+r)$.
\end{proposition}
\begin{proof}
	It is not quite the case that $\op{GL}_n(\CC)$ is $\mathbb G_m\times\op{SL}_n(\CC)$, but the natural covering map
	\[\mathbb G_m(\CC)\times\op{SL}_n(\CC)\to\op{GL}_n(\CC)\]
	has finite kernel $\mu_n\subseteq\CC^\times$. Thus, the irreducible representations of $\op{GL}_n(\CC)$ consist of the irreducible representations of $\mu_n\times\op{SL}_n(\CC)$ for which $\mu_n$ acts by the identity. Notably, such an irreducible representation can be described as a tensor product of a representation of $\mathbb G_m(\CC)$ and a representation of $\op{SL}_n(\CC)$, with some condition on the diagonal.
	
	However, irreducible representations of $\CC^\times$ are one-dimensional. Passing to the Lie algebra, we see that we are basically given the data of an operator $h$ for which $e^{2\pi ih}=1$, which means that the operator admits integer eigenvalues, so we see that the corresponding representations of $\CC^\times$ are given by the characters $\chi_m(z)\coloneqq z^m$.

	Thus, irreducible representations of $\CC^\times\op{SL}_n(\CC)$ can be described as tensor products
	\[L_{m,\lambda}\coloneqq\chi_m\otimes L_\lambda,\]
	where $\lambda$ is some weight of $\mf{sl}_n(\CC)$. Recall that such weights can be described as decreasing sequences of $n$ integers with $\lambda_n=0$. Notably, $\mu_n$ acts on $L_{m,\lambda}$ by sending the scalar $z$ to $z^{-m+\sum_i\lambda_i}$, which needs to be divisible by $n$ for our representation to descend to $n$. We can compute the highest weight of $L_{m,\lambda}$ as $(\lambda_1+r,\ldots,\lambda_n+r)$.
\end{proof}
\begin{remark}
	Equivalently, we may say that representations of $\op{GL}_n(\CC)$ are parameterized by decreasing sequences of integers $\lambda=(\lambda_1,\ldots,\lambda_n)$, which is the highest weight.
\end{remark}
\begin{example}
	We can compute that there is now a fundamental weight $(1,\ldots,1)$, which is the determinant representation.
\end{example}
Here is an important class of our representations.
\begin{definition}[polynomial]
	An irreducible representation $L_\lambda$ of $\op{GL}_n(\CC)$ is \textit{polynomial} if and only if $\lambda_n\ge0$.
\end{definition}
\begin{remark}
	We can think about the sequence $\lambda$ as instead being a partition
	\[\mu\coloneqq(\lambda_1-\lambda_2,\ldots,\lambda_{n-1}-\lambda_n)\]
	of $\lambda_1$. We may then write $L_\mu\coloneqq L_\lambda$.
\end{remark}
\begin{remark}
	Partitions are labeled by Young diagrams, which we notably have no use for.
\end{remark}
\begin{remark}
	One can check that $L_\lambda$ is polynomial if and only if $L_\lambda$ embeds into $V^{\otimes N}$ for some $N$, where $V$ is the standard representation. In particular, we see that $L_\lambda$ is contained in $V^{\otimes\left|\mu\right|}$, which we can see by the same argument as in \Cref{prob:find-reps-in-tensors}. The point is that the determinant representation can still be found in $V^{\otimes n}$, but the inverse of the determinant cannot!
\end{remark}
\begin{remark}
	One can check that polynomial representations are precisely those which extend continuously to $M_n(\CC)$; equivalently, we may say that these representations are those for which the matrix coefficients are polynomial (not just rational). Again, the point is that the determinant representation has polynomial matrix coefficients, but the inverse of the determinant does not.
\end{remark}

\subsection{Schur--Weyl Duality}
Having said something about the representations of $\op{GL}_n(\CC)$, we are ready to relate them to the symmetric group. Set $V\coloneqq\CC^n$. Then we note that both $\op{GL}_n(\CC)$ and $S_N$ act on $V^{\otimes N}$, where notably $S_N$ acts by permuting the coordinates.
\begin{theorem}[Schur--Weyl duality] \label{thm:schur-weyl}
	Fix positive integers $n$ and $N$, and set $V\coloneqq\CC^n$. Let $A$ be the image of the natural map $U\mf{gl}(V)\to\op{End}_\CC V^{\otimes N}$, and let $B$ be the image of the natural map $\CC[S_N]\to\op{End}_\CC V^{\otimes N}$.
	\begin{listalph}
		\item The algebras $A$ and $B$ are centralizers of each other.
		\item There is a decomposition
		\[V^{\otimes N}=\bigotimes_{\lambda\vdash N}L_\lambda\otimes\pi_\lambda\]
		of $V^{\otimes N}$ into irreducible representations of $A\times B$; the indexing consists of partitions of $N$ with at most $n$ parts.
		\item As $\lambda$ varies over partitions of at most $n$ parts, then the $\pi_\lambda$s are pairwise non-isomorphic irreducible representations of $S_N$.
		\item If $n\ge N$, then the $\pi_\lambda$s exhaust all irreducible representations of $S_N$.
	\end{listalph}
\end{theorem}
\begin{remark}
	Note that $n=N$ implies that all permutations of $N$ have at most $n$ parts, so both (c) and (d) apply.
\end{remark}
\begin{remark}
	The existence of the decomposition in (b) should not be a surprise: for any group $G$ and completely reducible representation $Y$ of $G$, the natural map
	\[\bigoplus_{W\in\op{IrRep}(G)}W\otimes\op{Hom}_G(W,Y)\to Y\]
	is an isomorphism; indeed, to prove this, one simply decomposes $Y$ and checks that the statement is true on irreducibles, where it follows from Schur's lemma. For (b), we see that we can simply let $\pi_\lambda$ be the space $\op{Hom}_{\op{GL}_n(\CC)}(L_\lambda,V^{\otimes N})$.
\end{remark}
\begin{definition}[Schur algebra]
	The algebra $A$ in \Cref{thm:schur-weyl} is the \textit{Schur algebra}, denoted $\mc S_{n,N}$. The algebra $B$ is then justly called the \textit{centralizer algebra}, denoted $C_{n,N}$.
\end{definition}
We are going to need a few lemmas. Our first goal is to show that $A$ is the centralizer of $B$.
\begin{lemma} \label{lem:get-sym-power-space}
	Let $U$ be a complex vector space. Then $S^NU$ is spanned by the pure tensors
	\[\{x\otimes\cdots\otimes x:x\in U\}.\]
\end{lemma}
\begin{proof}
	Any $u\in S^NU$ has finitely many terms, so by taking the finite number of vectors contained therein, we may reduce to the finite-dimensional case. Now, the span of our pure tensors forms a nonzero subrepresentation of $S^NU$ (for the group $\op{GL}(U)$), but this representation is already known to be irreducible by \Cref{lem:sym-power-irred}, so the span must cover everything.
\end{proof}
\begin{lemma} \label{lem:get-sym-power-alg}
	Let $R$ be an algebra over $\CC$, and let $S^NR$ be the $N$th symmetric power, which we note is also an algebra. Now, define $\Delta\colon R\to S^NR$ by
	\[\Delta(x)\coloneqq(x\otimes1\otimes\cdots\otimes1)+(1\otimes x\otimes\cdots\otimes1)+\cdots+(1\otimes1\otimes\cdots\otimes x).\]
	Then $R$ is generated as an algebra by elements of the form $\Delta(x)$.
\end{lemma}
\begin{proof}
	By Newton's identities, there is a polynomial $P_N$ (with rational coefficients) for which
	\[z_1\cdots z_N=P_N\left(\sum_{i=1}^Nz_i^1,\ldots,\sum_{i=1}^Nz_i^N\right).\]
	(This is an identity that takes place in $\QQ[z_1,\ldots,z_N]$.) It follows that
	\[x\otimes\cdots\otimes x=P_N\left(\Delta(x),\ldots,\Delta(x^N)\right)\]
	for any $x\in R$, so we may conclude by \Cref{lem:get-sym-power-space}.
\end{proof}
\begin{lemma} \label{lem:one-centralizer-schur-weyl}
	In the context of \Cref{thm:schur-weyl}, the algebra $A$ is the centralizer of $B$.
\end{lemma}
\begin{proof}
	Note that certainly $A$ and $B$ commute with each other with no effort because $B$ only permutes the factors of $V$. On the other hand, note that the centralizer $Z_B$ of $B$ is
	\begin{align*}
		Z_B &= \op{End}\left(V^{\otimes N}\right)^{S_N} \\
		&\stackrel*= \left(\op{End}_\CC(V)^{\otimes N}\right)^{S_N} \\
		&= S^N\op{End}_\CC(V).
	\end{align*}
	Here, $\stackrel*=$ holds because there is a natural map going upwards, and it can be seen to be an isomorphism on a basis. Thus, by \Cref{lem:get-sym-power-alg}, it is enough to show that the elements $\Delta(\varphi)$ are in $A$ for each $\varphi\in\op{End}_\CC(V)$, which is true by definition of $A$.
\end{proof}
The rest of Schur--Weyl duality follows from the following piece of algebra.
\begin{lemma}[Double centralizer] \label{lem:double-centralizer}
	Fix a finite-dimensional vector space $V$ over any field $k$, and choose two algebras $A,B\subseteq\op{End}_kV$. Suppose that $B$ is a sum of matrix algebras, and $A$ is the centralizer of $B$.
	\begin{listalph}
		\item The algebra $A$ is isomorphic to a sum of matrix algebras.
		\item The algebra $B$ is the centralizer of $A$.
		\item There is a decomposition
		\[V=\bigoplus_{i\in I}(X_i\otimes Y_i),\]
		where the lists $\{X_i\}_{i\in I}$ and $\{Y_i\}_{i\in I}$ are exactly lists of irreducible representations of $A$ and $B$, respectively.
	\end{listalph}
\end{lemma}
\begin{proof}
	Start by letting $\{Y_i\}_{i\in I}$ be the list of irreducible representations of $B$ so that we have a decomposition
	\[V=\bigoplus_{i\in I}(X_i\otimes Y_i),\]
	where $X_i\coloneqq\op{Hom}_B(Y_i,V)$. Because $B$ embeds in $\op{End}_kV$, we see that $Y_i$ does admit an embedding into $V$: note that $B$ is a finite-dimensional semisimple algebra, so one can see from the structure theory that any faithful representation contains every irreducible representation. Now, $A$ is the centralizer of $B$, which is $\op{End}_B(V)$, but this is
	\[\op{End}_B(V)=\bigoplus_{i\in I}\op{End}_k(W_i)\]
	because the $V_i$s cannot map to distinct other representations. Then (a) is immediate, and (b) and (c) follow by a computation reversing everything in sight to compute decompositions over $A$. Namely, one will find that $B=\bigoplus_{i\in I}\op{End}_kY_i$.
\end{proof}
\begin{proof}[Proof of \Cref{thm:schur-weyl}]
	Note that $B$ is a quotient of $\CC[S_N]$ and is therefore a sum of matrix algebras by the representation theory of finite groups. We now apply \Cref{lem:double-centralizer}, which we may do by \Cref{lem:one-centralizer-schur-weyl}. The decomposition in (b) follows immediately because the irreducible representations of $A$ are labeled by partitions $\lambda$ of at most $n$ parts, and we see that (c) also immediately follows.

	It remains to show (d). It is enough to show that the projection $\CC[S_N]\to B$ is injective so that irreducible representations of $B$ are irreducible representations of $S_N$. Well, let $\{e_1,\ldots,e_n\}$ be a basis of $V$, and then we see that $\CC[S_N]$ acts faithfully on $e_1\otimes\cdots\otimes e_N$ because the $S_N$-action produces linearly independent vectors! Thus, the map $\CC[S_N]\to\op{End}_\CC V^{\otimes N}$ is injective, and we are done.
\end{proof}
\begin{remark}
	\Cref{thm:schur-weyl} provides a parameterization of irreducible representations of $S_N$ by partitions of $N$ for every $n\ge N$. This parameterization does not depend on the choice of $n$, which one can see via the natural diagonal embedding $\op{GL}_n(\CC)\to\op{GL}_{n+1}(\CC)$. This is not too interesting, so we will not write it out.%\todo{}
\end{remark}

\subsection{Schur Functors}
We would now like to describe the representations $L_\lambda$ of $\op{GL}_n(\CC)$ more explicitly.
\begin{definition}[Schur functor]
	Fix a partition $\lambda$ of a positive integer $N$. Then we define the \textit{Schur functor} by
	\[S^\lambda\coloneqq\op{Hom}_{S_N}\left(\pi_\lambda,(-)^{\otimes N}\right).\]
\end{definition}
\begin{example} \label{ex:schur-single}
	If $\lambda=(N)$, then $L_\lambda=S^NV$, so $\pi_\lambda$ is the trivial representation, so $S^\lambda V=S^NV$.
\end{example}
\begin{example} \label{ex:schur-long}
	If $\lambda=(1,\ldots,1)$, then $L_\lambda=\land^NV$ (which is the determinant representation when $\dim V=N$), so $\pi_\lambda$ is the sign representation, so $S^\lambda V=\land^NV$.
\end{example}
\begin{remark}
	In general, \Cref{thm:schur-weyl} has given us a decomposition
	\[V^{\otimes N}=\bigoplus_{\lambda\vdash N}S^\lambda V\otimes\pi_\lambda.\]
\end{remark}
\begin{example}
	We have a decomposition
	\[V^{\otimes2}=\left(S^2V\otimes\CC_{\mathrm{triv}}\right)\oplus\left(\land^2V\otimes\CC_{\mathrm{sgn}}\right).\]
	Note that we have silently applied \Cref{ex:schur-single,ex:schur-long}.
\end{example}
\begin{example} \label{ex:describe-schur-2-1}
	We have a decomposition
	\[V^{\otimes3}=\left(S^3V\otimes\CC_{\mathrm{triv}}\right)\oplus\left(\land^3V\otimes\CC_{\mathrm{sgn}}\right)\oplus\left(S^{(2,1)}V\otimes\pi_{(2,1)}\right).\]
	By referencing a character table of $S_3$, we see that $\pi_{(2,1)}$ must be the standard representation of $S_3$, which is its action on the trace-zero hyperplane in $\CC^3$. For example, if we restrict to the group $\op{GL}_3(\CC)$, then we have a decomposition
	\[V^{\otimes3}=S^3V\oplus\land^3V\oplus S^{(2,1)}V^{\oplus2}.\]
	On the other hand, the left-hand side is $V^{\otimes2}\otimes V=\left(S^2V\otimes V\right)\oplus\left(\land^2V\otimes V\right)$. The former factor has a copy of $S^3V$, and the latter factor has a copy of $\land^3V$; none of these inclusions are equalities, so it follows that each has one copy of $S^{(2,1)}V$. This allows us to describe $S^{(2,1)}V\subseteq S^2V\otimes V$ as the subset whose total symmetrization is zero. (There is a similar description of $S^{(2,1)}V$ as embedded in $\land^2V\otimes V$.)
\end{example}
While it is a little annoying to describe the $S^\lambda V$s explicitly (though \Cref{ex:describe-schur-2-1} gives a method), there is a formula for their dimension.
\begin{proposition}
	Fix $V\coloneqq\CC^n$. Then
	\[\dim S^\lambda V=\prod_{1\le i<j\le N}\frac{(\lambda_i-\lambda_j)+(j-i)}{j-i}.\]
\end{proposition}
\begin{proof}
	Recall that $\rho=(N-1,N-2,\ldots,0)$ because the positive roots are $e_i-e_j$ where $i<j$. Thus, we see that $(\rho,e_i-e_j)=(j-i)$, so $(\lambda+\rho,e_i-e_j)=(\lambda_i-\lambda_j)+(j-i)$, and \Cref{thm:weyl-dim} gives the result.
\end{proof}
We take a moment to give some simplifications. If $\lambda$ has only $k$ parts, then we could split the product into
\[\prod_{1\le i<j\le k}\frac{(\lambda_i-\lambda_j)+(j-i)}{j-i}\cdot\prod_{1\le \le k<j\le N}\frac{\lambda_i+(j-i)}{j-i}.\]
The latter product now telescopes after $\lambda_i$ terms, allowing us to rewrite this as
\[\prod_{1\le i<j\le k}\frac{(\lambda_i-\lambda_j)+(j-i)}{j-i}\cdot\prod_{i=1}^k\frac{(N+1-i)\cdots(N+\lambda_i-i)}{(k+1-i)\cdots(k+\lambda_i-i}.\]
For example, it follows that the dimension is a polynomial $P_\lambda$ with rational coefficients, whose roots are integers contained in $[1-\lambda,\ldots,k-1]$.

\end{document}