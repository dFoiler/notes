% !TEX root = ../notes.tex

\documentclass[../notes.tex]{subfiles}

\begin{document}

\section{February 2}
Here we go.

\subsection{Review of Lie Groups}
We start with some quick review. Here are our groups.
\begin{definition}[Lie group]
	A \textit{Lie group} is a group object $G$ in the category of manifolds. One may specify a ``real'' or ``complex'' Lie group, which means that we are taking the category of real or complex manifolds. Explicitly, we are asking for $G$ to be equipped with regular maps $m\colon G\times G\to G$, $i\colon G\to G$, and an identity. A \textit{homomorphism of Lie groups} is a morphism of the group objects.
\end{definition}
\begin{example}
	One has the usual examples: $\RR^n$, $\op U(n)$, $\op{Sp}_{2n}(\RR)$, $\op O(p,q)$, and $\op{SU}(n)$ are all real Lie groups.
\end{example}
\begin{example}
	There are classical groups over $\CC$, such as $\op{SL}_n(\CC)$, which are all Lie groups.
\end{example}
\begin{definition}
	If $G$ is a Lie group, then its connected component $G^\circ$ is a normal Lie subgroup.
\end{definition}
\begin{remark}
	The quotient $\pi_0G\coloneqq G/G^\circ$ is a discrete topological group.
\end{remark}
\begin{remark}
	Given a Lie group $G$, the universal cover $\widetilde G\to G$ can be checked to a Lie group via some universal properties, so we receive a homomorphism $\pi\colon\widetilde G\to G$. It turns out that the kernel is a central discrete subgroup $Z\subseteq\widetilde G$. It notably follows that $\pi_1(G)$ is abelian.
\end{remark}
\begin{remark}
	One can check that $G^\circ$ is generated by any open neighborhood of the identity. Indeed, the generated subgroup can be seen to be both open and closed.
\end{remark}
\begin{example}
	With $G=S^1$, we have the universal cover $\widetilde G=\RR$, and the kernel is $\ZZ\subseteq\RR$.
\end{example}
We also have subgroups.
\begin{definition}[Lie subgroup]
	A \textit{Lie subgroup} is an immersed submanifold $H\subseteq G$ which is also a subgroup, meaning that $H\into G$ admits injective differentials. A \textit{closed Lie subgroup} is an embedded submanifold $H\subseteq G$ which is also a subgroup.
\end{definition}
\begin{remark}
	It turns out that closed Lie subgroups are in fact closed subsets, which can be checked locally.
\end{remark}
\begin{example}
	The subgroup $\QQ^n\subseteq\RR^n$ is a Lie subgroup, but it is not a closed Lie subgroup. The only closed Li subgroups are vector spaces.
\end{example}
\begin{example}
	The subgroup $\op O_n(\RR)\subseteq\op{GL}_n(\RR)$ is a closed real Lie subgroup.
\end{example}
\begin{remark}
	It turns out that a closed subgroup of $G$ is in fact a closed Lie subgroup. We will prove this later in the semester.
\end{remark}
\begin{definition}[quotient]
	Fix a closed Lie subgroup $H\subseteq G$. Then $G/H$ is a manifold with transitive $G$-action. If $H$ is normal, then $G/H$ is further a Lie group.
\end{definition}
\begin{remark}
	In general, if $G$ acts transitively on a manifold $X$, then for any $x\in X$, $\op{Stab}_G(x)\subseteq G$ is a closed Lie subgroup, and the quotient is isomorphic to $X$.
\end{remark}
\begin{remark}
	If $G$ acts on a space $X$ which is not transitive, then for any $x\in X$, the subset $Gx\subseteq X$ is at least an immersed submanifold.
\end{remark}
\begin{example}
	The group $\RR$ has an action on $\RR^2/\ZZ^2$ by $t\colon x\mapsto tx$. The orbit of (say), $(1/2,\sqrt2/2)$ is an immersed but not closed submanifold.
\end{example}
\begin{definition}[representation]
	Fix a Lie group $G$. A \textit{representation} of a Lie group is a homomorphism $G\to\op{GL}_n(\CC)$.
\end{definition}
\begin{example}
	Let $G$ act on itself by conjugation. Then each $g\in G$ acts on $T_1G\to T_1G$, so we receive an adjoint representation $\op{Ad}_\bullet\colon G\to\op{GL}(T_1G)$.
\end{example}
As usual, one can define morphisms of representations, subrepresentations, direct sums, duals, tensor products, irreducible representations, and so on. We also have a Schur's lemma.
\begin{lemma}
	Fix irreducible representations $V$ and $W$ of $G$.
	\begin{listalph}
		\item Then a $G$-equivariant map $\varphi\colon V\to W$ is either zero or an isomorphism.
		\item Any $G$-equivariant map $A\to A$ is a scalar.
	\end{listalph}
\end{lemma}
\begin{proof}
	Omitted.
\end{proof}
\begin{definition}[unitary]
	A \textit{unitary representation} is one admitting a $G$-invariant positive-definite Hermitian form.
\end{definition}
\begin{remark}
	Any unitary representation admits a decomposition into irreducible representations by taking orthogonal complements.
\end{remark}
\begin{nex}
	Let $B\subseteq\op{GL}_2(\CC)$ be the subgroup of upper-triangular matrices. Then the standard representation of $B$ does not admit a decomposition into irreducibles, so it cannot be made unitary.
\end{nex}
\begin{example} \label{ex:finite-unitary}
	If $G$ is finite, then any representation $V$ admits a unitary structure: given any unitary structure $\langle-,-\rangle_0$, one can define an invariant unitary structure
	\[\langle v,w\rangle\coloneqq\frac1{\left|G\right|}\sum_{g\in G}\langle gv,gw\rangle_0,\]
	where $dg$ is a choice of Haar measure.
\end{example}
\begin{theorem}[Maschke]
	Fix a finite group $G$. Then all representations admit decomposition into irreducible representations.
\end{theorem}
\begin{proof}
	This follows from \Cref{ex:finite-unitary}.
\end{proof}

\subsection{Review of Lie Algebras}
We now linearize our story.
\begin{remark}
	Note that $G$ acts on itself by left translations $\ell_g$, so the tangent bundle $TG$ can be given a global frame by the induced isomorphisms $d\ell_g\colon T_1G\to T_gG$.
\end{remark}
\begin{notation}
	For each $a\in T_1G$, we define the vector field $L_a$ by
	\[L_a\coloneqq ga\in T_aG.\]
\end{notation}
\begin{remark}
	One can check that all left-invariant vector fields take the form $L_a$.
\end{remark}
\begin{definition}[commutator]
	Fix a Lie group $G$. For each $a,b\in T_1G$, we may take the commutator $[L_a,L_b]$ to produce another left-invariant vector field, which we label $L_{[a,b]}$.
\end{definition}
\begin{remark}
	The formalism of the commutator tells us that $[-,-]$ is antisymmetric and satisfies the Jacobi identity
	\[[a,[b,c]]+[b,[c,a]]+[c,[a,b]]=0.\]
\end{remark}
\begin{definition}[Lie algebra]
	Fix a vector space $\mf g$ over a field $k$. Then a \textit{Lie algebra} is such a vector space $\mf g$ equipped with an antisymmetric pairing $[-,-]\colon\mf g\times\mf g\to\mf g$ satisfying the Jacobi identity
	\[[a,[b,c]]+[b,[c,a]]+[c,[a,b]]=0.\]
\end{definition}
\begin{example}
	For any Lie group $G$, we have seen that we may equip $\op{Lie}G\coloneqq T_1G$ with the structure of a Lie group.
\end{example}
\begin{example}
	If $G=\op{GL}_n(\CC)$, then $\mf g=M_n(\CC)$, and one can check that $[X,Y]=XY-YX$.
\end{example}
We  now define Lie subalgebras and morphisms of Lie algebras in the expected way.
\begin{definition}[Lie ideal]
	Fix a Lie algebra $\mf g$. Then a \textit{Lie ideal} $\mf h\subseteq\mf g$ is a subspace for which $[\mf h,\mf g]\subseteq\mf h$.
\end{definition}
\begin{example}
	For any closed Lie subgroup $H\subseteq G$, we see that $\op{Lie}H\subseteq\op{Lie}G$ is a Lie subalgebra. If $H$ is normal, then $\op{Lie}H$ is a Lie ideal.
\end{example}
As expected, there is some representation theory.
\begin{definition}
	Fix a Lie algebra $\mf g$ over a field $k$. Then a \textit{representation} of $\mf g$ is a morphism $\mf g\to\mf{gl}_n(k)$.
\end{definition}
One can relate $\op{Lie}G$ to $G$ more directly via exponentiation.
\begin{definition}[exponential]
	Fix a Lie group $G$ with Lie algebra $\mf g$. We define a map $\exp\colon\mf g\to G$ as follows. For each $a\in\mf g$, one can check that the differential equation
	\[\begin{cases}
		e'(t)=e(t)\cdot a, \\
		e(0)=1,
	\end{cases}\]
	admits a unique solution; we then define $\exp(ta)\coloneqq e(t)$. (This is independent of the choice of $t$.) It turns out that $t\mapsto\exp(ta)$ is a group homomorphism.
\end{definition}
\begin{example}
	If $G=\op{GL}_n(\CC)$, then $\exp\colon M_n(\CC)\to\op{GL}_n(\CC)$ is the usual matrix exponential.
\end{example}
\begin{remark}
	It turns out that $\exp$ is a local diffeomorphism (though not necessarily injective), so there is a local inverse $\log\colon U\to\mf g$, where $U$ is some open neighborhood of the identity.
\end{remark}
\begin{remark}
	For small $a$ and $b$, it turns out that
	\[\log(\exp(a)\exp(b))=a+b+\frac12[a,b]+\cdots,\]
	where $\cdots$ denotes cubic terms. For example, if $G$ is commutative, then we see that the Lie bracket $[-,-]$ vanishes; conversely, if $[-,-]$ vanishes, then $G$ can be checked to commute in an open neighborhood of the identity, so $G$ commutes.
\end{remark}

\subsection{Fundamental Theorems}
In a first course, one checks the following two fundamental theorems.
\begin{theorem}
	Fix a Lie group $G$. Then there is a bijection between connected closed Lie subgroups $H\subseteq G$ and Lie subalgebras $\mf h\subseteq\op{Lie}G$.
\end{theorem}
\begin{theorem}
	Fix Lie groups $G$ and $K$, with $G$ simply connected. Then taking the differential
	\[\op{Hom}(G,K)\to\op{Hom}(\op{Lie}G,\op{Lie}K)\]
	is an isomorphism.
\end{theorem}
There is a third fundamental theorem, which we will prove later.
\begin{theorem}
	For any finite-dimensional Lie algebra $\mf g$ (over $\RR$ or $\CC$), then there is a Lie group $G$ with $\op{Lie}G\cong\mf g$.
\end{theorem}
The three theorems provide an equivalence between the category of simply connected Lie groups and the category of Lie algebras, thereby classifying the former.
\begin{remark}
	It follows that one may classify connected Lie groups as quotients of simply connected Lie groups by discrete central subgroups.
\end{remark}

\subsection{Representations of Lie Algebras}
Let's start with the representation theory of $\mf{sl}_2(\CC)$.
\begin{theorem}
	Fix the usual basis $e\coloneqq\begin{bsmallmatrix}
		0 & 1 \\ 0 & 0
	\end{bsmallmatrix}$, $f\coloneqq\begin{bsmallmatrix}
		0 & 0 \\ 1 & 0
	\end{bsmallmatrix}$, and $h\coloneqq[e,f]$ of $\mf{sl}_2(\CC)$.
	\begin{listalph}
		\item Then all irreducible representations of $\mf{sl}_2(\CC)$ can be parameterized as $\{V_n\}_{n\ge0}$, where $V_n$ is the representation of homogeneous polynomials in $x$ and $y$ of degree $n$. 
		\item Every representation is a direct sum of irreducible representations.
		\item Clebsch--Gordon rule: for any $n$ and $m$, we have
		\[V_n\otimes V_m=\bigoplus_{i=0}^{\min\{m,n\}}V_{\left|m-n\right|+2i}.\]
	\end{listalph}
\end{theorem}
It will be helpful to turn representation theory of Lie algebras into a module category.
\begin{definition}[universal enveloping algebra]
	Fix a Lie algebra $\mf g$. Then we define $U\mf g$ as the quotient of the tensor algebra by the relation
	\[[x,y]=x\otimes y-y\otimes x.\]
\end{definition}
\begin{remark}
	It turns out that $\op{Rep}\mf g$ is the same category as $\op{Mod}U\mf g$.
\end{remark}
Even though we have taken a quotient by an inhomogeneous relation, $U\mf g$ still receives a natural filtration by degree.
\begin{theorem}[Poincar\'e--Birkhoff--Witt]
	Fix a Lie algebra $\mf g$, and equip $U\mf g$ with the natural filtration. For any basis $\{x_1,\ldots,x_n\}$ of $\mf g$, the ordered monomials in the basis form a basis of $U\mf g$.
\end{theorem}
To continue our story, we need some adjectives for Lie algebras.
\begin{definition}[solvable]
	A Lie algebra $\mf g$ is \textit{solvable} if and only if the derived series eventually vanishes. Here, the derived series is defined inductively by $D^0(\mf g)\coloneqq\mf g$ and $D^{n+1}(\mf g)\coloneqq[D^n(\mf g),D^n(\mf g)]$ for each $n\ge0$.
\end{definition}
\begin{definition}[nilpotent]
	A Lie algebra $\mf g$ is \textit{nilpotent} if and only if the lower central series eventually vanishes. Here, the derived series is defined inductively by $L_0(\mf g)\coloneqq\mf g$ and $L_{n+1}(\mf g)\coloneqq[L_n(\mf g),\mf g]$ for each $n\ge0$.
\end{definition}
\begin{remark}
	One can see that nilpotent implies solvable.
\end{remark}
The representation theory of solvable Lie algebras is quite easy.
\begin{theorem}[Lie]
	Fix a finite-dimensional solvable Lie algebra $\mf g$ over an algebraically closed field of characteristic zero.
	\begin{listalph}
		\item Then every irreducible representation of $\mf g$ is one-dimensional.
		\item Every representation admits a basis on which $\mf g$ acts by upper-triangular matrices.
	\end{listalph}
\end{theorem}
\begin{theorem}[Engel]
	Fix a finite-dimensional Lie algebra $\mf g$. Then $\mf g$ is nilpotent if and only if $\op{ad}_X\colon\mf g\to\mf g$ is nilpotent for all $X\in\mf g$.
\end{theorem}
Thus, we see that we will want to ignore solvable and nilpotent pieces.
\begin{definition}[radical]
	Fix a Lie algebra $\mf g$. Then the \textit{radical} $\op{rad}\mf g$ is the sum of all solvable ideals of $\mf g$.
\end{definition}
\begin{remark}
	One can check that $\op{rad}\mf g$ is a solvable ideal, so it is automatically the largest solvable ideal.
\end{remark}
\begin{definition}[semisimple]
	Fix a Lie algebra $\mf g$. Then $\mf g$ is \textit{semisimple} if and only if $\op{rad}\mf g=0$.
\end{definition}
\begin{remark}
	It turns out that
	\[\mf g_{\mathrm{ss}}\coloneqq\frac{\mf g}{\op{rad}\mf g}\]
	is always semisimple. It turns out that the induced exact sequence splits, so there is a decomposition $\mf g=\mf g_{\mathrm{ss}}\ltimes\op{rad}\mf g$, which is known as the Levi decomposition; we will prove this later.
\end{remark}
Having defined semisimple, we should define ``simple.''
\begin{definition}[simple]
	A Lie algebra $\mf g$ is \textit{simple} if and only if its only ideals are $0$ and $\mf g$.
\end{definition}
\begin{remark}
	One can check that semisimple Lie algebras are precisely the sums of simple Lie algebras.
\end{remark}
It turns out to be convenient to allow a little radical.
\begin{definition}[reductive]
	A Lie algebra $\mf g$ is \textit{reductive} if and only if its radical is its center.
\end{definition}
\begin{example}
	One can check that $\mf{sl}_n(\CC)$ is simple, and $\mf{gl}_n(\CC)$ is reductive.
\end{example}
To test for a Lie algebra being semisimple (and other adjectives), we introduce the Killing form.
\begin{definition}[Killing form]
	Fix a Lie algebra $\mf g$. Then we define the \textit{Killing form} by
	\[K(x,y)\coloneqq\tr({\op{ad}_x}\circ{\op{ad}_y}).\]
\end{definition}
\begin{remark}
	One can check that $K$ is $\mf g$-invariant.
\end{remark}
\begin{theorem}[Cartan criteria]
	Fix a Lie algebra $\mf g$.
	\begin{listalph}
		\item $\mf g$ is solvable if and only if $[\mf g,\mf g]\subseteq K$.
		\item $\mf g$ is semisimple if and only if $K$ is non-degenerate.
	\end{listalph}
\end{theorem}
\begin{proposition}
	A Lie algebra $\mf g$ is reductive if and only if it admits a representation $\rho\colon\mf g\to\mf{gl}(V)$ for which the bilinear form
	\[B_V(X,Y)\coloneqq\tr(\rho_X\circ\rho_Y)\]
	is non-degenerate.
\end{proposition}
We may as well state one of the main theorems of our representation theory.
\begin{theorem}
	Every finite-dimensional representation of a semisimple Lie algebra is completely reducible.
\end{theorem}

\subsection{Structure Theory of Lie Algebras}
Here is another piece of structure theory.
\begin{definition}[adjoint]
	Fix a semisimple Lie algebra $\mf g$. Then we define the \textit{adjoint} Lie group $G^{\mathrm{ad}}$ by $G^{\mathrm{ad}}\coloneqq\op{Aut}(\mf g)^\circ\subseteq\op{GL}(\mf g)$.
\end{definition}
\begin{remark}
	It turns out that $\op{Lie}G^{\mathrm{ad}}=\mf g$.
\end{remark}
In our setting, one can generalize the Jordan decomposition.
\begin{defihelper}[semisimple, nilpotent]
	An element $X\in\mf g$ is \textit{semisimple} or \textit{nilpotent} if and only if the operator $\op{ad}_XX$ is.
\end{defihelper}
\begin{theorem}
	Fix a Lie algebra $\mf g$. Then any $X\in\mf g$ can be written uniquely as a sum of a semisimple and nilpotent element.
\end{theorem}
\begin{remark}
	It turns out that semisimple elements always act semisimply on representations, and nilpotent elements always act nilpotently on representations.
\end{remark}
The notion of semisimple elements is important to define Cartan subalgebras.
\begin{definition}[Cartan]
	Fix a semisimple Lie algebra $\mf g$. Then a \textit{Cartan subalgebra} is a maximal commutative subalgebra of 
\end{definition}
\begin{proposition}
	Fix a semisimple Lie algebra $\mf g$. All Cartan subalgebras are conjugate by $G^{\mathrm{ad}}$.
\end{proposition}
\begin{definition}
	Fix a semisimple Lie algebra $\mf g$. Then the \textit{rank} of $\mf g$ is the dimension of the Cartan subalgebras.
\end{definition}
A choice of Cartan subalgebra $\mf h\subseteq\mf g$ produces a root decomposition, which we write as
\[\mf g=\mf h\oplus\bigoplus_{\alpha\in\mf h^*\setminus\{0\}}\mf g_\alpha.\]
\begin{definition}[root system]
	Fix a semisimple Lie algebra $\mf g$ and a Cartan subalgebra $\mf h\subseteq\mf g$. Then the \textit{root system} of $\mf g$ consists of those nonzero eigenvalues $\alpha\in\mf h^*$ for the adjoint action of $\mf h$ on $\mf g$. We write $\mf g_\alpha$ for this eigenspace, and we write $\Phi(G)$ for the root system.
\end{definition}
\begin{remark}
	One can check that $[\mf g_\alpha,\mf g_\beta]\subseteq\mf g_{[\alpha,\beta]}$. In fact, $\mf g_\alpha$ and $\mf g_\beta$ are orthogonal for the Killing form except when $\alpha=-\beta$, where it is a perfect pairing.
\end{remark}
\begin{remark}
	It turns out that $\dim\mf g_\alpha=1$ for each $\alpha$. It follows that
	\[\#\Phi(\mf g)=\dim\mf g-\op{rank}\mf g.\]
\end{remark}
\begin{remark}
	There are the usual pictures of root systems of various types.
\end{remark}

\subsection{Root Systems}
It is useful to write down what properties are satisfied by these root systems.
\begin{definition}[root system]
	Fix a Euclidean space $E$. Then a finite subset $\Phi\subseteq E$ is a \textit{root system} if and only if
	\begin{listalph}
		\item $\Phi$ spans $E$,
		\item for each $\alpha,\beta\in\Phi$, the number
		\[n_{\alpha\beta}\coloneqq\frac{2(\alpha,\beta)}{(\alpha,\alpha)}\]
		is an integer,
		\item for each $\alpha,\beta\in\Phi$, the reflection
		\[s_\alpha(\beta)\coloneqq\beta-n_{\alpha\beta}\beta\]
		is in $\Phi$.
	\end{listalph}
	We say that $\Phi$ is \textit{reduced} if and only if $\alpha\in\Phi$ implies that $2\alpha\notin\Phi$.
\end{definition}
\begin{definition}
	A root system $\Phi$ is \textit{reducible} if and only if it can be written as a disjoint union of root systems coming from a decomposition of the Euclidean space into a product of Euclidean spaces.
\end{definition}
The reflections are important enough to be placed into a group.
\begin{definition}[Weyl group]
	Fix a root system $\Phi\subseteq E$. Then the \textit{Weyl group} $W$ is the subgroup of $\op{GL}(E)$ generated by the reflections.
\end{definition}
\begin{example}
	The Weyl group associated to the root system of $\mf{sl}_{n+1}$ consists of the permutation matrices in $\op{GL}_n(\RR)$. Indeed, each reflection corresponds to a transposition. This root system is said to be of type $A_n$, where $n$ refers to the rank.
\end{example}
\begin{example}
	The root system associated to $\mf{so}_{2n+1}$ is $B_n$. The root system associated to $\mf{sp}_{2n}$ is $C_n$. Lastly, the root system associated to $\mf{so}_{2n}$ is $D_n$.
\end{example}
\begin{remark}
	There are also various exceptional reduced root systems, which we may say something about later.
\end{remark}
We can even break down irreducible root systems into more controlled pieces.
\begin{definition}[positive]
	Fix a root system $\Phi\subseteq E$. For a choice of $t\in E$ for which $(t,\alpha)\ne0$ for all $\alpha\in E$, we say that a root in $\Phi$ is \textit{positive} if and only if $(t,\alpha)>0$. Similarly, $\alpha$ is \textit{negative} if and only if $(t,\alpha)<0$. We let $\Phi^+$ and $\Phi^-$ denote the sets of positive and negative roots, respectively.
\end{definition}
\begin{definition}
	Fix a root system $\Phi\subseteq E$. A positive root is \textit{simple} if and only if it is not a sum of other positive roots (with positive integer coefficients). We let $\Pi$ denote the set of simple roots.
\end{definition}
\begin{proposition}
	Fix a root system $\Phi\subseteq E$. Then $\Pi$ is a basis, and every positive root $\alpha\in\Phi^+$ can be written as a unique sum of elements of $\Pi$ with positive integer coefficients.
\end{proposition}
Each root system also admits a dual.
\begin{definition}[dual root system]
	Fix a root system $\Phi\subseteq E$. Then we define the \textit{dual root system} $\Phi^\lor\subseteq E^\lor$ to be given by the points
	\[\alpha^\lor=\frac{2(\alpha,-)}{(\alpha,\alpha)}\]
	for each $\alpha\in\Phi$.
\end{definition}
\begin{remark}
	The reduced root system $B_n$ is dual to $C_n$.
\end{remark}
It will be helpful to have some lattices from our root systems.
\begin{definition}
	Fix a root system $\Phi\subseteq E$.
	\begin{itemize}
		\item The \textit{root lattice} $Q$ is spanned by $\phi$.
		\item The \textit{coroot lattice} $Q^\lor$ is spanned by the $\alpha^\lor$.
		\item The lattice $P\subseteq E$ is $(Q^\lor)^*$.
		\item The \textit{weight lattice} $P^\lor\subseteq E^*$ is $Q^*$.
	\end{itemize}
\end{definition}
In general, $Q\subseteq P$, but equality does not have to hold.
\begin{example}
	For $\mf{sl}_{n}$, the quotient $P/Q$ is isomorphic to $\ZZ/n\ZZ$.
\end{example}

\end{document}