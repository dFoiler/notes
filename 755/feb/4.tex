% !TEX root = ../notes.tex

\documentclass[../notes.tex]{subfiles}

\begin{document}

\section{February 4}
We continue our review today.

\subsection{Weights}
It will be helpful to have some lattices from our root systems.
\begin{definition}
	Fix a root system $\Phi\subseteq E$.
	\begin{itemize}
		\item The \textit{root lattice} $Q$ is spanned by $\Phi$.
		\item The \textit{coroot lattice} $Q^\lor$ is spanned by the $\alpha^\lor$.
		\item The \textit{weight} lattice $P\subseteq E$ is $(Q^\lor)^*$.
		\item The \textit{coweight lattice} $P^\lor\subseteq E^*$ is $Q^*$.
	\end{itemize}
\end{definition}
In general, $Q\subseteq P$, but equality does not have to hold.
\begin{example}
	For $\mf{sl}_{n}$, the quotient $P/Q$ is isomorphic to $\ZZ/n\ZZ$.
\end{example}
It is useful to have a basis for the weight lattice
\begin{definition}[fundamental weight]
	Fix a root system $\Phi\subseteq E$. A \textit{fundamental weight} is an element of the dual basis (in the weight lattice $P$) of the $\alpha_\bullet^\lor\in Q^\lor$ where $\alpha$ is a simple root. Explicitly, if $\{\alpha_1,\ldots,\alpha_r\}$ are the simple roots, then the fundamental weights $\{\omega_1,\ldots,\omega_r\}$ satisfy
	\[(\omega_i,\alpha_j^\lor)=1_{i=j}.\]
\end{definition}
\begin{definition}[dominant]
	Fix a root system $\Phi\subseteq E$. A weight $\lambda$ is \textit{dominant} if and only if $\langle\lambda,\alpha_i^\lor\rangle$ is a nonnegative integer for all $i$.
\end{definition}
\begin{remark}
	One can check that the dominant weights are exactly the $\ZZ_{\ge0}$-span of the fundamental weights.
\end{remark}

\subsection{The Dynkin Diagram}
Let's start trying to classify our Lie algebras.
\begin{definition}[Cartan matrix]
	Fix a root system $\Phi\subseteq E$, and order the simple roots as $\{\alpha_1,\ldots,\alpha_r\}$. The matrix $A$ with entries
	\[a_{ij}\coloneqq\frac{2(\alpha_i,\alpha_j)}{(\alpha_i,\alpha_i)}\]
	is the \textit{Cartan matrix}.
\end{definition}
\begin{remark}
	The Cartan matrix satisfies the following properties, all essentially by construction.
	\begin{itemize}
		\item We have $a_{ii}=2$ for each $i$.
		\item For distinct $i$ and $j$, we have $a_{ij}\le0$, and $a_{ij}$ vanishes if and only if $a_{ji}$ vanishes.
		\item Because the inner product $(-,-)$ on the $\alpha_\bullet$s is positive definite, the pairing $(v,w)\coloneqq v^\intercal Aw$ is positive definite.
	\end{itemize}
\end{remark}
\begin{remark}
	For any $i$ and $j$, a piece of the Cartan matrix
	\[\begin{bmatrix}
		2 & a_{ij} \\
		a_{ji} & 2
	\end{bmatrix}\]
	must have positive determinant by the positive-definiteness. Thus, $4-a_{ij}a_{ji}>0$, so $a_{ij}a_{ji}\in[0,1,2,3]$ follows.
\end{remark}
\begin{example}
	The Cartan matrix for $\mf{sl}_4$ is
	\[\begin{bmatrix}
		2 & -1 & 0 \\
		-1 & 2 & -1 \\
		0 & -1 & 2
	\end{bmatrix}.\]
\end{example}
The data of a Cartan matrix can be encoded combinatorially into a Dynkin diagram.
\begin{definition}[Dynkin diagram]
	Fix a root system $\Phi\subseteq E$, and order the simple roots as $\{\alpha_1,\ldots,\alpha_r\}$. Let $A$ be the Cartan matrix. Then we define the \textit{Dynkin diagram} to have $r$ vertices and the following connections.
	\begin{itemize}
		\item If $a_{ij}=0$, then vertices $i$ and $j$ have no connection.
		\item If $a_{ij}=-1$ and $a_{ji}=-1$, we draw a single line between $i$ and $j$.
		\item If $a_{ij}\le-2$, then we draw an arrow from vertex $i$ to $j$ with $\left|a_{ij}\right|$ heads.
	\end{itemize}
\end{definition}
There are the usual pictures of all Dynkin diagrams.
\begin{example}
	For example, $A_{n-1}$ is a path with $n-1$ vertices.
	% https://q.uiver.app/#q=WzAsNSxbMCwwLCJcXGJ1bGxldCJdLFsxLDAsIlxcYnVsbGV0Il0sWzIsMCwiXFxidWxsZXQiXSxbMywwLCJcXGNkb3RzIl0sWzQsMCwiXFxidWxsZXQiXSxbMCwxLCIiLDAseyJzdHlsZSI6eyJoZWFkIjp7Im5hbWUiOiJub25lIn19fV0sWzIsMywiIiwwLHsic3R5bGUiOnsiaGVhZCI6eyJuYW1lIjoibm9uZSJ9fX1dLFszLDQsIiIsMCx7InN0eWxlIjp7ImhlYWQiOnsibmFtZSI6Im5vbmUifX19XSxbMSwyLCIiLDAseyJzdHlsZSI6eyJoZWFkIjp7Im5hbWUiOiJub25lIn19fV1d&macro_url=https%3A%2F%2Fraw.githubusercontent.com%2FdFoiler%2Fnotes%2Fmaster%2Fnir.tex
	\[\begin{tikzcd}[cramped]
		\bullet & \bullet & \bullet & \cdots & \bullet
		\arrow[no head, from=1-1, to=1-2]
		\arrow[no head, from=1-2, to=1-3]
		\arrow[no head, from=1-3, to=1-4]
		\arrow[no head, from=1-4, to=1-5]
	\end{tikzcd}\]
\end{example}
\begin{remark} \label{rem:lie-alg-to-diagram}
	It turns out that the root system determines the Lie algebra, the Cartan matrix determines the root system, and the Dynkin diagram determines the Cartan matrix.
\end{remark}
\begin{remark}[Coxeter group]
	One can read the Weyl group off of the Cartan matrix: namely, the reflection $s_{i}s_j$ has order
	\[m_{ij}\coloneqq\begin{cases}
		2 & \text{if }a_{ij}a_{ji}=0, \\
		3 & \text{if }a_{ij}a_{ji}=1, \\
		4 & \text{if }a_{ij}a_{ji}=2, \\
		6 & \text{if }a_{ij}a_{ji}=3.
	\end{cases}\]
	Indeed, this claim amounts to checking that $s_is_j$ is a rotation with specified degree, which can be seen on the inner products. It turns out that the Weyl group is generated by the relations $s_i^2=1$ and $(s_is_j)^{m_{ij}}=1$ for each pair of distinct $i$ and $j$.
\end{remark}
To prove \Cref{rem:lie-alg-to-diagram}, a difficult step is to show that every Dynkin diagram does in fact give rise to a Lie algebra. This is the content of the following theorem.
\begin{theorem}[Serre presentation]
	Fix a finite-dimensional simple Lie algebra $\mf g$ over $\CC$ with root system $\Phi$, and choose an ordered set of simple roots $\Pi=\{\alpha_1,\ldots,\alpha_r\}$. Select basis vectors $e_i$ and $f_i$ in each space $\mf g_{\alpha_i}$ and $\mf g_{-\alpha_i}$, and define $h_i\coloneqq[e_i,f_i]$. Then we have the following relations.
	\begin{listalph}
		\item We have $[h_i,h_j]=0$, $[h_i,e_j]=a_{ij}e_j$, and $[h_i,f_j]=-a_{ij}f_j$. 
		\item We have $[e_i,f_j]=1_{i=j}h_i$.
		\item Serre relations: $\op{ad}_{e_i}^{1-a_{ij}}e_j=0$ and $\op{ad}_{f_i}^{1-a_{ij}}f_j=0$.
	\end{listalph}
	In fact, the free Lie algebra defined by these generators and relations (for any root system $\Phi$) produces a finite-dimensional Lie algebra.
\end{theorem}

\subsection{Back to Representations of Lie Algebras}
Throughout, we fix a finite-dimensional Lie algebra $\mf g$, choose a Cartan subalgebra $\mf h\subseteq\mf g$. Then we choose a collection of positive roots for the induced root system, which lets us split $\mf g=\mf h\oplus\mf n_+\oplus\mf n_-$, where $\mf n_+$ is the sum of the positive root spaces, and $\mf n_-$ is the sum of the negative root spaces.
\begin{definition}[Verma module]
	Fix a weight $\lambda\in\mf h^*$. Then we define the \textit{Verma module} $M_\lambda$ to be generated as a $U\mf g$-module by the vector $v_\lambda$, given the relations $ev_\lambda=0$ for $e\in\mf n_+$ and $hv_\lambda=\lambda(h)v_\lambda$ for $h\in\mf h$.
\end{definition}
\begin{remark}
	In other words, $M_\lambda=U\mf g\otimes_{U(\mf h\oplus\mf n_+)}\CC_\lambda$, where $\CC_\lambda$ is the representation of $\mf h\oplus\mf n_+$ on which $\mf h$ acts by $\lambda$ and $\mf n_+$ acts by zero. For example, the PBW theorem gives us an isomorphism
	\[U\mf n_-\otimes U(\mf h\oplus\mf n_+)\to U\mf g\]
	of vector spaces, so $M_\lambda$ is a free module over $U\mf n_-$ of rank $1$.
\end{remark}
\begin{remark}
	For generic $\lambda$, it turns out that $M_\lambda$ is irreducible.
\end{remark}
\begin{definition}[highest weight]
	Fix a representation $V$ of $\mf g$. Then a weight $\lambda$ is a \textit{highest weight} if and only if $V$ is a nonzero quotient of $M_\lambda$. In other words, there is a vector $v\in V$ on which $\mf h$ acts by $\lambda$ and $\mf n_+$ acts by zero.
\end{definition}
Our classification result is as follows.
\begin{theorem}
	Fix everything as above.
	\begin{listalph}
		\item The Verma modules $M_\lambda$ admit a unique irreducible quotient.
		\item The representation $L_\lambda$ is finite-dimensional if and only if $\lambda$ is a dominant weight.
		\item Every irreducible finite-dimensional representation of $\mf g$ takes the form $L_\lambda$ for a dominant weight $\lambda$.
	\end{listalph}
\end{theorem}
\begin{proof}
	The proof of (a) is not too hard, and it is also not hard to show that $L_\lambda$ being finite-dimensional, then $\lambda$ is dominant. The hard part is to show that $\lambda$ being dominant implies that $L_\lambda$ is finite-dimensional!

	Lastly, by iteratively acting by $\mf n_+$, one can show that every irreducible finite-dimensional representation $V$ of $\mf g$ admits some highest weight $\lambda$. Indeed, given a finite-dimensional irreducible representation $V$ of $\mf g$, the representation theory of $\mf{sl}_2$ implies that the action of $\mf h$ on $V$ diagonalizes and admits integer eigenvalues. It follows that there is a surjection $M_\lambda\onto V$, so there is a surjection $L_\lambda\onto V$, which must then be an isomorphism.
\end{proof}
\begin{remark}
	Here is another way to select the highest weight: one can give the weight lattice a partial ordering, and then the highest one is the maximal element among the weights $\lambda$ for which $V[\lambda]\ne0$. In particular, the maximality implies that $\mf n_+$ acts by zero on $V[\lambda]$.
\end{remark}
There is also a construction of the quotient $L_\lambda$ by taking relations.
\begin{theorem}
	Fix everything as above, and choose a dominant weight $\lambda$. Then $L_\lambda$ is the quotient of $M_\lambda$ by the relations
	\[f_i^{(\lambda,\alpha_i^\lor)+1}v_\lambda=0.\]
\end{theorem}
\begin{remark}
	These relations are forced by passing to the representation theory of $\mf{sl}_2$.
\end{remark}

\subsection{Weyl Formulae}
Our construction of $L_\lambda$ is not so explicit; for example, what is the dimension of $L_\lambda$? To answer this question, we will need the Weyl dimension formula. We start with the Weyl character formula.
\begin{definition}[character]
	Fix a representation $V$ of a semisimple Lie algebra $\mf g$ admitting a weight decomposition
	\[V=\bigoplus_{\mu\in P}V[\mu].\]
	Then we define the \textit{character}
	\[\op{ch}V\coloneqq\sum_{\mu\in P}\dim V[\mu]e^\mu\]
	living in the free power series vector space with basis given by the $e^\mu$s, with a relation $e^{\mu+\lambda}$.
\end{definition}
\begin{remark}
	If $V$ is finite-dimensional, then it admits a weight decomposition (by the representation theory of $\mf{sl}_2$), and the character is a finite sum.
\end{remark}
\begin{remark}
	Let's explain why this is a character. Suppose that $V$ is a representation of a Lie group $G$ with Lie algebra $\mf g$. For $h\in\mf h$, the exponential $\exp(h)\in G$ acts on $V$ diagonally, with trace
	\[\tr_V\exp(h)=\sum_{\mu\in P}\dim V[\mu]e^{\mu(h)},\]
	which is exactly $\op{ch}V(h)$. In fact, these formulae determine the entire character because semisimple elements can all be found in Cartan subalgebras and are dense in $G$.
\end{remark}
We are now ready to state the Weyl character formula.
\begin{theorem}[Weyl character formula] \label{thm:weyl-char}
	Fix a semisimple Lie algebra $\mf g$ with root system $\Phi$, and let $\Phi^+$ be a set of positive roots. Further, set $\rho\coloneqq\frac12\sum_{\alpha\in\Phi^+}\alpha$. Then for each dominant weight $\lambda$, we have
	\[\op{ch}L_\lambda=\dfrac{\displaystyle\sum_{w\in W}\det(w)e^{w(\lambda+\rho)-\rho}}{\displaystyle\prod_{\alpha\in\Phi_+}(1-e^{-\alpha})}.\]
\end{theorem}
\begin{proof}
	Omitted. The idea is that the character of $M_\lambda$ is easy to compute, and one can build a resolution of $L_\lambda$ in terms of Verma modules.
\end{proof}
\begin{remark}
	One can write the right-hand side more symmetrically as
	\[\dfrac{\displaystyle\sum_{w\in W}\det(w)e^{w(\lambda+\rho)}}{\displaystyle\prod_{\alpha\in\Phi_+}\left(e^{\alpha/2}-e^{-\alpha/2}\right)}.\]
\end{remark}
\begin{remark}
	A priori, the right-hand side is a rational function, but one can use the anti-symmetry of the Weyl group action in the numerator to show that the denominator does in fact divide the numerator.
\end{remark}
\begin{example}[Weyl denominator] \label{ex:weyl-denom}
	If $\lambda=0$, then $L_\lambda$ is the trivial representation, so it follows that
	\[\sum_{w\in W}\det(w)e^{w\rho}=\prod_{\alpha\in\Phi_+}\left(e^{\alpha/2}-e^{-\alpha/2}\right).\]
\end{example}
\begin{example}[Vandermonde determinant]
	Suppose that $\mf g=\mf{sl}_n$ so that $\alpha_{ij}=x_i-x_j$. Then we set $X_i\coloneqq e^{x_i}$, so we find that
	\[\sum_{w\in S_n}\Bigg(\op{sgn}(w)\prod_{i=1}^nX_{w(i)}^{i-1}\Bigg)=\prod_{i<j}(X_i-X_j).\]
	Note that the right-hand side is precisely the determinant of the matrix
	\[\begin{bmatrix}
		1 & X_1 & \cdots & X_1^{n-1} \\
		\vdots & \vdots & \ddots & \vdots \\
		1 & X_n & \cdots & X_n^{n-1}
	\end{bmatrix}.\]
\end{example}
By \Cref{thm:weyl-char}, we could attempt to compute $\dim L_\lambda$ by plugging in $h=0$ to the polynomial.
\begin{theorem}[Weyl dimension formula] \label{thm:weyl-dim}
	Fix a semisimple Lie algebra $\mf g$ with root system $\Phi$, and let $\Phi^+$ be a set of positive roots. Then for any dominant weight $\lambda$,
	\[\dim L_\lambda=\prod_{\alpha\in\Phi_+}\frac{(\alpha^\lor,\lambda+\rho)}{(\alpha^\lor,\rho)}.\]
\end{theorem}
\begin{proof}
	This is slightly complicated because the given denominator vanishes at $h=0$, so we are left to compute the limit
	\[\dim L_\lambda=\lim_{h\to0}\dfrac{\displaystyle\sum_{w\in W}\det(w)e^{(w(\lambda+\rho),h)}}{\displaystyle\prod_{\alpha\in\Phi_+}\left(e^{(\alpha,h)/2}-e^{-(\alpha,h)/2}\right)}.\]
	This limit looks difficult to compute, but we may merely compute it on the line $h\coloneqq 2th_\rho$, where $h_\rho$ is the dual element for $\rho\in h^*$. In particular, it turns out that the numerator will factor in this case! To see this, note $\alpha(h_\rho)=(\alpha,\rho)$, so
	\[\sum_{w\in W}\det(w)e^{(w(\lambda+\rho),2th_\rho)}=\sum_{w\in W}\det(w)e^{2t(w(\lambda+\rho),\rho)},\]
	which upon moving the $w$ around and plugging into \Cref{ex:weyl-denom} gives
	\[\prod_{\alpha\in\Phi_+}\left(e^{t(\alpha,\lambda+\rho)}-e^{-t(\alpha,\lambda+\rho)}\right).\]
	The denominator is now
	\[\prod_{\alpha\in\Phi_+}\left(e^{(\alpha,h)/2}-e^{-(\alpha,h)/2}\right)=\prod_{\alpha\in\Phi_+}\left(e^{t(\alpha,\rho)}-e^{-t(\alpha,\rho)}\right).\]
	Sending $t\to0$ (and cancelling out a factor of two) proves the result.
\end{proof}

\subsection{Representation Theory of \texorpdfstring{$\mf{sl}_n(\CC)$}{ sln(C)}}
The dominant weights $\omega_i$ are the ``simplest,'' so the simplest interesting representations will be the $L_{\omega_i}$s. For example, we may hope to find all other representations inside them.
\begin{proposition}
	Fix a semisimple Lie algebra $\mf g$ with root system $\Phi$, and let $\Pi=\{\alpha_1,\ldots,\alpha_r\}$ be an ordered set of simple roots. Further, let $\{\omega_1,\ldots,\omega_r\}$ be the set of dominant weights. Then the category finite-dimensional representations of $\mf g$ is $\otimes$-generated by the representations $\{L_{\omega_i}\}_{i=1}^r$.
\end{proposition}
\begin{proof}
	Because any finite-dimensional representation is a direct sum of the irreducible representations of the form $L_\lambda$, where $\lambda$ is a dominant weight, it is enough to find these representations inside some tensor product of $L_{\omega_i}$s. Well, $\lambda$ is dominant, so we may write $\lambda=\sum_{i=1}^rk_i\omega_i$ for some nonnegative integers $\{k_1,\ldots,k_r\}$. We then consider the representation
	\[V\coloneqq\bigotimes_{i=1}^rL_{\omega_i}^{k_i}.\]
	For any dominant weight $\mu$, recall that $\mu$ is the highest weight of $L_\mu$, meaning that the weights of $L_\mu$ are concentrated in $\mu-P^+$, where $P^+$ consists of the dominant weights. Additionally, the nature of the $\mf g$-action on the tensor product implies that $W[\mu]\otimes W'[\mu']=(W\otimes W')[\mu+\mu']$ for any $W$ and $W'$ and $\mu$ and $\mu'$. Thus, we can see that the highest weight of $V$ is $\lambda=\sum_{i=1}^rk_i\omega_i$, and it has multiplicity $1$. It follows that there is a map $M_\lambda\to V$, which must descend to an embedding $L_\lambda\into V$, as desired.
\end{proof}
\begin{remark}
	The proof also shows that the other $L_\mu$s appearing in the representation $V$ have $\mu<\lambda$. Indeed, one can simply calculate the weights appearing in $V/L_\lambda$ and note that they are all strictly smaller than $\lambda$. The multiplicities of the various $L_\mu$ appearing in $V$ can be interesting to calculate.
\end{remark}
% Namely, given a dominant weight $\lambda$, we may write $\lambda=\sum_{i=1}^rk_i\alpha_i$ and consider the representation
% \[L_{\omega_1}^{\otimes k_1}\otimes\cdots\otimes L_{\omega_r}^{\otimes k_r}.\]
% By decomposing all the weight decompositions of the various $L_{\omega_\bullet}$s, we see that the highest weight of this representation $L_{\omega_1}^{\otimes k_1}$ is $k_1\alpha_1$ (with multiplicity $1$),\todo{} so the highest weight of the above representation is exactly $\lambda$. It follows that $L_\lambda$ is a quotient of this representation by its universal property, and the multiplicity of $L_\lambda$ is $1$. Thus,
% \[\bigotimes_{i=1}^rL_{\omega_i}^{\otimes k_i}=L_\lambda\oplus\bigoplus_{\mu<\lambda}L_\mu^{\oplus m_{\lambda\mu}}\]
% for some multiplicities $m_{\lambda\mu}$.
\begin{remark}
	In fact, we can see that $L_\lambda$ is the irreducible subrepresentation generated by the highest weight vector
	\[v_\lambda=v_{\omega_1}^{\otimes k_1}\otimes\cdots\otimes v_{\omega_r}^{\otimes k_r}\]
	in $V$. Indeed, the image of $L_\lambda$ in $V$ certainly contains this vector because this is the unique line with weight $\lambda$, so it follows that the generated subrepresentation must be exactly the irreducible representation $L_\lambda$.
\end{remark}
Let's work out our general theory in an interesting example.
\begin{example} \label{ex:alt-power-sl}
	Let $V_{\mathrm{std}}$ be the standard representation of $\mf{sl}_n(\CC)$. There is a choice of ordered simple roots of $\mf{sl}_n(\CC)$ for which
	\[L_{\omega_i}\cong\land^iV_{\mathrm{std}}.\]
\end{example}
\begin{proof}
	We proceed in steps.
	\begin{enumerate}
		\item We set up notation. Here, $\mf h$ is the set of diagonal matrices in $\mf{sl}_n(\CC)$, which we embed into $\CC^n$ as the subspace with sum zero. Let $\{e_1,\ldots,e_n\}$ be the natural basis of $\CC^n$, so we find that
		\[\Phi=\{e_i-e_j:i\ne j\}.\]
		Indeed, the elementary matrix $E_{ij}$ has weight $e_i-e_j$. By choosing the vector $t\coloneqq(n,n-1,n-2,\ldots,1,0)$ in $\CC^n$, we see that $\Phi^+$ consists of those $e_i-e_j$ with $i<j$. Then the simple roots are given by $\alpha_i\coloneq e_i-e_{i+1}$, which we can check is in fact a basis.

		For example, we see that this implies that the positive spaces in $\mf g$ are given by
		\[\mf n_+\coloneqq\bigoplus_{p<q}\CC E_{pq}.\]

		\item We calculate the fundamental weights. Now, the dual space $\mf h^*$ is then the quotient of $\CC^n$ by the diagonal subspace $\CC(1,\ldots,1)$. To compute our fundamental weights $\omega_i$, we see that we are on the hunt for some vectors $\omega_i$ for which
		\[(\omega_i,e_j-e_{j+1})=1_{i=j}.\]
		Well, $\omega_i=(1,\ldots,1,0,\ldots,0)$ works, where there are $i$ total $1$s. Accordingly, our dominant weights are the $\ZZ_{\ge0}$-linear combinations of the $\omega_i$, so they are the decreasing sequences of nonnegative integers.

		\item We claim that the representation $\land^iV_{\mathrm{std}}$ is irreducible. Well, let $W\subseteq\land^iV_{\mathrm{std}}$ be some nonzero subrepresentation, and we want to show that $W=\land^iV_{\mathrm{std}}$. Give $V_{\mathrm{std}}$ the standard basis $\{v_1,\ldots,v_n\}$, and we note that this basis is an eigenbasis for the $\mf h$-action on $V_{\mathrm{std}}$. Thus,
		\[\{v_{n_1}\land\cdots\land v_{n_i}:n_1<\cdots<n_i\}\]
		is an eigenbasis for the $\mf h$-action on $\land^iV_{\mathrm{std}}$. Because $W\subseteq V$ is a $\mf g$-subrepresentation, we conclude that it is an $\mathfrak h$-subrepresentation, so $W$ must contain some eigenvector $v_{n_1}\land\cdots\land v_{n_i}$.

		Now, for any $p\ne q$, we see that $E_{pq}\in\mf{sl}_n(\CC)$ acts by zero on all basis vectors $v_\bullet$ except it sends $v_q\mapsto v_p$. Thus, we see that we may apply such elementary matrices to any given eigenvector $v_{n_1}\land\cdots\land v_{n_i}$ to move it to $v_1\land\cdots\land v_i$ (namely, apply $E_{1n_1}$, then $E_{2n_2}$, and so on), and one can apply the process in reverse to move any $v_1\land\cdots\land v_i$ to any other eigenvector. We conclude that $W$ being stable under $\mf g$ forces $W=\land^iV$.

		\item We are now ready to show that $L_{\omega_i}\cong\land^iV_{\mathrm{std}}$. Because $\land^iV_{\mathrm{std}}$ is already irreducible, it is enough to check that its highest weight is $\omega_i$. Well, we claim that the highest weight vector is
		\[w\coloneqq v_1\land\cdots\land v_i.\]
		Well, for some $p<q$, we can see that $E_{pq}w=0$ because $E_{pq}$ can only ever send a basis vector $v_q$ to the ``earlier'' basis vector $v_p$, which either causes $w$ to vanish outright (if $	>i$) or introduces a multiplicity into $w$, still causing $E_{pq}w=0$. Thus, we do indeed see that $\mf n_+w=0$. As for the weight, one merely needs to calculate
		\[\op{diag}(x_1,\ldots,x_n)w=(x_1+\cdots+x_i)w,\]
		which is precisely the action by $\omega_i$!
		\qedhere
	\end{enumerate}
\end{proof}
\begin{remark}
	The moral is the story is that the category $\op{Rep}\mf{sl}_n(\CC)$ is $\otimes$-generated by the representations $\land^iV_{\mathrm{std}}$. In fact, because $\land^iV_{\mathrm{std}}$ is a quotient of $V_{\mathrm{std}}^{\otimes i}$, it follows that the category is $\otimes$-generated by $V_{\mathrm{std}}$.
\end{remark}
\begin{remark}
	There is a perfect pairing
	\[\land^iV_{\mathrm{std}}\otimes\land^{n-i}V_{\mathrm{std}}\to\land^nV_{\mathrm{std}}=\CC\]
	given simply by $w\otimes w'\mapsto w\land w'$; indeed, we can see this is a perfect pairing where a basis vector $v_{n_1}\land\cdots\land v_{n_i}$ has dual basis vector $v_{m_1}\land\cdots\land v_{m_{n-i}}$, where $\{n_1,\ldots,n_i\}\cup\{m_1,\ldots,m_{n-i}\}=\{1,\ldots,n\}$. Thus, we see that $L_{\omega_i}^*=L_{\omega_{n-i}}$ for each $i$.
\end{remark}
Here is another interesting family of representations.
\begin{example}
	Let $V_{\mathrm{std}}$ be the standard representation of $\mf{sl}_n(\CC)$. There is a choice of ordered simple roots of $\mf{sl}_n(\CC)$ for which
	\[L_{m\omega_1}\cong\op{Sym}^mV_{\mathrm{std}}.\]
\end{example}
\begin{proof}
	We continue from \Cref{ex:alt-power-sl}.
	\begin{enumerate}
		\item We claim that the representation $\op{Sym}^mV_{\mathrm{std}}$ is irreducible. As in \Cref{ex:alt-power-sl}, we see that
		\[\{v_{i_1}\cdots v_{i_m}:i_1\le\cdots\le i_m\}\]
		is an eigenbasis for the action of $\mf h$ on $V_{\mathrm{std}}$. Thus, any nonzero subrepresentation $W$ of $V_{\mathrm{std}}$ contains such an eigenvector. However, given one such eigenvector $v_{i_1}\cdots v_{i_m}$, one can iteratively apply $E_{1i_1}$, then $E_{1i_2}$, and so on, eventually proving that $v_1^m\in W$. Moving this basis vector around, we see that $v_i^m\in W$ for each $i$.

		We now claim that each basis vector $v_{i_1}\cdots v_{i_m}$ lives in $W$. Indeed, by hitting $v_1^m$ with $E_{11}-E_{22}+\sum_{i=2}^nc_iE_{1i}$, we see that
		\[\left(v_1+\sum_{i=2}^nc_iv_2\right)^m\in W.\]
		Each coefficient varies with a different multivariate polynomial in the $c_\bullet$s, so by choosing specializations carefully (and generically), we see that each monomial lives in $W$. Formally, one can inductively allow more and more of the $c_i$s to be nonzero, one at a time.

		\item We complete the proof. In light of the previous step, it is enough to show that $v_1^m$ is the highest weight vector of $\op{Sym}^mV_{\mathrm{std}}$ and has weight $m\omega_1$. Certainly $E_{pq}$ kills $v_1^m$ for each $p<q$ because $q>1$. As for the weight calculation, we note that
		\[\op{diag}(x_1,\ldots,x_n)\cdot v_1^m=mx_1\cdot v_1^m,\]
		as desired.
		\qedhere
	\end{enumerate}
\end{proof}
\begin{remark}
	Another way to check that $\op{Sym}^mV_{\mathrm{std}}$ is irreducible would be to use the Weyl dimension formula (\Cref{thm:weyl-dim}) to compute that $\dim L_{m\omega_1}=\dim\op{Sym}^mV_{\mathrm{std}}$. Then the second step provides a canonical map $L_{m\omega_1}\to V_{\mathrm{std}}$, which must be injective and hence an isomorphism.
\end{remark}

\end{document}