\documentclass[../notes.tex]{subfiles}
\graphicspath{{\subfix{../figs/}}}

\begin{document}
% !TEX root = ../notes.tex

Good morning, everyone. Welcome to the last day of class.

\subsection{Complex Numbers and Their Topology}
Today we summarize the course. We began our story with the complex numbers.
\complexdefiagain*
\noindent However, we wanted to turn this into a space, more specifically a metric space.
\complexdistdefi*
\noindent From here, we could define open balls and open subsets.
\openballdefi*
% \noindent Here is the image.
% \begin{center}
% 	\begin{asy}
% 		unitsize(0.5cm);
% 		dot("$z_0$", (0,0), W);
% 		draw(circle((0,0),2));
% 		draw((0,0) -- 2*dir(45), dashed);
% 		label("$r$", dir(45), dir(90+45));
% 	\end{asy}
% \end{center}
\noindent Then the open balls formed a basis of our topological space, giving our open sets.
\opendefi*

\subsection{Complex Functions}
With a topology in hand, we were able to talk about continuity; here are a few equivalent conditions.
\continuitygrabbag*
\noindent However, to really be able to talk about complex analysis, we need to introduce our notion of differentiation.
\complexdiffdefi*
\noindent Complex differentiability (as above) turns out to be very strong because the limit is taking place in the two-dimensional plane $\CC$.

Functions differentiable everywhere had a special name.
\holodefi*
\noindent We were able to show that a variety of functions were holomorphic, from polynomials to power series. Not all ``smoothish'' functions were holomorphic, such as $z\mapsto|z|$ and $z\mapsto\op{Re}z$.

As our first taste of the power of complex differentiability, we saw that it was a strictly stronger condition than merely being differentiable as a function $\RR^2\to\RR^2$: we had to satisfy some partial differential equations.
\cauchyriemannnecessary*
\noindent The above result had a pretty natural proof, essentially by writing down what we need for complex differentiability on the real and imaginary axis.

However, it turns out that this real and imaginary information was also sufficient.
\cauchyriemannsufficient*

\subsection{Integration}
Having talked about derivatives, we were able to integrate.
\integraldefi*
\noindent This definition was extended to piecewise $C^1$ paths in the natural way.

The point of studying integration was for the Cauchy integral formula. More concretely, the story of integration tied nicely into the story of analytic functions.
\analyticdefi*
\noindent Because power series were differentiable, we were able to get the following result.
\analyticishololem*
\noindent It turns out that the converse is also true: holomorphic functions were analytic.

To codify our connection, we needed to talk about winding numbers. Roughly speaking, $\op{Ind}(\gamma,z_0)$ refers to the number of times $\gamma$ goes around $z_0$ (with counterclockwise orientation). So our first hint that integration would be helpful for us is that it actually let us compute winding numbers.
\windingnumberformulalem*
\noindent From here, we could define more generalized winding numbers.
\indexdefi*
\noindent The point of this? The Cauchy integral formula.
\thmcif*
\noindent From here, we could in fact, prove our goal.
\holoisanacor*
\noindent The main ingredient in the proof of \autoref{thm:cif} was the Cauchy--Goursat theorem.
\starlikecg*
\noindent The Cauchy--Goursat theorem was first proven for triangles by some geometric argument and then generalized to star-like domains.

The Cauchy integral formula gave us all sorts of lovely corollaries. Let's start with Liouville's theorem.
\liovillethm*
\noindent From here followed the Fundamental theorem of algebra.
\ftathm*
\noindent My personal favorite corollary was the Identity theorem.
\identitythm*

\subsection{Singularities}
Another consequence of the Cauchy integral formula was that it let us study singularities. The most basic form was removable singularities.
\removablethm*
\noindent More generally, we had the following classification of singularities.
\basicsingularitydefi*
\isosingularitydefi*
\noindent We could understand removable singularities from the Riemann removable singularity theorem above, but more work was required to understand poles and essential singularities.

To begin, we started with poles. The key to understanding them was the Laurent series.
\laurentdefi*
\noindent Having access to Laurent expansions gave us a Residue theorem.
\residuethm*
\noindent In particular, if $f$ were holomorphic within the interior of $\gamma$, then \autoref{thm:starlikecg} could tell us that the integral should be $0$.

It is possible to generalize \autoref{thm:residue} by simply removing the condition on poles.
\residueagainthm*
\noindent The main ingredient in the proof of \autoref{thm:residue2} was a more general version of the Cauchy integral formula.
\cifagainagainthm*
\noindent We then closed the course by discussing M\"obius transformations and complex dynamics, for fun.
\begin{remark}
	In a future course, one might see Weierstrass factorization, the Riemann mapping theorem, and much more. We'll see you there.
\end{remark}
\end{document}