% !TEX root = ../notes.tex

Good morning, everyone.
\begin{itemize}
	\item There are office hours today from 11AM--12PM and 1PM--2:30PM. There are also office hours tomorrow from 10PM--12PM and 2PM--4PM.
	\item The midterm is still on Friday.
\end{itemize}

\subsection{Integral Commentary}
It's a review session today. We will be computing a lot of integrals for the midterm, for which we have many techniques. Here are some guidelines for finding the quickest computation for
\[\oint_\gamma f(z)\,dz.\]
\begin{enumerate}
	\item Is $f$ holomorphic or meromorphic almost everywhere? If not, we basically have to parameterize $\gamma$ and proceeding with the definition. For example, integrals such as
	\[\oint_\gamma\overline z\,dz\quad\text{ or }\quad\oint_\gamma|z|\,dz\quad\text{or}\quad\oint_\gamma\op{Re}z\,dz\]
	all fall under this category.

	\item If $f$ is close to holomorphic, look at the integral. We might try to pattern-match with
	\[f^{(n)}(w)=\frac{n!}{2\pi i}\oint_\gamma\frac{f(z)}{(z-w)^{n+1}}\,dz,\]
	where $n$ is some nonnegative integer. In life, sometimes this fails, and we still have to parameterize.
	\item If $f$ is meromorphic, we should use the Residue theorem, which states
	\[\oint_\gamma f(z)\,dz=\sum_{\text{poles }z_0}\op{Ind}(\gamma,z_0)\op{Res}_{z_0}(f),\]
	and we can compute the winding numbers and the residues by hand.
	\item If $f$ is not quite holomorphic or meromorphic but has an essential singularity, we can reparametrize the path to make the function $f$ meromorphic. Alternatively, we can use a power series expansion and attempt to switch the sum with the integral, using the residue theorem by hand. For example, we claim
	\[\oint_{|z|=1}z^2\sin(1/z)\,dz=-\oint_\eta z^{-4}\sin z\,dz\]
	by sending the path $\gamma$ to $\eta:=1/\gamma$. Indeed, running this through, we compute $\eta'(t)=-\gamma'(t)/\gamma(t)^2$, so
	\[\oint_{\gamma}z^2\sin(1/z)\,dz=\oint_0^1\gamma(t)^2\sin(1/\gamma(t))\gamma'(t)\,dz=-\oint_0^1\eta(t)^{-4}\sin(\eta(t))\eta'(t)\,dz=\oint_\eta z^{-4}\sin z\,dz\]
	Now, we can just compute this directly via the Residue theorem.
\end{enumerate}

\subsection{Review}
Here are some questions from class.
\begin{itemize}
	\item There might be a more general version of \autoref{cor:cifhomotopy} allowing for derivatives of $f$.
	\item Technically speaking, the Cauchy integral formula is a subset of the Residue theorem.
	\item We will not need homotopy on the exam.
\end{itemize}
Let's see some practice problems.
\begin{exe}
	Fix a polynomial $f(z)\in\CC[z]$ of degree $d\ge2$. Taking $R>0$ such that $f$ does not vanish for all $|z|\ge R$, we show that
	\[\oint_{|z|=R}\frac{dz}{f(z)}.\]
\end{exe}
\begin{proof}
	The fact that $f(z)$ does not vanish for $|z|\ge R$ promises us that $1/f(z)$ is holomorphic for $|z|\ge R$. The point, now, is to use the Residue theorem to bound the integral. Explicitly, pick up some $r\ge R$, and we set $\gamma_r$ to the counter-clockwise path around $|z|=r$ so that
	\[\oint_{\gamma_r}\frac{dz}{f(z)}=2\pi i\sum_{\substack{z_0\text{ zero of }f\\|z_0|\le r}}\underbrace{\op{Ind}(\gamma_r,z_0)}_1\cdot\op{Res}_{z_0}(f)=2\pi i\sum_{\substack{z_0\text{ zero of }f\\|z_0|\le R}}\underbrace{\op{Ind}(\gamma_R,z_0)}_1\cdot\op{Res}_{z_0}(f)=\oint_{\gamma_R}\frac{dz}{f(z)}\]
	because all poles of $1/f(z)$ are zeroes of $f(z)$, and those all live in the region with $|z|\le R\le r$. As such, the estimation lemma tells us that
	\[\left|\oint_{|z|=R}\frac{dz}{f(z)}\right|=\left|\oint_{|z|=r}\frac{dz}{f(z)}\right|\le2\pi r\cdot\max_{|z|=r}\left\{\frac1{|f(z)|}\right\}.\]
	To bound the size of $f(z)$, we set
	\[f(z)=\sum_{k=0}^da_kz^k\]
	so that
	\[\left|\frac1{f(z)}\right|\le\frac1{|a_d|\cdot |z|^d-|a_{d-1}|\cdot|z|^{d-1}-\cdots-|a_0|},\]
	so
	\[\left|\oint_{|z|=R}\frac{dz}{f(z)}\right|\le2\pi r\cdot\frac1{|a_d|\cdot r^d-|a_{d-1}|\cdot r^{d-1}-\cdots-|a_0|},\]
	which goes to $0$ as $r\to\infty$ because $d\ge2$. This finishes.
\end{proof}
\begin{exe}
	Fix a polynomial $p(z)\in\CC[z]$ of degree $n$. Suppose that we have some $M$ such that $|p(z)|\le M$ for $|z|<1$. Then we show that $|p(z)|\le M|z|^n$ for all $z$ with $|z|\ge1$.
\end{exe}
\begin{proof}
	The main point is that we know how $p$ behaves on $B(0,1)$ as a bound, so we are going to want to use the Maximum modulus principle. As such, we set $f(z):=p(z)/z^n$ and $g(z):=f(1/z)$; notably, a computation shows that $g$ is holomorphic (it's the ``reversed'' version of $p$), so we see that we already have a bound on the behavior of large values of $p$ from this.

	So now we push harder. By the Maximum modulus principle, the maximum of $|g(z)|$ on $\overline{B(0,1)}$ will be achieved on $\del B(0,1)$. But now, the values of $g$ agree with the values of $p$ on $\del B(0,1)$ (because $z\mapsto1/z$ is a bijection $\del B(0,1)\to\del B(0,1)$), and we know that the values of $p$ are upper-bounded by $M$ on $\del B(0,1)$. As such, we know that
	\[|g(z)|\le M\]
	on $\overline{B(0,1)}$, which rearranges to showing $\left|z^np(1/z)\right|\le M$ for all $z\in\overline{B(0,1)}\setminus\{0\}$ and so $|p(z)|\ge M\cdot|z|^n$ for all $z$ with $|z|\ge1$.
\end{proof}

% how is your day going?
% compute residues without pain?
% formal winding numbers/homotopy?
% use of the schwarz lemma?