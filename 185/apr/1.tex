% !TEX root = ../notes.tex

Good morning everyone. It's April Fool's day.
\begin{itemize}
	\item Homework \#7 is still due on Sunday at 11:59PM.
	\item There are office hours today.
\end{itemize}

\subsection{The Schwarz Lemma}
We quickly review the following result.
\schwarzlemma*
\begin{proof}
	The main point is to use the Maximum modulus principle on a specially chosen holomorphic function. We define $g:B(0,1)\to\CC$ as
	\[g(z):=\begin{cases}
		f(z)/z & z\ne0, \\
		f'(0) & z=0.
	\end{cases}\]
	As usual, we note that $g$ is holomorphic: we are holomorphic at all $z\ne0$ by restriction from $f(z)/z$, and we are in fact holomorphic at $z=0$ by doing a power series expansion there, by hand.

	We now have two cases.
	\begin{itemize}
		\item Now, if $g$ is constant, then $f(z)=az$ for each $z\in\CC$, for some fixed $a\in\CC$. We get $|f(z)|\le|z|$ because $|f(z)|\le1$ forces $|a|\le1$ (namely, by sending $z$ to the boundary of $B(0,1)$).
		\item Otherwise, take $g$ to be non-constant. To create a compact space, set $r\in(0,1)$ so that $\overline{B(0,r)}\subseteq B(0,1)$. Now, by compactness, we see that $|g|$ has a maximum on $\overline{B(0,r)}$, so \autoref{cor:mmp} tells us that each of these $r$ has a $w\in\del B(0,r)$, so
		\[|g(z)|\le|g(w)|=\frac{|f(w)|}{|w|}\le\frac1{|w|}=\frac1r\]
		for all $z\in B(0,r)$. Now, sending $r\to 1$, we get the inequality $|g(z)|\le1$ for all $z\in B(0,1)$, so $|f(z)|\le|z|$ follows.
	\end{itemize}
	The above casework finishes the first sentence of the proof.

	We now show the second sentence. If $|f(z)|=|z|$ for some nonzero $z\in B(0,1)$, then $g$ achieves $1$ on its interior, which we know must be now be its maximum. So \autoref{thm:mmp} forces $g$ to be constant, giving the result. Otherwise, if $f'(0)=1$, then $g(0)=1$, so again $g$ achieves its maximum in $B(0,1)$, so \autoref{thm:mmp} still forces $g$ to be constant.
\end{proof}
\begin{remark}
	The above result is approximately what lets us talk intelligently about automorphisms of $B(0,1)$.
\end{remark}

\subsection{Singularities}
We will spend the rest of lecture today discussing singularities.
\begin{definition}[Regular, singular]
	Fix an open and connected subset $\Omega\subseteq\CC$ with a function $f:\Omega\to\CC$.
	\begin{itemize}
		\item A point $z_0\in\overline\Omega$ is \textit{regular} if and only if $f$ is holomorphic at $z_0$.
		\item A point $z_0\in\overline\Omega$ is a \textit{singularity} otherwise.
	\end{itemize}
\end{definition}
\begin{definition}[Isolated singularity] \nirindex{Pole} \nirindex{Essential singularity} \nirindex{Removable singularity}
	Fix an open and connected subset $\Omega\subseteq\CC$ with a function $f:\Omega\to\CC$. A point $z_0\in\overline\Omega$ is an \textit{isolated singularity} if and only if we can find $r>0$ with $B(z,r)\subseteq\CC$ such that $f$ is holomorphic on $B(z_0,r)\setminus\{z\}$.
	\begin{itemize}
		\item $z_0$ is \textit{removable} if and only if $f$ is bounded near $z_0$.
		\item $z_0$ is a \textit{pole} if and only if $f$ is not bounded near $z_0$, but $z_0$ is a removable singularity of $1/f(z)$.
		\item $z_0$ is an \textit{essential singularity} if and only if $z_0$ is neither removable nor a pole.
	\end{itemize}
\end{definition}
\begin{remark}
	Being a removable singularity means that we can extend $f$ to be holomorphic at the point, by \autoref{thm:riemannremove}.
\end{remark}
Here are some examples.
\begin{example}
	The point $z_0=0$ is an isolated singularity of $f:\CC\setminus\{0\}\to\CC$ defined by $f(z)=\cos(z)/z^2$.
\end{example}
\begin{example}
	The point $z_0=0$ is a removable singularity of $f:\CC\setminus\{0\}\to\CC$ defined by $f(z)=\sin(z)/z$, which we can check by bounding $\sin$ near $0$.
\end{example}
\begin{example}
	The function $e^{1/z}$ has an essential singularity at $z_0=0$.
\end{example}
The point of introducing these notions is to expand our study of holomorphic functions. We take the following definition.
\begin{definition}[Meromorphic]
	Fix an open and connected subset $\Omega\subseteq\CC$. Then $f:\Omega\to\CC$ is \textit{meromorphic} if and only if all the singularities of $f$ are isolated and poles.
\end{definition}
The short version of where we are going is that meromorphic functions will also be very nice; for example, though they will not be literally power series at the singularity, they will be some power series with a finite negative tail, of sorts.

Anyway, we should probably prove something today.
\begin{lemma} \label{lem:poletozero}
	Fix an open and connected subset $\Omega\subseteq\CC$ with a function $f:\Omega\to\CC$. If $z_0\in\Omega$ is a pole of $f$, then
	\[\lim_{z\to z_0}\frac1{f(z)}=0.\]
\end{lemma}
Intuitively, poles of $f$ transfer to zeroes of $1/f$.
\begin{proof}
	We expand out the definitions. By definition, $z_0$ is a removable singularity of $1/f(z)$, and because our singularity is removable, we are promised an open ball $B(z_0,r)$ so that $f$ is nonzero at $B(z_0,r)\setminus\{z_0\}$, so we note that $1/f(z)$ will be holomorphic on this punctured ball.
	
	Further, $1/f(z)$ is bounded near $z_0$, so \autoref{thm:riemannremove} tells us that we can extend $1/f(z)$ to be holomorphic fully on $B(z_0,r)$, so we know that
	\[w:=frac1{f(z_0)}=\lim_{z\to z_0}\frac1{f(z)}\]
	after extending $1/f$ appropriately. We want to show that $w=0$. Well, suppose for the sake of contradiction that $w\ne0$ so that we have
	\[\lim_{z\to z_0}f(z)=\frac1w.\]
	However, this contradicts the fact that $f$ needs to not be bounded near $z_0$ because it does tell us that $(z-z_0)f(z)\to0$ as $z\to z_0$. In particular, we are now invoking the fact that $z_0$ is a pole.
\end{proof}
And here is the dual to this lemma.
\begin{lemma}
	Fix an open and connected subset $\Omega\subseteq\CC$ and $z_0\in\Omega$ with a function $f:\Omega\setminus\{z_0\}\to\CC$ so that $z_0$ is an isolated singularity of $f$. Then $z_0$ is a pole of $f$ if and only if
	\[\lim_{z\to z_0}|f(z)|=\infty.\]
\end{lemma}
\begin{proof}
	In the forward direction, $z_0$ being a pole forces
	\[\lim_{z\to z_0}\frac1{|f(z)|}=0\]
	by \autoref{lem:poletozero}. As such, we are forced to have
	\[\lim_{z\to z_0}|f(z)|=\infty.\]
	The backwards direction will require some effort. We need to show that $z_0$ is a removable singularity of $1/f$ and that $f$ is not bounded near $z_0$. On one hand, we know
	\[\lim_{z\to z_0}|f(z)|=\infty,\]
	but then we can rearrange to
	\[\lim_{z\to z_0}\frac{z-z_0}{f(z)}=0,\]
	so $z_0$ is indeed a removable singularity of $1/f$. On the other hand, suppose for the sake of contradiction that $f$ is bounded near $z_0$; then \autoref{thm:riemannremove} promises us that we can extend $f$ to be holomorphic on $\Omega$, and therefore we see
	\[\lim_{z\to z_0}f(z)\]
	exists. But then
	\[\lim_{z\to z_0}\frac1{|f(z)|}\]
	cannot be zero (it is either nonzero or not defined at all), which contradicts what we just showed.
\end{proof}

\subsection{Laurent Expansion}
To deal with singularities, we have the following definition.
\begin{definition}[Order]
	Fix an open and connected subset $\Omega\subseteq\CC$ with a function $f:\Omega\to\CC$. Given a pole $z_0\in\overline\Omega$ of $f$, we define the \textit{order} of $z_0$ as a pole to equal the multiplicity of $z_0$ as a zero of $1/f(z)$.
\end{definition}
Note that we are implicitly using \autoref{lem:poletozero}.

We have the following lemma, which is intended to be analogous to the fact that holomorphic functions are analytic.
\begin{restatable}{lemma}{laurent} \label{lem:laurent}
	Fix an open and connected subset $\Omega\subseteq\CC$ with a function $f:\Omega\to\CC$, and suppose that $z_0$ is a pole of $f$ with order $m>0$. Then there exists any sufficiently small real number $r\in\RR^+$ and a unique sequence $\{a_k\}_{k=-m}^\infty\subseteq\CC$ such that $z\in B(z_0,r)\setminus\{z_0\}$ has
	\[f(z)=\sum_{k=-m}^\infty a_k(z-z_0)^k.\]
\end{restatable}
\noindent In particular, the order of our pole controls the length of our tail.

We will not prove the above result today, but we will give the parts names.
\begin{definition}[Laurent expansion]
	In the context of \autoref{lem:laurent}, the ``power series'' expansion
	\[f(z)=\sum_{k=-m}^\infty a_k(z-z_0)^k\]
	is the \textit{Laurent expansion} of $f$ at $z_0$.
\end{definition}
\begin{definition}[Principal part]
	In the context of \autoref{lem:laurent}, we call the negative tail
	\[p_{f,z_0}(z):=\sum_{k=-m}^{-1}a_k(z-z_0)^k\]
	the \textit{principal part} of $f$ at $z_0$.
\end{definition}
Notably, the principal part is the ``bad'' part of our power series expansion.
\begin{restatable}[Residue]{definition}{resideudef}
	In the context of \autoref{lem:laurent}, we call $a_{-1}$ the \textit{residue} of $f$ at $z_0$, denoted $\op{Res}_{z_0}(f)$.
\end{restatable}
Later on we will be able to compute residues via integrals.