% !TEX root = ../notes.tex

Good morning, everyone.
\begin{itemize}
	\item Midterm \#2 is on Friday. Both practice problems and a practice midterm were released.
	\item There are extra office hours.
	\item There is a review session on Wednesday.
\end{itemize}

\subsection{Rouch\'e's Theorem}
We proved the following result on the homework.
\begin{theorem}[Argument principle] \label{thm:argpri}
	Fix a domain $\Omega$ and a meromorphic function $f:\Omega\to\CC$, and pick up $z_0\in\Omega$ and $r>0$ such that $B(z_0,r)\subseteq\Omega$ and $f$ has no zeroes nor poles on $\del B(z_0,r)$. Further, set the following.
	\begin{itemize}
		\item $N_f$ is the number of zeroes of $f$, counted with multiplicity, in $B(z_0,r)$.
		\item $P_f$ is the number of poles of $f$, counted with multiplicity, in $B(z_0,r)$.
	\end{itemize}
	Then
	\[N_f-P_f=\frac1{2\pi i}\oint_\gamma\frac{f'(z)}{f(z)}\,dz=\op{Int}(f\circ\gamma,0),\]
	where $\gamma:[0,1]\to\CC$ is $\gamma(t):=z_0+r\exp(2\pi it)$.
\end{theorem}
\begin{proof}
	The point is to use the Residue theorem on $f'/f$. We can check that $f'/f$ will only have poles when either $f(z)$ has a pole or zero. Then, at a point $w\in\Omega$, we can write down
	\[f(z)=(z-w)^ng(z)\]
	for some integer $n$ and for some holomorphic function $g:\Omega\to\CC$ such that $g(w)\ne0$. Then we can see that
	\[\frac{f'(z)}{f(z)}=\frac n{z-w}+\frac{g'(z)}{g(z)}\]
	by taking the derivative by hand, so we can see that
	\[\op{Res}_w(f'/f)=n.\]
	Thus, $\op{Res}_w(f'/f)$ counts zeroes with multiplicity positively and counts poles with multiplicity negatively. Summing over these residues in $B(z_0,r)$ (via \autoref{thm:residue}) gives the result.
\end{proof}
Now, here is the statement we are going to prove today.
\begin{theorem}[Rouch\'e's] \label{thm:rou}
	Fix a domain $\Omega$ and two holomorphic functions $f,g:\Omega\to\CC$. Further, suppose that we have $z_0\in\Omega$ and $r>0$ such that $\overline{B(z_0,r)}\subseteq\Omega$ and
	\[|g(z)|<|f(z)|\]
	for each $z\in\del B(z_0,r)$. Then $f$ and $f+g$ have the same number of zeroes, counted with multiplicity, contained in the ball $B(z_0,r)$.
\end{theorem}
\begin{remark}
	As in \autoref{thm:cif}, the main point is that we can talk about the behavior of $f$ by only weak information at the boundary. In particular, perturbations by ``small'' functions $g$ are unable to alter how $f$ works in practice.
\end{remark}
\begin{warn}
	The proof in the Eterovi\'c notes is incorrect.
\end{warn}
The point of \autoref{thm:rou} is to be able to count and determine the location of zeroes of some holomorphic function $f$ by relating $f$ to a simpler function.
\begin{exe}
	We show that the roots of the polynomial $p(z)=z^4+5z+2$ all lie in $B(0,2)$.
\end{exe}
\begin{proof}
	To be able to use Rouch\'e's theorem, we need to choose some $f$ and $g$. Because $g$ should be some ``small'' perturbation to $f$, we take $f(z):=z^4$ and $g(z):=5z+2$ so that $p(z)=f(z)+g(z)$. Now, for $z\in\del B(0,2)$, we see that
	\[|g(z)|=|5z+2|\le5|z|+2=5\cdot2+2=12<16=2^4\le|z|^4=|f(z)|.\]
	Thus, \autoref{thm:rou} tells us that $f$ and $f+g$ have the same number of zeroes in $B(0,2)$, but $f$ has four zeroes in $B(0,2)$ when counted with multiplicity (namely, four zeroes at $z=0$), so we can say the same for $p=f+g$. This finishes.
\end{proof}
Now, let's prove \autoref{thm:rou}.
\begin{proof}[Proof of \autoref{thm:rou}]
	Note that $f$ has no zeroes on $\del B(z_0,r)$ because $f$ is strictly larger than $|g(z)|\ge0$ for each $z\in\del B(z_0,r)$. Similarly, $|f(z)+g(z)|\ge|f(z)|-|g(z)|>0$ for $z\in \del B(z_0,r)$ by assumption, so $f+g$ also has no zeroes on this boundary. As such, we define
	\[h(z):=\frac{f(z)+g(z)}{f(z)}.\]
	Further, set $\gamma:[0,1]\to\Omega$ by $\gamma(t):=z_0+r\exp(2\pi it)$ to trace out $\del B(z_0,r)$.

	Continuing, note that the zeroes of $h$ will only occur at zeroes of $f(z)+g(z)$, and the poles of $h(z)$ will occur only at poles of $f(z)$. Now, $h$ has neither zero nor pole on $\del B(z_0,r)$, so \autoref{thm:argpri} tells us that
	\[N_h-P_h=\op{Ind}(h\circ\gamma,0).\]
	Notably, $N_h-P_h$ is the number of zeroes of $f+g$ minus the number of zeroes of $f$, even if there is some cancellation with having a zero in the same place. Thus, we would like to show that the above integral vanishes.

	Well, for each $z\in\del B(z_0,r)$, we see
	\[|h(z)-1|=\left|\frac{f(z)+g(z)}{f(z)}-1\right|=\left|\frac{g(z)}{f(z)}\right|<1.\]
	Thus, $\im(h\circ\gamma)\subseteq B(1,1)$ and in particular is nonzero everywhere. In particular, $h\circ\gamma$ does not wind around $0$ at all, so $\op{Ind}(h\circ\gamma)=0$.
\end{proof}
And here is another example.
\begin{exe}
	We compute the number of zeroes of $p(z):=z^5+3z^2+1$ for $z$ in the ``annulus'' $1<|z|<2$.
\end{exe}
\begin{proof}
	Here is our image.
	\begin{center}
		\begin{asy}
			unitsize(0.8cm);
			fill(circle((0,0), 2), lightgray);
			fill(circle((0,0), 1), white);
			draw(circle((0,0), 2), dashed);
			draw(circle((0,0), 1), dashed);
			draw((-3,0)--(3,0), arrow=EndArrow);
			draw((0,-3)--(0,3), arrow=EndArrow);
			label("$\textrm{Re}$", (3,0), E);
			label("$\textrm{Im}$", (0,3), N);
		\end{asy}
	\end{center}
	The point is to find the zeroes in $B(0,2)$ and the zeroes in $B(0,1)$ and then subtract. As such, we do two computations.
	\begin{itemize}
		\item For $z\in\del B(0,1)$, we have $|z|=1$, so we note $g(z):=z^5+1$ and $f(z):=3z^2$ give
		\[|g(z)|=\left|z^5+1\right|\le2<3=3|z|^2=|f(z)|,\]
		so we conclude that $p=f+g$ has two zeroes in $B(0,1)$.
		\item For $z\in\del B(0,2)$, we have $|z|=2$, so we note $g(z):=3z^2+1$ and $f(z):=z^5$ give
		\[|g(z)|=\left|3z^2+1\right|\le3\cdot4+1=13\le32=|z|^5=|f(z)|,\]
		so we conclude that $p=f+g$ has all five zeroes in $B(0,2)$.
	\end{itemize}
	Subtracting, it follows that there are three zeroes in $B(0,2)\setminus B(0,1)$. To claim this as our answer, we check that there is no zero on $\del B(0,1)$. Well, if $|z|=1$, then we compute
	\[\left|z^5+3z^2+1\right|\ge\left|3z^2\right|-\left|z^5\right|-|1|=3-1-1>0,\]
	so there are no zeroes here. Thus, there are indeed $\boxed3$ total zeroes in the annulus.
\end{proof}

\subsection{The Open Mapping Theorem}
We close class with the following nice consequence of \autoref{thm:rou}.
\begin{theorem}[Open mapping]
	Fix a domain $\Omega$ and a non-constant holomorphic function $f:\Omega\to\CC$. For open subsets $U\subseteq\Omega$, the set $f(U)$ is also open.
\end{theorem}
This is very surprising! For example, this is very much not true in $\RR$: the function $f(x):=\sin x$ sends the open set $\RR$ to $[-1,1]$, which is closed. In general, continuous and even differentiable functions really not need be open---open is a very different notion.
\begin{proof}
	Fix $w_0\in f(U)$ with $z_0\in U$ such that $f(z_0)=w_0$, and we need to put a neighborhood around $w_0$ in $f(U)$. To help us our, we define $g:\Omega\to\CC$ by
	\[g(z):=f(z)-w_0\]
	so that $g(z_0)=0$. Now, $g$ is a non-constant holomorphic function, so \autoref{thm:id} tells us that $g$ cannot have zeroes accumulating to $z_0$ (lest $g$ be equivalent to $0$), so there is some $r>0$ such that $g$ does not vanish on
	\[\overline{B(z_0,r)}\setminus\{z_0\}.\]
	Further, by making $r$ small enough, we can also assume that $\overline{B(z_0,r)}\setminus\{z_0\}\subseteq U$. Now, $\del B(z_0,r)$ is compact, so we can find $\delta>0$ such that
	\[|g(z)|\ge\delta\]
	for all $z\in\del B(z_0,r)$ because continuous functions have achieved minimums, and $g$ never achieves $0$. This $\delta$ will give our neighborhood.

	We are now almost ready to apply Rouch\'e's theorem. In particular, we would like to show that $B(z_0,\delta)\subseteq f(U)$. Well, pick up some $w\in B(w_0,\delta)$, and we set $h_w:\Omega\to\CC$ by
	\[h_w(z):=g(z)+w_0-w.\]
	In particular, we can compute that
	\[|w_0-w|<\delta<|g(z)|\]
	for all $z\in\del B(z_0,r)$, so \autoref{thm:rou} promises us that $h_w(z)=g(z)+w_0-w$ has the same number of zeroes as $g$ on $B(w_0,\delta)$. However, by construction of $r$, we see that $g$ has a zero in $B(z_0,r)$, so $h$ does as well, so there exists $z\in B(z_0,r)\subseteq U$ such that $h_w(z)=0$ and hence $f(z)=w$, giving $w\in f(U)$. This finishes.
\end{proof}