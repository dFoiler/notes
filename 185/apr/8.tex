\documentclass[../notes.tex]{subfiles}
\graphicspath{{\subfix{../figs/}}}

\begin{document}
% !TEX root = ../notes.tex

Good morning everyone.
\begin{itemize}
	\item Homework \#8 is due tonight at 11:59PM.
	\item There are office hours today from 1PM to 2:30PM.
	\item Midterm \#2 is next Friday. A review has been posted, with review problems and a practice midterm to come.
\end{itemize}
\begin{remark}[Morrow]
	Fun life tip: if you show up 10 minutes to jury duty, they will have enough jurors, and you will not get in trouble, so you will be excused.
\end{remark}

\subsection{Integrals in Simply Connected Domains}
We continue our discussion of homotopy. We will not go over every single proof because they are somewhat laborious. Last time we showed the following.
\homoind*
\noindent Have some corollaries.
\begin{corollary} \label{cor:simplyconnectedpathints}
	Fix a simply connected domain $\Omega$ and a holomorphic function $f\colon \Omega\to\CC$. Given two paths $\gamma,\eta\colon [0,1]\to\CC$ with the same endpoints $\gamma(0)=\eta(0)$ and $\gamma(1)=\eta(1)$, we have
	\[\int_\gamma f(z)\,dz=\int_\eta f(z)\,dz.\]
\end{corollary}
\begin{proof}
	Because $\Omega$ is simply connected, $\gamma$ and $\eta$ have a homotopy with fixed endpoints between them.
\end{proof}
\begin{corollary} \label{cor:simpconnisprimitive}
	Fix a domain $\Omega$. If $\Omega$ is simply connected, then $\Omega$ is primitive.
\end{corollary}
\begin{proof}
	The point is to use \autoref{thm:getprimitive}. As usual, pick up some holomorphic function $f\colon \Omega\to\CC$ which we would like to give a primitive, and choose any closed piecewise $C^1$ path $\gamma\colon [0,1]\to\Omega$, so we want to show
	\[\oint_\gamma f(z)\,dz\stackrel?=0,\]
	which by \autoref{thm:getprimitive} will provide us with a primitive. Well, because $\Omega$ is simply connected and $\gamma(0)=\gamma(1)\eqqcolon z_0$, we see that $\gamma$ is homotopic with fixed endpoints to the path $c\colon [0,1]\to\CC$ defined by
	\[c(t)=z_0\]
	for all $t\in[0,1]$. However, \autoref{cor:simplyconnectedpathints} now tells us that
	\[\oint_\gamma f(z)\,dz=\oint_c f(z)\,dz=\int_0^1f(c(t))c'(t)\,dt,\]
	but this last integral is $0$ because $c'(t)=0$.
\end{proof}
\begin{example}
	We now know that the Residue theorem (\autoref{thm:residue}) applies to simply connected domains.
\end{example}

\subsection{A Better Cauchy Integral Formula}
One of the main goals of homotopy is to be able to get a more general version of the Cauchy integral formula. Take the following definition.
\begin{definition}[Homologous to zero]
	Fix a domain $\Omega$. Then a closed, piecewise $C^1$ path $\gamma\colon [0,1]\to\Omega$ is \textit{homologous to $0$} if and only if $\op{Ind}(\gamma,w)=0$ for all $w\in\CC\setminus\Omega$.
\end{definition}
Roughly speaking, we are requiring that a path homologous to $0$, 
\begin{theorem}[Cauchy integral formula] \label{thm:cif2}
	Fix a domain $\Omega$ and a closed, piecewise $C^1$ path $\gamma\colon [0,1]\to\Omega$ which is homologous to $0$. Then, given a holomorphic function $f\colon \Omega\to\CC$, we have
	\[\oint_\gamma f(z)\,dz=0,\]
	and for any $w\in\Omega\setminus\im\gamma$, we have
	\[f(w)\op{Ind}(\gamma,w)=\frac1{2\pi i}\oint_\gamma\frac{f(z)}{z-w}\,dz.\]
\end{theorem}
Note that we are able to recover the first version of the Cauchy integral formula (namely, \autoref{thm:cif}) by setting
\[\gamma(t)=z_0+r\exp(2\pi it),\]
where $\im\gamma=\overline{B(z_0,r)}\subseteq\Omega$. In this case, \autoref{thm:starlikecg} was roughly speaking able to give us
\[\oint_\gamma f(z)\,dz=0,\]
and \autoref{thm:cif} was able to give us
\[f(w)=\frac1{2\pi i}\oint_\gamma\frac{f(z)}{z-w}\,dz\]
for any $\gamma$ inside the loop. This last part we generalize past the loop $\gamma$ above to a more general closed, piecewise $C^1$ path homologous to $0$, but we have to add in a winding number, lest we do something silly like $\gamma*\gamma$.

To prove \autoref{thm:cif2}, we will need the following result, but we will not prove it because it is somewhat technical.
\begin{proposition} \label{prop:surgery}
	Fix a domain $\Omega$ and a closed, piecewise $C^1$ path $\gamma\colon [0,1]\to\Omega$. Given a $w\in\im\gamma$, then we can generate a closed, piecewise $C^1$ path $\eta$ with $w\notin\im\eta$ while
	\[\oint_\gamma f(z)\,dz=\oint_\eta f(z)\,dz\]
	for any holomorphic function $f\colon \Omega\to\CC$.
\end{proposition}
\begin{proof}
	The point is to do ``surgery'' on $\gamma$ to avoid $w$. Here is the image of $\gamma$ with some bad $w\in\im\gamma$.
	\begin{center}
		\begin{asy}
			unitsize(1cm);
			draw((1,0) .. (0,1) .. (-1,0) .. (0,-1) .. cycle, arrow=EndArrow);
			label("$\gamma$", (1,0), E);
			dot("$w$", (-1,0), W, red);
		\end{asy}
	\end{center}
	Now, we explode $w$ a little as follows to make our $\eta$, as follows.
	\begin{center}
		\begin{asy}
			unitsize(1cm);
			//draw((1,0) .. (0,1) .. (-1,0) .. (0,-1) .. cycle, arrow=EndArrow);
			label("$\eta$", (1,0), E);
			dot("$w$", (-1,0), W, red);
			draw(arc((0,0), 1, 0,150), arrow=EndArrow);
			draw(arc((0,0), 1, 210,360), arrow=EndArrow);
			draw(arc((-1,0), 0.5176, -75, 75), arrow=BeginArrow);
		\end{asy}
	\end{center}
	By making the ball small enough, we can ensure that the entire ball lives in $\Omega$, and this ball is simply connected, so the integrals over any $f$ are the same by \autoref{cor:simplyconnectedpathints}, roughly speaking.
\end{proof}
Anyway, here is our proof.
\begin{proof}[Proof of \autoref{thm:cif2}]
	We proceed in steps. Replace $\Omega$ with some bounded domain containing $\gamma$, which we can do because $\im\gamma$ is compact and hence bounded. This won't affect the content of the conclusions; we merely have to replace $f$ with its restriction.
	\begin{enumerate}
		\item We define some $\widetilde F$. Define $\widetilde F\colon (\CC\setminus\im\gamma)\to\CC$ by
		\[\widetilde F(w)\coloneqq \int_0^1\frac{f(\gamma(t))-f(w)}{\gamma(t)-w}\cdot\gamma'(t)\,dt.\]
		Notably, $\widetilde F$ is holomorphic on $\CC\setminus\im\gamma$ by writing out a power series expansion at each point and then integrating the power series expansion by hand using some local absolute convergence result.

		Philosophically, the point is to show that $\widetilde F\equiv0$, which will give the second desired equality
		\[f(w)\op{Ind}(\gamma,w)\stackrel?=\frac1{2\pi i}\oint_\gamma\frac{f(z)}{z-w}\,dz\]
		by rearranging. In particular, we will show that $\widetilde F$ can be extended to be entire and bounded (which by \autoref{thm:liouville} forces $\widetilde F$ to be constant), and then we will show that $\widetilde F$ takes the value $0$ somewhere.

		\item We now extend to $\Omega$. Give some $w\in\Omega$, we define
		\[g(z,w)\coloneqq \begin{cases}
			\frac{f(z)-f(w)}{z-w} & z\ne w, \\
			f'(w) & z=w.
		\end{cases}\]
		We can check by hand (e.g., using a power series expansion) that $g(-,w)$ is holomorphic on $\Omega$; notably, we are holomorphic at each $z\ne w$ for free, and then we can build a power series expansion at $w$ by hand. As such, we extend $\widetilde F$ to $F\colon \Omega\to\CC$ by
		\[F(w)\coloneqq \oint_\gamma g(z,w)\,dz.\]
		Notably, $F$ does indeed restrict down to
		\[F(w)=\int_0^1\frac{f(\gamma(t))-f(w)}{\gamma(t)-w}\cdot\gamma'(t)\,dt=\widetilde F(w)\]
		for $w\in\CC\setminus\im\gamma$.

		\item We check that $F$ is holomorphic on $\Omega$. Our only problem is to check points $w\notin\im\gamma$. By \autoref{prop:surgery}, there exists $\eta$ with $w\notin\im\eta$ such that
		\[F(w)=\oint_\gamma g(z,w)\,dz=\oint_\eta g(z,w)\,dz,\]
		but now this last integral is manifestly holomorphic at $w$ because $w\notin\im\eta$, where here we are appealing to the previous steps to note that
		\[F(w)=\oint_\eta f(z)\,dz=\int_0^1\frac{f(\eta(t))-f(w)}{\eta(t)-w}\,dt\]
		is holomorphic on $\CC\setminus\im\eta$ and in particular at $w\notin\im\eta$.

		\item We check that $F$ is entire. Well, we have shown that $F$ is holomorphic on $\im\gamma$, and we started with $\widetilde F$ which is holomorphic on $\CC\setminus\im\gamma$, so we can glue these together to get an actual entire function.

		\item We show the integral formulae. Note that $F$ is continuous and hence uniformly continuous on the compact set $\im\gamma$, so $F$ is bounded there. On the other hand, we see that taking $w\notin\CC\setminus\im\gamma$ gives
		\begin{align*}
			F(w) &= \int_0^1\frac{f(\gamma(t))-f(w)}{\gamma(t)-w}\cdot\gamma'(t)\,dt \\
			&= \oint_\gamma\frac{f(z)}{z-w}\,dz-f(w)\oint_\gamma\frac1{z-w}\,dz \\
			&= \oint_\gamma\frac{f(z)}{z-w}\,dz-2\pi i\cdot f(w)\op{Ind}(\gamma,w).\tag{$*$}\label{eq:rearrangef}
		\end{align*}
		Notably, for $w\notin\CC\setminus\Omega$, the term $\op{Ind}(\gamma,w)$ will vanish because $\gamma$ is homologous to $0$ (!). Because $\Omega$ is bounded, fix $R\in\RR^+$ with $\Omega\subseteq B(0,R)$, we can just say that $w$ with $|w|>R$ will have
		\[F(w)=\oint_\gamma\frac{f(z)}{z-w}\,dz.\]
		Now, $t\mapsto f(\gamma(t))$ is a continuous function $[0,1]\to\RR$ on a compact set and hence has a maximum $M$. As such, we use \autoref{lem:estimation} to write
		\[|F(w)|=\left|\oint_\gamma\frac{f(z)}{z-w}\,dz\right|\le\ell(\gamma)\cdot\max_{t\in[0,1]}\left\{\left|\frac{f(\gamma(t))}{\gamma(t)-w}\right|\right\}\le\frac{M\ell(\gamma)}{|w|-R}\]
		for $|w|>R$. Now, sending $|w|\to\infty$ causes $|F(w)|\to0$.

		To finish, being entire implies that $F$ is bounded on the compact set $\overline{B(0,R+1)}$. Further, we have bounded
		\[|F(w)|\le\frac{M\ell(\gamma)}{R+1-R}\]
		for $|w|\ge R+1$, so $F$ is a bounded, entire function and hence constant by \autoref{thm:liouville}. However, $|F(w)|\to0$ as $|w|\to\infty$, so we must have $F\equiv0$. So we conclude that any $w\notin\im\gamma$ will have
		\[f(w)\op{Ind}(\gamma,w)=\frac1{2\pi i}\oint_\gamma\frac{f(z)}{z-w}\,dz\]
		by rearranging $F\equiv0$ with \autoref{eq:rearrangef}.

		\item It remains to show that $\oint_\gamma f(z)\,dz=0$. Well, given $w\in\Omega\setminus\im\gamma$, we define
		\[g_w(z)=(z-w)f(z).\]
		Then we compute
		\[\frac1{2\pi i}\oint_\gamma f(z)\,dz=\frac1{2\pi i}\oint_\gamma\frac{g_w(z)}{z-w}\,dz\stackrel*=g_w(w)\op{Ind}(\gamma,w)=0\]
		by using the integral formula in $\stackrel*=$.
		\qedhere
	\end{enumerate}
\end{proof}
To close out class, have a corollary, where we impose conditions on $\Omega$ instead of $\gamma$.
\begin{corollary} \label{cor:cifhomotopy}
	Fix a simply connected domain $\Omega$ and a closed, piecewise $C^1$ path $\gamma\colon [0,1]\to\Omega$. Then, given a holomorphic function $f\colon \Omega\to\CC$, we have
	\[\oint_\gamma f(z)\,dz=0,\]
	and for any $w\in\Omega\setminus\im\gamma$, we have
	\[f(w)\op{Ind}(\gamma,w)=\frac1{2\pi i}\oint_\gamma\frac{f(z)}{z-w}\,dz.\]
\end{corollary}
\begin{proof}
	The main point is to show that $\gamma$ is in fact homologous to $0$, from which the result will follow directly from \autoref{thm:cif2}.

	As such, pick up $w\in\CC\setminus\Omega$, and we show that $\op{Ind}(\gamma,w)=0$. Because $w\in\CC\setminus\Omega$, the function $f(z)\coloneqq \frac1{z-w}$ is holomorphic on $\Omega$ as the quotient of nonzero holomorphic functions. Now, \autoref{cor:simpconnisprimitive} promises that $f$ has a primitive, so \autoref{cor:ftconclosed} forces
	\[\op{Ind}(\gamma,w)=\frac1{2\pi i}\oint_\gamma\frac1{z-w}\,dz=\frac1{2\pi i}\oint_\gamma f(z)\,dz=0,\]
	which is what we wanted.
\end{proof}
\end{document}