% !TEX root = ../notes.tex

Last class there was a midterm. Today we mourn.
\begin{itemize}
	\item Homework \#9 will be posted later today, due Sunday at 11:59PM.
	\item Homework \#10 will be the last homework.
	\item Midterm \#2 will be returned on Wednesday.
\end{itemize}

\subsection{Applications of Rouch\'e's Theorem}
We begin by recalling the statement, as follows.
\routhm*
\noindent The main use of \autoref{thm:rou} is to determine where there are zeroes of a given holomorphic function. We also showed \autoref{thm:openmap}; on the homework, we will prove the Fundamental theorem of algebra.

Before continuing, we give another example.
\begin{exe}
	We compute the number of roots of $h(z)=6z^3+\exp(z)+1$ in $B(0,1)$.
\end{exe}
\begin{proof}
	Note that $f$ is holomorphic, so although this is not a polynomial, we can still use \autoref{thm:rou}. Indeed, our largest term seems to be $f(z)\coloneqq 6z^3$ and $g(z)\coloneqq \exp(z)+1$ so that, for $z\in\del B(0,1)$,
	\[|\exp(z)+1|\le|\exp(z)|+1\stackrel*\le\exp(|z|)+1\le e+1<6=6\cdot\left|z^3\right|=|f(z)|,\]
	where $\stackrel*\le$ holds by expanding out $\exp$ as a series. It follows that $f$ and $h=f+g$ have the same number of zeroes, so $h$ has $\boxed3$ zeroes in $B(0,1)$.
\end{proof}
Anyway, let's prove something today.
\begin{proposition} \label{prop:locallyinjective}
	Fix a domain $\Omega$ and a non-constant holomorphic function $f\colon \Omega\to\CC$. Given $z_0\in\Omega$, then $f'(z_0)\ne0$ if and only if $f|_{B(z_0,r)}$ is injective for some $r>0$.
\end{proposition}
Intuitively, we are saying that having derivative zero means that $f$ is locally injective.
\begin{ex}
	The function $f(z)=z^2$ is not injective on $B(0,r)$ for any $r>0$.
\end{ex}
Anyway, let's prove this.
\begin{proof}[Proof of \autoref{prop:locallyinjective}]
	We show the directions independently.
	\begin{itemize}
		\item We start by taking $f'(z_0)\ne0$; we imitate the proof of \autoref{thm:openmap}. Let $w_0\coloneqq f(z_0)$ and define
		\[g(z)\coloneqq f(z)-w_0\]
		so that $g(z_0)=0$. Additionally, because $f$ is non-constant, $g$ is also non-constant and in particular not zero everywhere, so \autoref{thm:id} forces $z_0$ to be an isolated zero of $g$. As such, there is some $r_0>0$ such that $g$ does not vanish on
		\[\overline{B(z_0,r_0)}\setminus\{z_0\}.\]
		We now bring in the condition $f'(z_0)\ne0$: because $f'(z_0)\ne0$, we see $g'(z_0)\ne0$, so $z_0$ is a zero of $g'(z_0)$ of multiplicity $1$---indeed, if we had $g(z)=(z-z_0)^2h(z)$, then $g'(z)=(z-z_0)\big(2h(z)+(z-z_0)h'(z)\big)$, so $g'(z_0)=0$. It follows that $g$ has one zero in $B(z_0,r_0)$, at $z=z_0$, even when counted with multiplicity.
	
		We now continue as in \autoref{thm:openmap}. Because $\del B(z_0,r_0)$ is closed and bounded and hence compact, there exists $\delta>0$ so that
		\[|g(z)|\ge\delta\]
		for all $z\in\del B(z_0,r_0/2)$ by giving $|g|$ a minimum; we can set $\delta>0$ because $g$ does not vanish on $\del B(z_0,r_0/2)$.
		% By making $\delta$ small enough, we can make $\delta<r_0/2$; this $\delta$, again, will give our neighborhood.
	
		Now, to apply \autoref{thm:rou}, we pick up some $w\in B(w_0,\delta)$, and we would like to show 
		\[h_w(z)=g(z)+w_0-w\]
		has exactly one root in $B(w_0,\delta)$; this will be enough because it shows $g$ is injective on $g^{-1}(B(w_0,\delta))$, from which we can extract an open neighborhood around $z_0$. Well, we compute
		\[|w_0-w|<\delta\le|g(z)|,\]
		for $z\in\del B(z_0,r_0/2)$, so $h_w$ and $g$ have the same number of roots on $B(w_0,\delta)$ by \autoref{thm:rou}, which in particular is exactly one by our discussion above.
	
		\item Now, suppose that $f'(z_0)=0$, and we show that $f$ is not injective on some any neighborhood around $z_0$; as such, fix any $r>0$, and we show $f$ is not injective on $B(z_0,r)$. Because $f$ is holomorphic, it is analytic, so by taking $r$ small enough (which will not harm our conclusion because our injectivity is local), we have
		\[f(z)=\sum_{k=0}^\infty a_k(z-z_0)^k\]
		for $z\in B(z_0,r)$. Because $f'(z_0)=0$, we have $a_1=0$ above. However, $f'$ is holomorphic and non-constant (because $f$ is holomorphic and non-constant, so some $a_k\ne0$ for $k>1$ above due to $a_1=0$), so \autoref{thm:id} forces $z_0$ to be an isolated zero of $f'$. In particular, we may take $r$ even smaller so that $f'$ does not vanish on
		\[\overline{B(z_0,r)}\setminus\{z_0\}.\]
		Running through the argument in the previous point once more, we are told that, for some $w$ in the image of $f$ under $B(z_0,r)$ not equal to $f(w)$, we have that
		\[f(z)=w\]
		has at least two roots in $B(z_0,r)$, counted with multiplicity.

		We now push this further. If $f$ were in fact injective on $B(z_0,r)$, then $f(z)-w$ has a double root at some $z=z_1\in B(z_0,r)$, but then $f'(z_1)=0$ would follow, which contradicts our construction of $r$ because $f'$ does not vanish on $\overline{B(z_0,r)}\setminus\{z_0\}$.
		\qedhere
	\end{itemize}
\end{proof}
\begin{remark}
	We can measure the failure of the locally injective by staring carefully at the argument at the end: if $f(z)-f(z_0)$ has a root of multiplicity $m$ at $z=z_0$, then $f$ is $m$-to-$1$ in some neighborhood around $z_0$. 
\end{remark}
\begin{nex}
	In real analysis, this statement is not true. For example, $f(z)\coloneqq z^3$ is bijective on $\RR$ while $f'(0)=0$. The issue here is that working in $\RR$ is hiding the ``rotation'' that $f$ is doing.
\end{nex}

\subsection{The Inverse Function Theorem}
We close class with the following result.
\begin{theorem}[Inverse function] \label{thm:inv}
	Fix a domain $\Omega$ and an injective, holomorphic function $f\colon \Omega\to\CC$. If $g\colon \im f\to\Omega$ is the right inverse of $f$ (i.e., $f(g(z))=z$ for all $z\in\im f$), then $g$ is holomorphic, and
	\[g(w)=\frac1{f'(g(w))}\]
	for all $w\in\im $.
\end{theorem}
\begin{proof}
	We proceed in steps.
	\begin{enumerate}
		\item We show that $g$ is continuous. Well, take $U\subseteq\Omega$, and we need to show $g^{-1}(U)\subseteq\im f$ is open. For this, we simply write down the computation
		\[g^{-1}(U)=g^{-1}\left(f^{-1}(f(U))\right)=(f\circ g)^{-1}(f(U))=\id_{f(\Omega)}(f(U)).\]
		Notably, we are using the fact that $f$ surjects onto $U\subseteq\Omega$ to say that $U=f^{-1}(f(U))$. Now, $f(U)$ is open by \autoref{thm:openmap}, so we are done.
		\item We now compute the derivative of $g$ by hand. Note that $f$ is injective, so $f'$ is locally injective everywhere, so $f'(z)\ne0$ for all $z\in\Omega$ by \autoref{prop:locallyinjective}.
		
		Now, fix $z_0\in\Omega$ and $w_0\coloneqq f(z_0)$, which implies hat $g(w_0)=z_0$ is forced by the injectivity of $f$. Note that any $w\in\im f$ will have some unique $z\in\Omega$ with $f(z)=w$ by injectivity. As such, the continuity of $f$ and $g$ implies that a sequence $\{w_n\}_{n\in\NN}\subseteq\im f$ has some unique pullbacks $z_n\coloneqq g(w_n)$ so that $f(z_n)\to w_0$ if and only if $z_n\to z$. Thus, we can compute
		\[\lim_{w\to w_0}\frac{g(w)-g(w_0)}{w-w_0}=\lim_{z\to z_0}\frac{z-z_0}{f(z)-f(z_0)}=\frac1{\lim_{z\to z_0}\frac{f(z)-f(z_0)}{z-z_0}}=\frac1{f'(z_0)}=\frac1{f'(g(w_0))},\]
		which is what we wanted.
		\qedhere
	\end{enumerate}
\end{proof}
\begin{remark}
	This result is somewhat surprising: a priori, we should only expect our inverse to be some set-theoretic construction, but in our case this happens to be holomorphic.
\end{remark}