% !TEX root = ../notes.tex

Good morning everyone. Today's lecture was not recorded.
\begin{itemize}
	\item Homework \#5 will be uploaded today, due next Friday.
	\item The class average on the midterm was a $74$; it might have been a little long. There will probably be something approximately equal to a $6$-point curve.
\end{itemize}

\subsection{Arguments}
Today we talk more about the exponential function. Last time we proved the following.
\polarform*
\noindent As a brief review, we recall that we took $r=|z|$, and we computed $\theta$ in terms of some $\arctan$s. Essentially, this means that we can effectively compute polar form without tears.
\begin{remark}
	The interval $[-\pi,\pi)$ is somewhat arbitrary; we can choose any set of representatives for $\RR/2\pi\ZZ$. To see this, we note that the unique $\theta\in[-\pi,\pi)$ will have a unique representative in any set of representatives for $\RR/2\pi\ZZ$ and vice versa. For example, any half-open interval of length $2\pi$ (such as $[0,2\pi)$) will do the trick. To see this, 
\end{remark}

We can in fact use polar form to talk about the exponential map.
\begin{corollary} \label{cor:expsurj}
	For any $z\in\CC^\times$, there exists some $w\in\CC$ such that $\exp(w)=z$.
\end{corollary}
\begin{proof}
	To start, we know that we can write $z=r\exp(i\theta)$. So, using real analysis, we set
	\[w:=\log r+i\theta,\]
	where $\log:\RR^+\to\RR$ is the real logarithm. Thus,
	\[z=r\exp(i\theta)=\exp(\log r)\exp(i\theta)=\exp(\log r+i\theta)=\exp(w),\]
	which is what we wanted.
\end{proof}
Continuing to talk about polar form, we have the following definition.
\begin{definition}[Argument]
	Given a complex number $z\in\CC^\times$, we define the \textit{principal argument} $\arg z\in[-\pi,\pi)$ by writing $z:=r\exp(i\theta)$ and taking $\arg z:=\theta$. More generally, for any $\eta\in\RR$, we define
	\[\arg_\eta:\CC^\times\to[\eta,\eta+2\pi)\]
	by $\arg_\eta(z):=\arg z+\pi+\eta$.
\end{definition}
\begin{example}
	We have that $\arg_{-\pi}=\arg$.
\end{example}

\subsection{The Complex Logarithm}
The logarithm is somewhat subtle, so we have to be careful. We take the following definition.
\begin{definition}[Branch of the logarithm]
	Fix $\Omega\subseteq\CC\setminus\{0\}$ an open, connected subset. A \textit{branch of the logarithm} is a continuous function $f:\Omega\to\CC$ such that
	\[\exp(f(z))=z.\] 
\end{definition}
Intuitively, $f$ will ``look like'' an inverse for $\exp$.

Nevertheless, there is a fairly standard choice of branch.
\begin{definition}
	Taking $\Omega:=\CC\setminus\RR_{\le0}$, we define the \textit{principal branch of the logarithm} as $\op{Log}:\Omega\to\CC$ by
	\[z\mapsto\log|z|+i\arg z.\]
\end{definition}
\begin{remark}
	Again, $\log:\RR^+\to\RR$ here is the real logarithm, legal because $z\ne0$ so that $|z|>0$.
\end{remark}
In particular, we are essentially using the construction from back in \autoref{cor:expsurj}.

As some brief geometric commentary, we are calling these ``branches'' our open sets $\Omega$ are typically $\CC$ minus a single line, and the subtlety of why we have to do this is to make the logarithm continuous. For example, in the principal branch, we deleted $\RR_{\le0}$, which has the following image.
\begin{center}
	\begin{asy}
		unitsize(1cm);
		draw((0,0)--(1.5,0), arrow=EndArrow);
		draw((0,-1.5)--(0,1.5), arrow=EndArrow);
		label("$\textrm{Re}$", (1.5,0), E);
		label("$\textrm{Im}$", (0,1.5), N);
		draw((0,0)--(-1.5,0), red);
	\end{asy}
\end{center}
We should probably check that $\op{Log}$ is actually well-formed; namely, it turns out that we had some choice in our construction of $\op{Log}$.
\begin{lemma}
	Fix $z,w\in\CC$ such that $\exp(z)\in\CC\setminus\RR_{\le0}$ and $\op{Log}\exp(z)=w$. Then there is a $k\in\ZZ$ such that $z=w+2\pi ik$.
\end{lemma}
\begin{proof}
	Write $z=x+iy$ so that $\exp z=\exp(x)\exp(iy)$. Now, we know that $\exp(\alpha)=0$ if and only if $\alpha\in2\pi i\ZZ$, so for example, we can write
	\[\exp(yi+2\pi in)=\exp(iy)\]
	for any $n\in\ZZ$. So, by the division algorithm, we choose a $k\in\ZZ$ so that
	\[\widetilde y:=y+2\pi k\]
	has $\widetilde y\in[-\pi,\pi)$. But now, because $\exp(z)\notin\RR_{\ge0}$, we see that we cannot have $\widetilde y=-\pi$ because this would make $\exp(iy)=-1$ and therefore $\exp z=-\exp(x)\in\RR_{\le0}$.

	The point of choosing this $\widetilde y$ is that we still have $\exp(z)=\exp(x)\exp(iy)=\exp(x)=\exp(i\widetilde y)$, but now $\widetilde y\in(-\pi,\pi)$, so we are assured
	\[\arg\exp(z)=\widetilde y.\]
	At this point, we just write out
	\[w=\op{Log}\exp(z)=\op{Log}\exp(x+iy)=\log(|\exp(x)\exp(it)|)+i\arg\exp(z)=x+i\widetilde y.\]
	So now we can write $w=x+iy-2\pi ik$, which is what we wanted.
\end{proof}
Let's return to our discussion of branches. There are a few reasons why we want ``branches'' for $\op{Log}$. Roughly speaking, here is the reasoning.
\begin{itemize}
	\item The function $\exp$ is not injective: it has kernel $\ker\exp=2\pi i\ZZ$. In particular, if we wanted to define $\op{Log}$ on $1\in\CC$, then we need to make a choice among the representatives in $2\pi i\ZZ$.

	\item In order to avoid having to make a choice, we chose $\op{Log}$ to have imaginary part in $[-\pi,\pi)$ always (in fact, $-\pi$ is illegal because $\op{Log}$ doesn't take inputs in $\RR_{\le0}$).
	
	\item But making this choice makes $\op{Log}$ not continuous at values in $\RR_{\le0}$ because (notably!) $\arg z$ is not continuous on $\RR_{\le0}$. In particular, $z\to -1$ from above gives $\arg z\to\pi$ while $z\to-1$ from below gives $\arg z\to-\pi$.

	\item So the point of introducing the branch is to simply throw out the $\RR_{\le0}$ and recover our continuity.
\end{itemize}
In particular, we do indeed have that $\arg$ is continuous where we want it to be.
\begin{lemma}
	The restricted argument function $\arg:\CC\setminus\RR_{\le0}\to[-\pi,\pi)$ is continuous.
\end{lemma}
\begin{proof}
	We defined $\arctan$ piece-wise, so it is not hard to check that $\arg$ is continuous on each quadrant. So the problematic regions are the imaginary axes, for which we refer to Eterovi\'c's notes for the details.
\end{proof}
\begin{corollary}
	The function $\op{Log}:\CC\setminus\RR_{\le0}\to\CC$ is continuous.
\end{corollary}
\begin{proof}
	Well, we write
	\[\op{Log}z=\log|z|+i\arg z,\]
	and we now know that each component is continuous, so the total function is continuous.
\end{proof}
In fact, we get that $\op{Log}$ is holomorphic, essentially inherited from $\exp$.
\begin{lemma} \label{lem:invholo}
	Fix $\Omega_1,\Omega_2\subseteq\CC$ connected and open subsets. Further, suppose we have a continuous function $f:\Omega_1\to\Omega_2$ and a holomorphic function $g:\Omega_2\to\Omega_1$ such that $g(f(z))=z$ for each $z\in\Omega_1$. Then $f$ is holomorphic on $\Omega_1$ with derivative
	\[f'(z)=\frac1{g'(f(z))}.\]
\end{lemma}
\begin{proof}
	Omitted.
\end{proof}
\begin{proposition}
	The function $\op{Log}$ is holomorphic on $\CC\setminus\RR_{\le0}$.
\end{proposition}
\begin{proof}
	We simply apply \autoref{lem:invholo} with $\Omega_1=\CC\setminus\RR_{\le0}$ and $\Omega_2=\CC$ and $f=\op{Log}$ and $g=\exp$. The hypotheses are satisfied.
\end{proof}