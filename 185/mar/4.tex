% !TEX root = ../notes.tex

Good morning everyone. Today's lecture was not recorded.
\begin{itemize}
	\item Homework \#5 will be uploaded today, due next Friday.
	\item The class average on the midterm was a $74$; it might have been a little long. There will probably be something approximately equal to a $6$-point curve.
\end{itemize}
Before continuing, we make some remarks, as a review from real analysis.
\begin{remark}[Nir]
	Today, we will want to pick up some properties of the real logarithm. We define $\log:\RR^+\to\RR$ as the inverse of $\exp:\RR\to\RR^+$, for which we need to know $\exp:\RR\to\RR^+$ is a bijection.
	\begin{itemize}
		\item Note $\exp r>0$ for $r\ge0$ by the power series, and $\exp(r)=1/\exp(-r)>0$ for $r<0$. Thus, $\exp'(r)=\exp(r)>0$ everywhere, so $\exp$ is strictly increasing and therefore injective.
		\item For surjective, by continuity and $\exp(-r)=1/\exp(r)$, we need $\exp r\to\infty$ as $r\to\infty$, for which we note $\exp(1)>\exp(0)=1$ gives $\exp(1)>1+\varepsilon$ for some $\varepsilon>0$, so $\exp(n)>(1+\varepsilon)^n>1+n\varepsilon$.
	\end{itemize}
\end{remark}
\begin{remark}[Nir]
	We will also want to know that $\log:\RR^+\to\RR$ is continuous. Well, $x<y$ if and only if $\exp(x)<\exp(y)$ implies that $x<y$ requires $\log x<\log y$, so $\log$ is strictly increasing. Thus, it suffices to show that $\log$ satisfies the intermediate value property, but $\log$ is surjective (it's the inverse function of a bijection and hence a bijection), so we are done.
\end{remark}

\subsection{Arguments}
Today we talk more about the exponential function. Last time we proved the following.
\polarform*
\noindent As a brief review, we recall that we took $r=|z|$, and we computed $\theta$ in terms of some $\arctan$s. Essentially, this means that we can effectively compute polar form without tears.
\begin{remark}
	The interval $[-\pi,\pi)$ is somewhat arbitrary; we can choose any set of representatives for $\RR/2\pi\ZZ$. To see this, we note that the unique $\theta\in[-\pi,\pi)$ will have a unique representative in any set of representatives for $\RR/2\pi\ZZ$ and vice versa. For example, any half-open interval of length $2\pi$ (such as $[0,2\pi)$) will do the trick. To see this, 
\end{remark}

We can in fact use polar form to talk about the exponential map.
\begin{corollary} \label{cor:expsurj}
	For any $z\in\CC^\times$, there exists some $w\in\CC$ such that $\exp(w)=z$.
\end{corollary}
\begin{proof}
	To start, we know that we can write $z=r\exp(i\theta)$ by \autoref{prop:polarform}. So, using real analysis, we set
	\[w:=\log r+i\theta,\]
	where $\log:\RR^+\to\RR$ is the real logarithm. Thus,
	\[z=r\exp(i\theta)=\exp(\log r)\exp(i\theta)=\exp(\log r+i\theta)=\exp(w),\]
	which is what we wanted.
\end{proof}
Continuing to talk about polar form, we have the following definition.
\begin{definition}[Argument]
	Given a complex number $z\in\CC^\times$, we define the \textit{principal argument} $\arg z\in[-\pi,\pi)$ by writing $z:=|z|\exp(i\theta)$ (using \autoref{prop:polarform}) and taking $\arg z:=\theta$. More generally, for any $\eta\in\RR$, we define
	\[\arg_\eta:\CC^\times\to[\eta,\eta+2\pi)\]
	by $\arg_\eta(z):=\arg z+\pi+\eta$.
\end{definition}
\begin{remark}[Nir] \label{rem:unwindingarg}
	By definition, we see that $|z|\exp(i\arg z)=z$.
\end{remark}
\begin{example}
	We have that $\arg_{-\pi}=\arg$.
\end{example}

\subsection{Branches of the Complex Logarithm}
The logarithm is somewhat subtle, so we have to be careful. We take the following definition.
\begin{definition}[Branch of the logarithm]
	Fix $\Omega\subseteq\CC\setminus\{0\}$ an open, connected subset. A \textit{branch of the logarithm} is a continuous function $f:\Omega\to\CC$ such that
	\[\exp(f(z))=z.\] 
\end{definition}
Intuitively, $f$ will ``look like'' an inverse for $\exp$.

Nevertheless, there is a fairly standard choice of branch.
\begin{defihelper}[\texorpdfstring{$\op{Log}$}{Log}] \nirindex{Log@$\operatorname{Log}$}
	Taking $\Omega:=\CC\setminus\RR_{\le0}$, we define the \textit{principal branch of the logarithm} as $\op{Log}:\Omega\to\CC$ by
	\[z\mapsto\log|z|+i\arg z.\]
\end{defihelper}
\begin{remark} \label{rem:checkconnected}
	It is not too hard to check that $\CC\setminus\RR_{\le0}$ is connected. Indeed, it is path-connected: for any $a+bi\in\CC\setminus\RR_{\le0}$, we define $\gamma:[0,1]\to\CC\setminus\RR_{\le0}$ by
	\[\gamma(t):=(1-t)(a+bi)+t.\]
	Notably, $\Im\gamma(t)=(1-t)b$, so $\gamma(t)\in\RR_{\le0}$ would imply that $\Im\gamma(t)=0$ so that $t=1$ or $b=0$. We cannot have $t=0$ because $\gamma(1)=1$; we cannot have $b=0$ because $b=0$ requires $a>0$, so $\Re\gamma(t)=(1-t)a+t>0$ always.
\end{remark}
\begin{remark} \label{rem:checklogisinverse}
	We can check directly that $\exp\op{Log}z=z$ for $z\in\CC\setminus\RR_{\le0}$. In particular, \autoref{rem:unwindingarg} lets us write
	\[\exp\op{Log}z=\exp(\log|z|+i\arg z)=\exp(\log|z|)\exp(i\arg z)=|z|\exp(i\arg z)=z.\]
\end{remark}
We will check that $\op{Log}$ is actually a continuous later, in \autoref{cor:logcont}.
\begin{remark}
	Again, $\log:\RR^+\to\RR$ here is the real logarithm, which is legal because $z\ne0$ so that $|z|>0$.
\end{remark}
In particular, we are essentially using the construction from back in \autoref{cor:expsurj}.

As some brief geometric commentary, we are calling these ``branches'' our open sets $\Omega$ are typically $\CC$ minus a single line, and the subtlety of why we have to do this is to make the logarithm continuous. For example, in the principal branch, we deleted $\RR_{\le0}$, which has the following image.
\begin{center}
	\begin{asy}
		unitsize(1cm);
		draw((0,0)--(1.5,0), arrow=EndArrow);
		draw((0,-1.5)--(0,1.5), arrow=EndArrow);
		label("$\textrm{Re}$", (1.5,0), E);
		label("$\textrm{Im}$", (0,1.5), N);
		draw((0,0)--(-1.5,0), red);
	\end{asy}
\end{center}
We should probably check that $\op{Log}$ is actually well-formed; namely, it turns out that we had some choice in our construction of $\op{Log}$.
\begin{lemma}
	Fix $z,w\in\CC$ such that $\exp(z)\in\CC\setminus\RR_{\le0}$ and $\op{Log}\exp(z)=w$. Then there is a $k\in\ZZ$ such that $z=w+2\pi ik$.
\end{lemma}
\begin{proof}
	Write $z=x+iy$ so that $\exp z=\exp(x)\exp(iy)$. Now, we know that $\exp(\alpha)=0$ if and only if $\alpha\in2\pi i\ZZ$, so for example, we can write
	\[\exp(yi+2\pi in)=\exp(iy)\]
	for any $n\in\ZZ$. So, by the division algorithm, we choose a $k\in\ZZ$ so that
	\[\widetilde y:=y+2\pi k\]
	has $\widetilde y\in[-\pi,\pi)$. But now, because $\exp(z)\notin\RR_{\ge0}$, we see that we cannot have $\widetilde y=-\pi$ because this would make $\exp(iy)=-1$ and therefore $\exp z=-\exp(x)\in\RR_{\le0}$.

	The point of choosing this $\widetilde y$ is that we still have $\exp(z)=\exp(x)\exp(iy)=\exp(x)=\exp(i\widetilde y)$, but now $\widetilde y\in(-\pi,\pi)$, so we are assured
	\[\arg\exp(z)=\widetilde y.\]
	At this point, we just write out
	\[w=\op{Log}\exp(z)=\op{Log}\exp(x+iy)=\log(|\exp(x)\exp(it)|)+i\arg\exp(z)=x+i\widetilde y.\]
	So now we can write $w=x+iy-2\pi ik$, which is what we wanted.
\end{proof}
Let's return to our discussion of branches. There are a few reasons why we want ``branches'' for $\op{Log}$. Roughly speaking, here is the reasoning.
\begin{itemize}
	\item The function $\exp$ is not injective: it has kernel $\ker\exp=2\pi i\ZZ$. In particular, if we wanted to define $\op{Log}$ on $1\in\CC$, then we need to make a choice among the representatives in $2\pi i\ZZ$.

	\item In order to avoid having to make a choice, we chose $\op{Log}$ to have imaginary part in $[-\pi,\pi)$ always (in fact, $-\pi$ is illegal because $\op{Log}$ doesn't take inputs in $\RR_{\le0}$).
	
	\item But making this choice makes $\op{Log}$ not continuous at values in $\RR_{\le0}$ because (notably!) $\arg z$ is not continuous on $\RR_{\le0}$. In particular, $z\to -1$ from above gives $\arg z\to\pi$ while $z\to-1$ from below gives $\arg z\to-\pi$.

	\item So the point of introducing the branch is to simply throw out the $\RR_{\le0}$ and recover our continuity.
\end{itemize}

\subsection{The Principal Branch}
We now finish the checks that $\op{Log}$ is actually a branch of the logarithm. For this, it remains to check that $\op{Log}$ is continuous; in fact, we will extend and show that $\op{Log}$ is holomorphic. As discussed when we were talking about branches, the issue with extending the continuity of $\op{Log}$ to all of $\CC$ is $\arg$, so we pay $\arg$ some special attention.
\begin{lemma} \label{lem:argcont}
	The restricted argument function $\arg:\CC\setminus\RR_{\le0}\to[-\pi,\pi)$ is continuous.
\end{lemma}
\begin{proof}
	Fix some $z\in\CC\setminus\RR_{\le0}$, and we show that $\arg z$ is continuous at $z$. We do casework because we have to back-track through the definition of $\arg$ and therefore back through \autoref{prop:polarform}
	\begin{itemize}
		\item Suppose $\Re z>0$. Then it suffices to show that $\arg$ is continuous on $B(z,\Re z)\subseteq\{w:\Re w>0\}$. Well, on this region we defined $\arg w$ by
		\[\arg w=\arctan\left(\frac{\Im w}{\Re w}\right).\]
		On $\{w:\Re w>0\}$, we see that $\Re w\ne0$, so the continuity of $\Re$ and $\Im$ promise that $\Im w/\Re w$ is continuous. So because $\arctan$ is continuous, we conclude $\arctan(\Im w/\Re w)$ is continuous at $z$.
		\item Suppose $\Re z<0$ and $\Im z>0$. Then it suffices to show that $\arg$ is continuous on
		\[B(z,\min\{-\Re z,\Im z\})\subseteq\{w:\Re w<0,\Im z>0\}.\]
		Here, we defined $\arg z$ by shifting $\pi-\arctan(\Im w/-\Re w)$ into $[-\pi,\pi)$. But now, $\Im w/\Re w>0$, so $\arctan(\Im w/-\Re w)\in(0,\pi/2)$, so
		\[\arg w=\pi-\arctan(\Im w/-\Re w)\in[-\pi,\pi).\]
		The function $\arctan(\Im w/\Re w)$ is continuous for the same reasons as before, so the total function is continuous at $z$.
		\item Suppose $\Re z<0$ and $\Im z<0$. Then it suffices to show that $\arg$ is continuous on
		\[B(z,\min\{-\Re z,-\Im z\})\subseteq\{w:\Re w<0,\Im z<0\}.\]
		On this region, we defined $\arg z$ by shifting $\pi-\arctan(\Im w/-\Re w)$ into $[-\pi,\pi)$. However, $\Im w/-\Re w<0$, so $\arctan(\Im w/-\Re w)\in(-\pi/2,0)$, so
		\[\arg w=-\pi-\arctan(\Im w/-\Re w)\in[-\pi,\pi).\]
		The function $\arctan(\Im w/\Re w)$ is continuous for the same reasons as before, so the total function is continuous at $z$.
		\item Suppose $\Re z=0$ and $\Im z>0$. Then we defined $\arg z=\frac\pi2$. To check continuity here, we note that it suffices to look in the ball $B(0,\Im z)\subseteq\{w:\Im w>0\}$. Then
		\[\lim_{\substack{w\to z\\\Re w>0,\Im w>0}}\arg w=\lim_{\substack{w\to z\\\Re w>0,\Im w>0}}\arctan\left(\frac{\Im w}{\Re w}\right)=\lim_{x\to\infty}\arctan x=\frac\pi2\]
		while
		\[\lim_{\substack{w\to z\\\Re w<0,\Im w>0}}\arg w=\lim_{\substack{w\to z\\\Re w<0,\Im w>0}}\pi-\arctan\left(\frac{\Im w}{-\Re w}\right)=\pi-\lim_{x\to\infty}\arctan x=\pi-\frac\pi2=\frac\pi2,\]
		which both match $\arg z$. So, fixing some $\varepsilon>0$, we can use the two limits above to find suitable $\delta_1,\delta_2$ in each region, and then we take $\delta:=\min\{\delta_1,\delta_2\}$.
		\item Suppose $\Re z=0$ and $\Im z<0$. We repeat the previous argument. Then we defined $\arg z=-\frac\pi2$. To check continuity here, we note that it suffices to look in the ball $B(0,\Im z)\subseteq\{w:\Im w<0\}$. Then
		\[\lim_{\substack{w\to z\\\Re w>0,\Im w<0}}\arg w=\lim_{\substack{w\to z\\\Re w>0,\Im w>0}}\arctan\left(\frac{\Im w}{\Re w}\right)=\lim_{x\to-\infty}\arctan x=-\frac\pi2\]
		while
		\[\lim_{\substack{w\to z\\\Re w<0,\Im w<0}}\arg w=\lim_{\substack{w\to z\\\Re w<0,\Im w<0}}-\pi-\arctan\left(\frac{\Im w}{-\Re w}\right)=-\pi-\lim_{x\to-\infty}\arctan x=-\pi+\frac\pi2=-\frac\pi2,\]
		which both match with $\arg z$. So, fixing some $\varepsilon>0$, we can use the two limits above to find suitable $\delta_1,\delta_2$ in each region, and then we take $\delta:=\min\{\delta_1,\delta_2\}$.
	\end{itemize}
	The above casework finishes the proof.
\end{proof}
\begin{corollary} \label{cor:logcont}
	The function $\op{Log}:\CC\setminus\RR_{\le0}\to\CC$ is continuous.
\end{corollary}
\begin{proof}
	Well, we write
	\[\op{Log}z=\log|z|+i\arg z,\]
	and we now know that each component is continuous, so the total function is continuous. To be explicit, the function $\log|z|$ is the composite of two continuous functions and is therefore continuous; the function $\arg z$ is continuous by the previous lemma. So we may finish by \autoref{prop:combinecontfuncs}.
\end{proof}
In fact, we get that $\op{Log}$ is holomorphic, essentially inherited from $\exp$.
\begin{lemma} \label{lem:invholo}
	Fix $\Omega_1,\Omega_2\subseteq\CC$ connected and open subsets. Further, suppose we have a continuous function $f:\Omega_1\to\Omega_2$ and a holomorphic function $g:\Omega_2\to\Omega_1$ such that $g(f(z))=z$ and $g'(z)\ne0$ for each $z\in\Omega_1$. Then $f$ is holomorphic on $\Omega_1$ with derivative
	\[f'(z)=\frac1{g'(f(z))}.\]
\end{lemma}
\begin{proof}
	We quickly observe that $f$ is injective: if $z,w\in\Omega_1$ have $f(z)=f(w)$, then $z=g(f(z))=g(f(w))=w$. Now, the trick is that, for distinct $z,w\in\Omega_1$, we may write
	\[\frac{g(f(z))-g(f(w))}{z-w}=\frac{g(f(z))-g(f(w))}{f(z)-f(w)}\cdot\frac{f(z)-f(w)}{z-w}.\]
	In particular, note $z\ne w$ implies $f(z)\ne f(w)$ because $f$ is injective. We see that the left-hand side is merely $1$ because $g\circ f=\id_{\Omega_1}$. In particular, we may write
	\[\lim_{z\to w}\frac{f(z)-f(w)}{z-w}=\lim_{z\to w}\frac1{\frac{g(f(z))-g(f(z))}{f(z)-f(w)}}.\]
	Notably, the denominator here is legal because $z\ne w$ implies $f(z)\ne f(w)$ and $g(f(z))\ne g(f(w))$.
	
	To finish, imagine some sequence $\{z_n\}_{n\in\NN}\subseteq\Omega_1\setminus\{w\}$ such that $z_n\to w$. By continuity of $f$, we see that $f(z_n)\to f(w)$. However, we know that
	\[\lim_{z'\to f(w)}\frac{g(z')-g(f(w))}{z'-f(w)}=g'(f(w)),\]
	so $f(z_n)\to f(w)$ tells us that $\frac{g(f(z_n))-g(f(w))}{f(z_n)-f(w)}\to g'(f(z))$. Because our sequence $\{z_n\}_{n\in\NN}$ was arbitrary, we may conclude
	\[\lim_{z\to w}\frac{f(z)-f(w)}{z-w}=\lim_{z\to w}\frac1{\frac{g(f(z))-g(f(z))}{f(z)-f(w)}}=\frac1{\lim_{z\to w}\frac{g(f(z))-g(f(z))}{f(z)-f(w)}}=\frac1{g'(f(z))}.\]
	This finishes.
\end{proof}
\begin{proposition}
	The function $\op{Log}$ is holomorphic on $\CC\setminus\RR_{\le0}$ with derivative
	\[\frac d{dz}\op{Log}z=\frac1z.\]
\end{proposition}
\begin{proof}
	We simply apply \autoref{lem:invholo} with $\Omega_1=\CC\setminus\RR_{\le0}$ and $\Omega_2=\CC$ and $f=\op{Log}$ and $g=\exp$. We quickly check the hypotheses.
	\begin{itemize}
		\item Note $\Omega_2$ is connected and open, as discussed before.
		\item Note $\Omega_1$ is connected by \autoref{rem:checkconnected} and open because
		\[\Omega_1=\{z\in\CC:\Re z>0\}\cup\{z\in\CC:\Im z\ne\}=\Re^{-1}(\RR_{>0})\cup\Im^{-1}(\RR\setminus\{0\})\]
		is the union of two open sets by the continuity of $\Re$ and $\Im$.
		\item The function $f$ is continuous by \autoref{cor:logcont}.
		\item The function $g$ is holomorphic on $\Omega_2$ by \autoref{lem:expanalytic}.
		\item We have $g(f(z))=z$, essentially by construction; see \autoref{rem:checklogisinverse}.
		\item The function $g'=\exp$ is nonzero everywhere on $\Omega_2$ because $\exp(z)\exp(-z)=1$ for $z\ne0$.
	\end{itemize}
	Now, applying \autoref{lem:invholo}, we see that
	\[\frac d{dz}\op{Log}z=\frac1{\exp'(\op{Log}z)}=\frac1{\exp(\op{Log}z)}=\frac1z,\]
	where we have used the facts that $\exp'=\exp$ by \autoref{lem:expanalytic} and that $\exp(\op{Log}z)=z$ as shown above.
\end{proof}