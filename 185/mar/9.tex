% !TEX root = ../notes.tex

Here we go.

\subsection{Integrals from the Reals}
Today we start talking about path integration.
\begin{definition}[Integrable]
	Fix $\psi:[a,b]\to\CC$ a function (such as a path) with $\psi(t)=u(t)+iv(t)$, where $u,v:\RR\to\RR$. Then $\psi$ is \textit{integrable} over $[a,b]$ if and only if $u$ and $v$ are both integrable over $[a,b]$ (as real functions!). In this case, we define
	\[\int_a^b\psi(t)\,dt:=\int_a^bu(t)\,dt+i\int_a^bv(t)\,dt.\]
\end{definition}
% \begin{remark}[Nir]
% 	There is an inherent appeal to real analysis in the above definition, but it can be removed if we define a measure on $\CC$ which is independent on $\RR$. We will not do this because I don't want to rebuild the somewhat hefty machinery of integrals existing.
% \end{remark}
We have the following sanity checks.
\begin{lemma} \label{lem:integraldistribute}
	Fix $\psi_1,\psi_2:[a,b]\to\CC$ integrable functions with $\alpha_1,\alpha_2\in\CC$. Then
	\[\int_a^b(\alpha_1\psi_1(t)+\alpha_2\psi_2(t))\,dt=\alpha_1\int_a^b\psi_1(t)\,dt+\alpha_2\int_a^b\psi_2(t)\,dt.\]
\end{lemma}
\begin{proof}
	This is by brute force. Let $\alpha_1=x_1+y_1i$ and $\alpha_2=x_2+y_2i$ and $\psi_1(t)=u_1(t)+iv_1(t)$ and $\psi_2(t)=u_2(t)+iv_2(t)$. Then we see that
	\begin{align*}
		\alpha_1\psi_1(t)+\alpha_2\psi_2(t) &= (x_1+y_1i)(u_1(t)+iv_1(t))+(x_2
		+y_2i)(u_2(t)+iv_2(t)) \\
		&= (x_1u_1(t)+x_2u_2(t)-y_1v_1(t)-y_2v_2(t))+i(x_1v_1(t)+x_2v_2(t)+y_1u_1(t)+y_2u_2(t))
	\end{align*}
	has integrable components because $u_1,v_1,u_2,v_2$ are all integrable by hypothesis, and the components are just $\RR$-linear combinations of these. Doing a lot of expansion, the fact that linear combinations of real-valued integrals is legal, we see
	\begin{align*}
		\int_a^b(\alpha_1\psi_1(t)+\alpha_2\psi_2(t))\,dt &= \int_a^b(x_1u_1(t)+x_2u_2(t)-y_1v_1(t)-y_2v_2(t))\,dt \\
		&\qquad+i\int_a^b(x_1v_1(t)+x_2v_2(t)+y_1u_1(t)+y_2u_2(t))\,dt \\
		&= x_1\int_a^bu_1(t)\,dt+x_2\int_a^bu_2(t)\,dt-y_1\int_a^bv_1(t)\,dt-y_2\int_a^bv_2(t)\,dt \\
		&\qquad+ix_1\int_a^bv_1(t)\,dt+ix_2\int_a^bv_2(t)\,dt+iy_1\int_a^bu_1(t)\,dt+iy_2\int_a^bu_2(t)\,dt \\
		&= (x_1+y_1i)\left(\int_a^bu_1(t)\,dt+i\int_a^bv_1(t)\,dt\right)\\
		&\qquad+(x_2+y_2i)\left(\int_a^bu_2(t)\,dt+i\int_a^bv_2(t)\,dt\right) \\
		&= \alpha_1\int_a^b\psi_1(t)\,dt+\alpha_2\int_a^b\psi_2(t)\,dt,
	\end{align*}
	which is what we wanted.
\end{proof}
\begin{lemma}
	Fix $\psi:[a,b]\to\CC$ an integral function. Then
	\[\left|\int_a^b\psi(t)\,dt\right|\le\int_a^b|\psi(t)|\,dt.\]
\end{lemma}
\begin{proof}
	There is approximately one idea to this proof: the point is to create a real-valued integral equal to the norm. Note that $\int_a^b\psi(t)\,dt=0$ means we are done for free. Thus, we can put $\int_a^b\psi(t)\,dt=0$ into polar form as
	\[r\exp(i\theta)=\int_a^b\psi(t)\,dt\]
	for $r>0$. We would like to factor out a $\exp(i\theta)$ from this integral, so we compute (using $\cos(-\theta)=\cos\theta$ and $\sin(-\theta)=-\sin\theta$ from \autoref{lem:negtrig}) that
	\begin{align*}
		\psi(t)\exp(-i\theta) &= (u(t)+iv(t))(\cos(-\theta)+i\sin(-\theta)) \\
		&= (u(t)+iv(t))(\cos\theta-i\sin\theta) \\
		&= \underbrace{(u(t)\cos\theta+v(t)\sin\theta)}_{\alpha(t)}+i\cdot\underbrace{(v(t)\cos\theta-u(t)\sin\theta)}_{\beta(t)}
	\end{align*}
	In particular, $\alpha,\beta:\RR\to\RR$, and so by \autoref{prop:exphom}, we write
	\[\psi(t)=\alpha(t)\exp(i\theta)+i\beta(t)\exp(i\theta)=\exp(i\theta)(\alpha(t)+i\beta(t)).\]
	Thus, we can write
	\[r\exp(i\theta)=\int_a^b\psi(t)\,dt=\int_a^b\exp(i\theta)(\alpha(t)+i\beta(t))\,dt=\exp(i\theta)\int_a^b(\alpha(t)+i\beta(t))\,dt.\]
	Upon cancelling out the $\exp(i\theta)$, we see that
	\[r=\int_a^b(\alpha(t)+i\beta(t))\,dt=\int_a^b\alpha(t)\,dt+i\int_a^b\beta(t)\,dt.\]
	Because $\beta$ is still a real function, that integral evaluates to a real number, but because we have no imaginary part, we conclude
	\[r=\int_a^b\alpha(t)\,dt.\]
	So now we appeal to real analysis. We see
	\[\left|\int_a^b\psi(t)\,dt\right|=r=\int_a^b\alpha(t)\,dt\le\int_a^b|\psi(t)|\,dt,\]
	where $\alpha(t)\le|\psi(t)|$ is because
	\[\alpha(t)=\Re\psi(t)\exp(-i\theta)\le|\psi(t)|\cdot|\exp(-i\theta)|=|\psi(t)|,\]
	where $|\exp(-i\theta)|=1$ by \autoref{cor:expmagker}.
\end{proof}

\subsection{Path Integration}
We have the following definition.
\begin{definition}[Integration] \label{defi:integration}
	Fix $\Omega\subseteq\CC$ an open and connected subset with a $C^1$ path $\gamma:[a,b]\to\Omega$. Now, given a continuous function $f:\Omega\to\CC$, we define the \textit{integral}
	\[\int_\gamma f(z)\,dz:=\int_a^bf(\gamma(t))\gamma'(t)\,dt,\]
	if the integral exists.
\end{definition}
\begin{lemma}
	Under the hypotheses of \autoref{defi:integration}, the integral $\int_\gamma f(z)\,dz$ actually exists.
\end{lemma}
\begin{proof}
	Note that $f$ and $\gamma$ are both continuous, so $f\circ\gamma$ is continuous. Similarly, $\gamma'$ is continuous because $\gamma$ is $C^1$. In total, we can expand
	\[f(\gamma(t))\gamma'(t)\]
	to be a product of continuous functions and therefore must be continuous. It follows that $\Re(f(\gamma(t))\gamma'(t))$ and $\Im(f(\gamma(t))\gamma'(t))$ is also a continuous function, so these components are integrable, so the total integral
	\[\int_a^bf(\gamma(t))\gamma'(t)\,dt\]
	exists.
\end{proof}
Our goal is to show that the integral itself only depends on the equivalence class of $\gamma$.

We can extend this definition a little to piecewise $C^1$ paths.
\begin{definition}[Integration]
	Fix $\Omega\subseteq\CC$ an open and connected subset with a piecewise $C^1$ path $\gamma:[a,b]\to\Omega$, where we have the strictly increasing sequence $\{a_k\}_{k=1}^n$ such that $a=a_0$ and $b=a_n$ and $\gamma|_{a_k,a_{k+1}}$ are $C^1$. Then, given a continuous function $f:\Omega\to\CC$, we define the \textit{integral}
	\[\int_\gamma f(z)\,dz:=\sum_{k=0}^{n-1}\int_{a_k}^{a_{k+1}}f(\gamma(t))\gamma'(t)\,dt.\]
\end{definition}
Note that this integral exists because each component integral exists because $\gamma|_{[a_k,a_{k+1}]}$ is in fact $C^1$.
\begin{example} \label{ex:windaround0}
	Fix $f:\CC\setminus\{0\}\to\CC$ by $f(z):=\frac1z$ and $\gamma:[0,2\pi]\to\CC$ by $\gamma(t)=\exp(it)$ so that $\gamma'(t)=i\exp(it)$. It follows
	\[\int_\gamma f(z)\,dz=\int_0^{2\pi}\left(\frac1{\exp(it)}\cdot i\exp(it)\right)\,dt=\int_0^{2\pi}i\,dt=2\pi it.\]
\end{example}
Now let's show that the integral does not change on reparameterization.
\begin{lemma} \label{lem:reparam}
	Fix $\gamma_1:[a,b]\to\Omega$ and $\gamma_2:[c,d]\to\Omega$ two equivalent piecewise $C^1$ paths. Then, for any continuous function $f:\Omega\to\CC$,
	\[\int_{\gamma_1}f(z)\,dz=\int_{\gamma_2}f(z)\,dz.\]
\end{lemma}
\begin{proof}
	By equivalence, we are promised a function $s:[c,d]\to[a,b]$ which is continuously differentiable, bijection, and has positive derivative everywhere such that $\gamma_2=\gamma_1\circ s$.

	We will in the case where $\gamma_1$ is $C^1$, and the general case will follow. Then we compute
	\[\int_{\gamma_2}f(z)\,dz=\int_c^df(\gamma_1(s(t)))(\gamma_1\circ s)'(t)\,dt.\]
	Applying \autoref{lem:reparameterize}, we see
	\[\int_{\gamma_2}f(z)\,dz=\int_c^df(\gamma_1(s(t)))\gamma_1'(s(t))s'(t)\,dt\]
	By applying a $u$-substitution along $s$ (notably, this is now an integral from a real variable!),\footnote{Technically, we should expand out this integral into real and imaginary parts and then apply the $u$-substitution. Please don't make me do this.} we see
	\[\int_{\gamma_2}f(z)\,dz=\int_a^bf(\gamma_1(s))\gamma_1'(s)\,ds=\int_{\gamma_1}f(z)\,dz,\]
	which is what we wanted.
\end{proof}

\subsection{Path Arithmetic}
Let's blast through some lemmas.
\begin{warn}
	In the following statements, we will merely require our paths to be piecewise $C^1$, but the proofs will deal with the $C^1$ case. This can be amended by partitioning all the intervals to make everything $C^1$, but we will not write this out formally.
\end{warn}
\begin{lemma}
	Fix an open subset $\Omega\subseteq\CC$. If $\gamma:[a,b]\to\Omega$ is a piecewise $C^1$ path and $f,g:\Omega\to\CC$ are continuous functions and $\alpha,\beta\in\CC$, then we have
	\[\int_\gamma(\alpha f(z)+\beta g(z))\,dz=\alpha\int_\gamma f(z)\,dz+\beta\int_\gamma g(z)\,dz.\]
\end{lemma}
\begin{proof}
	We write
	\[\int_\gamma(\alpha f(z)+\beta g(z))\,dz=\int_a^b(\alpha f(\gamma(t))+\beta g(\gamma(t)))\gamma'(t)\,dt\]
	by definition, which expands by \autoref{lem:integraldistribute} into
	\[\alpha\int_a^bf(\gamma(t))\gamma'(t)\,dt+\beta\int_a^bg(\gamma(t))\gamma'(t)\,dt=\alpha\int_\gamma f(z)\,dz+\beta\int_\gamma f(z)\,dz,\]
	which is what we wanted.
\end{proof}
\begin{lemma} \label{lem:opposite}
	Fix an open subset $\Omega\subseteq\CC$. Further, fix $\gamma:[a,b]\to\Omega$ is a piecewise $C^1$ path with $\gamma^-(t):=\gamma(b+a-t)$ the opposite path. Then, for $f:\Omega\to\CC$ a continuous function,
	\[\int_\gamma f(z)\,dz=-\int_{\gamma^-}f(z)\,dz.\]
\end{lemma}
\begin{proof}
	The point is to do a $u$-substitution $t\mapsto b+a-t$. Indeed, we compute
	\[\int_\gamma f(z)\,dz=\int_a^bf(\gamma(t))\gamma'(t)\,dt=\int_b^af(\gamma(b+a-t))\gamma'(b+a-t)\,dt,\]
	where in the last step we have applied our $u$-substitution, legal from real analysis because our integral is from a real variable.\footnote{Again, to be formal, we should expand this into real and imaginary parts and then apply the $u$-substitution, but we won't bother.} However, we see $\gamma(b+a-t)=\gamma^-(b+a-t)$, so the right-hand integral is the desired one; notably, $(\gamma^-)'(t)=-\gamma(b+a-t)$, but this inherited minus sign reverses the order of the time to be from $t=a$ to $t=b$, as it should be.
\end{proof}
\begin{lemma} \label{lem:concatintegral}
	Fix an open subset $\Omega\subseteq\CC$. Further, fix $\gamma:[a,b]\to\Omega$ and $\eta:[c,d]\to\Omega$ to be a piecewise $C^1$ paths such that $\gamma(b)=\eta(c)$. Then, for a continuous function $f:\Omega\to\CC$, we have
	\[\int_{\gamma*\eta}f(z)\,dz=\int_\gamma f(z)\,dz+\int_\eta f(z)\,dz.\]
\end{lemma}
\begin{proof}
	Note that
	\[\int_{\gamma*\eta}f(z)\,dz=\int_a^{b+d-c}f((\gamma*\eta)(t))(\gamma*\eta)'(t)\,dt=\int_a^bf(\gamma(t))\gamma'(t)\,dt+\int_{b}^{b+d-c}f(\eta(t-b+c))\eta'(t-b+c)\,dt.\]
	This is what we want as soon as we apply the change of variables $t-b+c\mapsto t$.
\end{proof}
For our last lemma, we have the following definition.
\begin{definition}[Length]
	Fix a $C^1$ path $\gamma:[a,b]\to\CC$. Then we define the \textit{length of $\gamma$} as
	\[\ell(\gamma):=\int_a^b|\gamma'(t)|\,dt.\]
	More generally, if $\gamma$ is piecewise $C^1$, then we are promised a strictly increasing sequence $\{a_k\}_{k=0}^n$ where $a_0=a$ and $a_n=b$ such that $\gamma|_{[a_k,a_{k+1}]}$ is $C^1$. So we define the \textit{length} as
	\[\ell(\gamma):=\sum_{k=0}^{n-1}\int_{a_k}^{a_{k+1}}|\gamma'(t)\,dt.\]
\end{definition}
Let's use this definition a little.
\begin{proposition} \label{lem:estimation}
	Fix an open and connected subset $\Omega\subseteq\CC$. Then, for $\gamma:[a,b]\to\CC$ a piecewise $C^1$ path and a continuous function $f,g:\Omega\to\CC$, we have the following. Then we have
	\[\left|\int_\gamma f(z)\,dz\right|\le\sup_{t\in[a,b]}\{|f(\gamma(t))|\}\cdot\ell(\gamma).\]
\end{proposition}
\begin{proof}
	We omit the proofs of the first three. By composition, $|f\circ\gamma|$ is a continuous function. In particular, because $[a,b]$ is a compact set, the supremum will actually exist, thus bounding $f$ on $\gamma([a,b])$.  Now, estimating, we see
	\[\left|\int_\gamma f(z)\,dz\right|=\left|\int_a^bf(\gamma(z))\gamma'(z)\,dz\right|\le\int_a^b|f(\gamma(z))|\cdot|\gamma'(z)|\,dz.\]
	By real analysis, we bound this last integral (from real analysis) as
	\[\sup_{t\in[a,b]}\{|f(\gamma(t))|\}\cdot\int_a^b|\gamma'(t)|\,dt,\]
	which is what we wanted.
\end{proof}
We close with a definition, to advertise the fundamental theorem of calculus.
\begin{restatable}[Primitive]{defihelper}{primitivedef} \nirindex{Primitive}
	Fix a nonempty, open subset $\Omega\subseteq\CC$. Then, given two continuous functions $F,f:\Omega\to\CC$, we say that $F$ is a \textit{primitive} on $f$ if and only if $F$ is holomorphic on $\Omega$ and $F'=f$.
\end{restatable}