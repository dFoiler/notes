\documentclass[../notes.tex]{subfiles}
\graphicspath{{\subfix{../figs/}}}

\begin{document}
% !TEX root = ../notes.tex

Good morning everyone. Here are some house-keeping notes.
\begin{itemize}
	\item Homework \#6 is still due Friday.
	\item Class on Friday will be recorded.
\end{itemize}

\subsection{Cauchy Integral Formula Primer}
Today we're start with the Cauchy integral formula. Here's the statement.
\begin{restatable}[Cauchy integral formula]{theorem}{thmcif} \label{thm:cif}
	Fix an open, connected subset $\Omega\subseteq\CC$ and some $z_0\in\Omega$ with $r>0$ such that $\overline{B(z,r)}\subseteq\Omega$. Further, fix the path $\gamma\colon [0,1]\to\Omega$ given by
	\[\gamma(t)\coloneqq z_0+r\exp(2\pi it).\]
	Then, if $f\colon \Omega\to\CC$ is holomorphic, then any $w\in B(z_0,r)$ has
	\[f(w)=\frac1{2\pi i}\oint_\gamma\frac{f(z)}{z-w}\,dz=\op{Ind}_f(\gamma,w).\]
\end{restatable}
Namely, evaluating a holomorphic function $f$ at a point $w$ can be determined only from the value of $f$ on the path $\gamma$!

Here is a nice consequence.
\begin{corollary} \label{cor:holoisana}
	Holomorphic functions are analytic.
\end{corollary}
\begin{proof}
	Use \autoref{thm:cif} to show that any function $f$ differentiable at a point in an open set is equal to $\op{Ind}(w,\gamma)$ locally, from which \autoref{prop:indexanalytic} provides the local power series expansion.
\end{proof}

\subsection{The Cauchy--Goursat Theorem}
To prove \autoref{thm:cif}, we will proceed in steps. Here is one major step.
\begin{theorem}[Cauchy--Goursat] \label{thm:cg}
	Fix an open, connected subset $\Omega\subseteq\CC$ and $T$ a triangle in $\Omega$ (i.e., a closed path defined as the concatenation of three segments). If $f\colon \Omega\to\CC$ is holomorphic, then
	\[\oint_Tf(z)\,dz=0.\]
\end{theorem}
\begin{proof}
	Suppose for the sake of contradiction that the integral is nonzero. Set
	\[I\coloneqq \left|\oint_Tf(z)\,dz\right|\ne0.\]
	Here is the image. The idea is to subdivide our triangle $T\coloneqq T^0$ by midpoints.
	\begin{center}
		\begin{asy}
			unitsize(1.3cm);
			pair A = (0,0);
			pair B = (2,4);
			pair C = (5,0);
			draw(A -- B -- C -- cycle, red);
			dot(A ^^ B ^^ C, red);
			pair D = (A+B)/2;
			pair E = (B+C)/2;
			pair F = (C+A)/2;
			draw(D -- E -- F -- cycle);
			dot(D ^^ E ^^ F);
			label("$T_2^0$", (A+D+F)/3);
			label("$T_1^0$", (B+D+E)/3);
			label("$T_3^0$", (E+D+F)/3);
			label("$T_4^0$", (C+E+F)/3);
			label("$T^0$", (B+E)/2, NE, red);
		\end{asy}
	\end{center}
	By orienting everything properly, we get cancellation along the overlapped regions, so
	\[\oint_Tf(z)\,dz=\sum_{i=1}^4\oint_{T^0_i}f(z)\,dz.\]
	Because the norm here is nonzero, there is an index $i$ such that
	\[\frac I4\le\left|\oint_{T^0_i}f(z)\,dz\right|,\]
	so we set $T^1\coloneqq T^0_i$. Then we can repeat the process inductively to $T^1$; here is the iterated image for $T^1$, working with $T^1=T^0_2$.
	\begin{center}
		\begin{asy}
			unitsize(1.3cm);
			pair A = (0,0);
			pair B = (2,4);
			pair C = (5,0);
			dot(B ^^ C);
			pair D = (A+B)/2;
			pair E = (B+C)/2;
			pair F = (C+A)/2;
			draw(A -- D -- F -- cycle, red);
			draw(D -- B -- C -- F);
			draw(D -- E -- F);
			dot(E); dot(A ^^ D ^^ F, red);
			label("$T_2^1$", (A+D+F)/3/2);
			label("$T_1^1$", (B+D+E)/3/2);
			label("$T_3^1$", (E+D+F)/3/2);
			label("$T_4^1$", (C+E+F)/3/2);
			draw(D/2 -- E/2 -- F/2 -- cycle);
			dot(D/2 ^^ E/2 ^^ F/2);
			label("$T^1$", (D/2+D)/2, NW, red);
		\end{asy}
	\end{center}
	This gives a sequence of nested triangles $T^0,T^1,\ldots$ such that
	\[I_k\coloneqq \left|\oint_{T^k}f(z)\,dz\right|\ge\frac I{4^k}>0.\]
	As another bound, we note that $\ell\left(T^k\right)=2^{-k}\ell(T)$ by essentially geometry of midpoint triangles.

	The idea, now, is to find a point contained in all of our triangles. Let $V^k$ be the region enclosed by $T^k$ (i.e., the convex hull). Thus, we have a descending sequence of nested closed sets
	\[V^1\supseteq V^2\supseteq V^3\supseteq\cdots.\]
	Each of the $V^k$ is closed and bounded and therefore compact, so it follows that the intersection in total is nonempty from the following lemma.
	\begin{lemma}
		Fix a descending chain
		\[V_0\supseteq V_1\supseteq V_2\supseteq\cdots\]
		of nonempty compact subsets of $\CC$. Then the intersection is nonempty.
	\end{lemma}
	\begin{proof}
		Suppose for the sake of contradiction that
		\[\bigcap_{i=0}^\infty V_i=\emp.\]
		Then we can write
		\[V_0=V_0\mathbin{\bigg\backslash}\bigcap_{i=0}^\infty V_i=\bigcup_{i=0}^\infty(V_0\setminus V_i).\]
		In particular, $V_0\setminus V_i=V_0\cap(\CC\setminus V_i)$ is open in $V_0$, so the above provides an open cover of $V_0$. By compactness, this has a finite subcover $\{V_{i_k}\}_{k=1}^n$, so
		\[V_0=\bigcup_{k=1}^n(V_0\setminus V_{i_k})=V_0\mathbin{\bigg\backslash}\bigcap_{k=1}^nV_{i_k},\]
		so we see that
		\[\emp=\bigcap_{k=1}^nV_{i_k}\supseteq\bigcap_{k=1}^nV_{\max_ki_k}=V_{\max_ki_k}\]
		must be empty, which is a contradiction to the construction of the $V_i$.
	\end{proof}
	Now, put $z_0$ in the intersection of our descending chain. Now, $f$ is holomorphic and in particular complex differentiable at $z_0$, so \autoref{prop:cara} promises us a continuous function $h\colon \Omega\to\CC$ continuous at $z_0$ such that
	\[f(z)=f(z_0)+h(z)(z-z_0).\]
	Quickly, we expand
	\[\oint_{T^k}\big(h(z)-f'(z_0)\big)(z-z_0)\,dz=\oint_{T^k}f(z)\,dz-\oint_{T^k}f(z_0)\,dz-\oint_{T^k}f'(z_0)(z-z_0)\,dz.\]
	Now, the constant function $z\mapsto f(z_0)$ has $f(z_0)z$ as a primitive, and $f'(z_0)(z-z_0)$ has $\frac{f'(z_0)}2(z-z_0)^2$ as a primitive, so \autoref{cor:ftconclosed} tells us that the two right-hand integrals vanish. Thus, we can estimate (by \autoref{lem:estimation})
	\begin{align*}
		I_k &= \left|\oint_{T^k}\big(h(z)-f'(z_0)\big)(z-z_0)\,dz\right| \\
		&\le \sup_{z\in V^k}\{|h(z)-f'(z_0)|\cdot|z-z_0|\}\cdot\ell\left(T^k\right) \\
		&\le \sup_{z\in V^k}\{|h(z)-f'(z_0)|\} \cdot\sup_{z\in V^k}\{|z-z_0|\}\cdot\ell\left(T^k\right).
	\end{align*}
	Now, $\sup_{z\in V^k}\{|z-z_0|\}$ is less than the largest length in $V^k$, which we define to be $\op{diam}\left(V^k\right)$. Re-expanding out to $T$, we see $\op{diam}\left(V^k\right)=2^{-k}\op{diam}(V^0)$ and $\ell\left(T^k\right)\le\ell(T^0)$, so we get to bound
	\[I^k\le 4^{-k}\sup_{z\in V^k}\{|h(z)-f'(z_0)|\}\cdot\op{diam}(V)\cdot\ell(T).\]
	We now take a moment to acknowledge that the point $z_0$ is the unique point in the intersection of the $V^k$ because $\op{diam}\left(V^k\right)=2^{-k}\op{diam}\left(V^0\right)$ goes to $0$, zeroing in on $z_0$.

	As such, we now take $z\to z_0$ and then $k\to\infty$. In particular, the continuity of $h$ requires
	\[4^kI_k\le\sup_{z\in V^k}\{|h(z)-f'(z_0)|\}\]
	to go to $0$ as $k\to\infty$. But now, $I\le 4^kI_k$, so $I=0$ is forced, which is our final contradiction.
\end{proof}

\subsection{Not Just Triangles}
Triangles are a nice starting point for \autoref{thm:cg}, but most sets we deal with will not be triangles. Here's a more general definition to help us.
\begin{definition}[Star-like]
	A subset $X\subseteq\CC$ is \textit{star-like} with respect to $z_0\in X$ if and only if each $w\in X$ has a line segment to $z_0$ contained in $X$.
\end{definition}
\begin{example}
	Any convex set $X$ is star-like, for any point in its interior. To be explicit, fix any $z_0\in X$. Then, for any $x\in X$, the line segment connecting $z_0$ and $x$ lives in $X$, thus finishing. Here's the image.
	\begin{center}
		\begin{asy}
			unitsize(0.8cm);
			path p = (-2,0) .. (0,1) .. (1,1) .. (1.3,-2) .. (-1,-1.5) .. cycle;
			fill(p, lightgray);
			draw(p);
			pair z0 = (1,-1);
			pair x = (-1,0.3);

			draw(z0 -- x, red);
			dot("$z_0$", z0, SE);
			dot(x);
		\end{asy}
	\end{center}
\end{example}
\begin{ex}
	The star is star-like with respect to its center. Here is the image.
	\begin{center}
		\begin{asy}
			unitsize(0.6cm);
			path p;
			for(int i = 0; i < 5; ++i)
			{
				p = p -- dir(360/5 * i+198);
				p = p -- sqrt(5)*dir(360/5 * (i+0.5)+198);
			}
			fill(p -- cycle, lightgray);
			draw(p -- cycle);
			draw((0,0) -- 1.5*dir(90-5), red);
			dot("$z_0$", (0,0), S);
			dot(1.5*dir(90-5));
		\end{asy}
	\end{center}
\end{ex}
So here is our associated statement.
\begin{restatable}{theorem}{starlikecg} \label{thm:starlikecg}
	Fix an open, connected, star-like subset $\Omega\subseteq\CC$ with respect to $z_0$. Further, fix a closed, piecewise $C^1$ path $\gamma\colon [0,1]\to\Omega$. Then, if $f\colon \Omega\to\CC$ is holomorphic,
	\[\oint_\gamma f(z)\,dz=0.\]
\end{restatable}
\begin{proof}
	The point is to construct a primitive for $f$ by hand, similar to \autoref{thm:getprimitive}, using \autoref{thm:cg} instead of the listed condition. In particular, note that if we give $f$ a primitive on $\Omega$, then the conclusion will follow by \autoref{cor:ftconclosed}.
	
	We imitate the construction from \autoref{thm:getprimitive}. Indeed, we would like to integrate over a path to create our primitive, so we will use the star-like condition to get the desired path: for $w\in\Omega$, the star-like condition on $\Omega$ promises us the line segment $\gamma_w\colon [0,1]\to\Omega$ from $z_0$ to $w$, defined by
	\[\gamma_w(t)\coloneqq (1-t)z_0+tw.\]
	As such, we set
	\[F(w)\coloneqq \int_{\gamma_w}f(z)\,dz.\]
	We now claim that $F$ is our primitive, for which we have to show $F'(z_1)=f(z_1)$ for any $z_1\in\Omega$.
	
	For psychological reasons, we start by placing $z_1\in\Omega$ inside some open ball $B(z_1,r)\subseteq\Omega$. We would like to control the value of $F$ inside $B(z_1,r)$. Well, for any $w$ in $B(z_1,r)$, we have the following image.
	\begin{center}
		\begin{asy}
			unitsize(1.4cm);

			pair z0 = (0,0);
			pair z1 = (-0.7,2.3);
			pair w = z1 + dir(10);

			draw((-3,-1) .. (-1.3,0) .. (-2,2) .. (1,2) .. (1,0.5) .. (3,1) .. (2,0.3) .. (0,-1) .. cycle, dashed);
			draw(circle(z1, 1.2), dotted);

			draw(z0 -- w, arrow=EndArrow, p=rgb(0.2,0.9,0.2));
			draw(w -- z1, arrow=EndArrow, p=blue);
			draw(z1 -- z0, arrow=EndArrow, p=magenta);

			dot("$z_1$", z1, W);
			dot("$w$", w, ENE);
			dot("$z_0$", z0, S);
			label("$\gamma_w$", (z0+w)/2, ESE, rgb(0.2,0.9,0.2));
			label("$\eta$", (w+z1)/2, NNW, blue);
			label("$\gamma_{z_1}^-$", (0.6*z1+0.4*z0), WSW, magenta);
		\end{asy}
	\end{center}
	In words, we have the triangle
	\[T_{z_0,w,z_1}\coloneqq \gamma_w*\eta*\gamma_{z_1}^-\]
	contained in $\Omega$, where $\eta(t)\coloneqq (1-t)w+tz_1$ is the line segment connecting $w$ to $z_1$. In particular, $\gamma_w(1)=w=\eta(0)$ and $\eta(1)=z_1=\gamma_{z_1}^-(1)$, so we may concatenate these segments into a triangle. Further, this triangle lives in $\Omega$ because $\im\gamma_w\subseteq\Omega$ and $\im\gamma_{z_1}\subseteq\Omega$ by hypothesis on $\Omega$, and $\im\eta\subseteq B(z_1,r)\subseteq\Omega$ because $B(z_1,r)$ is convex.

	Thus, by \autoref{thm:cg}, we get to write
	\begin{align*}
		0 &= \oint_{T_{z_0,w,z_1}}f(z)\,dz \\
		&= \int_{\gamma_w}f(z)\,dz+\int_\eta f(z)\,dz+\int_{\gamma_{z_1}^-}f(z)\,dz \\
		&= \int_{\gamma_w}f(z)\,dz+\int_\eta f(z)\,dz-\int_{\gamma_{z_1}}f(z)\,dz \\
		&= F(w)-F(z_1)+\int_\eta f(z)\,dz.
	\end{align*}
	We are now ready to bound our difference quotient: by \autoref{lem:estimation}, we see
	\begin{align*}
		\left|\frac{F(z_1)-F(w)}{z_1-w}-f(z_1)\right| &= \left|\frac1{z_1-w}\int_\eta f(z)\,dz-f(z_1)\right| \\
		&= \left|\int_0^1\frac{f((1-t)w+tz_1)}{z_1-w}\cdot(z_1-w)\,dt-f(z_1)\right| \\
		&= \left|\int_0^1f((1-t)w+tz_1)-f(z_1)\,dt\right| \\
		&= \sup_{t\in[0,1]}\left\{\big|f((1-t)w+tz_1)-f(z_1)\big|\right\}.
	\end{align*}
	Now, as $w\to z_1$, we see that $f((1-t)w+tz_1)$ will be forced to approach $f(z_1)$ by continuity of $f$, bounded uniformly by $w$, so the quantity approaches $0$. More rigorously, for any $\varepsilon>0$, choose $\delta<r$ so that $|z'-z_1|<\delta$ implies $|f(z')-f(z_1)|<\varepsilon$. Then any $w$ with $|w-z_1|<\delta$ will have $|(1-t)w+tz_1-z_1|<\delta$ as well, so
	\[\sup_{t\in[0,1]}\left\{\big|f((1-t)w+tz_1)-f(z_1)\big|\right\}\le\varepsilon\]
	by taking the supremum everywhere. Sending $\varepsilon\to0$ gives the result.
\end{proof}
\end{document}