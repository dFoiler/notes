% !TEX root = ../notes.tex

Good morning everyone.

\subsection{The Fundamental Theorem of Calculus}
Today we continue talking about path integration. We want to talk about a Fundamental theorem of calculus, so we pick up the following definition.
\primitivedef*
\noindent As promised, we have the following statement.
\begin{theorem}[Fundamental theorem of calculus] \label{thm:ftc}
	Fix an open, connected, nonempty subset $\Omega\subseteq\CC$ with continuous functions $F,f:\Omega\to\CC$ such that $F$ is a primitive of $f$. If $\gamma:[a,b]\to\CC$ is piecewise $C^1$, then we can compute
	\[\int_\gamma f(z)\,dz=F(\gamma(b))-F(\gamma(a)).\]
\end{theorem}
\begin{proof}
	We proceed by force. Write
	\[\int_\gamma f(z)\,dz=\int_a^b f(\gamma(t))\gamma'(t)\,dt=\int_a^b\frac d{dt}F(\gamma(t))\,dt=F(\gamma(b))-F(\gamma(a)),\]
	where the last step is separating out $F\circ\gamma$ into real and imaginary parts and using the Fundamental theorem of calculus from $\RR$.
\end{proof}
\begin{remark}
	Importantly, \autoref{thm:ftc} asserts that the exact path $\gamma$ does not matter to this integral---only its endpoints!
\end{remark}
\begin{corollary}
	Fix an open, connected, nonempty subset $\Omega\subseteq\CC$ with continuous functions $F,f:\Omega\to\CC$ such that $F$ is a primitive of $f$. If $\gamma:[a,b]\to\CC$ is a closed path, then
	\[\int_\gamma f(z)\,dz=0.\]
\end{corollary}
\begin{proof}
	We compute
	\[\int_\gamma f(z)\,dz=F(\gamma(b))-F(\gamma(a))=0,\]
	which is what we wanted.
\end{proof}
\begin{example}
	The function $f(z)=\frac1z$ does not have a primitive on $\CC\setminus\{0\}$, which we can see formally because
	\[\int_\gamma\frac1z\,dz=2\pi i,\]
	for $\gamma:[0,2\pi]$ defined by $\gamma(t):=e^{it}$. Less formally, we would like $f(z)=\frac1z$ to have primitive given by $\op{Log}z$, but $\op{Log}z$ is only defined on $\CC\setminus\RR_{<0}$.
\end{example}
So we would like to determine when a function has a primitive root.

To start our discussion, we have the following technical result.
\begin{lemma}
	Fix a nonempty, open, and connected subset $\Omega\subseteq\CC$. Then any two points in $\CC$ are connected by a piecewise $C^1$ path contained in $\Omega$.
\end{lemma}
\begin{proof}
	The idea is to build path-connected components as in \autoref{prop:getpathconnected}, but this time, we force our paths to be piecewise $C^1$.

	Fix $z\in\Omega$, and let $U\subseteq\Omega$ denote the set of points $w\in \Omega$ such that there exists a piecewise $C^1$ path from $z$ to $w$ contained in $\Omega$. We want to show $U=\Omega$. The key is the following lemma.
	\begin{lemma} \label{lem:convexinconnectedcomp}
		Fix everything as above. If $X\subseteq\Omega$ is convex with $X\cap U\ne\emp$, then $X\subseteq U$.
	\end{lemma}
	\begin{proof}
		Fix any $w\in X\cap U$ so that we want to show $X\subseteq U$. In other words, for any $p\in X$, we want to show $p\in U$. The point, like with \autoref{prop:getpathconnected}, is the following image.
		\begin{center}
			\begin{asy}
				unitsize(1cm);
				dot("$z$", (-5,0), W);
				dot("$w$", (0,0), S);
				dot("$p$", dir(40), dir(40));
				draw((-5,0) .. (-4,-0.3) .. (-1,0.1) .. (0,0));
				draw(circle((0,0), 1.7), dashed);
				draw((0,0) -- dir(40));
			\end{asy}
		\end{center}
		Indeed, because $w\in U$, there exists a piecewise $C^1$ path $\gamma:[a,b]\to\Omega$ from $z$ to $w$. To finish, we set $\eta:[0,1]\to B(w,r)$ to be
		\[\eta(t):=w+t(p-w)\]
		so that $\eta(0)=w$ and $\eta(1)=p$ and $\eta'(t)=p-w$ is a constant function and therefore continuous. Because $X$ is convex, we see that $\eta$ lives in $X$ and therefore in $\Omega$.
		
		Thus, $\eta$ is a $C^1$ path from $w$ to $p$, so $\gamma*\eta$ is a piecewise $C^1$ path from $z$ to $w$ to $p$.\footnote{Technically, we should provide a partition for $\gamma*\eta:[a,b+1]\to\Omega$. Well, partition $[a,b]$ by the partition for $\gamma$, and then take $[b,b+1]$ to be the last portion.} Because $\gamma$ and $\eta$ both output to $\Omega$, we see that $\gamma*\eta$ does as well, so $p\in U$ follows.
	\end{proof}
	We now have the following checks on $U$.
	\begin{itemize}
		\item We see that $U$ is nonempty because $z\in U$. Namely, the path $\gamma_z:[0,1]\to\Omega$ by
		\[\gamma_z(t):=z\]
		has derivative $\gamma_z'(t)=0$, which is constant and hence continuous. Thus, $\gamma_z$ is a $C^1$ path from $z$ to $z$.

		\item We show that $U$ is open in $\Omega$. Indeed, suppose that $w\in U$. We need to find an open neighborhood around $w$ which lives in $U$; well, $\Omega$ is open, so there exists some $r>0$ such that $B(w,r)\subseteq\Omega$.
		
		But now, $B(w,r)$ is convex (by \autoref{ex:convexball}) and intersects $U$ nontrivially at $w\in U$, so $B(w,r)\subseteq U$ by \autoref{lem:convexinconnectedcomp}, so we are done.

		\item We show that $U$ is closed in $\Omega$. For this, we show that $\Omega\setminus U$ is open in $\Omega$. Well, given $w\in\Omega\setminus U$, we need to find an open neighborhood around $w$ contained in $\Omega\setminus U$; because $\Omega$ is open, we certainly may find some $r>0$ such that $B(w,r)\subseteq\Omega$.

		So we claim that $B(w,r)\subseteq\Omega\setminus U$ or equivalently that $B(w,r)\cap U=\emp$. Well, supposing for the sake of contradiction that we can find $p\in B(w,r)\cap U$, we see that $B(w,r)$ is convex by \autoref{ex:convexball} and intersects $U$ nontrivially at $U$, so $B(w,r)\subseteq U$ and $w\in U$ will follow by \autoref{lem:convexinconnectedcomp}. But this contradicts the construction of $w$, so we are done.
	\end{itemize}
	Thus, $U$ is a nonempty closed and open subset of $\Omega$, so because $\Omega$ is connected, we must have $U=\Omega$: we see that $\Omega=U\sqcup(\Omega\setminus U)$ is a disjoint union into open sets, so because $U$ is nonempty, we must have $\Omega\setminus U$ be empty, so $U=\Omega$. But this is exactly what we wanted, so we are done.
\end{proof}
\begin{remark}
	We can strengthen this to having a $C^1$ path, with a little more technical care.
\end{remark}
As such, we have the following.
\begin{theorem}
	Fix a nonempty, open, and connected subset $\Omega\subseteq\CC$. Further, fix a continuous function $f:\Omega\to\CC$ such that
	\[\int_\gamma f(z)\,dz=0\]
	for all closed paths $\gamma$. Then $f$ admits a primitive $F$.
\end{theorem}
\begin{proof}
	We construct our primitive $F$ by hand. Fix $z_0\in\Omega$. Then, for any $z\in\Omega$, we choose some piecewise $C^1$ path $\gamma:[a,b]\to\CC$ with $\gamma(a)=z_0$ and $\gamma(b)=z$ so that we can define
	\[F(z):=\int_\gamma f(z)\,dz.\]
	Of course, it is not immediately obvious that $F$ does not depend on the exact choice of path $\gamma$, but it does not: suppose $\gamma_1:[a,b]\to\CC$ and $\gamma_2:[c,d]\to\CC$ have $\gamma_1(a)=\gamma_2(c)=z_0$ and $\gamma_1(b)=\gamma_2(d)=z$. Now, the key observation is that
	\[\gamma:=\gamma_2*\gamma_1^-,\]
	which we can see is well-defined because $\gamma_1^-(b)=z_0$ and $\gamma_2(c)=z_0$ as well. Further, $\gamma$ is closed because $\gamma_1^-(a)=z$ while $\gamma_2(d)=z$, so we see that
	\[\int_{\gamma_2}f(z)\,dz-\int_{\gamma_1}f(z)\,dz=\int_{\gamma_2}f(z)\,dz+\int_{\gamma_1^-}f(z)\,dz=\int_{\gamma_2*\gamma_1^-}f(z)\,dz=0,\]
	where we have used (in order) \autoref{lem:opposite} and \autoref{lem:concatintegral} and the hypothesis.

	It remains to show that $F$ is holomorphic on $\CC$ with $F'=f$. Well, fix $w\in\Omega$ and $\varepsilon>0$ such that $B(w,\varepsilon)\subseteq\Omega$, which is legal because $\Omega$ is open. Now, we are promised a piecewise $C^1$ path $\gamma:[a,b]\to\Omega$ such that
	\[\gamma(a)=z_0\qquad\text{and}\qquad\gamma(b)=w.\]
	Now, for any $z_1\in B(w,\varepsilon)$, we set $s_1:[0,1]\to B(w,\varepsilon)$ be the line segment connecting $w$ to $z_1$; explicitly, we have
	\[s_1(t)=w+t(z_1-w).\]
	Then, we define $\gamma_1*=\gamma*s_1$, a path from $z_0$ to $w$ to $z_1$. In particular, we find
	\[F(z_1)-F(w)=\int_{\gamma_1}f(z)\,dz-\int_{\gamma_2}f(z)\,dz=\int_{s_1}f(z)\,dz,\]
	where we have used \autoref{lem:concatintegral} again. In particular, for $z_1\ne w$, we find
	\begin{align*}
		\left|\frac{F(z_1)-F(w)}{z_1-w}-f(w)\right| &= \left|\frac1{z_1-w}\int_0^1f(w+t(z_1-w))(z_1-w)\,dt-f(w)\right| \\
		&= \left|\int_0^1f(w+t(z_1-w))\,dt-f(w)\right|.
	\end{align*}
	By \autoref{lem:estimation}, we see
	\[\left|\frac{F(z_1)-F(w)}{z_1-w}-f(w)\right|\le\ell(s_1)\cdot\sup_{t\in[0,1]}\{|f(w+t(z_1-w))-f(w)|\}.\]
	We now need to show that this values goes to $0$ as $\varepsilon>0$ goes to $0$. Well, for some $\varepsilon'>0$, there is a $\delta>0$ such that $|z-w|<\delta$ for which
	\[|z-w|<\delta\implies|f(z)-f(w)|<\varepsilon'.\]
	In particular, we see that $|z_1-w|<\delta'$ implies that
	\[|w+t(z_1-w)-w|=|t(z_1-w)|=|z_1-w|<\delta'\implies|f(w+t(z_1-w))-f(w)|<\varepsilon',\]
	so
	\[\sup_{t\in[0,1]}\{|f(w+t(z_1-w))-f(w)|\}\le\varepsilon'.\]
	Putting this together, we see that
	\[F'(w)=\lim_{z_1\to w}\frac{F(z_1)-F(w)}{z_1-w}=f(w),\]
	so we are done.
\end{proof}
\begin{remark}
	This criterion might appear useless, but we promise that it isn't. It will turn out that we don't really have to check all paths.
\end{remark}

\subsection{Winding Numbers}
We start with a continuous version of the polar form of a complex number. This will be the major technical step in our construction of the winding number.
\begin{restatable}{lemma}{windingnumber} \label{lem:byhandwinding}
	Fix $\gamma:[0,1]\to\CC\setminus\{0\}$ a path. Then there is a continuous function $\theta_\gamma:[0,1]\to\RR$ such that
	\[\gamma(t)=|\gamma(t)|\exp(2\pi i\theta_\gamma(t)).\]
	Furthermore, if we have two such functions $\theta_\gamma$ and $\psi_\gamma$, then $\theta_\gamma-\psi_\gamma$ differ by a constant integer.
\end{restatable}
\begin{proof}
	The point is to choose $\theta_\gamma$ with various branches of $\op{Log}$. We proceed with the following steps.
	\begin{enumerate}
		\item For psychological reasons, we replace $\gamma(t)$ with $\frac{\gamma(t)}{|\gamma(t)|}$ so that $\left|\gamma(t)\right|=1$, and we are looking for a function $\theta:[0,1]\to\RR$ so that
		\[\gamma(t)=\exp(2\pi i\theta(t)).\]
		\item We now temper the speed of $\gamma$ by partitioning its interval $[0,1]$. Because $[0,1]$ is compact $\gamma$ is continuous, $\gamma$ is in fact, uniformly continuous by \autoref{prop:contcompactgetsuniform}. So, for example, we can find some $\delta>0$ such that $s,t\in[0,1]$ has
		\[|s-t|<\delta\implies|\gamma(s)-\gamma(t)|<1.\]
		As such, we set some $n\in\NN$ exceeding $\frac1\delta$ and partition $[0,1]$ by $\{a_k\}_{k=0}^n$ defined by $a_k:=k/n$ (note $a_0=0$ and $a_n=1$) so that $|a_{k+1}-a_k|=\frac1n<\delta$. The point is that our partition $\{a_k\}_{k=0}^n$ forces $\gamma$ to move at a reasonable pace.
		\item We now define $\theta$ piecewise by $\overline\theta_k:[a_k,a_{k+1}]\to\CC$ by
		\[\overline\theta_k(t):=\frac{\arg\gamma(a_k)+\arg(\gamma(t)/\gamma(a_k))}{2\pi}.\]
		Notably, $|\gamma(t)-\gamma(a_k)|<1/2$ implies that $\gamma(t)/\gamma(a_k)$ cannot be in $\RR_{<0}$ because $\gamma$ lives on the unit circle in $\CC$, so $\gamma(t)/\gamma(a_k)\in\RR_{<0}$ would imply that $\gamma(t)/\gamma(a_k)=-1$ and so $|\gamma(t)-\gamma(a_k)|=|1--1|=2$. Thus, $\overline\theta_k$ defined above is in fact a continuous function by \autoref{lem:argcont} because it is the composite of continuous functions.

		Importantly, we can check that $i\arg\gamma(a_k)=\op{Log}\gamma(a_k)$ because $|\gamma(a_k)|$, so we see
		\begin{align*}
			\exp(2\pi i\overline\theta_k(t)) &= \exp(i\arg\gamma(a_k)+i\arg(\gamma(t)/\gamma(a_k))) \\
			&= \exp(\op{Log}\gamma(a_k))\exp(\op{Log}\gamma(t)/\gamma(a_k)) \\
			&= \gamma(a_k)\cdot\gamma(t)/\gamma(a_k)=\gamma(t),
		\end{align*}
		so our $\overline\theta_k$ is chosen correctly.
		\item Next we glue our $\overline\theta_k$ functions. Fixing some $a_m$ for $0<m<n$, we see
		\[\exp(2\pi i\overline\theta_{m-1}(a_m))=\exp(2\pi i\overline\theta_m(a_m)),\]
		so $2\pi i(\overline\theta_m(a_m)-\overline\theta_{m-1}(a_m))\in\ker\exp$ by \autoref{prop:exphom}, so $\overline\theta_m(a_m)=s_m+\overline\theta_{m-1}(a_m)$ for some $s_m\in\ZZ$ by \autoref{prop:expker}. As such, we define
		\[\theta(t):=\overline\theta_m(t)+\sum_{k=1}^ms_k\qquad\text{where}\qquad t\in[a_m,a_{m+1}],\]
		where $t\in[a_m,a_{m+1}]$. Note that this function is well-defined on the endpoints $a_m$ for $0<m<n$ because $\overline\theta_m(a_m)+s_m=\overline\theta_{m-1}(a_m)$. On one hand,
		\[\exp(2\pi i\theta(t))=\exp(2\pi i\overline\theta_m(t))\exp\left(2\pi i\cdot\sum_{k=1}^ms_k\right)=\gamma(t)\cdot1=\gamma(t)\]
		as we showed above (using \autoref{prop:exphom} and \autoref{prop:expker}), so we see that this $\theta$ satisfies the needed equation.

		Lastly, to see that $\theta$ is continuous, we note that $\theta$ is continuous within each interval $(a_m,a_{m+1})$ because this turns into a shifted version of $\overline\theta_m$, which we know is continuous by construction. Then at each endpoint the well-definedness check shows that we can glue these intervals together.
	\end{enumerate}
	Thus, we have exhibited our continuous function $\theta$. It remains to show that this $\theta$ is unique up to shifting by an integer. Well, suppose $\theta$ and $\psi$ both satisfy
	\[\gamma(t)=|\gamma(t)|\exp(2\pi i\theta(t))=|\gamma(t)|\exp(2\pi i\psi(t)).\]
	Using \autoref{prop:exphom}, we see that
	\[\exp\big(2\pi i(\theta(t)-\gamma(t))\big)=1,\]
	so $\theta(t)-\gamma(t)\in\ZZ$ by \autoref{prop:expker}. However, $t\mapsto\theta(t)-\gamma(t)$ is a continuous function from the connected set $[0,1]$ to the set $\ZZ$, but because the image must be connected by \autoref{prop:contconnectedimage}, so the image must be a single point.\footnote{Any connected subset $S\subseteq\ZZ$ containing a point $a\in\ZZ$ cannot be disconnected by the open sets $(a-1/2,a+1/2)\cap\ZZ$ and $(-\infty,a-1/2)\cup(a+1/2,\infty)\cap\ZZ$, so the latter set must be empty, so $S\subseteq(a-1/2,a+1/2)$, so $S=\{a\}$.} Thus,
	\[\theta(t)=\gamma(t)+n\]
	for any $t\in[0,1]$ for some fixed integer $n$.
\end{proof}
This gives us the winding number.
\begin{restatable}[Winding number]{defihelper}{windingnumberdef} \nirindex{Winding number}
	Fix $\gamma:[0,1]\to\CC\setminus\{0\}$ a closed path and $\theta_\gamma:[0,1]\to\RR$ such that
	\[\gamma(t)=|\gamma(t)|\exp(2\pi i\theta_\gamma(t)).\]
	Then we define the \textit{winding number of $\gamma$ around $0$} by
	\[\op{Ind}(\gamma,0):=\frac{\theta_\gamma(1)-\theta_\gamma(0)}{2\pi i}.\]
	More generally, for a given path $\gamma:[0,1]\to\CC\setminus\{z_0\}$, the winding number of $\gamma$ around $z_0$ is
	\[\op{Ind}(\gamma,z_1):=\op{Ind}(\gamma-z_1,0).\]
\end{restatable}
\begin{remark}
	Because $\gamma$ is closed, we see that $\exp(\theta_\gamma(0))=\exp(\theta_\gamma(1))$, so $\theta_\gamma(0)\equiv\theta_\gamma(1)\pmod{2\pi i}$, so the winding number is in fact an integer. In fact, the winding number is independent of the chosen $\theta_\gamma$ because any two such functions differ by a constant integer, by \autoref{lem:byhandwinding}.
\end{remark}
Pictorially, the winding number of $\gamma:[0,1]\to\CC\setminus\{0\}$ is intended to be the number of times $\gamma$ ``winds'' around $0$. We have the following examples, which we will not justify formally.
\begin{example}
	The following path has winding number $0$.
	\begin{center}
		\begin{asy}
			unitsize(1cm);
			dot("$0$", (0,0), S);
			draw(circle((1,0),0.7), arrow=EndArrow);
		\end{asy}
	\end{center}
\end{example}
\begin{example}
	The following path has winding number $1$.
	\begin{center}
		\begin{asy}
			unitsize(1cm);
			dot("$0$", (0,0), S);
			draw((1,0) .. (0,1) .. (-1,0) .. (0,-1) .. cycle, arrow=EndArrow);
		\end{asy}
	\end{center}
\end{example}
\begin{example}
	The following path has winding number $-1$.
	\begin{center}
		\begin{asy}
			unitsize(1cm);
			dot("$0$", (0,0), S);
			draw((1,0) .. (0,-1) .. (-1,0) .. (0,1) .. cycle, arrow=EndArrow);
		\end{asy}
	\end{center}
\end{example}
\begin{example}
	The following path has winding number $2$.
	\begin{center}
		\begin{asy}
			unitsize(1cm);
			dot("$0$", (0,0), S);
			draw((1,0) .. (0,1) .. (-1,0) .. (0,-1) .. (1.5,0) .. (0,1.5) .. (-1.5,0) .. (0,-1.5) .. cycle, arrow=EndArrow);
		\end{asy}
	\end{center}
\end{example}