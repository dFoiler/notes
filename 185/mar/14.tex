% !TEX root = ../notes.tex

Good morning everyone. It's $\pi$ day. Here are some house-keeping notes.
\begin{itemize}
	\item Homework \#6 is due on Friday, at midnight.
	\item Class on Friday will be recorded.
	\item Next week is spring break!
\end{itemize}

\subsection{Winding Numbers by Integrals}
Today we finish our discussion of path integration; soon we will transition over to the Cauchy integral formula. We recall the following lemma.
\windingnumber*
\noindent We quickly recall that the function $\theta_\gamma$ in the statement is, roughly speaking, the composition of $\gamma(t)$ (normalized) with a suitably chosen branch of the logarithm.

This gave us the following definition.
\windingnumberdef*
\begin{remark}
	It is advisable to not really care about the definition given in \autoref{lem:byhandwinding} because we are about to give a more computational view of it. To be more explicit, \autoref{lem:byhandwinding} is bad for computation.
\end{remark}
Here's a better way to compute the winding number.
\begin{lemma} \label{lem:betterwinding}
	Fix $\gamma:[0,1]\to\CC$ a closed, piecewise $C^1$ path. Further, fix $z_0\in\CC\setminus\im\gamma$. Then
	\[\op{Ind}(\gamma,z_0)=\frac1{2\pi i}\oint_\gamma\frac1{z-z_0}\,dz.\]
\end{lemma}
Here are some example.
\begin{example}
	Fix $\gamma:[0,1]\to\CC$ by $\gamma(t):=\exp(2\pi it)$ to be the unit circle. We can compute that
	\[\oint_\gamma\frac1z\,dz=\int_0^1\frac{2\pi i\exp(2\pi it)}{2\exp(2\pi it)}\,dt=2\pi i,\]
	so the winding number of $\gamma$ around $z_0=0$ is $1$.
\end{example}
\begin{example}
	Fix $\gamma:[0,1]\to\CC$ by $\gamma(t):=\exp(-2\pi it)$ to be the clockwise unit circle. We can then compute that
	\[\oint_\gamma\frac1z\,dz=\int_0^1\frac{-2\pi i\exp(-2\pi it)}{\exp(-2\pi it)}\,dt=-2\pi i,\]
	so the winding number of $\gamma$ around $z_0=0$ is $-1$.
\end{example}
And here is our proof.
\begin{proof}[Proof of \autoref{lem:betterwinding}]
	This proof is a little slick. The point is to write $a(t)$ in terms of a branch of the logarithm. As from \autoref{lem:byhandwinding}, we have that
	\[\gamma(t)=z_0+r(t)\exp(2\pi ia(t))\]
	where $r(t):=|\gamma(t)-z_0|$ and $a(t):[0,1]\to\RR$ is a continuous function.

	Fix some $t_0\in[0,1]$. The idea is to show that everything in sight is differentiable. Because $\gamma(t_0)\ne z_0$ and $\gamma$ is continuous, we can find some $\delta>0$ and a suitable branch of the logarithm $\op{Log}$ so that $\op{Log}(\gamma(t)-z_0)$ is defined on all $B(\gamma(t_0),\delta)$. Here, we can compute
	\[r(t)=\exp\left(\frac12\log\left((\gamma(t)-z_0)(\overline\gamma(t)-\overline{z_0})\right)\right).\]
	Notably, this is the real-valued logarithm, so all of our standard logarithm rules apply (i.e., we are allowed to move the $\frac12$ outside without concern). Thus, we see that $r$ is a composite of continuous functions and therefore continuous here. Now, by the continuity of $r(t)$, we can build a branch of the logarithm so that $\op{Log}\left(\frac{\gamma(t)-z_0}{r(t)}\right)$ is defined near $\gamma(t_0)$. Because
	\[\exp\left(\op{Log}\left(\frac{\gamma(t)-z_0}{r(t)}\right)\right)=\exp(2\pi ia(t)),\]
	we conclude from \autoref{prop:expker} that
	\[a(t)-\frac1{2\pi i}\cdot\frac{\gamma(t)-z_0}{r(t)}\]
	is always an integer for each $t\in[0,1]$. But because $[0,1]$ is connected and $\ZZ$ is discrete, this must be constant, so there is a fixed integer $n\in\ZZ$ such that
	\[a(t)=\frac1{2\pi i}\cdot\frac{\gamma(t)-z_0}{r(t)}+n.\]
	Now we integrate. We see that
	\[\oint_\gamma\frac1{z-z_0}\,dz=\int_0^1\frac{\gamma'(t)}{\gamma(t)-z_0}\,dz=\int_0^1\frac{\big(r'(t)+2\pi ir(t)a'(t)\big)\exp(2\pi it)}{r(t)\exp(2\pi it)}\,dt.\]
	At this point, we notice that the exponential functions cancel, so we have that
	\[\oint_\gamma\frac1{z-z_0}\,dz=\int_0^1\frac{r'(t)+2\pi ir(t)a'(t)}{r(t)}\,dt=\int_0^1\frac{r'(t)}{r(t)}\,dt+2\pi i\int_0^1a'(t)\,dt.\]
	Now these integrals are completely real-valued. So we compute
	\[\int_0^1\frac{r'(t)}{r(t)}\,dt=\log r(1)-\log r(0)=0\]
	because $\gamma(1)=\gamma(0)$ (it's a closed path). Thus, we are left with
	\[\oint_\gamma\frac1{z-z_0}\,dz=2\pi i\int_0^1a'(t)\,dt=2\pi i(a(1)-a(0)),\]
	so the conclusion follows.
\end{proof}
Here are some corollaries.
\begin{corollary}
	Fix a closed piecewise $C^1$ path $\gamma:[0,1]\to\CC$ and $z_0\in\CC\setminus\im\gamma$. Then $\op{Ind}(\gamma^-,z_0)=-\op{Int}(\gamma,z_0)$.
\end{corollary}
\begin{proof}
	Applying \autoref{lem:opposite} to \autoref{lem:betterwinding}, we get
	\[\op{Ind}(\gamma^-,z_0)=\oint_{\gamma^-}\frac1z\,dz=-\oint_\gamma\frac1z\,dz=-\op{Ind}(\gamma,z_0),\]
	which is what we wanted.
\end{proof}
\begin{corollary}
	Fix closed piecewise $C^1$ paths $\gamma,\eta:[0,1]\to\CC$ such that $\gamma(1)=\eta(0)$, and pick up some $z_0\in\CC\setminus(\im\gamma\cup\im\eta)$. Then $\op{Ind}(\gamma*\eta,z_0)=\op{Int}(\gamma,z_0)+\op{Ind}(\eta,z_0)$.
\end{corollary}
\begin{proof}
	Applying \autoref{lem:concatintegral} to \autoref{lem:betterwinding}, we get
	\[\op{Ind}(\gamma*\eta,z_0)=\oint_{\gamma*\eta}\frac1z\,dz=\oint_\gamma\frac1z\,dz+\oint_\eta\frac1z\,dz=\op{Ind}(\gamma,z_0)+\op{Ind}(\eta,z_0),\]
	which is what we wanted.
\end{proof}

\subsection{More General Indices}
We will want a slightly more general version of the winding number for where we're going.
\begin{definition}[Index]
	Fix an open and connected subset $\Omega\subseteq\CC$ and a closed piecewise $C^1$ path $\gamma:[a,b]\to\Omega$. Given a function $f:\Omega\to\CC$ which is continuous on $\im\gamma$, we define
	\[\op{Ind}_f(\gamma,w):=\frac1{2\pi i}\oint_\gamma\frac{f(z)}{z-w}\,dz\]
\end{definition}
\begin{remark}
	This integral is equal to
	\[\int_0^1\frac{f(\gamma(t))}{\gamma(t)-w}\cdot\gamma'(t)\,dt,\]
	which is now more obviously well-defined. In particular, the inner function is piecewise continuous, so its real and imaginary parts are integrable.
\end{remark}
\begin{proposition} \label{prop:indexanalytic}
	Fix an open and connected subset $\Omega\subseteq\CC$ and a closed, piecewise $C^1$ path $\gamma:[a,b]\to\Omega$. Given a function $f:\Omega\to\CC$ a function continuous on $\im\gamma$, the function $\op{Ind}_f(\gamma,-)$ is analytic (!) at $w$ with power series around $z_0$ given by
	\[\op{Ind}_f(\gamma,z_0)=\sum_{n=0}^\infty\left(\frac1{2\pi i}\oint_\gamma\frac{f(z)}{(z-z_0)^{n+1}}\,dz\right)(z-z_0)^n\]
\end{proposition}
This is our first major step towards showing that all holomorphic functions are analytic: here we have been granted a way to conjure some magical power series.

For \autoref{prop:indexanalytic}, we will need the following lemma.
\begin{lemma} \label{lem:exchangeintegrallimit}
	Fix $\Omega$ and $\gamma$ as in \autoref{prop:indexanalytic}. Given a sequence of continuous functions $\{f_k\}_{k=1}^\infty$ which uniformly converge to $f$ on $\im\gamma$, then $f$ is integrable and
	\[\lim_{n\to\infty}\oint_\gamma f(z)\,dz=\oint_\gamma f(z)\,dz.\]
\end{lemma}
\begin{proof}
	Roughly speaking, the point is to look at
	\[\oint_\gamma(f-f_n)(z)\,dz\]
	and use \autoref{lem:estimation} and uniform convergence to show that this vanishes as $n\to\infty$.

	Let's be a little more precise. We need to show that
	\[\lim_{n\to\infty}\oint_\gamma\big(f(z)-f_n(z)\big)\,dz=0.\]
	By \autoref{lem:estimation}, we can say
	\[\left|\oint_\gamma\big(f(z)-f_n(z)\big)\,dz\right|\le\sup_{t\in[a,b]}\{|f(\gamma(t))-f_n(\gamma(t))|\}\cdot\ell(\gamma).\]
	If $\ell(\gamma)=0$, there is nothing to say. Otherwise, we set any $\varepsilon>0$ and note that uniform convergence of $f_n\to f$ promises us some $N$ for which $n>N$ has
	\[|f(z)-f_n(z)|<\frac{\varepsilon}{2\ell(\gamma)}\]
	for all $z\in\im\gamma$. In particular, we find that
	\[\left|\oint_\gamma\big(f(z)-f_n(z)\big)\,dz\right|\le\sup_{t\in[a,b]}\{|f(\gamma(t))-f_n(\gamma(t))|\}\cdot\ell(\gamma)\le\frac\varepsilon{2\ell(\gamma)}\cdot\ell(\gamma)<\varepsilon,\]
	so we have established the needed limit.
\end{proof}
And here is our proof.
\begin{proof}[Proof of \autoref{prop:indexanalytic}]
	Fix $z_0\in\CC\setminus\im\gamma$. For psychological reasons, we translate our path and $\Omega$ so that $z_0=0$. Now, $[a,b]$ is compact, so $\gamma([a,b])$ is compact and therefore closed, so $\CC\setminus\im\gamma$ is open, so we can find an $r>0$ such that $B(0,2r)\subseteq\CC\setminus\im\gamma$.

	Now, for any $w\in B(0,r)$ and $z\in\im\gamma$, we have $|w|<r$ and $|z|2r$, so we have $|w/z|<1/2$. Continuing with our estimation, we set
	\[M:=\sup_{t\in[a,b]}\{|f(\gamma(t))|\}\]
	which exists because $[a,b]$ is compact (namely, real-valued continuous functions always maximums on compact sets). Thus, we bound
	\[\left|\frac{f(z)w^n}{z^{n+1}}\right|=\left|\frac{f(z)}z\right|\cdot\left|\frac wz\right|^n\le\frac M{2r}\cdot\left(\frac12\right)^n.\tag{$*$}\label{eq:weierboundindex}\]
	Noting that $|w/z|<1$, it follows that
	\[\frac{f(z)}{z-w}=\frac{f(z)}z\cdot\frac1{1-w/z}=\sum_{n=0}^\infty\frac{f(z)}z\left(\frac wz\right)^n=\sum_{n=0}^\infty\frac{f(z)w^n}{z^{n+1}},\]
	by how we sum geometric series. In fact, by the Weierstrass $M$-test, this sum converges uniformly: by \autoref{eq:weierboundindex}, we can write
	\[\sum_{n=0}^\infty\left|\frac{f(z)w^n}{z^{n+1}}\right|\le\frac M{2r}\sum_{n=0}^\infty\left(\frac12\right)^n<\infty.\]
	Thus, \autoref{lem:exchangeintegrallimit} tells us that
	\[\frac1{2\pi i}\oint_\gamma\frac{f(z)}{z-w}\,dz=\frac1{2\pi i}\oint_\gamma\Bigg(\sum_{n=0}^\infty\frac{w^nf(z)}{z^{n+1}}\Bigg)\,dz=\sum_{n=0}^\infty\left(\frac1{2\pi i}\oint_\gamma\frac{f(z)}{z^{n+1}}\,dz\right)w^n,\]
	which gives the desired power series expansion.
\end{proof}