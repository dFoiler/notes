% !TEX root = ../notes.tex

Good morning everyone.

\subsection{Smooth Paths}
Today we are going to build some theory of paths. We recall the definition.
\pathdef*
\noindent Now that we have access to some differentiation, we can talk about the smoothness of our paths.
\begin{definition}[Differentiable for paths]
	Fix $[a,b]\subseteq\RR$. A path $\gamma:[a,b]\to\CC$ is \textit{differentiable} if and only if the limit
	\[\lim_{t\to t_0}\frac{\gamma(t)-\gamma(t_0)}{t-t_0}\]
	exists. If the limit exists, we set it equal to $\gamma'(t)$ and call it the \textit{derivative}. Further, $\gamma$ is \textit{differentiable} if and only if $\gamma$ is differentiable at all points $t\in[a,b]$.
\end{definition}
\begin{remark}
	When computing $\gamma'(a)$ and $\gamma'(b)$, the above limit is one-sided.
\end{remark}
There are still going to be some pathological paths that are differentiable, so we add more smoothness conditions.
\begin{defihelper}[\texorpdfstring{$C^1$}{C1}] \nirindex{C1@$C^1$}
	Fix $[a,b]\subseteq\RR$. A path $\gamma:[a,b]\to\CC$ is \textit{$C^1$} or \textit{smooth} if and only if $\gamma$ is differentiable and $\gamma'$ is continuous.
\end{defihelper}
This is perhaps a little too strong, but it is the correct notion. Here is a slightly weaker version.
\begin{definition}[Piecewise \texorpdfstring{$C^1$}{C1}]
	Fix $[a,b]\subseteq\RR$. A path $\gamma:[a,b]\to\CC$ is \textit{piecewise $C^1$} if and only if there exists a sequence $\{a_k\}_{k=0}^n$ with $a_0=a$ and $a_n=b$ such that
	\[\gamma|_{[a_k,a_{k+1}]}\]
	is $C^1$ for each $0\le k<n$.
\end{definition}
The point is that we are going to want to glue $C^1$ paths together in the future, and the resulting path need not be $C^1$.

This is a math class, so we should probably prove something today, so have some lemmas.
\begin{lemma}
	Fix $s:[a,b]\to[c,d]$ a function differentiable at $t_0\in[a,b]$. Further, if $\gamma:[c,d]\to\CC$ is differentiable at $\gamma(t_0)$, then $\gamma\circ s$ is differentiable at $t_0$ with
	\[(\gamma\circ s)'(t_0)=\gamma'(s(t_0))s'(t_0).\]
\end{lemma}
\begin{proof}
	As usual with compositions, we consider the difference quotient $v:[c,d]\to\CC$ defined as
	\[v(x):=\begin{cases}
		\frac{\gamma(x)-\gamma(s(t_0))}{t-s_0}-\gamma'(s(t_0)) & t\ne s(t_0), \\
		0 & x=s(t_0).
	\end{cases}\]
	Now, by definition of the differentiability of $\gamma$ at $s(t_0)$, we know that $v(x)\to0$ as $x\to s(t_0)$. Rearranging, we see that
	\[\gamma(x)-\gamma(s(t_0))=(x-s(t_0))\cdot\big(\gamma'(s(t_0))+v(x)\big)\]
	for all $x\in [c,d]$, so plugging in $s(t)\in[c,d]$, we see that
	\[\gamma(s(t))-\gamma(s(t_0))=(s(t)-s(t_0))\cdot\big(\gamma'(s(t_0))+v(s(t))\big).\]
	So now we rearrange backwards to see
	\[\frac{\gamma(s(t))-\gamma(s(t_0))}{t-t_0}=\frac{s(t)-s(t_0)}{t-t_0}\cdot\big(\gamma'(s(t_0))+v(s(t))\big).\]
	Upon taking the limit as $t\to t_0$, the differentiability of $s$ at $t_0$ assures us that
	\[\lim_{t\to t_0}\frac{\gamma(s(t))-\gamma(s(t_0))}{t-t_0}=\left(\lim_{t\to t_0}\frac{s(t)-s(t_0)}{t-t_0}\right)\left(\lim_{t\to t_0}\gamma'(s(t_0))+v(s(t))\right)=s'(t_0)\gamma'(s(t_0)).\]
	Notably, we are using the fact that $v\circ s$ is continuous at $t_0$ because $s$ is continuous and $v$ is continuous at $s(t_0)$.
\end{proof}
\begin{lemma} \label{lem:conjugatepath}
	Fix $\gamma:[a,b]\to\CC$ a path differentiable at $c\in(a,b)$. Then $\overline{\gamma}$ is differentiable on $\CC$ with derivative $\overline{\gamma'(c)}$.
\end{lemma}
\begin{proof}
	Note that the function $z\mapsto\overline z$ is continuous, so we compute
	\[\lim_{t\to c}\frac{\overline\gamma(t)-\overline\gamma(c)}{t-c}=\lim_{t\to c}\overline{\left(\frac{\gamma(t)-\gamma(c)}{t-c}\right)}=\overline{\lim_{t\to c}\frac{\gamma(t)-\gamma(c)}{t-c}}=\overline{\gamma'(c)},\]
	which is what we wanted.
\end{proof}
\begin{remark}
	\autoref{lem:conjugatepath} might seem surprising because conjugation itself is usually not complex differentiable. However, this is okay because we are only really taking limits in $\RR$, so the extra dimension of $\CC$ does not impede us.
\end{remark}
As a side remark, we note the following: we can approximate any path reasonably well.
\begin{theorem}
	For any path $\gamma:[a,b]\to\CC$, there exists a sequence of piecewise $C^1$ paths $\{\gamma_k\}_{k\in\NN}$ such that $\gamma_k\to\gamma$ uniformly.
\end{theorem}
\begin{proof}
	The main point is to use the Stone--Weierstrass theorem. We will not prove this in class, for it would sidetrack us somewhat significantly.
\end{proof}
The reason why we bring up the above result is that we can, roughly speaking, understand paths (and integration on paths) by reducing them to piecewise $C^1$ paths and then studying the $C^1$ paths individually.

\subsection{Reparameterization}
We are going to want to adjust our ``speed'' along a path, for which we have the following definition.
\begin{definition}[Reparameterization]
	Fix $s:[a,b]\to[c,d]$ a continuously differentiable function with $s(a)=c$ and $s(b)=d$. Then, given a path $\gamma:[c,d]\to\CC$ be a $C^1$ path. Then the path
	\[\widetilde\gamma:=\gamma\circ s:[a,b]\to\CC\]
	is again a $C^1$ path. We call $\widetilde{\gamma}$ a \textit{reparameterization} of $\gamma$.
\end{definition}
\begin{remark}
	We can also check that, in the context of the above definition, $\im\gamma=\im\widetilde{\gamma}$. Indeed, it suffices to show that $\varphi$ is surjective, for which we note $\gamma(a)=c$ and $\gamma(b)=d$ gives us surjectivity onto $[c,d]$ by the Intermediate value theorem.
\end{remark}
Reparameterization allows us a notion of equivalence.
\begin{definition}[Equivalent]
	Two paths $\gamma_1:[a,b]\to\CC$ and $\gamma_2:[c,d]\to\CC$ are \textit{equivalent} if and only if there is a continuously differentiable, bijective function $s:[a,b]\to[c,d]$ such that $s'>0$ and $\gamma_1=\gamma_2\circ s$. We denote this by $\gamma_1\sim_e\gamma_2$
\end{definition}
One can check that $\sim_e$ defined above is an equivalence relation.
\begin{lemma}
	The relation $\sim_e$ defined on paths is an equivalence relation.
\end{lemma}
\begin{proof}
	We have the following checks.
	\begin{itemize}
		\item Reflexive: given a path $\gamma:[a,b]\to\CC$, we show $\gamma\sim_e\gamma$. Indeed, the function $s:[a,b]\to[a,b]$ defined by $s(x):=x$ is bijective and continuously differentiable (with constant derivative $1$) and $\gamma(t)=\gamma(s(t))$. So $s$ witnesses $\gamma\sim_e\gamma$.
		\item Symmetric: fix paths $\gamma_1:[a,b]\to\CC$ and $\gamma_2:[c,d]\to\CC$ with $\gamma_1\sim_e\gamma_2$; i.e., we are given a bijective and continuously differentiable $s:[a,b]\to[c,d]$ such that $\gamma_1=\gamma_2\circ s$.

		Because $s$ is bijective, it has an inverse function $r:[c,d]\to[a,b]$, which we can check is also continuously differentiable by real analysis; the idea is to copy the proof of \autoref{lem:invholo} to show that $r'(t)=\frac1{s'(r(t))}$ which is continuously differentiable (using the condition $s'>0$).

		Thus, $\gamma_1=\gamma_2\circ s$ implies $\gamma_2=\gamma_1\circ r$, so $\gamma_2\sim_e\gamma_1$.

		\item Transitive: fix paths $\gamma_1:[a,b]\to\CC$ and $\gamma_2:[c,d]\to\CC$ and $\gamma_3:[e,f]\to\CC$ such that $\gamma_1\sim_e\gamma_2$ and $\gamma_2\sim_e\gamma_3$; i.e., we have bijective, continuously differentiable functions $r:[a,b]\to[c,d]$ and $s:[c,d]\to[e,f]$ such that $r',s'>0$ and
		\[\gamma_1=\gamma_2\circ s\qquad\text{and}\qquad\gamma_2=\gamma_3\circ r.\]
		But then we see $r\circ s$ is bijective and continuously differentiable (by the chain rule) with $(r\circ s)'>0$, so $\gamma_1=\gamma_3\circ(r\circ s)$ witnesses $\gamma_1\sim_e\gamma_3$.
	\end{itemize}
	The above sketchy checks finish the proof.
\end{proof}
\begin{definition}[Oriented curve]
	An equivalence class $[\gamma]_e$ of paths is an \textit{oriented curve}.
\end{definition}
Here are two basic curves.
\begin{example}
	Given $z_0\in\CC$, the set of constant paths $\gamma:[a,b]\to\CC$ by $\gamma\equiv z_0$ is an oriented curve.
\end{example}
\begin{example}
	Given $\alpha,\beta\in\CC$, we define the line segment $\gamma:[0,1]\to\CC$ by
	\[\gamma(t):=(1-t)\alpha+t\beta,\]
	which we can check is differentiable with constant derivative ($-\alpha+\beta$) and is therefore continuously differentiable.
\end{example}
There might not be a nice, canonical way to define a curve. Here are two circles.
\begin{example}
	Fix $z_0\in\CC$ and $r\in\RR_{>0}$. Then we define the circle of radius $r$ centered at $z_0$ by the path $\gamma:[0,2\pi]\to\CC$ by the path
	\[\gamma(t):=z_0+r\exp(it).\]
	Here is the image.
	\begin{center}
		\begin{asy}
			unitsize(1.5cm);
			pair z0=dir(20);
			real r=1;
			for(int i = 0+0; i < 6; ++i)
				draw(arc(z0,r,60*i,60*i+60), EndArrow);
			dot("$z_0$", z0, -z0);
			draw(z0--z0+z0, dotted);
			label("$r$", z0+z0/2, dir(20+90));
		\end{asy}
	\end{center}
\end{example}
\begin{example}
	Fix $z_0\in\CC$ and $r\in\RR_{>0}$. Then we define the circle of radius $r$ centered at $z_0$ by the path $\gamma:[0,2\pi]\to\CC$ by the path
	\[\gamma_0(t):=z_0+r\exp(-it).\]
	Here is the image.
	\begin{center}
		\begin{asy}
			unitsize(1.5cm);
			pair z0=dir(20);
			real r=1;
			for(int i = 0+0; i < 6; ++i)
				draw(arc(z0,r,60*i,60*i-60), EndArrow);
			dot("$z_0$", z0, -z0);
			draw(z0--z0+z0, dotted);
			label("$r$", z0+z0/2, dir(20+90));
		\end{asy}
	\end{center}
\end{example}
We close with the following theorem.
\begin{theorem}
	Fix $\Omega\subseteq\CC$ an open and connected subset. Further, fix two paths $\gamma_1,\gamma_2:[a,b]\to\CC$ two $C^1$ paths and some holomorphic function $f:\Omega\to\CC$. Now, suppose that $t_1,t_2\in[a,b]$ have $z_0:=\gamma_1(t_1)=\gamma_2(t_2)\ne0$ with $f'(z_0)\ne0$. Then
	\[\arg\left(\frac{\gamma_1'(t_2)}{\gamma_2'(t_1)}\right)=\arg\left(\frac{(f\circ\gamma)'(t_2)}{(f\circ\gamma)'(t_1)}\right).\]
\end{theorem}
\begin{proof}
	We omit this proof. It is in the Eterovi\'c notes.
\end{proof}