% !TEX root = ../notes.tex

Good morning everyone.
\begin{itemize}
	\item Homework \#7 is due on Sunday just before midnight.
	\item There will be office hours tomorrow from 2PM to 3:30PM, as usual.
\end{itemize}

\subsection{More on Zeroes}
We are talking about more consequences of the Cauchy integral formula. For example, last time we showed Liouville's theorem, the Fundamental theorem of algebra, and the Riemann removable singularity theorem. We are also about to show the following result.
\isolatezeroes*
\begin{proof}
	Fix some $z_0\in Z$. Now, because $f$ is holomorphic, $f$ is analytic at $z_0$ (by \autoref{cor:betterholoisana}), so we have some $r>0$ such that $\overline{B(z_0,r)}\subseteq\Omega$ with
	\[f(z)=\sum_{k=0}^\infty a_k(z-z_0)^k\tag{$*$}\label{eq:expandfatz0}\]
	for any $z\in B(z_0,r)$. It is technically possible for $a_k=0$ for all $k\in\NN$. But now, $f$ is zero on all of $B(z_0,r)$, which is one possibility for part (a). This is all we are going to say about this case.
	
	Otherwise, let $n$ be the minimum natural number such that $a_n\ne0$. As such, we simply define $g:\Omega\to\CC$ as
	\[g(z):=\begin{cases}
		f(z)/(z-z_0)^n & z\ne z_0, \\
		a_n & z=z_0.
	\end{cases}.\]
	This function is at least holomorphic at all points away from $z_0$ as the quotient of two holomorphic functions (by \autoref{prop:computederivs}), so we merely need to check that $g$ is holomorphic at $z_0$. However, on $B(z_0,r)$, we see that $z\ne z_0$ will have
	\[g(z)=\frac{f(z)}{(z-z_0)^n}=\sum_{k=n}^\infty a_k(z-z_0)^n\]
	by \autoref{eq:expandfatz0}. But of course, this also works at $g(z_0)=a_n$, so we see that the above power series expansion works for all $z\in B(z_0,r)$. So $g$ is in fact analytic at $z_0$ and hence holomorphic at $z_0$ by \autoref{eq:expandatz0}.

	We now show that $z_0$ is an isolated point of $Z$. Well, $g(z_0)\ne0$ and $g$ is continuous (in fact holomorphic), we are promised some $\varepsilon>0$ such that
	\[|g(z)-g(z_0)|>|g(z_0)|\]
	for all $z\in B(z_0,\varepsilon)$, so in particular $g(z)\ne0$ here. Thus, when we write
	\[f(z)=(z-z_0)^ng(z),\]
	the only time we can have $f(z)=0$ for $z\in B(z_0,\varepsilon)$ is at $z=z_0$ because $z\ne z_0$ implies $(z-z_0)^n\ne0$ and $g(z)\ne0$.

	Lastly, we get the uniqueness of the integer $n$ follows from its minimality. %\todo{}
\end{proof}
To use the above result, we show one of my personal favorite results from complex analysis.
\begin{theorem}[Identity]
	Fix an open, connected subset $\Omega\subseteq\CC$ with two holomorphic functions $f_1,f_2:\Omega\to\CC$. Further, set
	\[Z:=\{z\in\Omega:f_1(z)=f_2(z)\}.\]
	If $Z$ contains an accumulation point, then $f_2=f_2$ on $\Omega$.
\end{theorem}
\begin{proof}
	For psychological reasons, we set $f(z):=f_1(z)-f_2(z)$ so that $z\in Z$ if and only if $f_1(z)=f_2(z)$ if and only if $f(z)=0$. So $Z=f^{-1}(\{0\})$, and we are ripe to apply the previous result.

	Now, we are granted an accumulation point $w\in Z$, so we have some sequence $\{z_k\}_{k\in\NN}\subseteq Z\setminus\{w\}$ such that $z_k\to n$. In particular, $w$ is not isolated: for every open neighborhood $B(w,\varepsilon)$ around $w$, the fact that $z_k\to n$ promises that $B(w,\varepsilon)\cap(Z\setminus\{w\})\ne\emp$.

	Thus, \autoref{prop:isolatezeroes} kicks in, so there exists some $r>0$ such that $f(z)=0$ for all $B(w,r)$, so $B(w,r)\subseteq Z$. In other words, every accumulation point of $Z$ is contained in the interior of $Z$, which we will denote $Z^\circ$.

	As such, we claim that $Z^\circ$ is closed. Quickly, note that $Z=f^{-1}(\{0\})$ is the pre-image of a closed set and hence closed by \autoref{lem:topologicalcontinuity} because $f$ is continuous.
	% Thus, we see that
	% \[\del Z=\overline Z\setminus Z^\circ=Z\setminus Z^\circ.\]
	% Now, $Z$ is a union of isolated points and accumulation points, so because all accumulation points live in $Z^\circ$, we are forced to have $\del Z$ consist of isolated points.
	In particular, if $\alpha$ is any limit point of $Z^\circ$, then $\alpha$ is an accumulation point of $Z$ (because $Z$ is closed), so $\alpha\in Z^\circ$.

	Thus, $Z^\circ$ is indeed closed. But it is also open, so the connectivity of $\Omega$ forces $Z^\circ=\emp$ or $Z^\circ=\Omega$. But $Z^\circ$ is nonempty because we have an accumulation point, so $Z^\circ=\Omega$, so we are done.
\end{proof}
\begin{remark}
	This is really something special about holomorphic functions. For example, the function
	\[f(z)=\begin{cases}
		e^{-1/z^2} & z>0, \\
		0 & z\le0,
	\end{cases}\]
	is real analytic everywhere, and it agrees with the zero function on $\RR_{\le0}$, but of course $f$ is nonzero.
\end{remark}
We remark that the value of $n$ in \autoref{prop:isolatezeroes} is somewhat special.
\begin{definition}[Multiplicity]
	Fix an open, connected subset $\Omega\subseteq\CC$ and a holomorphic function $f:\Omega\to\CC$. If we have some $z_0\in\Omega$ such that $f(z_0)=0$ and $z_0$ is isolated in $f^{-1}(\{0\})$, then by \autoref{prop:isolatezeroes} there is a unique integer $n$ and holomorphic function $g:\Omega\to\CC$ with $g(z_0)\ne0$ such that
	\[f(z)=(z-z_0)^ng(z).\]
	This $n$ is called the \textit{multiplicity} of $z_0$ in $f$.
\end{definition}
We actually know how to compute $f$ because \autoref{prop:isolatezeroes} is fully constructive: we simply expanded out the power series expansion of $f$ at $z_0$ as
\[f(z)=\sum_{k=0}^\infty a_k(z-z_0)^k\]
and then looked for the minimal $n$ such that $a_n\ne0$. However, we also know that these coefficients of the power series can be computed via the proof of \autoref{cor:readcoefs} as
\[f^{(m)}(z_0)=m!a_m,\]
so we can alternatively look for the minimal $n$ such that $f^{(n)}(z_0)\ne0$.
\begin{example}
	By \autoref{lem:powerseriescossin}, we computed
	\[\sin z=\sum_{k=0}^\infty\frac{(-1)^{k-1}}{(2k+1)!}z^{2k+1}.\]
	We can check that $\sin0=0$ while the linear term is nonzero, so we have multiplicity $1$. Alternatively, we can compute the first derivative as
	\[\sin'(0)=\cos(0)=1\ne0.\]
\end{example}

\subsection{More on Zeroes}
We close by stating a theorem.
\begin{theorem}[Maximum modulus principle] \label{thm:mmp}
	Fix an open, connected subset $\Omega\subseteq\CC$ and a non-constant holomorphic function $f:\Omega\to\CC$. For each $z\in\Omega$ and $r>0$, there exists $w\in B(z,r)\cap\Omega$ such that
	\[|f(w)|>|f(z)|.\]
\end{theorem}
\begin{proof}
	We will not prove this result in class, but it might be on the homework.
\end{proof}

The rough idea here is that $f$ cannot obtain a maximum on an open set: we must always look to the boundary. More rigorously, we have the following statement.
\begin{corollary} \label{cor:mmp}
	Fix an open, connected subset $\Omega\subseteq\CC$ and a non-constant continuous function $f:\overline\Omega\to\CC$ such that $f|_\Omega$ is holomorphic. Now, if $z_0\in\overline\Omega$ such that $|f(z_0)|$ is maximal, then $z_0\in\del\Omega$.
\end{corollary}
\begin{proof}
	Once again, we omit this proof.
\end{proof}
\begin{restatable}[Schwarz's lemma]{cor}{schwarzlemma}
	Fix a holomorphic function $f:B(0,1)\to B(0,1)$ such that $f(0)=0$. Then actually $|f(z)|\le|z|$ for all $z\in B(0,1)$ and also $|f'(0)|\le1$. Further, if $|f'(0)|=1$ or $|f(z)|=|z|$ for some nonzero $z\in B(0,1)$, then $f(z)=az$ for all $z\in\CC$ for some fixed $a\in\CC$.
\end{restatable}
Intuitively, holomorphic functions $B(0,1)\to B(0,1)$ must contract.
\begin{proof}
	We will prove this next class.
\end{proof}