% !TEX root = ../notes.tex

It is reportedly close enough to start.

\subsection{Logistics}
We are online for the first two weeks, as with the rest of Berkeley. We will be using \href{https://bcourses.berkeley.edu/courses/1511845}{\texttt{bCourses}} a lot, so check it frequently. There is also a \href{https://sites.google.com/view/ucb-math185-005-spring2022/home}{website}. There is a homework due on Friday, but do not worry about it.

Here are some syllabus things.
\begin{itemize}
	\item Our main text is \textit{Complex Variables and Applications, 8th Edition} because it is the version that Professor Morrow used. There is a free copy online.
	\item The homework consists of readings (for each course day) and weekly problem sets. Late homework is never accepted.
	\item Lowest two homework scores are dropped.
	\item There are two midterms and a final. The final is cumulative, as usual. The final can replace one midterm if the score is higher.
	\item Regrade requests can be made in GradeScope within one week of being graded.
	\item The class is curved but usually only curved at the end. The average on exams is expected to be 80\%--83\%.
\end{itemize}

\subsection{Complex Numbers}
Welcome to complex analysis. What does that mean?
\begin{idea}
	In complex analysis, we study functions $f\colon\CC\to\CC$, usually analytic to some extent.
\end{idea}
There are two pieces here: we should study $\CC$ in themselves, and then we will study the functions.
\begin{restatable}[Complex numbers]{definition}{complexdef}
	The set of complex numbers $\CC$ is $\{a+bi:a,b\in\RR\}$, where $i^2=-1$.
\end{restatable}
\noindent Hopefully $\RR$ is familiar from real analysis. As an aside, we see $\RR\subseteq\CC$ because $a=a+0i\in\CC$ for each $a\in\RR$.

The complex numbers have an inherent geometry as a two-dimensional plane.
\begin{center}
	\begin{asy}
		unitsize(1cm);
		for(int i = -3; i <= 3; ++i)
		{
			dot("$"+string(i)+"$", (i,0), SE);
			if(i != 0)
			{
				dot("$"+string(i)+"i$", (0,i), SE);
				draw((-3,i)--(3,i), dashed);
				draw((i,-3)--(i,3), dashed);
			}
		}
		draw((-3,0)--(3,0));
		draw((0,-3)--(0,3));
		dot("$2+i$", (2,1), NE);
	\end{asy}
\end{center}
The point is that $\CC$ looks like the real plane $\RR^2$. More precisely, $\CC\cong\RR^2$ as an $\RR$-vector space, where our basis is $\{1,i\}$.

We would like to understand $\CC$ geometrically, ``as a space.'' The first step here is to create a notion of size.
\begin{definition}[Norm on \texorpdfstring{$\CC$}{C}]
	We define the \textit{norm map} $|\cdot|:\CC\to\RR_{\ge0}$ by $|z|\coloneqq \sqrt{z\overline z}$. In other words,
	\[|a+bi|\coloneqq \sqrt{a^2+b^2}.\]
\end{definition}
Note that this agrees with the absolute value on $\RR$: for $a\in\RR$, we have $\sqrt{a^2}=|a|$.

Norm functions, as in the real case, give us a notion of distance.
\begin{definition}[Metric on \texorpdfstring{$\CC$}{C}]
	We define the \textit{metric on $\CC$} to be $d_\CC(z_1,z_2)\coloneqq |z_1-z_2|$.
\end{definition}
\noindent One can check that this is in fact a metric, but we will not do so here.
\begin{remark}
	The distance in $\CC$ is defined to match the distance in $\RR^2$ under the basis $\{1,i\}$.
\end{remark}
Again as we discussed in real analysis, having a metric gives us a metric topology by open balls. Lastly it is this topology that our geometry will follow from: we have turned $\CC$ into a topological space.

\subsection{Complex Functions}
There are lots of functions on $\CC$, and lots of them are terrible. So we would like to focus on functions with some structure. We'll start with \textit{continuous functions}, which are more or less the functions that respect topology.

Then from continuous functions, we will be able to define \textit{holomorphic functions}, which are complex differentiable. This intended to be similar to being real differentiable, but complex differentiable turns out to be a very strong condition. Nevertheless, everyone's favorite functions are holomorphic.
\begin{example}
	Polynomials, $\exp$, $\sin$, and $\cos$ are all holomorphic.
\end{example}
To make concrete that complex differentiability is stronger than real differentiability, the Cauchy--Riemann equations which provides a partial differential equation to test complex differentiability.

From here we define \textit{analytic} functions, which essentially are defined as taking the form
\[f(z)\coloneqq \sum_{k=0}^\infty a_kz^k.\]
Analytic functions are super nice in that we have an ability to physically write them down, so the following theorem is amazing.
\begin{theorem}
	Holomorphic functions on $\CC$ are analytic.
\end{theorem}
To prove this, we will need the following result, which is what Professor Morrow calls the most fundamental result in complex analysis, the \textit{Cauchy integral formula}.

In short, the Cauchy integral formula lets us talk about the value of holomorphic functions (and derivatives) at a point in terms of integrals around the point. This will essentially let us build the power series for a holomorphic function by hand. But as described, we will need a notion of complex (path) integration to even be able to talk about the Cauchy integral formula.

The Cauchy integral formula has lots of applications; for example, \textit{Liouville's theorem} on holomorphic functions and the \textit{Fundamental theorem of algebra}.
\begin{remark}
	It is very hard to spell Liouville.
\end{remark}
Additionally, we remark that our study of holomorphic functions, via the Cauchy integral formula, will boil down to a study of complex path integrals. So we will finish out our story with the \textit{Residue theorem}, which provides a very convenient way to compute such integrals.

Then as a fun addendum, we talk about automorphisms of the complex numbers.
\begin{definition}[Automorphisms of \texorpdfstring{$\CC$}{C}]
	A function $f\colon\CC\to\CC$ is an \textit{automorphism of $\CC$} if it is bijective and both $f$ and $f^{-1}$ are holomorphic.
\end{definition}
\noindent What is amazing is that all of these functions have a concrete description in terms of \textit{M\"obius transformations}.

\subsection{Why Care?}
Whenever taking a class, it is appropriate to ask why one should care. Here are some reasons to care.
\begin{itemize}
	\item Algebraic geometry in its study of complex analytic spaces uses complex analysis.
	\item Analytic number theory (e.g., the Prime number theorem) makes heavy use of complex analysis.
	\item Combinatorics via generating functions can use complex analysis.
	\item Physics uses complex analysis.
\end{itemize}
The first two Professor Morrow is more familiar with, the last two less so.