\documentclass[../notes.tex]{subfiles}
\graphicspath{{\subfix{../figs/}}}

\begin{document}
% !TEX root = ../notes.tex

Hopefully we'll be in-person on Monday. Homework 2 will be released later today, due next Friday.

\subsection{Sequences}
Today we're talking about sequences, building towards a theory of sequences and series. Next week we will begin studying holomorphic functions and actually doing complex analysis.

Anyways, here is a series of definitions.
\begin{definition}[Sequence]
	A \textit{sequence} of complex numbers is a function $f\colon\NN\to\CC$. Often we will notate this by $\{z_n\}_{n\in\NN}$ where $z_n\coloneqq f(n)$.
\end{definition}
By convention, all of our sequences will be sequences of complex numbers unless otherwise stated.
\begin{definition}[Subsequence]
	A sequence $\{w_n\}_{n\in\NN}\subseteq\CC$ is a \textit{subsequence} of a sequence $\{z_n\}_{n\in\NN}\subseteq\CC$ if and only if there is some strictly increasing function $g\colon\NN\to\NN$ such that $w_n=z_{g(n)}$.
\end{definition}
\begin{definition}[Bounded]
	A sequence $\{z_n\}_{n\in\NN}\subseteq\CC$ is \textit{bounded} if and only if there exists a positive real number $M>0$ such that
	\[\{z_n\}_{n\in\NN}\subseteq B(0,M).\]
	In other words, $|z_n|<M$ for each $n\in\NN$.
\end{definition}
We are in particular interested in convergence in analysis.
\begin{definition}[Converges]
	A sequence $\{z_n\}_{n\in\NN}\subseteq\CC$ \textit{converges} to some $z\in\CC$ if and only if, for each $\varepsilon>0$, there exists some $N$ such that $n>N$ implies
	\[|z-z_n|<\varepsilon.\]
	We will notate this by $z_n\to z$ or $\lim_{n\to\infty}z_n=z$.
\end{definition}
Note that the definition of the limit above says that
\[\lim_{n\to\infty}z_n=z\iff\lim_{n\to\infty}|z_n-z|=0.\]
Intuitively, the distance between the $z_n$ and the $z$ has to ``narrow in'' on $z$. %\todo{maybe add image}

We would like some real-analytic tools for our complex analysis. Here is a convergence lemma.
\begin{lemma} \label{lem:reducetor}
	Fix $\{z_n\}_{n\in\NN}\subseteq\CC$ a sequence. Then, letting $z_n\coloneqq x_n+y_ni$, we have that $z_n\to z$ where $z=x+yi$ if and only if $x_n\to x$ and $y_n\to y$.
\end{lemma}
\begin{proof}
	This is essentially by definition of the metric on $\CC$. We take the directions one at a time.
	\begin{itemize}
		\item Suppose that $z_n\to z$ in $\CC$. Then we claim that $\op{Re}z_n\to\op{Re}z$ and $\Im z_n\to\Im z_n$ in $\RR$. Indeed, for any $\varepsilon>0$, there is $N$ such that
		\[n>N\implies|z-z_n|<\varepsilon.\]
		But now we see that $|\Re z_n-\Re z|,|\Im z_n-\Im z|\le\sqrt{(\Re z_n-\Re z)^2+(\Im z_n-\Im z)^2}$, so it follows
		\[n>N\implies|\Re z_n-\Re z|,|\Im z_n-\Im z|<\varepsilon,\]
		finishing.
		\item Suppose that $\Re z_n\to x$ and $\Im z_n\to y$. We claim that $z_n\to x+yi$. Indeed, for any $\varepsilon>0$, there exists $N_x$ such that
		\[n>N_x\implies|\Re z_n-x|<\varepsilon/2\]
		and $N_y$ such that
		\[n>N_y\implies|\Im z_n-y|<\varepsilon/2.\]
		It follows that
		\[n>\max\{N_x,N_y\}\implies|z_n-(x+yi)|=\sqrt{|\Re z_n-x|^2+|\Im z_n-y|^2}\le\sqrt{\left(\frac\varepsilon2\right)^2+\left(\frac\varepsilon2\right)^2}<\varepsilon.\]
		This finishes.
		\qedhere
	\end{itemize}
\end{proof}
Essentially, this means that checking convergence of complex numbers is the same as checking real and imaginary parts individually, so we can turn convergence questions into ones from real analysis.

We also have the following basic properties about convergence.
\begin{proposition}
	Fix $\{z_n\}_{n\in\NN}\subseteq\CC$ a convergent sequence. The following are true.
	\begin{listalph}
		\item $\{z_n\}_{n\in\NN}$ is bounded.
		\item The limit of $\{z_n\}_{n\in\NN}$ is unique.
		\item Every subsequence of $\{z_n\}_{n\in\NN}$ converges to $z$.
	\end{listalph}
\end{proposition}
\begin{proof}
	We take the claims one at a time. Let $z\in\CC$ be so that $z_n\to z$.
	\begin{listalph}
		\item Fix $\varepsilon=1$ so that there exists $N$ so that $n>N$ implies $|z_n-z|<1$. Now set
		\[M\coloneqq \max(\{|z_n|+1:n\le N\}\cup\{|z|+1\}).\]
		We claim that $|z_n|<M$ for each $n\in\NN$. We have two cases.
		\begin{itemize}
			\item If $n\le N$, then $|z_n|<|z_n|+1\le M$.
			\item Otherwise, $n>N$ so that
			\[|z_n|\le|z_n-z|+|z|<|z|+1\le M,\]
			so we are done.
		\end{itemize}

		\item Suppose that $z_n\to z'$ for some $z'\in\CC$, and we show $z=z'$. Indeed, if $z=z'$, then we are done, so suppose that $z\ne z'$ so that $|z-z'|\ne0$. Then we set $\varepsilon\coloneqq \frac12|z-z'|>0$, and we are promised some $N,N'$ such that
		\[n>N\implies|z-z_n|<\frac\varepsilon2\quad\text{and}\quad n>N'\implies|z'-z_n|<\frac\varepsilon2.\]
		In particular, we see that, for $n>\max\{N,N'\}$, we have
		\[|z-z'|\le|z-z_n|+|z_n-z'|<\frac\varepsilon2+\frac\varepsilon2=\varepsilon=\frac12|z-z'|.\]
		But because $0\le|z-z'|$, we see that this forces $|z-z'|=0$, so $z=z'$ follows. (Technically we have hit contradiction, but we do not need to use this.)

		\item Note that subsequences can be characterized by choosing a strictly increasing function $f\colon\NN\to\NN$ so that we want to show $z_{f(n)}\to z$. Indeed, for any $\varepsilon>0$, we are promised some $N$ so that
		\[n>N\implies|z-z_n|<\varepsilon.\]
		Now, for each $n\in\NN$, we have\footnote{We can show this by induction on $n$, for $f(0)\ge0$ and $f(n+1)>f(n)$ forces $f(n+1)\ge f(n)+1$.} $f(n)\ge n$, so we see that
		\[n>N\implies f(n)>N\implies|z-z_{f(n)}|<\varepsilon,\]
		which finishes.
		\qedhere
	\end{listalph}
\end{proof}

Sequences themselves have an arithmetic.
\begin{proposition}
	Fix $\{z_n\}_{n\in\NN},\{w_n\}_{n\in\NN}\subseteq\CC$ sequences such that $z_n\to z$ and $w_n\to w$. Then the following hold.
	\begin{listalph}
		\item $z_n+w_n\to z+w$.
		\item $z_nw_n\to zw$.
		\item If $w\ne0$ and $w_n\ne0$ for each $n\in\NN$, then $\frac1{w_n}\to\frac1w$.
	\end{listalph}
\end{proposition}
\begin{proof}
	We take these one at a time, essentially borrowing the proof from metric spaces.
	\begin{listalph}
		\item Fix some $\varepsilon>0$. We can find some $N_z$ such that
		\[n>N_z\implies|z-z_n|<\varepsilon/2\]
		and some $N_w$ such that
		\[n>N_w\implies|w-w_n|<\varepsilon/2.\]
		Now, taking $N\coloneqq \max\{N_z,N_w\}$ so that the triangle inequality gives
		\[n>N\implies|(z+w)-(z_n+w_n)|\le|z-z_n|+|w-w_n|<\varepsilon,\]
		which finishes.

		\item We have to use the fact that the sequences are bounded here. Because $w_n\to w$, the sequence is bounded, so there is an $M$ so that $|w_n|<M$ for each $n\in\NN$. Now, the key inequality is that
		\[|z_nw_n-zw|\le|z_nw_n-zw_n|+|zw_n-zw|\le M|z_n-z|+|z|\cdot|w_n-w|.\tag{$*$}\label{eq:multbound}\]
		So with this in mind, fix any $\varepsilon>0$, and we see that we are promised $N_z$ such that
		\[n>N_z\implies|z_n-z|<\frac\varepsilon{2M}\]
		and some $N_w$ such that
		\[n>N_w\implies|w_n-w|<\frac\varepsilon{2|z|}\]
		so that \hyperref[eq:multbound]{($*$)} implies
		\[n>\max\{N_x,N_w\}\implies|z_nw_n-zw|<\varepsilon,\]
		finishing.

		\item We begin with some motivating arithmetic. Observe that
		\[\left|\frac1w-\frac1{w_n}\right|=\frac{|w_n-w|}{|ww_n|}.\]
		We can upper-bound the numerator without tears, so we see the main difficulty is lower-bounding the denominator. Well, because $w\ne0$, we can set $\varepsilon=|w|/2$ so that there exists $N_0$ such that
		\[n>N_0\implies|w_n-w|<|w|/2.\]
		In particular, it follows that $|w_n|\ge|w|-|w-w_n|=|w|/2$ for $n>N_0$.

		With this in mind, fix any $\varepsilon>0$. Then we are promised some $N_1$ such that
		\[n>N_1\implies|w_n-w|<|w|^2\varepsilon/2\]
		so that we see
		\[n>\max\{N_0,N_1\}\implies\left|\frac1w-\frac1{w_n}\right|=\frac{|w_n-w|}{|w|\cdot|w_n|}\le\frac{|w|^2\varepsilon/2}{|w|\cdot|w|/2}=\varepsilon,\]
		finishing.
		\qedhere
	\end{listalph}
\end{proof}

\subsection{Limit Points}
Here is our main character.
\begin{definition}[Limit point]
	Fix $X\subseteq\CC$ and some $z\in\CC$. Then we say that $z$ is a \textit{limit point} if and only if there exists some sequence $\{z_n\}_{n\in\NN}\subseteq X$ such that $z_n\to z$.
\end{definition}
\begin{definition}[Accumulation point]
	Fix $X\subseteq\CC$ and some $z\in\CC$. Then we say that $z$ is an \textit{accumulation point} if and only if there exists some sequence $\{z_n\}_{n\in\NN}\subseteq X\setminus\{z\}$ such that $z_n\to z$.
\end{definition}
Essentially accumulation points do not allow isolated points while limit points do.

The above essentially gives a more directly topological definition of ``closed set.'' It also gives us a more directly topological definition of the closure.
\begin{lemma}
	Fix $X\subseteq\CC$ and $z\in\CC$. The following are equivalent.
	\begin{listalph}
		\item We have that $z\in\overline X$.
		\item For all $\varepsilon>0$, we have $B(z,\varepsilon)\cap X\ne\emp$.
		\item There exists $\{z_n\}_{n\in\NN}\subseteq X$ such that $z_n\to z$.
	\end{listalph}
\end{lemma}
\begin{proof}
	We show our directions one at a time.
	\begin{itemize}
		\item We show (a) implies (b). Suppose $z\in\overline X$, and for the sake of contradiction suppose we have $\varepsilon>0$ such that $B(z,\varepsilon)\cap X=\emp$. In particular, $z\notin X$.

		Now, $z\in\overline X$ implies that $z$ is contained in every closed set containing $X$ by definition of $\overline X$. But because $B(z,\varepsilon)$ is open and is disjoint from $X$, we see
		\[z\in\overline X\subseteq\CC\setminus B(z,\varepsilon),\]
		which is a contradiction.
		\item We show (b) implies (c). For each $n\in\NN$, we know that $B(z,1/n)\cap X\ne\emp$, so we find some $z_n\in B(z,1/n)$. Now, for any $\varepsilon>0$, choose $N\coloneqq 1/\varepsilon$ so that
		\[n>N\implies|z_n-z|<\frac1n<\frac1N=\varepsilon,\]
		so indeed $z_n\to z$.

		\item We show (b) implies (a). We proceed by contraposition. Suppose that $z\notin\overline X$. It follows that $z\in\CC\setminus\overline X$, which is open, so there exists an $r>0$ such that
		\[B(z,r)\subseteq\CC\setminus\overline X\subseteq\CC\setminus X.\]
		It follows that $B(z,r)\cap X=\emp$.

		\item We show (c) implies (b). Suppose $\{z_n\}_{n\in\NN}\subseteq X$ has $z_n\to z$ for some $z\in\CC$. For any $\varepsilon>0$, there exists $N$ such that
		\[n>N\implies|z_n-z|<\varepsilon,\]
		so in particular, choosing any $n\coloneqq \ceil N+1$ has $z_n\in B(z,\varepsilon)\cap X$, so $B(z,\varepsilon)\cap X\ne\emp$.
		\qedhere
	\end{itemize}
\end{proof}
The above discussion can give us a more directly topological definition of ``closed.''
\begin{lemma}
	A subset $X\subseteq\CC$ is closed in $\CC$ if and only if $X$ contains all of its limit points.
\end{lemma}
\begin{proof}
	By the previous lemma, we see that $z\in\overline X$ if and only if $z$ is a limit point of $X$, so $\overline X$ is the set of limit points of $X$. Now, $X$ is closed if and only if $X=\overline X$, so $X$ is closed if and only if all limit points of $X$ are in fact points of $X$. (Note that all points of $X$ are automatically limit points essentially because $X\subseteq\overline X$ for free.)
\end{proof}

While we're here, we can pick up a nice topological result.
\begin{lemma}
	Fix $X\subseteq\CC$ a connected subset. Then $\overline X$ is also connected.
\end{lemma}
\begin{proof}
	This argument is purely topological. We proceed by contraposition: suppose $\overline X$ is disconnected by $U_1,U_2\subseteq\CC$. We claim that $U_1,U_2$ disconnect $X$. Well, we already know that $A\subseteq\overline A\subseteq U_1\cup U_2$, and we already know that $U_1$ and $U_2$ are disjoint.

	We claim that, for $U\subseteq\CC$ an open subset, if $U\cap\overline X\ne\emp$, then $U\cap X\ne\emp$ as well. Indeed, we proceed by contraposition: if $U\cap X=\emp$, then $X\subseteq\CC\setminus U$, but $\CC\setminus U$ is closed, so
	\[\overline X\subseteq\CC\setminus U,\]
	so $\overline X\cap U=\emp$.

	Thus, it follows from $U_1\cap\overline X,U_2\cap\overline X\ne\emp$ that $U_1\cap X,U_2\cap X\ne\emp$. This finishes the proof that $U_1$ and $U_2$ disconnect $X$. Indeed, 
\end{proof}

\subsection{Cauchy Sequences}
Just like in real analysis, we will be interested in Cauchy sequences.
\begin{definition}[Cauchy sequence]
	A sequence $\{z_n\}_{n\in\NN}\subseteq\CC$ is a \textit{Cauchy sequence} if and only if, for each $\varepsilon>0$, there exists an $N$ such that
	\[n,m>N\implies|z_n-z_m|<\varepsilon.\]
\end{definition}
We have the following results on Cauchy sequences.
\begin{proposition}
	Fix $\{z_n\}_{n\in\NN}\subseteq\CC$ a sequence. If $\{z_n\}_{n\in\NN}$ converges, it is Cauchy.
\end{proposition}
\begin{proof}
	This proof uses no special properties of $\CC$. If $z_n\to z$, then for a given $\varepsilon>0$, there exists $N$ such that
	\[n>N\implies|z_n-z|<\varepsilon/2.\]
	It follows that
	\[n,m>N\implies|z_n-z_m|<|z_n-z|+|z_m-z|<\varepsilon,\]
	finishing.
\end{proof}
\begin{proposition}
	Every Cauchy sequence in $\CC$ converges.
\end{proposition}
\begin{proof}
	If $\{z_n\}_{n\in\NN}$ is Cauchy, then we claim $\{\op{Re}z_n\}_{n\in\NN}$ and $\{\op{Im}z_n\}_{n\in\NN}$ are Cauchy sequences. Indeed, for any $\varepsilon>0$, there exists $N$ so that
	\[n,m>N\implies|z_n-z_m|<\varepsilon,\]
	but then $|\Re z_n-\Re z_m|<|z_n-z_m|$ and $|\Im z_n-\Im z_m|<|z_n-z_m|$, so the same $N$ witnesses that $\{\Re z_n\}_{n\in\NN}$ and $\{\Im z_n\}_{n\in\NN}$ are Cauchy in $\RR$.

	Now, Cauchy sequences in $\RR$ converge, so there are reals $x,y\in\RR$ such that $\Re z_n\to x$ and $\Im z_n\to w$. It follows that $z_n\to x+yi$, finishing.
\end{proof}

\subsection{A Little More Topology}
We close with one more topological definition.
\begin{definition}[Sequentially compact]
	A subset $X\subseteq\CC$ is \textit{sequentially compact} if and only if every $\{z_n\}_{n\in\NN}\subseteq X$ has a convergent subsequence which converges in $X$.
\end{definition}
\begin{remark}
	This happens to be equivalent to $X$ is compact because $\CC\cong\RR^2$ satisfies the fact that all compact sets are closed and bounded.
\end{remark}
\begin{example}
	Every finite set is compact.
\end{example}
And here is a last definition.
\begin{definition}[Tends to infinity]
	A sequence $\{z_n\}_{n\in\NN}\subseteq\CC$ \textit{tends to infinity} (notated $z_n\to\infty$) if and only if each $M>0$ has some $N\in\NN$ such that
	\[n>N\implies|z_n|>M.\]
\end{definition}
Essentially the points of $\{z_n\}_{n\in\NN}$ wander infinitely away.
\end{document}