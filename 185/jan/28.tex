% !TEX root = ../notes.tex

Hopefully we'll be in-person on Monday. Homework 2 will be released later today, due next Friday.

\subsection{Sequences}
Today we're talking about sequences, building towards a theory of sequences and series. Next week we will begin studying holomorphic functions and actually doing complex analysis.

Anyways, here is a series of definitions.
\begin{definition}[Sequence]
	A \textit{sequence} of complex numbers is a function $f:\NN\to\CC$. Often we will notate this by $\{z_n\}_{n\in\NN}$ where $z_n:=f(n)$.
\end{definition}
By convention, all of our sequences will be sequences of complex numbers unless otherwise stated.
\begin{definition}[Convergence]
	A sequence $\{w_n\}_{n\in\NN}\subseteq\CC$ is a \textit{subsequence} of a sequence $\{z_n\}_{n\in\NN}\subseteq\CC$ if and only if there is some strictly increasing function $g:\NN\to\NN$ such that $w_n=z_{g(n)}$.
\end{definition}
\begin{definition}[Bounded]
	A sequence $\{z_n\}_{n\in\NN}\subseteq\CC$ is \textit{bounded} if and only if there exists a positive real number $M>0$ such that
	\[\{z_n\}_{n\in\NN}\subseteq B(0,M).\]
	In other words, $|z_n|<M$ for each $n\in\NN$.
\end{definition}
We are in particular interested in convergence in analysis.
\begin{definition}[Converges]
	A sequence $\{z_n\}_{n\in\NN}\subseteq\CC$ \textit{converges} to some $z\in\CC$ if and only if, for each $\varepsilon>0$, there exists some $N$ such that $n>N$ implies
	\[|z-z_n|<\varepsilon.\]
	We will notate this by $z_n\to z$ or $\lim_{n\to\infty}z_n=z$.
\end{definition}
Note that the definition of the limit above says that
\[\lim_{n\to\infty}z_n=z\iff\lim_{n\to\infty}|z_n-z|=0.\]
Intuitively, the distance between the $z_n$ and the $z$ has to ``narrow in'' on $z$. \todo{maybe add image}

We would like some real-analytic tools for our complex analysis. Here is a convergence lemma.
\begin{lemma}
	Fix $\{z_n\}_{n\in\NN}\subseteq\CC$ a sequence. Then, letting $z_n:=x_n+y_ni$, we have that $z_n\to z$ where $z=x+yi$ if and only if $x_n\to x$ and $y_n\to y$.
\end{lemma}
\begin{proof}
	Omitted.\todo{}
\end{proof}
Essentially, this means that checking convergence of complex numbers is the same as checking real and imaginary parts individually, so we can turn convergence questions into ones from real analysis.

Sequences themselves have an arithmetic.
\begin{proposition}
	Fix $\{z_n\}_{n\in\NN},\{w_n\}_{n\in\NN}\subseteq\CC$ sequences such that $z_n\to z$ and $w_n\to w$. Then the following hold.
	\begin{listalph}
		\item $z_n+w_n\to z+w$.
		\item $z_nw_n\to zw$.
		\item If $w\ne0$ and $w_n\ne0$ for each $n\in\NN$, then $\frac1{w_n}\to\frac1w$.
	\end{listalph}
\end{proposition}
\begin{proof}
	We take these one at a time, essentially borrowing the proof from metric spaces.
	\begin{listalph}
		\item Fix some $\varepsilon>0$. We can find some $N_z$ such that
		\[n>N_z\implies|z-z_n|<\varepsilon/2\]
		and some $N_w$ such that
		\[n>N_w\implies|w-w_n|<\varepsilon/2.\]
		Now, taking $N:=\max\{N_z,N_w\}$ so that the triangle inequality gives
		\[n>N\implies|(z+w)-(z_n+w_n)|\le|z-z_n|+|w-w_n|<\varepsilon,\]
		which finishes.
	\end{listalph}
	We leave the other two as exercises.\todo{}
\end{proof}

\subsection{Limit Points}
Here is our main character.
\begin{definition}[Limit point]
	Fix $X\subseteq\CC$ and some $z\in\CC$. Then we say that $z$ is a \textit{limit point} if and only if there exists some sequence $\{z_n\}_{n\in\NN}\subseteq X$ such that $z_n\to z$.
\end{definition}
\begin{definition}[Accumulation point]
	Fix $X\subseteq\CC$ and some $z\in\CC$. Then we say that $z$ is an \textit{accumulation point} if and only if there exists some sequence $\{z_n\}_{n\in\NN}\subseteq X\setminus\{z\}$ such that $z_n\to z$.
\end{definition}
Essentially accumulation points do not allow isolated points while limit points do.
\begin{lemma}
	A subset $X\subseteq\CC$ is closed in $\CC$ if and only if $X$ contains all of its limit points.
\end{lemma}
\begin{proof}
	We show one direction and leave the other as an exerise.
	\begin{itemize}
		\item Take $X$ closed and find $z\notin X$. Now, suppose that $\{z_n\}_{n\in\NN}\subseteq X$ and we show that $z_n$ does not converge to $z$, which will be enough. Suppose for the sake of contradiction that $z\notin X$.

		Now, $U=\CC\setminus X$ is open and contains $z$, so there is some $r>0$ such that
		\[B(z,r)\subseteq U.\]
		In particular, if $|z-w|<r$, then $w\notin X$. It follows that the sequence $\{z_n\}_{n\in\NN}\subseteq X$ cannot $r$-close to $z$, so we are done.
		\item \todo{}
	\end{itemize}
\end{proof}
The above essentially gives a more directly topological definition of ``closed set.'' It also gives us a more directly topological definition of the closure.
\begin{lemma}
	Fix $X\subseteq\CC$ and $z\in\CC$. The following are equivalent.
	\begin{listalph}
		\item We have that $z\in\overline X$.
		\item For all $\varepsilon>0$, we have $B(z,\varepsilon)\cap X\ne\emp$.
		\item There exists $\{z_n\}_{n\in\NN}\subseteq X$ such that $z_n\to z$.
	\end{listalph}
\end{lemma}
\begin{proof}
	We show our directions one at a time.
	\begin{itemize}
		\item We show (a) implies (b). Suppose $z\in\overline X$, and for the sake of contradiction suppose we have $\varepsilon>0$ such that $B(z,\varepsilon)\cap X=\emp$. In particular, $z\notin X$.

		Now, $z\in\overline X$ implies that $z$ is contained in every closed set containing $X$ by definition of $\overline X$. But because $B(z,\varepsilon)$ is open and is disjoint from $X$, we see
		\[z\in\CC\setminus B(z,\varepsilon),\]
		which is a contradiction.
		\item We show (b) implies (c). For each $n\in\NN$, we know that $B(z,1/n)\cap X\ne\emp$, so we find some $z_n\in B(z,1/n)$. Then we get that $z_n\to z$ by writing out the definition.

		\item We omit the proof that (c) implies (a).\todo{}
		\qedhere
	\end{itemize}
\end{proof}

While we're here, we can pick up a nice topological result.
\begin{lemma}
	Fix $X\subseteq\CC$ a connected subset. Then $\overline X$ is also connected.
\end{lemma}
\begin{proof}
	We proceed by contraposition: suppose $\overline X$ is disconnected by $U_1,U_2\subseteq\CC$. We claim that $U_1,U_2$ disconnect $X$. Well, we already know that $A\subseteq\overline A\subseteq U_1\cup U_2$, and we already know that $U_1$ and $U_2$ are disjoint.

	However, $U_1\cap\overline X\ne\emp$ and $U_2\cap\overline X\ne\emp$, so arguing about open balls in our topology force $U_1\cap X\ne\emp$ and $U_2\cap X\ne\emp$, finishing.
\end{proof}

\subsection{Cauchy Sequences}
Just like in real analysis, we will be interested in Cauchy sequences.
\begin{definition}
	A sequence $\{z_n\}_{n\in\NN}\subseteq\CC$ is a \textit{Cauchy sequence} if and only if, for each $\varepsilon>0$, there exists an $N$ such that
	\[n,m>N\implies|z_n-z_m|<\varepsilon.\]
\end{definition}
We have the following results on Cauchy sequences.
\begin{proposition}
	Fix $\{z_n\}_{n\in\NN}\subseteq\CC$ a sequence. The following are true.
	\begin{listalph}
		\item If $\{z_n\}_{n\in\NN}$ converges, it is bounded.
		\item If $\{z_n\}_{n\in\NN}$ converges, its limit is unique.
		\item If $\{z_n\}_{n\in\NN}$ converges to $z$, every subsequence converges to $z$.
		\item If $\{z_n\}_{n\in\NN}$ converges, it is Cauchy.
	\end{listalph}
\end{proposition}
\begin{proof}
	We take these one at a time.
	\begin{listalph}
		\item Choose some $\varepsilon=1$. Then maximize with the finitely many $z_n$ away from this.
		\item Standard.\todo{}
		\item Use the definition of subsequence and use the same $N$ promised by $z_n\to z$.
		\item If $z_n\to z$, then for a given $\varepsilon>0$, there exists $N$ such that
		\[n>N\implies|z_n-z|<\varepsilon/2.\]
		It follows that
		\[n,m>N\implies|z_n-z_m|<|z_n-z|+|z_m-z|<\varepsilon,\]
		finishing.
		\qedhere
	\end{listalph}
\end{proof}
\begin{proposition}
	Every Cauchy sequence in $\CC$ converges.
\end{proposition}
\begin{proof}
	If $\{z_n\}_{n\in\NN}$ is Cauchy, then it follows $\{\op{Re}z_n\}_{n\in\NN}$ and $\{\op{Im}z_n\}_{n\in\NN}$ are Cauchy sequnces, so they converge in the reals to some $x$ and $y$. It follows that $z_n\to x+yi$, finishing.
\end{proof}

\subsection{A Little More Topology}
We close with one more topological definition.
\begin{definition}[Sequentially compact]
	A subset $X\subseteq\CC$ is \textit{sequentially compact} if and only if every $\{z_n\}_{n\in\NN}\subseteq X$ has a convergent subsequence which converges in $X$.
\end{definition}
\begin{remark}
	This happens to be equivalent to $X$ is compact because $\CC\cong\RR^2$ satisfies the fact that all compact sets are closed and bounded.
\end{remark}
\begin{example}
	Every finite set is compact.
\end{example}
And here is a last definition.
\begin{definition}
	A sequence $\{z_n\}_{n\in\NN}\subseteq\CC$ \textit{tends to infinity} (notated $z_n\to\infty$) if and only if each $M>0$ has some $N\in\NN$ such that
	\[n>N\implies|z_n|>M.\]
\end{definition}
Essentially the points of $\{z_n\}_{n\in\NN}$ wander infinitely away.