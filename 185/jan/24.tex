% !TEX root = ../notes.tex

Good morning everyone.

\subsection{Algebraic Structure}
Today we are reviewing the complex numbers (reportedly, ``some basics''). Or at least it is hopefully mostly review. Here is our main character this semester.
\begin{definition}[Complex numbers]
	The set $\CC$ of \textit{complex numbers} is
	\[\CC:=\{a+bi:a,b\in\RR\}.\]
	Here $i$ is some symbol such that $i^2=-1$ formally.
\end{definition}
In particular, two complex numbers $a_1+b_1i$ and $a_2+b_2i$ are equal if and only if $a_1=a_2$ and $b_1=b_2$.

The complex numbers also have some algebraic structure.
\begin{definition}[\texorpdfstring{$+$}{} and \texorpdfstring{$\times$}{} in \texorpdfstring{$\CC$}{}]
	Given complex numbers $a_1+b_1i,a_2+b_2i\in\CC$, we define
	\[(a_1+b_1i)+(a_2+b_2i)=(a_1+a_2)+(b_1+b_2)i,\]
	and
	\[(a_1+b_1i)+(a_2+b_2i)=(a_1a_2-b_1b_2)+(a_1b_2+a_2b_1)i,\]
	defined essentially by direct expansion, upon recalling $i^2=-1$.
\end{definition}
Here is the corresponding algebraic structure.
\begin{proposition}
	The set $\CC$ with the above operations is a two-dimensional $\RR$-vector space with basis $\{1,i\}$.
\end{proposition}
\begin{proof}
	The elements $\{1,i\}$ span $\CC$ because all complex numbers in $\CC$ can be written as $a+bi=a\cdot1+b\cdot i$ by definition.
	
	To see that these elements are linearly independent, suppose $a+bi=0$. If $b=0$, then $a=0$ follows, and we are done. Otherwise, take $b\ne0$, but then we see $(-a/b)=i$, so
	\[(-a/b)^2=-1<0,\]
	which does not make sense for real numbers. This finishes.
\end{proof}
\begin{proposition}
	The set $\CC$ with the above operations is a field.
\end{proposition}
\begin{proof}
	We have the following checks.
	\begin{itemize}
		\item The element $0+0i$ is our additive identity. Indeed, one can check that $(0+0i)+(a+bi)=(a+bi)+(0+0i)=a+bi$.
		\item The element $1+0i$ is our multiplicative identity. Indeed, one can check that $(1+0i)(a+bi)=(a+bi)(1+0i)=a+bi$.
		\item Commutativity of addition and multiplication follow from by expansion.
		\item The distributive laws can again be checked by expansion.
		\item The additive inverse of $a+bi$ is $(-a)+(-b)i$.
		\item The multiplicative inverse of $a+bi$ can be found by wishing really hard and writing
		\[\frac1{a+bi}=\frac1{a+bi}\cdot\frac{a-bi}{a-bi}=\frac a{a^2+b^2}-\frac b{a^2+b^2}i.\]
		Then one can check this works.
		\qedhere
	\end{itemize}
\end{proof}
Sometimes we would like to extract our coefficients from our basis.
\begin{definition}[\texorpdfstring{$\op{Re}$}{Re} and \texorpdfstring{$\op{Im}$}{Im}]
	Given $z:=a+bi\in\CC$, we define the operations
	\[\op{Re}z:=a\qquad\text{and}\qquad\op{Im}z:=b.\]
	Importantly, $\op{Re}:\CC\to\RR$ and $\op{Im}:\CC\to\RR$.
\end{definition}
Because we are merely doing basis extraction, it makes sense that these operations will preserve some (additive) structure.
\begin{proposition}
	Fix $z=a+bi$ and $w=c+di$. Then the following.
	\begin{listalph}
		\item $\op{Re}(z+w)=\op{Re}z+\op{Re}w$.
		\item $\op{Im}(z+w)=\op{Im}z+\op{Im}w$.
	\end{listalph}
\end{proposition}
\begin{proof}
	We proceed by direct expansion. Observe
	\[\op{Re}(z+w)=\op{Re}((a+c)+(b+d)i)=a+c=\op{Re}z+\op{Re}w,\]
	and
	\[\op{Im}(z+w)=\op{Im}((a+c)+(b+d)i)=b+d=\op{Im}z+\op{Im}w.\]
	This finishes.
\end{proof}

It also turns out that the complex numbers have a very special transformation.
\begin{definition}[Conjugate]
	Given $z:=a+bi\in\CC$, we define the \textit{complex conjugate} to be $\overline z:=a-bi\in\CC$.
\end{definition}
We promised conjugation would be special, so here are some special things.
\begin{proposition} \label{prop:accessreandim}
	Fix $z=a+bi\in\CC$. Then the following.
	\begin{listalph}
		\item $z+\overline z=2\op{Re}z$.
		\item $z-\overline z=2i\op{Im}z$.
		\item $\overline{\overline z}=z$.
	\end{listalph}
\end{proposition}
\begin{proof}
	We take these one at a time.
	\begin{listalph}
		\item Write $a+bi+\overline{a+bi}=a+bi+a-bi=2a$.
		\item Write $a+bi-\overline{a+bi}=a+bi-(a-bi)=2bi$.
		\item Write $\overline{\overline{a+bi}}=\overline{a-bi}=a+bi$.
		\qedhere
	\end{listalph}
\end{proof}
In fact, more is true.
\begin{proposition}
	Fix $z=a+bi\in\CC$ and $w=c+di\in\CC$. Then the following.
	\begin{listalph}
		\item $\overline{z+w}=\overline z+\overline w$.
		\item $\overline{zw}=\overline z\cdot\overline w$.
	\end{listalph}
\end{proposition}
\begin{proof}
	We take these one at a time.
	\begin{itemize}
		\item Write
		\[\overline{z+w}=(a+c)-(b+d)i=(a-bi)+(c-di)=\overline z+\overline w.\]
		\item Write
		\begin{align*}
			\overline z\cdot\overline w &= (a-bi)(c-di) \\
			&= (ac-bd)-(ad+bc)i \\
			&= \overline{(ac-bd)+(ad+bc)i} \\
			&= \overline{zw}.
		\end{align*}
		This finishes.
		\qedhere
	\end{itemize}
\end{proof}

\subsection{Defining Distance}
Complex conjugation actually gives rise to a notion of size.
\begin{definition}[Norm on \texorpdfstring{$\CC$}{C}]
	Given $z:=a+bi$, we define the \textit{norm function on $\CC$} by
	\[|z|:=\sqrt{a^2+b^2}.\]
\end{definition}
Size actually gives distance.
\begin{definition}[Distance on \texorpdfstring{$\CC$}{C}]
	Given complex numbers $z=a+bi$ and $w=c+di$, we define the \textit{distance} between $z$ and $w$ to be
	\[|z-w|=\sqrt{(a-c)^2+(b-d)^2}.\]
\end{definition}
Here are some examples.
\begin{center}
	\begin{asy}
		unitsize(1cm);
		draw((-1,0)--(3,0));
		draw((0,-1)--(0,3));
		dot("$1$", (1,0), S);
		dot("$i$", (0,1), W);
		dot("$0$", (0,0), SW);
		dot("$1+i$", (1,1), E);
		draw((0,0)--(1,1), dashed+red);
		dot("$1+2i$", (1,2), E);
		draw((1,1)--(1,2), dashed+blue);
		label("$\textrm{Re}$", (3,0), E);
		label("$\textrm{Im}$", (0,3), N);
	\end{asy}
\end{center}
One can ask what is the distance between $0+0i$ and $1+1i$, and we can compute directly that this is $\color{red}\sqrt{1+1}=\sqrt2$. Similarly, the distance between $1+2i$ and $1+i$ is $\color{blue}|(1+2i)-(1+i)|=|i|=1.$ It should agree with our geometric intuition.

We mentioned complex conjugation is involved here, so we have the following lemma.
\begin{lemma}
	Fix $z,w\in\CC$. The following are true.
	\begin{listalph}
		\item $|z|^2=z\overline z$.
		\item $|\op{Re}z|\le|z|$ and $|\op{Im}z|\le|z|$.
		\item $|z|=|\overline z|=|-z|$.
		\item $|z|=0$ if and only if $z=0$.
		\item $|zw|=|z|\cdot|w|$.
	\end{listalph}
\end{lemma}
\begin{proof}
	We take these one at a time. Set $z=a+bi$.
	\begin{listalph}
		\item We have
		\[|z|^2=a^2+b^2=(a+bi)(a-bi)=z\overline z.\]
		Here we have used subtraction of two squares, which one can see when writing $a^2+b^2=a^2-(ib)^2$.
		\item We have $a^2\le a^2+b^2$ and $b^2\le a^2+b^2$ by the Trivial inequality, so
		\[|\op{Re}z|=|a|\le\sqrt{a^2+b^2}=|z|,\]
		and similarly,
		\[|\op{Im}z|=|b|\le\sqrt{a^2+b^2}=|z|.\]
		\item Note
		\[|\overline z|=|a-bi|=\sqrt{a^2+(-b)^2}=\sqrt{a^2+b^2}=|z|,\]
		and
		\[|-z|=|-a-bi|=\sqrt{(-a)^2+(-b)^2}=\sqrt{a^2+b^2}=|z|.\]
		\item From (b), we know that $|\op{Re}z|,|\op{Im}z|\le|z|$, but $|z|=0$ then forces $\op{Re}z=\op{Im}z=0$, so $z=0$.
		\item From (a), we can write $|zw|^2=zw\cdot\overline{zw}$, which will expand out into
		\[z\cdot w\cdot\overline z\cdot\overline w.\]
		We can collect this into $z\overline z\cdot w\overline w=|z|^2|w|^2$. Thus, by (a) again, $|zw|^2=|z|^2|w|^2$. But because all norms must be nonnegative real numbers, we may take square roots to conclude $|zw|=|z|\cdot|w|$.
		\qedhere
	\end{listalph}
\end{proof}
\begin{remark}
	Norms are actually more general constructions. For example, the requirement $|zw|=|z|\cdot|w|$ makes $|\cdot|$ into a ``multiplicative'' norm.
\end{remark}
To finish off, we actually show that our distance function is good: we show the triangle inequality.
\begin{lemma}[Triangle inequality]
	For every $x,y,z\in\CC$, we claim
	\[|z-x|\le|z-y|+|y-z|.\]
\end{lemma}
This claim should be familiar from real analysis. Intuitively, it means that travelling between $z$ and $x$ cannot be made into a shorter trip by taking a detour to some other point $y$ first.
\begin{proof}
	Let $a:=z-y$ and $b:=y-z$ so that $a+b=z-x$. Thus, we are showing that
	\[|a+b|\stackrel?\le|a|+|b|,\]
	which is nicer because it only has two letters. For this, because everything is a nonnegative real numbers, it suffices to show the square of this requirement; i.e., we show
	\[(|a|+|b|)^2-|a+b|^2\stackrel?\ge0.\]
	Fully expanding, it suffices to show
	\[|a|^2+|b|^2+2|a|\cdot|b|-|a+b|^2\stackrel?\ge0.\]
	Expanding out $|w|^2=w\overline w$ for $w\in\CC$, we are showing
	\[a\overline a+b\overline b+2|a|\cdot|b|-(a+b)(\overline a+\overline b)\stackrel?\ge0.\]
	This is nice because the expansion of the rightmost term will induce some cancellation: it expands into $a\overline a+a\overline b+\overline ab+b\overline b$, so we are left with showing
	\[2|a|\cdot|b|-(a\overline b+b\overline a)\stackrel?\ge0.\]
	Note that $\overline ab=\overline{a\overline b}$, so we can collect the final term as $2\op{Re}(a\overline b)$. Similarly, we can write $|a|\cdot|b|=|a|\cdot|\overline b|=|a\overline b|$, so we are showing
	\[2|a\overline b|-2\op{Re}(a\overline b)\ge0,\]
	which is true because the real part does exceed the norm. This finishes.
\end{proof}