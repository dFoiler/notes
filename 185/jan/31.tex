\documentclass[../notes.tex]{subfiles}
\graphicspath{{\subfix{../figs/}}}

\begin{document}
% !TEX root = ../notes.tex

So we are lecturing in-person today. Good morning everyone.
\begin{quot}
	If I don't fall off the stage, I will consider it a major accomplishment.
\end{quot}
Homework 2 is due Friday, the 4th of February. Office hours will occur at the usual times, but they will now occur in-person at Evans 749.

\subsection{Series}
Today we're mostly talking about series, and on Friday we'll talk about continuous functions.
\begin{restatable}[Series]{defihelper}{seriesdefi} \nirindex{Series}
	An \textit{infinite series} over $\CC$ is an infinite sum
	\[S\coloneqq\sum_{n=1}^\infty z_n\]
	where $\{z_n\}_{n\in\NN}\subseteq\CC$ is a sequence of complex numbers.
\end{restatable}
\noindent With respect to series, we really want to know when various series converge so that the series is well-defined.
\begin{restatable}[Converge, diverge]{defihelper}{convergedefi} \nirindex{Converge, diverge}
	Fix a sequence $\{z_n\}_{n\in\NN}\subseteq\CC$ of complex numbers, we define the $m$th partial sum to be
	\[S_m\coloneqq\sum_{n=0}^mz_m.\]
	Then we say that the infinite series \textit{converges} if and only if the sequence $\{S_m\}$ of partial sums converges. Otherwise, we say that $S$ is \textit{divergent}.
\end{restatable}

As usual, we start with some basic examples.
\begin{exe}
	Fix some $z\in\CC$ with $|z|<1$, we define $z_n\coloneqq z^n$. Then we have
	\[S=\sum_{k=0}^\infty z^k=\frac1{1-z}.\]
\end{exe}
\begin{proof}
	Fix some partial sum
	\[S_N\coloneqq\sum_{k=0}^Nz^k=1+z+z^2+\cdots+z^N.\]
	Multiplying by $z$, we see that
	\[zS_n=z+z^2+\cdots+ z^N+z^{N+1}.\]
	It follows that
	\[S_N-zS_N=1-z^{N+1}.\]
	Because $|z|<1$, we have $z\ne1$, so we may write
	\[S_N=\frac1{1-z}-\frac{z^{N+1}}{1-z}.\]
	However, we may note that as $N\to\infty$, the bad term $z^{N+1}$ will have size
	\[\left|z^{N+1}\right|=|z|^{N+1},\]
	which goes to $0$ (because $|z|<1$).\footnote{This is surprisingly annoying to rigorize with an $\varepsilon$-$\delta$ proof, so we won't do so here. The interested can try to use induction to manually bound $|z|^n$ by $\frac cn$ for some $c$.} It follows that
	\[\lim_{N\to\infty}S_N=\frac1{1-z},\]
	which is what we wanted.
\end{proof}
Anyways, here are some basic lemmas.
\begin{lemma}[Divergence test] \label{lem:divtest}
	Suppose that $\{z_n\}_{n\in\NN}$ is a sequence of complex numbers such that $\sum z_n$ converges. Then $z_n\to0$ as $n\to\infty$.
\end{lemma}
\begin{proof}
	Let $S_n$ be the $n$th partial sum so that we are given $S_n\to L$ for some $L\in\CC$. But now we see that
	\[z_{n+1}=\left(\sum_{k=0}^{N+1}z_k\right)-\left(\sum_{k=0}^{N}z_k\right)=S_{n+1}-S_n.\]
	Using limit laws, we see that
	\[\lim_{n\to\infty}z_{n+1}=\lim_{n\to\infty}S_{n+1}-\lim_{n\to\infty}S_n=L-L=0.\]
	Shifting the indices back gives $z_n\to0$ as $N\to\infty$.
\end{proof}
Here is an important example of a divergent series.
\begin{exe}
	We claim that
	\[S=\sum_{k=1}^\infty\frac1k\]
	does not converge.
\end{exe}
\begin{proof}
	We will show that the sequence of partial sums $\{S_n\}_{n=1}^\infty$ is not Cauchy, which will show that the series diverges. Well, observe that
	\[S_{2^{n+1}}-S_{2^n}=\sum_{k=2^n+1}^{2^{n+1}}\frac1k\]
	after cancelling out all of our terms. However, each term in the sum is at least $\frac1{2^{n+1}}$, so we bound
	\[S_{2^{n+1}}-S_{2^n}\ge\frac1{2^{n+1}}\left(2^{n+1}-2^n\right)=\frac12.\]
	We now show that the partial sums are not Cauchy. Fix $\varepsilon$. Supposing for the sake of contradiction that the sequence is Cauchy, there exists $N$ so that $n,m>N$ has
	\[|S_n-S_m|<\frac12.\]
	However, we can find some power of $2$ named $2^r$ which exceeds $N$, in which case we find $2^{r+1},2^r>N$ and
	\[|S_{2^{r+1}}-S_{2^r}|\ge\frac12,\]
	which is our contradiction.
\end{proof}
\begin{remark}
	Because a sequence will converge if and only if its real and imaginary parts do, we can turn a convergence test into a real-number test by taking the real and imaginary parts of the sum.
\end{remark}

\subsection{The Comparison Test}
Recall the comparison test in $\RR$.
\begin{theorem}[Comparison test]
	Fix $\{x_n\}_{n\in\NN},\{y_n\}_{n\in\NN}\subseteq\RR$ sequences of real numbers. Further, suppose that we there exists a positive constant $c>0$ such that $0\le x_n\le cy_n$. Then the following hold.
	\begin{itemize}
		\item If $\sum y_n$ converges, then $\sum x_n$ converges as well.
		\item If $\sum x_n$ diverges, then $\sum y_n$ diverges as well.
	\end{itemize}
\end{theorem}
\begin{proof}
	We appeal to real analysis. The interested can see Theorem 2.1.21 in Eterovi\'c. The main point is to use the Monotone sequence theorem.
\end{proof}
Here is an example.
\begin{exe}
	Fix $s>1$ an integer. Then the series
	\[S=\sum_{k=1}^\infty\frac1{k^s}\]
	converges.
\end{exe}
\begin{proof}
	Because $s$ is an integer, we have $s\ge2$. Namely, $\frac1{k^s}\le\frac1{k^2}$, so by the comparison test it suffices to just show the convergence of
	\[S'\coloneqq\sum_{k=1}^\infty\frac1{k^2}.\]
	For this, we apply some trickery. In particular, for $k>1$, we bound
	\[\frac1{k^2}<\frac1{k(k-1)}=\frac1{k-1}-\frac1k.\]
	In particular,
	\[S'=1+\sum_{k=2}^\infty\frac1{k^2}<1+\sum_{k=2}^\infty\left(\frac1{k-1}-\frac1k\right).\]
	Thus, by the comparison test, it suffices to show the convergence of
	\[T\coloneqq\sum_{k=2}^\infty\left(\frac1{k-1}-\frac1k\right).\]
	But the $n$th partial sum will telescope, giving
	\[T_n\coloneqq\sum_{k=2}^n\left(\frac1{k-1}-\frac1k\right)=1-\frac1n,\]
	so $T_n\to1$ as $n\to\infty$, and $T=1$. It follows that $S'$ is upper-bounded by $1+T\le2$.
\end{proof}

\subsection{Absolute Convergence}
The following kind of convergence is nontrivially stronger, but that makes it better.
\begin{definition}[Absolute convergence]
	Fix a sequence $\{z_n\}_{n\in\NN}\subseteq\CC$ of complex numbers. Then the sum $S\coloneqq\sum z_n$ converges \textit{absolutely} if and only if the series
	\[\sum_{n=0}^\infty|z_n|\]
	also converges. In other words, the partial sums of the above series converges.
\end{definition}
We have the following quick lemma to justify naming this ``convergence.''
\begin{lemma}
	If a series converges absolutely, then the series also converges.
\end{lemma}
\begin{proof}
	The idea is to use the triangle inequality. Fix our series
	\[S\coloneqq\sum_{n=0}^\infty z_n\]
	for which
	\[T\coloneqq\sum_{n=0}^\infty|z_n|\]
	converges. Let $S_n$ be the $n$th partial sum of $S$ and $T_n$ the $n$ the partial sum of $T$.
	
	Our goal is to show that $\{S_n\}_{n\in\NN}$ is Cauchy. Observe $\{T_n\}_{n\in\NN}$ is an increasing sequence of real numbers because $|z|\ge0$ always. To start off our arithmetic, we note that, for $n,m\in\NN$ with $n>mn$, we have
	\[|S_n-S_m|=\left|\sum_{k=m+1}^nz_k\right|,\]
	which by the triangle inequality can be bounded by
	\[|S_n-S_m|\le\sum_{k=m+1}^n|z_k|=T_m-T_n.\]
	But now we can use the fact that $\{T_n\}_{n\in\NN}$ must be Cauchy to finish: for any $\varepsilon>0$, there exists some $N$ such that $n>m>N$ implies $T_m-T_n<\varepsilon$. But then this same $N$ promises $n>m>N$ implies
	\[|S_n-S_m|<T_m-T_n<\varepsilon,\]
	which is what we wanted.
\end{proof}
Here is a surprise tool that will help us later.
\begin{lemma}
	Fix a sequence $\{a_n\}_{n\in\NN}\subseteq\CC$ of nonzero complex numbers. Further, suppose that the sequence $\{a_n\}_{n\in\NN}$ tends to infinity (i.e., $|a_n|\to\infty$ as $n\to\infty$), then for any positive real number $r\in\RR^+$, the series
	\[\sum_{k=0}^\infty\left(\frac r{|a_k|}\right)^k\]
	converges.
\end{lemma}
\begin{proof}
	We need the $a_n$ to be nonzero in order to allow division, so the real puzzle is to determine how to use the fact $|a_n|\to\infty$. Well, there exists some $N$ such that $n>N$ has
	\[|a_n|>2r.\]
	But then $\frac r{|a_n|}<\frac12$ for each $n>N$, so we can use the comparison test as follows: observe that
	\[\sum_{k=0}^\infty\frac1{2^k}\]
	will converge, and there will exist some $c>1$ so that
	\[\frac r{|a_k|}<\frac c{2^k}\]
	for $0\le k\le N$; and then for $n>N$, we get the above inequality anyways as discussed earlier (observe we took $c>1$).
\end{proof}
\begin{quot}
	I can't break math on the first day of class. I can do it later on.
\end{quot}
\begin{lemma}
	Suppose that we have two series $S\coloneqq\sum_{k\in\NN}z_k$ and $T\coloneqq\sum_{k\in\NN}w_k$ are both absolutely convergent. Then the sum
	\[P\coloneqq\sum_{k=0}^\infty\left(\sum_{i+j=k}z_iw_j\right)\]
	is absolutely convergent as well. In fact, $P$ will converge to $ST$.
\end{lemma}
\begin{proof}
	We sketch the result, and the remaining details are in Eterovi\'c. As usual, consider the partial sums
	\[A_n\coloneqq\sum_{k=0}^n|z_k|\qquad\text{and}\qquad B_n=\sum_{k=0}^n|w_k|,\]
	both of which will converge as $n\to\infty$. Brazenly multiplying these together, we see that
	\[A_nB_n=\sum_{i,j=0}^n|z_iw_j|=\sum_{k=0}^n\sum_{\substack{i+j=k\\0\le i,j\le n}}|z_iw_j|+\sum_{k>n}\sum_{\substack{i+j=k\\0\le i,j\le n}}|z_iw_j|.\]
	In the first sum, observe that any time $i+j=k$, we will automatically have $i,j\le k\le n$. It follows that
	\[A_nB_n=\sum_{k=0}^n\left(\sum_{i+j=k}z_iw_j\right)+\underbrace{\sum_{\substack{i+j>n\\0\le i,j\le n}}|z_iw_j|}_{R_n}.\]
	The key claim is that $R_n\to0$. The main idea is that $i+j>n$ implies that $i\ge n/2$ or $j\ge n/2$, so we can write
	\[|R_n|\le\sum_{i=0}^n\sum_{j=n/2}^n|z_iw_j|+\sum_{i=n/2}^n\sum_{j=0}^n|z_iw_j|=\left(\sum_{i=0}^n|z_i|\right)\left(\sum_{j=n/2}^n|w_j|\right)+\left(\sum_{i=n/2}^n|z_i|\right)\left(\sum_{j=0}^n|w_j|\right).\]
	Now, fix any $\varepsilon>0$, and we show there exists $X$ so that $n>X$ has $|R_n|<\varepsilon$. Note $A\coloneqq\sum|z_k|$ and $B\coloneqq\sum|w_k|$ both converge and hence have Cauchy partial sums. Because the partial sums are increasing, we bound
	\[|R_n|\le A\left(\sum_{j=n/2}^n|w_j|\right)+B\left(\sum_{i=n/2}^n|z_i|\right)\]
	So there exists $N$ such that $n>m>N$ has
	\[\sum_{i=m+1}^n|z_i|<\frac\varepsilon {2B}\]
	Similarly there exists $M$ so that $n>m>M$ has
	\[\sum_{j=m+1}^n|w_j|<\frac\varepsilon {2A},\]
	from which it follows that $n>n/2>\max\{N,M\}$ will have
	\[|R_n|\le A\cdot\frac\varepsilon {2A}+B\cdot\frac\varepsilon {2B}=\varepsilon,\]
	which finishes.

	Now, because $R_n\to0$, we see
	\[\lim_{n\to\infty}\sum_{k=0}^n\left(\sum_{i+j=k}|z_i|\cdot|w_j|\right)=\lim_{n\to\infty}A_nB_n-\lim_{n\to\infty}R_n,\]
	which does indeed converge, so indeed the series
	\[\sum_{k=0}^\infty\left(\sum_{i+j=k}|z_i|\cdot|w_j|\right)\]
	will converge. By the comparison test (using the triangle inequality), it follows that
	\[P=\sum_{k=0}^\infty\left(\sum_{i+j=k}z_iw_j\right)\]
	will also absolutely converge.

	To show that $P$ converges to $ST$, we observe that the difference of the $n$th partial sum is
	\[P_n-S_nT_n=\sum_{k=0}^n\left(\sum_{i+j=k}z_iw_j\right)-\sum_{i,j=0}^nz_iw_j=\sum_{k=0}^n\left(\sum_{i+j=k}z_iw_j\right)-\sum_{k=0}^n\left(\sum_{i+j=k}^nz_iw_j\right)+\sum_{\substack{0\le i,j\le n\\i+j<n}}z_iw_j,\]
	so
	\[P_n-S_nT_n=\sum_{\substack{0\le i,j\le n\\i+j<n}}z_iw_j.\]
	But by the triangle inequality, we see $|P_n-S_nT_n|\le R_n$, so $P_n-S_nT_n\to0$ as $n\to\infty$. It follows $P_n$ and $S_nT_n$ have the same limit as $n\to\infty$ (which exists because $S_n$ and $T_n$ have a limit). So indeed, $P=ST$.
\end{proof}
% \begin{ex}
% 	It is not the case that the product of two convergent series will be convergent.
% \end{ex}
\end{document}