% !TEX root = ../notes.tex

We're reviewing set theory today.

\subsection{Set Theory Notation}
We have the following definitions.
\begin{itemize}
	\item $\emp$ means the empty set.
	\item $a\in X$ means that $a$ is an element of the set $X$.
	\item $A\subseteq B$ means that $A$ is a subset of $B$.
	\item $A\subsetneq B$ means that $A$ is a proper subset of $B$.
	\item $A\cup B$ consists of the elements which are in at least one of $A$ or $B$.
	\item $A\cap B$ consists of the elements which are in both $A$ and $B$.
	\item $A\setminus B$ consists of the elements of $A$ which are not in $B$.
	\item Two sets $A$ and $B$ are \textit{disjoint} if and only if $A\cap B=\emp$.
	\item Given a set $X$, we define $\mathcal P(X)$ to be the set of all subsets of $X$.
	\item $|X|=\#X$ is the cardinality of $X$, or (roughly speaking) the number of elements of $X$.
\end{itemize}
As an example of unwinding notation, we have the following.
\begin{proposition}[De Morgan's Laws]
	Fix $\mathcal S\subseteq\mathcal P(X)$ a collection of subsets of a set $X$. Then
	\[X\mathbin{\bigg\backslash} \bigcap_{S\in\mathcal S}S=\bigcup_{S\in\mathcal S}(X\setminus S)\qquad\text{and}\qquad X\mathbin{\big\backslash}\bigcup_{S\in\mathcal S}S=\bigcap_{S\in\mathcal S}(X\setminus S).\]
\end{proposition}
\begin{proof}
	We take these one at a time.
	\begin{itemize}
		\item Note $a\in X\setminus\bigcap\mathcal S$ if and only if $a\in X$ and $a\notin\bigcap\mathcal S$. However, $a\notin\bigcap\mathcal S$ is merely saying that $a$ is not in all the sets $S\in\mathcal S$, which is equivalent to saying $a\notin S$ for one of the $S\in\mathcal S$.

		Thus, this is equivalent to saying $a\in X$ while $a\notin S$ for some $S\in\mathcal S$, which is equivalent to $a\in\bigcup_{S\in\mathcal S}(X\setminus S)$.
		\item Note $a\in X\setminus\bigcup\mathcal S$ if and only if $a\in X$ and $a\notin\bigcup\mathcal S$. However, $a\notin\bigcup\mathcal S$ is merely saying that $a$ is not in any of the sets $S\in\mathcal S$, which is equivalent to saying $a\notin S$ for each of the $S\in\mathcal S$.

		Thus, this is equivalent to saying $a\in X$ while $a\notin S$ for each $S\in\mathcal S$, which is equivalent to $a\in\bigcap_{S\in\mathcal S}(X\setminus S)$.
		\qedhere
	\end{itemize}
\end{proof}

\subsection{Some Conventions}
In this class, we take the following names of standard sets.
\begin{itemize}
	\item $\NN=\{0,1,2,\ldots\}$ is the set of natural numbers. Importantly, $0\in\NN$.
	\item $\NN^+=\{1,2,3,\ldots\}$ is the set of positive integers.
	\item $\ZZ=\{\ldots,-2,-1,0,1,2,\ldots\}$ is the set of integers.
	\item $\QQ=\{p/q:p,q\in\ZZ\text{ and }q\ne\}$ is the set of rationals.
	\item $\RR$ is the set of real numbers. We will not specify a construction here; see any real analysis class.
	\item $\RR^\times=\{x\in\RR:x\ne\}$ is the nonzero real numbers.
	\item $\RR^+=\{x\in\RR:x>0\}$ is the positive real numbers.
	\item $\RR_{\ge0}=\{x\in\RR:x\ge0\}$ is the nonnegative real numbers.
	\item $\RR_{\le0}=\{x\in\RR:x\le0\}$ is the nonpositive real numbers.
	\item $\CC$ is the complex numbers.
	\item $\CC^\times=\{z\in\CC:z\ne0\}$ is the set of nonzero complex numbers.
\end{itemize}

\subsection{Relations}
Let's review some set theory definitions.
\begin{definition}[Cartesian product]
	Given two sets $A$ and $B$, we define the \textit{Cartesian product} $A\times B$ to be the set of ordered pairs $(a,b)$ such that $a\in A$ and $b\in B$.
\end{definition}
\begin{definition}[Binary relation]
	A \textit{binary relation} on $A$ is any subset $R\subseteq A^2:=A\times A$. We may sometimes notate $(x,y)\in R$ by $xRy$, read as ``$x$ is related to $y$.''
\end{definition}
\begin{example}
	Equality is a binary relation on any set $A$; namely, it is the subset $\{(a,a):a\in A\}$.
\end{example}
The best relations are equivalence relations.
\begin{definition}[Equivalence relation]
	An \textit{equivalence relation} on $A$ is a binary relation $R$ satisfying the following three conditions.
	\begin{itemize}
		\item Reflexive: each $x\in A$ has $(x,x)\in R$.
		\item Symmetric: each $x,y\in A$ has $(x,y)\in R$ implies $(y,x)\in R$.
		\item Transitive: each $x,y,z\in A$ has $(x,y)\in R$ and $(y,z)\in R$ implies $(x,z)\in R$.
	\end{itemize}
\end{definition}
Equivalence relations are nice because they allow us to partition the set into ``equivalence classes.''
\begin{defi}[Equivalence class]
	Fix $A$ a set and $R\subseteq A^2$ an equivalence relation. Then, for given $x\in A$, we define
	\[[x]_R:=\{y\in A:(x,y)\in R\}\]
	to be the \textit{equivalence class} of $x$.
\end{defi}
The hope is that equivalence classes partition the set. What is a partition?
\begin{definition}[Parition]
	A \textit{partition} of a set $A$ is a collection of nonempty subsets $\mathcal S\subseteq\mathcal P(A)$ of $A$ such that any two distinct $S_1,S_2\in\mathcal S$ are disjoint while $A=\bigcup_{S\in\mathcal S}S$.
\end{definition}
And now let's manifest our hope.
\begin{lemma}
	Equivalence relations are in one-to-one correspondence with partitions of $A$.
\end{lemma}
\begin{proof}
	Given an equivalence relation $R$, we define the collection
	\[\mathcal S(R)=\{[x]_R:x\in A\}.\]
	We claim that $R\mapsto\mathcal S(R)$ is our needed bijection. We have the following checks.
	\begin{itemize}
		\item Well-defined: observe that $\mathcal S(R)$ does partition $A$: if we have $[x]_R,[y]_R\in\mathcal S$, then $[x]_R\cap[y]_R\ne\emp$ implies there is some $z$ with $(x,z)\in R$ and $(z,y)\in R$, so $x\in[y]_R$ and then $[x]_R\subseteq[y]_R$ follows. So by symmetry, $[y]_R\subseteq[x]_R$ as well, so we finish the disjointness check.
	
		Further, we see that
		\[A=\bigcup_{x\in A}\{x\}\subseteq\bigcup_{x\in A}[x]_R\subseteq A\]
		because $x\in[x]_R$, so indeed the equivalence classes cover $A$.

		\item Injective: suppose $R_1$ and $R_2$ have $\mathcal S(R_1)=\mathcal S(R_2)$. We show that $R_1\subseteq R_2$, and $R_2\subseteq R_1$ will follow by symmetry, finishing.
		
		We notice that, for any $\mathcal S$ partitioning $A$, being a partition, will have exactly one subset which contains $x$. But for $\mathcal S(R)$ for an equivalence relation $R$, we see $x\in[x]_R\in\mathcal S(R)$, so this equivalence class must be the one.

		So because $[x]_{R_1}$ and $[x]_{R_2}$ are the only subsets of $\mathcal S(R_1)$ and $\mathcal S(R_2)$ containing $x$ (respectively), we must have $[x]_{R_1}=[x]_{R_2}$. Thus, $(x,y)\in R_1$ implies $y\in[x]_{R_1}=[x]_{R_2}$ implies $(x,y)\in R_2$.

		\item Surjective: suppose that $\mathcal S$ is a partition of $A$. As noted above, each $x\in A$ is a member of exactly one set $S\in\mathcal S$, which we call $[x]$. Then we define $R\subseteq A^2$ by $(x,y)\in R$ if and only if $y\in[x]$. One can check that this is an equivalence relation, which we will not do here in detail.\footnote{Note $x\in[x]$ by definition of $[x]$. If $y\in[x]$, then note $y\in[y]$ as well, so $[x]=[y]$ is forced by uniqueness, so $x\in[y]$. If $y\in[x]$ and $z\in[y]$, then again by uniqueness $[x]=[y]=[z]$, so $z\in[x]$ follows.}

		The point is that
		\[[x]_R=\{y:(x,y)\in R\}=\{y:y\in[x]\}=[x],\]
		so $\mathcal S(R)=\mathcal S$. So our mapping is surjective.
		\qedhere
	\end{itemize}
\end{proof}
We continue our discussion.
\begin{definition}[Quotient set]
	Given an equivalence relation $R\subseteq A^2$, we define the \textit{quotient set $A/R$} is the set of equivalence classes of $R$. In other words,
	\[A/R=\{[x]_R:x\in A\}.\]
\end{definition}
Intuitively, the quotient set is the set where we have gone ahead and identified the elements which are ``similar'' or ``related.''

We would like a more concrete way to talk about equivalence classes, for which we have the following.
\begin{definition}[Representatives]
	Given an equivalence relation $R\subseteq A^2$, we say that $C\subseteq A$ is a \textit{set of representatives of $R$-equivalence classes of $A$} if and only if $C$ consists of exactly one element from each equivalence class in $A/R$.
\end{definition}

\subsection{Functions}
To finish off, we discuss functions.
\begin{definition}[Functions]
	A \textit{function} $f:X\to Y$ is a relation $f\subseteq X\times Y$ satisfying the following.
	\begin{itemize}
		\item For each $x\in X$, there is some $y\in Y$ such that $(x,y)\in f$. Intuitively, each $x\in X$ goes somewhere.
		\item For each $x\in X$ and given some $y_1,y_2\in Y$ such that $(x,y_1),(x,y_2)\in f$, then $y_1=y_2$. Intuitively, each $x\in X$ goes to at most one place.
	\end{itemize}
	We will write $f(x)=y$ as notational sugar for $(x,y)\in f$. Note this equality is legal because the value $y$ with $(x,y)\in f$ is uniquely given.
\end{definition}
We would like to create new functions from old. Here are two ways to do this.
\begin{definition}[Restriction]
	Given a function $f:X\to Y$ and a subset $A\subseteq X$, we define
	\[f|_A=\{(x,y)\in f:x\in A\}\subseteq A\times Y\]
	to be a function $f|_A:A\to Y$.
\end{definition}
We will not check that $f|_A$ is actually a function; it is, roughly speaking inherited from $f$.
\begin{definition}
	Given two functions $f:X\to Y$ and $g:Y\to Z$, we define the \textit{composition} of $f$ and $g$ to be some function $g\circ f:X\to Z$ defined by
	\[(g\circ f)(x):=g(f(x)).\]
\end{definition}
Again, we will not check that this makes a function; it is.

Functions can also help create new sets.
\begin{definition}[Image]
	Given a function $f:X\to Y$, we define the \textit{image} of $f$ to be
	\[\im f=f(X):=\{y\in Y:\text{there is }x\in X\text{ such that }f(x)y\}.\]
	Namely, $\im f$ consists of all elements hit by someone in $X$ hit by $f$.
\end{definition}
\begin{definition}[Fiber, pre-image]
	Given a function $f:X\to Y$ and some $y\in Y$, we define the \textit{fiber} of $f$ over $y$ to be
	\[f^{-1}(y)=\{x\in X:f(x)=y\}\subseteq X.\]
	In general, we define the \textit{pre-image} of a subset $A\subseteq X$ to be
	\[f^{-1}(A):=\{x\in A:f(x)\in A\}=\bigcup_{a\in A}\{x\in A:f(x)=a\}=\bigcup_{a\in A}f^{-1}(a).\]
\end{definition}

Some functions have nicer properties than others.
\begin{definition}[Inj-, sur-, bijective]
	Fix a function $f:X\to Y$. We have the following.
	\begin{itemize}
		\item Then $f$ is \textit{injective} or \textit{one-to-one} if and only if, given $x_1,x_2\in X$, $f(x_1)=f(x_2)$ implies $x_1=x_2$.
		\item Then $f$ is \textit{surjective} or \textit{onto} if and only if $\im f=Y$. In other words, for each $y\in Y$, there exists $x\in X$ with $f(x)=y$.
		\item Then $f$ is \textit{bijective} if and only if it is both injective and surjective.
	\end{itemize}
\end{definition}
Here is an example.
\begin{definition}[Identity]
	For a given set $X$, the function $\id_X:X\to X$ defined by $\id_X(x):=x$ is called the \textit{identity function}.
\end{definition}
For completeness, here are the checks that $\id_X$ is bijective.
\begin{itemize}
	\item Injective: given $x_1,x_2\in X$, we see $\id_X(x_1)=\id_X(x_2)$ implies $x_1=\id_X(x_1)=\id_X(x_2)=x_2$.
	\item Surjective: given $x\in X$, we see that $x\in\im\id_X$ because $x=\id_X$.
\end{itemize}

We leave with some lemmas, to be proven once in one's life.
\begin{lemma}
	Fix finite sets $X$ and $Y$ such that $\#X=\#Y$. Then a function $f:X\to Y$ is bijective if and only if it is injective or surjective.
\end{lemma}
\begin{proof}
	Certainly if $f$ is bijective, then it is both injective and surjective, so there is nothing to say.
	
	The reverse direction is harder. We proceed by induction on $\#X=\#Y$. If $\#X=\#Y=0$, then $X=Y=\emp$, and all functions $f:\emp\to\emp$ are vacuously bijective: for injective, note that any $x_1,x_2\in\emp$ have $x_1=x_2$; for surjective, note that any $x\in\emp$ has $f(x)=x$.

	Otherwise, $\#X=\#Y>0$. We have two cases.
	\begin{itemize}
		\item Take $f$ injective; we show $f$ is surjective. In this case, $\#X>0$, so choose some $a\in X$. Note that $x\in X$ with $x\ne a$ will have $f(x)\ne f(a)$ by injectivity, so we may define the restriction
		\[f|_{X\setminus\{a\}}:X\setminus\{a\}\to Y\setminus\{f(a)\}.\]
		Observe that $f|_{X\setminus\{a\}}$ is injective because $f$ is: if $x_1,x_2\in X\setminus\{a\}$ have
		\[f(x_1)=f|_{X\setminus\{a\}}(x_1)=f|_{X\setminus\{a\}}(x_2)=f(x_2),\]
		then $x_1=x_2$ follows.
		
		Now, $\#(X\setminus\{a\})=\#(Y\setminus\{f(a)\})=\#X-1$, so by induction $f|_{X\setminus\{a\}}$ will be bijective because it is injective. In particular, $f$ by way of $f|_{X\setminus\{a\}}$ fully hits $Y\setminus\{f(a)\}$ in its image, so because $f(a)\in\im f$ as well, we conclude $\im f=Y$. So $f$ is surjective.

		\item Take $f$ surjective; we show $f$ is injective. Define a function $g:Y\to X$ as follows: for each $y\in Y$, the surjectivity of $f$ promises some $x\in X$ such that $f(x)=y$, so choose any such $x$ and define $g(y):=x$.\footnote{Technically we are using the Axiom of Choice here. One can remove this with an induction because all sets are finite, but I won't bother.} Observe that $f(g(y))=y$ by construction.

		Now, we notice that $g$ is injective: if $y_1,y_2\in Y$ have $g(y_1)=g(y_2)$, then $y_1=f(g(y_1))=f(g(y_2))=y_2$. So the previous case tells us that $g$ is in fact bijective.

		So now choose any $x_1,x_2\in X$ such that $f(x_1)=f(x_2)$. The surjectivity of $f$ promises some $y_1,y_2\in Y$ such that $g(y_1)=x_1$ and $g(y_2)=x_2$, so we see that
		\[x_1=g(y_1)=g(f(g(y_1)))=g(f(x_1))=g(f(x_2))=g(f(g(y_2)))=g(y_2)=x_2,\]
		proving our injectivity.
		\qedhere
	\end{itemize}
\end{proof}
\begin{lemma}
	Fix $f:X\to Y$ a bijective function. Then there is a unique function $g:Y\to X$ such that $f\circ g=\id_Y$ and $g\circ f=\id_X$.
\end{lemma}
\begin{proof}
	We show existence and uniqueness separately.
	\begin{itemize}
		\item We show existence. Note that, because $f:X\to Y$ is surjective, each $y\in Y$ has some $x\in X$ such that $f(x)=y$. In fact, this $x\in X$ is uniquely defined because $f(x_1)=f(x_2)$ implies $x_1=x_2$, so we may define $g(y)$ as the value $x$ for which $f(x)=y$.

		By construction, $f(g(y))=y$, so $f\circ g=\id_Y$. Additionally, we note that, given any $x\in X$, the value $x_0$ for which $f(x)=f(x_0)$ is $x=x_0$ by the injectivity, so $g(f(x))=x$. Thus, $g\circ f=\id_X$, as claimed.

		\item We show uniqueness. Suppose that we have two functions $g_1,g_2:Y\to X$ which satisfy
		\[f\circ g_1=f\circ g_2=\id_Y\qquad\text{and}\qquad g_1\circ f=g_2\circ f=\id_X.\]
		Then we see that
		\[g_1=g_1\circ\id_Y=g_1\circ (f\circ g_2)=(g_1\circ f)\circ g_2=\id_X\circ g_2=g_2,\]
		where we have used the fact that function composition associates. This finishes.
		\qedhere
	\end{itemize}
\end{proof}