\documentclass[../notes.tex]{subfiles}
\graphicspath{{\subfix{../figs/}}}

\begin{document}
% !TEX root = ../notes.tex

Good morning everyone. We are doing review today.

\subsection{Review Highlights}
Here are some of the answers to questions asked in class.
\begin{itemize}
	\item The midterm will be like the practice midterm.
	\item You do not need to be stressed about the midterm.
	\item Things proven in class we will not be asked to prove on the midterm.
	\item We will probably will not be asked to do anything involving summation by parts.
	\item Professor Morrow will not curve downward.
	\item Please write more on the exam for the sake of partial credit.
	\item For words with multiple definitions (like continuity and compactness), the first definition is preferred, though other definitions will likely be accepted.
	\item We may cite facts from real analysis, which is a requirement for this class; e.g., $[0,1]$ is compact.
	\item Lemmas elided from class we will not be responsible for. Essentially, please know the things on the review.
	\item Please know the definitions of things is important. They will be graded fairly harshly because these are critical to know to going forwards.
\end{itemize}
Let's do a practice problem.
\begin{exe}
	Find the possible functions $v(x,y)\colon \RR^2\to\RR$ such that
	\[f(z)=f(x+iy)=x^2-y^2+iv(x,y)\]
	is entire and $f(0)=0$.
\end{exe}
\begin{proof}[Proof 1]
	The point is to use the Cauchy--Riemann equations. We set $u(x,y)\coloneqq x^2-y^2$ so that $f(x+yi)=u(x,y)+iv(x,y)$. If we want this to be differentiable, we want
	\[u_x(x,y)=2x=v_y(x,y)\]
	by \autoref{thm:crnecessary}. This means that $v(x,y)=2xy+h(x)$ for some function $h(x)\colon \RR\to\RR$. Again, we note
	\[u_y(x,y)=-2y=-v_x(x,y)=-2y-h'(x),\]
	so we want $h'(x)=0$. So $h$ is a constant function, so we set $h(x)=c$ for some $c\in\RR$.

	It remains to determine $c$. Well, so far the story is that
	\[f(x+iy)=x^2-y^2+i(2xy+c).\]
	Plugging in $x=y=0$ forces $c=0$, so we see that we get $\boxed{f(x+iy)=x^2-y^2-i\cdot 2xy}$.
\end{proof}
\begin{remark}
	The current form of the answer is fine: we do not have to simplify in terms of $z$ or something. More generally, we will not have to spend large amounts of time simplifying on the exam.
\end{remark}
Let's present another proof.
\begin{proof}[Proof 2]
	The point is to use the $x$ information to fully piece together $f'(z)$. As before, set, $u(x,y)=x^2-y^2$. Namely, the Cauchy--Riemann equations promise
	\[f'(z)=f'(x+yi)=u_x(x,y)+iv_x(x,y)=u_x(x,y)-iu_y(x,y).\]
	Taking partial derivatives of $u$ implies that
	\[f(z)=2x-i(-2y)=2x+i\cdot 2y=2(x+yi)=2z.\]
	So from here, we can take the ``antiderivative'' (i.e., guess) that $f(z)=z^2+c$. Lastly, plugging in $f(0)=0$, we get $c=0$, so $\boxed{f(z)=z^2}$.
\end{proof}
\begin{remark}
	We can rigorize that this is the only possible solution because any other solution $g(z)$ must have $g(z)-z^2$ with constant derivative $0$, from which we can argue that $g(z)-z^2$ is constant using the Cauchy--Riemann equations and the fact that $\CC$ is path-connected. To be explicit, we are using \autoref{cor:getconstant}.
\end{remark}
\end{document}