% !TEX root = ../notes.tex

The wheel of time marches on. Today, we actually start talking about complex analysis.

\subsection{Differentiability}
We are going to talk about holomorphic functions.
\begin{convention}
	We set $\Omega$ to be some open subset of $\CC$.
\end{convention}
This gives the following definition.
\begin{restatable}[Differentiable]{definition}{complexdiffdefi}
	Fix an open subset $\Omega\subseteq\CC$ and $f\colon \Omega\to\CC$ a function. Then $f$ is \textit{complex differentiable at} $z_0\in\Omega$ \textit{with derivative} $\alpha\in\CC$ if and only if
	\[\lim_{h\to0}\frac{f(z_0+h)-f(z_0)}h=\alpha.\]
	We write this as $f'(z_0)=\alpha$.
\end{restatable}
\noindent If $f'$ is itself a differentiable function, then $f$ would be ``twice'' differentiable, and we denote this function by $f''$. In general, if $f$ can be differentiated $n$ times, we denote the corresponding function by $f^{(n)}$.
\begin{warn}
	In the definition of complex differentiability, we are taking the limit with $h\in\CC$, not $h\in\RR$. This will make complex differentiability significantly more structured.
\end{warn}
Differentiability gives rise to the following definition.
\begin{restatable}[Holomorphic, entire]{defi}{holodefi}
	Fix an open subset $\Omega\subseteq\CC$ and $f\colon \Omega\to\CC$ a function. Then $f$ is \textit{holomorphic on $\Omega$} if and only if $f$ is complex differentiable at each $z_0\in\CC$. If $\Omega=\CC$, then we say $f$ is \textit{entire}.
\end{restatable}
Here is a small usual lemma.
\begin{lemma}
	Fix an open subset $\Omega\subseteq\CC$ and $f\colon \Omega\to\CC$ a function. Then if $f$ is differentiable at $z_0\in\Omega$, then $f$ is continuous at $z_0\in\CC$.
\end{lemma}
\begin{proof}
	We compute that
	\begin{align*}
		\lim_{z\to z_0}\big(f(z)-f(z_0)\big) &= \lim_{z\to z_0}\frac{f(z)-f(z_0)}{z-z_0}\cdot\lim_{z\to z_0}(z-z_0) \\
		&\stackrel*= \lim_{h\to0}\frac{f(z_0+h)-f(z_0)}h\cdot\lim_{z\to z_0}(z-z_0) \\
		&= f'(z_0)\cdot0 \\
		&= 0.
	\end{align*}
	It follows by rearrangement that $\lim_{z\to z_0}f(z)=f(z_0)$, which is what we wanted. Notably, $\stackrel*=$ sets $h\coloneqq z-z_0$.
\end{proof}

\subsection{Basic Properties}
As usual, differentiable functions have an arithmetic.
\begin{proposition} \label{prop:computederivs}
	Fix an open subset $\Omega\subseteq\CC$ and $f,g\colon \Omega\to\CC$ functions differentiable at $z_0\in\CC$.
	\begin{listalph}
		\item We have that $(af+bg)'(z_0)=af'(z_0)+bg'(z_0)$, where $a,b\in\CC$.
		\item We have that $(fg)'(z_0)=f'(z_0)g(z_0)+f(z_0)g'(z_0)$.
		\item If $g'(z_0)\ne0$, then
		\[(f/g)'(z_0)=\frac{f'(z_0)g(z_0)-f(z_0)g'(z_0)}{g(z_0)^2}.\]
	\end{listalph}
\end{proposition}
\begin{proof}
	We copy the proofs from real analysis.
	\begin{listalph}
		\item We check that
		\begin{align*}
			\lim_{h\to0}\frac{(af+bg)(z_0+h)+(af+bg)(z_0)}h &= a\cdot\lim_{h\to0}\frac{f(z_0+h)+f(z_0)}h+b\cdot\lim_{h\to0}\frac{g(z_0+h)-g(z_0)}h \\
			&= a\cdot f'(z_0)+b\cdot g'(z_0),
		\end{align*}
		which is what we wanted.
		\item The key idea is to add and subtract $f(z_0)g(z_0+h)$. Indeed, we see
		\begin{align*}
			\lim_{h\to0}\frac{(fg)(z_0+h)-(fg)(z_0)}{h} &= \lim_{h\to0}\frac{f(z_0+h)g(z_0+h)-f(z_0)g(z_0+h)}{h} \\
			&\qquad+\lim_{h\to0}\frac{f(z_0)g(z_0+h)-f(z_0)g(z_0)}h \\
			&= \left(\lim_{h\to0}\frac{f(z_0+h)-f(z_0)}{h}\right)\left(\lim_{h\to0}g(z_0+h)\right) \\
			&\qquad+f(z_0)\left(\lim_{h\to0}\frac{g(z_0+h)-g(z_0)}h\right) \\
			&= f'(z_0)g(z_0)+f(z_0)g'(z_0),
		\end{align*}
		which is what we wanted.
		\item This will follow from applying the product rule to $f\cdot\frac1g$, where we can compute the derivative of $\frac1g$ by the chain rule. We refer to Eterovi\'c's notes for the details.
		\qedhere
	\end{listalph}
\end{proof}
\begin{remark}[Nir]
	Technically part (c) will require us to compute the derivative of $f(z)\coloneqq \frac1z$ for $z\ne0$ to finish the proof. Well, for any $z\in\CC\setminus\{0\}$, we see that
	\[\frac{f(z+h)-f(z)}h=\frac{\frac1{z+h}-\frac1z}{h}=\frac{z-(z+h)}{hz(z+h)}=-\frac1{z(z+h)}.\]
	Taking $h\to0$ reveals that the derivative is in fact $f'(z)=-\frac1{z^2}$.
\end{remark}
Let's give some examples of entire functions.
\begin{exe}
	Fix $n$ some positive integer. We show that the function $f(z)\coloneqq z^n$ is entire with derivative $f'(z)\coloneqq nz^{n-1}$.
\end{exe}
\begin{proof}
	We employ the usual proof involving the binomial theorem. Note that
	\[f(z+h)=(z+h)^n=\sum_{k=0}^n\binom nkz^{n-k}h^k,\]
	so
	\[\frac{f(z+h)-f(z)}h=\sum_{k=1}^n\binom nkz^{n-k}h^{k-1},\]
	where notably the $k=0$ term was killed by the $-f(z)$. Thus,
	\[\lim_{h\to0}\frac{f(z+h)-f(z)}h=\sum_{k=1}^n\binom nkz^{n-k}\left(\lim_{h\to0}h^{k-1}\right),\]
	but all terms except $k=1$ will now vanish as $h\to0$, so we are left with $nz^{n-1}$ as our limit.
\end{proof}
\begin{remark}[Nir]
	One could also show this by induction, using the product rule.
\end{remark}
\begin{corollary}
	Any polynomial function is entire.
\end{corollary}
\begin{proof}
	Polynomials are (finite) linear combinations of the monomials $f_n(z)\coloneqq z^n$, so this follows from combining the above two results.
\end{proof}

\subsection{Advanced Properties}
We also have a notion of L'H\^opital's rule.
\begin{proposition}[L'H\^opital's rule]
	Fix $\Omega\subseteq\CC$ an open subset with $f,g\colon \Omega\to\CC$ holomorphic functions. Then, given $z_0\in\Omega$ with $f(z_0)=g(z_0)=0$ while $g'(z_0)\ne0$, we have that
	\[\lim_{z\to z_0}\frac{f(z)}{g(z)}=\frac{f'(z_0)}{g'(z_0)}.\]
\end{proposition}
\begin{proof}
	Note that, because $f(z_0)=g(z_0)=0$, we see that
	\[f'(z_0)=\lim_{z\to z_0}\frac{f(z)}{z-z_0}\qquad\text{and}\qquad g'(z_0)=\lim_{z\to z_0}\frac{g(z)}{z-z_0}.\]
	Dividing, we see that
	\[\lim_{z\to z_0}\frac{f(z)}{g(z)}=\lim_{z\to z_0}\frac{f(z)/(z-z_0)}{g(z)/(z-z_0)}=\lim_{z\to z_0}\frac{f'(z_0)}{g'(z_0)}=\frac{f'(z_0)}{g'(z_0)},\]
	which is what we wanted.
\end{proof}
\begin{remark}[Nir]
	The above proof technically does not work because we have not ruled out the possibility that $g$ might vanish arbitrarily close to $z_0$, thus making the limits not actually make sense. We will not fix this problem, but we will remark that a holomorphic function will only have finitely many zeroes on a compact set, which we could use to create a neighborhood for $z_0$ on which $g$ doesn't vanish.
\end{remark}
And here is our chain rule.
\begin{proposition}[Chain rule]
	Fix $\Omega$ and $U$ open subsets of $\CC$ with functions $f\colon \Omega\to U$ and $g\colon U\to\CC$. Further, suppose that $f$ is differentiable at $z_0\in\Omega$ and that $g$ is differentiable at $f(z_0)\in U$. Then $(g\circ f)$ is differentiable at $z_0$ with derivative
	\[(g\circ f)'(z_0)=g'(f(z_0))f'(z_0).\]
\end{proposition}
\begin{proof}
	This proof is long, so we will try to be brief. The main idea is to consider the auxiliary function $r\colon U\setminus\{f(z_0)\}\to\CC$ defined by
	\[r(w)\coloneqq \frac{g(w)-g(f(z_0))}{w-f(z_0)}-g'(f(z_0)).\]
	We extend $r$ to $f(z_0)$ by setting $r(f(z_0))\coloneqq 0$. Now, the differentiability of $g$ at $f(z_0)$ implies that
	\[\lim_{z\to z_0}\frac{g(z)-g(f(z_0))}{z-z_0}=g'(f(z_0)),\]
	so in particular rearranging implies that $r$ is continuous on at $f(z_0)\in U$.

	The reason we used the letter $r$ is that we should think of $r$ as a remainder term. Indeed, we see
	\[g(w)-g(f(z_0))=g'(f(z_0))(w-f(z_0))+r(w)(w-f(z_0)).\]
	Plugging in $w=f(z)$, we get
	\[g(f(z))-g(f(z_0))=g'(f(z_0))(f(z)-f(z_0))+r(f(z))(f(z)-f(z_0)),\]
	so
	\[\frac{g(f(z))-g(f(z_0))}{z-z_0}=g'(f(z_0))\cdot\frac{f(z)-f(z_0)}{z-z_0}+r(f(z))\cdot\frac{f(z)-f(z_0)}{z-z_0}.\]
	Sending $z\to z_0$ makes the rightmost term vanish by continuity because $r(f(z_0))=0$ and the limit is $f'(z_0)$, so we are left with
	\[(g\circ f)'(z_0)=g'(f(z_0))f'(z_0),\]
	which is what we wanted.
\end{proof}
\begin{remark}[Nir]
	Let's complete the proof of quotient rule. Note that the derivative of $\frac1{g(z)}$ will be, by the chain rule, $-\frac1{g(z)^2}\cdot g'(z)$. Thus, the derivative of $\frac{f(z)}{g(z)}=f(z)\cdot\frac1{g(z)}$ will be
	\[f'(z)\cdot\frac1{g(z)}-f(z)\cdot\frac{g'(z)}{g(z)^2}=\frac{f'(z)g(z)-f(z)g'(z)}{g(z)^2}.\]
\end{remark}
And we finish with a result which is less common in real analysis, essentially saying that differentiable functions are ``approximately'' linear.
\begin{proposition}[Carath\'eodory] \label{prop:cara}
	Fix $\Omega\subseteq\CC$ an open subset with a function $f\colon \Omega\to\CC$ and point $z_0\in\Omega$. Then $f$ is differentiable at $z_0$ if and only if there exists a function $h\colon \Omega\to\CC$ which is continuous at $z_0$ such that
	\[f(z)-f(z_0)=h(z)(z-z_0).\]
	In particular, $h(z_0)=f'(z_0)$.
\end{proposition}
\begin{proof}
	We show the directions independently.
	\begin{itemize}
		\item Suppose $f$ is differentiable at $z_0$. We construct the function $h$ manually. We define
		\[h(z)\coloneqq \begin{cases}
			(f(z)-f(z_0))/(z-z_0) & z\in\Omega\setminus\{z_0\}, \\
			f'(z_0) & z=z_0.
		\end{cases}\]
		In particular, we note that $h$ is continuous at $z_0$ because $h(z)\to f'(z_0)$ as $z\to z_0$ by differentiability of $f$.
		\item Suppose $h$ is such a function. Then
		\[\lim_{z\to z_0}\frac{f(z)-f(z_0)}{z-z_0}=\lim_{z\to z_0}h(z)=h(z_0)\]
		by continuity. Formally, the first equality is holding for the limit in $\Omega\setminus\{z_0\}$, and the second equality is continuity for $h|_{\Omega\setminus\{z_0\}}$.
	\end{itemize}
	To finish, we note that the second part shows that $h(z_0)=f'(z_0)$.
\end{proof}