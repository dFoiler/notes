% !TEX root = ../notes.tex

Good morning everyone. A few announcements.
\begin{itemize}
	\item Homework \#3 is due on Friday.
	\item There will be no in-person class on Wednesday or Friday.
	\item Office hours this week are today (1:00PM--2:30PM) and tomorrow (2:00PM--3:30PM).
\end{itemize}

\subsection{Connectedness}
Today we're going to talk more about continuous functions.

Last time we ended with the following proposition.
\begin{proposition}
	A space $X$ is path-connected implies that $X$ is connected. If $X$ is open and connected, then $X$ is path-connected.
\end{proposition}
\begin{proof}
	We do these separately.
	\begin{itemize}
		\item Suppose that $X=U_1\sqcup U_2$ is disconnected, and we show that $X$ is not path-connected. Namely, we have $U_1,U_2\subseteq X$ open subsets (in $X$) which are disjoint and nonempty. Because $U_1$ and $U_2$ are nonempty, find $x_1\in U_1$ and $x_2\in U_2$.

		However, we claim there is no continuous path $\gamma:[0,1]\to X$ with $\gamma(0)=x_1$ and $\gamma(1)=x_2$. Indeed, the image of $\gamma([0,1])$ must be connected, but then we can disconnect $\gamma([0,1])$ by $U_1$ and $U_2$: $\gamma([0,1])\subseteq U_1\cup U_2$ and $x_\bullet\in U_\bullet\cap\gamma([0,1])$ and $U_1\cap U_2=\emp$.

		At a high level, here is the image that a disconnected $X$ cannot have a path between any two pair points: there is no possible red path below which stays in the gray region.
		\begin{center}
			\begin{asy}
				unitsize(1cm);
				draw(circle((-1,0),1), dashed);
				draw(circle((2,0), 1.7), dashed);
				fill((-1.5,0) .. (-0.7,0.3) .. (-1, -0.7) .. cycle, lightgray);
				fill((2,-1.3) .. (3.5,-0.5) .. (1, 0.7) .. cycle, lightgray);

				pair u1 = (-1,0) + 0.3*dir(200);
				pair u2 = (2,0) + 1*dir(30);
				dot("$u_1$", u1, W);
				dot("$u_2$", u2, E);
				label("$U_1$", (-1,-1), S);
				label("$U_2$", (2,-1.7), S);
				draw(u1 -- u2, red);
			\end{asy}
		\end{center}

		\item Suppose we have a point $z\in X$, and we set
		\[C(z):=\{w\in X:\text{there is a path from }z\text{ to }w\}.\]
		We claim that $C(z)$ is closed and open in $X$, which will force $C(z)=X$ because $X$ is connected and $C(z)$ is nonempty ($z\in C(z)$ by the trivial path $\gamma:t\mapsto z$).

		We start by showing $C(z)$ is open: because $X$ is open, there exists $r>0$ such that $B(w,r)\subseteq X$. But with $w\in C(z)$, there will be a path between any point in $p\in B(z,w)$ and $w$, so there is a path from $z$ to $w$ to $p$. Here is the image.
		\begin{center}
			\begin{asy}
				unitsize(1cm);
				dot("$z$", (-5,0), W);
				dot("$w$", (0,0), S);
				dot("$p$", dir(40), dir(40));
				draw((-5,0) .. (-4,-0.3) .. (-1,0.1) .. (0,0));
				draw(circle((0,0), 1.7), dashed);
				draw((0,0) -- dir(40));
			\end{asy}
		\end{center}

		Now we show that $C(z)$ is closed. Suppose that $w\in X\setminus C(z)$, and we have to show that there is an open ball around $w$ in $X\setminus C(z)$. To see this, fix an open ball $B(w,r)\subseteq X$ for $r>0$, but now there can be no path from $z$ to anywhere in $B(w,r)$, for then we could just run the above argument again to show that $w\in C(z)$.
		\qedhere
	\end{itemize}
\end{proof}
\begin{remark}
	The proof for the second part merely needs $X$ to be locally path-connected, not a metric space.
\end{remark}
\begin{corollary}
	We have that $\CC$ is path-connected and therefore connected.
\end{corollary}
\begin{proof}
	Given any two points $z,w\in\CC$, we choose the path $\gamma:[0,1]\to\CC$ by
	\[\gamma(t)=tz+(1-t)w.\]
	Indeed, $\gamma(0)=w$ and $\gamma(1)=z$, and $\gamma$ is somewhat clearly continuous by, say, checking componentwise.
\end{proof}

\subsection{Compactness}
Let's do compactness better this time.
\begin{lemma}
	Fix $X\subseteq\CC$ (sequentially) compact. Then $X$ is both closed and bounded.
\end{lemma}
\begin{proof}
	We start by showing $X$ is closed. For this, we show that $X$ contains all of its limit points.
	
	Well, suppose that $z\in X$ is a limit point so that we have a sequence $\{z_n\}_{n\in\NN}\subseteq X$ such that $z_n\to z$. But by the (sequential) compactness of $X$, this sequence has a convergent subsequence $\{z_{\sigma n}\}_{n\in\NN}$ which does converge in $X$. But any subsequence will converge to the same limit (!), so $z_{\sigma n}\to z$ as well, so $z\in X$ is forced.

	We now show that $X$ is bounded. We proceed by contraposition: if $X$ is unbounded, then for any $n\in\NN$, then we can find some $z_n\in X\setminus B(0,n)$. But then we can check that $\{z_n\}_{n\in\NN}$ has no convergent subsequence, essentially because it tends off to infinity.
\end{proof}

Our goal for the rest of class is to prove the following two results.
\begin{restatable}{prop}{closedbounded}
	A subset $X\subseteq\CC$ is (sequentially) compact if and only if it is closed and bounded.
\end{restatable}
\begin{restatable}{theorem}{heineborel}[Heine--Borel] \label{thm:hb}
	A subset $X\subseteq\CC$ is (sequentially) compact if and only if every open cover of $X$ has a finite subcover.
\end{restatable}
On the homework, we showed the backward direction of \autoref{thm:hb}.
\begin{remark}
	Our hope is to have lots of equivalent characterizations of compactness so that we can have easier proofs of statements about compact sets.
\end{remark}
To start off, here are some lemmas we will need.
\begin{lemma}
	Fix $X\subseteq\CC$ (sequentially) compact. For any $\varepsilon>0$, there exist only finitely many points $z_1,\ldots,z_n\in X$ such that
	\[X\subseteq\bigcup_{k=1}^nB(z_k,\varepsilon).\]
\end{lemma}
\begin{proof}
	The point is to build some inductive argument: one fixes an $\varepsilon>0$ and then continues choosing random points out of $X$ until we cover $X$. Indeed, if the process does not terminate, then the sequence we generate has no convergent subsequence.

	Rigorously, if $X$ is empty, then just choose no points at all and be done. Otherwise, we can find some $z_1\in X$. Inductively, suppose we have a sequence $\{z_1,\ldots,z_m\}$. If
	\[X\subseteq\bigcup_{k=1}^nB(z_k,\varepsilon),\]
	then we are done. Otherwise, we can find $z_{m+1}\in X\setminus\bigcup_{k=1}^nB(z_k,\varepsilon)$.

	If the above inductive process terminates, then we get the result. Otherwise, there is a sequence $\{z_n\}_{n\in\NN}$ such that
	\[z_{n+1}\in X{\mathbin\bigg\backslash}\bigcup_{k=1}^nB(z_k,\varepsilon).\]
	We claim that $\{z_n\}_{n\in\NN}$ has subsequence converging in $X$. Indeed, suppose for the sake of contradiction that $z_{\sigma n}\to z$ for some strictly increasing $\sigma$ and $z\in X$. Then there exists $N$ such that $n>N$ implies
	\[|z_{\sigma n}-z|<\varepsilon/2.\]
	But then, finding some $n+1,n>N$, we have
	\[|z_{\sigma(n+1)}-z_{\sigma n}|<|z_{\sigma(n+1)}-z|+|z_{\sigma n}-z|<\varepsilon,\]
	so
	\[z_{\sigma(n+1)}\in\bigcup_{k=1}^{\sigma(n+1)-1}B(z_k,\varepsilon),\]
	which is our contradiction to the construction of $z_\bullet$.
\end{proof}
\begin{lemma} \label{lem:singleepsilon}
	Fix $X\subseteq\CC$ (sequentially) compact with some open cover $\mathcal U$ of $X$. Then there is an $\varepsilon>0$ such that, for every $z\in X$, there is $U\in\mathcal U$ such that $B(z,\varepsilon)\subseteq U$.
\end{lemma}
\begin{proof}
	Suppose that, for all $\varepsilon>0$, there exists some $z\in X$ such that no $U\in\mathcal U$ has $B(z,\varepsilon)\subseteq U$. We construct a sequence in $X$ with no subsequence converging in $X$. Indeed, for any $n\in\NN$, we find $z_n\in X$ such that no $U\in\mathcal U$ has $B(z_n,1/n)\subseteq U$. We claim that $\{z_n\}_{n\in\NN}$ has no subsequence converging in $X$.

	Indeed, suppose that we have $z\in X$ and strictly increasing $\sigma:\NN\to\NN$ such that $z_{\sigma n}\to z$. We will then be able to find some $z_n$ such that $B(z_n,1/n)\subseteq U$ for some $U\in\mathcal U$, which will be a contradiction. Indeed, $z\in X$, and $\mathcal U$ covers $z$, so there is some $U\in\mathcal U$ with $z\in U$. In fact, $U$ is open, so there is an $\varepsilon>0$ such that
	\[B(z,\varepsilon)\subseteq U.\]
	Now, there is $N$ such that for $n>N$, we can gaurantee that $|z-z_n|<\varepsilon/2$. Further, for $n>2/\varepsilon$, we have $1/n<\varepsilon/2$. So $n>\max\{N,2/\varepsilon\}$ will have $\sigma n>\max\{N,2/\varepsilon\}$, implying
	\[|w-z_n|<1/n<\varepsilon/2\implies|w-z|<|w-z_n|+|z-z_n|=\varepsilon\implies w\in B(z,\varepsilon)\subseteq U,\]
	so $B(z_n,1/n)\subseteq U$. This contradiction finishes.
\end{proof}
This is saying that there is a universal $\varepsilon$ that we can find for our open cover.
\begin{lemma} \label{lem:easycover}
	Fix $X$ a bounded set. Then, for any $\varepsilon>0$, there exist finitely many points $z_1,\ldots,z_n$ such that
	\[X\subseteq\bigcup_{k=1}^nB(z_k,\varepsilon).\]
\end{lemma}
\begin{proof}
	The point is to reduce this to the case of $[-M,M]^2$ which can cover $X$ by boundedness, and then we can create the cover for $X$ by hand.
\end{proof}
Now let's attack one of our equivalent conditions for compactness.
\closedbounded*
\begin{proof}
	The forwards direction we have already done.

	In the backwards direction, suppose that $\{z_n\}_{n\in\NN}\subseteq X$ is some sequence. Our main goal is to construct a convergent subsequence. By boundedness, we can choose $w_{1,1},\ldots,w_{1,\ell_1}$ such that
	\[X\subseteq\bigcup_{k=1}^{\ell_1}B(w_{1,k},1/2).\]
	Now, because $\{z_n\}_{n\in\NN}$ is infinite, there must be some index $w_1:=w_{1,k_1}$ such that
	\[L_1=\{n\in\NN:z_n\in B(w_{1,k_1},1/2)\}\]
	is infinite. The important point is that $\{z_n\}_{n\in L_1}$ has formed a subsequence which lives inside of a ball of radius $1/2$. We can continue this process: by boundedness, we can find some $w_{2,1},\ldots,w_{2,\ell_2}\in B(w_{1,k_1},1/2)$ such that
	\[B(w_{1,k_1},1/2)\subseteq\bigcup_{k=1}^{\ell_2}B(w_{2,k},1/4).\]
	Then we can choose $L_2$ from here by choosing one of the $w_{2,k}$ with infinitely many indices. Continuing this process forces our sequence to converge.

	To more explicitly appeal to choice, we note that we can always find some sequence $\{w_{k,i}\}\subseteq X$ such that
	\[X\subseteq\bigcup_{k=1}^{\ell_n}B(w_{k,i},1/2^k),\]
	but $L_{i-1}$ is infinite, so there is a specific $w_k:=w_{k,i}$ such that
	\[L_i:=\{n\in L_{i-1}:|z_n-w_{k}|<2^{-k}\}\]
	is infinite. To actually construct our sequence from these infinite subsets, we define a choice function over our indices: define $\varphi:\NN\to\NN$ such that $\varphi(n+1)$ is the smallest number exceeding $\varphi(n)$ with $\varphi(n+1)\in L_{n+1}$. Then we know that
	\[|z_{\varphi(n)}-w_{k}|<2^{-n}\]
	for each $1\le k\le n$. Thus, for $n\ge m>N$, we have
	\[|z_{\varphi(n)}-z_{\varphi(m)}|\le|z_{\varphi(n)}-w_m|+|z_{\varphi(n)}-w_m|<2\cdot2^{-m}<2^{-N+1},\]
	so for any $\varepsilon>0$, we can choose $N:=1-\log_2\varepsilon$ sufficiently large so that $n,m>N$ implies
	\[|z_{\varphi(n)}-z_{\varphi(m)}|<2^{-N+1}=\varepsilon.\]
	It follows that the subsequence defined by $\varphi$ is Cauchy and hence converges. But because $X$ is closed, any convergent sequence in $X$ will be in $X$, so our sequence in $X$ has a convergent subsequence.
\end{proof}