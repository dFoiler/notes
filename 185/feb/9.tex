% !TEX root = ../notes.tex

\subsection{More Compactness}
To wrap up from last class, we show the following.
\heineborel*
\begin{proof}
	The direction that sequentially compact implies closed and bounded was done on the homework.
	
	We focus on the other direction. Fix $\mathcal U$ an open cover of $X$. By \autoref{lem:singleepsilon}, we know there exists $\varepsilon>0$ such that, for each $z\in X$, there is some $U\in\mathcal U$ such that $B(z,\varepsilon)\subseteq U$. But in fact, with this $\varepsilon>0$, \autoref{lem:easycover} tells us that there exists finitely many points $z_1,\ldots,z_\ell$ such that
	\[X\subseteq\bigcup_{k=1}^\ell B(z_k,\varepsilon).\]
	But now, finding $U_k$ such that $B(z_k,\varepsilon)\subseteq U_k$ (possible by construction of $\varepsilon$), we see that $\{U_k\}_{k=1}^\ell$ will be our finite subcover.
\end{proof}
\begin{remark}
	The conclusion of the above theorem is the usual notion of compactness, so I will stop writing ``(sequentially)'' whenever I say ``compact.''
\end{remark}
Let's see a use for our definitions of compactness.
\begin{corollary}
	Let $X\subseteq\CC$ be a compact space and $f:X\to\CC$ continuous. Then $f(X)$ is compact.
\end{corollary}
\begin{proof}
	Give $f(X)$ some open cover $\mathcal U$. Because $f$ is continuous, we see that
	\[\left\{f^{-1}(U)\right\}_{U\in\mathcal U}\]
	is an open cover for $X$. But $X$ is compact, so we can find some finite subcover $\{U_k\}_{k=1}^n\subseteq\mathcal U$ so that $\left\{f^{-1}(U_k)\right\}_{k=1}^n$ covers $X$. But then the $\{U_k\}_{k=1}^n$ will cover $X$ by taking the union over our open subcover.
\end{proof}

\subsection{Uniform Continuity}
The point of uniform convergence is to make fewer choices in our notion of continuity.
\begin{definition}[Uniform continuity]
	Fix $X\subseteq\CC$ a nonempty subset. Then a function $f:X\to\CC$ is \textit{uniformly continuous} if and only if, for each $\varepsilon>0$, there exists a single $\delta>0$ so that $z,w\in X$ have
	\[|z-w|<\delta\implies|f(z)-f(w)|<\varepsilon.\]
\end{definition}
Importantly, this definition has $\delta$ not depend on either $z$ nor $w$, where continuity would allow $\delta$ to depend on one of them.
\begin{example} \label{ex:conjcont}
	The functions $\id_\CC$ and $z\mapsto\overline z$ are both uniformly continuous on $\CC$. Letting $f$ be either of these functions, we see that, for any $\varepsilon>0$, we may take $|z-w|<\varepsilon$ to imply
	\[|f(z)-f(w)|=|z-w|<\varepsilon.\]
\end{example}

Here is a nice result.
\begin{proposition} \label{prop:contcompactgetsuniform}
	Fix $X$ a nonempty, compact subset. Then any continuous function $f:X\to\CC$ is uniformly continuous.
\end{proposition}
\begin{proof}
	The point is to let $\delta\to0$ until we can fit some prescribed $\varepsilon$ bound. Choose $\delta=1/n$ as $n$ varies over positive integers, and we imagine fixing sequences $\{z_n\}_{n=1}^\infty$ and $\{w_n\}_{n=1}^\infty$ such that
	\[|z_n-w_n|<1/n.\]
	Now we use the sequential compactness of $X$, which forces $\{z_n\}_{n=1}^\infty$ to have a convergent subsequence, so we conjure $\alpha\in X$ and a strictly increasing $\sigma:\NN\to\NN$ such that $z_{\sigma n}\to\alpha$ as $n\to\infty$.

	We now claim that $w_{\sigma n}\to\alpha$ as well. In particular, for any $\delta>0$, there is some $N_1$ so that $n>N_1$ implies
	\[|z_{\sigma n}-\alpha|<\delta/2.\]
	Choosing $N$ to be larger than $N_1$ and $2/\delta$, we see that $n>N$ will have
	\[|w_{\sigma n}-\alpha|\le|z_{\sigma n}-w_{\sigma n}|+|z_{\sigma n}-\alpha|<\frac1{\sigma n}+\frac\delta2\le\frac1n+\frac\delta2<\frac\delta2+\frac\delta2=\delta,\]
	so indeed $w_{\sigma n}\to\alpha$ as $n\to\infty$.

	Only now we suppose for the sake of contradiction we have some $\varepsilon>0$ such that any $\delta>0$ has some $z$ and $w$ such that $|z-w|<\delta$ actually has $|f(z)-f(w)|\ge\varepsilon$. Taking $\delta:=1/n$, we are promised some sequences $\{z_n\}_{n=1}^\infty$ and $\{w_n\}_{n=1}^\infty$ so that
	\[|z_n-w_n|<\delta\qquad\text{and}\qquad|f(z_n)-f(w_n)|\ge\varepsilon.\]
	Using the above machinery, we see that $z_{\sigma n}\to\alpha$ and $w_{\sigma n}\to\alpha$ force $f(z_{\sigma n})\to f(\alpha)$ and $f(w_{\sigma n})\to f(\alpha)$ by continuity of $f$, but the sequences $f(z_{\sigma n})$ and $f(w_{\sigma n})$ are supposed to be $\varepsilon$ far apart! Explicitly, we can find sufficiently large $N_1$ and $N_2$ such that
	\begin{align*}
		n>N_1&\implies|f(z_{\sigma n})-\alpha|<\varepsilon/4, \\
		n>N_2&\implies|f(w_{\sigma n})-\alpha|<\varepsilon/4,
	\end{align*}
	which by the triangle inequality means that any $n>\max\{N_1,N_2\}$ will give
	\[|f(z_{\sigma n})-f(w_{\sigma n})|\le|f(z_{\sigma n})-\alpha|+|f(w_{\sigma n})-\alpha|\le\frac\varepsilon4+\frac\varepsilon4<\varepsilon,\]
	which is a contradiction to the construction of $z_{\sigma n}$ and $w_{\sigma n}$.
\end{proof}

\subsection{Uniform Convergence}
We next talk about uniform convergence for functions. Here is our starter pack.
\begin{definition}[Sequence of functions]
	Fix $X\subseteq\CC$ a nonempty subset. Then a sequence of functions $\{f_n\}_{n\in\NN}$ is a function $\varphi:\NN\to(X\to\CC)$. Namely, for each $n\in\NN$, we are given a function $\varphi(n):X\to\CC$.
\end{definition}
\begin{definition}[Pointwise convergence]
	Fix $\{f_n\}_{n\in\NN}$ a sequence of functions $X\to\CC$. Then \textit{$\{f_n\}$ converges to some $g:X\to\CC$ pointwise} if and only if, for each $z\in X$, we have $f_n(z)\to g(z)$ as $n\to\infty$. We write this as $f_n\to g$.
\end{definition}
This is called pointwise convergence because we are only checking convergence at each individual point $z\in X$, without caring about the larger structure of the function. This will cause problems later but not now.
\begin{definition}[Bounded]
	We say that a function $f:X\to\CC$ is \textit{bounded} if and only if $f(X)\subseteq\CC$ is bounded. In other words, there is some $M>0$ so that $f(X)\subseteq B(0,M)$.
\end{definition}
\begin{restatable}[Uniform convergence]{defihelper}{uniconvergedefi} \nirindex{Uniform convergence}
	Fix $\{f_n\}_{n\in\NN}$ a sequence of functions $X\to\CC$. Then \textit{$\{f_n\}$ converges to some $g:X\to\CC$ pointwise} if and only if, for each $\varepsilon>0$, there is some $N$ so that
	\[n>N\implies|f_n(z)-g(z)|<\varepsilon\]
	for each $z\in X$.
\end{restatable}
The uniformity here is that the value of $N$ is no longer allowed to depend on $z$. Here is an alternate definition.
\begin{proposition} \label{prop:definefuncdist}
	Fix $\{f_n\}_{n\in\NN}$ a sequence of functions $X\to\CC$ and $g:X\to\CC$ some function. Then $\{f_n\}_{n\in\NN}$ converges uniformly to $g$ if and only if
	\[\lim_{n\to\infty}\sup_{z\in X}\{|f_n(z)-g(z)|\}=0.\]
\end{proposition}
\begin{proof}
	We take the directions independently.
	\begin{itemize}
		\item In the forward direction, we know that there is an $N_1$ so that $n>N_1$ implies each $z\in X$ has
		\[|f_n(z)-g(z)|<1.\]
		In particular, for $n>N_1$, the set $\{|f_n(z)-g(z)|:z\in X\}$ is bounded above by $1$, so the supremum will exist; set $y_n:=\sup\{|f_n(z)-g(z)|:z\in X\}$ so that we want to show $y_n\to0$ as $n\to\infty$.
	
		More generally, for any $\varepsilon>0$, there exists some $N$ so that $n>N_0$ implies
		\[|f_n(z)-g(z)|<\varepsilon/2.\]
		So $n>\max\{N_0,N_1\}$, we will have that $y_n=\sup\{|f_n(z)-g(z)|:z\in X\}$ both exists and has $y_n\le\varepsilon/2<\varepsilon$. So we do get $y_n\to0$ as $n\to\infty$.

		\item In the reverse direction, set $y_n:=\sup\{|f_n(x)-g(x)|:x\in X\}$ so that $y_n\to0$ as $n\to\infty$. Namely, for each $\varepsilon>0$, there exists some $N$ so that $n>N$ has $y_n<\varepsilon$. In particular, we see $n>N$ has
		\[|f_n(x_0)-g(x_0)|\le\sup\{|f_n(x)-g(x)|:x\in X\}=y_n<\varepsilon\]
		for each $x_0\in X$. So indeed, $f_n$ converges to $g$ uniformly.
		\qedhere
	\end{itemize}
\end{proof}

\subsection{Distances Between Functions}
Later in life it will be nice to view functions as forming a metric under $d(f,g):=\sup\{|f(x)-g(x)|\}$. However, this supremum need not only exist; here is one condition in which it does.
\begin{lemma} \label{lem:distdefined}
	Fix $f,g:X\to\CC$ bounded functions. Then $\sup\{|f(x)-g(x)|:x\in X\}$ exists.
\end{lemma}
\begin{proof}
	Because $f$ is bounded, there exists $M_f$ so that each $x\in X$ has $|f(x)|<M_f$. Similarly, because $g$ is bounded, there exists $M_g$ so that each $x\in X$ has $|g(x)|<M_g$. It follows that, for each $x\in X$,
	\[|f(x)-g(x)|\le|f(x)|+|g(x)|\le M_f+M_g,\]
	so the set $\{|f(x)-g(x)|:x\in X\}$ is bounded above and in particular has a supremum.
\end{proof}
\begin{proposition}
	Fix $f,g,h:X\to\CC$ all bounded functions. Then
	\[\sup_{x\in X}\{|f(x)-h(x)|\}\le\sup_{x\in X}\{|f(x)-g(x)|\}+\sup_{x\in X}\{|g(x)-h(x)|\}.\]
\end{proposition}
Note that all the suprema above exist by \autoref{lem:distdefined}
\begin{proof}
	The point is to reduce to the typical triangle inequality. Indeed, for any $x\in X$, we see that
	\[|f(x)-h(x)|\le|f(x)-g(x)|+|g(x)-h(x)|.\]
	Thus,
	\begin{align*}
		\sup_{x\in X}\{|f(x)-h(x)|\} &\le \sup_{x\in X}\{|f(x)-g(x)|+|g(x)-h(x)|\} \\
		&\le \sup_{x\in X}\{|f(x)-g(x)|\}+\sup_{x\in X}\{|g(x)-h(x)|\},
	\end{align*}
	which is what we wanted. We have used the fact that $\sup(A+B)\le\sup A+\sup B$ for $A,B\subseteq\RR$, which is not hard to show: if $a+b\in A+B$, then $a\le\sup A$ and $b\le\sup B$, so $a+b\le\sup A+\sup B$; thus, $\sup(A+B)\le\sup A+\sup B$.\footnote{In fact, $\sup A+\sup B\le\sup(A+B)$ as well. We show $\sup A\le\sup(A+B)-\sup B$. Fixing $a\in A$, we need $a\le\sup(A+B)-\sup B$, so we show $\sup B\le\sup(A+B)-a$. Fixing $b\in B$, we need $b\le\sup(A+B)-a$, which is clear.}
\end{proof}
\begin{remark}[Nir] \label{rem:metriconfuncs}
	Viewing \autoref{lem:distdefined} as providing a distance metric on the space of bounded functions $X\to\CC$, the above proposition proves the triangle inequality for this metric. The other checks as follows; fix two bounded functions $f,g:X\to\CC$.
	\begin{itemize}
		\item Note that $\sup\{|f(x)-g(x)|:x\in X\}=0$ if and only if $|f(x)-g(x)|=0$ for all $x\in X$ if and only if $f=g$.
		\item Note that $|f(x)-g(x)|=|g(x)-f(x)|$ for each $x\in X$, so $\{|f(x)-g(x)|:x\in X\}=\{|g(x)-f(x)|:x\in X\}$, so they also have equal suprema.
	\end{itemize}
\end{remark}
We can also build a Cauchy criterion for uniform convergence.
\begin{prop} \label{prop:betteruniformconverge}
	A sequence of functions $\{f_n\}_{n\in\NN}$ a sequence of functions $X\to\CC$. Then $\{f_n\}_{n\in\NN}$ converges to some function uniformly if and only if the quantity $\sup_{x\in X}\{f_n(x)-f_m(x)\}$ exists and, for any $\varepsilon>0$, there exists some $N$ so that $n,m>N$ implies
	\[\sup_{x\in X}\{|f_n(x)-f_m(x)|\}<\varepsilon\]
	for any $x\in X$.
\end{prop}
We note that the hypothesis that the supremum exists can be removed if the functions are presupposed to be bounded.
\begin{proof}
	We again take the directions independently.
	\begin{itemize}
		\item Suppose that the sequence of functions $\{f_n\}_{n\in\NN}$ converges to a function $g$ uniformly. Then, for any $\varepsilon>0$, we are promised some $N$ so that $n>N$ will have
		\[|f_n(x)-g(x)|<\varepsilon/4\]
		for any $x\in X$. In particular, for any $n,m>N$, we see
		\[|f_n(x)-f_m(x)|<|f_n(x)-g(x)|+|f_m(x)-g(x)|<\frac\varepsilon4+\frac\varepsilon4=\frac\varepsilon2,\]
		so
		\[\sup_{x\in X}\{|f_n(x)-f_m(x)|\}\le\frac\varepsilon2<\varepsilon,\]
		which is what we wanted.
		\item There are two steps.
		\begin{itemize}
			\item We begin by constructing $g$. Well, for each $x\in X$, we note that any $\varepsilon>0$ will have some $N$ so that $n,m>N$ implies
			\[|f_n(x)-f_m(x)|\le\sup_{x\in X}\{|f_n(x)-f_m(x)|\}<\varepsilon,\]
			so the sequence $\{f_n(x)\}_{n\in\NN}$ is a Cauchy sequence and hence converges in $\CC$. We define $g(x)$ to be the limit of $f_n(x)$ as $n\to\infty$.
			\item Next we show the uniform convergence. Fix some $\varepsilon>0$. Then we are promised some $N$ so that $n,m>N$ has
			\[\sup_{x\in X}\{|f_n(x)-f_m(x)|\}<\varepsilon.\]
			In particular, for any $x\in X$
			\[f_n(x)-g(x)=\lim_{m\to\infty}(f_n(x)-f_m(x)),\]
			so because $z\mapsto|z|$ is continuous, any $n>N$ will have
			\[|f_n(x)-g(x)|=\lim_{m\to\infty}|f_n(x)-f_m(x)|\le\lim_{m\to\infty}\sup_{z\in X}\{|f_n(x)-f_m(x)|\}<\lim_{m\to\infty}\varepsilon=\varepsilon,\]
			where the last inequality holds by taking $m$ sufficiently large (that is, $m>N$). So we have been provided uniform convergence.
			\qedhere
		\end{itemize}
	\end{itemize}
\end{proof}
\begin{remark}[Nir]
	In the language of metric topology, the above proposition asserts that the space of (bounded) functions is metrically complete. For this, one must technically show that $\{f_n\}_{n\in\NN}$ being bounded implies that the convergent $g$ is bounded, but this is not hard: there is $N$ so that $n>N$ has $|f_n(x)-g(x)|<1$ for each $x\in X$.
\end{remark}
\begin{remark}
	In lecture, Professor Morrow asserted that we require these functions to be bounded. I do not think this is the case; indeed, the above proof never uses this hypothesis.
\end{remark}
We close with one result which shows that uniform continuity is nice.
\begin{proposition} \label{prop:uniformgoodcontinuity}
	Fix $\{f_n\}_{n\in\NN}$ a sequence of functions $X\to\CC$ all continuous at some $x\in X$. If $\{f_n\}_{n\in\NN}$ converges uniformly to some function $g:X\to\CC$, then $g$ is also continuous.
\end{proposition}
\begin{proof}
	The idea is to well-approximate $g$ by $f_n$s. Fix any $\varepsilon>0$. By the uniform convergence, there will be some $N$ so that
	\[|f_n(z)-g(z)|<\varepsilon/3\]
	for any $n>N$ and $z\in X$; fix some $n>N$. Because $f_n$ is continuous, we are promised some $\delta>0$ (allowed to vary with our chosen $x\in X$) so that
	\[|z-x|<\delta\implies|f_n(z)-f_n(x)|<\varepsilon/3\]
	for any $z\in X$. Well, if $|z-x|<\delta$, then the triangle inequality gives
	\[|g(z)-g(x)|\le|g(z)-f_n(z)|+|f_n(z)-f_n(x)|+|f_n(x)-g(x)|<\frac\varepsilon3+\frac\varepsilon3+\frac\varepsilon3=\varepsilon,\]
	which is what we needed.
\end{proof}
\begin{remark}[Nir]
	In fact, if the $\{f_n\}_{n\in\NN}$ are uniformly continuous, then $g$ will also be uniformly continuous. The argument is similar.
\end{remark}