% !TEX root = ../notes.tex

Good morning everyone. Here is some house-keeping.
\begin{itemize}
	\item Homework \#2 is due at Friday at 11:59, on GradeScope. The assignment has just been added.
	\item There are office hours to talk about the homework. Please come if you have questions.
\end{itemize}

\subsection{Summation Review}
Today we finish our discussion of series, for now. We quickly recall the definitions.
\begin{definition}[Series]
	An \textit{infinite series} over $\CC$ is an infinite sum
	\[S:=\sum_{n=1}^\infty z_n\]
	where $\{z_n\}_{n\in\NN}\subseteq\CC$ is a sequence of complex numbers.
\end{definition}
\begin{definition}[Converge, diverge]
	Fix a sequence $\{z_n\}_{n\in\NN}\subseteq\CC$ of complex numbers, we define the $m$th partial sum to be
	\[S_m:=\sum_{n=0}^mz_m.\]
	Then we say that the infinite series \textit{converges} if and only if the sequence $\{S_m\}$ of partial sums converges. Otherwise, we say that $S$ is \textit{divergent}.
\end{definition}
Today we are building towards proving Dirichlet's convergence theorem. We pik up the following lemmas.
\begin{lemma}
	Fix sequences $\{z_{k,\ell}\}_{k,\ell\in\NN}$ a collection of complex numbers satisfying the following conditions.
	\begin{itemize}
		\item Fixing any $k$, the sum $\sum_{\ell=0}^\infty|z_{k,\ell}|$ converges.
		\item The sum $\sum_{k=0}^\infty\sum_{\ell=0}^\infty|z_{k,\ell}|$ converges.
	\end{itemize}
	Then the following are true.
	\begin{listalph}
		\item Fix any $\ell$, the sum $\sum_{k=0}^\infty|z_{k,\ell}|$ converges; i.e., the terms in the left sum below are well-defined.
		\item We have that
		\[\sum_{\ell=0}^\infty\sum_{k=0}^\infty z_{k,\ell}=\sum_{k=0}^\infty\sum_{\ell=0}^\infty z_{k,\ell},\]
		and both sums converge.
	\end{listalph}
\end{lemma}
Intuitively, the first condition is requiring that the series horizontally does not grow too fast. The second condition is requiring an absolute convergence condition. Then the first conclusion says we can sum vertically instead, and the second conclusion says that we can move around the order of summation.
\begin{proof}
	We will sketch this proof; we prove (a) and (b) more or less simultaneously. To turn the infinite double sum into something we can consider finite partial sums of, we set, for each natural $N$,
	\[S_n:=\sum_{k=0}^n\sum_{\ell=0}^n|z_{k,\ell}|.\]
	The main claim is that
	\[\lim_{n\to\infty}S_n\stackrel?=\sum_{k=0}^\infty\sum_{\ell=0}^\infty|z_{k,\ell}|.\]
	Indeed, fix any $\varepsilon>0$. Because the latter sum converges, there exists some natural $A$ such that
	\[\sum_{k>A}\sum_{\ell=0}^\infty|z_{k,\ell}|<\frac\varepsilon2.\]
	Further, there exists some natural $B_k$ such that
	\[\sum_{\ell>B_k}|z_{k,\ell}|<\frac\varepsilon{2A}\]
	for each $k\in\NN$. Take $B:=\max_{0\le k<A}B_k$. Now, we set $N:=\max\{A,B\}$. To start off our inequalities, we note that
	\[S_n=\sum_{k=0}^n\sum_{\ell=0}^n|z_{k,\ell}|\le\sum_{k=0}^n\sum_{\ell=0}^\infty|z_{k,\ell}|\le\sum_{k=0}^\infty\sum_{\ell=0}^\infty|z_{k,\ell}|,\]
	so we know the sign of our difference. In particular, for any $n>N$, we see that
	\[S_n=\sum_{k=0}^N\sum_{\ell=0}^N|z_{k,\ell}|\ge\sum_{k=0}^K\sum_{\ell=0}^L|z_{k,\ell}|.\]
	Thus,
	\[0\le\sum_{k=0}^\infty\sum_{\ell=0}^\infty|z_{k,\ell}|-S_n\le\sum_{k=0}^\infty\sum_{\ell=0}^\infty|z_{k,\ell}|-\sum_{k=0}^K\sum_{\ell=0}^L|z_{k,\ell}|=\sum_{k>K}\sum_{\ell=0}^\infty|z_{k,\ell}|+\sum_{k=0}^K\sum_{\ell>L}|z_{k,\ell}|\]
	after some cancellation. But we can upper-bound the last quantity by $\frac\varepsilon2+K\cdot\frac\varepsilon{2K}=\varepsilon$, so we are done.

	The main point of the above lemma is that we are told each $\varepsilon>0$ has some $N$ so that $n>N$ implies
	\[\sum_{k=0}^\infty\sum_{\ell=0}^\infty|z_{k,\ell}|-S_n=\sum_{\substack{(k,\ell)\in\ZZ^2\\k>n\text{ or }\ell>n}}|z_{k,\ell}|<\varepsilon.\]
	We now take the two parts in sequence.
	\begin{listalph}
		\item Fix an index $\ell'$; we show absolute convergence by showing that the partial sums of $\sum_{k=0}^\infty|z_{k,\ell'}|$ are Cauchy. Indeed, fix some $\varepsilon>0$, and we know there exists $N$ so that each $n>N$ has
		\[\sum_{k=0}^\infty\sum_{\ell=0}^\infty|z_{k,\ell}|-S_n<\varepsilon.\]
		Now, we see that any $n>m>N$ will have
		\[\sum_{k=m+1}^n|z_{k,\ell'}|\le\sum_{k=m+1}^N\sum_{\ell=0}^\infty|z_{k,\ell}|\le\sum_{k=N+1}^\infty\sum_{\ell=0}^\infty|z_{k,\ell}|+\sum_{k=0}^N\sum_{\ell=N+1}^\infty|z_{k,\ell}|<\varepsilon,\]
		so we are done.
		\item As above, fix some $\varepsilon>0$, and we are promised $N$ so that
		\[\sum_{\substack{(k,\ell)\in\ZZ^2\\k>N\text{ or }\ell>N}}|z_{k,\ell}|<\varepsilon/2.\]
		Observe, for $K,L>N$, we have by the triangle inequality tht
		\[\left|\sum_{\ell=0}^L\sum_{k=0}^Kz_{k,\ell}-S_N\right|<\varepsilon/2.\]
		This bounds holds for any $K$, so we can send $K$ arbitrarily large; that inner sum will converge, so in fact we can send $K$ to $\infty$ without ill effect. (Formally, the inner term is an increasing sequence bounded above, so it will converge as $K\to\infty$.) This gives
		\[\left|\sum_{\ell=0}^L\sum_{k=0}^\infty z_{k,\ell}-S_N\right|\le\varepsilon/2.\]
		Again, the inner term is an increasing sequence as $L\to\infty$ but still bounded above as $\varepsilon/2$, so the inner sum will converge as $L\to\infty$ and still give the inequality
		\[\left|\sum_{\ell=0}^\infty\sum_{k=0}^\infty z_{k,\ell}-S_N\right|<\varepsilon.\]
		Now as we send $\varepsilon\to0$, we see that $\lim_{N\to\infty}S_N=\sum_{\ell=0}^\infty\sum_{k=0}^\infty z_{k,\ell}$, which finishes.\qedhere
	\end{listalph}
\end{proof}

\subsection{Dirichlet Test}
We now go directly for the Dirichlet test for convergence.
\begin{lemma}[Summation by parts] \label{lem:sumparts}
	Fix sequences $\{a_n\}_{n\in\NN}$ and $\{b_n\}_{n\in\NN}$ sequences of complex numbers. Then we define
	\[B_n:=\sum_{k=0}^Nb_n,\]
	and $B_{-1}=0$ to be the empty sum. It follows that, for any $n,m\in\NN$ with $n>m$,
	\[\sum_{k=m}^na_kb_k=a_nB_n-a_mB_{m-1}+\sum_{k=m}^{n-1}B_k(a_k-a_{k+1}).\]
\end{lemma}
\begin{proof}
	This comes down to a direct computation. The main point is that $b_k=B_k-B_{k-1}$, which even works with $k=0$. Indeed,
	\begin{align*}
		\sum_{k=m}^na_kb_k &= \sum_{k=m}^na_k(B_k-B_{k-1}) \\
		&= \sum_{k=m}^na_kB_k-\sum_{k=m}^na_kB_{k-1} \\
		&\stackrel*= a_nB_n+\sum_{k=m}^{n-1}a_kB_k-a_mB_{m-1}-\sum_{k=m}^na_{k+1}B_k \\
		&= a_nB_n-a_mB_{m-1}+\sum_{k=m}^{n-1}B_k(a_k-a_{k+1}),
	\end{align*}
	which is what we wanted. The important step to pay attention to is the rearrangement we did in $\stackrel*=$ in order to massage the sums together.
\end{proof}
And here is our theorem.
\begin{theorem}[Dirichlet's test]
	Fix $\{a_n\}_{n\in\NN}\subseteq\RR$ a sequence of real numbers and $\{b_n\}_{n\in\NN}\subseteq\CC$ a sequence of complex numbers satisfying the following conditions.
	\begin{itemize}
		\item The sequence $\{a_n\}_{n\in\NN}$ is decreasing.
		\item We have $a_n\to0$ as $n\to\infty$.
		\item Bounded partial sums: there exists a positive real number $M$ such that
		\[\left|\sum_{k=0}^nb_k\right|<M\]
		for each $n\in\NN$.
	\end{itemize}
	Then we claim that
	\[\sum_{k=0}^\infty a_kb_k\]
	converges.
\end{theorem}
\begin{proof}
	As usual, fix our partial sums
	\[S_n:=\sum_{k=0}^na_kb_k\qquad\text{and}\qquad B_n:=\sum_{k=0}^nb_k.\]
	We are given that the $B_k$ are bounded, so we are going to want to use \autoref{lem:sumparts}, which tells us that
	\[S_n=a_nB_n+\sum_{k=0}^{n-1}B_k(a_k-a_{k+1}).\]
	We examine the convergence of thesex terms individually.
	\begin{itemize}
		\item We will show that the summation term absolutely converges. We are given that the partial sums $B_n$ are bounded by $M$, so we note $|B_k(a_k-a_{k+1})|<M|a_k-a_{k+1}|$, so it suffices to show that
		\[M\sum_{k=0}^{n-1}\left|a_k-a_{k+1}\right|\]
		converges as $n\to\infty$. It would be great if this would telescope, and indeed it does! Because the $a_k$ are decreasing,
		\[\sum_{k=0}^\infty|a_k-a_{k+1}|=\sum_{k=0}^\infty(a_k-a_{k+1})=a_0-a_{n+1}.\]
		Because $a_n\to0$ as $n\to\infty$, we see that this sum will converge to $a_0$. It follows that
		\[\sum_{k=0}^\infty|B_k(a_k-a_{k+1})|\]
		will converge by the Comparison test, so
		\[\sum_{k=0}^\infty B_k(a_k-a_{k+1})\]
		converges by absolute convergence.
		\item Note that the $B_n$ are bounded in norm by $M$, so $|a_nB_n|\le M|a_n|$, but $|a_n|\to0$ as $n\to\infty$, so $|a_nB_n|\to0$.
		\qedhere
	\end{itemize}
\end{proof}
Eterovi\'c has lots of different convergence tests in his notes, but we don't care about most of them. Here is one that we do care about.
\begin{theorem}[Integral test]
	Fix a decreasing function $f:\left[1,\infty\right)\to\RR^+$ and for which
	\[\int_k^{k+1}f(x)\,dx\]
	always exists. Then the sequence of integrals $I_n:=\int_1^nf(x)\,dx$ converges if and only if the summation
	\[\sum_{k=1}^\infty f(k)\]
	converges.
\end{theorem}
\begin{proof}
	We omit this proof; it's a reasonably standard real-analytic test.
\end{proof}