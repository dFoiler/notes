% !TEX root = ../notes.tex

Good morning, everyone. Here are some announcements.
\begin{itemize}
	\item Midterm grades will be posted today or tomorrow, on \texttt{bCourses}.
	\item Class on Wednesday will be a recording. Professor Morrow will be giving a talk, at 9AM as decided by the powers that be.
	\item There is no homework due Friday because we haven't covered anything since the midterm.
\end{itemize}

\subsection{Holomorphic Power Series}
Today we actually talk about analytic functions. Professor Morrow promises that it is actually complex analysis today, and once we talk about analytic functions and path integration, we will prove the Cauchy integral formula, which is one of the major results of the course.

We recall the following definition.
\powerseriesdefi*
\noindent So far we've talked about the radius of convergence of a power series as well as some properties of series of functions in general (e.g., the Weierstrass $M$-test).

Today we are showing the following result.
\begin{proposition} \label{prop:powerseriesholo}
	Fix $S(z)=\sum_{k=0}^\infty a_kz^k$ a (complex) power series with radius of convergence $R>0$. Then $S(z)$ is holomorphic on $B(0,R)$ with derivative
	\[S'(z)=\sum_{k=1}^\infty ka_kz^{k-1}.\]
	Further, $S'(z)$ also has radius of convergence $R$.
\end{proposition}
Note that this derivative is essentially the ``term-wise'' derivative of $S(z)$, so it is more or less the best thing that we could want.
\begin{proof}
	We will symbolically define
	\[S'(z):=\sum_{k=1}^\infty ka_kz^{k-1}\]
	and show that it is equal to the requested derivative. We start by noting the radius of convergence of $S'$ is
	\[\frac1{\lim_{k\to\infty}\sqrt[k]{|(k+1)a_k|}}=\frac1{\lim_{k\to\infty}\sqrt[k]k}\cdot\frac1{\lim_{k\to\infty}\sqrt[k]{|a_k|}}=1\cdot R=R,\]
	so at the very least our radius of convergence matches, as claimed.
	
	Fix $0<r<R$ a real number (i.e., we don't want to deal with $R=+\infty$), so that it suffices to show $S$ is holomorphic with the given derivative on $B(0,r)$. (Namely, for a given $w\in B(0,R)$, choose any $r$ with $|w|<r<R$.)
	
	Indeed, given $w\in B(0,r)$, it suffices to show that $S$ is differentiable at $w$ with the requested derivative, for which we claim
	\[\left(\lim_{z\to w}\frac{S(z)-S(w)}{z-w}\right)-S'(w)\stackrel?=0,\]
	where $S'(z)$ is the claimed derivative. To set up our computation, we fix a positive integer $m$ and work with the $m$th partial sum, computing
	\begin{align*}
		\frac{S_m(z)-S_m(w)}{z-w}-S_m'(w) &= \sum_{k=0}^m\frac{a_kz^k-a_kw^k}{z-w}-\sum_{k=1}^mka_kw^{k-1} \\
		&= (a_0-a_0)+\sum_{k=1}^ma_k\left(\frac{z^k-w^k}{z-w}-kw^{k-1}\right) \\
		&= \sum_{k=1}^ma_k\left(\sum_{a+b=k-1}z^bw^a-\sum_{a+b=k-1}w^{k-1}\right) \\
		&= \sum_{k=1}^ma_k\left(\sum_{a+b=k-1}\left(z^bw^a-w^{k-1}\right)\right) \\
		&= \sum_{k=1}^ma_k\left(\sum_{a+b=k-1}w^a\left(z^b-w^b\right)\right).
	\end{align*}
	With this in mind, we set
	\[h_k(z)=\sum_{a+b=k-1}w^a\left(z^b-w^b\right),\]
	which we note is a polynomial in $z\in B(0,r)$ because we fixed $w$ to be constant. In particular, we have
	\[\frac{S_m(z)-S_m(w)}{z-w}-S_m'(w)=\sum_{k=1}^ma_kh_k(z).\]
	We now show that this series converges uniformly as $m\to\infty$; we will use \autoref{thm:mtest}. For this, we bound
	\[|h_k(z)|=\left|\sum_{a+b=k-1}w^a\left(z^b-w^b\right)\right|\le\sum_{a+b=k-1}|w|^a\left(|z|^b+|w|^b\right)<\sum_{a+b=k-1}r^a\left(r^b+r^b\right)=2(k-1)r^{k-1},\]
	so we bound $|a_kh_k(z)|<|a_k|\cdot2(k-1)r^{k-1}$. Namely, by \autoref{thm:mtest}, it suffices to show that the series
	\[\sum_{k=1}^\infty2(k-1)|a_k|r^{k-1}\]
	converges. Well, $\sum_{k=1}^\infty2(k-1)|a_k|x^{k-1}$ is a power series with radius of convergence
	\[\frac1{\lim_{k\to\infty}\left(\sqrt[k]{2k}\cdot\sqrt[k]{|a_{k+1}|}\right)}=\frac1{\lim_{k\to\infty}\sqrt[k]{2k}}\cdot\frac1{\lim_{k\to\infty}\sqrt[k]{|a_{k+1}|}}=R,\]
	so indeed the power series $\sum_{k=1}^\infty2(k-1)|a_k|x^{k-1}$ converges at $x=r<R$.

	So in total, we see that the series of functions
	\[\sum_{k=1}^\infty a_kh_k(z)\]
	uniformly converges as $m\to\infty$. Because each component function $a_kh_k(z)$ is continuous, we see that the entire series will converge to a continuous function by \autoref{rem:continuousseries}. In other words, we can evaluate
	\[\lim_{z\to w}\lim_{m\to\infty}\left(\frac{S_m(z)-S_m(w)}{z-w}-S_m'(w)\right)=\lim_{z\to w}\sum_{k=1}^\infty a_kh_k(z)=\sum_{k=1}^\infty a_kh_k(w).\]
	But now we notice that $h_k(w)=0$ for each $h_k$, so this sum does indeed vanish.

	We are now essentially done. We compute
	\begin{align*}
		\lim_{z\to w}\frac{S(z)-S(w)}{z-w} &= \lim_{z\to w}\left(\frac{S(z)-S(w)}{z-w}-S'(w)\right)+\lim_{z\to w}S'(w) \\
		&= \lim_{z\to w}\left(\frac{\lim_{m\to\infty} S_m(z)-\lim_{m\to\infty}S_m(w)}{z-w}-\lim_{m\to\infty}S_m'(w)\right)+S'(w) \\
		&= \lim_{z\to w}\lim_{m\to\infty}\left(\frac{S_m(z)-S_m(w)}{z-w}-S_m'(w)\right)+S'(w) \\
		&= S'(w),
	\end{align*}
	so we are done.
\end{proof}
So indeed, power series are holomorphic. Here is nice application of this fact.
\begin{corollary}
	Fix
	\[S(z)=\sum_{k=0}^\infty a_kz^k\qquad\text{and}\qquad T(z)=\sum_{k=0}^\infty b_kz^k\]
	two complex power series with radius of convergence $R>0$. If $S(z)=T(z)$ for all $z\in B(0,R)$, then $a_k=b_k$ for each $k$.
\end{corollary}
\begin{proof}
	We proceed inductively, in spirit. For example $a_0=S(0)=T(0)=b_0$, so these are equal as our base case. Further, we could take one derivative to see that
	\[S'(z)=\sum_{k=1}^\infty ka_kz^{k-1}\qquad\text{and}\qquad T'(z)=\sum_{k=1}^\infty kb_kz^{k-1},\]
	so $a_1=S'(0)=T'(0)=b_1$. More generally, setting $S^{(m)}$ to be the $m$th derivative, we can see that
	\[S^{(m)}(z)=\sum_{k=m}^\infty k(k-1)\cdots(k-m+1)a_kz^{k-m}\qquad\text{and}\qquad T^{(m)}=\sum_{k=m}^\infty k(k-1)\cdots(k-m+1)a_kz^{k-m},\]
	and both of these have the same radius of convergence. So now $a_m=\frac1{m!}S^{(m)}(0)=\frac1{m!}T^{(m)}(0)=b_m$.
\end{proof}

\subsection{Analytic Functions}
To define analytic, we need one more definition.
\begin{definition}[Power series expansion]
	Fix $X\subseteq\CC$ a nonempty open subset and $f:X\to\CC$ a function. We say that $f$ has a \textit{power series expansion centered at $z_0\in X$}
	if and only if there is a positive real number $r$ such that $B(z_0,r)\subseteq X$ and further there is a power series defined by
	$\{a_k\}_{k\in\NN}$ which has
	\[f(z)=\sum_{k=0}^\infty a_k(z-z_0)^k\]
	for each $z\in B(z_0,r)$.
\end{definition}
And here is our definition.
\begin{definition}[Analytic]
	Fix $X\subseteq\CC$ a nonempty open subset and $f:X\to\CC$ a function. Then $f$ is \textit{analytic} at $z_0\in\CC$ if and only if $f$ has a power series expansion at $z_0$. Explicitly, there is a power series $S(z)=\sum_{k=0}^\infty a_kz^k$ and positive real number $r>0$ (less than the radius of convergence) such that
	\[f(z)=S(z-z_0)=\sum_{k=0}^\infty a_k(z-z_0)^k\]
	for any $z\in B(z_0,r)$. Then $f$ is \textit{analytic} if and only if it is analytic at each $z_0\in\CC$.
\end{definition}
Here is the idea.
\begin{idea}
	Analytic functions are locally power series.
\end{idea}
Being analytic is a very nice condition. For example, we have the following.
\begin{proposition} \label{prop:anaisholo}
	Analytic functions are holomorphic on their domain.
\end{proposition}
\begin{proof}
	Fix $f:X\to\CC$ an analytic function. For each $x\in X$, we note that $f$ is locally equal to a power series at $x$ (i.e., $f|_{B(x,r)}$ is a power series), which is holomorphic by \autoref{prop:powerseriesholo}. Because $f$ is locally differentiable at each point, $f$ will be actually differentiable at each point.
\end{proof}
\begin{remark}
	It will turn out that the converse is also true, but this is a pretty deep result. We will prove it from the Cauchy integral formula. The main obstacle is how we should construct the power series, which the Cauchy integral formula will tell us how to do.
\end{remark}
Anyways, let's prove something of substance.
\begin{lemma} \label{lem:ddzanaisana}
	Fix $X\subseteq\CC$ a nonempty open subset and $f:X\to\CC$ an analytic function. Then $f'$ is also analytic.
\end{lemma}
\begin{proof}
	Fix $z_0\in X$. Because $f$ is analytic, there is a positive real number $r>0$ and power series $S(z)=\sum_{k=0}^\infty a_k(z-z_0)^k$ (with radius of convergence at least $r$) such that
	\[f(z)=S(z-z_0)=\sum_{k=0}^\infty a_k(z-z_0)^k\]
	for each $z\in B(z_0,r)$. By \autoref{prop:powerseriesholo}, we see that
	\[f'(z)=S'(z-z_0)=\sum_{k=1}^\infty ka_k(z-z_0)^{k-1}\]
	for each $z\in B(0,r)$. So we see that $f'$ has a power series expansion at our arbitrarily chosen $z_0\in X$, so $f'$ is analytic at each $z_0\in X$, so $f'$ is analytic.
\end{proof}
\begin{remark}
	We can iterate the above lemma to show that an analytic function is infinitely differentiable.
\end{remark}
\begin{remark}
	In fact, because analytic will turn out to be equivalent to holomorphic, we will see that being once differentiable implies being analytic implies being infinitely differentiable. This is pretty nice.
\end{remark}
Next class we will start talking about the exponential function, a very important analytic function.