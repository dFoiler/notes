% !TEX root = ../notes.tex

Today we are talking about continuity.

\subsection{Continuity}
We have the following lemam.
\begin{lemma}
	Suppose that $f:X\to\CC$.
	\begin{listalph}
		\item Then $f$ is continuous at $w$ if and only if every sequence $\{z_n\}\subseteq X$ such that $z_n\to z$ implies $f(z_n)\to f(z)$.
		\item We have that $f$ is continuous on $X$ if and only if every open open set $U\subseteq\CC$ has $f^{-1}(U)$ open in $X$.
		\item We have that $f$ is continuous on $X$ if and only if each closed set $V\subseteq X$ has $f^{-1}(V)$ closed in $X$.
		\item We have that $f$ is continuous at if and only if, for each $\varepsilon>0$ and $z\in\CC$, we have that $f^{-1}(B(z,\varepsilon))$ is open in $X$.
	\end{listalph}
\end{lemma}
\begin{proof}
	We take the parts one at a time.
	\begin{listalph}
		\item For the forwards direction, suppose that $\{z_n\}_{n\in\NN}\subseteq X$ converges to some $w$. Then let $\varepsilon>0$. By assumption, there exists some $\delta>0$ such that
		\[|z-w|<\delta\implies|f(x)-f(w)|<\varepsilon.\]
		It follows from $z_n\to w$ that there exists some $N$ such that
		\[n>N\implies|z_n-w|<\delta\implies|f(z_n)-f(z)|<\varepsilon,\]
		so it fllows that $f(z_n)\to f(z)$.
		
		In the reverse direction, take $f$ not continuous at $w$, so there exists $\varepsilon>0$ so that for all $n\in\NN$, there exists some chosen $z_n$ with
		\[|z_n-w|<\delta\implies|f(z_n)-f(w)|>\varepsilon.\]
		But as $z_n\to w$, we see that $f(z_n)$ does not approach $f(w)$, so we are done.

		\item In the forwards direction, suppose that $U\subseteq\CC$ is open, and we show that $f^{-1}(U)$ is open in $X$. Well, suppose that $z\in f^{-1}(U)$, and we will find $\delta>0$ such that $B(z,\delta)\subseteq f^{-1}(U)$. 
		
		Well, $f(z)\in U$, so there exists $\varepsilon>0$ such that $B(f(z),\varepsilon)\subseteq U$. Thus, continuity of $f$ requires some $\delta>0$ such that
		\[|w-z|<\delta\implies|f(w)-f(z)|<\varepsilon,\]
		which implies $f(w)\in B(f(z),\varepsilon)\subseteq U$ implies $w\in f^{-1}(U)$. So indeed, $B(z,\delta)\subseteq f^{-1}(U)$.

		In the reverse direction, suppose that each open $U\subseteq\CC$ has $f^{-1}(U)$ is open. Now fix any $z\in X$ and $\varepsilon>0$. The set $B(f(z),\varepsilon)$ is open, so
		\[f^{-1}(B(f(z),\varepsilon))\]
		is open. But $z\in f^{-1}(B(f(z),\varepsilon))$, so we can find $\delta>0$ such that $B(z,\delta)\subseteq f^{-1}(B(f(z),\varepsilon))$. Thus, $|w-z|<\delta$ implies $w\in f^{-1}(B(f(z)),\varepsilon)$ implies $f(w)\in B(f(z),\varepsilon)$ implies $|f(w)-f(z)|<\varepsilon$, finishing.

		\item In the forwards direction, suppose $f$ is continuous so that $U\subseteq\CC$ open implies $f^{-1}(U)\subseteq X$ is open. But then, if $V\subseteq\CC$ is closed, then $\CC\setminus V$ is open, so\footnote{To see $f^{-1}(A\setminus B)=f^{-1}(A)\setminus f^{-1}(B)$, note that $x\in f^{-1}(A\setminus B)$ if and only if $f(x)\in A\setminus B$ if and only if $f(x)\in A$ but $f(x)\notin B$ if and only if $x\in f^{-1}(A)$ but $x\notin f^{-1}(B)$.}
		\[f^{-1}(\CC\setminus V)=f^{-1}(\CC)\setminus f^{-1}(V)=X\setminus f^{-1}(V)\]
		is open, so $f^{-1}(V)$ is closed.

		In the backwards direction, suppose $f^{-1}$ preserves closed sets. Then, if $U\subseteq\CC$ is open, $\CC\setminus U$ is closed, so
		\[f^{-1}(\CC\setminus U)=f^{-1}(\CC)\setminus f^{-1}(U)=X\setminus f^{-1}(U)\]
		is closed, so $f^{-1}(U)$ is open. Thus, $f^{-1}$ preserves open sets, so $f$ must be continuous.

		\item In the forwards direction, fix $\varepsilon>0$ and $z\in\CC$, so $B(z,\varepsilon)$ is open, so $f^{-1}(B(z,\varepsilon))$ is open in $X$, finishing.
		
		In the other direction fix $\varepsilon>0$ and $z\in\CC$ to consider $B(f(z),\varepsilon)\subseteq U$. Thus, continuity of $f$ requires some $\delta>0$ such that
		\[|w-z|<\delta\implies|f(w)-f(z)|<\varepsilon,\]
		which implies $f(w)\in B(f(z),\varepsilon)\subseteq U$ implies $w\in f^{-1}(U)$. So indeed, $B(z,\delta)\subseteq f^{-1}(U)$.

		In the reverse direction, fix $U$ open, and we show that $f^{-1}(U)$ is open. Well, each $z\in U$ has some $\varepsilon_z$ such that $B(z,\varepsilon_z)\subseteq U$. But $f^{-1}(B(z,\varepsilon_z))$ is open by hypothesis, so
		\[f^{-1}(U)=f^{-1}\left(\bigcup_{z\in U}B(z,\varepsilon_z)\right)=\bigcup_{z\in U}f^{-1}(B(z,\varepsilon_z))\]
		is an arbitrary union of open sets and hence open.
		\qedhere
	\end{listalph}
\end{proof}

And here are some more lemmas.
\begin{lemma}
	The following are true.
	\begin{listalph}
		\item Fix some $z_0\in\CC$. Then $f(z):=|z-z_0|$ is continuous on $\CC$.
		\item The functions $\Re$ and $\Im$ are continuous on $\CC$.
	\end{listalph}
\end{lemma}
\begin{proof}
	We leave these proofs as an exercise. The main point is that each $\varepsilon$ should be chosen equal to $\delta$.
\end{proof}
Continuous functions also have some arithmetic.
\begin{proposition}
	Fix $f,g:X\to\CC$ to functions continuous at $z_0\in X$. Then $f+g,f\cdot g$ are both continuous at $z_0\in X$, and $f/g$ is continuous at $z_0$ provided $g(z_0)\ne0$.
\end{proposition}
\begin{proof}
	The point is to appeal to the corresponding results on convergence of sequences. In particular, we use the idea that $f$ is continuous at $z_0$ if and only if each sequence $z_n\to z_0$ in $X$ has $f(z_n)\to f(z_0)$.
\end{proof}
\begin{cor}
	Every polynomial in one variable is a continuous function $X\to\CC$ for any $X\subseteq\CC$.
\end{cor}
\begin{proof}
	Note that $x\mapsto x$ is continuous, so by induction $x\mapsto x^n$ is continuous for each $n\in\NN$.
\end{proof}
Here is another sort of arithmetic.
\begin{lemma}
	The composition of two continuous functions is continuous.
\end{lemma}
\begin{proof}
	Omitted.
\end{proof}

\subsection{Connectedness}
We want to build towards a particular type of continuous function.
\begin{proposition}
	Fix $X\subseteq\CC$ a connected subset. Then a continuous function $f:X\to\CC$ has connected image $f(X)$.
\end{proposition}
\begin{proof}
	The main point is to use the topological characterization of continuity. In particular, suppose that $f(X)$ is disconnected, and we show that $X$ is disconnected. In particular, suppose that $U_1$ and $U_2$ disconnect $f(X)$, and we have that $f^{-1}(U_1)$ and $f^{-1}(U_2)$ disconnect $X$. We will not run all the checks here; the main point is that $f^{-1}(U_1)$ and $f^{-1}(U_2)$ are open because $f$ is continuous.
\end{proof}
\begin{definition}[Path]
	A \textit{path} in $\CC$ is a continuous function $\gamma:[a,b]\to\CC$ where $a<b$ are real numbers.
\end{definition}
\begin{definition}
	We say that a path $\gamma$ is \textit{closed} if and only if $\gamma(a)=\gamma(b)$. We say that $\gamma$ is \textit{simple} if and only if $\gamma:(a,b)\to\CC$ is injective.
\end{definition}
\begin{remark}
	We restrict $\gamma$ to the closed interval at the end so that closed, simple paths are allowed to exist.
\end{remark}
\begin{example}
	Here is a path.
	\begin{center}
		\begin{asy}
			unitsize(1cm);
			draw((-1,0) .. (1,0) .. (1,1) .. (0,1/2) .. (1,0) .. (3/2,1/2) .. (2,0));
		\end{asy}
	\end{center}
\end{example}
\begin{example}
	Here is a closed path.
	\begin{center}
		\begin{asy}
			unitsize(1cm);
			draw((-1,0) .. (1,0) .. (1,1) .. (0,1/2) .. (1,0) .. (3/2,1/2) .. (2,0) .. (0,0) .. cycle);
		\end{asy}
	\end{center}
\end{example}
\begin{example}
	Here is a simple path.
	\begin{center}
		\begin{asy}
			unitsize(1cm);
			draw((-1,0) .. (1,0) .. (1,1) .. (0,1/2) .. (-0.5,1));
		\end{asy}
	\end{center}
\end{example}
\begin{example}
	Here is a closed, simple path, also called a loop.
	\begin{center}
		\begin{asy}
			unitsize(1cm);
			draw((-1,0) .. (1,0) .. (1,1) .. (0,1/2) .. (-0.5,1) .. cycle);
		\end{asy}
	\end{center}
\end{example}
\begin{definition}[Concatenation]
	Fix $\gamma_1:[a,b]\to\CC$ and $\gamma_2:[c,d]$ paths in $\CC$ such that $\gamma_1(b)=\gamma_2(c)$. Then we define the \textit{concatenation} of $\gamma_1$ and $\gamma_2$ to be
	\[(\gamma_1*\gamma_2)(t):=\begin{cases}
		\gamma_1(t) & t\in[a,b], \\
		\gamma_2(t) & t\in[b,d+c-b].
	\end{cases}\]
\end{definition}
The main point is that we are doing one path after the other.

Paths give us the following notion.
\begin{definition}[Path-connected]
	A subset $X\subseteq\CC$ is \textit{path connected} if and only if, for any two $x_0,x_1\in X$, there exists a path $\gamma:[0,1]\to X$ such taht $\gamma(0)=x_0$ and $\gamma(1)=x_1$.
\end{definition}
\begin{lemma}
	The open ball $B(z,r)$ and closed ball $\overline{B(z,r)}$ are both path-connected.
\end{lemma}
\begin{proof}
	The point is that $B(z,r)$ and $\overline{B(z,r)}$ are both convex, so the path
	\[\gamma(t):=z_0+t(z_1-z_0)\]
	will work.
\end{proof}
Here is the basic result.
\begin{proposition}
	A space $X$ is path-connected implies that $X$ is connected. If $X$ is open and connected, then $X$ is path-connected.
\end{proposition}
\begin{proof}
	We will show this next class.
\end{proof}