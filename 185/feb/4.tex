% !TEX root = ../notes.tex

Today we are talking about continuity.
\begin{warn}
	The first half of this lecture was transcribed from Professor Morrow's notes because I had to miss class for a job interview
\end{warn}

\subsection{Limits}
Before defining continuity, we have the following definitions.
\begin{definition}[Limit]
	Fix $f:X\to\CC$ a function and $z_0\in\overline X$. Then we say \textit{the limit of $f(z)$ as $z$ approaches $z_0$ equals $w$}, denoted
	\[\lim_{z\to z_0}f(z)=w,\]
	if and only if, for each $\varepsilon>0$, there exists $\delta>0$ such that
	\[|z-z_0|<\delta\implies|f(z)-w|<\varepsilon\]
	for $z\in X$.
\end{definition}
This is the standard $\varepsilon$-$\delta$ definition.

We also pick up the following convention as a surprise tool that may help us later.
\begin{definition}[Infinite limits]
	Fix $f:X\to\CC$ a function. Then we say \textit{the limit of $f(z)$ as $z$ tends to infinity equals $w$}, denoted
	\[\lim_{z\to\infty}f(z)=w,\]
	if and only if, for each $\varepsilon>0$, there exists $N>0$ such that
	\[|z|>N\implies|f(z)-w|<\varepsilon\]
	for $z\in X$.
\end{definition}

As in real analysis, the $\varepsilon$-$\delta$ definition of a limit can be translated to a statement about sequences.
\begin{proposition} \label{prop:altlimit}
	Fix $\alpha\in\overline X$. Then $\lim_{z\to\alpha}f(z)=w$ if and only if, for each $\{z_n\}_{n\in\NN}\subseteq X$ such that $z_n\to\alpha$ as $n\to\infty$, we have $f(z_n)\to w$ as $n\to\infty$.
\end{proposition}
\begin{proof}
	In the forwards direction, fix $\{z_n\}_{n\in\NN}\subseteq X$ such that $z_n\to\alpha$, and we show that $f(z_n)\to w$. Well, for any $\varepsilon>0$, there exists $\delta>0$ such that
	\[|z-\alpha|<\delta\implies|f(z)-f(\alpha)|<\varepsilon,\]
	where $z\in X$. But for this $\delta>0$, there exists $N$ such that
	\[n>N\implies|z_n-\alpha|<\delta\implies|f(z_n)-f(\alpha)|<\varepsilon.\]
	So indeed, $f(z_n)\to f(\alpha)$.

	In the reverse direction, suppose that $f(z)$ does not approach $w$ as $z\to\alpha$. Then, there exists $\varepsilon_0>0$ such that, for any $\delta>0$, there is $z\in X$ such that $|z-\alpha|<\delta$ while $|f(z)-w|>\varepsilon_0$. Well, for any $n\in\NN$, taking $\delta=1/(n+1)$, we are promised $z_n\in X$ such that
	\[|z_n-\alpha|<\frac1{n+1}qquad\text{and}\qquad|f(z_n)-w|>\varepsilon_0.\]
	So to finish, we claim that $z_n\to\alpha$ as $n\to\infty$, but $f(z_n)$ does not approach $w$ as $n\to\infty$.
	\begin{itemize}
		\item For any $\varepsilon>0$, we note that $N:=1/\varepsilon$ has $n>N$ implies
		\[|z_n-\alpha|<\frac1{n+1}<\frac1{N+1}<\frac1N=\varepsilon,\]
		so indeed $z_n\to\alpha$ as $N\to\infty$.
		\item We note that $\varepsilon_0>0$ satisfies that
		\[|f(z_n)-w|>\varepsilon_0\]
		for any $n\in\NN$, so no $N$ will have $n>N$ implies $|f(z_n)-w|<\varepsilon_0$. Thus, $f(z_n)$ does not approach $w$ as $n\to\infty$.
	\end{itemize}
	The sequence $\{z_n\}_{n\in\NN}$ now completes the proof by showing the reverse direction by contraposition.
\end{proof}

While we're here, we pick up the following definitions.
\begin{definition}[Bounded]
	A function $f:X\to\CC$ is \textit{bounded} if there exists $R>0$ such that $\im f\subseteq B(0,R)$.
\end{definition}
\begin{definition}[Bounded near]
	Fix a nonempty open subset $\Omega\subseteq\CC$ and $z_0\in\Omega$. Then $f:\Omega\setminus\{z_0\}\to\CC$ is \textit{bounded near} $z_0$ if and only if
	\[\lim_{z\to z_0}(z-z_0)f(z)=0.\]
\end{definition}

\subsection{Continuity}
And here is our central definition for today.
\begin{definition}[Continuous]
	A function $f:X\to\CC$ is \textit{continuous at $z_0\in X$} if and only if, for each $\varepsilon>0$, there exists $\delta>0$ such that
	\[|z-z_0|<\delta\implies|f(z)-f(z_0)|<\varepsilon,\]
	where $z\in X$. Further, $f$ is \textit{continuous on $X$} if and only if $f$ is continuous at each $z_0\in X$.
\end{definition}
We have the following lemma of equivalent definitions.
\begin{restatable}{lemma}{continuitygrabbag} \label{lem:topologicalcontinuity}
	Suppose that $f:X\to\CC$.
	\begin{listalph}
		\item Then $f$ is continuous at $w$ if and only if every sequence $\{z_n\}\subseteq X$ such that $z_n\to z$ implies $f(z_n)\to f(z)$.
		\item We have that $f$ is continuous on $X$ if and only if every open set $U\subseteq\CC$ has $f^{-1}(U)$ open in $X$.
		\item We have that $f$ is continuous on $X$ if and only if each closed set $V\subseteq X$ has $f^{-1}(V)$ closed in $X$.
		\item Lastly, we have that $f$ is continuous at if and only if, for each $\varepsilon>0$ and $z\in\CC$, we have that $f^{-1}(B(z,\varepsilon))$ is open in $X$.
	\end{listalph}
\end{restatable}
\begin{proof}
	We take the parts one at a time.
	\begin{listalph}
		\item We could use \autoref{prop:altlimit}, but we will just do this by hand. For the forwards direction, suppose that $\{z_n\}_{n\in\NN}\subseteq X$ converges to some $w$. Then let $\varepsilon>0$. By assumption, there exists some $\delta>0$ such that
		\[|z-w|<\delta\implies|f(x)-f(w)|<\varepsilon.\]
		It follows from $z_n\to w$ that there exists some $N$ such that
		\[n>N\implies|z_n-w|<\delta\implies|f(z_n)-f(z)|<\varepsilon,\]
		so it follows that $f(z_n)\to f(z)$.
		
		In the reverse direction, take $f$ not continuous at $w$, so there exists $\varepsilon>0$ so that for all $n\in\NN$, there exists some chosen $z_n$ with
		\[|z_n-w|<\delta\implies|f(z_n)-f(w)|>\varepsilon.\]
		But as $z_n\to w$, we see that $f(z_n)$ does not approach $f(w)$, so we are done.

		\item In the forwards direction, suppose that $U\subseteq\CC$ is open, and we show that $f^{-1}(U)$ is open in $X$. Well, suppose that $z\in f^{-1}(U)$, and we will find $\delta>0$ such that $B(z,\delta)\subseteq f^{-1}(U)$. 
		
		Well, $f(z)\in U$, so there exists $\varepsilon>0$ such that $B(f(z),\varepsilon)\subseteq U$. Thus, continuity of $f$ requires some $\delta>0$ such that
		\[|w-z|<\delta\implies|f(w)-f(z)|<\varepsilon,\]
		which implies $f(w)\in B(f(z),\varepsilon)\subseteq U$ implies $w\in f^{-1}(U)$. So indeed, $B(z,\delta)\subseteq f^{-1}(U)$.

		In the reverse direction, suppose that each open $U\subseteq\CC$ has $f^{-1}(U)$ is open. Now fix any $z\in X$ and $\varepsilon>0$. The set $B(f(z),\varepsilon)$ is open, so
		\[f^{-1}(B(f(z),\varepsilon))\]
		is open. But $z\in f^{-1}(B(f(z),\varepsilon))$, so we can find $\delta>0$ such that $B(z,\delta)\subseteq f^{-1}(B(f(z),\varepsilon))$. Thus, $|w-z|<\delta$ implies $w\in f^{-1}(B(f(z)),\varepsilon)$ implies $f(w)\in B(f(z),\varepsilon)$ implies $|f(w)-f(z)|<\varepsilon$, finishing.

		\item In the forwards direction, suppose $f$ is continuous so that $U\subseteq\CC$ open implies $f^{-1}(U)\subseteq X$ is open. But then, if $V\subseteq\CC$ is closed, then $\CC\setminus V$ is open, so\footnote{To see $f^{-1}(A\setminus B)=f^{-1}(A)\setminus f^{-1}(B)$, note that $x\in f^{-1}(A\setminus B)$ if and only if $f(x)\in A\setminus B$ if and only if $f(x)\in A$ but $f(x)\notin B$ if and only if $x\in f^{-1}(A)$ but $x\notin f^{-1}(B)$.}
		\[f^{-1}(\CC\setminus V)=f^{-1}(\CC)\setminus f^{-1}(V)=X\setminus f^{-1}(V)\]
		is open, so $f^{-1}(V)$ is closed.

		In the backwards direction, suppose $f^{-1}$ preserves closed sets. Then, if $U\subseteq\CC$ is open, $\CC\setminus U$ is closed, so
		\[f^{-1}(\CC\setminus U)=f^{-1}(\CC)\setminus f^{-1}(U)=X\setminus f^{-1}(U)\]
		is closed, so $f^{-1}(U)$ is open. Thus, $f^{-1}$ preserves open sets, so $f$ must be continuous.

		\item In the forwards direction, fix $\varepsilon>0$ and $z\in\CC$, so $B(z,\varepsilon)$ is open, so $f^{-1}(B(z,\varepsilon))$ is open in $X$, finishing.
		
		In the other direction fix $\varepsilon>0$ and $z\in\CC$ to consider $B(f(z),\varepsilon)\subseteq U$. Thus, continuity of $f$ requires some $\delta>0$ such that
		\[|w-z|<\delta\implies|f(w)-f(z)|<\varepsilon,\]
		which implies $f(w)\in B(f(z),\varepsilon)\subseteq U$ implies $w\in f^{-1}(U)$. So indeed, $B(z,\delta)\subseteq f^{-1}(U)$.

		In the reverse direction, fix $U$ open, and we show that $f^{-1}(U)$ is open. Well, each $z\in U$ has some $\varepsilon_z$ such that $B(z,\varepsilon_z)\subseteq U$. But $f^{-1}(B(z,\varepsilon_z))$ is open by hypothesis, so
		\[f^{-1}(U)=f^{-1}\left(\bigcup_{z\in U}B(z,\varepsilon_z)\right)=\bigcup_{z\in U}f^{-1}(B(z,\varepsilon_z))\]
		is an arbitrary union of open sets and hence open.
		\qedhere
	\end{listalph}
\end{proof}

And here are some special examples.
\begin{example}
	Fix some $z_0\in\CC$. Then $f(z):=|z-z_0|$ is continuous on $\CC$. Indeed, fix any $w\in\CC$. Then for any $\varepsilon>0$, we set $\delta:=\varepsilon$ so that $|z-w|<\delta$ implies
	\[|f(z)-f(w)|=\big||z-z_0|-|w-z_0|\big|\le|z-w|<\delta=\varepsilon.\]
\end{example}
\begin{example}
	The functions $\Re$ and $\Im$ is continuous. Indeed, fix any $w\in\CC$. Then, for any $\varepsilon>0$, take $\delta:=\varepsilon$ so that $|z-w|<\delta$ implies
	\[|\Re z-\Re w|=|\Re(z-w)|\le|z-w|<\delta=\varepsilon,\]
	and similarly,
	\[|\Im z-\Im w|=|\Im(z-w)|\le|z-w|<\delta=\varepsilon,\]
\end{example}
Continuous functions also have some arithmetic.
\begin{proposition} \label{prop:combinecontfuncs}
	Fix $f,g:X\to\CC$ to functions continuous at $z_0\in X$. Then $f+g,f\cdot g$ are both continuous at $z_0\in X$, and $f/g$ is continuous at $z_0$ provided $g(z_0)\ne0$.
\end{proposition}
\begin{proof}
	The point is to appeal to the corresponding results on convergence of sequences. In particular, we use the idea that $f$ is continuous at $z_0$ if and only if each sequence $z_n\to z_0$ in $X$ has $f(z_n)\to f(z_0)$. We omit the details because they are essentially the same as in a real analysis class.
\end{proof}
\begin{cor}
	Every polynomial in one variable is a continuous function $X\to\CC$ for any $X\subseteq\CC$.
\end{cor}
\begin{proof}
	Note that $x\mapsto x$ is continuous, so by induction $x\mapsto x^n$ is continuous for each $n\in\NN$. Taking a $\CC$-linear combination gives arbitrary polynomials.
\end{proof}
Here is another sort of arithmetic.
\begin{lemma}
	The composition of two continuous functions is continuous.
\end{lemma}
\begin{proof}
	Omitted.
\end{proof}

\subsection{Connectedness}
We want to build towards a particular type of continuous function.
\begin{proposition} \label{prop:contconnectedimage}
	Fix $X\subseteq\CC$ a connected subset. Then a continuous function $f:X\to\CC$ has connected image $f(X)$.
\end{proposition}
\begin{proof}
	The main point is to use the topological characterization of continuity. In particular, suppose that $f(X)$ is disconnected, and we show that $X$ is disconnected. In particular, suppose that $U_1$ and $U_2$ disconnect $f(X)$, and we have that $f^{-1}(U_1)$ and $f^{-1}(U_2)$ disconnect $X$. We will not run all the checks here; the main point is that $f^{-1}(U_1)$ and $f^{-1}(U_2)$ are open because $f$ is continuous.
\end{proof}
\begin{restatable}[Path]{defihelper}{pathdef} \nirindex{Path}
	A \textit{path} in $\CC$ is a continuous function $\gamma:[a,b]\to\CC$ where $a<b$ are real numbers.
\end{restatable}
\begin{definition}
	We say that a path $\gamma$ is \textit{closed} if and only if $\gamma(a)=\gamma(b)$. We say that $\gamma$ is \textit{simple} if and only if $\gamma:(a,b)\to\CC$ is injective.
\end{definition}
\begin{remark}
	The point of restricting $\gamma$ to the open interval at the end so that closed, simple paths are allowed to exist.
\end{remark}
\begin{example}
	Here is a path.
	\begin{center}
		\begin{asy}
			unitsize(1cm);
			draw((-1,0) .. (1,0) .. (1,1) .. (0,1/2) .. (1,0) .. (3/2,1/2) .. (2,0));
		\end{asy}
	\end{center}
\end{example}
\begin{example}
	Here is a closed path.
	\begin{center}
		\begin{asy}
			unitsize(1cm);
			draw((-1,0) .. (1,0) .. (1,1) .. (0,1/2) .. (1,0) .. (3/2,1/2) .. (2,0) .. (0,0) .. cycle);
		\end{asy}
	\end{center}
\end{example}
\begin{example}
	Here is a simple path.
	\begin{center}
		\begin{asy}
			unitsize(1cm);
			draw((-1,0) .. (1,0) .. (1,1) .. (0,1/2) .. (-0.5,1));
		\end{asy}
	\end{center}
\end{example}
\begin{example}
	Here is a closed, simple path, also called a loop.
	\begin{center}
		\begin{asy}
			unitsize(1cm);
			draw((-1,0) .. (1,0) .. (1,1) .. (0,1/2) .. (-0.5,1) .. cycle);
		\end{asy}
	\end{center}
\end{example}
\begin{definition}[Concatenation]
	Fix $\gamma_1:[a,b]\to\CC$ and $\gamma_2:[c,d]$ paths in $\CC$ such that $\gamma_1(b)=\gamma_2(c)$. Then we define the \textit{concatenation} of $\gamma_1$ and $\gamma_2$ to be
	\[(\gamma_1*\gamma_2)(t):=\begin{cases}
		\gamma_1(t) & t\in[a,b], \\
		\gamma_2(t-b+c) & t\in[b,d-c+b].
	\end{cases}\]
\end{definition}
The main point is that we are doing one path after the other.
\begin{example}
	The following shows an example concatenation of $\gamma_1*\gamma_2$, where $\gamma_1,\gamma_2:[0,1]\to\CC$.
	\begin{center}
		\begin{asy}
			unitsize(0.7cm);
			draw((0,0) .. (1,0.3) .. (1.7,-0.5) .. (4,0.1) .. (6,0), arrow=EndArrow, p=red);
			draw((6,0) .. (6,-1) .. (8,0) .. (10,-1) .. (9,1), arrow=EndArrow, p=blue);
			dot("$t=0$", (0,0), W);
			dot("$t=1$", (6,0), NE);
			dot("$t=2$", (9,1), NW);
			label("$\gamma_1$", (4,0.1), NNW, red);
			label("$\gamma_2$", (11,0.1), E, blue);
		\end{asy}
	\end{center}
	The entire path is $\gamma_1*\gamma_2$.
\end{example}

Paths give us the following notion.
\begin{definition}[Path-connected]
	A subset $X\subseteq\CC$ is \textit{path connected} if and only if, for any two $x_0,x_1\in X$, there exists a path $\gamma:[0,1]\to X$ such that $\gamma(0)=x_0$ and $\gamma(1)=x_1$.
\end{definition}
\begin{lemma}
	The open ball $B(z,r)$ and closed ball $\overline{B(z,r)}$ are both path-connected.
\end{lemma}
\begin{proof}
	The point is that $B(z,r)$ and $\overline{B(z,r)}$ are both convex, so the path
	\[\gamma(t):=z_0+t(z_1-z_0)\]
	will work.
\end{proof}
Here is the basic result.
\begin{restatable}{proposition}{openconnected} \label{prop:getpathconnected}
	A space $X$ is path-connected implies that $X$ is connected. If $X$ is open and connected, then $X$ is path-connected.
\end{restatable}
\begin{proof}
	We will show this next class.
\end{proof}