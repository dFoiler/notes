% LTeX: enabled=false

\documentclass[openany]{book}
\usepackage[utf8]{inputenc}

\newcommand{\nirpdftitle}{185 Notes}
\usepackage{import}
\inputfrom{..}{nir}
\graphicspath{{figs/}}

\pagestyle{contentpage}

\title{185: Introduction to Complex Analysis}
\author{Nir Elber}
\date{Spring 2022}
\rhead{\textit{185: INTRO. TO COMPLEX ANALYSIS}}

\begin{document}

\maketitle

\toctrue
\tableofcontents
\tocfalse

\newpage

\chapter{Introduction}

\epigraph{Our reality isn't about what's real, it's about what we pay attention to.}
{---Hank Green}
% A Beautifully Foolish Endeavor

\foreach \n in {19,21}
{
	\section{January \n}
	\input{jan/\n}
}

\chapter{Complex Numbers and Their Topology}

\epigraph{This somewhat laborious proof could have been avoided if one had defined a complex analytic structure}
{---Jean-Pierre Serre}
% A Course in Arithmetic

\foreach \n in {24,26,28,31}
{
	\section{January \n}
	\input{jan/\n}
}

\foreach \n in {2,4,7,9}
{
	\section{February \n}
	\input{feb/\n}
}

\chapter{Differentiation}

\epigraph{I turn with terror and horror from this lamentable scourge of continuous functions with no derivatives.}
{---Charles Hermite}

\foreach \n in {11,14,16,18,23,25,28}
{
	\section{February \n}
	\input{feb/\n}
}

\foreach \n in {2,4}
{
	\section{March \n}
	\input{mar/\n}
}

\chapter{Integration}

\epigraph{How strange to actually have to see the path of your journey in order to make it.}
{---Neal Shusterman}

\foreach \n in {7,9,11,14,16,18,28}
{
	\section{March \n}
	\input{mar/\n}
}

\chapter{Smoothing Over}

\epigraph{What we didn't do is make the construction at all usable in practice! This time we will remedy this.}
{---Kiran S. Kedlaya}

\foreach \n in {30}
{
	\section{March \n}
	\input{mar/\n}
}

\foreach \n in {1,4,6,8,11,13}
{
	\section{April \n}
	\input{apr/\n}
}

\chapter{Extra Topics}

\epigraph{You take the red pill, you stay in wonderland, and I show you how deep the rabbit hole goes.}
{---Morpheus}

\foreach \n in {18,20,22,25,27,29}
{
	\section{April \n}
	\input{apr/\n}
}

% until "The exponential function II"

% "Integration" from "Paths" to "Consequences of Cauchy Integral Formula II"
% might use "How strange to actually have to see the path of your journey in order to make it." from Scythe for the Paths section
% not sure how to organize the rest of this ...

% the definition of "bounded near" seems off? f(z) = 1/sqrt(z) is bounded near 0?

% do we need to know about differentiation of power series?
% does "knowing the theorem statements" merely mean knowing the the theorem statements? what does knowing "continuous on compact implies uniformly continuous" mean?
% should we be expected to be able to give proofs of the listed theorems?
% should it be obvious that C--R is sufficient?

% compactification
% manifolds and branch cuts

% this analytic check for exp seems *really* annoying

% copy and pasted proofs of inverse derivatives and chain rule---surely there is a more general framework we are accessing?
% conformal map theorem shouldn't need args?
% conformal when things are equal to 0?
% does preserving right angles impoly preserving all angles?

% rigorize changing order of summation?
% why is the Cauchy--Goursat theorem so hard?

% sharpen entire has dense image
% something about zeta with arbitrary approximation
% entire equals power series

% argument principle if we go where there is a zero
% the uniform thing, citing Miles
% brag about diagrams
% homotopy things from residue theorem?

\nirprintindex

\end{document}