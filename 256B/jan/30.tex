% !TEX root = ../notes.tex

\documentclass[../notes.tex]{subfiles}

\begin{document}

\section{January 30}

Homework was assigned and still due on Friday, sadly.
\begin{remark}
	Let $k$ be an algebraically closed field of positive characteristic $p$. It turns out $G\coloneqq\pi_1^{\text{\'et}}(\AA^1_k)$ is profinite but not topologically finitely generated---it's very large. In fact, one can show that any finite $p$-group arises as a quotient of $\pi_1^{\text{\'et}}(\AA^1_k)$. More generally, any finite quasi-$p$-group is a quotient, where a quasi-$p$-group is a finite group generated by its Sylow $p$-subgroups cover $G$.
\end{remark}

\subsection{Everything Is Frobenius}
Thus far we roughly understand finite separable morphisms of curves. We now investigate the purely inseparable case. In particular, today $k$ will be an algebraically closed field of positive characteristic $p$. Note there is a canonical embedding $\FF_p\into k$, which gives rise to the Frobenius automorphism as follows.
\begin{definition}[Frobenius]
	Let $k$ be a field of characteristic $p>0$. Given a $k$-scheme $X$, we define the \textit{Frobenius automorphism} $F\colon X\to X$ as being the identity on topological spaces and being the $p$th-power map $F^\sharp_U\colon\OO_X(U)\to\OO_X(U)$ for all open $U\subseteq X$.
\end{definition}
We can see that this map takes units of $\OO_{X,x}$ to units of $\OO_{X,x}$ for any $x\in X$, so we have defined a morphism of locally ringed spaces.
\begin{example}
	Let $X=\Spec A$ be an $\mathbb F_p$-scheme. Then the ring homomorphism $F\colon A\to A$ given by the $p$th power map is the Frobenius $F\colon X\to X$. To show the map is the identity on the topological space, we note $a^p\in\mf p$ is equivalent to $a\in\mf p$ for $a\in A$, where we are using the primality of $\mf p$. Thus, $F^{-1}(\mf p)=\mf p$ for any prime $\mf p\in\Spec A$.
\end{example}
The Frobenius map defined above is not $k$-linear because it is the $p$th power map on $k$ too. To make this $k$-linear, we essentially cheat.
\begin{definition}
	Fix a scheme $X$ over a field $k$ of characteristic $p$. Then we define the $k$-scheme $X_p$ which is equal to $X$ as a scheme but whose structure morphism to $\Spec k$ is given by
	\[X_p\to\Spec k\stackrel F\to\Spec k.\]
\end{definition}
The point is that the diagram
% https://q.uiver.app/?q=WzAsNCxbMCwwLCJYX3AiXSxbMSwwLCJYIl0sWzEsMSwiXFxTcGVjIGsiXSxbMCwxLCJcXFNwZWMgayJdLFswLDEsIkYiXSxbMywyLCJGIl0sWzEsMl0sWzAsM10sWzAsMl1d&macro_url=https%3A%2F%2Fraw.githubusercontent.com%2FdFoiler%2Fnotes%2Fmaster%2Fnir.tex
\begin{equation}
	% https://q.uiver.app/?q=WzAsNCxbMCwwLCJYX3AiXSxbMSwwLCJYIl0sWzEsMSwiXFxTcGVjIGsiXSxbMCwxLCJcXFNwZWMgayJdLFswLDEsIkYiXSxbMywyLCJGIl0sWzEsMl0sWzAsM11d&macro_url=https%3A%2F%2Fraw.githubusercontent.com%2FdFoiler%2Fnotes%2Fmaster%2Fnir.tex
	\begin{tikzcd}
		{X_p} & X \\
		{\Spec k} & {\Spec k}
		\arrow["F", from=1-1, to=1-2]
		\arrow["F", from=2-1, to=2-2]
		\arrow[from=1-2, to=2-2]
		\arrow[from=1-1, to=2-1]
	\end{tikzcd} \label{eq:frob-square}
\end{equation}
commutes, so we do genuinely have a Frobenius morphism $F\colon X_p\to X$.
\begin{remark}
	Explicitly, for each affine open $\Spec A\subseteq X$, we get an identical affine open $\Spec A_p\subseteq X_p$, but if the $k$-algebra structure on $A$ is given by $g\colon k\to A$, then the $k$-algebra structure on $A$ is given by $g_p(x)\cdot\alpha\coloneqq g\left(x^p\right)\cdot\alpha$.
\end{remark}
Namely, with everything being contravariant on functions, we see that
% https://q.uiver.app/?q=WzAsNCxbMCwwLCJBIl0sWzEsMCwiQSJdLFsxLDEsImsiXSxbMCwxLCJrIl0sWzIsMywiRiIsMl0sWzEsMCwiRiIsMl0sWzIsMV0sWzMsMF1d&macro_url=https%3A%2F%2Fraw.githubusercontent.com%2FdFoiler%2Fnotes%2Fmaster%2Fnir.tex
\[\begin{tikzcd}
	A & A \\
	k & k
	\arrow["F"', from=2-2, to=2-1]
	\arrow["F"', from=1-2, to=1-1]
	\arrow[from=2-2, to=1-2]
	\arrow[from=2-1, to=1-1]
\end{tikzcd}\]
commutes.
\begin{remark}
	Certainly $X_p\cong X$ as schemes because they are literally the same data. If $k$ is perfect, then $X_p\cong X$ is an isomorphism of $k$-schemes because the $p$th-power map on $k$ is an isomorphism.\todo{Why?}
\end{remark}
\begin{example} \label{ex:perf-xp-is-x}
	Let $k$ be a perfect field of characteristic $p>0$. Set $X=\AA^1_k=\Spec k[t]$. Then $F\colon X_p\to X$ is given by the morphism $k[t_p]\to k[t]$ by $f\mapsto f^p$. Thus, to witness our isomorphism of $k$-schemes, we note that we can post-compose $k[t_p]\to k[t]$ with the morphism which extends $k\cong k$ by $a^p\mapsto a$ by $t_p\mapsto t_p$, so we have made a $k$-linear isomorphism. One can essentially extend this construction to work in general when $k$ is perfect.
\end{example}
\begin{example}
	Let $k$ be a field of characteristic $p>0$, and let $X$ be an integral $k$-scheme. Then $F\colon X_p\to X$ is given by a morphism $K(X)\to K(X_p)$ by $\alpha\mapsto\alpha^p$. (Yes, this is $k$-linear because the $k$-action on $K(X_p)$ is by $p$th powers.) As such, we have defined an embedding of $K(X)$ into an algebraic extension $K(X_p)$: namely, every $\alpha\in K(X_p)$ is a root of the polynomial $t^p-\alpha^p=0$ in $K(X)[t]$.
	
	Conversely, $t^p-\beta\in K(X)[t]$ always has a root in $K(X_p)$ because $\beta\in K(X)$ embeds into $K(X_p)$ as $\beta^p$, so this polynomial ``looks like'' $t^p-\beta^p$ in $K(X_p)$, where $\beta$ is our root. Thus,
	\[K(X_p)=K(X)^{1/p}.\]
\end{example}
The point is that a $k$-curve $X$ will have the Frobenius morphism $X_p\to X$ induced by the embedding $K(X)\to K(X)^{1/p}$.

Now, here is our main result.
\begin{theorem} \label{thm:all-is-frob}
	Let $k$ be an algebraically closed field of characteristic $p>0$. Further, let $f\colon X\to Y$ be a finite map of $k$-curves which induces a purely inseparable extension $f^\sharp\colon K(Y)\to K(X)$. Then $f$ is some iterate of the Frobenius morphism. In particular, $X$ and $Y$ are isomorphic as schemes and thus have the same genus.
\end{theorem}
\begin{proof}
	Note $\deg f=[K(X):K(Y)]$ is a power of $p$, which we call $p^\nu$. Being purely inseparable then enforces $K(X)\subseteq K(Y)^{1/p^\nu}$; namely, the minimal polynomial if all elements $\alpha\in K(X)$ must be the minimal polynomial of the form $x^{p^r}-\beta=0$, for otherwise there is a separable subextension, violating our pure inseparability.

	Now, consider iterated Frobenius morphisms
	\[Y_{p^\nu}\to Y_{p^{\nu-1}}\to\cdots\to Y_p\to Y,\]
	which corresponds to the inclusion of fields
	\[K(Y)\subseteq K(Y)^{1/p}\subseteq\cdots\subseteq K(Y)^{1/p^{\nu-1}}\subseteq K(Y)^{1/p^\nu},\]
	where the inclusions have reversed.
	
	Thus, to conclude, we would like to show $K(X)=K(Y)^{1/p^\nu}$. By degree arguments, it's enough to conclude $\left[K(Y)^{1/p^\nu}:K(Y)\right]=p^\nu$. By induction, it's enough to show $\left[K(Y)^{1/p}:K(Y)\right]=p$. We now must use the fact that $Y$ is a smooth $k$-curve, so we will push its proof into the following lemma.
	\begin{lemma}
		Let $k$ be an algebraically closed field of characteristic $p>0$. If $Y$ is a smooth $k$-curve, then $\left[K(Y)^{1/p}:K(Y)\right]=p$.
	\end{lemma}
	\begin{proof}
		Equivalently, we would like to show that $\left[K(Y):K(Y)^p\right]=p$. Note that $\Omega_{K(Y)/k}$ is a $K(Y)$-vector space of dimension $1$; we refer to Theorem~II.8.6.A, where the point is that $k$ being perfect tells us $K(Y)/k$ is separably generated, so $\dim_k\Omega_{K(Y)/k}$ is the transcendence degree of $K(Y)$ over $k$, which is $1$.

		We now note that $dx$ generates $\Omega_{K(Y)/k}$ if and only if $x\in K(Y)$ yields a power basis $\{1,x,\ldots,x^{p-1}\}$ of $K(Y)$ over $K(Y)^p$, which completes the proof.\todo{Okay.}
	\end{proof}
	It remains to show that the genus does not change. Well, $g=\dim_kH^1(X,\OO_X)$, and we see that the only difference between $X$ and $Y$ is the structure morphism, and this dimension does not change if we change the structure morphism.
\end{proof}
\begin{remark}
	The above proof basically shows that the Frobenius morphism $F\colon X_p\to X$ in our setting is a finite morphism of degree $p$. In particular, it cannot be an isomorphism.
\end{remark}

\end{document}