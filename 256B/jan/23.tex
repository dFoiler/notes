% !TEX root = ../notes.tex

\documentclass[../notes.tex]{subfiles}

\begin{document}

\section{January 20}

Today we apply the Riemann--Roch theorem.
\begin{remark}
	Here is a quick hint for the homework: fix a Weil divisor $D=\sum_Pn_PP$ on a $k$-curve $X$, where $k$ is algebraically closed. Then $\Gamma(X,\OO_X(D))$ can be described as space of rational functions $f$ on $X$ such that $D+\op{div}(f)$ is effective. In other words, for each point $P\in X$, we see $f$ has a pole of order at worse $n_P$ at $P$.
\end{remark}

\subsection{Applications of Riemann--Roch}
Let's give a few applications of \Cref{thm:rr}.
\begin{example} \label{ex:compute-deg-k}
	Fix a $k$-curve $X$, where $k$ is algebraically closed. Further, let $g$ be the genus of $X$ and $K$ the canonical divisor. We can compute $\deg K$ as follows: plugging into \Cref{thm:rr}, we see
	\[g-1=\ell(K)-\ell(0)=\deg K-1+g,\]
	so $\deg K=2g-2$.
\end{example}
\begin{remark}
	More generally, we can see that plugging in $K-D$ into \Cref{thm:rr} is only able to deduce $\deg K=2g-2$.
\end{remark}
\begin{example}
	Let $D$ be a divisor on a $k$-curve $X$, where $k$ is algebraically closed. Further, let $g$ be the genus of $X$ and $K$ the canonical divisor. We would like to study $\dim|nD|=\ell(nD)-1$ for $n\in\ZZ^+$. We have the following cases.
	\begin{itemize}
		\item If $\deg D<0$, then $\deg(nD)<0$ still, so $\ell(nD)=0$, so $\dim|nD|=-1$ always.
		\item If $\deg D=0$, then there are two possibilities. Namely, if $nD$ is linearly equivalent to $0$, then $\ell(nD)=1$, so $\dim|nD|=0$; otherwise, $D$ will not be linearly equivalent to any effective divisor (the only effective divisor with degree $0$ is $0$), so $\dim|nD|=-1$.
		\item If $\deg D>0$, then for $n$ large enough, we see $\deg(K-nD)<0$, so $\ell(K-nD)=0$, so \Cref{thm:rr} implies $\ell(nD)=n\deg D+1-g$, so $\dim|nD|=n\deg D-g$. Here, ``$n$ large enough'' is just $n>\deg K/\deg D$.
	\end{itemize}
\end{example}
Here is a more interesting corollary.
\begin{lemma} \label{lem:get-to-p1}
	Let $X$ be a $k$-curve, where $k$ is algebraically closed. Suppose that two distinct closed points $P$ and $Q$ produce linearly equivalent Weil divisors. Then $X\cong\PP^1_k$.
\end{lemma}
\begin{proof}
	We are given that $\op{div}(f)=P-Q$ for some $f\in K(X)$. Thus, we induce a map $k(t)\to K(X)$ given by $t\mapsto f$, where we view $k(t)$ as the fraction field of $\PP^1_k$. Notably, $t$ has a zero at $0$ and a pole at $\infty$, and $f$ has a zero at $Q$ and a pole at $P$. This will induce a finite map $g\colon X\to\PP^1$, which we can compute to have degree $1$ by the following discussion (notably, the pull-back of the divisor $[0]$ is $[P]$), so $g$ is a birational map and hence an isomorphism.

	Now, for any finite map of curves $g\colon X\to Y$, recall there is a map on divisors $g^*\colon\op{Cl}(Y)\to\op{Cl}(X)$ as follows: given point $P\in Y$ inside an affine open subscheme $V\subseteq Y$, we can take the pre-image to $X$ to produce a Weil divisor.\footnote{Alternatively, one can view this operation as the pullback $g^*\colon\op{Pic}Y\to\op{Pic}X$ and then recall that each element of the class group corresponds to an isomorphism class in $\op{Pic}$.} More formally, we send $P$ to
	\[g^*(P)\coloneqq\sum_{Q\in g^{-1}(\{P\})}v_Q(t)Q,\]
	where $t$ is a uniformizer parameter for $Q_{X,P}$, and $v_Q(t)$ is its valuation at the local ring $\OO_{X,Q}$.\todo{What?}
	
	In fact, we showed last semester that a divisor $D$ on $Y$ has
	\[\deg g^*D=(\deg g)(\deg D),\]
	where $\deg g=[K(X):K(Y)]$ is the degree of the morphism $g$. Let's recall the proof: it suffices to show this in the case where $D=P$ is a point. Plugging into the definition of $g^*$, we are showing that
	\[\sum_{Q\in g^{-1}(\{P\})}v_Q(t)=\deg g^*P\stackrel?=\deg g.\]
	This statement is local at $P$, so we may assume that $Y=\Spec B$, whereupon taking the pre-image along $g$ enforces $X=\Spec A$ for some $A$. For dimension-theory reasons, we see that $g$ is dominant, so the induced map $B\to A$ is injective.
	
	Localizing, we set $A'\coloneqq A\otimes_B\OO_{Y,P}$, so we are really interested in the map $\OO_{Y,P}\to A'$, which is still injective. It follows that $A'$ is a finite (by $g$) torsion-free (by this injectivity argument) module over $\OO_{Y,P}$. But $\OO_{Y,P}$ is a principal ideal domain, so we may appeal to the structure theorem. Namely, we want to compute the rank of $A'$ over $\OO_{Y,P}$, for which it suffices to take fraction fields everywhere and instead compute
	\[\op{rank}_{\OO_{Y,P}}A'=[\op{Frac}A':\op{Frac}\OO_{Y,P}]=[\op{Frac}A:\op{Frac}B].\]
	On the other hand, given uniformizer $t$ of $\OO_{X,P}$, we can compute the corresponding rank of $A'/tA'$ over $k=k(P)$ is $\deg g$. However, $\Spec A'/tA'$ is the pre-image of $P$, so we go ahead and note $A'/tA'$ is a product of local Artinian rings which are quotients corresponding to points in $g^{-1}(\{P\})$. In particular, for each $Q\in g^{-1}(P)$, we see $v_Q(t)=m$ means that $\OO_{X,Q}$ appears in $A'/tA'$ as $\OO_Q/(\varpi_Q)^{m}$, where $\varpi_Q$ is a uniformizer at $Q$. So we can write
	\[A'/tA'=\prod_{Q\in g^{-1}(\{P\})}\OO_{X,Q}/(\varpi_Q)^{v_Q(t)}.\]
	But the $k$-rank of this is $\deg g^*(\{P\})$ by definition of $g^*$, which must equal the $\OO_{Y,P}$-rank of $A'$, so we are done.
\end{proof}
\begin{corollary}
	Let $X$ be a $k$-curve of genus $0$, where $k$ is algebraically closed. Then $X$ is isomorphic to $\PP^1_k$.
\end{corollary}
\begin{proof}
	As an aside, we note that $\PP^1_k$ is certainly a $k$-curve of genus $0$.

	Quickly, choose any two points $P$ and $Q$ on $X$. As such, we take $D\coloneqq P-Q$ so that $\deg(K-D)=-2<0$, so $\ell(K-D)=0$. Thus, \Cref{thm:rr} implies $\ell(D)=1$. Thus, $D$ is linearly equivalent to an effective divisor, but the only effective divisor with degree $0$ is $0$ itself, so we see that $P-Q$ is linearly equivalent to $0$. This is enough to conclude that $X\cong\PP^1_k$ by \Cref{lem:get-to-p1}; note that $X$ being a curve requires $X$ to have infinitely many points and thus distinct points.
\end{proof}
Lastly, let's give a corollary for elliptic curves.
\begin{definition}[elliptic]
	A (proper) $k$-curve $X$ is \textit{elliptic} if and only if $X$ has genus $1$.
\end{definition}
\begin{corollary}
	Let $X$ be an elliptic $k$-curve, where $k$ is algebraically closed. We give $X(k)$ a group law arising from $\op{Pic}X$.
\end{corollary}
\begin{proof}
	Let $K$ be a canonical divisor for $X$, and we see $\deg K=0$ by \Cref{ex:compute-deg-k}. However, $\ell(K)=1$ is the genus, so $K$ is linearly equivalent to some effective divisor, so as usual we note that $K$ is linearly equivalent to $0$.

	Quickly, we note that the group structure on the Picard group $\op{Pic}X$ of isomorphism classes of line bundles on $X(k)$ induces a group law on $X$. Indeed, fix some $k$-point $P_0\in X$. We now claim that the map $X(k)\to\op{Pic}^0X$ given by
	\[P\mapsto\OO_X(P-P_0)\]
	is a bijection. (Here, $\op{Pic}^0X$ is the subgroup of degree-$0$ line bundles.) This will give $X(k)$ a group law by stealing it from $\op{Pic}X$.
	
	Because we already know that $\op{Pic}X$ is in bijection with divisors more generally, it's enough to show that any divisor $D$ of degree $0$ is linearly equivalent to a divisor of the form $P-P_0$ for $P\in X(k)$. Well, we use \Cref{thm:rr} with $D+P_0$, which yields
	\[\ell(D+P_0)-\ell(K-D-P_0)=1+1-g=1,\]
	but $K-D-P_0$ has degree $-1$ and so $\ell(K-D-P_0)=0$. Thus, $\dim|D+P_0|=0$, so there is a unique effective divisor of degree $1$ linearly equivalent to $D+P_0$. However, an effective divisor of degree $1$ is just a point $P$, so we are done.
\end{proof}

\end{document}