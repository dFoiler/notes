% !TEX root = ../notes.tex

\documentclass[../notes.tex]{subfiles}

\begin{document}

\section{January 27}

There is homework due tonight. Today we keep talking towards the Riemann--Hurwitz formula.

\subsection{The Riemann--Hurwitz Formula}
Throughout, $f\colon X\to Y$ is a finite separable morphism of $k$-curves, where $k$ is algebraically closed.
\begin{definition}[ramification divisor]
	Let $f\colon X\to Y$ be a finite separable morphism of $k$-curves, where $k$ is algebraically closed. Then the \textit{ramification divisor} is the divisor
	\[R(f)\coloneqq\sum_{x\in X}\op{length}(\Omega_{Y/X})_xx.\]
	Note that there are only finitely many ramified points, so this is indeed a divisor.
\end{definition}
In particular, in tame ramification, this length is in fact exactly our ramification.
\begin{lemma} \label{lem:use-ramification}
	Let $f\colon X\to Y$ be a finite separable morphism of $k$-curves, where $k$ is algebraically closed. Further, let $K_X$ and $K_Y$ denote the canonical divisors. Then $K_X$ is linearly equivalent to $f^*K_Y+R$.
\end{lemma}
\begin{proof}
	We can see by hand that the structure sheaf $\OO_R$ of $R$ as a closed subscheme is exactly $\Omega_{Y/X}$, so the exact sequence
	\[0\to f^*\Omega_{Y/k}\to\Omega_{X/k}\to\Omega_{Y/X}\to0\]
	can be tensored with $\Omega_{X/k}^{-1}$ to give\todo{What?}
	\[0\to f^*\Omega_{Y/k}\otimes\Omega_{X/k}^{-1}\to\OO_X\to\OO_R\to0.\]
	Now, we see $\OO_R=\OO_X(-R)$ by unwinding definitions, so $f^*\Omega_{Y/k}\otimes\Omega_{X/k}^{-1}=\OO_X(-R)$, which is what we wanted.
\end{proof}
\begin{theorem}[Hurwitz] \label{thm:rh}
	Let $f\colon X\to Y$ be a finite separable map of $k$-curves, where $k$ is algebraically closed. Letting $n\coloneqq\deg f$, we have
	\[2g(X)-2=n\cdot(2g(Y)-2)+\deg R(f).\]
\end{theorem}
\begin{proof}
	Take degrees of \Cref{lem:use-ramification}.
\end{proof}
Let's derive some corollaries.
\begin{defihelper}[\'etale] \nirindex{etale@\'etale}
	A morphism locally of finite presentation $f\colon X\to Y$ of schemes is \textit{\'etale} at $x\in X$ if and only if the following conditions hold.
	\begin{itemize}
		\item $f$ is flat at $x$. In other words, $\OO_{X,x}$ is flat as an $\mathcal O_{Y,f(x)}$-module.
		\item $\OO_{X,x}/\mf m_{f(x)}\OO_{X,x}$ is a field and separable over $k(f(x))$. Equivalently, we are requiring $\mf m_{f(x)}\OO_{X,x}=\mf m_x$ and for the residue field extension to be separable.
	\end{itemize}
	Then $f$ is \'etale if and only if $f$ is \'etale at all points.
\end{defihelper}
\begin{remark}
	The locus of points for which a morphism is \'etale is open essentially because this is true for both conditions individually.
\end{remark}
\begin{remark}
	If the morphism of $k$-curves $f\colon X\to Y$ is \'etale, then it is unramified because we are essentially saying $\op{length}\Omega_{X/Y}=0$. In fact, unramified morphisms are \'etale, which should roughly be our intuition. Flatness is a bit mysterious, but such is life.
\end{remark}
\begin{remark}
	Equivalently, a morphism $f\colon X\to Y$ is \'etale if and only if $f$ is a smooth morphism of relative dimension $0$. For example, if both $X$ and $Y$ are varieties, then we are asking for the varieties to have the same sheaf of differentials.
\end{remark}
Geometrically, we imagine finite \'etale morphisms as covering space maps. This motivates the following definition.
\begin{definition}[simply connected]
	A finite \'etale morphism $f\colon X\to Y$ is \textit{trivial} if and only if $X$ is a disjoint union of copies of $Y$, and $f$ is a disjoint union of automorphisms. Then a curve $Y$ is \textit{simply connected} if and only if all finite \'etale morphisms $X\to Y$ are trivial.
\end{definition}
Roughly speaking, we are trying to say that the fundamental group is trivial.
\begin{proposition}
	Let $k$ be an algebraically closed field. Then $\PP^1_k$ is simply connected.
\end{proposition}
\begin{example}
	We can see that the sphere $\PP^1_\CC$ is simply connected.
\end{example}
\begin{proof}
	Fix some finite \'etale morphism $f\colon X\to Y$. By breaking this morphism into connected components, we may assume that $X$ is connected. We show that $X=\PP^1_k$. Because $f$ is smooth, we see that the structure map $X\to Y\to k$ is smooth, so $X$ is a (smooth) curve. (In particular, we can also see that $X$ is irreducible.) Now, because $f$ is \'etale, it is \'etale at the generic point, so $f$ is also separable.

	We are now ready to apply \Cref{thm:rh}. Here, $R(f)=0$ and $g(Y)=0$, so we are left with
	\[2g(X)-2=(\deg f)(0-2)=-2\deg f.\]
	However, $g(X)\ge0$ and $\deg f\ge1$, so we must have $g(X)=0$ and $\deg f=1$, so $f$ is in fact an isomorphism $X\cong\PP^1_k$.
\end{proof}
\begin{remark}
	In characteristic $0$, we will have $\AA^1_k$ is simply connected. However, in positive characteristic, this is no longer true; indeed, $\pi_1^{\text{\'et}}(\AA^1_{\FF_p})$ is infinite.
\end{remark}

\end{document}