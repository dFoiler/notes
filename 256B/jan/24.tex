% !TEX root = ../notes.tex

\documentclass[../notes.tex]{subfiles}

\begin{document}

Today we continue deriving functors.

\subsection{The Long Exact Sequence}
Here is the main result on cohomology.
\begin{theorem} \label{thm:derived-les}
	Fix a left exact functor $F\colon\mc A\to\mc B$ of abelian categories with enough injectives. Given a short exact sequence
	\[0\to A'\to A\to A''\to0\]
	in $\mc A$, there are natural morphisms $\delta^i\colon R^iF(A'')\to R^{i+1}F(A')$ for $i\ge0$ (i.e., the $\delta^i$ are natural in the short exact sequence) such that there is a long exact sequence as follows.
	% https://q.uiver.app/#q=WzAsOCxbMCwwLCIwIl0sWzEsMCwiUl4wRihBJykiXSxbMiwwLCJSXjBGKEEpIl0sWzMsMCwiUl4wRihBJycpIl0sWzEsMSwiUl4xRihBJykiXSxbMiwxLCJSXjFGKEEpIl0sWzMsMSwiUl4xRihBJycpIl0sWzQsMSwiXFxjZG90cyJdLFswLDFdLFsxLDJdLFsyLDNdLFszLDQsIlxcZGVsdGFeMCIsMV0sWzQsNV0sWzUsNl0sWzYsN11d&macro_url=https%3A%2F%2Fraw.githubusercontent.com%2FdFoiler%2Fnotes%2Fmaster%2Fnir.tex
	\[\begin{tikzcd}
		0 & {R^0F(A')} & {R^0F(A)} & {R^0F(A'')} \\
		& {R^1F(A')} & {R^1F(A)} & {R^1F(A'')} & \cdots
		\arrow[from=1-1, to=1-2]
		\arrow[from=1-2, to=1-3]
		\arrow[from=1-3, to=1-4]
		\arrow["{\delta^0}"{description}, from=1-4, to=2-2]
		\arrow[from=2-2, to=2-3]
		\arrow[from=2-3, to=2-4]
		\arrow[from=2-4, to=2-5]
	\end{tikzcd}\]
\end{theorem}
\begin{proof}
	We use \Cref{prop:get-les-complex}. The main obstacle is that we need to produce a short exact sequence of injective resolutions for $A'$, $A$, and $A''$. We begin by fixing injective resolutions $0\to A'\to I^\bullet$ and $0\to A''\to J^\bullet$, which we would like to glue together into an injective resolution for $A$ as well. In particular, we would like a sequence of morphisms to go into the middle of the following diagram.
	% https://q.uiver.app/#q=WzAsMjEsWzAsMSwiMCJdLFsxLDEsIkEnIl0sWzEsMiwiQSJdLFsxLDMsIkEnJyJdLFswLDIsIjAiXSxbMCwzLCIwIl0sWzEsMCwiMCJdLFsxLDQsIjAiXSxbMiwxLCJJXjAiXSxbMiwzLCJKXjAiXSxbMiwyLCJJXjBcXG9wbHVzIEpeMCJdLFsyLDAsIjAiXSxbMiw0LCIwIl0sWzMsMCwiMCJdLFszLDEsIkleMSJdLFszLDIsIkleMVxcb3BsdXMgSl4xIl0sWzMsMywiSl4xIl0sWzMsNCwiMCJdLFs0LDEsIlxcY2RvdHMiXSxbNCwyLCJcXGNkb3RzIl0sWzQsMywiXFxjZG90cyJdLFswLDFdLFs0LDJdLFs1LDNdLFs2LDFdLFsxLDJdLFsyLDNdLFszLDddLFsxLDhdLFs4LDE0XSxbMTQsMThdLFsyLDEwLCIiLDIseyJzdHlsZSI6eyJib2R5Ijp7Im5hbWUiOiJkYXNoZWQifX19XSxbMTAsMTUsIiIsMix7InN0eWxlIjp7ImJvZHkiOnsibmFtZSI6ImRhc2hlZCJ9fX1dLFsxNSwxOSwiIiwyLHsic3R5bGUiOnsiYm9keSI6eyJuYW1lIjoiZGFzaGVkIn19fV0sWzMsOV0sWzksMTZdLFsxNiwyMF0sWzExLDhdLFs4LDEwXSxbMTAsOV0sWzksMTJdLFsxMywxNF0sWzE0LDE1XSxbMTUsMTZdLFsxNiwxN11d&macro_url=https%3A%2F%2Fraw.githubusercontent.com%2FdFoiler%2Fnotes%2Fmaster%2Fnir.tex
	\[\begin{tikzcd}
		& 0 & 0 & 0 \\
		0 & {A'} & {I^0} & {I^1} & \cdots \\
		0 & A & {I^0\oplus J^0} & {I^1\oplus J^1} & \cdots \\
		0 & {A''} & {J^0} & {J^1} & \cdots \\
		& 0 & 0 & 0
		\arrow[from=2-1, to=2-2]
		\arrow[from=3-1, to=3-2]
		\arrow[from=4-1, to=4-2]
		\arrow[from=1-2, to=2-2]
		\arrow[from=2-2, to=3-2]
		\arrow[from=3-2, to=4-2]
		\arrow[from=4-2, to=5-2]
		\arrow[from=2-2, to=2-3]
		\arrow[from=2-3, to=2-4]
		\arrow[from=2-4, to=2-5]
		\arrow[dashed, from=3-2, to=3-3]
		\arrow[dashed, from=3-3, to=3-4]
		\arrow[dashed, from=3-4, to=3-5]
		\arrow[from=4-2, to=4-3]
		\arrow[from=4-3, to=4-4]
		\arrow[from=4-4, to=4-5]
		\arrow[from=1-3, to=2-3]
		\arrow[from=2-3, to=3-3]
		\arrow[from=3-3, to=4-3]
		\arrow[from=4-3, to=5-3]
		\arrow[from=1-4, to=2-4]
		\arrow[from=2-4, to=3-4]
		\arrow[from=3-4, to=4-4]
		\arrow[from=4-4, to=5-4]
	\end{tikzcd}\]
	Here, the downward morphisms, except the ones on the far left, are all given by having a split short exact sequence. (Note $I^i\oplus J^i$ is injective for each $i$ because the sum of injective objects must be injective; this can be seen directly from the definition of injective objects.)
	
	Working inductively, the main point is as follows: suppose we have a diagram as follows, where we would like to induce the vertical morphism $f$ making the diagram commute.
	% https://q.uiver.app/#q=WzAsMTAsWzAsMCwiMCJdLFsxLDAsIksnIl0sWzIsMCwiSyJdLFszLDAsIksnJyJdLFs0LDAsIjAiXSxbMSwxLCJJIl0sWzMsMSwiSiJdLFs0LDEsIjAiXSxbMCwxLCIwIl0sWzIsMSwiSVxcb3BsdXMgSiJdLFs4LDVdLFs1LDldLFs5LDZdLFs2LDddLFswLDFdLFsxLDJdLFsyLDNdLFszLDRdLFsxLDUsImYnIiwyXSxbMyw2LCJmJyciXSxbMiw5LCJmIiwwLHsic3R5bGUiOnsiYm9keSI6eyJuYW1lIjoiZGFzaGVkIn19fV1d&macro_url=https%3A%2F%2Fraw.githubusercontent.com%2FdFoiler%2Fnotes%2Fmaster%2Fnir.tex
	\[\begin{tikzcd}
		0 & {K'} & K & {K''} & 0 \\
		0 & I & {I\oplus J} & J & 0
		\arrow[from=2-1, to=2-2]
		\arrow[from=2-2, to=2-3]
		\arrow[from=2-3, to=2-4]
		\arrow[from=2-4, to=2-5]
		\arrow[from=1-1, to=1-2]
		\arrow[from=1-2, to=1-3]
		\arrow[from=1-3, to=1-4]
		\arrow[from=1-4, to=1-5]
		\arrow["{f'}"', from=1-2, to=2-2]
		\arrow["{f''}", from=1-4, to=2-4]
		\arrow["f", dashed, from=1-3, to=2-3]
	\end{tikzcd}\]
	Here, $I$ and $J$ are injective, and $f'$ and $f''$ is injective; the Snake lemma will imply that $f$ is injective too. Well, by summing, all one needs is maps $g'\colon K\to I$ and $g''\colon K\to J$ making the following diagram commute.
	% https://q.uiver.app/#q=WzAsMTAsWzAsMCwiMCJdLFsxLDAsIksnIl0sWzIsMCwiSyJdLFszLDAsIksnJyJdLFs0LDAsIjAiXSxbMSwxLCJJIl0sWzMsMSwiSiJdLFs0LDEsIjAiXSxbMCwxLCIwIl0sWzIsMSwiSVxcb3BsdXMgSiJdLFs4LDVdLFs1LDldLFs5LDZdLFs2LDddLFswLDFdLFsxLDJdLFsyLDNdLFszLDRdLFsxLDUsImYnIiwyXSxbMyw2LCJmJyciXSxbMiw1LCJnJyIsMSx7InN0eWxlIjp7ImJvZHkiOnsibmFtZSI6ImRhc2hlZCJ9fX1dLFsyLDYsImcnJyIsMSx7InN0eWxlIjp7ImJvZHkiOnsibmFtZSI6ImRhc2hlZCJ9fX1dXQ==&macro_url=https%3A%2F%2Fraw.githubusercontent.com%2FdFoiler%2Fnotes%2Fmaster%2Fnir.tex
	\[\begin{tikzcd}
		0 & {K'} & K & {K''} & 0 \\
		0 & I & {I\oplus J} & J & 0
		\arrow[from=2-1, to=2-2]
		\arrow[from=2-2, to=2-3]
		\arrow[from=2-3, to=2-4]
		\arrow[from=2-4, to=2-5]
		\arrow[from=1-1, to=1-2]
		\arrow[from=1-2, to=1-3]
		\arrow[from=1-3, to=1-4]
		\arrow[from=1-4, to=1-5]
		\arrow["{f'}"', from=1-2, to=2-2]
		\arrow["{f''}", from=1-4, to=2-4]
		\arrow["{g'}"{description}, dashed, from=1-3, to=2-2]
		\arrow["{g''}"{description}, dashed, from=1-3, to=2-4]
	\end{tikzcd}\]
	For this, we see that $g''$ is given by composition, and $g'$ is given because $K'\subseteq K$ and $I$ is injective object.

	We now explain how the previous step proves the result. We immediately produce the needed map $A\to I^0\oplus J^0$. Now to go from having the map $I^i\oplus J^i\to I^{i+1}\oplus J^{i+1}$ to having the map $I^{i+1}\oplus J^{i+1}$, we use the above paragraph on the following diagram.
	% https://q.uiver.app/#q=WzAsMTAsWzAsMCwiMCJdLFsxLDAsIlxcZGZyYWN7SV57aSsxfX17SV5pfSJdLFsyLDAsIlxcZGZyYWN7SV57aSsxfX17SV5pfVxcb3BsdXNcXGRmcmFje0pee2krMX19e0peaX0iXSxbMywwLCJcXGRmcmFje0pee2krMX19e0peaX0iXSxbNCwwLCIwIl0sWzEsMSwiSV57aSsyfSJdLFszLDEsIkpee2krMn0iXSxbMiwxLCJJXntpKzJ9XFxvcGx1cyBKXntpKzJ9Il0sWzAsMSwiMCJdLFs0LDEsIjAiXSxbMCwxXSxbMSwyXSxbMiwzXSxbMyw0XSxbOCw1XSxbNSw3XSxbNyw2XSxbNiw5XSxbMSw1LCIiLDEseyJzdHlsZSI6eyJ0YWlsIjp7Im5hbWUiOiJob29rIiwic2lkZSI6InRvcCJ9fX1dLFszLDYsIiIsMSx7InN0eWxlIjp7InRhaWwiOnsibmFtZSI6Imhvb2siLCJzaWRlIjoidG9wIn19fV0sWzIsNywiIiwxLHsic3R5bGUiOnsiYm9keSI6eyJuYW1lIjoiZGFzaGVkIn19fV1d&macro_url=https%3A%2F%2Fraw.githubusercontent.com%2FdFoiler%2Fnotes%2Fmaster%2Fnir.tex
	\[\begin{tikzcd}
		0 & {\dfrac{I^{i+1}}{I^i}} & {\dfrac{I^{i+1}}{I^i}\oplus\dfrac{J^{i+1}}{J^i}} & {\dfrac{J^{i+1}}{J^i}} & 0 \\
		0 & {I^{i+2}} & {I^{i+2}\oplus J^{i+2}} & {J^{i+2}} & 0
		\arrow[from=1-1, to=1-2]
		\arrow[from=1-2, to=1-3]
		\arrow[from=1-3, to=1-4]
		\arrow[from=1-4, to=1-5]
		\arrow[from=2-1, to=2-2]
		\arrow[from=2-2, to=2-3]
		\arrow[from=2-3, to=2-4]
		\arrow[from=2-4, to=2-5]
		\arrow[hook, from=1-2, to=2-2]
		\arrow[hook, from=1-4, to=2-4]
		\arrow[dashed, from=1-3, to=2-3]
	\end{tikzcd}\]
	This completes the construction of the needed short exact sequence of injective resolutions, from which the result follows upon using \Cref{prop:get-les-complex} on the short exact sequence of complexes
	\[0\to FI^\bullet\to FI^\bullet\oplus FJ^\bullet\to FJ^\bullet\to0.\]
	(This is still short exact because additive functors preserve split short exact sequences.) Note that we have not checked that the $\delta^\bullet$s are natural in the short exact sequence; this follows from the naturality of \Cref{prop:get-les-complex}.
\end{proof}

\subsection{Acyclic Objects}
We note the following computation.
\begin{proposition} \label{prop:inj-is-acyclic}
	Fix a left exact functor $F\colon\mc A\to\mc B$ of abelian categories with enough injectives. If $I\in\mc A$ is injective, then $R^iF(I)=0$ for all $i\ge1$.
\end{proposition}
\begin{proof}
	There is an injective resolution
	\[0\to I\to I\to 0\to 0\to\cdots\]
	of $I$. Upon taking $F$, we see that $R^0F(I)=I/0$ and $R^1F(I)=0/I$ and $R^iF(I)=0/0$ for $i\ge2$. This proves the result.
\end{proof}
We now get the following definition.
\begin{definition}[acyclic]
	Fix a left exact functor $F\colon\mc A\to\mc B$ of abelian categories with enough injectives. We say an object $A\in\mc A$ is \textit{acyclic for $F$} if and only if $R^iF(A)=0$ for all $i\ge1$.
\end{definition}
\begin{example}
	If $A\in\mc A$ is injective, then \Cref{prop:inj-is-acyclic} implies that $A$ is acyclic for any left exact functor $F\colon\mc A\to\mc B$.
\end{example}
Here is the point of defining acyclic objects.
\begin{proposition}
	Fix a left exact functor $F\colon\mc A\to\mc B$ of abelian categories with enough injectives. For any acyclic resolution $0\to A\to I^\bullet$, there are canonical isomorphisms
	\[R^iF(A)\cong h^i(FJ^\bullet).\]
\end{proposition}
\begin{proof}
	Induct on $i$ using the long exact sequences. For example, there is nothing to say for $i=0$. To get up to $i=1$, use the exact sequence
	\[0\to A\stackrel\varepsilon\to J^0\to\coker\varepsilon\to0\]
	to produce the needed long exact sequence
	\[0\to FA\to FJ^0\to F\coker\varepsilon\to R^1F(A)\to0,\]
	and $h^1(FJ^\bullet)$ becomes the needed quotient. This process continues upwards.
\end{proof}

\subsection{A Little \texorpdfstring{$\delta$}{ Delta}-Functors}
Here is our definition.
\begin{defihelper}[$\delta$-functor] \nirindex{delta-functor@$\delta$-functor}
	Fix abelian categories $\mc A$ and $\mc B$. A \textit{$\delta$-functor} consists of the data of some additive functors $T^i\colon\mc A\to\mc B$ for each $i\in\NN$ and some morphisms $\delta^i\colon T^iA''\to T^{i+1}A$ for each short exact sequence $0\to A'\to A\to A''\to0$ such that there is a long exact sequence as follows.
	% https://q.uiver.app/#q=WzAsOCxbMCwwLCIwIl0sWzEsMCwiVF4wQSciXSxbMiwwLCJUXjBBIl0sWzMsMCwiVF4wQScnIl0sWzEsMSwiVF4xQSciXSxbMiwxLCJUXjFBIl0sWzMsMSwiVF4xQScnIl0sWzQsMSwiXFxjZG90cyJdLFszLDQsIlxcZGVsdGFeMCIsMV0sWzAsMV0sWzEsMl0sWzIsM10sWzQsNV0sWzUsNl0sWzYsN11d&macro_url=https%3A%2F%2Fraw.githubusercontent.com%2FdFoiler%2Fnotes%2Fmaster%2Fnir.tex
	\[\begin{tikzcd}
		0 & {T^0A'} & {T^0A} & {T^0A''} \\
		& {T^1A'} & {T^1A} & {T^1A''} & \cdots
		\arrow["{\delta^0}"{description}, from=1-4, to=2-2]
		\arrow[from=1-1, to=1-2]
		\arrow[from=1-2, to=1-3]
		\arrow[from=1-3, to=1-4]
		\arrow[from=2-2, to=2-3]
		\arrow[from=2-3, to=2-4]
		\arrow[from=2-4, to=2-5]
	\end{tikzcd}\]
\end{defihelper}
\begin{example}
	If $\mc A$ has enough injective, the derived functors provide examples of $\delta$-functors by \Cref{thm:derived-les}.
\end{example}
The following definition will be very helpful.
\begin{definition}[initial]
	Fix abelian categories $\mc A$ and $\mc B$. A $\delta$-functor $(T^\bullet,\delta_T^\bullet)$ is \textit{initial} if and only if any other $\delta$-functor $(U^\bullet,\delta_U^\bullet)$ together with a map $\varphi\colon T^0\Rightarrow U^0$ has a unique sequence of natural transformations $\eta^\bullet\colon T^\bullet\Rightarrow U^\bullet$ extending $\varphi$ and commute with the formation of the long exact sequences. Explicitly, a short exact sequence $0\to A'\to A\to A''\to0$ induces the following morphism of long exact sequences.
	% https://q.uiver.app/#q=WzAsMTQsWzAsMCwiMCJdLFsxLDAsIlReMEEnIl0sWzIsMCwiVF4wQSJdLFszLDAsIlReMEEnJyJdLFs0LDAsIlReMUEnIl0sWzUsMCwiVF4xQSJdLFs2LDAsIlxcY2RvdHMiXSxbMCwxLCIwIl0sWzEsMSwiVV4wQSciXSxbMiwxLCJVXjBBIl0sWzMsMSwiVV4wQScnIl0sWzQsMSwiVV4xQSciXSxbNSwxLCJVXjFBIl0sWzYsMSwiXFxjZG90cyJdLFswLDFdLFsxLDJdLFsyLDNdLFszLDQsIlxcZGVsdGFeMF9UIl0sWzQsNV0sWzUsNl0sWzcsOF0sWzgsOV0sWzksMTBdLFsxMCwxMSwiXFxkZWx0YV4wX1UiXSxbMTEsMTJdLFsxMiwxM10sWzEsOCwiZl4wIiwyXSxbMiw5LCJmXjAiLDJdLFszLDEwLCJmXjAiLDJdLFs0LDExLCJmXjEiLDJdLFs1LDEyLCJmXjEiLDJdXQ==&macro_url=https%3A%2F%2Fraw.githubusercontent.com%2FdFoiler%2Fnotes%2Fmaster%2Fnir.tex
	\[\begin{tikzcd}
		0 & {T^0A'} & {T^0A} & {T^0A''} & {T^1A'} & {T^1A} & \cdots \\
		0 & {U^0A'} & {U^0A} & {U^0A''} & {U^1A'} & {U^1A} & \cdots
		\arrow[from=1-1, to=1-2]
		\arrow[from=1-2, to=1-3]
		\arrow[from=1-3, to=1-4]
		\arrow["{\delta^0_T}", from=1-4, to=1-5]
		\arrow[from=1-5, to=1-6]
		\arrow[from=1-6, to=1-7]
		\arrow[from=2-1, to=2-2]
		\arrow[from=2-2, to=2-3]
		\arrow[from=2-3, to=2-4]
		\arrow["{\delta^0_U}", from=2-4, to=2-5]
		\arrow[from=2-5, to=2-6]
		\arrow[from=2-6, to=2-7]
		\arrow["{f^0}"', from=1-2, to=2-2]
		\arrow["{f^0}"', from=1-3, to=2-3]
		\arrow["{f^0}"', from=1-4, to=2-4]
		\arrow["{f^1}"', from=1-5, to=2-5]
		\arrow["{f^1}"', from=1-6, to=2-6]
	\end{tikzcd}\]
\end{definition}
Note that initial $\delta$-functors are unique up to unique isomorphism when they exist.

\end{document}