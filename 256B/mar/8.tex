% !TEX root = ../notes.tex

\documentclass[../notes.tex]{subfiles}

\begin{document}

\section{March 8}

We continue discussing \v Cech cohomology.

\subsection{Finishing \v Cech Cohomology's Correctness}
Here is the precise statement we will show.
\begin{theorem}
	Fix a Noetherian separated scheme $X$ with affine open cover $\mf U=\{U_\alpha\}_{\alpha\in\lambda}$. For any quasicoherent sheaf $\mc F$ on $X$, we have $\check H^p(\mf U,\mc F)=H^p(X,\mc F)$ for any index $p\ge0$.
\end{theorem}
\begin{proof}
	Because $X$ is separated, the intersection of any two affine open subschemes $U$ and $V$ is still an affine open subscheme because $U\cap V=\Delta^{-1}(U\times V)$ where $\Delta\colon X\times X\to X$ is the diagonal map. (Notably, $\Delta$ is an affine morphism because it is a closed embedding.)
	
	Now, we apply dimension-shifting. By checking affine-locally, we may embed $\mc F$ into a flasque quasicoherent sheaf $\mc G$; letting $\mc R$ denote the cokernel, we have the exact sequence
	\[0\to\mc F\to\mc G\to\mc R\to0.\]
	Now, for any indices $\alpha_0<\cdots<\alpha_p$, we get the exact sequence
	\[0\to\mc F(U_{\alpha_0,\ldots,\alpha_p})\to\mc G(U_{\alpha_0,\ldots,\alpha_p})\to\mc R(U_{\alpha_0,\ldots,\alpha_p})\to0\]
	because $U_{\alpha_0,\ldots,\alpha_p}$ is affine and everything is a quasicoherent sheaf. Taking the product over all such $(p+1)$-tuples of indices and looping over all indices, we get an exact sequence of \v Cech cohomology
	\[0\to C^\bullet(\mf U,\mc F)\to C^\bullet(\mf U,\mc G)\to C^\bullet(\mf U,\mc R)\to0.\]
	This exact sequence of complexes now produces a long exact sequence in cohomology as follows.
	% https://q.uiver.app/?q=WzAsOCxbMSwwLCJcXGNoZWNrIEhecChcXG1mIFUsXFxtYyBGKSJdLFsxLDEsIlxcY2hlY2sgSF57cCsxfShcXG1mIFUsXFxtYyBGKSJdLFsyLDAsIlxcY2hlY2sgSF5wKFxcbWYgVSxcXG1jIEcpIl0sWzMsMCwiXFxjaGVjayBIXnAoXFxtZiBVLFxcbWMgUikiXSxbMiwxLCJcXGNoZWNrIEhee3ArMX0oXFxtZiBVLFxcbWMgRykiXSxbMywxLCJcXGNoZWNrIEhee3ArMX0oXFxtZiBVLFxcbWMgUikiXSxbNCwxLCJcXGNkb3RzIl0sWzAsMCwiXFxjZG90cyJdLFswLDJdLFsyLDNdLFszLDFdLFsxLDRdLFs0LDVdLFs3LDBdLFs1LDZdXQ==&macro_url=https%3A%2F%2Fraw.githubusercontent.com%2FdFoiler%2Fnotes%2Fmaster%2Fnir.tex
	\[\begin{tikzcd}
		\cdots & {\check H^p(\mf U,\mc F)} & {\check H^p(\mf U,\mc G)} & {\check H^p(\mf U,\mc R)} \\
		& {\check H^{p+1}(\mf U,\mc F)} & {\check H^{p+1}(\mf U,\mc G)} & {\check H^{p+1}(\mf U,\mc R)} & \cdots
		\arrow[from=1-2, to=1-3]
		\arrow[from=1-3, to=1-4]
		\arrow[from=1-4, to=2-2]
		\arrow[from=2-2, to=2-3]
		\arrow[from=2-3, to=2-4]
		\arrow[from=1-1, to=1-2]
		\arrow[from=2-4, to=2-5]
	\end{tikzcd}\]
	Using the transition maps $\check H^\bullet(\mf U,\mc F)\to H^\bullet(X,\mc F)$ given last class, we get the following morphism of exact sequences.
	% https://q.uiver.app/?q=WzAsMTAsWzAsMCwiXFxjaGVjayBIXntwLTF9KFxcbWYgVSxcXG1jIEYpIl0sWzEsMCwiXFxjaGVjayBIXntwLTF9KFxcbWYgVSxcXG1jIEcpIl0sWzIsMCwiXFxjaGVjayBIXntwLTF9KFxcbWYgVSxcXG1jIFIpIl0sWzMsMCwiXFxjaGVjayBIXnAoXFxtZiBVLFxcbWMgRikiXSxbNCwwLCIwIl0sWzAsMSwiSF57cC0xfShYLFxcbWMgRikiXSxbMSwxLCJIXntwLTF9KFgsXFxtYyBHKSJdLFsyLDEsIkhee3AtMX0oWCxcXG1jIFIpIl0sWzMsMSwiSF57cC0xfShYLFxcbWMgRikiXSxbNCwxLCIwIl0sWzAsNSwiXFxzaW1lcSJdLFsxLDYsIlxcc2ltZXEiXSxbMiw3LCJcXHNpbWVxIl0sWzMsOF0sWzAsMV0sWzEsMl0sWzIsM10sWzMsNF0sWzUsNl0sWzYsN10sWzcsOF0sWzgsOV1d&macro_url=https%3A%2F%2Fraw.githubusercontent.com%2FdFoiler%2Fnotes%2Fmaster%2Fnir.tex
	\[\begin{tikzcd}
		{\check H^{p}(\mf U,\mc F)} & {\check H^{p}(\mf U,\mc G)} & {\check H^{p}(\mf U,\mc R)} & {\check H^{p+1}(\mf U,\mc F)} & 0 \\
		{H^{p}(X,\mc F)} & {H^{p}(X,\mc G)} & {H^{p}(X,\mc R)} & {H^{p+1}(X,\mc F)} & 0
		\arrow["\simeq", from=1-1, to=2-1]
		\arrow["\simeq", from=1-2, to=2-2]
		\arrow["\simeq", from=1-3, to=2-3]
		\arrow[from=1-4, to=2-4]
		\arrow[from=1-1, to=1-2]
		\arrow[from=1-2, to=1-3]
		\arrow[from=1-3, to=1-4]
		\arrow[from=1-4, to=1-5]
		\arrow[from=2-1, to=2-2]
		\arrow[from=2-2, to=2-3]
		\arrow[from=2-3, to=2-4]
		\arrow[from=2-4, to=2-5]
	\end{tikzcd}\]
	(Note that $\check H^{p+1}(\mf U,\mc G)=0$ by \Cref{lem:flasque-vanishes-cech}. Additionally, we are natural in the $\mc R\to\mc F$ boundary map because the transition maps we defined actually come from a complex, so we are saying that a morphism of complexes is giving rise to a morphism of long exact sequences, which we know.) By the inductive hypothesis (note that our base case is degree-$0$ by \Cref{lem:cech-correct-deg-zero}), the three morphisms on the left are isomorphisms. Thus, the desired morphism $\check H^{p+1}(\mf U,\mc F)\to H^{p+1}(X,\mc F)$ is an isomorphism by the Five lemma or something, which completes our induction.
\end{proof}

\subsection{Review of Projective Space}
For a Noetherian ring $A$, we will be computing the cohomology of $\PP^r_A\coloneqq\Proj A[x_0,\ldots,x_r]$. We will be using the usual facts about the $\Proj$ construction. In particular, for a graded ring $S$, for any homogeneous $f\in S_+$ of positive degree, we have the distinguished open set $D_+(f)=\{\mf p\in\Proj S:f\notin\mf p\}$, which has structure sheaf given by $\Spec (S_f)_0$.
\begin{example}
	We work with $\PP^n_A=\Proj A[x_0,\ldots,x_r]$. If we take $f=x_i$ for some index $i$, then the distinguished open subscheme is given by
	\[D_+(x_i)=\Spec(A[x_0,\ldots,x_n]_{x_i})_0=\Spec\frac{A[x_{0/i},\ldots,x_{r/i}]}{(x_{i/i}-1)}\]
	where the isomorphism is given by $x_j/x_i\mapsto x_{j/i}$.
\end{example}
Now, for a graded $S$-module $M$, then one can follow the above construction almost verbatim to produce a quasicoherent sheaf $\widetilde M$ on $\Proj S$. Namely, for homogeneous $f\in S_+$ of positive degree, we set $\widetilde M|_{D_+(f)}\coloneqq\widetilde{(M_f)_0}$. As such, one can check that any $\mf p\in\Proj S$ yields
\[\widetilde M_\mf p\coloneqq(M_\mf p)_0.\]
Here, the localization $M_\mf p$ only permits denominators which are homogeneous elements outside $\mf p$.
\begin{remark}
	If $S$ is Noetherian, and $M$ is finitely generated, then one can check that $\widetilde M$ is coherent.
\end{remark}
At the end of the day, we care about the quasicoherent sheaves $\OO(n)\coloneqq\widetilde{S(n)}$, where $S(n)$ is the $S$-module $S$ with grading given by $S(n)_i=S_{n+i}$.

Our goal is to compute the cohomology of these quasicoherent sheaves $\OO(n)$. Here is a start.
\begin{proposition}
	Fix a ring $A$. We compute $H^0\left(\PP^r_A,\bigoplus_{n\in\ZZ}\OO(n)\right)=S$.
\end{proposition}
\begin{proof}
	Fix some $n$, and we will compute $H^0\left(\PP^r_A,\OO(n)\right)=\Gamma\left(\PP^r_A,\OO(n)\right)$; because cohomology commutes with direct sums, this will complete the proof. Checking affine-locally, we note that a global section $t_i\in\Gamma\left(D_+(x_i),\OO(n)\right)$ is simply of an element which is homogeneous of degree $n$ in $S_{x_i}$ by the definition of $\OO(n)$ as $\widetilde{S(n)}$. Now to glue our $\{t_i\}_{i=1}^n$ together, we are requiring that
	\[t_i|_{D_+(x_ix_j)}=t_j|_{D_+(x_ix_j)}\]
	which means that $t_i$ and $t_j$ have the same image in $S_{x_ix_j}$. Patching these together, we are saying that we want our global section to live in the intersection
	\[\bigcap_{i=1}^nS_{x_i},\]
	which means that we want an element of total degree $n$ with nonnegative degree in each coordinate. Thus, we are just looking at the degree-$n$ homogeneous polynomials in $S$. Thus, $H^0\left(\PP^r_A,\OO(n)\right)$ consists of the degree-$n$ homogeneous polynomials in $A[x_0,\ldots,x_n]$.
\end{proof}

\end{document}