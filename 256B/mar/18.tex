% !TEX root = ../notes.tex

\documentclass[../notes.tex]{subfiles}

\begin{document}

Today we continue discussing curves.

\subsection{Properties of Curves}
We quickly note that projective and proper are the same for curves.
\begin{proposition}
	Fix a regular curve $X$ over an algebraically closed field $k$. Then the following are equivalent.
	\begin{listroman}
		\item $X$ is projective.
		\item $X$ is proper.
		\item $X\cong t(C_{K(x)})$ where $t\colon\mathrm{Var}_k\to\mathrm{Sch}_k$ is the equivalence of \cite[Proposition~2.6]{hartshorne}.
	\end{listroman}
\end{proposition}
\begin{proof}
	We won't show the equivalences of (i) and (iii). The implication (i) to (ii) is simply because projective implies proper. The implication (ii) to (i) is by Chow's lemma.
	\begin{lemma}[Chow] \label{lem:chow}
		Fix a scheme $X$ proper over a Noetherian scheme $S$. Then there exists a birational morphism $g\colon X\to X'$ where $X'$ is projective over $S$.
	\end{lemma}
	We won't prove \Cref{lem:chow} because it is pretty hard. To see how \Cref{lem:chow} shows the implication, note that $g$ extends to a full morphism $g\colon X\to X'$ which is an isomorphism on an open subscheme $U$ of $X$. Checking our isomorphism on stalks reveals that $X'$ continues to be regular, and then we finish by noting that the inverse rational map for $g$ also extends to a full map $X'\to X$, so we are done.
\end{proof}
This allows us to understand morphisms of curves.
\begin{proposition} \label{prop:curve-maps}
	Fix a morphism $f\colon X\to Y$ of curves over an algebraically closed field $k$, and suppose $X$ is proper and regular. Then exactly one of the following hold.
	\begin{listroman}
		\item $f(X)$ is a single point.
		\item $f$ is surjective. In this case, $f^\sharp\colon K(Y)\to K(X)$ is a finite field extension, and $Y$ is proper, and $f$ is finite.
	\end{listroman}
\end{proposition}
\begin{proof}
	Note $f(X)$ must be a closed subset of $Y$ because $X$ is proper, and because $Y$ is a curve, all closed subsets are either finite sets of points or all of $Y$. However, $X$ is connected, so $f(X)$ is connected, so the image of $f(X)$ is either a point or all of $Y$ (and not both).

	It remains to do the other checks of (ii) in the case that $f$ is surjective. Because $f$ is surjective, we see that $Y$ is proper (in particular, $Y$ is universally closed because the image of any morphism from $Y$ will continue to have closed image because we can realize this as an image from $X$). Now, $f$ is dominant, so $f^\sharp\colon K(Y)\to K(X)$ is injective, so we have an actual field extension; because these two field extensions have transcendence degree $1$ by Noether normalization, we see that this is an algebraic extension. Because these rings are finitely generated over $k$, we see that the extension $K(Y)\subseteq K(X)$ is also finite generated, so our degree is in fact finite.

	It remains to check that $f$ is finite. This is somewhat involved. Let $V=\Spec B$ be some nonempty affine open subscheme of $Y$, and let $A$ be the integral closure of $B$ in $K(X)$. Note $A$ is finite over $B$ because our field extension is finite. As such, we would like to know that $U\coloneqq\Spec A$ is $f^{-1}(V)$. Well, $K(U)=K(X)$, so $U$ is birational to $X$. Further $U$ is normal and hence regular by construction, so properness of $X$ means that the birational map $U\to X$ becomes a full morphism $i\colon U\to X$. Notably, the diagram
	% https://q.uiver.app/#q=WzAsNCxbMCwwLCJVIl0sWzAsMSwiViJdLFsxLDAsIlgiXSxbMSwxLCJZIl0sWzAsMiwiaSJdLFsyLDMsImYiXSxbMCwxXSxbMSwzLCJcXHN1YnNldGVxIiwzLHsic3R5bGUiOnsiYm9keSI6eyJuYW1lIjoibm9uZSJ9LCJoZWFkIjp7Im5hbWUiOiJub25lIn19fV1d&macro_url=https%3A%2F%2Fraw.githubusercontent.com%2FdFoiler%2Fnotes%2Fmaster%2Fnir.tex
	\[\begin{tikzcd}
		U & X \\
		V & Y
		\arrow["i", from=1-1, to=1-2]
		\arrow["f", from=1-2, to=2-2]
		\arrow[from=1-1, to=2-1]
		\arrow["\subseteq"{marking, allow upside down}, draw=none, from=2-1, to=2-2]
	\end{tikzcd}\]
	commutes by construction of the map $i$, so $i(U)\subseteq X$ lands inside $f^{-1}(V)$. We would like for $i$ to be an open embedding, which we do as follows: note $U$ will embed as an open subset into some unique proper regular curve $\ov U$,\footnote{More precisely, place $U$ into some projective space, and let $\ov U$ be the closure there.} and universality then provides a rational map $X\to\ov U$ (which extends to a full morphism $g\colon X\to\ov U$) making the diagram
	% https://q.uiver.app/#q=WzAsNSxbMCwwLCJVIl0sWzEsMCwiXFxvdiBVIl0sWzIsMSwiWCJdLFswLDEsIlYiXSxbMiwyLCJZIl0sWzAsMSwiXFxzdWJzZXRlcSIsMyx7InN0eWxlIjp7ImJvZHkiOnsibmFtZSI6Im5vbmUifSwiaGVhZCI6eyJuYW1lIjoibm9uZSJ9fX1dLFsyLDEsImciLDJdLFswLDIsImkiLDFdLFswLDNdLFsyLDQsImYiXSxbMyw0XV0=&macro_url=https%3A%2F%2Fraw.githubusercontent.com%2FdFoiler%2Fnotes%2Fmaster%2Fnir.tex
	\[\begin{tikzcd}
		U & {\ov U} \\
		V && X \\
		&& Y
		\arrow["\subseteq"{marking, allow upside down}, draw=none, from=1-1, to=1-2]
		\arrow["g"', from=2-3, to=1-2]
		\arrow["i"{description}, from=1-1, to=2-3]
		\arrow[from=1-1, to=2-1]
		\arrow["f", from=2-3, to=3-3]
		\arrow[from=2-1, to=3-3]
	\end{tikzcd}\]
	commute. Tracking around this diagram reveals that $g^{-1}(U)\subseteq f^{-1}(V)$.

	We quickly claim that $g^{-1}(U)=f^{-1}(V)$. Well, suppose for the sake of contradiction that we have strict containment. Because generic points map to generic points, we at least are a nonempty open subset, so at worst we are missing some finite set of closed points $\{p_1,\ldots,p_r\}$. Well, set $W\coloneqq f^{-1}(V)\setminus\{p_2,\ldots,p_r\}$ to be $g^{-1}(U)$ with $p_1$ appended. Then $g$ sends $W\setminus\{p_1\}$ to $U$, and $f$ maps $W$ to $V$, so because $U\to V$ is proper (in fact finite), we know that $g$ must extend to a map $W\to U$ by the valuative criterion for properness (as usual). But $g'(p_1)\in U$ while $g(p)\notin U$, which contradicts the fact that the map $\ov U\to k$ is separated (again using a valuative criterion for points on $k$).

	We now complete the proof that $f$ is finite. Note $g|_{g^{-1}(U)
	}\circ i$ sends $U$ back to $U$, and $\varphi\circ g$ sends $f^{-1}(V)$ to $f^{-1}(V)$, and these maps are over $X$ and $Y$ respectively, so we have mutually inverse isomorphisms. Namely, we have provided our isomorphism of $U$ with $f^{-1}(V)$ over $Y$, so $f^{-1}(V)$ is finite over $V$ because $U$ is.
\end{proof}
This allows us to define the degree of a morphism.
\begin{definition}[degree]
	Fix a finite morphism $f\colon X\to Y$ over a field $k$. Then the \textit{degree} of $f$, denoted $\deg f$, is the degree of the field extension $f^\sharp\colon K(Y)\to K(X)$.
\end{definition}
\begin{proposition}
	Fix a regular curve $U$ over an algebraically closed field $k$. Then there is an open embedding $U$ into a proper regular curve $X$.
\end{proposition}
\begin{proof}
	Let $V\subseteq U$ be some nonempty affine open subscheme. Then we can place $V$ into some projective space and takes its closure, which we call $\ov V$. Let $\pi\colon X\to\ov V$ be the normalization map so that $\pi$ is finite (note that this is a birational map). Now, $X$ is normal and proper (in particular, it follows that $X$ is regular), so one can argue as in the previous point to see that the rational map
	\[U\to V\subseteq\ov V\to X\]
	extends uniquely to an open embedding. Indeed, this is the main content of the proof of \Cref{prop:curve-maps}.
\end{proof}

\end{document}