% !TEX root = ../notes.tex

\documentclass[../notes.tex]{subfiles}

\begin{document}

\section{March 24}

Spring break is upon us. There will not be a problem set over spring break.

\subsection{Stating Serre Duality}
Today we would like to finish proving Serre duality. Given a projective scheme $i\colon X\to\PP^n_k$, we would like to show that $\omega_X^\circ\coloneqq R^ri^!\omega_{\PP^n_k}$ is in fact a dualizing sheaf.
\begin{proposition} \label{prop:dualize-sheaf-proj-scheme}
	Set $\PP\coloneqq\PP^N_k$ for a field $k$, and fix a closed subscheme $i\colon X\to\PP$ of codimension $r$. Then set $\omega_X^\circ\coloneqq R^ri^!\omega_\PP$. Then $\omega^\circ_X$ is a dualizing sheaf on $X$.
\end{proposition}
\begin{proof}
	For brevity, set $n\coloneqq N-r$, which is $\dim X$. To begin, we require a trace morphism $t\colon H^n(X,\omega^\circ_X)\to k$ inducing the desired perfect pairing
	\[\op{Hom}(\mc F,\omega^\circ_X)\to H^n(X,\mc F)\to H^n(X,\omega^\circ_X)\stackrel t\to k.\]
	Now, \Cref{lem:almost-serre-duality} and applying \Cref{thm:serre-duality-pn} tells us that
	\begin{equation}
		\op{Hom}(\mc F,\omega^\circ_X)\simeq\op{Ext}^r_\PP(i_*\mc F,\omega_\PP)\simeq H^{N-r}(\PP,i_*\mc F)^\lor\simeq H^n(X,\mc F)^\lor \label{eq:serre-pairing}
	\end{equation}
	for any coherent sheaf $\mc F$, and these isomorphisms are natural in $\mc F$.
	
	Thus, we appear to have a perfect pairing, but we need to figure out what the corresponding trace map was supposed to be. Well, the identity $\id$ in $\op{Hom}_X(\omega_X^\circ,\omega_X^\circ)$ by the above computation produces some map $t\in H^n(X,\mc F)^\lor$, which we claim is our trace map. Indeed, for some $\alpha\colon\mc F\to\omega_X^\circ$, we have the following commutative diagram.
	% https://q.uiver.app/?q=WzAsNixbMCwwLCJcXG9we0hvbX0oXFxvbWVnYV9YXlxcY2lyYyxcXG9tZWdhX1heXFxjaXJjKSJdLFswLDEsIlxcb3B7SG9tfShcXG1jIEYsXFxvbWVnYV9YXlxcY2lyYykiXSxbMSwwLCJcXG9we0V4dH0oaV8qXFxvbWVnYV9YXlxcY2lyYyxcXG9tZWdhX1xcUFApIl0sWzIsMCwiSF5uKFgsXFxvbWVnYV9YXlxcY2lyYyleXFxsb3IiXSxbMiwxLCJIXm4oWCxcXG1jIEYpXlxcbG9yIl0sWzEsMSwiXFxvcHtFeHR9KGlfKlxcbWMgRixcXG9tZWdhX1xcUFApIl0sWzAsMl0sWzIsM10sWzMsNF0sWzUsNF0sWzIsNV0sWzEsNV0sWzAsMV1d&macro_url=https%3A%2F%2Fraw.githubusercontent.com%2FdFoiler%2Fnotes%2Fmaster%2Fnir.tex
	\[\begin{tikzcd}
		{\op{Hom}(\omega_X^\circ,\omega_X^\circ)} & {\op{Ext}^r(i_*\omega_X^\circ,\omega_\PP)} & {H^n(X,\omega_X^\circ)^\lor} \\
		{\op{Hom}(\mc F,\omega_X^\circ)} & {\op{Ext}^r(i_*\mc F,\omega_\PP)} & {H^n(X,\mc F)^\lor}
		\arrow[from=1-1, to=1-2]
		\arrow[from=1-2, to=1-3]
		\arrow[from=1-3, to=2-3]
		\arrow[from=2-2, to=2-3]
		\arrow[from=1-2, to=2-2]
		\arrow[from=2-1, to=2-2]
		\arrow[from=1-1, to=2-1]
	\end{tikzcd}\]
	Tracking around $\id_{\omega^\circ_X}$, we have the following.
	% https://q.uiver.app/?q=WzAsNCxbMCwwLCJcXGlkIl0sWzAsMSwiXFxhbHBoYSJdLFsyLDEsIlxccHNpX1xcYWxwaGE9dFxcY2lyYyBIXm4oXFxhbHBoYSkiXSxbMiwwLCJ0Il0sWzEsMiwiIiwyLHsic3R5bGUiOnsidGFpbCI6eyJuYW1lIjoibWFwcyB0byJ9fX1dLFswLDEsIiIsMix7InN0eWxlIjp7InRhaWwiOnsibmFtZSI6Im1hcHMgdG8ifX19XSxbMCwzLCIiLDAseyJzdHlsZSI6eyJ0YWlsIjp7Im5hbWUiOiJtYXBzIHRvIn19fV0sWzMsMiwiIiwwLHsic3R5bGUiOnsidGFpbCI6eyJuYW1lIjoibWFwcyB0byJ9fX1dXQ==&macro_url=https%3A%2F%2Fraw.githubusercontent.com%2FdFoiler%2Fnotes%2Fmaster%2Fnir.tex
	\[\begin{tikzcd}
		\id && t \\
		\alpha && {\psi_\alpha=t\circ H^n(\alpha)}
		\arrow[maps to, from=2-1, to=2-3]
		\arrow[maps to, from=1-1, to=2-1]
		\arrow[maps to, from=1-1, to=1-3]
		\arrow[maps to, from=1-3, to=2-3]
	\end{tikzcd}\]
	So we conclude that the natural isomorphism in \eqref{eq:serre-pairing} exactly sends $\alpha$ to $t\circ H^n(\alpha)$, which is exactly what we needed for $t$ to be a trace map.
\end{proof}
And here is Serre duality.
\begin{restatable}[Serre duality]{thm}{serreduality} \label{thm:serre-duality}
	Set $\PP\coloneqq\PP^N_k$ for a field $k$, and fix a closed subscheme $i\colon X\to\PP$ of codimension $r$. Further, set $n\coloneqq n-r=\dim X$ and $\omega^\circ_X\coloneqq R^ri^!\omega_\PP$ to be our dualizing sheaf with trace map $t$. For any coherent sheaf $\mc F$ on $X$ and index $i\ge0$, there are natural maps
	\[\theta^i\colon\op{Ext}^i(\mc F,\omega^\circ_X)\to H^{n-i}(X,\mc F)^\lor\]
	is induced by the trace map of our dualizing sheaf. In fact, following are equivalent.
	\begin{listalph}
		\item $X$ is Cohen--Macaulay and equidimensional.
		\item For every locally free sheaf $\mc F$ and $a<n$, we have $H^a(X,\mc F(-q))=0$ for $q$ sufficiently large.
		\item The maps $\theta^i$ are isomorphisms for all $i\ge0$ and $\mc F$ coherent.
		\item We have $R^pi^!\omega_\PP=0$ for $p\ne r$.
	\end{listalph}
\end{restatable}
\begin{proof}[Proof of existence]
	We have $\theta^0$ defined as an isomorphism at \Cref{prop:dualize-sheaf-proj-scheme}. Now, $\op{Ext}^i(-,\omega_X^\circ)$ is a contravariant $\delta$-functor by \Cref{prop:contravariant-ext-les}, as is $H^{n-i}(X,-)^\lor$ because higher-dimensional cohomology vanishes by \Cref{thm:dimension-bound-cohom}. Further, $\op{Ext}^i(-,\omega_X^\circ)$ is universal because it is coeffaceable by \Cref{lem:ext-is-uni} and hence universal by \Cref{thm:eff-is-uni}, and this induces the desired $\theta^i$ maps by definition of being universal.
\end{proof}

\subsection{Depth and All That}
To continue, we must discuss Cohen--Macaulay rings.
\begin{definition}[regular sequence]
	Fix a ring $A$ and an $A$-module $M$. Then a \textit{regular sequence} for $M$ is a sequence of elements $x_1,\ldots,x_n\in A$ such that the $x_i$ are not zero-divisors of $M$, and $x_i$ is not a zero-divisor on $M/(x_1,\ldots,x_{i-1})M$ for each $i$.
\end{definition}
\begin{definition}[depth, Cohen--Macaulay]
	Fix a Noetherian local ring $A$. The \textit{depth} of an $A$-module $M$ is the length of a maximal regular sequence for $M$. The ring $A$ is \textit{Cohen--Macaulay} if and only if $\dim A=\op{depth}A$.
\end{definition}
It is a lemma that the depth of $M$ is well-defined; we will not show it here.
\begin{remark}
	One can show that any regular local ring is Cohen--Macaulay. This is essentially following directly from the definition of regular.
\end{remark}
\begin{remark}
	In fact, any localization of a Cohen--Macaulay ring at a prime ideal is Cohen--Macaulay.
\end{remark}
Let's discuss depth in some detail.
\begin{definition}[projective resolution, length]
	Fix a ring $A$. A \textit{projective resolution} of an $A$-module $M$ is a complex $P_\bullet$ of projective modules exact in all degrees except having cohomology isomorphic to $M$ in degree $0$. The \textit{length} of $P_\bullet$ is the smallest nonnegative integer $n$ such that $P_k=0$ for $k>n$.
\end{definition}
\begin{definition}[projective dimension]
	Fix a ring $A$. The \textit{projective dimension} $\op{pd}(M)$ of an $A$-module $M$ is the least length of a projective resolution for $M$  or is $\infty$ if $M$ has no finite projective resolution.
\end{definition}
\begin{proposition} \label{prop:local-ext-bounds-pd}
	Fix a regular local ring $A$. For a finitely generated $A$-module $M$, we have $\op{pd}(M)\le n$ if and only if $\op{Ext}^i(M,A)=0$ for $i>n$.
\end{proposition}
\begin{proof}
	We explain the easy direction. If $\op{pd}(M)\le n$, we have a projective resolution
	\[0\to P_n\to P_{n-1}\to\cdots\to P_1\to P_0\to M\to0.\]
	Computing $\op{Ext}^i(M,A)$ with this projective resolution (namely, by applying $\op{Hom}(-,A)$ here), we see that terms higher than $n$ vanish.
\end{proof}
\begin{proposition} \label{prop:ext-bounds-pd}
	Fix a module $M$ over a ring $A$. Then $\op{pd}M\le n$ if and only if $\op{Ext}^i(M,N)=0$ for $i>n$ and any $A$-module $N$.
\end{proposition}
\begin{proof}
	Similar to the previous bound.
\end{proof}
\begin{proposition} \label{prop:pd-depth-bound}
	Fix a regular local Noetherian ring $A$. Then for each $A$-module $M$, we have
	\[\op{pd}M+\op{depth}M=\dim A.\]
\end{proposition}

\end{document}