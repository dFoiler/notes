% !TEX root = ../notes.tex

\documentclass[../notes.tex]{subfiles}

\begin{document}

\section{March 20}

Drew has joined us for class today, along with the goose.

\subsection{Serre Duality for \texorpdfstring{$\PP^n_k$}{ Pn}}
Serre duality is the most technical result we will prove in this course. Let's first discuss the case for $\PP\coloneqq\PP^n_A$. The statement of Serre duality requires the notion of a dualizing sheaf, which in our case is $\omega_\PP\coloneqq\Omega_{\PP/A}^{\land n}$. One can show (using \Cref{thm:proj-cohom}) that $\omega=\OO_\PP(-n-1)$: indeed, the short exact sequence
\[0\to\Omega\to\OO(-1)^{\oplus(n+1)}\to\OO\to0\]
induces the short exact sequence
\[0\to\Omega^{\land(n+1)}\to\OO(-n-1)\to\Omega^{\land n}\to0\]
by taking the standard filtration associated to the exterior power. Now, the statement of Serre duality amounts to isomorphisms
\[\mathrm{Ext}^i(\mc F,\omega_\PP)\cong H^{n-i}(X,\mc F)^\lor,\]
where $(-)^\lor$ refers to the dual as an $A$-module.

In general, we will have a locally closed embedding $i\colon X\to\PP^n_A$, where $X$ has codimension $r$. To generalize the above statement of Serre duality, the issue is to replace the dualizing sheaf $\omega_\PP$ with some other sheaf $\omega_X$. (Indeed, we know how to replace all other objects here.) Well, we will define
\[\omega_X^\circ\coloneqq R^ri^!\omega_\PP,\]
where $i^!$ is the sheaf with supports in $Z$, where $i^!\coloneqq\mc Hom_{\OO_\PP}(i_*\OO_X,-)$. In particular, we can see visually that this is a left-exact functor, and we can check locally on open sets that
\[\mc Hom_{\OO_\PP}(i_*\OO_X,-)=\mc Hom_{\OO_X}(\OO_X,i^*-)\]
by the adjunction, so $i^!$ naturally produces $\OO_X$-modules.

Anyway, here is our statement of Serre duality for $\PP^n_k$.
\begin{lemma} \label{lem:ext-is-uni}
	Fix a projective scheme $X$ over a field. Then $\op{Ext}^\bullet(-,\mc G)$ is co-effaceable for any coherent sheaf $\mc G$.
\end{lemma}
\begin{proof}
	Fix some coherent sheaf $\mc F$, and we may use \cite[Corollary~II.5.18]{hartshorne} to find a surjection $\mc E\onto\mc F$ where $\mc E=\bigoplus_{k=1}^N\OO(-q_k)$ for various $q_k>0$, and we may make the $q_k$ as large as we please. Now, we compute
	\[\op{Ext}^i(\mc E,\mc G)=\bigoplus_{k=1}^N\op{Ext}^i(\OO(-q_k),\mc G)=\bigoplus_{k=1}^NH^i(X,\mc G(q_k)).\]
	For $i>n$, this vanishes by \Cref{thm:dimension-bound-cohom}. Then by \Cref{thm:very-ample-kills-cohom}, we can ensure that the $q_\bullet$ are each large enough so that we can kill all the finitely many cohomology groups $H^n(X,\mc G(q_k))$. This completes our check.
\end{proof}
\begin{theorem}[Serre duality for \texorpdfstring{$\PP^n_k$}{Pn}] \label{thm:serre-duality-pn}
	Fix a field $k$, and set $X\coloneqq\PP^n_k$. Define $\omega\coloneqq\Omega_{X/k}^{\land n}$.
	\begin{listalph}
		\item We have $H^n(X,\omega)\cong k$.
		\item For any coherent sheaf $\mc F$, we have a perfect pairing
		\[\op{Ext}^i(\mc F,\omega)\times H^{n-i}(X,\mc F)\to H^n(X,\Omega)\cong k\]
		for each $0\le i\le n$.
	\end{listalph}
\end{theorem}
\begin{proof}
	We know (a) because $\omega=\OO(-n-1)$, so this is simply \Cref{thm:proj-cohom}. For (b), we start in index $0$. We remark that the pairing in index $0$ is induced because a morphism $\mc F\to\omega$ will induce a morphism $H^n(X,\mc F)\to H^n(X,\omega)$. For our proof, we have the following cases.
	\begin{enumerate}
		\item Suppose $\mc F=\OO(q)$ for some $q\in\ZZ$. Remembering that we are looking in index $0$, the statement is
		\[\op{Hom}(\OO,\OO(-q-n-1))\times H^n(X,\OO(q))\to H^n(X,\OO(-n-1))\cong k.\]
		However, this follows directly from \Cref{thm:proj-cohom}.
		\item In general, we know from Corollary~II.5.18 in \cite{hartshorne} that we can write $\mc F$ as the cokernel in an exact sequence
		\[\mc E'\to\mc E\to\mc F\to0,\]
		where $\mc E'$ and $\mc E$ are sums of line bundles $\OO(q)$ for various $q$. Using the left-exactness of $\op{Hom}(\mc F,-)$, we produce the short exact sequence
		\[0\to\op{Hom}(\mc F,\OO(-n-1))\to\op{Hom}(\mc E,\OO(-n-1))\to\op{Hom}(\mc E',\OO(-n-1)).\]
		But now we already have a natural map $\op{Hom}(\mc F,\OO(-n-1))\to H^n(X,\mc F)^\lor$ when we described the pairing above, so we produce the following morphisms.
		% https://q.uiver.app/?q=WzAsOCxbMCwwLCIwIl0sWzEsMCwiXFxvcHtIb219KFxcbWMgRixcXE9PKC1uLTEpKSJdLFsyLDAsIlxcb3B7SG9tfShcXG1jIEUsXFxPTygtbi0xKSkiXSxbMywwLCJcXG9we0hvbX0oXFxtYyBFJyxcXE9PKC1uLTEpKSJdLFswLDEsIjAiXSxbMSwxLCJIXm4oWCxcXG1jIEYpXlxcbG9yIl0sWzIsMSwiSF5uKFgsXFxtYyBFKV5cXGxvciJdLFszLDEsIkhebihYLFxcbWMgRScpXlxcbG9yIl0sWzAsMV0sWzEsMl0sWzIsM10sWzMsNywiXFxzaW0iXSxbMiw2LCJcXHNpbSJdLFs2LDddLFs1LDZdLFs0LDVdLFsxLDVdXQ==&macro_url=https%3A%2F%2Fraw.githubusercontent.com%2FdFoiler%2Fnotes%2Fmaster%2Fnir.tex
		\[\begin{tikzcd}
			0 & {\op{Hom}(\mc F,\OO(-n-1))} & {\op{Hom}(\mc E,\OO(-n-1))} & {\op{Hom}(\mc E',\OO(-n-1))} \\
			0 & {H^n(X,\mc F)^\lor} & {H^n(X,\mc E)^\lor} & {H^n(X,\mc E')^\lor}
			\arrow[from=1-1, to=1-2]
			\arrow[from=1-2, to=1-3]
			\arrow[from=1-3, to=1-4]
			\arrow["\sim", from=1-4, to=2-4]
			\arrow["\sim", from=1-3, to=2-3]
			\arrow[from=2-3, to=2-4]
			\arrow[from=2-2, to=2-3]
			\arrow[from=2-1, to=2-2]
			\arrow[from=1-2, to=2-2]
		\end{tikzcd}\]
		The goal now is to show that the bottom sequence is exact, which will finish the proof. This requires some care. By dualizing, it is enough to show that we have a right-exact sequence
		\[H^n(X,\mc E')\to H^n(X,\mc E)\to H^n(X,\mc F)\to0.\]
		Well, we let $\mc R$ denote the kernel of the map $\mc E\onto\mc F$, and let $\mc R'$ denote the kernel of the map $\mc E'\onto\mc R$. We now have two checks.
		\begin{itemize}
			\item The long exact sequence of $0\to\mc R\to\mc E\to\mc F\to0$ produces the exact sequence
			\[H^n(X,\mc R)\to H^n(X,\mc E)\to H^n(X,\mc F)\to0.\]
			We want to replace $H^n(X,\mc R)$ with $H^n(X,\mc E')$ here.
			\item The long exact sequence of $0\to\mc R''\to\mc E'\to\mc R\to0$ produces the surjection $H^n(X,\mc E')\onto H^n(X,\mc R)$, so plugging this in grants the exact sequence
			\[H^n(X,\mc E')\to H^n(X,\mc E)\to H^n(X,\mc F)\to0.\]
		\end{itemize}
		The above checks complete the proof in degree $0$.
	\end{enumerate}
	It remains to extend past $i=0$. In fact, we will exhibit a natural isomorphism
	\[\mathrm{Ext}^i(-,\omega)\simeq H^{n-i}(X,-)^\lor.\]
	We are going to make some kind of $\delta$-functor argument. In particular, we can see automatically that both sides are contravariant $\delta$-functors (for example, the left-hand side is a contravariant $\delta$-functor by \Cref{prop:contravariant-ext-les}). Further, we have natural isomorphisms in degree $0$ by the above argument.
	
	It remains to show that these functors are universal, for which by \Cref{thm:eff-is-uni} it is enough to show that these functors are co-effaceable. This has two checks; fix some coherent sheaf $\mc F$, and we may use \cite[Corollary~II.5.18]{hartshorne} to find a surjection $\mc E\onto\mc F$ where $\mc E=\bigoplus_{k=1}^N\OO(-q_k)$ for various $q_k>0$.
	\begin{itemize}
		\item We check that $\op{Ext}^i(-,\omega)$ is co-effaceable. This is exactly \Cref{lem:ext-is-uni}.
		\item We check that $H^{n-i}(X,-)^\lor$ is co-effaceable. Again, we compute
		\[H^{n-i}(X,\mc E)^\lor\cong\bigoplus_{k=1}^NH^{n-i}(X,\OO(-q_k))^\lor.\]
		Again, for $0<i<n$, these factors vanish automatically for \Cref{thm:proj-cohom}; for $i=n$, we still have $H^0(X,\OO(-q_k))=0$ because $q_k>0$. Thus, the surjection $\mc E\onto\mc F$ witnesses that $\op{Ext}^i(-,\omega)$ is co-effaceable.
	\end{itemize}
	In total, the uniqueness of our universal (contravariant) $\delta$-functors completes the proof.
\end{proof}

\end{document}