% !TEX root = ../notes.tex

\documentclass[../notes.tex]{subfiles}

\begin{document}

Here we go.

\subsection{Ample Line Bundles via Cohomology}
Let's return to cohomology but now with some ample flavor.
\begin{proposition}
	Fix a proper $X$ over a Noetherian ring $A$. Given a line bundle $\mc L$ on $X$, the following are equivalent.
	\begin{listroman}
		\item $\mc L$ is ample.
		\item For any coherent sheaf $\mc F$ on $X$, there is an integer $n_0(\mc F)$ such that $H^i(X,\mc F\otimes\mc L^{\otimes n})=0$ for $i>0$ and $n>n_0(\mc F)$.
	\end{listroman}
\end{proposition}
\begin{proof}
	We have two implications to show. Quickly, we show (i) implies (ii) so that $\mc L$ is ample. Because $X$ has an ample sheaf $\mc L$, there is a very ample power $\mc L^{\otimes m}$ by \Cref{prop:power-ample-is-very-ample}. Now using \Cref{thm:coh-cohom-proj-space}, for each $k$ between $0$ and $n-1$, we can find some $n_k$ such that $H^i(X,\mc F\otimes\mc L^{\otimes k}(n))=0$ for $n>n_k$, where $\OO_X(1)\coloneqq\mc L^{\otimes m}$. Taking $n_0(\mc F)\coloneqq\max\{(m+1)(n_k+1):0\le k\le n-1\}$ will now work.

	We show (ii) implies (i). This is harder. We proceed in steps.
	\begin{enumerate}
		\item Fix a coherent sheaf $\mc F$ on $X$ a closed point $p\in X$. We claim that there is an integer $n_0$ (depending on $\mc F$ and $p$) such that any $n\ge n_0$ has an open neighborhood $U\subseteq X$ such that $\Gamma(X,\mc F\otimes\mc L^{\otimes n})$ generates $(\mc F\otimes\mc L^{\otimes n})|_U$.

		To begin, the fact that $p$ is closed grants us an exact sequence
		\[0\to\mc I_p\to\OO_X\to k(p)\to0,\]
		so
		\[\mc I_p\otimes\mc F\to\mc F\to\mc F\otimes k(p)\to0\]
		is exact. Computing the image of $\mc I_p\otimes\mc F$ in $\mc F$, we see that
		\[0\to\mc I_p\mc F\to\mc F\to\mc F\otimes k(p)\to0\]
		is exact. The point is that (ii) tells us that some $n_0$ has $H^1(X,\mc I_p\mc F\otimes\mc L^{\otimes n})=0$ for $n\ge n_0$, meaning
		\[\Gamma(X,\mc F\otimes\mc L^{\otimes n})\to\Gamma(X,\mc F\otimes\mc L^{\otimes n}\otimes k(p))\]
		is surjective for $n\ge n_0$. Thus, Nakayama's lemma tells us that $\Gamma(X,\mc F\otimes\mc L^{\otimes n})$ generates $(\mc F\otimes\mc L^{\otimes n})_p$.
		
		As such, there is a map $\pi\colon\OO_X^{\oplus N}\to\mc F\otimes\mc L^{\otimes n}$ which is surjective at $p$. It remains to spread out from $p$. Well, choose $U$ to be the complement of the support of $\coker\pi$, meaning that $(\coker\pi)|_U$ vanishes, so $\pi$ continues to be surjective on $U$, and we are done.
		
		\item We upgrade the previous step to make $U$ independent of $n$. We still make allow $U$ to depend on $\mc F$ and $p$, which we fix. Notably, the first claim does grant $m>0$ and open $V\subseteq X$ such that $\mc L^{\otimes m}$ is globally generated over $V$.

		Now, using the previous step on $\mc F\otimes\mc L^{\otimes r}$ for each $0\le r<m$, we get some $n_0$ very large and open subsets $U_r$ for each $r$ such that $\mc F\otimes\mc L^{\otimes(n_0+r)}$ is globally generated over $U_r$ for each $r$. We now set
		\[U\coloneqq V\cap\bigcap_{r=0}^{m-1}U_r,\]
		which is now independent of $n$: for any $n\ge n_0$, write $n=n_0+km+r$ with $k\ge0$ and $0\le r<m$, and we see that
		\[\mc F\otimes\mc L^{n}=\mc F\otimes\mc L^{\otimes(n_0+r)}\otimes\mc L^{\otimes mk}\]
		is the tensor product of sheaves globally generated over $U$, which continues to be globally generated.

		\item We now finish the proof. Because $X$ is Noetherian, it is quasicompact, so any nonempty closed subset has a closed point by \Cref{lem:qc-has-closed-point}. So we claim that the open neighborhoods $U_p$ of the previous step cover $X$: if not, then the complement is closed, which has a closed point $p$, but then $U_p$ means that it can have no closed point.

		Thus, quasicompactness of $X$ grants finitely many closed points $\{p_1,\ldots,p_r\}$ such that the $U_{p_i}$ cover $X$, and each $p_i$ has been granted some $n_i$ such that $\mc F\otimes\mc L^{\otimes n}$ is globally generated over $U_i$ for each $n\ge n_i$. Choosing $n\coloneqq\max\{n_i:1\le i\le r\}$ completes the argument because we have an open cover.
		\qedhere
	\end{enumerate}
\end{proof}
\begin{remark}
	A close examination of the proof of (ii) implies (i) shows that we only use (ii) in the first step, and we only use it to show $H^1$ vanishes.
\end{remark}

\subsection{The Euler Characteristic}
While we're here, we talk about the Euler characteristic quickly.
\begin{definition}[Euler characteristic]
	Fix a $k$-scheme $X$. Then given a coherent sheaf $\mc F$ on $X$, the \textit{Euler characteristic} $\chi(\mc F)$ is
	\[\chi(\mc F)\coloneqq\sum_{i\ge0}\dim_kH^i(X,\mc F).\]
	Note that these dimensions are finite by \Cref{thm:coh-cohom-proj-space}.
\end{definition}
\begin{remark}
	If one has an exact sequence
	\[0\to\mc F_0\to\mc F_1\to\cdots\to\mc F_n\to0\]
	of coherent sheaves on $X$, then $\sum_{i=0}^n(-1)^i\chi(\mc F_i)=0$. Indeed, if $n=2$, then this is direct from the long exact sequence; for the general case, one inducts on $n$. (For example, if $n=0$ or $n=1$, there is nothing to do, and for higher $n$, one can divide up the long exact sequence into short ones.)
\end{remark}
There is a notion of Hilbert polynomial arising from the Euler characteristic.
\begin{lemma}
	Fix a projective $k$-scheme $X$, and let $\OO_X(1)$ be a very ample line bundle. Given a coherent sheaf $\mc F$ on $X$, there is a polynomial $P\in\QQ[x]$ such that $\chi(\mc F(n))=P(n)$ for $n$ sufficiently large.
\end{lemma}
\begin{proof}
	Morally, one inducts on the support of $\mc F$.
\end{proof}

\end{document}