% !TEX root = ../notes.tex

\documentclass[../notes.tex]{subfiles}

\begin{document}

Here we go.

\subsection{Degree of Divisors}
We begin by defining degree in the most geometric situation.
\begin{definition}[degree]
	Fix a field $k$ and positive integer $n>0$. Then let $Y$ be a prime divisor of $X\coloneqq\PP^n_k$, which we remark is smoothish. Because $Y$ is integral, we may write $Y=V((f))$ for some irreducible homogeneous polynomial $f$ (unique up to multiplication by an element of $k^\times$), so we define the \textit{degree} as
	\[\deg Y\coloneqq\deg f.\]
\end{definition}
\begin{remark}
	To see that $Y$ takes the form $V((f))$, note that $Y\into X$ is an integral closed subscheme of codimension $1$, so we can affine-locally realize it via a quotient of $k[x_{0/i},\ldots,\widehat{x_{i/i}},\ldots,x_{n/i}]$ of codimension $1$, and here dimension theory tells us that we will be cut out by a single irreducible polynomial. Then we can glue together our irreducible polynomials to complete.
\end{remark}
Let's understand this notion of degree.
\begin{proposition}
	Fix a field $k$ and positive integer $n>0$ so that $X\coloneqq\PP^n_k$ is smoothish. Let $H$ be the hyperplane cut out by $x_0$.
	\begin{listalph}
		\item For any divisor $D$ on $X$, we have $D\sim dH$ where $d=\deg D$.
		\item All principal divisors on $f$ have degree $0$.
		\item $\deg$ induces an isomorphism $\op{Cl}X\to\ZZ$.
	\end{listalph}
\end{proposition}
\begin{proof}
	Here we go.
	\begin{listalph}
		\item By linearity, we may assume that $D$ is a prime divisor $Y$ of the form $V((f))$ for some irreducible homogeneous polynomial of degree $d$. Then $f/x_0^d\in K(X)^\times$ (it is a rational section defined on the distinguished open subscheme cut out by $x_0\ne0$), so the principal divisor associated to $f/x_0^d$ is $Y-dH$, so we are done.
		\item We omit this argument.
		\item By (a), we have a surjective homomorphism $\deg\colon\op{Div}X\to\ZZ$. Note that this is well-defined up to equivalence: if $D\sim D'$, then we see $(\deg D)H\sim(\deg D')H$ by (a), so to show that $\deg D=\deg D'$, it suffices to show that $H$ is not equivalent to $0$. But $H$ has degree $1$ (it's cut out by $x_0$), so we get the claim by (b).

		Thus, we have a well-defined surjection $\deg\op{Cl}X\to\ZZ$. Setting $U\coloneqq X\setminus H$ (which we note is isomorphic to $\AA^n_k$), we see that \Cref{prop:cl-exact-seq} provides an exact sequence
		\[\ZZ[H]\to\op{Cl}X\to\op{Cl}U\to0,\]
		but the last term vanishes because $\AA^n_k$ is the spectrum of a unique factorization domain. So the map $\ZZ[H]\to\op{Cl}X$ is surjective and has trivial kernel because any kernel would give kernel in $\ZZ[H]\to\op{Cl}X\to\ZZ$ (where the last map is $\deg$), but the composite $\ZZ[H]\to\ZZ$ is the identity.
		\qedhere
	\end{listalph}
\end{proof}
\begin{remark}
	One can show that $X\times\AA^1$ satisfies $\op{Cl}(X\times\AA^1)\cong\op{Cl}X$.
\end{remark}
\begin{remark}
	It is not generally true that $\op{Cl}(X\times Y)\cong\op{Cl}X\times\op{Cl}Y$. For example, take $X\subseteq\PP^2_k$ to be cut out by $y^2z=x^3-xz^2$, and let $\Delta\subseteq X\times_kX$ be the diagonal. Then $\Delta$ is a prime divisor but not linearly equivalent to any divisor coming from $\op{Cl}X\times\op{Cl}X$. Namely, we are claiming that $\Delta$ fails to equivalent to one of the form $\sum_iP_i\times X+\sum_jX\times Q_j$ for closed points $\{P_i\}$ and $\{Q_j\}$ of $X$. Otherwise, we get $f\in K(X\times X)$ such that
	\[\Delta=\sum_iP_i\times X+\sum_jX\times Q_j+\op{div}f.\]
	Choosing $R_1,R_2\in X$ distinct from the $P_i$ and $Q_j$, then $f|_{R_1\times X}$ produces a rational function on $X$ with $\op{div}f=R_1-\sum_jQ_j$, and similarly one gets $f_2$ with $\op{div}f_2=R_2-\sum_jQ_j$, but then $R_1\sim R_2$, which contradicts $X$ not being isomorphic to $\PP^1_k$.
\end{remark}

\subsection{Rational Maps}
We will also want to discuss rational maps in some detail before continuing.
\begin{definition}[variety]
	Fix an algebraically closed field $k$. Then a \textit{variety} over $k$ is an integral separated $k$-scheme of finite type.
\end{definition}
We want to define rational maps, which we do as follows.
\begin{definition}[dominant]
	A morphism $f\colon X\to Y$ of integral schemes is \textit{dominant} if and only if $f(X)$ is Zariski dense in $Y$.
\end{definition}
\begin{remark}
	Suppose that $X$ and $Y$ are integral schemes with generic points $\xi$ and $\eta$, respectively. Then we claim that $f\colon X\to Y$ is dominant if and only if $f(\xi)=\eta$. Certainly if $f(\xi)=\eta$ than the image of $f$ is Zariski dense. Conversely, if $\xi$ specializes to some $x\in X$, then we note that $f(\xi)$ will specialize to some $f(x)$ by continuity of $f$, so $\ov{f(X)}\subseteq\ov{\{f(\xi)}$, so for this to be all of $Y$ we must have $f(\xi)=\eta$.
\end{remark}
\begin{definition}[rational]
	Fix integral separated schemes $X$ and $Y$. Then a \textit{rational map} is an equivalence class of maps $(U,\varphi)$ where $\varphi\colon U\to Y$ is a bona fide morphism, where we declare $(U,\varphi)\sim(V,\psi)$ if and only if $\varphi|_{U\cap V}=\psi|_{U\cap V}$. The rational map $\varphi\colon X\to Y$ is dominant if its representatives are.
\end{definition}
\begin{remark}
	Let's discuss how to check that this is an equivalence relation. Reflexivity and symmetry have little content, but transitivity requires us to remark that any two rational maps will agree on a closed subset of their domain (this is where being separated is used), so agreeing on a closed subset means that they will actually agree.
\end{remark}
\begin{remark}
	One can compose rational maps exactly as expected, with the caveat that we need to work in smaller and smaller Zariski open subsets.
\end{remark}
\begin{remark}
	Any rational map $\varphi\colon X\to Y$ has a unique largest open subset $U$ where it is defined; indeed, one can simply take the union of all the $U$s appearing in the equivalence class. The point here is that one can specify the entire rational map by specifying how it behaves on any open subscheme.
\end{remark}
With maps in one direction, we should discuss having maps in both directions.
\begin{definition}[birational]
	Fix integral separated schemes $X$ and $Y$. A \textit{birational} map is a rational map $\varphi\colon X\to Y$ with a birational inverse map.
\end{definition}
\begin{remark}
	One can see that birational maps are necessarily dominant because the image needs to surject onto some Zariski open subset.
\end{remark}

\end{document}