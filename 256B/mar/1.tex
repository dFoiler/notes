% !TEX root = ../notes.tex

\documentclass[../notes.tex]{subfiles}

\begin{document}

Today we discuss line bundles.

\subsection{Ample Line Bundles}
The following result helps motivate the notion of ample.
\begin{proposition}[Serre] \label{prop:very-ample-is-ample}
	Fix a projective scheme $X$ over a Noetherian ring $A$, and let $\OO_X(1)$ be some very ample line bundle. For any coherent quasicoherent sheaf $\mc F$ on $X$, there is an integer $n_0(\mc F)$ such that $\mc F(n)$ is generated by finitely many global sections for $n\ge n_0(\mc F)$.
\end{proposition}
\begin{proof}
	We proceed in steps.
	\begin{enumerate}
		\item We reduce to the case of $X=\PP^r_A$. Because $X$ is projective, and $\OO_X(1)$ is very ample, we are promised some closed embedding $i\colon X\to\PP^r_A$ such that $\OO_X(1)=i^*\OO_{\PP^r_A}(1)$. Because $i$ is finite (and $A$ Noetherian), we see that $i_*\mc F$ continues to be coherent. Notably, $H^0(X,\mc F)=H^0(\PP^r_A,i_*\mc F)$.

		For all affine open subschemes $U=\Spec B$ of $\PP^r_A$, the fact that $i$ is closed allows us to say $V\coloneqq i^{-1}(U)$ is an affine open subscheme $\Spec B/i$ of $X$ (where $I\subseteq B$ is some ideal). Now, $\mc F$ is coherent, so $\mc F|_V=\widetilde M$ for some finitely generated $(B/I)$-module $M$. Further, $(i_*\mc F)|_U=\widetilde M$, where $M$ is now viewed as a $B$-module via $B\onto B/I$, so if we have global sections $s_0,\ldots,s_n\in H^0(\PP^r_A,i_*\mc F)$ globally generating $i_*\mc F$, they will restrict to some $m_1,\ldots,m_n\in M$ which generate $M$ as a $B$-module and hence generate as a $(B/I)$-module. Thus, $\mc F|_V$ is generated by these global sections for all affine open subschemes $V$, which is good enough.

		\item We complete the proof. Cover $X=\PP^r_A$ with the standard affine open subschemes $D_+(x_i)$ where $0\le i\le r$; say $D_+(x_i)=B_i\coloneqq A[x_0/x_i,\ldots,x_n/x_i]$. Notably, $\mc F|_{D_+(x_i)}\cong\widetilde{M_i}$ for some finitely generated $B_i$-module $M_i$, so we may let $\{s_{ij}\}_{j=1}^{m_i}$ be a set of generators. Notably, for each $s_{ij}$, we can imagine cancelling out any possible pole by multiplying by $x_i^{n_{ij}}$ for some integer $n_{ij}$, making $x_i^{n_{ij}}s_{ij}$ a global section of $\mc F(n_{ij})$. (See \cite[Exercise~II.5.14]{hartshorne}.) Let $n$ be the maximum of all the $n_{ij}$s so that $x^ns_{ij}$ is a global section of $\mc F(n)$.

		Now, $\mc F(n)|_{D_+(x_i)}$ is isomorphic to $\widetilde{M_i'}$ for some $B_i$-module $M_i$, and $x_i^n\colon\mc F\to\mc F(n)$ will produce an isomorphism $M_i\to M_i'$ by how twisting works, so the global sections $x_i^ns_{ij}$ are able to generate each $\widetilde{M_i'}$. So our global sections are able to generate $\mc F(n)$, as required.
		\qedhere
	\end{enumerate}
\end{proof}
So let's codify the conclusion of the above result.
\begin{definition}[globally generated]
	Fix a scheme $X$. An $\mathcal O_X$-module $\mc F$ is \textit{globally generated} if and only if $\Gamma(X,\mc F)$ generates $\mc F$.
\end{definition}
\begin{remark}
	Perhaps one is worried about finite generation. Well, if $\mc F$ is a coherent sheaf on a Noetherian scheme $X$, then being globally generated means that $\mc F$ is actually generated by a finite set of global sections. Morally, letting $S$ be a generating subset of global sections, we cover $X$ by finitely many open affine Noetherian subschemes, and then over each subscheme, only finitely many global sections of $S$ need to be used because $\mc F$ is coherent. So we only pick out finitely many global sections from $S$ that are needed to generate it.
\end{remark}
This leads to the following definition.
\begin{definition}[ample]
	A line bundle $\mc L$ on a Noetherian scheme $X$ is \textit{ample} if and only if any coherent sheaf $\mc F$ on $X$ has some integer $n_0(\mc F)\in\ZZ$ such that $\mc F\otimes\mc L^{\otimes n}$ is globally generated for any $n\ge n_0(\mc F)$.
\end{definition}
\begin{remark}
	A very ample line bundle is always relative to the (locally closed) embedding of $X$ into projective space.
\end{remark}
\begin{example}
	\Cref{prop:very-ample-is-ample} explains that very ample line bundles on projective schemes are ample.
\end{example}
\begin{example}
	Fix an affine Noetherian scheme $X=\Spec A$. Then every coherent sheaf on $X$ is globally generated, so actually any line bundle $\mc L$ is ample: for any coherent sheaf $\mc F$ on $X$, we see $\mc F\otimes\mc L^{\otimes n}$ continues to be coherent for any $n\ge0$ (say), so it is globally generated.
\end{example}
Let's run some basic checks on ample line bundles.
\begin{lemma} \label{lem:power-ample}
	Fix a line bundle $\mc L$ on a Noetherian scheme $X$. Then the following are equivalent.
	\begin{listroman}
		\item $\mc L$ is ample.
		\item $\mc L^{\otimes m}$ is ample for all $m>0$.
		\item $\mc L^{\otimes m}$ is ample for some $m>0$.
	\end{listroman}
\end{lemma}
\begin{proof}
	Note (ii) implies (iii) has little content. Also, for (i) implies (ii), we proceed via the definitions directly: for any coherent sheaf $\mc F$, we know that there is a nonnegative integer $n_0(\mc F)$ such that $\mc F\otimes\mc L^{\otimes n}$ for $n\ge n_0(\mc F)$, but then $\mc F\otimes\left(\mc L^{\otimes m}\right)^{\otimes n}$ for $n\ge n_0(\mc F)$ also. Thus, $\mc L^{\otimes m}$ is ample.

	So it remains to show that (iii) implies (i), which requires a trick. Fix a coherent sheaf $\mc F$, and we want to show that $\mc F\otimes\mc L^{\otimes n}$ is globally generated for sufficiently large $n$. Well, $\mc L^{\otimes m}$ is ample, so for each $i\in\{0,\ldots,m-1\}$, there is some $n_i(\mc F)\ge0$ such that
	\[\left(\mc F\otimes\mc L^{\otimes i}\right)\otimes\left(\mc L^{\otimes m}\right)^{\otimes n}=\mc F\otimes\mc L^{\otimes(i+mn)}\]
	is globally generated for $n\ge n_i(\mc F)$. (Namely, we recall $\mc F\otimes\mc L^{\otimes i}$ is coherent!) Thus, for $n\ge m(\max\{n_i\}+1)$, we see that $\mc F\otimes\mc L^{\otimes n}$ is globally generated by finding the $i\in\{0,\ldots,m-1\}$ with $n\equiv i\pmod m$ and applying the previous sentence.
\end{proof}
\begin{lemma} \label{lem:restrict-ample}
	Fix an ample line bundle $\mc L$ on a Noetherian scheme $X$. For any open subscheme $U\subseteq X$, the line bundle $\mc L|_U$ remains ample.
\end{lemma}
\begin{proof}
	This is not as easy as it might look. Let $\mc F$ be a coherent sheaf on $U$. The difficulty is that we must use \cite[Exercise~II.5.15]{hartshorne} in order to build a coherent sheaf $\mc F'$ on $X$ such that $\mc F'|_U=\mc F$; morally, one reduces to the affine case by some technical argument, and then some sort of extension by zero can work. Now, there is some $n_0(\mc F')$ such that $\mc F'\otimes\mc L^{\otimes n}$ is globally generated for $n\ge n_0(\mc F')$, meaning that
	\[\left(\mc F'\otimes\mc L^{\otimes n}\right)|_U=\mc F\otimes(\mc L|_U)^{\otimes n}\]
	continues to be globally generated for $n\ge n_0(\mc F')$, as desired. (The restriction is globally generated basically by viewing the global generation condition as asking for a surjective map $\OO_X^{\oplus m}\onto\mc F'\otimes\mc L^{\otimes n}$.)
\end{proof}
We now use ample line bundles to build very ample line bundles. Because we are allowed to ``take roots'' of ample line bundles, we will have to ``take powers'' below.
\begin{proposition}
	Fix a Noetherian scheme $X$ of finite type over a Noetherian ring $A$. Given a line bundle $\mc L$ on $X$, then $\mc L$ is ample if and only if and only if $\mc L^{\otimes m}$ is very ample for some positive integer $m>0$.
\end{proposition}
\begin{proof}
	The forward direction is essentially \Cref{prop:very-ample-is-ample}. Indeed, if $\mc L^{\otimes m}$ is very ample, find some locally closed embedding $i\colon X\to\PP^r_A$ such that $\mc L^{\otimes m}\cong i^*\OO_){\PP^r_A}(1)$. Because $i$ is locally closed, we can write $i$ as $i_2\circ i_1$ where $i_1\colon X\to\ov X$ is open, and $i_2\colon\ov X\to\PP^r_A$ is closed. Then $i_2^*\OO_{\PP^r_A}(1)$ is very ample on $\ov X$, hence ample by \Cref{prop:very-ample-is-ample}, so its restriction to $X$ is ample by \Cref{lem:restrict-ample}, so $\mc L$ is ample by \Cref{lem:power-ample}.

	We will show the other direction next class.
\end{proof}

\end{document}