% !TEX root = ../notes.tex

\documentclass[../notes.tex]{subfiles}

\begin{document}

Today we finish discussing line bundles.

\subsection{More on Ample Line Bundles}
We now use ample line bundles to build very ample line bundles. Because we are allowed to ``take roots'' of ample line bundles, we will have to ``take powers'' below.
\begin{proposition} \label{prop:power-ample-is-very-ample}
	Fix a Noetherian scheme $X$ of finite type over a Noetherian ring $A$. Given a line bundle $\mc L$ on $X$, then $\mc L$ is ample if and only if and only if $\mc L^{\otimes m}$ is very ample for some positive integer $m>0$.
\end{proposition}
\begin{proof}
	The forward direction is essentially \Cref{prop:very-ample-is-ample}. Indeed, if $\mc L^{\otimes m}$ is very ample, find some locally closed embedding $i\colon X\to\PP^r_A$ such that $\mc L^{\otimes m}\cong i^*\OO_){\PP^r_A}(1)$. Because $i$ is locally closed, we can write $i$ as $i_2\circ i_1$ where $i_1\colon X\to\ov X$ is open, and $i_2\colon\ov X\to\PP^r_A$ is closed. Then $i_2^*\OO_{\PP^r_A}(1)$ is very ample on $\ov X$, hence ample by \Cref{prop:very-ample-is-ample}, so its restriction to $X$ is ample by \Cref{lem:restrict-ample}, so $\mc L$ is ample by \Cref{lem:power-ample}.

	It remains to show the forward direction. The point is that $\mathcal L$ and some generating sections will determine a morphism to projective space, which we eventually want to be a closed embedding. In particular, we need to separate points and tangent vectors to be a closed embedding. Anyway, we proceed in steps.
	\begin{enumerate}
		\item As such, as a starting step, we claim that any $p\in X$ has $n>0$ and a section $s\in\Gamma\left(X,\mathcal L^{\otimes n}\right)$ such that $X_s$ contains $p$ and is affine. (Recall $X_s$ consists of $q\in X$ such that $s_q\notin\mf m_q\mc L^{\otimes n}_q$; in particular, $X_s$ is open.) To begin, let $U$ be an affine open neighborhood of $p$ such that $\mc L|_U=\OO_U$, and let $Y\coloneqq X\setminus U$ be the complement, which we give the reduced scheme structure. While we're here, we also let $\mathcal I_Y$ be the coherent ideal sheaf corresponding to $Y$ (note $\mathcal I_Y$ is coherent because $X$ is Noetherian).

		Thus, ampleness of $\mc L$ tells us that $\mc I_Y\otimes\mc L^{\otimes n}$ is globally generated for $n$ sufficiently large. In particular, we may find a global section $s\in\Gamma(X,\mc I_Y\otimes\mc L^{\otimes n})$ such that $s_p\notin\mf m_p(\mc I_Y\otimes\mc L^{\otimes n})_p$. Now, $\mc I_Y\subseteq\OO_X$, so $\Gamma(X,\mc I_Y\otimes\mc L^{\otimes n})\to\Gamma(X,\mc L^{\otimes n})$ is injective, so we may view $s$ as a global section of $\mc L^{\otimes n}$.

		Continuing, $p\notin Y$ means $(i_*\OO_Y)_p=0$ (here, $i\colon Y\to X$ is the embedding), so the short exact sequence
		\[0\to\mc I_Y\to\OO_X\to i_*\OO_Y\to0\]
		being exact at the stalk at $p$ forces the inclusion $\mc I_{Y,p}\to\OO_{X,p}$ to be an isomorphism. As such, we see $s_p\notin\mf m_p\mc L^{\otimes n}_p$. However, the failure of this to be an isomorphism for $q\in Y$ means that $s_q\in\mf m_q\mc L^{\otimes n}_q$ for $q\in Y$. So we are able to conclude that $p\in X_s$ and $X_s\subseteq U$; in particular, $X_s$ is now affine because it is a distinguished open subscheme of the affine scheme $U$, where $s|_U$ is being viewed as a global section of $\OO_U$ because $\mc L^{\otimes n}|_U\cong\OO_U$.

		\item We upgrade the open cover produced by the previous step. Now, because $X$ is quasicompact, we can find finitely many sections $\{s_1,\ldots,s_k\}$ such that the $X_{s_i}$ are affine and cover $X$. Now, by construction, $s_i\in\Gamma(X,\mc L^{\otimes n_i})$ for some $n_i>0$, so we set $n\coloneqq\lcm(n_1,\ldots,n_k)$ and replace $s_i$ with a power so that $s_i\in\Gamma(X,\mc L^{\otimes n})$. We also note that $\mc L^{\otimes n}$ is ample here, so we may as well replace $\mc L$ with $\mc L^{\otimes n}$ so that actually $n=1$, and the $s_\bullet$ are global sections of $\mc L$.

		Continue with the open cover $X_i\coloneqq X_{s_i}$ where $1\le i\le k$ as in the previous step. Because $X$ is of finite type over $A$, we may write $B_i=A[b_{i1},\ldots,b_{ik_i}]$. As we stated earlier, each $i$ and $j$ have some $n_{ij}>0$ such that $s_i^{n_{ij}}b_{ij}$ is the restriction of a global section $c_{ij}\in\Gamma(X,\mc L^{\otimes n_{ij}})$; again, by taking $\lcm$s and replacing $\mc L$ with a power of itself, we may assume that $n_{ij}=1$ for all $i$ and $j$. As such, we see that the $s_i$ and $c_{ij}$ are global sections which generate $\mc L$.

		\item At this point, we have a list of global sections generating $\mc L$, so we produce a morphism $\varphi\colon X\to\PP^N_A$ to projective space. In particular, we can write $\PP^N_A=\Proj k[x_i,y_{ij}]$ so that $\varphi^*x_i=s_i$ and $\varphi^*y_{ij}=c_{ij}$ for each $i$ and $j$; for example, $X_{s_i}$ is the pre-image of $D_+(x_i)$. Notably, the induced map $X_{s_{i_0}}\to D_+(x_{i_0})$ is a closed embedding: by construction, on rings, this map looks like $A[x_i/x_{i_0},y_{ij}/x_{i_0}]\to B_{i_0}$, which is surjective by construction of the map and the $c_{ij}$s. As such, $X$ is a closed subscheme of an open subscheme of $\PP^N_A$ via the map given by global sections of $\mc L$, so we conclude that $\mc L$ is very ample.
		\qedhere
	\end{enumerate}
\end{proof}
Let's see some extra miscellaneous facts about ample and very ample line bundles.
\begin{lemma}
	Fix a morphism $\pi\colon X\to Y$ of Noetherian schemes.
	\begin{listalph}
		\item If $\mc L$ is $\pi$-very ample, then $\mc L^{\otimes n}$ is $\pi$-very ample for any $n\ge0$.
		\item If $\mc L$ and $\mc M$ are both very ample, then $\mc L\otimes\mc M$ is very ample. 
	\end{listalph}
\end{lemma}
\begin{proof}
	For (a), use the $n$-uple embedding, which is notably closed. More explicitly, if $\mc L$ is very ample, then it comes from some locally closed embedding $i\colon X\to\PP^N_Y$, so we can post-compose with the closed embedding $\PP^N_Y\to\PP^{nN}_Y$, which effectively takes powers of $\mc L$ as needed. For (b), proceed as above, but now we note that there is a closed embedding $\PP^N_A\times\PP^M_A\to\PP^{NM+M+N-1}_A$ which we want to post-compose by to get $\mc L\otimes\mc M$.
\end{proof}
And here are some facts about ampleness.
\begin{lemma}
	Fix a Noetherian scheme $X$.
	\begin{listalph}
		\item If $\mc L$ and $\mc M$ are ample, then $\mc L\otimes\mc M$ is also ample.
		\item If $\mc L$ is ample, and $i\colon X'\to X$ is a locally closed embedding, then $i^*\mc L$ continues to be ample.
		\item For $r>0$ and $n\le0$, then $\OO_{\PP^r_A}(n)$ is not ample on $\PP^r_A$.
	\end{listalph}
\end{lemma}
\begin{proof}
	One proceeds directly from the definitions for (a), and one pulls back according to the definitions for (b). Lastly, for (c), we simply have to note that $\OO_{\PP^r_A}(n)$ never has global sections for $n<0$, so no tensor power of it can succeed to be very ample, so it can never succeed to be ample.
\end{proof}

\end{document}