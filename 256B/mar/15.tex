% !TEX root = ../notes.tex

\documentclass[../notes.tex]{subfiles}

\begin{document}

\section{March 15}

We continue discussing line bundles with cohomology.

\subsection{A Criterion for Being Ample}
As another application, we will be able to give a cohomological criterion for a sheaf to ample.
\begin{theorem}
	Fix a proper scheme $X$ over a Noetherian ring $A$. For any line bundle $\mc L$, the following are equivalent.
	\begin{listalph}
		\item $\mc L$ is ample.
		\item For any coherent $\mc F$, we have $H^i\left(X,\mc F\otimes\mc L^{\otimes n}\right)=0$ for $n$ sufficiently large, for any $i$.
		\item For any coherent $\mc F$, we have $H^1\left(X,\mc F\otimes\mc L^{\otimes n}\right)=0$ for $n$ sufficiently large.
	\end{listalph}
\end{theorem}
\begin{proof}
	We show our implications independently.
	\begin{itemize}
		\item Suppose that $\mc L$ is ample. Then $\mc M\coloneqq\mc L^{\otimes m}$ is very ample for some $m$. Now, for a fixed very ample sheaf $\mc F$ and index $i$, for each $k\in[0,m)$, we may find some $n_k$ such that
		\[H^i\left(X,\mc F\otimes\mc L^k\otimes\mc M^n\right)=0\]
		for $n\ge n_k$. Letting $N$ be the maximum of all the $n_i$, we see that we have looped over all our residue classes, so for $n>m+N$, we see that $H^i\left(X,\mc F\otimes\mc L^n\right)=0$, which is what we wanted.
		\item Conversely, suppose (b). Given a coherent sheaf $\mc F$, we want to know that $\mc F\otimes\mc L^{\otimes n}$ is globally generated for $n$ sufficiently large. Well, fix some closed point $p$. This grants us the exact sequence
		\[0\to\mc I_p\to\OO_X\to k(p)\to0,\]
		where $\mc I_p$ is the corresponding ideal sheaf. Tensoring with $\mc F$, we get the right-exact sequence
		\[\mc I_p\otimes\mc F\to\mc F\to\to\mc F\otimes k(p)\otimes\mc F\to0.\]
		Now, the image of $\mc I_p\otimes\mc F$ in $\mc F$ is simply sections of $\mc F$ multiplied by sections of $\mc I_p$, so we are looking at
		\[0\to\mc I_p\mc F\to\mc F\to\mc F\otimes k(p)\to0.\]
		Notably, because $\mc L^{\otimes n}$ is flat (it's locally trivial), we get the exact sequence
		\[0\to\mc I_p\mc F\otimes\mc L^{\otimes n}\to\mc F\otimes\mc L^{\otimes n}\to\mc F\otimes k(p)\otimes\mc L^{\otimes n}\to0\]
		for any $n$, so because cohomology vanishes on the left for $n$ large enough, we get a surjection on global sections
		\[\Gamma\left(X,\mc F\otimes\mc L^{\otimes n}\right)\to\Gamma\left(X,\mc F\otimes\mc L^{\otimes n}\otimes k(p)\right).\]
		Notably, we are seeing that global sections are surjecting onto $\mc F\otimes\mc L^{\otimes n}\otimes k(p)=\mc F_p\otimes\mc L^n_p\otimes\OO_{X,p}/\mf m_p$, which by Nakayama's lemma tells us that it must lift to a surjection onto the stalk $\mc F_p\otimes\mc L_n^p$; in particular, we are looking over the local ring $\OO_{X,p}$ with residue field $\OO_{X,p}/\mf m_p$, so a basis will lift to a generating set. So we are indeed globally generated at closed points for $n$ large enough, say larger than $n_p$.

		Now, our fixed closed point $p$, we can pick up our global sections to build a surjection
		\[\OO_X^{\oplus k}\onto\mc F\otimes\mc L^{\oplus n},\]
		so the cokernel has vanishing stalk at $p$, so the cokernel must vanish in an open neighborhood of $p$, so this map is surjective on an open neighborhood $U_{p,n}$ of $p$ (for $n$ larger than $n_p$).

		We would like for $U_{p,n}$ to not depend on $n$. Taking $\mc F=\OO_X$, for each $p\in X$, we may find some $m_p$ and an open neighborhood $U_p$ so that $\mc L^{\otimes m_p}$ is generated by global sections on $U_p$. Refining this to a finite open cover and taking the product of all the relevant $m_\bullet$ produces some $m$ such that $\mc L^{\otimes m}$ is globally generated.\footnote{We are using the fact that the tensor product of globally generated sheaves remains globally generated. Indeed, if $\mc F$ and $\mc G$ are globally generated, then we get surjections $\OO^{\oplus n}\onto\mc F$ and $\OO^{\oplus m}\onto\mc G$, so we get a surjection $\OO^{\oplus(n+m)}\onto\mc F\otimes\mc G$.}

		We now apply our residue class trick again. For each $i\in\{0,1,\ldots,m-1\}$, and $p\in U$, we find an open neighborhood $U_{i}$ of $p$ upon which $\mc F\otimes\mc L^{\otimes(i+n_im)}$ is globally generated for some $n_i$. By taking the intersection of all these $U_i$, we produce an open neighborhood $U_p$ such that
		\[\mc F\otimes\mc L^{\otimes i}\]
		is globally generated for all $i$ large enough (say, larger than $N_p$), where the point is that multiplying through by the globally generated sheaf $\mc L^{\otimes m}$ is safe.

		To finish, we use the compactness trick one more time. Because $X$ is quasicompact, we only need to care about finitely many of the $U_p$ and $N_p$, so we can take the maximum of all the relevant $N_p$ producing an open cover to finish the proof.
		\qedhere
	\end{itemize}
\end{proof}

\subsection{Ext Sheaves}
We are going to want a few different kinds of cohomology before we continue.
\begin{definition}[Ext]
	Fix a ringed space $(X,\OO_X)$ and an $\mathcal O_X$-module $\mc F$.
	\begin{itemize}
		\item The functor $\mathrm{Hom}(\mc F,-)\colon\mathrm{Mod}_{\OO_X}\to\mathrm{Ab}$ is left-exact, so we define $\mathrm{Ext}^i(\mc F,-)$ to be the right-derived functors.
		\item The functor $\mathcal Hom(\mc F,-)\colon\mathrm{Mod}_{\OO_X}\to\mathrm{Mod}_{\OO_X}$ is left-exact, so we define $\mathcal Ext^i(\mc F,-)$ to be the right-derived functors.
	\end{itemize}
\end{definition}
Here are our usual basic facts.
\begin{lemma}
	Fix a ringed space $(X,\OO_X)$ and an $\mathcal O_X$-module $\mc F$.
	\begin{itemize}
		\item For any $\OO_X$-module $\mc G$, we have $\mathrm{Ext}^0(\mc F,\mc G)=\op{Hom}(\mc F,\mc G)$.
		\item For any $\OO_X$-module $\mc G$, we have $\mathcal Ext^0(\mc F,\mc G)=\mathcal Hom(\mc F,\mc G)$.
	\end{itemize}
\end{lemma}
\begin{proof}
	This follows directly from our discussion of derived functors.
\end{proof}
\begin{example}
	Because $\mathcal Hom(\OO_X,\mc G)=\mc G$ for any $\mc G$, we note that this functor is the identity and hence exact, so $\mathcal Ext^i(\OO_X,-)$ is the zero functor for $i>0$.
\end{example}
\begin{example}
	Note that $\mathrm{Hom}(\OO_X,\mc G)=\Gamma(X,\mc G)$ is global sections, so $\mathrm{Ext}^i(\OO_X,\mc G)=H^i(X,\mc G)$ for any $i$.
\end{example}
We are going to want the following abstract nonsense result.
\begin{lemma}
	Fix abelian categories $\mc A$ and $\mc B$. If the exact functor $f\colon\mc A\to\mc G$ is left adjoint to the functor $g\colon\mc G\to\mc F$, then $g$ preserves injectives.
\end{lemma}
\begin{proof}
	Fix an injective object $I\in\mc B$. We want to show that $g(I)$ is injective. Well, suppose we that we have a monomorphism $M\into N$ and a map $M\to g(I)$ so that we want to extend this map to $N\to g(I)$ as follows.
	% https://q.uiver.app/?q=WzAsMyxbMCwwLCJNIl0sWzEsMCwiTiJdLFsxLDEsImcoSSkiXSxbMCwyXSxbMCwxLCIiLDIseyJzdHlsZSI6eyJ0YWlsIjp7Im5hbWUiOiJob29rIiwic2lkZSI6InRvcCJ9fX1dLFsxLDIsIiIsMix7InN0eWxlIjp7ImJvZHkiOnsibmFtZSI6ImRhc2hlZCJ9fX1dXQ==&macro_url=https%3A%2F%2Fraw.githubusercontent.com%2FdFoiler%2Fnotes%2Fmaster%2Fnir.tex
	\[\begin{tikzcd}
		M & N \\
		& {g(I)}
		\arrow[from=1-1, to=2-2]
		\arrow[hook, from=1-1, to=1-2]
		\arrow[dashed, from=1-2, to=2-2]
	\end{tikzcd}\]
	Well, hitting this with the adjunction, we really want a morphism filling in the following triangle.
	% https://q.uiver.app/?q=WzAsMyxbMCwwLCJmKE0pIl0sWzEsMCwiZihOKSJdLFsxLDEsIkkiXSxbMCwyXSxbMCwxLCIiLDIseyJzdHlsZSI6eyJ0YWlsIjp7Im5hbWUiOiJob29rIiwic2lkZSI6InRvcCJ9fX1dLFsxLDIsIiIsMix7InN0eWxlIjp7ImJvZHkiOnsibmFtZSI6ImRhc2hlZCJ9fX1dXQ==&macro_url=https%3A%2F%2Fraw.githubusercontent.com%2FdFoiler%2Fnotes%2Fmaster%2Fnir.tex
	\[\begin{tikzcd}
		{f(M)} & {f(N)} \\
		& I
		\arrow[from=1-1, to=2-2]
		\arrow[hook, from=1-1, to=1-2]
		\arrow[dashed, from=1-2, to=2-2]
	\end{tikzcd}\]
	However, $I$ is now injective, so we can fill in this arrow: $f$ is exact, so the $f(M)\into f(N)$ remains injective. Reflecting back with the adjunction finishes.
\end{proof}
For our purposes, we will be interested in using the fact that $f^*$ and $f_*$ are adjoints here, and restriction to an open set is left adjoint to extension by $0$.
\begin{proposition}
	Fix a ringed space $(X,\OO_X)$. For an exact sequence of $\OO_X$-modules
	\[0\to\mc F''\to\mc F\to\mc F'\to0,\]
	any $\OO_X$-module $\mc G$ produces the long exact sequences as follows.
	\[\arraycolsep=1.4pt\begin{array}{ccccccccccccccc}
		0 &\to& \op{Hom}(\mc F'',\mc G) &\to& \op{Hom}(\mc F,\mc G) &\to& \op{Hom}(\mc F',\mc G) &\to& \op{Ext}^1(\mc F'',\mc G) &\to& \cdots \\
		0 &\to& \mathcal Hom(\mc F'',\mc G) &\to& \mathcal Hom(\mc F,\mc G) &\to& \mathcal Hom(\mc F',\mc G) &\to& \mathcal Ext^1(\mc F'',\mc G) &\to& \cdots \\
	\end{array}\]
\end{proposition}
\begin{proof}
	Provide $\mc G$ with an augmented injective resolution $0\to\mc G\to\mc I^\bullet$. Then the functor $\op{Hom}(-,\mc I)$ is exact for any injective $\mc I$---this is direct from the definition of injectivity---so we get the exact sequence of complexes
	\[0\to\op{Hom}(\mc F'',\mc I^\bullet)\to\op{Hom}(\mc F,\mc I^\bullet)\to\op{Hom}(\mc F',\mc I^\bullet)\to0.\]
	Taking the long exact sequence arising from cohomology here produces our first long exact sequence. For the sheaf $\mc Hom$, we see that we can check the exactness of
	\[0\to\mc Hom(\mc F'',\mc I^\bullet)\to\mc Hom(\mc F,\mc I^\bullet)\to\mc Hom(\mc F',\mc I^\bullet)\to0\]
	by restricting to some open set $U$ and evaluating at global sections (which allows us to check at stalks), where the point is that $\op{Hom}(-,\mc I|_U)$ remains exact because $\mc I|_U$ remains injective because restriction to an open subset is an exact left adjoint to extension by $0$.
\end{proof}

\end{document}