% !TEX root = ../notes.tex

\documentclass[../notes.tex]{subfiles}

\begin{document}

Today we'll continue our discussion of birational maps and then move on to divisors on curves.

\subsection{More on Birational Maps}
Last class we discussed birational maps. It will be helpful to have the following stronger notion.
\begin{definition}[birational]
	A morphism $f\colon X\to Y$ of integral separated schemes is \textit{birational} if and only if it is a birational map; i.e., it has a rational map as an inverse.
\end{definition}
\begin{example} \label{ex:affine-proj-birational}
	Fix an algebraically closed field $k$. Let $X\subseteq\AA^2_k\times\PP^1_k$ be the closed $k$-subvariety
	\[\left\{((x,y),[t:u])\in\AA^2_k\times\PP^1_k:xu=yt\right\}.\]
	(Formally, this is cut out by some equation.) Let $\varphi$ be the projection onto $\AA^2_k$. Notably, if $(x,y)\ne(0,0)$, then the fiber is the single point $((x,y),[x:y])$, but if $(x,y)=(0,0)$, then the fiber is a full $\PP^1_k$. Thus, $\varphi$ is a birational morphism: certainly it is a morphism, and its inverse can be represented by the rational map is given by $(x,y)\mapsto((x,y),[x:y])$, defined on $\AA^2_k\setminus\{(0,0)\}$.
\end{example}
% \begin{example}
% 	For an integer $n$, there is a birational morphism $\varphi\colon(\AA^2_k\setminus\{(0,0)\})\to\PP^1_k$ given by $\varphi(x,y)\coloneqq[x:y]$.
% \end{example}
\begin{definition}[graph]
	Given a morphism $f\colon X\to Y$ of $S$-schemes, the \textit{graph} $\Gamma_f$ of $f$ is the morphism $({\id_X},f)\colon X\to X\times_S Y$.
\end{definition}
\begin{example}
	If $f\colon\Spec A\to\Spec B$ is a morphism of affine $R$-schemes, then the graph is the map $\Spec A\to\Spec(A\otimes_R B)$ given by the map $A\otimes_R B\to A$ defined by $a\otimes b\mapsto a\cdot f^\sharp(b)$.
\end{example}
\begin{remark}
	If $f$ is separated, then $({\id_X},f)\colon X\to X\times Y$ is a closed embedding. Indeed, the square
	% https://q.uiver.app/#q=WzAsNCxbMCwwLCJYIl0sWzAsMSwiWSJdLFsxLDEsIllcXHRpbWVzX1NZIl0sWzEsMCwiWFxcdGltZXNfU1kiXSxbMCwxLCJmIiwyXSxbMSwyLCJcXERlbHRhX2YiXSxbMCwzLCJcXEdhbW1hX2YiXSxbMywyLCIoZix7XFxpZF9ZfSkiXV0=&macro_url=https%3A%2F%2Fraw.githubusercontent.com%2FdFoiler%2Fnotes%2Fmaster%2Fnir.tex
	\[\begin{tikzcd}
		X & {X\times_SY} \\
		Y & {Y\times_SY}
		\arrow["f"', from=1-1, to=2-1]
		\arrow["{\Delta_f}", from=2-1, to=2-2]
		\arrow["{\Gamma_f}", from=1-1, to=1-2]
		\arrow["{(f,{\id_Y})}", from=1-2, to=2-2]
	\end{tikzcd}\]
	is a pullback square, so $\Gamma_f$ is a closed embedding because $\Delta_f$ is.
\end{remark}
\begin{example}
	Continuing from \Cref{ex:affine-proj-birational}, define $U\coloneqq\AA^2_k\setminus\{(0,0)\}$ and then $f\colon U\to\PP^1_k$ given by $\varphi(x,y)\coloneqq[x:y]$. (Notably, $f$ is a rational map $\AA^2_k\to\AA^1_k$.) Then we set $\psi\coloneqq({\id_U},f)$ so that $\psi$ is the graph of $f$; notably, everything in sight is separated, so this graph is a closed embedding onto the closed subscheme $X\cap(U\times\PP^1_k)$ of $U\times\PP^1_k$. Notably, $X\subseteq\AA^2_k\times\PP^1_k$ is then a closed subset agreeing with this graph in a Zariski dense subset, so $X$ is the closure of the graph.
\end{example}
\begin{remark}
	Continuing, the example, we can view $f$ as an element of $K(\AA^2_k\times\PP^1_k)$, whereupon we may compute $f(x,y)=y/x$ has $\op{div}(f)=(y)-(x)$. More generally, for a scheme $X$, rational maps $X\to\AA^1$ correspond to rational functions on $X$ by passing to an open subset and using the adjunction
	\[\op{Mor}(X,\AA^1)\cong\op{Hom}(\ZZ[x],\Gamma(X,\OO_X))\cong\Gamma(X,\OO_X).\]
\end{remark}
We now note that birational maps do indeed (locally) behave like isomorphisms.
\begin{proposition}
	Fix a field $k$ and a birational map $\varphi\colon X\to Y$ of $k$-varieties. Then there are nonempty open subschemes $U\subseteq X$ and $V\subseteq Y$ such that $\varphi$ induces an isomorphism $\varphi\colon U\cong V$. Conversely, any isomorphism on nonempty open subschemes induces a birational map.
\end{proposition}
\begin{proof}
	The second claim just uses the given isomorphism on nonempty open subschemes to define the birational maps. For the first claim, if $\varphi$ has rational inverse given by $\psi$, then take some open subscheme $U\subseteq X$ where $\varphi$ is defined and some open subscheme $V\subseteq Y$ where $\psi$ is defined. Then $U\cap\varphi^{-1}(V)$ and $V\cap\psi^{-1}(U)$ should do the trick because we know that composing $\varphi$ and $\psi$ (when defined) will yield the identity.
\end{proof}
We now take a moment to note that rational maps are determined purely by the functions on the variety.
\begin{proposition}
	Fix a field $k$. A dominant rational maps $f\colon X\to Y$ of $k$-varieties functorially defines a $k$-algebra homomorphism $f^\sharp\colon K(Y)\to K(X)$ given by taking stalks at the generic points.
\end{proposition}
\begin{proof}
	All that remains to be checked is functoriality, which is true because $(f\circ g)^\sharp=g^\sharp\circ f^\sharp$ as maps of sheaves, and taking stalks is functorial.
\end{proof}
\begin{remark}
	In fact, we note that a rational function $g\in K(Y)$ thought of as a rational map $g\colon Y\to\AA^1$ will go to the induced rational map $(g\circ f)\colon X\to\AA^1$. This is a matter of tracking through the adjunction. Indeed, assuming that everything is defined over $U\subseteq X$ and $V\subseteq Y$, we merely need to note that the following diagram commutes.
	% https://q.uiver.app/#q=WzAsOCxbMSwwLCJcXEdhbW1hKFYsXFxPT19WKSJdLFsxLDEsIlxcR2FtbWEoVSxcXE9PX1UpIl0sWzAsMCwiXFxvcHtNb3J9KFYsXFxTcGVjIGtbeF0pIl0sWzAsMSwiXFxvcHtNb3J9KFUsXFxTcGVjIGtbeF0pIl0sWzIsMCwiZyJdLFszLDAsImdeXFxzaGFycCh4KSJdLFsyLDEsIihnXFxjaXJjIGYpIl0sWzMsMSwiKGdcXGNpcmMgZileXFxzaGFycCh4KSJdLFswLDEsImZeXFxzaGFycCIsMl0sWzIsMF0sWzMsMV0sWzIsMywiKC1cXGNpcmMgZikiLDJdLFs0LDUsIiIsMix7InN0eWxlIjp7InRhaWwiOnsibmFtZSI6Im1hcHMgdG8ifX19XSxbNiw3LCIiLDIseyJzdHlsZSI6eyJ0YWlsIjp7Im5hbWUiOiJtYXBzIHRvIn19fV0sWzQsNiwiIiwxLHsic3R5bGUiOnsidGFpbCI6eyJuYW1lIjoibWFwcyB0byJ9fX1dLFs1LDcsIiIsMSx7InN0eWxlIjp7InRhaWwiOnsibmFtZSI6Im1hcHMgdG8ifX19XV0=&macro_url=https%3A%2F%2Fraw.githubusercontent.com%2FdFoiler%2Fnotes%2Fmaster%2Fnir.tex
	\[\begin{tikzcd}
		{\op{Mor}(V,\Spec k[x])} & {\Gamma(V,\OO_V)} & g & {g^\sharp(x)} \\
		{\op{Mor}(U,\Spec k[x])} & {\Gamma(U,\OO_U)} & {(g\circ f)} & {(g\circ f)^\sharp(x)}
		\arrow["{f^\sharp}"', from=1-2, to=2-2]
		\arrow[from=1-1, to=1-2]
		\arrow[from=2-1, to=2-2]
		\arrow["{(-\circ f)}"', from=1-1, to=2-1]
		\arrow[maps to, from=1-3, to=1-4]
		\arrow[maps to, from=2-3, to=2-4]
		\arrow[maps to, from=1-3, to=2-3]
		\arrow[maps to, from=1-4, to=2-4]
	\end{tikzcd}\]
\end{remark}
\begin{theorem}
	Let $k$ be an algebraically closed field. The functor $X\mapsto K(X)$  from the category of $k$-varieties (equipped with dominant rational maps) to the category of $k$-algebras (equipped with $k$-algebra homomorphisms) is fully faithful.
\end{theorem}
\begin{proof}
	See \cite[Theorem~I.4.4]{hartshorne}.
\end{proof}

\subsection{Divisors on Curves}
Here is our definition.
\begin{definition}[curve]
	Fix a field $k$. A \textit{curve} over $k$ is a one-dimensional $k$-variety $X$. Then $X$ is a complete if and only if proper and nonsingular if and only if regular.
\end{definition}
Having properness (and being dimension $1$) allows us to use the valuative criterion to produce the following result.
\begin{proposition}
	Ant rational map $f\colon C\to X$ from a regular $k$-curve $C$ to a proper $k$-variety $X$ extends uniquely to a full morphism $C\to X$.
\end{proposition}
\begin{proof}
	This is exactly the valuative criterion for properness for curves.
\end{proof}

\end{document}