% !TEX root = ../notes.tex

\documentclass[../notes.tex]{subfiles}

\begin{document}

Today we continue our discussion of Cartier divisors.

\subsection{More on Cartier Divisors}
We begin with a continuation of \Cref{rem:concrete-cartier}.
\begin{remark} \label{rem:eq-for-cartier}
	The representation $\{f_U\}_{U\in\mc U}$ is far from unique. Namely, it is very possible that two different $\{f_U\}_{U\in\mc U}$ and $\{g_V\}_{V\in\mc V}$ give the same Cartier divisor: indeed, this merely asks that they are the same up to $\OO_X^\times$ after refining the open cover; i.e., we are asking for
	\[f_U/g_V\in\OO_X(U\cap V)^\times\]
	for any $U\in\mc U$ and $V\in\mc V$.
\end{remark}
As with Weil divisors, we will want a notion of effectivity.
\begin{defihelper}[effective] \nirindex{effective!Cartier divisor}
	A Cartier divisor $D$ on an integral scheme $X$ is \textit{effective} if and only if $D$ is represented by some $\{f_U\}_{U\in\mc U}$ where $f_U\in\OO_X(U)$.
\end{defihelper}
\begin{remark}
	Using \Cref{rem:eq-for-cartier}, we see if $\{g_V\}_{V\in\mc V}$ represents an effective Cartier divisor $\{f_U\}_{U\in\mc U}$ where $f_U\in\OO_X(U)$, then $g_V\in\OO_X(V)^\times$ for all $V$. Indeed, we are given that
	\[g_V/f_U\in\OO_X(U\cap V)^\times\]
	for any $U$ or $V$, meaning that $g_V\in\OO_X(U\cap V)$ for any $U$, so $g_V\in\OO_X(V)$ by gluing.
\end{remark}
\begin{defihelper}[support] \nirindex{support!Cartier divisor}
	Fix a Cartier divisor $D$ on an integral scheme $X$. If $\{f_U\}_{U\in\mc U}$, we say that the \textit{support} of $D$ is
	\[\op{supp}D\coloneqq\{x\in X:(f_U)_x\notin\OO_{X,x}^\times\text{ for some }U\text{ with }x\in U\}.\]
	By the coherence of the Cartier divisor and equality of Cartier divisors, we see that $x\in\op{supp}D$ if and only if $(g_V)_x\notin\OO_{X,x}^\times$ for any $x\in V$ for any $\{g_V\}_{V\in\mc V}$ representing $D$.
\end{defihelper}
We forgot to define what support means for Weil divisors, so here it is.
\begin{defihelper}[support] \nirindex{support!Weil divisor}
	Fix a Weil $D=\sum_{Y\in X^{(1)}}n_YY$ on a smoothish scheme $X$. Then the \textit{support} of $D$ is
	\[\op{supp}D\coloneqq\bigcup_{\substack{Y\in X^{(1)}\\n_Y\ne0}}Y.\]
	Note that this is a closed subset of $X$ of codimension $1$.
\end{defihelper}
We also have a notion of principal divisors and hence equivalence.
\begin{defihelper}[principal] \nirindex{principal!Cartier divisor}
	A Cartier divisor $D$ on an integral scheme $X$ is \textit{principal} if and only if it is represented by a single global rational function $f\in K(X)^\times$.
\end{defihelper}
\begin{defihelper}[linearly equivalent] \nirindex{linearly equivalent!Cartier divisor}
	We say that two Cartier divisors $D$ and $D'$ on an integral scheme $X$ are \textit{linearly equivalent} if and only if $D/D'$ is principal. The quotient $\op{CaDiv}(X)/K(X)^\times$ of linear equivalence classes of Cartier divisors is the \textit{Cartier divisor class group} $\op{CaCl}(X)$.
\end{defihelper}
\begin{remark}[Vojta]
	The notation $\op{CaCl}(X)$ is not to be confused with $\op{CaCl}_2$, which is calcium chloride.
\end{remark}
Now, one of the key benefits of Cartier divisors is that it is easier to describe how to pull them back.
\begin{definition}
	Fix a line bundle $\mc L$ on an integral scheme $X$, and let $s$ be a nonzero rational section of $\mc L$. Then for a trivializing open subset $U\subseteq X$ for $\mc L$, we choose an isomorphism $\varphi_U\colon\mc L|_U\to\OO_X|_U$, and we see that $\varphi_U(s)$ is a rational section of $\OO_X|_U$, which is an element $f_U$ of $K(X)$. Note that the class $f_U\in K(X)/\OO_X(U)^\times$ does not depend on the choice of $\varphi_U$ (either on the isomorphism or really on the open subset $U$) because automorphisms of $\OO_X(U)$ are given by $\OO_X(U)^\times$. Thus, we let $\{f_U\}_{U\in\mc U}$ define the \textit{Cartier divisor} $\op{div}_\mc L(s)$.
\end{definition}
\begin{remark}
	Tracking through the definition reveals that
	\[\op{div}_\mc L(s)\cdot\op{div}_{\mc L'}(s')=\op{div}_{\mc L\otimes\mc L'}(s\otimes s').\]
	In particular, taking $(\mc L',s')=(\mc L^\lor,s^\lor)$, we see that the output is a principal Cartier divisor.
\end{remark}
\begin{remark}
	If $s$ is a global section of $\mc L$, then we see that $\op{div}_\mc L(s)$ is an effective Cartier divisor, basically by construction.
\end{remark}

\subsection{Cartier Divisors to Weil Divisors}
We now describe a map $\op{CaDiv}X\to\op{Div}X$. In particular, we let $X$ denote an integral smoothish scheme defined by sending
\[D\mapsto\sum_{Y\in X^{(1)}}\nu_Y(f_U)Y_i,\]
where $\{f_U\}_{U\in\mc U}$ represents $D$, and we only take $\nu_Y(f_U)$ when $U\cap Y\ne\emp$; note that $\nu_Y(f_U)$ does not depend on the choice of $U$ because updating $f_U$ by a unit does not change the valuation.
\begin{remark} \label{rem:normal-means-compare-div-inj}
	If $X$ is in addition normal, then the above map is injective. Indeed, suppose that $D$ is in the kernel of this map and is represented by $\{f_U\}_{U\in\mc U}$. Then $f_U,f_U^{-1}\in\OO_X(U)^\times$ for all $U$ to have trivial valuation, but then this means that $D$ is the trivial Cartier divisor.
\end{remark}
Under additional smoothness hypotheses, we get the following.
\begin{proposition} \label{prop:compare-cartier-weil}
	Fix an integral, Noetherian, separated, and locally factorial scheme $X$. (Note that $X$ is then regular and hence smoothish.) Then we have the following.
	\begin{listalph}
		\item The map $\op{CaDiv}X\to\op{Div}X$ is an isomorphism.
		\item Fix a Cartier divisor $D$ with image $D'\in\op{Div}X$. Then $D$ is effective/principal if and only if $D'$ is.
		\item We have $\op{supp}D=\op{supp}D'$, where $D'\in\op{Div}X$ denotes the image of $D\in\op{CaDiv}X$.
	\end{listalph}
\end{proposition}
\begin{proof}
	We omit the proof of (c) (which is a matter of expanding the definitions).

	Let's show (a). \Cref{rem:normal-means-compare-div-inj} tells us that this map is injective (note locally factorial implies normal), so we really only need to show surjectivity. Fix some Weil divisor $D'$ we want to hit by a Cartier divisor in $\op{CaDiv}X$. The point is to look sufficiently locally to make the Weil divisor $D'$ ``locally principal,'' which is legal because $X$ is locally factorial. Explicitly, at each $x\in X$, we know that $D'$ restricts a divisor $D'_x$ on the stalk $\Spec\OO_{X,x}$, but $\OO_{X,x}$ is factorial, and $D'_x$ is height one, so $D'_x$ is principal, meaning we get $f_x\in K(X)^\times$ such that $\op{div}(f_x)=D'_x$.

	We now spread out from stalks. Note $D'-\op{div}(f_x)$ has support not containing $x$, so we can find an open neighborhood $U_x$ such that $D'|_U=\op{div}(f_x)|_U$. Thus, one sees that $D\coloneqq\{(U_x,f_x)\}_{x\in X}$ will assemble into a Cartier divisor (on overlaps $U_x\cap U_y$ we have $f_x/f_y\in\OO_X(U_x\cap U_y)$ because over in $\op{Div}X$ the divisor vanishes, so the injectivity of our map from \Cref{rem:normal-means-compare-div-inj} makes us okay), and $D$ by construction of the map $\op{CaDiv}X\to\op{Div}X$ goes to $D'$.

	We now turn our attention to (b). For the principal assertion, the point is that the diagram
	% https://q.uiver.app/#q=WzAsMyxbMCwwLCJLKFgpXlxcdGltZXMiXSxbMSwwLCJcXG9we0NhRGl2fVgiXSxbMSwxLCJcXG9we0Rpdn1YIl0sWzEsMl0sWzAsMV0sWzAsMl1d&macro_url=https%3A%2F%2Fraw.githubusercontent.com%2FdFoiler%2Fnotes%2Fmaster%2Fnir.tex
	\[\begin{tikzcd}
		{K(X)^\times} & {\op{CaDiv}X} \\
		& {\op{Div}X}
		\arrow[from=1-2, to=2-2]
		\arrow[from=1-1, to=1-2]
		\arrow[from=1-1, to=2-2]
	\end{tikzcd}\]
	commutes, so the set of principal Cartier divisors and Weil divisors correspond.

	We now address effectivity. If $D$ is effective, then of course $D'$ is effective by construction of the map: writing $D=\{f_U\}_{U\in\mc U}$, having $f_U\in\Gamma(U,\OO_U)$ means that all its valuations are nonnegative, so $D'$ is effective. In the other direction, if $D'$ is effective with $D=\{f_U\}_{U\in\mc U}$ where the $U\in\mc U$ are all affine. We know that $\nu_Y(f_U)\ge0$ whenever $Y$ intersects $U$ nontrivially, so algebraic Hartog's lemma tells us that $f_U\in\Gamma(U,\OO_U)$ for each $U$, yielding our effectivity of $D$. (Note we have used that $X$ is normal, which follows because $X$ is locally factorial.)
\end{proof}

\end{document}