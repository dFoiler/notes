% !TEX root = ../notes.tex

\documentclass[../notes.tex]{subfiles}

\begin{document}

Today we discuss duality for many projective schemes.

\subsection{Serre Duality}
We begin by providing our dualizing sheaves.
\begin{proposition} \label{prop:get-dualizing-sheaf}
	Fix a nonempty projective scheme $X$ over a field $k$, and set $n\coloneqq\dim X$. Then $X$ has a dualizing sheaf.
\end{proposition}
\begin{proof}
	Choose an embedding $X\subseteq\PP^N_k$ with $N$ very large, and let $r\coloneqq N-n$ be the codimension. Then we define
	\[\omega_X^\circ\coloneqq\mc Ext_{\PP^N_k}^r\left(\OO_X,\omega_{\PP^N_k}\right).\]
	Then \Cref{lem:check-dualizing-sheaf} shows
	\[\op{Hom}_X\left(\mc F,\omega_X^\circ\right)\cong\op{Ext}^r_{\PP^N_k}\left(\mc F,\omega_{\PP^N_k}\right)\]
	for any $\OO_X$-module $\mc F$. If $\mc F$ is coherent, then duality of $\omega_{\PP^N_k}$ on $\PP^N_k$ tells us that this is
	\[\op{Ext}^r_{\PP^N_k}\left(\mc F,\omega_{\PP^N_k}\right)\cong H^{N-r}(\PP^N_k,\mc F)=H^n(X,\mc F).\]
	All of these isomorphisms are functorial in $\mc F$, so we see that $\omega_X^\circ$ is a dualizing sheaf via \Cref{rem:yoneda-for-dualizing}.
\end{proof}
At long last, here is our duality theorem.
\begin{theorem}[Serre duality]
	Fix a nonempty projective scheme $X$ over a field $k$ with very ample sheaf $\OO_X(1)$, and set $n\coloneqq\dim X$. Let $\omega_X^\circ$ be a dualizing sheaf for $X$, which exists by \Cref{prop:get-dualizing-sheaf}.
	\begin{listalph}
		\item For all $i\ge0$, there are natural morphisms
		\[\theta^i\colon\op{Ext}^i_X(\mc F,\omega_X^\circ)\to H^{n-i}(X,\mc F)^\lor.\]
		\item The following are equivalent.
		\begin{listroman}
			\item $X$ is Cohen--Macaulay and equidimensional.
			\item For any locally free $\mc F$ on $X$ of finite rank, we have $H^i(X,\mc F(-q))=0$ for $i>n$ and $q$ sufficiently large.
			\item $H^i(X,\OO_X(-q))=0$ for all $i<n$ and $q$ sufficiently large.
			\item The maps $\theta^\bullet$ defined above are isomorphisms for all $i$ and $\mc F$.
		\end{listroman}
	\end{listalph}
\end{theorem}
\begin{proof}
	Here we go.
	\begin{listalph}
		\item It is enough to show that $\op{Ext}^\bullet(-,\omega_X^\circ)$ is coeffaceable and hence a universal contravariant $\delta$-functor because this makes the morphism at $t=0$ extend uniquely to a morphism of $\delta$-functors.
		
		As usual, we note that any coherent $\mc F$ has some vector bundle $\mc E\coloneqq\bigoplus_i\OO_X(-q_i)$ for $q_\bullet$ sufficiently large such that there is a surjection $\mc E\onto\mc F$. Then we see that
		\[\op{Ext}^\bullet_X(\mc E,\omega_X^\circ)=\bigoplus_i\op{Ext}^\bullet_X(\OO_X,\omega_X^\circ(q_i))=H^\bullet(X,\omega_X^\circ(q_i))\]
		vanishes for $q_\bullet$ sufficiently large (even for all indices), which completes our check.

		\item We show the implications one at a time.
		\begin{itemize}
			\item Note (ii) implies (iii) with no content.
			\item We check that (iii) implies (iv). By the discussion above utilizing universal $\delta$-functors, it is enough to show that $H^{n-\bullet}(X,-)^\lor$ is also coeffaceable and hence universal. Well, choose any coherent $\mc F$ and then select $\mc E$ as above. Then (iii) tells us that
			\[H^{n-\bullet}(X,\mc E)^\lor=\bigoplus_iH^{n-\bullet}(X,\OO_X(-q_i))\]
			vanishes for all $i>0$ and $q_\bullet$ sufficiently large, which completes our check. So universality makes the morphisms $\theta^\bullet$ into isomorphisms.
			\item We check that (iv) implies (ii). Well, we just directly compute
			\[H^\bullet(X,\mc F(-q))\stackrel\theta\cong\op{Ext}^{n-\bullet}_X(\mc F(-q),\omega_X^\circ)^\lor\cong\op{Ext}^{n-\bullet}_X(\OO_X,\omega_X^\circ\otimes\mc F^\lor(q))^\lor\cong H^{n-\bullet}(X,\omega_X^\circ\otimes\mc F^\lor(q))^\lor,\]
			which vanishes for large $q$ by \Cref{thm:coh-cohom-proj-space}.
		\end{itemize}
		We won't bother to prove that (i) and (ii) are equivalent because we don't want to define Cohen--Macaulay.
		\qedhere
	\end{listalph}
\end{proof}
\begin{remark}
	If $X$ is regular (for example, if $X$ is smooth), then $X$ is Cohen--Macaulay.
\end{remark}
Here, then, is the ``headline'' result.
\begin{corollary}
	Fix a nonempty projective scheme $X$ over a field $k$ with very ample sheaf $\OO_X(1)$, and set $n\coloneqq\dim X$. Let $\omega_X^\circ$ be a dualizing sheaf for $X$, which exists by \Cref{prop:get-dualizing-sheaf}. If $X$ is Cohen--Macaulay and equidimensional, then any locally free coherent sheaf $\mc F$ has
	\[H^\bullet(X,\mc F)\cong H^{n-\bullet}(X,\mc F^\lor\otimes\omega_X^\circ)^\lor.\]
\end{corollary}
\begin{proof}
	As before, we compute
	\[H^\bullet(X,\mc F)\stackrel\theta\cong\op{Ext}^{n-\bullet}_X(\mc F,\omega_X^\circ)^\lor\cong\op{Ext}^{n-\bullet}_X(\OO_X,\omega_X^\circ\otimes\mc F^\lor)^\lor\cong H^{n-\bullet}(X,\omega_X^\circ\otimes\mc F^\lor)^\lor,\]
	so we are done.
\end{proof}

\subsection{The Koszul Complex}
We will want the Koszul complex in order to actually compute $\omega_X^\circ$ in some cases.
\begin{definition}[Koszul complex]
	Fix a ring $A$ and some elements $f_1,\ldots,f_r\in A$. Then the \textit{Koszul complex} $K_\bullet(f_1,\ldots,f_r)$ is defined by
	\[K_p(f_1,\ldots,f_r)\coloneqq\land^pA^r\]
	on objects. To define the differential, we let $\{e_1,\ldots,e_r\}$ denote the standard basis on $K_1=A^r$ and then define $d\colon K_p\to K_{p-1}$ by
	\[d(e_{i_1}\land\cdots\land e_{i_p})\coloneqq\sum_{j=1}^p(-q)^{j-1}f_{i_j}(e_{i_1}\land\cdots\land\widehat e_{i_j}\land\cdots\land e_{i_p}).\]
\end{definition}

\end{document}