% !TEX root = ../notes.tex

\documentclass[../notes.tex]{subfiles}

\begin{document}

\section{April 21}

Today we continue moving towards smoothness, which we will continue to do until the end of the semester.

\subsection{Smooth Morphisms}
Fix a morphism $f\colon X\to S$ locally of finite presentation so that we can factor $f$ as
\[X\stackrel j\to\AA^n_S\to S\]
where $j\colon X\to\AA^n_S$ is some locally closed embedding. Working affine-locally, we may suppose that, in the neighborhood of $j(x)\in\AA^n_S$ for some $x\in X$, the scheme $X$ is cut out by some $n-r$ equations $g_{r+1},\ldots,g_n$ in $\OO_{\AA^n_S}$. Further, suppose that
\[dg_{r+1}(j(x)),\ldots,dg_n(j(x))\]
are linearly independent over the $k(j(x))$-vector space $\Omega^1_{\AA^n_S/S}\otimes k(j(x))$. We now claim that $f^{-1}(f(x))$ is equidimensional with dimension $r$.

For this, require the following lemma.
\begin{lemma}
	Fix a local ring $(A,\mf m)$. Given elements $f_1,\ldots,f_i\in\mf m$ which generate an $i$-dimensional subspace in the $(A/\mf m)$-vector space $\mf m/\mf m^2$. Then $A/(f_1,\ldots,f_i)$ is a regular local ring of dimension $\dim A-i$.
\end{lemma}
\begin{proof}
	The point is that $(f_1,\ldots,f_i)$ produces a regular sequence, so modding out by a regular sequence produces for us a regular local ring with the required dimension.
\end{proof}
To continue our story, recall that any locally closed embedding of finite presentation is going to be unramified; indeed, one can check this by (d) in \Cref{prop:unramified-grab-bag}. However, closed embeddings are not flat, so for flatness we add a new word.
\begin{definition}[\'etale]
	A scheme morphism $f\colon X\to S$ locally of finite presentation is \textit{\'etale} if and only if it is flat and unramified.
\end{definition}
Notably, closed embeddings tend not to be flat (the fibers are far from continuous). Explicitly, one can check that \'etale locally closed embeddings are just open embeddings.
\begin{remark}
	Finite \'etale maps are open and closed (because proper), so finite \'etale maps are able to play the role of covering spaces in algebraic geometry. Namely, this allows us to discuss the \'etale fundamental group via these maps.
\end{remark}
We can now define smoothness and explain how it relates to our existing notions.
\begin{definition}[smooth]
	Fix a morphism $f\colon X\to S$ locally of finite presentation. Then $f$ is \textit{smooth at $x\in X$ of relative dimension $r$} if and only if there is an open neighborhood $U$ of $x$ and a factorization of $f|_U$ by
	\[U\stackrel j\to\AA^n_S\to S\]
	satisfying the following constraints.
	\begin{itemize}
		\item For each $x\in X$, the sheaf of ideals $\mc I$ defining the locally closed embedding $j\colon U\to\AA^n_S$ is generated by $(n-r)$ sections $g_{r+1},\ldots,g_n\in\OO_{\AA^n_S}$.
		\item The differentials $dg_{r+1}(j(x)),\ldots,dg_n(j(x))$ in the $k(j(x))$-vector space $\Omega_{\AA^n_S/S}\otimes k(j(x))$ are linearly independent.
	\end{itemize}
	The full morphism $f$ is \textit{smooth} if and only if it is smooth at all points in $X$.
\end{definition}
\begin{remark}
	It turns out that $f$ is \'etale if and only if $f$ is smooth of relative dimension $0$. We will not prove this here.
\end{remark}
\begin{remark}
	We will shortly see that smooth morphisms are flat and hence open.
\end{remark}
\begin{remark}
	One can show that a morphism is smooth if and only if it is flat and has smooth fibers; this is nice because it is easier to check smoothness at fibers.
\end{remark}
\begin{remark}
	As with flatness, the smooth locus is open. Similar holds for being unramified and for being \'etale.
\end{remark}
\begin{remark}
	One can show that a smooth morphism of relative dimension $r$ (at all points) has relative dimension $r$.
\end{remark}
Let's pick up some facts.
\begin{proposition}
	Fix a smooth morphism $f\colon X\to Y$ of $S$-schemes.
	\begin{listalph}
		\item $\Omega_{X/Y}$ is locally free with rank equal to the relative dimension.
		\item There is an exact sequence
		\[0\to f^*\Omega_{Y/S}\to\Omega_{X/S}\to\Omega_{X/Y}\to0\]
		is exact and locally split (namely, on affine open subschemes).
	\end{listalph}
\end{proposition}
\begin{proof}
	Here we go.
	\begin{listalph}
		\item We may assume that $Y$ and then $X$ is affine. Now, choose a factorization $U\to\AA^n_Y\to Y$ promised by smoothness, where $j\colon U\to\AA^n_Y$ has the required properties. Let $X\subseteq\AA^n_Y$ be locally cut out by the sheaf of ideals $\mc I$. We now note that we have the right-exact sequence
		\[\mc I/\mc I^2\to j^*\Omega_{\AA^n_Y/Y}\to\Omega_{X/Y}\to 0.\]
		Now, by hypothesis, $\mc I$ is generated by $n-r$ sections where $r$ is the relative dimension, which we label as $g_{r+1},\ldots,g_n$. Further, we note that $\Omega_{\AA^n_Y/Y}$ is free of rank $n$ over $Y$, so its pullback to $X$ along $j$ is as well.

		We now note that the $dg_{r+1}(j(x)),\ldots,dg_n(j(x))$ are linearly independent in $\Omega_{\AA^n_Y/Y}\otimes k(j(x))$, so we can extend this to a basis $m_1,\ldots,m_r,\ldots,dg_{r+1},\ldots,dg_r$ looking in the stalk $(\Omega_{\AA^n_Y/Y})_{j(x)}$. By Nakayama's lemma, these germs generate, but $\Omega_{\AA^n_Y/Y}$ is free of rank $n$, so the germs $m_1,\ldots,m_r,\ldots,dg_{r+1},\ldots,dg_r$ are actually a basis of $(\Omega_{\AA^n_Y/Y})_{j(x)}$.

		Now, extend all our germs to be in some open neighborhood $V$ of $j(x)$. Now, taking the $m_i$ to $\Omega_{X/Y}|_V$, we see that they generate (because we have a surjection), and they generate with no relations from $V$ by the construction of the $m_i$, so we conclude that $\Omega_{X/Y}|_V$ is in fact free of rank $r$. In fact, we have seen that
		\[0\to\mc I/\mc I^2\to j^*\Omega_{\AA^n_Y/Y}\to\Omega_{X/Y}\to0\]
		is a locally split exact sequence of locally free modules, of ranks $n-r$, $n$, and $r$, respectively.

		\item We have right-exactness in general, so it remains to discuss left-exactness. From part (a), everything involved is locally free of known dimensions (by computing relative dimensions), and we can compute ranks to get our injectivity by checking at stalks and using Nakayama's lemma.

		Explicitly, denote $\alpha\colon f^*\Omega_{Y/S}\to\Omega_{X/S}$ by the left map. Now, for each $x\in X$, we see $\left(f^*\omega_{Y/S}\right)_x$ that embeds into $\ker\alpha_x\oplus\im\alpha_x$ because $(\Omega_{X/S})_x$ embeds into $\im\alpha_x\oplus(\Omega_{X/Y})_x$, so computing ranks enforces $\left(f^*\omega_{Y/S}\right)_x=\ker\alpha_x$.
		\qedhere
	\end{listalph}
\end{proof}

\end{document}