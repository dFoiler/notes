% !TEX root = ../notes.tex

\documentclass[../notes.tex]{subfiles}

\begin{document}

\section{April 5}

Today we compute the dualizing sheaf.

\subsection{Some Cohen--Macaulay Facts}
We pick up a few more facts about Cohen--Macaulay rings whose proofs we omit.
\begin{lemma} \label{lem:reg-by-dim}
	Fix a Noetherian Cohen--Macaulay local ring $A$. Then a sequence $x_1,\ldots,x_r\in A$ forms a regular sequence if and only if
	\[\dim\frac A{(x_1,\ldots,x_r)}=\dim A-r.\]
\end{lemma}
\begin{lemma} \label{lem:divide-by-reg-is-cm}
	Fix a Noetherian Cohen--Macaulay local ring $A$. Then any regular sequence $x_1,\ldots,x_r$ of elements in $A$ yields a Cohen--Macaulay ring $A/(x_1,\ldots,x_r)$.
\end{lemma}
\begin{lemma} \label{lem:reg-quotient-is-free}
	Fix a Noetherian Cohen--Macaulay local ring $A$. Given a regular sequence $x_1,\ldots,x_r$, set $I\coloneqq(x_1,\ldots,x_r)$. Then $I/I^2$ is a free $(A/I)$-module of rank $r$.
\end{lemma}
Today we will also want the following definition.
\begin{definition}[local complete intersection]
	Fix a closed subscheme $Y\subseteq X$, where $X$ is a nonsingular $k$-variety. Then $Y$ is a \textit{local complete intersection} if and only if the ideal sheaf $\mc I_Y$ of $Y$ is locally generated by $\codim_XY$ elements.
\end{definition}
\begin{remark}
	By checking locally at each point, we see that local complete intersections are equidimensional. Namely, the dimension must be fully equal to $\dim Y$ at each closed point.
\end{remark}
Note that the Principal ideal theorem says that $\mc I_Y$ cannot be locally generated by any fewer than $\codim_XY$ elements.
\begin{proposition}
	Fix a local complete intersection $Y$ of a nonsingular $k$-variety $X$. Then $Y$ is Cohen--Macaulay.
\end{proposition}
\begin{proof}
	At each point $p\in X$, the local ring $\OO_{X,p}$ is regular local. Then $\OO_{Y,p}=\OO_{X,p}/\mc I_{Y,p}$ for each $p\in Y$, so because $Y$ is a local complete intersection, we note that the generators of $\mc I_{Y,p}$ must form a regular sequence in $\OO_{X,p}$ by \Cref{lem:reg-by-dim}, so the quotient is Cohen--Macaulay by \Cref{lem:divide-by-reg-is-cm}.
\end{proof}

\subsection{Computing the Dualizing Sheaf}
Before continuing, here's a quick corollary of Serre duality.
\begin{corollary}
	Fix a Cohen--Macaulay projective $k$-scheme $X$ of equidimension $n$. For any locally free sheaf $\mc F$ of finite rank on $X$, we have
	\[H^i(X,\mc F)\cong H^{n-i}\left(\mc F^\lor\otimes\omega^\circ_X\right),\]
	where $\omega^\circ_X$ is the dualizing sheaf.
\end{corollary}
\begin{proof}
	By \Cref{thm:serre-duality}, observe that
	\[H^i(X,\mc F)=\op{Ext}^{n-i}_X\left(\mc F,\omega^\circ_X\right)^\lor=\op{Ext}^{n-i}_X\left(\OO_X,\mc F^\lor\otimes\omega^\circ_X\right)^\lor=H^{n-i}\left(\mc F^\lor\otimes\omega^\circ_X\right)^\lor,\]
	which is what we wanted.
\end{proof}
This is a surprise tool which will help us later. We now work with nonsingular projective varieties.
\begin{proposition} \label{prop:loc-comp-intersection-dualizing-sheaf}
	Fix a closed subscheme $X$ of $\PP^N_k$ which is a local complete intersection of codimension $r$; let $\mc I$ be the ideal sheaf of $X$. Then
	\[\omega^\circ_X\cong\omega_\PP\otimes\left(\bigland^r\mc I/\mc I^2\right)^\lor.\]
	Thus, $\omega^\circ_X$ is an invertible sheaf on $X$.
\end{proposition}
\begin{proof}
	Note that (locally) $\mc I/\mc I^2$ is free of rank $r$ by \Cref{lem:reg-quotient-is-free}, so the right-hand side of the desired expression is in fact locally free of rank $1$. So the last sentence does follow from the desired isomorphism.

	It remains to compute $\omega^\circ_X$. Recall that this is $\mathcal Ext^r(\OO_X,\omega_\PP)=R^ri^!\omega_\PP$ by definition. Now, to compute $\mathcal Ext$, we would like a locally free resolution of $\OO_X$, for which we use the Koszul complex. To be explicit, because $\mc I$ is locally generated by $r$ elements, so for any point $p\in X$, we may find $r$ germs $f_1,\ldots,f_r$ generating locally; taking the intersection of the open sets upon which the $f_\bullet$ are defined, we get to work with an affine open subscheme $U$.
	
	Thus, setting $A\coloneqq\Gamma(U,\OO_U)$ and $I\coloneqq\Gamma(U,\mc I)$, we let $\varepsilon\colon J\to A$ (where $J\coloneqq A^{\oplus r}$) be the map given by $(f_1,\ldots,f_r)$ so that we have the Koszul complex
	\[0\to J^{\land r}\stackrel d\to\cdots J^{\land2}\stackrel d\to J^{\land 1}\stackrel\varepsilon\to J^{\land 0}\to A/I\to0.\]
	To be explicit, the maps $d$ are given by
	\[d(j_1\land\cdots\land j_k)\coloneqq\sum_{i=1}^k(-1)^{i+1}\varepsilon(j_i)(j_1\land\cdots\land\widehat{j_i}\land\cdots\land j_k).\]
	On the homework, we showed that this is exact when the sequence $f_1,\ldots,f_r$ is regular, the above sequence is exact; for example, this will be exact when localized at our point $p$. Because localization commutes with taking homology, we see that the Koszul homology is killed upon localizing at $p$; however, we can instead localize by precisely the elements needed to kill homology (which exist because they are in the denominators of $A_p$) and then replace $U=\Spec A$ by the distinguished affine open subscheme given by localizing these elements. In total, by shrinking $U$ a little further, we may assume that our Koszul complex is fully exact.

	Now, we can take $\widetilde\cdot$ everywhere to produce a locally free resolution $\widetilde K_\bullet$ of $\OO_X|_U$, so we see
	\[\mathcal Ext^r_U(\OO_X,\omega_\PP)=H^r\mathcal Hom(\widetilde K_\bullet,\omega_\PP)=H^r\left(\omega_\PP\otimes\widetilde K_\bullet^\lor\right).\]
	We now have to actually compute what these maps are given by. Writing everything out at degree $r$, we are looking at the following.
	% https://q.uiver.app/?q=WzAsNixbMCwwLCJcXG9tZWdhX1xcUFBcXG90aW1lc1xcd2lkZXRpbGRle1xcbGVmdChcXGxhbmRee3ItMX1KXFxyaWdodCl9XlxcbG9yIl0sWzEsMCwiXFxvbWVnYV9cXFBQXFxvdGltZXNcXHdpZGV0aWxkZXtcXGxlZnQoXFxsYW5kXntyfUpcXHJpZ2h0KX1eXFxsb3IiXSxbMCwxLCJcXGRpc3BsYXlzdHlsZVxcYmlnb3BsdXNfe2k9MX1eclxcb21lZ2FfXFxQUCJdLFsxLDEsIlxcb21lZ2FfXFxQUCJdLFsyLDAsIjAiXSxbMiwxLCIwIl0sWzAsMV0sWzEsNF0sWzMsNV0sWzIsM10sWzEsMywiIiwxLHsibGV2ZWwiOjIsInN0eWxlIjp7ImhlYWQiOnsibmFtZSI6Im5vbmUifX19XSxbMCwyLCIiLDEseyJsZXZlbCI6Miwic3R5bGUiOnsiaGVhZCI6eyJuYW1lIjoibm9uZSJ9fX1dXQ==&macro_url=https%3A%2F%2Fraw.githubusercontent.com%2FdFoiler%2Fnotes%2Fmaster%2Fnir.tex
	\[\begin{tikzcd}
		{\omega_\PP\otimes\widetilde{\left(\land^{r-1}J\right)}^\lor} & {\omega_\PP\otimes\widetilde{\left(\land^{r}J\right)}^\lor} & 0 \\
		{\displaystyle\bigoplus_{i=1}^r\omega_\PP} & {\omega_\PP} & 0
		\arrow[from=1-1, to=1-2]
		\arrow[from=1-2, to=1-3]
		\arrow[from=2-2, to=2-3]
		\arrow[from=2-1, to=2-2]
		\arrow[Rightarrow, no head, from=1-2, to=2-2]
		\arrow[Rightarrow, no head, from=1-1, to=2-1]
	\end{tikzcd}\]
	We can compute out what this dualized map does, but the point is that it will send the $i$th copy of $\omega_\PP$ to $f_i\omega_\PP$, so our cohomology here is given by $\omega_\PP/(f_1,\ldots,f_r)\omega_\PP=\omega_\PP\otimes\OO_X|_U$, as  desired.

	We would like to glue this local computation into a global computation, but this is a little tricky. Namely, the concern is that we made a choice in choosing the $f_i$, so we need to consider what happens when we choose some different $f_i$ given by $f_i'$. Then we have some transition function $T\colon J\to J$ sending the basis of the $f_i$ to the basis of the $f_i'$. Notably, taking determinants produces the map $(\det T)\colon J^{\land r}\to J^{\land r}$ corresponding to the determinant of our change-of-basis matrix. In particular, we can fix our non-canonical isomorphism
	\[\mathcal Ext^r_U(\OO_X,\omega_\PP)\cong\omega_\PP\otimes\OO_X|_U\]
	into a more canonical isomorphism
	\[\mathcal Ext^r_U(\OO_X,\omega_\PP)\cong\left(\omega_\PP\otimes\OO_X|_U\right)\otimes\left(\left(\mc I/\mc I^2\right)^{\land r}\right)^\lor|_U.\]
	Namely, over $U$, the change of coordinates arising from $\OO_X$ will cause $\OO_X$ to change by $\det T$, but then the isomorphism $\land^r\mc I/\mc I^2\cong\OO_X$ will end up changing by exactly $\det T$ as well, so dualizing fixes this change. Thus, we do have a canonical isomorphism as stated.
\end{proof}
\begin{corollary}
	Fix an integral projective nonsingular $k$-variety $X$. Then $\omega_X^\circ=\omega_X$.
\end{corollary}
\begin{proof}
	Because $X$ is nonsingular, it is regular and hence a local complete intersection. Thus, fixing an embedding $i\colon X\into\PP^N_k$ giving $X$ codimension $r$, we see
	\[\omega^\circ_X=\omega_\PP\otimes\left(\bigland^r\mc I/\mc I^2\right)^\lor.\]
	However, this is indeed $\omega_X$. Namely, being smooth grants us the short exact sequence
	\[0\to\mc I/\mc I^2\to\Omega_\PP\otimes\OO_X\to\Omega_X\to0.\]
	Then we can take determinants everywhere to produce
	\[\underbrace{\Omega_X^{\land(N-r)}}_{\omega_X}\otimes\left(\mc I/\mc I^2\right)^{\land r}\cong\underbrace{\Omega_\PP^{\land N}}_{\omega_\PP},\]
	which is what we wanted after combining with \Cref{prop:loc-comp-intersection-dualizing-sheaf}.
\end{proof}

\end{document}