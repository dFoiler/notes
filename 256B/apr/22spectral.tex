% !TEX root = ../notes.tex

\documentclass[../notes.tex]{subfiles}

\begin{document}

We now switch gears and start talking about spectral sequences.

\subsection{Spectral Sequences}
Let $\mc A$ be an abelian category; for example, one does not lose much by taking $\mc A=\mathrm{Ab}$. Here is our definition.
\begin{definition}[spectral sequence]
	Fix an abelian category $\mc A$. A \textit{spectral sequence} is a sequence of pairs $\{(E_r,d_r)\}_{r\ge0}$ of bi-graded objects $E_r\coloneqq\bigoplus_{p,q\in\NN}E_r^{p,q}$ equipped with ``differential'' homomorphisms $d_r\colon E_r^{p,q}\to E_r^{p+q,q-r+1}$ such that the following hold.
	\begin{listalph}
		\item $d_r^2=0$.
		\item $H(E_r)=E_{r+1}$. In other words,
		\[E_r^{pq}=\frac{\ker\left(d_r\colon E_r^{p,q}\to E_r^{p+r,q-r+1}\right)}{\im\left( d_r\colon E_r^{p-r,q+r-1}\to E_r^{p,q}\right)}.\]
	\end{listalph}
\end{definition}
Visually, here are the morphisms $d_0$.
% https://q.uiver.app/#q=WzAsMTIsWzAsMCwiKDAsMikiXSxbMCwxLCIoMCwxKSJdLFswLDIsIigwLDApIl0sWzEsMiwiKDEsMCkiXSxbMiwyLCIoMiwwKSJdLFszLDIsIigzLDApIl0sWzEsMSwiKDEsMSkiXSxbMiwxLCIoMiwxKSJdLFszLDEsIigzLDEpIl0sWzMsMCwiKDMsMikiXSxbMiwwLCIoMiwyKSJdLFsxLDAsIigxLDIpIl0sWzIsMV0sWzMsNl0sWzEsMF0sWzYsMTFdLFs0LDddLFs3LDEwXSxbNSw4XSxbOCw5XV0=&macro_url=https%3A%2F%2Fraw.githubusercontent.com%2FdFoiler%2Fnotes%2Fmaster%2Fnir.tex
\[\begin{tikzcd}
	{(0,2)} & {(1,2)} & {(2,2)} & {(3,2)} \\
	{(0,1)} & {(1,1)} & {(2,1)} & {(3,1)} \\
	{(0,0)} & {(1,0)} & {(2,0)} & {(3,0)}
	\arrow[from=2-1, to=1-1]
	\arrow[from=2-2, to=1-2]
	\arrow[from=2-3, to=1-3]
	\arrow[from=2-4, to=1-4]
	\arrow[from=3-1, to=2-1]
	\arrow[from=3-2, to=2-2]
	\arrow[from=3-3, to=2-3]
	\arrow[from=3-4, to=2-4]
\end{tikzcd}\]
Here are the morphisms $d_1$.
% https://q.uiver.app/#q=WzAsMTIsWzAsMCwiKDAsMikiXSxbMCwxLCIoMCwxKSJdLFswLDIsIigwLDApIl0sWzEsMiwiKDEsMCkiXSxbMiwyLCIoMiwwKSJdLFszLDIsIigzLDApIl0sWzEsMSwiKDEsMSkiXSxbMiwxLCIoMiwxKSJdLFszLDEsIigzLDEpIl0sWzMsMCwiKDMsMikiXSxbMiwwLCIoMiwyKSJdLFsxLDAsIigxLDIpIl0sWzIsM10sWzMsNF0sWzQsNV0sWzEsNl0sWzYsN10sWzcsOF0sWzAsMTFdLFsxMSwxMF0sWzEwLDldXQ==&macro_url=https%3A%2F%2Fraw.githubusercontent.com%2FdFoiler%2Fnotes%2Fmaster%2Fnir.tex
\[\begin{tikzcd}
	{(0,2)} & {(1,2)} & {(2,2)} & {(3,2)} \\
	{(0,1)} & {(1,1)} & {(2,1)} & {(3,1)} \\
	{(0,0)} & {(1,0)} & {(2,0)} & {(3,0)}
	\arrow[from=1-1, to=1-2]
	\arrow[from=1-2, to=1-3]
	\arrow[from=1-3, to=1-4]
	\arrow[from=2-1, to=2-2]
	\arrow[from=2-2, to=2-3]
	\arrow[from=2-3, to=2-4]
	\arrow[from=3-1, to=3-2]
	\arrow[from=3-2, to=3-3]
	\arrow[from=3-3, to=3-4]
\end{tikzcd}\]
Here are the morphisms $d_2$.
% https://q.uiver.app/#q=WzAsMTQsWzAsMCwiKDAsMikiXSxbMCwxLCIoMCwxKSJdLFswLDIsIigwLDApIl0sWzEsMiwiKDEsMCkiXSxbMiwyLCIoMiwwKSJdLFszLDIsIigzLDApIl0sWzEsMSwiKDEsMSkiXSxbMiwxLCIoMiwxKSJdLFszLDEsIigzLDEpIl0sWzMsMCwiKDMsMikiXSxbMiwwLCIoMiwyKSJdLFsxLDAsIigxLDIpIl0sWzQsMSwiKDQsMSkiXSxbNCwyLCIoNCwyKSJdLFswLDddLFsxMSw4XSxbMSw0XSxbNiw1XSxbMTAsMTJdLFs3LDEzXV0=&macro_url=https%3A%2F%2Fraw.githubusercontent.com%2FdFoiler%2Fnotes%2Fmaster%2Fnir.tex
\[\begin{tikzcd}
	{(0,2)} & {(1,2)} & {(2,2)} & {(3,2)} \\
	{(0,1)} & {(1,1)} & {(2,1)} & {(3,1)} & {(4,1)} \\
	{(0,0)} & {(1,0)} & {(2,0)} & {(3,0)} & {(4,2)}
	\arrow[from=1-1, to=2-3]
	\arrow[from=1-2, to=2-4]
	\arrow[from=1-3, to=2-5]
	\arrow[from=2-1, to=3-3]
	\arrow[from=2-2, to=3-4]
	\arrow[from=2-3, to=3-5]
\end{tikzcd}\]
One can imagine how this continues.
\begin{remark}
	The ``antidiagonal'' of terms with $n=p+q$ for fixed $n$ will be quite important.
\end{remark}
\begin{remark}
	Set $n\coloneqq p=q$. If $r>n+1$, then $q-r+1<0$ and $p-r<0$, so $d_r^{p,q}=d_r^{p-r,q+r-1}=0$ because these morphisms are ``outside'' of our $E_r^{\bullet,\bullet}$. Thus, $E_{r+1}^{p,q}=E_r^{p,q}$, allowing us to define
	\[E_\infty^{p,q}\coloneqq E_{n+1}^{p,q}=E_{n+2}^{p,q}=\cdots.\]
\end{remark}
The previous remark suggests that we will be able to define later $E_r$s using prior ones by induction, but we note that $d_0$ and $d_1$ move in orthogonal directions, so there is basically no hope of beginning this induction at the $E_0$ stage. However, this turns out to be the only obstruction.
\begin{definition}[double complex]
	Fix an abelian category $\mc A$. A \textit{double complex} is a bi-graded object $K^{\bullet,\bullet}\coloneqq\bigoplus_{p,q\in\NN}K^{p,q}$ equipped with differentials $d^{p,q}\colon K^{p,q}\to K^{p,q+1}$ and $\delta^{p,q}\colon K^{p,q}\to K^{p+1,q}$ such that $d\circ d=0$ and $\delta\circ\delta=0$ and $d\circ\delta=-\delta\circ d$. (This last condition means that $d$ and $\delta$ ``anti-commute.'')
\end{definition}
\begin{theorem}
	A double complex $(K,d,\delta)$ functorially determines a spectral sequence $\{(E_r,d_r)\}_{r\ge0}$ such that $E_0=K$, $d_0=d$, and $d_1$ is induced by $\delta$ (and the later differentials are also uniquely defined).
\end{theorem}
We will prove this later by working in slightly more generality.
\begin{definition}[filtration]
	Fix a cocomplex $(K^\bullet,d)$ of abelian groups. A \textit{filtration} of $K^\bullet$ is an $\mathbb N$-graded filtration of the form
	\[K^n=F^0K^0\supseteq F^1K^n\supseteq\cdots\]
	of $K^n$ for all $n$ such that $D\colon F^pK\to F^pK$ for all $p$. We will throughout assume that each $n$ has $F^pK^n=0$ for sufficiently large $p$. We call the triple $(K^\bullet,d,F^\bullet)$ a \textit{filtered complex}.
\end{definition}
\begin{example}
	The double complex $(K^{\bullet,\bullet},d,\delta)$ gives rise to a cocomplex
	\[K^n\coloneqq\bigoplus_{p+q=n}K^{p,q}\]
	with canonical filtration by
	\[F^{p_0}K^n=\bigoplus_{\substack{p+q=n\\p\ge p_0}}K^{p,q}=K^{p_0,n-p_0}\oplus K^{p_0+1,n-p_0-1}\oplus\cdots\oplus K^{n-p_0+1,p_0+1}\oplus K^{n-p_0,p_0}.\]
\end{example}

\end{document}