% !TEX root = ../notes.tex

\documentclass[../notes.tex]{subfiles}

\begin{document}

We began by finishing up our discussion of duality.

\subsection{The Kodaira Vanishing Theorem}
Here is our statement.
\begin{theorem}[Kodaita vanishing] \label{thm:kodaira-c}
	Fix a nonsingular projective variety $X$ over $\CC$, and set $n\coloneqq\dim X$. Given an ample line bundle $\mc L$ on $X$, the following are true.
	\begin{listalph}
		\item $H^i\left(X,\mc L^i\otimes\omega_X\right)=0$ for all $i>0$.
		\item $H^i\left(X,\mc L^{-i}\right)=0$.
	\end{listalph}
\end{theorem}
\begin{proof}
	We omit the proof because it uses analytic methods, but we do remark that (a) and (b) are equivalent by \Cref{cor:serre-duality}.
\end{proof}
\begin{corollary}
	Fix a nonsingular projective variety $X$ over a field $k$ of characteristic $0$, and set $n\coloneqq\dim X$. Given an ample line bundle $\mc L$ on $X$, the following are true.
	\begin{listalph}
		\item $H^i\left(X,\mc L^i\otimes\omega_X\right)=0$ for all $i>0$.
		\item $H^i\left(X,\mc L^{-i}\right)=0$.
	\end{listalph}
\end{corollary}
\begin{proof}
	We use the ``Lefschetz principle'' to reduce our computations to $\CC$. Because $X$ is projective, we are granted a closed embedding $i\colon X\subseteq\PP^N_k$ for some $N>0$.
	
	Now, listing out the various coefficients used in the definition of the closed embedding $i$ and the construction of the line bundle $\mc L$, we are granted a field $k_0$ finitely generated over $\QQ$, a smooth scheme $X_0$ over $k_0$, a closed embedding $i_0\colon X_0\to\PP^N_{k_0}$, and a line bundle $\mc L_0$ on $X_0$ such that $i_0\times_{k_0}k=i$ and $\mc L_0$ pulls back to $X$. Explicitly, note the ideal sheaf of $X$ is finitely generated and so can be described with only finitely many elements over $\QQ$ (from $k$), and $\mc L$ can similarly be described in terms of a trivializing open cover and how to transition between them (via a Cartier divisor) which can again be fit into a finitely generated field extension of $\QQ$. (Smoothness of $X$ can be given by the full rank of the Jacobian via local defining equations, so again it can be witnessed using finitely many elements of $k_0$.)

	Now, because $\CC$ has infinite transcendence degree over $\QQ$, we can find a field embedding $k_0\into\CC$, allowing us to base-change $X_0$ and $\mc L_0$ (over $k_0$) to $X_\CC$ and $\mc L_\CC$ (over $\CC$). We are now able to apply \Cref{thm:kodaira-c} because proper base change implies
	\[H^\bullet(X,\mc L)=H^\bullet(X_0,\mc L_0)\otimes_{k_0}k\qquad\text{and}\qquad H^\bullet(X_\CC,\mc L_\CC)=H^\bullet(X_0,\mc L_0)\otimes_{k_0}\CC\]
	for any index. More directly, one can note that these cohomology groups can be computed via \v Cech cohomology by \Cref{thm:cech-comparison}, and the \v Cech complex can be base-changed by a field to adjust cohomology exactly as needed to produce the above isomorphisms.
\end{proof}

\end{document}