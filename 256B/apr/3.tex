% !TEX root = ../notes.tex

\documentclass[../notes.tex]{subfiles}

\begin{document}

\section{April 3}

Today we complete our discussion of Serre duality.

\subsection{Finishing Serre Duality}
Recall our statement.
\serreduality*
\noindent Last class we showed the existence of the $\theta^i$.
\begin{proof}[Proof of equivalences]
	Here are our implications.
	\begin{itemize}
		\item We show that (a) implies (b) and (d). Well, fix some locally free sheaf $\mc F$ on $X$. For any closed point $x\in X$, we note $\mc F_x$ has depth as an $\mathcal O_{X,x}$-module is equal to $n=\dim X$ by \Cref{prop:pd-depth-bound} because $X$ is Cohen--Macaulay and equidimensional. Now, we quickly note that we have a surjection
		\[\OO_{\PP,x}\onto\OO_{X,x}.\]
		Further, $\PP$ is regular, so $\OO_{\PP,x}$ is regular local of dimension $N$, so by pulling back a regular sequence from $\OO_{X,x}$ we see that $\op{depth}_{\OO_{\PP,x}}\mc F_x=n$ as well. (Indeed, pulling back the regular sequence gives $\ge n$, but we can also push forward a regular sequence to get $\le n$; the point is that the $\OO_{\PP,x}$-action on $\mc F_x$ is happening through the quotient, so an element is a zero-divisor if and only if it is a zero-divisor after projected to $\OO_{X,x}$.) It follows from \Cref{prop:pd-depth-bound} that $\op{pd}_{\OO_{\PP,x}}\mc F_x=N-n$, so
		\[\mathcal Ext^a_{\OO_\PP}(\mc F,\mc G)_x=\mathcal Ext^a_{\OO_{\PP,x}}(\mc F_x,\mc G_x)\]
		vanishes for $a>N-n$ and quasicoherent sheaf $\mc G$. We already showed that this vanishes for $a<r$ by \Cref{prop:lower-codim-dies} (again, we use that $\mc F$ is locally free here), so we only have cohomology in $a=r$; taking $\mc F=\omega^\circ_X$ produces (d).

		It remains to show (b). Well, \Cref{thm:serre-duality-pn} tells us that
		\[H^a(X,\mc F(-q))^\lor=H^a(\PP,\mc F(-q))^\lor=\op{Ext}^{N-i}_{\OO_\PP}(\mc F(-q),\omega_\PP)=\op{Ext}^{N-i}(\mc F,\omega(q)).\]
		(Note the first equality is legal: $X\into\PP$ is a closed embedding, so \Cref{cor:closed-embed-preserve-cohom} says we're okay; the last equality is okay by \Cref{lem:twist-ext-by-vector-bundle}.) Now, by \Cref{prop:global-secs-ext}, we see that this is
		\[\Gamma\left(\PP,\mathcal Ext^{N-a}_{\OO_\PP}(\mc F,\omega_\PP(q))\right)\]
		for $q$ large enough. But this dies for $N-a>r$ (which means $a<n$) by \Cref{thm:dimension-bound-cohom}, where we are moving cohomology back along our closed embedding.

		\item We show that (b) implies (d) implies (a). Define $\mc E^{N-a}\coloneqq\mathcal Ext^{N-a}_{\OO_\PP}(\mc F,\omega_\PP)$. Note that building some kind of locally free resolution and then computing $\mathcal Ext$ from that lets us move twists in and out of or $\mathcal Ext$, so we see
		\[\mathcal E^{N-a}(q)=\mathcal Ext^{N-a}_{\OO_\PP}(\mc F,\omega_\PP(q)).\]
		Now, $\mathcal E^{N-a}(q)$ is globally generated for $q$ large enough because $\OO(1)$ is ample, but this vanishes for $q$ large enough by (b), so we conclude that $\mathcal E^{N-i}(q)=0$ and thus $\mathcal E^{N-i}=0$ by twisting back. By using $\mc F=\OO_X$, we conclude that $\mathcal E^a(\OO_X,\omega_\PP)=0$ for $a>N-n$, from which (d) follows by \Cref{lem:almost-serre-duality}. (Again, we note that we vanish for $a<r$ already by \Cref{prop:lower-codim-dies}.)

		It remains to show (a) from (d). Now, taking stalks from what we just had, we have concluded that
		\[\mathcal Ext^a_{\OO_{\PP,x}}(\OO_{X,x},\omega_{\PP,x})=0\]
		for any closed point $x\in X$ and any $a>r$. Notably, $\omega_\PP=\OO_\PP(-N-1)$, so $\omega_{\PP,x}=\OO_{\PP,x}$. It follows by \Cref{prop:local-ext-bounds-pd} that $\op{pd}_{\OO_{\PP,x}}\OO_{X,x}\le r$, so \Cref{prop:pd-depth-bound} tells us that $\op{depth}_{\OO_{\PP,x}}\OO_{X,x}\ge n$. But then it follows that $\op{depth}_{\OO_{\PP,x}}\OO_{X,x}=n$ because the depth cannot exceed dimension, so we conclude that
		\[\op{depth}\OO_{X,x}\OO_{X,x}=n\]
		again by doing some argument about regular sequences in quotient rings. Because each closed point has the same depth equal to the dimension, we conclude that $X$ is equidimensional and Cohen--Macaulay.

		\item We show that (ii) implies (iii). It is enough to show that $H^{n-a}(X,-)^\lor$ is co-effaceable (and thus universal), using the fact that $\theta^0$ is an isomorphism already. Well, for our coherent sheaf, we pick up some surjection
		\[\bigoplus_{i=1}^k\OO(-q_i)\to\mc F,\]
		where we can make the $q_i$ as large as we please by twisting. But these $\OO(-q_i)$ have vanishing cohomology for $q_i$ large enough by (ii), so we get co-effaceable.

		\item We show that (iii) implies (ii). Fix $\mc F$ locally free. Well, using our duality, we see
		\[H^{n-a}(X,\mc F(-q))^\lor=\op{Ext}^a(\mc F(-q),\omega_X^\circ).\]
		Using \Cref{lem:twist-ext-by-vector-bundle}, this is
		\[\op{Ext}^i_{\OO_X}(\OO_X,\mc F^\lor\otimes\omega_X^\circ(q))=H^i(X,\mc F^\lor\otimes\omega^\circ_X(q)),\]
		but this dies after high enough twisting by \Cref{thm:very-ample-kills-cohom}.
		\qedhere
	\end{itemize}
\end{proof}


\end{document}