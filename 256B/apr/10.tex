% !TEX root = ../notes.tex

\documentclass[../notes.tex]{subfiles}

\begin{document}

We began class by completing the proof of \Cref{thm:serre-dual-proj-space}.
\begin{remark}
	The isomorphism $t\colon H^n(X,\omega_X)\to k$ is invariant under automorphisms of $\PP^n_k$ (which are in bijection with $\op{PGL}_n(k)$); see \cite[Example~7.1.1]{hartshorne}.
\end{remark}

\subsection{Dualizing Sheaves}
The sheaf $\omega_X$ was the main character of \Cref{thm:serre-dual-proj-space}. For more general $X$, it is not obvious what $\omega_X$ should be, so we will simply pick up a definition for when we have such a sheaf.
\begin{definition}[dualizing sheaf]
	Fix a proper $k$-scheme $X$ of dimension $n$. Then a \textit{dualizing sheaf} of $X$ is a pair $(\omega_X^\circ,t)$ where $\omega_X^\circ$ is a coherent sheaf on $X$, and $t\colon H^n(X,\omega_X^\circ)\to k$ is a morphism such that the composite
	\[\op{Hom}(\mc F,\omega_X^\circ)\otimes H^n(X,\mc F)\to H^n(X,\omega_X^\circ)\stackrel t\to k\]
	is a perfect pairing for all coherent sheaves $\mc F$; in other words, the coherent sheaf $\omega_X^\circ$ represents the functor $\mc F\mapsto H^n(X,\mc F)^\lor$, where choice of $t$ amounts to a choice of the natural isomorphism from $\op{Hom}(-,\omega_X^\circ)$. More generally, for $n\ge\dim X$, we say that $\omega_X^\circ$ is an $n$-dualizing sheaf if and only if it represents the functor $\mc F\mapsto H^n(X,\mc F)^\lor$.
\end{definition}
\begin{remark}
	Technically, our definition permits $X$ to have irreducible components of differing dimensions. This is bad, so in the future, we will require $X$ to have pure dimension $n$.
\end{remark}
\begin{remark}
	If $n>\dim X$, then the functor $\mc F\mapsto H^n(X,\mc F)^\lor$ is the zero functor, so we take $\omega_X^\circ=0$ to represent it.
\end{remark}
\begin{remark} \label{rem:yoneda-for-dualizing}
	Let's explain why dualizing sheaves amount to representing objects. This is essentially the Yoneda lemma.
	\begin{itemize}
		\item Suppose $(\omega_X^\circ,t)$ is a dualizing sheaf. Then our perfect pairing amounts to saying that we have a natural isomorphism
		\[\op{Hom}(-,\omega_X^\circ)\Rightarrow H^n(X,-)^\lor\]
		by $\varphi\mapsto\left(c\mapsto t(H^n(\varphi)c)\right)$. So $\omega_X^\circ$ represents our functor.
		\item Suppose $\omega_X^\circ$ represents our functor via the natural isomorphism $\Psi\colon\op{Hom}(-,\omega_X^\circ)\Rightarrow H^n(X,-)^\lor$. Then $t\coloneqq\Psi_{\omega_X^\circ}({\id_{\omega_X^\circ}})$ is a morphism $t\colon H^n(X,\omega_X^\circ)\to k$, and we note that the following diagram commutes for any morphism $\varphi\in\op{Hom}(\mc F,\omega_X^\circ)$.
		% https://q.uiver.app/#q=WzAsOCxbMCwwLCJcXG9we0hvbX0oXFxvbWVnYV9YXlxcY2lyYyxcXG9tZWdhX1heXFxjaXJjKSJdLFswLDEsIlxcb3B7SG9tfShcXG1jIEYsXFxvbWVnYV9YXlxcY2lyYykiXSxbMSwwLCJIXm4oWCxcXG9tZWdhX1heXFxjaXJjKV5cXGxvciJdLFsxLDEsIkhebihYLFxcbWMgRileXFxsb3IiXSxbMiwwLCJcXGlkX3tcXG9tZWdhX1heXFxjaXJjfSJdLFsyLDEsIlxcdmFycGhpIl0sWzMsMCwidCJdLFszLDEsInRcXGNpcmMgSF5uKFxcdmFycGhpKSJdLFswLDJdLFswLDEsIi1cXGNpcmNcXHZhcnBoaSIsMl0sWzEsM10sWzIsMywiSF5uKFxcdmFycGhpKSJdLFs0LDUsIiIsMCx7InN0eWxlIjp7InRhaWwiOnsibmFtZSI6Im1hcHMgdG8ifX19XSxbNSw3LCIiLDAseyJzdHlsZSI6eyJ0YWlsIjp7Im5hbWUiOiJtYXBzIHRvIn19fV0sWzQsNiwiIiwyLHsic3R5bGUiOnsidGFpbCI6eyJuYW1lIjoibWFwcyB0byJ9fX1dLFs2LDcsIiIsMix7InN0eWxlIjp7InRhaWwiOnsibmFtZSI6Im1hcHMgdG8ifX19XV0=&macro_url=https%3A%2F%2Fraw.githubusercontent.com%2FdFoiler%2Fnotes%2Fmaster%2Fnir.tex
		\[\begin{tikzcd}
			{\op{Hom}(\omega_X^\circ,\omega_X^\circ)} & {H^n(X,\omega_X^\circ)^\lor} & {\id_{\omega_X^\circ}} & t \\
			{\op{Hom}(\mc F,\omega_X^\circ)} & {H^n(X,\mc F)^\lor} & \varphi & {t\circ H^n(\varphi)}
			\arrow[from=1-1, to=1-2]
			\arrow["{-\circ\varphi}"', from=1-1, to=2-1]
			\arrow[from=2-1, to=2-2]
			\arrow["{H^n(\varphi)}", from=1-2, to=2-2]
			\arrow[maps to, from=1-3, to=2-3]
			\arrow[maps to, from=2-3, to=2-4]
			\arrow[maps to, from=1-3, to=1-4]
			\arrow[maps to, from=1-4, to=2-4]
		\end{tikzcd}\]
		This unravels into telling us that $(\omega_X^\circ,t)$ is a dualizing sheaf; notably, we have produced the needed perfect pairing because the bottom row is an isomorphism because $\Psi_{\mc F}$ is a natural isomorphism.
	\end{itemize}
\end{remark}

\end{document}