% !TEX root = ../notes.tex

\documentclass[../notes.tex]{subfiles}

\begin{document}

Today we try to compute the dualizing sheaf in some cases.

\subsection{Computing the Dualizing Sheaf}
We will want the following notion.
\begin{definition}[regular]
	Fix a ring $A$ and an $A$-module $M$. Then a \textit{regular sequence} for $M$ is a sequence $\{f_1,\ldots,f_r\}\subseteq A$ such that $f_i$ is a non-zero-divisor on $M/(f_1,\ldots,f_{i-1}M)$ for all $i$.
\end{definition}
A priori, the regular sequence depends on the ordering of the $f_\bullet$s.
\begin{proposition}
	Fix a ring $A$ and an $A$-module $M$ and a regular sequence $\{f_1,\ldots,f_r\}$ for $M$. Then
	\[h_i(K_\bullet(f_1,\ldots,f_r)\otimes_AM)=\begin{cases}
		0 & \text{if }i>0, \\
		M/(f_1,\ldots,f_r)M & \text{if }i=0.
	\end{cases}\]
\end{proposition}
\begin{proof}
	Omitted for time. See, for example, \cite[Section~17.5]{eisenbud-comm-alg}. This is more or less just a long computation.
\end{proof}
Noting that exterior powers are free over $A$, we see that we have basically produced a free resolution for $M/(f_1,\ldots,f_r)M$.

Anyway, here is our main result.
\begin{theorem}
	Fix a field $k$, and let $X$ be a nonempty closed subscheme of $\PP^N_K$ cut out by the ideal sheaf $\mc I$. If $X$ is a locally complete intersection of codimension $r$, then
	\[\omega_X^\circ\cong\omega_{\PP^N_k}\otimes\land^r\left(\mc I/\mc I^2\right)^\lor.\]
	In particular, $\omega_X^\circ$ is a line bundle on $X$.
\end{theorem}
\begin{proof}[Sketch]
	We know that $\omega_X^\circ=\mc Ext_{\PP^N_k}^r(\OO_X,\omega_{\PP^N_k})$, so we just need to compute this sheaf.
	\begin{enumerate}
		\item We begin by computing our stalks. Fix some closed point $x\in X$, and let $U\subseteq\PP^N_k$ be an affine open neighborhood of $x$; set $U=\Spec A$. Being a locally complete intersection means that we may shrink $X$ so that $X\cap U=V(f_1,\ldots,f_r)$ inside $\Spec A$. With $x\in U$, we may find the corresponding maximal ideal $\mf m\in\Spec A$. Because $\PP^N_k$ is smooth, it is regular, so $A_\mf m$ is regular, so it is Cohen--Macaulay, so $\{f_1,\ldots,f_r\}$ is a regular sequence by some commutative algebra.

		Thus, $K_\bullet(f_1,\ldots,f_r)\otimes A_\mf m$ is a free resolution of $A_\mf m/(f_1,\ldots,f_r)A_\mf m$, which is $\OO_{X,x}$. The property of being free just means that there is some set of elements which provide an isomorphism from some free module, which is a condition that can be spread out (namely, we are asking for some kernel and cokernel to vanish as soon as we fix the required image); thus, we can shrink $U$ so that $K_\bullet(f_1,\ldots,f_r)$ is just a free resolution of $A$. We now use this resolution to compute
		\begin{align*}
			\mc Ext^r_{\PP^N_k}\left(\OO_X,\omega_{\PP^N_k}\right)|_U &\cong h^r\left(\mc Hom(K_\bullet(f_1,\ldots,f_r)^{\sim},\omega_{\PP^N_k}|)U\right) \\
			&= \frac{\ker d^r}{\im d^{r-1}} \\
			&= \frac{\mc Hom\left(K_r(f_1,\ldots,f_r)^{\sim},\omega_{\PP^N_k}|_U\right)}{\im\left(\mc Hom\left(K_{r-1}(f_1,\ldots,f_r)^{\sim},\omega_{\PP^N_k}|_U\right)\to\mc Hom\left(K_r(f_1,\ldots,f_r)^{\sim},\omega_{\PP^N_k}|_U\right)\right)} \\
			&= \frac{\omega_{\PP^N_k}|_U}{\im\left((\omega_{\PP^N_k}|_U)^r\stackrel{(f_1,\ldots,f_r)}\to\omega_{\PP^N_k}|_U\right)}
		\end{align*}
		by some computation with the Koszul complex; for example, $\ker d^r$ is as written because the next term of the sequence is $0$, and $\im d^{r-1}$ is as written by tracking through what happens on the Koszul complex at this stage (with the standard basis of the exterior product). We now note that this last sheaf is isomorphic to $(\omega_{\PP^N_k}\otimes\OO_X)|_U$.

		\item We now must check that the isomorphisms provided in the previous step glue to a global isomorphism (namely, that they are compatible on intersections). The main check is that one can more naturally identify
		\[K_r(f_1,\ldots,f_r)^{\sim}\otimes\OO_X\cong\land^r\left(\mc I/\mc I^2\right)\]
		by noting that the quotient on the right is basically generated by $(f_1,\ldots,f_r)$. We won't write out these details.
		\qedhere
	\end{enumerate}
\end{proof}
\begin{corollary}
	Fix a regular, projective, nonempty projective scheme. Then $\omega_X^\circ\cong\omega_X$, where $\omega_X\coloneqq\land^{\dim X}\Omega_X$.
	In particular, $H^r(X,\OO_X)\cong H^0(X,\omega_X)^\lor$.
\end{corollary}
\begin{proof}
	Because $X$ is projective, we may embed it into $\PP^N_k$ for some $N>0$. A regular scheme is a locally complete intersection, so
	\[\omega_X^\circ\cong\omega_{\PP^N_k}\otimes\land^r\left(\mc I/\mc I^2\right)^\lor\]
	as above, but the adjunction formula \cite[Proposition~II.8.20]{hartshorne} explains that this is $\omega_X$. The last sentence now follows from applying \Cref{cor:serre-duality}.
\end{proof}
\begin{example}
	If $X$ has dimension $1$, then we see that
	\[p_a(X)=h^1(X,\OO_X)=h^0(X,\omega_X)=p_g(X).\]
	However, if $X$ is singular, then it turns out $p_a(X)>p_g(X)$; notably, if $\pi\colon\widetilde X\to X$ is a normalization of $X$, then $p_g(X)=p_g(\widetilde X)$. It is known that $p_a(X)-p_g(X)$ is controlled by the singularities of $X$: one has
	\[p_a(X)-p_g(X)=\sum_{\text{singular }x\in X}[\kappa(x):k]\delta_x,\]
	where $\delta_x\coloneqq\dim_{\kappa(x)}(\pi_*\OO_{\widetilde X}/\OO_{X})_x$.
\end{example}

\end{document}