% !TEX root = ../notes.tex

\documentclass[../notes.tex]{subfiles}

\begin{document}

\section{April 19}

Today we start talking about smooth morphisms.

\subsection{Unramified Morphisms}
Morally speaking, a morphism $f\colon X\to S$ locally of finite presentation\footnote{Recall that a morphism of schemes $f\colon X\to S$ is locally of finite presentation if and only if each $x\in X$ has an affine open neighborhood $\Spec A\subseteq X$ which goes to an affine open neighborhood $\Spec B\subseteq S$ so that $A$ is finitely presented as an $A$-algebra.} is smooth at a point $x\in X$ if and only if we can find an open neighborhood $U$ of $x$ so that $f|_U$ factors as
\[U\to\AA^n_S\to S\]
where the map $U\to\AA^n_S$ is \'etale. Here, \'etale morphisms are flat and unramified, where unramified is a word we have not defined. Roughly speaking, we are trying to say that $f$ is locally the projection from affine space, up to some \'etale map (which should be thought of as almost an open embedding). Then $f$ is smooth if and only if it is smooth everywhere.

To turn this into a less clunky definition, we will want to use the relative differentials. The unramified map $U\to\AA^n_S$ preserves relative differentials, so we will want to use the ``free-ness'' of the relative differentials $\Omega_{\AA^n_S/S}$.

Anyway, let's begin our definitions.
\begin{definition}[relative dimension]
	A scheme morphism $f\colon X\to S$ locally of finite presentation has \textit{relative dimension $d$} if and only if
	\[\dim f^{-1}(f(x))=d\]
	for any $x\in X$. In other words, we are saying that all fibers have dimension $d$.
\end{definition}
For example, \'etale maps will be smooth of relative dimension $0$.
\begin{definition}[unramified]
	A scheme morphism $f\colon X\to S$ locally of finite presentation is \textit{unramified at $x\in X$} if and only if $(\Omega_{X/S})_x=0$. Thus, the morphism $f$ is unramified if and only if $\Omega_{X/S}=0$.
\end{definition}
\begin{remark}
	Checking affine-locally, because $f$ is locally of finite presentation, at $x\in X$, we can place $x\in\Spec A\subseteq X$ and $f(U)\subseteq\Spec B\subseteq S$. Because $A$ is finitely presented over $B$, we may factor $f|_U$ here as
	\[U\into\AA^n_B\to B\]
	where $U\into\AA^n_B$ is a closed embedding, say given by the finitely generated ideal $I\subseteq B[x_1,\ldots,x_n]$. Thus, we see we have the right-exact sequence
	\[\mc I/\mc I^2\to\Omega_{\AA^n_S/S}\to\Omega_{X/S}\to0,\]
	by gluing these pieces together. Bringing this together, we see that being unramified is equivalent to asking for the map $\mc I/\mc I^2\to\Omega^1_{\AA^n_S/S}$ to be surjective.
\end{remark}
As usual, one can check that unramified morphisms have open locus (because the definition is stalk-local on a sheaf), and we can check on stalks that we are stable under composition. Because the pullback of the sheaf of differentials behaves in squares, we also see that unramified morphisms are stable under base-change.

It will be easier to describe being unramified in terms of the diagonal.
\begin{proposition}
	Fix a morphism $f\colon X\to S$ locally of finite presentation. Then the following are equivalent.
	\begin{listalph}
		\item $f$ is unramified at a point $x\in X$.
		\item The diagonal $\Delta_f\colon X\to X\times_SX$ is an open embedding in an open neighborhood of $x\in X$.
		\item We can check on fibers: $X_s$ is unramified over $k(s)$ at $x\in X$.
		\item Letting $\mf m_s\subseteq\OO_{S,s}$ be the maximal ideal for some $s\in S$ where $s\coloneqq f(x)$, then $\mf m_s\OO_{X,x}=\mf m_x$ and $k(x)$ thus becomes a finite separable extension of $k(s)$.
	\end{listalph}
\end{proposition}
\begin{proof}
	Let's have fun with our adjectives.
	\begin{itemize}
		\item We show that (a) and (b) are equivalent. Let $\mc I$ be the ideal sheaf of the diagonal in $X\times_SX$. Then by definition, we have $\Omega_{X/S}=\Delta^*\left(\mc I/\mc I^2\right)$. This essentially completes the proof. For example, if $(\Omega_{X/S})_x=0$, then $\left(\mc I/\mc I^2\right)_x=0$, so $\mc I_x=0$ by Nakayama's lemma. (Note we are using the fact that $\mc I_x$ is finitely generated here!) In the other direction, if $\mc I_x=0$, then we can read back $(\Omega_{X/S})_x=0$, which is what we wanted.
		\item We note that (a) implies (c) by base-change to a fiber.
		\item We show (c) implies (a). Let $s\colon k(s)\to S$ be the corresponding bound. We have the following pullback square.
		% https://q.uiver.app/?q=WzAsNCxbMCwxLCJrKHMpIl0sWzEsMSwiUyJdLFswLDAsIlhfcyJdLFsxLDAsIlgiXSxbMiwzXSxbMCwxXSxbMywxXSxbMiwwXSxbMiwxLCIiLDEseyJzdHlsZSI6eyJuYW1lIjoiY29ybmVyIn19XV0=&macro_url=https%3A%2F%2Fraw.githubusercontent.com%2FdFoiler%2Fnotes%2Fmaster%2Fnir.tex
		\[\begin{tikzcd}
			{X_s} & X \\
			{k(s)} & S
			\arrow[from=1-1, to=1-2]
			\arrow[from=2-1, to=2-2]
			\arrow[from=1-2, to=2-2]
			\arrow[from=1-1, to=2-1]
			\arrow["\lrcorner"{anchor=center, pos=0.125}, draw=none, from=1-1, to=2-2]
		\end{tikzcd}\]
		Now, $s^*\Omega_{X/S}=\Omega_{X_s/k(s)}$, so we are being given that $\left(s^*\Omega_{X/S}\right)_x=0$. Computing this pullback, we see
		\[\left(s^*\Omega_{X/S}\right)_x=(\Omega_{X/S})_x\otimes_{\OO_{S,s}}k(s)=\left(\Omega_{X/S}\right)_x/\mf m_s\left(\Omega_{X/S}\right)_x=0,\]
		so Nakayama's lemma again tells us that $\left(\Omega_{X/S}\right)_x=0$.
		\item We show that (d) implies (a). Note that (c) allows us to pullback along $s\colon k(s)\to\Spec S$ so that we can check being unramified $k(s)$. In other words, we may assume that $\OO_{S,s}$ is a field, whereupon looking locally we are assuming that $S=\Spec k$ for some field $k\coloneqq k(s)$. (One should check that the hypotheses in (d) are also preserved by this base-change, which is true.) Now, we may compute that
		\[\left(\Omega_{X/S}\right)_x=\Omega_{k(x)/k(s)}\]
		will vanish because $k(x)/k(s)$ is a finite separable extension.
		\item We show that (b) implies (d). As usual, by taking the base-change to the fiber, we may assume that $S=\Spec k$ for a field $k\coloneqq k(s)$. (One should check that $X_s\to X_s\times_kX_s$ remains an open embedding, but the point is that we are just computing some base-change of $X\to X\times_SX$.) Our goal now is to show that $X=X_s$ is a finite disjoint union of spectra of finite separable extensions of $k$. By checking locally, we may assume that $X=\Spec A$.

		Continuing, we may base-change by $k\to\overline k$ to allow us to assume that $k$ is algebraically closed. Thus, we now want to show that $A$ is a finite product of fields isomorphic to $k$ because $k$ is algebraically closed. (Notably, we will have shown that $A$ before the base-change to the algebraic closure is a product of fields, and if $A$ has any factor which was inseparable, then we would obtain nilpotents in this base-change, so we are actually showing that $A$ before this base-change only has factors which are finite separable extensions of $k$.)

		Now, the map $X\to X\times_kX$ is a closed embedding and an open embedding, so it will follow that $X$ is some finite union of closed points. Well, for a closed point $z\in X$, we can take the pre-image of $\Delta(X)$ along the map
		\[X\to\{z\}\to X\]
		to see that $\{z\}$ is open. So closed points are open in $X$, from which it follows that $X$ is a finite union of closed points.

		It remains to show that we have no nilpotents. By shrinking $X$ more, we may assume that $X$ is a single point. However, the diagonal becomes the map $A\otimes_kA\to A$ which must be an isomorphism as both an open and closed embedding, so comparing dimensions enforces $\dim_kA=1$. Thus, we see that $A$ has no nilpotents.
		\qedhere
	\end{itemize}
\end{proof}
\begin{remark}
	The above proposition explains why $f$ being unramified implies being relative dimension $0$. Indeed, I think this follows more or less from (d) because we are essentially checking things on fibers.
\end{remark}

\end{document}