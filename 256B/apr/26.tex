% !TEX root = ../notes.tex

\documentclass[../notes.tex]{subfiles}

\begin{document}

We now turn to a discussion of higher direct images.

\subsection{Higher Direct Images}
Here is our definition.
\begin{definition}[higher direct image]
	Fix a continuous map $f\colon X\to Y$ of topological spaces. Then the \textit{higher direct images} are the right-derived functors of the functor $f_*\colon\mathrm{Ab}(X)\to\mathrm{Ab}(Y)$.
\end{definition}
\begin{remark}
	Perhaps we should check that $f_*$ is left exact. Left exactness can be checked on the level of sections, so pick up a left-exact sequence $0\to\mc F'\to\mc F\to\mc F''$ of sheaves on $X$, and to check that
	\[0\to f_*\mc F'\to f_*\mc F\to f_*\mc F''\]
	is exact, it suffices to pick up an open subset $V\subseteq Y$ and note that
	\[0\to f_*\mc F'(V)\to f_*\mc F(V)\to f_*\mc F''(V)\]
	is exact because this is the exact sequence
	\[0\to \mc F'\left(f^{-1}(V)\right)\to\mc F\left(f^{-1}(V)\right)\to\mc F''\left(f^{-1}(V'')\right).\]
\end{remark}
\begin{example}
	Frequently we will take $f\colon X\to Y$ to be a morphism of schemes, which can be thought of as a family of schemes over $Y$ by taking the fiber $X_y$ for each $y\in Y$. For example, given a functor $F\colon\mathrm{Sch}_Y\opp\to\mathrm{Set}$ (for example, outputting abelian varieties or similar), one can hope for a morphism $f\colon X\to Y$ so that $X$ represents this functor.
\end{example}
\begin{example}
	A projective $\QQ$-variety $X$ can be extended to a morphism $X'\to\Spec\ZZ$ basically by clearing denominators to define our equations. One has $X(\QQ)=X'(\Spec\ZZ)$ by the valuative criterion of properness for $X'$. This is the beginning of Arakelov theory.
\end{example}
It will be helpful to relate our higher direct images to our previously studied cohomology.
\begin{proposition} \label{prop:higher-direct-image-as-sheafification}
	Fix a continuous map $f\colon X\to Y$ of topological spaces. For any sheaf $\mc F$ on $X$ of abelian groups and index $i$, the sheaves $R^\bullet f_*\mc F$ are the sheafification of the presheaves
	\[V\mapsto H^\bullet\left(f^{-1}(V),\mc F|_{f^{-1}(V)}\right).\]
\end{proposition}
\begin{proof}
	Let $\mc H^i(\mc F)$ be the given sheaves. Note $\mc H^0(\mc F)$ is the sheafification of the sections of $\mc F$ pushed from $X$, which is exactly $f_*\mc F$ already. Note that $R^\bullet f_*$ yields a universal $\delta$-functor, so it is enough to check that $\mc H^\bullet(\mc F)$ provides an effaceable (and hence universal) $\delta$-functor.

	To check that we have a $\delta$-functor, fix an exact sequence $0\to\mc F'\to\mc F\to\mc F''\to0$ of sheaves of abelian groups on $X$, and then we note that each open subset $V\subseteq Y$ produces a long exact sequence
	\[0\to H^0\left(f^{-1}(V),\mc F'|_{f^{-1}(V)}\right)\to H^0\left(f^{-1}(V),\mc F|_{f^{-1}(V)}\right)\to H^0\left(f^{-1}(V),\mc F''|_{f^{-1}(V)}\right)\to\cdots.\]
	Thus, the sequence of presheaves will be exact, so the sequence of sheaves from $\mc H^\bullet$ will be exact upon taking sheafification. This completes our construction of long exact sequence, making $\mc H^\bullet$ into a $\delta$-functor.

	It remains to check $\mc H^\bullet$ is effaceable. The point is that $\mathrm{Ab}(X)$ has enough injectives. Namely, any sheaf can be embedded into an injective object $\mc I$, so it will be enough to check that $\mc H^\bullet(\mc I)=0$ in higher degrees. Well, \Cref{lem:restrict-injective-open} tells us that $\mc I|_{f^{-1}(V)}$ is injective when $V\subseteq Y$ is open, so $H^i\left(f^{-1}(V),\mc I|_{f^{-1}(V)}\right)=0$ for $i>0$ and open $V\subseteq Y$, so $\mc H^i(\mc I)=0$ for $i>0$ by taking the sheafification of the zero presheaf.
\end{proof}
\begin{corollary} \label{cor:res-higher-direct-image}
	Fix a continuous map $f\colon X\to Y$ of topological spaces. For any sheaf $\mc F$ on $X$ of abelian groups and open subset $V\subseteq Y$, we have
	\[R^\bullet f_*\mc F|_V\cong R^\bullet f_*\left(\mc F|_{f^{-1}(V)}\right).\]
\end{corollary}
\begin{proof}
	Use \Cref{prop:higher-direct-image-as-sheafification} and note that the construction of the presheaf in the statement commutes with restriction. Explicitly, constructing it for $f\colon X\to Y$ and then restricting to $f^{-1}(V)$ produces the same presheaf as constructing it for $f\colon f^{-1}(V)\to V$.
\end{proof}
\begin{corollary} \label{cor:higher-flasque-is-acyclic}
	Fix a continuous map $f\colon X\to Y$ of topological spaces. If $\mc F$ is a flasque sheaf of abelian groups on $X$, then $R^if_*\mc F=0$ for all $i>0$.
\end{corollary}
\begin{proof}
	Note that $\mc F|_U$ is flasque for any open $U\subseteq X$ by its definition, so because flasque sheaves are acyclic by \Cref{lem:flasque-is-acyclic}, we see that the presheaf given in \Cref{prop:higher-direct-image-as-sheafification} vanishes, so we are done.
\end{proof}
Technically, we constructed $R^\bullet f_*$ for sheaves of abelian groups, but we may want to work with the functor $f_*\colon\mathrm{Mod}(\OO_X)\to\mathrm{Mod}(YOO_Y)$ instead (e.g., in the case of schemes). This causes no problems.
\begin{proposition}
	Fix a morphism $f\colon(X,\OO_X)\to(Y,\OO_Y)$ of ringed spaces. Then the higher direct images $R^\bullet f_*$ on $\mathrm{Mod}(\OO_X)$ are the right-derived functors of the functor $f_*\colon\mathrm{Mod}(\OO_X)\to\mathrm{Mod}(\OO_Y)$.
\end{proposition}
\begin{proof}
	Repeat the proof of \Cref{prop:ox-cohomology} because these two functors have a common set of acyclic objects given by flasque sheaves (by \Cref{cor:higher-flasque-is-acyclic} combined with \Cref{lem:inj-to-flasque}).
\end{proof}
% \begin{remark}
% 	For schemes, one can make the same remark for $f_*\colon\mathrm{QCoh}(\OO_X)\to\mathrm{Mod}(\OO_Y)$.
% \end{remark}
\Cref{prop:higher-direct-image-as-sheafification} provides some means to compute these higher direct images, but it will be helpful to have more special cases.
\begin{proposition} \label{prop:higher-direct-image-affine}
	Fix a morphism $f\colon X\to Y$ of schemes. Assume that $X$ is Noetherian and that $Y=\Spec A$ is affine. Then for any quasicoherent sheaf $\mc F$ on $X$, we have
	\[R^\bullet f_*\mc F\cong\widetilde{H^\bullet(X,\mc F)}.\]
\end{proposition}
\begin{proof}
	We will show an isomorphism of $\delta$-functors defined $\mathrm{QCoh}(X)\to\mathrm{Mod}(Y)$. For our base case of this isomorphism, we note that $i=0$ has $f_*\mc F$ quasicoherent on $Y$, and its module of global sections is in fact $H^0(X,\mc F)$, so the isomorphism follows.

	For the induction, we do the usual dimension-shifting. Embed $\mc F$ into a quasicoherent flasque sheaf $\mc G$ on $X$, which is possible by \Cref{cor:embed-to-qcoh-flasque}, and let $\mc Q$ be the cokernel of this embedding, which will produce a short exact sequence
	\[0\to\mc F\to\mc G\to\mc Q\to0.\]
	Then $R^if_*\mc G=0$ for $i>0$ by \Cref{cor:higher-flasque-is-acyclic}, and $H^i(X,\mc G)=0$ by \Cref{lem:flasque-is-acyclic}, so for each $i\ge0$, we get the following commutative diagram.
	% https://q.uiver.app/#q=WzAsOCxbMiwwLCJSXntpKzF9Zl8qXFxtYyBGIl0sWzMsMCwiMCJdLFszLDEsIjAiXSxbMiwxLCJcXHdpZGV0aWxkZXtIXntpKzF9KFgsXFxtYyBGKX0iXSxbMSwwLCJSXmlmXypcXG1jIFEiXSxbMCwwLCJSXmlmXypcXG1jIEciXSxbMCwxLCJcXHdpZGV0aWxkZXtIXmkoWCxcXG1jIEcpfSJdLFsxLDEsIlxcd2lkZXRpbGRle0heaShYLFxcbWMgUSl9Il0sWzAsMywiIiwwLHsic3R5bGUiOnsiYm9keSI6eyJuYW1lIjoiZGFzaGVkIn19fV0sWzUsNF0sWzQsMF0sWzUsNl0sWzYsN10sWzQsN10sWzcsM11d&macro_url=https%3A%2F%2Fraw.githubusercontent.com%2FdFoiler%2Fnotes%2Fmaster%2Fnir.tex
	% https://q.uiver.app/#q=WzAsOCxbMiwwLCJSXntpKzF9Zl8qXFxtYyBGIl0sWzMsMCwiMCJdLFszLDEsIjAiXSxbMiwxLCJcXHdpZGV0aWxkZXtIXntpKzF9KFgsXFxtYyBGKX0iXSxbMSwwLCJSXmlmXypcXG1jIFEiXSxbMCwwLCJSXmlmXypcXG1jIEciXSxbMCwxLCJcXHdpZGV0aWxkZXtIXmkoWCxcXG1jIEcpfSJdLFsxLDEsIlxcd2lkZXRpbGRle0heaShYLFxcbWMgUSl9Il0sWzAsMywiIiwwLHsic3R5bGUiOnsiYm9keSI6eyJuYW1lIjoiZGFzaGVkIn19fV0sWzUsNF0sWzQsMF0sWzUsNl0sWzYsN10sWzQsN10sWzcsM10sWzAsMV0sWzMsMl1d&macro_url=https%3A%2F%2Fraw.githubusercontent.com%2FdFoiler%2Fnotes%2Fmaster%2Fnir.tex
	\[\begin{tikzcd}
		{R^if_*\mc G} & {R^if_*\mc Q} & {R^{i+1}f_*\mc F} & 0 \\
		{\widetilde{H^i(X,\mc G)}} & {\widetilde{H^i(X,\mc Q)}} & {\widetilde{H^{i+1}(X,\mc F)}} & 0
		\arrow[from=1-1, to=1-2]
		\arrow[from=1-1, to=2-1]
		\arrow[from=1-2, to=1-3]
		\arrow[from=1-2, to=2-2]
		\arrow[from=1-3, to=1-4]
		\arrow[dashed, from=1-3, to=2-3]
		\arrow[from=2-1, to=2-2]
		\arrow[from=2-2, to=2-3]
		\arrow[from=2-3, to=2-4]
	\end{tikzcd}\]
	Here, the right vertical map is induced, and the left square commutes with vertical isomorphisms by induction. Namely, granting the result for $i$, we achieve the result for $i+1$ by the uniqueness of the cokernel.
\end{proof}
\begin{corollary}
	Let $f\colon X\to Y$ be a morphism of schemes. Assume that $X$ is Noetherian. If $\mc F$ is a quasicoherent sheaf on $X$, then $R^\bullet f_*\mc F$ is a quasicoherent sheaf on $Y$ for any index.
\end{corollary}
\begin{proof}
	Being quasicoherent can be checked on an affine open cover of $Y$. Because forming the higher direct image commutes with restriction (see \Cref{cor:res-higher-direct-image}), we complete by \Cref{prop:higher-direct-image-affine}.
\end{proof}
\begin{remark}
	If $\mc F$ is not quasicoherent, we of course cannot ask for $R^\bullet f_*\mc F$ to be quasicoherent. For example, take $f=\id_X$ and let $\mc F$ be any sheaf which fails to be quasicoherent.
\end{remark}

\end{document}