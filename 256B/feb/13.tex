% !TEX root = ../notes.tex

\documentclass[../notes.tex]{subfiles}

\begin{document}

\section{February 13}

Today we finish discussing curves. The fourth assignment will be released later today.

\subsection{Finishing Clifford's Theorem}
We are interested in the equality cases of \Cref{prop:clifford}. Here is the result.
\begin{theorem}[Clifford]
	Fix an effective special divisor $D$ on the $k$-curve $X$. Then $\dim|D|=\frac12\deg D$ if and only if $D=0$, or $D=K$, or $X$ is hyperelliptic where $D$ is linearly equivalent to a multiple of $X$'s $g^1_2$.
\end{theorem}
\begin{proof}
	We showed each of these equality cases work in \Cref{rem:clifford-eqs}. It remains to show that the equality implies one of these conditions.

	Well, suppose that $\dim|D|=\frac12\deg D$ and $D\ne0$ and $D\ne K$; we will show that $X$ is hyperelliptic, and $D$ is a multiple of $g^1_2$. Note \Cref{thm:rh} tells us that
	\[\dim|D|-\dim|K-D|\le\deg D+1-g\]
	must achieve equality, by the proof of \Cref{prop:clifford} above. Now, we see $\deg D$ must be even. We will induct on $\deg D$. For example, if $\deg D=2$ and $\ell(D)=1$ forces $D$ to induce a double-cover $X\to\PP^1_k$ (meaning $X$ is hyperelliptic), so $D\sim g^1_2$ follows from this uniqueness of this divisor from \Cref{thm:hyper-canonical-divisor}.

	To finish the proof, suppose $\deg D\ge4$ so that $\dim|D|\ge2$. Because $\dim|K-D|\ge0$ (recall $D$ is special), we get an effective divisor $E\in|K-D|$. Without loss of generality, we may assume that $\deg E>0$ (otherwise $D=K$), so we can find a point $P$ in the support of $E$ and a point $Q$ not in the support of $D$. Without loss of generality, we can find an effective divisor linearly equivalent to $D$ with support containing $P$ and $Q$, which is doable because $\dim|D|\ge2$: indeed, $|D|$ determines a projective morphism $f\colon X\to\PP^n_k$ for some $n\ge2$, and then one can pull back a hyperplane $H$ of $\PP^n_k$ containing $f(P)$ and $f(Q)$ so that $D\sim f^*H$ contains $P$ and $Q$ in its support.

	We now set $D'\coloneqq D\cap E$, which by definition is the maximal effective divisor such that $D'\le D$ and $D'\le E$. Note that $D'<D$ in fact because $Q$ is in the support of $D$ but not $E$, but $D'>0$ because $P$ is in the support of both $D$ and $E$. We are going to want to use the inductive hypothesis on $D'$. Notably, we claim we have an exact sequence
	\[0\to\OO_X(D')\to\OO_X(D)\oplus\OO_X(E)\to\OO_X(D+E-D')\to0.\]
	Here, $\OO_X(D')\to\OO_X(D)\oplus\OO_X(E)$ is the diagonal map, and the second map is $(f,g)\mapsto(f-g)$. Notably, for any valuation $v$ on $K(X)$, we see $v(a+b)\ge\min\{v(a),v(b)\}$ with equality provided $v(a)\ne v(b)$.

	To see our exactness, let's work an example.
	\begin{example}
		Fix an open subscheme $U\subseteq X$, and take $D=P+2Q$ and $E=2P+Q$ on restriction to $U$ so that $D'=P+Q$ here. Given some $f$ with a pole of order $1$ at $P$ and order $2$ at $Q$, and some $g$ with a pole of order $2$ at $P$ and of order $1$ at $Q$. Indeed, we see that $f-g$ has poles of order $2$ at both $P$ and $Q$, which tells us $f-g$ is in fact in $\OO_X(D+E-D')$.
	\end{example}
	This example explains our exactness: by checking surjectivity at the stalks of some point $R$, we may assume that $D=aR$ and $E=bR$ where $a\ge b$ without loss of generality. Then the exact sequence looks roughly like
	\[0\to\OO_X(bR)_R\to\OO_X(aR)\oplus\OO_X(E)_{bR}\to\OO_X(aR)\to0,\]
	which is more clearly exact: we are injective on the left by definition, we are surjective on the right by taking elements of the form $(0,-g)$, and we are exact in the middle because the only elements which vanish in the right map are diagonal ones.

	Now, taking global sections, we get the left-exact sequence
	\[0\to\Gamma(X,\OO_X(D'))\to\Gamma(X,\OO_X(D))\oplus\Gamma(X,\OO_X(E))\to\Gamma(X,\OO_X(D+E-D')).\]
	Taking dimensions everywhere, we see
	\[\ell(D)+\ell(E)\le\ell(D')+\ell(D+E+D'),\]
	which implies
	\[g-1=\dim|K|=\dim|D|+\dim|D-K|\le\dim|D'|+\dim|K-D'|\le\dim|K|=g-1,\]
	where we have used \Cref{lem:add-divisors-dim} repeatedly. Thus, equalities hold everywhere, and we are forced to have $\dim|D'|+\dim|K-D'|=g-1$ as well, which means that $D'$ also satisfies $\dim|D'|=\frac12\deg D'$ by the argument of \Cref{prop:clifford}. However, $\deg D'<\deg D$, and $D'\ne0,K$, so the inductive hypothesis now applies to $D'$, telling us that $X$ is hyperelliptic. It remains to show that $D$ is a multiple of $g^1_2$.

	The trick is to relate $D$ to $K$ and use \Cref{thm:hyper-canonical-divisor} because we already know that $K$ is a multiple of $g^1_2$. Well, set $r\coloneqq\dim|D|$. Note that $D$ being special implies that $\ell(K-D)>0$, so $\deg D\le2g-2$, so we go ahead and study like to show $S\coloneqq D+(g-1-r)g^1_2$ is linearly equivalent to $K$, where the point is that $\deg S=2g-2$. We would like to compute $\dim|S|$. On one hand, we see
	\[\dim|S|\ge\dim|D|\ge\dim|D|+(g-1-r)=g-1.\]
	On the other hand, \Cref{thm:rh} implies
	\[|S|-|K-S|=(2g-2)+(1-g)=g-1,\]
	so $|K-S|\ge0$. Thus, $K-S$ is linearly equivalent to an effective divisor of degree $0$, which actually forces $K-S\sim0$, so $S\sim K$ follows. However, this then implies that
	\[D\sim K-(g-1-r)g^1_2,\]
	which finishes because $K$ is a multiple of $g^1_2$ by \Cref{thm:hyper-canonical-divisor}. This completes the proof.
\end{proof}
\begin{remark}
	This concludes our discussion of curves. We hope it motivates our discussion of cohomology, which will follow.
\end{remark}

\end{document}