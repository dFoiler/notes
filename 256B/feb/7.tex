% !TEX root = ../notes.tex

\documentclass[../notes.tex]{subfiles}

\begin{document}

Today we will compute cohomology on affine schemes.

\subsection{Affine via Cohomology}
To build our cohomology on $\Spec A$, we pick up the following checks.
\begin{proposition} \label{prop:inj-to-flasque-qcho}
	Fix an injective $A$-module $I$, where $A$ is Noetherian. Then $\widetilde I$ is a flasque sheaf on $\Spec A$.
\end{proposition}
\begin{proof}
	This proof is somewhat annoying, so we omit and refer to \cite[Proposition~III.3.4]{hartshorne}. The main idea is to do Noetherian induction on $\op{Supp}\widetilde I$.
\end{proof}
\begin{proposition} \label{prop:qchoh-affine-acyclic}
	Fix a Noetherian ring $A$. Then quasicoherent sheaves on $X\coloneqq\Spec A$ are acyclic.
\end{proposition}
\begin{proof}
	Fix an $A$-module $M$, and we want to show that $\widetilde M$ is acyclic. Well, fix an injective resolution $0\to M\to I^\bullet$, which by \Cref{prop:inj-to-flasque-qcho} produces an acyclic resolution
	\[0\to\widetilde M\to\widetilde I^\bullet.\]
	\Cref{lem:flasque-is-acyclic} followed by \Cref{prop:acyclic-res} allows us to compute cohomology using this resolution, but this is just
	\[0\to M\to I^\bullet\]
	upon taking global sections, which is exact, so the cohomology of $\widetilde M$ must vanish.
\end{proof}
Thus, we see quasicoherent sheaves on affine schemes are well-behaved. This turns out to characterize affine schemes.
\begin{theorem}[Serre]
	Fix a Noetherian scheme $X$. Then the following are equivalent.
	\begin{listroman}
		\item $X$ is affine.
		\item $H^i(X,\mc F)=0$ for all quasicoherent sheaves $\mc F$ on $X$ and indices $i>0$.
		\item $H^1(X,\mc I)=0$ for all quasicoherent sheaves $\mc I$ of ideals on $X$.
	\end{listroman}
\end{theorem}
\begin{proof}
	Note (i) implies (ii) is \Cref{prop:qchoh-affine-acyclic}, and (ii) implies (iii) with no content. So the main content of the argument is (iii) implies (i). We proceed in steps.
	\begin{enumerate}
		\item To set ourselves up, we recall \cite[Exercise~II.2.17]{hartshorne}, which is on the homework, which asserts that $X$ is affine if and only if there is a finite set $\{f_1,\ldots,f_r\}$ of global sections in $A\coloneqq\Gamma(X,\OO_X)$ such that the open subschemes
		\[X_{f_i}\coloneqq\{x\in X:(f_i)_x\notin\mf m_x\}\]
		are affine and cover $X$. (Note that covering $X$ is equivalent to the ideal $(f_1,\ldots,f_r)$ localizing to the full local ring $\OO_{X,x}$ at each $x\in X$, so this is equivalent to $(f_1,\ldots,f_r)=A$.)

		\item We claim that all closed points $p\in X$ have some $f\in A$ such that $X_f$ is affine and $p\in X_f$. Well, let $U\subseteq X$ be some affine open neighborhood of $p\in X$, and let $Y\coloneqq X\setminus U$. Note that we have a short exact sequence
		\[0\to\mc I_{Y\cup\{p\}}\to\mc I_Y\to k(p)\to0\]
		of sheaves on $X$, where $\mc I_\bullet$ is the ideal sheaf of a closed subscheme, and $k(p)$ refers to the skyscraper sheaf at $p$. Notably exactness of this sequence can be checked on stalks; more explicitly, the left map is just an isomorphism on $X\setminus\{p\}$, and on $\{p\}$ this is $0\to\mf m_p\to\OO_{X,p}\to k(p)\to0$, which is exact because $p$ is closed (!). Now, by (iii), we have an exact sequence
		\[H^0(X,\mc I_Y)\to H^0(X, k(p))\to\underbrace{H^1(X,\mc I_{Y\cup\{p\}})}_0,\]
		so we can find some $f\in\Gamma(X,\mc I_Y)$ such that $f\notin\mf m_p$ by this surjectivity, so $p\notin X_f$ by construction. But now $f\in\mc I_Y$ means that $Y\cap X_f\ne\emp$, so $X_f\subseteq U$, and in fact, we see $U_f\subseteq X_f\subseteq U_f$, so $X_f$ is affine (because $U$ is affine means $U_f$ is affine), so we have completed the proof of the claim.

		\item We now need to produce lots of closed points on $X$. For this, we claim that any quasicompact scheme $X$ has a closed point. Well, fix a finite affine open cover $\{U_i\}_{i=1}^n$ of $X$, doable because $X$ is quasicompact, and we may suppose that no affine open subset is covered by the union of the other ones (for then we could remove this open subset from the finite collection). Then we see that
		\[U_1\setminus(U_2\cup U_3\cup\cdots\cup U_n)\]
		is a closed nonempty subset of $U_1$, so it has a closed point $p\in U_1$ corresponding to a maximal ideal of $\Gamma(U_1,\OO_{U_1})$ which contains the ideal cut out by the complement of $(U_2\cup\cdots\cup U_n)$. We claim that $p$ is still closed in $X$.
		
		Well,
		\[X\setminus\{p\}=\bigcup_{i=1}^n(U_i\setminus\{p\})=(U_1\setminus\{p\})\cup\bigcup_{i=2}^nU_i\]
		is open, so we are okay.
	\end{enumerate}
	Anyway, we will complete the proof next class.
\end{proof}

\end{document}