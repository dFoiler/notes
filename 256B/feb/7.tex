% !TEX root = ../notes.tex

\documentclass[../notes.tex]{subfiles}

\begin{document}

Today we will compute cohomology on affine schemes.

\subsection{Cohomology on Affine Schemes}
To build our cohomology on $\Spec A$, we pick up the following checks.
\begin{proposition} \label{prop:inj-to-flasque-qcoh}
	Fix an injective $A$-module $I$, where $A$ is Noetherian. Then $\widetilde I$ is a flasque sheaf on $\Spec A$.
\end{proposition}
\begin{proof}
	This proof is somewhat annoying, so we omit and refer to \cite[Proposition~III.3.4]{hartshorne}. The main idea is to do Noetherian induction on $\op{Supp}\widetilde I$.
\end{proof}
\begin{proposition} \label{prop:qchoh-affine-acyclic}
	Fix a Noetherian ring $A$. Then quasicoherent sheaves on $X\coloneqq\Spec A$ are acyclic.
\end{proposition}
\begin{proof}
	Fix an $A$-module $M$, and we want to show that $\widetilde M$ is acyclic. Well, fix an injective resolution $0\to M\to I^\bullet$, which by \Cref{prop:inj-to-flasque-qcoh} produces an acyclic resolution
	\[0\to\widetilde M\to\widetilde I^\bullet.\]
	\Cref{lem:flasque-is-acyclic} followed by \Cref{prop:acyclic-res} allows us to compute cohomology using this resolution, but this is just
	\[0\to M\to I^\bullet\]
	upon taking global sections, which is exact, so the cohomology of $\widetilde M$ must vanish.
\end{proof}
Thus, we see quasicoherent sheaves on affine schemes are well-behaved. This turns out to characterize affine schemes.
\begin{theorem}[Serre] \label{thm:serre}
	Fix a Noetherian scheme $X$. Then the following are equivalent.
	\begin{listroman}
		\item $X$ is affine.
		\item $H^i(X,\mc F)=0$ for all quasicoherent sheaves $\mc F$ on $X$ and indices $i>0$.
		\item $H^1(X,\mc I)=0$ for all quasicoherent sheaves $\mc I$ of ideals on $X$.
	\end{listroman}
\end{theorem}
\begin{proof}
	Note (i) implies (ii) is \Cref{prop:qchoh-affine-acyclic}, and (ii) implies (iii) with no content. So the main content of the argument is (iii) implies (i). We proceed in steps.
	\begin{enumerate}
		\item To set ourselves up, we recall \cite[Exercise~II.2.17]{hartshorne}, which is on the homework, which asserts that $X$ is affine if and only if there is a finite set $\{f_1,\ldots,f_r\}$ of global sections generating $A\coloneqq\Gamma(X,\OO_X)$ such that the open subschemes
		\[X_{f_i}\coloneqq\{x\in X:(f_i)_x\notin\mf m_x\}\]
		are affine.

		\item We claim that all closed points $p\in X$ have some $f\in A$ such that $X_f$ is affine and $p\in X_f$. Well, let $U\subseteq X$ be some affine open neighborhood of $p\in X$, and let $Y\coloneqq X\setminus U$. Note that we have a short exact sequence
		\[0\to\mc I_{Y\cup\{p\}}\to\mc I_Y\to k(p)\to0\]
		of sheaves on $X$, where $\mc I_\bullet$ is the ideal sheaf of a closed subscheme, and $k(p)$ refers to the skyscraper sheaf at $p$. Notably exactness of this sequence can be checked on stalks; more explicitly, the left map is just an isomorphism on $X\setminus\{p\}$, and on $\{p\}$ this is $0\to\mf m_p\to\OO_{X,p}\to k(p)\to0$, which is exact because $p$ is closed (!). Now, by (iii), we have an exact sequence
		\[H^0(X,\mc I_Y)\to H^0(X, k(p))\to\underbrace{H^1(X,\mc I_{Y\cup\{p\}})}_0,\]
		so we can find some $f\in\Gamma(X,\mc I_Y)$ such that $f\notin\mf m_p$ by this surjectivity, so $p\notin X_f$ by construction. But now $f\in\mc I_Y$ means that $Y\cap X_f\ne\emp$, so $X_f\subseteq U$, and in fact, we see $U_f\subseteq X_f\subseteq U_f$, so $X_f$ is affine (because $U$ is affine means $U_f$ is affine), so we have completed the proof of the claim.

		\item We now need to produce lots of closed points on $X$. For this, we claim that any quasicompact scheme $X$ has a closed point. Well, fix a finite affine open cover $\{U_i\}_{i=1}^n$ of $X$, doable because $X$ is quasicompact, and we may suppose that no affine open subset is covered by the union of the other ones (for then we could remove this open subset from the finite collection). Then we see that
		\[U_1\setminus(U_2\cup U_3\cup\cdots\cup U_n)\]
		is a closed nonempty subset of $U_1$, so it has a closed point $p\in U_1$ corresponding to a maximal ideal of $\Gamma(U_1,\OO_{U_1})$ which contains the ideal cut out by the complement of $(U_2\cup\cdots\cup U_n)$. We claim that $p$ is still closed in $X$.
		
		Well,
		\[X\setminus\{p\}=\bigcup_{i=1}^n(U_i\setminus\{p\})=(U_1\setminus\{p\})\cup\bigcup_{i=2}^nU_i\]
		is open, so we are okay.

		\item We exhibit a finite set $\{f_1,\ldots,f_r\}\subseteq A$ of global sections such that $X_{f_\bullet}$ are affine and cover $X$. Well, let $X_{\mathrm{cl}}$ denote the set of closed points of $X$, and the second step shows that each $p\in X_{\mathrm{cl}}$ has some $f_p\in A$ such that $p\in X_{f_p}$ and $X_{f_p}$ is affine.
		
		We want $\{X_{f_p}\}_{p\in X_{\mathrm{cl}}}$ to cover $X$, so we consider
		\[Z\coloneqq X\mathbin{\bigg\backslash}\bigcup_{p\in X_{\mathrm{cl}}}X_{f_p},\]
		which is a closed subset of $X$, so we give $Z$ the reduced subscheme structure. Well, suppose for the sake of contradiction that $Z$ is nonempty. Because $X$ is quasicompact, so is the closed subset $Z$, so $Z$ has a closed point $p\in Z$. But then $p$ is still closed in $X$: we know $Z\setminus\{p\}$ is open in $X$, so there is an open $U\subseteq X$ such that $Z\setminus\{p\}=Z\cap U$, so $X\setminus\{p\}=(X\setminus Z)\cup U$. This is a contradiction because $p\in X_{f_p}$ and so cannot be in $Z$.

		We are now done: $X$ is quasicompact, so we can reduce $\{X_{f_p}\}_{p\in X_{\mathrm{cl}}}$ to a finite subcover, which completes this step and hence the proof.

		\item We complete the proof. In particular, to plug into the first step, fix $\{f_1,\ldots,f_r\}$ as in the previous step, and we must show that $(f_1,\ldots,f_r)=A$.
		
		We want to show that the map $\alpha\colon\OO_X^{\oplus r}\to\OO_X$ given by $(a_1,\ldots,a_r)\mapsto(a_1f_1+\cdots+a_rf_r)$ is surjective on global sections. Certainly $\alpha$ is surjective on stalks: any $x\in X$ can be placed in some $X_{f_r}$, and then $f_r\colon\OO_{X,x}\to\OO_{X,x}$ is already an isomorphism. But now $\mc K\coloneqq\ker\alpha$ produces the exact sequence
		\[\Gamma\left(X,\OO_X^{\oplus r}\right)\stackrel\alpha\to\Gamma(X,\OO_X)\to H^1(X,\mc K).\]
		To complete this step, we would like to know that $H^1(X,\mc K)=0$.
		
		Namely, we will claim that $H^1\left(X,\mc K\cap\OO_X^{\oplus i}\right)=0$ for each $i\in\{0,\ldots,r\}$ by induction. There is nothing to say for $i=0$. Then if $H^1(X,\mc F\cap\OO_X^{\oplus(i-1)})=0$, we have an exact sequence
		\[0\to\mc K\cap\OO_X^{\oplus(i-1)}\to\mc K\cap\OO_X^{\oplus i}\to\mc Q_i\to0,\]
		where $\mc Q_i$ is the needed sheaf. One can see that $\mc Q_i$ is a subsheaf of $\OO_X$ (namely, it is an ideal sheaf) by applying the Snake lemma to the following diagram.
		% https://q.uiver.app/#q=WzAsMTAsWzAsMCwiMCJdLFs0LDAsIjAiXSxbMSwwLCJcXG1jIEtcXGNhcFxcT09fWF57XFxvcGx1cyhpLTEpfSJdLFsyLDAsIlxcbWMgS1xcY2FwXFxPT19YXntcXG9wbHVzIGl9Il0sWzMsMCwiXFxtYyBRX2kiXSxbMCwxLCIwIl0sWzEsMSwiXFxPT19YXntcXG9wbHVzKGktMSl9Il0sWzIsMSwiXFxPT19YXntcXG9wbHVzIGl9Il0sWzMsMSwiXFxPT19YIl0sWzQsMSwiMCJdLFswLDJdLFsyLDNdLFszLDRdLFs0LDFdLFs1LDZdLFs2LDddLFs3LDhdLFs4LDldLFsyLDYsIiIsMSx7InN0eWxlIjp7InRhaWwiOnsibmFtZSI6Imhvb2siLCJzaWRlIjoidG9wIn19fV0sWzMsNywiIiwxLHsic3R5bGUiOnsidGFpbCI6eyJuYW1lIjoiaG9vayIsInNpZGUiOiJ0b3AifX19XSxbNCw4XV0=&macro_url=https%3A%2F%2Fraw.githubusercontent.com%2FdFoiler%2Fnotes%2Fmaster%2Fnir.tex
		\[\begin{tikzcd}
			0 & {\mc K\cap\OO_X^{\oplus(i-1)}} & {\mc K\cap\OO_X^{\oplus i}} & {\mc Q_i} & 0 \\
			0 & {\OO_X^{\oplus(i-1)}} & {\OO_X^{\oplus i}} & {\OO_X} & 0
			\arrow[from=1-1, to=1-2]
			\arrow[from=1-2, to=1-3]
			\arrow[from=1-3, to=1-4]
			\arrow[from=1-4, to=1-5]
			\arrow[from=2-1, to=2-2]
			\arrow[from=2-2, to=2-3]
			\arrow[from=2-3, to=2-4]
			\arrow[from=2-4, to=2-5]
			\arrow[hook, from=1-2, to=2-2]
			\arrow[hook, from=1-3, to=2-3]
			\arrow[from=1-4, to=2-4]
		\end{tikzcd}\]
		Thus, $H^1(X,\mc Q_i)=0$ by our hypothesis (iii), so the long exact sequence implies that $H^1\left(X,\mc K\cap\OO_X^{\oplus i}\right)=0$, where we are now using the inductive hypothesis.
		\qedhere
	\end{enumerate}
\end{proof}

\end{document}