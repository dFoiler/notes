% !TEX root = ../notes.tex

\documentclass[../notes.tex]{subfiles}

\begin{document}

Here we go.

\subsection{Grothendieck Vanishing}
Last class we began the proof of the following result, which I have moved to today because it will be our focus for today.
\begin{theorem}[Grothendieck vanishing] \label{thm:gr-van}
	Fix a Noetherian topological space $X$ of dimension $n$. Given $\mc F\in\mathrm{Ab}(X)$, we have $H^i(X,\mc F)=0$ for $i>n$.
\end{theorem}
\begin{proof}
	We proceed by induction on the collection of pairs $(n,m)\in\{(-1,0)\}\cup\NN\times\ZZ^+$, where $n=\dim X$ and $m$ is the number of irreducible components of $X$. For our induction, we order $\{(-1,0)\}\cup\NN\times\ZZ^+$ lexicographically; here $\dim\emp=-1$. In other words, we will induct on the dimension, and within that induction, we will induct on the number of irreducible components. %Notably, for $(n,m)=(-1,0)$, we have $X=\emp$, and there is nothing to do.

	Anyway, we proceed in steps.
	\begin{enumerate}
		\item We begin by reducing to $X$ being irreducible; fix $X$ of dimension $n$, and assume all lower results (lower dimension, fewer irreducible components if of dimension $n$). We may assume that $X$ is nonempty, so choose an irreducible component $Z\subseteq X$, and set $U\coloneqq X\setminus Z$. Notably, for this paragraph (making the reduction), we are assuming the statement for $Z$, so any sheaf $\mc F\in\mathrm{Ab}(X)$ has the exact sequence
		\[0\to\mc F_U\to\mc F\to\mc F_Z\to0\]
		by \Cref{rem:ses-res-sheaf}. By the long exact sequence, it will be enough to show that $H^i(X,\mc F_Z)=H^i(X,\mc F_U)=0$ for all $i>n$. For one, note that $H^i(X,\mc F_Z)=0$ for all $i>n$ because $H^i(X,\mc F_Z)=H^i(Z,\mc F|_Z)$ by \Cref{lem:closed-cohomology}, and we have assumed the conclusion for $Z$.
		
		So it remains to discuss $H^\bullet(X,\mc F_U)$. We begin by claiming that there is a sheaf $\mc G$ on $\ov U$ such that $\mc F_U=j_*\mc G$ where $j\colon\ov U\into X$ is the inclusion. Indeed, \Cref{rem:ses-res-sheaf} provides an exact sequence
		\[0\to(\mc F_U)_{X\setminus\ov U}\to\mc F_U\to(\mc F_U)_{\ov U}\to0,\]
		but $(\mc F_U)_{X\setminus\ov U}=0$ by computing stalks: any nonzero stalk must have $p\in X\setminus\ov U$ and $(\mc F|_U)_p\ne0$ and hence $p\in U$ also, but no such $p$ suffices. Thus, we see
		\[\mc F_U\cong(\mc F_U)_{\ov U}\cong j_*(\mc F_U|_{\ov U}),\]
		so $\mc G\coloneqq\mc F_U|_{\ov U}$ will do the trick.

		We are now ready to show that $H^i(X,\mc F_U)=0$ for $i>n$. Well, by the previous paragraph, we see
		\[H^i(X,\mc F_U)=H^i(X,j_*(\mc F_U|_{\ov U}))=H^i(\ov U,\mc F_U|_{\ov U})\]
		by \Cref{lem:closed-cohomology}. But now, by inductive hypothesis, this vanishes for $i<n$ because $\ov U$ has one fewer irreducible component than $X$ and no higher dimension.
		
		\item We handle some base cases. When $\dim X=-1$, we have $X=\emp$, where there is nothing to do. We will also handle $\dim X=0$ while we're here. The previous step allows us to assume that $X$ is irreducible.

		Quickly, we remark that the only closed subsets of $X$ are $\{\emp,X\}$. Indeed, of course these sets are closed. Conversely, if $Z\subseteq X$ is a minimally closed subset, then minimality forces $Z$ to be irreducible, but then $\emp\subseteq Z\subseteq X$ requires $Z\in\{\emp,X\}$ because $\dim X=0$.

		Thus, the previous paragraph implies that $X$ has the indiscrete topology. In particular, all sheaves are flasque because evaluating a sheaf on $\emp$ makes a single point, so $H^i(X,-)$ vanishes for $i>0$, as needed.

		\item Fix a sheaf $\mc F$ of abelian groups on $X$; we need $H^i(X,\mc F)=0$ for $i>\dim X$. We reduce to the case where $\mc F$ is finitely generated (as a sheaf---namely, there are finitely many sections such that the restrictions of those sections generate $\mc F(U)$ for all open $U\subseteq X$). Well, define the set
		\[B\coloneqq\bigcup_{\text{open }U\subseteq X}\mc F(U),\]
		and let $A$ denote the collection of finite subsets. Notably, by restriction, $A$ becomes a set which is directed by the collection of open sets on $X$. Anyway, for $\alpha\in A$, let $\mc F_\alpha$ denote the sheaf generated by the sections in $\alpha$, and we conclude by noting that $\mc F=\colimit\mc F_\alpha$, so
		\[H^i(X,\mc F)=\colimit H^i(X,\mc F_\alpha)\]
		by \Cref{prop:cohomology-directed-colimit}, which vanishes for $i>\dim X$ by assumption of this step.

		\item Fix a finitely generated sheaf $\mc F$ of abelian groups on $X$; we need $H^i(X,\mc F)=0$ for $i>\dim X$. We reduce to the case where $\mc F$ is generated by a single section (and its restrictions). Indeed, assuming we have the case of generated by a single section, we may proceed by induction: if $\mc F$ is generated by $n$ sections $\alpha$ (so that $\mc F=\mc F_\alpha$), let $\alpha'\subsetneq\alpha$ be a proper subset of sections, and let $\mc F_{\alpha'}$ be the sheaf generated by $\alpha'$. Then we have the exact sequence
		\[0\to\mc F_{\alpha'}\to\mc F_\alpha\to\mc G\to0,\]
		and we note that $\mc G$ is generated by fewer than $n$ elements; explicitly, $\mc G$ is generated by the images of the sections in $\alpha\setminus\alpha'$. To be explicit, one can see that $\mc F_{\alpha\setminus\alpha'}\onto\mc G$ by checking on stalks. Thus, $H^i(X,\mc F_{\alpha'})=H^i(X,\mc G)=0$ for $i>\dim X$ by assumption, so the long exact sequence enforces $H^i(X,\mc F)=0$ for $i>\dim X$.

		\item Fix a sheaf $\mc F$ of abelian groups on $X$ generated by a single section $s\in\mc F(U)$ where $U\subseteq X$ is open; we need $H^i(X,\mc F)=0$ for $i>\dim X$. We reduce to the case of subsheaves of $\underline{\ZZ}_U$.

		Well, we may assume that $U$ is nonempty (or else $\mc F=0$, and there is nothing to be done). Now, there is a map $\underline{\ZZ}_U\to\mc F$ given by sending $1\mapsto s$ on $U$ (working on the presheaf) and then appropriately restricting elsewhere. This map is surjective by hypothesis on $\mc F$ (indeed, it is surjective on the level of the presheaves), so we let $\mc K$ denote the kernel, providing the short exact sequence
		\[0\to\mc K\to\underline{\ZZ}_U\to\mc F\to0.\]
		Now, $H^i(X,\mc K)=H^i(X,\underline{\ZZ}_U)=0$ for $i>\dim X$ by assumption of this section, so $H^i(X,\mc F)=0$ for $i>\dim X$ by the long exact sequence.

		\item In this reduction step, we will use that $X$ is irreducible. Fix a subsheaf $\mc F$ of $\underline{\ZZ}_U$ for open $U\subseteq X$; we need $H^i(X,\mc F)=0$ for $i>\dim X$. We reduce to the case $\mc F=\underline{\ZZ}_U$. If $\mc F=0$, there is nothing to do.

		Otherwise, we may let $d$ be the smallest positive integer such that $d\in\mc F_x$ as $x\in U$ varies (notably, some $\mc F_x$ is nonzero, so a $d$ exists). Now,
		\[V\coloneqq\{x\in U:d\in\mc F_x\}\]
		is nonempty and open ($d\in\mc F_x$ means that $d\in\mc F(U')$ for some open $U'\subseteq U$), and $\mc F_x=d\ZZ$ for each $x\in V$, so $\mc F|_V=d\underline{\ZZ}$, so we have an equality $\mc F_V=d\underline{\ZZ}_V$. So we have an exact sequence
		\[0\to d\underline{\ZZ}_V\to\mc F\to\mc F/d\underline{\ZZ}_V\to0\]
		of sheaves on $X$. By assumption, we have the result for $d\underline{\ZZ}_V$. Now, $\mc F/d\underline{\ZZ}_V$ is supported on $\ov{U\setminus V}$ by construction of $V$, and $\ov{U\setminus V}$ will have smaller dimension than $X$ because $X$ is irreducible, so we get the result for $\mc F/d\underline{\ZZ}_V$ by the inductive hypothesis on $X$. So the long exact sequence purchases the result for $\mc F$.

		\item We complete the induction. We may assume that $X$ is irreducible of dimension $n$, and we may assume that $\mc F=\underline{\ZZ}_U$. Well, we have an exact sequence
		\[0\to\underline{\ZZ}_U\to\underline\ZZ\to\underline{\ZZ}_{X\setminus U}\]
		by \Cref{rem:ses-res-sheaf}. Because $X$ is irreducible, $X\setminus U$ has smaller dimension, $H^i(X,\underline{\ZZ}_{X\setminus U})=0$ for $i>\dim X-1$ (also using \Cref{lem:closed-cohomology}). Additionally, $\underline{\ZZ}$ is flasque because $X$ is irreducible---all open subsets of $X$ are connected, so $\underline{\ZZ}(U)=\ZZ$ always---and hence acyclic by \Cref{lem:flasque-is-acyclic}, so we conclude $H^i(X,\underline{\ZZ}_U)=0$ for $i>\dim X$ by the long exact sequence.
		\qedhere
	\end{enumerate}
\end{proof}
While we're here, let's do an example computation to show \Cref{thm:gr-van} is sharp.
\begin{exe}
	Fix a field $k$, and set $X\coloneqq\AA^1_k$. Given distinct points $P,Q\in X$, set $U\coloneqq X\setminus\{P,Q\}$, and we see
	\[H^1(X,\underline{\ZZ}_U)\ne0.\]
\end{exe}
\begin{proof}
	Let $j\colon U\into X$ denote the inclusion so that $\underline{\ZZ}_U=j_!(\underline\ZZ)$. Note that $U$ is irreducible, so any open subsets are connected, so we may as well have
	\[\underline{\ZZ}_U(V)=\begin{cases}
		\ZZ & \text{if }V\subseteq U\text{ and }V\ne\emp, \\
		0 & \text{otherwise}.
	\end{cases}\]
	Anyway, note that we have the exact sequence
	\[0\to\underline{\ZZ}_U\to\underline\ZZ\to\underline\ZZ_{\{P,Q\}}\to0\]
	by \Cref{rem:ses-res-sheaf}, so we get a long exact sequence
	\[H^0(X,\underline\ZZ_U)\to H^0(X,\underline\ZZ)\to H^0(X,\underline\ZZ_{\{P,Q\}})\to H^1(X,\underline{\ZZ}_U)\to H^1(X,\underline{\ZZ}).\]
	Because $X$ is irreducible, we see that $\underline{\ZZ}$ is flasque (all open subsets are connected), so the rightmost term vanishes by \Cref{lem:flasque-is-acyclic}. Also, above we noted that $H^0(X,\underline\ZZ_U)=0$, and computing global sections on the other sheaves implies that we have
	\[0\to\ZZ\to\ZZ\oplus\ZZ\to H^1(X,\underline{\ZZ}_U)\to0,\]
	so the rightmost map cannot be surjective.
\end{proof}

\end{document}