% !TEX root = ../notes.tex

\documentclass[../notes.tex]{subfiles}

\begin{document}

\section{February 22}

Another homework will be assigned probably later today.

\subsection{Universal \texorpdfstring{$\delta$}{ Delta}-Functors}
Quickly, we state a useful abstract nonsense result.
\begin{definition}[effaceable]
	Fix an additive functor $F\colon\mc A\to\mc B$ of abelian categories. Then $F$ is \textit{effaceable} if and only if each $A\in\mc A$ has some monomorphism $u\colon A\into M$ such that $F(u)=0$. We define \textit{coeffaceable} dually.
\end{definition}
Of course, no left-exact functor will be effaceable, but we do have the following.
\begin{theorem}[Grothendeick]
	Fix a $\delta$-functor $\left\{T^i,\delta^i\right\}_{i\ge0}\colon\mc A\to\mc B$ between abelian categories. If each $T^i$ is effaceable, then $\left\{T^i,\delta^i\right\}_{i\ge0}$ is universal.
\end{theorem}
The point here is that we can show two $\delta$-functors are the same by showing that one is universal and that the other is effaceable.

\subsection{Sheaf Cohomology}
On the homework, we established the following results.
\begin{theorem}
	Fix a ring $A$. Then $\mathrm{Mod}_A$ has enough injectives.
\end{theorem}
\begin{theorem} \label{thm:mod-ox-enough-inj}
	Fix a ringed space $(X,\OO_X)$. Then $\mathrm{Mod}_{\OO_X}$ has enough injectives.
\end{theorem}
\begin{corollary}
	Fix a topological space $X$. Then the category $\mathrm{Ab}_X$ of abelian sheaves on $X$ has enough injectives.
\end{corollary}
\begin{proof}
	Turn $X$ into a ringed space $(X,\OO_X)$ by $\OO_X=\underline\ZZ$. Then \Cref{thm:mod-ox-enough-inj} completes the proof.
\end{proof}
In particular, we can use our abstract nonsense on the left-exact functor $\Gamma(X,-)\colon\mathrm{Mod}_{\OO_X}\to\mathrm{Ab}$. So we may define sheaf cohomology.
\begin{definition}
	Fix a topological space. The right-derived functors of the functor $\Gamma(X,-)\colon\mathrm{Ab}_X\to\mathrm{Ab}$ exist and are denoted by $H^i(X,-)$.
\end{definition}
\begin{remark}
	If $\mc F$ is also a sheaf of $\OO_X$-modules, then notably our abstract nonsense is forgetting about this extra structure. Nonetheless, we will shortly see that these cohomology groups are the same.
\end{remark}
We are going to want to compute some cohomology groups. It will be helpful to be able to more easily build some acyclic sheaves.
\begin{definition}[flasque]
	Fix a topological space $X$. A sheaf $\mc F$ on $X$ is \textit{flasque} if and only if the restriction maps $\mc F(U)\to\mc F(V)$ are always surjective.
\end{definition}
Roughly speaking, it will turn out that these flasque sheaves are acyclic for our cohomology. Here is a quick result which suggests something along these lines.
\begin{proposition}
	Fix a ringed space $(X,\OO_X)$. Then every injective $\OO_X$-module $\mc F$ is flasque.
\end{proposition}
\begin{proof}
	Fix an open subset $i\colon U\subseteq X$. Then we note $\OO_U$ is $i_!i^*\OO_X=i_!\OO_U$. The point is that $\op{Hom}_X(\OO_U,\mc F)=\op{Hom}_U(\OO_U,\mc F|_U)$, which is $\mc F(U)$ because $\mc F$ is an $\mathcal O_U$-module. However, for any other open subset $j\colon V\subseteq U$, then we note our embedding $\OO_V=j_!j^*\OO_U\into\OO_U$ allows us to push forward a section $s\in\mc F(V)$ inducing a morphism $\OO_V\to\mc F$ to a section $s'\in\mc F(U)$ inducing a morphism $\OO_U\to\mc F$ by the injectivity of $\mc F$. Unwinding, we see that our restriction map $\mc F(U)\to\mc F(V)$ is surjective.\todo{What?}
\end{proof}
We quickly note the following properties of flasque sheaves.
\begin{lemma}
	Fix an exact sequence
	\[0\to\mc F'\to\mc F\to\mc F''\to0\]
	of sheaves on a topological space $X$. If $\mc F'$ and $\mc F$ is flasque, then so is $\mc F''$ is flasque. Additionally, if $\mc F'$ is flasque, then
	\[0\to\Gamma(U,\mc F'0\to\Gamma(U,\mc F)\to\Gamma(U,\mc F'')\to0\]
	is exact for any open $U\subseteq X$.
\end{lemma}
\begin{proof}
	We showed these last class. Notably, the first statement follows from the second by hitting ourselves with the Five lemma on the following diagram.
	% https://q.uiver.app/?q=WzAsMTAsWzAsMCwiMCJdLFsxLDAsIlxcR2FtbWEoXFxtYyBGJyxVKSJdLFsyLDAsIlxcR2FtbWEoXFxtYyBGLFUpIl0sWzMsMCwiXFxHYW1tYShcXG1jIEYnJyxVKSJdLFs0LDAsIjAiXSxbMSwxLCJcXEdhbW1hKFxcbWMgRicsVikiXSxbMiwxLCJcXEdhbW1hKFxcbWMgRixWKSJdLFszLDEsIlxcR2FtbWEoXFxtYyBGJycsVikiXSxbNCwxLCIwIl0sWzAsMSwiMCJdLFswLDFdLFsxLDJdLFsyLDNdLFszLDRdLFs5LDVdLFs1LDZdLFs2LDddLFs3LDhdLFsxLDUsIiIsMSx7InN0eWxlIjp7ImhlYWQiOnsibmFtZSI6ImVwaSJ9fX1dLFsyLDYsIiIsMSx7InN0eWxlIjp7ImhlYWQiOnsibmFtZSI6ImVwaSJ9fX1dLFszLDddXQ==&macro_url=https%3A%2F%2Fraw.githubusercontent.com%2FdFoiler%2Fnotes%2Fmaster%2Fnir.tex
	\[\begin{tikzcd}
		0 & {\Gamma(\mc F',U)} & {\Gamma(\mc F,U)} & {\Gamma(\mc F'',U)} & 0 \\
		0 & {\Gamma(\mc F',V)} & {\Gamma(\mc F,V)} & {\Gamma(\mc F'',V)} & 0
		\arrow[from=1-1, to=1-2]
		\arrow[from=1-2, to=1-3]
		\arrow[from=1-3, to=1-4]
		\arrow[from=1-4, to=1-5]
		\arrow[from=2-1, to=2-2]
		\arrow[from=2-2, to=2-3]
		\arrow[from=2-3, to=2-4]
		\arrow[from=2-4, to=2-5]
		\arrow[two heads, from=1-2, to=2-2]
		\arrow[two heads, from=1-3, to=2-3]
		\arrow[from=1-4, to=2-4]
	\end{tikzcd}\]
	Indeed, the left two morphisms being surjective forces the right morphism to also be surjective.
\end{proof}
\begin{proposition}
	Fix a flasque sheaf $\mc F$ of abelian groups on a topological space $X$. Then $\mc F$ is acyclic for the functor $\Gamma(X,-)$.
\end{proposition}
\begin{proof}
	Embed $\mc F$ into an injective sheaf $\mc I$. Taking quotients, we get an exact sequence
	\[0\to\mc F\to\mc I\to\mc G\to0,\]
	so the previous lemma does tell us that $\mc G$ must be flasque. We now finish by dimension-shifting: by the long exact sequence, we see that $H^i(X,\mc G)=H^{i+1}(X,\mc F)$. Furthermore, our long exact sequence has
	\[\Gamma(X,\mc I)\onto\Gamma(X,\mc G)\to H^1(X,\mc F)\to\underbrace{H^1(X,\mc I)}_0,\]
	so we see $H^1(X,\mc F)=0$ follows. But then $H^1(X,\mc G)=0$ because $\mc G$ is flasque, so $H^2(X,\mc F)=0$ also, and we can continue our induction upwards.
\end{proof}
\begin{corollary}
	The right-derived functors of $\Gamma(X,-)\colon\mathrm{Mod}_{\OO_X}\to\mathrm{Ab}$ coincide with $H^\bullet(X,-)$.
\end{corollary}
\begin{proof}
	Take an injective resolution of some $\OO_X$-module $\mc F$. But this is an acyclic resolution for $\Gamma(X,-)\colon\mathrm{Ab}_X\to\mathrm{Ab}$, so we are done by \Cref{prop:acyclic-res}.
\end{proof}
\begin{remark}
	Notably, if $X$ actually has the structure of a ringed space $(X,\OO_X)$, then deriving with $\Gamma(X,-)\colon\mathrm{Mod}_{\OO_X}\to\mathrm{Ab}$ grants $H^\bullet(X,-)$ a module structure over $\Gamma(X,\OO_X)$. For example, if $X$ is a $k$-variety, this tells us that the $H^\bullet(X,-)$ should be $k$-vector spaces.
\end{remark}

\subsection{Dimension-Bounding Cohomology}
While we're here, we pick up the following vanishing result.
\begin{theorem} \label{thm:dimension-bound-cohom}
	Fix a Noetherian topological space $X$ of dimension $n$. For any abelian sheaf $\mc F$, we have $H^i(X,\mc F)=0$ for $i>n$.
\end{theorem}
We are going to want to pick up a few starting facts.
\begin{lemma} \label{lem:pull-preserves-flasque}
	Fix a continuous map $f\colon X\to Y$ of topological spaces. Given a flasque abelian sheaf $\mc F$ on $X$, then $f_*\mc F$ is flasque on $Y$.
\end{lemma}
\begin{proof}
	For any open subsets $V\subseteq U\subseteq Y$. Then the restriction map $f_*\mc F(U)\to f_*\mc F(V)$ is simply the restriction map $\mc F\left(f^{-1}(U)\right)\to\mc F\left(f^{-1}(V)\right)$, which is surjective because $\mc F$ is flasque.
\end{proof}
\begin{corollary}
	Fix a closed embedding $j\colon Z\to X$ of schemes. Given an abelian sheaf $\mc F$ on $Z$, we have $H^\bullet(X,j_*\mc F)=H^\bullet(Z,\mc F)$.
\end{corollary}
\begin{proof}
	Note that $j\colon Z\to X$ being a closed embedding means that $j_*$ is just extension by zero and is thus exact by checking everything at stalks---in particular, $(j_*\mc F)_x=0$ for $x\notin j(Z)$. However, $j_*$ also preserves flasque sheaves by \Cref{lem:pull-preserves-flasque}, so we can push forward a flasque resolution of $\mc F$ on $Z$ to a flasque resolution of $j_*\mc F$ on $X$.
	
	Computing cohomology with these two resolutions completes the proof. Let's be more explicit: give $\mc F$ an injective resolution as follows
	\[0\to\mc F\to\mc I^0\to\mc I^1\to\mc I^2\to\cdots.\]
	Hitting this with $j_*$ gives us an exact resolution of flasque sheaves
	\[0\to j_*\mc F\to j_*\mc I^0\to j_*\mc I^1\to j_*\mc I^2\to\cdots.\]
	Taking global sections of these two resolutions will give the same complex
	\[0\to\mc I^0(Z)\to\mc I^1(Z)\to\mc I^2(Z)\to\cdots,\]
	so they yield the same cohomology groups.
\end{proof}
We also want to know how cohomology compares with directed colimits so that we can take about stalks. This is the content of the next lemma.
\begin{definition}[directed set]
	A partially ordered set $(\lambda,\le)$ is \textit{directed} if and only if any two $a,b\in\lambda$ have some $c\in\lambda$ such that $a,c\le c$.
\end{definition}
\begin{lemma}
	Fix a Noetherian topological space $X$. If $\{\mc F_\alpha\}_{\alpha\in\lambda}$ is a collection of sheaves indexed by a directed set, then
	\[H^\bullet(X,\colim\mc F)=\colim H^\bullet(X,\mc F_i).\]
\end{lemma}
This result is useful because it more or less bounds the size of $H^\bullet(X,\colim\mc F)$ by the various $H^\bullet(X,\mc F_i)$s. Roughly speaking, we are going to use this result to prove \Cref{thm:dimension-bound-cohom} by replacing $\mc F$ with some kind of directed colimit of finitely generated subsheaves and work there instead.

\end{document}