% !TEX root = ../notes.tex

\documentclass[../notes.tex]{subfiles}

\begin{document}

\section{February 3}

We continue discussing the theory of things which go bump in the night.

\subsection{Line Bundle Review}
We begin by recalling a couple of facts.
\begin{proposition} \label{prop:get-proj-morphism}
	Fix an $A$-scheme $X$, where $A$ is an affine scheme. We recall that a morphism $X\to\PP^n_A$ has equivalent data to giving a line bundle $\mc L$ on $X$ together with global sections $(s_0,\ldots,s_n)$ which generate $\mc L$.
\end{proposition}
Recall that the global sections generate $\mc L$ if and only if the induced map $\OO^{n+1}_X\to\mc L$ is surjective, which means that the global sections generate all the talks of $\mc L$.
\begin{remark}
	If $A=k$ is a field, then defining a morphism $X\to\PP^n_k$ can be defined (up to automorphism on $\PP^n_k$), then it suffices to just provide a globally generated line bundle $\mc L$, and the precise choice of spanning set $(s_0,\ldots,s_n)$ merely adds an automorphism.
\end{remark}
\begin{remark}
	The chosen global sections technically need not fully span $\Gamma(X,\mc L)$.
\end{remark}
\begin{remark}
	We remark that the pullback of the line bundle $\OO_{\PP^n_A}(1)$ under $X\to\PP^1_A$ is (canonically) $\mc L$, and the pullback of the global section $x_i$ is $s_i$. Explicitly, there is morphism $x_i\colon\OO_{\PP^n_A}\to\OO_{\PP^n_A}(1)$ which will pull back to $s_i\colon\OO_X\to\mc L$ upon applying $\varphi^*$.
\end{remark}
We might want to upgrade \Cref{prop:get-proj-morphism} to give a closed embedding into projective space. Here are the corresponding conditions.
\begin{proposition}
	Fix an algebraically closed field $k$ and a $k$-variety $X$. Fix a morphism $\varphi\colon X\to\PP^n_k$ corresponding to the line bundle $\mc L$ equipped with global sections $s_0,\ldots,s_n\in\Gamma(X,\mc L)$. Further, set $V=\op{span}(s_0,\ldots,s_n)$. Then $\varphi$ is a closed immersion if and only if $V$ satisfies the following.
	\begin{itemize}
		\item Separates points: for any distinct $x,x'\in X$, there is a section $s\in V$ such that $x\in\op{supp}\op{div}(\mc L,s)$ but $x'\notin\op{supp}\op{div}(\mc L,s)$ (i.e., $s\in\mf m_x\mc L_x\setminus\mf m_{x'}\mc L_{x'}$).
		\item Separates tangent vectors: for every $x\in X$, the set
		\[\{s\in V:s\in\mf m_x\mc L_x\}\]
		spans the Zariski tangent place $\mf m_x\mc L_x/(\mf m_x\mc L_x)^2$.
	\end{itemize}
\end{proposition}
Notably, when $X$ is a curve, our Zariski tangent space has dimension $1$, so we just want some section to show up in there.

We will also want the following definitions.
\begin{definition}[ver ample]
	A line bundle $\mc L$ on a scheme $X$ is \textit{very ample} relative to a scheme $Y$ if and only if there is a locally closed embedding $\iota\colon X\to\PP^n_Y$ for some $n>0$ such that $\mc L=\iota^*\OO_{\PP^n_Y}(1)$.
\end{definition}
\begin{remark}
	In the case where $X$ is a $k$-curve, because $X$ is proper, any locally closed embedding $\iota\colon X\to\PP^n_k$ is automatically closed. As such, in this situation we may as well talk about closed embeddings.
\end{remark}
\begin{definition}[ample]
	A line bundle $\mc L$ on a scheme $X$ is \textit{ample} if each coherent sheaf $\mc F$ on $X$ makes $\mc F\otimes\mc L^n$ globally generated for $n$ sufficiently large.
\end{definition}
Here is how these notions relate.
\begin{proposition}
	Fix a scheme $X$ of finite type over a Noetherian ring $A$. Then a line bundle $\mc L$ on $X$ is ample if and only if $\mc L^n$ is very ample (relative to $A$) for some $n>0$.
\end{proposition}

\subsection{Projective Embeddings for Curves}
We now return to talk about curves. We quickly extend our definition of ample.
\begin{defihelper}[ample, very ample] \nirindex{ample} \nirindex{very ample}
	Fix a $k$-curve $X$. Then a divisor $D$ on $X$ is \textit{ample} or \textit{very ample} if and only if $\OO_X(D)$ is as well.
\end{defihelper}
So let's translate what we know about projective embeddings into our language of divisors.
\begin{prop} \label{prop:div-to-proj-embed}
	Fix a divisor $D$ on a $k$-curve $X$.
	\begin{listalph}
		\item The complete linear system $|D|$ is base-point-free if and only if
		\[\dim|D|=\dim|D-P|+1\]
		for all closed points $P\in X$.
		\item The divisor $D$ is very ample if and only if
		\[\dim|D|=\dim|D-P-Q|+2\]
		for all $P,Q\in X$.
	\end{listalph}
\end{prop}
Roughly speaking, (a) asks for us to separate points somehow, and (b) asks if we can separate points with the right multiplicity. Namely, we're asking for something to be in $\mf m_P$ but not $\mf m_P^2$.
\begin{proof}
	We have two parts to show.
	\begin{listalph}
		\item Note $\dim|D|=\dim|D-P|+1$ is equivalent to
		\[\ell(D)=\ell(D-P)+1,\]
		which is equivalent to $\OO_X(D)(X)\setminus\OO_X(D-P)(X)$ being nonempty. (Recall that $\dim\OO_X(D)\le\OO_X(D-P)+1$ at the very least.) Thus, we are saying there is $f\in K(X)$ such that $P\in\op{supp}D$ but $P\in\op{supp}(D+(f))$, so we have a global section $f$ which does not vanish at $P$. Repeating this for all points $P$ shows that $|D|$ is base-point-free, and running this argument in reverse gives us the converse implication.

		\item Certainly if $D$ is very ample, then $\mc L(D)$ is base-point-free. Additionally, note that the conclusion of (b) implies the conclusion of (a) because
		\[\dim|D-P-Q|+2\le\dim|D-P|+1\le\dim|D|,\]
		so equalities must hold everywhere and in particular on the right inequality. In particular, we can assume that $|D|$ is base-point-free in either direction.

		As such, in either direction, we already know that $D$ determines a morphism to projective space, so we need to check that we have defined a closed embedding.
		\begin{itemize}
			\item Separate points: for every distinct $P,Q\in X$, some $s\in\Gamma(X,\OO_X(D))$ has $P\in\op{supp}\op{div}(\mc L,s)$ but $Q\in\op{supp}\op{div}(\mc L,s)$ (namely, we vanish at $P$) is equivalent to $\op{div}(\mc L,s)\in\Gamma(X,\OO_X(D-P))$ but $Q$ is not a base-point of $\OO_X(D-P)$ at $s$ (namely, we do not vanish at $Q$). Hitting this with (a) again, we are asking for
			\[\dim|D-P|=\dim|D-P-Q|+1.\]
			We already have (a), so we conclude $\dim|D|=\dim|D-P-Q|+2$. Running this argument in reverse gets the other implication.
			\item Separates tangent vectors: for every $P\in X$, we are asking for $s\in\Gamma(X,\OO_X(D))$ which vanishes at order $1$ at $P$. (Indeed, this is saying $s\in\mf m_P\mc L_P/\mf m_P^2\mc L_P$ is nonzero.) Arguing as above, we are asking for $P$ to not be a base-point for $D-P$. So we can again hit this condition with (a) to say that we are asking for
			\[\dim|D-P|=\dim|D-P-P|+1\]
			and use the fact that $D$ is base-point-free already to finish.
		\end{itemize}
		The above discussion completes the proof.
		\qedhere
	\end{listalph}
\end{proof}
\begin{corollary} \label{cor:big-is-very-ample}
	Fix a divisor $D$ on a $k$-curve $X$ of genus $g$.
	\begin{listalph}
		\item If $\deg D\ge2g$, then $D$ is base-point-free.
		\item If $\deg D\ge2g+1$, then $D$ is very ample.
	\end{listalph}
\end{corollary}
\begin{proof}
	We use \Cref{thm:rr}. Recall $\deg K=2g-2$, where $K$ is the canonical divisor for $X$.
	\begin{listalph}
		\item Note $\deg(K-D)\le0$ and $\deg(K-(D-P))\le0$, so $\ell(K-D)=\ell(K-D-P)=0$, so we conclude that
		\[\ell(D)=\deg D+1-G=1+\deg(D-P)+1-g=1+\ell(D-P),\]
		so we are done by \Cref{prop:div-to-proj-embed}.
		\item Here, we also get $\deg(K-(D-P-Q))\le0$, so $\ell(K-(D-P-Q))=0$ again, so arguing as above completes the proof by \Cref{prop:div-to-proj-embed}.
		\qedhere
	\end{listalph}
\end{proof}
In particular, we see that every $k$-curve $X$ has a closed embedding into projective space with a morphism of degree at most $2g+1$.

\end{document}