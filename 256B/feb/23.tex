% !TEX root = ../notes.tex

\documentclass[../notes.tex]{subfiles}

\begin{document}

Here we go.

\subsection{More on Cohomology on Projective Space}
Today we prove \Cref{thm:cohom-proj-space}.
% \cohomprojthm*
\begin{proof}[Proof of \Cref{thm:cohom-proj-space}]
	As suggested by the remark, our proof of \Cref{thm:cohom-proj-space} will use \v Cech cohomology. It will be helpful to glue everything together into
	\[\mc F\coloneqq\bigoplus_{n\in\ZZ}\OO_X(n),\]
	which is a $\ZZ$-graded quasicoherent sheaf of $S$-modules. Taking cohomology, which commutes with infinite sums because taking infinite sums is exact, we see that
	\[H^i(X,\mc F)=\bigoplus_{n\in\ZZ}H^i(X,\OO_X(n))\]
	for each $i$. Notably, one has an $A$-module structure everywhere by \Cref{rem:cech-module}.

	Now, for our open cover $\mc U$ (for \v Cech cohomology), we take $U_j\coloneqq D_+(x_j)$ for $0\le j\le r$; note then that $U_{i_0,\ldots,i_p}\coloneqq U_{i_0}\cap\cdots\cap U_{i_p}=D_+(x_{i_0}\cdots x_{i_p})$. In particular, $\PP^r_A$ is Noetherian and separated (note $A$ is Noetherian), so
	\[\check H^\bullet(\mc U,\OO_X(n))\cong H^\bullet(X,\OO_X(n))\]
	by \Cref{thm:cech-comparison}. Notably, we find that
	\[\mc F(U_{i_0\cdots i_p})=S_{x_{i_0}\cdots x_{i_p}}\]
	by tracking through the localizations on $\OO_X(n)$, meaning that our \v Cech complex looks like
	\[0\to\prod_{i_0}S_{x_{i_0}}\stackrel{d^0}\to\prod_{i_0,i_1}S_{x_{i_0}x_{i_1}}\stackrel{d^1}\to\cdots\stackrel{d^{r-1}}\to S_{x_0\cdots x_r}\to0\]
	of graded $S$-modules.

	We now proceed with our arguments.
	\begin{enumerate}[label=(\alph*)]
		\item We compute
		\[\check H^0(\mc U,\mc F)=\ker d^0=\bigcap_iS_{x_i}.\]
		Here, the intersections are being taken in $S_{x_0\cdots x_r}$, legal because the relevant localization maps are injective. Now, this last $S$-module is just $S$ because (for example) $S=S_{x_0}\cap S_{x_1}$ by a direct computation.

		\setcounter{enumi}{2}
		\item We begin by claiming that $\check H^r(\mc U,\mc F)$ is the free graded $A$-module with basis given by elements of the form $x_0^{\ell_0}\cdots x_r^{\ell_r}$ where $\ell_i$ are negative integers. Notably, in degree $-r-1$, we are searching for solutions to $\ell_0+\cdots+\ell_r=-r-1$, which is only $\ell_0=\cdots=\ell_r=-1$, so we will have rank $1$, which is (c).

		So it remains to show the claim. Well, looking at our \v Cech complex,
		\[\check H^r(\mc U,\mc F)=\coker\Bigg(\prod_jS_{x_0\cdots\widehat x_j\cdots x_r}\stackrel{d^{r-1}}\to S_{x_0\cdots x_r}\Bigg).\]
		Now, $S_{x_0\cdots x_r}$ is a free $A$-module with basis given by terms of the form $x_0^{\ell_0}\cdots x_r^{\ell_r}$ where $\ell_0,\ldots,\ell_r\in\ZZ$, so we want the cokernel to kill all undesired terms. Well, tracking through $d^{r-1}$, we find that it is just inclusion (up to sign), so the image is the free $A$-module with basis given by terms of the form $x_0^{\ell_0}\cdots x_r^{\ell_r}$ with at least one nonnegative exponent, so the claim follows.

		\item Let's begin by describing the pairing. Given some $s\in H^0(X,\OO_X(n))$ (which is just a global section), we produce a map $s\colon\OO_X(-r-1-n)\to\OO_X(-r-1)$. Moving up to cohomology, we get an $A$-linear map $H^r(X,\OO_X(-r-1-n))\to H^r(X,\OO_X(-r-1))$, which upon letting $s$ vary produces the required bilinear pairing
		\[H^0(X,\OO_X(n))\to\op{Hom}_A(H^r(X,\OO_X(-r-1-n)),H^r(X,\OO_X(-r-1))).\]
		It remains to check that this pairing is perfect. If $n<0$, then both terms in our pairing will vanish, so there will be nothing left to check. For example, $H^0(X,\OO_X(n))=0$ by (a), and $H^r(X,\OO_X(-r-1-n))=0$ by the claim in (c): we know $H^r(X,\OO_X(-r-1-n))=0$ is a free $A$-module with basis given by $x_0^{\ell_0}\cdots x_r^{\ell_r}$ where the $\ell_\bullet$s are negative integers summing to $-r-1-n>-r-1$, but there is no such basis element.

		So we may take $n\ge0$. Tracking through the description of our pairing on \v Cech cohomology, we see that the basis element $x_0^{m_0}\cdots x_r^{m_r}\in H^0(X,\OO_X(n))$ (of total degree $n$) will send the basis element $x_0^{\ell_0}\cdots x_r^{\ell_r}\in H^r(X,\OO_X(-r-1-n))$ to the element $x_0^{m_0+\ell_0}\cdots x_r^{m_r+\ell_r}\in H^r(X,\OO_X(-r-1))$ (which means $0$ if any exponent is nonnegative). Let's explain this. To begin, we need to show that we actually have a well-defined map. On \v Cech cohomology, we are attempting to describe a map
		\[\left(\ker d^0\right)_n\otimes_A\frac{C^r(\mc U,\mc F)_{-r-1-n}}{\left(\im d^{r-1}\right)_{-r-1-n}}\to\frac{C^r(\mc U,\mc F)_{-r-1}}{\left(\im d^{r-1}\right)_{-r-1}}.\]
		We can now see that $\ker d^0=S$, so it does have basis elements in the form $x_0^{m_0}\cdots x_r^{m_r}$ of total degree $n$, and tensoring by this element will indeed send basis elements $x_0^{\ell_0}\cdots x_r^{\ell_r}\in H^r(X,\OO_X(-r-1-n))$ as described. (Notably, we do go to $0$ if any exponent is nonnegative because this is the image of $d^{r-1}$. Perhaps one might also want to note that if we input some element $x_0^{\ell_0}\cdots x_r^{\ell_r}$ with a nonnegative exponent, then the corresponding product will have a nonnegative exponent in the same spot.) Formally, perhaps one should go through the following commutative diagram, as follows.
		% https://q.uiver.app/#q=WzAsNCxbMCwwLCJDXlxcYnVsbGV0KFxcbWMgVSxcXE9PX1goLXItMS1uKSkiXSxbMSwwLCJDXlxcYnVsbGV0KFxcbWMgVSxcXE9PX1goLXItMSkpIl0sWzEsMSwiQ15cXGJ1bGxldChcXG1jIFUsXFxtYyBGKV97LXItMX0iXSxbMCwxLCJDXlxcYnVsbGV0KFxcbWMgVSxcXG1jIEYpX3stci0xLW59Il0sWzAsMywiIiwwLHsibGV2ZWwiOjIsInN0eWxlIjp7ImhlYWQiOnsibmFtZSI6Im5vbmUifX19XSxbMSwyLCIiLDAseyJsZXZlbCI6Miwic3R5bGUiOnsiaGVhZCI6eyJuYW1lIjoibm9uZSJ9fX1dLFszLDIsIiIsMSx7InN0eWxlIjp7ImJvZHkiOnsibmFtZSI6ImRhc2hlZCJ9fX1dLFswLDEsIigtXFxvdGltZXMgcykiXV0=&macro_url=https%3A%2F%2Fraw.githubusercontent.com%2FdFoiler%2Fnotes%2Fmaster%2Fnir.tex
		\[\begin{tikzcd}
			{C^\bullet(\mc U,\OO_X(-r-1-n))} & {C^\bullet(\mc U,\OO_X(-r-1))} \\
			{C^\bullet(\mc U,\mc F)_{-r-1-n}} & {C^\bullet(\mc U,\mc F)_{-r-1}}
			\arrow[Rightarrow, no head, from=1-1, to=2-1]
			\arrow[Rightarrow, no head, from=1-2, to=2-2]
			\arrow[dashed, from=2-1, to=2-2]
			\arrow["{(-\otimes s)}", from=1-1, to=1-2]
		\end{tikzcd}\]
		We will continue the proof next class.
		\qedhere
	\end{enumerate}
\end{proof}
\begin{remark}
	The choice of isomorphism in (c) is notably not canonical. In particular, it depends on our choice of basis element for the cokernel.
\end{remark}

\end{document}