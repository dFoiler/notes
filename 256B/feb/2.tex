% !TEX root = ../notes.tex

\documentclass[../notes.tex]{subfiles}

\begin{document}

Let's just get this over with.

\subsection{More on Directed Colimits}
We continue our discussion towards Grothendieck vanishing. We can now see that directed colimits commutes with cohomology.
\begin{proposition} \label{prop:cohomology-directed-colimit}
	Fix a Noetherian topological space $X$. Given a directed system $\{\mc F_\alpha\}_{\alpha\in\Lambda}$ of sheaves in $\mathrm{Ab}(X)$, there is a natural isomorphism
	\[\colimit H^\bullet(X,\mc F_\alpha)=H^\bullet\left(X,\colimit\mc F_\alpha\right)\]
	compatible in the long exact sequence.
\end{proposition}
\begin{proof}
	For convenience, let $\mc C$ be the category of directed systems in $\mathrm{Ab}(X)$ indexed by $\Lambda$. We would like to exhibit an isomorphism
	\[\colimit H^\bullet(X,-)\simeq H^\bullet\left(X,\colimit-\right)\]
	of $\delta$-functors $\mc C\to\mathrm{Ab}$.

	Quickly, we note that we can take a sheaf $\mc F$ and map it to its ``sheaf of discontinuous sections'' given by
	\[U\mapsto\prod_{x\in U}\mc F_x.\]
	This construction is functorial and can be repeated, so we get functorial flasque resolutions in $\mc F$.

	In particular, let $\mc G^\bullet_\alpha$ be the produced flasque resolution of $\mc F_\alpha$. Thus, using \Cref{lem:flasque-is-acyclic} with \Cref{prop:acyclic-res} to compute our cohomology, we see
	\[\colimit H^i(X,\mc F_\alpha)\simeq\colimit h^i(\Gamma(X,\mc G^\bullet_\alpha)).\]
	Now, taking directed colimits is exact, so this is
	\[h^i\left(\colimit\Gamma(X,\mc G_\alpha^\bullet)\right).\]
	Taking global sections commutes with directed colimits (here we use that $X$ is Noetherian with \cite[Exercise~1.11]{hartshorne}), so this is
	\[h^i\left(\Gamma\left(X,\colimit\mc G_\alpha^\bullet\right)\right).\]
	Now, taking these directed colimits commutes with taking stalks, so it will be exact on sheaves, so we have the resolution
	\[0\to\colimit\mc F_\alpha\to\colimit\mc G_\alpha^\bullet,\]
	so our last cohomology is the desired $H^i\left(X,\colimit\mc F_\alpha\right)$. Everything has been done on the level of resolutions, so we have produced a bona fide isomorphism of $\delta$-functors.
\end{proof}
\begin{example}
	Cohomology also commutes with infinite direct sums because these are directed colimits (of the finite sums!).
\end{example}

\subsection{Cohomology on Closed Subsets}
Next up, one reduction we will want to make is to go down closed subschemes, so we have the following.
\begin{lemma} \label{lem:closed-cohomology}
	Fix a closed subset $j\colon Y\to X$ of a topological space. Given a sheaf $\mc F\in\mathrm{Ab}(Y)$, there is a natural isomorphism
	\[H^\bullet(Y,\mc F)=H^\bullet(X,j_*\mc F)\]
	compatible in the long exact sequence.
\end{lemma}
\begin{proof}
	We are asking for an isomorphism of $\delta$-functors $\mathrm{Ab}(Y)\to\mathrm{Ab}$. The point is that, by computing stalks, $j_*$ is an exact functor, and by computing sections, $j_*$ sends flasque sheaves to flasque sheaves. So we use the usual combination of \Cref{lem:flasque-is-acyclic} with \Cref{prop:acyclic-res} so that a flasque resolution $\mc G^\bullet$ of $\mc F$ produces the sequence of natural isomorphisms
	\[H^i(Y,\mc F)\simeq h^i(\Gamma(Y,\mc G^\bullet))=h^i(\Gamma(X,j_*\mc G^\bullet))\simeq H^i(X,j_*\mc F).\]
	Everything was done on the level of resolutions, so this is an isomorphism of $\delta$-functors.
\end{proof}
\begin{remark}
	If $Y\subseteq X$ is not closed, $j_*$ need not be exact.
\end{remark}
With our closed subsets, we will want the notion of restricting sheaves.
\begin{definition}
	Fix a topological space $X$ and closed subset $i\colon Z\to X$ and open subset $j\colon U\to X$ where $U=X\setminus Z$. For a sheaf $\mc F$ on $X$, we set $\mc F_Z\coloneqq i_*(\mc F|_Z)$ and $\mc F_U\coloneqq j_!(\mc F|_U)$.
\end{definition}
\begin{remark} \label{rem:ses-res-sheaf}
	Fix everything as above. Computing stalks, we see that there is an exact sequence
	\[0\to\mc F_U\to\mc F\to\mc F_Z\to0\]
	of sheaves on $X$, provided $\mc F\in\mathrm{Ab}(X)$. We will not bother to give the construction of the maps; they can be given on the level of presheaves, where the left map is essentially an inclusion, and the right map is essentially a restriction.
\end{remark}
% Anyway, here is our result.
% \begin{theorem}[Grothendieck vanishing]
% 	Fix a Noetherian topological space $X$ of dimension $n$. Given $\mc F\in\mathrm{Ab}(X)$, we have $H^i(X,\mc F)=0$ for $i>n$.
% \end{theorem}
% \begin{proof}
% 	We proceed by induction on the collection of pairs $(n,m)\in\{(-1,0)\}\cup\NN\times\ZZ^+$, where $n=\dim X$ and $m$ is the number of irreducible components of $X$. For our induction, we order $\{(-1,0)\}\cup\NN\times\ZZ^+$ lexicographically; here $\dim\emp=-1$. In other words, we will induct on the dimension, and within that induction, we will induct on the number of irreducible components. Notably, for $(n,m)=(-1,0)$, we have $X=\emp$, and there is nothing to do.

% 	We begin by reducing to $X$ being irreducible; fix $X$ of dimension $n$, and assume all lower results (lower dimension, fewer irreducible components if of dimension $n$). We may assume that $X$ is nonempty, so choose an irreducible component $Z\subseteq X$, and set $U\coloneqq X\setminus Z$. Notably, for this paragraph (making the reduction), we are assuming the statement for $Z$, so any sheaf $\mc F\in\mathrm{Ab}(X)$ has the exact sequence
% 	\[0\to\mc F_U\to\mc F\to\mc F_Z\to0.\]
% 	By the long exact sequence, it will be enough to show that $H^i(X,\mc F|_Z)=H^i(X,\mc F|_U)=0$ for all $i>n$. For one, note that $H^i(X,\mc F_Z)=0$ for all $i>n$ because $H^i(X,\mc F_Z)=H^i(Z,\mc F|_Z)$ by \Cref{lem:closed-cohomology}, and we have assumed the conclusion for $Z$. We will complete the proof next class.
% \end{proof}

\end{document}