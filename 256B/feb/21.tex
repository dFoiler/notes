% !TEX root = ../notes.tex

\documentclass[../notes.tex]{subfiles}

\begin{document}

We spent most class completing a proof from the previous class, and I have edited directly into the notes of the previous class for continuity.

\subsection{Cohomology on Projective Space}
We are now moving towards the proof of Serre duality, for which we will want to have computed the cohomology of some line bundles on projective space. Throughout, we will take $A$ to be a Noetherian ring (for example, a field), set $S\coloneqq A[x_0,\ldots,x_r]$ to be the graded ring, and we set $X\coloneqq\PP^r_A=\Proj S$. Our goal is to compute the cohomology of the sheaves $\OO_X(n)\coloneqq\widetilde{S(n)}$. We also recall the following construction: for an $\mathcal O_X$-module $\mc F$, we can define
\[\Gamma_\bullet(\mc F)\coloneqq\bigoplus_{n\in\ZZ}\Gamma(X,\mc F(n)),\]
which is a $\ZZ$-graded $S$-module. Let's go ahead and state our desired theorem.
\begin{theorem} \label{thm:cohom-proj-space}
	Fix a Noetherian ring $A$, and set $S\coloneqq A[x_0,\ldots,x_r]$ and $X\coloneqq\PP^r_A$.
	\begin{listalph}
		\item The natural map
		\[S\to\bigoplus_{n\in\ZZ}\Gamma(X,\OO_X(n))\]
		is an isomorphism of $\ZZ$-graded $S$-modules.
		\item $H^i(X,\OO_X(n))=0$ for all $0<i<r$ and $n\in\ZZ$.
		\item $H^r(X,\OO_X(-r-1))=A$.
		\item For each $n\in\ZZ$, the natural map
		\[H^0(X,\OO_X(n))\times H^r(X,\OO_X(-r-1-n))\to H^r(X,\OO_X(-r-1))\cong A\]
		is a perfect pairing of free $A$-modules of finite rank.
	\end{listalph}
\end{theorem}
\begin{remark}
	The standard affine open covering on $\PP^n_A$ tells us that $H^i(X,\mc F)=0$ for any $i>r$ and any quasicoherent sheaf $\mc F$. Notably, we are using \v Cech cohomology via the comparison theorem \Cref{thm:cech-comparison}, which applies because $X$ is in fact Noetherian and separated.
\end{remark}
As suggested by the remark, our proof of \Cref{thm:cohom-proj-space} will use \v Cech cohomology. For example, we may set
\[\mc F\coloneqq\bigoplus_{n\in\ZZ}\OO_X(n)\]
to be a quasicoherent sheaf. Taking cohomology, which commutes with infinite sums because taking infinite sums is exact, we see that
\[H^i(X,\mc F)=\bigoplus_{n\in\ZZ}H^i(X,\OO_X(n))\]
for each $i$. Notably, one has an $A$-module structure everywhere by \Cref{rem:cech-module}.

\end{document}