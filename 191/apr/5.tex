% !TEX root = ../notes.tex

\documentclass[../notes.tex]{subfiles}

\begin{document}

\section{April 5}

We began class in the middle of an argument; I have edited directly into those notes for continuity.

\subsection{Sieving in Minor Arcs: Type II}
One can argue as we did in the Type I sums to see that the contribution here is
\[\ll N^{1/2}(\log N)^3\max_{U\le M\le N/V}\left(M+\sum_{1\le m\le N/M}\min\left\{\frac Nm,\frac1{\lfloor m\alpha\rceil}\right\}\right)^{1/2},\]
where the point is that we have managed to sum our geometric series using the same bound as before. Anyway, we can use the bound achieved in the previous argument to achieve
\[\ll\left(NV^{-1/2}+NU^{-1/2}+Nq^{-1/2}+N^{1/2}q^{1/2}\right)(\log qN)^4,\]
so in total we achieve
\[|S(\alpha)|\ll\left(UV+q+NU^{-1/2}+NV^{-1/2}+Nq^{-1/2}+N^{1/2}q^{1/2}\right)(\log qN)^4\]
by combining with the Type I contribution. Taking $U=V=N^{2/5}$ completes the proof of \Cref{thm:vinogradov-minor}.

\subsection{Vinogradov's Sieve}
Putting everything together, we see that it remains to prove \Cref{thm:vinogradov-sieve}. The main idea is that we can ``sieve'' via M\"obius inversion as
\[f(1)+\sum_{\sqrt N<p\le N}f(p)=\sum_{\substack{n\le N\\\gcd(n,P_{\sqrt N})=1}}f(n)=\sum_{\substack{t\mid P_{\sqrt N}\\t\le N}}\Bigg(\mu(t)\sum_{r\le N/t}f(rt)\Bigg).\]
Here $P_{\sqrt N}$ is the product of all the primes less than $\sqrt N$. This explains why we might expect we want bounds on Type I sums. However, when $t$ is relatively close to $N$, the inner sum will have essentially no cancellation---it's too short!

This is where Type II sums come in.
\begin{proposition}[Vaughan's identity] \label{prop:vaughan}
	Fix $F(s)\coloneqq\sum_{n\le U}\Lambda(n)n^{-s}$ and $G(s)\coloneqq\sum_{n\le V}\mu(n)n^{-s}$. Then
	\[-\frac{\zeta'(s)}{\zeta(s)}=F(s)-\zeta(s)F(s)G(s)-\zeta'(s)G(s)+\left(-\frac{\zeta'(s)}{\zeta(s)}-F(s)\right)(1-\zeta(s)G(s)).\]
\end{proposition}
\begin{proof}
	Fully expand our the product on the right-hand side and cancel.
\end{proof}
The fourth term here turns out to be Type II sums. Namely, we use the ``$TT^*$'' method to estimate the operator norm of
\[(f(mn))_{M\le m\le 2M,N\le n\le 2N}.\]

\end{document}