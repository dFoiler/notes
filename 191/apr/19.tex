% !TEX root = ../notes.tex

\documentclass[../notes.tex]{subfiles}

\begin{document}

\section{April 19}

We continue the proof of the Bombieri--Vinogradov theorem.

\subsection{Using the Large Sieve}
We are now interested in proving the following inequality.
\begin{align*}
	&\sum_{q\le Q}\frac q{\varphi(q)}{\sum_\chi}^*\max_U\bigg|\max_{\substack{1\le m\le M\\1\le n\le N\\mn\le U}}a_mb_n\chi(mn)\bigg| \stackrel?\ll \\
	&\qquad\left(M+Q^2\right)^{1/2}\left(N+Q^2\right)^{1/2}\log(2MN)\Bigg(\sum_{1\le m\le M}|a_m|^2\Bigg)^{1/2}\Bigg(\sum_{1\le n\le N}|b_n|^2\Bigg)^{1/2}.
\end{align*}
For this, we use a little Fourier analysis. As discussed before, the issue with our bounding is dealing with $mn\le U$. To give us more to work with, we will try to indicate with functions like $(mn)^{it}$. To set us up, for $T,\beta>0$, we note
\[\int_{-T}^Te^{-it\alpha}\frac{\sin t\beta}{\pi t}\,dt=\begin{cases}
	\pi+O\left(T^{-1}(\beta-|\alpha|)^{-1}\right) & \text{if }|\alpha|<\beta, \\
	O\left(T^{-1}(|\alpha-\beta)^{-1}\right) & \text{if }|\alpha>\beta.
\end{cases}\]
Thus, we set $A(t,\chi)\coloneqq\sum_{m=1}^Ma_m\chi(m)m^{it}$ and $B(t,\chi)\coloneqq\sum_{n=1}^Nb_m\chi(n)n^{it}$ so that multiplying $A(t,\chi)$ and $B(t,\chi)$ will detect values of $mn$ by some kind of Fourier analysis. Explicitly,
\[\int_{-T}^TA(t,\chi)B(t,\chi)\frac{\sin(t\log u)}{\pi t}\,dt=\sum_{\substack{1\le m\le M\\1\le n\le N\\mn\le U}}a_mb_n\chi(mn)+O\Bigg(T^{-1}\sum_{\substack{1\le m\le M\\1\le n\le N}}|a_mb_n\log(mn/u)|^{-1}\Bigg),\]
where we take $u$ to be (say) a half-integer. For the error term, we see we may as well assume that $u\le MN$, so we note $\log(mn/u)\gg\frac1{MN}$ and $\sin(t\log u)\ll\min\{1,|t|\log2MN\}$ (by staring at the graph of $\sin x$ either close to $0$ or away from $0$). Rearranging, we achieve
\[\Bigg|\sum_{\substack{1\le m\le M\\1\le n\le N\\mn\le u}}a_mb_n\chi(mn)\Bigg|\ll\int_{-T}^T|A(t,\chi)|\cdot|B(t,\chi)|\cdot\min\{1/|t|,\log2MN\}\,dt+\frac{MN}T\sum_{\substack{1\le m\le M\\1\le n\le N}}|a_mb_n|.\]
At this point, we set $T\coloneqq(MN)^{3/2}$. We now use the large sieve inequality to bound
\[\sum_{q\le Q}\frac q{\varphi(q)}{\sum_\chi}^*|A(t,\chi)B(t,\chi)|\]
as we did last class when we did not have to deal with the $\chi$s; we omit the details. Further, we see that
\[\int_{-T}^T\min\left\{\frac1{|t|},\log2MN\right\}\,dt\ll\log2MN\]
by first integrating over the region $[-1,1]$ to achieve $\ll\log2MN$ and then recalling $T=(MN)^{3/2}$ to integrate outside $[-1,1]$ to be sure that we do not overcome $\ll\log2MN$. Additionally, summing over all $q$ and $\chi$, our term is bounded by
\begin{align*}
	\frac{MN}T\sum_{\substack{1\le m\le M\\1\le n\le N}}|a_mb_n|Q^2 &\le \frac{Q^2MN}T\Bigg(\sum_{1\le m\le M}|a_m|^2\Bigg)^{1/2}\Bigg(\sum_{1\le n\le N}|b_n|^2\Bigg)^{1/2}M^{1/2}N^{1/2} \\
	&\le \left(Q^2+M\right)^{1/2}\left(Q^2+N\right)^{1/2}\Bigg(\sum_{1\le m\le M}|a_m|^2\Bigg)^{1/2}\Bigg(\sum_{1\le n\le N}|b_n|^2\Bigg)^{1/2}.
\end{align*}
Combining the previous bounds on $|A(t,\chi)B(t,\chi)|$ is able to complete the proof of the inequality.

\subsection{Using the Bilinear Method}
We use Vaughan's identity to continue. Arguing as before, we may write $\psi(y,\chi)=S_1+S_2+S_3+S_4$ where
\begin{align*}
	S_1 &= \sum_{n\le U}\Lambda(n)\chi(n) \\
	S_2 &= -\sum_{t\le UV}\Bigg(\sum_{\substack{t=md\\m\le U,d\le V}}\mu(d)\Lambda(m)\Bigg)\sum_{r\le y/t}\chi(rt) \\
	S_3 &= \sum_{\substack{d\le V\\dh\le y}}\mu(d)\log h\chi(dh) \\
	S_4 &= -\sum_{U\le m\le y/V}\Lambda(m)\sum_{V\le k\le y/m}\Bigg(\sum_{d\mid k,d\le V}\mu(d)\Bigg)\chi(mk).
\end{align*}
Quickly, we see that $|S_1|\ll U$ by the Prime number theorem. Also, as in the proof of Vinogradov's theorem, summation by parts produces
\[|S_3|\ll\log y\sum_{d\le V}\max_w\left|\sum_{w\le h\le y/d}\chi(h)\right|.\]
Next up, we bound $S_4$. We'll do this next class.

\end{document}