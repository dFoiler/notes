% !TEX root = ../notes.tex

\documentclass[../notes.tex]{subfiles}

\begin{document}

\section{April 21}

We continue.

\subsection{Continuing the Bilinear Method}
The theme of today is to use our large sieve inequality to bound as many of the $S_\bullet$ terms with bilinear sums because the large sieve inequality is quite efficient. From last time, our next task is to bound
\[S_4 = -\sum_{U\le m\le y/V}\Lambda(m)\sum_{V\le k\le y/m}\Bigg(\sum_{d\mid k,d\le V}\mu(d)\Bigg)\chi(mk).\]
We use a dyadic decomposition. For each $M\in[U,y/V]$, we see
\[\sum_{q\le Q}\frac q{\varphi(q)}{\sum_\chi}^*\max_{y\le x}\Bigg|\sum_{\substack{U\le m\le y/V\\M\le m\le2M}}\Lambda(m)\sum_{V\le k\le y/M}\Bigg(\sum_{\substack{d\mid k\\d\le V}}\mu(d)\Bigg)\chi(mk)\Bigg|^2,\]
which is our averaged version of $S_4$. Now, by \eqref{eq:large-sieve-ineq}, this is
\[\ll\left(Q^2+M\right)^{1/2}\left(Q^2+x/M\right)^{1/2}\Bigg(\sum_{M\le m\le2M}\Lambda(m)^2\Bigg)\Bigg(\sum_{k\le x/M}d(k)^2\Bigg)^{1/2}\log2x,\]
which we can now upper-bound as in the proof of Vinogradov's theorem. After doing so, we achieve
\[\ll\left(Q^2+M\right)^{1/2}\left(Q^2+x/M\right)^{1/2}x^{1/2}(\log2x)^3.\]
``Expanding'' the square root using something like H\"older, up to a constant term we get
\[\ll\left(Q^2x^{1/2}+QxM^{-1/2}+Qx^{1/2}M^{1/2}+x\right)(\log2x)^3.\]
We now sum over all $U/2\le M\le x/V$ via our dyadic intervals to compute
\[\sum_{q\le Q}\frac q{\varphi(q)}{\sum_\chi}^*\max_{y\le x}|S_4(y,\chi)|\ll\left(Q^2x^{1/2}QxU^{-1/2}+Qx^{-1/2}+x\right)(\log2x)^4,\]
where the extra $\log$ factor comes from summing over $\log$ many dyadic intervals.

Continuing, we bound
\[S_2 = -\sum_{t\le UV}\Bigg(\sum_{\substack{t=md\\m\le U,d\le V}}\mu(d)\Lambda(m)\Bigg)\sum_{r\le y/t}\chi(rt).\]
For this, we split the sum into the two pieces
\begin{align*}
	S_2' \coloneqq{}& -\sum_{t\le U}\Bigg(\sum_{\substack{t=md\\m\le U,d\le V}}\mu(d)\Lambda(m)\Bigg)\sum_{r\le y/t}\chi(rt), \\
	S_2'' \coloneqq{}& -\sum_{U\le t\le UV}\Bigg(\sum_{\substack{t=md\\m\le U,d\le V}}\mu(d)\Lambda(m)\Bigg)\sum_{r\le y/t}\chi(rt).
\end{align*}
The same technique that we used for $S_4$ also works for $S_2''$. Namely, repeating the above argument with some $M$ lets us estimate the averaged version of $S_2''$
\[\sum_{q\le Q}\frac q{\varphi(q)}{\sum_\chi}^*\max_{y\le x}\Bigg|-\sum_{U\le t\le UV}\Bigg(\sum_{\substack{t=md\\m\le U,d\le V\\M\le m\le2M}}\mu(d)\Lambda(m)\Bigg)\sum_{r\le y/t}\chi(rt)\Bigg|^2.\]
Summing over the dyadic ranges appropriately, keeping track of the number of $\log$s, we produce
\[\ll\left(Q^2+QxU^{-1}2+Qx^{1/2}U^{1/2}V^{1/2}+x\right)(\log2x)^3\]
as the bound for averaged version of $S_2''$. For $S_2'$, our averaged version can be bounded as
\begin{align*}
	&\sum_{1<q\le Q}\frac q{\varphi(q)}{\sum_\chi}^*\max_{y\le x}|S_2'(y,\chi)| \\
	\ll{}& Q^2\max_{\text{prim. }\chi\pmod q}\max_{y\le x}|S_2'(y,\chi)|.
\end{align*}
Staring at $S_2'(y,\chi)$, we are able to bound via the P\'olya--Vinogradov inequality to produce
\[\ll Q^{5/2}U(\log xU)^2.\]
Adding in $q=1$ with the trivial character, we see that the maximum of $|S_2'(y,\chi_0)|$ is bounded by $x(\log xU)^2$ by simply evaluating the inner sum via the Prime number theorem. In total, we achieve
\[\sum_{q\le Q}\frac q{\varphi(q)}{\sum_\chi}^*\max_{y\le x}|S_2'(y,\chi)|\ll\left(Q^{5/2}U+x\right)(\log xU)^2.\]
For $S_3$, we use the same techniques of $S_2'$ (combining with the upper bound we achieved last class) in order to achieve
\[\sum_{q\le Q}\frac q{\varphi(q)}{\sum_\chi}^*\max_{y\le x}\left|S_3(y,\chi)\right|\ll\left(Q^{5/2}V+x\right)(\log xV)^2.\]
Combining all of our bounds, we may upper-bound
\[\sum_{q\le Q}\frac1{\varphi(q)}{\sum_\chi}^*\max_{y\le x}|\psi'(y,\chi)|\]
by
\[\ll\left(Q^2x^{1/2}+QxU^{-1/2}+QxV^{-1/2}+Qx^{1/2}U^{1/2}V^{1/2}+Q^{5/2}U+Q^{5/2}V\right)(\log xUV)^4.\]
To optimize, over $x^{1/3}\le Q\le x^{1/2}$, we take $U=V\coloneqq x^{2/3}Q^{-1}$ to achieve $\ll\left(Q^2x^{1/2}+x\right)(\log x)^4$. Then for $Q\le x^{1/3}$, we take $U=V\coloneqq x^{1/3}$ to achieve $\ll\left(x+x^{5/6}Q\right)(\log x)^4$. Summing, we produce the desired inequality.
\begin{remark}
	There is an Elliott--Halberstam conjecture which asserts that any $0<\theta<1$ has
	\[\sum_{q<x^\theta}\max_{\gcd(a,q)=1}\left|\psi(x;q,a)-\frac x{\varphi(q)}\right|\ll_A x(\log x)^{-A}.\]
	The Bombieri--Vinogradov theorem achieves the result for $\theta<1/2$. Any progress for larger $\theta$ would mark extreme progress. For example, this conjecture has connections to bounded gaps between primes.
\end{remark}

\end{document}