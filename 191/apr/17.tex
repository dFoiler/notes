% !TEX root = ../notes.tex

\documentclass[../notes.tex]{subfiles}

\begin{document}

\section{April 17}

We continue discussing the Bombieri--Vinogradov theorem.

\subsection{}
Recall the following statement.
\begin{theorem}
	For all $A>0$ and $Q\in\left[\sqrt x(\log x)^{-A},x^{1/2}\right]$, we have
	\[\sum_{q\le Q}\max_{y\le x}\max_{\substack{1\le a\le q\\\gcd(a,q)=1}}\left|\psi(y;q,a)-\frac y{\varphi(q)}\right|\ll\sqrt xQ(\log x)^5.\]
\end{theorem}
\begin{proof}
	During the Friday lecture, we found
	\[\max_{\substack{1\le a\le q\\\gcd(a,q)=1}}\left|\psi(y;q,a)-\frac y{\varphi(q)}\right|\le\frac1{\varphi(q)}\sum_\chi|\psi'(y,\chi)|,\]
	for some function $\psi'$. We would like to only use primitive characters, so observe that if $\chi$ is an imprimitive character induced by $\overline\chi$, then we can upper-bound the difference as
	\[|\psi'(y,\chi)-\psi'(y,\overline\chi)|\ll(\log qy)^2,\]
	so we achieve
	\[S\coloneqq\max_{\substack{1\le a\le q\\\gcd(a,q)=1}}\left|\psi(y;q,a)-\frac y{\varphi(q)}\right|\ll(\log qy)^2+\frac1{\varphi(q)}\sum_\chi|\psi'(y,\overline\chi)|.\]
	Summing gives
	\[\sum_{q\le Q}\max_{y\le x}\max_{\substack{1\le a\le q\\\gcd(a,q)=1}}\left|\psi(y;q,a)-\frac y{\varphi(q)}\right|\ll\sum_{q\le Q} Q(\log qx)^2+{\sum_{\chi\pmod q}}^*\max_{y\le x}|\psi'(y,\chi)|\sum_{kq\le Q}\frac1{\varphi(kq)}.\]
	We now note that $\varphi(kq)\ge\varphi(k)\varphi(q)$, so the last summation can be controlled like a harmonic series. Namely, we observe
	\[\sum_{k\le z}\frac1{\varphi(k)}\le\prod_{p\le z}\sum_{\nu=0}^\infty\frac1{\varphi(p^\nu)}=\prod_{p\le z}\left(1+\frac1{p-1}\sum_{\nu=0}^\infty\frac1{p^\nu}\right)=\prod_{p\le z}\frac1{1-\frac1p}\left(1+\frac1{p(p-1)}\right)\ll\log z.\]
	Notably, the second term here converges absolutely, and the $\frac1{1-\frac1p}$ term produces the $\log z$. Inputting this inequality grants
	\[S\ll Q(\log Qx)^2+\log x\sum_{q\le Q}{\sum_\chi}^*\max_{y\le x}|\psi'(y,\chi)|^2.\]
	Thus, it remains to show
	\[\sum_{q\le Q}\frac1{\varphi(q)}{\sum_\chi}^*\max_{y\le x}|\psi'(y,\chi)|\ll\sqrt xQ(\log x)^4.\]
	Instead, we will prove the ``dyadic'' inequality
	\[\sum_{U\le q\le 2U}\frac1{\varphi(q)}{\sum_\chi}^*\max_{y\le x}|\psi'(y,\chi)|\ll\left(\frac xU+x^{5/6}+x^{1/2}U\right)(\log x)^4.\]
	To see how this implies the desired inequality, we sum over our dyadic intervals: note that the sum of the $x^{5/6}$ terms do not matter (they are strictly less than $x^{1/2}Q$). Similarly, the $x^{1/2}U$ term will sum to $\ll x^{1/2}Q$ and also does not matter. The last term is harder. To write it out, for $U\in[Q_1,Q]$ for some $Q_1$, we have
	\[\sum_{Q_1\le q\le Q}\frac1{\varphi(q)}{\sum_\chi}^*\max_{y\le x}|\psi'(y,\chi)|\ll(\log x)^4\left(\frac x{Q_1}+x^{5/6}\log x+x^{1/2}Q\right).\]
	For example, taking $Q_1=(\log x)^A$, we note that Siegel's theorem lets us bound
	\[|\psi'(y,\chi)|\ll x(\log x)^{-2A-4},\]
	so we can deal with the small values of $q$ as
	\[\sum_{q\le(\log x)^A}\frac1{\varphi(q)}{\sum_\chi}^*\max_{y\le x}|\psi'(y,\chi)|\ll x(\log x)^{-A-4},\]
	which is good enough for our purposes.

	It remains to prove our dyadic bound. We split this into two parts. To begin, we use the large sieve. We claim that
	\[\sum_{q\le Q}\frac q{\varphi(q)}{\sum_\chi}^*\max_U\bigg|\max_{\substack{1\le m\le M\\1\le n\le N\\mn\le U}}a_mb_n\chi(mn)\bigg|\ll\left(M+Q^2\right)^{1/2}\left(N+Q^2\right)^{1/2}\log(2MN)\Bigg(\sum_{1\le m\le M}|a_m|^2\Bigg)^{1/2}\Bigg(\sum_{1\le n\le N}|b_n|^2\Bigg)^{1/2}.\]
	The main difficulty is dealing with the $mn\le U$ constraint. Indeed, without this requirement, we could use Cauchy--Schwarz by writing
	\begin{align*}
		\sum_{q\le Q}\frac q{\varphi(q)}{\sum_\chi}^*\Bigg|\sum_{1\le m\le M}\sum_{1\le n\le N}a_mb_n\chi(mn)\Bigg| &\le \Bigg(\sum_{q\le Q}\frac q{\varphi(q)}{\sum_\chi}^*\Bigg|\sum_{1\le m\le M}a_m\chi(m)\Bigg)^{1/2}\cdot \\
		&\qquad\qquad\Bigg(\sum_{q\le Q}\frac q{\varphi(q)}{\sum_\chi}^*\Bigg|\sum_{1\le n\le N}b_n\chi(n)\Bigg)^{1/2}.
	\end{align*}
	Then the large sieve grants us
	\[\ll\left(M+Q^2\right)^{1/2}\left(N+Q^2\right)^{1/2}\Bigg(\sum_{1\le m\le M}|a_m|^2\Bigg)^{1/2}\Bigg(\sum_{1\le n\le N}|b_n|^2\Bigg)^{1/2}.\]
	To achieve the desired inequality, the point is to ``complete the sum.'' Intuitively, we are essentially adding in a term of $1_{mn\le U}$ to our sum in order to bound via Fourier analysis. Namely, the Mellin transform of $1_{[0,1]}(t)$ is $\frac1s$. We will continue this next class.
\end{proof}

\end{document}