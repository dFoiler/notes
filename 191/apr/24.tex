% !TEX root = ../notes.tex

\documentclass[../notes.tex]{subfiles}

\begin{document}

\section{April 21}

It is the last week of instruction.

\subsection{Least Nonquadratic Residues}
Here is the conjecture.
\begin{conj}[Vinogradov] \label{conj:vinogradov}
	For each prime $p>2$, let $n(p)$ be the least positive integer which is not a quadratic residue$\pmod p$. Then $n(p)\ll_\varepsilon p^\varepsilon$ for all $\varepsilon>0$.
\end{conj}
\begin{remark}
	Continuing our discussion of the Elliott--Halberstam conjecture, Tao was able to conditionally show \Cref{conj:vinogradov}. Also, the grand Riemann hypothesis implies \Cref{conj:vinogradov}. The argument is akin to our discussion of primality testing.
\end{remark}
\begin{remark}
	Quickly, note that $n(p)$ is prime. Indeed, suppose we have a prime factorization
	\[n(p)=\prod_{q\text{ prime}}q^{\nu_q(n(p))}.\]
	Then $\left(\frac{n(p)}p\right)=-1$ implies that $\left(\frac qp\right)=-1$ for some prime $q\mid n(p)$. Thus, $q\le n(p)$, so minimality enforces $q=n(p)$, so $n(p)$ is prime.
\end{remark}
\begin{remark}
	The Burgess bound is able to achieve $n(p)\ll_\varepsilon p^{1/4+\varepsilon}$.
\end{remark}
Vinogradov's sieving trick is able to achieve $n(p)\ll_\varepsilon p^{1/(4\sqrt e)+\varepsilon}$, which is the current record. Let's see this.
\begin{theorem}
	For any $\varepsilon>0$, we have $n(p)\ll_\varepsilon p^{1/(4\sqrt e)+\varepsilon}$.
\end{theorem}
\begin{proof}
	Intuitively, the point is to compute the character sum
	\[\sum_{k=1}^n\left(\frac kp\right),\]
	and as soon as we can show that this is less than $n$, we get a nonquadratic residue. Some extra structure about primality of $n(p)$ is able to sharpen the bound.

	Let's be more explicit. Set $\chi\coloneqq\left(\frac\cdot p\right)$. The Burgess bound is able to achieve
	\[\sum_{1\le n\le y}\chi(n)=o_\varepsilon(y)\]
	for $y=\floor{p^{1/4+\varepsilon}}$. To see our nonquadratic residues, we write
	\[\sum_{1\le n\le y}\chi(n)=y-2\sum_{\substack{1\le n\le y\\\chi(n)}}1\ge y-2\sum_{\substack{1\le q\le y\\q\text{ prime}\\\chi(q)=-1}}\frac yq=y\Bigg(1-2\sum_{\substack{1\le q\le y\\q\text{ prime}\\\chi(q)=-1}}\frac yq\Bigg).\]
	The point is that if we get too many $q$s in the sum which are too big, then we're not going to achieve $o_\varepsilon(y)$. With this in mind, we use the definition of $n(p)$ to write
	\[o_\varepsilon(y)=\sum_{1\le n\le y}\chi(n)\ge y\Bigg(1-2\sum_{\substack{n(p)\le q\le y\\q\text{ prime}\\\chi(q)=-1}}\frac yq\Bigg).\]
	Thus, we achieve
	\[\frac12\le\sum_{\substack{n(p)\le q\le y\\q\text{ prime}}}\frac1q+o_\varepsilon(1).\]
	Because $\sum_{p\le n}\frac1p=\log\log n$, we see
	\[\frac12\le\log(\log_{n(p)}y)+o_\varepsilon(1),\]
	so $\log_{n(p)}y\ge e^{1/2+o_\varepsilon(1)}\gg_\varepsilon e^{1/2}$. Rearranging, we see $n(p)\ll_\varepsilon y^{1/\sqrt e+\varepsilon}=y^{1/(4\sqrt e)+\varepsilon}$, which is what we wanted.
\end{proof}

\end{document}