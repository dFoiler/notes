% !TEX root = ../notes.tex

\documentclass[../notes.tex]{subfiles}

\begin{document}

\section{April 28}

Today we discuss Hua's inequality for Waring's problem.

\subsection{Hua's Inequality}
Here is the problem we want to solve: for fixed positive integer $k$, compute the smallest positive integer $s$ such that every natural number is the sum of $\le s$ powers of $k$; we call this positive integer $g(k)$; it is a result of Waring--Hilbert that $g(k)<\infty$. A more interesting question to ask is to find $G(k)$ so that every sufficiently large integer is the sum of $\le G(k)$ powers of $k$; this is more interesting because it turns out that small values will dominate so that $g(k)$ is more easily understood but less representative of the underlying structure.

We will be using the circle method, which will even get us an asymptotic formula. To be explicit, we want to study powers of the function
\[g_k(\alpha,X))=\sum_{1\le x\le X}e\left(\alpha x^k\right).\]
The main difficulty, as expected, is bounding the minor arcs. For example, consider the integral of the moment
\[\int_0^1\left|g_k(\alpha,X)\right|^{2s}\,d\alpha=\#\left\{((x_i),(y_i))\in[1,X]^s\times[1,X]^s:x_1^k+\cdots+x_s^k:y_1^k+\cdots+y_s^k\right\}\]
by expanding out $\left|g_k(\alpha,X)\right|^{2s}=g_k(\alpha,X)^s\overline{g_k(\alpha,X)}^s$ and integrating. These combinatorial objects are quite difficult to understand; for example, not much is even known at $k=3$. Our result for today is Hua's inequality, as follows.
\begin{theorem}[Hua's inequality]
	For any $k\ge0$, we have
	\[\int_0^1\left|g_k(\alpha,X)\right|^{2^k}\,d\alpha\ll_\varepsilon X^{2^k-k+\varepsilon}.\]
\end{theorem}
\begin{proof}[Sketch]
	We will sketch the main ideas. The $2^k$ here is going to arise from squaring this inequality repeatedly. Explicitly, note
	\begin{equation}
		(x+y)^k-x^k=y\sum_{j=0}^{k-1}\binom kjy^{k-j-1}x^j. \label{eq:use-binom-thm}
	\end{equation}
	The point here is that control over $y$ has made our polynomial have less degree; this method is called ``differencing.'' To apply this, we need two lemmas.
	\begin{lemma}
		Fix $f(x)\in\ZZ[x]$ of positive degree with nonnegative coefficients. For $y\in\ZZ$, define the polynomial $\Delta_yf$ by $\Delta_yf(x)\coloneqq f(x+y)-f(x)$, and define $\Delta_{y_1,\ldots,y_n}\coloneqq\Delta_{y_1}\cdots\Delta_{y_n}$. Given positive integers $y_1,\ldots,y_v$ where $v\le\deg f$, then $\Delta_{y_1,\ldots,y_v}$ is a polynomial of degree $\deg f-v$, with nonnegative coefficients, and is divisible by $y_1\cdots y_v$.
	\end{lemma}
	\begin{proof}
		Induct on $v$, using \eqref{eq:use-binom-thm}.
	\end{proof}
	\begin{lemma}
		For $1\le v\le k-1$, we have
		\[\left|g_k(\alpha,X)\right|^{2^v}\ll_vX^{2^v-1}+X^{2^v-v-1}\Re\Bigg(\sum_{y_1,\ldots,y_v\in[1,X]}\sum_xe\left(\alpha\Delta_{y_1,\cdots,y_v}x^k\right)\Bigg),\]
		where the summation of $x$ is in some interval of $[1,X]$ depending on the $y_i$.
	\end{lemma}
	The inequality now follows from the previous lemmas.
\end{proof}
\begin{remark}
	Vinogradov was able to improve the $2^k$ in the inequality to $O\left(k^2\log k\right)$ by considering solutions to systems
	\[x_1^j+\cdots+x_s^j=y_1^j+\cdots+y_s^j\]
	for $1\le j\le k$. This system appears somewhat unmotivated, but it turns out to be helpful; for example, this system of equations turns out to satisfy some form of translation-invariance.
\end{remark}

\end{document}