% !TEX root = ../notes.tex

\documentclass[../notes.tex]{subfiles}

\begin{document}

\section{April 10}

We began class by proving some small technicality, so I have edited last class's notes for continuity.
\begin{remark}
	For $\sqrt x\le n\le x$, one can use the bilinear method to show
	\[\Lambda(n)=\sum_{\substack{m\le\sqrt x\\m\mid n}}\mu(x)\log(n/m)-\sum_{m\le\sqrt x}\mu(m)\sum_{\substack{k\le\sqrt x\\km\mid n}}\Lambda(k).\]
\end{remark}

\subsection{A Duality Theorem}
We are going to talk about the large sieve. To give a flavor for what we will achieve, here is an application of the large sieve. Recall that
\[\psi(x;q,a)\coloneqq\sum_{\substack{m\le x\\n\equiv a\pmod q}}\Lambda(n).\]
\begin{theorem}[Bombieri--Vinogradov]
	For given $A$, there is a constant $B$ such that we have
	\[\psi(x;q,a)=\frac x{\varphi(q)}\left(1+O\left((\log x)^{-A}\right)\right)\]
	for almost every $q\le x^{1/2}(\log x)^{-B}$.
\end{theorem}
\begin{remark}
	As another example, one can use the large sieve to show that there is a finite set of integers which contains a primitive root$\pmod p$ for any prime $p$. This is a weaker version of Artin's conjecture. In general, one is able to achieve reasonably strong results with the large sieve if one is willing to allow a few exceptions.
\end{remark}
Our approach to the large sieve will be to take some hard analysis techniques and combine with stuff like Vaughan's identity to turn our results into number theory.

Anyway, here is our duality theorem.
\begin{theorem}
	Fix some countable sets $A$ and $B$. Additionally, fix some sequence $\{x_{mn}\}_{(m,n)\in A\times B}$ such that $\norm x_2<\infty$ and $X>0$. The following are equivalent.
	\begin{listalph}
		\item For each sequence $\{a_m\}_{m\in A}$ of complex numbers such that $\norm a_2<\infty$, we have
		\[\sum_{n\in B}\left|\sum_{m\in A}x_{mn}a_m\right|^2\le X\norm a_2^2.\]
		\item For each sequence $\{b_m\}_{m\in A}$ of complex numbers such that $\norm b_2<\infty$, we have
		\[\sum_{m\in A}\left|\sum_{n\in B}x_{mn}b_m\right|^2\le X\norm b_2^2.\]
	\end{listalph}
	Here, $\norm\cdot$ refers to the $L^2$-norm.
\end{theorem}
\begin{proof}
	By symmetry, it is enough to show that (a) implies (b). The idea is to introduce an auxiliary vector defined by
	\[c_m\coloneqq\sum_{n\in A}x_{mn}b_n.\]
	Note that this sum is finite because (by Cauchy--Schwarz) we have $\norm{c}_2^2\le\norm x_2^2\cdot\norm b_2^2$; in particular, each coordinate in $c$ must be finite. We now compute
	\[\norm c_2^2=\sum_{m\in A}\overline{c_m}\sum_{n\in B}b_nx_{mn}=\sum_{n\in B}b_n\sum_{m\in A}\overline{c_m}x_{mn},\]
	where the exchange of summation is fine by Fubini's theorem. Now, using Cauchy--Schwarz, we see that the above is an inner product (indexed over $n$) which can be bounded as
	\[\norm c_2^2\le\norm b_2^2\sqrt{\sum_{n\in B}\left|\sum_{m\in A}x_{mn}\overline{c_m}\right|^2}.\]
	Then by (a), we achieve the bound $\norm c_2^2\le\sqrt X\cdot\norm c_2\cdot\norm b_2$, which rearranges into $\norm c_2\le\sqrt X\cdot\norm b_2$. Upon squaring, this is what we wanted.
\end{proof}
As an example application, next class we will show the following.
\begin{theorem}
	Fix $\delta$-separated real numbers $x_1,\ldots,x_R$ on $\RR/\ZZ$. Then
	\[\sum_{r=1}^R\left|\sum_{n=M}^{M+N-1}a_ne(nx_r)\right|\ll\left(N+\delta^{-1}\right)\norm a_2^2.\]
\end{theorem}

\end{document}