% !TEX root = ../notes.tex

\documentclass[../notes.tex]{subfiles}

\begin{document}

\section{January 23}

Today we finish the proof of \Cref{thm:dirichlet}.

\subsection{The Dirichlet Convolution}
As motivation, we might be interested in the product of two Dirichlet series. Formally, we might write
\[\Bigg(\sum_{k=1}^\infty\frac{a_k}{k^s}\Bigg)\Bigg(\sum_{\ell=1}^\infty\frac{b_\ell}{\ell^s}\Bigg)=\sum_{k=1}^\infty\sum_{\ell=1}^\infty\frac{a_kb_\ell}{(k\ell)^s}=\sum_{n=1}^\infty\Bigg(\sum_{k\ell=n}a_kb_\ell\Bigg)\frac1{n^s}.\]
Of course, we will want to formalize this intuitive argument to give the corresponding series the correct analytic properties, but we have at least arrived at the correct definition.
\begin{definition}[Dirichlet convolution]
	Fix functions $f,g\colon\NN\to\CC$. Then the \textit{Dirichlet convolution} $(f*g)\colon\NN\to\CC$ is defined by
	\[(f*g)(n)\coloneqq\sum_{k\ell=n}f(k)g(\ell)=\sum_{d\mid n}f(d)g(n/d).\]
\end{definition}
And we may now take products of Dirichlet series.
\begin{proposition} \label{prop:prod-dir-series}
	Fix functions $f,g\colon\NN\to\CC$ such that $|f(n)|,|g(n)|=O\left(n^\sigma\right)$ for some $\sigma\in\RR$. Then define the series
	\[F(s)\coloneqq\sum_{n=1}^\infty\frac{f(n)}{n^s},\qquad G(s)\coloneqq\sum_{n=1}^\infty\frac{g(n)}{n^s},\qquad D(s)\coloneqq\sum_{n=1}^\infty\frac{(f*g)(n)}{n^s}.\]
	Then $D$ converges absolutely for $\op{Re}s>\sigma+1$, where it defines a holomorphic function given by $D(s)=F(s)G(s)$.
\end{proposition}
\begin{proof}
	Fix $s$ with $\op{Re}s>\sigma+1$. We show that $D(s)$ converges absolutely and yields $D(s)=F(s)G(s)$, from which it follows that $D(s)$ is holomorphic over the region by using \Cref{prop:basic-dirichlet-series} on $F$ and $G$. Let $F_n(s)$, $G_n(s)$, and $D_n(s)$ denote the $n$th partial sums. Then we see
	\[F_N(s)G_N(s)=\Bigg(\sum_{k=1}^N\frac{f(k)}{k^s}\Bigg)\Bigg(\sum_{\ell=1}^N\frac{g(\ell)}{\ell^s}\Bigg)=\underbrace{\sum_{n=1}^N\Bigg(\sum_{k\ell=n}f(k)g(\ell)\Bigg)\frac1{n^s}}_{D_N(s)}+\underbrace{\sum_{\substack{1\le k,\ell\le N\\k\ell>N}}\frac{f(k)g(\ell)}{(k\ell)^s}}_{R_N(s)\coloneqq}.\]
	Thus, the key claim is that $R_N(s)\to0$ as $N\to\infty$. The main point is that $k\ell>N$ requires $k>\sqrt N$ or $\ell>\sqrt N$, so
	\[|R_N(s)|\le\sum_{\substack{1\le k,\ell\le N\\k\ell>N}}\frac{|f(k)|\cdot|g(\ell)|}{(k\ell)^{\op{Re}s}}\le\Bigg(\sum_{k>\sqrt N}\frac{|f(k)|}{k^{\op{Re}s}}\Bigg)\Bigg(\sum_{\ell\ge1}\frac{|g(\ell)|}{\ell^{\op{Re}s}}\Bigg)+\Bigg(\sum_{k\ge1}\frac{|f(k)|}{k^{\op{Re}s}}\Bigg)\Bigg(\sum_{\ell>\sqrt N}\frac{|g(\ell)|}{\ell^{\op{Re}s}}\Bigg).\]
	The absolute convergence of $F$ and $G$ at $s$ now causes the right-hand side to be
	\[\Bigg(\sum_{k=1}^\infty\frac{|f(k)|}{k^{\op{Re}s}}\Bigg)\cdot0+0\cdot\Bigg(\sum_{\ell=1}^\infty\frac{|g(\ell)|}{\ell^{\op{Re}s}}\Bigg)=0\]
	as $N\to\infty$, so we conclude $R_N(s)\to0$ as $N\to\infty$. Thus, we conclude
	\[F(s)G(s)=\lim_{N\to\infty}(F_N(s)G_N(s))=\lim_{N\to\infty}D_N(s)+\lim_{N\to\infty}R_N(s)=D(s).\]
	Lastly, we need to show that $D(s)$ actually converges absolutely. Well, we note that we can replace $f$ with $|f|$ and $g$ with $|g|$ and $s$ with $\op{Re}s$ everywhere in the above bounding to show that
	\[\sum_{n=1}^\infty\left|\frac{(f*g)(n)}{n^s}\right|\le\sum_{n=1}^\infty\frac{(|f|*|g|)(n)}{n^{\op{Re}s}}=\Bigg(\sum_{k=1}^\infty\frac{|f(k)|}{k^{\op{Re}s}}\Bigg)\Bigg(\sum_{\ell=1}^\infty\frac{|g(\ell)|}{\ell^{\op{Re}s}}\Bigg),\]
	and the right-hand side converges because $F(s)$ and $G(s)$ converge absolutely. Thus, $D(s)$ converges absolutely.
\end{proof}
\begin{example}
	Let $d(n)$ denote the number of divisors of $n$. Then we see
	\[\zeta(s)^2=\sum_{n=1}^\infty\frac{(1*1)(n)}{n^s}=\sum_{n=1}^\infty\frac{d(n)}{n^s}.\]
	Here, $1\colon\NN\to\CC$ is the function which constantly returns $1$.
\end{example}
We might be interested in an Euler product factorization for a product of two Dirichlet series (as in \Cref{prop:euler-factor}), but this notably requires the relevant functions to be multiplicative. Thus, we now show that the Dirichlet convolution sends multiplicative functions to multiplicative functions.
\begin{lemma} \label{lem:dir-conv-mult}
	Let $f,g\colon\NN\to\CC$ be multiplicative functions. Then $(f*g)\colon\NN\to\CC$ is still multiplicative.
\end{lemma}
\begin{proof}
	Let $n$ and $m$ be coprime positive integers. We must show $(f*g)(nm)=(f*g)(n)\cdot(f*g)(m)$. The key point is that there is a bijection between divisors $d\mid nm$ and pairs of divisors $d_n\mid n$ and $d_m\mid m$ by sending $(d_n,d_m)$ to $d$. We quickly show formally that this is a bijection.
	\begin{itemize}
		\item Well-defined: certainly $d_n\mid n$ and $d_m\mid m$ implies $d_nd_m\mid nm$.
		\item Injective: suppose $d_nd_m=d_n'd_m'$ for $d_n,d_n'\mid n$ and $d_m,d_m'\mid m$. We show $d_n=d_n'$, and $d_m=d_m'$ follows by symmetry. Well, for each $p\mid n$, we see $p\nmid m$ because $\gcd(n,m)=1$, so $p\nmid d_m,d_m'$ as well, meaning
		\[\nu_p(d_n)=\nu_p(d_nd_m)=\nu_p(d'_nd'_m)=\nu_p(d'_n)\]
		for all $p\mid n$. However, $p\mid d_n,d_n'$ implies $p\mid n$, so we see that the prime factorizations of $d_n$ and $d_n'$ are the same, so $d_n=d_n'$.
		\item Surjective: for each $d\mid nm$, define $d_n\coloneqq\gcd(d,n)$ and $d_m\coloneqq\gcd(d,m)$. Certainly $d_n\mid n$ and $d_m\mid m$, so it remains to show $d=d_nd_m$. Well, for each $p\mid n$, we see $\nu_p(d_n)=\nu_p(d)$ because $d\mid n$; and similarly, each $p\mid m$ has $\nu_p(d_m)=\nu_p(m)$. Because each prime $p\mid nm$ divides exactly one of $n$ or $m$, we see that
		\[\nu_p(d_nd_m)=\nu_p(d_n)+\nu_p(d_m)=\nu_p(d)\]
		by doing casework on $p\mid n$ or $p\mid m$.
	\end{itemize}
	We have written down all of this so that we may compute
	\begin{align*}
		(f*g)(nm) &= \sum_{d\mid nm}f(d)g(nm/d) \\
		&= \sum_{d_n\mid n}\sum_{d_m\mid m}f(d_nd_m)g\left(\frac n{d_n}\cdot\frac m{d_m}\right) \\
		&\stackrel*= \Bigg(\sum_{d_n\mid n}f(d_n)g(n/d_n)\Bigg)\Bigg(\sum_{d_m\mid m}f(d_m)g(m/d_m)\Bigg) \\
		&= (f*g)(n)\cdot(f*g)(m).
	\end{align*}
	Here, we have used the multiplicativity at $\stackrel*=$, noting that $d_n\mid n$ and $d_m\mid m$ implies $\gcd(d_n,d_m)=1$ because $\gcd(n,m)=1$.
\end{proof}

\subsection{The Mellin Transform}
In this subsection, we pick up a few facts about the Mellin transform. Roughly speaking, we are doing Fourier analysis on the group $\RR^+$ whose operation is multiplication. As such, the Haar measure is $dx/x$: for any Borel set $S\subseteq\RR^+$ and $a\in\RR^+$, we see
\[\int_{aS}\frac{dx}x=\int_S\frac{d(ax)}{ax}=\int_S\frac aa\cdot\frac{dx}x=\int_S\frac{dx}x,\]
so $dx/x$ is in fact a translation-invariant measure on $\RR^+$. Anyway, here is our definition of the Mellin transform.
\begin{definition}[decaying]
	A function $\varphi\colon(0,\infty)\to\CC$ is \textit{decaying at a rate of $(\alpha,\beta)$} for real numbers $\alpha<\beta$ if and only if the functions $x^\alpha f(x)$ and $x^\beta f(x)$ are bounded.
\end{definition}
\begin{definition}[Mellin transform]
	Let $\varphi\colon(0,\infty)\to\CC$ be a continuous function decaying at  a rate of $(\alpha,\beta)$. Then the \textit{Mellin transform} is the function $\mc M\varphi$ given by
	\[(\mc M\varphi)(s)\coloneqq\int_0^\infty\varphi(x)x^s\,\frac{dx}x\]
	for $\alpha<\op{Re}s<\beta$.
\end{definition}
\begin{remark} \label{rem:bound-mellin}
	We quickly check that the integral $\mc M\varphi$ (absolutely) converges for $\alpha<\op{Re}s<\beta$. For each $\gamma\in\{\alpha,\beta\}$, find a constant $C_\gamma\in\RR$ such that $\left|x^\gamma\varphi(x)\right|\le C_\gamma$ for all $x\in(0,\infty)$. For our absolute convergence, we set $\sigma\coloneqq\op{Re}s\in(\alpha,\beta)$ and compute
	\[\int_0^\infty\left|\varphi(x)x^s\right|\,\frac{dx}x\le \int_0^1C_\alpha x^{-\alpha+\sigma-1}\,dx+\int_1^\infty C_\beta x^{-\beta+\sigma-1}\,dx,\]
	so both of the right-hand integrals converge because $-\alpha+\sigma-1>-1$ and $-\beta+\sigma-1<-1$. Notably, this shows that $(\mc M\varphi)$ is uniformly bounded by
	\[\int_0^1C_\alpha x^{-\alpha+\alpha_0-1}\,dx+\int_1^\infty C_\beta x^{-\beta+\beta_0-1}\,dx\]
	whenever $\sigma\in[\alpha_0,\beta_0]$.
\end{remark}
\begin{remark} \label{rem:mellin-to-fourier}
	Fixing some $\sigma\in(\alpha,\beta)$, let $\psi(u)\coloneqq e^{-\sigma u}\varphi\left(e^{-u}\right)$. Provided that $\psi$ is Schwarz, changing variables by $x=e^{-u}$ gives
	\[(\mc M\varphi)(\sigma+2\pi it)=\int_0^\infty\varphi(x)x^{\sigma+2\pi it}\,\frac{dx}x=\int_\RR\varphi\left(e^{-u}\right)e^{-\sigma u-2\pi itu}\,du=(\mc F\psi)(t).\]
\end{remark}
Here is a basic result on the Mellin transform.
\begin{lemma} \label{lem:deriv-mellin-transform}
	Fix a differentiable function $\varphi\colon(0,\infty)\to\CC$ such that $\varphi$ decays at a rate of $(\alpha,\beta)$. Defining $\psi(x)\coloneqq x\varphi'(x)$, for any $\alpha<\op{Re}s<\beta$, the integral defining $(\mc M\psi)(s)$ converges, and
	\[(\mc M\psi)(s)=-s(\mc M\varphi)(s).\]
\end{lemma}
\begin{proof}
	This is by integration by parts. Indeed, we compute
	\begin{align*}
		(\mc M\psi)(s) &= \int_0^\infty x\varphi'(x)x^s\,\frac{dx}x \\
		&= x^s\varphi(x)\bigg|_0^\infty-s\int_0^\infty\varphi(x)x^s\,\frac{dx}x \\
		&= -s(\mc M\varphi)(s),
	\end{align*}
	which is what we wanted. Note, $x^s\varphi(x)\to0$ as $x\to0^+$ and $x\to\infty$ because $\varphi$ decays at a rate of $(\alpha,\beta)$ and $\op{Re}s\in(\alpha,\beta)$.
\end{proof}
We will need two key properties of the Mellin transform.
\begin{proposition} \label{prop:bound-mellin-transform}
	Let $\varphi\colon(0,\infty)\to\CC$ be a continuous function decaying at a rate of $(\alpha,\beta)$. 
	\begin{listalph}
		\item The function $\mc M\varphi$ is holomorphic on $\{s:\alpha<\op{Re}s<\beta\}$.
		\item Suppose that $\varphi$ is infinitely differentiable, and the $n$th derivatives decays at a rate of $(\alpha-n,\beta-n)$. Then for any integer $A\ge0$ and $[\alpha_0,\beta_0]\subseteq(\alpha,\beta)$, the set
		\[\left\{|s|^A(\mc M\varphi)(s):\alpha_0\le\op{Re}s\le\beta_0\right\}\]
		is bounded.
	\end{listalph}
\end{proposition}
\begin{proof}
	These are essentially bounding results.
	\begin{listalph}
		\item We use \Cref{lem:diff-under-sign}. Here, $f(s,t)\coloneqq\varphi(x)x^{s-1}$. We will show that $\mc M\varphi$ is holomorphic on the vertical strip $U\coloneqq\{s:-\alpha_0<\op{Re}s<\beta_0\}$ for any $\alpha<\alpha_0<\beta_0<\beta$, and the result will follow by taking the union over all $\alpha_0$ and $\beta_0$.

		By hypothesis on $\varphi$, we can find a constant $C$ such that $\left|x^\alpha\varphi(x)\right|\le C$ and $\left|x^\beta\varphi(x)\right|\le C$ for each $x$. As such, we set
		\[g(t)\coloneqq\begin{cases}
			0 & x\le0 \\
			C(\log x)x^{-\alpha+\alpha_0-1} & \text{if }x\in(0,1], \\
			C(\log x)x^{-\beta+\beta_0-1} & \text{if }x>1.
		\end{cases}\]
		Note $\int_\RR g(t)\,dt<\infty$ by \Cref{rem:annoying-log-power-ints} because $-\alpha+\alpha_0-1>-1$ and $-\beta+\beta_0-1<-1$. Thus, we see that $x\in(0,1]$ gives
		\[\left|\frac{\del}{\del s}\varphi(x)x^{s-1}\right|=\left|\varphi(x)(\log x)x^{s-1}\right|\le C(\log x)x^{-\alpha+\op{Re}s-1}\le C(\log x)x^{-\alpha+\alpha_0+1},\]
		and similar for $x\in(1,\infty)$ comparing with $\beta_0$. The result now follows from \Cref{lem:diff-under-sign}.
		\item This follows from \Cref{lem:deriv-mellin-transform}. Define $\varphi_0\coloneqq\varphi$ and $\varphi_{n+1}(x)\coloneqq x\varphi_n'(x)$ for each $n$. By induction, $\varphi_n$ decays at a rate of $(\alpha,\beta)$ for each $n$, and for each $n$, we see
		\[\left|s^n(\mc M\varphi)(s)\right|=\left|(\mc M\varphi_n)(n)\right|\]
		by \Cref{lem:deriv-mellin-transform}. However, for each $n$, we see that $(\mc M\varphi_n)$ is uniformly bounded on $[\alpha_0,\beta_0]$ by \Cref{rem:bound-mellin}, which is what we wanted.
		\qedhere
	\end{listalph}
\end{proof}
\begin{theorem} \label{thm:inv-mellin}
	Let $\varphi\colon(0,\infty)\to\CC$ be a function such that $\psi(u)\coloneqq e^{-\sigma u}\varphi\left(e^{-u}\right)$ is Schwarz. For any $\sigma\in\RR$ and $x\in(0,\infty)$, we have
	\[\varphi(x)=\frac1{2\pi i}\int_{\sigma-i\infty}^{\sigma+i\infty}(\mc M\varphi)(s)x^{-s}\,ds.\]
\end{theorem}
\begin{proof}
	We translate everything to the Fourier setting with \Cref{rem:mellin-to-fourier}, where \Cref{thm:fourier-inversion} finishes. Following this outline, we compute
	\begin{align*}
		\frac1{2\pi i}\int_{\sigma-i\infty}^{\sigma+i\infty}(\mc M\varphi)(s)x^{-s}\,ds &= \int_\RR (\mc M\varphi)(\sigma+2\pi it)x^{-\sigma-2\pi it}\,dt \\
		&= x^{-\sigma}\int_\RR (\mc F\psi)(t)e^{2\pi i(-\log x)t}\,dt \\
		&= x^{-\sigma}\cdot\psi(-\log x) \\
		&= \varphi(x),
	\end{align*}
	which is what we wanted.
\end{proof}

\subsection{Finishing Dirichlet's Theorem}
We finish the proof of \Cref{thm:dirichlet}. By \Cref{prop:reduce-dir-to-l} and \Cref{lem:complex-l-chi}, we have left to show $L(1,\chi)\ne0$ for real characters $\chi$. We provide a slick proof of this result.
\begin{lemma}
	Let $\chi\pmod q$ be a ``real'' non-principal Dirichlet character, meaning $\chi=\overline\chi$. We show 
\end{lemma}
\begin{proof}
	We combine two techniques called ``positivity'' and ``smoothing.'' The main point is that $L(1,\chi)=0$ implies that the zero of $L(s,\chi)$ at $s=1$ is able to cancel the pole of $\zeta(s)$ as $s=1$, implying that the function $\zeta(s)L(s,\chi)$ is holomorphic on $\{s:\op{Re}s>0\}$ by combining \Cref{prop:continue-l-chi,prop:continue-zeta}.

	Anyway, we divide the proof in three steps.
	\begin{enumerate}
		\item Let's begin with our positivity result. Because we are interested in $\zeta(s)L(s,\chi)$, we will want to study the coefficients of this Dirichlet series, which are given by $(1*\chi)$ by \Cref{prop:prod-dir-series}. Note $(1*\chi)$ is multiplicative by \Cref{lem:dir-conv-mult}.

		To set up our bounding, we claim that $(1*\chi)(n)\ge0$ for all $n\in\NN$, and $(1*\chi)\left(n^2\right)\ge1$. Because $(1*\chi)$ is multiplicative, we may write
		\[(1*\chi)(n)=(1*\chi)\Bigg(\prod_{p\mid n}p^{\nu_p(n)}\Bigg)=\prod_{p\mid n}(1*\chi)\left(p^{\nu_p(n)}\right).\]
		Thus, it suffices to show $(1*\chi)\left(p^k\right)\ge0$ for each prime-power $p^k$, and $(1*\chi)\left(p^k\right)\ge1$ when $k$ is even. Well, we can compute this directly as
		\[(1*\chi)\left(p^k\right)=\sum_{d\mid p^k}\chi(d)=\sum_{\nu=0}^k\chi\left(p^\nu\right)=\sum_{\nu=0}^k\chi(p)^\nu.\]
		Now, $\chi(p)=\overline{\chi(p)}$ by hypothesis on $\chi$, so because $|\chi(p)|=1$ by \Cref{rem:chars-to-s1}, we conclude $\chi(p)\in\{\pm1\}$. Thus, on one hand, if $\chi(p)=1$, then $(1*\chi)\left(p^\nu\right)=\nu+1\ge1$ always. On the other hand, if $\chi(p)=-1$, then $(1*\chi)\left(p^\nu\right)$ is $1$ when $\nu$ is even and $0$ if $\nu$ is odd. The claim follows.
	
		To finish, our positivity claim is that
		\[\sum_{x<n\le2x}(1*\chi)(n)\ge\sum_{x<n^2\le2x}(1*\chi)\left(n^2\right)\ge\sum_{\sqrt x<n\le\sqrt{2x}}1=\floor{\sqrt{2x}}-\floor{\sqrt x}\ge(\sqrt2-1)\sqrt x-2.\]
		Thus, for $x$ large enough, we see
		\[\sum_{x<n\le2x}(1*\chi)(n)\ge\frac13\sqrt x.\]

		\item We now apply smoothing Let $\varphi\colon(0,\infty)\to[0,\infty)$ be an infinitely differentiable function with support contained in $[0.9,2.1]$ such that $\varphi(x)=1$ for $x\in[1,2]$. Then one sees
		\[\sum_{n=1}^\infty\varphi(n/x)(1*\chi)(m)\ge\sum_{x<n\le2x}(1*\chi)(n)\ge\frac13\sqrt x.\]
		Note that this sum is finite because only finitely many $n$ have $n/x\le2.1$.
	
		We now use the Mellin transform $\mc M\varphi$. Indeed, note that $\varphi$ is decaying at a rate of $(\alpha,\beta)$ for any $\alpha<\beta$ because $\varphi$ is continuous and has compact support, so the function $x^\alpha\varphi(x)$ is continuous and has compact support for any $\alpha$ and is thus bounded.
		
		Further, for any $\sigma>0$, the function $\psi(u)\coloneqq e^{-\sigma u}\varphi\left(e^{-u}\right)$ has compact support and is infinitely differentiable, so $x^k\psi^{(\ell)}(x)$ is continuous of compact support for all $k$ and $\ell$ and hence bounded. Thus, $\psi$ is Schwarz, so we can use \Cref{thm:inv-mellin} to compute
		\[\sum_{n=1}^\infty\psi(n/x)(1*\chi)(n)=\frac1{2\pi i}\sum_{n=1}^\infty\int_{2-i\infty}^{2+i\infty}\left((\mc M\varphi)(s)x^{s}\cdot\frac{(1*\chi)(n)}{n^s}\right)ds.\]
		Thus, we see that we would like to exchange the integral and the sum so that we can sum over $(1*\chi)$ to finally make $\zeta(s)L(s,\chi)$ appear. It suffices to show that this iterated ``integral'' absolutely converges, so for any $\sigma>0$, we may compute
		\[I_\sigma(x)\coloneqq\int_{\sigma-i\infty}^{\sigma+i\infty}\sum_{n=1}^\infty\left|(\mc M\varphi)(s)x^{s}\cdot\frac{(1*\chi)(n)}{n^s}\right|ds=\int_{\sigma-i\infty}^{\sigma+i\infty}\left|(\mc M\varphi)(s)x^s\cdot\zeta(s)L(s,\chi)\right|\,ds\]
		by \Cref{prop:prod-dir-series}. To bound this, we see $\left|x^s\right|\le x^{\op{Re}s}=x^\sigma$ and
		\[|\zeta(s)L(s,\chi)|\le q\cdot\frac{|s|}\sigma\cdot|s|\left(\frac1{|1-\sigma|}+\frac1\sigma\right)=C_0(q,\sigma)|s|^2\]
		by \Cref{rem:bound-l-chi,rem:bound-zeta}, where $C_0(q,\sigma)$ is some constant. Thus,
		\[I_\sigma(x)\le C_0(q,\sigma)x^c\int_{\sigma-i\infty}^{\sigma+i\infty}(|\mc M\varphi)(s)|\left(\sigma^2+(\op{Im}s)^2\right)\,ds.\]
		However, by \Cref{prop:bound-mellin-transform}, there is $C$ such that $|(\mc M\varphi)(s)|\le C|s|^{-4}\le C(\op{Im}s)^{-4}$ on the vertical strip of interest, so we bound
		\begin{align*}
			\frac{I(x)}{C_0(q,\sigma)x^c} &\le 
			C\left(\int_{\sigma-i\infty}^{\sigma-i}\frac{\left(\sigma^2+(\op{Im}s)^2\right)}{(\op{Im}s)^4}\,ds\right)
			+C\left(\int_{\sigma+i}^{\sigma+i\infty}\frac{\left(\sigma^2+(\op{Im}s)^2\right)}{(\op{Im}s)^4}\,ds\right) \\
			&\qquad+\left(\int_{\sigma-i}^{2+i}(|\mc M\varphi)(s)|\left(\sigma^2+(\op{Im}s)^2\right)\,ds\right).
		\end{align*}
		The integrals on the top row are finite by direct computation (they are improper integrals avoiding $0$ of decaying on the order of $x^{-2}$ or faster), and the bottom integral is finite because it is a finite integral of a continuous function. We conclude that $I(x)$ converges, so we have absolute convergence.
		
		In fact, the entire right-hand side of the above bound is merely some function of $\sigma$, so we have actually shown that
		\begin{equation}
			\int_{\sigma-i\infty}^{\sigma+i\infty}\left|(\mc M\varphi)(s)x^s\cdot\zeta(s)L(s,\chi)\right|\,ds\le C(q,\sigma)x^c \label{eq:want-push-left}
		\end{equation}
		for some constant $C(q,\sigma)$.

		\item Anyway, we now know we can write
		\[\frac13\sqrt x\le\frac1{2\pi i}\int_{2-i\infty}^{2+i\infty}\underbrace{(\mc M\varphi)(s)x^{s}\zeta(s)L(s,\chi)}_{D(s)}\,ds\]
		by exchanging the sum and the integral and using \Cref{prop:prod-dir-series}. In order to use \eqref{eq:want-push-left}, we would like to push the vertical line left from $\op{Re}s=2$ to $\op{Re}s=1/3$ (for example).
		
		We will be allowed to do this by Cauchy's theorem because the function $D(s)=(\mc M\varphi)(s)x^s\zeta(s)L(s,\chi)$ is holomorphic on $\{s:\op{Re}s>0\}$. Indeed, the only possible pole among these functions is the pole of order $1$ at $s=1$ for $\zeta(s)$, but $L(s,\chi)$ has a zero there by assumption and thus cancels this out!

		We now apply Cauchy's theorem. For any $T>0$, we see
		\[\left|\int_{1/3-iT}^{1/3+iT}D(s)\,ds-\int_{2-iT}^{2+iT}D(s)\,ds\right|\le\int_{1/3+iT}^{2+iT}|D(s)|\,ds+\int_{1/3-iT}^{2-iT}|D(s)|\,ds.\]
		We would like to show that this right-hand side vanishes as $T\to\infty$. Because the length of each of these paths is finite, it suffices to show that $|D(s)|$ vanishes as $\op{Im}s\to\infty$ on these paths. Well, utilizing our bounds from before, we see
		\[|D(s)|\le|(\mc M\varphi)(s)|\cdot x^2\cdot C_0(q,\sigma)\left(4+(\op{Im}s)^2\right).\]
		Because $(\mc M\varphi)(s)$ is rapidly decaying as $\op{Im}s\to\infty$ (recall \Cref{prop:bound-mellin-transform}), we see that this indeed goes to $0$ as $\op{Im}s\to\infty$.

		In total, we see
		\[\frac13\sqrt x\le\frac1{2\pi i}\int_{2-i\infty}^{2+i\infty}D(s)\,ds=\frac1{2\pi i}\int_{1/3-i\infty}^{1/3+i\infty}D(s)\,ds\le C(q,1/3)x^{1/3},\]
		where we have used \eqref{eq:want-push-left} at the end. However, for $x$ large enough, this is impossible: $x^{1/2-1/3}\to\infty$ as $x\to\infty$. So we have hit our contradiction.
		\qedhere
	\end{enumerate}
\end{proof}
\begin{remark}
	The product $\zeta(s)L(s,\chi)$ is the Dedekind $\zeta$-function associated to a real quadratic field.
\end{remark}

\subsection{A Little on Quadratic Forms}
To say something in the direction of Dirichlet's class number formula, we discuss quadratic forms. In particular, we will discuss the reduction theory, which shows that there are finitely many classes of binary quadratic forms of given discriminant.
\begin{definition}[binary quadratic form]
	A \textit{binary quadratic form} is a function $f\colon\ZZ^2\to\ZZ$ where $f(x,y)\coloneqq ax^2+bxy+cy^2$ where $a,b,c\in\ZZ$. If $\gcd(a,b,c)=1$, then we call the quadratic form \textit{primitive}.
\end{definition}
It is a problem of classical interest to determine when a quadratic form achieves a particular integer.

It is another problem of classical interest to count the number of binary quadratic forms. However, some binary quadratic forms are ``the same,'' in the sense that they are just a variable change away.
\begin{example}
	The quadratic forms $x_1^2+x_2^2$ and $y_1^2+2y_1y_2+2y_2^2$ are roughly the same by the change of variables given by
	\[(y_1,y_2)=(x_1-x_2,x_2).\]
\end{example}
To define this correctly, we define a group action on the set of quadratic forms.
\begin{lemma}
	Let $\mc Q$ be the set of binary quadratic forms. Then $\op{SL}_2(\ZZ)$ acts on the set of binary quadratic forms by
	\[(\gamma\cdot f)\coloneqq f\circ\gamma^{-1},\]
	where $f\in\mc Q$ and $\gamma\in\op{SL}_2(\ZZ)$.
\end{lemma}
\begin{proof}
	We have the following checks.
	\begin{itemize}
		\item Identity: note $({\id}\cdot f)=f\circ{\id}^{-1}=f\circ{\id}=f$.
		\item Composition: note $((\gamma\gamma')\cdot f)=f\circ(\gamma\gamma')^{-1}=f\circ(\gamma')^{-1}\circ\gamma^{-1}=\gamma\cdot(\gamma'\cdot f)$.
		\qedhere
	\end{itemize}
\end{proof}
\begin{definition}[equivalent]
	Two binary quadratic forms $f_1,f_2\colon\ZZ^2\to\ZZ$ are \textit{equivalent} if and only if $f_1$ and $f_2$ live in the same orbit under the $\op{SL}_2(\ZZ)$-action. In other words, $f_1$ and $f_2$ are equivalent if and only if there exists $\gamma\in\op{SL}_2(\ZZ)$ such that
	\[f_1=f_2\circ\gamma.\]
\end{definition}
Note that this is in fact an equivalence relation because the orbits of a group action form a partition.
\begin{remark} \label{rem:get-matrix-from-form}
	For a binary quadratic form $f(x,y)\coloneqq ax^2+bxy+cy^2$, note that
	\[f(v)=ax^2+bxy+cy^2=\begin{bmatrix}
		x & y
	\end{bmatrix}\underbrace{\begin{bmatrix}
		a & b/2 \\
		b/2 & c
	\end{bmatrix}}_{M\coloneqq}\begin{bmatrix}
		x \\ y
	\end{bmatrix}=v^\intercal Mv\]
	for any $v=(x,y)\in\ZZ^2$. In fact, this symmetric matrix $M$ is unique to $f$: if $v^\intercal Mv=v^\intercal M'v$ for all $v=(x,y)\in\ZZ^2$, then writing $M=(a_{ij})$ and $M'=(a'_{ij})$, we see
	\[a_{11}x^2+2a_{12}xy+a_{22}^2=v^\intercal Mv=v^\intercal M'v=a'_{11}x^2+2a'_{12}xy+a'_{22}y^2.\]
	Plugging in $(x,y)\in\{(1,0),(0,1),(1,1)\}$ shows $M=M'$.
\end{remark}
\begin{remark} \label{rem:matrix-of-equiv-form}
	Associate a binary quadratic form $f$ the matrix $M$ as in \Cref{rem:get-matrix-from-form}. Thus, for any $\gamma\in\op{SL}_2(\ZZ)$,
	\[(\gamma\cdot f)(v)=f\left(\gamma^{-1}v\right)=\left(\gamma^{-1}v\right)^\intercal M\gamma^{-1}v=v^\intercal\left(\gamma^{-\intercal}M\gamma^{-1}\right)v,\]
	so we can associate $\gamma\cdot f$ to the matrix $\gamma^{-\intercal}M\gamma^{-1}$. (Notably, this is still a symmetric matrix!) This allows for relatively easy computation of $\gamma\cdot f$.
\end{remark}
So we would like to count the number of quadratic forms, up to equivalence. However, we will soon see that there are still infinitely many of equivalence classes, so we will want some stronger invariant to distinguish between them.
\begin{definition}[discriminant]
	The \textit{discriminant} of the binary quadratic form $f(x,y)\coloneqq ax^2+bxy+cy^2$ is given by $\op{disc}f\coloneqq b^2-4ac$. The number of equivalence classes of quadratic forms of discriminant $d$ is notated by $h(-d)$.
\end{definition}
\begin{remark} \label{rem:disc-on-equiv-class}
	By definition, note that the discriminant of the binary quadratic form $f$ is $4\det M$, where $M$ is the matrix associated to $f$ as in \Cref{rem:get-matrix-from-form}. Using \Cref{rem:matrix-of-equiv-form}, we see that the discriminant of $\gamma\cdot f$ is thus
	\[4\det\left(\gamma^{-\intercal}\right)\det(M)\det\left(\gamma^{-1}\right)=4\det M\]
	for any $\gamma\in\op{SL}_2(\ZZ)$.
\end{remark}
\Cref{rem:disc-on-equiv-class} shows that the discriminant is invariant to equivalence class. Thus, for example, for each $d\in\ZZ$, we set
\[f_d(x,y)\coloneqq dxy\]
so that $\op{disc}f=d^2$. Now letting $d$ vary of $\ZZ$, we see that there are infinitely many equivalence classes of quadratic forms.

But once we bound our discriminant, there will be finitely many quadratic forms. Here is our goal.
\begin{restatable}{theorem}{classnumberfinite} \label{thm:class-number-finite}
	Let $d<0$ be an integer. Then $h(d)$ is finite.
\end{restatable}
\begin{remark}
	It is also true that $h(d)$ is finite when $d\ge0$, but we will not show it here.
\end{remark}

\subsection{The Upper-Half Plane}
To show \Cref{thm:class-number-finite}, we will want to relate the action of $\op{SL}_2(\ZZ)$ on quadratic forms with the action of $\op{SL}_2(\RR)$ on $\mathbb H\coloneqq\{z\in\CC:\op{Im}z>0\}$ given by
\[\begin{bmatrix}
	a & b \\
	c & d
\end{bmatrix}z\coloneqq\frac{az+b}{cz+d}.\]
Here are some checks on this action.
\begin{lemma} \label{lem:basic-upper-half-plane}
	Let $\HH\coloneqq\{z\in\CC:\op{Im}z>0\}$ denote the upper-half plane.
	\begin{listalph}
		\item The group $\op{SL}_2(\RR)$ acts on $\HH$ by
		\[\begin{bmatrix}
			a & b \\
			c & d
		\end{bmatrix}z\coloneqq\frac{az+b}{cz+d}.\]
		\item The orbit of $i\in\HH$ under $\op{SL}_2(\RR)$ is all of $\HH$.
		\item The stabilizer of $i\in\HH$ is $\op{SO}_2(\RR)$, the group of rotations.
	\end{listalph}
\end{lemma}
\begin{proof}
	We show the parts one at a time.
	\begin{listalph}
		\item To begin, we show that the action is well-defined: given $z$ with $z\in\HH$, we need to show that $\gamma\cdot z\in\HH$ for any $\gamma\in\op{SL}_2(\RR)$. Well, giving coefficients to $\gamma$, we compute
		\[\gamma\cdot z=\begin{bmatrix}
			a & b \\
			c & d
		\end{bmatrix}z=\frac{az+b}{cz+d}=\frac{(az+b)(c\overline z+d)}{|cz+d|^2}=\frac{\left(ac|z|^2+bd\right)+(adz+bc\overline z)}{|cz+d|^2}.\]
		To check $\gamma\cdot z\in\HH$, we must check that the imaginary part here is positive. Well, we see
		\[\op{Im}(\gamma\cdot z)=\frac{(ad-bc)\op{Im}(z)}{|cz+d|^2}=\frac{\op{Im}(z)}{|cz+d|^2},\]
		where the last equality is because $\det\gamma=1$.

		We now run our checks to have a group action.
		\begin{itemize}
			\item Identity: we compute
			\[\begin{bmatrix}
				1 & 0 \\
				0 & 1
			\end{bmatrix}z=\frac{z+0}{0+1}=z.\]
			\item Composition: we compute
			\begin{align*}
				\begin{bmatrix}
					a & b \\
					c & d
				\end{bmatrix}\left(\begin{bmatrix}
					a' & b' \\
					c' & d'
				\end{bmatrix}z\right) &= \begin{bmatrix}
					a & b \\
					c & d
				\end{bmatrix}\frac{a'z+b'}{c'z+d'} \\
				&= \frac{a\cdot\frac{a'z+b'}{c'z+d'}+b}{c\cdot\frac{a'z+b'}{c'z+d'}+d} \\
				&= \frac{a(a'z+b')+b(c'z+d')}{c(a'z+b')+d(c'z+d')} \\
				&= \frac{(aa'+bc')z+(ab'+bd')}{(ca'+dc')z+(cb'+dd')} \\
				&= \begin{bmatrix}
					aa'+bc' & ab'+bd' \\
					ca'+dc' & cb'+dd'
				\end{bmatrix}z \\
				&= \left(\begin{bmatrix}
					a & b \\
					c & d
				\end{bmatrix}\begin{bmatrix}
					a' & b' \\
					c' & d'
				\end{bmatrix}\right)z.
			\end{align*}
		\end{itemize}

		\item Giving coefficients to some $\gamma\in\op{SL}_2(\ZZ)$, we use the computation in (a) to see
		\[\gamma\cdot i=\begin{bmatrix}
			a & b \\
			c & d
		\end{bmatrix}i=\frac{\left(ac|i|^2+bd\right)+(adi+bc\overline i)}{|ci+d|^2}=\frac{(ac+bd)+(ad-bc)i}{c^2+d^2}=\frac{(ac+bd)+i}{c^2+d^2}.\]
		Thus, for any $a+bi\in\HH$, we see
		\[\begin{bmatrix}
			\sqrt b & a/\sqrt b \\
			0 & 1/\sqrt b
		\end{bmatrix}i=\frac{a/b+i}{1/b}=a+bi,\]
		so the orbit of $i$ is indeed all of $\HH$.

		\item Using the computation of (b), we see that $\gamma\cdot i=i$ if and only if the usual coefficients of $\gamma$ have $ac+bd=0$ and $c^2+d^2=1$. Thus, we see that any $\theta\in[0,2\pi)$ will give
		\[\begin{bmatrix}
			\cos\theta & -\sin\theta \\
			\sin\theta & \cos\theta
		\end{bmatrix}i=i\]
		because $(\cos\theta)(\sin\theta)+(\cos\theta)(-\sin\theta)=0$ and $(\cos\theta)^2+(\sin\theta)^2=1$. It follows that $\op{SO}_2(\RR)$ is certainly contained in the stabilizer of $i$.

		Conversely, suppose $\gamma$ stabilizes $i$ and has the usual coefficients. Note that the pair $(c,d)$ with $c^2+d^2=1$ has a unique $\theta\in[0,2\pi)$ such that $c=\sin\theta$ and $d=\cos\theta$. To solve for $a$ and $b$, we divide our work in two cases.
		\begin{itemize}
			\item If $c\ne0$, then we see $a=-bd/c$. Further, $ad-bc=1$, so we see $-bd^2/c-bc=1$, which gives
			\[b=-\frac1{d^2/c+c}=-\frac c{c^2+d^2}=-c=-\sin\theta.\]
			Thus, we see $a=-bd/c=d=\cos\theta$. Plugging everything in, we see $\gamma\in\op{SO}_2(\RR)$.
			\item If $c=0$, then $d\ne0$, so we see $b=-ac/d$. Thus, $ad-bc=1$, so we see $ad+ac^2/d=1$, which gives
			\[a=\frac1{d+c/d}=\frac d{c^2+d^2}=d=\cos\theta.\]
			Thus, we see $b=-ac/d=-c=-\sin\theta$. Plugging everything in, we again see $\gamma\in\op{SO}_2(\RR)$.
		\end{itemize}
		The above cases complete the proof.
		\qedhere
	\end{listalph}
\end{proof}
% One can check that this is in fact a group action. As such, we will as an intermediate goal try to find a fundamental domain for the action of $\op{SL}_2(\ZZ)\subseteq\op{SL}_2(\RR)$ on $\HH$.
\begin{remark} \label{rem:h-as-quotient}
	Parts (b) and (c) of \Cref{lem:basic-upper-half-plane} roughly show
	\[\frac{\op{SL}_2(\RR)}{\op{SO}_2(\RR)}\cong\HH.\]
\end{remark}
Next class we will discuss how to build a fundamental domain for the induced action of $\op{SL}_2(\ZZ)\subseteq\op{SL}_2(\RR)$ on $\HH$.

\end{document}