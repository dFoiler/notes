% !TEX root = ../notes.tex

\documentclass[../notes.tex]{subfiles}

\begin{document}

\section{January 30}

Today we actually prove the functional equation.

\subsection{The Functional Equation}
The functional equation will be derived from the functional equation for the $\Theta\colon\CC\to\CC$ function, defined by
\[\Theta(x)\coloneqq\sum_{n\in\ZZ}e^{-\pi n^2x}.\]
In particular, \Cref{thm:ps} implies
\[\Theta(x)=\sum_{n\in\ZZ}e^{-\pi n^2x}=\sum_{n\in\ZZ}e^{-\pi\left(n\sqrt x\right)^2}=\sum_{n\in\ZZ}\frac1{\sqrt x}e^{-\pi\left(n/\sqrt x\right)^2}=\frac1{\sqrt x}\Theta\left(\frac1x\right)\]
for $x>0$. The uniqueness of analytic continuation now implies $\Theta(z)=\frac1{\sqrt z}\Theta(z)$ for all $z\in\CC$.
\begin{remark}
	For $z\in\HH$, we note that $\Theta(z+2)=\Theta(z)$ and $\Theta(-1/z)=\sqrt{z/i}\cdot\Theta(z)$, which shows that $\Theta$ is a modular form of weight $1/2$ and level
	\[\left\langle\begin{bmatrix}
		1 & 2 \\
		0 & 1
	\end{bmatrix},\begin{bmatrix}
		0 & 1 \\
		-1 & 0
	\end{bmatrix}\right\rangle.\]
\end{remark}
We now use $\Theta$ for fun and profit. Consider the integral
\[\int_0^\infty\left(\frac{\Theta(x)-1}2\right)x^{s/2}\,\frac{dx}x.\]
This will converge absolutely for $\op{Re}s>1$. However, for $\op{Re}s>1$, we can unwind this expression as
\begin{align*}
	\int_0^\infty\left(\frac{\Theta(x)-1}2\right)x^{s/2}\,\frac{dx}x &= \int_0^\infty\Bigg(\sum_{n=1}^\infty e^{-\pi n^2x}x^{s/2}\Bigg)\frac{dx}x \\
	&= \zeta(s)\int_0^\infty e^{-\pi x}x^{s/2}\,\frac{dx}x \\
	&= \zeta(s)\pi^{-s/2}\Gamma(s/2)
\end{align*}
after enough rearranging. (Note we interchanged the sum and integral validly because everything absolutely converges.) Now, we define
\[\xi(s)\coloneqq\zeta(s)\pi^{-s/2}\Gamma(s/2).\]
The key result is as follows.
\begin{theorem}
	The function $\Xi(s)\coloneqq\xi(s)s(1-s)$ has an analytic continuation to all of $\CC$ and satisfies the functional equation
	\[\xi(s)=\xi(1-s).\]
\end{theorem}
\begin{proof}
	The trick is to break the integral
	\[\int_0^\infty\left(\frac{\Theta(x)-1}2\right)x^{s/2}\,\frac{dx}x\]
	into pieces $(0,1)$ and $(1,\infty)$.
	\begin{itemize}
		\item On $(1,\infty)$, we see
		\[I(s)\coloneqq\int_1^\infty\left(\frac{\Theta(x)-1}2\right)dx\]
		converges absolutely to a holomorphic function.
		\item We deal with $(0,1)$. Here,
		\[\int_0^1\frac{x^{s/2}}2\,\frac{dx}x=\frac1s,\]
		and for the $\Theta$ term, we use modularity to write
		\begin{align*}
			\int_0^1\left(\frac{\Theta(x)}2\right)x^{s/2}\,\frac{dx}x &= \int_0^1\left(\frac{\Theta(1/x)}{2\sqrt x}\right)x^{s/2}\,\frac{dx}x \\
			&= \int_1^\infty\left(\frac{\Theta(x)}2\right)x^{(1-s)/2}\,\frac{dx}x \\
			&= I(1-s)-\frac1{1-s}.
		\end{align*}
		Summing our two pieces, we get $I(1-s)-\frac1s-\frac1{1-s}$.
	\end{itemize}
	In total, we see
	\[\xi(s)=I(s)+I(1-s)-\frac1s-\frac1{1-s},\]
	where $I(s)$ is holomorphic. This completes the proof.
\end{proof}

\subsection{Zeroes of \texorpdfstring{$\zeta$}{ Zeta}}
We would like to understand the zeroes of $\zeta(s)$, for which we use Cauchy's formula. Roughly speaking, we will study integrals
\[\frac1{2\pi i}\oint_\gamma\frac{\zeta'(s)}{\zeta(s)}\,ds,\]
where $\gamma$ is a contour over a very tall vertical strip in $\CC$.

We will want the following bounds.
\begin{lemma}[Stirling]
	We have
	\[\log\Gamma(s)=\left(s-\frac12\right)\log s+\frac12\log2\pi+O_\delta(1/|s|),\]
	and
	\[\frac{\Gamma'(s)}{\Gamma(s)}=\log s+O_\delta(1/|s|)\]
	as $|s|\to\infty$ for $-\pi+\delta<\arg s<\pi-\delta$.
\end{lemma}
\begin{proof}
	Omitted.
\end{proof}
\begin{remark}
	For $\sigma\in[a,b]$ for some fixed $a<b$, we have $\Gamma(\sigma+it)\sim e^{-\pi|t|/2}|t|^{\sigma-1/2}$ as $\to\infty$. Again, we omit this proof.
\end{remark}
We also want access to the Hadamard factorization theorem.
\begin{theorem}
	Fix an entire function $\varphi\colon\CC\to\CC$. Let $n(r)$ denote the number of zeroes $z$ such that $|z|<r$ and $\alpha$ the order of an entire function $\varphi$, one has $n(R)=O_\varepsilon\left(R^{\alpha+\varepsilon}\right)$ for any $\varepsilon>0$.
\end{theorem}
Notably, for $\op{Re}s>1/2$, one has $|s(s-1)\zeta(s)|\ll|s|^3$. Further, one can check that $\Gamma$ has order $1$ as an entire function, so $s(1-s)\xi(s)$ has order at most $1$. Thus, Hadamard's factorization theorem enforces
\[s(1-s)\xi(s)=e^{A+Bs}\prod_{\zeta(\rho)=0}\left(\left(1-\frac s\rho\right)e^{s/\rho}\right).\]
Notably, this product will converge absolutely. For example, absolute convergence tells us
\[\sum_{\zeta(\rho)=0}\frac1{|\rho|^{1+\varepsilon}}<\infty\]
for any $\varepsilon>0$. One also has the following result on the distribution of our $\rho$.
\begin{theorem}
	Define
	\[N(T)\coloneqq\#\{\rho:0\le\op{Re}\rho\le1,\op{Im}\rho\ge0\zeta(\rho)=0\}.\]
	Then
	\[N(T)=\frac T{2\pi}\log\left(\frac T{2\pi}\right)-\frac T{2\pi}+O(\log T)\]
	as $T\to\infty$.
\end{theorem}
We will first show the following lemma.
\begin{lemma}
	We have
	\[\sum_\rho\frac1{1+|\op{Im}\rho-T|^2}\ll\log(T+3).\]
\end{lemma}
\begin{proof}
	This is by smoothing. By taking logarithmic differentiation
	\[\frac{\Xi'(s)}{\Xi(s)}=B+\sum_\rho\left(\frac1{s-\rho}+\frac1\rho\right).\]
	However, by using the above estimates, we see
	\[\frac{\Xi'(s)}{\Xi(s)}=\frac1s-\frac1{1-s}-\frac12\log\pi+\frac12\frac{\Gamma'(s/2)}{\Gamma(s/2)}+\frac{\zeta'(s)}{\zeta(s)}\]
	by definition of $\xi$. Now, the term $\zeta'/\zeta$ is well-behaved for $\op{Re}s$ large: we set $s\coloneqq2+it$, and one can see that $|\zeta'(s)/\zeta(s)|$ is bounded by an absolute constant. Thus, we understand on the right-hand side here.

	Continuing, we see
	\[\op{Re}\left(\frac1{s-\rho}+\frac1\rho\right)=\frac{2-\op{Im}s}{(2-\op{IRe}s)^2+(T-\op{Im}s)^2}+\frac\beta{(\op{Re}s+\op{Im}s)^2}\gg\frac1{1+|T-\op{Im}s|^2}.\]
	However, $\Gamma'(s)/\Gamma(s)\ll\log(T+3)$ by Stirling, so the result follows.
\end{proof}

\end{document}