% !TEX root = ../notes.tex

\documentclass[../notes.tex]{subfiles}

\begin{document}

\section{January 25}

Today we continue discussing quadratic forms.

\subsection{A Fundamental Domain}
Recall from \Cref{rem:h-as-quotient} that
\[\frac{\op{SL}_2(\ZZ)}{\op{SO}_2(\RR)}\cong\HH.\]
Now, $\op{SL}_2(\ZZ)\subseteq\op{SL}_2(\RR)$ has a natural action on $\HH$; this is a ``discrete subgroup,'' so one might say that the action is discrete. (Concretely, we can see that the orbit of any $z\in\HH$ under the action of $\op{SL}_2(\ZZ)$ is a discrete set.) We will be interested in a fundamental domain for the action of $\op{SL}_2(\ZZ)$ on $\HH$. Here is an example.
\begin{proposition} \label{prop:fund-domain}
	Define the subset
	\[D\coloneqq\{z\in\HH:|z|>1,-1/2\le\op{Re}z<1/2\}\cup\{z\in\HH:|z|=1,-1/2\le\op{Re}z\le0\}.\]
	Then $D$ is a fundamental domain for the action of $\op{SL}_2(\ZZ)$ on $\HH$. In other words, for each $z\in\HH$, there exists a unique $z_0\in\HH$ such that there exists $\gamma\in\op{SL}_2(\ZZ)$ such that $z=\gamma\cdot z_0$.
\end{proposition}
\begin{proof}
	Omitted. Roughly speaking, one has to show that $\op{SL}_2(\ZZ)$ is generated by the elements
	\[S\coloneqq\begin{bmatrix}
		0 & 1 \\
		-1 & 0
	\end{bmatrix}\qquad\text{and}\qquad T\coloneqq\begin{bmatrix}
		1 & 1 \\
		0 & 0
	\end{bmatrix}.\]
	Then one can use $T$ to push all elements of $\HH$ to $\{z\in\HH:-1\le\op{Re}z<1\}$ and use $S$ to push what's left over to $S$. We refer to \cite{course-arithmetic} for details.
\end{proof}

\subsection{Gauss Reduced Forms}
We now use \Cref{prop:fund-domain} for fun and profit.
\classnumberfinite*
\begin{proof}
	Roughly speaking, a quadratic form $f(x,y)\coloneqq ax^2+bxy+cz^2$ where $a,c>0$ without loss of generality, we can study $f(x,1)$ to have a root
	\[z_f\coloneqq\frac{-b+\sqrt{b^2-4ac}}{2a}=\frac{-b+\sqrt{d}}{2a}.\]
	Now, in our case of interest, we have $d<0$, so this describes an element of $\HH$. (There is also a negative root, but we focus on $z_f$.) In fact, one can check that $z_{\gamma f}=\gamma z_f$, which is how we relate quadratic forms to $\HH$.

	In fact, by \Cref{prop:fund-domain}, we know there is some $\gamma f$ such that $z_{\gamma f}\in D$. The point here is that the number of quadratic forms up to equivalence is bounded above by the number of points in $D$ with imaginary part $\sqrt{|d|}$. For example, the condition $|z_f|\ge1$ implies that
	\[\frac{b^2-d}{4a^2}=\frac ca,\]
	so $a\le c$. Further, the condition $-1/2\le\op{Re}z\le1/2$ implies $|b|\le2a$. Thus, we are counting the number of triples $(a,b,c)$ with $a,c>0$ such that $b^2-4ac=d$ and $|b|\le a\le c$, which we can see immediately is finite. Indeed, $b^2\le d$, so there are only finitely many possible $b$, but then for each $b$, we see $4ac=b^2-d$, so there are only finitely many possible $a$ and $c$.
\end{proof}
\begin{remark}
	A quadratic form satisfying the above conditions on $a,b,c$ is called ``Gauss reduced.''
\end{remark}

\subsection{Dirichlet's Class Number Formula}
We take a moment to record Dirichlet's class number formula for completeness, though we will not prove it.
\begin{theorem}[class number formula]
	Let $d$ be a ``fundamental discriminant,'' meaning that $d\equiv1\pmod4$ and is squarefree or $d=4q$ where $q\equiv2,3\pmod 4$ and is squarefree. Let $\chi_d=\left(\frac d\bullet\right)$ be the Kronecker symbol.
	\begin{listalph}
		\item If $d<0$,
		\[h(d)=\frac{w_d|d|^{1/2}}{2\pi}\cdot L(1,\chi_d),\]
		where $w_d=2$ if $d<-4$ and $w_d=4$ if $d=-r$ and $w_d=6$ if $d=-3$. (Namely, $w_d$ is the number of roots of unity in $\QQ(\sqrt d)$.)
		\item If $d>0$, then
		\[h(d)\log\varepsilon_d=|d|^{1/2}L(1,\chi_d),\]
		where $\varepsilon_d$ is a fundamental unit for $\OO_{\QQ(\sqrt d)}$. (Namely, $\varepsilon_d=(t_0+u_0\sqrt d)/2$ yields the least positive solution to $t_0^2-du_0^2=4$.)
	\end{listalph}
\end{theorem}
The point behind the fundamental discriminant is that $\op{disc}\OO_{\QQ(\sqrt d)}=d$.
\begin{remark}
	The interested should now be able to do the first part of the first problem set.
\end{remark}

\end{document}