% !TEX root = ../notes.tex

\documentclass[../notes.tex]{subfiles}

\begin{document}

\section{January 20}

A syllabus was posted. There are some extra references posted.

\subsection{Dirichlet \texorpdfstring{$L$}{ L}-Functions}
We continue discussing the Dirichlet $L$-functions
\[L(s,\chi)=\sum_{n=1}^\infty\frac{\chi(n)}{n^s}\]
from last class. In particular, we move towards a proof of \Cref{thm:dirichlet}.
\begin{remark}
	Note $L(s,\chi)\ne0$ for $\op{Re}s>1$ due to the Euler product factorization.
\end{remark}
Arguing as in \Cref{thm:inf-primes}, we see that
\[\log L(s,\chi)=\sum_{p\text{ prime}}-\log\big(1-\chi(p)p^{-s}\big)=\sum_{p\text{ prime}}\Bigg(\sum_{k=1}^\infty\frac{\chi(p)^k}{kpk^{ks}}\Bigg)\]
for $\op{Re}s>1$. By isolating the $k>1$ terms in the same way, we see
\[\log L(s,\chi)=\Bigg(\sum_{p\text{ prime}}\frac{\chi(p)}{p^s}\Bigg)+O(1)\]
for $\op{Re}s>1$. Now, the idea is that we can sum over characters $\chi$ to get a sum over primes $p\equiv a\pmod q$, as described in \Cref{cor:indicate-g}.

Thus, we are motivated to look at how $L(s,\chi)$ behaves as $s\to^+1$. In fact, for non-principal characters, we can easily extend to $\op{Re}s>0$.
\begin{proposition}
	Fix a non-principal Dirichlet character $\chi\pmod q$. Then $L(s,\chi)$ has an analytic continuation to $\op{Re}s>0$.
\end{proposition}
\begin{proof}
	The idea is that
	\[S(x)\coloneqq\sum_{n=1}^{\floor x}\chi(n)=O(1)\]
	for any $N$. As such, we apply summation by parts. Let $L_n(s,\chi)$ denote the $n$th partial sum
	\[L(s,\chi)=\sum_{n=1}^\infty\frac{\chi(n)}{n^s}=\int_1^\infty x^{-s}\,dS(n)=s\int_1^\infty S(x)x^{-s-1}\,dx\]
	by integrating by parts via Riemann--Stieltjes integration. Thus, because $S(x)$ is bounded, we see that the integral converges absolutely for $\op{Re}s>0$.
\end{proof}
\begin{remark}
	It follows from the proof that, for any $\sigma_0>0$, we see
	\[L(s,\chi)<C(\sigma_0)\cdot(|s|+1)q\]
	for $\op{Re}s>\sigma_0$, where $C(\sigma_0)\in\RR$ is some constant.
\end{remark}
And here is the situation for the principal character. Here, we can quickly reduce to the situation for $\zeta$.
\begin{proposition} \label{prop:principal-character}
	Let $\chi_0\pmod q$ be the principal character. Then
	\[L(s,\chi_0)=\Bigg(\prod_{p\mid q}\left(1-p^{-s}\right)\Bigg)\zeta(s)\]
\end{proposition}
\begin{proof}
	Expand with the Euler product.
\end{proof}
And now let's extend $\zeta$.
\begin{proposition}
	We provide a meromorphic continuation for $\zeta(s)$ over $\op{Re}s>0$, where we have a pole of order $1$ and residue $1$ at $s=1$.
\end{proposition}
\begin{proof}
	Integrating by parts again, we see
	\begin{align*}
		\zeta(s) &= \sum_{n=1}^\infty\frac1{n^s} \\
		&= \int_1^\infty x^{-s}\,d\floor x \\
		&= x^{-s}\floor x\bigg|_1^\infty+\int_1^\infty sx^{-s-1}\floor x\,dx \\
		&= 0+\int_1^\infty sx^{-s}\,dx+\int_1^\infty s\{x\}x^{-s-1}\,dx \\
		&= \frac s{s-1}+\int_1^\infty s\{x\}x^{-s-1}\,dx.
	\end{align*}
	The left term here is analytic for $\op{Re}s>0$ with the prescribed pole behavior, and the right term is some bounded analytic function. In fact, we can upper-bound the extra term by $c(\sigma_0)(|s|+1)$, where $\op{Re}s>\sigma_0>0$.
\end{proof}
\begin{remark}
	Doing repeated integration by parts, one can extend the continuations provide above further to the left, but we will not do this. Instead, we will use a functional equation to continue to the left in one fell swoop.
\end{remark}

\subsection{Towards \texorpdfstring{$L(1,\chi)$}{ L(1,chi)}}
We now attack \Cref{thm:dirichlet} directly. Take $\op{Re}s>1$. Working in $G\coloneqq(\ZZ/q\ZZ)^\times$, \Cref{cor:indicate-g} tells us
\[1_{[a]}(p)=\frac1{\varphi(q)}\sum_{\chi\in\widehat G}\overline{\chi(a)}\chi(p),\]
so it follows from our preceding discussion that
\begin{equation}
	\sum_{p\equiv a\pmod q}\frac1{p^s}=\frac1{\varphi(q)}\sum_{\chi\in\widehat G}\overline{\chi(a)}\log L(s,\chi)+O(1). \label{eq:reduce-to-l}
\end{equation}
Thus, we want to study what happens to $L(s,\chi)$ as $s\to1^+$. Philosophically, if $L(s,\chi)$ is not zero or a pole at $s=1$, then we can legally take $\log L(s,\chi)$ and push the result into the $O(1)$ term.

By \Cref{prop:principal-character}, we know that $L(s,\chi_0)\to\infty$ as $s\to1^+$ because $\zeta(s)\to1^+$. More precisely, we can write
\[\frac1{\varphi(q)}\log L(s,\chi_0)=\frac1{\varphi(q)}\log\zeta(s)+O_q(1)=-\frac1{\varphi(q)}\log(1-s)+O_q(1).\]
It remains to deal with the other characters, which is the meat of the proof. Notably, we know $L(1,\chi)$ is a legitimate complex number. Our goal is to show $L(1,\chi)\ne0$ for non-principal $\chi$ because otherwise we get another pole in \Cref{eq:reduce-to-l}, which could potentially cancel out the contribution of $L(s,\chi_0)$. As an outline, for ``complex'' characters $\chi$, we are able to compare $\chi$ and $\overline\chi$ to enforce $L(1,\chi)\ne0$, but the real characters require more ingenuity.

Let's see more explicitly what happens with complex characters. Suppose $\chi\ne\overline\chi$ so that $\chi(G)\not\subseteq\RR$. If $L(1,\chi)=0$, then we see $L(1,\overline\chi)=0$ as well, so
\[\sum_{p\equiv a\pmod q}\frac1{p^s}=\frac1{\varphi(q)}\sum_{\chi\in\widehat G}\overline{\chi(a)}\log L(s,\chi)+O(1),\]
so the right-hand side goes to $-\infty$ as $s\to1^+$. But this doesn't make any sense because the left-hand side is surely positive. It remains to deal with the real characters.

\end{document}