% !TEX root = ../notes.tex

\documentclass[../notes.tex]{subfiles}

\begin{document}

\section{January 25}

We now shift gears and move towards the Prime number theorem. Today, we begin by discussing Riemann's original paper on the topic.
\begin{remark}
	For the rest of this course, any sum or product over an unnamed $p$ will be a sum over primes.
\end{remark}

\subsection{The Statement}
So far we have established the following facts about $\zeta$.
\begin{itemize}
	\item By \Cref{cor:euler-factor-zeta}, for $\op{Re}s>1$, there is an Euler product factorization
	\[\zeta(s)=\sum_{n=1}^\infty\frac1{n^s}=\prod_p\frac1{1-p^{-s}}.\]
	\item By \Cref{prop:continue-zeta}, there is a meromorphic continuation of $\zeta(s)$ to $\op{Re}s>1$, where $\zeta(s)$ is analytic everywhere except for a pole of order $1$ at $s=1$.
\end{itemize}
Roughly speaking, we will show the Prime number theorem by being able to study $\zeta'(s)/\zeta(s)=\frac d{ds}\log\zeta(s)$. Let's establish some notation.
\begin{definition}
	For $x\in\RR$, we define the following functions.
	\begin{align*}
		\pi(x) &\coloneqq \sum_{p\le x}1, \\
		\Lambda(n) &\coloneqq\begin{cases}
			\log p & \text{if }n=p^\nu\text{ for }\nu\in\ZZ^+, \\
			0 & \text{else},
		\end{cases} \\
		\psi(x) &\coloneqq \sum_{n\le x}\Lambda(n).
	\end{align*}
\end{definition}
Quickly, we note that the prime-powers we have included in $\Lambda(n)$ and $\psi(x)$ don't actually matter.
\begin{lemma} \label{lem:no-prime-powers}
	For any $x\ge2$, we have
	\[\psi(x)=\sum_{p\le x}\log p+O\left(\sqrt x(\log x)^2\right).\]
\end{lemma}
\begin{proof}
	Note
	\[\psi(x)=\sum_{n\le x}\Lambda(n)=\sum_{k=1}^\infty\Bigg(\sum_{p^k\le x}\log p\Bigg).\]
	Now, note $k\ge\log_2x$ implies that $p^k\ge2^k\ge x$ for all primes $p$, so we only need to sum up to $\log_2x$. As such, we upper-bound the $k>1$ sum as
	\[\Bigg|\sum_{k=2}^{\log_2x}\Bigg(\sum_{p^k\le x}\log p\Bigg)\Bigg|\le|\log_2x-1|\cdot\Bigg|\sum_{n\le\sqrt x}\log(\sqrt x)\Bigg|\le\frac{\sqrt x(\log x)^2}{2\log2}.\]
	Adding the $k=1$ sum back in, we see that
	\[\psi(x)=\sum_{p\le x}\log p+O\left(\sqrt x(\log x)^2\right),\]
	which is what we wanted.
\end{proof}
Now, here is our statement.
\begin{theorem}[Prime number]
	We have $\pi(x)\sim x/\log x$ as $x\to\infty$.
\end{theorem}
Here is why we mentioned $\psi$.
\begin{proposition}
	The following are equivalent.
	\begin{listalph}
		\item $\pi(x)\sim x/\log x$ as $x\to\infty$.
		\item $\psi(x)\sim x$ as $x\to\infty$.
	\end{listalph}
\end{proposition}
\begin{proof}
	This is by summation by parts. We have two implications to show.
	\begin{listalph}
		\item Suppose (b), and we'll show (a). Let $a_n=\log n$ be $1$ if $n$ is prime and $0$ otherwise, and let $A(x)\coloneqq\sum_{n\le x}a_n$ denote the partial sums. By \Cref{lem:no-prime-powers}, we see
		\[A(x)=\sum_{p\le x}\log p=\psi(x)+O\left(\sqrt x(\log x)^2\right).\]
		In particular, $A(x)/x=\psi(x)+O\left(x^{-1/2}(\log x)^2\right)$ goes to $1$ as $x\to\infty$ by hypothesis.

		Further, let $f(n)\coloneqq1/\log n$ if $n>1$ and $0$ at $n=1$. Then \Cref{thm:abel-summation} tells us
		\[\pi(x)=\sum_{n\le x}a_nf(n)=A(x)f(x)-\int_0^xA(t)f'(t)\,dt=\frac{A(x)}{\log x}+\int_2^x\frac{A(t)}{(\log t)^2}\,dt,\]
		so
		\[\frac{\pi(x)}{x/\log x}=\frac{A(x)}{x}+\frac{\log x}x\int_2^x\frac{A(t)}{(\log t)^2}\,dt.\]
		As $x\to\infty$, we know $A(x)/x\to 1$, so we have left to show that the right-hand term goes to $0$. Because $A(x)/x\to1$, we can find $N$ such that $x>N$ implies $A(x)/x\le2$. Thus, we can upper-bound
		\[\frac{\log x}x\int_2^x\frac{A(t)}{(\log t)^2}\,dt\le2(\log x)\int_2^x\frac1{(\log t)^2}\,dt\]
		for $x>N$.\todo{}
	\end{listalph}
\end{proof}

\subsection{Poisson Summation}
Starting with the easier parts of Riemann's paper, we will show the functional equation for $\zeta(s)$. For this, we use the Poisson summation formula.
% Let's state it. We will need to establish a little Fourier analysis.
% \begin{definition}[Schwarz]
% 	Let $f\colon\RR\to\RR$ be an infinitely differentiable functions. Then $f$ is \textit{Schwarz} if and only if the $n$th derivative $f^{(n)}$ is rapidly decaying at $\pm\infty$ for all $n$.
% \end{definition}
% \begin{definition}[Fourier transform]
% 	Let $f\colon\RR\to\RR$ be a Schwarz function. Then the \textit{Fourier transform} is the function
% 	\[\widehat f(s)\coloneqq\int_{-\infty}^\infty f(x)x^{-2\pi its}\,dt.\]
% \end{definition}
% \begin{example}
% 	Fix $a>0$, and let $f\colon\RR\to\RR$ be a Schwarz function. Then define $f_a(t)\coloneqq f(at)$ for $t\in\RR$. One can compute that $\widehat{f_a}(s)=\frac1a\widehat f\left(\frac sa\right)$.
% \end{example}
\begin{theorem}[Poisson summation] \label{thm:ps}
	Let $f\colon\RR\to\CC$ be a Schwarz function. Then
	\[\sum_{n\in\ZZ}f(n)=\sum_{n\in\ZZ}(\mc Ff)(n).\]
\end{theorem}
\begin{proof}
	Consider the function
	\[F(x)\coloneqq\sum_{n\in\ZZ}f(x+n).\]
	The point is to compute the Fourier series of $F\colon\RR\to\CC$. Thus, we divide the proof into steps.
	\begin{enumerate}
		% \item Note the series converges (absolutely). The Schwartz condition on $f$ lets us find a constant $C\in\RR$ such that $\left|x^kf(x)\right|\le C$ for any $x$ and $k\in\{0,2\}$. Thus, $|f(x)|\le2C/\left(x^2+1\right)$, so
		% \[\sum_{n\in\ZZ}|f(x+n)|\le2C\sum_{n\in\ZZ}\frac1{(x+n)^2+1}\le2C\sum_{n\in\ZZ}\frac1{\floor{|x+n|}^2+1}.\]
		% Now, for each nonnegative integer $k$, we note $k=\floor{|x+n|}$ is equivalent to $k\le|x+n|<k+1$, which is equivalent to either $k-x\le n<k-x+1$ or $-k-x-1<n\le-k-x$. Thus, there are at most $2$ values of $n\in\ZZ$ such that $\floor{|x+n|}=k$. As such,
		% \[\sum_{n\in\ZZ}|f(x+n)|\le2C\sum_{k=0}^\infty\frac1{k^2+1}\le2C\Bigg(1+\sum_{k=1}^\infty\frac1{k^2}\Bigg)<\infty.\]
		\item Note that $F$ is continuous. Indeed, we will essentially see that the series (absolutely) converges uniformly on compact sets: let $F_N$ denote the $N$th partial sum, where $N\ge1$. Thus, to show that $F$ is continuous on some closed interval $[a,b]$, it suffices to show that $F_N\to F$ uniformly on $[a,b]$ because each $F_N$ is continuous. This will be enough because each $x\in\RR$ is contained in some closed interval $[x-1,x+1]$, so $F$ is continuous at each $x\in\RR$.

		Before doing anything, note $x\in[a,b]$ implies $|x|\le m$ where $m\coloneqq\max\{|a|,|b|\}$, so we will take $N>2m$ throughout. Now, the Schwartz condition on $f$ lets us find a constant $C\in\RR$ such that $\left|x^2f(x)\right|\le C$, so
		\[\left|F(x)-F_N(x)\right|\le\sum_{|n|>N}|f(x+n)|\le2C\sum_{|n|>N}\frac1{(x+n)^2}.\]
		The sum now splits into
		\[\left|F(x)-F_N(x)\right|\le\sum_{n<-N}\frac1{(x+n)^2}+\sum_{n>N}\frac1{(x+n)^2}=\sum_{n>N}\left(\frac1{(n-x)^2}+\frac1{(n+x)^2}\right).\]
		The summand is now decreasing in $n$, so we may upper-bound this by the integral test, writing
		\[\left|F(x)-F_N(x)\right|\le\int_{N-1}^\infty\left(\frac1{(t-x)^2}+\frac1{(t+x)^2}\right)\,dt=\frac1{2(N-1-x)}+\frac1{2(N-1+x)},\]
		which does vanish as $N\to\infty$.

		As an aside, note that the above bounding has also shown that the series $F(x)$ absolutely converges because we showed that $\sum_{|n|>N}|f(x+n)|$ converges for some $N$ depending on $x$ (though this dependency is irrelevant here).

		\item Note $F$ is $1$-periodic because rearranging the sum gives
		\[F(x+1)=\sum_{n\in\ZZ}f(x+n+1)=\sum_{n\in\ZZ}f(x+n)=F(x).\]

		\item The next step is compute the Fourier coefficients of $F$, which for some $n\in\ZZ$ is
		\[a_n(F)=\int_0^1\Bigg(\sum_{k\in\ZZ}f(x+k)e^{-2\pi inx}\Bigg)dx.\]
		We would like to exchange the integral and the sum, so we check the absolute convergence as
		\[\int_0^1\Bigg(\sum_{k\in\ZZ}\left|f(x+k)e^{-2\pi inx}\right|\Bigg)dx=\int_0^1\Bigg(\sum_{k\in\ZZ}|f(x+k)|\Bigg)dx.\]
		Now, we showed that the series $x\mapsto\sum_{k\in\ZZ}|f(x+k)|$ converges uniformly on compact closed intervals $[a,b]$, so it defines a continuous function on the closed interval $[0,1]$, so this integral converges. As such, we may now apply Fubini's theorem to get
		\[a_n(F)=\sum_{k\in\ZZ}\int_0^1f(x+k)e^{-2\pi inx}\,dx=\sum_{k\in\ZZ}\int_k^{k+1}f(x)e^{-2\pi inx}\,dx=\int_{-\infty}^\infty f(x)e^{-2\pi inx}\,dx=(\mc Ff)(n).\]

		\item We would like to build the Fourier series using \Cref{thm:fourier-series}, but for this we must show that $S_F$ converges absolutely and uniformly. Well, by \Cref{lem:fourier-checks}, we see that $n\ne0$ have
		\[(\mc Ff)(n)=\frac1{2\pi in}\cdot(\mc Ff')(n)=\frac1{-4\pi^2n^2}\cdot(\mc Ff'')(n).\]
		Now, $(\mc Ff'')$ is bounded by \Cref{rem:fourier-transform-converges}, so find $M$ such that $|(\mc Ff'')(s)|\le M$ for all $s$. Checking the absolute and uniform convergence, we see $N>0$ lets us upper-bound
		\[\sum_{|n|>N}\left|a_n(F)e^{2\pi inx}\right|\le\frac M{4\pi^2}\sum_{|n|>N}\frac1{n^2}=\frac{2M}{4\pi^2}\sum_{n>N}\frac1{n^2}=\frac{2M}{4\pi^2}\int_N^\infty\frac1{x^2}\,dx=\frac{2M}{4\pi^2N},\]
		which does vanish as $N\to\infty$.

		\item The previous step gives our absolute and uniform convergence, so \Cref{thm:fourier-series} tells us
		\[\sum_{n\in\ZZ}f(x+n)=F(x)=\sum_{n\in\ZZ}a_n(F)e^{2\pi inx}=\sum_{n\in\ZZ}(\mc Ff)(n)e^{-2\pi inx}\]
		for all $x\in\RR$. Setting $x=0$ completes the proof.
		\qedhere
	\end{enumerate}
	% We compute the trace of a special operator in two ways. Define
	% \[K(x,y)\coloneqq\sum_{n\in\ZZ}f(x-y+n).\]
	% Note that $K$ is $1$-periodic in $x$ and $y$, so $K$ defines a smooth function on $(\RR/\ZZ)^2$. Then one can define the operator
	% \[T_Kf(x)\coloneqq\int_{y\in\RR/\ZZ}K(x,y)f(y)\,dy.\]
	% The point is that $T_K\colon L^2(\RR/\ZZ)\to L^2(\RR/\ZZ)$ defines a compact operator. We now compute the trace of $T_K$ in two ways.
	% \begin{itemize}
	% 	\item ``Summing'' along the diagonal, we see
	% 	\[\tr K=\int_0^1K(x,x)\,dx=\sum_{n\in\ZZ}f(n).\]
	% 	More explicitly, $K$ is a normal operator, so we can find an orthonormal basis $\{\varphi_i\}_i$ which are an eigenbasis for $K$, we let $\lambda_i$ be the eigenvalue of $\varphi_i$, and we see
	% 	\[K(x,y)=\sum_i\lambda_i\varphi_i(x)\overline{\varphi_j(y)}.\]
	% 	Integrating appropriately completes the proof.
	% 	\item We may use $\varphi_i(x)\coloneqq e^{-2\pi inx}$ as an orthonormal basis so that $T_Kf$ becomes a Fourier transform. Computing the trace in the same way as above, we see
	% 	\[\tr K=\sum_{n\in\ZZ}\widehat f(n).\]
	% \end{itemize}
	% Equating the above two expressions for $\tr K$ completes the proof.
\end{proof}
% \begin{remark}
% 	The above proof generalizes to the Selberg trace formula. In particular, we are interested in $L^2(\Gamma\backslash\op{SL}_2(\RR))$, where $\Gamma$ is some subgroup of finite index. One recover the Selberg trace formula by computing the trace of the ``point pair invariant.''
% \end{remark}
\begin{example}
	Let $f$ be a Schwarz function, and define $f_x(t)\coloneqq f(tx)$ for any $x>0$. Then $(\mc Ff_x)(s)=\frac1x(\mc Ff)\left(\frac sx\right)$, so \Cref{thm:ps} yields
	\[\sum_{n\in\ZZ}f(nx)=\sum_{n\in\ZZ}f_x(n)=\sum_{n\in\ZZ}(\mc Ff_x)(n)=\frac1x\sum_{n\in\ZZ}(\mc Ff)(n/x).\]
\end{example}

\end{document}