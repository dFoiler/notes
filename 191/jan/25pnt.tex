% !TEX root = ../notes.tex

\documentclass[../notes.tex]{subfiles}

\begin{document}

\section{January 25}

We now shift gears and move towards the Prime number theorem. Today, we begin by discussing Riemann's original paper on the topic.
\begin{remark}
	For the rest of this course, any sum or product over an unnamed $p$ will be a sum over primes.
\end{remark}

\subsection{The Statement}
So far we have established the following facts about $\zeta$.
\begin{itemize}
	\item By \Cref{cor:euler-factor-zeta}, for $\op{Re}s>1$, there is an Euler product factorization
	\[\zeta(s)=\sum_{n=1}^\infty\frac1{n^s}=\prod_p\frac1{1-p^{-s}}.\]
	\item By \Cref{prop:continue-zeta}, there is a meromorphic continuation of $\zeta(s)$ to $\op{Re}s>1$, where $\zeta(s)$ is analytic everywhere except for a pole of order $1$ at $s=1$.
\end{itemize}
Roughly speaking, we will show the Prime number theorem by being able to study $\zeta'(s)/\zeta(s)=\frac d{ds}\log\zeta(s)$. Let's establish some notation.
\begin{definition}
	For $x\in\RR$, we define the following functions.
	\begin{align*}
		\pi(x) &\coloneqq \sum_{\substack{p\text{ prime}\\p\le x}}1, \\
		\Lambda(x) &\coloneqq\begin{cases}
			\log p & \text{if }n=p^\nu\text{ for }\nu\in\ZZ^+, \\
			0 & \text{else},
		\end{cases} \\
		\psi(x) &\coloneqq \sum_{n\le x}\Lambda(n).
	\end{align*}
\end{definition}
\begin{remark}
	Note that
	\[\psi(x)=\sum_{p\le x}\log p+\sum_{p\le x^{1/2}}\log p+\sum_{p\le x^{1/3}}\log p+\cdots=\sum_{p\le x}\log p+O_\varepsilon\left(x^{1/2+\varepsilon}\right).\]
	The point is that the prime-powers don't actually contribute anything.
\end{remark}
Now, here is our statement.
\begin{theorem}[Prime number]
	We have $\pi(x)\sim x/\log x$ as $x\to\infty$.
\end{theorem}
Here is why we mentioned $\psi$.
\begin{proposition}
	The statement $\psi(x)\sim x$ as $x\to\infty$ is equivalent to $\pi(x)\sim x/\log x$ as $x\to\infty$.
\end{proposition}
\begin{proof}
	Summation by parts.
\end{proof}

\subsection{Poisson Summation}
Starting with the easier parts of Riemann's paper, we will use the Poisson summation formula. Let's state it. We will need to establish a little Fourier analysis.
\begin{definition}[Schwarz]
	Let $f\colon\RR\to\RR$ be an infinitely differentiable functions. Then $f$ is \textit{Schwarz} if and only if the $n$th derivative $f^{(n)}$ is rapidly decaying at $\pm\infty$ for all $n$.
\end{definition}
\begin{definition}[Fourier transform]
	Let $f\colon\RR\to\RR$ be a Schwarz function. Then the \textit{Fourier transform} is the function
	\[\widehat f(s)\coloneqq\int_{-\infty}^\infty f(x)x^{-2\pi its}\,dt.\]
\end{definition}
\begin{example}
	Fix $a>0$, and let $f\colon\RR\to\RR$ be a Schwarz function. Then define $f_a(t)\coloneqq f(at)$ for $t\in\RR$. One can compute that $\widehat{f_a}(s)=\frac1a\widehat f\left(\frac sa\right)$.
\end{example}
\begin{theorem}[Poisson summation]
	Let $f\colon\RR\to\RR$ be a Schwarz function. Then
	\[\sum_{n\in\ZZ}f(n)=\sum_{n\in\ZZ}\widehat f(n).\]
\end{theorem}
\begin{proof}
	We compute the trace of a special operator in two ways. Define
	\[K(x,y)\coloneqq\sum_{n\in\ZZ}f(x-y+n).\]
	Note that $K$ is $1$-periodic in $x$ and $y$, so $K$ defines a smooth function on $(\RR/\ZZ)^2$. Then one can define the operator
	\[T_Kf(x)\coloneqq\int_{y\in\RR/\ZZ}K(x,y)f(y)\,dy.\]
	We'll finish this next class.
\end{proof}
\begin{example}
	Let $f$ be a Schwarz function. Then $x>0$ yields
	\[\sum_{n\in\ZZ}f(nx)=\sum_{n\in\ZZ}f_x(n)=\sum_{n\in\ZZ}\widehat{f_x}(n)=\frac1x\sum_{n\in\ZZ}\widehat f(n/x)=\frac{\widehat f(0)}x+O_{f,A}\left(x^A\right).\]
\end{example}

\end{document}