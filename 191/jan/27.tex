% !TEX root = ../notes.tex

\documentclass[../notes.tex]{subfiles}

\begin{document}

\section{January 27}

We began class finishing the proof of \Cref{thm:ps}. I have edited directly into that proof for continuity.

\subsection{An Abstract Functional Equation}
We now use \Cref{thm:ps} in order to show the functional equation for $\zeta$, which provides us with its meromorphic continuation.

To work somewhat abstractly, suppose $f\colon\RR\to\RR$ is a Schwartz function which has both $f$ and $\widehat f$ even and satisfies $f(0)=\widehat f(0)=0$. Now define
\[I(f,s)\coloneqq\int_0^\infty\Bigg(\sum_{n=1}^\infty f(nx)\Bigg)x^s\,\frac{dx}x.\]
One can show by hand that $I(f,s)$ is absolutely convergent and then analytic for $\op{Re}s>0$. However, if we take the Fourier transform and use \Cref{thm:ps}, we are able to continue this to all of $\CC$. Now, for $\op{Re}s>1$, we have absolute convergence, so Fubini's theorem lets us write
\[I(f,s)=\sum_{n=1}^\infty\Bigg(\int_0^\infty f(nx)x^s\,\frac{dx}x\Bigg)=\sum_{n=1}^\infty\Bigg(\frac1{n^s}\int_0^\infty f(x)x^s\,\frac{dx}x\Bigg)=\zeta(s)(\mc Mf)(s).\]
Notably, we used $\op{Re}s>1$ in order to write $\zeta(s)$ as the series. Now, everything has a continuation to $\op{Re}s>0$, so uniqueness of extension lets us extend to $\op{Re}s>0$.

However, we were able to continue $I(f,s)$ to all of $\CC$ by applying Poisson summation. Indeed, \Cref{thm:ps} yields
\[I(f,s)=\int_0^\infty\Bigg(\sum_{n=1}^\infty\widehat f(n/x)\Bigg)x^{s-1}\,\frac{dx}x.\]
Now, $\widehat f$ is also Schwarz, so we have absolute convergence for $\op{Re}s<1$, so we have indeed produced our analytic continuation to all of $\CC$. Manipulating, we see
\[I(f,s)=\int_0^\infty\Bigg(\sum_{n=1}^\infty\widehat f(nx)\Bigg)x^{1-s}\,\frac{dx}x=I(\widehat f,s-1).\]
Comparing, we see
\[(\mc M\widehat f)(1-s)\zeta(1-s)=I(\widehat f,s-1)=I(f,s)=\zeta(s)(\mc Mf)(s)\]
on $0<\op{Re}s<1$. Thus, we see that we will have a good functional equation for $\zeta$ by choosing a sufficiently good $f$. Indeed, choosing $f$ which is nonzero everywhere except for some finite set of points will grant us a meromorphic continuation of $\zeta$ to all of $\CC$.

\subsection{The Functional Equation}
We now go back and use a specific value of $f$ to give a functional equation we can really write down. In particular, we will know that $\zeta$ only has simple poles at $s=0$ and $s=1$, each of residue $1$. Indeed, set $f\colon\RR\to\RR$ by $f(x)\coloneqq e^{-\pi x^2}$ so that $\widehat f(x)=f(x)$. Motivated by the above work, we set
\[\xi(s)\coloneqq\pi^{-s/2}\Gamma(s/2)\zeta(s),\]
where
\[\Gamma(s)=\int_0^\infty e^{-t}t^s\,\frac{dt}t\]
for $\op{Re}s>0$, and we can continue $\Gamma$ to the left by the functional equation $\Gamma(s)=s\Gamma(s-1)$. (This functional equation is proven by integration by parts.)
\begin{remark}
	In some sense, $\Gamma$ is a continuous version of a Gauss sum: it's an integral of an additive character multiplied by a multiplicative character, over a suitable Haar measure.
\end{remark}
\begin{remark}
	Notably, $\Gamma(s)$ has simple poles for $s\in\{0,-1,-2,\ldots\}$. In fact, $1/\Gamma(s)$ is entire.
\end{remark}
\begin{remark}
	In some sense, we want to write
	\[\xi(s)=\pi^{-s/2}\Gamma(s/2)\prod_{p\text{ prime}}\frac1{1-p^{-s}}.\]
	Here, $\pi^{-s/2}\Gamma(s/2)$ is an ``archimedean local factor'' corresponding to the infinite place $\infty$ of $\QQ$, and each of the $\left(1-p^{-s}\right)^{-1}$ are ``nonarchimedean local factors.'' Roughly speaking, the rigorization of this intuition is Tate's thesis.
\end{remark}
Now working through the arguments of the previous subsection, we see that $\xi(s)$ is entire except for simple poles at $s\in\{0,1\}$, and
\[\xi(1-s)=\xi(s).\]
Let's give a few consequences.
\begin{remark}
	Doing logarithmic differentiation, one finds
	\[\frac d{ds}(-\log\zeta(s))=-\frac{\zeta'(s)}{\zeta(s)}=\sum_{n=1}^\infty\frac{\Gamma(n)}{n^s}.\]
	This explains why $\psi$ is a ``better'' prime-counting function than $\pi$.
\end{remark}
\begin{remark}
	Ignoring convergence issues, we may compute
	\[\psi(x)=\sum_{n\le x}\Gamma(n)=\sum_{n=1}^\infty1_{[0,1]}(n/x)\Gamma(n)=\frac1{2\pi i}\int_{2-i\infty}^{2+i\infty}\left(-\frac{\zeta'(s)}{\zeta(s)}\right)x^s\,\frac{ds}s.\]
	Now, if we imagine that we could push this integral all the way to the left of $\CC$, we will eventually vanish and only pick up on the poles of $\zeta'/\zeta$. As such, we expect to achieve a formula of the form
	\[\psi(x)=x-\sum_\rho\frac{x^\rho}\rho,\]
	where the sum is over the roots $\rho$ of $\zeta$. Thus, we see that having more control over the zeroes of $\zeta$ will be able to get good bounds on $\psi(x)-x$. In particular, the Riemann hypothesis is equivalent to $\psi(x)=x+O(\sqrt x)$. As another application, the discontinuity of $\psi$ will imply that $\zeta$ must have infinitely many roots.
\end{remark}
\begin{remark}
	The previous remark, after some summation by parts, tells us that $\pi(x)-\op{Li}(x)$ has a better error term than $\pi(x)-x/\log x$, where
	\[\op{Li}(x)=\int_2^x\frac{dt}{\log t}.\]
\end{remark}
Next class we will show the functional equation.

\end{document}