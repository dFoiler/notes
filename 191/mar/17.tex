% !TEX root = ../notes.tex

\documentclass[../notes.tex]{subfiles}

\begin{document}

\section{March 17}

We began class discussing \Cref{thm:mordell}.

\subsection{Introducing the Circle Method}
Roughly speaking, the idea is that
\[\int_{[0,1]^k}\sum_{j=1}^ne\left(n_j\cdot\alpha\right)\,d\alpha=\#\{j:n_j=0\}\]
by a direct integration. As an example, if we want to count the number of ways to write some $N$ as the sum of ten cubes, we can use the function
\[e(-N\alpha)\Bigg(\sum_{n=1}^pe\left(n^3\alpha\right)\Bigg)^{10}.\]
Now, these exponential sums can be studied separately, which allows us to bound the integral. In particular, the sum is large essentially only when we are close to rational numbers with reasonably small denominator; these are the major arcs. Then we call everything else a minor arc. Analyzing our integral on the major and minor arcs separately is occasionally able to produce novel bounds. This is the circle method.
\begin{remark}
	The above function suggests that the circle method will have applications in additive number theory, which is indeed the case. By trying harder, one can achieve results in multiplicative number theory as well. (Our main application will be Vinogradov's three primes theorem.)
\end{remark}
Historically, the circle method was introduced to study partitions.
\begin{definition}[partition]
	Given a positive integer $n$, we let $p(n)$ denote the number of \textit{partitions} of $n$, where a partition is a summation
	\[n=\lambda_1+\cdots+\lambda_k\]
	where the order here does not matter.
\end{definition}
From the definition, an analysis of the generating function tells us that
\[\sum_{n=0}^\infty p(n)x^n=\prod_{k=1}^\infty\left(1+x^k+x^{2k}+x^{3k}+\cdots\right)=\prod_{k=1}^\infty\frac1{1-x^k},\]
where the choice of factor in the $k$th summation factor communicates the number of terms in our partition equal to $k$. Indeed, Hardy and Littlewood showed
\[p(n)\sim\frac1{4^n\sqrt3}\exp\left(\pi\sqrt{\frac{2n}3}\right).\]

\end{document}