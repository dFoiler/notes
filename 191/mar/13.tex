% !TEX root = ../notes.tex

\documentclass[../notes.tex]{subfiles}

\begin{document}

\section{March 13}

Let's briefly introduce the Brun sieve.

\subsection{The Brun Sieve}
In contrast to the Selberg sieve, the Brun sieve provides a lower bound.
\begin{theorem}[Brun sieve]
	Fix a sequence $\mc A=\{a_n\}_{n=1}^N$ of nonnegative real numbers in $[0,1]$. Further, suppose that we have (large) integers $X$ and $D$ such that we may write
	\[\sum_{n\le N}a_n=X\qquad\text{and}\qquad\sum_{n\equiv0\pmod d}g(d)X+r(d,\mc A),\]
	for each squarefree $d$, where we satisfy the following conditions.
	\begin{listroman}
		\item $d$ is squarefree.
		\item $g$ is multiplicative and $0\le g(p)<1$ for all $p$.
	\end{listroman}
	Then we can lower-bound $\mc S(\mc A,P_z)\coloneqq\sum_{n\le N,\gcd(n,P_z)=1}a_n$.
\end{theorem}
\begin{proof}[Idea]
	We will not prove this rigorously. As before, we note
	\[S(\mc A,P_z)=\sum_{n=1}^Na_n\Bigg(\sum_{d\mid\gcd(n,P_z)}\mu(d)\Bigg),\]
	where the internal sum is an indicator for $\gcd(n,P_z)=1$. For the Brun sieve, we will optimize real numbers $\lambda_d^+$ and $\lambda_d^-$ for $1\le d\le D$ such that $\lambda_1^+=\lambda_1^-=1$ and
	\[\sum_{d\mid n}\lambda_d^-\le0\le\sum_{d\mid n}\lambda_d^+\]
	for each $n$. The point is that the $\lambda^\pm_\bullet$ behave as a truncated M\"obius function.

	Assuming the existence of these real numbers, we get some error terms $R^+$ and $R^-$ such that
	\[X\sum_{\substack{d\mid P_z\\d\le D}}\lambda_d^-g(d)-R^-\le S(\mc A,P_z)\le X\sum_{\substack{d\mid P_z\\d\le D}}\lambda_d^+g(d)+R^+,\]
	where
	\[R^\pm\coloneqq\sum_{\substack{d\mid P_z\\d\le D}}\left|\lambda_d^\pm r(d,\mc A)\right|.\]
	We would like these remainder terms to be relatively small.

	The point is that the magic goes into choosing the $\lambda^\pm_\bullet$. Roughly speaking, we will want to choose them to approximately agree with $\mu$ on small values. As such, we choose real parameters $y_m>0$, and let $D^+$ denote the set of squarefree positive integers $d=q_1\cdots q_r$ (with $q_i<q_{i+1}$) such that $q_m<y_m$ for each odd $m$, and we define $D^-$ analogously as the set of squarefree positive integers $d=q_1\cdots q_r$ (with $q_i<q_{i+1}$) such that $q_m<q_m$ for $m$ even. We now want to define
	\[\lambda_d^\pm\coloneqq\mu(d)|_{D^\pm}\]
	for each $d$. To ensure that this is nonzero for enough $d$, we set our parameters $y_m$ by
	\[y_m\coloneqq\left(\frac D{p_1\cdots p_m}\right)^{1/\beta}\]
	where $\beta\ge1$.
\end{proof}

\subsection{Twin Primes}
We now bound the remainder term of our Selberg sieve.
\begin{theorem}
	Fix everything as in the Selberg sieve. Suppose we have an explicit constant $\kappa\in\NN$ such that $g(p)\ge\kappa/p$ for all but finitely many primes $p$. Then, letting $q$ denote the product of these finitely many primes, we have
	\[J\ge\frac{(\frac12\log D)^\kappa}{k!H_q}\left(1-\frac{\kappa\ell(q)}{\frac12\log D}\right),\]
	where $H_q$ and $\ell(q)$ are effective constants depending on $q$.
\end{theorem}
\begin{proof}
	Explicitly, we will show that
	\[H_q=\prod_{p\mid q}(1-g(p))\left(1-\frac1p\right)^{-\kappa}\qquad\text{and}\qquad\ell(q)=\sum_{p\mid q}g(p)\log p.\]
	Define $\tau_\kappa(n)$ denote the number of $\kappa$-tuples of positive integers $(a_1,\ldots,a_k)$ such that $a_1\cdots a_\kappa=n$. (The point is to be able to rearrange summations with some ease.) Recall that
	\[h(b)\coloneqq\prod_{p\mid b}\Bigg(\sum_{k=1}^\infty g(p)^k\Bigg)\]
	for squarefree $b$. As an aside, a direct computation with sticks-and-stones tells us that
	\begin{equation}
		\sum_{n\ge0}\frac{\tau_\kappa\left(p^n\right)}{p^n}=\sum_{n\ge0}\frac{\binom{\kappa+n-1}{n-1}}{p^n}=\left(1-\frac1p\right)^{-\kappa}=\left(\frac{\varphi(p)}p\right)^{-\kappa}. \label{eq:compare-tau-kappa}
	\end{equation}
	Now, again by definition, we see
	\[J=\sum_{\substack{\text{squarefree }a,b\\ab<\sqrt D\\a\mid q\\\gcd(b,q)=1}}h(a)h(b)\]
	by unwinding the Selberg sieve. As such, we see that we want to find a lower bound for
	\[F(x)\coloneqq\sum_{\substack{\text{squarefre }b<x\\\gcd(b,q)=1}}h(b).\]
	Extending $g$ to be completely multiplicative for technical reasons, we can define $h(b)$ in the same way, and we note that we still have $g(b)\ge\tau_\kappa(b)/b$ by checking locally at prime-powers and then multiplying. Notably, by expanding out the products in $h(b)$, we see
	\[F(x)\ge\sum_{\substack{b<x\\\gcd(b,q)=1}}g(b),\]
	but our inequality $g(b)\ge\tau_\kappa(b)/b$ now grants
	\[F(x)\ge\sum_{\substack{b<x\\\gcd(b,q)=1}}\frac{\tau_\kappa(b)}b.\]
	Adding in the remainder terms by multiplying through with \eqref{eq:compare-tau-kappa}, we see that actually
	\[F(x)\ge\left(\frac{\varphi(q)}q\right)^{\kappa}\sum_{b<x}\frac{\tau_\kappa(b)}b,\]
	so upon expanding out $\tau_\kappa$ and using the integral bound for a sum, we see that
	\[F(x)\ge\left(\frac{\varphi(q)}q\right)^{\kappa}\int_{\substack{x_1\cdots x_k<x\\x_1,\ldots,x_k\ge1}}\frac{dx_1\cdots dx_\kappa}{x_1\cdots x_\kappa}=\left(\frac{\varphi(q)}q\right)^\kappa\cdot\frac1{\kappa!}(\log x)^\kappa,\]
	where we have omitted the computation of the integral. In total, we conclude
	\[J\ge\frac1{k!}\sum_{\substack{\text{squarefree }a<\sqrt D\\a\mid q}}h(a)\left(\frac{\varphi(q)}q\log\left(\frac{\sqrt D}a\right)\right)^\kappa.\]
	However, we can lower-bound $(1-y)^\kappa\ge1-\kappa y$, which turns this bound into
	\[J\ge\frac1{k!}\left(\frac{\varphi(q)}q\log\sqrt D\right)^\kappa\sum_{\text{squarefree }a\mid q}h(a)\left(1-\kappa\cdot\frac{\log a}{\log\sqrt D}\right).\]
	We will continue next class.
\end{proof}

\end{document}