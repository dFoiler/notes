% !TEX root = ../notes.tex

\documentclass[../notes.tex]{subfiles}

\begin{document}

\section{March 24}
We continue discussing the Hardy--Littlewood circle method.

\subsection{Singular Things}
Last class we showed that we have the contribution
\[\int_{\mf M(a,q)}S(\alpha)^3e(-N\alpha)\,d\alpha=\left(\int_{-1/Q}^{1/Q}T(\beta)^3e(-N\beta)\,d\beta\right)\Bigg(\sum_{\substack{1\le a<q\le p\\\gcd(a,q)=1}}\left(\frac{\mu(q)}{\varphi(q)^3}\right)e\left(-N\frac aq\right)\Bigg).\]
Our goal for now is to compute the ``singular integral''
\[\int_{-1/Q}^{1/Q}T(\beta)^3e(-N\beta)\,d\beta.\]
Roughly speaking, the intuition is that these singular integrals are supposed to give rise to the archimedean factor in our major-arc contribution, which is a bit surprising because we are only looking at such a small interval. Well, we note that
\[\int_0^1T(\beta)^3e(-N\beta)\,d\beta=\frac12(N-1)(N-2)=\frac12N^2+O(N)\]
by the definition of $T$ as some geometric series which mostly vanishes. As such, we go compute
\[\int_{1/Q}^{1-1/Q}T(\beta)^3e(-N\beta)\,d\beta\ll\int_{x>1/Q}x^{-3}\,dx\ll Q^2=N^2(\log N)^{-2\beta},\]
where we have again used the definition of $T$ in the second inequality. Thus, our singular integral is approximately
\[\int_{-1/Q}^{1/Q}T(\beta)^3e(-N\beta)\,d\beta=\frac12N^2+O\left(N^2(\log N)^{-2\beta}\right)\sim\frac12N^2,\]
which is indeed our archimedean density.

It remains to sum over the other major arcs to get the rest of our density contribution. This is our ``singular series''
\[\sum_{\substack{1\le a<q\le p\\\gcd(a,q)=1}}\left(\frac{\mu(q)}{\varphi(q)^3}\right)e\left(-N\frac aq\right).\]
To agree with the literature, we will be interested in evaluating
\[C_q(n)\coloneqq\sum_{\gcd(a,q)=1}e\left(-\frac aqn\right).\]
Notably, without the constraint $\gcd(a,q)=1$, we can just use the Chinese remainder theorem to easily compute this sum; to add in this condition, we must sieve via M\"obius inversion. As such, we observe
\[\sum_{n=1}^qe\left(-\frac aqn\right)=\sum_{d\mid q}\sum_{\substack{a=1\\\gcd(a,q)=1}}^qe\left(-\frac aqn\right)=\sum_{d\mid q}c_d(n).\]
However, the left-hand side vanishes when $q\nmid n$ and is $q$ when $q\mid n$, so M\"obius inversion yields
\[c_q(n)=\sum_{d\mid q,n}d\mu(q/d).\]
Namely, we morally should have $d\mid q$ over all divisors $d$ here, but when $d\nmid n$, the contribution in the sum vanishes by our previous computation of the M\"obius inversion; thus, we only pay attention to the terms with $d\mid n$ which yield $d$ in the summation. We can check by hand that $C_q(n)$ is multiplicative in $q$ (with $n$ fixed), which now lets us fully compute $C_q(n)$. We can now compute
\[C_{p^\beta}(n)=\begin{cases}
	\varphi\left(p^\beta\right) & \text{if }\beta\le\nu_p(n), \\
	-p^\alpha & \text{if }\beta=\alpha+1, \\
	0 & \text{else}.
\end{cases}\]
Now, in our application, we really only care about the cases where $q$ is semiprime (because we have a $\mu(q)$ term in our summation). Anyway, one can show that
\[c_q(n)=\frac{\mu(q/(n,q))\varphi(q)}{\varphi(q/(n,q))}\]
by checking on prime powers and extending multiplicatively. Thus, we bound $\varphi(q)\gg_\varepsilon q^{1-\varepsilon}$ for any $\varepsilon>0$, so our summation is small enough in the sense
\[\left|\sum_{a>p}\frac{\mu(q)}{\varphi(q)^3}C_q(N)\right|\le\sum_{q>p}\varphi(q)^{-2}\ll(\log N)^{-B/2}\]
by using the bound that we just achieved. Thus, our summation converges.

We are now ready to compute our infinite sum as
\[\mf S(N)\coloneqq\sum_{q-1}^\infty\frac{\mu(q)}{\varphi(q)^2}C_q(N)=\prod_p\left(1-\frac{C_p(N)}{(p-1)^3}\right)=\prod_{p\mid N}\left(1-\frac1{(p-1)^2}\right)\prod_{p\nmid N}\left(1+\frac1{(p-1)^3}\right)\]
by factoring the infinite geometric series in the usual way. As such, we morally expect $\mf S(N)$

\end{document}