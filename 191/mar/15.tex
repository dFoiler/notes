% !TEX root = ../notes.tex

\documentclass[../notes.tex]{subfiles}

\begin{document}

\section{March 15}

We began class by finishing the proof from last class. I have directly edited into those notes for continuity reasons.

\subsection{More Counting by Geometry}
As usual, fix a polynomial $F\in\ZZ[x,y]$. Roughly speaking, we are interested in the number of solutions to $F(x,y)\pmod p$ as a prime $p$ varies. For analytic number theory, we care because these sorts of local factors appear when sieving or applying the circle method.
\begin{example}
	Take $F(x,y)\coloneqq x^n+y^n-a$. Then one can show that $N_F(p)=p+O\left(n^2\sqrt p\right)$. The corresponding character sums here as $x$ varies turns into a Gauss sum computation, which we do know how to bound already.
\end{example}
\begin{remark}
	If $F$ is irreducible, then the Weil conjecture grants
	\[N_F(p)=p+O(g\sqrt p),\]
	where $g$ is the genus of the corresponding Riemann surface cut out by $F$. However, this is quite difficult to prove.
\end{remark}
Here is the result that we will show.
\begin{theorem}
	Fix an irreducible polynomial $f\in\ZZ[x]$ of degree $3$ or $4$, and define $F(x,y)\coloneqq y^2-f(x)$. Then
	\[N_F(p)-p\ll p^{2/3}.\]
	Here, the implied constant is independent of $f$.
\end{theorem}
\begin{proof}[Proof in degree $4$]
	Set $N\coloneqq N_F(p)$ for brevity. The idea is to average over a lot of $f$s and compute some moments. By looping over all $x$ and $y$, we see that
	\[\sum_{t\in\FF_p}\sum_{x,y\in\FF_p}e\left(\frac{tF(x,y)}p\right)=pN_F(p),\]
	where the point is that the summation over $t$ cause the sums to vanish whenever $F(x,y)=0$. Now, removing the contribution at $t=0$, we see
	\[\sum_{t=1}^{p-1}\sum_{x,y\in\FF_p}e\left(\frac{tF(x,y)}p\right)=p(N-p).\]
	Now, going to the moment at $r\coloneqq6$, so we note we can fully expand everything out as
	\[p^r(N-p)^r=\sum_{t_1,\ldots,t_r=1}^{p-1}\sum_{\substack{x_1,\ldots,x_r=0\\y_1,\ldots,y_r=0}}^{p-1}e\left(\frac1p\sum_{k=1}^rt_kF(x_k,y_k)\right).\]
	In order to smooth over some issues, we will work over a family of $f$s given by
	\[F_a(x,y)=y^2-a_1x^4-a_2x^3-a_3x^2-a_4x-a_5\]
	where the coefficients fully vary. As such, summing over all $a$, we will see that we cancel everything out unless $t_1x^d+\cdots+t_6x^d=0$ for $d\in\{0,1,2,3,4\}$ already (meaning that our term was going to die anyway), but then here the contribution is given by
	\[\sum_{y,t}e\left(\frac1p\sum_{k=1}^6t_ky_k^2\right),\]
	which is a Gauss sum. We will finish this bounding next class.
\end{proof}

\end{document}