% !TEX root = ../notes.tex

\documentclass[../notes.tex]{subfiles}

\begin{document}

\section{March 15}

We began class by finishing the proof from last class. I have directly edited into those notes for continuity reasons.

\subsection{More Counting by Geometry}
As usual, fix a polynomial $F\in\ZZ[x,y]$. Roughly speaking, we are interested in the number of solutions to $F(x,y)\pmod p$ as a prime $p$ varies. For analytic number theory, we care because these sorts of local factors appear when sieving or applying the circle method.
\begin{example}
	Take $F(x,y)\coloneqq x^n+y^n-a$. Then one can show that $N_F(p)=p+O\left(n^2\sqrt p\right)$. The corresponding character sums here as $x$ varies turns into a Gauss sum computation, which we do know how to bound already.
\end{example}
\begin{remark}
	If $F$ is irreducible, then the Weil conjecture grants
	\[N_F(p)=p+O(g\sqrt p),\]
	where $g$ is the genus of the corresponding Riemann surface cut out by $F$. However, this is quite difficult to prove.
\end{remark}
Here is the result that we will show.
\begin{theorem} \label{thm:mordell}
	Fix an irreducible polynomial $f\in\ZZ[x]$ of degree $3$ or $4$, and define $F(x,y)\coloneqq y^2-f(x)$. Then
	\[N_F(p)-p\ll p^{2/3}.\]
	Here, the implied constant is independent of $f$.
\end{theorem}
\begin{proof}[Proof in degree $4$]
	Set $N\coloneqq N_F(p)$ for brevity. The idea is to average over a lot of $f$s and compute some moments. By looping over all $x$ and $y$, we see that
	\[\sum_{t\in\FF_p}\sum_{x,y\in\FF_p}e\left(\frac{tF(x,y)}p\right)=pN_F(p),\]
	where the point is that the summation over $t$ cause the sums to vanish whenever $F(x,y)=0$. Now, removing the contribution at $t=0$, we see
	\[\sum_{t=1}^{p-1}\sum_{x,y\in\FF_p}e\left(\frac{tF(x,y)}p\right)=p(N-p).\]
	Now, going to the moment at $r\coloneqq6$, so we note we can fully expand everything out as
	\[p^r(N-p)^r=\sum_{t_1,\ldots,t_r=1}^{p-1}\sum_{\substack{x_1,\ldots,x_r=0\\y_1,\ldots,y_r=0}}^{p-1}e\left(\frac1p\sum_{k=1}^rt_kF(x_k,y_k)\right).\]
	In order to smooth over some issues, we will work over a family of $f$s given by
	\[F_a(x,y)=y^2-a_1x^4-a_2x^3-a_3x^2-a_4x-a_5\]
	where the coefficients fully vary over $\FF_p^5$. As such, summing over all $a\in\FF_p^5$, we will see that we cancel everything out (namely, fix $x$ and $y$ and $t$, letting $a$ vary) unless $t_1x^d+\cdots+t_6x^d=0$ for $d\in\{0,1,2,3,4\}$ already. Call the set of $(x,t)$ satisfying this system to be $S$.
	
	Now, summing over $y$ where $(x,t)\in S$, we see our contribution over $y$ is
	\[\left|p^5\prod_{s=1}^6\sum_{y_s=0}^{p-1}e\left(\frac{t_sy_s^2}p\right)\right|\le p^8\]
	by factoring everything appropriately. Thus, we see that
	\[\sum_{a\in\FF_p^5}p^6(N_{F_a}-p)^6\le Mp^8,\]
	where $M$ is the number of solutions $(x,t)\in S$.

	The game, now, is to bound $M$. For example, if the $x_\bullet$ are all distinct, then a computation of the Vandermonde determinant tells us that the only possible solution is $t_1=\cdots=t_6=0$; this gives about $p^7$ total solutions. If some of $x_\bullet$ are the same, then one can compute what happens in our degenerate cases, but they only contribute $O\left(p^6\right)$ solutions total,\footnote{Roughly speaking, if some $x_\bullet$ are equal, then the ``Lagrange interpolation problem'' we are solving for the $t_\bullet$ has more degrees of freedom, in particular $p$ more degrees of freedom. Nonetheless, we will dominate the main term.} so this term does not matter. In total, we get
	\[\sum_{a\in\FF_p^5}(N_a-p)^6\le p^9+O\left(p^8\right)\]
	in total.

	We now examine the $a\in\FF_p^5$ more closely. Notably, we don't actually care about all $a\in\FF_p^5$ because we require $F_a$ to be irreducible for our argument. Let $B\subseteq\FF_p^5$ be the set of the ``worst'' $a\in\FF_p^5$ where $F_a$ fails to be irreducible. In particular, we take the following cases for $f_a\coloneqq a_1x^4+\cdots+a_5x^0$.
	\begin{itemize}
		\item We might have $f_a=r\left(x^2+ex+f\right)^2$ for $r$ a quadratic residue and $E,F\in\FF_p$. Here, $N_{F_a}-p=p+O(1)$ by taking the square root directly. The number of solutions $a$ here is given by $p^3/2+O\left(p^2\right)$.
		\item We might have $f_a=n\left(x^2+ex+f\right)^2$, where $n$ is a non-quadratic residue, and $E,F\in\FF_p$. Here, $N_{F_a}-p=-p+O(1)$, and the total number of $a$ is the same.
	\end{itemize}
	Totaling the above contributions, we see that these ``worst'' $a$s in fact total to $p^9$ in contribution, so we see
	\[\sum_{a\notin B}(N_a-p)^6=O\left(p^8\right).\]
	To complete the argument, we require one more idea: some $a\notin B$ are essentially the same curve $F_a$, up to a fractional linear transformation, which will in particular not change the number of our points. Explicitly, our fractional linear transformation is given by
	\[\begin{bmatrix}
		a & b \\
		c & d
	\end{bmatrix}x=\frac{ax+b}{cx+d}\qquad\text{and}\qquad\begin{bmatrix}
		a & b \\
		c & d
	\end{bmatrix}y=\frac y{(cx+d)^2}.\]
	Notably, hitting a curve with such a transformation by an element $\gamma\in\op{GL}_2(\FF_p)$, the number of points on the projective plane curve will not change, so the number of points in $N_a$ will only change by $O(1)$. One can also compute directly that each $a\in\FF_p^5$ gets transformed to at least $\gg p^4$ different $a'\in\FF_p^5$.\footnote{Rigorously, one can see that the action of $\op{GL}_2(\FF_p)$ on the space of degree-$4$ polynomials provides a symmetric action on the roots most of the time. The fact that our action is symmetric and hence essentially transitive will do the trick.} Looking at a particular class of $p^4$ different $a\in\FF_p^5$, we see that particular $a\in\FF_p^5$ cannot have $N_a-p$ exceeding $O\left(p^{2/3}\right)$. This completes the proof.
\end{proof}

\end{document}