% !TEX root = ../notes.tex

\documentclass[../notes.tex]{subfiles}

\begin{document}

\section{March 3}

We began class by finishing the proof of Siegel's theorem. I have edited directly into those notes for completeness.

\subsection{The Prime Number Theorem in Arithmetic Progressions}
We are now ready to prove the Prime number theorem in arithmetic progressions. The point here is to input our zero-free region (improved by Siegel's theorem) into our Prime number theorem machine to get out a prime number theorem for arithmetic progressions.
\begin{definition}
	Fix coprime integers $a$ and $q$. Then we define
	\[\psi(x;q,a)\coloneqq\sum_{\substack{n<x\\n\equiv a\pmod q}}\Lambda(n).\]
\end{definition}
Let's be more explicit about our zero-free regions. Fix a primitive Dirichlet character $\chi\pmod q$.
\begin{itemize}
	\item For some constant $c>0$, we know that $L(s,\chi)$ has no zeroes in the region
	\[\left\{s:\op{Re}s>1-\frac c{\log(q(|t|+2))}\right\}\]
	except for possibly a real zero if $\chi$ is a real character.
	\item In the event that $\chi$ is a real character, then each $\varepsilon>0$ provides an ineffective constant $c(\varepsilon)$ such that $L(s,\chi)$ does not have a zero in the interval $\left(1-c(\varepsilon)/q^\varepsilon,1\right]$ for any $\varepsilon>0$.
\end{itemize}
These inputs give the following result.
\begin{theorem}[Siegel--Walfisz]
	Fix coprime integers $a$ and $q$, and fix some $\varepsilon>0$. Then we see
	\[\psi(x;q,a)=\frac x{\varphi(q)}+O_\varepsilon\left(\frac x{(\log x)^{1+\varepsilon}}\right),\]
	where the implied constant is ineffective.
\end{theorem}
\begin{remark}
	One can improve this result in various ways. For example, averaging over $a\pmod q$ is the Bombieri--Vinogradov theorem. Under the assumption of the generalized Riemann hypothesis, we have
	\[\psi(x;q,a)=\frac x{\varphi(q)}+O_\varepsilon\left(x^{1/2+\varepsilon}\right),\]
	where the implied constant is effective.
\end{remark}
For the remainder of the class, we are going to turn towards sieve theory.

\end{document}