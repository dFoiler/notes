% !TEX root = ../notes.tex

\documentclass[../notes.tex]{subfiles}

\begin{document}

\section{February 10}

We continue discussing applications of the Gauss sum.

\subsection{The P\'olya--Vinogradov Inequality}
As an aside, we set up the P\'olya--Vinogradov inequality.
\begin{theorem}[P\'olya--Vinogradov inequality] \label{thm:polya-vinogradov}
	Fix a prime $p$ and a nontrivial character $\chi\pmod p$. Then for any $a,b$, we have
	\[\left|\sum_{a\le n\le b}\chi(n)\right|\ll\sqrt p\log p,\]
	where the implicit constant does not depend on anything.
\end{theorem}
\begin{proof}
	Roughly speaking, we are computing the inner product of $\chi$ and the indicator function of an interval. Using ``Plancherel's formula'' to bound completes the proof. The trick here is to ``complete the sum.'' Because $\sum_{n=0}^{p-1}\chi(n)=0$, we may assume that $a,b\le p$. (If shifting yields $a\le p\le b\le p+a$, then we can flip the entire sum to make it  $b-p\le p-a\le p$.)

	Now, the main point is to take the Fourier transform
	\[\widehat{1_{[a,b]}}(m)=\sum_{a\le n\le b}e\left(-\frac{mn}p\right)\ll\frac pm,\]
	where we have expanded out the geometric series to get this bound; namely, we are noting $\frac1{1-e(-1/p)}\approx p$. As such, we use the Fourier inversion formula \Cref{cor:indicate-g} to see
	\[\left|\sum_{x\in\FF_p}1_{[a,b]}(x)\chi(x)\right|=\frac1p\left|\sum_{m,x\in\FF_p}\widehat{1_{[a,b]}}(m)e\left(\frac{mx}p\right)\chi(x)\right|.\]
	Now, by \Cref{prop:mag-of-gauss-sum}, this is bounded above by
	\[\frac1p\sum_{m=1}^{p-1}\widehat{1_{[a,b]}}\sqrt p\ll\frac1{\sqrt p}\cdot\sqrt p\sum_{m=1}^{p-1}\frac1m\ll\sqrt p\log p.\]
	(Notably, the $m=0$ term provides no contribution.) This completes the proof.
\end{proof}
\begin{corollary}
	The least nonquadratic residue is $O(\sqrt x\log x)$.
\end{corollary}
\begin{proof}
	This is direct from \Cref{thm:polya-vinogradov}.
\end{proof}
\begin{remark}
	For ``short'' intervals, one can do better, which is the point of the Burgess bound.
\end{remark}
\begin{remark}[Nir]
	Here is another nice corollary: fix real numbers $0\le\alpha<\beta\le1$. For a prime $p\equiv1\pmod3$, define the character $\chi\pmod p$ by $\chi(g)\coloneqq\zeta_3$ where $g\in\FF_p^\times$ is a generator. Then the function $f\coloneqq\frac13\left(1+\chi+\chi^2\right)$ has $f(a)$ indicating if $a\pmod p$ is a cube. Thus, the proportion of cubic residues$\pmod p$ in the interval $[\alpha p,\beta p]$ is
	\[\frac{\#\{a\pmod p\text{ is a square}:\alpha p\le a\le \beta p\}}{(\beta-\alpha)p}=\frac1{(\beta-\alpha)p}\cdot\frac13((\beta-\alpha)p+O(\sqrt p\log p))=\frac13+O\left(p^{-1/2}\log p\right).\]
	In other words, cubic residues are equidistributed$\pmod p$. A similar argument works for higher powers.
\end{remark}

\subsection{The Functional Equation for \texorpdfstring{$L(s,\chi)$}{ L(s, chi)}}
Given a primitive Dirichlet character $\chi\pmod q$, we would like to provide a functional equation akin to \Cref{thm:xi-func-eq}. It will be convenient to divide our analysis in the cases $\chi(-1)=1$ and $\chi(-1)=-1$. Notably, $\chi(-1)^2=\chi(1)=1$, so $\chi(-1)\in\{\pm1\}$ is indeed forced.
\begin{remark} \label{rem:why-sign-chi}
	Roughly speaking, the case of $\chi(-1)=1$ makes $L(s,\chi)$ a factor of the Dedekind $\zeta$-function of a real quadratic field, but the case of $\chi(-1)=-1$ makes $L(s,\chi)$ a factor of the Dedekind $\zeta$-function of an imaginary quadratic field. Because the infinite place of $\QQ$ splits differently in these cases, the $\Gamma$-factors used to complete our $L$-function will be different, which is why we must separate our analysis into cases.
\end{remark}
Despite the philosophical remark of \Cref{rem:why-sign-chi}, our entire discussion will avoid discussing number fields. As in the proof of \Cref{thm:xi-func-eq}, our functional equation will follow from the functional equation of a suitably defined $\Theta$-function.
\begin{definition}
	Fix a Dirichlet character $\chi\pmod q$.
	\begin{itemize}
		\item If $\chi(-1)=1$, we define, for $\op{Re}s>0$,
		\[\Theta(s,\chi)\coloneqq\sum_{n\in\ZZ}\chi(n)e^{-\pi n^2s/q}.\]
		\item If $\chi(-1)=-1$, we define, for $\op{Re}s>0$,
		\[\Theta(s,\chi)\coloneqq\sum_{n\in\ZZ}n\chi(n)e^{-\pi n^2s/q}.\]
	\end{itemize}
\end{definition}
Note that $\chi(-1)=-1$ would imply that
\[\sum_{n\in\ZZ}\chi(n)e^{-\pi n^2s}=0,\]
which explains why we should add the factor of $n$ here.
\begin{lemma}
	For any Dirichlet character $\chi\pmod q$ and $\varepsilon>0$, the sum defining $\Theta(s,\chi)$ converges absolutely and uniformly on $\{s:\op{Re}s\ge\varepsilon\}$. Thus, $\Theta(s,\chi)$ defines a holomorphic function on $\{s:\op{Re}s>0\}$.
\end{lemma}
\begin{proof}
	Note that the second sentence follows from the first: by \Cref{lem:holo-limit}, we see that $\Theta(s,\chi)$ is holomorphic on $\{s:\op{Re}s\ge\varepsilon\}$ for any $\varepsilon>0$, so taking the union over all $\varepsilon>0$ achieves the result. We now show our convergences by the Weierstrass $M$-test.
	\begin{itemize}
		\item For $\chi(-1)=1$, we see
		\[\sum_{n\in\ZZ}\left|\chi(n)e^{-\pi n^2s/q}\right|\le\sum_{n\in\ZZ}e^{-\pi n^2\op{Re}s/q}\le\sum_{n\in\ZZ}e^{-\pi n^2\varepsilon/q}=\Theta(\varepsilon/q),\]
		which already converges by \Cref{rem:theta-converges}. Note that the bound $e^{-\pi n^2\varepsilon}$ is independent of $s$, which gives our uniformity.
		\item For $\chi(-1)=-1$, we see
		\[\sum_{n\in\ZZ}\left|n\chi(n)e^{-\pi n^2s/q}\right|\le\sum_{n\in\ZZ}|n|e^{-\pi n^2\op{Re}s/q}\le1+2\sum_{n=1}^\infty ne^{-\pi n\varepsilon/q}.\]
		Because $ne^{-\pi n\varepsilon/q}$ is independent of $s$, it remains to show that the rightmost sum converges. Well, note that $n^3/e^{\pi n\varepsilon/q}\to0$ as $n\to\infty$, so because this is a continuous function, we actually see that it is bounded by some constant $C>0$, so we see
		\[\sum_{n=1}^\infty ne^{-\pi n\varepsilon/q}\le\sum_{n=1}^\infty n\cdot\frac M{n^3}=M\sum_{n=1}^\infty\frac1{n^2},\]
		which converges. For example, this is $\zeta(2)$, which converges by \Cref{prop:basic-dirichlet-series}, say.
		\qedhere
	\end{itemize}
\end{proof}
And here is our functional equation for $\Theta(s,\chi)$.
\begin{proposition}
	Fix a primitive Dirichlet character $\chi\pmod q$.
	\begin{itemize}
		\item If $\chi(-1)=1$, then for any $\op{Re}s>0$,
		\[\Theta(s,\chi)=\frac{\sqrt{q}}{\tau(\overline\chi)}\cdot s^{-1/2}\Theta\left(\frac1s,\overline\chi\right).\]
		\item If $\chi(-1)=-1$, then for any $\op{Re}s>0$,
		\[\Theta(s,\chi)=\frac{i\sqrt{q}}{\tau(\overline\chi)}\cdot s^{-3/2}\Theta\left(\frac1s,\overline\chi\right).\]
	\end{itemize}
\end{proposition}
\begin{proof}
	The argument is similar to \Cref{prop:theta-func-eq}. Note that $\Theta(s,\chi)$ is holomorphic on the region $\{s:\op{Re}s>0\}$ by \Cref{rem:theta-converges}. On the other side, we note that $\op{Re}s>0$ implies that $\op{Re}(1/s)>0$ as well: writing $s=a+bi$ for $a>0$, we have $\frac1s=\frac{a-bi}{|s|^2}$, which also has positive real part. Thus, we see that $\Theta(1/s,\overline\chi)$ is the composite of holomorphic functions and is therefore holomorphic, as are $\frac{\sqrt{q}}{\tau(\overline\chi)}\cdot s^{-1/2}\Theta\left(\frac1s,\overline\chi\right)$ and $\frac{i\sqrt{q}}{\tau(\overline\chi)}\cdot s^{-3/2}\Theta\left(\frac1s,\overline\chi\right)$.

	In total, by the uniqueness of analytic continuation, it therefore suffices to show that our holomorphic functions are equal on $\RR_{>0}$. As such, fix some $t>0$, and we use Poisson summation, roughly in the form of \Cref{cor:use-ps}.
	\begin{itemize}
		\item Suppose $\chi(-1)=1$. Then we note \Cref{lem:adjust-quad-gauss-sum} implies
		\begin{align*}
			\tau(\overline\chi)\Theta(t,\chi) &= \sum_{n\in\ZZ}\chi(n)\tau(\overline\chi)e^{-\pi n^2t/q} \\
			&= \sum_{n\in\ZZ}\tau(\overline\chi,n)e^{-\pi n^2t/q} \\
			&= \sum_{r=0}^{q-1}\Bigg(\overline\chi(r)\sum_{n\in\ZZ}e^{-\pi n^2t/q+2\pi inr/q}\Bigg).
		\end{align*}
		Now, taking the conjugate of \Cref{cor:use-ps}, the inner sum becomes
		\[\sum_{n\in\ZZ}e^{-\pi n^2(t/q)+2\pi in(r/q)}=\frac1{\sqrt{t/q}}\sum_{n\in\ZZ}e^{-\pi(n+r/q)^2/(t/q)},\]
		so
		\begin{align*}
			\tau(\overline\chi)\Theta(t,\chi) &= \sum_{r=0}^{q-1}\Bigg(\overline\chi(r)\cdot\frac1{\sqrt{t/q}}\sum_{n\in\ZZ}e^{-\pi(n+r/q)^2/(t/q)}\Bigg) \\
			&= \frac{\sqrt q}{\sqrt t}\sum_{r=0}^{q-1}\Bigg(\sum_{n\in\ZZ}\overline\chi(r)e^{-\pi(qn+r)^2(1/t)}\Bigg) \\
			&= \frac{\sqrt q}{\sqrt t}\underbrace{\sum_{n\in\ZZ}\overline\chi(n)e^{-\pi n^2(1/t)}}_{\Theta(1/t,\overline\chi)},
		\end{align*}
		where in the last equality we have used the absolute convergence of $\Theta(1/t,\overline\chi)$ to rearrange the sum. Rearranging our equality, we see
		\[\Theta(t,\chi)=\frac{\sqrt{q}}{\tau(\overline\chi)}\cdot t^{-1/2}\Theta\left(\frac1t,\overline\chi\right).\]
		\item Suppose $\chi(-1)=-1$. Again, we note that \Cref{lem:adjust-quad-gauss-sum} implies
		\begin{align*}
			\tau(\overline\chi)\Theta(t,\chi) &= \sum_{n\in\ZZ}\chi(n)\tau(\overline\chi)ne^{-\pi n^2t/q} \\
			&= \sum_{n\in\ZZ}\tau(\overline\chi,n)ne^{-\pi n^2t/q} \\
			&= \sum_{r=0}^{q-1}\Bigg(\overline\chi(r)\sum_{n\in\ZZ}ne^{-\pi n^2t/q+2\pi inr/q}\Bigg).
		\end{align*}
		This time around, Poisson summation in the form of \Cref{cor:use-ps} is not good enough for our purposes, but Poisson summation still suffices. Set $f\colon\RR\to\CC$ by $f(x)\coloneqq xe^{-\pi x^2(t/q)+2\pi ix(r/q)}$. Setting $g(x)\coloneqq e^{-\pi x^2}$, we note that $g'(x)=-2\pi xe^{-\pi x^2}$, so
		\[f(x)=-\frac1{2\pi\sqrt{t/q}}g'(\sqrt{t/q}x)e^{2\pi ix(r/q)}.\]
		Thus, \Cref{lem:fourier-checks} and \Cref{exe:gaussian} tell us that $f$ is Schwarz with Fourier transform given by
		\begin{align*}
			(\mc Ff)(s) &= -\frac1{2\pi\sqrt{t/q}}\cdot\frac1{\sqrt{t/q}}(\mc Fg')\left(\frac{s-r/q}{\sqrt{t/q}}\right) \\
			&= -\frac1{2\pi(t/q)}\cdot2\pi i\left(\frac{s-r/q}{\sqrt{t/q}}\right)e^{-\pi(s-r/q)^2/(t/q)} \\
			&= -i\sqrt qt^{-3/2}(qs-r)e^{-\pi(qs-r)^2/t}.
		\end{align*}
		Now applying \Cref{thm:ps}, we get
		\[\sum_{n\in\ZZ}ne^{-\pi n^2t/q+2\pi inr/q}=\sum_{n\in\ZZ}f(n)=\sum_{n\in\ZZ}(\mc Ff)(n)=\sum_{n\in\ZZ}-i\sqrt qt^{-3/2}(qn-r)e^{-\pi(qn-r)^2/t}.\]
		As such, we get rid of the sign here by noting $\overline\chi(-r)=-\overline\chi(r)$, so
		\begin{align*}
			\tau(\overline\chi)\Theta(t,\chi) &= \sum_{r=0}^{q-1}\Bigg(\overline\chi(r)\sum_{n\in\ZZ}-i\sqrt qt^{-3/2}(qn-r)e^{-\pi(qn-r)^2/t}\Bigg) \\
			&= i\sqrt qt^{-3/2}\sum_{r=0}^{q-1}\Bigg(\sum_{n\in\ZZ}\overline\chi(qn-r)(qn-r)e^{-\pi(qn-r)^2/t}\Bigg) \\
			&= i\sqrt qt^{-3/2}\underbrace{\sum_{n\in\ZZ}\overline\chi(n)ne^{-\pi n^2(1/t)}}_{\Theta(1/t,\overline\chi)}.
		\end{align*}
		Rearranging our equality, we see
		\[\Theta(t,\chi)=\frac{\sqrt q}{\tau(\overline\chi)}\cdot t^{-3/2}\Theta\left(\frac1t,\overline\chi\right),\]
		which is what we wanted.
		\qedhere
	\end{itemize}
\end{proof}

We will focus on the case of $\chi\ne\chi_0$. Now, for $\alpha\in\RR$, an argument similar to \Cref{cor:use-ps} yields
\[\sum_{n\in\ZZ}e^{-\pi(n+\alpha)^2/x}=x^{1/2}\sum_{n\in\ZZ}e^{-\pi n^2x+2\pi in\alpha}.\]
In particular, taking $\alpha=m/p$ for $m\in\ZZ$ grants
\[\sum_{n\in\ZZ}e^{-\pi(n+m/p)^2/x}=\left(\frac xp\right)^{1/2}\sum_{n\in\ZZ}e^{-\pi n^2x/p+2\pi inm/p}.\]
To continue, we will work with $\chi(-1)=1$.\footnote{This is called the ``unramified at $\infty$ case'' because the place here at infinity is totally real.} Here, we set
\[\Theta_\chi(x)\coloneqq\sum_{n\in\ZZ}\chi(n)e^{-\pi n^2x/p}.\]
Roughly speaking, in the case where $\chi(-1)=-1$, this sum would completely vanish, so we would have to add a factor of $n$ or similar to make this summation behave. We will not say more about this case.

Using \Cref{cor:indicate-g}, we see
\begin{align*}
	\Theta_\chi(x) &= \sum_{n\in\ZZ}\frac1{\tau(\overline\chi,1)}\Bigg(\sum_{m=0}^{p-1}\overline\chi(m)\Bigg)e^{mn/p}e^{-\pi n^2x/p} \\
	&= \frac1{\tau(\overline\chi,1)}\sum_{m=0}^{p-1}\chi(m)\sum_{n\in\ZZ}e^{2\pi imn/p-\pi n^2x/p} \\
	&\stackrel*= \frac1{\tau(\overline\chi,1)}\cdot\left(\frac xp\right)^{1/2}\sum_{m=0}^{p-1}\chi(m)\sum_{n\in\ZZ}e^{-\pi(n+m/p)^2p/x} \\
	&= \frac1{\tau(\overline\chi,1)}\cdot\left(\frac xp\right)^{1/2}\sum_{m=0}^{p-1}\chi(m)\sum_{n\in\ZZ}e^{-\pi(pn+m)^2/(px)},
\end{align*}
where we have applied Poisson summation at $\stackrel*=$. Now, the summations loop over all residue classes in $pn+m$, so we see
\[\Theta_\chi(x)=\frac1{\tau(\overline\chi,1)}\cdot\left(\frac xp\right)^{1/2}\sum_{t\in\ZZ}\chi(t)e^{-\pi r^2/(px)}=\frac1{\tau(\overline\chi,1)}\cdot\left(\frac xp\right)^{1/2}\Theta_{\overline\chi}(1/x),\]
where we are also using the fact that $\chi$ is periodic$\pmod p$.

Now, to find our functional equation, we write
\[\Xi_\chi(s)\coloneqq p^{s/2}\pi^{-s/2}\Gamma(s/2)L(s,\chi).\]
Here, the factor of $p$ roughly comes from some kind of conductor, and the $\pi^{-s/2}\Gamma(s/2)$ is our real $\Gamma$-factor. In particular, our functional equation will turn out to be the following result.
\begin{theorem}[Functional equation for $\Xi_\chi$]
	Fix a nontrivial Dirichlet character $\chi\pmod p$ such that $\chi(-1)=-1$. Then $\Xi_\chi(s)$ is entire and satisfies the functional equation
	\[\Xi_\chi(s)=\varepsilon(\overline\chi)\Xi_{\overline\chi}(1-s),\]
	where $\varepsilon_\chi\coloneqq\sqrt q/\tau(\chi)$.
\end{theorem}
\begin{proof}
	We follow the proof of \Cref{thm:xi-func-eq}. One can compute the integral
	\[\Gamma(s/2)(p/\pi)^{s/2}n^{-s}=\int_0^\infty e^{-\pi n^2x/p}x^{s/2}\,\frac{dx}x\]
	for $\op{Re}s>1$. Summing, we see
	\[\Xi_\chi(s)=\frac12\int_0^\infty\Theta_\chi(x)x^{s/2}\,\frac{dx}x.\]
	At this point, one can see directly that this right-hand side is entire for all $s\in\CC$: indeed, $\Theta_\chi(x)$ rapidly decays at both $0$ and $\infty$, so its Mellin transform is safe for all $s\in\CC$. Thus, we already see that $\Xi_\chi$ is entire. In particular, this equality now holds for all $s\in\CC$.

	Anyway, applying the usual variable change $x\mapsto1/x$, we see
	\begin{align*}
		\Xi_\chi(s) &= \frac12\int_0^\infty\Theta_\chi(1/x)x^{-s/2}\,\frac{dx}x \\
		&= \frac12\int_0^\infty\left(\frac{p^{1/2}}{\tau(\overline\chi)}\Theta_{\overline\chi}(x)\right)x^{-s/2}\,\frac{dx}x,
	\end{align*}
	where we have used the functional equation for $\Theta_\chi$ at the last equality. Upon using the analytic continuation for $\Xi_{\overline\chi}$ provided by the previous paragraph, we get
	\[\Xi_\chi(s)=\varepsilon_{\overline\chi}\Xi_{\overline\chi}(1-s).\]
	This completes the proof.
\end{proof}
\begin{remark}
	The element $\varepsilon_\chi$ in the functional equation is called ``the root number.'' There is a wealth of research trying to understand their behavior.
\end{remark}
\begin{remark}
	We omit the case of $\chi(-1)=-1$.
\end{remark}

\end{document}