% !TEX root = ../notes.tex

\documentclass[../notes.tex]{subfiles}

\begin{document}

\section{February 15}

We open class by remarking that the arguments of Gauss sums turn out to be equidistributed. As references, we mention ``Sato--Tate Theorems for Finite Field Mellin Transforms'' and ``Gauss sums, Kloosterman Sums, and Monodromy Groups.'' Let's give a quick algebraic proof of the supplement.
\begin{proposition}
	Fix an odd prime $p$. Then $\left(\frac2p\right)=(-1)^{\left(p^2-1\right)/8}$.
\end{proposition}
\begin{proof}
	Let $\zeta\in\overline{\FF_p}$ be an eighth root of unity. (We can also find $\zeta$ in $\FF_{p^4}$ because $\zeta$ is a root of the equation $\zeta^4+1=0$.) Because $\zeta^4=-1$, we see $\zeta^2=\zeta^{-2}$, so
	\[\left(\zeta+\zeta^{-1}\right)^2=\zeta^2+\zeta{-2}+2=2.\]
	Now, we see $\left(\frac2p\right)=1$ is equivalent to having $\zeta+\zeta^{-1}\in\FF_p$, which because $\op{Gal}(\overline{\FF_p}/\FF_p)$ is topologically generated by the Frobenius, is equivalent to
	\[\zeta^p+\zeta^{-p}=\left(\zeta+\zeta^{-1}\right)^p\stackrel?=\zeta+\zeta^{-1}.\] 
	Now, this is equivalent to $p\equiv\pm1\pmod8$; indeed, if $p\equiv\pm3\pmod8$, then we see $\zeta^3+\zeta^{-3}=-\zeta-\zeta^{-1}$ is the negative root.
\end{proof}

\subsection{Counting Zeroes of \texorpdfstring{$L(s,\chi)$}{ L(s, chi)}}
We would like to generalize \Cref{thm:num-zeroes-zeta}. For this, we have the following definition.
\begin{notation}
	Fix a primitive character $\chi\pmod q$. Then we define
	\[N_\chi(T)\coloneqq\#\{|\rho|\le T:\Xi_\chi(\rho)=0\}.\]
\end{notation}
\begin{remark}
	Note that the functional \Cref{thm:l-chi-func-eq} does tell us that all zeroes live in the critical strip, but we no longer have conjugate symmetry because $\chi$ might not be real.
\end{remark}
Arguing as before, the set
\[\left\{\frac{L'(2+iT,\chi)}{L(2+iT,\chi)}:T\in\RR,q,\chi\pmod q\right\}\]
is uniformly bounded above, and one can show as in \Cref{lem:bound-zeroes} that
\[\sum_{\Xi_\chi(\rho)=0}\frac1{1+|\op{Im}\rho-T|^2}\ll\log(|T|+2)+\log q,\]
where the implied constant is absolute. Following the rest of the proof of \Cref{thm:num-zeroes-zeta} from the argument principle again gives us
\[\frac12N_\chi(T)=\frac T{2\pi}\log\left(\frac{qT}{2\pi}\right)-\frac T{2\pi}+O(\log q(|T|+2)),\]
where the implied constants are still absolute.

\subsection{Primality Testing}
As an application of what we've done so far, we describe a primality test assuming the Generalized Riemann Hypothesis (GRH).
\begin{ques}
	Can one determine if an integer $n$ is prime in $\op{Poly}(\log n)$ time?
\end{ques}
This is known unconditionally (via the AKS algorithm), and there are fast probabilistic algorithms, but we describe an algorithm which works assuming GRH. Here is the statement of GRH.
\begin{conj}[Generalized Riemann Hypothesis]
	For any primitive character $\chi\pmod q$, the zeroes of $\Xi_\chi(s)$ all lie on the vertical line
	\[\op{Re}s=\frac12.\]
\end{conj}
And now here is our result.
\begin{theorem}[Miller--Rabin]
	Assume GRH. Then we can test if an integer $n$ is prime in $\op{Poly}(\log n)$ time.
\end{theorem}
For this, we describe the Miller--Rabin primality test, which is one of the more efficient probabilistic primality tests.
\begin{lemma}[Fermat's little theorem] \label{lem:fermat-little}
	Fix a prime $p$. If $\gcd(a,p)=1$, then $a^{p-1}\equiv1\pmod p$.
\end{lemma}
\begin{proof}
	This follows from Lagrange's theorem: the order of $a\in\FF_p^\times$ divides $\left|\FF_p^\times\right|=p-1$.
\end{proof}
\begin{remark}
	It turns out that there are infinitely many integers $n$ such that
	\[a^n\equiv a\pmod n.\]
	For example, $n=561$ works. Thus, one cannot really use \Cref{lem:fermat-little} to test for primality.
\end{remark}
Instead, we will want to use \Cref{prop:euler-criterion}, which implies
\[a^{(p-1)/2}\equiv\left(\frac ap\right)\pmod p\]
for $\gcd(a,p)=1$. In order to compute Legendre symbols efficiently, we must introduce the Jacobi symbol.
\begin{definition}[Jacobi]
	Fix an odd integer $n$. Then for integers $a$, we define the \textit{Jacobi symbol}
	\[\left(\frac an\right)=\prod_p\left(\frac ap\right)^{\nu_p(n)}.\]
\end{definition}
\begin{remark} \label{rem:jacobi-recip}
	The Jacobi symbol, like the Legendre symbol, is multiplicative in the numerator, but it is also multiplicative in the denominator. One can use this to show ``Jacobi reciprocity,'' which asserts that odd $a,b\in\ZZ$ grant
	\[\left(\frac ab\right)\left(\frac ba\right)=(-1)^{(a-1)(b-1)/4}.\]
	One also gets the supplement $\left(\frac2n\right)=(-1)^{\left(n^2-1\right)/8}$.
\end{remark}
\begin{remark}
	\Cref{rem:jacobi-recip} allows us to compute Jacobi symbols efficiently via reciprocity. Roughly speaking, we are basically just doing the Euclidean algorithm and keeping track of some signs.
\end{remark}
We now do primality testing with the Euler criterion.
\begin{lemma}
	Fix an odd integer $n$. If $n$ is not prime, there exists some $a$ such that
	\[a^{(n-1)/2}\not\equiv\left(\frac an\right)\pmod n.\]
\end{lemma}
\begin{proof}
	Omitted. The idea is to split this into two cases: if $n$ is squarefree, one can divide the result via the Chinese remainder theorem into various$\pmod p$ statements; if $n$ is not squarefree, then one can look$\pmod{p^2}$ somewhere to get the result.
\end{proof}
This suggests the following algorithm to test if $n$ is prime.
\begin{enumerate}
	\item Choose some random $a\in[1,n)$.
	\item Compute the Jacobi symbol $\left(\frac an\right)$. Via the Euclidean algorithm, one can compute this in $O\left((\log n)^2\right)$ time.
	\item Compute $a^{(n-1)/2}\pmod n$ using exponentiation by repeated squarings.
\end{enumerate}
In fact, one can show (using GRH), that one only has to check $a\ll(\log n)^2$, so the entire algorithm runs in $O\left((\log n)^4\right)$.

\end{document}