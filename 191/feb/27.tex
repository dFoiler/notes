% !TEX root = ../notes.tex

\documentclass[../notes.tex]{subfiles}

\begin{document}

\section{February 27}

Last class we showed Siegel's theorem. In particular, for real primitive characters $\chi$ with conductor $q$, we have $|L(1,\chi)|\gg q^{-\varepsilon}$ for any $\varepsilon>0$, where the implied constant is ineffective.
\begin{remark}
	As a correction, Mordell's conjecture that curves of genus at least $2$ have at most finitely many rational solutions. For example, for any $n\ge4$, the equation $x^n+y^n=1$ has finitely many rational solutions. We mention this because computing the number of points exactly (though finite) is ineffective; this is an active area of research in arithmetic geometry.
\end{remark}
In this second half of the semester, we are going to cover quite a few disparate topics. In particular, after doing a few more applications (the Burgess bound and elementary counting over finite fields), we will turn to sieve theory (e.g., the weak Goldbach conjecture).

\subsection{The Burgess Bound}
% iwaniec ch 11 and 12
We are going to do the analytic part of the argument here. Roughly speaking, we are interested in bounding sums which look like
\[\left|\sum_{n=1}^{N}\chi(n)\right|.\]
From \Cref{thm:polya-vinogradov}, we do have a bound of $\sqrt q\log q$, but we would like to do better for smaller $N$.
\begin{remark}
	Under the Generalized Riemann Hypothesis, one can achieve
	\[\left|\sum_{n=1}^{N}\chi(n)\right|\ll\sqrt q\log\log q.\]
	This is due to Montgomery--Vaughan. However, one expects to have $\ll_\varepsilon\sqrt Nq^\varepsilon$ for any $\varepsilon>0$.
\end{remark}
For applications, we are primarily interested in the smallest $q$ for which $\chi(q)\ne1$, for which these sums help, up to some constants. (For example, this implies an upper bound on the least nonquadratic residue.) We are going to achieve some cancellation for $N$ bigger than $q^{1/4}$, which is better than what is given by \Cref{thm:polya-vinogradov}.
\begin{theorem}[Burgess] \label{thm:burgess}
	Fix $\delta,\varepsilon>0$. There exists some $p_0(\delta,\varepsilon)>0$ such that, for primes $p>p_0(\delta,\varepsilon)$ and $N>p^{1/4+\delta}$, we have
	\[\left|\sum_{n=M+1}^{M+N}\chi(n)\right|<\varepsilon N\]
	for nontrivial characters $\chi\pmod p$.
\end{theorem}
\begin{remark}
	One can even allow the conductor $p$ to be cube-free, where $\chi\pmod p$ is now forced to be primitive.
\end{remark}

\subsection{A Little on Curves}
We are going to show \Cref{thm:burgess} in the real character case, where $\chi=\left(\frac\bullet p\right)$. For this, we are going to want the following ingredient.
\begin{definition}[hyperelliptic curve]
	Fix a polynomial $P(x)\in\ZZ[x]$ of degree $r\ge3$. Given a prime $p$, we may consider \textit{the hyperelliptic curve}
	\[C_p\coloneqq\left\{(x,y)\in\FF_p^2:y^2=P(x)\right\}.\]
	Further, we say that $C_p$ is \textit{irreducible} if and only if $P(x)\pmod p$ is not a square.
\end{definition}
\begin{theorem}[Riemann hypothesis for curves] \label{thm:rh-curves}
	Fix an irreducible projective curve $C/\FF_p$, and define $N_p\coloneqq\#C(\FF_p)$. Then
	\[|N_p-(p+1)|<8(\deg C)\sqrt p\]
	for $p$ large enough.
\end{theorem}
\begin{remark}
	Our definition of $N_p$ is including points at infinity, so one must be careful about just counting $\FF_p$-points on a curve. Our following argument
\end{remark}
We are not going to prove \Cref{thm:rh-curves} in full generality (e.g., this should hold for arbitrary projective varieties), but we will be able to show a somewhat weaker statement in our case, which will be good enough for our purposes.

Quickly, let's explain why we are looking at these hyperelliptic curves at all.
\begin{lemma} \label{lem:char-sum-from-point}
	Fix a polynomial $P(x)\in\ZZ[x]$ of degree $r$. Further, fix a prime $p$.
	\begin{listalph}
		\item There are most $r$ points $(x,y)\in\FF_p^2$ such that $y^2=f(x)$ such that $P(x)\equiv0\pmod p$.
		\item There are either zero or two points $(x,y)\in\FF_p^2$ such that $y^2=f(x)$ if $P(x)\not\equiv0\pmod p$; we have zero if $\left(\frac{P(x)}p\right)=1$ and $0$ if $\left(\frac{P(x)}p\right)=-1$.
		\item In total, the number of points $(x,y)\in\FF_p^2$ is
		\[p+\sum_{x\pmod p}\left(\frac{P(x)}p\right).\]
	\end{listalph}
\end{lemma}
\begin{proof}
	This is somewhat direct. The first statement (a) holds because $P(x)$ has at most $r$ roots$\pmod p$, and then $y=0$ is forced. The second statement (b) holds by tracking if $P(x)$ is a quadratic residue or not. Then the third statement (c) holds by the above casework on $x\pmod p$.
\end{proof}
Comparing \Cref{lem:char-sum-from-point} with \Cref{thm:rh-curves} produces the bound
\[\left|\sum_{x\pmod p}\left(\frac{P(x)}p\right)\right|\le 2+r\sqrt p.\]
One can improve the constant here with some effort. With elementary methods, one can actually achieve
\[\left|\sum_{x\pmod p}\left(\frac{P(x)}p\right)\right|\le 9r\sqrt p.\]

\end{document}