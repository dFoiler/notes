% !TEX root = ../notes.tex

\documentclass[../notes.tex]{subfiles}

\begin{document}

\section{February 22}

This lecture was recorded.

\subsection{Real Primitive Characters}
We continue discussing primitive characters $\chi$ modulo prime powers $p^\nu$. Recall that these are controlled for $p$ odd because $\left(\ZZ/p^\nu\right)^\times$ is still cyclic. In particular, there is exactly one real primitive Dirichlet character$\pmod{p}$ and none for $p^\nu$ for $\nu>1$.

For $p=2$, one must be a little more careful.
\begin{itemize}
	\item There is a unique primitive real character$\pmod2$ which is the trivial one.
	\item There is a real primitive character $\chi_4\pmod4$ given by $\left(\frac{-1}\bullet\right)$.
	\item Further, there are two real primitive characters$\pmod8$ given by $\left(\frac2\bullet\right)$ and $\left(\frac{-2}\bullet\right)$.
	\item There are no more real primitive characters$\pmod{2^\nu}$.
\end{itemize}
Roughly speaking, these characters can give all primitive characters by the Chinese remainder theorem. We can be a little more explicit about this.
\begin{remark}
	Using the Kronecker symbol, one can write all real primitive Dirichlet characters as $\left(\frac d\bullet\right)$. The values $d$ yielding primitive characters are the ones which are fundamental discriminants; namely,
	\[d\equiv\begin{cases}
		N & \text{if }N\equiv1\pmod4\text{ and is squarefree}, \\
		4N & \text{if }N\equiv2,3\pmod4\text{ and is squarefree}.
	\end{cases}\]
	Note that we permit $N<0$.
\end{remark}
\begin{remark}
	Fix a real primitive character $\chi=\left(\frac d\bullet\right)$. Then $\zeta(s)L(s,\chi)$ is the Dedekind $\zeta$-function for the quadratic field $\QQ(\sqrt d)$. Roughly speaking, this explains why $\zeta(s)L(s,\chi)$ should have positive coefficients. See \cite[Chapter~6]{davenport-mult-nt} for details.
\end{remark}

\subsection{Zero-Free Regions for \texorpdfstring{$L(s,\chi)$}{ L(s, chi)}}
We now establish some zero-free regions for $L(s,\chi)$. For complex primitive Dirichlet characters $\chi$, one can use the product
\[\zeta(s)^3L(s+it_0,\chi)^2L(s-it_0,\overline\chi)^2L(s+2it_0,\chi^2)L(s-2it_0,\overline\chi^2)\]
combined with \Cref{lem:zero-free-region} to see that $L(s,\chi)$ has no zeroes in the region
\[\left\{s\in\CC:\op{Re}s>1-\frac c{\log(q(|t|+2))}\right\}.\]
Note that any imprimitive character $\chi$ can be reduced to a primitive one by adjusting finitely many Euler factors of $L(s,\chi)$, which do not change a vanishing region.

In the case of real primitive Dirichlet characters $\chi$, one does not do as well. Notably, $\chi^2=\overline\chi^2$ is the principal character, so $L(s+2it_0,\chi^2)$ is lacking a pole at $s=1$. Nonetheless, an argument will still work except at $t_0=0$, where we see that we have at most one zero in the real numbers in the region
\[\left\{s\in\CC:\op{Re}s>1-\frac c{\log(q(|t|+2))}\right\}.\]
However, at most one zero is still not good enough for our purposes. Well, to deal with a possibly real zero, we can apply \Cref{lem:zero-free-region} to
\[\zeta(s)L(s,\chi),\]
and we can produce a lower bound (using the proof of $L(1,\chi)\ne0$) to produce the lower bound
\[|L(1,\chi)|>cq^{-1/2}.\]
In particular, from the summation, our Dirichlet series is $1$ on squares, which is where the square root comes from.
\begin{remark}
	In contrast, we can upper-bound $L(s,\chi)$ relatively easily as
	\[|L'(\sigma,\chi)|\ll(\log q)^2\qquad\text{for}\qquad1-\frac1{\log q}\le\sigma\le1,\]
	and
	\[|L(\sigma,\chi)\ll\log q,\qquad\text{for}\qquad1-\frac1{\log q}\le\sigma\le1.\]
	For details, see \cite[Chapter~14]{davenport-mult-nt}. Roughly speaking, one can just use the Dirichlet series for $L(s,\chi)$. In particular, early terms rotate quickly and can be bounded as sines and cosines, and the later terms are small.
\end{remark}
The idea above is that we can use the above derivative combined with our lower bound for $L(1,\chi)$ in order to get some very small interval in the real numbers where we are nonzero. This is indeed technically a zero-free region; in the next lecture, we will cover Siegel's theorem, which is ineffective but will do a little better.

In total, we get that any problematic real zero $\beta$ of $L(s,\chi)$ must satisfy
\[\beta<1-\frac c{(\log q)^2\sqrt q}.\]
Of course, this is much worse when compared to $c/\log q$, but it does give us a zero-free region to work with.
\begin{remark}[Landau]
	Fix $\chi_d\coloneqq\left(\frac d\bullet\right)$. Then one can use
	\[\zeta(s)L(s,\chi_{d_1})L(s,\chi_{d_2})L(s,\chi_{d_1}\chi_{d_2}),\]
	for $d_1\ne d_2$, which is the $\zeta$ function of a biquadratic field. Now, because each of these $L$-functions have at most one real zero in the desired region $\left(1-\frac c{\log|d_1d_2|},1\right]$, we note that a zero for $L(s,\chi_{d_1})$ will force the other $L$-functions to not have zeroes! An idea like this is able to produce a zero-free region, and it is the key input to Siegel's theorem.
	% ch 14 of davenport
\end{remark}

\end{document}