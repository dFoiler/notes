% !TEX root = ../notes.tex

\documentclass[../notes.tex]{subfiles}

\begin{document}

\section{February 10}

Today we conclude discussing the functional equation for $L(s,\chi)$.

\subsection{All Functional Equations}
We will want to state our functional equation for primitive Dirichlet characters, so here is the definition of primitive characters.
\begin{definition}[primitive]
	A Dirichlet character $\chi\pmod q$ is \textit{primitive} if and only if $\chi\colon(\ZZ/q\ZZ)^\times\to\CC^\times$ has no smaller period than $q$.
\end{definition}
\begin{remark}
	In more general contexts, $q$ is called the ``conductor'' of $\chi$.
\end{remark}
Here is our statement.
\begin{theorem} \label{thm:l-chi-func-eq}
	Fix a primitive Dirichlet character $\chi\pmod q$ for $q>1$. Then set $a_\chi\coloneqq\frac12(1-\chi(-1))\in\{0,1\}$, and define
	\[\Xi_\chi(s)\coloneqq\left(\frac q\pi\right)^{(s+a)/2}\Gamma\left(\frac{s+a}2\right)L(s,\chi).\]
	Then $\Xi_\chi$ is entire and satisfies the functional equation
	\[\Xi_\chi(s)=\varepsilon_{\overline\chi}\Xi_{\overline\chi}(1-s),\]
	where $\varepsilon_\chi\coloneqq i^aq^{1/2}/\tau(\chi)$, where $\tau(\overline\chi)$ is a Gauss sum.
\end{theorem}
\begin{proof}
	Omitted.
\end{proof}
We should probably say a few words about these more general Gauss sums.
\begin{lemma}
	Fix a primitive Dirichlet character $\chi\pmod q$. Then $\tau(\chi,m)=\overline\chi(m)\tau(\chi,1)$.
\end{lemma}
\begin{proof}
	If $\gcd(m,q)=1$, then the same proof from earlier goes through, so we may assume that $\gcd(m,q)>1$. Further, if $q\mid m$, then everything vanishes, so we may assume that $q\nmid m$.

	In this case, we would like to show that $\tau(\chi,m)=0$, or in other words,
	\[\sum_{k=0}^{q-1}\chi(k)e\left(\frac{mk}q\right)\stackrel?=0.\]
	For this, we must use the fact that $\chi$ is primitive. Now, take $d\coloneqq\gcd(m,q)$ and $m'\coloneqq m/d$ and $q'\coloneqq q/d$ so that $\gcd(m',q')=1$. As such, we compute
	\begin{align*}
		\tau(\chi,m) &= \sum_{k=0}^{q-1}\chi(k)e\left(\frac mkq\right) \\
		&= \sum_{k=0}^{q'-1}\sum_{r=0}^{d-1}\chi(kq'+r)e\left(\frac{m'(kq'+r)}{q'}\right) \\
		&= \sum_{r=0}^{d-1}\Bigg(e\left(\frac{m'r}{q'}\right)\sum_{k=0}^{q'-1}\chi(kq'+r)\Bigg).
	\end{align*}
	Because $\chi$ is primitive, $\chi\pmod{q'}$ is not periodic, so we can find $r_1\equiv r_2\pmod{q'}$ such that $\chi(r_1)$ and $\chi(r_2)$ are nonzero and not equal. It follows from the above expression that
	\[\chi(r_1)\tau(\chi,m)=\chi(r_2)\tau(\chi,m)\]
	by taking and then re-indexing the sums over $r$ in our last summation.

	Alternatively, if $\chi(tq'+1)=1$ for all $t\in\ZZ$, then it actually follows that $\chi$ is periodic$\pmod{q'}$, which is a contradiction, so it is enough to check that
	\[\tau(\chi,m)=\chi(tq'+1)\tau(\chi,m)\]
	for all $t$ by again just re-indexing the relevant sum.
\end{proof}
\begin{lemma}
	Fix a primitive Dirichlet character $\chi\pmod q$. Then $|\tau(\chi)|^2=\sqrt q$.
\end{lemma}
\begin{proof}
	Again sum over the $\tau(\chi,m)$ as in \Cref{lem:mag-of-gauss-sum}.
\end{proof}
The rest of the proof of \Cref{thm:l-chi-func-eq} now follows exactly as we proceeded last class.

\subsection{A Zero-Free Region for Complex Characters}
We continue to let $\chi\pmod q$ be a primitive character.
\begin{remark}
	When $\chi\pmod q$ is not primitive, then $L(s,\chi)$ is equal, up to a few Euler factors, to some $L(s,\chi')$, where $\chi'$ has smaller modulus. These finite Euler factors do not affect our zero-free region.
\end{remark}
\begin{example}
	Fix
	\[D(s)\coloneqq\zeta(s)^3L(s,\chi)^2L(s,\overline\chi)^2L(s,\chi^2)L(s,\overline\chi^2).\]
	When $\chi$ is a complex character, one can check that this has nonnegative real coefficients, so \Cref{lem:zero-free-region} goes through and grants us a zero-free region using the same argument we used for $\zeta$. Namely, we still get a zero-free region which looks like
	\[\left\{\sigma+it:\sigma>1-\frac c{\log|q(|t|+2)|}\right\},\]
	where $c$ is some fixed constant. In particular, we get a smaller zero-free region for larger $q$.
\end{example}

\subsection{The Quadratic Gauss Sum}
We take a moment to evaluate some Gauss sums.
\begin{lemma} \label{lem:adjust-quad-gauss-sum}
	Fix a prime $p$. Then
	\[\tau\coloneqq\sum_{k=1}^{p-1}\left(\frac kp\right)e\left(\frac kp\right)=\sum_{k=0}^{p-1}e\left(\frac{k^2}p\right).\]
\end{lemma}
\begin{proof}
	Quickly, note that $\sum_{k=0}^{p-1}e(k/p)=0$, so we can add this to our original sum to see
	\[\tau=\sum_{k=0}^{p-1}e\left(\frac{k^2}p\right)\]
	by counting the number of times quadratic and nonquadratic residues appear. Explicitly, $k^2=0$ happens once, $k^2$ hits quadratic residues twice, and $k^2$ hits nonquadratic residues zero times; this exactly matches $1+\left(\frac ap\right)$.
\end{proof}
The benefit of \Cref{lem:adjust-quad-gauss-sum} is that we can use this as our ``definition'' of the Gauss sum, namely remove the hypothesis that $p$ is prime.
\begin{proposition} \label{prop:evaluation-of-gauss-sum}
	Fix some odd integer $q$. Then
	\[\sum_{r=0}^{q-1}e\left(\frac{r^2}q\right)=\frac{1-i^q}{1-i}\cdot\sqrt q.\]
\end{proposition}
\begin{proof}
	We will use Poisson summation; the point is to turn our Gauss sum into an infinite sum by adding some dampening factor. As such, define $f(t)\coloneqq\Theta(t/\sqrt\pi)$ so that \Cref{prop:theta-func-eq} tells us that
	\begin{equation}
		f(t)=\left(\frac\pi t\right)^{1/2}f\left(\frac{\pi^2}t\right) \label{eq:func-eq-almost-theta}
	\end{equation}
	for $t>0$. Everything here is holomorphic on $\op{Re}s>0$, so we can extend this identity to work for $\op{Re}s>0$ by the uniqueness of analytic continuation.

	Notably, for $\varepsilon>0$, we see
	\begin{align*}
		f\left(\varepsilon+\frac{2\pi i}q\right) &= 1+2\sum_{n=1}^\infty e^{-n^2}e^{-2\pi in^2/q} \\
		&= 1+2\sum_{r=0}^{q-1}\Bigg(e^{-2\pi ir^2/q}\sum_{m=0}^\infty e^{-(r+mq)^2\varepsilon}\Bigg).
	\end{align*}
	Now, one can check that $\varepsilon\to0^+$ enforces
	\[\sum_{m=0}^\infty e^{-(r+mq)^2\varepsilon}\sim\frac{\sqrt\pi}{2q\sqrt\varepsilon}\]
	by doing Poisson summation and only focusing on the leading term. In particular, we see that
	\[f\left(\varepsilon+\frac{2\pi i}q\right)\sim\frac{\sqrt\pi}{2q\sqrt\varepsilon}\sum_{r=1}^qe^{-2\pi ir^2/q}\]
	as $\varepsilon\to0^+$. As such, using the functional equation \eqref{eq:func-eq-almost-theta} for $f$, as well as the computation
	\[\frac{\pi^2}{\varepsilon+2\pi i/q}=-\frac{\pi iq}2+\frac{q^2}4\varepsilon+O\left(\varepsilon^2\right)\]
	to take the asymptotics on the other side. This will complete the proof.
\end{proof}
\begin{remark}
	One can use \Cref{prop:evaluation-of-gauss-sum} to prove the law of quadratic reciprocity, essentially by comparing Gauss sums$\pmod p$ and $\pmod q$ for distinct primes $p$ and $q$.
\end{remark}

\end{document}