% !TEX root = ../notes.tex

\documentclass[../notes.tex]{subfiles}

\begin{document}

\section{February 17}

The next two classes (Wednesday and Friday) will be recorded and posted online.

\subsection{Deterministic Solovay--Strassen}
Here is our main theorem, which tells us that the Solovay--Strassen primality test can be made deterministic.
\begin{theorem} \label{thm:miller}
	Suppose that $q$ is an odd integer which is not prime. Assuming GRH, then there exists $a\ll(\log q)^2$ such that
	\[\left(\frac aq\right)\not\equiv a^{(q-1)/2}\pmod q.\]
\end{theorem}
We will require the following result.
\begin{proposition} \label{prop:get-subgroup-from-small}
	Let $q$ be an odd integer. Given a subgroup $A\subseteq(\ZZ/q\ZZ)^\times$. Assuming GRH, there exists an absolute constant $c$ such that
	\[\left\{[k]\in(\ZZ/n\ZZ)^\times:k\le c(\log q)^2\right\}\subseteq A\]
	implies $A=(\ZZ/q\ZZ)^\times$.
\end{proposition}
One can quickly prove \Cref{thm:miller} from \Cref{prop:get-subgroup-from-small}.
\begin{proof}[Proof of \Cref{thm:miller}]
	Indeed, the set
	\[A\coloneqq\left\{a\in(\ZZ/q\ZZ)^\times:\left(\frac aq\right)\not\equiv a^{(n-1)/2}\pmod q\right\}\]
	is a subgroup of $(\ZZ/q\ZZ)^\times$. Because $q$ is not prime, we know that $A\ne(\ZZ/q\ZZ)^\times$, so it follows from \Cref{prop:get-subgroup-from-small} that $A$ does not contain all $[k]\in(\ZZ/q\ZZ)^\times$ such that $k\le c(\log q)^2$. The conclusion follows.
\end{proof}
We now attack \Cref{prop:get-subgroup-from-small}.
\begin{proof}[Proof of \Cref{prop:get-subgroup-from-small}]
	Suppose for the sake of contradiction that $A\ne(\ZZ/q\ZZ)^\times$. Now, the quotient $(\ZZ/q\ZZ)^\times/A$ is nontrivial and abelian, so it has some nonzero character. Pulling this character to $(\ZZ/q\ZZ)^\times$, we have a Dirichlet character $\chi\pmod q$ such that
	\[\chi(k)=1\]
	for each $0\le k\le c(\log q)^2$ coprime to $n$.

	As in the proof of the Prime number theorem, we want to consider the infinite sum
	\[\sum_{n\le x}\Lambda(n)\chi(n)=\frac1{2\pi i}\int_{\op{Re}s=c}\left(-\frac{L'(s,\chi)}{L(s,\chi)}\right)x^s\frac{ds}s\]
	and shift the contour over to the left. To do this, we apply smoothing $\psi$ to get a smooth function compactly supported on $[1/4,3/4]$. Arguing as in Dirichlet's theorem, we see
	\[\sum_{n\ge0}\Lambda(n)\chi(n)\psi\left(\frac nx\right)=\frac1{2\pi i}\int_{\op{Re}s=2}\left(-\frac{L'(s,\chi)}{L(s,\chi)}\right)(\mc M\psi)(s)x^s\,ds.\]
	Shifting the contour is somewhat delicate, but it can be done similarly as in our proof of the Prime number theorem. This gives
	\[\sum_{n\ge0}\Lambda(n)\chi(n)\psi\left(\frac nx\right)=-\sum_{L(\rho,\chi)=0}(\mc M\psi)(\rho_\chi)x^{\rho_\chi}\]
	plus some smaller error terms. By GRH, we may assume that all the roots lie on $\op{Re}s=\frac12$, so this is bounded (up to a constant) by
	\[\sqrt x\log q.\]
	Notably, the number of zeroes does not increase very much, especially in comparison to the rapid decay of $\mc M\psi$. As such, we see
	\[x\ll\left|\sum_{n\ge0}\Lambda(n)\chi(n)\psi\left(\frac nx\right)\right|\ll\sqrt x\log q,\]
	so $x\ll(\log q)^2$. However, setting $x=c(\log q)^2$ for $c$ large enough will break this bound, which is our contradiction.
\end{proof}

\subsection{Imprimitive Characters}
Let's talk a little more about our characters.
\begin{definition}[conductor] \nirindex{imprimitive}
	Fix a Dirichlet character $\chi\pmod q$. Then the \textit{conductor} $f(\chi)$ is the minimal period of $\chi$ restricted $\{n\in\ZZ:\gcd(n,q)=1\}$. If $f(\chi)\ne q$, then $\chi$ is said to be \textit{imprimitive}.
\end{definition}
Roughly speaking, one can take characters and reduce them to the primitive case by pretending they are Dirichlet characters modulo their conductor.
\begin{definition}[induces]
	Fix a Dirichlet character $\chi\pmod q$ with conductor $f$. Then the Dirichlet character ``$\chi\pmod f$'' is primitive (by construction) and is said to \textit{induce} $\chi\pmod q$.
\end{definition}
\begin{example}
	The principal Dirichlet characters are induced by the constantly $1$ character.
\end{example}
\begin{remark}
	The point is that a Dirichlet character $\chi\pmod q$ induced by a primitive Dirichlet character $\chi'\pmod f$ has $L$-function given by
	\[L(s,\chi')=L(s,\chi)\prod_{\substack{p\nmid f\\p\mid q}}\frac1{1-\chi'(p)p^{-s}}.\]
	Note that these finitely many Euler factors do not add any zeroes or poles or similar.
\end{remark}
\begin{remark}
	Under our philosophy that the real characters are the hard ones. By the Chinese remainder theorem, it suffices to understand characters modulo prime powers $p^\nu$. If $p$ is odd, then $\left(\ZZ/p^\nu\ZZ\right)^\times$ is cyclic, so the only real characters are either principal or is $\left(\frac\bullet p\right)$ depending on if the generator $g\in\left(\ZZ/p^\nu\ZZ\right)^\times$ gets sent to $-1$ or $1$. If $p=2$, then $\left(\ZZ/2^\nu\ZZ\right)\cong\langle-1\rangle\times\langle5\rangle=\ZZ/2\ZZ\times\ZZ/2^{\nu-2}\ZZ$ for $\nu\ge2$, and we can decompose this as one might expect. Namely, there is one modulo $4$ and two modulo $8$. 
\end{remark}

\end{document}