% !TEX root = ../notes.tex

\documentclass[../notes.tex]{subfiles}

\begin{document}

\section{February 24}

Again, this lecture was recorded. Our goal here is to state and prove Siegel's theorem. This will be our first ``ineffective'' theorem.

\subsection{Siegel's Theorem}
Fix a primitive Dirichlet character $\chi\pmod q$. Recall that there is an effective constant $c>0$ such that $L(s,\chi)$ has no zeroes in the region
\[\left\{s:\op{Re}s>1-\frac c{\log(q|t|+2)}\right\}\]
except possibly a real zero in the case where $\chi$ is a real character.

We also recall from Landau that distinct primitive Dirichlet characters $\chi_1$ and $\chi_2$ with coprime moduli will make
\[\zeta_K(s)\coloneqq\zeta(s)L(s,\chi_1)L(s,\chi_2)L(s,\chi_1\chi_2)\]
have nonnegative coefficients.
\begin{remark}
	If we set $\chi_1=\left(\frac{d_1}\bullet\right)$ and $\chi_2=\left(\frac{d_2}\bullet\right)$, then the above Dirichlet series is the Dedekind $\zeta$-function associated to the biquadratic field $K\coloneqq\QQ(\sqrt{d_1},\sqrt{d_2})$. In particular, the fact that $\zeta_K$ has nonnegative coefficients is direct from this definition.
\end{remark}
Now, one can repeat the argument from \Cref{lem:zero-free-region} to get that there is at most one zero of $\zeta_K$ in the desired region, that it must be real, and that it must live in $\left(1-c/\log(q_1q_2),1\right]$. As an aside, we do note that $L(1,\chi)\gg1/\sqrt q$, so we get somewhat automatically (from also bounding $L'(s,\chi)$) that $L(s,\chi)$ does not vanish on $\left(1-c/(\sqrt q(\log q)^2),1\right)$ for some constant $c>0$.
\begin{remark}
	As an example, one can estimate the first prime $p$ which is $p\equiv a\pmod q$ for some $(a,q)$ with $\gcd(a,q)=1$. From the above considerations, one gets about $e^q$, but we expect $q^{1+\varepsilon}$ for any $\varepsilon>0$.
\end{remark}
With these preliminaries, we are ready to state Siegel's theorem.
\begin{theorem}[Siegel] \label{thm:siegel}
	Fix a primitive Dirichlet character $\chi\pmod q$. For any $\varepsilon>0$ with $\varepsilon\le1/2$, there is a(n ineffective) constant $c(\varepsilon)>0$ such that
	\[L(1,\chi)\ge c(\varepsilon)q^{-\varepsilon}.\]
\end{theorem}
\begin{remark}
	From this lower bound, one sees that $L(s,\chi)$ has no zeroes in the region $\left(1-c(\varepsilon)/q^\varepsilon,1\right]$. Namely, we also have a bound on $L'(s,\chi)$.
\end{remark}
\begin{remark}
	One can use this result, combined with the class number formula, to show that there are only finitely many imaginary quadratic fields with class number equal to $1$.
\end{remark}
We now turn to the proof of \Cref{thm:siegel}.
\begin{proof}[Proof of \Cref{thm:siegel}]
	We follow Goldfeld's proof of this result. Recall
	\[\zeta_K(s)\coloneqq\zeta(s)L(s,\chi_1)L(s,\chi_2)L(s,\chi_1\chi_2)\]
	has nonnegative coefficients. Choose $\beta\in[3/4,1)$, which is a surprise tool which will help us later. The trick is to look at
	\[I(\beta)\coloneqq\frac1{2\pi i}\int_{\op{Re}s=2}\zeta_K(s+\beta)\Gamma(s)x^s\,ds.\]
	Notably, as $|t|\to\infty$, we have $|\Gamma(\sigma+it)|\sim e^{-\pi|t|/2}|t|^{\sigma-1/2}$ for bounded $|\sigma|$, so everything is going to converge absolutely. Now, absolute convergence everywhere allows us to exchange the sum and integral to give
	\[I(\beta)=\sum_{n=1}^\infty\frac{a_n}{n^\beta}e^{-n/x},\]
	where $a_n$ are the coefficients of $\zeta_K(s)$. Notably, by positivity, we have the lower bound $I(\beta)\ge1$.

	We now shift the contour of $I$ to $\op{Re}s=1/2-\beta$. 
\end{proof}

\end{document}