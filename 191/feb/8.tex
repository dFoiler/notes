% !TEX root = ../notes.tex

\documentclass[../notes.tex]{subfiles}

\begin{document}

\section{February 8}

We began class by proving the Prime number theorem. I have edited directly into those notes for continuity.
\begin{warn}
	In the class, the professor spread discussion of things about characters throughout this chapter. In order to isolate the elementary parts of the discussion from the analytic ones, I have attempted to collect all the character theory in today's lecture.
\end{warn}

\subsection{Quadratic Residues}
Fix a prime $p$, for simplicity. Let $\chi\pmod p$ be a Dirichlet character. Our goal is to prove an analytic continuation and functional equation for $L(s,\chi)$.
\begin{proposition} \label{prop:fp-cyclic}
	Fix a prime $p$. Then $\FF_p^\times$ is cyclic.
\end{proposition}
\begin{proof}
	We proceed in steps.
	\begin{enumerate}
		\item Given $a,b\in\FF_p^\times$ of orders $k$ and $\ell$, we claim that there is an element $x\in\FF_p^\times$ of order $\lcm(k,\ell)$. Roughly speaking, the idea is that $\gcd(k,\ell)=1$ will imply that $ab$ has order $k\ell$: of course, $(ab)^{k\ell}=1$, and to see that $k\ell$ is the smallest exponent, note $(ab)^n=1$ implies $(ab)^{nk}=1$, so $b^{nk}=1$, so $\ell\mid nk$, so $\ell\mid n$ because $\gcd(k,\ell)=1$. Analogously, $k\mid n$, so $k\ell\mid n$.
		
		To extend the above proof to the case of $\gcd(k,\ell)>1$, we use unique prime factorization. Set
		\[k'\coloneqq\prod_{\nu_p(k)\ge\nu_p(\ell)}p^{\nu_p(k)}\qquad\text{and}\qquad\ell'\coloneqq\prod_{\nu_p(k)<\nu_p(\ell)}p^{\nu_p(\ell)}.\]
		In particular, we see that $\nu_p(k')>0$ if and only if $\nu_p(k)\ge\nu_p(\ell)$, and $\nu_p(\ell')>0$ if and only if $\nu_p(k)<\nu_p(\ell)$. Thus, no prime $p$ divides both $k'$ and $\ell'$, so $\gcd(k',\ell')=1$. Further, by construction, we see $k'\mid k$ and $\ell'\mid\ell$ and
		\[k'\ell'=\prod_{\nu_p(k)\ge\nu_p(\ell)}p^{\nu_p(k)}\cdot\prod_{\nu_p(k)<\nu_p(\ell)}p^{\nu_p(\ell)}=\prod_pp^{\max\{\nu_p(k),\nu_p(\ell)\}}=\lcm(k,\ell).\]
		Thus, we see that $a^{k/k'}$ has order $k'$, and $b^{\ell/\ell'}$ has order $\ell'$, so their product $x\coloneqq a^{k/k'}b^{\ell/\ell'}$ has order $k'\ell'=\gcd(k,\ell)$.

		\item Inductively applying the previous step to every $a\in\FF_p^\times$, we produce an element $g\in\FF_p^\times$ with order $n$ which is the least common multiple of the orders of all $a\in\FF_p^\times$. In particular, the order of $a\in\FF_p^\times$ divides into $n$, so we see that
		\[a^n\equiv1\pmod p.\]
		In particular, the equation $x^n-1=0$ has $p-1$ roots in $\FF_p$ given by the elements of $\FF_p^\times$. However, for a field, the number of roots of a polynomial is bounded by the degree, so $x^n-1=0$ has at most $n$ solutions, so we conclude $n\ge p-1$. Because the order of $g$ must divide $\#\FF_p^\times=p-1$, we conclude that $n\le p-1$ as well, so $n=p-1$ is forced. So $g$ is a generator of $\FF_p^\times$.
		\qedhere
	\end{enumerate}
\end{proof}
We can extend this result as follows.
\begin{proposition} \label{prop:prime-power-cyclic}
	Fix an odd prime $p$. For any $\nu>0$, the group $\left(\ZZ/p^\nu\ZZ\right)^\times$ is cyclic.
\end{proposition}
\begin{proof}
	We induct in steps.
	\begin{itemize}
		\item The case of $\nu=1$ is from \Cref{prop:fp-cyclic}.

		\item The case of $\nu=2$ requires some care. Let $g\in\FF_p^\times$ be a generator from the $\nu=1$ case. If $g\pmod{p^2}$ already has order $p(p-1)$, then we are done. Otherwise, $g\pmod{p^2}$ has order $n$ strictly less than $p(p-1)$. However, note that $g^n\equiv1\pmod{p^2}$ implies
		\[g^n\equiv1\pmod p,\]
		so $p-1\mid n$ because the order of $g\pmod p$ is $p-1$. Thus, $p-1\mid n$ and $k\mid p(p-1)$ but $n<p(p-1)$ forces $n=p-1$, so $g^{p-1}\equiv1\pmod{p^2}$.

		Now, the trick is to consider $g+p$. Note $g+p$ is still a generator of $\FF_p^\times$, so its order is divisible by $(p-1)$ but divides $p(p-1)$ and so equals $p(p-1)$ or $(p-1)$. To see that the order is not $(p-1)$, we note
		\begin{align*}
			(g+p)^{p-1} &= \sum_{k=0}^{p-1}\binom{p-1}kg^{(p-1)-k}p^k \\
			&\equiv g^{p-1}+(p-1)g^{p-2}p \\
			&\equiv 1-g^{p-2}p\pmod{p^2}.
		\end{align*}
		However, $g^{p-2}\not\equiv0\pmod p$, so $(g+p)^{p-1}\not\equiv1\pmod{p-1}$. We conclude that the order of $g+p$ must be $p(p-1)$.

		\item To help the following induction, we note that some $g\in\ZZ$ which is a generator of $(\ZZ/p^2\ZZ)^\times$ will have $g^{p-1}\equiv1\pmod p$ but $g^{p-1}\not\equiv1\pmod{p^2}$, so we can write
		\[g^{p-1}=1+pa\]
		for some integer $a$ not divisible by $p$. Thus, we see
		\begin{align*}
			g^{p(p-1)} &= (1+pa)^p \\
			&= \sum_{k=0}^p\binom pk(pa)^k \\
			&\equiv 1+p\cdot pa+\frac{p(p-1)}2\cdot(pa)^2 \\
			&\equiv 1+p^2a\pmod{p^3}.
		\end{align*}
		Thus, we actually see $g^{p(p-1)}\not\equiv1\pmod{p^3}$. Note that we have used $p\ne2$ in the above computation.

		\item The induction in the remaining cases $\nu\ge2$ is easier. Suppose that we have $g\in\ZZ$ which is a generator $g\in(\ZZ/p^\nu\ZZ)^\times$ with $g^{p^{\nu-1}(p-1)}\not\equiv1\pmod{p^{\nu+1}}$. Then we claim that $g$ is also a generator of $(\ZZ/p^{\nu+1}\ZZ)^\times$ with $g^{p^{\nu}(p-1)}\not\equiv1\pmod{p^{\nu+2}}$. This will complete the proof by induction, where the base case was shown in the previous two steps.

		Well, we note that the order of $g\pmod{p^{\nu+1}}$ must certainly divide $p^\nu(p+1)$, and we want to show equality. For this, we see $g^n\equiv1\pmod{p^{\nu+1}}$ will imply $g^n\equiv1\pmod{p^\nu}$, so $n$ is divisible by $p^{\nu-1}(p-1)$. Thus, the order of $g\pmod{p^{\nu+1}}$ is divisible by $p^{\nu-1}(p-1)$, but the order is not actually equal to $p^{\nu-1}(p-1)$ because
		\[g^{p^{\nu-1}(p-1)}\not\equiv1\pmod{p^{\nu+1}}.\]
		So the order is a divisor of $p^\nu(p+1)$ divisible by but strictly greater than $p^{\nu-1}(p-1)$, so the order must actually be $p^\nu(p+1)$. We conclude that $g\pmod{p^{\nu+1}}$ is a generator.

		To complete the induction, we must show $g^{p^\nu(p-1)}\not\equiv1\pmod{p^{\nu+2}}$. Well, by hypothesis, we may write $g^{p^\nu(p-1)}=1+p^{\nu+1}a$ for some $a$ not divisible by $p$. Then
		\begin{align*}
			g^{p^{\nu+1}(p-1)} &= \left(1+p^{\nu+1}a\right)^p \\
			&= \sum_{k=0}^p\binom kp\left(p^{\nu+1}a\right)^k \\
			&\equiv 1+p^{\nu+1}a\pmod{p^{\nu+2}},
		\end{align*}
		where we don't care about the other terms because $p^{(\nu+1)k}\equiv0\pmod{p^{\nu+2}}$ for $k\ge2$. Because $p\nmid a$, the conclusion follows.
		\qedhere
	\end{itemize}
\end{proof}
\begin{proposition}
	For any $\nu\ge2$, we have $(\ZZ/2^\nu\ZZ)^\times\cong\ZZ/2\ZZ\oplus\ZZ/2^{\nu-2}\ZZ$.
\end{proposition}
\begin{proof}
	We proceed in steps.
	\begin{enumerate}
		\item For any $\nu\ge0$, we claim that
		\[5^{2^\nu}\equiv1+2^{\nu+2}\pmod{2^{\nu+3}}.\]
		We proceed by induction. For $\nu=0$, the statement reads $5\equiv1+4\pmod8$, which is true. Then for the induction, we are given that $5^{2^\nu}=1+(1+2a)2^{\nu+2}$ for some integer $a$ and compute
		\begin{align*}
			5^{2^{\nu+1}} &= \left(1+(1+2a)2^{\nu+2}\right)^2 \\
			&= 1+(1+2a)2^{\nu+3}+(1+2a)^22^{2\nu+4} \\
			&\equiv 1+2^{\nu+3}\pmod{2^{\nu+4}}.
		\end{align*}
		Notably, $2\nu+4\ge\nu+4$, so the rightmost term vanishes in the last equivalence. Anyway, this completes the induction.

		\item For any $\nu\ge0$, we claim that the order of $5\pmod{2^{\nu+2}}$ is $2^{\nu}$. Certainly the order divides $2^\nu$ because
		\[5^{2^\nu}\equiv1\pmod{2^{\nu+2}}\]
		from the previous step. If $\nu=0$, there is nothing else to say. Otherwise, we see the order must exceed $2^{\nu-1}$ because
		\[5^{2^{\nu-1}}\equiv1+2^{\nu+1}\pmod{2^{\nu+2}}\]
		by the previous step again, so we conclude the order actually equals $2^\nu$.

		\item For any $\nu\ge2$, we claim that
		\[\ZZ/2\ZZ\oplus(\ZZ/2^{\nu-2}\ZZ)\cong\langle\pm1\rangle\oplus\langle5\rangle\to\left(\ZZ/2^\nu\ZZ\right)^\times\]
		is an isomorphism, which will complete the proof. Note that the left map is an isomorphism because $-1$ has order $2$, and $5\pmod{2^{\nu}}$ has order $2^{\nu-2}$ by the previous step. As such, it remains to show that the right map given by $(a,b)\mapsto ab$ is an isomorphism.

		To begin, note that the map is a group homomorphism because it is the product map induced by the inclusions $\langle\pm1\rangle\subseteq(\ZZ/2^\nu\ZZ)^\times$ and $\langle5\rangle\subseteq(\ZZ/2^\nu\ZZ)^\times$. Further, we note that our two groups both have size $2\cdot2^{\nu-2}=2^{\nu-1}=\varphi\left(2^\nu\right)$, so it is enough to show that our map is injective to show that we have a bijection. Well, suppose that
		\[(-1)^a\cdot5^b\equiv1\pmod{2^\nu}\]
		for some $(a,b)$; we must show that $(a,b)=(0,0)$. Well, $\nu\ge2$, so we may reduce$\pmod4$ to give us that $(-1)^a\equiv1\pmod4$, so $a\equiv0\pmod2$. We then see $5^b\equiv1\pmod{2^\nu}$, so $b\equiv0\pmod{2^{\nu-2}}$. This completes the proof.
		\qedhere
	\end{enumerate}
\end{proof}
Anyway, let's start talking about quadratic residues.
\begin{corollary}
	Fix a prime $p$ and some $d\in\ZZ^+$.
	\begin{listalph}
		\item The function $\mu_d\colon\FF_p^\times\to\FF_p^\times$ given by $\mu_d\colon x\mapsto x^d$ is a homomorphism.
		\item If $\gcd(d,p-1)=1$, then $\mu_d$ is an isomorphism.
		\item If $d\mid p-1$, then each $a\in\FF_p^\times$ makes $x^d\equiv a\pmod p$ have either $0$ or $d$ solutions.
	\end{listalph}
\end{corollary}
\begin{proof}
	Here we go.
	\begin{listalph}
		\item This holds because $\FF_p^\times$ is abelian: note $\mu_d(xy)=(xy)^d=x^dy^d=\mu_d(x)\mu_d(y)$.
		\item Because $\gcd(d,p-1)=1$, we can find $k\pmod{p-1}$ such that $dk\equiv1\pmod{p-1}$. It follows that
		\[\mu_k(\mu_d(x))=x^{dk}=x\qquad\text{and}\qquad\mu_d(\mu_k(x))=x^{kd}=x\]
		for each $x\in\FF_p^\times$, where we are using the fact that the order of $x$ divides $p-1$. Thus, $\mu_k$ provides the inverse homomorphism for $\mu_d$, which shows that $\mu_d$ is an isomorphism.
		\item If $x^d\equiv a\pmod p$ has no solutions, then there is nothing to say.
		
		Otherwise, fix a generator $g\in\FF_p^\times$ by \Cref{prop:fp-cyclic}, and suppose that $g^\ell\in\FF_p^\times$ is a solution to $x^d\equiv a\pmod p$ so that $a=g^{d\ell}$. Then we note some $x=g^k$ is a solution to $x^d=a$ if and only if
		\[g^{dk}=x^d=a=g^{d\ell},\]
		which is equivalent to $p-1\mid(dk-d\ell)$. Because $d\mid p-1$, this is equivalent to $\frac{p-1}d\mid (k-\ell)$, or $\ell\equiv k\pmod{\frac{p-1}d}$. As $\ell$ varies through $\ZZ/(p-1)\ZZ$, we see that there are exactly $(p-1)/d$ total options present for $\ell$.
		\qedhere
	\end{listalph}
\end{proof}
This motivates the Legendre symbol.
\begin{definition}[quadratic residue]
	Fix an odd prime $p$ and some $a\in\ZZ$ not divisible by $p$.
	\begin{itemize}
		\item If $x^2\equiv a\pmod p$ has a solution, then $a$ is a \textit{quadratic residue}.
		\item If $x^2\equiv a\pmod p$ does not have a solution, then $a$ is a \textit{nonquadratic residue}.
	\end{itemize}
	We will be silent about the case of $p\mid a$.
\end{definition}
\begin{remark} \label{rem:qr-means-even}
	Suppose $p$ is an odd prime. Given $a\in\FF_p^\times$, write $a=g^k$, where $g\in\FF_p^\times$ is a generator.
	\begin{itemize}
		\item If $k$ is even, then note $a$ is a quadratic residue because $a\equiv\left(g^{k/2}\right)^2\pmod p$.
		\item Conversely, if $a$ is a quadratic residue, then $k$ is even. Indeed, if we can write $a\equiv x^2\pmod p$, then we see $p\nmid a$ enforces $p\nmid x$, so writing $x=g^\ell$ for some integer $\ell$, we must have
		\[g^k=a=x^2=g^{2\ell}.\]
		Rearranging, we have $k-2\ell\equiv0\pmod{p-1}$, but $p-1$ is even, so this forces $k$ to be even.
	\end{itemize}
\end{remark}
\begin{definition}[Legendre symbol]
	Fix an odd prime $p$ and some $a\in\ZZ$. Then we define the \textit{Legendre symbol} by
	\[\left(\frac ap\right)\coloneqq\begin{cases}
		\phantom+0 & \text{if }p\mid a, \\
		\phantom+1 & \text{if }a\text{ is a quadratic residue}, \\
		-1 & \text{if }a\text{ is a nonquadratic residue}.
	\end{cases}\]
\end{definition}
Here is a quick way to evaluate Legendre symbols.
\begin{proposition}[Euler's criterion] \label{prop:euler-criterion}
	Fix an odd prime $p$. For any $a\in\ZZ$, we have
	\[\left(\frac ap\right)\equiv a^{(p-1)/2}\pmod p.\]
\end{proposition}
\begin{proof}
	We proceed in cases.
	\begin{itemize}
		\item If $p\mid a$, then we see $0\equiv 0^{(p-1)/2}\pmod p$.
		\item If $a\pmod p$ is a quadratic residue, then we can write $a\equiv b^2\pmod p$ for some $b\pmod p$. Note $p\nmid a$ forces $p\nmid b$, so we can compute
		\[a^{(p-1)/2}=b^{p-1}\equiv1=\left(\frac ap\right)\pmod p,\]
		as desired.
		\item If $a\pmod p$ is a nonquadratic residue, then we pick up a generator $g\in\FF_p^\times$ from \Cref{prop:fp-cyclic}. As such, we can write $a=g^k$ for some integer $k$; note that $k$ is odd by \Cref{rem:qr-means-even}. As such, we compute
		\[a^{(p-1)/2}\equiv g^{k(p-1)/2}=\left(g^{(p-1)/2}\right)^k\equiv(-1)^k=-1\pmod p.\]
		Notably, $g^{(p-1)/2}\equiv-1\pmod p$ because $g^{(p-1)/2}$ cannot be $1\pmod p$ (because the order of $g$ is $p-1$), but $g^{(p-1)/2}$ must square to $1\pmod p$, which forces$g^{(p-1)/2}\equiv-1\pmod p$.
		\qedhere
	\end{itemize}
\end{proof}
\begin{remark}
	Requiring $p\ne2$ might look concerning, but every residue in $\FF_2^\times=\{1\}$ is a square anyway, so the analysis here is somewhat trivial.
\end{remark}
\begin{corollary}
	Fix an odd prime $p$. Then $\left(\frac{-1}p\right)=1$ if $p\equiv1\pmod 4$, and $\left(\frac{-1}p\right)=-1$ if $p\equiv-1\pmod 4$.
\end{corollary}
\begin{proof}
	If $p\equiv1\pmod4$, we write $p=1+4k$ and note
	\[\left(\frac{-1}p\right)\equiv(-1)^{(p-1)/2}=(-1)^{2k}=1\pmod p,\]
	so $p>2$ forces $\left(\frac{-1}p\right)=1$. Similarly, if $p\equiv-1\pmod4$, we write $p=-1+4k$ and note
	\[\left(\frac{-1}p\right)\equiv(-1)^{(p-1)/2}=(-1)^{-1+2k}=-1\pmod p,\]
	so $p>2$ forces $\left(\frac{-1}p\right)=-1$. This is what we wanted.
\end{proof}
In our discussion of $L$-functions, the following result explains why we care about Legendre symbols.
\begin{proposition} \label{prop:legendre-is-character}
	Fix a prime $p$. Then the Legendre symbol $\left(\frac\bullet p\right)$ is the unique non-principal real Dirichlet character$\pmod p$.
\end{proposition}
\begin{proof}
	Fix a real Dirichlet character $\chi\pmod p$. In particular, $\chi$ arises from a character $\chi\colon\FF_p^\times\to\RR^\times$, but by \Cref{rem:chars-to-s1}, we see that $\chi$ must output to $S^1\cap\RR^\times=\{\pm1\}$. Fixing a generator $g\in\FF_p^\times$ by \Cref{prop:fp-cyclic}, we have two cases.
	\begin{itemize}
		\item Suppose $\chi(g)=1$. Then for any $g^k\in\FF_p^\times$, we see $\chi\left(g^k\right)=\chi(g)^k=1$. Thus, $\chi\pmod p$ is the principal character.
		\item Suppose $\chi(g)=-1$. Then for any $g^k\in\FF_p^\times$, we see
		\[\chi\left(g^k\right)=\chi(g)=(-1)^k.\]
		Now, comparing \Cref{rem:qr-means-even} with the definition of the Legendre symbol, we see that $\chi(a)=\left(\frac ap\right)$ for each $a\in\FF_p^\times$ because, upon writing $g=a^k$ for some integer $k$, both are $1$ when $k$ is even, and both are $-1$ when $k$ is odd. Lastly, both $\chi$ and $\left(\frac\bullet p\right)$ vanish on multiples of $p$, so we conclude that $\chi=\left(\frac\bullet p\right)$.
	\end{itemize}
	The above classification of real Dirichlet characters$\pmod p$ completes the proof.
\end{proof}
\begin{remark}
	More generally studying when $f(x)\equiv0\pmod p$ has solutions (and how it factors) has connections directly with the Langlands program and similar. We will not say more because this is (very) far outside the scope of the course.
\end{remark}

\subsection{Primitive Characters}
Motivated by \Cref{prop:legendre-is-character}, we introduce primitive Dirichlet characters. This requires discussing conductors.
\begin{definition}[conductor]
	Fix a Dirichlet character $\chi\pmod q$. The \textit{conductor} $f(\chi)$ of $\chi$ is the smallest positive integer $f$ such that
	\[\chi(a)=\chi(b)\]
	for any $a\equiv b\pmod f$ such that $\gcd(a,q)=\gcd(b,q)=1$.
\end{definition}
Here is a basic check.
\begin{lemma} \label{lem:conductor-divides-modulus}
	Fix a Dirichlet character $\chi\pmod q$. Then $f(\chi)\mid q$.
\end{lemma}
\begin{proof}
	Set $f\coloneqq\gcd(f(\chi),q)$. We claim any $a\equiv b\pmod f$ with $\gcd(a,q)=\gcd(b,q)=1$ will have $\chi(a)=\chi(b)$. This will finish the proof because it will force $f\ge f(\chi)$, but $f\mid f(\chi)$ enforces $f=f(\chi)$.
	
	To show the claim, write $f=xf(\chi)+yq$ for some integers $x,y\in\ZZ$. Then note $a=b+fk$ for some integer $k$, which implies
	\[\chi(a)=\chi(b+fk)=\chi(b+xf(\chi)k+yqk)=\chi(b+xf(\chi)).\]
	Now, $\chi(a)\ne0$, so $\chi(b+xf(\chi))\ne0$, so $b+xf(\chi)$ must be coprime with $q$. Thus, by definition of $f(\chi)$, we conclude that $\chi(a)=\chi(b+xf(\chi))=\chi(b)$, which is what we wanted.
\end{proof}
\begin{definition}[primitive]
	A Dirichlet character $\chi\pmod q$ is \textit{primitive} if and only if $f(\chi)=q$. In other words, $q$ is the smallest integer $f$ such that $a\equiv b\pmod f$ implies $\chi(a)=\chi(b)$ for $\gcd(a,q)=\gcd(b,q)=1$.
\end{definition}
We are going to work almost exclusively with primitive characters in the sequel; let's justify why.
\begin{proposition} \label{prop:all-chars-induced-from-primitive}
	Fix a Dirichlet character $\chi\pmod q$ of conductor $f$. Then there is a unique Dirichlet character $\chi_f\pmod f$ such that
	\[\chi(n)=\begin{cases}
		\chi_f(n) & \text{if }\gcd(n,q)=1, \\
		0 & \text{if }\gcd(n,q)>1.
	\end{cases}\]
	In fact, $\chi_f$ is primitive.
\end{proposition}
\begin{proof}
	We show uniqueness, existence, and primitivity separately.
	\begin{itemize}
		\item We show that $\chi_f$ is unique. For each $n\in\ZZ$ such that $\gcd(n,f)=1$, we claim that there is some $n'$ such that $\gcd(n',q)=1$ but $n\equiv n'\pmod f$. This is surprisingly technical and arises from the fact $f\mid q$ from \Cref{lem:conductor-divides-modulus}. Well, the Chinese remainder theorem allows us to find some $n'\pmod q$ such that
		\[n'\equiv\begin{cases}
			n \pmod{p^{\nu_p(q)}} & \text{if }p\mid f, \\
			1 \pmod{p^{\nu_p(q)}} & \text{if }p\mid q\text{ and }p\mid f.
		\end{cases}\]
		Notably, the moduli are coprime, and their product is $q$. Further, for each $p\mid q$, we claim $p\nmid n'$: if $p\mid f$, then $p\mid n'-n$, so $p\nmid n$ implies $p\nmid n'$; otherwise if $p\nmid f$, so $p\mid n'-1$, so $p\nmid n'$. However, for each $p\mid f$, we see that
		\[n'\equiv n\pmod{p^{\nu_p(f)}}\]
		because $\nu_p(f)\le\nu_p(q)$, so we conclude that $n'\equiv n\pmod f$.

		To complete the proof, we note that each $n\in\ZZ$ such that $\gcd(n,f)=1$ has some $n'$ such that $\gcd(n',q)=1$ while $n\equiv n'\pmod f$, so
		\[\chi_f(n)=\chi_f(n')=\chi(n').\]
		Otherwise, for $n\in\ZZ$ such that $\gcd(n,f)>1$, we must have $\chi_f(n)=0$ because $\chi_f\pmod f$ is a Dirichlet character. So $\chi_f$ is uniquely determined by $\chi$.

		\item We show that $\chi_f$ exists. Well, for each $n\in\ZZ$ such that $\gcd(n,f)=1$, we note from the previous step that there is some $n'$ such that $\gcd(n',q)=1$ and $n\equiv n'\pmod f$. But by the definition of $f$, we note that the value of $\chi(n')$ is uniquely determined, so we may define
		\[\chi_f(n)\coloneqq\begin{cases}
			\chi(n) & \text{if }\gcd(n,q)=\gcd(n',f)=1\text{ and }n\equiv n'\pmod f, \\
			0 & \text{if }\gcd(n,f)>1.
		\end{cases}\]
		Quickly, note that $\chi_f\pmod f$ is a Dirichlet character: certainly $\chi_f$ vanishes on $n$ such that $\gcd(n,f)>1$. Further, given $a,b\in\ZZ$ such that $\gcd(a,f)=\gcd(b,f)=1$, we find $a'$ and $b'$ such that $a'\equiv a\pmod f$ and $b'\equiv b\pmod f$ and $\gcd(a',q)=\gcd(b',q)=1$. Thus, we see
		\[\chi_f(a)\chi_f(b)=\chi(a')\chi(b')=\chi(a'b')=\chi_f(ab),\]
		where the last equality is valid because $a'b'\equiv ab\pmod f$ and $\gcd(a'b',q)=1$ because $\gcd(a',q)=\gcd(b',q)=1$.

		Lastly, we check that $\chi$ is built from $\chi_f$ as claimed. Well, for $n\in\ZZ$, if $\gcd(n,q)>1$, then of course $\chi(n)=0$. Alternatively, if $\gcd(n,q)=1$, then $\gcd(n,f)=1$ as well, so the construction of $\chi_f$ grants $\chi(n)=\chi_f(n)$.

		\item We show that $\chi_f\pmod f$ is a primitive Dirichlet character, using the construction of the previous step. Indeed, let $f'$ be the conductor of $\chi_f$; we want to show $f=f'$. Certainly $f'\mid f$ by \Cref{lem:conductor-divides-modulus}, so $f'\le f$. On the other hand, if $a\equiv b\pmod{f'}$ and $\gcd(a,q)=\gcd(b,q)=1$, then we see
		\[\chi(a)=\chi_f(a)\stackrel*=\chi_f(b)=\chi(b),\]
		where $\stackrel*=$ holds because $\gcd(a,f)=\gcd(b,f)=1$ as well by \Cref{lem:conductor-divides-modulus}. Thus, because $f$ is the conductor of $\chi$, we see $f'\ge f$, so we conclude $f=f'$.
		\qedhere
	\end{itemize}
\end{proof}

\subsection{Gauss Sums}
We mentioned in \Cref{rem:gamma-is-gauss-sum} that $\Gamma$ is more or less a continuous version of a Gauss sum: it's some kind of multiplicative Fourier transform of an additive character. Well, here are the usual Gauss sums.
\begin{definition}[Gauss sum]
	Fix a Dirichlet character $\chi\pmod q$. Then the \textit{Gauss sum} is
	\[\tau(\chi,m)\coloneqq\sum_{n=0}^{q-1}e\left(\frac{nm}q\right)\chi(n).\]
	For brevity, we set $\tau(\chi)\coloneqq\tau(\chi,1)$.
\end{definition}
Namely, $\psi_m\colon n\mapsto e\left(\frac{nm}p\right)$ is our additive character, our measure is the counting measure, so we are indeed just looking at the multiplicative Fourier transform of an additive character.

Let's show a few basic facts. To set up our discussion, we emphasize that primitive characters are better-behaved.
\begin{lemma} \label{lem:primitive-facts}
	Fix a Dirichlet character $\chi\pmod q$. The following are equivalent.
	\begin{listalph}
		\item $\chi$ is primitive.
		\item For each proper divisor $q'\mid q$, there exists $k\equiv1\pmod{q'}$ such that $\chi(k)\notin\{0,1\}$.
		\item For each proper divisor $q'\mid q$ and integer $r$, we have
		\[\sum_{k=0}^{q/q'-1}\chi(kq'+r)=0.\]
	\end{listalph}
\end{lemma}
\begin{proof}
	We show our implications in sequence.
	\begin{listalph}
		\item We show (a) implies (b). Because $\chi$ is primitive, we know that the nonzero values of $\chi$ are not periodic$\pmod{q'}$. In particular, we may find $r\equiv s\pmod{q'}$ such that $\chi(r)$ and $\chi(s)$ are distinct and nonzero. In particular, we must have $r,s\in(\ZZ/q\ZZ)^\times$, so we let $s'\in\ZZ$ denote a multiplicative inverse of $s\pmod q$ so that $rs'\equiv ss'\equiv1\pmod{q'}$, but
		\[\chi(rs')=\chi(r)\chi(s')=\chi(r)/\chi(s)=1.\]
		This is what we wanted.
		\item We show (b) implies (c). Well, fix an integer $\ell\equiv1\pmod{q'}$ such that $\chi(\ell)\notin\{0,1\}$. Writing $\ell=1+q'\ell'$ and $d\coloneqq q/q'$, we see that
		\begin{align*}
			\chi(\ell)\sum_{k=0}^{q/q'-1}\chi(kq'+r) &= \sum_{k=0}^{d-1}\chi(\ell kq'+\ell r) \\
			&= \sum_{k=0}^{d-1}\chi(\ell kq'+(1+q'\ell')r) \\
			&= \sum_{k=0}^{d-1}\chi((\ell k+\ell')q'+r).
		\end{align*}
		Now, we claim that $k\mapsto\ell k+\ell'$ is a bijection $\ZZ/d\ZZ\to\ZZ/d\ZZ$. Indeed, it is enough to show injectivity because this is a map from a finite set to itself, so we see that $\ell k_1+\ell'\equiv\ell k_2+\ell'\pmod d$ implies
		\[\ell k_1\equiv\ell k_2\pmod d,\]
		which implies $k_1\equiv k_2\pmod d$ because $\gcd(\ell,d)=\gcd(\ell,q)=1$. This completes the proof of the claim, so we can re-index our sum as
		\[\chi(\ell)\sum_{k=0}^{q/q'-1}\chi(kq'+r)=\sum_{k=0}^{q/q'-1}\chi(kq'+r)\]
		because the value of $k$ in the sum only matters$\pmod d$. (Recall $d=q/q'$, and $\chi$ is periodic$\pmod q$.) Because $\chi(\ell)\ne1$, we conclude that the value of the sum must be $0$.
		\item We show (c) implies (a) by contraposition. Indeed, if $\chi$ is not primitive, then \Cref{prop:all-chars-induced-from-primitive} grants us a Dirichlet character $\chi_f\pmod f$ where $f\coloneqq f(\chi)$ such that
		\[\chi(n)=\begin{cases}
			\chi_f(n) & \text{if }\gcd(n,q)=1, \\
			0 & \text{if }\gcd(n,q)>1.
		\end{cases}\]
		Now, by \Cref{lem:conductor-divides-modulus}, we see $f\mid q$, and because $\chi$ fails to be primitive, we have $f<q$, so $f$ is a proper divisor. As such, we see that
		\[\sum_{k=0}^{q/f-1}\chi(kf+1)=\sum_{k=0}^{q/f-1}1_{\gcd(kf+1,q)=1}.\]
		This sum has nonnegative terms and at least one positive term at $k=0$, so the total sum is at least $1$ and hence is nonzero.
		\qedhere
	\end{listalph}
\end{proof}
We can now relate our Gauss sums together.
\begin{lemma} \label{lem:relate-gauss-sums}
	Fix a primitive Dirichlet character $\chi\pmod q$. Then for any integer $m$, we have
	\[\tau(\chi,m)=\overline\chi(m)\tau(\chi).\]
\end{lemma}
\begin{proof}
	We have the following cases.
	\begin{itemize}
		\item Suppose that $\gcd(m,q)=1$. Here, the argument is a change of variables. Indeed, we note $\chi(m)\ne0$, so we may write
		\begin{align*}
			\tau(\chi,m) &= \sum_{n=0}^{q-1}e\left(\frac{nm}q\right)\chi(n) \\
			&= \frac1{\chi(m)}\sum_{n=0}^{q-1}e\left(\frac{nm}q\right)\chi(nm) \\
			&= \frac1{\chi(m)}\sum_{n=0}^{q-1}e\left(\frac{n}q\right)\chi(n),
		\end{align*}
		where we have re-indexed our sum in the last step; in particular, we see that multiplication by $m\in(\ZZ/q\ZZ)^\times$ is a bijection $(\ZZ/q\ZZ)^\times\to(\ZZ/q\ZZ)^\times$. In total, we note $\chi(m)^{-1}=\overline{\chi(m)}$ because $|\chi(m)|=1$ by \Cref{rem:chars-to-s1}, so we conclude that $\tau(\chi,m)=\overline{\chi(m)}\tau(\chi)$, as desired.
		% \item Suppose that $q\mid m$. In this case, we see
		% \[\tau(\chi,m)=\sum_{n=0}^{q-1}e\left(\frac{nm}q\right)\chi(n)=\sum_{n=0}^{q-1}\chi(n)=0,\]
		% where the last equality is by \Cref{prop:ortho-relations}. But this is also $\overline\chi(m)\tau(\chi)=0\tau(\chi)$.
		\item Suppose that $\gcd(m,q)>1$. This is harder and requires using that $\chi\pmod q$ is primitive. Quickly, note that $\overline\chi(m)\tau(\chi)=0$ because $\gcd(m,q)>1$, so we must show that $\tau(\chi,m)=0$.

		Well, take $d\coloneqq\gcd(m,q)$ and $m'\coloneqq m/d$ and $q'\coloneqq q/d$ so that $\gcd(m',q')=1$. As such, we write each $n\in\ZZ/q\ZZ$ as $kq'+r$ by the division algorithm, which gives
		\begin{align*}
			\tau(\chi,m) &= \sum_{n=0}^{q-1}e\left(\frac{nm}q\right)\chi(n) \\
			&= \sum_{k=0}^{d-1}\Bigg(\sum_{r=0}^{q'-1}e\left(\frac{(kq'+r)m'}{q'}\right)\chi(kq'+r)\Bigg) \\
			&= \sum_{r=0}^{q'-1}\Bigg(e\left(\frac{rm'}{q'}\right)\sum_{k=0}^{d-1}\chi(kq'+r)\Bigg).
		\end{align*}
		We now note that the inner sums vanish by \Cref{lem:primitive-facts} because $q'$ is a proper divisor of $q$. Note that we have used the fact that $\chi$ is primitive here.
		\qedhere
	\end{itemize}
\end{proof}
\begin{proposition} \label{prop:mag-of-gauss-sum}
	Fix a primitive Dirichlet character $\chi\pmod q$. Then $|\tau(\chi)|^2=q$.
\end{proposition}
\begin{proof}
	This is essentially the Plancherel formula. Using \Cref{lem:relate-gauss-sums}, we note $|\chi(m)|=1$ if $\gcd(m,q)=1$ (by \Cref{rem:chars-to-s1}) and $|\chi(m)|=0$ otherwise, so
	\begin{align*}
		\varphi(q)|\tau(\chi)|^2 &= \sum_{m=0}^{q-1}|\overline\chi(m)\tau(\chi)|^2 \\
		&= \sum_{m=0}^{q-1}|\tau(\chi,m)|^2 \\
		&= \sum_{m=0}^{q-1}\tau(\chi,m)\overline{\tau(\chi,m)} \\
		&= \sum_{m=0}^{q-1}\Bigg(\sum_{k,\ell=0}^{q-1}e\left(\frac{km}q\right)\chi(k)\overline{e\left(\frac{\ell m}q\right)\chi(\ell)}\Bigg) \\
		&= \sum_{k,\ell=0}^{q-1}\Bigg(\chi(k)\overline\chi(\ell)\sum_{m=0}^{q-1}e\left(\frac{(k-\ell)m}q\right)\Bigg).
	\end{align*}
	Now, if $k\ne\ell$, then the inner sum is
	\[\sum_{m=0}^{q-1}e\left(\frac{(k-\ell)m}q\right)=\frac{e\left(\frac{(k-\ell)q}q\right)-1}{e\left(\frac{k-\ell}q\right)-1}=0,\]
	so we only care about the terms with $k=\ell$, where the inner sum is $q$. Thus,
	\[\varphi(q)|\tau(\chi)|^2=\sum_{k=0}^{q-1}\underbrace{\chi(k)\overline{\chi(k)}}_{|\chi(k)|^2}q=\varphi(q)q,\]
	which yields $|\tau(\chi)|^2=q$. This is what we wanted.
\end{proof}

\end{document}