\documentclass{amsart}
\usepackage[utf8]{inputenc}

\newcommand{\nirpdftitle}{191 Term Paper}
\usepackage{import}
\inputfrom{.}{pre}
\usepackage[backend=biber,
    style=alphabetic,
    sorting=ynt
]{biblatex}
\addbibresource{../bib.bib}
\addtolength{\headheight}{12.0pt}

\pagestyle{contentpage}

\title{Hecke's Converse Theorem}
\author{Nir Elber}
\date{5 May 2023}
\lhead{} \chead{} \rhead{\textit{HECKE'S CONVERSE THEOREM}}

\begin{document}

\begin{abstract}
	\noindent We introduce the basics of modular forms in order to state and prove Hecke's converse theorem \cite{hecke-converse}. Throughout, we will give commentary at a high level, but all proofs will assume little background beyond complex analysis and Fourier analysis.
\end{abstract}

\maketitle

\setcounter{tocdepth}{2}
% \tableofcontents

\section{Introduction}
The goal of this paper is to introduce modular forms in order to state and prove Hecke's converse theorem \cite{hecke-converse}. Roughly speaking, a modular form is some highly symmetric holomorphic function $f$. For example, one can express $f$ via a Fourier expansion
\[f(z)=\sum_{n=0}^\infty a_ne^{2\pi inz}.\]
For now, we will assume that $a_0=0$. One can attach to $f$ to the $L$-function
\[L(s,f)\coloneqq\sum_{n=1}^\infty\frac{a_n}{n^s}.\]
The symmetries enforced on $f$ turn out to rather formally imply that the ``completed $L$-function'' $\Lambda(s,f)\coloneqq(2\pi)^{-s}\Gamma(s)L(s,f)$ satisfies a functional equation
\begin{equation}
	\Lambda(s,f)=(-1)^{k/s}\Lambda(k-s,f) \label{eq:l-func-eq}
\end{equation}
for some integer $k$ depending on $f$. Approximately, the main ingredient in this result is the computation that the Mellin transform of $f$ equals $\Lambda(s,f)$. As such, it is not too surprising that applying Mellin inversion allows us to go the other way.
\begin{theorem}[Hecke]
	Suppose that the function $L(s,f)\coloneqq\sum_{n=1}^\infty\frac{a_n}{n^s}$ satisfies the functional equation \eqref{eq:l-func-eq}. Then, under suitable growth conditions, the function $f(s)\coloneqq\sum_{n=1}^\infty a_ne^{2\pi inz}$ is a modular form.
\end{theorem}
This is our converse theorem. We will state and prove this result more formally later in \Cref{thm:hecke}.
% but the idea is that one can turn desirable properties about a holomorphic function (namely, modularity) into desirable properties about an $L$-function (namely, a functional equation), and vice versa. One can take this analogy further: for example, one can even encode $L$ having an Euler product into $f$ being some kind of eigenvector \cite[Proposition~1.5.1]{bump-forms}.
For motivation, note that in number theory, modularity theorems (taking data and attaching to it a modular form) are becoming increasingly important because they allow one to move questions about objects of interest into questions about modular forms. As such, having tools like converse theorems like the one above are useful: indeed, the point of a converse theorem is to take desirable properties of an $L$-function and produce form. For example, Weil's converse theorem \cite[Theorem~1.5.1]{bump-forms} merely takes twists of the functional equation to produce a modular form, and there are also generalizations to $\op{GL}_n$.

\section{Modular Forms}
In this section, we define and proof some basic properties of modular forms.

\subsection{Congruence Subgroups}
We said modular forms are highly symmetric functions, so we ought to discuss the symmetries we want to obey.
\begin{definition}[congruence subgroup]
	Fix a positive integer $N$. Then we define the subgroup
	\[\Gamma(N)\coloneqq\left\{\begin{bmatrix}
		a & b \\
		c & d
	\end{bmatrix}\in\op{SL}_2(\ZZ):\begin{bmatrix}
		a & b \\
		c & d
	\end{bmatrix}\equiv\begin{bmatrix}
		1 & 0 \\
		0 & 1
	\end{bmatrix}\pmod N\right\}.\]
	Then a subgroup $\Gamma\subseteq\op{SL}_2(\ZZ)$ is a \textit{congruence subgroup} if and only if $\Gamma$ contains $\Gamma(N)$ for some positive integer $N$; the least $N$ such that $\Gamma\supseteq\Gamma(N)$ is the \textit{level} of $\Gamma$.
\end{definition}
Note that the $\Gamma(N)$ are all normal subgroups of $\op{SL}_2(\ZZ)$ because $\Gamma(N)$ is the kernel of the reduction map $\op{SL}_2(\ZZ)\to\op{SL}_2(\ZZ/N\ZZ)$. In this article, our focus will be on level $1$, which means $\Gamma=\Gamma(1)=\op{SL}_2(\ZZ)$, but we will do what we can in higher level. Nonetheless, because we will primarily focus on level $1$, the following result will alleviate headaches later.
\begin{proposition} \label{prop:gens-for-sl2z}
	The group $\op{SL}_2(\ZZ)$ is generated by the matrices
	\[S\coloneqq\begin{bmatrix}
		0 & -1 \\
		1 & 0
	\end{bmatrix}\qquad\text{and}\qquad T\coloneqq\begin{bmatrix}
		1 & 1 \\
		0 & 1
	\end{bmatrix}.\]
\end{proposition}
\begin{proof}
	This is essentially the Euclidean algorithm. We must show that
	\[\gamma\coloneqq\begin{bmatrix}
		a & b \\
		c & d
	\end{bmatrix}\]
	can be written as a product of $S$s and $T$s for any $\gamma\in\op{SL}_2(\ZZ)$. By multiplying $\gamma$ by $S^2=-I$, we may assume that $a\ge0$. We now do the two computations
	\[S\gamma=\begin{bmatrix}
		-c & -b \\
		a & d
	\end{bmatrix}\qquad\text{and}\qquad T^n\gamma=\begin{bmatrix}
		a+nc & b+nd \\
		c & d
	\end{bmatrix}.\]
	Thus, multiplying $\gamma$ by $S$ on the left lets us also require $a=|a|\ge|c|$, and using $T$ to apply the division algorithm to $\gamma$ lets us assume that $0\le a<|c|$ or $c=0$. Namely, repeatedly multiplying $\gamma$ by $S$ and $T$ on the left (applying the Euclidean algorithm) lets us enforce $c=0$. But then $ad=\det\gamma=1$ and $a\ge0$ enforces $a=d=1$. However, with $a=d=1$ and $c=0$, we conclude that $\gamma=T^b$, which finishes the proof.
\end{proof}

Having described our groups, we next describe the space on which they act.
\begin{definition}
	The group $\op{GL}_2(\RR)$ acts on $\PP^1_\CC\coloneqq\CC\cup\{\infty\}$ by the multiplication
	\[\begin{bmatrix}
		a & b \\
		c & d
	\end{bmatrix}z\coloneqq\frac{az+b}{cz+d}\qquad\text{and}\qquad\begin{bmatrix}
		a & b \\
		c & d
	\end{bmatrix}\infty\coloneqq\frac bd.\]
	Namely, if the denominator is zero, we say that the output is $\infty$. In other words, we have $\op{GL}_2(\RR)$ act on $\CC^2\setminus\{(0,0)\}$ by matrix-vector multiplication and then project this action along $\CC^2\setminus\{(0,0)\}\to\PP^1_\CC$ by $(z,w)\mapsto z/w$ (and $(z,0)\mapsto\infty$).
\end{definition}

We are going to want to restrict our action to $\mathbb H\coloneqq\{z\in\CC:\op{Im}z>0\}$, so we compute
\[\op{Im}\left(\begin{bmatrix}
	a & b \\
	c & d
\end{bmatrix}z\right)=\op{Im}\left(\frac{az+b}{cz+d}\right)=\frac{(ad-bc)\op{Im}z}{|cz+d|^2}.\]
Thus, for any $\gamma\in\op{SL}_2(\ZZ)$ so that $\det\gamma=1$, we see that $z\in\mathbb H\cup\PP^1_\QQ$ implies that $\gamma z\in\mathbb H\cup\PP^1_\QQ$. In total, we have defined an action of $\op{SL}_2(\ZZ)$ on $\mathbb H\cup\PP^1_\QQ\subseteq\PP^1_\CC$. For brevity, we define $\mathbb H^*\coloneqq\mathbb H\cup\PP^1_\QQ$. Elements of $\PP^1_\QQ$ are called ``cusps.''
\begin{remark}
	Notably, the orbit of $\infty$ under the action of $\op{SL}_2(\ZZ)$ is all of $\PP^1_\QQ$, which explains why we are forcing ourselves to include it. It turns out that $\mathbb H^*$ has the structure of a complex manifold.
\end{remark}

% \section{Modular Forms}

\subsection{The Definition}
We are now ready to define a modular form as advertized: it is a highly symmetric holomorphic function.
\begin{definition}[modular form, cusp form]
	Fix an integer $k$ and congruence subgroup $\Gamma\subseteq\op{SL}_2(\ZZ)$. A \textit{meromorphic modular form of weight $k$ and level $\Gamma$} is a meromorphic function $f\colon\mathbb H^*\to\CC$ such that
	\begin{equation}
		f\left(\begin{bmatrix}
			a & b \\
			c & d
		\end{bmatrix}z\right)=(cz+d)^kf(z)\qquad\text{for all}\qquad\begin{bmatrix}
			a & b \\
			c & d
		\end{bmatrix}\in\Gamma. \label{eq:symmetry-condition}
	\end{equation}
	If $f$ is holomorphic on $\mathbb H^*$, then $f$ is a \textit{modular form of weight $k$ and level $\Gamma$}. In other words, we need $f$ to be holomorphic on $\mathbb H$ as well as ``holomorphic at each cusp $c\in\PP^1_\QQ$'' in that $|f(z)|$ is bounded as $z\to c$; namely, $z\to\infty$ means that $\op{Im}z\to\infty$. If $f$ vanishes at each cusp, then $f$ is a \textit{cusp form}.
\end{definition}
\begin{remark}
	Morally, one would define being holomorphic on $\mathbb H^*$ by providing a complex manifold structure on $\mathbb H^*$ so that we may talk about holomorphic maps $\mathbb H^*\to\CC$. To reconcile this with the above, note boundedness at cusps means that cusps are removable singularities.
\end{remark}
% It is not totally clear what is meant by ``holomorphic on $\mathbb H^*$'' because of the addition of the mysterious points $\PP^1_\QQ$. The correct formulation of this condition is to regard $\mathbb H^*$ as a complex manifold so that we are asking for $f$ to be a holomorphic map of complex manifolds. More concretely, we may ask for $f$ to be holomorphic on the open subset $\mathbb H\subseteq\CC$ and to be ``holomorphic on the cusps'' in the sense that $|f(z)|$ is bounded as $z\to c$ for any $c\in\PP^1_\QQ$; here, $z\to\infty$ means that $\op{Im}z\to\infty$. We might write $z\to i\infty$.
% To see that this boundedness is the same notion as being holomorphic as a complex manifold, we should place some complex manifold structure on an open neighborhood of $\infty$ (or any of the other elements of $\PP^1_\QQ$) and check that boundedness is the same thing as 
The chief mystery in our definition of a modular form is the presence of the factor $(cz+d)^k$. As a first explanation, there are simply not many modular forms of weight $0$: we will not prove it, but all these modular forms are constant functions. One might complain that it is not clear that there are modular forms of higher weights, but we assure the reader that such modular forms exist.
\begin{example}
	The ``Eisenstein series''
	\[G_k(z)\coloneqq\sum_{\substack{(m,n)\in\ZZ^2\\(m,n)\ne(0,0)}}\frac1{(mz+n)^k}\]
	converges uniformly for $k\ge4$ and produces a nonzero modular form of weight $k$ and level $\op{SL}_2(\ZZ)$ when $k\ge4$ is even. The proof of modularity arises because an element of $\op{SL}_2(\ZZ)$ merely permutes the lattice $\ZZ^2\setminus\{(0,0)\}$, but we will not be more rigorous.
	% \footnote{Plugging in $-I$ to the symmetry condition in \eqref{eq:symmetry-condition} reveals that $f(z)=(-1)^kf(z)$, so all nonzero modular forms for $\op{SL}_2(\ZZ)$ have even weight.}
\end{example}
% An alternate perspective is that we are trying to find $\Gamma$-invariant $k$-fold differential forms on $\mathbb H^*$. Indeed, if $f(z)\cdot(dz)^k$ is a $\Gamma$-invariant $k$-fold differential form on $\mathbb H^*$, then we can compute
% \[f(z)\cdot(dz)^k=f\left(\frac{az+b}{cz+d}\right)\cdot \left(d\,\frac{az+b}{cz+d}\right)^k=(cz+d)^{-2k}f(z)\cdot(dz)^k\]
% for any matrix in $\Gamma$. (We have used the fact that $\Gamma\subseteq\op{SL}_2(\ZZ)$ in the above computation.) Rearranging the above equation produces the desired symmetry condition for the weight $2k$.

\subsection{The \texorpdfstring{$q$}{q}-Expansion}
It is not entirely obvious how to write down a modular form; for example, we were able to write down the Eisenstein series $G_k$ via some kind of infinite series, but such series expansions have no reason to generalize to other modular forms, even of level $\op{SL}_2(\ZZ)$. It turns out that the best way to write down a modular form is via Fourier analysis.
\begin{proposition}[{$q$-expansion}] \label{prop:q-expansion}
	Let $f\colon\mathbb H\to\CC$ be a holomorphic function which is also holomorphic at $\infty$. If $f(z)=f(z+1)$, then there are complex numbers $a_0,a_1,\ldots$ such that
	\[f(z)=\sum_{n=0}^\infty a_nq^n\]
	where $q\coloneqq e^{2\pi iz}$. Such an expansion is called a ``$q$-expansion.''
\end{proposition}
\begin{proof}
	Roughly speaking, having $f(z)=f(z+1)$ implies that we can give $f$ a Fourier expansion. Then being bounded as $z\to\infty$ implies that the negative terms of our Fourier series must vanish.

	Regardless, we provide an argument avoiding Fourier analysis. Define the function $\overline f$ by $\overline f\left(e^{2\pi inz}\right)\coloneqq f(z)$ for $z\in\mathbb H$; note the condition $f(z)=f(z+1)$ implies $\overline f$ is well-defined. Notably, the domain of $\overline f$ is
	\[\left\{e^{2\pi inz}:z\in\mathbb H\right\}=\left\{e^{-z}:\op{Re}z>0\right\}=\{q:0<|q|<1\}.\]
	Thus, we have defined a function $\overline f$ from the punctured open disk of radius $1$ to $\CC$. In fact, we can check by hand that $\overline f$ is holomorphic: for any $q$ and $q'$ with $q=e^{2\pi inz}$ and $q'=e^{2\pi inz'}$, we see
	\[\frac{\overline f(q)-\overline f(q')}{q-q'}=\frac{f(z)-f(z')}{z-z'}\cdot\frac{z-z'}{e^{2\pi inz}-e^{2\pi inz'}}.\]
	Because $f$ is holomorphic and $z\mapsto e^{2\pi inz}$ are holomorphic, we see that the right-hand side converges as $z'\to z$, so the left-hand side converges as $q'\to q$, meaning that $\overline f$ is holomorphic.

	Now, we would like to extend $\overline f$ to the full open disk of radius $1$. For this, we note that $|f(z)|$ being bounded as $\op{Im}z\to\infty$ translates into $|\overline f(q)|$ being bounded as $q\to0$, so we may extend $\overline f$ to a holomorphic function on the open disk of radius $1$. But we are now done: the Taylor expansion of $\overline f$ at $q=0$ is the desired $q$-expansion for $f$.
	% Being holomorphic, we may give $\overline f$ a Taylor expansion
	% \[\overline f(q)=\sum_{n=0}^\infty a_nq^n,\]
	% so plugging in $f(z)=\overline f\left(e^{2\pi inz}\right)$ completes the proof.
\end{proof}
\begin{example} \label{ex:gk-q-expansion}
	For even $k\ge4$, the $q$-expansion of $G_k(z)$ is given by
	\[G_k(z)=2\zeta(k)+\frac{2(2\pi i)^k}{(k-1)!}\sum_{n=1}^\infty\sigma_{k-1}(n)q^n.\]
	% We will not need this for the main theorem, so we will not prove it.
\end{example}
\begin{remark}
	If $f$ is a modular form of level $\Gamma(N)$, then we see $T^N\in\Gamma(N)$ implies $f(z+N)=f(z)$, so the holomorphic function $g(z)\coloneqq f(z/N)$ will have a $q$-expansion by \Cref{prop:q-expansion}. Thus, we are able to give $f$ a $q$-expansion of the form $f(z)=\sum_{n=0}^\infty a_ne^{2\pi nz/N}$.
\end{remark}
\Cref{prop:q-expansion} only uses a single symmetry condition of a modular form. One can ask if it is possible to ``see'' the other symmetries in this $q$-expansion as well, though in practice this is hard. As a partial result, we note these symmetry conditions are able to place constraints on the size of our coefficients.
\begin{proposition} \label{prop:coeff-bound}
	Let $f\colon\mathbb H^*\to\CC$ be a modular form of level $\Gamma(N)$ and weight $k$ have $q$-expansion given by $f(z)=\sum_{n=0}^\infty a_ne^{2\pi inz/N}$. If $f$ is a cusp form (i.e., vanishes at all cusps) or $k<0$, then $a_n=O_f\left(Ne^{2\pi ny/N}y^{-k/2}\right)$ for any real $y>0$.
\end{proposition}
\begin{proof}
	Define the function $h\colon\mathbb H\to\RR$ by $h(z)\coloneqq|f(z)|(\op{Im}z)^{k/2}$. A direct computation shows that $h$ is invariant under $\Gamma(N)$. Further, $h$ vanishes at each cusp $c\in\PP^1_\QQ$: 
	% The point is that $f$ is invariant under $\Gamma(N)$ and vanishes as $z\to c$ for any cusp $c\in\PP^1_\QQ$. To see the invariance, we compute
	% \[\left|f\left(\begin{bmatrix}
	% 	a & b \\
	% 	c & d
	% \end{bmatrix}z\right)\right|\left(\op{Im}\begin{bmatrix}
	% 	a & b \\
	% 	c & d
	% \end{bmatrix}z\right)^{k/2}=|cz+d|^k|f(z)|\cdot\frac{(\op{Im}z)^{k/2}}{|cz+d|^{k}}=|f(z)|(\op{Im}z)^{k/2}.\]
	% To see that $h$ vanishes as $z\to c$ at a cusp $c\in\PP^1_\QQ$,
	note that $f$ is surely bounded as $z\to c$, so we get the result for free if $k<0$. Otherwise, translating the given cusp to $i\infty$ via $\op{SL}_2(\ZZ)$, we note that $f$ vanishing at $c$ turns into $a_0=0$ of the corresponding $q$-expansion. As such, $e^{-2\pi iz/N}f(z)\to a_1$ as $z\to i\infty$, so $|f(z)|(\op{Im}z)^{k/2}\to 0$ as $z\to i\infty$ because $\left|f(z)\right|\to0$ exponentially as $z\to i\infty$.

	Thus, we claim that $h$ is bounded. Note that $f(z+N)=f(z)$ means that we only have to look in a fixed vertical strip of $\mathbb H$, and the fact that $h$ vanishes at the cusps implies that $h$ is bounded for large values of $\op{Im}z$ and close to the other cusps. Thus, it suffices to subtract out neighborhoods of the cusps and observe $h$ is bounded on the resulting compact set.

	Because $h$ is bounded on $\mathbb H^*$, let $C$ denote its maximum. We now extract our Fourier coefficients. On one hand, the Cauchy integral formula applied to the $q$-expansoin of $f$ yields
	\[\left|a_n\right|=\left|\int_0^Nf(x+yi)e^{-2\pi in(x+yi)/N}\,dx\right|\le e^{2\pi ny/N}\int_0^N\left|f(x+yi)\right|\,dx\le CNe^{2\pi ny/N}y^{-k/2},\]
	which is what we wanted.
\end{proof}
\begin{corollary}
	The only modular form $f$ of level $\Gamma(N)$ and weight $k$ for $k<0$ is the zero function.
\end{corollary}
\begin{proof}
	Sending $y\to0^+$ in \Cref{prop:coeff-bound} implies that the $q$-expansion of $f$ vanishes.
\end{proof}
\begin{corollary} \label{cor:cusp-form-coeffs}
	Let $f\colon\mathbb H^*\to\CC$ be a modular form of level $\Gamma(N)$ and weight $k$ have $q$-expansion given by $f(z)=\sum_{n=0}^\infty a_ne^{2\pi inz/N}$. If $f$ vanishes at all cusps, then $a_n=O_f\left(Ne^{2\pi/N}n^{k/2}\right)$.
\end{corollary}
\begin{proof}
	Set $y=1/n$ in \Cref{prop:coeff-bound}.
\end{proof}
\begin{remark} \label{rem:mod-forms-cusps}
	Suppose $f$ is a modular form of level $\op{SL}_2(\ZZ)$. Then one can see that any nonzero modular form $f(z)=\sum_{n=0}^\infty a_nq^n$ of weight $k>0$ has $k\ge4$ and $k$ even; in particular, one can show with some effort that there are no modular nonzero modular forms of weight $2$. As such there exists a constant $a$ such that $g(z)\coloneqq f(z)-aG_k(z)$ is a cusp form. Upon comparing $q$-expansions of the cusp form $g$ in \Cref{cor:cusp-form-coeffs} and of $G_k$ in \Cref{ex:gk-q-expansion}, we see that $f(z)=g(z)+aG_k(z)$ must have $a_n=O\left(n^{k-1}\right)$.
\end{remark}

% \begin{proposition} \label{prop:weight-0}
% 	Let $f$ be a modular form of weight $0$. Then $f$ is a constant function.
% \end{proposition}
% \begin{proof}
% 	Omitted for now.\todo{}
% \end{proof}

\section{Hecke's Converse Theorem}
In this section, we state and prove Hecke's converse theorem. As convention, we require that our modular forms have level $\Gamma=\op{SL}_2(\ZZ)$; many results and definitions generalize, but this will ease headaches. For example, by \Cref{prop:gens-for-sl2z}, it suffices to check the symmetry conditions
\begin{equation}
	f(z+1)=f(Tz)=f(z)\qquad\text{and}\qquad f(-1/z)=f(Sz)=z^kf(z). \label{eq:level-1-symmetry-condition}
\end{equation}
Indeed, one can directly compute that if $f$ satisfies the symmetry condition of \eqref{eq:symmetry-condition} for $\gamma$ and $\gamma'$, then $f$ satisfies the symmetry condition for $\gamma\gamma'$ and $\gamma^{-1}$.

\subsection{The \texorpdfstring{$L$}{ L}-Function}
We fix some notation. Throughout, $f$ is a modular form of weight $k$ with $q$-expansion $f(z)=\sum_{n=0}^\infty a_nq^n$. Then we define the $L$-function
\begin{equation}
	L(s,f)\coloneqq\sum_{n=1}^\infty\frac{a_n}{n^s}. \label{eq:define-l}
\end{equation}
By \Cref{rem:mod-forms-cusps}, we see that $a_n=O\left(n^{k-1}\right)$, so $L(s,f)$ converges absolutely and uniformly on compacts to a holomorphic function on $\{s\in\CC:\op{Re}s>k\}$. As with the story of the $\zeta$-function and the Jacobi $\theta$-function, we might hope that the symmetry conditions of $f$ will give translate to a functional equation for $L$. Our means of translation is the Mellin transform.
\begin{lemma} \label{lem:computed-completed-l-func}
	Fix everything as above. For $\op{Re}s>k$, we have
	\[\int_0^\infty(f(yi)-a_0)y^s\,\frac{dy}y=(2\pi)^{-s}\Gamma(s)L(s,f).\]
\end{lemma}
\begin{proof}
	This is a direct computation. Expanding and substituting $y\mapsto 2\pi ny$ yields
	\[(2\pi)^{-s}\Gamma(s)L(s,f) = \sum_{n=1}^\infty\left(a_nn^{-s}(2\pi)^{-s}\int_0^\infty e^{-y}y^s\,\frac{dy}y\right)=\sum_{n=1}^\infty\int_0^\infty a_ne^{2\pi in(yi)}y^s\,\frac{dy}y.\]
	Now, we note that we have absolute convergence of this sum of integral: reversing the computation, we are trying to show that $(2\pi)^{-s}\Gamma(s)\sum_{n=1}^\infty\frac{|a_n|}{n^s}$ converges, which is true because $a_n=O\left(n^{k-1}\right)$ and $\op{Re}s>k$. Exchanging the sum and the integral produces the desired result.
\end{proof}
The point is that we are able to recover the ``completed'' $L$-function
\begin{equation}
	\Lambda(s,f)\coloneqq(2\pi)^{-s}\Gamma(s)L(s,f) \label{eq:define-lambda}
\end{equation}
as the Mellin transform of $f(yi)-a_0$. Thus, the Mellin transform provides a bridge between $f$ and $L$, through which we can pass $f$'s symmetry condition.
\begin{proposition} \label{prop:fe}
	Fix everything as above. Then $\Lambda(s,f)$ has a meromorphic continuation to $\CC$ with at worst simple poles at $s=0$ and $s=k$ with residues $-a_0$ and $(-1)^{k/2}a_0$, respectively. (Namely, if $a_0=0$, then $\Lambda$ is actually holomorphic on all $\CC$.) Furthermore,
	\[\Lambda(s,f)=(-1)^{k/2}\Lambda(k-s,f).\]
\end{proposition}
\begin{proof}
	% One can see the functional equation more or less directly from the symmetry condition $f(i/y)=(yi)^kf(iy)$ from \eqref{eq:level-1-symmetry-condition} plugged into \Cref{lem:computed-completed-l-func}. However, it will be more efficient in exposition to produce the analytic continuation and the functional equation at once. As such, 
	We follow \cite[Theorem~9.5]{conrad-forms}. The plan is to roughly follow Riemann's proof of the functional equation of $\zeta$, writing everything as integrals on $[1,\infty)$. With this in mind, we use \Cref{lem:computed-completed-l-func} to write
	\[\Lambda(s,f)=\int_0^1(f(yi)-a_0)y^s\,\frac{dy}y+\int_1^\infty(f(yi)-a_0)y^s\,\frac{dy}y.\]
	We want to push the left-hand integral to be over $[1,\infty)$. Well, recall
	\[f(i/y)=(yi)^kf(yi)=(-1)^{k/2}y^kf(yi)\]
	by plugging $yi$ into \eqref{eq:level-1-symmetry-condition}. Now, for $\op{Re}s>k$ (where everything converges by \Cref{lem:computed-completed-l-func}), we use the above symmetry condition combined with the substitution $y\mapsto1/y$ to write
	\begin{align*}
		\int_0^1(f(yi)-a_0)y^s\,\frac{dy}y &= (-1)^{k/2}\int_0^1f(i/y)y^{s-k}\,\frac{dy}y-\frac{a_0}s \\
		&= (-1)^{k/2}\int_1^\infty f(yi)y^{k-s}\,\frac{dy}y-\frac{a_0}s \\
		&= (-1)^{k/2}\int_1^\infty(f(yi)-a_0)y^{k-s}\,\frac{dy}y-\frac{a_0}s-(-1)^{k/2}\frac{a_0}{k-s}.
	\end{align*}
	At this point, we may collect our terms in $\Lambda(s,f)$ back together into
	\[\Lambda(s,f)=\left(\int_1^\infty(f(yi)-a_0)y^s\,\frac{dy}y-\frac{a_0}s\right)+(-1)^{k/2}\left(\int_1^\infty(f(yi)-a_0)y^{k-s}\,\frac{dy}y-\frac{a_0}{k-s}\right).\]
	Now, $(f(z)-a_0)/q\to a_1$ as $z\to i\infty$, so $f(yi)-a_0$ vanishes rapidly as $y\to\infty$, so the integrals above converge to a holomorphic function on all $\CC$, from which the meromorphic continuation of $\Lambda(s,f)$ follows. The above expression for $\Lambda(s,f)$ also implies that $\Lambda(s,f)=(-1)^{k/2}\Lambda(k-s,f)$ upon plugging in $k-s$.
\end{proof}
\begin{remark}
	\Cref{prop:fe} is somewhat remarkable in that it tells us that $L(s,f)$ ``remembers'' the constant term $a_0$ of the modular form $f$ even though it is not featured in the actual $L$-function.
\end{remark}

\subsection{The Converse Theorem}
The main input to \Cref{prop:fe} was the computation of $\Lambda(s,f)$ as a Mellin transform of $f(yi)-a_0$ plus the symmetry condition \eqref{eq:level-1-symmetry-condition}. To go in the other direction, we will invert the Mellin transform in order to use the functional equation of $\Lambda$ to produce a symmetry condition for $f$.
\begin{theorem}[Hecke] \label{thm:hecke}
	Let $\{a_n\}_{n=0}^\infty$ be a sequence of complex numbers such that $|a_n|=O\left(n^c\right)$ for some real number $c>0$. For $\op{Re}s>c+1$, define $L(s,f)$ and $\Lambda(s,f)$ as in \eqref{eq:define-l} and \eqref{eq:define-lambda}. Further, suppose the following of $\Lambda$ for some even positive integer $k$.
	\begin{itemize}
		\item $\Lambda$ has a meromorphic continuation to $\CC$ with at worst simple poles at $s=0$ and $s=k$ with residues $-a_0$ and $(-1)^{k/2}a_0$, respectively.
		\item $\Lambda$ is bounded in the vertical strips $\{s:\sigma_1\le\op{Re}s\le\sigma_2,\left|\op{Im}s\right|\ge1\}$ for any real numbers $\sigma_1<\sigma_2$.
		\item $\Lambda$ satisfies the functional equation $\Lambda(s,f)=(-1)^{k/2}\Lambda(k-s,f)$.
	\end{itemize}
	Then $f(z)\coloneqq\sum_{n=0}^\infty a_ne^{2\pi inz}$ is a modular form of weight $k$.
\end{theorem}
\begin{proof}
	The infinite series defining $f$ converges absolutely and uniformly on compacts for $\op{Im}z>0$ because $\left|a_n\right|=O\left(n^c\right)$. Thus, $f$ defines a holomorphic function on $\mathbb H$. Furthermore, as $z\to i\infty$, we see that $f(z)\to a_0$, so $f$ is holomorphic on $\mathbb H^*$. It remains to check the symmetry conditions in \eqref{eq:level-1-symmetry-condition}. The $q$-expansion yields $f(z)=f(z+1)$ automatically, so it remains to check $f(-1/z)=z^kf(z)$. Because everything is holomorphic on $\mathbb H$, it suffices to check that $f(-1/z)=z^kf(z)$ on the positive imaginary axis. In other words, we check
	\begin{equation}
		f(i/y)\stackrel?=i^ky^kf(yi) \label{eq:needed-s-symmetry}
	\end{equation}
	for any $y>0$. For this, we note \Cref{lem:computed-completed-l-func} still computes the Mellin transform of $f(yi)-a_0$ for $\op{Re}s>c$, so for $\sigma>\max\{c,k\}$, Mellin inversion implies
	\[f(iy)-a_0=\frac1{2\pi i}\int_{\sigma-i\infty}^{\sigma+i\infty}\Lambda(s,f)y^{-s}\,ds.\]
	We will achieve \eqref{eq:needed-s-symmetry} after using the functional equation and moving the contour. This needs the following.
	\begin{proposition}[Phragmen--Lindel\"of principle] \label{prop:pl}
		Suppose $f$ is a function holomorphic over $\{s\in\CC:\op{Re}s\in[\sigma_1,\sigma_2],\op{Im}s>c\}$. Further, suppose there is $\alpha>0$ such that $f(\sigma+it)=O\left(e^{t^\alpha}\right)$ for $\sigma\in[\sigma_1,\sigma_2]$. If there is $M$ such that $f(\sigma+it)=O_\sigma\left(t^M\right)$ for $\sigma\in\{\sigma_1,\sigma_2\}$, then $f(\sigma+it)=O\left(t^M\right)$ for $\sigma\in[\sigma_1,\sigma_2]$.
	\end{proposition}
	\begin{proof}
		The proof is not long, but it would take us too far afield. See \cite[p.~262]{lang-alg-nt}.
	\end{proof}
	We now attempt to move the contour from $\op{Re}s=\sigma$ to $\op{Re}s=k-\sigma$. Because $\sigma>k$, we see $L(s,f)$ absolutely converges and is therefore bounded on $\op{Re}s=\sigma$; thus, because $\Gamma(\sigma+it)\to0$ as $t\to\pm\infty$, we see that $\Lambda(\sigma+it,f)y^{-(\sigma+it)}\to0$ as $t\to\pm\infty$. The functional equation for $\Lambda$ implies that $\Lambda(k-(\sigma+it),f)y^{-(k-(\sigma+it))}\to0$ as $t\to\pm\infty$ as well, so \Cref{prop:pl} tells us that $\Lambda(k-s,f)y^{-s}\to0$ as $\op{Im}s\to\pm\infty$ uniformly for $\op{Re}s\in[k-\sigma,\sigma]$. Thus, noting that $\Lambda(s,f)$ has only at worst simple poles at $s=0$ and $s=k$ with known residues, the Residue theorem grants
	\begin{align*}
		\lim_{N\to\infty}\frac1{2\pi i}\left(\int_{\sigma-iN}^{\sigma+iN}\Lambda(s,f)y^{-s}\,ds-\int_{(k-\sigma)-iN}^{(k-\sigma)+iN}\Lambda(s,f)y^{-s}\,ds\right) &= -a_0+(-1)^{k/2}a_0y^{-k}.
	\end{align*}
	Bringing everything together, we move the contour and use the functional equation to see
	\begin{align*}
		f(iy)-a_0 &= \frac1{2\pi i}\int_{(k-\sigma)-i\infty}^{(k-\sigma)+i\infty}\Lambda(s,f)y^{-s}\,ds-a_0+(-1)^{k/2}a_0y^{-k} \\
		&= \frac{(-1)^{k/2}}{2\pi i}\int_{(k-\sigma)-i\infty}^{(k-\sigma)+i\infty}\Lambda(k-s,f)y^{-s}\,ds-a_0+(-1)^{k/2}a_0y^{-k} \\
		&= \frac{(-1)^{k/2}}{2\pi i}\int_{\sigma-i\infty}^{\sigma+i\infty}\Lambda(s,f)y^{-(k-s)}\,ds-a_0+(-1)^{k/2}a_0y^{-k} \\
		&= (-1)^{k/2}y^{-k}(f(i/y)-a_0)-a_0+(-1)^{k/2}a_0y^{-k}.
	\end{align*}
	Rearranging yields \eqref{eq:needed-s-symmetry}, so we are done.
\end{proof}

% \section{Hecke Operators}

\printbibliography[title={References}]

\end{document}