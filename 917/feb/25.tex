% !TEX root = ../notes.tex

\documentclass[../notes.tex]{subfiles}

\begin{document}

\section{February 25}
Here we go.

\subsection{Hopf Algebroids}
Given an $\mathbb E_\infty$-ring $R$, one has an $\infty$-category of $R$-modules. If $R$ is a discrete ring, then this is the derived category of $R$-modules. One may want such an explicit description of the $R$-modules in more general situations.
\begin{definition}[totalization]
	The \textit{totalization} of a cosimplicial discrete ring $\Delta\opp\to\mathrm{Ring}$ is the limit of this functor.
\end{definition}
\begin{example}
	One can define a ring $R$ as the limit of the \v{C}ech nerve
	\[\mathrm{BP}_*\to\mathrm{BP}_*\mathrm{BP}\to\cdots.\]
\end{example}
\begin{remark}
	Before totalization, a cosimplicial discrete ring is some ``affine stack,'' meaning that it is a functor which takes a commutative ring to simplicial sets.
\end{remark}
It will sometimes be true that such an affine stack will actually output to classical groupoids instead of arbitrary simplicial sets. In this situation, the cosimplicial ring is a Hopf algebroid.
\begin{definition}[Hopf algebroid]
	A \textit{Hopf algebroid} is a pair of commutative rings $(A,\Gamma)$ equipped with maps $\eta_L,\eta_R\colon A\to\Gamma$, a comultiplication $\Delta\colon\Gamma\to\Gamma\otimes_A\Gamma$, a counit $\varepsilon\colon\Gamma\to A$, and an inversion $\iota\colon\Gamma\to\Gamma$, all of which satisfy the following.
	\begin{itemize}
		\item $\eta_L$ makes $\Gamma$ into a flat $A$-module.
		\item There are various other identities so that $\op{Hom}(A,R)$ is the set of objects and $\op{Hom}(\Gamma,R)$ is the set of morphisms of a groupoid.
	\end{itemize}
	When no confusion is possible, we will simply call $\Gamma$ the Hopf algebroid.
\end{definition}
\begin{remark}
	A Hopf algebroid $(A,\Gamma)$ produces a cosimplicial ring as follows.
	% https://q.uiver.app/#q=WzAsNCxbMCwwLCJBIl0sWzEsMCwiXFxHYW1tYSJdLFsyLDAsIlxcR2FtbWFcXHRpbWVzX0FcXEdhbW1hIl0sWzMsMCwiXFxjZG90cyJdLFsxLDBdLFswLDEsIlxcZXRhX0wiLDAseyJvZmZzZXQiOi0xfV0sWzAsMSwiXFxldGFfUiIsMix7Im9mZnNldCI6MX1dLFsxLDJdLFsyLDEsIiIsMix7Im9mZnNldCI6LTF9XSxbMiwxLCIiLDAseyJvZmZzZXQiOjF9XSxbMSwyLCIiLDAseyJvZmZzZXQiOi0yfV0sWzEsMiwiIiwwLHsib2Zmc2V0IjoyfV1d&macro_url=https%3A%2F%2Fraw.githubusercontent.com%2FdFoiler%2Fnotes%2Fmaster%2Fnir.tex
	\[\begin{tikzcd}[cramped]
		A & \Gamma & {\Gamma\times_A\Gamma} & \cdots
		\arrow["{\eta_L}", shift left, from=1-1, to=1-2]
		\arrow["{\eta_R}"', shift right, from=1-1, to=1-2]
		\arrow[from=1-2, to=1-1]
		\arrow[from=1-2, to=1-3]
		\arrow[shift left=2, from=1-2, to=1-3]
		\arrow[shift right=2, from=1-2, to=1-3]
		\arrow[shift left, from=1-3, to=1-2]
		\arrow[shift right, from=1-3, to=1-2]
	\end{tikzcd}\]
	The totalization is an $\mathbb E_\infty$-$\ZZ$-algebra.
\end{remark}
\begin{example}
	Any Hopf algebra $\Gamma$ over a field $k$ produces a Hopf algebroid by taking $\eta_L=\eta_R$ to both be the same identity map. For example, one can consider the Hopf algebra $\alpha_p\coloneqq\FF_p[x]/\left(x^p\right)$ over $\FF_p$, which is a subscheme of $\mathbb G_{a,\FF_p}$.
\end{example}
\begin{example}
	If we take $A=\pi_*\mathrm{BP}$ and $\Gamma=\pi_*(\mathrm{BP}_*\mathrm{BP})$, then the Hopf algebroid (co)represents the moduli space of $p$-typical formal groups up to strict isomorphism.
\end{example}
We thus will satisfy ourselves with trying to understand the category of modules for such a Hopf algebroid. The point is to have some module for $A$ and $\Gamma$ which communicate with each other in some way.
\begin{definition}[module]
	Fix a Hopf algebroid $\Gamma$. Then a \textit{discrete left $\Gamma$-comodule} is a discrete left $A$-module $M$ equipped with an $A$-module map
	\[\psi\colon M\to\Gamma\otimes_AM,\]
	which is counital and coassociative.
\end{definition}
\begin{remark}
	The idea is that one can always present a module for the totalization as a sequence of modules, each for the discrete ring $A$ and then $\Gamma$ and then $\Gamma\otimes_A\Gamma$, and so on, always with some homotopies for coherence. But one can think of any module (indeed, any spectrum) by filtering it with its Postnikov tower, which in the case of $\ZZ$-modules allows one to recover chain complexes. The same thing works for $\Gamma$, thereby granting us our description of $\mathrm{Mod}(R)$, which are now the inductive systems of the bounded chain complexes. (Technically speaking, bounded chain complexes are those $R$-modules which are finite limits of just $R$, and we need to close this category under filtered colimits to get all of $\mathrm{Mod}(R)$.)
\end{remark}
\begin{remark}
	The (discrete) $A$-linear dual $\Gamma^\lor$ has the natural structure of a Hopf algebroid. Then the $A$-linear dual of a comodule $M$ is simply a right $\Gamma^\lor$-module.
\end{remark}
\begin{remark}[Hahn]
	Let's not pretend we care about homological algebra. Homological algebra is some fiddly way of pointing out modules.
\end{remark}

\subsection{Vanishing Lines}
Let's return to $\alpha_p$.
\begin{example}
	Take $(A,\Gamma)=(\FF_p,\FF_p[x]/\left(x^p\right))$, and let $R$ be the totalization. Then a perfect $R$-module is a bounded chain complex of finite-dimensional $\FF_p$-modules $M$ equipped with the structure map $\psi\colon M\to\FF_p[x]/\left(x^p\right)\otimes_{\FF_p}M$. However, such a map is the same as giving $M$ the structure of an $\mathbb F_p[x]/\left(x^p\right)$-module because $\Gamma$ is its own self-dual (indeed, $\alpha_p=\alpha_p^\lor$).
\end{example}
\begin{example}
	We continue from the previous example. The homotopy groups of the totalization $R$ of this Hopf algebroid can be computed to be $\op{Ext}_\Gamma(\FF_p,\FF_p)$ via the Dold--Kan correspondence. This turns out to be $\FF_p[x]\otimes\Lambda(h)$, where $h$ is in degree $2$ and $\beta$ is in degree approximately $p$. Letting $x$ have some grading (e.g., a $\ZZ/p\ZZ$-grading), then $\pi_*R$ admits a bigrading, so we see that $\pi_*R$ is supported on two lines of positive slope in the bigrading.
\end{example}
\begin{warn}
	By convention, in all of our bigradings, homological degree is the vertical degree.
\end{warn}
One can use the above examples to show the following.
\begin{proposition}
	Let $R$ be the totalization of the Hopf algebroid $\FF_p[x]$, and choose a perfect $R$-module $M$. Then either
	\begin{listalph}
		\item $\pi_*M$ has a horizontal vanishing line in its bigraded homotopy groups, or
		\item there is a map $\Theta\colon\Sigma^{x,y}M\to M$ of $R$-modules which is not nilpotent and of the same slope as $\beta$. In this case, $M/\Theta\cong\op{cofiber}\Theta$. 
	\end{listalph}
\end{proposition}
\begin{proof}
	Simply classify all $R$-modules, as above. Roughly speaking, by the Dold--Kan correspondence, the homotopy groups of a discrete $R$-module $M$ are computed as $\op{Ext}_{\FF_p[x]/\left(x^p\right)}(M,\FF_p$. Thus, everything is built from the following two simple discrete modules.
	\begin{itemize}
		\item For the module $\FF_p$, we see that its homotopy groups have the second property.
		\item For the module $\FF_p[x]/\left(x^p\right)$, its homotopy groups are finitely supported.
		\qedhere
	\end{itemize}
\end{proof}
One can even pass to higher powers.
\begin{proposition}
	Let $R$ be the totalization of the Hopf algebroid $\FF_p[x]/\left(x^{p^2}\right)$. For every perfect $R$-module $M$, the homotopy groups $\pi_*M$ admit a vanishing line with minimal slope and an endomorphism $\Theta$ with slope parallel to that vanishing line. The cofiber of $\Theta$ will have a smaller slope vanishing line.
\end{proposition}
\begin{proof}
	Note that $R$ is an extension of the totalization $R_0$ of $\FF_p[x]/\left(x^p\right)$ by itself. The point is that this sort of statement is immune to taking extensions in some formal way.
\end{proof}
Here is another example.
\begin{example}
	Let $\mc P_*\coloneqq\pi_{2*}(\FF_p\otimes_{\mathbb S}\FF_p)$ be the ``even'' Steenrod algebra, which is a free polynomial ring over $\FF_p$ with generators $\xi_i$ in degree $2p^i-2$. It turns out that one has a Hopf algebra coming from $(A,\Gamma)\coloneqq(\pi_{2*}\FF_p,\mc P_*)$. It is difficult to explicitly write down the coproduct $\mc P_*\to\mc P_*\otimes_{\FF_p}\mc P_*$, which come from the Adem relations. Nonetheless, we can define this map by taking homotopy groups of the composite
	\[\FF_p\otimes_{\mathbb S}\FF_p=\FF_p\otimes_{\mathbb S}\mathbb S\otimes_{\mathbb S}\FF_p\to\FF_p\otimes_{\mathbb S}\FF_p\otimes_{\mathbb S}\FF_p=(\FF_p\otimes_{\mathbb S}\FF_p)\otimes_{\FF_p}(\FF_p\otimes_{\mathbb S}\FF_p).\]
	In particular, because $\FF_p$ is a field, taking homotopy groups of the target produces $\mc P_*\otimes_{\FF_p}\mc P_*$.
\end{example}
\begin{remark}
	It turns out that the Hopf algebra $\mc P_*$ (co)represents the additive group with its automorphisms.
\end{remark}
\begin{remark}
	It turns out that the Hopf algebra $\mc P_*$  is an iterated extension of Hopf algebras, and this process gives us the associated graded which is a tensor product of $\FF_p\left[\xi_i^{p^j}\right]/\left(\xi_i^{p^{j+1}}\right)$s.
\end{remark}
One can now use our ideas from $\alpha_p$ and $\FF_p[x]/\left(x^{p^2}\right)$ to show the following.
\begin{theorem}[Palmieri]
	Fix a perfect module $M$ over the graded $\mathbb E_\infty$-ring $R$ which is the totalization of the Hopf algebra $\pi_2(\FF_p\otimes_{\mathbb S}\FF_p)$. Then the homotopy groups $\pi_*M$ admit a vanishing line of some slope contained in the set
	\[\left\{\frac1{p^{j+1}(p^i-1)-1}:i>j\ge0\right\}\cup\{\infty\}.\]
	Further, there is an endomorphism $\Theta$ of $M$ with this slope which is not nilpotent, and the cofiber of $\Theta$ is an $R$-module with smaller slope.
\end{theorem}
\begin{example}
	If $X$ is a finite CW-complex, then
	\[\lim\pi_{2*}\left(\Sigma_+^\infty X\otimes_{\mathbb S}\FF_p^{\otimes(i+1)}\right).\]
\end{example}

\end{document}