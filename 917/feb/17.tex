% !TEX root = ../notes.tex

\documentclass[../notes.tex]{subfiles}

\begin{document}

\section{February 17}
Today we continue.

\subsection{Maps from \texorpdfstring{$\mathrm{MU}$}{ MU}}
Let's try to better understand $\mathrm{MU}$. For example, we can try to understand its functor of points.
\begin{remark} \label{rem:pts-of-mu}
	Choose an $\mathbb E_\infty$-ring $R$. If the composite
	\[\mathrm{BU}\to\op{Pic}\mathbb S\to\op{Pic}R\]
	is nullhomotopic, then taking the colimit to $\mathrm{Mod}(R)$ shows that $R\otimes_{\mathbb S}\mathrm{MU}$ is a constant colimit
	\[\colim_{\mathrm{BU}}R=R\otimes\colim_{\mathrm{BU}}\mathbb S=R\otimes\Sigma^\infty_+\mathrm{BU}.\]
	But note $\mathrm{BU}$ is already pointed, so $\Sigma^\infty_+\mathrm{BU}=\Sigma^\infty\mathrm{BU}\oplus\mathbb S$, so we receive an $R$-module map $R\otimes_{\mathbb S}\mathrm{MU}\to R$, which is the same data as an $\mathbb S$-module map $\mathrm{MU}\to R$.
\end{remark}
\begin{remark}
	This discussion of nullhomotopies has a precursor: the data of a complex orientation on $E$ is equivalent to the data of a nullhomotopy of the composite $\CP^\infty\to\op{Pic}\mathbb S\to\op{Pic}E$. Indeed, simply apply the adjunction between $\Sigma^2$ and $\Omega^2$.
\end{remark}
\begin{remark}
	One can upgrade \Cref{rem:pts-of-mu} to show that the space of unital maps $\mathrm{MU}\to R$ is equivalent to the data of a nullhomotopy of the composite
	\[\mathrm{BU}\to\op{Pic}\mathbb S\to\op{Pic}R\]
	in $\mathrm{Spaces}_*$. (Indeed, providing a map out of $\mathrm{MU}$ is a map out of a certain colimit, and unwinding the data finds the required nullhomotopy.) Furthermore, the space of $\mathbb E_n$-ring maps $\mathrm{MU}\to R$ is equivalent to the space of nullhomotopies of the same composite in the category of $n$-fold loop spaces.
\end{remark}
We can work out the above remark in some controlled situations. Here are the ``$\ZZ$-points'' of $\mathrm{MU}$.
\begin{example}
	The map $\mathrm{BU}\to\op{Pic}\mathbb S\to\op{Pic}\ZZ$ is nullhomotopic in the category of infinite loop spaces because $\op{Pic}\ZZ$ only has homotopy in two degrees (namely $0$ and $1$), and $\mathrm{BU}$ has no homology there. Thus, there is an isomorphism $\ZZ\otimes_{\mathbb S}\Sigma^\infty_+\mathrm{BU}\cong\ZZ\otimes_{\mathbb S}\mathrm{MU}$ of $\mathbb E_\infty$-rings, so one can compute
	\[\mathrm H_*(\mathrm{MU};\ZZ)\cong\mathrm H_*(\mathrm{BU};\ZZ),\]
	and then one can calculate $\mathrm H^*(\mathrm{BU};\ZZ)\cong\ZZ[b_1,b_2,\ldots]$.
\end{example}
\begin{remark}
	Let's explain where the $b_i$s may come from. The natural map $\op{MU}(1)\to\Sigma^2\mathrm{MU}$ (placing $\op{MU}(1)$ in the correct degree) is the same as the complex orientation $\Sigma^\infty\CP^\infty\to\Sigma^2\mathrm{MU}$. This then induces a map
	\[\ZZ\otimes_{\mathbb S}\Sigma^\infty\CP^\infty\to\ZZ\otimes_{\mathbb S}\Sigma^2\mathrm{MU}.\]
	But now the homology of $\CP^\infty$ is the free $\ZZ$-module on the generators $b_1,b_2,\ldots$.
\end{remark}
Here are the $\mathrm{MU}$-points of $\mathrm{MU}$.
\begin{example}
	There is a canonical identity $\mathrm{MU}\to\mathrm{MU}$, which thus induces a nullhomotopy $\mathrm{BU}\to\op{Pic}\mathbb S\to\op{Pic}\mathrm{MU}$ in the category of infinite loop  spaces. Thus, we see that
	\[\mathrm{MU}_*\mathrm{MU}=\mathrm{MU}_*\Sigma^\infty_+\mathrm{BU}.\]
	Now, the Atiyah--Hirzebruch spectral sequence degenerates, so this is $\pi_*\mathrm{MU}[b_1,b_2,\ldots]$. One way to see it degenerates is to know that $\pi_*\mathrm{MU}$ is concentrated in even degrees (via \Cref{thm:quillen}). There is also an easier way to see this, basically by directly showing that the $b_i$s are permanent cycles, which by the Leibniz rule is equivalent to showing that $b_1$ is permanent, which can be checked more directly from $\op{MU}(1)$.
\end{example}
The moral of the previous example is that $\mathrm{MU}_*\mathrm{MU}$ is the universal ring for a pair of formal group laws and a choice of strict isomorphism between them.
\begin{proposition}
	Fix an even $\mathbb E_\infty$-ring $E$. Then a complex orientation of $E$ extends to an $\mathbb E_2$-map $\mathrm{MU}\to E$.
\end{proposition}
\begin{proof}
	Note that there is a canonical line bundle $\mc L\colon\CP^\infty\to\mathrm{BU}\times\ZZ$, so there is a map $(\mc L-1)\colon\CP^\infty\to\mathrm{BU}\times\ZZ$, which then maps to $\op{Pic}\mathbb S$ as before. As discussed last class, we see the Thom spectrum of $(\mc L-1)$ is $\Sigma^{-2}\mathrm{MU}(1)$. The moral is that we have some composite
	\[\CP^\infty\to\mathrm{BU}\to\op{Pic}\mathbb S\to\op{Pic}E.\]
	Taking $\Sigma^2$ yields
	\[\Sigma^2\CP^\infty\to\mathrm{BSU}\to\mathrm B^2\op{Pic}\mathbb S\to\mathrm B^2\op{Pic}E.\]
	A nullhomotopy of the full sequence is a complex orientation on $E$, but a choice of nullhomotopy $\mathrm{BSU}\to\mathrm B^2\op{Pic}E$ on its own is an $\mathbb E_2$-map $\mathrm{MU}\to E$. Thus, we want to show that the former maps always extend to the latter maps. By splitting up $\mathrm B^2\op{Pic}E=\Omega^\infty\Sigma^2\op{pic}E$ into its Postnikov tower, it is enough to check that the maps
	\[\mathrm H^*(\mathrm{BSU};-)\to\mathrm H^*\left(\Sigma^2\CP^\infty;-\right)\]
	are surjective, which can be checked by calculations with the spectral sequence.
\end{proof}
\begin{remark}
	As a consequence, we note that an even $\mathbb E_\infty$-ring $E$ makes $E^*\CP^\infty$ into a formal group. Choosing coordinates means choosing an isomorphism $E^*\CP^\infty\cong\pi_{-*}E[[t]]$, which means choosing a formal group law up to isomorphism and thus induces a map $L\to\pi_*E$. Our discussion is intended to motivate the fact that choosing a coordinate alternatively should come from a choice of complex orientation and therefore the map $L\to\pi_*E$ comes from an $\mathbb E_2$-ring map $\mathrm{MU}\to E$. It is an interesting (open) question when one can upgrade this $\mathbb E_2$-ring map into an $\mathbb E_\infty$-ring map.
\end{remark}

\subsection{Localization}
Let's try to apply some of this discussion.
\begin{notation}
	Given a formal group law $f$, we define the power series $[n]_f(x)$ as $x+_f\cdots+_fx$, where $x$ is added to itself $n$ times. We will write $[n]$ for $[n]_f$ if there is no possibility of confusion.
\end{notation}
\begin{example}
	In the Lazard ring $L_{(p)}$, one finds that
	\[[3](x)=3x-8x_2x^3+72x_2^2x^5-840x_2^3x^7+\cdots,\]
	where the $x_\bullet$s are the Hazewinkel generators of $L_{(p)}\cong\ZZ[x_1,x_2,x_3,\ldots]$. (Here, $x_i$ has degree $2i$, and $x$ has degree $-2$, so the above polynomial is homogeneous.)
\end{example}
\begin{remark}
	With the Hazewinkel generators, one can show that the universal formal group law $f$ over $L_{(p)}$ is isomorphic to one that only involves the coefficients $x_{p^\bullet-1}$, and in fact, up to units, $x_{p^i-1}$ is the coefficient of $x^{p^{i}}$. Thus, setting $v_i\coloneqq x^{p^i-1}$, we see that the ring
	\[\ZZ_{(p)}[v_1,v_2,\ldots]\]
	carries the universal formal $p$-typical group law up isomorphism! (Here, $p$-typical means that our coefficients $x_j$ vanish for $j$ which is not one less than a power of $p$.) The point is that we only have to care about these coefficients $v_i$ to define such a formal group law.
\end{remark}
\begin{example}
	There is an idempotent ring endomorphism $f\colon L_{(p)}\to L_{(p)}$ sending $x_i$ to itself if $i$ is one less than a power of $p$ and sending $x_i$ to zero otherwise. Note that the colimit of the diagram
	\[L_{(p)}\stackrel f\to L_{(p)}\stackrel f\to\cdots\]
	is just $\ZZ_{(p)}[v_1,v_2,\ldots]$.
\end{example}
The above remarks prove the following.
\begin{proposition}
	There is an $\mathbb E_2$-ring map $\mathrm{MU}_{(p)}\to\mathrm{MU}_{(p)}$ whose telescope (in the category of $\mathbb E_2$-rings) is an $\mathbb E_2$-ring $\mathrm{BP}$ with
	\[\pi_*\mathrm{BP}\cong\ZZ_{(p)}[v_1,v_2,\ldots],\]
	where $v_i$ lives in degree $2p^i-2$.
\end{proposition}
\begin{proof}
	Let's explain how to produce the map $\op{MU}_{(p)}\to\op{MU}_{(p)}$. Indeed, the algebraic map $L\to\pi_*\op{MU}_{(p)}$ (given by the change of coordinates just described) lifts to an $\mathbb E_2$-map $\op{MU}\to\op{MU}_{(p)}$ (because this is how one maps out of $\mathrm{MU}$), which is then further localized.
\end{proof}
The moral of $\mathrm{BP}$ is that we have removed all of our ``junk'' generators.
\begin{remark}
	It is known that $\mathrm{BP}$ is not $\mathbb E_\infty$, but it is $\mathbb E_4$. Exactly the highest $\mathbb E_n$ which works is currently open.
\end{remark}
\begin{remark}
	As before, one can compute $\mathrm{BP}_*\mathrm{BP}$ as $\ZZ_{(p)}[v_1,v_2,\ldots][t_1,t_2,\ldots]$, which classifies two formal $p$-local formal group laws and a choice of $p$-typical strict isomorphism between them. Namely, the isomorphism is given by $x\mapsto x+_ft_1x^p+_ft_2x^{p^2}+_f\cdots$.
\end{remark}

\end{document}