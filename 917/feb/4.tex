% !TEX root = ../notes.tex

\documentclass[../notes.tex]{subfiles}

\begin{document}

\section{February 2}
Our next goal is to prove \Cref{thm:nishida-nilpotence}.

\subsection{Spectra}
Let's set some notation. We are interested in the category $\mathrm{Spaces}$ of spaces, also called anima.
\begin{definition}[pointed space]
	The category of pointed spaces is denoted $\mathrm{Space}_*$.
\end{definition}
\begin{remark}
	There is a functor $(-)_+\colon\mathrm{Spaces}\to\mathrm{Spaces}_*$ which simply adds a basepoint to any topological space.
\end{remark}
Our next category will be spectra.
\begin{definition}[spectra]
	A \textit{spectrum} is an infinite tuple $(X_0,X_1,\ldots)$ of spaces, equipped with isomorphisms $X_0\cong\Omega X_1\cong\Omega^2X_1\cong\cdots$. The category of spectra is, unsurprisingly, denoted $\mathrm{Spectra}$.
\end{definition}
\begin{remark}
	The functor $\Omega\colon\mathrm{Spectra}\to\mathrm{Spectra}$, given by shifting all the spaces down, is an auto-equivalence. Its inverse functor is $\Sigma$. Thus, $\mathrm{Spectra}$ is a stable category.
\end{remark}
\begin{remark}
	The functor $\Omega^\infty\colon\mathrm{Spectra}\to\mathrm{Spaces}$ given by sending a spectrum $X$ to $X_0$ admits an adjoint $\Sigma^\infty$. We may denote the composite functor $X\mapsto\Sigma^\infty(X_+)$ by $\Sigma^\infty_+$.
\end{remark}
\begin{remark}
	The category $\mathrm{Spectra}$ admits products and coproducts, which are the same and denoted $\oplus$.
\end{remark}
\begin{example}
	We let $\mathbb S$ denote the sphere spectrum $\Sigma^\infty S^0$. We can also think about this as $\Sigma^\infty_+\mathrm{pt}$.
\end{example}
\begin{remark}
	The category $\mathrm{Spectra}$ admits internal $\op{Hom}$s which has a left adjoint $\otimes_{\mathbb S}$. This gives a symmetric monoidal structure to $\mathrm{Spectra}$.
\end{remark}
\begin{remark}
	For any space $X$, the constant map $X\to\mathrm{pt}$ induces a map $\Sigma^\infty_+X\to\mathbb S$. On the other hand, for any pointed space $X$, the canonical map $\mathrm{pt}\to X$ induces a map $\mathbb S\to\Sigma^\infty X$. Thus, when $X$ is pointed, we find
	\[\Sigma_+^\infty X=\Sigma^\infty X\oplus\mathbb S.\]
\end{remark}

\subsection{Module Categories}
We now pass to other rings.
\begin{defihelper}[$\mathbb E_\infty$-ring] \nirindex{Einf ring@$\mathbb E_\infty$-ring}
	An \textit{$\mathbb E_\infty$-ring} $R$ is a spectrum $R$ equipped with a multiplication $R\otimes_{\mathbb S}R\to R$ and a unit $\mathbb S\to R$, as well as much other data (e.g., many other operations all required to cohere with each other).
\end{defihelper}
One can define modules in an expected way.
\begin{example}
	For any $\mathbb E_\infty$-ring $R$, the category $\mathrm{Mod}(R)$ of $R$-modules continues to be a stable, symmetric monoidal category, where the symmetric monoidal structure is given by some tensor product $\otimes_R$.
\end{example}
\begin{remark} \label{rem:cat-to-ring}
	Conversely, given a symmetric monoidal stable category $\mc C$, we can recover $\mc C$ as a module category over the ring $R\coloneqq\op{End}_{\mc C}1$, where $1$ is a tensor unit. Indeed, if $\mc C=\mathrm{Mod}_R$, we see that
	\[\op{Hom}_{\mc C}(R,R)=\op{Hom}_{\mathbb S}(\mc S,R)\]
	by the tensor--hom adjunction. This is then $\op{Hom}_{\mathbb S}(\Sigma_+^\infty\mathrm{pt},R)$, which is just $\Omega^\infty R$ because $\Sigma^\infty$ is left adjoint to $\Omega^\infty$.
\end{remark}
Here are some constructions with modules.
\begin{definition}
	Fix an $\mathbb E_\infty$-ring $R$. For any $x\in\pi_iR$, we receive a spectrum map $x\colon\Sigma^i\mathbb S\to R$ and hence an $R$-module map $\Sigma^iR\to R$. We now define $R\left[x^{-1}\right]$ to be the $R$-module
	\[\colim\left(R\stackrel x\to\Sigma^{-i}R\stackrel x\to\Sigma^{-2i}R\to\cdots\right).\]
\end{definition}
\begin{remark}
	By construction, there is a unit $R\to R\left[x^{-1}\right]$. Additionally, by understanding maps from $\mathbb S$ to the colimit, we see that $\pi_*R\left[x^{-1}\right]=(\pi_*R)\left[x^{-1}\right]$.
\end{remark}
Note that we have only defined $R\left[x^{-1}\right]$ as a module. One may hope to upgrade it to a ring.
\begin{remark}
	One can check that the canonical map
	\[R\otimes_RR\left[x^{-1}\right]\to R\left[x^{-1}\right]\otimes_RR\left[x^{-1}\right]\]
	is an equivalence: on homotopy groups, both sides are $(\pi_*R)\left[x^{-1}\right]$. It follows that $R\left[x^{-1}\right]$ is an idempotent algebra, defined by taking the inverse of the equivalence.
\end{remark}
\begin{remark}
	Here is another way to find that $R\left[x^{-1}\right]$ is a ring. Let $\mc C$ be the full subcategory of $R$-modules $M$ such that the action of $x$ on $\pi_*M$ is invertible. Then $\mc C$ is closed under the tensor product, so it is a stable symmetric monoidal category, and then \Cref{rem:cat-to-ring} allows us to find $R\left[x^{-1}\right]$ as the corresponding ring.
\end{remark}

\subsection{Some Free Constructions}
Here is a way to construct an $\mathbb E_\infty$-ring.
\begin{remark}
	Given a spectrum $R$, note that $X\otimes_{\mathbb S}X$ admits a natural action by the symmetric group $\Sigma_2$. This produces a functor $\mathrm B\Sigma_2\to\mathrm{Spectra}$ which sends the one object to $X\otimes_{\mathbb S}X$ and the nontrivial morphism to the swapping action. We may let $(X\otimes_{\mathbb S}X)_{h\Sigma_2}$ denote the ``quotient.'' In general, one can form the ``quotient'' $\left(R^{\otimes_{\mathbb S}n}\right)_{h\Sigma_n}$.
\end{remark}
\begin{remark}
	If $R$ is an $\mathbb E_\infty$-ring, then there is a canonical map $(R\otimes_{\mathbb S}R)_{h\Sigma_2}\to R$ given by the commutativity. The associativity and commutativity produce a map
	\[\left(R^{\otimes_{\mathbb S}n}\right)_{h\Sigma_n}\to R.\]
	Note that the existence of this map requires us to remember the higher coherences present in the definition of the $\mathbb E_\infty$-ring $R$.
\end{remark}
\begin{definition}[free ring]
	Fix a spectrum $X$. Then we define the \textit{free $\mathbb E_\infty$-ring} $\op{Free}_{\mathbb E_\infty}X$ to be
	\[\op{Free}_{\mathbb E_\infty}X\coloneqq\mathbb S\oplus X\oplus(X\otimes_{\mathbb S}X)_{h\Sigma_2}\oplus\cdots.\]
	The operation is induced by the construction.
\end{definition}
\begin{remark}
	The free rings have a grading, but the general ones do not.
\end{remark}
\begin{remark}
	There is an analogous notion of a free $\mathbb E_\infty$-space of a space $X$, which is
	\[\op{Free}_{\mathbb E_\infty}X\coloneqq\mathrm{pt}\sqcup X\sqcup\left(X\times X\right)_{h\Sigma_2}\sqcup\cdots.\]
	Notably, $\Sigma^\infty_+$ preserves the symmetric monoidal structure and is a left adjoint, so one knows by pure nonsense that
	\[\op{Free}_{\mathbb E_\infty}\Sigma^\infty_+X=\Sigma_+^\infty\op{Free}_{\mathbb E_\infty}X.\]
\end{remark}
Recall that $\mathbb E_\infty$-spaces are interesting because they allow us to access many spectra.
\begin{remark}
	If $X$ is an $\mathbb E_\infty$-space, then $\pi_0X$ is a commutative monoid.
\end{remark}
\begin{definition}[group-like]
	An $\mathbb E_\infty$-space is \textit{group-like} if and only if $\pi_0X$ is a group.
\end{definition}
\begin{remark}
	The embedding from group-like $\mathbb E_\infty$-spaces to $\mathbb E_\infty$-spaces admits a right adjoint $\mathrm B\Omega$.
\end{remark}
\begin{theorem}[May] \label{thm:may-space-recog}
	Consider the category of group-like $\mathbb E_\infty$-spaces.
	\begin{listalph}
		\item There is a fully faithful embedding from the category of group-like $\mathbb E_\infty$-spaces to the category of spectra.
		\item Its essential image consists of those spectra with no homotopy groups in negative degrees.
		\item The inverse functor to the fully faithful embedding is $\Omega^\infty$.
	\end{listalph}
\end{theorem}
Note that we currently have two ways to construct an $\mathbb E_\infty$-space: there is the free construction, but we could also take $\Omega^\infty\Sigma^\infty_+$ because $\Omega^\infty$ automatically outputs $\mathbb E_\infty$-spaces. These constructions are related, but a modification is required because $\Omega^\infty$ outputs group-like $\mathbb E_\infty$-spaces.
\begin{theorem} \label{thm:may-free}
	For any space $X$, there is an isomorphism
	\[\Omega\mathrm B\op{Free}_{\mathbb E_\infty}X\cong\Omega^\infty\Sigma_+^\infty X\]
	of group-like $\mathbb E_\infty$-spaces.
\end{theorem}
\begin{proof}
	We use the Yoneda lemma: for any group-like $\mathbb E_\infty$-space $A$, which we may naturally view as a spectrum, we see
	\begin{align*}
		\op{Hom}_{\mathbb E_\infty}(\Omega^\infty\Sigma^\infty_+X,A) &= \op{Hom}_{\mathbb S}(\Sigma_+^\infty X,A) \\
		&= \op{Hom}_{\mathrm{Spaces}}(X,A) \\
		&= \op{Hom}_{\mathbb E_\infty}(\op{Free}_{\mathbb E_\infty}X,A) \\
		&= \op{Hom}_{\mathbb E_\infty}(\Omega\mathrm B\op{Free}_{\mathbb E_\infty}X,A),
	\end{align*}
	where the last equality holds because $\Omega^\infty A$ is group-like.
\end{proof}
\begin{example}
	Note that $\Omega\mathrm B\op{Free}_{\mathbb E_\infty}\mathrm{pt}=\Omega^\infty\Sigma^\infty_+\mathrm{pt}$, which is $\Omega^\infty\mathbb S$. On the other hand, $\op{Free}_{\mathbb E_\infty}\mathrm{pt}$ by definition(!) is
	\[\mathrm{pt}\sqcup\mathrm{pt}\sqcup(\mathrm{pt}\times\mathrm{pt})_{h\Sigma_2}\sqcup\cdots,\]
	which is $\mathrm B\Sigma_0\sqcup\mathrm B\Sigma_1\sqcup\mathrm B\Sigma_2\sqcup\cdots$, also known as the category of finite sets up to isomorphism. Thus, we see that the ``group completion'' of the category of finite sets (up to isomorphism) is $\Omega^\infty\mathbb S$, indicating that the group completion of the category of finite sets ought to be $\mathbb S$.
\end{example}
We can also take some free $\mathbb E_\infty$-spaces of pointed spaces.
\begin{definition}
	Fix a pointed space $X$. Then $\op{Free}_{\mathbb E_\infty,*}X$ is the pushout of the following diagram.
	% https://q.uiver.app/#q=WzAsNCxbMCwwLCJcXG9we0ZyZWV9X3tcXG1hdGhiYiBFX1xcaW5mdHl9XFxtYXRocm17cHR9Il0sWzEsMCwiXFxvcHtGcmVlfV97XFxtYXRoYmIgRV9cXGluZnR5fVgiXSxbMCwxLCJcXG1hdGhybXtwdH0iXSxbMSwxLCJcXG9we0ZyZWV9X3tcXG1hdGhiYiBFX1xcaW5mdHksKn1YIl0sWzAsMl0sWzAsMV0sWzIsM10sWzEsM10sWzMsMCwiIiwxLHsic3R5bGUiOnsibmFtZSI6ImNvcm5lciJ9fV1d&macro_url=https%3A%2F%2Fraw.githubusercontent.com%2FdFoiler%2Fnotes%2Fmaster%2Fnir.tex
	\[\begin{tikzcd}[cramped]
		{\op{Free}_{\mathbb E_\infty}\mathrm{pt}} & {\op{Free}_{\mathbb E_\infty}X} \\
		{\mathrm{pt}} & {\op{Free}_{\mathbb E_\infty,*}X}
		\arrow[from=1-1, to=1-2]
		\arrow[from=1-1, to=2-1]
		\arrow[from=1-2, to=2-2]
		\arrow[from=2-1, to=2-2]
		\arrow["\lrcorner"{anchor=center, pos=0.125, rotate=180}, draw=none, from=2-2, to=1-1]
	\end{tikzcd}\]
\end{definition}
\begin{theorem}[May] \label{thm:may-pointed-free}
	If $X$ is a pointed space, then
	\[\Omega\mathrm B\op{Free}_{\mathbb E_\infty,*}X=\Omega^\infty\Sigma^\infty X.\]
\end{theorem}
\begin{example}
	If $X$ is a connected pointed space, then one can compute that $\op{Free}_{\mathbb E_\infty,*}X$ is connected (intuitively, the connected components of $\op{Free}_{\mathbb E_\infty}X$ have collapsed). Thus, $\op{Free}_{\mathbb E_\infty,*}X$ is already group-like (it's $\pi_0$ is just a point!), so $\op{Free}_{\mathbb E_\infty,*}X=\Omega^\infty\Sigma^\infty X$. Thus, we may view $\Omega^\infty\Sigma^\infty X$ as the ``group completion'' of $X$.
\end{example}
Here is an application of our ``free'' constructions which does not use the word ``free.''
\begin{theorem}[Snaith, Jones] \label{thm:snaith-jones}
	Fix a connected pointed space $X$. Then
	\[\Sigma_+^\infty\Omega^\infty\Sigma^\infty X=\mathbb S\oplus\Sigma^\infty X\oplus\left(\Sigma^\infty X\right)^{\otimes2}_{h\Sigma_2}\oplus\cdots.\]
\end{theorem}
\begin{proof}
	Recall that we have a pushout
	% https://q.uiver.app/#q=WzAsNCxbMCwwLCJcXG9we0ZyZWV9X3tcXG1hdGhiYiBFX1xcaW5mdHl9XFxtYXRocm17cHR9Il0sWzEsMCwiXFxvcHtGcmVlfV97XFxtYXRoYmIgRV9cXGluZnR5fVgiXSxbMCwxLCJcXG1hdGhybXtwdH0iXSxbMSwxLCJcXG9we0ZyZWV9X3tcXG1hdGhiYiBFX1xcaW5mdHksKn1YIl0sWzAsMl0sWzAsMV0sWzIsM10sWzEsM10sWzMsMCwiIiwxLHsic3R5bGUiOnsibmFtZSI6ImNvcm5lciJ9fV1d&macro_url=https%3A%2F%2Fraw.githubusercontent.com%2FdFoiler%2Fnotes%2Fmaster%2Fnir.tex
	\[\begin{tikzcd}[cramped]
		{\op{Free}_{\mathbb E_\infty}\mathrm{pt}} & {\op{Free}_{\mathbb E_\infty}X} \\
		{\mathrm{pt}} & {\op{Free}_{\mathbb E_\infty,*}X}
		\arrow[from=1-1, to=1-2]
		\arrow[from=1-1, to=2-1]
		\arrow[from=1-2, to=2-2]
		\arrow[from=2-1, to=2-2]
		\arrow["\lrcorner"{anchor=center, pos=0.125, rotate=180}, draw=none, from=2-2, to=1-1]
	\end{tikzcd}\]
	of $\mathbb E_\infty$-spaces. Note that the bottom-right is $\Omega^\infty\Sigma^\infty X$ by \Cref{thm:may-pointed-free}. Hitting this with the left adjoint $\Sigma_+^\infty$ produces a pushout
	% https://q.uiver.app/#q=WzAsNCxbMCwwLCJcXG9we0ZyZWV9X3tcXG1hdGhiYiBFX1xcaW5mdHl9XFxtYXRoYmIgUyJdLFswLDEsIlxcbWF0aGJiIFMiXSxbMSwwLCJcXG9we0ZyZWV9X3tcXG1hdGhiYiBFX1xcaW5mdHl9XFxTaWdtYV8rXlxcaW5mdHkgWCJdLFsxLDEsIlxcU2lnbWFfK15cXGluZnR5XFxPbWVnYV5cXGluZnR5XFxTaWdtYV5cXGluZnR5IFgiXSxbMSwzXSxbMCwxXSxbMCwyXSxbMiwzXSxbMywwLCIiLDEseyJzdHlsZSI6eyJuYW1lIjoiY29ybmVyIn19XV0=&macro_url=https%3A%2F%2Fraw.githubusercontent.com%2FdFoiler%2Fnotes%2Fmaster%2Fnir.tex
	\[\begin{tikzcd}[cramped]
		{\op{Free}_{\mathbb E_\infty}\mathbb S} & {\op{Free}_{\mathbb E_\infty}\Sigma_+^\infty X} \\
		{\mathbb S} & {\Sigma_+^\infty\Omega^\infty\Sigma^\infty X}
		\arrow[from=1-1, to=1-2]
		\arrow[from=1-1, to=2-1]
		\arrow[from=1-2, to=2-2]
		\arrow[from=2-1, to=2-2]
		\arrow["\lrcorner"{anchor=center, pos=0.125, rotate=180}, draw=none, from=2-2, to=1-1]
	\end{tikzcd}\]
	of $\mathbb E_\infty$-rings, where the top-right is has $\Sigma_+^\infty X=\Sigma^\infty X\oplus\mathbb S$ and so
	\begin{align*}
		\op{Free}_{\mathbb E_\infty}\Sigma_+^\infty X &= \op{Free}_{\mathbb E_\infty}(\Sigma^\infty X\oplus\mathbb S) \\
		&= \op{Free}_{\mathbb E_\infty}\Sigma^\infty X\otimes_{\mathbb S}\op{Free}_{\mathbb E_\infty}\mathbb S \\
		&= \op{Free}_{\mathbb E_\infty}\Sigma^\infty X.
	\end{align*}
	The result now follows by computing our pushout.
\end{proof}
\begin{remark}
	This is a remarkable theorem! Basically, it tells us that taking $\Sigma_+^\infty$ of some group-like $\mathbb E_\infty$-space should have homology which splits in a very natural way.
\end{remark}
\begin{remark}
	It turns out that $\Sigma_+^\infty\Omega^\infty\mathbb S=\op{Free}_{\mathbb E_\infty}\mathbb S\left[x^{-1}\right]$ for some class $x$ arising from the class of a point (in degree one). Note that \Cref{thm:snaith-jones} does not apply here because the space $S^0$ is not connected! The proof of this statement amounts to ``forcing'' $\pi_0$ to be a group.
\end{remark}

\subsection{The Transfer}
Here is a more complicated free example.
\begin{example}
	Consider the category of finite $\Sigma_2$-sets, considered up to isomorphism. Such a set is a union of some points and some two-element sets with a swapping action. Enumerating all such finite sets, we find that this category is $\op{Free}_{\mathbb E_\infty}(\mathrm{pt}\sqcup\mathrm B\Sigma_2)$. By \Cref{thm:may-free}, we thus see that
	\[\Omega\mathrm B\op{Free}_{\mathbb E_\infty}(\mathrm{pt}\sqcup\mathrm B\Sigma_2)=\Omega^\infty\left(\mathbb S\oplus\Sigma^\infty_+\mathrm B\Sigma_2\right).\]
\end{example}
Continuing, the example, we note that there is a forgetful functor $\mathrm{FinSet}(\Sigma_2)\to\mathrm{FinSet}$ which descends to isomorphism classes. By taking group completion, one receives a map
\[\Sigma^\infty_+\mathrm B\Sigma_2\oplus\mathbb S\to\mathbb S,\]
so there is a ``transfer'' map $\tr\colon\Sigma^\infty_+\mathrm B\Sigma_2\to\mathbb S$. (Note that we have silently removed our $\Omega^\infty$, which we can do because the forgetful functor preserves disjoint unions, so $\Omega^\infty$ can be removed by \Cref{thm:may-space-recog} because these things are group-like.)
\begin{remark}
	The canonical map $\RP^\infty\to\mathrm{pt}$ produces a different map $\varepsilon\colon\Sigma^\infty_+\mathrm B\Sigma_2\to\mathbb S$.
\end{remark}
\begin{remark}[Siegel's conjecture]
	It turns out that all maps $\Sigma^\infty_+\mathrm B\Sigma_2\to\mathbb S$ can be obtained from $\tr$ and $\varepsilon$.
\end{remark}
\begin{theorem}[Kahn--Priddy]
	The map $\pi_*\tr\colon\pi_*\Sigma_+^\infty\mathrm B\Sigma_2\to\pi_*\mathbb S$ is surjective in positive degrees.
\end{theorem}

\end{document}