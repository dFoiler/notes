% !TEX root = ../notes.tex

\documentclass[../notes.tex]{subfiles}

\begin{document}

\section{February 2}
Here we go.

\subsection{Idempotent Algebras}
The goal of this class is to understand some topics related to the chromatic splitting conjecture. Thus, the first half of the class will try to understand the statement, and the second half of the class will explain how it relates to other problems in algebra.
\begin{warn}
	All categories in this course are $\infty$-categories.
\end{warn}
\begin{example}
	Given a ring $R$, we have a stable, symmetric monoidal $\infty$-category $D(R)$ of chain complexes of $R$-modules, considered up to quasi-isomorphism. Notably, the symmetric monoidal structure is given by the derived tensor product.
\end{example}
We begin our story with idempotent algebras.
\begin{definition}[idempotent algebra]
	Fix a ring $R$. An \textit{idempotent algebra} is an object $E\in D(R)$ equip\-ped with a unit map $R\to E$ such that the composite
	\[E=E\otimes_RR\to E\otimes_RE\]
	is an equivalence.
\end{definition}
\begin{remark}
	Such an object $E$ grants $E$ a multiplication structure $E\otimes_RE\to E$, and $E$ gains the structure of a differentially graded algebra.
\end{remark}
\begin{example}
	Consider $R=\ZZ$. Then for each prime $p$, the algebra $\ZZ_{(p)}$ is idempotent: localizing $\ZZ_{(p)}$ further at $(p)$ does nothing!
\end{example}
\begin{nex}
	The $\ZZ$-algebra $\FF_p$ is not idempotent because the tensor product we are considering is derived. Indeed, we computed $\FF_p\otimes_\ZZ\FF_p$ last semester.
\end{nex}
Here is a quick reason why one might care about idempotent algebras.
\begin{theorem}[Neeman] \label{thm:idemp-as-spec}
	Fix a Noetherian ring $R$. Then the lattice of idempotent algebras is equivalent to the data of $\Spec R$ as a topological space.
\end{theorem}
\begin{example}
	For $R=\ZZ$, it turns out that the idempotent algebras are either $\ZZ_{(p)}$ or $\QQ$, and the maps between them look like the specializations of $\Spec R$.
\end{example}
Of course, we are homotopy theorists, so we have less reason to care about $\ZZ$. Recall that $\ZZ$ is obtained from $\NN$ by formally adding inverses. But $\NN$ is basically isomorphism classes of $\mathrm{FinSet}$; if we had instead formally added inverses directly to $\mathrm{FinSet}$ (instead of taking isomorphism classes first), we would have found the sphere spectrum $\mathbb S$. In particular, we will be interested in the category $D(\mathbb S)$ of $\mathbb S$-modules, also called spectra.

We now no longer have access to algebraic geometry directly on $\mathbb S$. Instead, \Cref{thm:idemp-as-spec} motivates us to look for the idempotent algebras for $\mathbb S$.
\begin{remark}
	For any $x\in\pi_*\mathbb S$, there is an idempotent algebra $\mathbb S\left[x^{-1}\right]$. For example, $\pi_0\mathbb S=\ZZ$, so there is an idempotent algebra $\mathbb S_{(p)}$.
\end{remark}
Here is our first main theorem.
\begin{theorem}[Nishida] \label{thm:nishida-nilpotence}
	Fix some $x\in\pi_*\mathbb S$ of positive degree. Then $x$ is nilpotent.
\end{theorem}
Thus, the idempotent algebras $\mathbb S\left[x^{-1}\right]$ do not look genuinely ``new.'' To get other idempotent algebras, we need more tools.

\subsection{The Adams--Novikov Spectral Sequence}
Recall the $\mathbb S$-algebra $\mathrm{MU}$ defined as the colimit of the embedding $\mathrm{BU}\to\mathrm{BLG}(\mathbb S)\subseteq\mathrm{Mod}(\mathbb S)$. Let's compute its homotopy groups.
\begin{definition}[formal group law]
	Fix a commutative ring $R$. Then a \textit{commutative formal group law} over $R$ is a power series $f(x,y)\in R[[x,y]]$ satisfying
	\begin{listalph}
		\item $f(x,0)=x$ and $f(0,y)=y$,
		\item $f(x,y)=f(y,x)$, and
		\item $f(x,f(y,z))=f(x,f(y,z))$.
	\end{listalph}
\end{definition}
\begin{definition}[Lazard ring]
	The \textit{Lazard ring} is the ring $L$ which is exactly the quotient of $\ZZ[\{a_{ij}\}_{ij}]$ by the relations dictating that
	\[f(x,y)=\sum_{i,j\ge0}a_{ij}x^iy^j\]
	is a commutative formal group law.
\end{definition}
\begin{remark}
	In other words, $L$ represents the collection formal group laws, in the sense that the data of a formal group law for a ring $R$ amounts to the data of a ring homomorphism $L\to R$.
\end{remark}
\begin{remark}
	By definition, there is a ``universal'' formal group law $f_L$ in $L$ given exactly by
	\[f_L(x,y)=\sum_{i,j\ge0}a_{ij}x^iy^j.\]
\end{remark}
\begin{theorem}[Quillen]
	The ring $\pi_*\mathrm{MU}$ is exactly the Lazard ring.
\end{theorem}
\begin{remark}
	Quillen also computed $\pi_*(\mathrm{MU}\otimes_{\mathbb S}\mathrm{MU})$ as well as the two natural maps $\pi_*\mathrm{MU}\to\pi_*(\mathrm{MU}\otimes_{\mathbb S}\mathrm{MU})$. It turns out that this is more or less related to some notion of isomorphism of the formal group laws.
\end{remark}
The use of $\mathrm{MU}$ is that it produces a spectral sequence with which we can understand $\pi_*\mathbb S$. By \v{C}ech descent along the map $\mathbb S\to\mathrm{MU}$, we see that $\mathbb S$ is the limit of the diagram
% https://q.uiver.app/#q=WzAsNCxbMCwwLCJcXG1hdGhybXtNVX0iXSxbMSwwLCJcXG1hdGhybXtNVX1cXG90aW1lc197XFxtYXRoYmIgU31cXG1hdGhybXtNVX0iXSxbMiwwLCJcXG1hdGhybXtNVX1cXG90aW1lc197XFxtYXRoYmIgU31cXG1hdGhybXtNVX1cXG90aW1lc197XFxtYXRoYmIgU31cXG1hdGhybXtNVX0iXSxbMywwLCJcXGNkb3RzIl0sWzEsMF0sWzAsMSwiIiwwLHsib2Zmc2V0IjotMX1dLFswLDEsIiIsMSx7Im9mZnNldCI6MX1dLFsyLDFdLFsxLDIsIiIsMSx7Im9mZnNldCI6LTF9XSxbMiwxLCIiLDEseyJvZmZzZXQiOi0yfV0sWzEsMiwiIiwxLHsib2Zmc2V0IjoxfV0sWzIsMSwiIiwxLHsib2Zmc2V0IjoyfV1d&macro_url=https%3A%2F%2Fraw.githubusercontent.com%2FdFoiler%2Fnotes%2Fmaster%2Fnir.tex
\[\begin{tikzcd}[cramped]
	{\mathrm{MU}} & {\mathrm{MU}\otimes_{\mathbb S}\mathrm{MU}} & {\mathrm{MU}\otimes_{\mathbb S}\mathrm{MU}\otimes_{\mathbb S}\mathrm{MU}} & \cdots
	\arrow[shift left, from=1-1, to=1-2]
	\arrow[shift right, from=1-1, to=1-2]
	\arrow[from=1-2, to=1-1]
	\arrow[shift left, from=1-2, to=1-3]
	\arrow[shift right, from=1-2, to=1-3]
	\arrow[from=1-3, to=1-2]
	\arrow[shift left=2, from=1-3, to=1-2]
	\arrow[shift right=2, from=1-3, to=1-2]
\end{tikzcd}\]
which we can then truncate as $\op{fil}^n\mathbb S$ in order to get a descending filtration to $\op{fil}^0\mathbb S$. Computing homotopy along this filtration produces the desired spectral sequence, as soon as we compute homotopy groups of the various tensor powers of the $\mathrm{MU}$s and so on.
\begin{theorem}[Adams--Novikov]
	Let $\mc M_{\mathrm{fg}}$ be the moduli space of formal groups. Then there is a spectral sequence
	\[E_2=\mathrm H^s(\mc M_{\mathrm{fg}};\omega^{\otimes t})\Rightarrow\pi_{2t-s}\mathbb S.\]
\end{theorem}
\begin{remark}
	It turns out that the spectral sequence is concentrated in the region $s\le 2t-s$.
\end{remark}
\begin{example}
	Along the line $s=2t-s$, there is some $h_1$ at $(s,2t-s)=(1,1)$, and then we can take powers of it to go up the line. It turns out that $h_1$ survives the spectral sequence, and it goes to the ``Hopf map'' $\eta\in\pi_1\mathbb S$; however, $\eta^4=0$, though the Adams--Novikov spectral sequence cannot see it!
\end{example}
Thus, we see that the Adams--Novikov spectral sequence is not an amazing approximation: the $E_2$ page sees many classes which we know abstractly must vanish! Life is better if we pass to $E_\infty$ instead; the following is our first main theorem.
\begin{theorem}[Devinatz--Hopkins--Smith]
	The $E_\infty$ page of the Adams--Novikov spectral sequence lies under a curve which grows more slowly than any line.
\end{theorem}
Note that this immediately implies \Cref{thm:nishida-nilpotence}. On the other hand, we will see that the topological input of \Cref{thm:nishida-nilpotence} plus some algebraic facts about formal group laws will prove the above big theorem.
\begin{remark}
	The curve is known to be faster than logarithmic, but not much else is known. Our proof will not help us much because our proof of \Cref{thm:nishida-nilpotence} will be ineffective.
\end{remark}

\subsection{Back to Idempotent Algebras}
Let's return to trying to find some idempotent algebras.
\begin{notation}
	Define the power series $[n]\in L[[x]]$ to be adding with $f$ a total of $n$ times.
\end{notation}
\begin{example}
	We see that $[2](x)=f(x,x)$ and $[5](x)=f(f(f(f(x,x),x),x),x)$.
\end{example}
\begin{notation}
	Fix a prime $p$. Then we define the class $v_n\in\pi_*\mathrm{MU}$ to be the coefficient of $x^{p^n}$ in the power series $[p](x)$.
\end{notation}
Now, because localization is exact, we see that $\mathbb S_{(p)}$ is the limit of the nerve
% https://q.uiver.app/#q=WzAsNCxbMCwwLCJcXG1hdGhybXtNVX1feyhwKX0iXSxbMSwwLCJcXG1hdGhybXtNVX1feyhwKX1cXG90aW1lc197XFxtYXRoYmIgU31cXG1hdGhybXtNVX1feyhwKX0iXSxbMiwwLCJcXG1hdGhybXtNVX1feyhwKX1cXG90aW1lc197XFxtYXRoYmIgU31cXG1hdGhybXtNVX1feyhwKX1cXG90aW1lc197XFxtYXRoYmIgU31cXG1hdGhybXtNVX1feyhwKX0iXSxbMywwLCJcXGNkb3RzIl0sWzEsMF0sWzAsMSwiIiwwLHsib2Zmc2V0IjotMX1dLFswLDEsIiIsMSx7Im9mZnNldCI6MX1dLFsyLDFdLFsxLDIsIiIsMSx7Im9mZnNldCI6LTF9XSxbMiwxLCIiLDEseyJvZmZzZXQiOi0yfV0sWzEsMiwiIiwxLHsib2Zmc2V0IjoxfV0sWzIsMSwiIiwxLHsib2Zmc2V0IjoyfV1d&macro_url=https%3A%2F%2Fraw.githubusercontent.com%2FdFoiler%2Fnotes%2Fmaster%2Fnir.tex
\[\begin{tikzcd}[cramped]
	{\mathrm{MU}_{(p)}} & {\mathrm{MU}_{(p)}\otimes_{\mathbb S}\mathrm{MU}_{(p)}} & {\mathrm{MU}_{(p)}\otimes_{\mathbb S}\mathrm{MU}_{(p)}\otimes_{\mathbb S}\mathrm{MU}_{(p)}} & \cdots
	\arrow[shift left, from=1-1, to=1-2]
	\arrow[shift right, from=1-1, to=1-2]
	\arrow[from=1-2, to=1-1]
	\arrow[shift left, from=1-2, to=1-3]
	\arrow[shift right, from=1-2, to=1-3]
	\arrow[from=1-3, to=1-2]
	\arrow[shift left=2, from=1-3, to=1-2]
	\arrow[shift right=2, from=1-3, to=1-2]
\end{tikzcd}\]
so it is not unreasonable to consider the following limit.
\begin{notation}
	Fix a prime $p$ and some $n\ge0$. Then we define $L_n\mathbb S_{(p)}$ as the limit of the following diagram.
	% https://q.uiver.app/#q=WzAsNCxbMCwwLCJcXG1hdGhybXtNVX1feyhwKX1cXGxlZnRbdl9uXnstMX1cXHJpZ2h0XSJdLFsxLDAsIlxcbWF0aHJte01VfV97KHApfVxcbGVmdFt2X25eey0xfVxccmlnaHRdXFxvdGltZXNfe1xcbWF0aGJiIFN9XFxtYXRocm17TVV9X3socCl9XFxsZWZ0W3Zfbl57LTF9XFxyaWdodF0iXSxbMiwwLCJcXG1hdGhybXtNVX1feyhwKX1cXGxlZnRbdl9uXnstMX1cXHJpZ2h0XVxcb3RpbWVzX3tcXG1hdGhiYiBTfVxcbWF0aHJte01VfV97KHApfVxcbGVmdFt2X25eey0xfVxccmlnaHRdXFxvdGltZXNfe1xcbWF0aGJiIFN9XFxtYXRocm17TVV9X3socCl9XFxsZWZ0W3Zfbl57LTF9XFxyaWdodF0iXSxbMywwLCJcXGNkb3RzIl0sWzEsMF0sWzAsMSwiIiwwLHsib2Zmc2V0IjotMX1dLFswLDEsIiIsMSx7Im9mZnNldCI6MX1dLFsyLDFdLFsxLDIsIiIsMSx7Im9mZnNldCI6LTF9XSxbMiwxLCIiLDEseyJvZmZzZXQiOi0yfV0sWzEsMiwiIiwxLHsib2Zmc2V0IjoxfV0sWzIsMSwiIiwxLHsib2Zmc2V0IjoyfV1d&macro_url=https%3A%2F%2Fraw.githubusercontent.com%2FdFoiler%2Fnotes%2Fmaster%2Fnir.tex
	\[\begin{tikzcd}[cramped]
		{\mathrm{MU}_{(p)}\left[v_n^{-1}\right]} & {\mathrm{MU}_{(p)}\left[v_n^{-1}\right]\otimes_{\mathbb S}\mathrm{MU}_{(p)}\left[v_n^{-1}\right]} & {\mathrm{MU}_{(p)}\left[v_n^{-1}\right]\otimes_{\mathbb S}\mathrm{MU}_{(p)}\left[v_n^{-1}\right]\otimes_{\mathbb S}\mathrm{MU}_{(p)}\left[v_n^{-1}\right]}
		\arrow[shift left, from=1-1, to=1-2]
		\arrow[shift right, from=1-1, to=1-2]
		\arrow[from=1-2, to=1-1]
		\arrow[shift left, from=1-2, to=1-3]
		\arrow[shift right, from=1-2, to=1-3]
		\arrow[from=1-3, to=1-2]
		\arrow[shift left=2, from=1-3, to=1-2]
		\arrow[shift right=2, from=1-3, to=1-2]
	\end{tikzcd}\]
	We may abbreviate $L_n\mathbb S_{(p)}$ to $L_n\mathbb S$ if there is no possibility of confusion.
\end{notation}
\begin{remark}
	It turns out that there are natural maps $L_{n+1}\mathbb S\to L_n\mathbb S$.
\end{remark}
These spectra $L_n\mathbb S$ give us new idempotent algebras, more or less granting us further understanding of the ``spectrum'' of $\mathbb S$.
\begin{theorem}[Hopkins--Ravenel]
	Fix a prime $p$ and some $n\ge0$. Then $L_n\mathbb S_{(p)}$ is an idempotent algebra.
\end{theorem}
\begin{remark}
	Ravenel has conjectured that if $E$ is a nonzero idempotent algebra under $\mathbb S_{(p)}$, then $E$ is either $\QQ$ or one of the $L_n\mathbb S$s. This was recently disproved. It is current work to attempt a classification.
\end{remark}
Nonetheless, $\mathbb S_{(p)}$ can be understood well from the $L_n\mathbb S$s.
\begin{theorem}[Hopkins--Ravenel]
	Fix a prime $p$. Then $\mathbb S_{(p)}$ is the limit of the diagram
	\[\cdots\to L_3\mathbb S_{(p)}\to L_2\mathbb S_{(p)}\to L_1\mathbb S_{(p)}.\]
\end{theorem}

\subsection{Completion}
Continue with our fixed prime $p$. For motivation, we return to abelian groups.
\begin{remark}
	For any $M\in D(\ZZ)$, the $p$-localization sits in a pullback square
	% https://q.uiver.app/#q=WzAsNCxbMCwwLCJNX3socCl9Il0sWzEsMCwiTV9wXlxcbGFuZCJdLFsxLDEsIk1fcF5cXGxhbmRcXG90aW1lc19cXFpaXFxRUSJdLFswLDEsIk1cXG90aW1lc19cXFpaXFxRUSJdLFswLDNdLFszLDJdLFswLDFdLFsxLDJdXQ==&macro_url=https%3A%2F%2Fraw.githubusercontent.com%2FdFoiler%2Fnotes%2Fmaster%2Fnir.tex
	\[\begin{tikzcd}[cramped]
		{M_{(p)}} & {M_p^\land} \\
		{M\otimes_\ZZ\QQ} & {M_p^\land\otimes_\ZZ\QQ}
		\arrow[from=1-1, to=1-2]
		\arrow[from=1-1, to=2-1]
		\arrow[from=1-2, to=2-2]
		\arrow[from=2-1, to=2-2]
	\end{tikzcd}\]
	more or less corresponding to finding the ``lattice'' $\ZZ_{(p)}\subseteq\QQ$.
\end{remark}
Analogously, there is a completion of $L_n\mathbb S$ which fits into a diagram
% https://q.uiver.app/#q=WzAsNCxbMCwwLCJMX25FIl0sWzAsMSwiTF97bi0xfUUiXSxbMSwwLCJMX3tLKG4pfUUiXSxbMSwxLCJMX3tuLTF9TF97SyhuKX1FIl0sWzAsMV0sWzEsM10sWzAsMl0sWzIsM11d&macro_url=https%3A%2F%2Fraw.githubusercontent.com%2FdFoiler%2Fnotes%2Fmaster%2Fnir.tex
\begin{equation}
	\begin{tikzcd}[cramped]
		{L_nE} & {L_{K(n)}E} \\
		{L_{n-1}E} & {L_{n-1}L_{K(n)}E}
		\arrow[from=1-1, to=1-2]
		\arrow[from=1-1, to=2-1]
		\arrow[from=1-2, to=2-2]
		\arrow[from=2-1, to=2-2]
	\end{tikzcd} \label{eq:l-k-n-completion-square}
\end{equation}
where $L_nE\coloneqq L_n\mathbb S\otimes_{\mathbb S}E$. We are now ready to state the chromatic splitting conjecture.
\begin{conj}[Chromatic splitting] \label{conj:chromatic-splitting}
	For any $n\ge2$, the inclusion
	\[L_{K(n)}\mathbb S\to L_{n-1}L_{K(n)}\mathbb S\]
	is an inclusion of a direct summand.
\end{conj}
\begin{remark}
	This implies that the natural map
	\[\mathbb S_p^\land\to\prod_{n\ge1}L_{K(n)}\mathbb S\]
	is the inclusion of a direct summand. The point is that the squares \eqref{eq:l-k-n-completion-square} are rather degenerate, which would let us compute the homotopy groups of $L_n\mathbb S$ from the completions.
\end{remark}
\begin{remark}
	\Cref{conj:chromatic-splitting} is known at $n=2$ and all primes, by work of many people.
\end{remark}
The goal of the present class is to review the homotopy theory required to understand the statement of \Cref{conj:chromatic-splitting} formally, and then we will discuss why perfectoid geometry may be useful to prove it.

Let's see why passing to $L_{K(n)}\mathbb S$ is genuinely easier.
\begin{example}
	For $p>2$, we can define $L_{K(1)}\mathbb S$ as the homotopy fiber of the endomorphism $\psi^g-1$ of $\mathrm{KU}^\land_p$, where $g$ is a choice of topological generator of $\ZZ_p^\times$, and $\psi$ is some action of $\ZZ_p^\times$ on $\mathrm{KU}^\land_p$.
\end{example}
\begin{theorem}[Goerss--Hopkins--Miller, Rogres]
	Fix a prime $p$. For each $n\ge1$, there is an $\mathbb S$-algebra $E_n$ and a profinite group $\mathbb G_n$ for which
	\[L_{K(n)}\mathbb S=(E_n)^{\mathbb G_n}.\]
	In fact, $E_n$ is a Galois extension of $L_{K(n)}\mathbb S$.
\end{theorem}
\begin{remark}
	We will only be able to keep track of this sort of ``infinite Galois theory'' with condensed mathematics.
\end{remark}
\begin{remark}
	The profinite group $\mathbb G_n$ is some subgroup of automorphisms of formal group laws.
\end{remark}
\begin{remark} \label{rem:galois-descent-completed-homotopy}
	For any spectrum $X$, there is some ``Galois descent''
	\[L_{K(n)}X=\left(L_{K(n)}(E_n\otimes X)\right)^{\mathbb G_n}.\]
	This generalizes to a spectral sequence
	\[\mathrm H^*_{\mathrm{cts}}(\mathbb G_n;\pi_*(L_{K(n)}(E_n\otimes X)))\Rightarrow\pi_*L_{K(n)}X.\]
\end{remark}
The previous remark produces a spectral sequence
\[\mathrm H^*_{\mathrm{cts}}(\mathbb G_n;\pi_*E_n)\Rightarrow\pi_*L_{K(n)}\mathbb S.\]
If $p$ is large compared to $n$, then it turns out that the spectral sequence collapses for degree reasons, so we are reduced to a pure algebra problem.

The end of the course will be interested $L_{K(n-1)}L_{K(n)}\mathbb S_{(p)}$ for general $n$ but $p$ very large. \Cref{conj:chromatic-splitting} tells us that this should be fairly easy to understand, so we can view the end of the course as trying to provide some evidence for the conjecture. For example, work in progress by many people has recently culminated in the following strategy.
\begin{notation}
	Fix $\mathbb B\coloneqq E_{n-1}\otimes_{\mathbb S} L_{K(n-1)}E_n$.
\end{notation}
\begin{remark}
	It turns out that $\mathbb B$ is Galois over $L_{K(n-1)}L_{K(n)}\mathbb S$ with Galois group $\mathbb G_{n-1}\times\mathbb G_n$. Thus, we can hope to be able to use some Galois descent spectral sequence to understand $L_{K(n-1)}L_{K(n)}\mathbb S$, as in \Cref{rem:galois-descent-completed-homotopy}.
\end{remark}
Now, $\pi_*\mathbb B$ is a local ring, so one becomes motivated to consider a perfection $\widehat{\mathbb B}$. In particular, it turns out that there is a $\mathbb G_n\times\mathbb G_{n-1}$-equivariant map $\mathbb B\to\widehat{\mathbb B}$, so taking fixed points produces a map out of $L_{K(n-1)}L_{K(n)}\mathbb S$. This is the sort of thing that \Cref{conj:chromatic-splitting} asks us to do! Of course, the target is related to the perfection $\widehat{\mathbb B}$, which we now want to understand.
\begin{theorem}
	The groups $\mathrm H^*_{\mathrm{cts}}(\mathbb G_n\times\mathbb G_{n-1};\pi_*\widehat{\mathbb B})$ is the same as the cohomology of the structure sheaf of some diamond related to the Fargues--Fontaine curve.
\end{theorem}
Let's explain the application to \Cref{conj:chromatic-splitting}: this calculation tells us that $(\widehat{\mathbb B})^{\mathbb G_n\times\mathbb G_{n-1}}$ is $L_{K(n-1)}\mathbb S\oplus\Sigma L_{K(n-1)}\mathbb S$, from which our small piece of \Cref{conj:chromatic-splitting} follows!

\end{document}