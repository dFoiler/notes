% !TEX root = ../notes.tex

\documentclass[../notes.tex]{subfiles}

\begin{document}

\section{February 9}
Here we go.
\begin{remark}
	The functor $\Sigma_+^\infty$ taking $\mathbb E_\infty$-spaces to $\mathbb E_\infty$-rings admits a left adjoint known as $\Omega^\infty$. Note that $\Omega^\infty R$ is merely a (multiplicative) $\mathbb E_\infty$-space, which need not be grouplike.
\end{remark}

\subsection{Fun with \texorpdfstring{$\Sigma_2$}{ S2}}
Let's do a little combinatorics with $\mathrm{FinSet}(\Sigma_2)$.
\begin{example}
	There is a single point set, which has automorphism group $\Sigma_2$.
\end{example}
\begin{example}
	The two-elements sets are $\mathrm{pt}\sqcup\mathrm{pt}$, whose stabilizer is $\Sigma_2$, and $\Sigma_2$, whose stabilizer is also $\Sigma_2$.
\end{example}
\begin{example}
	There is a three-element set $\mathrm{pt}^{\sqcup3}$, whose stabilizer is $\Sigma_3$. There is also the three-element set $\mathrm{pt}\sqcup\Sigma_2$, whose stabilizer is $\Sigma_2$.
\end{example}
\begin{example}
	Consider the four-element set $\Sigma_2\sqcup\Sigma_2$. Then the automorphism group has eight elements (one can swap within the $\Sigma_2$s or also swap the two copies of $\Sigma_2$), and one can draw a square with each $\Sigma_2$ on the diagonals to show that the stabilizer is $D_8$.
\end{example}
\begin{remark}
	There is a fiber sequence
	\[\mathrm BC_2\times\mathrm BC_2\to\mathrm BD_8\to\mathrm BC_2\]
	induced by the short exact sequence of groups. It follows that $\mathrm BD_8=(\mathrm BC_2\times\mathrm BC_2)_{h\Sigma_2}$. In particular, $\mathrm BD_8$ lives in $\op{Free}_{\mathbb E_\infty}(\mathrm BC_2)$.
\end{remark}
\begin{example}
	Consider the full category of finite $\Sigma_2$-sets, considered up to isomorphism. Such a set is a union of some points and some two-element sets with a swapping action. Enumerating all such finite sets, we find that this category is $\op{Free}_{\mathbb E_\infty}(\mathrm{pt}\sqcup\mathrm B\Sigma_2)$. By \Cref{thm:may-free}, we thus see that
	\[\Omega\mathrm B\op{Free}_{\mathbb E_\infty}(\mathrm{pt}\sqcup\mathrm B\Sigma_2)=\Omega^\infty\left(\mathbb S\oplus\Sigma^\infty_+\mathrm B\Sigma_2\right).\]
\end{example}
Here are some linear maps with $\mathrm{FinSet}^{\cong}$.
\begin{itemize}
	\item There is a forgetful functor $\mathrm{FinSet}(\Sigma_2)\to\mathrm{FinSet}$ which descends to isomorphism classes. It also sends disjoint unions to disjoint unions, so by taking group completion, one receives a map
	\[\Sigma^\infty_+\mathrm B\Sigma_2\oplus\mathbb S\to\mathbb S,\]
	so there is a ``transfer'' map $\tr\colon\Sigma^\infty_+\mathrm B\Sigma_2\to\mathbb S$. (Note that we have silently removed our $\Omega^\infty$, which we can do because the forgetful functor preserves disjoint unions, so $\Omega^\infty$ can be removed by \Cref{thm:may-space-recog} because these things are group-like.) It turns out that this map is given by $({\tr},1)$.
	\item There is a map $\mathrm{FinSet}\to\mathrm{FinSet}(\Sigma_2)$ giving a finite set the trivial $\Sigma_2$-action. It sends disjoint unions to disjoint unions, so one can argue as above to show that we receive a map
	\[\Omega^\infty\mathbb S\to\Omega^\infty\mathbb S\times\Omega^\infty\Sigma^\infty_+\mathrm B\Sigma_2\]
	on group completions. It is the inclusion into the first factor, as one can see by tracking around our isomorphisms.
	\item There is a map $\mathrm{FinSet}(\Sigma_2)\to\mathrm{FinSet}$ which takes the quotient by the $\Sigma_2$-action. This time, the corresponding map on group completion
	\[\Sigma^\infty_+\mathrm B\Sigma_2\oplus\mathbb S\to\mathbb S,\]
	and it takes the form $(\varepsilon,1)$.
\end{itemize}
\begin{remark}
	It turns out that the map $\varepsilon$ is an unstable map, coming from the natural projection $\RP^\infty\to\mathrm{pt}$. The map $\tr$ is stable.
\end{remark}
\begin{remark}[Siegel's conjecture]
	It turns out that all maps $\Sigma^\infty_+\mathrm B\Sigma_2\to\mathbb S$ can be obtained from $\tr$ and $\varepsilon$.
\end{remark}

\subsection{The Kahn--Priddy Theorem}
Here are some nonlinear maps.
\begin{example}[squaring] \label{ex:squaring}
	There is a squaring map $x\mapsto x^2$ on $\Omega^\infty\mathbb S$. This squares on $\pi_0$, so it is not linear. Nonetheless, it sends $0$ to $0$, so it is at least pointed if $0$ is the basepoint. We claim that it vanishes on higher homotopy groups.
\end{example}
\begin{proof}
	An element in $\pi_i\Omega^\infty\mathbb S$ amounts to the data of a map $f\colon S^i\to\Omega^\infty\mathbb S$, and we can see that squaring is the map
	\[S^i\stackrel{\Delta}\to S^i\times S^i\stackrel{(f,f)}\to\Omega^\infty\mathbb S\times\Omega^\infty\mathbb S\stackrel m\to\Omega^\infty\mathbb S.\]
	For $i>0$, it turns out that the map $S^i\to S^i\times S^i$ factors (up to homotopy) with $S^i\vee S^i$, which we can see by pinching some corners. Now, mapping out of $S^i\vee S^i$ is determined by its map on each of the factors, which we can see is nullhomotopic in the composite. It follows that the squaring map $x\mapsto x^2$ is trivial on higher homotopy groups.
\end{proof}
\begin{example}[Norm]
	There is a canonical factorization
	% https://q.uiver.app/#q=WzAsMyxbMSwwLCJcXE9tZWdhXlxcaW5mdHkoXFxTaWdtYV5cXGluZnR5XytcXG1hdGhybSBCXFxTaWdtYV8yXFxvcGx1c1xcbWF0aGJiIFMpIl0sWzEsMSwiXFxPbWVnYV5cXGluZnR5XFxtYXRoYmIgUyJdLFswLDEsIlxcT21lZ2FeXFxpbmZ0eVxcbWF0aGJiIFMiXSxbMCwxLCIoe1xcdHJ9LDEpIl0sWzIsMSwieFxcbWFwc3RvIHheMiIsMl0sWzIsMCwiXFxvcHtOb3JtfSIsMCx7InN0eWxlIjp7ImJvZHkiOnsibmFtZSI6ImRhc2hlZCJ9fX1dXQ==&macro_url=https%3A%2F%2Fraw.githubusercontent.com%2FdFoiler%2Fnotes%2Fmaster%2Fnir.tex
	\[\begin{tikzcd}[cramped]
		& {\Omega^\infty(\Sigma^\infty_+\mathrm B\Sigma_2\oplus\mathbb S)} \\
		{\Omega^\infty\mathbb S} & {\Omega^\infty\mathbb S}
		\arrow["{({\tr},1)}", from=1-2, to=2-2]
		\arrow["{\op{Norm}}", dashed, from=2-1, to=1-2]
		\arrow["{x\mapsto x^2}"', from=2-1, to=2-2]
	\end{tikzcd}\]
	which defines the norm.
\end{example}
\begin{proof}
	Combinatorially, the point is that there is a composite
	\[\mathrm{FinSet}^{\cong} \to \mathrm{FinSet}(\Sigma_2)^{\cong} \to \mathrm{FinSet}\]
	where the left map is squaring, and the right map is forgetful. However, squaring does not preserve disjoint unions, so we cannot just take group completion. However, it turns out that we can move down to the telescope (as in \Cref{rem:telescope})
	\[\ZZ\times\mathrm B\Sigma_\infty\to\op{Telescope}(\mathrm{FinSet}(\Sigma_2))\to\ZZ\times\mathrm B\Sigma_\infty,\]
	where the middle object is the colimit of $\mathrm{FinSet}(\Sigma_2)$ where we are adding in $\mathrm{pt}\sqcup\mathrm B\Sigma_2$ at all levels. We are interested in getting this composite to map to $\Omega^\infty\mathbb S$, but by the adjunction, we can instead attempt to get the composite
	\[\Sigma^\infty_+(\ZZ\times\mathrm B\Sigma_\infty)\to \Sigma_+^\infty\op{Telescope}(\mathrm{FinSet}(\Sigma_2))\to\Sigma^\infty_+(\ZZ\times\mathrm B\Sigma_\infty)\]
	to $\mathbb S$. However, the above composite is made of group completions---we are using the ``Group completion theorem,'' whose proof is some messing around with the Yoneda lemma.
\end{proof}
We will want the following result.
\begin{theorem}[Kahn--Priddy] \label{thm:kahn-priddy}
	The map $\pi_*\tr\colon\pi_*\Sigma_+^\infty\mathrm B\Sigma_2\to\pi_*\mathbb S$ is surjective in positive degrees.
\end{theorem}
\begin{proof}
	Consider the map $\Omega^\infty\mathbb S\to\Omega^\infty\Sigma^\infty_+\mathrm B\Sigma_2$ by $X\mapsto X^2\setminus X$, where we are thinking about this as coming from finite sets (mapping to free $\Sigma_2$-sets), which upgrades to the group completion (at least as a map to $\Omega^\infty\mathbb S$) by the above telescoping trick. The total composite is now $x\mapsto x^2-x$ on $\pi_0$ and so an isomorphism on higher homotopy groups because we showed earlier in \Cref{ex:squaring} that squaring vanishes on higher homotopy.
\end{proof}

\subsection{Nishida Nilpotence}
Before we jump into Nishida nilpotence, we will need the $D_2$ construction.
\begin{defihelper}[$D_2$ construction] \nirindex{D2 construction@$D_2$ construction}
	Suppose $R$ is an $\mathbb E_\infty$-ring, and choose an element $x\in\pi_0R$, which provides the data of a map $x\colon\mathbb S\to R$. We can then form a composite
	\[\left(\mathbb S^{\otimes2}\right)_{h\Sigma_2}\to\left(R^{\otimes_{\mathbb S}2}\right)_{h\Sigma_2}\to R,\]
	where the left map is $x\otimes x$, and the right map is multiplication. Note that the left object is $\colim_{\mathrm B\Sigma_2}\mathbb S=\Sigma^\infty_+\colim_{\mathrm B\Sigma_2}\mathrm{pt}=\Sigma_+^\infty\mathrm B\Sigma_2$, so we see that we have found an element $D_2(x)\in R^0(\RP^\infty)$.
\end{defihelper}
\begin{example}
	Take $R=\mathbb S$. Then $D_2(1)=\varepsilon$. Indeed, the multiplication map $\mathbb S^{\otimes2}_{h\Sigma_2}\to\mathbb S$ is $\Sigma^\infty_+$ of the projection $\mathrm B\Sigma_2\to\mathrm{pt}$, so we can see that the total composite is in fact $\varepsilon$.
\end{example}
\begin{example} \label{ex:d2-2}
	Take $R=\mathbb S$. To compute $D_2(2)$, we note that
	\[(\mathbb S\oplus\mathbb S)^{\otimes2}=\mathbb S\oplus\mathbb S\oplus\mathbb S\oplus\mathbb S,\]
	and the canonical swapping map on the left fixes two factors and swaps the other two. (Indeed, the isomorphism is basically given by $(x+y)^2=x^2+xy+yx+y^2$.) Taking $(-)_{h\Sigma_2}$, it turns out that we get
	\[\Sigma_+^\infty\RP^\infty\to\Sigma_+^\infty\RP^\infty\oplus\mathbb S\oplus\Sigma_+^\infty\RP^\infty\to\Sigma_+^\infty\RP^\infty.\]
	In particular, $(\mathbb S\oplus\mathbb S)_{h\Sigma_2}$ can be seen to be $\mathbb S$ again by considering a left Kan extension of the map $\mathbb S\colon\mathrm{pt}\to\mathrm{Spectra}$ to $\mathrm B\Sigma_2$. Now, it turns out that the left map is $(1,{\tr},1)$ (one can see that the middle map is transfer by returning to combinatorics of finite sets), so by mapping back to $\mathbb S$, we see that $D_2(2)=2\varepsilon+\tr$.
\end{example}
\begin{remark}
	The method of \Cref{ex:d2-2} can show that $D_2(a+b)=D_2(a)+D_2(b)+ab\tr$ for any $a,b\in\pi_0\mathbb S$. For example, $D_2(4)=r\varepsilon+6\tr$.
\end{remark}
And here is our theorem.
\begin{theorem}[Nishida nilpotence] \label{thm:nishida-nilpotence-proved}
	If $x\in\pi_i\mathbb S$ for $i>0$, then $x$ is nilpotent.
\end{theorem}
\begin{proof}
	It suffices to show that $\mathbb S\left[x^{-1}\right]$ is the trivial ring, meaning that we want to show that $1=0$ in $\pi_0\mathbb S\left[x^{-1}\right]$. Now, every higher homotopy group is torsion, so there is at least some positive integer $N$ for which $N=0$ in $\pi_0\mathbb S\left[x^{-1}\right]$. If $N$ admits no prime factors, then we are done. Otherwise, choose a prime $p$ dividing $N$, and we would like to replace $N$ with $N/p$, for which it is enough to check that $\mathbb S_{(p)}\left[x^{-1}\right]=0$.

	We will give the proof in an example. For example, let's work with $p=2$ so that we want to show $\mathbb S_{(2)}\left[x^{-1}\right]=0$. By the discussion of the previous paragraph, we may even assume that some power of $2$ vanishes. Further specializing with loss of generality, let's say that $4$ vanishes. (The same idea turns out to work in general.) This means that the composite
	\[\mathbb S\stackrel 4\to\mathbb S\to\mathbb S_{(2)}\left[x^{-1}\right]\]
	is nullhomotopic. Squaring, we can form the diagram
	% https://q.uiver.app/#q=WzAsNSxbMCwwLCJcXG1hdGhiYiBTXntcXG90aW1lczJ9X3toXFxTaWdtYV8yfSJdLFsxLDAsIlxcbWF0aGJiIFNee1xcb3RpbWVzMn1fe2hcXFNpZ21hXzJ9Il0sWzIsMCwiXFxtYXRoYmIgU197KDIpfVxcbGVmdFt4XnstMX1cXHJpZ2h0XV57XFxvdGltZXMyfSJdLFsyLDEsIlxcbWF0aGJiIFNfeygyKX1cXGxlZnRbeF57LTF9XFxyaWdodF0iXSxbMSwxLCJcXG1hdGhiYiBTIl0sWzAsMSwiNCJdLFsyLDMsIm0iXSxbMSw0LCIiLDAseyJsZXZlbCI6Miwic3R5bGUiOnsiaGVhZCI6eyJuYW1lIjoibm9uZSJ9fX1dLFsxLDJdLFs0LDNdXQ==&macro_url=https%3A%2F%2Fraw.githubusercontent.com%2FdFoiler%2Fnotes%2Fmaster%2Fnir.tex
	\[\begin{tikzcd}[cramped]
		{\mathbb S^{\otimes2}_{h\Sigma_2}} & {\mathbb S^{\otimes2}_{h\Sigma_2}} & {\mathbb S_{(2)}\left[x^{-1}\right]^{\otimes2}} \\
		& {\mathbb S} & {\mathbb S_{(2)}\left[x^{-1}\right]}
		\arrow["4", from=1-1, to=1-2]
		\arrow[from=1-2, to=1-3]
		\arrow[equals, from=1-2, to=2-2]
		\arrow["m", from=1-3, to=2-3]
		\arrow[from=2-2, to=2-3]
	\end{tikzcd}\]
	where the top composite is nullhomotopic. But the zigzag path is $D_2(4)$, so we conclude that the composite
	\[\mathbb S^{\otimes2}_{h\Sigma_2}\stackrel{D_2(4)}\to\mathbb S\to\mathbb S_{(2)}\left[x^{-1}\right]\]
	is nullhomotopic. Now, $D_2(4)=4+6\tr$, so because $4=0$, we further see that the composite
	\[\mathbb S^{\otimes2}_{h\Sigma_2}\stackrel{2\tr}\to\mathbb S\to\mathbb S_{(2)}\left[x^{-1}\right]\]
	is nullhomotopic. But now \Cref{thm:kahn-priddy} grants us a map $\mathbb S^i\to\Sigma_+^\infty\RP^\infty$ which goes back to $x$ under $\pi_*\tr$, so it follows that
	\[\mathbb S^i\stackrel{2x}\to\mathbb S\to\mathbb S_{(2)}\left[x^{-1}\right]\]
	is nullhomotopic. Thus, $2x=0$ in $\mathbb S_{(2)}\left[x^{-1}\right]$, so $2=0$.

	Let's at least say a sentence or two about why this should work in general. Continuing with $p=2$, the point is that $D_2(2a)=2D_2(a)+a^2\tr$, and it turns out that the trace term again has lower $2$-adic valuation. Thus, one can expect to be able to induct and handle general powers of $2$. For other primes $p$, we need a $D_p$ construction, and again, the point is that the multinomial coefficients in an expansion $(a_1+\cdots+a_p)^p$ will have smaller $p$-adic valuation.
\end{proof}
\begin{remark}
	There is a specific element in $\kappa\in\pi_{24}\mathbb S$ for which we do not know if $\kappa^{11}=0$ or if merely $\kappa^{12}=0$.
\end{remark}
\begin{remark}
	One can upgrade this argument to show the following: for an $\mathbb E_\infty$-ring $R$, one has $R=0$ if and only if $\ZZ\otimes_{\mathbb S}R=0$. (This truly uses the $\mathbb E_\infty$-structure: it is false if one tries to work with $\mathbb E_n$-rings for any finite $n$!) In particular, plugging in $\mathbb S\left[x^{-1}\right]$ for $R$ recovers \Cref{thm:nishida-nilpotence-proved}.
\end{remark}

\end{document}