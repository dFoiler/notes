% !TEX root = ../notes.tex

\documentclass[../notes.tex]{subfiles}

\begin{document}

\section{February 11}
Today, we say a little bit about $\mathrm{MU}$. There are many existing sources: there is Adams's red book, Ravenel's green book, Wilson's \textit{An Introduction and Sampler}, Lurie's chromatic homotopy notes, Pstragowki's book, and Peterson's book.

\subsection{Formal Group Laws}
Let's start with some pure algebra.
\begin{definition}[formal group law]
	Fix a commutative ring $R$. Then a \textit{formal group law} over $R$ is a power series $f\in R[[x,y]]$ such that
	\begin{listalph}
		\item $f(x,0)=f(0,x)=x$,
		\item $f(x,y)=f(y,x)$, and
		\item $f(f(x,y),z)=f(x,f(y,z))$.
	\end{listalph}
	We may write $x+_fy$ for $f(x,y)$.
\end{definition}
\begin{remark}
	One is supposed to think of $f$ as providing the formula for an actual group law, under the assumption that $f$ always converges. More rigorously, these are supposed to be additions on $\widehat{\AA}^1$.
\end{remark}
\begin{definition}[isomorphism]
	Fix a commutative ring $R$. An \textit{isomorphism} $g\colon f_1\to f_2$ of formal group laws is a power series $g\in R[[x]]$ for which $g(0)=0$, $g'(0)\in R^\times$ and
	\[g(x+_{f_1}y)=x+_{f_2}y.\]
	We say that $g$ is \textit{strict} if and only if $g'(0)=1$.
\end{definition}
\begin{example}
	There is the additive formal group law $f(x,y)\coloneqq x+y$.
\end{example}
\begin{example}
	There is the multiplicative formal group law $f(x,y)=x+y+xy$.
\end{example}
There is an important ``universal'' formal group law.
\begin{definition}[Lazard ring]
	The \textit{Lazard ring} $L$ is defined to co-represent the functor sending a ring $R$ of all formal group laws over $R$. Explicitly, we can realize it as a quotient of the ring $\ZZ[\{a_{ij}\}_{ij}]$ by the relations required to make the power series
	\[\sum_{i,j\ge0}a_{ij}x^iy^j\]
	into a formal group law. We give $L$ a grading by $\deg a_{ij}\coloneqq2(i+j-1)$.
\end{definition}
\begin{definition}
	We define $\ZZ[b_1,b_2,\ldots]$ to be the universal ring equipped with a formal group law and an isomorphism of this formal group law. Explicitly, it is a ring $\ZZ[b_1,b_2,\ldots]$ equipped with a map $\varphi\colon L\to\ZZ[b_1,b_2,\ldots]$ to classify $g\left(g^{-1}(x)+g^{-1}(y)\right)$, where $g(x)\coloneqq x+b_1x^2+b_2x^3$. We give this ring a grading by $\deg b_i=2i$, so the map $\varphi$ is graded.
\end{definition}
\begin{remark}
	One can see that $L[b_1,b_2,\ldots]$ classifies two formal group laws and a choice of strict isomorphism between them. Indeed, $L$ provides the formal group law, and the $b_\bullet$s provide the strict isomorphism.
\end{remark}
\begin{theorem}[Lazard] \label{thm:lazard}
	For all $n>0$, the map $\varphi_{2n}\colon L_{2n}\to\ZZ[b_1,b_2,\ldots]_{2n}$ is an injection after taking a quotient by the decomposable elements. Explicitly, the target is isomorphic to $\ZZ b_{2n}$, and the image is full if $n+1$ is not a prime power; otherwise, the image is $p\ZZ$ if $n+1$ is a power of $p$.
\end{theorem}
\begin{proof}[Idea]
	The hard part is to show that $L_{2n}$ modulo decomposable elements is isomorphic to $\ZZ$. Accordingly, let this quotient be $Q_{2n}$. Then one needs to exhibit a natural isomorphism $\op{Hom}_\ZZ(Q_{2n},M)\cong M$, but one finds that maps $Q_{2n}\to M$ are given by formal group laws over the square zero extension $\ZZ\oplus M$ of the form
	\[x+y+\sum_{i+j=n+1}x^iy^j.\]
	Then one needs to classify such things, which is an algebra problem with binomial coefficients.
\end{proof}
\begin{corollary}
	For each $i$, choose $x_i\in L$ of degree $2i$ which generates $L_{2i}$ modulo decomposable elements. Then the induced map
	\[\ZZ[x_1,x_2,\ldots]\to L\]
	is an isomorphism of rings.
\end{corollary}
\begin{proof}
	This follows from \Cref{thm:lazard}: indeed, \Cref{thm:lazard} shows that we have produced an isomorphism after passing to the associated graded ring for the relevant filtration, which is enough to show that we have an isomorphism.
\end{proof}
Thus, $L$ is just a polynomial ring! However, this presentation is not very helpful because it requires infinitely many choices. Alternatively, the map $\varphi$ is an isomorphism after taking tensor product with $\QQ$ by \Cref{thm:lazard}, but taking this tensor product may lose information we care about.

\subsection{The Spectrum \texorpdfstring{$\mathrm{MU}$}{ MU}}
We now return to algebraic topology. We would like to make sense of the following theorem.
\begin{theorem}[Quillen] \label{thm:quillen}
	There is an $\mathbb E_\infty$-ring $\mathrm{MU}$ satisfying the following.
	\begin{listalph}
		\item $\pi_*\mathrm{MU}\cong L$.
		\item The induced map $\pi_*\mathrm{MU}\to\pi_*(\ZZ\otimes_{\mathbb S}\mathrm{MU})$ is isomorphic to $\varphi$.
		\item There are isomorphisms $\pi_*(\mathrm{MU}\otimes_{\mathbb S}\mathrm{MU})\cong\pi_*\mathrm{MU}[b_1,b_2,\ldots]\cong L[b_1,b_2,\ldots]$ such that the two induced maps $\pi_*\mathrm{MU}\to\pi_*(\mathrm{MU}\otimes_{\mathbb S}\mathrm{MU})$ correspond to the source and target of the universal formal group laws of $L[b_1,b_2,\ldots]$.
	\end{listalph}
\end{theorem}
Of course, we will have to start by constructing $\mathrm{MU}$.
\begin{remark}
	There is a functor $S\colon\mathrm{Vec}_\CC\to\mathrm{Top}_*$ of ordinary category given by taking the one-point compactification. It satisfies $S(V\oplus W)=SV\land SW$, so $S$ is symmetric monoidal, and we get a map
	\[\bigsqcup_{n\ge0}\mathrm {BU}(n)\to\mathrm{Top}_*\to\mathrm{Spaces}_*\]
	sending $V$ to $\Sigma^\infty S^V$.
\end{remark}
\begin{definition}[Picard group]
	Fix an $\mathbb E_\infty$-ring $R$. Then $\op{Pic}R$ is the full subcategory of $\mathrm{Mod}(R)$ of $\otimes$-invertible objects, only equipped with the isomorphisms.
\end{definition}
\begin{example}
	The only invertible objects in $\op{Pic}\mathbb S$ are powers of $\mathbb S$. Here is a sketch: any invertible object remains invertible after $-\otimes_{\mathbb S}\ZZ$, so it becomes an invertible object in $\mathrm{Mod}(\ZZ)$. But the only invertible chain complexes are shifts of $\ZZ$, so we know that any invertible object has homology supported in one degree. By the Hurewicz theorem, it therefore has homotopy starting in one degree (and zero below), so we receive a map from a sphere, and one can check that it is an isomorphism by a variant of Whitehead's theorem.
\end{example}
\begin{remark}
	Note that $\op{Pic}R$ is in fact a group-like $\mathbb E_\infty$-space, where the operation is given by $\otimes$.
\end{remark}
\begin{remark}
	It thus turns out that the map $\bigsqcup_{n\ge0}\mathrm{BU}(n)\to\mathrm{Spectra}$ given by $V\mapsto\Sigma^\infty SV$ factors through $\op{Pic}\mathbb S$ because it only produces powers of spheres.
\end{remark}
\begin{definition}[Thom spectrum]
	Fix an $\mathbb E_\infty$-ring $R$. If $f\colon X\to\op{Pic}R$ is a map of groupoids, then we define the \textit{Thom spectrum} $\op{Thom}(f)$ to be the colimit of the diagram
	\[X\to\op{Pic}R\to\mathrm{Mod}(R).\]
\end{definition}
\begin{remark}
	Intuitively, the map $f\colon X\to\op{Pic}R$ is a bundle of invertible spectra on $R$, so we are recovering a classical notion.
\end{remark}
\begin{example}
	We define the map $\bigsqcup_{n\ge0}\mathrm{BU}(n)\to\op{Pic}\mathbb S$ to have Thom spectrum $\bigoplus_{n\ge0}\op{MU}(n)$.
\end{example}
\begin{example}
	We claim that $\op{MU}(1)$ is $\Sigma^\infty\CP^\infty$.
\end{example}
\begin{proof}
	We are looking at the colimit which $\mathrm{BU}(1)\to\mathrm{Spectra}$ which sends a line bundle $\mc L\in\mathrm BS^1$ to the spectrum $\Sigma^\infty S\mc L$. Let $S^1\mc L$ be the unit sphere in $\mc L$, and then there is a homotopy fiber sequence
	\[S^1\mc L_+\to\mathrm{pt}_+\to S\mc L.\]
	Intuitively, this homotopy fiber sequence is gluing some part of $S\mc L$ against a point to produce a sphere with a point.

	It follows that $\op{MU}(1)$ can be computed as
	\begin{align*}
		\colim_{\mathrm{B}S^1}\Sigma^\infty S\mc L &= \Sigma^\infty\colim_{\mathrm BS^1}S\mc L \\
		&= \Sigma^\infty\colim_{\mathrm BS^1}\op{cofiber}\left(S^1\mc L_+\to\mathrm{pt}_+\right) \\
		&= \op{cofiber}\left(\colim_{\mathrm BS^1}\Sigma_+^\infty S^1\mc L\to\colim_{\mathrm BS^1}\Sigma^\infty_+\mathrm{pt}\right).
	\end{align*}
	Now, the right colimit has $\Sigma_+^\infty$ come out, so it is $\Sigma_+^\infty\CP^\infty$. The left colimit is the colimit of the group $S^1$ acting on itself, which is the sphere $\mathbb S$. Now, the map $\mathbb S\to\Sigma_+^\infty\CP^\infty$ just gets rid of the point, so the cofiber is $\Sigma^\infty\CP^\infty$.
\end{proof}
Now, we recognize that we have a map $\bigsqcup_{n\ge0}\mathrm{BU}(n)\to\op{Pic}\mathbb S$ which has target equal to a grouplike $\mathbb E_\infty$-space. Thus, we can take a group completion to produce a map
\[\Omega\mathrm B\bigsqcup_{n\ge0}\mathrm{BU}(n)\to\op{Pic}\mathbb S.\]
The left-hand side is just $\mathrm{BU}\times\ZZ$ (here, $\mathrm{BU}$ is the colimit $\mathrm{BU}_\infty$), and it has no negative homotopy groups, so we can realize it as some $\Omega^\infty\mathrm{ku}$ for some $\mathrm{ku}$.
\begin{definition}
	Fix everything as above. Then $\mathrm{MU}$ is the Thom spectrum of the sequence
	\[\mathrm{BU}\to\mathrm{BU}\times\ZZ\to\op{Pic}\mathbb S.\]
\end{definition}
\begin{remark}
	One can realize the left map as $\Omega^\infty$ applied to the sequence $\tau_{\ge2}\mathrm{ku}\to\mathrm{ku}\to\mathrm{pic}\mathbb S$, where $\op{pic}\mathbb S$ is the space for which $\Omega^\infty\op{pic}\mathbb S=\op{Pic}\mathbb S$.
\end{remark}
\begin{remark}
	It was important to mention $\op{Pic}$ so that we could take group completion.
\end{remark}
\begin{remark}
	Our definition of $\mathrm{MU}$ automatically gives it the structure of a group-like $\mathbb E_\infty$-space. It is already a colimit valued in $\mathbb S$-modules, so $\mathrm{MU}$ becomes an $\mathbb E_\infty$-ring!
\end{remark}
\begin{remark}
	The Thom spectrum of the map
	\[\Omega\mathrm B\bigsqcup_{n\ge0}\mathrm{BU}(n)\to\op{Pic}\mathbb S\]
	produces a space $\mathrm{MUP}$, which is isomorphic to $\bigoplus_{k\in\ZZ}\Sigma^k\mathrm{MU}$. Its ring structure is not unique! It is an open question if the ring structure on $\mathrm{MU}$ is unique.
\end{remark}

\subsection{Complex Orientations}
Let's find some structure on $\mathrm{MU}$.
\begin{remark} \label{rem:mu-orientation}
	By taking Thom spectra of the map
	\[\CP^\infty\to\bigsqcup_{n\ge0}\mathrm{BU}(n)\to\mathrm{BU}\times\ZZ\to\op{Pic}\mathbb S\]
	produces a map $\op{MU}(1)\to\op{MUP}$. Note that the copy of $\ZZ$ here keeps track of the dimension of the ambient vector bundle, so our copy of $\mathrm{BU}(1)$ will land in the first component of $\mathrm{BU}\times\ZZ$. Expanding out $\op{MUP}$, we see that we have induced a map $\Sigma^\infty\CP^\infty\to\Sigma^2\mathrm{MU}$.
\end{remark}
The sort of gadget in \Cref{rem:mu-orientation} is important enough to deserve a name.
\begin{definition}[complex orientation]
	Fix an $\mathbb E_\infty$-ring $R$. A \textit{complex orientation} is a map $\Sigma^\infty\CP^\infty\to\Sigma^2R$ for which the composite
	\[\Sigma^\infty\CP^1\to\Sigma^\infty\CP^\infty\to\Sigma^2R\]
	is the unit map $\Sigma^2\mathbb S\to\Sigma^2R$.
\end{definition}
Complex orientations give rise to formal group laws. Let's explain how this is done.
\begin{remark}
	A choice of complex orientation yields an isomorphism
	\[R^*\CP^\infty\cong\pi_{-*}R[[t]],\]
	where $t\in R^2\CP^\infty$ is the class induced by the complex orientation. Indeed, this is some spectral sequence calculation. The point is that the Atiyah--Hirzebruch spectral sequence tells us that $R^*\CP^\infty$ certainly looks like $\pi_{-*}R[[t]]$ on $E_2$, and there cannot be differentials making this smaller because $t$ is a genuine cohomology class.
\end{remark}
\begin{remark}
	One can similarly show that a complex orientation on $R$ produces an isomorphism
	\[R^*(\CP^\infty\times\CP^\infty)\cong\pi_{-*}R[[x,y]].\]
\end{remark}
\begin{remark}
	The map $\CP^\infty\times\CP^\infty\to\CP^\infty$ given by the tensor product of lines is the same data as a map $\Sigma^2\ZZ\oplus\Sigma^2\ZZ\to\Sigma^2\ZZ$ after $\Omega^\infty$. This then induces a map $\pi_{-*}R[[t]]\to\pi_{-*}R[[x,y]]$. The image of $t$ turns out to be a formal group law, which one can see all the back from the line bundle maps.
\end{remark}
Thus, we have a formal group law on $\pi_*\mathrm{MU}$, so we receive a map $L\to\pi_*\mathrm{MU}$. \Cref{thm:quillen} will tell us that this map is an isomorphism. One next proceeds by producing maps
\[\pi_*\mathrm{MU}\to\mathrm H_*(\mathrm{MU};\ZZ)\to\mathrm H_*(\mathrm{MU};\QQ).\]
Thus, it is profitable to understand the target rings. To this end, note that we have a square
% https://q.uiver.app/#q=WzAsNCxbMCwwLCJcXG9we1BpY31cXG1hdGhiYiBTIl0sWzEsMCwiXFxtYXRocm17TW9kfShcXG1hdGhiYiBTKSJdLFswLDEsIlxcb3B7UGljfVxcWloiXSxbMSwxLCJcXG1hdGhybXtNb2R9KFxcWlopIl0sWzEsMywiLVxcb3RpbWVzXFxaWiJdLFswLDJdLFsyLDNdLFswLDFdXQ==&macro_url=https%3A%2F%2Fraw.githubusercontent.com%2FdFoiler%2Fnotes%2Fmaster%2Fnir.tex
\[\begin{tikzcd}[cramped]
	{\op{Pic}\mathbb S} & {\mathrm{Mod}(\mathbb S)} \\
	{\op{Pic}\ZZ} & {\mathrm{Mod}(\ZZ)}
	\arrow[from=1-1, to=1-2]
	\arrow[from=1-1, to=2-1]
	\arrow["{-\otimes\ZZ}", from=1-2, to=2-2]
	\arrow[from=2-1, to=2-2]
\end{tikzcd}\]
where the vertical maps are given by base-change, and the horizontal maps are inclusions. Observe that $\pi_0\op{Pic}\ZZ=\ZZ$ because the invertible chain complexes are all just $\ZZ$ shifted by some degree. It then turns out that $\tau_{\ge1}\op{Pic}\ZZ\cong\ZZ/2\ZZ$ embeds into homomorphisms $\ZZ\to\ZZ$, which is $\Omega^\infty\ZZ$.

Now, to compute $\mathrm H_*(\mathrm{MU};\ZZ)=\pi_*(\mathrm{MU}\otimes_{\mathbb S}\ZZ)$, we are interested in making $\mathrm{BU}$ map all the way to $\mathrm{Mod}(\ZZ)$. But $\op{Pic}\ZZ$ only has homotopy groups below $1$, and $\mathrm{BU}$ is simply connected, so the induced map $\mathrm{BU}\to\mathrm{Mod}(\ZZ)$ is nullhomotopic. It follows that
\[\mathrm{MU}\otimes_{\mathbb S}\ZZ=\Sigma^\infty_+\mathrm{BU}\otimes_{\mathbb S}\ZZ,\]
so $\pi_*(\mathrm{MU}\otimes_{\mathbb S}\ZZ)=\mathrm H_*(\mathrm{BU};\ZZ)$. Then one can make some arguments with spectral sequences and so on.

\end{document}