% !TEX root = ../notes.tex

\documentclass[../notes.tex]{subfiles}

\begin{document}

\section{February 18}
Let's start using $\mathrm{MU}$ to understand $\mathbb S$.

\subsection{Filtrations for Fun and Profit}
Here is how we will use $\mathrm{MU}$ to get to $\mathbb S$.
\begin{lemma} \label{lem:sphere-by-mu}
	The spectrum $\mathbb S$ is the limit of the following diagram.
	% https://q.uiver.app/#q=WzAsNCxbMCwwLCJcXG1hdGhybXtNVX0iXSxbMSwwLCJcXG1hdGhybXtNVX1ee1xcb3RpbWVzMn0iXSxbMiwwLCJcXG1hdGhybXtNVX1ee1xcb3RpbWVzM30iXSxbMywwLCJcXGNkb3RzIl0sWzEsMF0sWzAsMSwiIiwwLHsib2Zmc2V0IjotMX1dLFswLDEsIiIsMSx7Im9mZnNldCI6MX1dLFsxLDJdLFsyLDEsIiIsMSx7Im9mZnNldCI6LTF9XSxbMiwxLCIiLDEseyJvZmZzZXQiOjF9XSxbMSwyLCIiLDEseyJvZmZzZXQiOi0yfV0sWzEsMiwiIiwxLHsib2Zmc2V0IjoyfV1d&macro_url=https%3A%2F%2Fraw.githubusercontent.com%2FdFoiler%2Fnotes%2Fmaster%2Fnir.tex
	\[\begin{tikzcd}[cramped]
		{\mathrm{MU}} & {\mathrm{MU}^{\otimes2}} & {\mathrm{MU}^{\otimes3}} & \cdots
		\arrow[shift left, from=1-1, to=1-2]
		\arrow[shift right, from=1-1, to=1-2]
		\arrow[from=1-2, to=1-1]
		\arrow[from=1-2, to=1-3]
		\arrow[shift left=2, from=1-2, to=1-3]
		\arrow[shift right=2, from=1-2, to=1-3]
		\arrow[shift left, from=1-3, to=1-2]
		\arrow[shift right, from=1-3, to=1-2]
	\end{tikzcd}\]
\end{lemma}
\begin{proof}
	This follows from the following general fact: if $A\to B$ is a map of $\mathbb E_\infty$-rings whose fiber has homotopy groups only in positive degree, then $A$ is the limit of the \v{C}ech nerve of $B$ over $A$ (which is the given colimit). This fact is checked by considering the partial limits and showing that they induce isomorphisms on higher and higher homotopy groups, so we get an isomorphism in the limit. Anyway, it is enough to check that $\mathbb S\to\mathrm{MU}$ is an isomorphism in degree zero, which is true by \Cref{thm:quillen}.
\end{proof}
\begin{notation}
	We let $\op{fil}^n\mathbb S$ be the limit of the following diagram.
	\[\begin{tikzcd}[cramped]
		{\tau_{\ge2n}\mathrm{MU}} & {\tau_{\ge2n}\mathrm{MU}^{\otimes2}} & {\tau_{\ge2n}\mathrm{MU}^{\otimes3}} & \cdots
		\arrow[shift left, from=1-1, to=1-2]
		\arrow[shift right, from=1-1, to=1-2]
		\arrow[from=1-2, to=1-1]
		\arrow[from=1-2, to=1-3]
		\arrow[shift left=2, from=1-2, to=1-3]
		\arrow[shift right=2, from=1-2, to=1-3]
		\arrow[shift left, from=1-3, to=1-2]
		\arrow[shift right, from=1-3, to=1-2]
	\end{tikzcd}\]
	Using $2n$ instead of $n$ is a convention, motivated by \Cref{thm:quillen}.
\end{notation}
\begin{example}
	Note that $\mathrm{MU}$ has only homotopy groups in nonnegative degrees, we see that $\op{fil}^n\mathbb S=\mathbb S$ for $n\le0$. It follows that the colimit of the diagram
	\[\cdots\to\op{fil}^3\mathbb S\to\op{fil}^2\mathbb S\to\op{fil}^1\mathbb S\to\op{fil}^0\mathbb S\]
	is $\mathbb S$. In fact, the limit of this diagram is $0$, which more or less follows from \Cref{lem:sphere-by-mu}.
\end{example}
We are thus motivated to say something about filtrations of spectra.
\begin{definition}
	A \textit{filtered spectrum} $X_*$ is a diagram of the form
	\[\cdots\to X_{-2}\to X_{-1}\to X_0\to X_1\to X_2\to\cdots.\]
	A morphism of such filtered spectra is a morphism of the underlying diagrams, including homotopies for all the relevant squares.
\end{definition}
\begin{remark}
	It turns out that the category of filtered spectra is then a stable $\infty$-category. It is further symmetric monoidal, with $\otimes$ given by
	\[(X_*\otimes Y_*)_n\coloneqq\colim_{i+j\ge n}(X_i\otimes Y_j).\]
	For example, if the filtered spectrum is in fact given from a ``graded spectrum'' (so that the internal maps between the $X_*$s and $Y_*$s are all split embeddings), then the right-hand side is $\bigoplus_{i+j=n}(X_i\otimes Y_j)$. This formula shows that the object $\cdots0\to0\to\mathbb S\to\mathbb S\to\cdots$ is the identity.
\end{remark}
\begin{example}
	If $E$ is an $\mathbb E_\infty$-ring, then $\{\tau_{\ge2*}E\}_n$ is an $\mathbb E_\infty$-algebra valued in filtered complexes.
\end{example}
\begin{example}
	It now makes sense to say that $\op{fil}^*\mathbb S$ is the limit of the \v{C}ech nerve
	\[\tau_{\ge2*}\mathrm{MU}\to\tau_{\ge2*}\mathrm{MU}^{\otimes2}\to\cdots\]
	in the category of filtered $\mathbb E_\infty$-rings. It follows that $\op{fil}^*\mathbb S$ gains the structure of an $\mathbb E_\infty$-algebra valued in the category of filtered spectra.
\end{example}
The benefit of doing category theory is that we can make categories.
\begin{definition}[synthetic spectrum]
	A \textit{synthetic spectrum} is a module of $\op{fil}^*\mathbb S$ in the category of filtered spectra. The (stable $\infty$-)category of synthetic spectra is denoted $\op{SynSpectra}$.
\end{definition}
\begin{example}
	There is a functor $\nu$ which takes a spectrum $X$ and produces the synthetic spectrum
	\[\lim\tau_{\ge2*}\left(X\otimes\mathrm{MU}^{\otimes\bullet+1}\right).\]
\end{example}
Here is the point of working with synthetic spectra.
\begin{proposition}
	If $X$ is spectrum with homotopy groups bounded below, then $\nu(X)$ is a filtered spectrum $F_\bullet$, and its limit is $0$ and colimit is $X$. There is then a spectral sequence
	\[E_2\colon\pi_*C_*\Rightarrow\pi_*X,\]
	where $C_i$ is defined as the cofiber of $F_{i-1}\to F_i$.
\end{proposition}
\begin{remark}
	There are also higher coherences present in this spectral sequence arising because $\nu(E)$ has some higher coherences in its construction. For example, the fact that $\op{fil}^*\mathbb S$ is a ring produces the Leibniz rule.
\end{remark}

\subsection{The Adams--Novikov Spectral Sequence}
Later in life, we will want to localize at a prime $p$.
\begin{lemma}
	For each prime $p$,
	\[\lim\tau_{\ge2*}\op{MU}^{\otimes\bullet}_{(p)}=\lim\tau_{\ge2*}\op{BP}^{\otimes\bullet+1}_{(p)}.\]
\end{lemma}
\begin{proof}
	Certainly there is a natural map $\mathrm{BP}\to\mathrm{MU}$, so we just have to check that it induces an isomorphism of the filtered spectra. By some argument with the Five lemma, it is enough to check that it gives an isomorphism on the associated graded spectra. Well, taking this grading passes through the limit, so it becomes a natural map
	\[\Sigma^{2i}\lim\pi_{2i}\op{BP}^{\otimes\bullet+1}_{(p)}\to\Sigma^{2i}\lim\pi_{2i}\op{MU}^{\otimes\bullet+1}_{(p)},\]
	which is now just a morphism taking place in $\mathrm{Mod}(\ZZ)$. We can now use the Dold--Kan correspondence to pass to chain complexes. It turns out that the homology on the left computes cohomology of $\mc M_{\mathrm{fg},(p)}^{\mathrm{strict}}$ (which is the moduli stack of formal group laws up to strict isomorphism, and then we invert everything away from $p$), and the homology on the right computes moduli of $p$-typical formal group laws up to strict isomorphism.
\end{proof}
\begin{remark}
	In fact, the homology of the aforementioned complex is exactly the starting page of the spectral sequence.
\end{remark}
\begin{example}
	At $p=2$, our $\mathrm{MU}$ complex looks like
	\[\ZZ_{(2)}[v_1,\ldots]\to\ZZ_{(2)}[v_1,\ldots][t_1,\ldots]\to\ZZ_{(2)}[v_1,\ldots][t_1\otimes t_1,\ldots].\]
	For example, one can compute that $d(v_1)=\pm 2t_1$, which because $\ZZ_{(2)}$ is $\ZZ$-torsion-free, we see $d(t_1)=0$. (Formulae for the differentials have been computed for many degrees.) One then sees that $t_1$ represents a class in the first homology which is $2$-torsion. Unravelling the spectral sequence, one sees that an $i$th homology class in degree $2j$ (of the polynomial ring) maps to $\pi_{2j-i}\mathbb S$. For example, $t_1$ goes to the Hopf map in $\pi_1\mathbb S$.
\end{example}
\begin{remark}
	Even though $\mathrm{BP}$ is not an $\mathbb E_\infty$-ring, it turns out that $\op{fil}^*\mathbb S_{(p)}=\lim\tau_{\ge2*}\mathrm{BP}^{\otimes\bullet+1}$ is a filtered $\mathbb E_\infty$-ring, as we just showed. There are some reasons why $\op{fil}^*\mathbb S_{(p)}$ may be ``more canonical'' than $\mathbb S$.
	\begin{itemize}
		\item One can find $\op{fil}^*\mathbb S$ is the limit of the filtered spectra $\tau_{\ge2*}E$ as $E$ varies over all even $\mathbb E_\infty$-rings.
		\item The $p$-completed category $(\mathrm{SynSp})_p^\land$ embeds fully faithfully into the $p$-completed category of $\CC$-motivic spectra.
	\end{itemize}
\end{remark}
One will frequently write down the $E_2$ page of our associated spectral sequence for $\lim\tau_{\ge2*}\mathrm{BP}^{\otimes\bullet+1}$ by graphing along axes $i$ and $2j-i$, where $i$ is the homological degree, and $2j$ is the top degree of the relevant polynomial in the homotopy groups of $\mathrm{MU}$. (Recall that a class in homological degree $i$ and polynomial degree $2j$ maps to $\pi_{2j-i}\mathbb S$.) For example, the $E_2$ page for $p=2$ looks something like the following.
% https://q.uiver.app/#q=WzAsMTUsWzAsMywidF8xIl0sWzEsMywiPyJdLFsyLDMsIj8iXSxbMSwyLCJ0XzFcXG90aW1lcyB0XzEiXSxbMiwyLCI/Il0sWzIsMSwidF8xXFxvdGltZXMgdF8xXFxvdGltZXMgdF8xIl0sWzAsMiwiMCJdLFswLDEsIjAiXSxbMSwxLCIwIl0sWzAsMCwiXFx2ZG90cyJdLFsxLDAsIlxcdmRvdHMiXSxbMiwwLCJcXHZkb3RzIl0sWzMsMSwiXFxjZG90cyJdLFszLDIsIlxcY2RvdHMiXSxbMywzLCJcXGNkb3RzIl1d&macro_url=https%3A%2F%2Fraw.githubusercontent.com%2FdFoiler%2Fnotes%2Fmaster%2Fnir.tex
\[\begin{tikzcd}[cramped]
	\vdots & \vdots & \vdots & \\
	0 & 0 & {t_1\otimes t_1\otimes t_1} & \cdots \\
	0 & {t_1\otimes t_1} & {?} & \cdots \\
	{t_1} & {?} & {?} & \cdots
\end{tikzcd}\]
\begin{remark}
	It turns out that multiplication by $t_1$ is an isomorphism on the $E_2$ page, above a line of slope $1/5$. But $t_1^4=0$ (even though $E_2$ does not show this), it follows that the $E_n$ page of the Adams--Novikov spectral sequence for large $n$ (which can see $t_1^4=0$ vanishes) above some line of slope $1/5$.
\end{remark}
Our next target is the following.
\begin{theorem}
	For any positive slope, there is a page $E_n$ which vanishes above a line of that slope.
\end{theorem}
One key idea in the proof will be to not merely view the spectral sequence as some formal algebraic gadget but as an actual object in the category of filtered spectra.

\subsection{Towards Hopf Algebroids}
We now see that we are interested in understanding the \v{C}ech nerve
\[\pi_*\mathrm{BP}\to\pi_*\mathrm{BP}^{\otimes2}\to\pi_*\mathrm{BP}^{\otimes3}\to\cdots.\]
This has the data of a ``Hopf algebroid.'' Indeed, maps to $R$ will get some kind of Agroup structure.
\begin{itemize}
	\item A map $\pi_*\mathrm{BP}\to R$ is the data of a $p$-typical formal group law.
	\item The data of a map from
	\[\pi_*\mathrm{BP}^{\otimes2}=\pi_*\mathrm{BP}\otimes_{\pi_*\mathrm{BP}}\pi_*\mathrm{BP}\]
	(note $\mathrm{BP}\otimes\mathrm{BP}$ is free as a module over $\pi_*\mathrm{BP}$) classifies two formal group laws with an isomorphism between them. For example, the canonical map from $\pi_*\mathrm{BP}$ explains what the identity is.
	\item The data of a map from $\pi_*\mathrm{BP}^{\otimes3}$ similarly explains how to compose the two formal group laws from $\pi_*\mathrm{BP}^{\otimes2}$.
\end{itemize}
In general, given any $\mathbb E_\infty$-ring $R$, if $R\otimes R$ is free over $R$, then one gets a Hopf algebroid.

\end{document}