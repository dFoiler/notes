% !TEX root = ../notes.tex

\documentclass[../notes.tex]{subfiles}

\begin{document}

\section{November 2}

Today we discuss Stone spaces. I will not record any topological background.

\subsection{The Stone Topology}
As usual, we will let $\mathcal M$ be an $\mathcal L$-structure, and we let $A\subseteq M$ be a subset. Recall we defined $S_n^{\mathcal M}(A)$ to be the set of complete $n$-types with parameters from $A$.
\begin{definition}[Stone topology]
	Fix an $\mathcal L$-structure $\mathcal M$ and a subset $A\subseteq M$. For each $\mathcal L_A$-formula $\varphi(\overline x)$ with $n$ free variables, we define
	\[[\varphi]\coloneqq\left\{P\in S_n^{\mathcal M}(A):\varphi\in P\right\}.\]
	The \textit{Stone topology} is the topology generated by the $[\varphi]$ as a sub-basis.
\end{definition}
\begin{remark} \label{rem:conjunct-is-intersect}
	Note that we are able to bring in some semantics: for a complete type $P$, observe $\varphi\land\psi\in P$ if and only if $\varphi\in P$ and $\psi\in P$, where both directions can be argued by contradiction. For example, if $\varphi\in P$ and $\psi\in P$ but $\varphi\land\psi\notin P$, then $\lnot(\varphi\land\psi)\in P$, but this is now impossible to satisfy along with $\varphi$ and $\psi$. These arguments tell us that
	\[[\varphi]\cap[\psi]=[\varphi\land\psi]\]
	Similarly, one can see that $S_n^{\mathcal M}(A)\setminus[\varphi]=[\lnot\varphi]$ because $\lnot\varphi\in P$ if and only if $\varphi\notin P$ by completeness. Combining, we get
	\[[\varphi]\cup[\psi]=S_n^{\mathcal M}(A)\setminus([\lnot\varphi]\cap[\lnot\psi])=[\lnot(\lnot\varphi\land\lnot\psi)]=[\varphi\lor\psi],\]
	which of course we could also have proven directly similarly to the argument with $\land$.
\end{remark}
In fact, we have a basis.
\begin{lemma} \label{lem:stone-is-basis}
	Fix an $\mathcal L$-structure $\mathcal M$ and a subset $A\subseteq M$. For a given nonnegative integer $n$, the sets $[\varphi]$ form a basis of a topology on $S_n^{\mathcal M}(A)$
\end{lemma}
\begin{proof}
	It is enough to show that the intersection of any two basic sets $[\varphi]$ and $[\psi]$ can be written as the union of basic open sets. But this is automatic from \Cref{rem:conjunct-is-intersect}.
\end{proof}
\begin{remark}
	Thus, if $\mathcal L$ is finite or even countable, we have provided a countable basis for the topology on $S_n^{\mathcal M}(A)$.
\end{remark}
So the open sets of the stone topology on $S_n^{\mathcal M}(A)$ are unions of the basic open sets $[\varphi]$ for various $\mathcal L$-formulae $\varphi$ with $n$ free variables. For example, this allows us to explicitly describe convergence: for a net $\{p_\alpha\}_{\alpha\in\Lambda}$, we have $p_\alpha\to q$ if and only if any basic open set $[\varphi]$ containing $q$ (i.e., $\varphi\in q$), there is some $\lambda\in\Lambda$ such that $\alpha\ge\lambda$ implies $p_\alpha\in[\varphi]$ (i.e., $\varphi\in p_\alpha$).

Let's discuss the topology on $S_n^{\mathcal M}(A)$.
\begin{proposition} \label{prop:stone-tot-disc}
	Fix an $\mathcal L$-structure $\mathcal M$ and a subset $A\subseteq M$. Then $S_n^{\mathcal M}(A)$ is totally disconnected when given the Stone topology.
\end{proposition}
\begin{proof}
	We show that $S_n^{\mathcal M}(A)$ is totally disconnected. In other words, we have to show that singletons are the largest connected sets. So for any set $S\subseteq S_n^{\mathcal M}(A)$ with more than one point, we want to show that $S$ is not connected. Well, we are given two points $P,Q\in S$; they are distinct, so find $\varphi\in P$ with $\varphi\notin Q$. But then $S\subseteq[\varphi]\cup[\lnot\varphi]$ even though $[\varphi]\cap[\lnot\varphi]$ and $P\in S\cap[\varphi]$ and $Q\in S\cap[\lnot\varphi]$. So $[\varphi]$ and $[\lnot\varphi]$ disconnect $S$.
\end{proof}
\begin{remark}
	The argument above in fact shows that $S_n^{\mathcal M}(A)$ is Hausdorff as well: we have placed any two distinct complete types $P$ and $Q$ into disjoint open subsets $[\varphi]$ and $[\lnot\varphi]$ where $\varphi\in P\setminus Q$.
\end{remark}
For something a little more interesting, let's use compactness.
\begin{theorem} \label{thm:stone-compact}
	Fix an $\mathcal L$-structure $\mathcal M$ and a subset $A\subseteq M$. Then $S_n^{\mathcal M}(A)$ is compact and totally disconnected when given the Stone topology.
\end{theorem}
\begin{proof}
	We show that $S_n^{\mathcal M}(A)$ is compact.
	\begin{enumerate}
		\item We translate covers into semantics. Fix a subset of $\mathcal L_A$-formulae $\Phi$. The main claim is that $\{[\varphi]:\varphi\in\Phi\}$ covers $S_n^{\mathcal M}(A)$ if and only if $\{\lnot\varphi:\varphi\in\Phi\}\cup\op{Th}_A\mathcal M$ is not satisfiable.

		In one direction, suppose that $\{\lnot\varphi:\varphi\in\Phi\}\cup\op{Th}_A\mathcal M$ is satisfiable by a structure $\mathcal N$ and tuple $\overline a$. Then we let $P$ be the type $\op{tp}_A^{\mathcal N}(\overline a)$. By construction, $P$ is complete, and $\mathcal N\models\lnot\varphi(\overline a)$ for each $\varphi\in\Phi$, so we conclude that $P$ is a complete type not covered by one of the $[\varphi]$ for $\varphi\in\Phi$.

		In the other direction, suppose that $\{[\varphi]:\varphi\in\Phi\}$ fails to cover $S_n^{\mathcal M}(A)$. So choose $P$ which does not live in any $[\varphi]$ for $\varphi\in\Phi$, implying that $\lnot\varphi\in P$ for each $\varphi\in\Phi$. Now, \Cref{cor:realize-types} grants us some elementary superstructure $\mathcal N$ of $\mathcal M$ and some $\overline a\in N^n$ so that $P=\op{tp}_A^{\mathcal N}(\overline a)$. Thus, by construction, $\mathcal N\models\lnot\varphi(\overline a)$ for each $\varphi\in\Phi$ and $\mathcal N\models\op{Th}_A\mathcal M$ because we have an elementary superstructure, so we are done.

		\item Now, Suppose that we have an open cover $\mathcal U$ of $S_n^{\mathcal M}(A)$ which we would like to reduce to a finite subcover. By writing each open set in $\mathcal U$ as a union of basic open subsets, we may assume that $\mathcal U$ has only basic open subsets, which we enumerate as $[\varphi]$ for various $\varphi\in\Phi$. We would like to extract a finite subcover. The previous step implies that
		\[\{\lnot\varphi:\varphi\in\Phi\}\]
		fails to be satisfiable, so by compactness, a finite subset fails to be satisfiable, so we have some finite $\Phi_0\subseteq\Phi$ such that
		\[\{\lnot\varphi:\varphi\in\Phi_0\}\]
		fails to be satisfiable, so by the previous step once again, we see that $\{[\varphi]:\varphi\in\Phi_0\}$ is the needed finite subcover.
		\qedhere
	\end{enumerate}
\end{proof}
\begin{remark}
	Notably, the main input to the above proof was the compactness theorem! In some sense, this is where the compactness theorem gets its name.
\end{remark}
There are a bunch of other functoriality checks one can do with continuous maps.

\subsection{Isolated Types}
Topology motivates the following definition.
\begin{definition}[isolated]
	Fix an $\mathcal L$-structure $\mathcal M$ and a subset $A\subseteq M$. Then a type $P\in S_n^{\mathcal M}(A)$ is \textit{isolated} if and only if $P$ is isolated in the Stone topology. In other words, there exists an open subset around $P$ only containing $P$.
\end{definition}
Let's get a better understanding of this term.
\begin{proposition} \label{prop:how-to-isolated-type}
	Fix an $\mathcal L$-structure $\mathcal M$ and a subset $A\subseteq M$. Let $P\in S_n^{\mathcal M}(A)$ be a type. Then the following are equivalent.
	\begin{listalph}
		\item $P$ is isolated.
		\item $\{P\}=[\varphi]$ for some $\mathcal L_A$-formula $\varphi$.
		\item There is an $\mathcal L_A$-formula $\varphi\in P$ such that, for any other $\mathcal L_A$-formula $\psi$, we have $\psi\in P$ if and only if $\op{Th}_A(\mathcal M)\models(\varphi\to\psi)$.
	\end{listalph}
	Note that the completeness of $\op{Th}_A(\mathcal M)$ makes $\op{Th}_A(\mathcal M)\models(\varphi\to\psi)$ equivalent to $(\varphi\to\psi)$ being in $\op{Th}_A(\mathcal M)$, which is equivalent to $\mathcal M\models(\varphi\to\psi)$.
\end{proposition}
\begin{proof}
	Note that (b) implies (a) by the definition of being isolated. For (a) implies (b), we note that $\{P\}$ is an open set by definition, so we can find some $\varphi$ such that $P\in[\varphi]$ and $[\varphi]\subseteq\{P\}$ by using our basis, so of course $\{P\}=[\varphi]$ follows.

	So the interesting part is showing that (c) is equivalent to the other two. The main claim is that $[\varphi]\subseteq[\psi]$ if and only if $\op{Th}_A(\mathcal M)\models(\varphi\to\psi)$. We show the implications separately.
	\begin{itemize}
		\item In one direction, if $\op{Th}_A(\mathcal M)\models(\varphi\to\psi)$, then $\mathcal M\models(\varphi\to\psi)$, then if $P\in[\varphi]$, we have $\varphi\in P$, so $\psi\in P$ by completeness, so $P\in[\psi]$.
		\item In the other direction, if $\mathcal M$ fails to satisfy $(\varphi\to\psi)$, then there is some $\overline a\in M$ such that $\mathcal M\models\varphi(\overline a)\land\lnot\psi(\overline a)$. Thus, $\op{tp}_A^{\mathcal M}(\overline a)\in[\varphi]\setminus[\psi]$.
	\end{itemize}
	We now show (b) implies (c): we have $\{P\}=[\varphi]$, so $\psi\in P$ if and only if $P\subseteq[\psi]$ if and only if $[\varphi]\subseteq[\psi]$, which by the claim is equivalent to $\mathcal M\models(\varphi\to\psi)$. Lastly, we show (c) implies (b): given our special $\varphi$, we want to show that $\{P\}=[\varphi]$. Well, certainly $P\in[\varphi]$. Conversely, if $Q\ne P$ for some complete $n$-type $Q$, pick up $\psi\in P\setminus Q$, but then $\mathcal M\models(\varphi\to\psi)$, so by the claim, $[\varphi]\subseteq[\psi]$, but then $Q\notin[\psi]$, so $Q\notin[\varphi]$.
\end{proof}
\begin{remark}
	The point is that isolated types are determined by a single formula. Note that the formula $\varphi$ yielding $P$ is unique up to equivalence by (c) because then $\op{Th}_A(\mathcal M)\models(\varphi\leftrightarrow\psi)$ if $\{P\}=[\varphi]=[\psi]$.
\end{remark}
\begin{example}
	Let $\mathcal M=(\RR,0,1,+,\times,\le)$. Then $P=\op{tp}_\emp^{\mathcal M}(A)$ is isolated given by formula $x=0$.
\end{example}
More generally, we have the following.
\begin{proposition}
	Fix an $\mathcal L$-structure $\mathcal M$ and a subset $A\subseteq M$. Suppose that $\overline b\in M$ is definable over $A$. Then $\op{tp}_A^{\mathcal M}(\overline b)$ is an isolated type.
\end{proposition}
\begin{proof}
	Well, suppose $\varphi(\overline x)$ defines $\overline b$, and we claim that $\op{tp}_A^{\mathcal M}(\overline b)=[\varphi]$, for which we use \Cref{prop:how-to-isolated-type}. Well, we see that $\psi(\overline x)\in\op{tp}_A^{\mathcal M}(\overline b)$ if and only if $\mathcal M\models\psi(\overline b)$ if and only if $\mathcal M\models\forall\overline x(\varphi(\overline x)\to\psi(\overline x))$, which is equivalent to $\mathcal M\models(\varphi\to\psi)$. To finish, we note that this is equivalent to $\varphi\to\psi$ living in $\op{Th}_A(\mathcal M)$, which is equivalent to $\op{Th}_A(\mathcal M\models(\varphi\to\psi)$ by completeness.
\end{proof}
In fact, we have the following partial converse.
\begin{proposition} \label{prop:isolated-is-realized}
	Fix an $\mathcal L$-structure $\mathcal M$ and a subset $A\subseteq M$. If $P$ is an isolated type, then $P=\op{tp}_A^{\mathcal M}(\overline a)$ for some $\overline a$.
\end{proposition}
\begin{proof}
	Suppose that $\{P\}=[\varphi]$. But now $\op{Th}_A(\mathcal M)\cup\{\exists\overline x\,\varphi(\overline x)\}$ is satisfiable because this is the same as $\op{Th}_A(\mathcal M)\cup P$. So $\exists\overline x\,\varphi(\overline x)\in\op{Th}_A(\mathcal M)$ by completeness, so $\mathcal M\models\varphi(\overline a)$ for some $\overline a\in M$. To complete the argument, we note that $\psi\in P$ if and only if $\op{Th}_A(\mathcal M)\models(\varphi(\overline a)\to\psi(\overline a))$, so $\mathcal M\models\psi(\overline a)$, so $\psi\in\op{tp}_A^{\mathcal M}(\overline a)$. So $P\subseteq\op{tp}_A^{\mathcal M}(\overline a)$, and equality follows by completeness.
\end{proof}
\begin{example}
	One can use the above proposition to show that there are types which aren't isolated in $\mathcal M=(\RR,0,1,+,\times,\le)$. For example, take the type given by any transcendental.
\end{example}

\end{document}