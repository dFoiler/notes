% !TEX root = ../notes.tex

\documentclass[../notes.tex]{subfiles}

\begin{document}

\section{November 21}

Today we will finish our discussion of saturated models.

\subsection{Miscellaneous Saturated Models}
We begin with the following lemmas.
\begin{lemma}
	Fix a complete $\mathcal L$-theory $T$. If $\mathcal M$ is $\kappa$-saturated, then $\mathcal M$ is $\kappa$-homogeneous.
\end{lemma}
\begin{proof}
	Fix some subset $A\subseteq M$ of cardinality less than $\kappa$ and a partial elementary embedding $f\colon A\to M$ which we would like to extend to an element $a\in M$.

	To do this, set $p(x)\coloneqq\op{tp}^\mathcal M_A(a)$. We now push the parameters to $B$, defining $q(x)$ to be the set of formulae defined by taking any constants from $A$ in any formula $\varphi(x)\in p(x)$ and pushing them to $B$. We would like to show that $q(x)$ is actually a type (with parameters in $B$), and $\mathcal M$ being $\kappa$-saturated implies that it will be realized by some $b\in\mathcal M$, which will complete the argument by extending $f$ to send $a\mapsto b$.

	So it remains to show that $q(x)$ is consistent with $\op{Th}_B\mathcal M$. Well, it suffices to show that $q(x)\cup\op{Th}_B\mathcal M$ is satisfiable, for which it is enough to show that it is finitely satisfiable, so it is enough to show that any single formula $\varphi(x,\overline b)\in q(x)$ is consistent with $\op{Th}_B\mathcal M$. Well, this is $\varphi(x,f(\overline a))$ for some $\overline a\in A$, and $f$ being a partial elementary embedding then implies that $\varphi(x,f(\overline a))\in\op{Th}_B\mathcal M$ is equivalent to $\varphi(x,\overline a)\in\op{Th}_A\mathcal M$, which is true by construction of $\varphi(x,\overline b)\in q(x)$.
\end{proof}
\begin{lemma} \label{lem:saturated-is-uniq}
	If $\mathcal M$ and $\mathcal N$ are saturated models of the same cardinality, then $\mathcal M\cong\mathcal N$.
\end{lemma}
\begin{proof}
	This is a back-and-forth argument proceeding via transfinite induction. Let $\kappa$ be the cardinality of $M$ and $N$, and enumerate $M=\{m_\alpha:\alpha\in\kappa\}$ and $N=\{n_\alpha:\alpha\in\kappa\}$. Then one proceeds as in \Cref{prop:prime-is-atomic}.
\end{proof}
\begin{remark}
	The above proof actually shows that any partial elementary embedding $f_0\colon A\to\mathcal N$ with $\left|A\right|<\kappa$, then we can extend $f_0$ to an isomorphism. Indeed, just replace the $0$th step of the transfinite induction with $f_0$.
\end{remark}
While we're here, we also show \Cref{thm:saturated-is-univ-homo}.
\satisunivhomo*
\begin{proof}
	We already show that $\kappa$-saturated implies $\kappa$-homogeneous. To show that $\kappa$-saturated implies $\kappa^+$-universal, argue as in \Cref{lem:saturated-is-uniq} but only use embeddings in one direction (i.e., do the forth but not the back).

	We now show the other direction; choose $\mathcal M$ as in (b). We would like to show that $\mathcal M$ realizes any type $P(x)\in S_1^\mathcal M(A)$ for some $A\subseteq M$ of cardinality less than $\kappa$. By \Cref{prop:realize-type-in-elem-ext}, we know that $p(x)$ will be realized in some elementary extension $\mathcal N$ of $\mathcal M$, and \Cref{thm:down-skolem} allows us to reduce the size of $\mathcal N$ back down to $\kappa$, so $\kappa^+$-universality grants an elementary embedding $\varphi\colon\mathcal N\into\mathcal M$. However, this elementary embedding need not fix $A$, so we use the homogeneity of $\mathcal M$ to move $f(A^{\mathcal N})$ back to $A\subseteq M$, and extending this up to automorphism moves the element realizing $p(x)$ in $\mathcal N$ to the needed element of $\mathcal M$.
\end{proof}

\subsection{Construction of Saturated Models}
We now have the following step to construct saturated models.
\begin{lemma} \label{lem:almost-saturated-model}
	Fix an infinite cardinal $\kappa$ and a language $\mathcal L$ with $\left|\mathcal L\right|<\kappa$ and a complete $\mathcal L$-theory $T$. For any model $\mathcal M$, there is a $\kappa^+$ saturated model $\mathcal N$ which is an elementary extension of $\mathcal M$ and with $\left|N\right|\le\left|M^\kappa\right|$.
\end{lemma}
\begin{proof}
	As an intermediate step, we claim that there is an elementary extension $\mathcal M'\ge\mathcal M$ which realizes any type $p(x)\in S_1^\mathcal M(A)$ where $A\subseteq M$ is a subset of cardinality $\left|A\right|\le\kappa$. We will do this by compactness: let $T_0$ be the theory of $\op{elDiag}\mathcal M$ plus the sentences in all the types $p(c_p)$ we need to satisfy, where $c_p$ is a new constant we added. We want to show that $T_0$ is satisfiable, for which we note that it is enough to check that only finitely many sentences in finally many of these types is satisfiable with $\op{elDiag}\mathcal M$, for which we use \Cref{prop:realize-type-in-elem-ext}.
	
	We take a moment to recognize that there are $\left|M\right|^\kappa$ subsets $A$ available, and for each subset $A$, there are $2^{\left|A\right|+\left|\mathcal L\right|+\aleph_0}=2^\kappa$ possible subsets of sentences (and hence possible types), so the number of added constants in the above construction was at most
	\[\left|M\right|^\kappa\cdot2^\kappa=\left|M\right|^\kappa.\]
	So our $\mathcal M'$ may have size at most $\left|M\right|^\kappa$.

	We now define an elementary sequence using the above steps
	\[\underbrace{\mathcal M_0}_{\mathcal M}\le\mathcal M_1\le\mathcal M_2\le\cdots\le\mathcal M_\alpha\le\cdots\]
	for each $\alpha\in\kappa^+$, where at successor stages we use the above claim, and at limit stages we take unions (which remains an elementary extension due to the chain). Then we define $\mathcal N_{\kappa^+}$ as the union of all these chains.
	
	We take a moment to compute the size of these models. Looking at our stages, we claim that $\left|N_\alpha\right|\le\left|M\right|^\kappa$ by induction: at limit stages, we are taking the union of at most $\kappa$-many sets of cardinality at most $\left|M\right|^\kappa$, which is okay; at successor steps, we note that
	\[\left|N_{\alpha+1}\right|^\kappa\le\left|N_\alpha\right|^\kappa\le\left|M\right|^{\kappa\times\kappa}=\left|M\right|^{\kappa}\]
	by the induction.

	It remains to check that $\mathcal N_{\kappa^+}$ is $\kappa^+$-saturated. Well, for any subset $A\subseteq\mathcal N_{\kappa^+}$ of size at most $\kappa$, there is some $\alpha<\kappa^+$ such that $A\subseteq\mathcal N_\alpha$,\footnote{Namely, finding a bijection of $A$ to a subset $\lambda\subseteq\kappa^+$, of cardinality at most $\kappa$, we note that the supremum of $\lambda$ is an ordinal which is the union of all elements in $A$, which has cardinality at most $\kappa\times\kappa=\kappa$, so we must have $\lambda<\kappa^+$.} but then $\mathcal N_{\alpha+1}$ realizes any type with parameters in $A$.
\end{proof}
\begin{theorem}
	Fix an infinite cardinal $\kappa$ and a language $\mathcal L$ with $\left|\mathcal L\right|<\kappa$ and a complete $\mathcal L$-theory $T$.
	\begin{itemize}
		\item If $\kappa^+=2^\kappa$, then there is a saturated model of cardinality $\kappa^+$.
		\item If $\lambda$ is infinite, and each $\tau<\lambda$ has $2^\tau\le\lambda$, then there is a saturated model of cardinality $\lambda$.
	\end{itemize}
\end{theorem}
\begin{proof}
	Note $2^\kappa=\kappa^\kappa$, so one can iterate the construction of \Cref{lem:almost-saturated-model} to produce the needed model. Namely, in the first case, there is nothing to do because being $\kappa^+$-saturated implies just being saturated by the size condition. For the second case, one builds a sequence $\{\mathcal M_\alpha:\alpha\le\lambda\}$ where $\mathcal M_\alpha$ is $\aleph_\alpha^+$-saturated but $\left|\mathcal M_\alpha\right|\le\lambda$ and then take the union.
\end{proof}
\begin{remark}
	Some cardinal arithmetic shows that cardinals satisfying the second case exist. For example, one can find $\lambda$ with $\aleph_\lambda=\lambda$ (take the limit of $\aleph_0,\aleph_{\aleph_0},\aleph_{\aleph_0},\ldots$), where the result is true.
\end{remark}

\end{document}