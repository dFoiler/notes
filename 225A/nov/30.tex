% !TEX root = ../notes.tex

\documentclass[../notes.tex]{subfiles}

\begin{document}

\section{November 30}

It's the last class for this course.

\subsection{Skolem Functions}
We note that we can always add in Skolem functions to a theory.
\begin{proposition}[Skolemization] \label{prop:skolemize}
	Fix a language $\mathcal L$ and theory $T$. Then we can extend the language and theory to $\mathcal L^*$ and $T^*$, respectively, so that $T^*$ has built-in Skolem functions.
\end{proposition}
\begin{proof}
	Set $\mathcal L_0\coloneqq\mathcal L$ and $T_0\coloneqq T$. Then for each $i$, we can make $\mathcal L_{i+1}$ from $\mathcal L_i$ by adding a function symbol $f_\varphi$ for each formula $\varphi(\overline x,y)$ and then make $T_{i+1}$ from $T_i$ by adding the sentences
	\[\forall\overline x(\exists y\,\varphi(\overline x,y)\to\varphi(\overline x,f(\overline x))).\]
	Taking the union over all the $\mathcal L_i$ and $T_i$ will produce the needed $\mathcal L^*$ and $T^*$.
\end{proof}
\begin{remark}
	A theory $T$ with Skolem functions has quantifier elimination: any formula $\varphi(\overline x,y)$ with Skolem function $f_\varphi$ has
	\[T\models\forall\overline x(\exists y\,\varphi(\overline x,y)\leftrightarrow\varphi(\overline x,f_\varphi(\overline x))),\]
	effectively eliminating the outer existential from $\exists y\,\varphi(\overline x,y)$. So we can eliminate quantifiers, by hand, one at a time. For example, it follows that $T$ is model-complete, recovering \Cref{rem:skolem-implies-model-complete}.
\end{remark}
Anyway, we are permitted the following definitions.
\begin{definition}[Skolemization]
	Fix an $\mathcal L$-theory $T$. Then a \textit{Skolemization} of $T$ is an extension $T^*$ of $T$ to a new language $\mathcal L^*\supseteq\mathcal L$ which has built-in Skolem functions.
\end{definition}
To continue, we pick up the following definition.
\begin{definition}
	Fix an $\mathcal L$-theory $T$ with built-in Skolem functions. Given a subset $X$ of a model $\mathcal M\models T$, we let $H(X)$ be the $\mathcal L$-substructure generated by $X$.
\end{definition}
\begin{remark}
	Explicitly, the universe of $H(X)$ is the collection of terms whose inputs come from $X$.
\end{remark}
\begin{remark}
	By \Cref{rem:skolem-implies-model-complete}, we note that $H(X)\le\mathcal M$ in the above situation.
\end{remark}

\subsection{Building Automorphisms}
We now get the following theorem on automorphisms of indiscernibles.
\begin{theorem} \label{thm:get-autos}
	Fix an $\mathcal L$-theory $T$ with built-in Skolem functions. Further, fix a model $\mathcal M$ and a sequence of indiscernibles $\{x_i\}_{i\in I}$, with $X\coloneqq\{x_i:i\in I\}$. Given any order-preserving automorphism $\sigma\colon I\to I$, there is a unique automorphism $\tau\colon H(X)\to H(X)$ such that $\tau(x_i)=x_{\sigma(i)}$ for each $i\in I$.
\end{theorem}
\begin{proof}
	The uniqueness is easier, and only needs us to preserve the relevant structure (namely, not bijectivity): any element of $H(X)$ is of the form $t(\overline x)$ where $t$ is a term and $\overline x\in X$, but then any automorphism $\tau\colon H(X)\to H(X)$ with $\tau(x_i)=x_{\sigma(i)}$ must have
	\[\tau(t(\overline x))=t(\tau(\overline x)),\]
	thus defining $\tau$.

	For existence, we need to show that the $\tau$ defined above is well-defined and an automorphism. Here are our checks.
	\begin{itemize}
		\item Suppose we have equal terms $t(x_{i_1},\ldots,x_{i_k})=s(x_{j_1},\ldots,x_{j_\ell})$, and we need to show that
		\[t(x_{\sigma i_1},\ldots,x_{\sigma i_k})\stackrel?=s(x_{\sigma j_1},\ldots,x_{\sigma j_\ell}).\]
		Well, this follows from $X$ being a sequence of indiscernibles: we see that the sentence
		\[t(y_{i_1},\ldots,y_{i_k})=s(y_{j_1},\ldots,y_{j_\ell})\]
		is going to be in the type of $\{x_{i_1},\ldots,x_{i_k},x_{j_1},\ldots,x_{j_\ell}\}$ (suitably ordered and so on), so because applying $\sigma$ (which is order-preserving!) cannot change the type, we see that the sentence still holds upon applying $\sigma$, achieving the needed equality.
		\item We show that $\tau$ preserves the structure. Essentially the same argument goes through. For example, $\mathcal M\models R(t_1(\overline x),\ldots,t_n(\overline x))$ is equivalent to $\mathcal M\models R(t_1(\sigma\overline x),\ldots,t_n(\sigma\overline x))$ because we are asking for $R(t_1(\overline x),\ldots,t_n(\overline x))$ to be in the type of $\overline x$, which does not change because $X$ is a sequence of indiscernibles, and $\sigma$ is order-preserving.
		\item We show that $\tau$ is an isomorphism. Well, we can define some automorphism $\tau'$ which sends $x_i$ to $x_{\sigma^{-1}i}$ using the argument above, but then $\tau$ and $\tau'$ are inverse maps of structures: $\tau\circ\tau'$ and $\tau'\circ\tau$ are morphisms sending $x_i\mapsto x_i$ and so must be the identity by uniqueness, so we do indeed have an automorphism.
		\qedhere
	\end{itemize}
\end{proof}
As a corollary, we get to build models that have lots of automorphisms!
\begin{corollary}
	Fix an $\mathcal L$-theory $T$ with $\mathcal L$ countable, where $T$ has infinite models. For any $\kappa\ge\aleph_0$, there is a model $\mathcal M\models T$ of cardinality $\kappa$ and $2^\kappa$ automorphisms.
\end{corollary}
\begin{proof}
	We may as well extend $T$ to be a complete theory which has built-in Skolem functions by taking the Skolemization. (Note that the language remains countable due to the construction in \Cref{prop:skolemize}.) Now, chose a linear ordering $(I,\le)$ of cardinality $\kappa$, and then \Cref{thm:indisc-exist} promises the existence of a model $\mathcal M\models T$ of cardinality $\kappa$ with a sequence of indiscernibles $\{x_i:i\in I\}$. We now may replace $\mathcal M$ with $H(\{x_i:i\in I\})$, which we note still has cardinality $I$.

	Now, \Cref{thm:get-autos} promises that $\mathcal M=H(\{x_i:i\in I\})$ will have at least as many automorphisms as order-preserving maps $I\to I$, so it remains to choose $I$ so that $I$ has $2^\kappa$ order-preserving maps $I\to I$. Well, choose $I\coloneqq\ZZ\times\kappa$ to be $\kappa$ many copies of $\ZZ$, named $\ZZ_\alpha$ for $\alpha\in\ZZ$. Then for any subset $S\subseteq\kappa$, we can build an order-preserving map $\sigma_S$ by applying a $+1$ shift to $\ZZ_\alpha$ for each $\alpha\in S$ and do nothing to $\ZZ_\alpha$ for $\alpha\notin S$.
\end{proof}
Here is another application.
\begin{lemma}
	Fix an $\mathcal L$-theory $T$, and let $X=\{x_i:i\in I\}$ be a sequence of indiscernibles. For any linear ordering $J$, there is a model $\mathcal N$ with a sequence of indiscernibles $Y=\{y_j:j\in J\}$ such that
	\[\op{tp}^\mathcal M(X)=\op{tp}^\mathcal N(Y).\]
	Further, if $T$ has built-in Skolem functions and $\mathcal M$ omits any type $p(x)$, then we may require that $\mathcal N$ omits the same type $p(x)$.
\end{lemma}
\begin{proof}
	We use compactness. For the first claim, build $\mathcal N$ by compactness, as in \Cref{thm:indisc-exist}. Namely, add in constants $\{c_j:j\in J\}$ and force $\mathcal N$ to satisfy $\op{elDiag}\mathcal M$ along with the sentences $\varphi(c_{j_1},\ldots,c_{j_n})$ when $j_1<\cdots<j_n$ and $\varphi$ lives in $\op{tp}^\mathcal M(X)$. This is finitely satisfiable by $\mathcal M$ because any finite segment of $J$ looks just like a finite segment of $I$.

	To get the second claim, move everything up to a Skolemization first and then replace the given model $\mathcal N$ with $H(Y)$. Indeed, the type of any sequence of elements in $\mathcal N$ only uses finitely many elements of $\mathcal N$, so we can move the used subsequence of $y_\bullet$s back to $x_\bullet$s to show that the types in $\mathcal N$ are a subset of the types in $\mathcal M$.
\end{proof}
\begin{theorem}
	Fix a countable language $\mathcal L$ and an $\mathcal L$-theory $T$ with built-in Skolem functions. Further, suppose that any $\alpha\in\omega_1$ has a model $\mathcal M\models T$ of size at least $\beth_\alpha$ which omits a given type $p(x)\in S_1(T)$. Then there is a model $\mathcal N\models T$ still omitting $p(x)$ with an infinite sequence of indiscernibles.
\end{theorem}
\begin{proof}
	Intuitively, we will pick up some large subset of $\mathcal M$ and slowly make it smaller and smaller in order to avoid the type and all satisfy the same formulae (to be a sequence of indiscernibles). The process of making it smaller but maintaining an infinite size is some kind of coloring problem with every formula colored by which formulae they satisfy. So we will want the following purely combinatorial result.
	\begin{theorem}
		Fix a set $B$ of size $\beth_n(\kappa)^+$ and a coloring $c\colon[B]^{n+1}\to\kappa$. Then there is a monochromatic subset $A\subseteq B$ of cardinality $\kappa^+$.
	\end{theorem}
	Here, we recall that $\beth_n$ means repeating $\mathcal P$ a total of $n$ times.

	Let's provide a few more details. Add in constants $\{c_i:i\in\NN\}$ to our language. To begin our compactness argument, we add to our theory $\Sigma\supseteq T$ the requirements that each the $c_i$ are distinct and that we are making the $\{c_i:i\in\NN\}$ into a sequence of indiscernibles. This is still satisfiable, so we will try to complete $\Sigma$ in a way that has this continue to be satisfiable. Now, the main idea is to add in the constraints that each term $t(\overline v)$ has $\varphi(y)\in p(y)$ such that $\lnot\varphi(t(\overline c))$.

	To do this construction, we will build $\Sigma$ inductively via a sequence $\Sigma_0\subseteq\Sigma_1\subseteq\cdots$. At each step of the construction $\alpha\in\omega_1$, we ensure that we have a model $\mathcal N_\alpha$ omitting $p(x)$ but satisfying $\Sigma_s$, and we ensure that $\mathcal N_\alpha$ has a sequence of $\beth_\alpha$ indiscernibles.
\end{proof}
\begin{remark}
	The above theorem answers many of the questions of \Cref{subsec:motivate-indisc} on trying to construct models with lots of automorphisms but still omitting a type.
\end{remark}

\end{document}