% !TEX root = ../notes.tex

\documentclass[../notes.tex]{subfiles}

\begin{document}

\section{November 16}

Today we continue with saturated models.

\subsection{Construction of Saturated Models}
Whenever necessary, we will continue with $\mathcal L$ being a countable language and $T$ being a complete $\mathcal L$-theory with infinite models.
\begin{example}
	Let's describe a model of $\mathrm{DLO}$ which is $\aleph_1$-saturated. Let $\mathcal M$ consist of functions $f\colon\omega_1\to\QQ$ with countable support, ordered lexicographically: namely, $f<g$ if and only if the least $i$ with $f(i)\ne g(i)$ has $f(i)<g(i)$. This is a dense linear order (between any two functions $f$ and $g$, they differ at some least point $i$, and so define a function between the two by sending $i\mapsto\frac12(f(i)+g(i))$).

	To see that $\mathcal M$ is $\aleph_1$-saturated, one makes some argument using the countable support. Now, we note that it has cardinality $2^{\aleph_0}$: there are certainly at least $2^{\aleph_0}$ functions, and the number of functions can be upper-bounded by the number of countable ``graphs'' living in $\omega_1\times\QQ$, which simply has size $\aleph_1^{\aleph_0}\le2^{\aleph_0\times\aleph_0}=2^{\aleph_0}$. So under the continuum hypothesis, $\mathcal M$ achieves a cardinality of $\aleph_1$, so $\mathcal M$ is indeed saturated.
\end{example}
\begin{example}
	The algebraically closed field $\CC$ is $\aleph_1$-saturated. The point is that quantifier elimination means that we only need to worry about types which either say that an element is a root of some polynomial or avoids being a root of some set of polynomials with parameters in a countable set. But $\CC$ is big enough that it has enough transcendental elements, so we are okay.
\end{example}
Here are some other ways of thinking about saturated types.
\begin{proposition}
	Fix an $\mathcal L$-structure $\mathcal M$ and an infinite cardinal $\kappa$. Then the following are equivalent.
	\begin{listalph}
		\item $\mathcal M$ is $\kappa$-saturated.
		\item For all $A\subseteq M$ of cardinality less than $\kappa$, all (possibly partial) types $P$ with parameters in $A$ are realized in $\mathcal M$.
		\item For all $A\subseteq M$ of cardinality less than $\kappa$, any type $P\in S_1^\mathcal M(A)$ is realized in $\mathcal M$.
	\end{listalph}
\end{proposition}
\begin{proof}
	Of course, (b) implies (a) with no content because complete types are types. To show that (a) implies (b), note that a type can be extended to a complete type, so having (a) is enough to realize the needed partial type.

	Similarly, we note that (a) implies (c) again with no content. For the converse, suppose we have a complete $n$-type $P(\overline x)\in S_n^\mathcal M(A)$ which we want to realize. Then we will add in one variable at a time. Namely, start with $P_1(x_1)\in S_1^\mathcal M(A)$ be the type of formulae in $P$ only mentioning $x_1$, and we see that this is a complete $1$-type, which is then realized in $\mathcal M$ by some $a_1$.

	Now, because $\kappa$ is infinite, we may consider the complete type $P_2(a_1,x_2)\in S_1^\mathcal M(A\cup\{a_1\})$ of formulae in $P$ only mentioning $x_1$ and $x_2$ but with $x_1$ replaced with $a_1$. Notably, $P_2(a_1,x_2)$ is indeed consistent with $\mathcal M$: for each $\varphi(x_1,x_2)\in P_2(x_1,x_2)$, we see that $\exists x_2\,\varphi(x_1,x_2)$ would need to live in $P_1$, so $\mathcal M\models\exists x_2\,\varphi(a_1,x_2)$, so $\varphi(a_1,x_2)$ is consistent. So we may find $a_2$ so that $(a_1,a_2)$ realizes $P_2$. From here, we can continue the argument inductively up to $P_3$ and so on.
\end{proof}
\begin{remark}
	We only needed $\kappa$ to be infinite in order to show that (c) implies (a).
\end{remark}
Let's go ahead and start to construct saturated models.
\begin{example}
	Consider the theory $\mathrm{ACF}_0$ of algebraically closed fields of characteristic $0$. Then an extension $F/\overline\QQ$ with countable transcendence degree will be countable and so in particular countable and saturated. The point is that we are only going to consider parameter sets $A$ which are finite, and then $F$ realizes all types to consider, which by quantifier elimination either is asking to be the root of some polynomial in $A$ or to be transcendental with respect to parameters in the finite set $A$.
\end{example}
\begin{theorem}
	Fix a countable language $\mathcal L$ and a complete $\mathcal L$-theory $T$. Then $T$ has a countable saturated model if and only if $S_n(T)$ is countable.
\end{theorem}
\begin{proof}
	The forward direction is easier: if $T$ has a countable saturated model $\mathcal M$, then each distinct complete $n$-type $P$ must correspond to a distinct $n$-tuple in $\mathcal M$ (note distinct complete types have distinct sentences, so they cannot be satisfied by the same $n$-type!), of which there are only countably many, so there can only be countably many types.

	We now show the reverse direction. Suppose we have a countable model $\mathcal M\models T$. Then we claim that there is a countable model $\mathcal M'\ge\mathcal M$ realizing any countable set of types. Indeed, add in (countably many) new constants $\overline c^P$ corresponding to each of the countably many types $P(\overline x)$, and everything is finitely satisfiable by $\mathcal M$ because $T$ is complete. So we can find a model $\mathcal M'$ satisfying everything including $\op{elDiag}\mathcal M$.

	Iterating this construction grants an ascending sequence
	\[\mathcal M_0\le\mathcal M_1\le\mathcal M_2\le\cdots,\]
	where $\mathcal M_{i+1}$ is constructed from $\mathcal M_i$ by asking for all the types in $S_1^{\mathcal M_i}(A)$ to be realized for each finite subset $A\subseteq M_i$. The key point is that $S_1^{\mathcal M_i}(A)$ is still countable: for each $1$-type $P(x,\overline a)\in S_1^{\mathcal M_i}(A)$ (where $\overline a$ is an enumeration of $A$), we can turn this into an $(n+1)$-type $P(x,\overline y)\in S_{n+1}(T)$, of which there are only countably many. This function is countable-to-one, so we conclude that there are only countably many types in $S_1^{\mathcal M_i}(A)$.

	To complete the proof, we let $\mathcal M$ be the union of the $\mathcal M_i$s, which is an elementary extension of all the $\mathcal M_i$, so $\mathcal M$ will realize all the types from all the $\mathcal M_i$. Now $\mathcal M$ is countable, so we still only need to worry about finite parameter sets $A\subseteq M$, which means that we might as well put the parameter set inside some fixed level $\mathcal M_i$, so we know the types will be realized in $\mathcal M_{i+1}$.
\end{proof}
\begin{remark}
	In fact, the above proof has shown that any countable model $\mathcal M_0$ can be embedded (in an elementary way) into a saturated countable model (provided a single saturated countable model exists).
\end{remark}

\subsection{Universal Models}
The above proof motivates the following definition.
\begin{defihelper}[$\kappa$-universal] \nirindex{Kappa-universal@$\kappa$-universal}
	Fix a complete $\mathcal L$-theory $T$. A model $\mathcal M$ of $T$ is \textit{$\kappa$-universal} if and only if any model $\mathcal N$ of $T$ of cardinality less than $\kappa$ has some elementary embedding $\mathcal N\into\mathcal M$. Then we say $\mathcal M$ is \textit{universal} if only if it is $\left|M\right|^+$ universal.
\end{defihelper}
\begin{example}
	Consider the theory $T$ of infinitely many equivalence classes with infinite size. Then any model $\mathcal M$ with $\aleph_1$ many classes of size $\aleph_1$ will be universal by embedding in the necessary number of equivalence classes. Notably, this example informs us that universal models are not isomorphic despite having some kind of universal property.
\end{example}
So in the above language, we now know that countable and saturated implies universal. More generally, we have the following.
\begin{theorem}
	Fix a complete $\mathcal L$-theory $T$ and a model $\mathcal M$ of $T$. Then the following are equivalent.
	\begin{listalph}
		\item $\mathcal M$ is saturated.
		\item $\mathcal M$ is universal and homogeneous.
	\end{listalph}
\end{theorem}
We will show this next class.

\end{document}