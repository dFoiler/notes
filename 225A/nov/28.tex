% !TEX root = ../notes.tex

\documentclass[../notes.tex]{subfiles}

\begin{document}

\section{November 28}

The final exam will be released Wednesday the 13th of December morning (at 5AM) and due on Friday the 15th of December at 10PM. It will be released and submitted by email. It is expected to be the length of a regular exam, open-note.

\subsection{Primer on Indiscernibles} \label{subsec:motivate-indisc}
We are going to do a little from chapter 5, which is on indiscernibles. Let $T$ be an $\mathcal L$-theory, where $\mathcal L$ is a countable language. Recall that if $T$ has an infinite model, compactness implies that $T$ has an infinite model of cardinality $\kappa$ for each infinite cardinal $\kappa$. However, if we require that $T$ omit some type, then we may no longer have arbitrarily large models.
\begin{example}
	Take the language $\mathcal L$ consisting of countably many constants $c_i$ for each $i\in\NN$, where $T$ is the theory that $c_i\ne c_j$ whenever $i\ne j$. Then omitting the type $p(x)$ given by $x\ne c_i$ for each $i\in\NN$ requires that any model of $T$ is now countable!
\end{example}
\begin{example}
	Take the language $\mathcal L$ consisting of countably many constants $c_i$ for each $i\in\NN$, and we add in a unary relation $\omega$ and a binary relation $E$. The point is that we will require models to look like they have some $\omega$ and also subsets of $\omega$. With this in mind, we require our theory $T$ to have the following.
	\begin{itemize}
		\item Constants are distinct: $c_i\ne c_j$ for each $i\ne j$.
		\item Constants are in $\omega$: $\omega(c_i)$ for each $i$.
		\item Subsets are outside $\omega$: for all $x$ and $y$ with $xEy$, we have $\omega(x)$ and $\lnot\omega(y)$.
		\item Subsets satisfy extensionality: given $x$ and $y$ with $\lnot\omega(x)\land\lnot\omega(y)$, if $wEx\liff wEy$ for each $w$, then $x=y$.
	\end{itemize}
	We now ask to omit the type $p(x)$ given by requiring $\omega(x)$ and $x\ne c_i$. This means that any model $\mathcal M$ of $T$ omitting $p(x)$ has that $\mathcal M\models\omega(a)$ implies $a=c_i^\mathcal M$ for any $a\in M$. Now every element outside $\omega$ in $M$ can be read off as living in $\mathcal P(M)\subseteq\mathcal P(\omega)$.
\end{example}
One can also iterate the above example, adding in a relation for subsets of subsets of $\omega$ and so on. Namely, there is a language $\mathcal L$ and theory $T$ such that omitting a given type $p(x)$ has only models of size at most $\beth_\alpha$ for $\alpha<\omega$. With some effort, we can extend this to any countable $\alpha$. However, we will show that if $T$ has a model of size $\beth_{\omega_1}$ omitting a given type $p(x)$, then $T$ has a model of any larger size.

As another motivation, take $T$ to be a theory in a countable language. We would like models with ``lots'' of automorphisms.
\begin{example}
	Consider $T=\op{Th}(\NN,0,1,+,\cdot,<)$. There are only countably many types, so find a model $\mathcal M$ of $T$ of cardinality $\aleph_1$; then we can surely find two elements of the same type, so the two elements have an automorphism between them in an elementary extension by \Cref{prop:type-gives-aut}.
\end{example}
We might want to enforce having countable models, which the above example does not require.

\subsection{Introducing Indiscernibles}
Anyway, let's provide our definition.
\begin{definition}
	Fix a countable language $\mathcal L$ and $\mathcal L$-theory $T$. Further, fix a model $\mathcal M$ and a linear ordering $(I,\le)$. Then $\{x_i\}_{i\in I}\subseteq\mathcal M$ is a \textit{sequence of indiscernibles} if and only if the $x_i$ are distinct, and any ordered sequences $i_1<\cdots<i_n$ and $j_1<\cdots<j_n$ will have
	\[\mathcal M\models\left(\varphi(x_{i_1},\ldots,x_{i_n})\liff\varphi(x_{j_1},\ldots,x_{j_n})\right)\]
	for any $\mathcal L$-formula $\varphi$ with $n$ free variables.
\end{definition}
\begin{example}
	For any dense linear order, any subset will produce a sequence of indiscernibles, using the ordering provided by the linear ordering. Indeed, one can write down an automorphism sending any finite subset to another finite subset.
\end{example}
\begin{example}
	Give $\ZZ\times\ZZ$ the lexicographical ordering. Then $\{(x,0):x\in\ZZ\}$ is a sequence of indiscernibles: there is no first-order way to tell these elements apart because they are already ``infinitely'' apart.
\end{example}
\begin{remark}
	Given a set of indiscernibles $\{x_i:i\in I\}$ as above, one can define its type as
	\[\op{tp}(I)\coloneqq\left\{\varphi(v_1,\ldots,v_n):\mathcal M\models\varphi(x_{i_1},\ldots,x_{i_n})\text{ for any }i_1<\cdots<i_n\right\}.\]
	Note the choice of $x_{i_1},\ldots,x_{i_n}$ does not matter by definition of the indiscernibles.
\end{remark}
Let's try to find indiscernibles.
\begin{theorem} \label{thm:indisc-exist}
	Let $T$ be an $\mathcal L$-theory with infinite models. Fix a linear ordering $(I,\le)$. Then there is a model $\mathcal M\models T$ with a sequence $\{x_i:i\in I\}$ of indiscernibles.
\end{theorem}
This result requires Ramsey's theorem. For notation, let $[X]^n$ denote the sets of $X$ with $n$ elements.
\begin{theorem}[Ramsey] \label{thm:ramsey}
	Any $k$-coloring $c\colon[\NN]^n\to\{0,1,\ldots,k\}$ has an infinite monochromatic set $H\subseteq\NN$. Namely, there is $\ell$ such that any $S\in[H]^n$ will have $c(S)=\ell$.
\end{theorem}
For example, $n=0$ has nothing to show, and $n=1$ is the Pigeonhole principle. If we have time, we will prove \Cref{thm:ramsey} next class. Anyway, here is our proof of \Cref{thm:indisc-exist}.
\begin{proof}[Proof of \Cref{thm:indisc-exist} using \Cref{thm:ramsey}]
	We use compactness. Add to our language new constants $c_i$ for each $i\in I$, and let $T'$ be the theory $T$ adding in the requirements that $c_i\ne c_j$ for each $i\ne j$ and also the sequences
	\[\varphi(c_{i_1},\ldots,c_{i_n})\liff\varphi(c_{j_1},\ldots,c_{j_n})\]
	for each pair of ordered sequences $i_1<\cdots<i_n$ and $j_1<\cdots<j_n$ in $I$. We need to show that $T'$ is satisfiable, so it is enough by compactness to show that $T'$ is finitely satisfiable. Well, any finite subset $\Delta\subseteq T'$ will only mention finitely many formulae $\varphi_1,\ldots,\varphi_k$ and finitely many constants $c_1,\ldots,c_n$.

	We now show that $\Delta$ is satisfiable. Well, pick up some model $\mathcal M$, and we may as well assume that $\mathcal M$ is countable.\footnote{Even if $\mathcal L$ is countable, $\Delta$ only uses a finite subset of this language, so we may restrict $T$ to this language when concerned with the satisfiability of $\Delta$.} We would now like to assign the finitely many constants $c_\bullet$ to distinct places to satisfy $\Delta$. For this, we use \Cref{thm:ramsey}. Namely, order $M$ as $M=\{a_i:i\in\NN\}$, and we define the coloring
	\[c\colon[M]^n\to 2^k,\]
	where $c(\{a_1,\ldots,a_n\})$ is the subset of $\{1,2,\ldots,k\}$ determined by having $\ell$ if and only if $\mathcal M\models\varphi_\ell(a_1,a_2,\ldots)$ and not having $\ell$ otherwise. (The arity of $\varphi_\ell$ matters here, which is why we have written $a_1,a_2,\ldots$.) Now, \Cref{thm:ramsey} produces an infinite subset $H\subseteq M$ such that $c$ is constant on $[H]^m$. Thus, we may send the constants $c_1,\ldots,c_n$ wherever we please in $H$ (ordered properly).
\end{proof}
\begin{remark}
	The above proof in fact promises a model $\mathcal M$ of cardinality $\left|\mathcal L\right|+\left|I\right|+\aleph_0$.
\end{remark}
Let's think a little about using indiscernibles to produce automorphisms. This requires an adjective to our theory.
\begin{definition}[Skolem functions]
	Fix an $\mathcal L$-theory $T$. Then $T$ has \textit{built-in Skolem functions} if and only if any $\mathcal L$-formula $\varphi(\overline x,\overline y)$ has a term $t(\overline x)$ such that
	\[T\models\forall\overline x(\exists\overline y\,\varphi(\overline x,\overline y)\to\varphi(\overline x,t(\overline x))).\]
	We call $t(\overline x)$ a \textit{Skolem function}.
\end{definition}
\begin{remark} \label{rem:skolem-implies-model-complete}
	A theory $T$ with built-in Skolem functions is model-complete: for any substructures $\mathcal M\subseteq\mathcal N$, we can use the Tarski--Vaught test to show $\mathcal M\le\mathcal N$. Namely, looking at \Cref{lem:tarski-vaught}, choosing some formula $\varphi(\overline x,y)$ and $\overline a\in M^n$ with $\mathcal N\models\exists y\,\varphi(\overline a,y)$, then the existential is witnessed by a Skolem function, meaning that we can find it in $\mathcal M$.
\end{remark}
\begin{example}
	Peano arithmetic has ``definable'' Skolem functions by finding the least element satisfying some formula. Adding in function symbols produces an extension of the theory with built-in Skolem functions.
\end{example}

\end{document}