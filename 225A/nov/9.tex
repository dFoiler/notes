% !TEX root = ../notes.tex

\documentclass[../notes.tex]{subfiles}

\begin{document}

\section{November 9}

We continue.
\begin{example}
	Let $\NN=(\NN,0,1,+,\times,\le)$, and choose a countable proper elementary extension $\mathcal M$ of $\NN$, so for example, $\mathcal M$ is still a model of $\mathrm{PA}$. Now, we build a proper elementary extension $\mathcal N$ of $\mathcal M$. For example, for each $m\in M$, we can try to omit the type $p_m$ defined by
	\[\{(v<m)\}\cup\{(v\ne h):h<m\}.\]
	We could then build a model $\mathcal N\models\op{elDiag}\mathcal M$ omitting all the types $p_m$ (which are not isolated because they are not realized in $\mathcal M$).
\end{example}

\subsection{Prime and Atomic Models}
For the next few weeks, we examine prime, atomic, and saturated models.
\begin{definition}[prime]
	Fix an $\mathcal L$-theory $T$. Then a model $\mathcal M$ of $T$ is a \textit{prime model} if and only if $\mathcal N\models T$ implies that there is an elementary embedding $\mathcal M\to\mathcal N$.
\end{definition}
We would like to know when these exist and that they are unique. These require proof.
\begin{example}
	In the theory $\mathrm{ACF}_p$ of algebraically closed fields of characteristic $p$, then $\overline{\FF_p}$ is a prime model: it will embed into any other algebraically closed field of characteristic $p$, and these are elementary extensions because $\mathrm{ACF}_p$ eliminates quantifiers and hence is model-complete.
\end{example}
\begin{example}
	Similarly, the theory $\mathrm{DLO}$ has $\QQ$ as a prime model for essentially the same reason.
\end{example}
\begin{remark}
	A theory $T$ needs to be complete to have prime models. Namely, suppose $\mathcal M$ is a prime model. Then $\mathcal N\models T$ implies that $\mathcal M\le\mathcal N$, so $\op{Th}\mathcal M=\op{Th}\mathcal N$. Thus, $\mathcal M\models\varphi$ if and only if $\mathcal N\models\varphi$ for all models $\mathcal N$ of $T$, which is then equivalent to $T\models\varphi$, so completeness of $T$ follows.
\end{remark}
\begin{remark} \label{rem:prime-implies-atomic}
	Suppose our language $\mathcal L$ is countable. Given a prime model $\mathcal M$ of $T$, then $\op{tp}^{\mathcal M}(\overline a)$ must be isolated. Indeed, if not, then there is a model $\mathcal M'$ omitting $\op{tp}^{\mathcal M}(\overline a)$, but then the promised elementary extension $\mathcal M\le\mathcal M'$ requires there to be an element in $\mathcal M'$ with the same type! Note that this implies that the isolated types are dense in $S_n(T)$, which need not be the case in general.
\end{remark}
And now for atomic models.
\begin{definition}[atomic]
	Fix an $\mathcal L$-theory $T$. Then a model $\mathcal M$ of $T$ is an \textit{atomic model} if and only if $\op{tp}^{\mathcal M}(\overline a)$ is isolated for all $\overline a\in M$.
\end{definition}
Actually, these are the same.
\begin{proposition} \label{prop:prime-is-atomic}
	Let $\mathcal L$ be a countable language, and let $T$ be a complete $\mathcal L$-theory with an infinite model. Then a model $\mathcal M$ is prime if and only if it is atomic and countable.
\end{proposition}
\begin{proof}
	\Cref{rem:prime-implies-atomic} tells us that prime models are atomic, and \Cref{thm:down-skolem} tells us that $T$ does have countable models, so any prime model must be countable in order to embed into such models.
	
	Thus, the difficulty will come from the converse direction. Suppose $\mathcal M$ is atomic and countable, and let $\mathcal N\models T$. We want to show that there is an elementary embedding $\mathcal M\le\mathcal N$. Well, enumerate $\mathcal M$ as $\{m_i:i\in\omega\}$. We will create our elementary embedding inductively: namely, we want to define a sequence $n_0,n_1,\ldots\in N$ such that
	\[\op{tp}^{\mathcal M}(m_0,\ldots,m_k)=\op{tp}^{\mathcal N}(n_0,\ldots,n_k).\]
	This will imply our elementary embedding: for any $\varphi(\overline x)$, we see that $\mathcal M\models\varphi(m_0,\ldots,m_k)$ if and only if $\mathcal N\models\varphi(n_0,\ldots,n_k)$. (Even if a formula $\varphi$ does not use every single variable in $\{m_0,\ldots,m_k\}$, we might as well include them anyway.) Let's do our induction.
	\begin{itemize}
		\item At step $0$, we may find $n_0$ because the isolated type $\op{tp}^{\mathcal M}(m_0)$ is realized in $\mathcal N$ by \Cref{prop:isolated-is-realized}.
		\item At step $k+1$, we want to find $n_{k+1}$ such that
		\[\op{tp}^{\mathcal M}(m_0,\ldots,m_k,m_{m+1})=\op{tp}^{\mathcal M}(n_0,\ldots,n_k,n_{k+1}).\]
		The issue here is that we want to find $n_{k+1}$ without adjusting $n_0,\ldots,n_k$. To get around this, we find $\varphi(x_0,\ldots,x_k)$ isolate $\op{tp}^{\mathcal M}(m_0,\ldots,m_k)$. We want $n_{k+1}$ such that $\mathcal N\models\varphi(n_0,\ldots,n_k)$, but we know that
		\[\mathcal M\models\exists x_{k+1}\,\varphi(m_0,\ldots,m_k,x_{k+1}),\]
		so the sentence $\exists x_{k+1}\,\varphi(x_0,\ldots,x_k,x_{k+1})$ belongs to $\op{tp}^{\mathcal M}(m_0,\ldots,x_k)$, so the inductive hypothesis implies that it belongs to $\op{tp}^{\mathcal N}(n_0,\ldots,n_k)$ too, so
		\[\mathcal N\models\exists x_{k+1}\,\varphi(n_0,\ldots,n_k,x_{k+1}),\]
		which provides us with the needed $n_{k+1}$.
		\qedhere
	\end{itemize}
\end{proof}
If $\mathcal M$ and $\mathcal N$ are both countable and atomic, one can turn the above argument into a genuine back-and-forth, allowing us to conclude that $\mathcal M\cong\mathcal N$.
\begin{proposition} \label{prop:countable-atomic-uniq}
	Let $\mathcal L$ be a countable language, and let $T$ be a complete $\mathcal L$-theory with an infinite model. Then any two countable and atomic models $\mathcal M$ and $\mathcal N$ are isomorphic.
\end{proposition}
\begin{proof}
	Turn \Cref{prop:prime-is-atomic} into a back-and-forth argument. Essentially, enumerate both $\mathcal M$ and $\mathcal N$ and then alternate steps going back and forth to make sure we produce a bijection.
\end{proof}
\begin{theorem}
	Let $\mathcal L$ be a countable language, and let $T$ be a complete $\mathcal L$-theory with an infinite model. Then the following are equivalent.
	\begin{listalph}
		\item $T$ has a prime model.
		\item $T$ has an atomic model.
		\item The isolated types in $S_n(T)$ are dense for all $n$.
	\end{listalph}
\end{theorem}
\begin{proof}
	Note that (a) implies (b) by \Cref{rem:prime-implies-atomic}. For (b) implies (a), we note that any atomic model $\mathcal M$ by \Cref{thm:down-skolem} has some $\mathcal M_0\le\mathcal M$ which is countable, but then all types will remain isolated, so \Cref{prop:prime-is-atomic} completes. Next, (a) implies (c) by \Cref{rem:prime-implies-atomic}.

	Thus, the difficulty comes from showing that (c) implies (b). For the proof, say that a sentence $\varphi(\overline x)$ is \textit{isolating} if and only if $T\models\exists\,\varphi(\overline x)$ and any $\psi(\overline x)$ has either $T\models\forall\overline x(\varphi(\overline x)\to\psi(\overline x))$ or $T\models\forall\overline x(\varphi(\overline x)\to\lnot\psi(\overline x))$. Namely, $\varphi(\overline x)$ implies a complete set of formulae.

	Now, consider the set $P_n$ of formulae $\lnot\varphi(\overline x)$ where this is an isolating formula $\varphi(\overline x)$ with $n$ free variables. This is countably many types, so we would like to use \Cref{rem:omit-countable-types} to provide a model $\mathcal M\models T$ which omits all the types $P_n$. For this, we must check that $P_n$ is not an isolated (partial) $n$-type. (If they are not consistent with $T$, they are omitted automatically, so we don't have to worry about that.)
	
	Well, suppose for the sake of contradiction that $P_n$ is isolated by $\psi(\overline x)$. This means that $T\models\exists\,\psi(\overline x)$ and $T\models(\psi(\overline x)\to\varphi(\overline x))$ for each $\varphi(\overline x)$ in $P$. But by the hypothesis (c), we may find an isolated $n$-type $Q$ containing $\psi(\overline x)$, meaning that it is isolated by the formula $\theta(\overline x)$, so $T\models\forall\overline x(\theta(\overline x)\to\psi(\overline x))$. This is contradiction because $\lnot\theta(\overline x)$ lives in $P_n$, so $T\models\forall\overline x(\psi(\overline x)\to\lnot\theta(\overline x))$, meaning we have shown $\theta$ implies $\lnot\theta$.

	We now complete the proof. \Cref{rem:omit-countable-types} now grants us a model $\mathcal M$ omitting all the types $P_n$. Thus, each $\overline a\in M^n$ cannot realize $P_n$, means that $\mathcal M\models\theta(\overline a)$ for some isolating formula $\theta(\overline x)$, but then $\op{tp}^{\mathcal M}(\overline a)$ must be the type isolated by $\theta(\overline x)$. So $\mathcal M$ is an atomic model.
\end{proof}
\begin{remark}
	Now combining with \Cref{prop:countable-atomic-uniq} assures us that these prime models are in fact unique.
\end{remark}

\end{document}