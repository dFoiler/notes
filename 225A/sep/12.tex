% !TEX root = ../notes.tex

\documentclass[../notes.tex]{subfiles}

\begin{document}

\section{September 12}

We started class by showing that \Cref{thm:los} implies the compactness theorem. Professor Scanlon's proof is distinct from the one in my notes, but I have not bothered to record his proof.

\subsection{Elementary Equivalence}
The following notion will be helpful.
\begin{definition}[theory]
	Fix a language $\mathcal L$ and an $\mathcal L$-structure $\mathcal M$. Then the \textit{theory} $\mathrm{Th}_\mathcal L(\mathcal M)$ is the set of sentences $\varphi$ such that $\mathcal M\models\varphi$. For a subset $A\subseteq M$, we may abbreviate $\op{Th}_{\mathcal L_A}(\mathcal M)$ to just $\op{Th}_A(\mathcal M)$ for brevity.
\end{definition}
The following notions are also sometimes helpful.
\begin{definition}[diagram]
	Fix a language $\mathcal L$ and an $\mathcal L$-structure $\mathcal M$. The \textit{diagram} $\mathrm{Diag}(\mathcal M)$ is the set $\varphi$ of atomic $\mathcal L_M$-sentences (in the expanded language $\mathcal L_M$) or negations of atomic sentences such that $\mathcal M\models\varphi$. The \textit{elementary diagram} is the theory $\mathrm{Th}_{\mathcal L_M}(\mathcal M_M)$.
\end{definition}
The theory is in some sense everything that a structure can see. As such, we make the following definition.
\begin{definition}[elementarily equivalent]
	Fix a language $\mathcal L$. Then two $\mathcal L$-structures $\mathcal M$ and $\mathcal N$ are \textit{elementarily equivalent}, written $\mathcal M\equiv\mathcal N$ if and only if $\mathrm{Th}_\mathcal L(\mathcal M)=\mathrm{Th}_\mathcal L(\mathcal N)$.
\end{definition}
\begin{remark} \label{rem:equiv-by-containment}
	In fact, it is enough to merely have $\mathrm{Th}_\mathcal L(\mathcal M)\supseteq\mathrm{Th}_\mathcal L(\mathcal N)$. Indeed, suppose for the sake of contradiction that $\mathrm{Th}_\mathcal L(\mathcal M)\supsetneq\mathrm{Th}_\mathcal L(\mathcal N)$. Then there is a sentence $\varphi$ with $\mathcal M\models\varphi$ but $\mathcal N$ does not satisfy $\varphi$. But then $\mathcal N\models\lnot\varphi$, so $\mathcal M\models\lnot\varphi$ too! But this does not make sense because $\mathcal M$ cannot satisfy both $\varphi$ and $\lnot\varphi$.
\end{remark}
\begin{proposition} \label{prop:iso-is-equiv}
	Fix a language $\mathcal L$ and isomorphic $\mathcal L$-structures $\mathcal M$ and $\mathcal N$. Then $\mathcal M$ and $\mathcal N$ are elementarily equivalent.
\end{proposition}
\begin{proof}
	We show this by induction. Fix an isomorphism $f\colon\mathcal M\to\mathcal N$. We will actually show that $\mathcal M_M\equiv\mathcal N_M$, where $\mathcal N_M$ means $\mathcal M$ viewed as an $\mathcal L_M$-structure where the constants $a\in M$ are interpreted as $a^\mathcal N\coloneqq f(a)$.

	Anyway, we induct on $\varphi$.
	\begin{itemize}
		\item Suppose that $\varphi$ is atomic of the form $t_1(\overline a)=t_2(\overline a)$. If $\mathcal M_M\models(t_1(\overline a)=t_2(\overline a))$, then an induction on terms $t$ shows that
		\[t^\mathcal N(\overline a)=f\left(t^\mathcal M(\overline a)\right).\]
		Indeed, if $t$ is a constant term, then this follows directly from $f$ being an isomorphism. Otherwise, $t$ takes the form $g(t_1,\ldots,t_n)$ for a function symbol $g$, and the interpretation of $g$ is also respected by $f$ because it is an isomorphism.

		Now, $\mathcal M_M\models(t_1(\overline a)=t_2(\overline a))$ if and only if $t_1^\mathcal M(\overline a)=t_2^\mathcal M(\overline a)$, which is equivalent to $\mathcal N_M\models(t_1(\overline a)=t_2(\overline a))$ by passing through $f$ as above.

		\item Suppose that $\varphi$ is atomic of the form $R(t_1(\overline a),\ldots,t_n(\overline a))$. Well, $\mathcal M_M\models R(t_1(\overline a),\ldots,t_n(\overline a))$ if and only if $\left(t_1^\mathcal M(\overline a),\ldots,t_n^\mathcal M(\overline a)\right)\in R^\mathcal M$, and then passing everything through $f$ shows that this is equivalent to
		\[\left(t_1^\mathcal N(\overline a),\ldots,t_n^\mathcal N(\overline a)\right)\in R^\mathcal N,\]
		which is $\mathcal N_M\models R(t_1(\overline a),\ldots,t_n(\overline a))$.

		\item Suppose that $\varphi$ takes the form $\lnot\psi$. Then the usual semantic argument takes care of us.

		\item Suppose that $\varphi$ takes the form $\psi\land\theta$. Then the usual semantic argument takes care of us.

		\item Suppose that $\varphi$ takes the form $\exists x\,\psi(x)$. Then $\mathcal M_M$ models this if and only if there is some $a\in M$ such that $\mathcal M_M\models\psi(a)$, but $\psi(a)$ is a perfectly valid sentence in our language because we expanded our constants, so this is equivalent to $\mathcal N_M\models\psi(a)$ for some $a\in M$. This last assertion is equivalent to $\mathcal N_M\models\exists x\,\psi(x)$ (the forward direction is clear, and the backward direction is because any $b\in\mathcal N$ witnessing takes the form $f(a)$ for some $a\in\mathcal M$ because $f$ is a bijection on the universe).
	\end{itemize}
	The above induction completes the argument.
\end{proof}
\Cref{prop:iso-is-equiv} is a nice result. We might hope for a converse, but it is false in general. There is a converse for finite structures.
\begin{proposition}
	Fix a finite language $\mathcal L$ and a finite structure $\mathcal M$. Then $\mathcal M\equiv\mathcal N$ if and only if $\mathcal M\cong\mathcal N$.
\end{proposition}
\begin{proof}
	Say that $\mathcal M$ has $n$ elements. Then we build a sentence which asserts that there are exactly $n$ elements $x_1,\ldots,x_n$, and then add on conditions for each $m$-ary function symbol $f$ what $f(x_{i_1},\ldots,x_{i_m})$ should equal, for each $m$-ary function symbol $R$ whether $R(x_{i_1},\ldots,x_{i_m})$ should be, and so on.

	Let's write this out. The start of this sentence
	\[\exists x_1\cdots\exists x_n\left(\Bigg(\bigland_{i\ne j}\lnot(x_i\ne x_j)\Bigg)\land\Bigg(\forall y\biglor_{i=1}^n(y=x_i)\Bigg)\land\cdots\right)\]
	dictates that any model satisfying this sentence has exactly $n$ elements. (Namely, the first part asserts that the model has at least $n$ elements, and the second bit says that any element equals one of the given $n$ elements.) Next we write in function symbols. Enumerate $\mathcal M$ as $a_1,\ldots,a_n$. For each $m$-ary function symbol $f$ in the language $\mathcal L$, and $m$ elements $a_{i_1},\ldots,a_{i_m}$ of $M$, we note that $f^\mathcal M(a_{i_1},\ldots,a_{i_m})$ is some element of $M$, which by abuse of notation we will write as $a_{\overline f(i_1,\ldots,i_m)}$. As such, we next tack on the sentence
	\[\bigland_{m\text{-ary }f}\bigland_{1\le i_1,\ldots,i_m\le n}\left(f(x_{i_1},\ldots,x_{i_m})=x_{\overline f(i_1,\ldots,i_m)}\right).\]
	Next up, we interpret constant symbols: by abuse of notation, let $c^\mathcal M$ be $a_{\overline c}$, so we add on the sentence
	\[\bigland_{c\text{ constant}}(c=x_{\overline c}).\]
	Lastly, we interpret relations: we need the sentence
	\[\bigland_{m\text{-ary }R}\bigland_{\substack{1\le i_1,\ldots,i_m\le n\\R(a_{i_1},\ldots,a_{i_m})}}R(x_{i_1},\ldots,x_{i_m}).\]
	In total, our sentence looks like
	\begin{align*}
		\exists x_1\cdots\exists x_n&\Bigg(\Bigg(\bigland_{i\ne j}\lnot(x_i\ne x_j)\Bigg)\land\Bigg(\forall y\biglor_{i=1}^n(y=x_i)\Bigg) \\
		&\qquad\land\bigland_{m\text{-ary }f}\bigland_{1\le i_1,\ldots,i_m\le n}\left(f(x_{i_1},\ldots,x_{i_m})=x_{\overline f(i_1,\ldots,i_m)}\right) \\
		&\qquad\land\bigland_{c\text{ constant}}(c=x_{\overline c}) \\
		&\qquad\land\bigland_{m\text{-ary }R}\bigland_{\substack{1\le i_1,\ldots,i_m\le n\\R(a_{i_1},\ldots,a_{i_m})}}R(x_{i_1},\ldots,x_{i_m}) \\
		&\qquad\land\bigland_{m\text{-ary }R}\bigland_{\substack{1\le i_1,\ldots,i_m\le n\\\lnot R(a_{i_1},\ldots,a_{i_m})}}\lnot R(x_{i_1},\ldots,x_{i_m}) \Bigg).
	\end{align*}
	Let's quickly explain why this works. Notably, $\mathcal M$ satisfies the above sentence by taking $x_i$ to be $a_i$. On the other hand, for any $\mathcal N$ which is an $\mathcal L$-structure satisfying the above sentence, the first line dictates that $\mathcal N$ must have exactly $n$ elements $b_1,\ldots,b_n$. The second line dictates what $f^\mathcal N(b_{i_1},\ldots,b_{i_m})$ must equal for each $m$-ary function symbol $f$. The third line dictates what $c^\mathcal N$ for each constant symbol $c$. Lastly, the last two lines dictate what $R^\mathcal N(b_{i_1},\ldots,b_{i_m})$ for each $m$-ary relation symbol $R$. Thus, we see that we have an isomorphism $\rho\colon\mathcal M\to\mathcal N$ by $a_i\mapsto b_i$.

	Writing this out a bit, let's check that $\rho$ preserves function symbols. The other checks are no harder. By construction, we see that
	\begin{align*}
		\rho\left(f^\mathcal M(a_{i_1},\ldots,a_{i_m})\right) &= \rho\left(a_{\overline f(i_1,\ldots,i_m)}\right) \\
		&= b_{\overline f(i_1,\ldots,i_m)} \\
		&= f^\mathcal N(b_{i_1},\ldots,b_{i_m}),
	\end{align*}
	which is what we wanted. Notably, the last equality holds because it was required by our sentence.
\end{proof}
\begin{remark}
	The infinite language case might be an interesting question for the midterm exam. The proof should be quite similar.
\end{remark}
Let's verify that infinite structures are not determined by their theories.
\begin{proposition} \label{prop:go-up-elementary-substructure}
	Fix a language $\mathcal L$ and infinite $\mathcal L$-structure $\mathcal M$. Then there exists an $\mathcal L$-structure $\mathcal N$ such that $\mathcal M\not\cong\mathcal N$ but $\mathcal M\equiv\mathcal N$.
\end{proposition}
\begin{proof}
	We will choose $\mathcal N$ to simply be larger than $\mathcal M$. Choose a cardinal $\kappa$ strictly larger than $\left|M\right|$, and let $\mathcal L'$ be an expanded language with $\kappa$ new constants $c_\alpha$ for each $\alpha\in\kappa$.

	We now use compactness to construct $\mathcal N$. Choose the theory $T$ to be
	\[\mathrm{Th}_\mathcal L(\mathcal M)\sqcup\{c_\alpha\ne c_\beta:\alpha\ne\beta\text{ for }\alpha,\beta\in\kappa\}.\]
	We claim that $T$ is finitely satisfiable. Indeed, for any finite subset $\Delta$, we claim that $\mathcal M$ can be made into a model for $\Delta$. Well, $\mathcal M$ certainly satisfies $T\cap\Delta\subseteq\mathrm{Th}_\mathcal L(\mathcal M)$, and then $\Delta\setminus\mathrm{Th}_\mathcal L(\mathcal M)$ is just asserting that $\mathcal M$ has some finite number of distinct elements, which is true
	
	More explicitly, let $\lambda\subseteq\kappa$ be a finite subset such that any $c_\alpha$ appearing in a sentence of $\Delta$ has $\alpha\in\lambda$. Then choose some element $a_0\in\mathcal M$ and then $\left|\lambda\right|$ distinct elements $a_\alpha$ for each $\alpha\in\lambda$. We interpret $c_\alpha$ as $a_\alpha$ for each $\alpha\in\lambda$ and interpret each $c_\beta$ as $a_0$ for each $\beta\notin\lambda$. We can see that this new model $\mathcal M'$ models $\Delta$, so we are safe.

	Anyway, \Cref{thm:compactness} now provides us with a model $\mathcal N'$ of $T$. Notably, $\mathcal N'$ can be restricted to an $\mathcal L$-structure by simply forgetting how to interpret the $\kappa$ new constants, and we see that $\op{Th}_\mathcal L(\mathcal N)\supseteq\op{Th}_\mathcal L(\mathcal M)$, so $\mathcal M\equiv\mathcal N$ follows by \Cref{rem:equiv-by-containment}. However, $\left|\mathcal N\right|\ge\kappa>\left|\mathcal M\right|$ requires that $\mathcal M$ and $\mathcal N$ are not isomorphic.
\end{proof}
Here are some follow-up questions. Fix a language $\mathcal L$.
\begin{enumerate}
	\item If we have $\mathcal M\equiv\mathcal N$ and $\left|\mathcal M\right|=\left|\mathcal N\right|$, can we construct an example with $\mathcal M\not\cong\mathcal N$? This is true for some theories $\mathrm{Th}_\mathcal L(\mathcal M)$ where this is true but not always. For example, for countable models, this is (roughly speaking) the theory of types.
	\item If $\mathcal M\equiv\mathcal N$, can we find a nonempty index set $I$ and an ultrafilter $\mathcal U$ such that $\mathcal M^I/\mathcal U\cong\mathcal N^I/\mathcal U$? The converse is certainly true by \Cref{thm:los}. This forward direction turns out to be yes and is Keisler--Shelah. By the end of the course, we will be able to show this under some assumptions (countable languages, countable structures, and assuming the continuum hypothesis).
\end{enumerate}

\end{document}