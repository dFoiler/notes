% !TEX root = ../notes.tex

\documentclass[../notes.tex]{subfiles}

\begin{document}

\section{September 28}

Let's talk about some game.

\subsection{Ehrenfeucht--Fra\"iss\'e Games}
For today's lecture, let's discuss Ehrenfeucht--Fra\"iss\'e Games. Recall the following definition.
\begin{definition}[unnested]
	An atomic $\mathcal L$-formula $\varphi$ is \textit{unnested} if and only if it takes one of the following forms.
	\begin{itemize}
		\item Equalities: $t_i=t_j$ or $x_i=c$ where the $t_\bullet$ are variables or constants.
		\item Relations: $R(t_1,\ldots,t_n)$ where the $t_\bullet$ are variables or constants.
		\item Functions: $f(t_1,\ldots,t_n)=t_{n+1}$ where the $t_\bullet$ are variables or constants.
	\end{itemize}
\end{definition}
For our discussion today, we let $U_0$ denote the set of finite boolean combinations of unnested atomic formulae, up to provable equivalence (e.g., we don't want to include $\varphi\land\varphi$ from $\varphi$), and we inductively set $U_{n+1}$ to be finite boolean combinations (again, up to provable equivalence) of formulae of the form $\exists x\psi$ where $\psi\in U_n$ and $x$ is a variable.
\begin{proposition} \label{prop:finiteness-of-u-n}
	Fix a finite language $\mathcal L$. Then for each $n$ and $m$, there are only finitely many formulae in $U_n$ with the variables $x_1,\ldots,x_m$ (up to provable equivalence).
\end{proposition}
\begin{proof}
	Fix $m$, and we induct on $n$. We start with $n=0$. For number unnested atomic formulae is finite because the problem is just combinatorics to count sentences of each type. As for the boolean combinations, we note that the boolean algebra generated by a finite set is finite,\todo{} so there are only finitely many classes up to provable equivalence. Then to go up, we place $\exists x_\bullet$ or not in front of each formula, so there continue to be only finitely many formulae, and the boolean algebra generated continues to be finite, so we are okay.
\end{proof}
Our observation, now, is that every $\mathcal L$-formula is equivalent to some formula in one of the $U_n$.
\begin{proposition} %\label{prop:finiteness-of-u-n}
	Fix a language $\mathcal L$. Then any $\mathcal L$-formula $\varphi$ is equivalent to some $\psi\in U_n$ for some $n$.
\end{proposition}
\begin{proof}
	It suffices to check this for atomic formulae; all other formulae follow by adding enough quantifiers and taking boolean combinations. Here are our cases.
	\begin{itemize}
		\item Take sentences of the form $t_1=t_2$. We now have to induct on the complexity of the terms. If we have an equality of variables $x_i=x_j$ or an equality $x_i=c$ for constant $c$, there is nothing to say. If we have $c=x_i$, then this is equivalent to the unnested formula $x_i=c$. Lastly, $c=d$ is equivalent to the sentence $\exists x(x=c\land x=d)$.

		Now if we have something of the type $t_1=f(s_1,\ldots,s_n)$, then by induction, we can achieve any of the formulae $x_{n+1}=t_1$ and $x_i=s_i$ for each $i$ where the $x_\bullet$ are variables. So $t_1=f(s_1,\ldots,s_n)$ is equivalent to
		\[\exists x_1\cdots\exists x_n\left(\bigland_{i=1}^nx_i=s_i\land x_{n+1}=t_1\land x_{n+1}=f(x_1,\ldots,x_n)\right).\]
		This induction completes this case.

		\item For relations, one does essentially the same trick. If we have $R(t_1,\ldots,t_n)$, we simply look at the sentences $x_i=t_i$ combined with $R(x_1,\ldots,x_n)$, reducing to the previous case.
		\qedhere
	\end{itemize}
\end{proof}
Now let's play a game. Fix a language $\mathcal L$ with two $\mathcal L$-structures $\mathcal A$ and $\mathcal B$, and we fix a natural number $n$. The game $EF_n(\mathcal A,\mathcal B)$ of length $n$ is played as follows.
\begin{itemize}
	\item Player I picks $\mathcal A$ or $\mathcal B$ and chooses some $a_1\in A$ or $b_1\in B$. Then Player II chooses an element $b_1\in B$ or $a_1\in A$ from the opposite universe to the one Player I chose.
	\item Then the above move is repeated until we have two $n$-tuples $(a_1,\ldots,a_n)$ or $(b_1,\ldots,b_n)$.
	\item Player II wins if, for any unnested atomic formula $\psi(x_1,\ldots,x_n)$, we have $\mathcal A\models\psi(\overline a)$ is equivalent to $\mathcal B\models\psi(\overline b)$. Otherwise, Player I wins.
\end{itemize}
Roughly speaking, Player I wants to make $\mathcal A$ and $\mathcal B$ look different, and Player II wants them to look similar. We write $\mathcal A\equiv^n\mathcal B$ to mean that Player II can win the $EF_n$ game.
\begin{example}
	Fix the language $\mathcal L=\{<\}$, and take $\mathcal A$ to be $\omega+\omega^*$, where the $\omega^*$ means we have concatenated $\omega$ on top of $\omega^*$ but in reverse (so that $0^*$ is the largest element). We then let $\mathcal B$ be the set $\{0,1,2,\ldots,6\}$ for some natural $m$, and we play the game. Player I can win the game in four moves, but Player II can win in three moves.
\end{example}
Here is our result.
\begin{proposition} \label{prop:winning-ef-game}
	Fix a finite language $\mathcal L$. For each $n$ and structures $\mathcal A$ and $\mathcal B$, Player II has a winning strategy in the $EF_n(\mathcal A,\mathcal B)$ game if and only if $\mathcal A\models\psi$ is equivalent to $\mathcal B\models\psi$ for all sentences $\psi\in U_n$.
\end{proposition}
\begin{proof}
	We prove this by induction on $n$, but the inductive hypothesis will allow $\mathcal A$ and $\mathcal B$ to vary. At $n=0$, we are asking for $\mathcal A\models\varphi$ if and only if $\mathcal B\models\varphi$ where $\varphi$ is an unnested atomic formula, so Player II wins if and only if this is satisfied.

	For our induction, suppose $n$, and we get $n+1$. There are two implications to show.
	\begin{itemize}
		\item In one direction, suppose Player II has a winning strategy. Suppose Player I has picked $a_1\in A$ (without loss of generality). Then Player II responds with some $b_1\in B$ according to the winning strategy. Now, the rest of the game is a length-$n$ game in the language $\mathcal L'$ expanded by a constant symbol $c$ with the structures $\mathcal A'$ and $\mathcal B'$ have $c^{\mathcal A'}=a_1$ and $c^{\mathcal B'}=b_1$. So we are now playing $EF_n(\mathcal A',\mathcal B')$. So Player II has a winning strategy in $EF_{n+1}(\mathcal A,\mathcal B)$ if and only if, for all $a\in A_1$, there exists $b_1\in B$ such that Player II has a winning strategy in $EF_{n}(\mathcal A',\mathcal B')$. Anyway, by the induction, we get $\mathcal A'\equiv^n\mathcal B'$ in $\mathcal L'$.

		We now show that $\mathcal A\equiv^{n+1}\mathcal B$. Thus far we are given that $\mathcal A'\models\psi$ if and only if $\mathcal B'\models\psi$ for any $\mathcal L'$-sentence $\psi\in U_n$. We now do our check. Fix a sentence $\theta\in U_{n+1}$ of the form $\exists x_1\varphi$ where $\varphi\in U_n$. Then $\mathcal A\models\theta$ is equivalent to having some $a_1\in A$ such that $\mathcal A\models\varphi(a_1)$. Let $b_1$ be the resulting choice of Player II. But now using our hypothesis at the beginning of the paragraph, we achieve $\mathcal A'\models\varphi(c)$, so $\mathcal B'\models\varphi(c)$, so $\mathcal B\models\varphi(b_1)$. The reverse implication is similar.

		\item Conversely, suppose that $\mathcal A\models\psi$ is equivalent to $\mathcal B\models\psi$ for all sentences $\psi\in U_n$. We give a winning strategy for Player II. Let's say $a_1\in A$ is chosen by Player I. Let $\Psi$ be the set of formulae $\psi(x_1)\in U_n$ with at most $(n+1)$ variables such that $\mathcal A\models\psi(a_1)$, which is a finite set up to provable equivalence by \Cref{prop:finiteness-of-u-n}. It is important that $\Psi$ is finite because now
		\[\mathcal A\models\exists x_1\bigland_{\psi\in\Psi}\psi(x_1).\]
		This formula lives in $U_{n+1}$, so by hypothesis, we get
		\[\mathcal B\models\exists x_1\bigland_{\psi\in\Psi}\psi(x_1),\]
		so we get $b_1\in B$ satisfying all $\mathcal B\models\psi(b_1)$ for $\psi\in\Psi$.
	
		Now build $\mathcal L'$ and structures $\mathcal A'$ and $\mathcal B'$ as before. We claim that $\mathcal A'\models\varphi$ if and only if $\mathcal B'\models\varphi$ for all $\mathcal L'$-sentences $\varphi\in U_n$. Indeed, simply view $\varphi$ as an $\mathcal L$-formula $\widetilde\varphi(x)$ by extracting out the constant $c$ and replacing it with $c$, and we see $\mathcal A'\models\varphi$ is equivalent to $\mathcal A\models\widetilde\varphi(a_1)$, which is indeed equivalent to $\mathcal B\models\widetilde\varphi(b_1)$.

		Now by induction, Player II has a winning strategy in the game $EF_n(\mathcal A',\mathcal B')$, which is equivalent to winning the original game, as discussed in the previous implication.
		\qedhere
	\end{itemize}
\end{proof}
\begin{corollary}
	Fix a language $\mathcal L$. Then $\mathcal A\equiv\mathcal B$ if and only if, for all finite language $\mathcal L'\subseteq\mathcal L$, we have $\mathcal A|_{\mathcal L'}\equiv\mathcal B|_{\mathcal L'}$.
\end{corollary}
\begin{proof}
	Play the above game. Note $\mathcal A\equiv\mathcal B$ if and only if they satisfy the same formulae, which is equivalent to having $\mathcal A|_{\mathcal L'}\equiv\mathcal B|_{\mathcal L'}$ for all finite $\mathcal L'\subseteq\mathcal L$ because any formula will only contain finitely many symbols. Then this is in fact equivalent to satisfying the same $\mathcal L'$-sentences in $U_n$ for all $n$, which finishes by \Cref{prop:winning-ef-game}.
\end{proof}
\begin{remark}
	Here is a challenge problem: for which $m$ and $n$ does Player II win the game of length $n$ between the groups $\ZZ$ and $\ZZ/m\ZZ$? There does exist some $n$ such that Player I will always win this game. Approximately speaking, one needs a sentence true in $\ZZ$ which is false in the $\ZZ/m\ZZ$s.
\end{remark}

\end{document}