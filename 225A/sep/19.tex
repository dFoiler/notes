% !TEX root = ../notes.tex

\documentclass[../notes.tex]{subfiles}

\begin{document}

\section{September 19}

Here we go.

\subsection{An Example of the Back-and-Forth Method}
For our example, let $\mathcal L$ be a language with one binary relation $E$, which will be considered to be an equivalence relation. Consider the structure $\mathcal M_0$ with universe $(x,y)\in\NN^2$ where $x<y$, where $(x,y)E(x',y')$ if and only if $y=y'$.

We claim that there is another countable model with the same theory. For example, we consider $\mathcal M_\omega$ which is $\mathcal M_0$ with a disjoint copy of $\NN^2\times\{0\}$ where $(x,y,0)E(x',y',0)$ if and only if $y=y'$. Let's check that the theory of $\mathcal M_0$ has the same theory of $\mathcal M_\omega$. This essentially follows from compactness (\Cref{thm:compactness}) and \Cref{thm:down-skolem} to the theory $T$ consisting of the elementary diagram of $\mathcal M_0$ plus the sentences
\[\{c_{xy}\ne c_{x'y'}:\text{ for }(x,y)\ne(x',y')\}\cup\{c_{xy}Ec_{x'y}:x,x',y\in\NN\}\cup\{c_{xy}Ec_{x'y'}:x,x',y,y'\in\NN\text{ where }y\ne y'\},\]
where we have introduced these new constants $c_{xy}$ to an extended language $\mathcal L'$. Namely, \Cref{thm:compactness} permits us to find a countable model of this above theory: to see that the above set of sentences is satisfiable, we note that $\mathcal M_0$ is able to model any finite subset of the above theory is only asking for arbitrary many arbitrarily large equivalence classes, which $\mathcal M_0$ provides.

So we produce a countable model $\mathcal M'$ of $T$. We claim that $\mathcal M'\cong\mathcal M_\omega$ in the language $\mathcal L$. This will use the back-and-forth method.
\begin{lemma}
	Fix everything as above. Then $\mathcal M'\cong\mathcal M_\omega$, where $\mathcal M'$ is considered as an $\mathcal L$-structure.
\end{lemma}
\begin{proof}
	We build our isomorphism via approximations $f_i\colon X_i\to Y_i$ for $i\in\NN$, where $X_i\subseteq\mathcal M'$ and $Y_i\subseteq\mathcal M_\omega$. We require that $i\le j$ means $X_i\subseteq X_j$ and $Y_i\subseteq Y_j$ and then $f_j|_{X_i}=f_i$, and we also want $f_i$ to be an isomorphism of $\mathcal L$-strucutres for $i>0$. By the end of this process, we will want $\bigcup_{i\in\NN}X_i=\mathcal M'$ and $\bigcup_{i\in\NN}Y_i=\mathcal M_\omega$ so that we have a well-defined isomorphism $f\colon\mathcal M'\to\mathcal M_\omega$ at the end. This last bit is going to be a little tricky. For this, we enumerate $M'=\{a_i\}_{i=0}^\infty$ and $M_\omega=\{b_i\}_{i=0}^\infty$, and we will ask that each $n$ have $\{a_j:j<n\}\subseteq X_{2n}$ and $\{b_j:j<n\}\subseteq\op{im}f_{2n+1}$.

	Alright, let's get started. Take $f_0$ to be the unique function $X_0\to Y_0$ where $X_0=Y_0=\emp$. One can check that this trivially works for all of our hypotheses. We now induct in two cases.
	\begin{itemize}
		\item Suppose we have $f_{2n}\colon X_{2n}\to Y_{2n}$, and we want to produce $f_{2n+1}\colon X_{2n+1}\to Y_{2n+1}$. The point is that $b_n$ now needs to appear in the range of $f_{2n+1}$. We have the following cases.
		\begin{itemize}
			\item If $b_n$ is already in the range, do nothing. In the following cases, we suppose that $b_n$ is not in the range of $f_{2n}$ already.
			\item Suppose that $b_n$ is not equivalent to some element of $\im f_{2n}$. If $b_n$ is in a finite equivalence class, map it to the corresponding unique equivalence class in $\mathcal M'$, which cannot have been chosen so far because $f_{2n}$ is an isomorphism. If $b_n$ lives in an infinite equivalence class, then go find an unused infinite equivalence class in $\mathcal M'$, which is possible because $f_{2n}$ has finite domain currently.
			\item Suppose that $b_n$ is equivalent to some element $b'\in\im f_{2n}$. By the nature of $f_\bullet$ being an isomorphism, we are arranging so that the size of the equivalence class of $a$ and $f_\bullet(a)$ are always the same. So the size of the equivalence class of $f^{-1}_{2n}(b')$ must have space (even if finite!) because the element of $b_n$ not being hit so far requires us to have space in the equivalence class of $f^{-1}_{2n}(b')$.
		\end{itemize}
		% if it is already in there, we do nothing. Well, if $b_n$ belongs to a finite equivalence class of size $N$, then we know because $\mathcal M'$ models $T$, there is exactly one equivalence class of size $N$, so send $b_n$ to one of those elements that we have not chosen yet. Notably, $\mathcal M_\omega$ also has exactly one equivalence class of size $N$, so as long as we keep track of the correct number of elements in this class we have chosen, we will be okay.

		% Otherwise, $b_n$ belongs to an infinite equivalence class. If $b_n$ is equivalent to some already chosen $b_\bullet$, then select some corresponding $a_\bullet$ in the correct equivalence class back in $\mathcal M_\omega$.

		\item Going forward the argument is essentially the same just talking in reverse.
	\end{itemize}
	Assembling the $f_\bullet$ together produces the desired result.
\end{proof}
We now conclude by remarking that $\op{Th}_{\mathcal L_{\mathcal M_0}}(\mathcal M_0)=\op{Th}_{\mathcal L_{\mathcal M_0}}(\mathcal M_\omega)$, so $\mathcal M_0\le\mathcal M_\omega$.
\begin{remark}
	We can now define $\mathcal M_n\coloneqq\mathcal M_0\sqcup\NN\times\{0,1,\ldots,n-1\}\times\{0\}$ as a substructure of $\mathcal M_\omega$. One can repeat the above argument with $\mathcal M_0$ replaced by $\mathcal M_n$ to conclude that $\mathcal M_n\le\mathcal M_\omega$ again. We conclude that $\mathcal M_0\equiv\mathcal M_n$ for each $n$. In total, we have produced countably many non-isomorphic models. It turns out that these are all the countable ones.
\end{remark}
One might now go back and ask for the number of models of $\mathrm{Th}_{\mathcal L_{\mathcal M_0}}(\mathcal M_0)$ of cardinality $\aleph_1$. It turns out that there are again countably many. The point is that a model $\mathcal M$ of cardinality $\aleph_1$ can be attached the two invariants
\begin{align*}
	\kappa_0(\mathcal M) &\coloneqq \#\{[x]\in M/E:[x]_E\text{ has size }\aleph_0\}, \\
	\kappa_1(\mathcal M) &\coloneqq \#\{[x]\in M/E:[x]_E\text{ has size }\aleph_1\}.
\end{align*}
One can show that $\mathcal M_1\cong\mathcal M_2$ if and only if $\kappa_0(\mathcal M_1)=\kappa_0(\mathcal M_2)$ and $\kappa_1(\mathcal M_1)\cong\kappa_1(\mathcal M_2)$ by using some set theory, and then one can produce a model with given invariants $\kappa_0$ and $\kappa_1$ arbitrarily provided that $\aleph_0\kappa_0+\aleph_1\kappa_1=\aleph_1$.

\subsection{Dense Linear Orders Without Endpoints}
Let's see another example.
\begin{proposition} \label{prop:q-is-aleph-0-categorical}
	Fix a language $\mathcal L$ with a single binary relation $<$. Then $\op{Th}_{\mathcal L}(\QQ,<)$ is $\aleph_0$-categorical.
\end{proposition}
We should perhaps define $\aleph_0$-categorical.
\begin{defihelper}[$\kappa$-categorical] \nirindex{kappa-categorical@$\kappa$-categorical}
	A theory $T$ of a language $\mathcal L$ is \textit{$\kappa$-categorical} if and only if $T$ has exactly one isomorphism class of models of cardinality $\kappa$.
\end{defihelper}
In fact, we will show the following.
\begin{proposition} \label{prop:countable-dlo-w-e}
	Fix a language $\mathcal L$ with a single binary relation $<$, and let $\mathrm{DLO}$ be the following theory, of dense linear orders without endpoints.
	\begin{itemize}
		\item $<$ is a total ordering.
		\item Dense: $\forall x\forall y(x<y\to\exists z(x<z\land z<y))$.
		\item Without endpoints: $\forall x\exists y(y<x)$ and $\forall x\exists y(x<y)$.
	\end{itemize}
	Then $\mathrm{DLO}$ is $\aleph_0$-categorical.
\end{proposition}
Note that $\QQ$ models $\mathrm{DLO}$, so \Cref{prop:q-is-aleph-0-categorical} will follow. Anyway, let's show \Cref{prop:countable-dlo-w-e}.
\begin{proof}[Proof of \Cref{prop:countable-dlo-w-e}]
	Let $\mathcal A$ and $\mathcal B$ be models of $\mathrm{DLO}$. Enumerate $\mathcal A=\{a_i\}_{i=0}^\infty$ and $\mathcal B=\{b_i\}_{i=0}^\infty$, and we will work in the same set-up as the back-and-forth argument previously described. Namely, we describe a sequence of compatible isomorphisms $f_i\colon X_i\to Y_i$ where $X_{2n}$ contains $\{a_1,\ldots,a_{n-1}\}$ and $Y_{2n+1}$ contains $\{b_1,\ldots,b_{n-1}\}$. Take $f_0$ to be the unique function $\emp\to\emp$.
	\begin{itemize}
		\item Suppose we have $f_{2n-1}$, and we want to build $f_{2n}$. If $a_n$ is already in the domain of $f_{2n-1}$, do nothing. We have three cases.
		\begin{itemize}
			\item If $a_n<x$ for all $x\in X_{2n-1}$, use that $\mathcal B$ has no endpoints to find $f(a_n)$ less than everyone in $Y_{2n-1}$.
			\item If $a_n>x$ for all $x\in X_{2n-1}$, make a similar argument as the previous case.
			\item Otherwise, find $x,y\in X_{2n-1}$ so that $x<a_n<y$, and nothing in $X_{2n-1}$ lives between $x$ and $y$; this is possible because $<$ is a total ordering. Then use the density of $\mathcal B$ to find some $f(a_n)$ strictly between $x$ and $y$ to complete.
		\end{itemize}
		\item To extend $f_{2n}$ to $f_{2n+1}$, repeat the above argument in reverse.
	\end{itemize}
	Now, assembling our $f_\bullet$ produces our isomorphism.
\end{proof}
\begin{remark}
	We now might ask how many models $\mathrm{DLO}$ has of cardinality $\aleph_1$. There are apparently $2^{\aleph_1}$ many up to isomorphism. Of course, this is an upper bound on the number of models because an ordering is asking for a subset of $\aleph_1\times\aleph_1$. So the name of the game now is to produce enough models; one cannot really hope to precisely describe all the models.
\end{remark}
To wrap us up, let's pick up the following result.
\begin{proposition} \label{prop:kappa-categorical-is-complete}
	Fix an $\mathcal L$-theory $T$ which is $\kappa$-categorical for cardinality $\kappa$. If $T$ has only infinite models, then $T$ is complete; i.e., any $\mathcal L$-sentence $\varphi$ has either $T\models\varphi$ or $T\models\lnot\varphi$.
\end{proposition}
\begin{proof}
	Let $\mathcal M$ be a model of $T$ of cardinality $\kappa$. Now, for any sentence $\varphi$, if $T\models\varphi$ and $T\models\lnot\varphi$, then there is a model $\mathcal M_+$ and $\mathcal M_-$ satisfying $T\cup\{\varphi\}$ and $T\cup\{\lnot\varphi\}$, respectively. By \Cref{thm:down-skolem}, we may bring $\mathcal M_+$ and $\mathcal M_-$ to have cardinality $\kappa$, so being $\kappa$-categorical requires $\mathcal M_+\cong\mathcal M_-$, which is a contradiction because then $\mathcal M_+\equiv\mathcal M_-$.
\end{proof}
\begin{example}
	Thus, \Cref{prop:countable-dlo-w-e} requires that $\mathrm{DLO}$ is complete. As such, the theory $\mathrm{DLO}$ must complete to exactly $\mathrm{Th}(\QQ,<)$.
\end{example}

\end{document}