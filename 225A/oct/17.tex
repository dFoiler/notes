% !TEX root = ../notes.tex

\documentclass[../notes.tex]{subfiles}

\begin{document}

\section{October 17}

There is an exam on Thursday. It will be about four or five questions similar to ones on the homework but hopefully of the more reasonable kind (namely, solvable in something like 20 minutes).

\subsection{The Cell Decomposition Theorem}
Let's continue the proof from last class. We continue with our definable function $f$ which we are trying to show is piecewise constant or strictly monotone.
\begin{enumerate}
	\setcounter{enumi}{1}
	\item For notation, let $\lambda(x)$ be a function outputting $+$ if $f$ is locally increasing to the left of $x$, $-$ if  $f$ is locally decreasing to the left of $x$, $0$ if $f$ is locally constant to the left of $x$, and $*$ otherwise. We define $\mu(x)$ to the needed right versions of these properties. This produces $16$ cases for the pair $(\lambda(x),\mu(x))$.

	Quickly, we argue that $*$ is in fact never outputted; this follows from $o$-minimality. By symmetry, we might as well argue this for $\lambda$. Define $\rho(y,z)$ to be $<$ or $>$ or $=$ depending on how $y$ and $z$ relate. Now, if $\lambda(x)=*$, then any $\delta>0$ produces $y$ and $z$ between $x-\delta$ and $x$ such that $\rho(f(y),f(x))\ne\rho(f(z),f(x))$. This allows us to build ascending sequences $\{y_i\}_{i=0}^\infty$ and $\{z_i\}_{i=0}^\infty$ (always less than $x$) such that $\rho(f(y_i),f(x))\ne\rho(f(z_i),f(x))$ always. By the pigeonhole principle, we may reduce to a subsequence so that $\rho(f(y_i),f(x))$ and $\rho(f(z_i),f(x))$ are each constant and not equal. However, the definable set
	\[\{y<x:\rho(f(y),f(x))=\rho(f(y_i),f(x))\text{ for each }i\}\]
	cannot be the union of finitely many intervals because the sequence $\{z_i\}$ puts infinitely many holes in it around $x$. Explicitly, any interval containing infinitely many of the $y_\bullet$ (which must be possible by the pigeonhole principle) will also contain infinitely many of the $z_\bullet$.

	So we have left to deal with the $9$ cases for $(\lambda(x),\mu(x))$.

	\item Continuing, by $o$-minimality, we may decompose $\mathcal R$ into intervals and points so that $\lambda$ and $\mu$ are both constant on these intervals, essentially using all $9$ cases. Let $I$ be such an interval upon which $\lambda$ and $\mu$ are constant. We would like to show that $\lambda$ and $\mu$ are the same on $I$. Looking locally, we may as well assume that $I$ is a bounded interval.
	
	For example, take $\lambda(x)=0$ on $I$; a similar argument works if $\mu(x)=0$. We claim that $\mu(x)=0$ for each $x\in I$. Then we get $\delta<x$ to check constant to the left of $x$, we may as well assume that $\delta\in I$, and we are promised that $f(y)=f(x)$ for all $y$ between $\delta$ and $x$. Now, choosing any $x>\varepsilon>y$ such that $\rho(f(z),f(y))$ corresponds to $\mu(y)$ for any $y<z$, which is doable because $\mu$ is constant in this region. But then $f(z)=f(x)$ is forced because $\delta<z<x$, and $f(x)=f(y)$, so $f(z)=f(y)$, so $\mu(y)=0$ follows. But $\mu$ is constant on $I$, so $\mu$ vanishes everywhere on $I$.

	Next up, suppose $\lambda$ is $+$ on $I$ but $\mu$ is $-$ on $I$; in other words, every point is a local maximum!\footnote{This is weird but not immediately a contradiction: the function $\QQ\to\QQ$ defined by $p/q\mapsto1/q$ (where $\gcd(p,q)=1$ and $q>0$) has every point as a local maximum. We will have to use $o$-minimality.} The other cases will be analogous. We claim that for all sufficiently large $x$, there is some $y>x$ such that $f(y)\ge f(x)$. Well, let $B$ be the set of all $x\in I$ such that all $y>x$ have $f(y)<f(x)$. Now, if our claim were false, then $B$ would have infinitely many elements and hence contain an interval. But then looking locally at some point in the interval would require that $\lambda(y)=-$, which is a contradiction.

	So because we worked with $x$ large enough, we might as well replace $I$ with an interval upon which the claim was true. We now claim that each $x\in X$ and $y$ sufficiently larger than $x$ will have $f(y)>f(x)$. Well, consider the set
	\[B\coloneqq\{y>x:f(y)<f(x)\}.\]
	Because $\mu$ is constantly $-$, this set is nonempty; similarly, the complement is nonempty. Further, the set is certainly definable, so we may let $z$ denote the largest boundary point. If $f(x)>f(z)$, then one finds that $f(x)>f(u)$ for some $u$ close but below $x$, say above $\delta<x$. But then an interval consisting of elements from $\delta$ to a little above $z$ does not live in the complement of $B$, so $z$ is in fact not a boundary point. (Namely, everything after $z$ needs to be less than $f(x)$.)\todo{What?}

	As such, $f(x)\le f(z)$. By the previous claim, we produce $w\ge z$ such that $f(z)\le f(w)$. Further, if there is $\delta<z$ such that each $u$ between $\delta$ and $z$ such that $f(u)<f(x)$. But then any $v$ between $z$ and $u$ has $f(v)<f(x)$, so again $B$ contains points beyond $z$, which is a contradiction.

	Similarly, if one has $f(v)\ge f(x)$ for all $v>z$, then because $\mu$ is $+$ on $I$, we see that the set of all $v$ such that $f(v)=f(x)$ must be finite, so there is $t\ge z$ such that $u>t$ implies $f(u)\le g(x)$.\todo{What?}

	We are now ready to define a function $\beta\colon I\to I$ sending $x\in I$ to the least element of the set $B_x$ consisting of $y>x$ such that all $z>y$ has $f(z)>f(x)$, which exists by what we've just shown. Quickly, note that there is $\delta<\beta(x)$ such that any $w$ between $\delta$ and $\beta(x)$ has $f(w)>f(x)$. Indeed, if no such thing exists, then instead there is some $\delta<\beta(x)$ such that any $w$ between $\delta$ and $\beta(x)$ has $f(w)<f(x)$. Choosing any such $w$ will violate the fact that $\beta(x)$ is supposed to be the infimum of $B_x$.

	So we have a property $\theta_{-,+}(v)$ such that we have $\delta_1<x<\delta_2$ with $\delta_1<u<v<w<\delta_2$ has $f(u)<f(w)$. We have checked that $\theta_{-,+}(\beta(x))$ by the above argument.

	Now, $\beta$ is definable, so $\beta(I)$ is definable, satisfying $\beta(x)>x$ (and hence infinite), so we can find an interval $J\subseteq\beta(I)$ which is a ``cofinal'' interval, meaning that any point in $I$ has a larger point living in $J$. Because $J$ lives in the image of $\beta$, we see that, in $J$, having $\mu(v)=-$ and $\lambda(v)=+$ implies $\theta_{-,+}(v)$. Now, we go ahead and replace $J$ with $I$ because we can.

	As a weird trick, we now reverse the ordering and rerun all our arguments. For example, any sufficiently small $x$ has some $y<x$ such that $f(y)\ge f(x)$, and we are able to restrict $J$ to a ``coinitial'' interval upon which the above statement is true. Continuing, we can show as before that any $x\in J$ and $y$ sufficiently smaller than $x$ has $f(y)>f(x)$, so we are able to define a function $\alpha$ equal to the supremum of all $y$ such that any $z<y$ has $f(z)>f(x)$. As before, we are able to find an interval $K\subseteq\alpha(J)$, and we again get the analogous property $\theta_{+,-}$ everywhere on $K$. But this is a contradiction because we already have $\theta_{-,+}$.

	\item Thus, we have shown that any interval $I$ as defined at the top of the previous step has $\mu=\lambda$ if $\mu$ and $\lambda$ are constant. It remains to show that $f$ is strictly increasing or strictly decreasing or constant on such an interval. The constant case is relatively easy, so without loss of generality, we take $\lambda=\mu=+$. Well, select $x\in I$, and define
	\[B_x\coloneqq\{y>x:f(y)>f(x)\}.\]
	Certainly $B_x$ is nonempty because $\mu=+$. We would like to show that $B_x$ contains everything above $x$. If there is an element of $I$ bigger than $x$ but not in $B_x$, we may as well as choose some $z$ the minimum of the boundary of $B_x$. If $f(z)\le f(x)$, then everything between $z$ and $x$ must have the same value, but this is not okay because we are locally increasing at $z$. Similarly, if $f(z)>f(x)$, we note that locally increasing at $z$ causes similar problems.
\end{enumerate}
We can now prove (a) of \Cref{thm:cell-decomp}, assuming (b). Namely, suppose that definable functions $\mathcal R^n\to\mathcal R$ are piecewise continuous, and we prove the cell decomposition theorem in $\mathcal R^{n+1}$. Well, suppose $X\subseteq\mathcal R^{n+1}=\mathcal R^n\times\mathcal R$ is definable. Then for $b\in\mathcal R^n$, we define $X_b$ to be the set of $a\in\mathcal R$ such that $(b,a)\in X$; note that $X_b$ is a definable subset of $\mathcal R^n$.

Now, we note that there is an upper bound $N$ (only depending on $X$) such that each $b\in\mathcal R^n$ with $\#\del X_b\le N$; this is by some compactness argument. Then we can choose $B_0,\ldots,B_N\subseteq\mathcal R^n$ such that $B_i$ is the set of $b$ with $\del X_b$ having $i$ elements. Now, for any $x\in\mathcal R$, define a function $g_i\colon\mathcal R^n\to\mathcal R$ as sending $b$ to the $i$th element of $\del X_b$, which is definable and hence piecewise continuous. Namely, one has a cell decomposition $\mathcal C'$ of $\mathcal R^n$ such that $g_i|_{\mathcal C'}$ is continuous for each $i$, and we may as well assume that the $\mathcal C'$ decomposes the $B_\bullet$. One can now decompose $X$ using $\mathcal C'$. Explicitly, take $\widehat{\mathcal C}$ to be the graphs of the $g_i$ on $C$ for each $C\in\mathcal C'$ and also the cells between the $g_\bullet$s (and also the cells below $g_1$ and the cell above $g_N$).

\end{document}