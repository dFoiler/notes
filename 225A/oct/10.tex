% !TEX root = ../notes.tex

\documentclass[../notes.tex]{subfiles}

\begin{document}

\section{October 10}

The exam is in a little over a week. Exercises will be focused on content covered in class (and harder exercises will be chosen from there), but it is possible to be asked about other topics in Marker. Some exercises on the midterm will be taken from exercises assigned to us.

The class began by showing that 

\subsection{Back to Back to Real Closed Fields}
For the time being, take $a$ to be algebraic. We claim that there is $\alpha\in\mathrm{Cut}^-(a/A_1)$ and $\beta\in\op{Cut}^+(a/A_1)$ with $\op{sgn}(P(\alpha))\ne\op{sgn}(P(\beta))$, which is a sign change that we will be able to push over to $\mathcal R_2$ in order to produce a root over there. Well, note $P'(a)\ne0$ because we are in characteristic $0$, so everything is separable. We take $P'(a)>0$; otherwise simply reverse all signs. Then \Cref{prop:poly-increases-locally} grants $b<a<c$ with $P(x)<0<P(y)$ whenever $b<x<a<c<y$, but technically the argument only gives $b,c\in R_1$, and the same holds for everything between.

It remains to bring these down to $A_1$. For this, we use the following lemma.
\begin{lemma}
	Fix a real closed field $\mathcal R$. For $Q[x]\in R[x]$ and $\alpha<\beta$ with $Q(\alpha)=Q(\beta)=0$, there is $\gamma\in[\alpha,\beta]$ such that $Q'(\gamma)=0$.
\end{lemma}
\begin{proof}
	If $Q'(\alpha)=0$ or $Q'(\beta)=0$, there is nothing to do. Now, if $Q'(\alpha)$ and $Q'(\beta)$ have different signs, $\mathcal R$ being a real closed field grants us our $\gamma$.

	Lastly, suppose $Q'(\alpha)$ and $Q'(\beta)$ have the same sign. Without loss of generality, make both of them positive. Then there is $\varepsilon>0$ such that $\alpha<\alpha+\varepsilon<\beta-\varepsilon<\beta$ such that having $\alpha<\gamma<\alpha+\varepsilon$ implies $Q(\gamma)>0$ and having $\beta-\varepsilon<\gamma<\beta$ implies $Q(\gamma)<0$. So $Q$ has another root strictly between $\alpha$ and $\beta$, so we replace $\beta$ with this root $\beta'$.
	
	Namely, check if $Q'(\beta')\le0$, we get our root of $Q'$; otherwise, we repeat the process for $[\alpha,\beta']$ to get yet another root $\beta''$. This process must eventually terminate because $Q$ can only have finitely many roots, so we get our needed root of $Q$.
\end{proof}
Now, choose $\alpha\in\op{Cut}^-(a/A_1)$ and $\beta\in\op{Cut}^+(a/A_1)$. For concreteness, list the roots of $P(x)$ as $a_1<a_2<\cdots<a_n$, and suppose $a=a_i$ for some $i$. Now, set $\lambda\coloneqq\alpha$ if $a=a_1$ or instead a root of $P'(x)$ between $a_{i-1}$ and $a_i=a$ of $i>1$. Similarly, set $\mu\coloneqq\beta$ if $a=a_n$ or instead a root of $P'(x)$ between $a_i$ and $a_{i+1}$ if $i<n$. (These exist by the above lemma.) Notably, $\lambda$ and $\mu$ are at worst roots of polynomials of $P'$, which has degree less than $a$, so $\lambda,\mu\in A_1$!

As such, we have $\lambda<a<\mu$ with $\lambda,\mu\in A_1$. Note that $P(\lambda)$ and $P(\mu)$ have different sign: certainly these are not roots, so we have sign in $\{\pm1\}$, and if they had the same sign, say they are both of sign $P'(a)$, then $P$ being locally strictly monotone at $a$ will produce a root either between $\lambda$ and $a$ or between $a$ and $\mu$, which contradicts the construction of $\lambda$ and $\mu$.

The point is that the data $(P,\lambda,\mu)$ uniquely determine $a$, and these are data we can push through the isomorphism $f\colon A_1\to A_2$. Namely, the sign of $P(\lambda)$ and $P(\mu)$ continue to be different after passing through our isomorphism, so the intermediate value property in $\mathcal R_2$ grants us some $b\in\mathcal R_2$ between $f(\lambda)$ and $f(\mu)$. So we get an isomorphism of fields
\[A_1[a]\cong\frac{A_1[x]}{(P(x))}\cong\frac{A_2[x]}{(P(x))}\cong A_2[b].\]
We will later upgrade this to an isomorphism of ordered fields, which will complete the argument in this case.

Before running this check, though, let's take care of the transcendental case. Add a new constant symbol $b$ to our language. We claim that
\begin{equation}
	\op{elDiag}(\mathcal R_2)\cup\{f(\alpha)<b:\alpha\in\op{Cut}^-(a/A_1)\}\cup\{b<f(\beta):\beta\in\op{Cut}^+(a/A_1)\} \label{eq:get-elem-ext-r2}
\end{equation}
is satisfiable. It's enough to check that this is finitely satisfiable. Upon using the linear order in $A_1$, it is enough to check that there is $b^{\mathcal R_2}\in\mathcal R_2$ with $f(\alpha)<b^{\mathcal R_2}<f(\beta)$ for some $\alpha\in\op{Cut}^-(a/A_1)$ and $\beta\in\op{Cut}^+(a/A_1)$, for which $\frac12(\alpha+\beta)$ will do. Now, let $\mathcal R_2^*$ model \eqref{eq:get-elem-ext-r2}; by construction, $\mathcal R_2$, and we let $b$ denote the interpretation of the corresponding constant, and we get an isomorphism $A_1[a]\cong A_2[b]$. (Note that we can promise $b$ is also transcendental because of yet another compactness argument avoiding the root of any polynomial.) Now choose some $\lambda,\mu\in A_1$ so that $\lambda<a<\mu$, provided they exist.

We now check that our field isomorphism $A_1[a]\cong A_2[b]$ extends to an isomorphism of ordered fields. Well, for any $Q\in A_1[x]$ such that $\deg Q<\deg P$ (take $\deg P=+\infty$ in the transcendental case), we need to check that $\op{sgn}Q(a)=\op{sgn}Q(b)$. Quickly, if $a>A_1$ always, then the sign of $Q(a)$ is the sign of the leading coefficient (we have gone off to infinity), and $f(a)>A_2$ also, so the sign of $Q(b)$ is also the sign of the leading coefficient. The case of $a<A_2$ is similar.

Now, we may recall that we have some extra information $\lambda$ and $\mu$. Certainly $Q(a)\ne0$ because $\deg Q<\deg P$. Without loss of generality, we take $Q(a)>0$. Now, all roots of $Q$ will live in $A_1$ by our induction, so we let $\lambda_Q$ denote the maximum of $\lambda$ and also all the roots $y$ of $Q$ with $y<a$, and we construct $\mu_Q$ dually as a minimum greater than $a$. Now, $Q$ has no roots between $\lambda_Q$ and $\mu_Q$ by construction, so the intermediate value property promises that $Q$ maintains sign over this entire interval, and this sign is the sign of $Q((\lambda_Q+\mu_Q)/2)\in A_1$. The same holds over in $A_2$, and we note that this sign will agree with $Q((f(\lambda_Q)+f(\mu_Q))/2)\in A_2$. So the sign of $Q(a)$ is the same as the sign of $Q(f(a))$.

\subsection{Cell Decomposition}
In life one might want explicitly eliminate quantifiers, perhaps with few quantifiers and modest complexity. For this, one can use cell decomposition.
\begin{defihelper}[$o$-minimal] \nirindex{o-minimal@$o$-minimal}
	A theory $T$ in a language $\mathcal L$ extending the language of ordered sets is \textit{$o$-minimal} if and only if the following conditions are satisfied.
	\begin{enumerate}
		\item $T$ restricted to the language of ordered sets is equivalent to $\op{DLO}$
		\item Any model $\mathcal R\models T$ with an $\mathcal L_R$-formula $\varphi(x)$ has some partition $-\infty=a_0<a_1<\cdots<a_n=+\infty$ and subsets $I\subseteq\{1,2,\ldots,n-1\}$ and $J\subseteq\{0,\ldots,n\}$ such that
		\[R\models\varphi\leftrightarrow\left(\biglor{i\in I}x=a_i\lor\biglor_{j\in J}(a_i<x<a_{i+1})\right).\]
	\end{enumerate}
\end{defihelper}

\end{document}