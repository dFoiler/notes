% !TEX root = ../notes.tex

\documentclass[../notes.tex]{subfiles}

\begin{document}

\section{October 3}

Let's play the game to start off the class.
\begin{example}
	We work with ordinals in the language $\mathcal L=\{<\}$.
	\begin{itemize}
		\item We play with $\varepsilon_0=\sup\left\{\omega,\omega^\omega,\ldots\right\}$ and $2$. Then Player II loses after, say, 2 moves: Player I selects anything, Player II selects (say) $0$, and then Player I chooses something smaller than what they chose in $\varepsilon_0$.
		\item We play with $\varepsilon_0$ and $\omega_1$. Then Player II can always win. The point is that there is some kind of finite-length back-and-forth argument
	\end{itemize}
\end{example}

\subsection{Real Closed Fields}
Let's discuss real closed fields because they, in some sense, will tell us that Euclidean geometry is decidable (approximately speaking). Our language will be the language $\mathcal L=\{+,-,\cdot,<,0,1\}$ of ordered rings. The theory of ordered fields $\mathrm{OrdFld}$ is axiomatized by writing the axioms for fields, for a total order, and requiring that addition and multiplication respect this ordering. We won't bother writing down the first two lists of axioms, but the third list is given as follows.
\begin{itemize}
	\item $\forall a\forall b\forall c((a<b)\to (a+c<b+c))$.
	\item $\forall a\forall b\forall c(((a<b)\land(c>0))\to (a\cdot c<b\cdot c))$.
\end{itemize}
So we have a finitely axiomatized our theory $\mathrm{OrdFld}$.
\begin{example}
	Any subfield of $\RR$ will do is a model.
\end{example}
\begin{example}
	We can use compactness to provide a model of $\RR$ with an element larger than any other element but the same cardinality.
\end{example}
We will be actually be interested in the theory $\mathrm{RCF}$ of real closed fields, which is the theory $\mathrm{OrdFld}$ plus the intermediate value theorem for polynomials. This is an infinite list of axioms, approximately saying that, for any model $\mathcal R$ with universe $R$, and polynomial $f\in R[x]$ with inputs $a,b\in R$ such that $f(a)<0<f(b)$ has some $c\in R$ such that $f(c)=0$.

To write this out, we choose a degree of $n$ and write down the sentence
\begin{align*}
	\forall a_0\cdots\forall a_n\forall a\forall b&\big(\left((a<b)\land \left(a_0a^0+\cdots+a_na^n<0<a_0b^0+\cdots+a_nb^n\right)\right) \\
	&\to\exists c\left(a<c<b\land a_0c^0+\cdots+a_nc^n=0\right)\big).
\end{align*}
We cannot finitely axiomatize these sentences using an argument like \Cref{lem:try-to-finitely-axiom}.
\begin{remark}
	Any ordered field $(\mathcal R,+,-,0,1,<)$ has $(\mathcal R,<)$ satisfying $\mathrm{DLO}$. We know that we are a linear order, we have no endpoints because $x+-1<x<x+1$ for ant $x\in R$, and we are dense because $x<\frac{x+y}2<y$ for any $x,t\in R$. Note that checking $x<x+1$ (for example) requires knowing that $0<1$, which is a nontrivial fact on its own (one should use trichotomy and rule our $0=1$ by fields and rule our $0>1$ because this would imply $-1>0$ and then $1>0$ by squaring). There are lots of these nontrivial facts (e.g., we also want to know $0<1/2<1$), but we won't bother to show this.
\end{remark}
For ordered fields, there is an order topology, and one can show that various functions like $+$ and $\cdot$ and polynomials are all continuous.

We will define the function $\left|\cdot\right|\colon R\to R$ given by
\[\left|x\right|\coloneqq\begin{cases}
	+x & \text{if }x\ge0, \\
	-x & \text{if }x\le0.
\end{cases}\]
Now, if $x<0$, then $-x>0$ by subtraction, so we see that $\left|x\right|>0$ for all $x\ne0$. The standard casework is also able to prove the triangle inequality $\left|x+y\right|\le\left|x\right|+\left|y\right|$ by some casework. If both nonpositive or nonnegative, then we have equality, and if they have different signs (say, $x>0>y$ without loss of generality and $\left|x\right|\ge\left|y\right|$), then we are looking at $x+y\le x-y$, which is true.

For notation, we will also want the function $\op{sgn}$ given by
\[\op{sgn}(x)\coloneqq\begin{cases}
	+1 & \text{if }x>0, \\
	0 & \text{if }x=0, \\
	-1 & \text{if }x<0.
\end{cases}\]
Now, one is able to check the following, which tells us that polynomials ``go off to infinity.''
\begin{proposition} \label{prop:off-to-infinity}
	Fix an ordered field $\mathcal R$ and a polynomial $f(x)\in R[x]$ of positive degree $d$, written
	\[f(x)=\sum_{i=0}^dc_dx^d\]
	where $c_d\ne0$. If
	\[x>1+\frac1{\left|c_d\right|}\sum_{i=0}^{d-1}\left|c_i\right|,\]
	then $\op{sgn}f(x)=\op{sgn}c_d$.
\end{proposition}
\begin{proof}
	Boring bounding. Note
	\[\op{sgn}f(x)=\op{sgn}c_d\cdot\op{sgn}\Bigg(x^d+\sum_{i=0}^{d-1}\frac{c_i}{c_d}x^d\Bigg),\]
	so by scaling down, it is enough to consider the case where $c_d=1$.

	As an aside, we note that any $x\ge1$ and nonnegative integer $n$ will have $x^n\ge x$, which is true by induction because $x^{n+1}\ge x^n$, where our base case is $x^1=x\ge 1=x^0$. With this in mind, we see that $x$ satisfying the desired inequality will have
	\[x^d=x\cdot x^{d-1}>\sum_{i=0}^{d-1}\left|c_i\right|x^{d-1}\ge\sum_{i=0}^{d-1}\left|c_i\right|x^i\ge\sum_{i=0}^{d-1}-c_ix^i,\]
	so $f(x)>0$ follows.
\end{proof}
\begin{corollary}
	If $\mathcal R$ is a real closed field and $a\ge0$, then there exists $b\ge0$ such that $b^2=a$.
\end{corollary}
\begin{proof}
	If $a=0$, set $b=0$. Otherwise, consider the polynomial $f(x)\coloneqq x^2-a$. Note $f(0)<0$, and \Cref{prop:off-to-infinity} tells us that $f(1+a)>0$, so the intermediate value theorem for polynomials tells us that there is some $b$ such that $f(b)=0$, so $b^2=a$.
\end{proof}
\begin{corollary}
	If $\mathcal R$ is a real closed field, then any polynomial $f(x)$ of odd degree has a root.
\end{corollary}
\begin{proof}
	Write
	\[f(x)=\sum_{i=0}^dc_ix^i\]
	where $d$ is odd and $c_d\ne0$, and let $N\coloneqq2+\frac1{\left|c_d\right|}\sum_{i=0}^{d-1}\left|c_i\right|$. By \Cref{prop:off-to-infinity}, we have $N$ such that $\op{sgn}f(N)=\op{sgn}c_d$, and we see similarly that the polynomial $f(-x)$ will now have $\op{sgn}f(-N)=\op{sgn}(-c_d)$. Thus, $f(N)$ and $f(-N)$ have different signs, so the intermediate value theorem for polynomials grants $f$ a root.
\end{proof}
The above two corollaries turn out to characterize real closed fields.
\begin{remark}
	We can now remove the ordering from our real closed fields by declaring that squares are exactly the nonnegative elements. It is in general an interesting question when we can give a field an order; for example, $-1$ cannot be a sum of squares because $-1<0$. This turns out to be good enough to make a field orderable!
\end{remark}

\end{document}