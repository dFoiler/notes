% !TEX root = ../notes.tex

\documentclass[../notes.tex]{subfiles}

\begin{document}

\section{August 31}

Professor Scanlon is back. Let's prove the Compactness theorem. We are going to prove $2.5$ times.

\subsection{Proof of Compactness}
Recall the statement.
\compact*
\begin{remark}
	This result is special to first-order logic: in some sense, \Cref{thm:compactness} combined with a corollary characterizes first-order logic among various logics. Roughly speaking, one wants to formalize what a logic is with its various structures and sentences should do.
\end{remark}
\begin{proof}[Proof with completeness]
	We can prove this result fairly quickly given the Completeness theorem. Recall that the Completeness theorem says that any theory fails to be satisfiable if and only if there is a proof of contradiction $\perp$; one writes that a theory $T$ proves a sentence $\varphi$ by $T\vdash\varphi$. We have not discussed how formal proofs work, and we won't discuss it further because this is not a proofs class. Approximately speaking, a formal proof is a list of steps one can use the sentence sin $T$ to produce $\varphi$ syntactically.

	Now, suppose that $T$ fails to be satisfiable. Then there is a proof of $\perp$. But then one can look at the proof, which is necessarily finite in length, and then we pick up any sentence $\varphi$ occurring in the proof of $\perp$. But then we have a proof of $\perp$ using only finitely many sentences in $T$, so $T$ fails to be finitely satisfiable! This completes the proof.
\end{proof}
Anyway, let's present an actual proof.
\begin{definition}[witness]
	Fix a theory $T$ of a language $\mathcal L$. Then $T$ has \textit{witnesses} (or \textit{Henkin constants}) if and only if each formula $\varphi(x)$ in one free variable $x$ has a constant symbol $c$ such that $\exists x\varphi(x)\to\varphi(c)$ lives in $T$.
\end{definition}
\begin{remark}
	If $T$ has witnesses, then $T'\supseteq T$ also has witnesses for any theory $T'$ extending $T$.
\end{remark}
Let's quickly sketch our proof.
\begin{enumerate}
	\item We will show that if $T$ is finitely satisfiable, then there is an expanded language $\mathcal L'\supseteq\mathcal L$ and expanded finitely satisfiable $\mathcal L'$-theory $T'\supseteq T$ of $\mathcal L'$ such that $\left|\mathcal L'\right|\le\left|\mathcal L\right|+\aleph_0$, and $T'$ has witnesses (as does any extended theory $T''$ of $T'$).

	\item Next, suppose $T$ is a maximally finitely satisfiable theory (i.e., $T$ is finitely satisfiable, and any proper extension $T'\supseteq T$ fails to be finitely satisfiable\footnote{Such a thing exists by some sort of Zorn's lemma argument: note that there is a theory containing $T$ which fails to be finitely satisfiable: take the set of all sentences!}). Then we will show $T$ is complete (i.e., each sentence $\varphi$ has either $\varphi\in T$ or $\lnot\varphi\in T$).

	\item From here, we want to extend maintaining being complete: if $T$ is finitely satisfiable, then there is an extended language $\mathcal L'\supseteq\mathcal L$ of size $\left|\mathcal L'\right|=\left|\mathcal L\right|+\aleph_0$ and extended theory $T'$ of $T$ which is complete, finitely satisfiable, and has witnesses.

	We can prove this using the previous two steps. Indeed, the first step provides an extended language $\mathcal L'$ (of cardinality at most $\left|\mathcal L\right|+\aleph_0$) and extended theory $T'$ which is finitely satisfiable and has witnesses. Then we use Zorn's lemma to become maximally finitely satisfiable. Let $\mathcal P$ denote the set of finitely satisfiable $\mathcal L'$-theories $T''$ extending $T'$ which is finitely satisfiable. Containment shows that $\mathcal P$ is a partial order, and it's nonempty because $T'\in\mathcal P$. Next up, we note that any nonempty chain $\{T_\alpha\}_{\alpha\in\lambda}$ is upper-bounded by
	\[\bigcup_{\alpha\in\lambda}T_\alpha,\]
	which we can see continues to be finitely satisfiable (any finite set belongs to some fix $T_\beta$ for $\beta$ perhaps large) and thus lives in $\mathcal P$ and succeeds to upper-bound our chain. Thus, Zorn's lemma provides a maximally finitely satisfiable theory $T''$ containing $T'$, which will be complete by the previous step. Because $T''$ contains $T'$, we continue to have witnesses.

	\item We are now ready to do our construction. At this point, we may assume that our theory $T$ is finitely satisfiable, complete, and has witnesses. Then we claim that there is a model $\mathcal M$ such that $\left|M\right|\le\left|\mathcal L\right|$.
	
	In fact, the model can be described somewhat explicitly. Take $M\coloneqq\mathcal C/{\sim}$ where $\mathcal C$ is our set of constants, and our equivalence relation $\sim$ is given by $c\sim d$ if and only if $(c=d)\in T$. Notably, constants $c\in\mathcal C$ are interpreted as $c^\mathcal M\coloneqq[c]$. To interpret functions $f$, we have $f^\mathcal M([a_1],\ldots[a_n])=[d]$ if and only if $(f(a_1,\ldots,a_n)=d)\in T$. Lastly, to interpret relations $R$, we have $R^\mathcal M([a_1],\ldots,[a_n])$ if and only if $(R(a_1,\ldots,a_n))\in T$.
\end{enumerate}
Let's start implementing the details.
\begin{remark}
	In logic, the answer to a question is often the question. For example, in step 4, we see that $T$ has a model because $T$ says that it has a model.
\end{remark}
Here is our first step.
\begin{lemma}
	Fix a finitely satisfiable theory $T$ of a language $\mathcal L$. Then there is an expanded language $\mathcal L'\supseteq\mathcal L$ and expanded finitely satisfiable $\mathcal L'$-theory $T'\supseteq T$ of $\mathcal L'$ such that $\left|\mathcal L'\right|\le\left|\mathcal L\right|+\aleph_0$, and $T'$ has witnesses.
\end{lemma}
\begin{proof}
	We would like to just set $T'$ to be $T$ together with new constants providing witnesses for all formulae. But these new constants will make new formulae, so we need to do some kind of induction to go upwards.

	With this in mind, we will build an increasing sequence of languages
	\[\mathcal L_0\coloneqq\mathcal L\subseteq\mathcal L_1\subseteq\mathcal L_2\subseteq\cdots\]
	and theories
	\[T_0\coloneqq T\subseteq T_1\subseteq T_2\subseteq\cdots\]
	such that $T_n$ is always a finitely satisfiable $\mathcal L_n$-theory, and each $\mathcal L_n$-formula $\varphi$ with one free variable has a constant $c\in\mathcal C_{\mathcal L_n}$ such that $\exists x\varphi(x)\to\varphi(c)$ lives in $T_n$. We will then set $\mathcal L'$ to be the union of the $\mathcal L_\bullet$ and $T'$ to be the union of the $T_\bullet$, and this will complete the proof.

	We have already built the $n=0$ stage, as above. Then to build $\mathcal L_{n+1}$ from $\mathcal L_n$, add in new constant symbols $c_{\varphi(x)}$ for each $\mathcal L_n$-formula $\varphi(x)$ with one free variable; all the functions and relations remain the same. Note $\mathcal L_{n+1}$ is now the size of the formulae with one free variable in $\mathcal L_n$, so $\left|\mathcal L_{n+1}\right|=\left|\mathcal L_n\right|+\aleph_0$.

	As for our theory, let $T_{n+1}$ be $T_n$ plus the sentences $\exists x\varphi(x)\to\varphi\left(c_{\varphi(x)}\right)$ for each $\mathcal L_n$-formula $\varphi(x)$ with one free variable. We are now ready to set
	\[\mathcal L'\coloneqq\bigcup_{n\in\NN}\mathcal L_n\qquad\text{and}\qquad T'\coloneqq\bigcup_{n\in\NN}T_n.\]
	We see that $\mathcal L'$ then has the right size, and $T'$ has witnesses: for any $\mathcal L'$-formula $\varphi(x)$ with one free variable, note that $\varphi(x)$ has only finitely many symbols, so we can find some fixed level $\mathcal L_n$ containing all the symbols used in $\varphi(x)$. But then $\varphi(x)$ has a witness from $T_{n+1}\subseteq T'$, as needed.

	It remains to show that $T'$ is finitely satisfiable. It suffices to show that $T_n$ is finitely satisfiable for any $n\in\NN$ because any finite set will be contained in some $T_n$. We show this by induction. For $n=0$, there is nothing to say. Now suppose $T_n$ is finitely satisfiable, and we show that $T_{n+1}$ is finitely satisfiable.
	
	Fix some finite subset $\Delta\subseteq T_{n+1}$ which we would like to build a model for. Now, $\Delta$ will be built by some sentences in $T_n$ plus some new sentences from $T_{n+1}$. Looking hard at $T_{n+1}\setminus T_n$, we see that we can enumerate $\Delta\setminus T_n$ as some sentences
	\[\exists x\psi_k(x)\to \psi_k(c_k)\]
	where $\{\psi_k\}_{k=1}^m$ is some $\mathcal L_n$-formulae in one free variable.
	
	We now begin building our model. Note $\Delta\cap T_n$ is a finite subset of $T_n$, so it is satisfiable by some model $\mathcal M$. Now, for each $k$, if there is some $a\in M$ such that $\mathcal M\models\varphi_k(a)$, set $a\coloneqq a_k$; otherwise, set $a_k\coloneqq m$ for any chosen $m\in M$. (Note structures are nonempty.) We now let $\mathcal M'$ be the $\mathcal L_{n+1}$-structure with universe $M$, interpretations of functions and relations the same as in $\mathcal M$, and all old constant symbols are also all still interpreted the same way. Then for each new constant symbol, we interpret $c_k^\mathcal M\coloneqq a_k$, and each other new constant symbol can also go to $m$. Now $\mathcal M'$ is a model for $\Delta$ because it models everything in $\Delta\cap T_n$ for free, and it has satisfied $\Delta\setminus T_{n+1}$ by construction, so we are done.
\end{proof}
To show the second step, we begin with the following lemma.
\begin{lemma} \label{lem:fs-can-extend-complete}
	Fix a finitely satisfiable theory $T$ of a language $\mathcal L$. For any $\mathcal L$-sentence $\varphi$, then either $T\cup\{\varphi\}$ or $T\cup\{\lnot\varphi\}$ is finitely satisfiable.
\end{lemma}
\begin{proof}
	Suppose that both $T\cup\{\varphi\}$ and $T\cup\{\lnot\varphi\}$ both fail to be finitely satisfiable. We will show that $T$ fails to be finitely satisfiable.

	Well, we are given finite subsets $\Delta_+\subseteq T\cup\{\varphi\}$ and $\Delta_-\subseteq T\cup\{\lnot\varphi\}$ which are not satisfiable. If $\Delta_+$ fails to contain $\varphi$, then $\Delta_+$ is a finite subset of $T$ which is not satisfiable, so $T$ fails to be satisfiable. Thus, we may assume that $\varphi\in\Delta_+$. Analogously, we may assume that $\lnot\varphi\in\Delta_-$. Now, $(\Delta_+\cup\Delta_-)\setminus\{\varphi\}$ and $(\Delta_+\cup\Delta_-)\setminus\{\lnot\varphi\}$ both fail to be satisfiable.

	But now suppose for the sake of contradiction that $T$ is finitely satisfiable. Then $(\Delta_+\cup\Delta_-)\setminus\{\varphi,\lnot\varphi\}$ has a model $\mathcal M$. But $\mathcal M\models\varphi$ or $\mathcal M\models\lnot\varphi$, so we see that $\mathcal M$ will model at least one of $(\Delta_+\cup\Delta_-)\setminus\{\varphi\}$ or $(\Delta_+\cup\Delta_-)\setminus\{\lnot\varphi\}$, which is the desired contradiction.
\end{proof}
The second step now follows from a Zorn's lemma argument.
\begin{lemma}
	Fix a maximally finitely satisfiable theory $T$ of a language $\mathcal L$. Then $T$ is complete.
\end{lemma}
\begin{proof}
	Let $\varphi$ be any $\mathcal L$-sentence. Then either $T\cup\{\varphi\}$ or $T\cup\{\lnot\varphi\}$ is finitely satisfiable by \Cref{lem:fs-can-extend-complete}, so by maximality, we may conclude that either $T=T\cup\{\varphi\}$ or $T=T\cup\{\lnot\varphi\}$, so either $\varphi\in T$ or $\lnot\varphi\in T$, which is what we wanted.
\end{proof}

\end{document}